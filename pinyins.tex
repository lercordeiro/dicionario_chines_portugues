
%%%%%%%%%%%%%%%%%%%%%%%%%%%%%%%%%%%%%%%%%
% LuaLaTex
%
% Dicionário Chinês -> Português
% Autor: Luiz Eduardo Roncato Cordeiro
%
% Licença:
% CC BY-NC-SA 3.0 (http://creativecommons.org/licenses/by-nc-sa/3.0/)
%%%%%%%%%%%%%%%%%%%%%%%%%%%%%%%%%%%%%%%%%

\documentclass[a4paper,9pt,twoside,openright,book]{memoir}

\usepackage[brazilian]{babel}
\usepackage{fontspec}
\usepackage[dvipsnames]{xcolor}
\usepackage{imakeidx}
\usepackage[inline]{enumitem}
\usepackage{zhnumber}
\usepackage{tikz}
\usepackage[hyperindex]{hyperref}
\usepackage{pifont}
\usepackage{xstring}
\usepackage{xifthen}
\usepackage{tabularray}
\usepackage[most]{tcolorbox}
\usepackage{luacode}
\usepackage{stackengine}

% Meus Comandos
%%%%%%%%%%%%%%%%%%%%%%%%%%%%%%%%%%%%%%%%%%%%%%%%%%%%%%%%%%%%%%%%%%%%%%%%%%%%%%%
%%%%%%%%%%%%%%%%%%%%%%%%%%%%%%%%%%%%%%%%%%%%%%%%%%%%%%%%%%%%%%%%%%%%%%%%%%%%%%%
%%%%%                                                                     %%%%%
%%%%% Funções e Ajustes dos Documentos do Dicionário                      %%%%%
%%%%%                                                                     %%%%%
%%%%%%%%%%%%%%%%%%%%%%%%%%%%%%%%%%%%%%%%%%%%%%%%%%%%%%%%%%%%%%%%%%%%%%%%%%%%%%%
%%%%%%%%%%%%%%%%%%%%%%%%%%%%%%%%%%%%%%%%%%%%%%%%%%%%%%%%%%%%%%%%%%%%%%%%%%%%%%%

%%% Espaçamento das linhas normal
\SingleSpacing

%%% Hyperref em modo 'draft' não gera os hiperlinks
\hypersetup{final}

%%% Ajustes da separação das colunas quando em modo texto de 2 colunas
\setlength{\columnsep}{1.2em}
\setlength{\columnseprule}{0.1mm}

%%% Estilo do capítulo, o melhor que encontrei
\chapterstyle{dash}

%%% Sem identação
\setlength{\parindent}{0cm}
\setlength{\parskip}{0.15\baselineskip}

%%% Ajuste das margens do documento
\setlrmarginsandblock{3cm}{2cm}{*}
\setulmarginsandblock{2cm}{*}{1}
\checkandfixthelayout

%%% Pra evitar viúvas e órfãs
\clubpenalty=10000
\widowpenalty=10000
\raggedbottom

%%% Usando a fonte NoTofu do Google.
\babelfont{rm}[
 Renderer=Harfbuzz,
 Ligatures=TeX,
 BoldFont={NotoSerifCJKsc-Bold},
 BoldSlantedFont={NotoSansCJKsc-Regular},
 AutoFakeSlant=0.25,
 SlantedFeatures={FakeSlant=0.25},
 BoldSlantedFeatures={FakeSlant=0.25}]{Noto Serif CJK SC}
\babelfont{sf}[Renderer=Harfbuzz,Ligatures=TeX]{Noto Sans CJK SC}
\babelfont{tt}[Renderer=Harfbuzz,Ligatures=TeX]{Noto Sans Mono CJK SC}

%%% Ajustes do MultiCol: parar com a indentação do primeiro parágrafo
%\AddToHook{env/multicols/begin}{\AddToHookNext{para/begin}{\OmitIndent}}

%%% Ajustes do Sumário
\setcounter{secnumdepth}{2}
\makeatletter
\renewcommand{\@pnumwidth}{2em} 
\renewcommand{\@tocrmarg}{4em}
\makeatother
\renewcommand\cftbeforechapterskip{5pt plus 1pt}

%%% Cria 'Lista de Hanzis'
\newcommand{\listlohname}{Primeiros Hanzis}
\newlistof{listoffirsthanzis}{loh}{\listlohname}

%%% Ajustes de Cabeçalhos e Rodapés
\setheadfoot{14pt}{28pt}

% Estilo "plain"
\makefootrule{plain}{\textwidth}{\normalrulethickness}{2pt}
\ifdraftdoc
 \makeevenfoot{plain}{\thepage}{汉葡词典}{Draft}
 \makeoddfoot{plain}{Draft}{汉葡词典}{\thepage}
\else
 \makeevenfoot{plain}{\thepage}{汉葡词典}{}
 \makeoddfoot{plain}{}{汉葡词典}{\thepage}
\fi

% Estilo "dictionary"
\makepagestyle{dictionary}
\makeheadrule{dictionary}{\textwidth}{\normalrulethickness}
\makefootrule{dictionary}{\textwidth}{\normalrulethickness}{2pt}
\ifdraftdoc
 \makeevenhead{dictionary}{\rightmark}{Draft}{\leftmark}
 \makeoddhead{dictionary}{\rightmark}{Draft}{\leftmark}
 \makeevenfoot{dictionary}{\thepage}{汉葡词典}{Draft}
 \makeoddfoot{dictionary}{Draft}{汉葡词典}{\thepage}
\else
 \makeevenhead{dictionary}{\rightmark}{}{\leftmark}
 \makeoddhead{dictionary}{\rightmark}{}{\leftmark}
 \makeevenfoot{dictionary}{\thepage}{汉葡词典}{}
 \makeoddfoot{dictionary}{}{汉葡词典}{\thepage}
\fi

\newcommand{\boxedsec}[1]
 {%
  \begin{tcolorbox}%
   [%
    enhanced,%
    nobeforeafter,%
    before={\noindent},%
    colframe=black,%
    colback=black!20!white,%
    boxrule=2pt,%
    leftrule=4mm,%
    left=0mm,%
    right=0mm,%
    top=0mm,%
    bottom=0mm%
   ]
   \hfill\LARGE\bfseries#1
  \end{tcolorbox}
 }
\setsecheadstyle{\boxedsec}
\newcommand{\sectionbreak}{\phantomsection}

\newcommand{\boxedsubsec}[1]
 {%
  \begin{tcolorbox}%
   [%
    enhanced,%
    nobeforeafter,%
    before={\noindent},%
    colframe=black,%
    colback=black!10!white,%
    boxrule=2pt,%
    leftrule=2mm,%
    left=0mm,%
    right=0mm,%
    top=0mm,%
    bottom=0mm%
   ]
   \hfill\Large#1
  \end{tcolorbox}
 }
\setsubsecheadstyle{\boxedsubsec}

%%% Estilo das caixas dos verbetes
\newtcolorbox{lightbox}%
 {%
  enhanced,%
  size=fbox,%
  colframe=black,%
  colback=white,%
  boxrule=1pt,%
  toprule=3pt,%
  left=0mm,%
  right=0mm,%
  top=0mm,%
  bottom=0mm,%
  middle=0mm,%
  nobeforeafter,%
  segmentation empty,%
  before={\noindent}%
 }
\newtcolorbox{darkbox}%
 {%
  enhanced,%
  size=fbox,%
  colframe=black,%
  colback=black!5!white,%
  boxrule=1pt,%
  toprule=3pt,%
  left=0mm,%
  right=0mm,%
  top=0mm,%
  bottom=0mm,%
  middle=0mm,%
  nobeforeafter,%
  segmentation empty,%
  before={\noindent}%
 }


%%% Variáveis tipo "bool" para dizer se tem ou não os campos
%%% "Veja", "Veja também", "Synonym"" e "Antonym"
%%% nas definições dos verbetes
\newbool{hassee}
\newbool{hasseealso}
\newbool{hassynonym}
\newbool{hasantonym}

%%% Converte os pinyins numéricos em pinyins com marcação de tom
\directlua{dofile "include/tex-sx-pinyin-tonemarks.lua"}

%%% Comandos genéricos usados no Dicionário

% Função "\pinyin" faz a conversão
\protected\def\pinyin#1{\directlua{packagedata.pinyintones.convert ([==[#1]==])}}

\ExplSyntaxOn

% Comando "\dictpinyin", coloca o pinyin entre «»
\NewDocumentCommand{\dictpinyin}{m}{\guillemotleft\pinyin{#1}\guillemotright} 

% Comando "\dpy", gera a string do pinyin utilizada no Dicionário
% Este comando realiza uma série de substituições antes
\NewDocumentCommand{\dpy}{m}%
 {%
  \StrSubstitute{#1}{5}{}[\result]%
  \StrSubstitute{\result}{v}{ü}[\result]%
  \StrSubstitute{\result}{V}{Ü}[\result]%
  \edef\py{\dictpinyin{\result}}%
  \mbox{}\py
 }

% Yin, Yang e Os Oito Trigramas
\newfontfamily\dejavusans{DejaVu Sans}
\DeclareRobustCommand{\Yin}{{\dejavusans\symbol{"268A}}}
\DeclareRobustCommand{\Yang}{{\dejavusans\symbol{"268B}}}
\DeclareRobustCommand{\TrigramHeaven}{{\dejavusans\symbol{"2630}}}
\DeclareRobustCommand{\TrigramLake}{{\dejavusans\symbol{"2631}}}
\DeclareRobustCommand{\TrigramFire}{{\dejavusans\symbol{"2632}}}
\DeclareRobustCommand{\TrigramThunder}{{\dejavusans\symbol{"2633}}}
\DeclareRobustCommand{\TrigramWind}{{\dejavusans\symbol{"2634}}}
\DeclareRobustCommand{\TrigramWater}{{\dejavusans\symbol{"2635}}}
\DeclareRobustCommand{\TrigramMountain}{{\dejavusans\symbol{"2636}}}
\DeclareRobustCommand{\TrigramEarth}{{\dejavusans\symbol{"2637}}}

% Comando "\&", insere o caracgter "&"
%\DeclareRobustCommand{\&}%
% {%
%  \ifdim\fontdimen1\font>0pt%
%   \textsl{\symbol{`\&}}%
%  \else%
%   \symbol{`\&}%
%  \fi%
% }

\NewDocumentCommand{\setvar}{mm}
 {
  % clear an existing variable or allocate a new one
  \tl_clear_new:c { g__youthdoo_var_#1_tl }
  % set to the stated value
  \tl_gset:cn { g__youthdoo_var_#1_tl } { #2 }
 }

\NewExpandableDocumentCommand{\usevar}{m}
 {
  % deliver the contents
  \tl_use:c { g__youthdoo_var_#1_tl }
 }

% Ambiente "enumerate" especial utilizado no dicionário, coloca as definições 
% do verbete em uma lista numerada em linha
\NewDocumentCommand{\dictenumerate}{>{\SplitList{|}}m}
 {%
  \begin{enumerate*}[nosep,label={\arabic*},left=0pt,mode=boxed,font=\bfseries]
   \ProcessList{#1}{\insertitem}
  \end{enumerate*}
 }
\NewDocumentCommand{\insertitem}{>{\TrimSpaces}m}{\item #1}

% Ambiente "enumerate" especial utilizado no dicionário, coloca os exemplos
% das definições do verbete em uma lista numerada em linha, utilizando
% algarismos romanos
\makeatletter
\NewDocumentCommand{\dictexamples}{m>{\SplitList{|}}m}
 {%
  \def\@theword{#1}% 
  \begin{sloppypar}
   \begin{enumerate}[nosep,label=\alph*),left=0pt,mode=boxed,font=\bfseries]
    \ProcessList{#2}{\insertexample}
   \end{enumerate}
  \end{sloppypar}
 }
\NewDocumentCommand{\insertexample}{>{\TrimSpaces}m}
 {%
   \IfSubStr{#1}{===}
   {% Com traducao
     \StrCut{#1}{===}\csH\csP%
     \StrSubstitute{\csH}{\@theword}{\underline{\@theword}}[\csHUL]%
     \item\foreignlanguage{chinese-hans}{\csHUL}\\*\textit{\footnotesize``\csP''}
   }
   {% Sem traducao
     \StrSubstitute{#1}{\@theword}{\underline{\@theword}}[\csHUL]%
     \item\foreignlanguage{chinese-hans}{\csHUL}
   }
 }  
\makeatother

%%% Cria listas especializadas (seelist, seealsolist, synonymlist e antonymlist)
\newlist{seelist}{enumerate}{1}
\newlist{seealsolist}{enumerate}{1}
\newlist{synonymlist}{enumerate}{1}
\newlist{antonymlist}{enumerate}{1}

\setlist[seelist]{label={\roman*)},topsep=0pt,nosep,noitemsep,font=\bfseries}
\setlist[seealsolist]{label={\roman*)},topsep=0pt,nosep,noitemsep,font=\bfseries}
\setlist[synonymlist]{label={\roman*)},topsep=0pt,nosep,noitemsep,font=\bfseries}
\setlist[antonymlist]{label={\roman*)},topsep=0pt,nosep,noitemsep,font=\bfseries}

%%% Cria e inicializa a lista "Veja", "Veja também", "Sinônimo(s)" e "Antônimo(s)"
\newcommand\seerefl{}
\newcommand\seealsorefl{}
\newcommand\synonymrefl{}
\newcommand\antonymrefl{}

%%% Comando "\areference", adiciona um item "Veja", "Veja também", "Antônimo(s)" ou "Sinônimo(s)" na lista,
%%% com os pinyins abaixo dos caracteres
\newcommand{\areference}[2]
 {
  \StrLen{#1}[\hlen]%
  \StrLen{#2}[\plen]%
  \ifnumcomp{\hlen+\plen}{>}{24}
   {%
    \foreignlanguage{chinese-hans}{#1}\ (pág.~\pageref{\l_label_tl : #1 : #2})\\*
    \dpy{#2}
   }
   {%
    \foreignlanguage{chinese-hans}{#1}\ \dpy{#2}\ (pág.~\pageref{\l_label_tl : #1 : #2})
   }
 }

%%% Comando "\definition", gera o texto da definição
\NewDocumentCommand{\definition}{sommo}
 {%
  \begin{midsloppypar}
   \IfBooleanTF{#1}%
    {% Substantivo Próprio
     {\small\ding{108}}\ (\textit{S.P.})\IfValueT{#2}{~[clas.:~#2]}{\ \dictenumerate{#4}}\par
    }%
    {%
     {\small\ding{108}}\ (\textit{#3})\IfValueT{#2}{~[clas.:~#2]}{\ \dictenumerate{#4}}\par
    }%
  \end{midsloppypar}
  \IfValueT{#5}%
   {%
    \IfSubStr{#5}{|}{\textbf{Exemplos:}}{\textbf{Exemplo:}}\dictexamples{\l_hanzi_tl}{#5}\par
   }%
 }

%%% Comando "Variante de"
\NewDocumentCommand{\variantof}{m}
 {
  {\small\ding{108}}\ Variante\ de\ \foreignlanguage{chinese-hans}{#1}\ (p.~\pageref{\l_label_tl : #1 : \l_pinyin_tl})\par
 }

%%% Comando "Veja"
\NewDocumentCommand{\seeref}{m}{\booltrue{hassee}\listgadd{\seerefl}{\l_hanzi_tl : #1}}

%%% Comando "Veja também"
\NewDocumentCommand{\seealsoref}{mm}{\booltrue{hasseealso}\listgadd{\seealsorefl}{#1 : #2}}

%%% Comando "Sinônimo(s)"
\NewDocumentCommand{\synonymref}{mm}{\booltrue{hassynonym}\listgadd{\synonymrefl}{#1 : #2}}

%%% Comando "Antônimo(s)"
\NewDocumentCommand{\antonymref}{mm}{\booltrue{hasantonym}\listgadd{\antonymrefl}{#1 : #2}}

%%% Ambiente "DictionaryEntries" para definir o início das entradas dos verbetes
\NewDocumentEnvironment{DictionaryEntries}{m}%
 {%
  \tl_set:Nn \l_label_tl {#1}
  \pagestyle{dictionary}
  \twocolumn
 }%
 {%
  \onecolumn
 }%

%%% Ambiente "EntryWithPhonetic", para os verbetes
\NewDocumentEnvironment{EntryWithPhonetic}{mO{}mO{}mmooo}%
 {%
  \leavevmode
  \markboth{#1{\tiny\dpy{#3}}}{#1{\tiny\dpy{#3}}}
  \tl_set:Nn \l_hanzi_tl {#1}
  \tl_set:Nn \l_pinyin_tl {#3}
  \tl_set:Nn \l_strokes_tl {#5}
  \boolfalse{hassee}\renewcommand\seerefl{}
  \boolfalse{hasseealso}\renewcommand\seealsorefl{}
  \boolfalse{hassynonym}\renewcommand\synonymrefl{}
  \boolfalse{hasantonym}\renewcommand\antonymrefl{}
  \label{\l_label_tl : #1 : #3}
  \StrLen{#1}[\hlen]%
  \StrLen{#3}[\plen]%
  \begin{lightbox}
   \ifnumcomp{\hlen}{>}{10}
    {%
     \mbox{}\hfill\textsuperscript{\tiny(#5画)}\\
     {\Large\foreignlanguage{chinese-hans}{#1}}
    }
    {%
     {\Large\foreignlanguage{chinese-hans}{#1}}\hfill\textsuperscript{\tiny(#5画)}
    }
   \tcblower
   \ifnumcomp{\plen}{>}{25}
    {%
     {\footnotesize#2\ \dpy{#3}\ #4}\\
    }
    {%
      {\footnotesize#2\ \dpy{#3}\ #4}
    }
   \IfValueT{#7}{\mbox{}\hfill{\tiny#7}}{}%
   \IfValueT{#8}{\mbox{}\hfill{\tiny#8}}{}%
   \IfValueT{#9}{\mbox{}\hfill{\tiny#9}}{}%
   \mbox{}\hfill\IfSubStr{#6}{、}{\tiny Radicais\ #6}{\tiny Radical\ #6}
  \end{lightbox}
 }%
 {%
  \ifbool{hassee}%
   {% Processa as referências "Veja"
    \RenewDocumentCommand\do{>{\SplitArgument{1}{:}}m}{\item \areference ##1}
    \textbf{Veja:}%
    \begin{seelist}
     \dolistloop{\seerefl}
    \end{seelist}
   }{}%
  \ifbool{hasseealso}%
   {% Processa as referências "Veja também"
    \RenewDocumentCommand\do{>{\SplitArgument{1}{:}}m}{\item \areference ##1}
    \textbf{Veja\ também:}%
    \begin{seealsolist}
     \dolistloop{\seealsorefl}
    \end{seealsolist}
   }{}%
  \ifbool{hassynonym}%
   {% Processa as referências "Sinônimo"
    \RenewDocumentCommand\do{>{\SplitArgument{1}{:}}m}{\item \areference ##1}
    \textbf{Sinônimo(s):}%
    \begin{synonymlist}
     \dolistloop{\synonymrefl}
    \end{synonymlist}
   }{}%
  \ifbool{hasantonym}%
   {% Processa as referências "Antônimo"
    \RenewDocumentCommand\do{>{\SplitArgument{1}{:}}m}{\item \areference ##1}
    \textbf{Antônimo(s):}%
    \begin{antonymlist}
     \dolistloop{\antonymrefl}
    \end{antonymlist}
   }{}%
 }

%%% Ambiente "Entry", para os verbetes
\NewDocumentEnvironment{Entry}{mmmooo}%
 {%
  \leavevmode
  \markboth{#1{\tiny(#2画)}}{#1{\tiny(#2画)}}
  \tl_set:Nn \l_hanzi_tl {#1}
  \tl_set:Nn \l_strokes_tl {#2}
  \StrLen{#1}[\hlen]%
  \begin{lightbox}
   \ifnumcomp{\hlen}{>}{10}
    {%
     \mbox{}\hfill\textsuperscript{\tiny(#2画)}\\
     {\Large\foreignlanguage{chinese-hans}{#1}}
    }
    {%
     {\LARGE#1}\hfill\textsuperscript{\tiny(#2画)}
    }
   \tcblower
   \IfValueT{#4}{\mbox{}\hfill{\tiny#4}}{}%
   \IfValueT{#5}{\mbox{}\hfill{\tiny#5}}{}%
   \IfValueT{#6}{\mbox{}\hfill{\tiny#6}}{}%
   \mbox{}\hfill\IfSubStr{#3}{,}{\tiny Radicais\ #3}{\tiny Radical\ #3}
  \end{lightbox}\par
 }{}

%%% Ambiente "Phonetics", para as diversas entradas de fonética da palavra
\NewDocumentEnvironment{Phonetics}{mO{}mO{}O{}}%
 {%
  \tl_set:Nn \l_pinyin_tl {#3}
  \boolfalse{hassee}\renewcommand\seerefl{}
  \boolfalse{hasseealso}\renewcommand\seealsorefl{}
  \boolfalse{hassynonym}\renewcommand\synonymrefl{}
  \boolfalse{hasantonym}\renewcommand\antonymrefl{}
  \label{\l_label_tl : #1 : #3}
   \ding{93}\ #2\ \dpy{#3}\ #4\ \ding{93}\hfill \textbf{#5}
 }%
 {%  
  \ifbool{hassee}%
   {% Processa as referências "Veja"
    \RenewDocumentCommand\do{>{\SplitArgument{1}{:}}m}{\item \areference ##1}
    \textbf{Veja:}%
    \begin{seelist}
     \dolistloop{\seerefl}
    \end{seelist}
   }{}%
  \ifbool{hasseealso}%
   {% Processa as referências "Veja também"
    \RenewDocumentCommand\do{>{\SplitArgument{1}{:}}m}{\item \areference ##1}
    \textbf{Veja\ também:}%
    \begin{seealsolist}
     \dolistloop{\seealsorefl}
    \end{seealsolist}
   }{}%
  \ifbool{hassynonym}%
   {% Processa as referências "Sinônimo"
    \RenewDocumentCommand\do{>{\SplitArgument{1}{:}}m}{\item \areference ##1}
    \textbf{Sinônimo(s):}%
    \begin{synonymlist}
     \dolistloop{\synonymrefl}
    \end{synonymlist}
  }{}%
  \ifbool{hasantonym}%
   {% Processa as referências "Antônimo"
    \RenewDocumentCommand\do{>{\SplitArgument{1}{:}}m}{\item \areference ##1}
    \textbf{Antônimo(s):}%
    \begin{antonymlist}
     \dolistloop{\antonymrefl}
    \end{antonymlist}
   }{}%
 }

\ExplSyntaxOff

%%%%% EOF %%%%%


% Ajustes do PDF
\hypersetup{
  linktoc=page,
  colorlinks=true,
  urlcolor=blue,
  linkcolor=blue,
  citecolor=blue,
  pdftitle={汉葡词典 - Dicionário Chinês-Português},
  pdfsubject={Dicionário Chinês-Português -- Ordenado por Pinyin},
  pdfauthor={罗学凯, Luiz Eduardo Roncato Cordeiro},
  pdfkeywords={dicionário, chinês, português, instituto confúcio}
}

%%%
%%% Documento começa aqui!
%%%

\begin{document}
\addfontfeatures{CharacterWidth=Proportional}

\input{title.tex}

\clearpage
\pagestyle{empty}
\tableofcontents

\clearpage
\pagestyle{empty}
\chapter{汉葡词典}

%%%%%%%%%%%%%%%%%%%%%%%%
%
% https://en.wikipedia.org/wiki/Chinese_character_orders
%
%%%%%%%%%%%%%%%%%%%%%%%%

Dicionário Chinês-Português ordenado primeiro pelo pinyin de cada
caracter, depois pelo número de traços e, finalmente, pela ordem do
caracter na tabela UTF-8.

\clearpage
\begin{DictionaryEntries}{pinyins}
 %%%
%%% A
%%%

\section*{A}\addcontentsline{toc}{section}{A}

\begin{EntryWithPhonetic}{阿}{a1}{7}{⾩}
  \definition{pref.}{em dialetos do sul para formar termos carinhosos, antes de nomes de animais de estimação, sobrenomes monossilábicos ou números que denotam ordem de antiguidade em uma; anexado a 大, 二, 三,\dots\ para indicar classificação (e, às vezes, intimidade) | antes dos termos de parentesco; na frente de um sobrenome, de um nome próprio ou de um determinado título, com uma conotação de intimidade | em alguns contextos, pode soar infantil ou muito informal (por exemplo, chamar um colega de trabalho por ``阿 + Nome'' sem intimidade)}[阿妈===mamãe | 阿明 ===forma carinhosa de chamar uma pessoa chamada Ming]
  \seeref{e1}
\end{EntryWithPhonetic}

\begin{EntryWithPhonetic}{阿哥}{a1ge1}{7,10}{⾩、⼝}
  \definition{s.}{irmão mais velho (afetivo) | forma afetuosa de tratamento entre homens da mesma idade}[阿哥,帮我拿一下书包!===Irmão, ajude-me com minha mochila escolar!]
\end{EntryWithPhonetic}

\begin{EntryWithPhonetic}{阿拉伯语}{a1la1bo2 yu3}{7,8,7,9}{⾩、⼿、⼈、⾔}[HSK 7-9]
  \definition{s.}{árabe; idioma árabe}
\end{EntryWithPhonetic}

\begin{EntryWithPhonetic}{阿姨}{a1yi2}{7,9}{⾩、⼥}[HSK 4]
  \definition[个,位,名,些]{s.}{tia; uma forma de tratamento para uma mulher da geração dos pais; dirigir-se a uma mulher que tem aproximadamente a mesma idade da sua mãe, geralmente não é parente | babá em uma família; professora em um jardim de infância | tia; irmã da mãe (mais comum no sul da China)}[阿姨,生日快乐!===Tia, feliz aniversário! | 阿姨,这个苹果多少钱一斤?===Tia/Senhora, quanto custa o quilo dessas maçãs? | 阿姨,我想喝水。===Tia/Babá, eu quero beber água.]
\end{EntryWithPhonetic}

\begin{EntryWithPhonetic}{呵}{a1}{8}{⼝}
  \variantof{啊}
  \seeref{he1}
\end{EntryWithPhonetic}

\begin{EntryWithPhonetic}{啊}{a1}{10}{⼝}[HSK 2]
  \definition{interj.}{Ah!; Oh!; expressar surpresa ou admiração}
  \seeref{a2}
  \seeref{a3}
  \seeref{a4}
  \seeref{a5}
\end{EntryWithPhonetic}

\begin{EntryWithPhonetic}{啊呀}{a1ya1}{10,7}{⼝、⼝}
  \definition{interj.}{Oh meu Deus! | interjeição de surpresa}
\end{EntryWithPhonetic}

\begin{EntryWithPhonetic}{啊哟}{a1yo5}{10,9}{⼝、⼝}
  \definition{interj.}{Meu Deus! | Oh! | Ai! | interjeição de surpresa ou dor}
\end{EntryWithPhonetic}

\begin{EntryWithPhonetic}{啊}{a2}{10}{⼝}[HSK 2]
  \definition{interj.}{Eh?; Ei?; Que?; Por que?; expressar questionamento, dúvida ou solicitar opinião}
  \seeref{a1}
  \seeref{a3}
  \seeref{a4}
  \seeref{a5}
\end{EntryWithPhonetic}

\begin{EntryWithPhonetic}{嗄}{a2}{13}{⼝}
  \variantof{啊}
  \seeref{sha4}
\end{EntryWithPhonetic}

\begin{EntryWithPhonetic}{啊}{a3}{10}{⼝}[HSK 2]
  \definition{interj.}{Eh?; Meu!; E aí?; Que?; expressar surpresa e dúvida}
  \seeref{a1}
  \seeref{a2}
  \seeref{a4}
  \seeref{a5}
\end{EntryWithPhonetic}

\begin{EntryWithPhonetic}{啊}{a4}{10}{⼝}[HSK 2]
  \definition{interj.}{Bem!; Sim!; expressa concordância, pronúncia mais curta | Oh!; Ah!; indica que compreendeu, com pronúncia mais longa | Oh!; expressa surpresa ou admiração, com pronúncia mais longa | Desgraça!; expressa tristeza ou pesar}
  \seeref{a1}
  \seeref{a2}
  \seeref{a3}
  \seeref{a5}
\end{EntryWithPhonetic}

\begin{EntryWithPhonetic}{啊}{a5}{10}{⼝}[HSK 2,4]
  \definition{part.}{usado no final da frase para expressar admiração | usado no final da frase para expressar afirmação, justificativa, insistência, recomendação, etc. | usado no final da frase para indicar dúvida | usado para fazer uma pequena pausa na frase, chamando a atenção para o que vem a seguir | usado após os itens enumerados | usado após verbos repetitivos, indica um processo longo}
  \seeref{a1}
  \seeref{a2}
  \seeref{a3}
  \seeref{a4}
\end{EntryWithPhonetic}

\begin{EntryWithPhonetic}{哎}{ai1}{8}{⼝}[HSK 7-9]
  \definition{interj.}{Por que?; Ei!; Ai!; expressar surpresa ou insatisfação | Ei!; Cuidado!}
\end{EntryWithPhonetic}

\begin{EntryWithPhonetic}{哎呀}{ai1ya1}{8,7}{⼝、⼝}[HSK 7-9]
  \definition{interj.}{expressar surpresa ou espanto | expressar reclamação, impaciência, etc.}
\end{EntryWithPhonetic}

\begin{EntryWithPhonetic}{哀}{ai1}{9}{⼝}
  \definition*{s.}{Sobrenome Ai}
  \definition{adj.}{triste; pesaroso}
  \definition{adv.}{tristemente; lamentavelmente}
  \definition{s.}{luto | tristeza; pesar | pena; misericórdia}
  \definition{v.}{lamentar; lamentar-se por | Literário: estar triste}
\end{EntryWithPhonetic}

\begin{EntryWithPhonetic}{哀求}{ai1qiu2}{9,7}{⼝、⽔}[HSK 7-9]
  \definition{v.}{suplicar; implorar | suplicar; implorar piedosamente}
\end{EntryWithPhonetic}

\begin{EntryWithPhonetic}{唉}{ai1}{10}{⼝}
  \definition{interj.}{Sim!; Certo!; Bem! | Ai de mim!; o som dos suspiros}
  \seeref{ai4}
\end{EntryWithPhonetic}

\begin{EntryWithPhonetic}{挨}{ai1}{10}{⼿}
  \definition{prep.}{por turnos; em sequência; indica sequencialmente}
  \definition{v.}{estar próximo de; estar (ou chegar) perto de; abordar}
  \seeref{ai2}
\end{EntryWithPhonetic}

\begin{EntryWithPhonetic}{挨家挨户}{ai1jia1-ai1hu4}{10,10,10,4}{⼿、⼧、⼿、⼾}[HSK 7-9]
  \definition{expr.}{ir de casa em casa, de porta em porta ; um após o outro}
\end{EntryWithPhonetic}

\begin{EntryWithPhonetic}{挨着}{ai1 zhe5}{10,11}{⼿、⽬}[HSK 6]
  \definition{adv.}{ao lado de; perto de; imediatamente depois}
\end{EntryWithPhonetic}

\begin{EntryWithPhonetic}{挨}{ai2}{10}{⼿}[HSK 6]
  \definition{v.}{sofrer; suportar | arrastar-se; lutar para sobreviver (tempos difíceis); passar (tempo) com dificuldade | parar; atrasar; adiar; procrastinar}
  \seeref{ai1}
\end{EntryWithPhonetic}

\begin{EntryWithPhonetic}{挨打}{ai2/da3}{10,5}{⼿、⼿}[HSK 6]
  \definition{v.+compl.}{levar uma surra; ser atacado; ser espancado}
\end{EntryWithPhonetic}

\begin{EntryWithPhonetic}{癌}{ai2}{17}{⽧}[HSK 7-9]
  \definition{s.}{câncer; carcinoma; tumor maligno}
\end{EntryWithPhonetic}

\begin{EntryWithPhonetic}{癌症}{ai2zheng4}{17,10}{⽧、⽧}[HSK 7-9]
  \definition[种]{s.}{câncer; tumores malignos no corpo}
\end{EntryWithPhonetic}

\begin{EntryWithPhonetic}{矮}{ai3}{13}{⽮}[HSK 4]
  \definition{adj.}{baixo; baixa estatura; pequeno em altura | baixo; refere-se a grau, classificação, nível, etc.}[他比我矮。===Ele é mais baixo que eu. | 这栋楼很矮,只有三层。===Esse prédio é baixo, tem só três andares. | 她虽然矮,但是跑得很快!===Ela pode ser baixinha, mas corre muito rápido!]
\end{EntryWithPhonetic}

\begin{EntryWithPhonetic}{矮凳}{ai3deng4}{13,14}{⽮、⼏}
  \definition{s.}{banquinho baixo | banqueta}[这个矮凳是木制的,很结实。===Este banquinho é de madeira e bem resistente.]
\end{EntryWithPhonetic}

\begin{EntryWithPhonetic}{矮林}{ai3lin2}{13,8}{⽮、⽊}
  \definition{s.}{mata rasteira | bosque baixo}[这片矮林里有很多野兔和鸟类。===Neste bosque baixo há muitos coelhos selvagens e pássaros. | 山坡上长满了矮林,远看像绿色的地毯。===A encosta está coberta de mata rasteira, que de longe parece um tapete verde.]
\end{EntryWithPhonetic}

\begin{EntryWithPhonetic}{矮胖}{ai3pang4}{13,9}{⽮、⾁}
  \definition{adj.}{atarracado; gorducho; rechonchudo; roliço; baixo e robusto | chamar alguém diretamente de 矮胖 pode ser ofensivo}[我家猫矮胖矮胖的,像个毛球。===Meu gato é baixinho e gordinho, parece uma bolinha de pelo.]
\end{EntryWithPhonetic}

\begin{EntryWithPhonetic}{矮人}{ai3ren2}{13,2}{⽮、⼈}
  \definition{s.}{anão; pessoa de baixa estatura (indivíduo) | homúnculo; figuras criadas artificialmente pelos alquimistas em frascos de destilação | nanismo}[他虽然是矮人,但很有力气。===Embora ele seja baixo, é muito forte. | 北欧神话中的矮人是技艺高超的工匠。===Na mitologia nórdica, os anões são artesãos habilidosos. | 他因为身高被嘲笑为‘矮人’,这让他很伤心。===Ele foi zombado por ser chamado de ‘anão’ devido à sua altura, o que o magoou.]
\end{EntryWithPhonetic}

\begin{EntryWithPhonetic}{矮树}{ai3shu4}{13,9}{⽮、⽊}
  \definition[棵]{s.}{arbusto | árvore pequena, baixa}[矮树比高树更容易修剪。===Árvores baixas são mais fáceis de podar do que árvores altas. | 我们种了些矮树作为花园的边界。===Plantamos alguns arbustos como cerca natural do jardim.]
\end{EntryWithPhonetic}

\begin{EntryWithPhonetic}{矮小}{ai3 xiao3}{13,3}{⽮、⼩}[HSK 4]
  \definition{adj.}{subdimensionado; curto e pequeno; baixo e pequeno | quando usado para pessoas, pode soar depreciativo se não for em contexto neutro ou afetuoso}[这位矮小的老人是村里的智者。===Este idoso baixinho é o sábio da vila. | 这种矮小的灌木适合盆栽。===Este tipo de arbusto pequeno é ideal para vasos. | 山脚下有一片矮小的房屋,显得格外宁静。===Ao pé da montanha, havia casas baixas que transmitiam uma tranquilidade única.]
\end{EntryWithPhonetic}

\begin{EntryWithPhonetic}{矮星}{ai3xing1}{13,9}{⽮、⽇}
  \definition{s.}{estrela anã}[白矮星是恒星演化的最终阶段之一。===Anãs brancas são um dos estágios finais da evolução estelar.]
\end{EntryWithPhonetic}

\begin{EntryWithPhonetic}{矮子}{ai3zi5}{13,3}{⽮、⼦}
  \definition{s.}{pessoa baixa; anão; baixinho}[白雪公主和七个小矮子住在森林里。===Branca de Neve e os sete anões vivem na floresta. | 用`矮子'称呼他人是不礼貌的。===Chamar alguém de `baixinho' é falta de educação.]
\end{EntryWithPhonetic}

\begin{EntryWithPhonetic}{艾}{ai4}{5}{⾋}
  \definition*{s.}{Botânica: Artemísia chinesa (Artemisia argyi)}
  \definition*{s.}{Sobrenome Ai}
  \definition{adj.}{Literário: gracioso; bonito; lindo}
  \definition{s.}{artemísia; absinto; artemísia chinesa}
  \definition{v.}{Literário: parar; terminar}
  \seeref{yi4}
\end{EntryWithPhonetic}

\begin{EntryWithPhonetic}{艾滋病}{ai4zi1bing4}{5,12,10}{⾋、⽔、⽧}[HSK 7-9]
  \definition*{s.}{Síndrome da Imunodeficiência Adquirida (AIDS)}
\end{EntryWithPhonetic}

\begin{EntryWithPhonetic}{唉}{ai4}{10}{⼝}[HSK 7-9]
  \definition{interj.}{Oh!; Ah!; Bem!; interjeição que expressa tristeza ou arrependimento | Bem!; Argh!; usado para responder ou reconhecer}
  \seeref{ai1}
\end{EntryWithPhonetic}

\begin{EntryWithPhonetic}{爱}{ai4}{10}{⽖}[HSK 1]
  \definition*{s.}{Sobrenome Ai}
  \definition[个]{s.}{amor; afeição profunda; preocupação profunda; especialmente amor entre pessoas}[爱是理解和包容。===O amor é compreensão e tolerância.]
  \definition{v.}{amar; ter sentimentos profundos por pessoas ou coisas | gostar; gostar de; estar interessado em |  cuidar; valorizar; ter em alta estima; cuidar bem de | estar apto a; ter o hábito de}[他们深深爱着对方。===Eles se amam profundamente. | 我爱我的家人。===Eu amo minha família. | 我爱旅行。===Eu adoro viajar.]
\end{EntryWithPhonetic}

\begin{EntryWithPhonetic}{爱爱}{ai4'ai5}{10,10}{⽖、⽖}
  \definition{v.}{coloquial: fazer amor ou relações íntimas | pode ser usado como um apelido entre casais, transmitindo ternura | pode soar vulgar se usado em contextos inadequados}[他们俩刚结婚,天天都想爱爱。===Eles acabaram de se casar e querem fazer amor todo dia. | 爱爱,你今天好漂亮!===Amor, você está linda hoje!]
\end{EntryWithPhonetic}

\begin{EntryWithPhonetic}{爱不释手}{ai4bu2shi4shou3}{10,4,12,4}{⽖、⼀、⾤、⼿}[HSK 7-9]
  \definition{expr.}{``Não consigo parar de ler.''; ``Amo tanto que não consigo deixar passar.''; gostar (amar) algo tanto que não se pode suportar separar-se dele}
\end{EntryWithPhonetic}

\begin{EntryWithPhonetic}{爱抚}{ai4fu3}{10,7}{⽖、⼿}
  \definition{s.}{carinho; carícia}
  \definition{v.}{acariciar; afagar; cuidar (com ternura)}[他轻轻爱抚她的头发。===Ele afagou suavemente o cabelo dela. | 母亲爱抚婴儿的脸颊。===A mãe acaricia a bochecha do bebê. | 她爱抚着小猫的耳朵。===Ela acariciou as orelhas do gatinho.]
\end{EntryWithPhonetic}

\begin{EntryWithPhonetic}{爱国}{ai4 guo2}{10,8}{⽖、⼞}[HSK 4]
  \definition{adj.}{patriótico; patriotismo}[爱国是每个公民的责任。===O patriotismo é o dever de todo cidadão. | 这部电影讲述了英雄的爱国故事。===Este filme conta a história patriótica de um herói.]
  \definition{v.}{ser patriota; amar o seu país}
\end{EntryWithPhonetic}

\begin{EntryWithPhonetic}{爱好}{ai4 hao4}{10,6}{⽖、⼥}[HSK 1]
  \definition[个,种]{s.}{passatempo; interesse; \emph{hobby}; sentimentos de interesse especial ou afeição por algo | 爱好 é mais usado para atividades regulares (esportes, música), enquanto 喜欢 é para preferências gerais}[他的爱好是收集邮票。===Seu hobby era colecionar selos.  | 我的爱好是读书和旅行。===Meus hobbies são ler e viajar.]
  \definition{v.}{estar interessado em; ter prazer em; ter um forte interesse em algo; ter sentimentos profundos por alguém ou algo}
  \seealsoref{喜欢}{xi3huan5}
\end{EntryWithPhonetic}

\begin{EntryWithPhonetic}{爱好者}{ai4 hao4 zhe3}{10,6,8}{⽖、⼥、⽼}
  \definition{s.}{hobbista; amador; entusiasta; fã; amante (de arte, esportes, etc.)}[他是一位摄影爱好者。===Ele é um entusiasta de fotografia. | 她是位潜水爱好者,经常去东南亚潜水。===Ela é uma mergulhadora amadora e frequentemente mergulha no Sudeste Asiático.  | 我们为书法爱好者创建了一个微信群。===Criamos um grupo no WeChat para amantes de caligrafia.]
\end{EntryWithPhonetic}

\begin{EntryWithPhonetic}{爱护}{ai4hu4}{10,7}{⽖、⼿}[HSK 4]
  \definition{v.}{acalentar; valorizar; salvaguardar; cuidar bem de}[全社会都应爱护老年人。===Toda a sociedade deve tratar os idosos com cuidado e respeito. | 请爱护公园里的小动物。===Por favor, tratem os animais do parque com cuidado.]
\end{EntryWithPhonetic}

\begin{EntryWithPhonetic}{爱理不理}{ai4li3-bu4li3}{10,11,4,11}{⽖、⽟、⼀、⽟}[HSK 7-9]
  \definition{expr.}{frio e indiferente; distante}
\end{EntryWithPhonetic}

\begin{EntryWithPhonetic}{爱面子}{ai4/mian4zi5}{10,9,3}{⽖、⾯、⼦}[HSK 7-9]
  \definition{v.+compl.}{estar preocupado em salvar a face; ser vigilante em relação à reputação; ser sensível ao próprio orgulho; valorizar minha própria dignidade e ter medo que os outros me desprezem}[他爱面子,怕别人笑话他。===Ele se importa com sua reputação e tem medo que os outros riam dele.]
\end{EntryWithPhonetic}

\begin{EntryWithPhonetic}{爱情}{ai4qing2}{10,11}{⽖、⼼}[HSK 2]
  \definition{s.}{amor (entre pessoas); afeição}[爱情是盲目的。===O amor é cego. | 爱情如同玫瑰,美丽却带刺。===O amor é como uma rosa, bela mas com espinhos.  | 这首歌讲述了破碎的爱情故事。===Esta música conta uma história de amor fracassado.]
\end{EntryWithPhonetic}

\begin{EntryWithPhonetic}{爱人}{ai4 ren5}{10,2}{⽖、⼈}[HSK 2]
  \definition[个]{s.}{amante; \emph{dollbaby}; namorado(a) | marido ou esposa; mais usado em ocasiões formais}[这是我的爱人。===Este é o meu/minha esposo/companheiro. | 她是我一生的爱人。===Ela é o amor da minha vida. | 请携带爱人出席晚宴。===Por favor, traga seu cônjuge para o jantar.]
\end{EntryWithPhonetic}

\begin{EntryWithPhonetic}{爱上}{ai4shang4}{10,3}{⽖、⼀}
  \definition{v.}{perder o coração por; apaixonar-se por}[他在旅行时爱上了一位法国女孩。===Ele se apaixonou por uma garota francesa durante a viagem.  | 来到杭州后,我爱上了龙井茶。===Depois de chegar em Hangzhou, me apaixonei pelo chá Longjing. | 我从来没想过自己会爱上健身。===Eu nunca imaginei que iria me apaixonar por academia.]
\end{EntryWithPhonetic}

\begin{EntryWithPhonetic}{爱惜}{ai4xi1}{10,11}{⽖、⼼}[HSK 7-9]
  \definition{v.}{valorizar; prezar; estimar; usar com moderação; não desperdiçar}
\end{EntryWithPhonetic}

\begin{EntryWithPhonetic}{爱心}{ai4xin1}{10,4}{⽖、⼼}[HSK 3]
  \definition[片]{s.}{amor; carinho; compaixão; um sentimento de preocupação e carinho por outras pessoas ou animais}
\end{EntryWithPhonetic}

\begin{EntryWithPhonetic}{碍}{ai4}{13}{⽯}
  \definition{v.}{atrapalhar; dificultar; obstruir; estar no caminho de | levar em consideração}
\end{EntryWithPhonetic}

\begin{EntryWithPhonetic}{碍事}{ai4/shi4}{13,8}{⽯、⼅}[HSK 7-9]
  \definition{s.}{importa; tem consequências | (usualmente em frases negativas) sem consequência, não importa}[这决定不碍什么事。===Esta decisão não importa.]
  \definition{v.+compl.}{ser um obstáculo; estar no caminho; manter-se sob os pés de alguém; afetar o trabalho; causar inconveniência}
\end{EntryWithPhonetic}

\begin{EntryWithPhonetic}{厂}{an1}{2}{⼚}[Kangxi 27]
  \definition{s.}{usado principalmente em nomes pessoais}[他名中有个厂字。===O nome dele contém a palavra `An'.]
  \seeref{chang3}
  \seeref{han3}
\end{EntryWithPhonetic}

\begin{EntryWithPhonetic}{广}{an1}{3}{⼴}[Kangxi 53]
  \definition{s.}{mais comum em nomes de pessoas; o mesmo que 庵}[广安是我的朋友。===An'an é meu amigo.]
  \seeref{guang3}
  \seeref{yan3}
  \seealsoref{庵}{an1}
\end{EntryWithPhonetic}

\begin{EntryWithPhonetic}{安}{an1}{6}{⼧}[HSK 4]
  \definition*{s.}{Sobrenome An}
  \definition{adj.}{pacífico; quieto; tranquilo; calmo | seguro; protegido (oposto a 危) | com boa saúde | em paz; bem}
  \definition{adv.}{pacificamente; silenciosamente | com segurança; em segurança | em perguntas retóricas: como?}
  \definition{pron.}{usado como pronome interrogativo, como em 哪里,怎么; 谁,何,如何}
  \definition{s.}{segurança; proteção; paz | ampère; abreviação de ampère, 安培}
  \definition{v.}{tranquilizar (a mente de alguém); acalmar | contentar-se; ficar satisfeito | colocar em uma posição adequada; encontrar um lugar para | instalar; consertar; encaixar; configurar | trazer (uma acusação contra alguém); dar (a alguém um apelido); reivindicar (crédito por algo) | abrigar (uma intenção) | acalmar; estabilizar | sentir-se satisfeito e à vontade}
  \seealsoref{安培}{an1pei2}
  \seealsoref{何}{he2}
  \seealsoref{哪里}{na3 li3}
  \seealsoref{如何}{ru2he2}
  \seealsoref{谁}{shei2}
  \seealsoref{危}{wei1}
  \seealsoref{怎么}{zen3me5}
\end{EntryWithPhonetic}

\begin{EntryWithPhonetic}{安定}{an1ding4}{6,8}{⼧、⼧}[HSK 7-9]
  \definition{adj.}{estável; tranquilo; estabelecido; pacífico e em ordem; estável e normal, sem flutuações}
  \definition{s.}{tranquilizante; medicina ocidental comumente usada, com efeitos sedativos e anticonvulsivantes}
  \definition{v.}{acalmar; estabilizar; manter}
\end{EntryWithPhonetic}

\begin{EntryWithPhonetic}{安抚}{an1fu3}{6,7}{⼧、⼿}[HSK 7-9]
  \definition{v.}{pacificar; consolar; apaziguar; tranquilizar e acalmar}
\end{EntryWithPhonetic}

\begin{EntryWithPhonetic}{安家}{an1/jia1}{6,10}{⼧、⼧}
  \definition{v.+compl.}{montar uma casa | estabelecer-se}
\end{EntryWithPhonetic}

\begin{EntryWithPhonetic}{安检}{an1 jian3}{6,11}{⼧、⽊}[HSK 6]
  \definition{s.}{verificação de segurança}
  \definition{v.}{realizar verificação de segurança}
\end{EntryWithPhonetic}

\begin{EntryWithPhonetic}{安静}{an1jing4}{6,14}{⼧、⾭}[HSK 2]
  \definition{adj.}{silencioso; tranquilo; sem som; sem barulho e sem algazarra}
\end{EntryWithPhonetic}

\begin{EntryWithPhonetic}{安眠药}{an1mian2yao4}{6,10,9}{⼧、⽬、⾋}[HSK 7-9]
  \definition[片,颗,粒,瓶]{s.}{comprimido para dormir; soporífero; pílula para dormir; medicamentos que podem suprimir o córtex cerebral e induzir o sono}
\end{EntryWithPhonetic}

\begin{EntryWithPhonetic}{安宁}{an1ning2}{6,5}{⼧、⼧}[HSK 7-9]
  \definition{adj.}{pacífico; tranquilo; descreve um estado de ordem normal sem fatores externos que causem desordem ou inquietação | calmo; composto; livre de preocupações; sem preocupações, ansiedades ou inquietações}
\end{EntryWithPhonetic}

\begin{EntryWithPhonetic}{安排}{an1pai2}{6,11}{⼧、⼿}[HSK 3]
  \definition{s.}{plano; programação; organização; tabela do plano de atividades ou horários}
  \definition{v.}{organizar (assuntos) de acordo com a sequência ou regras; tratar as coisas de acordo com uma determinada ordem ou regras | atribuir tarefas a alguém; colocar as pessoas nos cargos de trabalho determinados, conforme planejado}
\end{EntryWithPhonetic}

\begin{EntryWithPhonetic}{安培}{an1pei2}{6,11}{⼧、⼟}
  \definition{clas.}{A; empréstimo linguístico: ampere; física: unidade de corrente elétrica}
\end{EntryWithPhonetic}

\begin{EntryWithPhonetic}{安全}{an1quan2}{6,6}{⼧、⼊}[HSK 2]
  \definition{adj.}{seguro; protegido; sem perigo; sem ameaças; sem acidentes}
  \definition{s.}{segurança; proteção; refere-se a um estado ou conceito, geralmente indicando ausência de ameaças ou perigo}
\end{EntryWithPhonetic}

\begin{EntryWithPhonetic}{安神}{an1/shen2}{6,9}{⼧、⽰}
  \definition{v.+compl.}{acalmar os nervos | aliviar a inquietação pela tranquilização da mente e do corpo}
\end{EntryWithPhonetic}

\begin{EntryWithPhonetic}{安慰}{an1wei4}{6,15}{⼧、⼼}[HSK 5]
  \definition{adj.}{confortar; tranquilizar; consolar; apaziguar;}
  \definition[句,通,番,声,个]{s.}{conforto; consolo; comportamento que alivia a dor de alguém e o acalma com palavras ou gestos}
  \definition{v.}{confortar; consolar; acalmar e confortar; deixar a mente tranquila}
\end{EntryWithPhonetic}

\begin{EntryWithPhonetic}{安稳}{an1wen3}{6,14}{⼧、⽲}[HSK 7-9]
  \definition{adj.}{seguro; suave e estável; estável | composto; calmo e equilibrado; (comportamento) estável; calmo}
\end{EntryWithPhonetic}

\begin{EntryWithPhonetic}{安心}{an1xin1}{6,4}{⼧、⼼}[HSK 7-9]
  \definition{adj.}{aliviado; à vontade; tranquilo; seguro}
  \definition{v.}{abrigar (más) intenções; acalentar certas intenções; ter (pensamentos ruins) em mente}
\end{EntryWithPhonetic}

\begin{EntryWithPhonetic}{安逸}{an1yi4}{6,11}{⼧、⾡}[HSK 7-9]
  \definition{adj.}{fácil; fácil e confortável; relaxado e confortável}
  \definition{s.}{conforto e facilidade; conforto}
\end{EntryWithPhonetic}

\begin{EntryWithPhonetic}{安置}{an1zhi4}{6,13}{⼧、⽹}[HSK 4]
  \definition{v.}{providenciar; encontrar um lugar para; ajudar a estabelecer-se; colocar pessoas ou coisas em uma determinada posição ou organizá-las adequadamente}
\end{EntryWithPhonetic}

\begin{EntryWithPhonetic}{安装}{an1zhuang1}{6,12}{⼧、⾐}[HSK 3]
  \definition{v.}{instalar; consertar; configurar; fixar máquinas ou equipamentos (geralmente conjuntos) em um determinado local, de acordo com métodos e especificações específicos}
\end{EntryWithPhonetic}

\begin{EntryWithPhonetic}{庵}{an1}{11}{⼴}
  \definition*{s.}{Sobrenome An}
  \definition[个,座]{s.}{cabana | convento de freiras; templos budistas, principalmente onde vivem as freiras}
\end{EntryWithPhonetic}

\begin{EntryWithPhonetic}{岸}{an4}{8}{⼭}[HSK 5]
  \definition{adj.}{arrogante; orgulhoso; grandioso (de maneira sombria ou condescendente)}
  \definition[条,道,段,面]{s.}{margem; costa; litoral; terreno à beira da água}
\end{EntryWithPhonetic}

\begin{EntryWithPhonetic}{岸上}{an4 shang4}{8,3}{⼭、⼀}[HSK 5]
  \definition{s.}{em terra; costa; margem | na margem do rio; na beira do rio}
\end{EntryWithPhonetic}

\begin{EntryWithPhonetic}{按}{an4}{9}{⼿}[HSK 3]
  \definition{prep.}{de acordo com; à luz de; com base em; em conformidade com}
  \definition{v.}{pressionar; empurrar para baixo; pressionar ou apertar com a mão ou os dedos | pôr de parte; deixar de lado; deixar para mais tarde | restringir; controlar; inibir; parar | verificar; consultar | comentar ou anotar (por um editor ou autor)}
\end{EntryWithPhonetic}

\begin{EntryWithPhonetic}{按键}{an4jian4}{9,13}{⼿、⾦}[HSK 7-9]
  \definition[个]{s.}{tecla; botão; teclas pressionadas manualmente}[键盘上的按键非常灵敏。===As teclas do teclado são muito responsivas.]
\end{EntryWithPhonetic}

\begin{EntryWithPhonetic}{按理}{an4li3}{9,11}{⼿、⽟}
  \definition{adv.}{de acordo com o princípio ou a razão; no curso normal dos eventos; normalmente | de acordo com a razão; de acordo com a prática comum; por (bons) direitos}
\end{EntryWithPhonetic}

\begin{EntryWithPhonetic}{按理说}{an4li3 shuo1}{9,11,9}{⼿、⽟、⾔}[HSK 7-9]
  \definition{adv.}{de acordo com o princípio ou a razão; no curso normal dos eventos; normalmente | é razoável dizer que\dots}
\end{EntryWithPhonetic}

\begin{EntryWithPhonetic}{按摩}{an4mo2}{9,15}{⼿、⼿}[HSK 5]
  \definition{s.}{massagem; empurrar, pressionar, beliscar e amassar o corpo de uma pessoa com as mãos para promover a circulação sanguínea, aumentar a resistência da pele e regular a função dos nervos}
\end{EntryWithPhonetic}

\begin{EntryWithPhonetic}{按时}{an4shi2}{9,7}{⼿、⽇}[HSK 4]
  \definition{adv.}{na hora; no horário; pontualmente; de acordo com o tempo estipulado}
\end{EntryWithPhonetic}

\begin{EntryWithPhonetic}{按说}{an4shuo1}{9,9}{⼿、⾔}[HSK 7-9]
  \definition{adv.}{no curso normal dos eventos; ordinariamente; normalmente | de acordo com o fato (senso comum); refere-se a falar de acordo com fatos ou senso comum; como uma questão de razão; expressões semelhantes incluem 按理 e 按理说}
  \seealsoref{按理}{an4li3}
  \seealsoref{按理说}{an4li3 shuo1}
\end{EntryWithPhonetic}

\begin{EntryWithPhonetic}{按照}{an4zhao4}{9,13}{⼿、⽕}[HSK 3]
  \definition{prep.}{de acordo com; em conformidade com; à luz de; com base em; apresentar os fundamentos ou critérios de julgamento para fazer algo}
\end{EntryWithPhonetic}

\begin{EntryWithPhonetic}{案}{an4}{10}{⽊}
  \definition{s.}{mesa; escrivaninha; mesa longa | caso; caso de direito (legal) | registro; arquivo; arquivo de caso | um plano submetido para consideração; proposta; um documento que propõe planos, sugestões, métodos, etc.}
\end{EntryWithPhonetic}

\begin{EntryWithPhonetic}{案件}{an4jian4}{10,6}{⽊、⼈}[HSK 7-9]
  \definition[个,起,件,类]{s.}{caso; caso de direito; caso legal; contencioso e eventos ilegais}
\end{EntryWithPhonetic}

\begin{EntryWithPhonetic}{暗}{an4}{13}{⽇}[HSK 4]
  \definition{adj.}{escuro; opaco; sem graça; pouca luz | escondido; secreto; não revelado | pouco claro; nebuloso; vago; confuso | subterrâneo}
  \definition{adv.}{secretamente | no escuro}
\end{EntryWithPhonetic}

\begin{EntryWithPhonetic}{暗地里}{an4di4li3}{13,6,7}{⽇、⼟、⾥}[HSK 7-9]
  \definition{adv.}{secretamente; interiormente; às escondidas}
\end{EntryWithPhonetic}

\begin{EntryWithPhonetic}{暗恋}{an4lian4}{13,10}{⽇、⼼}
  \definition{s.}{amor secreto}
  \definition{v.}{estar secretamente apaixonado por}
\end{EntryWithPhonetic}

\begin{EntryWithPhonetic}{暗杀}{an4sha1}{13,6}{⽇、⽊}[HSK 7-9]
  \definition{s.}{assassinato}
  \definition{v.}{assassinar | matar secretamente}
\end{EntryWithPhonetic}

\begin{EntryWithPhonetic}{暗示}{an4shi4}{13,5}{⽇、⽰}[HSK 4]
  \definition[种]{s.}{sugestão; insinuação; intimação; (psicologia) refere-se ao uso de palavras, gestos, expressões, etc. para fazer as pessoas aceitarem involuntariamente uma determinada opinião ou fazerem algo}
  \definition{v.}{dar uma dica; sugerir secretamente; indicar algo a alguém usando outras palavras, expressões faciais ou gestos sem dizer em voz alta}
\end{EntryWithPhonetic}

\begin{EntryWithPhonetic}{暗香}{an4xiang1}{13,9}{⽇、⾹}
  \definition{s.}{fragrância sutil}
\end{EntryWithPhonetic}

\begin{EntryWithPhonetic}{暗中}{an4zhong1}{13,4}{⽇、⼁}[HSK 7-9]
  \definition{adv.}{no escuro; na escuridão | em segredo; às escondidas; sorrateiramente | furtivamente; nos bastidores}
\end{EntryWithPhonetic}

\begin{EntryWithPhonetic}{昂}{ang2}{8}{⽇}
  \definition{adj.}{alto; subindo}
  \definition{v.}{manter (a cabeça) erguida | elevar; levantar; olhar para cima}
\end{EntryWithPhonetic}

\begin{EntryWithPhonetic}{昂贵}{ang2gui4}{8,9}{⽇、⾙}[HSK 7-9]
  \definition{adj.}{caro; dispendioso; algo é muito caro, o preço é particularmente alto; metaforicamente, o custo de fazer algo é particularmente alto}
\end{EntryWithPhonetic}

\begin{EntryWithPhonetic}{凹}{ao1}{5}{⼐}[HSK 7-9]
  \definition{adj.}{afundado; amassado (oposto a 凸) | côncavo; oco; amassado; mais baixo que a área circundante}
  \definition{v.}{cavar; chanfrar | desabar; afundar}
  \seeref{wa1}
  \seealsoref{凸}{tu1}
\end{EntryWithPhonetic}

\begin{EntryWithPhonetic}{熬}{ao1}{14}{⽕}
  \definition{v.}{ensopar; ferver; cozinhar em água}
  \seeref{ao2}
\end{EntryWithPhonetic}

\begin{EntryWithPhonetic}{熬}{ao2}{14}{⽕}[HSK 7-9]
  \definition{v.}{ferver; ensopar; fazer uma decocção; cozinhar em fogo baixo por muito tempo | preparar; infundir; extrair a essência por fervura longa | resistir; suportar (angústia, tempos difíceis, etc.)}
  \seeref{ao1}
\end{EntryWithPhonetic}

\begin{EntryWithPhonetic}{熬夜}{ao2/ye4}{14,8}{⽕、⼣}[HSK 7-9]
  \definition{v.+compl.}{ficar acordado a noite toda ou até tarde da noite}
\end{EntryWithPhonetic}

\begin{EntryWithPhonetic}{傲}{ao4}{12}{⼈}[HSK 7-9]
  \definition{adj.}{orgulhoso; altivo | arrogante}
  \definition{v.}{recusar-se a ceder; desafiar}
\end{EntryWithPhonetic}

\begin{EntryWithPhonetic}{傲慢}{ao4man4}{12,14}{⼈、⼼}[HSK 7-9]
  \definition{adj.}{altivo; arrogante; autoritário}
\end{EntryWithPhonetic}

\begin{EntryWithPhonetic}{奥}{ao4}{12}{⼤}
  \definition*{s.}{Oersted, a unidade eletromagnética de intensidade do campo magnético; abreviação de 奥斯特 | Sobrenome Ao}
  \definition{adj.}{profundo e difícil de entender; abstruso | significado profundo, não é fácil de entender}
  \definition{s.}{canto secreto da casa; antigamente, referia-se ao canto sudoeste de uma casa e também, de modo geral, à profundidade de uma casa}
  \seealsoref{奥斯特}{ao4 si1 te4}
\end{EntryWithPhonetic}

\begin{EntryWithPhonetic}{奥林匹克运动会}{ao4lin2pi3ke4 yun4dong4hui4}{12,8,4,7,7,6,6}{⼤、⽊、⼖、⼗、⾡、⼒、⼈}
  \definition*{s.}{Jogos Olímpicos, Olimpíadas}
\end{EntryWithPhonetic}

\begin{EntryWithPhonetic}{奥秘}{ao4mi4}{12,10}{⼤、⽲}[HSK 7-9]
  \definition[个]{s.}{enigma; mistério profundo; fenômenos ou princípios profundos e misteriosos}
\end{EntryWithPhonetic}

\begin{EntryWithPhonetic}{奥斯特}{ao4 si1 te4}{12,12,10}{⼤、⽄、⽜}
  \definition{s.}{Oersted}
\end{EntryWithPhonetic}

\begin{EntryWithPhonetic}{奥特曼}{ao4te4man4}{12,10,11}{⼤、⽜、⽈}
  \definition*{s.}{Ultraman,  super-herói de ficção científica japonesa}
\end{EntryWithPhonetic}

\begin{EntryWithPhonetic}{奥运}{ao4yun4}{12,7}{⼤、⾡}
  \definition*{s.}{Jogos Olímpicos, Olimpíadas; Abreviação de 奥林匹克运动会}
  \seealsoref{奥林匹克运动会}{ao4lin2pi3ke4 yun4dong4hui4}
\end{EntryWithPhonetic}

\begin{EntryWithPhonetic}{奥运会}{ao4yun4hui4}{12,7,6}{⼤、⾡、⼈}[HSK 7-9]
  \definition*[届,次]{s.}{Jogos Olímpicos, Olimpíadas; Abreviação de 奥林匹克运动会}
  \seealsoref{奥林匹克运动会}{ao4lin2pi3ke4 yun4dong4hui4}
\end{EntryWithPhonetic}

\begin{EntryWithPhonetic}{澳}{ao4}{15}{⽔}
  \definition*{s.}{Abreviação de Austrália, 澳大利亚 | Sobrenome Ao}
  \definition{s.}{baía; uma entrada do mar; um lugar curvo na costa onde os barcos podem ser atracados, frequentemente usado em nomes de lugares}
  \seealsoref{澳大利亚}{ao4da4li4ya4}
\end{EntryWithPhonetic}

\begin{EntryWithPhonetic}{澳大利亚}{ao4da4li4ya4}{15,3,7,6}{⽔、⼤、⼑、⼆}
  \definition*{s.}{Austrália}
\end{EntryWithPhonetic}

%%%%% EOF %%%%%


 %%%
%%% B
%%%

\section*{B}\addcontentsline{toc}{section}{B}

\begin{EntryWithPhonetic}{八}{ba1}{2}{⼋}[HSK 1][Kangxi 12]
  \definition{num.}{oito; 8}
\end{EntryWithPhonetic}

\begin{EntryWithPhonetic}{八八六}{ba1 ba1 liu4}{2,2,4}{⼋、⼋、⼋}
  \definition{expr.}{\emph{Bye bye!}, em salas de bate-papo e mensagens de texto}
\end{EntryWithPhonetic}

\begin{EntryWithPhonetic}{巴}{ba1}{4}{⼰}
  \definition*{s.}{Ba, um estado da Dinastia Zhou | Nome antigo de Sichuan Oriental | Sobrenome Ba}
  \definition{s.}{píton (cobra) | crosta; formação semelhante a uma crosta | parte de órgãos biológicos que denotam ponta, extremidade, cauda | bar, unidade física de pressão, 1 atm = 1,01325 bar}
  \definition{v.}{esperar sinceramente; esperar ansiosamente por; aguardar ansiosamente | agarrar-se a; grudar em; manter-se fiel a | escalar física e socialmente | estar perto de; estar ao lado de | abrir; dividir; rachar; quebrar}
\end{EntryWithPhonetic}

\begin{EntryWithPhonetic}{巴勒斯坦}{ba1le4si1tan3}{4,11,12,8}{⼰、⼒、⽄、⼟}
  \definition*{s.}{Palestina}
\end{EntryWithPhonetic}

\begin{EntryWithPhonetic}{巴士}{ba1 shi4}{4,3}{⼰、⼠}[HSK 4]
  \definition[辆]{s.}{ônibus, transliteração da palavra inglesa \emph{bus}}
\end{EntryWithPhonetic}

\begin{EntryWithPhonetic}{巴西}{ba1xi1}{4,6}{⼰、⾑}
  \definition*{s.}{Brasil}
\end{EntryWithPhonetic}

\begin{EntryWithPhonetic}{巴西人}{ba1xi1 ren2}{4,6,2}{⼰、⾑、⼈}
  \definition[个,位]{s.}{brasileiro | pessoa ou povo do Brasil}[他是巴西人。===Ele é brasileiro.]
\end{EntryWithPhonetic}

\begin{EntryWithPhonetic}{巴西战舞}{ba1xi1zhan4wu3}{4,6,9,14}{⼰、⾑、⼽、⾇}
  \definition{s.}{capoeira}
\end{EntryWithPhonetic}

\begin{EntryWithPhonetic}{扒}{ba1}{5}{⼿}
  \definition{v.}{segurar; agarrar-se a | cavar; varrer; puxar para baixo | empurrar para o lado | despir-se; tirar}
  \seeref{扒}{pa2}
\end{EntryWithPhonetic}

\begin{EntryWithPhonetic}{吧}{ba1}{7}{⼝}
  \definition{s.}{som de estalo, som crepitante |  abreviação de bar, 酒吧 | cibercafé; um local público que fornece computadores e serviços de \emph{Internet} onde as pessoas podem navegar, jogar, etc.}
  \definition{v.}{fumar; dar uma tragada (puxar) no cachimbo}
  \seeref{吧}{ba5}
  \seealsoref{酒吧}{jiu3ba1}
\end{EntryWithPhonetic}

\begin{EntryWithPhonetic}{拔}{ba2}{8}{⼿}[HSK 5]
  \definition{v.aux.}{puxar para cima; puxar para fora; arrastar para fora | extrair; sugar | escolher; selecionar | superar; destacar-se entre | apreender; capturar | esfriar na água; mergulhar algo em água fria para que esfrie}
\end{EntryWithPhonetic}

\begin{EntryWithPhonetic}{拔尖}{ba2jian1}{8,6}{⼿、⼩}
  \definition{adj.}{topo de linha | fora do comum | o melhor}
  \definition{v.+compl.}{empurrar-se para a frente | sentir que é superior aos outros}
\end{EntryWithPhonetic}

\begin{EntryWithPhonetic}{把}{ba3}{7}{⼿}[HSK 3]
  \definition{adj.}{referindo-se à relação de irmandade}
  \definition{clas.}{usado antes de objetos com alças ou coisas para segurar | um punhado de; a quantidade que se pode pegar com uma mão | usado antes de coisas abstratas | usado em coisas feitas com as mãos | número de ações, coisas}
  \definition{part.}{adicionado após quantificadores como 百, 千, 万 e 里, 斤, 个, indica que a quantidade é próxima dessa unidade (não pode ser adicionado outro quantificador antes)}
  \definition{prep.}{fazer uma determinada alteração em um objeto; causar uma determinada mudança em um objeto | fazer com que os outros façam/sintam algo}
  \definition{s.}{alça; punho; a parte que se segura | feixe; molho; algo que se segura com as mãos ou se amarra em pequenos feixes}
  \definition{v.}{agarrar; segurar | segurar (um bebê enquanto ele urina) | controlar; dominar; monopolizar | encostar-se; apoiar-se | vigiar (locais importantes); observar; guardar | dar | usar algo como; considerar como; tratar como; conter o significado de 拿 | acorrentar; trancar}
  \seeref{把}{ba4}
  \seealsoref{百}{bai3}
  \seealsoref{个}{ge4}
  \seealsoref{斤}{jin1}
  \seealsoref{里}{li3}
  \seealsoref{拿}{na2}
  \seealsoref{千}{qian1}
  \seealsoref{万}{wan4}
\end{EntryWithPhonetic}

\begin{EntryWithPhonetic}{把柄}{ba3bing3}{7,9}{⼿、⽊}
  \definition[个]{s.}{alça; a parte de um objeto que é fácil de segurar com as mãos | evidências que podem ser obtidas em ações judiciais ou argumentos; uma metáfora para um erro ou falha que pode ser usada para chantagear alguém}
\end{EntryWithPhonetic}

\begin{EntryWithPhonetic}{把持}{ba3chi2}{7,9}{⼿、⼿}
  \definition{v.}{(frequentemente pejorativo) dominar; monopolizar | controlar (os próprios sentimentos, etc.) | manter sob controle}
\end{EntryWithPhonetic}

\begin{EntryWithPhonetic}{把风}{ba3feng1}{7,4}{⼿、⾵}
  \definition{v.}{vigiar (em uma atividade clandestina) | estar atento}
\end{EntryWithPhonetic}

\begin{EntryWithPhonetic}{把关}{ba3guan1}{7,6}{⼿、⼋}
  \definition{v.}{verificar rigorosamente; examinar cuidadosamente para ver se algo está sendo feito de acordo com o padrão fixo; fazer a verificação final | proteger uma fronteira, passagem, portões, etc.}
\end{EntryWithPhonetic}

\begin{EntryWithPhonetic}{把脉}{ba3mai4}{7,9}{⼿、⾁}
  \definition{v.}{resolver problemas por meio de investigação e estudo | sentir o pulso | para tomar o pulso de alguém}
\end{EntryWithPhonetic}

\begin{EntryWithPhonetic}{把式}{ba3shi4}{7,6}{⼿、⼷}
  \definition{s.}{pessoa qualificada em um comércio}
\end{EntryWithPhonetic}

\begin{EntryWithPhonetic}{把守}{ba3shou3}{7,6}{⼿、⼧}
  \definition{v.}{vigiar | guardar}
\end{EntryWithPhonetic}

\begin{EntryWithPhonetic}{把玩}{ba3wan2}{7,8}{⼿、⽟}
  \definition{v.}{brincar com | mexer com | girar nas mãos}
\end{EntryWithPhonetic}

\begin{EntryWithPhonetic}{把稳}{ba3wen3}{7,14}{⼿、⽲}
  \definition{adj.}{confiável}
  \definition{v.}{ter certeza de; ser firme; manter-se firme}
\end{EntryWithPhonetic}

\begin{EntryWithPhonetic}{把握}{ba3wo4}{7,12}{⼿、⼿}[HSK 3]
  \definition[的]{s.}{seguro; garantia; certeza; confiabilidade do sucesso}
  \definition{v.}{agarrar; segurar; apreender |  (algo abstrato) agarrar; segurar}
\end{EntryWithPhonetic}

\begin{EntryWithPhonetic}{把戏}{ba3xi4}{7,6}{⼿、⼽}
  \definition{s.}{acrobacia | malabarismo | truque barato}
\end{EntryWithPhonetic}

\begin{EntryWithPhonetic}{把}{ba4}{7}{⼿}
  \definition{s.}{punho; alça; empunhadura; parte do utensílio que é fácil de segurar com a mão |haste (de uma folha, flor ou fruto) | motivo de ridículo; alvo; comportamentos e declarações que servem de assunto para piadas}
  \seeref{把}{ba3}
\end{EntryWithPhonetic}

\begin{EntryWithPhonetic}{爸}{ba4}{8}{⽗}[HSK 1]
  \definition[个,位]{s.}{(informal) pai}
  \seealsoref{爸爸}{ba4ba5}
\end{EntryWithPhonetic}

\begin{EntryWithPhonetic}{爸爸}{ba4ba5}{8,8}{⽗、⽗}[HSK 1]
  \definition[个,位,名,群]{s.}{(informal) pai; papai; papa}
  \seealsoref{爸}{ba4}
\end{EntryWithPhonetic}

\begin{EntryWithPhonetic}{爸妈}{ba4ma1}{8,6}{⽗、⼥}
  \definition{s.}{pai e mãe}
\end{EntryWithPhonetic}

\begin{EntryWithPhonetic}{罢}{ba4}{10}{⽹}
  \definition{v.}{parar; cessar | revogar; destituir; encerrar | terminar | abandonar uma ideia; esqueçer sobre algo; deixar estar (passar)}
  \seeref{罢}{ba5}
\end{EntryWithPhonetic}

\begin{EntryWithPhonetic}{罢工}{ba4gong1}{10,3}{⽹、⼯}[HSK 6]
  \definition{v.}{parar de trabalhar; entrar em greve; abandonar o emprego}
\end{EntryWithPhonetic}

\begin{EntryWithPhonetic}{罢了}{ba4 le5}{10,2}{⽹、⼅}[HSK 6]
  \definition{part.}{usado no final de uma frase, significa 仅此而已, geralmente seguido de 无非, 不过, 只是}
  \seeref{罢了}{ba4 liao3}
  \seealsoref{不过}{bu2guo4}
  \seealsoref{仅此而已}{jin3ci3'er2yi3}
  \seealsoref{无非}{wu2fei1}
  \seealsoref{只是}{zhi3 shi4}
\end{EntryWithPhonetic}

\begin{EntryWithPhonetic}{罢了}{ba4 liao3}{10,2}{⽹、⼅}
  \definition{part.}{uma partícula modal indicando (não se preocupe, ok)}
\end{EntryWithPhonetic}

\begin{EntryWithPhonetic}{霸}{ba4}{21}{⾬}
  \definition*{s.}{Sobrenome Ba}
  \definition{adj.}{arrogante; dominador; tirânico}
  \definition{s.}{líder dos senhores feudais; suserano | tirano; déspota; valentão; \emph{bully} | poder hegemônico; hegemonismo; hegemonia | chefe dos príncipes feudais; líder da antiga aliança feudal}
  \definition{v.}{dominar; tiranizar; governar (ocupar) pela força}
\end{EntryWithPhonetic}

\begin{EntryWithPhonetic}{霸权}{ba4quan2}{21,6}{⾬、⽊}
  \definition{s.}{hegemonia | supremacia}
\end{EntryWithPhonetic}

\begin{EntryWithPhonetic}{吧}{ba5}{7}{⼝}[HSK 1]
  \definition{part.}{indica discussão, sugestão, solicitação ou comando no final de uma frase | indica concordância ou aprovação no final de uma frase | indica uma pergunta ou especulação no final de uma frase | indica incerteza no final de uma frase | em uma frase, indica uma pausa, carrega um tom hipotético, frequentemente apresenta um contraste e implica um dilema}
  \seeref{吧}{ba1}
\end{EntryWithPhonetic}

\begin{EntryWithPhonetic}{罢}{ba5}{10}{⽹}
  \definition{part.}{partícula final, a mesma que 吧}
  \seeref{罢}{ba4}
  \seealsoref{吧}{ba5}
\end{EntryWithPhonetic}

\begin{EntryWithPhonetic}{白}{bai2}{5}{⽩}[HSK 1,3][Kangxi 106]
  \definition*{s.}{Sobrenome Bai}
  \definition{adj.}{branco | claro; entendível; compreendível | puro; claro; simples; sem mistura; em branco | branco (como símbolo de reação) | escrito incorretamente ou pronunciado incorretamente}
  \definition{adv.}{em vão; sem propósito; sem resultados | gratuito; sem custos}
  \definition{s.}{parte falada em ópera, etc.; frases de peças de teatro, etc. | dialeto local | funeral}
  \definition{v.}{explicar; apresentar; esclarecer; declarar | branquear | olhar para as pessoas com o branco dos olhos (olhar vazio, de desaprovação); olhar para alguém com desdém}
\end{EntryWithPhonetic}

\begin{EntryWithPhonetic}{白菜}{bai2 cai4}{5,11}{⽩、⾋}[HSK 3]
  \definition[棵,种]{s.}{couve chinesa | \emph{pak choi}, um tipo de couve}
\end{EntryWithPhonetic}

\begin{EntryWithPhonetic}{白痴}{bai2chi1}{5,13}{⽩、⽧}
  \definition{adj.}{idiota; uma pessoa que sofre de idiotice; frequentemente usado para menosprezar alguém que é incompetente ou incapaz de fazer as coisas}
  \definition{s.}{idiotice; uma doença caracterizada por retardo mental, demência, fala arrastada, movimentos lentos e até mesmo incapacidade de cuidar de si mesmo}
\end{EntryWithPhonetic}

\begin{EntryWithPhonetic}{白蛋白}{bai2dan4bai2}{5,11,5}{⽩、⾍、⽩}
  \definition{s.}{albumina}
\end{EntryWithPhonetic}

\begin{EntryWithPhonetic}{白鹄}{bai2hu2}{5,12}{⽩、⿃}
  \definition{s.}{cisne branco}
\end{EntryWithPhonetic}

\begin{EntryWithPhonetic}{白拣}{bai2jian3}{5,8}{⽩、⼿}
  \definition{s.}{uma escolha barata}
  \definition{v.}{escolher algo que não custa nada}
\end{EntryWithPhonetic}

\begin{EntryWithPhonetic}{白酒}{bai2 jiu3}{5,10}{⽩、⾣}[HSK 5]
  \definition[瓶,杯,壶]{s.}{aguardente branca; aguardente (geralmente destilada de sorgo ou milho); bebidas destiladas tradicionais chinesas, feitas de sorgo, milho, etc., transparentes e incolores, com alto teor alcoólico}
\end{EntryWithPhonetic}

\begin{EntryWithPhonetic}{白领}{bai2 ling3}{5,11}{⽩、⾴}[HSK 6]
  \definition[个,名,位,些]{s.}{colarinho branco; trabalhador de colarinho branco; refere-se a funcionários cujo trabalho principal envolve trabalho intelectual, são conhecidos por suas roupas elegantes, colarinhos e camisas brancas; atualmente é frequentemente usado para se referir àqueles que trabalham em cargos de gestão ou técnicos em empresas e ganham salários relativamente altos}
\end{EntryWithPhonetic}

\begin{EntryWithPhonetic}{白萝卜}{bai2luo2bo5}{5,11,2}{⽩、⾋、⼘}
  \definition{s.}{rabanete branco | \emph{daikon}}
\end{EntryWithPhonetic}

\begin{EntryWithPhonetic}{白色}{bai2 se4}{5,6}{⽩、⾊}[HSK 2]
  \definition{s.}{a cor branca}
\end{EntryWithPhonetic}

\begin{EntryWithPhonetic}{白天}{bai2 tian1}{5,4}{⽩、⼤}[HSK 1]
  \definition{adv.}{dia;  de dia}
  \definition[个]{s.}{dia; horário diurno; durante o dia}
\end{EntryWithPhonetic}

\begin{EntryWithPhonetic}{白苋}{bai2xian4}{5,7}{⽩、⾋}
  \definition{s.}{amaranto branco | brotos e folhas tenras de espinafre chinês usados como alimento}
\end{EntryWithPhonetic}

\begin{EntryWithPhonetic}{百}{bai3}{6}{⽩}[HSK 1]
  \definition{adj.}{todos; todos os tipos de; multifacetados; numerosos}
  \definition{adv.}{muito; sempre}
  \definition{num.}{cem; 100}
\end{EntryWithPhonetic}

\begin{EntryWithPhonetic}{百般}{bai3ban1}{6,10}{⽩、⾈}
  \definition{adv.}{de todas as maneiras possíveis | por todos os meios}
\end{EntryWithPhonetic}

\begin{EntryWithPhonetic}{百分}{bai3fen1}{6,4}{⽩、⼑}
  \definition{s.}{por cento | nota máxima; pontuação máxima; 100 pontos (em um sistema de classificação de cem pontos) | um jogo específico; um jogo de pôquer}
\end{EntryWithPhonetic}

\begin{EntryWithPhonetic}{百分点}{bai3 fen1 dian3}{6,4,9}{⽩、⼑、⽕}[HSK 6]
  \definition[个]{s.}{ponto percentual; em estatística, um por cento é chamado de ponto percentual}
\end{EntryWithPhonetic}

\begin{EntryWithPhonetic}{百货}{bai3 huo4}{6,8}{⽩、⾙}[HSK 4]
  \definition{s.}{mercadorias em geral; loja de departamentos; um termo geral para bens que incluem principalmente roupas, utensílios e necessidades diárias gerais}
\end{EntryWithPhonetic}

\begin{EntryWithPhonetic}{柏}{bai3}{9}{⽊}
  \seeref{柏}{bo2}
  \seeref{柏}{bo4}
  \seealsoref{柏树}{bai3shu4}
\end{EntryWithPhonetic}

\begin{EntryWithPhonetic}{柏树}{bai3shu4}{9,9}{⽊、⽊}
  \definition{s.}{cipreste}
\end{EntryWithPhonetic}

\begin{EntryWithPhonetic}{摆}{bai3}{13}{⼿}[HSK 4]
  \definition*{s.}{Festival de Ganbai; uma reunião realizada nas áreas Dai durante festivais religiosos, para celebrar uma boa colheita ou para trocar materiais; geralmente se refere a uma reunião em massa | Sobrenome Bai}
  \definition{s.}{pêndulo; dispositivo mecânico que controla a frequência de oscilação em relógios e instrumentos |  a bainha inferior de um vestido, jaqueta ou saia}
  \definition{v.}{colocar; posicionar; organizar | assumir; mostrar intencionalmente | balançar; ondular; balançar para frente e para trás | revelar; listar; afirmar claramente | dizer; falar; declarar | libertar-se; livrar-se}
\end{EntryWithPhonetic}

\begin{EntryWithPhonetic}{摆动}{bai3 dong4}{13,6}{⼿、⼒}[HSK 4]
  \definition{v.}{balançar; balançar para frente e para trás; oscilar; vibrar}
\end{EntryWithPhonetic}

\begin{EntryWithPhonetic}{摆烂}{bai3lan4}{13,9}{⼿、⽕}
  \definition{v.}{(neologismo, gíria) parar de lutar (especialmente quando se sabe que não pode ter sucesso) | deixar tudo ir para o inferno}
\end{EntryWithPhonetic}

\begin{EntryWithPhonetic}{摆手}{bai3shou3}{13,4}{⼿、⼿}
  \definition{v.+compl.}{gesticular com a mão (acenando, acenando adeus, etc.) | balançar os braços | acenar com as mãos}
\end{EntryWithPhonetic}

\begin{EntryWithPhonetic}{摆脱}{bai3tuo1}{13,11}{⼿、⾁}[HSK 4]
  \definition{v.}{sacudir; rejeitar; romper com; libertar-se (ou desembaraçar-se) de; livrar-se de dificuldades, escravidão, controle, etc.}
\end{EntryWithPhonetic}

\begin{EntryWithPhonetic}{败}{bai4}{8}{⾒}[HSK 4]
  \definition{adj.}{ruim; deteriorado; murcho; dilapidado; decadente}
  \definition{v.}{ser derrotado; perder (oposto a 胜) | derrotar; bater | falha (oposto a 成) | estragar; arruinar | decair; murchar | quebrar; neutralizar; dissipar}
  \seealsoref{成}{cheng2}
  \seealsoref{胜}{sheng4}
\end{EntryWithPhonetic}

\begin{EntryWithPhonetic}{拜}{bai4}{9}{⼿}
  \definition*{s.}{Sobrenome Bai}
  \definition{adv.}{respeitosamente (usado na comunicação interpessoal);}
  \definition{v.}{fazer uma visita de cortesia | adorar; prestar homenagem | fazer uma chamada cerimonial | ligar; fazer uma visita | intitular alguém com cerimônia; conceder uma posição oficial ou um determinado título com certa etiqueta | estabelecer ou jurar formalmente relacionamentos}
\end{EntryWithPhonetic}

\begin{EntryWithPhonetic}{拜访}{bai4fang3}{9,6}{⼿、⾔}[HSK 5]
  \definition{v.}{visitar; fazer uma visita (respeitosamente)}
\end{EntryWithPhonetic}

\begin{EntryWithPhonetic}{班}{ban1}{10}{⽟}[HSK 1]
  \definition*{s.}{Sobrenome Ban}
  \definition{adj.}{regular; programado; executado regularmente; com horários fixos (meios de transporte)}
  \definition{clas.}{um grupo de; uma classe de; usado para pessoas | meios de transporte com horários fixos}
  \definition[个]{s.}{equipe; turma; organização estruturada | dever; turno; período de trabalho dentro de um dia | equipe; esquadrão; unidade básica das forças armadas | nome usado antigamente para designar uma companhia teatral}
  \definition{v.}{mover; implantar; implementar}
\end{EntryWithPhonetic}

\begin{EntryWithPhonetic}{班级}{ban1 ji2}{10,6}{⽟、⽷}[HSK 3]
  \definition[个]{s.}{classe; série (na escola); o nome geral para as séries e turmas da escola}
\end{EntryWithPhonetic}

\begin{EntryWithPhonetic}{班长}{ban1 zhang3}{10,4}{⽟、⾧}[HSK 2]
  \definition[个,位,名]{s.}{monitor de turma; líder de equipe; alunos responsáveis nas turmas da escola | líder de esquadrão; responsável por uma turma de soldados, geralmente com patente de sargento}
\end{EntryWithPhonetic}

\begin{EntryWithPhonetic}{般}{ban1}{10}{⾈}
  \definition{clas.}{tipo; classe; gênero; amostra}
  \definition{part.}{(o mesmo) que; como; semelhante}
  \seeref{般}{bo1}
  \seeref{般}{pan2}
\end{EntryWithPhonetic}

\begin{EntryWithPhonetic}{搬}{ban1}{13}{⼿}[HSK 3]
  \definition{v.}{tirar; mover; remover | mudar-se (de casa) | aplicar indiscriminadamente; copiar mecanicamente}
\end{EntryWithPhonetic}

\begin{EntryWithPhonetic}{搬动}{ban1dong4}{13,6}{⼿、⼒}
  \definition{v.}{mover (algo ao redor) | mudar de casa}
\end{EntryWithPhonetic}

\begin{EntryWithPhonetic}{搬家}{ban1jia1}{13,10}{⼿、⼧}[HSK 3]
  \definition{v.+compl.}{mudar de casa; mudar-se para outro lugar}
\end{EntryWithPhonetic}

\begin{EntryWithPhonetic}{搬口}{ban1kou3}{13,3}{⼿、⼝}
  \definition{v.}{tagarelar | (idioma) transmitir histórias;  semear dissensão | contar histórias}
\end{EntryWithPhonetic}

\begin{EntryWithPhonetic}{搬弄}{ban1nong4}{13,7}{⼿、⼶}
  \definition{v.}{causar problemas | mexer com alguém | mostrar (o que se pode fazer)}
\end{EntryWithPhonetic}

\begin{EntryWithPhonetic}{搬运}{ban1yun4}{13,7}{⼿、⾡}
  \definition{v.}{carregar; transportar}
\end{EntryWithPhonetic}

\begin{EntryWithPhonetic}{搬走}{ban1zou3}{13,7}{⼿、⾛}
  \definition{v.}{carregar}
\end{EntryWithPhonetic}

\begin{EntryWithPhonetic}{板}{ban3}{8}{⽊}[HSK 3]
  \definition{adj.}{rígido; não natural; inflexível}
  \definition[块,个]{s.}{tábua; placa; prato; objeto rígido em forma de placa | veneziana; persiana; refere-se especificamente aos painéis de portas de lojas | badalos (instrumento musical que marca o ritmo) | uma batida acentuada (ritmo) na música e na ópera tradicional | chefe}
  \definition{v.}{parecer sério | corrigir maus hábitos ou defeitos | ser rígido como uma tábua}
\end{EntryWithPhonetic}

\begin{EntryWithPhonetic}{版}{ban3}{8}{⽚}[HSK 5]
  \definition{clas.}{usado como uma palavra de medida para materiais impressos (por exemplo, livros, jornais, edições)}
  \definition{s.}{chapa, placa ou bloco de impressão | edição (livros impressos) | página (de um jornal) | moldes ou fromas de construção}
\end{EntryWithPhonetic}

\begin{EntryWithPhonetic}{办}{ban4}{4}{⼒}[HSK 2]
  \definition{v.}{fazer; lidar com; gerenciar; cuidar de | executar; configurar | preparar algo; comprar uma quantidade razoável de | punir; levar à justiça; punir com medidas}
\end{EntryWithPhonetic}

\begin{EntryWithPhonetic}{办法}{ban4fa3}{4,8}{⼒、⽔}[HSK 2]
  \definition[个,种]{s.}{método; meio; medida; caminho; maneira; método de lidar com situações ou resolver problemas}
\end{EntryWithPhonetic}

\begin{EntryWithPhonetic}{办公}{ban4 gong1}{4,4}{⼒、⼋}[HSK 6]
  \definition{v.}{trabalhar; fazer trabalho de escritório; lidar com negócios oficiais; tratar de assuntos oficiais}
\end{EntryWithPhonetic}

\begin{EntryWithPhonetic}{办公室}{ban4gong1shi4}{4,4,9}{⼒、⼋、⼧}[HSK 2]
  \definition[个,间]{s.}{órgãos, escolas, grupos, empresas e outras entidades que lidam com assuntos administrativos cotidianos | escritório; sala de escritório}
\end{EntryWithPhonetic}

\begin{EntryWithPhonetic}{办理}{ban4li3}{4,11}{⼒、⽟}[HSK 3]
  \definition{v.}{conduzir; lidar; transacionar; negociar; solicitar um documento ou realizar um procedimento específico}
\end{EntryWithPhonetic}

\begin{EntryWithPhonetic}{办事}{ban4 shi4}{4,8}{⼒、⼅}[HSK 4]
  \definition{v.}{trabalhar | lidar com assuntos; manipular transações}
\end{EntryWithPhonetic}

\begin{EntryWithPhonetic}{办事处}{ban4 shi4 chu4}{4,8,5}{⼒、⼅、⼡}[HSK 6]
  \definition[个,家]{s.}{escritório; agência; agências enviadas pelo governo, militares, grupos, etc.}
\end{EntryWithPhonetic}

\begin{EntryWithPhonetic}{办学}{ban4 xue2}{4,8}{⼒、⼦}[HSK 6]
  \definition{v.+compl.}{administrar uma escola}
\end{EntryWithPhonetic}

\begin{EntryWithPhonetic}{半}{ban4}{5}{⼗}[HSK 1]
  \definition{adv.}{parcialmente; usado antes de verbos ou adjetivos para indicar incompletude}
  \definition{num.}{(depois de um número) ``e meio'' | meio; metade | na metade; no meio | muito pouco; o mínimo}
\end{EntryWithPhonetic}

\begin{EntryWithPhonetic}{半决赛}{ban4 jue2 sai4}{5,6,14}{⼗、⼎、⾙}[HSK 6]
  \definition{s.}{semifinais}
\end{EntryWithPhonetic}

\begin{EntryWithPhonetic}{半年}{ban4 nian2}{5,6}{⼗、⼲}[HSK 1]
  \definition{s.}{meio ano}
\end{EntryWithPhonetic}

\begin{EntryWithPhonetic}{半球}{ban4qiu2}{5,11}{⼗、⽟}
  \definition{s.}{hemisfério}
\end{EntryWithPhonetic}

\begin{EntryWithPhonetic}{半天}{ban4 tian1}{5,4}{⼗、⼤}[HSK 1]
  \definition{s.}{metade do dia; metade do dia dividida pelo meio-dia | um longo tempo; bastante tempo; refere-se a um período de tempo relativamente longo (com um tom exagerado)}
\end{EntryWithPhonetic}

\begin{EntryWithPhonetic}{半夜}{ban4 ye4}{5,8}{⼗、⼣}[HSK 2]
  \definition{s.}{no meio da noite; metade da noite | por volta da meia-noite, também se refere à madrugada}
\end{EntryWithPhonetic}

\begin{EntryWithPhonetic}{半音}{ban4yin1}{5,9}{⼗、⾳}
  \definition{s.}{semitom; na música, uma oitava é dividida em doze notas e o intervalo entre duas notas adjacentes é chamado de semitom}
\end{EntryWithPhonetic}

\begin{EntryWithPhonetic}{伴}{ban4}{7}{⼈}
  \definition[个,位]{s.}{companheiro; parceiro}
  \definition{v.}{acompanhar; estar perto}[伴君如伴虎。===Acompanhar o rei é como acompanhar um tigre.]
\end{EntryWithPhonetic}

\begin{EntryWithPhonetic}{伴侣}{ban4lv3}{7,8}{⼈、⼈}
  \definition{s.}{companheiro | parceiro}
\end{EntryWithPhonetic}

\begin{EntryWithPhonetic}{扮}{ban4}{7}{⼿}
  \definition{v.}{vestir-se como; desempenhar o papel de | maquiar-se; disfarçar-se como | (expressão facial) fazer cara de}
\end{EntryWithPhonetic}

\begin{EntryWithPhonetic}{扮演}{ban4yan3}{7,14}{⼿、⽔}[HSK 5]
  \definition{v.}{desempenhar o papel de; ter um papel (em uma peça, etc.); atuar}
\end{EntryWithPhonetic}

\begin{EntryWithPhonetic}{帮}{bang1}{9}{⼱}[HSK 1]
  \definition*{s.}{Sobrenome Bang}
  \definition{clas.}{um grupo de; um bando de; uma gangue de; um grupo de pessoas}
  \definition{s.}{lateral; superior; partes ao lado ou ao redor do objeto | folha externa; parte mais grossa das folhas externas dos vegetais | gangue; banda; grupo; conglomerado}
  \definition{v.}{ajudar; assistir; auxiliar | trabalho; refere-se ao envolvimento em trabalho assalariado}
\end{EntryWithPhonetic}

\begin{EntryWithPhonetic}{帮教}{bang1jiao4}{9,11}{⼱、⽁}
  \definition{v.}{orientar}
\end{EntryWithPhonetic}

\begin{EntryWithPhonetic}{帮忙}{bang1 mang2}{9,6}{⼱、⼼}[HSK 1]
  \definition{v.+compl.}{ajudar; dar uma mão; dar uma mãozinha; fazer um favor; fazer uma boa ação; ajudar os outros a fazer algo, referindo-se, de maneira geral, a oferecer ajuda quando alguém está com dificuldades}
\end{EntryWithPhonetic}

\begin{EntryWithPhonetic}{帮佣}{bang1yong1}{9,7}{⼱、⼈}
  \definition{s.}{trabalhador doméstico; empregada doméstica; servo; servente}
  \definition{v.}{trabalhar ou ser contratado como trabalhador doméstico, servo, etc.}
\end{EntryWithPhonetic}

\begin{EntryWithPhonetic}{帮助}{bang1zhu4}{9,7}{⼱、⼒}[HSK 2]
  \definition[个,次,回,份,种]{s.}{ajuda; auxílio; socorro; função de promoção ou auxílio}
  \definition{v.}{ajudar; assistir; apoiar; quando alguém está passando por dificuldades, oferecer apoio financeiro ou material, ou ainda apoio moral, dar conselhos, pensar em soluções, fazer coisas por essa pessoa, etc.}
\end{EntryWithPhonetic}

\begin{EntryWithPhonetic}{傍}{bang4}{12}{⼈}
  \definition*{s.}{Sobrenome Bang}
  \definition{v.}{estar perto de (à distância); aproximar-se | estar perto de (no tempo) | depender de; confiar em}
\end{EntryWithPhonetic}

\begin{EntryWithPhonetic}{傍晚}{bang4wan3}{12,11}{⼈、⽇}[HSK 6]
  \definition[个]{s.}{ao entardecer; ao cair da noite; (tarde) refere-se ao momento em que se aproxima o anoitecer, frequentemente usado na linguagem escrita}
\end{EntryWithPhonetic}

\begin{EntryWithPhonetic}{棒}{bang4}{12}{⽊}[HSK 5]
  \definition{adj.}{bom; forte; excelente}
  \definition[根]{s.}{porrete; bastão; cajado; clava}
\end{EntryWithPhonetic}

\begin{EntryWithPhonetic}{棒棒糖}{bang4bang4tang2}{12,12,16}{⽊、⽊、⽶}
  \definition[根]{s.}{pirulito}
\end{EntryWithPhonetic}

\begin{EntryWithPhonetic}{棒冰}{bang4bing1}{12,6}{⽊、⼎}
  \definition{s.}{picolé}
\end{EntryWithPhonetic}

\begin{EntryWithPhonetic}{包}{bao1}{5}{⼓}[HSK 1]
  \definition*{s.}{Sobrenome Bao}
  \definition{clas.}{pacote; embalagem; embrulho; usado para coisas empacotadas}
  \definition[个,只]{s.}{feixe; pacote; encomenda; algo embrulhado | saco; sacola; saco para guardar coisas | caroço; inchaço; protuberância; inchaço ou protuberância no corpo ou em objetos | tenda; tenda com cúpula feita de feltro}
  \definition{v.}{embrulhar; envolver com papel, tecido, etc. | cercar; rodear; envolver; envelopar | incluir; conter | realizar todo o processo; assumir toda a responsabilidade | assegurar; garantir | contratar; reservar; fretar; comprar ou alugar tudo; acordar uso exclusivo}
\end{EntryWithPhonetic}

\begin{EntryWithPhonetic}{包办}{bao1ban4}{5,4}{⼓、⼒}
  \definition{v.}{cuidar de tudo que diz respeito a um trabalho | comandar todo o espetáculo; monopolizar tudo | assumir tudo; manter tudo em suas próprias mãos}
\end{EntryWithPhonetic}

\begin{EntryWithPhonetic}{包干}{bao1gan1}{5,3}{⼓、⼲}
  \definition{s.}{tarefa alocada}
  \definition{v.}{ter a responsabilidade total sobre um trabalho}
\end{EntryWithPhonetic}

\begin{EntryWithPhonetic}{包裹}{bao1guo3}{5,14}{⼓、⾐}[HSK 4]
  \definition[个,件]{s.}{pacote; embrulho}
  \definition{v.}{embrulhar; amarrar; enrolar coisas em pano ou outra coisa}
\end{EntryWithPhonetic}

\begin{EntryWithPhonetic}{包含}{bao1han2}{5,7}{⼓、⼝}[HSK 4]
  \definition{v.}{conter; implicar; incluir; conter dentro, resumir, enfatizar o que está contido dentro, focar em relações internas, muitas vezes coisas abstratas}
\end{EntryWithPhonetic}

\begin{EntryWithPhonetic}{包括}{bao1kuo4}{5,9}{⼓、⼿}[HSK 4]
  \definition{v.}{incluir; compreender; consistir em; conter, conter dentro, resumir junto, enfatizar a listagem de todas as partes, ou a citação de uma parte delas, que podem ser coisas abstratas ou concretas}
\end{EntryWithPhonetic}

\begin{EntryWithPhonetic}{包容}{bao1rong2}{5,10}{⼓、⼧}
  \definition{adj.}{inclusivo}
  \definition{v.}{perdoar | mostrar tolerância | conter | segurar}
\end{EntryWithPhonetic}

\begin{EntryWithPhonetic}{包围}{bao1wei2}{5,7}{⼓、⼞}[HSK 5]
  \definition{v.}{circundar; cercar; rodear}
\end{EntryWithPhonetic}

\begin{EntryWithPhonetic}{包装}{bao1zhuang1}{5,12}{⼓、⾐}[HSK 5]
  \definition[个,款]{s.}{embalagem; materiais usados para embalar produtos, como papel, sacolas, garrafas ou caixas}
  \definition{v.}{embalar; embrulhar; empacotar | aumentar a fama e o apelo de alguém ou algo por meio de publicidade | tornar alguém ou algo mais comercialmente viável ou atraente por meio de embelezamento ou publicidade}
\end{EntryWithPhonetic}

\begin{EntryWithPhonetic}{包子}{bao1 zi5}{5,3}{⼓、⼦}[HSK 1]
  \definition[个]{s.}{pão recheado cozido no vapor; alimentos, com recheio de vegetais, carne ou açúcar, etc., com massa levedada como invólucro, embrulhados e cozidos no vapor}
\end{EntryWithPhonetic}

\begin{EntryWithPhonetic}{包租}{bao1zu1}{5,10}{⼓、⽲}
  \definition{s.}{aluguel fixo para terras agrícolas}
  \definition{v.}{fretar | alugar | alugar um terreno ou uma casa para subarrendar}
\end{EntryWithPhonetic}

\begin{EntryWithPhonetic}{炮}{bao1}{9}{⽕}
  \definition{v.}{processar; o método de preparação da medicina chinesa é colocar as ervas cruas em uma panela de ferro em alta temperatura e fritá-las até que fiquem marrons e estourem | secar alimentos pelo calor; refogar}
  \seeref{炮}{pao2}
  \seeref{炮}{pao4}
\end{EntryWithPhonetic}

\begin{EntryWithPhonetic}{薄}{bao2}{16}{⾋}[HSK 4]
  \definition{adj.}{fino; frágil | frio; indiferente; carente de calor | leve; fraco | pobre; infértil}
  \seeref{薄}{bo2}
  \seeref{薄}{bo4}
\end{EntryWithPhonetic}

\begin{EntryWithPhonetic}{宝}{bao3}{8}{⼧}[HSK 4]
  \definition*{s.}{Sobrenome Bao}
  \definition{adj.}{antigo; precioso; estimado}
  \definition{pron.}{estimado; um termo educado usado para se referir à família, loja, etc. de alguém}
  \definition[个,件]{s.}{tesouro; objeto estimado; coisa preciosa | dinheiro; moeda; moeda antiga com furo quadrado no centro; moeda de prata}
\end{EntryWithPhonetic}

\begin{EntryWithPhonetic}{宝宝}{bao3 bao5}{8,8}{⼧、⼧}[HSK 4]
  \definition[个,位]{s.}{querida; \emph{darling}; \emph{baby}; apelido para crianças}
\end{EntryWithPhonetic}

\begin{EntryWithPhonetic}{宝贝}{bao3bei4}{8,4}{⼧、⾙}[HSK 4]
  \definition{adj.}{excêntrico; estranho; imprestável; um termo depreciativo para uma pessoa incompetente ou ridícula}
  \definition[个,件]{s.}{tesouro; objeto estimado; coisa preciosa | querida; \emph{darling}; \emph{baby}; apelido para crianças}
\end{EntryWithPhonetic}

\begin{EntryWithPhonetic}{宝贵}{bao3gui4}{8,9}{⼧、⾙}[HSK 4]
  \definition{adj.}{precioso; extremamente valioso, muito raro, pode ser usado para descrever coisas específicas, também pode ser usado para descrever coisas abstratas | valioso; como um tesouro}
\end{EntryWithPhonetic}

\begin{EntryWithPhonetic}{宝石}{bao3 shi2}{8,5}{⼧、⽯}[HSK 4]
  \definition[颗,枚,块,粒]{s.}{gema; jóia; pedra preciosa; mineral precioso que tem um brilho lindo e uma dureza de mais de sete graus, não é afetado pela atmosfera ou por produtos químicos e pode ser usado como decoração, suporte de instrumentos ou abrasivos}
\end{EntryWithPhonetic}

\begin{EntryWithPhonetic}{饱}{bao3}{8}{⾷}[HSK 2]
  \definition{adj.}{cheio; comer até ficar satisfeito | cheio; rechonchudo}
  \definition{adv.}{totalmente; completamente; plenamente}
  \definition{v.}{satisfazer}
\end{EntryWithPhonetic}

\begin{EntryWithPhonetic}{保}{bao3}{9}{⼈}[HSK 3]
  \definition*{s.}{Sobrenome Bao}
  \definition{s.}{fiador; babá ou responsável pela guarda de crianças | oficial responsável; sistema administrativo; unidade administrativa do antigo registro civil}
  \definition{v.}{defender; proteger | manter; preservar; conservar em boas condições | assegurar; garantir | ser fiador de alguém}
\end{EntryWithPhonetic}

\begin{EntryWithPhonetic}{保安}{bao3 an1}{9,6}{⼈、⼧}[HSK 3]
  \definition[个,位,名]{s.}{guarda de segurança; segurança}
  \definition{v.}{proteger; manter em segurança; defender a segurança social | garantir a segurança; proteger a segurança dos trabalhadores e prevenir acidentes durante o processo de produção}
\end{EntryWithPhonetic}

\begin{EntryWithPhonetic}{保持}{bao3chi2}{9,9}{⼈、⼿}[HSK 3]
  \definition{v.}{manter; conservar; reter; preservar; manter um determinado estado, para que não desapareça ou não se altere}
\end{EntryWithPhonetic}

\begin{EntryWithPhonetic}{保存}{bao3cun2}{9,6}{⼈、⼦}[HSK 3]
  \definition{v.}{salvar; preservar; conservar; manter a existência com ênfase em que as coisas, as propriedades, os significados, os estilos, etc. não sofram perdas ou mudanças | (computação) salvar (um arquivo, etc.)}
\end{EntryWithPhonetic}

\begin{EntryWithPhonetic}{保护}{bao3hu4}{9,7}{⼈、⼿}[HSK 3]
  \definition{v.}{proteger, guardar, cuidar; salvaguardar; cuidar ao máximo, para que não seja danificado, referindo-se principalmente a coisas concretas}
\end{EntryWithPhonetic}

\begin{EntryWithPhonetic}{保护国}{bao3hu4guo2}{9,7,8}{⼈、⼿、⼞}
  \definition{s.}{protetorado}
\end{EntryWithPhonetic}

\begin{EntryWithPhonetic}{保护剂}{bao3hu4ji4}{9,7,8}{⼈、⼿、⼑}
  \definition{s.}{agente protetor; protetor}
\end{EntryWithPhonetic}

\begin{EntryWithPhonetic}{保护区}{bao3hu4qu1}{9,7,4}{⼈、⼿、⼖}
  \definition[个,片]{s.}{zona de proteção | área de preservação; reserva natural}
\end{EntryWithPhonetic}

\begin{EntryWithPhonetic}{保护色}{bao3hu4se4}{9,7,6}{⼈、⼿、⾊}
  \definition{s.}{camuflagem | coloração protetora}
\end{EntryWithPhonetic}

\begin{EntryWithPhonetic}{保护神}{bao3hu4shen2}{9,7,9}{⼈、⼿、⽰}
  \definition{s.}{anjo da guarda | santo patrono}
\end{EntryWithPhonetic}

\begin{EntryWithPhonetic}{保护物}{bao3hu4 wu4}{9,7,8}{⼈、⼿、⽜}
  \definition{s.}{protetor}
\end{EntryWithPhonetic}

\begin{EntryWithPhonetic}{保护性}{bao3hu4xing4}{9,7,8}{⼈、⼿、⼼}
  \definition{s.}{proteção; protetor}
\end{EntryWithPhonetic}

\begin{EntryWithPhonetic}{保护者}{bao3hu4zhe3}{9,7,8}{⼈、⼿、⽼}
  \definition{s.}{protetor | segurador}
\end{EntryWithPhonetic}

\begin{EntryWithPhonetic}{保护主义}{bao3hu4zhu3yi4}{9,7,5,3}{⼈、⼿、⼂、⼂}
  \definition{s.}{protecionismo}
\end{EntryWithPhonetic}

\begin{EntryWithPhonetic}{保健}{bao3 jian4}{9,10}{⼈、⼈}[HSK 6]
  \definition{s.}{cuidados de saúde; proteção da saúde}
  \definition{v.}{cuidar da sua saúde; proteger sua saúde}
\end{EntryWithPhonetic}

\begin{EntryWithPhonetic}{保留}{bao3liu2}{9,10}{⼈、⽥}[HSK 3]
  \definition{v.}{manter; continuar a ter; manter o estado original inalterado | conter; reter; deixar ficar; não tirar | reservar; colocar os direitos, opiniões, etc. de lado, não exercê-los ou expressá-los por enquanto}
\end{EntryWithPhonetic}

\begin{EntryWithPhonetic}{保密}{bao3mi4}{9,11}{⼈、⼧}[HSK 4]
  \definition{v.}{manter segredo; manter algo em segredo; manter a confidencialidade}
\end{EntryWithPhonetic}

\begin{EntryWithPhonetic}{保守}{bao3shou3}{9,6}{⼈、⼧}[HSK 4]
  \definition{adj.}{retrógrado; conservador; pensamentos e conceitos que são retrógrados e não conseguem acompanhar o desenvolvimento da situação}
  \definition{v.}{manter; guardar; evitar perder}
\end{EntryWithPhonetic}

\begin{EntryWithPhonetic}{保卫}{bao3wei4}{9,3}{⼈、⼙}[HSK 5]
  \definition{v.}{defender; proteger; salvaguardar; proteger-se de ser violado}
\end{EntryWithPhonetic}

\begin{EntryWithPhonetic}{保险}{bao3xian3}{9,9}{⼈、⾩}[HSK 3]
  \definition{adj.}{seguro; pode ficar tranquilo}
  \definition[个,份,种]{s.}{seguro; um tipo de seguro comercial que garante que o segurado receba uma indenização em caso de prejuízo}
  \definition{v.}{ter certeza; estar obrigado a; garantir que algo aconteça (o que as pessoas desejam)}
\end{EntryWithPhonetic}

\begin{EntryWithPhonetic}{保养}{bao3yang3}{9,9}{⼈、⼋}[HSK 5]
  \definition{v.}{preservar; cuidar bem (ou conservar) da saúde |  fazer manutenção; conservar; manter; manter em bom estado de conservação}
\end{EntryWithPhonetic}

\begin{EntryWithPhonetic}{保证}{bao3zheng4}{9,7}{⼈、⾔}[HSK 3]
  \definition[种,份]{s.}{compromisso; garantia; caução; aval; condições ou coisas que garantem a realização de algo}
  \definition{v.}{prometer; garantir; assegurar; certamente concluir algo; garantir que determinados padrões e requisitos sejam alcançados}
\end{EntryWithPhonetic}

\begin{EntryWithPhonetic}{报}{bao4}{7}{⼿}[HSK 3]
  \definition[份,张]{s.}{jornal | revista; periódico; referência a uma publicação específica | relatório; boletim; algo que transmite alguma informação | telegrama | julgamento; retribuição}
  \definition{v.}{relatar; declarar; anunciar; informar; comunicar | responder; retribuir; revidar | retribuir; recompensar | vingar-se; retaliar | relatar; condenar de acordo com a lei e reportar às autoridades superiores | enviar; submeter; especificamente, relatar ao superior}
\end{EntryWithPhonetic}

\begin{EntryWithPhonetic}{报酬}{bao4chou5}{7,13}{⼿、⾣}
  \definition{s.}{recompensa | remuneração}
\end{EntryWithPhonetic}

\begin{EntryWithPhonetic}{报答}{bao4da2}{7,12}{⼿、⽵}[HSK 5]
  \definition{v.}{reembolsar; devolver; retribuir; pagar de volta; mostrar seu apreço de forma tangível}
\end{EntryWithPhonetic}

\begin{EntryWithPhonetic}{报到}{bao4dao4}{7,8}{⼿、⼑}[HSK 3]
  \definition{v.+compl.}{apresentar-se ao serviço; fazer o check-in; registrar-se; assinar o livro de presença; informar à organização que você já chegou}
\end{EntryWithPhonetic}

\begin{EntryWithPhonetic}{报道}{bao4dao4}{7,12}{⼿、⾡}[HSK 3]
  \definition[个,篇,分]{s.}{história; reportagem; comunicado de imprensa publicado por escrito ou transmitido pela rádio}
  \definition{v.}{cobrir; reportar (notícias); divulgar notícias ao público através de jornais, rádio, etc.}
\end{EntryWithPhonetic}

\begin{EntryWithPhonetic}{报告}{bao4gao4}{7,7}{⼿、⼝}[HSK 3]
  \definition[份,篇]{s.}{relatório; discurso; palestra; consultivo; declaração formal feita a superiores ou ao público}
  \definition{v.}{relatar; divulgar; informar; informar formalmente sobre um assunto ou opinião aos superiores ou ao público em geral}
\end{EntryWithPhonetic}

\begin{EntryWithPhonetic}{报警}{bao4jing3}{7,19}{⼿、⾔}[HSK 5]
  \definition{v.}{relatar (um incidente) à polícia; relatar uma situação crítica ou sinalizar uma emergência às autoridades competentes}
\end{EntryWithPhonetic}

\begin{EntryWithPhonetic}{报刊}{bao4 kan1}{7,5}{⼿、⼑}[HSK 6]
  \definition{s.}{a imprensa; jornais e periódicos}
\end{EntryWithPhonetic}

\begin{EntryWithPhonetic}{报考}{bao4 kao3}{7,6}{⼿、⽼}[HSK 6]
  \definition{v.}{inscrever-se para um exame}
\end{EntryWithPhonetic}

\begin{EntryWithPhonetic}{报名}{bao4ming2}{7,6}{⼿、⼝}[HSK 2]
  \definition{v.+compl.}{inscrever-se; alistar-se; registrar seu nome; cadastrar-se; matricular-se; informar seu nome à pessoa responsável, órgão, grupo etc., indicando que você deseja participar de alguma atividade ou organização}
\end{EntryWithPhonetic}

\begin{EntryWithPhonetic}{报纸}{bao4zhi3}{7,7}{⼿、⽷}[HSK 2]
  \definition[分,期,张]{s.}{jornal; publicações periódicas cujo conteúdo principal é notícias, geralmente referem-se a jornais diários | papel jornal; um tipo de papel usado para imprimir jornais ou publicações em geral}
\end{EntryWithPhonetic}

\begin{EntryWithPhonetic}{抱}{bao4}{8}{⼿}[HSK 4]
  \definition*{s.}{Sobrenome Bao}
  \definition{clas.}{braçada; medida dos dois braços}
  \definition{v.}{carregar no peito; segurar com ambos os braços; abraçar | ter o primeiro filho ou neto | adotar um bebê ou criança | ficar juntos, unidos | encaixar ou servir perfeitamente (roupas e sapatos do tamanho certo) | estimar; nutrir; abrigar; ter em mente | continuar; sobrecarregar com | chocar ovos}
\end{EntryWithPhonetic}

\begin{EntryWithPhonetic}{抱歉}{bao4qian4}{8,14}{⼿、⽋}[HSK 6]
  \definition{adj.}{pesaroso; arrependido; sentir pena de alguém porque você causou perda, inconveniência ou não atendeu às suas necessidades}
\end{EntryWithPhonetic}

\begin{EntryWithPhonetic}{抱怨}{bao4yuan4}{8,9}{⼿、⼼}[HSK 5]
  \definition{v.}{reclamar ou expressar descontentamento ou insatisfação; falar com os outros sobre pessoas ou coisas com as quais você não está satisfeito}
\end{EntryWithPhonetic}

\begin{EntryWithPhonetic}{豹}{bao4}{10}{⾘}
  \definition*{s.}{Sobrenome Bao}
  \definition[只]{s.}{leopardo; pantera | espécies de gato da montanha}
\end{EntryWithPhonetic}

\begin{EntryWithPhonetic}{豹子}{bao4zi5}{10,3}{⾘、⼦}
  \definition[头]{s.}{leopardo}
\end{EntryWithPhonetic}

\begin{EntryWithPhonetic}{暴}{bao4}{15}{⽇}
  \definition*{s.}{Sobrenome Bao}
  \definition{adj.}{repentino e violento | cruel; selvagem; feroz | temperamental | severo e tirânico; brutal | irritável; irascível; impaciente}
  \definition{adv.}{de repente e ferozmente}
  \definition{s.}{violência; ferocidade}
  \definition{v.}{sobressair; destacar-se; inchar | expor; transmitir | desperdiçar; arruinar; estragar}
\end{EntryWithPhonetic}

\begin{EntryWithPhonetic}{暴风雨}{bao4 feng1 yu3}{15,4,8}{⽇、⾵、⾬}[HSK 6]
  \definition{s.}{tempestade; tormenta; temporal; borrasca; vento e chuva fortes e violentos}
\end{EntryWithPhonetic}

\begin{EntryWithPhonetic}{暴力}{bao4li4}{15,2}{⽇、⼒}[HSK 6]
  \definition{s.}{violência; força (usada em tempos de conflito); poder de coerção}
\end{EntryWithPhonetic}

\begin{EntryWithPhonetic}{暴露}{bao4lu4}{15,21}{⽇、⾬}[HSK 6]
  \definition{adj.}{reveladoras (roupas inadequadas que expõem muito o corpo)}
  \definition{v.}{expor; desnudar; revelar; tornar público algo oculto}
\end{EntryWithPhonetic}

\begin{EntryWithPhonetic}{暴乱}{bao4luan4}{15,7}{⽇、⼄}
  \definition{s.}{rebelião | revolta | tumulto}
\end{EntryWithPhonetic}

\begin{EntryWithPhonetic}{暴行}{bao4xing2}{15,6}{⽇、⾏}
  \definition{s.}{ato selvagem | atrocidade | indignação}
\end{EntryWithPhonetic}

\begin{EntryWithPhonetic}{暴雨}{bao4yu3}{15,8}{⽇、⾬}[HSK 6]
  \definition[场,次,阵]{s.}{tempestade; chuva torrencial; chuva forte com precipitação intensa; em meteorologia, refere-se a chuvas de 16 mm ou mais em uma hora ou 50 mm ou mais em 24 horas}
\end{EntryWithPhonetic}

\begin{EntryWithPhonetic}{暴躁}{bao4zao4}{15,20}{⽇、⾜}
  \definition{adj.}{irascível | irritável}
\end{EntryWithPhonetic}

\begin{EntryWithPhonetic}{瀑}{bao4}{18}{⽔}
  \definition{s.}{chuva torrencial; tempestade}
  \seeref{瀑}{pu4}
\end{EntryWithPhonetic}

\begin{EntryWithPhonetic}{爆}{bao4}{19}{⽕}[HSK 6]
  \definition{v.}{explodir; estourar | fritar rapidamente; ferver rapidamente | aparecer (ou ocorrer) inesperadamente}
\end{EntryWithPhonetic}

\begin{EntryWithPhonetic}{爆发}{bao4fa1}{19,5}{⽕、⼜}[HSK 6]
  \definition{v.}{entrar em erupção; explodir | estourar; irromper; ocorrer de forma repentina e violenta}
\end{EntryWithPhonetic}

\begin{EntryWithPhonetic}{爆米花}{bao4mi3hua1}{19,6,7}{⽕、⽶、⾋}
  \definition{s.}{pipoca (de milho) | pipoca de arroz}
\end{EntryWithPhonetic}

\begin{EntryWithPhonetic}{爆炸}{bao4zha4}{19,9}{⽕、⽕}[HSK 6]
  \definition{s.}{explosão}
  \definition{v.}{explodir; explodir; detonar | aumentar bruscamente em um curto espaço de tempo (de quantidade)}
\end{EntryWithPhonetic}

\begin{EntryWithPhonetic}{杯}{bei1}{8}{⽊}[HSK 1]
  \definition{clas.}{para certos recipientes de líquidos: copo, xícara, etc.}
  \definition[只,个]{s.}{copo; caneca; xícara | taça; troféu; prêmio em forma de taça}
\end{EntryWithPhonetic}

\begin{EntryWithPhonetic}{杯具}{bei1ju4}{8,8}{⽊、⼋}
  \definition{s.}{parachoque | fiasco | (gíria) tragédia}
\end{EntryWithPhonetic}

\begin{EntryWithPhonetic}{杯子}{bei1 zi5}{8,3}{⽊、⼦}[HSK 1]
  \definition[个,只,种]{s.}{xícara; copo; recipiente para bebidas ou outros líquidos, geralmente cilíndrico ou com a parte inferior ligeiramente mais estreita, com capacidade geralmente pequena}
\end{EntryWithPhonetic}

\begin{EntryWithPhonetic}{背}{bei1}{9}{⾁}[HSK 2]
  \definition{clas.}{carga; pacote; para transportar coisas nas costas}
  \definition{v.}{carregar nas costas | suportar; carregar}
  \seeref{背}{bei4}
\end{EntryWithPhonetic}

\begin{EntryWithPhonetic}{背包}{bei1 bao1}{9,5}{⾁、⼓}[HSK 5]
  \definition[个,只,款]{s.}{mochila; mochila de ataque; mochila de infantaria; pacotes de roupas carregados nas costas quando marcham}
\end{EntryWithPhonetic}

\begin{EntryWithPhonetic}{悲}{bei1}{12}{⽕}
  \definition{adj.}{triste; pesaroso; melancólico | compassivo; misericordioso}
\end{EntryWithPhonetic}

\begin{EntryWithPhonetic}{悲惨}{bei1can3}{12,11}{⽕、⽕}[HSK 6]
  \definition{adj.}{trágico; miserável; extremamente doloroso e triste}
\end{EntryWithPhonetic}

\begin{EntryWithPhonetic}{悲观}{bei1guan1}{12,6}{⽕、⾒}
  \definition{adj.}{pessimista; negativismo, falta de confiança no futuro (oposto a 乐观)}
  \seealsoref{乐观}{le4guan1}
\end{EntryWithPhonetic}

\begin{EntryWithPhonetic}{悲剧}{bei1 ju4}{12,10}{⽕、⼑}[HSK 5]
  \definition[出,部]{s.}{tragédia; drama trágico; uma das principais categorias de teatro, caracterizada basicamente pela representação do conflito irreconciliável entre o protagonista e a realidade e seu final trágico | tragédia; evento triste; metáfora para encontro infeliz}
\end{EntryWithPhonetic}

\begin{EntryWithPhonetic}{悲伤}{bei1 shang1}{12,6}{⽕、⼈}[HSK 5]
  \definition{adj.}{triste; pesaroso}
\end{EntryWithPhonetic}

\begin{EntryWithPhonetic}{北}{bei3}{5}{⼔}[HSK 1]
  \definition*{s.}{Norte (os países desenvolvidos) | Sobrenome Bei}
  \definition{s.}{norte; uma das quatro direções básicas, a esquerda quando se está de frente para o sol pela manhã (oposta ao 南)}
  \definition{v.}{ser derrotado}
  \seealsoref{南}{nan2}
\end{EntryWithPhonetic}

\begin{EntryWithPhonetic}{北边}{bei3 bian1}{5,5}{⼔、⾡}[HSK 1]
  \definition{s.}{norte; o lado norte}
\end{EntryWithPhonetic}

\begin{EntryWithPhonetic}{北部}{bei3 bu4}{5,10}{⼔、⾢}[HSK 3]
  \definition{s.}{parte norte de uma região ou país}
\end{EntryWithPhonetic}

\begin{EntryWithPhonetic}{北大西洋公约组织}{bei3 da4xi1 yang2 gong1 yue1 zu3zhi1}{5,3,6,9,4,6,8,8}{⼔、⼤、⾑、⽔、⼋、⽷、⽷、⽷}
  \definition*{s.}{Organização do Tratado do Atlântico Norte, OTAN}
\end{EntryWithPhonetic}

\begin{EntryWithPhonetic}{北方}{bei3fang1}{5,4}{⼔、⽅}[HSK 2]
  \definition{s.}{norte; indicando a direção norte | o Norte; a parte norte da China, especialmente a área ao norte do rio Huang He}
\end{EntryWithPhonetic}

\begin{EntryWithPhonetic}{北极}{bei3ji2}{5,7}{⼔、⽊}[HSK 5]
  \definition*{s.}{Polo Norte; Polo Ártico}
  \definition{s.}{polo norte magnético; o ponto mais setentrional da Terra, também se refere à região mais setentrional da Terra}
\end{EntryWithPhonetic}

\begin{EntryWithPhonetic}{北京}{bei3 jing1}{5,8}{⼔、⼇}[HSK 1]
  \definition*{s.}{Pequim (Beijing), Capital da República Popular da China | Capital da China, localizada no nordeste do país, fundada em 700 a.C., a cidade é um importante centro comercial, industrial e cultural}
\end{EntryWithPhonetic}

\begin{EntryWithPhonetic}{北面}{bei3mian4}{5,9}{⼔、⾯}
  \definition{s.}{norte; o lado norte}
\end{EntryWithPhonetic}

\begin{EntryWithPhonetic}{北约}{bei3yue1}{5,6}{⼔、⽷}
  \definition*{s.}{OTAN, Organização do Tratado do Atlântico Norte; Abreviação de 北大西洋公约组织}
  \seealsoref{北大西洋公约组织}{bei3 da4xi1 yang2 gong1 yue1 zu3zhi1}
\end{EntryWithPhonetic}

\begin{EntryWithPhonetic}{贝}{bei4}{4}{⾙}[Kangxi 154]
  \definition*{s.}{Sobrenome Bei}
  \definition{clas.}{bel (= dez decibéis)}
  \definition{s.}{marisco; crustáceo; um termo geral para moluscos com concha | moeda antiga feita de conchas}
\end{EntryWithPhonetic}

\begin{EntryWithPhonetic}{备}{bei4}{8}{⼡}
  \definition*{s.}{Sobrenome Bei}
  \definition{adv.}{totalmente; de todas as maneiras possíveis | todos; tudo}
  \definition{s.}{equipamento}
  \definition{v.}{estar equipar com; ter; possuir | preparar; ficar pronto | providenciar (ou preparar) contra; tomar precauções contra}
\end{EntryWithPhonetic}

\begin{EntryWithPhonetic}{备份}{bei4fen4}{8,6}{⼡、⼈}
  \definition{s.}{cópia de segurança | \emph{backup}}
\end{EntryWithPhonetic}

\begin{EntryWithPhonetic}{备胎}{bei4tai1}{8,9}{⼡、⾁}
  \definition{s.}{pneu sobressalente | (gíria) substituto}
\end{EntryWithPhonetic}

\begin{EntryWithPhonetic}{背}{bei4}{9}{⾁}[HSK 3]
  \definition{adj.}{azarado | fora do caminho; um lugar muito distante do centro movimentado, onde poucas pessoas aparecem | deficiente auditivo}
  \definition{s.}{parte posterior do corpo; costas; coluna vertebral; parte do tronco entre os ombros e a região lombar | parte de trás de um objeto}
  \definition{v.}{afastar-se; virar as costas | decorar; memorizar; recitar de memória | esconder algo de; fazer algo em segredo | sair, ir embora; partir; abandonar | quebrar; violar; agir de forma contrária a}
  \seeref{背}{bei1}
\end{EntryWithPhonetic}

\begin{EntryWithPhonetic}{背后}{bei4 hou4}{9,6}{⾁、⼝}[HSK 3]
  \definition{s.}{parte posterior; parte de trás; traseira | pelas costas de alguém}
\end{EntryWithPhonetic}

\begin{EntryWithPhonetic}{背景}{bei4jing3}{9,12}{⾁、⽇}[HSK 4]
  \definition[种]{s.}{pano de fundo; fundo; cenário de teatro, filme ou drama de TV | fundo; cenário que permeia a imagem principal na tela | condições sociais; ambientes históricos (significativamente influentes para algo ou alguém) | poder que dá suporte a alguém}
\end{EntryWithPhonetic}

\begin{EntryWithPhonetic}{背心}{bei4 xin1}{9,4}{⾁、⼼}[HSK 6]
  \definition[件]{s.}{colete; vestimenta sem mangas; \emph{tops} sem gola e sem mangas}
\end{EntryWithPhonetic}

\begin{EntryWithPhonetic}{背着}{bei4 zhe5}{9,11}{⾁、⽬}[HSK 6]
  \definition{adv.}{pelas costas; atrás de alguém}
  \definition{v.}{carregar nas costas}
\end{EntryWithPhonetic}

\begin{EntryWithPhonetic}{倍}{bei4}{10}{⼈}[HSK 4]
  \definition{adv.}{ainda mais; especialmente | (antes de certos adjetivos) muito; particularmente; é pronunciado como um som erhua e é usado antes de certos adjetivos para expressar um alto grau de profundidade, equivalente a 非常 ou 特别}
  \definition{clas.}{vezes; usado após um numeral, significa que o valor anterior é multiplicado por este número}[增长了五倍。===Aumentou cinco vezes. | 二的三倍是六。===Três vezes dois é seis.]
  \seealsoref{非常}{fei1chang2}
  \seealsoref{特别}{te4bie2}
\end{EntryWithPhonetic}

\begin{EntryWithPhonetic}{被}{bei4}{10}{⾐}[HSK 3]
  \definition{part.}{usada antes de verbos para formar frases verbais passivas}
  \definition{prep.}{usado em uma estrutura passiva para introduzir o executor da ação ou apenas a ação | usado em frases para expressar passividade, com o sujeito sendo o objeto}
  \definition{s.}{colcha}
  \definition{v.}{cobrir; espalhar | sofrer}
\end{EntryWithPhonetic}

\begin{EntryWithPhonetic}{被单}{bei4dan1}{10,8}{⾐、⼗}
  \definition[床]{s.}{lençol (de cama) | envelope para uma colcha acolchoada}
\end{EntryWithPhonetic}

\begin{EntryWithPhonetic}{被动}{bei4dong4}{10,6}{⾐、⼒}[HSK 5]
  \definition{adj.}{passivo;  agir com base em um impulso externo (oposto de 主动) | passivo; impossibilidade de prosseguir como pretendido devido a resistência ou interferência}
  \seealsoref{主动}{zhu3dong4}
\end{EntryWithPhonetic}

\begin{EntryWithPhonetic}{被告}{bei4gao4}{10,7}{⾐、⼝}[HSK 6]
  \definition{s.}{réu; indiciado; acusado (oposto a 原告)}
  \seealsoref{原告}{yuan2gao4}
\end{EntryWithPhonetic}

\begin{EntryWithPhonetic}{被迫}{bei4 po4}{10,8}{⾐、⾡}[HSK 4]
  \definition{v.}{ser forçado; ser coagido; ser compelido; ser constrangido; ser forçado a fazer algo por força externa}
\end{EntryWithPhonetic}

\begin{EntryWithPhonetic}{被套}{bei4tao4}{10,10}{⾐、⼤}
  \definition{s.}{capa de \emph{edredon}}
  \definition{v.}{ter dinheiro preso (em ações, imóveis, etc.)}
\end{EntryWithPhonetic}

\begin{EntryWithPhonetic}{被窝}{bei4wo1}{10,12}{⾐、⽳}
  \definition{s.}{colcha}
\end{EntryWithPhonetic}

\begin{EntryWithPhonetic}{被子}{bei4zi5}{10,3}{⾐、⼦}[HSK 3]
  \definition[条,床]{s.}{colcha; cobertor; algo com que você se cobre quando dorme, geralmente feito de pano ou seda, com forro de pano e preenchido com algodão ou fio de seda}
\end{EntryWithPhonetic}

\begin{EntryWithPhonetic}{辈}{bei4}{12}{⾞}[HSK 5]
  \definition*{s.}{Sobrenome Bei}
  \definition{s.}{pessoas de um certo tipo; semelhantes | geração; geração na família | duração da vida | círculo familiar}
\end{EntryWithPhonetic}

\begin{EntryWithPhonetic}{奔}{ben1}{8}{⼤}
  \definition{v.}{correr rápido; correr com pressa | apressar | fugir; escapar | galopar | fugir; termo antigo para uma mulher que foge com um homem}
  \seeref{奔}{ben4}
\end{EntryWithPhonetic}

\begin{EntryWithPhonetic}{奔驰}{ben1chi2}{8,6}{⼤、⾺}
  \definition*{s.}{Benz de Mercedes-Benz}
  \definition{v.}{acelerar; galopar; (carro, cavalo, etc.) mover-se ou correr rapidamente}
  \seealsoref{梅赛德斯-奔驰}{mei2sai4de2si1-ben1chi2}
\end{EntryWithPhonetic}

\begin{EntryWithPhonetic}{奔跑}{ben1 pao3}{8,12}{⼤、⾜}[HSK 6]
  \definition{v.}{correr; correr muito rápido, com uma gama de aplicações mais ampla do que 奔驰, usado principalmente na linguagem falada}
  \seealsoref{奔驰}{ben1chi2}
\end{EntryWithPhonetic}

\begin{EntryWithPhonetic}{本}{ben3}{5}{⽊}[HSK 1,6]
  \definition*{s.}{Sobrenome Ben}
  \definition{adj.}{original; inerente | principal; central}
  \definition{adv.}{originalmente}
  \definition{clas.}{para livros, dicionários, periódicos, arquivos, etc. | para vídeos de uma determinada duração | para peças de teatro, ópera}
  \definition{prep.}{de acordo com; em consonância com; em conformidade com; equivalentes a 依照 e 按照}
  \definition{pron.}{nativo; próprio; refere-se ao próprio interlocutor ou ao grupo, instituição, empresa, local, etc. ao qual o interlocutor pertence | isto; atual; presente}
  \definition[个]{s.}{caule ou raiz de plantas | base; origem; fundamento; fundação;  alicerce | capital; capital social | livro; caderno; livreto | edição; versão | cópia; roteiro; manuscrito | memorial do trono; na era feudal, referia-se a um documento oficial}
  \definition{v.}{seguir; basear-se em; estar de acordo com}
  \seealsoref{按照}{an4zhao4}
  \seealsoref{依照}{yi1 zhao4}
\end{EntryWithPhonetic}

\begin{EntryWithPhonetic}{本地}{ben3 di4}{5,6}{⽊、⼟}[HSK 6]
  \definition{s.}{local; nativo; localidade; a área onde as pessoas e as coisas estão localizadas; uma área específica referida em uma narrativa}
\end{EntryWithPhonetic}

\begin{EntryWithPhonetic}{本金}{ben3 jin1}{5,8}{⽊、⾦}
  \definition{s.}{capital; capital para a operação do comércio e da indústria; capital para a operação de negócios | valor principal; dinheiro retirado ao depositar ou tomar emprestado (diferente de 利息)}
  \seealsoref{利息}{li4xi1}
\end{EntryWithPhonetic}

\begin{EntryWithPhonetic}{本科}{ben3ke1}{5,9}{⽊、⽲}[HSK 4]
  \definition{s.}{graduação; bacharelado; o curso básico de uma universidade ou faculdade}
\end{EntryWithPhonetic}

\begin{EntryWithPhonetic}{本来}{ben3lai2}{5,7}{⽊、⽊}[HSK 3]
  \definition{adj.}{original}
  \definition{adv.}{anteriormente; originalmente; indica que antes disso | claro; em primeiro lugar; como deveria ser; indica que algo é natural ou óbvio}
\end{EntryWithPhonetic}

\begin{EntryWithPhonetic}{本领}{ben3 ling3}{5,11}{⽊、⾴}[HSK 3]
  \definition[项,个,种]{s.}{habilidade; capacidade; faculdade; poder; destreza; talento}
\end{EntryWithPhonetic}

\begin{EntryWithPhonetic}{本期}{ben3 qi1}{5,12}{⽊、⽉}[HSK 6]
  \definition{adv.}{o período atual | este prazo (geralmente em finanças)}
\end{EntryWithPhonetic}

\begin{EntryWithPhonetic}{本人}{ben3ren2}{5,2}{⽊、⼈}[HSK 5]
  \definition{pron.}{eu (mim, mim mesmo); o orador refere-se a si mesmo | a si mesmo; em pessoa; refere-se à própria pessoa ou à pessoa mencionada anteriormente}
\end{EntryWithPhonetic}

\begin{EntryWithPhonetic}{本身}{ben3shen1}{5,7}{⽊、⾝}[HSK 6]
  \definition{pron.}{próprio; em si mesmo; refere-se à pessoa, unidade ou coisa em si}
\end{EntryWithPhonetic}

\begin{EntryWithPhonetic}{本事}{ben3shi4}{5,8}{⽊、⼅}
  \definition{s.}{habilidade; aptidão; capacidade; competência; refere-se às habilidades, capacidades ou talentos que uma pessoa possui em determinada área | habilidade; aptidão; capacidade; competência; a capacidade e os meios necessários para atingir um determinado objetivo ou concluir uma determinada tarefa | status; poder; posição; autoridade; refere-se à identidade, posição ou poder de uma pessoa.}
  \seeref{本事}{ben3shi5}
\end{EntryWithPhonetic}

\begin{EntryWithPhonetic}{本事}{ben3shi5}{5,8}{⽊、⼅}[HSK 3]
  \definition{s.}{habilidade; capacidade; talento; aptidão}
  \seeref{本事}{ben3shi4}
\end{EntryWithPhonetic}

\begin{EntryWithPhonetic}{本土}{ben3 tu3}{5,3}{⽊、⼟}[HSK 6]
  \definition{s.}{território metropolitano; pátria-mãe; refere-se ao território do país | país (ou terra) natal de alguém; nativo; cidade natal; local original de crescimento}
\end{EntryWithPhonetic}

\begin{EntryWithPhonetic}{本质}{ben3zhi4}{5,8}{⽊、⾙}[HSK 6]
  \definition{s.}{essência; natureza; caráter inato; qualidade intrínseca; refere-se aos atributos fundamentais inerentes às próprias coisas, que desempenham um papel decisivo na natureza, condição e desenvolvimento das coisas (distinguido de 现象)}
  \seealsoref{现象}{xian4xiang4}
\end{EntryWithPhonetic}

\begin{EntryWithPhonetic}{本子}{ben3 zi5}{5,3}{⽊、⼦}[HSK 1]
  \definition[个,本]{s.}{livro; caderno | edição | impressão | licença; certificado de competência emitido por uma instituição especializada, obtido após aprovação no exame | \emph{script}; roteiro}
\end{EntryWithPhonetic}

\begin{EntryWithPhonetic}{奔}{ben4}{8}{⼤}
  \definition{prep.}{em direção a}
  \definition{v.}{ir direto em direção a; seguir em direção a; ir direto para o seu destino | aproximar-se; estar prestes a | estar ocupado correndo por aí; correr por algo}
  \seeref{奔}{ben1}
\end{EntryWithPhonetic}

\begin{EntryWithPhonetic}{笨}{ben4}{11}{⽵}[HSK 4]
  \definition{adj.}{estúpido; sem graça; tolo; de pouca habilidade; sem inteligência | desajeitado; tosco; inflexível | incômodo; pesado; desajeitado; difícil de manejar; trabalhoso}
\end{EntryWithPhonetic}

\begin{EntryWithPhonetic}{笨蛋}{ben4dan4}{11,11}{⽵、⾍}
  \definition{s.}{bobalhão | cabeça-oca | cabeça-dura}
  \definition{v.}{iludir | enganar}
\end{EntryWithPhonetic}

\begin{EntryWithPhonetic}{崩}{beng1}{11}{⼭}
  \definition{v.}{colapsar |  estourar; quebrar | atingir por explosão | matar atirando; atirar; executar | (de um imperador) morrer | rachar; romper | atingir | executar atirando}
\end{EntryWithPhonetic}

\begin{EntryWithPhonetic}{绷}{beng1}{11}{⽷}
  \definition{s.}{estrutura de cama amarrada com cordas, tiras de vime, etc.}
  \definition{v.}{esticar (ou puxar) com força | saltar; quicar | alinhavar; fixar | (dialeto) conseguir fazer algo com dificuldade | (roupas) apertar | costurar ou alfinetar com parcimônia | (dialeto) fraudar; roubar dinheiro}
  \seeref{绷}{beng3}
\end{EntryWithPhonetic}

\begin{EntryWithPhonetic}{绷带}{beng1dai4}{11,9}{⽷、⼱}
  \definition{s.}{curativo | (empréstimo linguístico) \emph{bandage}}
\end{EntryWithPhonetic}

\begin{EntryWithPhonetic}{甭}{beng2}{9}{⽤}
  \definition{adv.}{não; não precisa; não tem que; contração de 不用}
  \seealsoref{不用}{bu2 yong4}
\end{EntryWithPhonetic}

\begin{EntryWithPhonetic}{绷}{beng3}{11}{⽷}
  \definition{v.}{mostrar uma cara sombria, tensa; parecer descontente | conter o próprio temperamento}
\end{EntryWithPhonetic}

\begin{EntryWithPhonetic}{蹦}{beng4}{18}{⾜}
  \definition{v.}{pular; saltar; quicar}
\end{EntryWithPhonetic}

\begin{EntryWithPhonetic}{蹦极}{beng4ji2}{18,7}{⾜、⽊}
  \definition{s.}{\emph{bungee jumping}}
\end{EntryWithPhonetic}

\begin{EntryWithPhonetic}{逼}{bi1}{12}{⾡}[HSK 6]
  \definition{adj.}{estreito}
  \definition{v.}{forçar; pressionar; compelir | extorquir; pressionar por | fechar em; pressionar em direção a; aproximar-se}
\end{EntryWithPhonetic}

\begin{EntryWithPhonetic}{鼻}{bi2}{14}{⿐}[Kangxi 209]
  \definition{s.}{nariz}
\end{EntryWithPhonetic}

\begin{EntryWithPhonetic}{鼻子}{bi2zi5}{14,3}{⿐、⼦}[HSK 5]
  \definition[个,只]{s.}{nariz; órgão da face, responsável pela respiração e pelo olfato}
\end{EntryWithPhonetic}

\begin{EntryWithPhonetic}{比}{bi3}{4}{⽐}[HSK 1][Kangxi 81]
  \definition*{s.}{Abreviação de Bélgica, 比利时}
  \definition{adj.}{específico}
  \definition{adv.}{recentemente}
  \definition{part.}{partícula usada para comparação (superioridade)}
  \definition{prep.}{que; do que | (seguido por um substantivo e adjetivo) mais \{adj.\} do que \{s.\}}
  \definition{s.}{razão; proporção | contraste; comparação | metáfora em poesia; técnica de composição poética}
  \definition{v.}{estar ao lado de; estar próximo a | igualar; comparar; competir; contrastar; emular; comparar superioridade, inferioridade, comprimento, distância, qualidade, etc. | assemelhar-se a; comparar com; fazer uma analogia | gesticular; fazer gestos | ser treinado em; ser direcionado a | copiar; imitar | poder ser comparado | apegar-se a; depender}
  \seealsoref{比利时}{bi3li4shi2}
\end{EntryWithPhonetic}

\begin{EntryWithPhonetic}{比方}{bi3fang1}{4,4}{⽐、⽅}[HSK 5]
  \definition{conj.}{se; suponha que; expressa uma hipótese, equivalente a 如果 (com eufemismos)}
  \definition{s.}{analogia; exemplo; instância; expressão que usa uma coisa para descrever outra (expressão idiomática); (figurativo) usar uma coisa para descrever outra}
  \definition{v.}{ilustrar; exemplificar; fazer uma analogia; usar uma coisa para descrever outra (expressão idiomática)}
  \seealsoref{如果}{ru2guo3}
\end{EntryWithPhonetic}

\begin{EntryWithPhonetic}{比分}{bi3 fen1}{4,4}{⽐、⼑}[HSK 4]
  \definition{s.}{pontuação; comparação de pontuações entre as duas equipes em uma partida}
\end{EntryWithPhonetic}

\begin{EntryWithPhonetic}{比较}{bi3jiao4}{4,10}{⽐、⾞}[HSK 3]
  \definition{adv.}{razoavelmente; relativamente; bastante; um pouco; comparativamente; indica um certo grau, com o significado de 相当}
  \definition{prep.}{usado para comparar uma diferença de grau. para distinguir as diferenças ou superioridades entre duas ou mais coisas semelhantes}
  \definition{v.}{comparar; contrastar; usado para comparar diferenças em propriedades e graus; para distinguir semelhanças, diferenças ou superioridade entre duas ou mais coisas semelhantes}
  \seealsoref{相当}{xiang1dang1}
\end{EntryWithPhonetic}

\begin{EntryWithPhonetic}{比利时}{bi3li4shi2}{4,7,7}{⽐、⼑、⽇}
  \definition*{s.}{Bélgica}
\end{EntryWithPhonetic}

\begin{EntryWithPhonetic}{比例}{bi3li4}{4,8}{⽐、⼈}[HSK 3]
  \definition{s.}{escala; razão; relação de múltiplos entre dois números | porporção; a quantidade que uma parte representa no todo | proporção; na descrição do grau de coordenação das coisas}
\end{EntryWithPhonetic}

\begin{EntryWithPhonetic}{比拼}{bi3pin1}{4,9}{⽐、⼿}
  \definition{s.}{concurso}
  \definition{v.}{competir ferozmente}
\end{EntryWithPhonetic}

\begin{EntryWithPhonetic}{比如}{bi3ru2}{4,6}{⽐、⼥}[HSK 2]
  \definition{conj.}{por exemplo; tal como; suponha; digamos; a seguir, apresentamos alguns exemplos; na linguagem coloquial, também se pode dizer 比如说}
  \seealsoref{比如说}{bi3 ru2 shuo1}
\end{EntryWithPhonetic}

\begin{EntryWithPhonetic}{比如说}{bi3 ru2 shuo1}{4,6,9}{⽐、⼥、⾔}[HSK 2]
  \definition{adv.}{por exemplo}
  \seealsoref{比如}{bi3ru2}
\end{EntryWithPhonetic}

\begin{EntryWithPhonetic}{比萨饼}{bi3sa4bing3}{4,11,9}{⽐、⾋、⾷}
  \definition[张]{s.}{pizza}
\end{EntryWithPhonetic}

\begin{EntryWithPhonetic}{比赛}{bi3sai4}{4,14}{⽐、⾙}[HSK 3]
  \definition[场,次,轮,站,个]{s.}{competição; atividades da competição}
  \definition{v.}{competir; disputar; comparar o nível e a qualidade das habilidades e competências}
\end{EntryWithPhonetic}

\begin{EntryWithPhonetic}{比亚迪}{bi3ya4di2}{4,6,8}{⽐、⼆、⾡}
  \definition*{s.}{Montadora BYD}
\end{EntryWithPhonetic}

\begin{EntryWithPhonetic}{比重}{bi3zhong4}{4,9}{⽐、⾥}[HSK 5]
  \definition{s.}{proporção; o peso da parte em relação ao todo | Física: densidade específica; a relação entre o peso de um objeto e seu volume}
\end{EntryWithPhonetic}

\begin{EntryWithPhonetic}{彼}{bi3}{8}{⼻}
  \definition{s.}{aquele; aquilo (oposto a 此) ; outro | a outra parte}
  \seealsoref{此}{ci3}
\end{EntryWithPhonetic}

\begin{EntryWithPhonetic}{彼此}{bi3ci3}{8,6}{⼻、⽌}[HSK 5]
  \definition{pron.}{um ao outro; uns com os outros; este e aquele têm algum tipo de relacionamento; ambas as partes}
\end{EntryWithPhonetic}

\begin{EntryWithPhonetic}{笔}{bi3}{10}{⽵}[HSK 2]
  \definition{clas.}{usado para grandes quantias de dinheiro, compras, negócios, propriedades, etc. | usado em caligrafia e pintura, etc.}
  \definition[支,枝]{s.}{caneta; lápis; pincel para escrever; ferramentas para escrever ou desenhar | técnica de escrita; caligrafia ou desenho | traço}
  \definition{v.}{escrever à mão}
\end{EntryWithPhonetic}

\begin{EntryWithPhonetic}{笔记}{bi3 ji4}{10,5}{⽵、⾔}[HSK 2]
  \definition[篇,本,个]{s.}{notas; anotações feitas durante aulas, palestras e leituras | ensaios; esboços}
  \definition{v.}{tomar nota (por escrito)}
\end{EntryWithPhonetic}

\begin{EntryWithPhonetic}{笔记本}{bi3ji4ben3}{10,5,5}{⽵、⾔、⽊}[HSK 2]
  \definition[个,本]{s.}{caderno para anotações | \emph{laptop}; refere-se a um computador portátil}
  \definition{s.}{\emph{laptop}}
\end{EntryWithPhonetic}

\begin{EntryWithPhonetic}{笔试}{bi3 shi4}{10,8}{⽵、⾔}[HSK 6]
  \definition{s.}{exame escrito; um tipo de exame que exige respostas escritas; diferente de 口试}
  \seealsoref{口试}{kou3 shi4}
\end{EntryWithPhonetic}

\begin{EntryWithPhonetic}{必}{bi4}{5}{⼼}[HSK 5]
  \definition{adv.}{certamente; necessariamente; indica que algo é certo ou que alguém acredita que esteja correto | deve; tem que}
\end{EntryWithPhonetic}

\begin{EntryWithPhonetic}{必定}{bi4ding4}{5,8}{⼼、⼧}
  \definition{adv.}{sem falta | certamente | com certeza | definitivamente | inevitavelmente | com determinação}
  \definition{v.}{estar vinculado a | ter certeza de}
\end{EntryWithPhonetic}

\begin{EntryWithPhonetic}{必将}{bi4 jiang1}{5,9}{⼼、⼨}[HSK 6]
  \definition{adv.}{certamente; certamente irá; usado para expressar inevitabilidade (ou necessidade)}
\end{EntryWithPhonetic}

\begin{EntryWithPhonetic}{必然}{bi4ran2}{5,12}{⼼、⽕}[HSK 3]
  \definition{adj.}{certo; inevitável; necessário; definido e inalterável; imutável}
  \definition{adv.}{inevitavelmente}
  \definition{s.}{necessidade; em filosofia, refere-se às leis objetivas do desenvolvimento que não são influenciadas pela vontade humana}
\end{EntryWithPhonetic}

\begin{EntryWithPhonetic}{必修}{bi4 xiu1}{5,9}{⼼、⼈}[HSK 6]
  \definition{adj.}{(de um curso acadêmico) obrigatório; compulsório; mandatório; obrigatório estudar de acordo com os regulamentos (em oposição a 选修)}
  \seealsoref{选修}{xuan3 xiu1}
\end{EntryWithPhonetic}

\begin{EntryWithPhonetic}{必须}{bi4xu1}{5,9}{⼼、⾴}[HSK 2]
  \definition{adv.}{necessariamente; obrigatoriamente; indica a necessidade lógica e emocional | deve; tem que; é obrigado a}
\end{EntryWithPhonetic}

\begin{EntryWithPhonetic}{必需}{bi4 xu1}{5,14}{⼼、⾬}[HSK 5]
  \definition{adj.}{essencial; indispensável}
  \definition{v.}{ser essencial; ser indispensável}
\end{EntryWithPhonetic}

\begin{EntryWithPhonetic}{必要}{bi4yao4}{5,9}{⼼、⾑}[HSK 3]
  \definition{adj.}{necessário; essencial; indispensável}
  \definition[个,些]{s.}{necessidade; características indispensáveis}
\end{EntryWithPhonetic}

\begin{EntryWithPhonetic}{毕}{bi4}{6}{⽐}
  \definition*{s.}{Bi, uma das mansões lunares; a décima nona das vinte e oito constelações em que a esfera celeste foi dividida, consistindo de oito estrelas, seis em Híades e duas em Touro | Sobrenome Bi}
  \definition{adv.}{tudo; completamente; totalmente}
  \definition{v.}{terminar; realizar; concluir  | completar; terminar}
\end{EntryWithPhonetic}

\begin{EntryWithPhonetic}{毕竟}{bi4jing4}{6,11}{⽐、⾳}[HSK 5]
  \definition{adv.}{afinal de contas; quando tudo estiver dito e feito; em última análise; indica um resultado que não pode ser alterado, enfatizando que se trata de uma causa ou fato que precisa ser enfocado para referência | significa 到底, 究竟, 终究, indicando a conclusão final alcançada}
  \seealsoref{到底}{dao4di3}
  \seealsoref{究竟}{jiu1jing4}
  \seealsoref{终究}{zhong1jiu1}
\end{EntryWithPhonetic}

\begin{EntryWithPhonetic}{毕业}{bi4ye4}{6,5}{⽐、⼀}[HSK 4]
  \definition{v.+compl.}{formar-se}
\end{EntryWithPhonetic}

\begin{EntryWithPhonetic}{毕业生}{bi4 ye4 sheng1}{6,5,5}{⽐、⼀、⽣}[HSK 4]
  \definition[个,名,位,些]{s.}{diplomado; graduado; bacharel; pessoa que recebeu um diploma, grau ou certificado}
\end{EntryWithPhonetic}

\begin{EntryWithPhonetic}{闭}{bi4}{6}{⾨}[HSK 6]
  \definition*{s.}{Sobrenome Bi}
  \definition{v.}{fechar; encerrar | bloquear; obstruir; parar}
\end{EntryWithPhonetic}

\begin{EntryWithPhonetic}{闭幕}{bi4 mu4}{6,13}{⾨、⼱}[HSK 5]
  \definition{v.+compl.}{fechar; concluir; (conferência, exposição, etc.) terminar | cair a cortina; abaixar a cortina; terminar a apresentação e a cortina se fechar em frente ao palco}
\end{EntryWithPhonetic}

\begin{EntryWithPhonetic}{闭幕式}{bi4 mu4 shi4}{6,13,6}{⾨、⼱、⼷}[HSK 5]
  \definition{s.}{cerimônia de encerramento; cerimônia formal realizada no final de uma conferência ou exposição}
\end{EntryWithPhonetic}

\begin{EntryWithPhonetic}{闭嘴}{bi4zui3}{6,16}{⾨、⼝}
  \definition{expr.}{Cale-se!; Pare de falar!}
\end{EntryWithPhonetic}

\begin{EntryWithPhonetic}{秘}{bi4}{10}{⽲}
  \definition*{s.}{Abreviação de Peru, 秘鲁 | Sobrenome Bi}
  \seeref{秘}{mi4}
  \seealsoref{秘鲁}{bi4lu3}
\end{EntryWithPhonetic}

\begin{EntryWithPhonetic}{秘鲁}{bi4lu3}{10,12}{⽲、⿂}
  \definition*{s.}{Peru}
\end{EntryWithPhonetic}

\begin{EntryWithPhonetic}{壁}{bi4}{16}{⼟}
  \definition*{s.}{Bi, a décima quarta das vinte e oito constelações em que a esfera celeste foi dividida, consistindo em duas estrelas em linha reta, uma em Pégaso e a outra em Andrômeda | A Estrela Bìxìu, uma das Vinte e Oito Mansões da astronomia tradicional chinesa}
  \definition[道]{s.}{parede | superfície plana como uma parede | penhasco | muralha; parapeito | barreira}
\end{EntryWithPhonetic}

\begin{EntryWithPhonetic}{壁虎}{bi4hu3}{16,8}{⼟、⾌}
  \definition{s.}{lagartixa}
\end{EntryWithPhonetic}

\begin{EntryWithPhonetic}{壁纸}{bi4zhi3}{16,7}{⼟、⽷}
  \definition{s.}{papel de parede; papel colado em paredes internas para decoração ou proteção, com diversos tipos e cores}
\end{EntryWithPhonetic}

\begin{EntryWithPhonetic}{避}{bi4}{16}{⾌}[HSK 4]
  \definition{v.}{evitar; evadir; esquivar-se; buscar abrigo; fugir | impedir; manter afastado; repelir; previnir}
\end{EntryWithPhonetic}

\begin{EntryWithPhonetic}{避免}{bi4mian3}{16,7}{⾌、⼉}[HSK 4]
  \definition{v.}{evitar; desviar; abster-se de; tentar não fazer com que algo aconteça; prevenir; tentar impedir (que algo ruim aconteça) com antecedência}
\end{EntryWithPhonetic}

\begin{EntryWithPhonetic}{边}{bian1}{5}{⾡}[HSK 2]
  \definition*{s.}{Sobrenome Bian}
  \definition{adv.}{dois ou mais 边 são usados separadamente antes de diferentes verbos, indicando que diferentes ações ocorrem simultaneamente}
  \definition[条,个]{s.}{lado (de uma figura geométrica) | borda; lado; margem; aba; rebordo | fronteira; limite | ao lado de; lugar próximo a; perto de um objeto; lateral | aro; aba; borda; decoração em forma de faixa incrustada ou pintada na borda de um objeto}
  \definition{suf.}{lado; anexado a palavras de localização monossilábicas, formando palavras de localização dissílabas}
  \seeref{边}{bian5}
\end{EntryWithPhonetic}

\begin{EntryWithPhonetic}{边防}{bian1fang2}{5,6}{⾡、⾩}
  \definition{s.}{defesa da fronteira}
\end{EntryWithPhonetic}

\begin{EntryWithPhonetic}{边关}{bian1guan1}{5,6}{⾡、⼋}
  \definition{s.}{posto de fronteira | posição defensiva estratégica na fronteira}
\end{EntryWithPhonetic}

\begin{EntryWithPhonetic}{边境}{bian1jing4}{5,14}{⾡、⼟}[HSK 5]
  \definition{s.}{fronteira; borda; perto da fronteira}
\end{EntryWithPhonetic}

\begin{EntryWithPhonetic}{边缘}{bian1yuan2}{5,12}{⾡、⽷}[HSK 6]
  \definition{s.}{borda; beira; franja; uma área ou objeto próximo ao extremo |  borda; beira; algo está muito próximo de uma situação perigosa | interdisciplinar; relacionado a muitas coisas}
\end{EntryWithPhonetic}

\begin{EntryWithPhonetic}{编}{bian1}{12}{⽷}[HSK 4]
  \definition*{s.}{Sobrenome Bian}
  \definition{s.}{livro; volume; parte de um livro | organização e pessoal; estabelecimento}
  \definition{v.}{tecer; trançar; entrançar | fazer uma lista; organizar em uma lista; organizar; agrupar | editar; compilar | compor; escrever | fabricar; inventar; fazer; preparar}
\end{EntryWithPhonetic}

\begin{EntryWithPhonetic}{编程}{bian1cheng2}{12,12}{⽷、⽲}
  \definition{v.}{programar computador}
\end{EntryWithPhonetic}

\begin{EntryWithPhonetic}{编辑}{bian1ji2}{12,13}{⽷、⾞}[HSK 5]
  \definition[名,位,个]{s.}{editor; compilador; uma pessoa que organiza e processa dados ou trabalhos existentes}
  \definition{v.}{editar; compilar; organizar e processar dados ou trabalhos existentes}
  \seeref{编辑}{bian1ji5}
\end{EntryWithPhonetic}

\begin{EntryWithPhonetic}{编辑}{bian1ji5}{12,13}{⽷、⾞}[HSK 5]
  \definition{s.}{editor; compilador; pessoa que organiza e processa dados ou trabalhos existentes}
  \seeref{编辑}{bian1ji2}
\end{EntryWithPhonetic}

\begin{EntryWithPhonetic}{编制}{bian1 zhi4}{12,8}{⽷、⼑}[HSK 6]
  \definition{s.}{estabelecimento; organização e pessoal; refere-se à estrutura organizacional de uma unidade, cotas de pessoal, alocação de tarefas, etc.}
  \definition{v.}{tecer; trançar; entrelaçar tiras de vime, salgueiro, bambu, etc. para fazer objetos | resolver; realizar; elaborar; fazer de acordo com os dados (procedimentos, planos, etc.)}
\end{EntryWithPhonetic}

\begin{EntryWithPhonetic}{邉}{bian1}{17}{⾡}
  \variantof{边}
\end{EntryWithPhonetic}

\begin{EntryWithPhonetic}{扁}{bian3}{9}{⼾}[HSK 6]
  \definition{adj.}{plano}
  \definition{v.}{(coloquial)  bater em alguém}
  \seeref{扁}{pian1}
\end{EntryWithPhonetic}

\begin{EntryWithPhonetic}{变}{bian4}{8}{⼜}[HSK 2]
  \definition{adj.}{alterado; mutável; que pode mudar; que está mudando ou já mudou}
  \definition{s.}{uma reviravolta inesperada nos acontecimentos; mudanças significativas repentinas}
  \definition{v.}{mudar; tornar-se diferente; fazer mudanças | tornar-se; transformar-se; natureza, estado ou situação diferentes dos originais | alterar; mudar; transformar}
\end{EntryWithPhonetic}

\begin{EntryWithPhonetic}{变成}{bian4 cheng2}{8,6}{⼜、⼽}[HSK 2]
  \definition{v.}{crescer; tornar-se; fazer; desenvolver-se; revelar-se; resultar; acontecer; passar a ser; passar para; acumular-se; converter-se; transformar-se; transformar-se em; mudar-se em; transformação da situação ou condição anterior para a situação ou condição atual}
\end{EntryWithPhonetic}

\begin{EntryWithPhonetic}{变动}{bian4 dong4}{8,6}{⼜、⼒}[HSK 5]
  \definition{s.}{mudança; alteração; oscilação; modificação; variação}
  \definition{v.}{mudar; alterar; oscilar; modificar; variar}
\end{EntryWithPhonetic}

\begin{EntryWithPhonetic}{变更}{bian4 geng1}{8,7}{⼜、⽈}[HSK 6]
  \definition{v.}{alterar; mudar; modificar}
\end{EntryWithPhonetic}

\begin{EntryWithPhonetic}{变化}{bian4hua4}{8,4}{⼜、⼔}[HSK 3]
  \definition[个]{s.}{mudança; variação; a nova situação após uma mudança em pessoas ou coisas}
  \definition{v.}{mudar;  variar}
\end{EntryWithPhonetic}

\begin{EntryWithPhonetic}{变换}{bian4 huan4}{8,10}{⼜、⼿}[HSK 6]
  \definition{v.}{variar; alternar; mudar a forma ou o conteúdo de algo de uma coisa para outra}
\end{EntryWithPhonetic}

\begin{EntryWithPhonetic}{变节}{bian4jie2}{8,5}{⼜、⾋}
  \definition{s.}{traição | deserção | vira-casaca}
  \definition{v.}{retratar-se e submeter-se; renunciar e render-se | mudar de lado politicamente}
\end{EntryWithPhonetic}

\begin{EntryWithPhonetic}{变迁}{bian4qian1}{8,6}{⼜、⾡}
  \definition{s.}{mudanças; transição; vicissitudes; mudança em tendências ou condições; mudança de situação ou estágio}
\end{EntryWithPhonetic}

\begin{EntryWithPhonetic}{变数}{bian4shu4}{8,13}{⼜、⽁}
  \definition{s.}{(matemática) variável | fatores variáveis}
\end{EntryWithPhonetic}

\begin{EntryWithPhonetic}{变为}{bian4 wei2}{8,4}{⼜、⼂}[HSK 3]
  \definition{v.}{transformar-se em; tornar-se | mudar para}
\end{EntryWithPhonetic}

\begin{EntryWithPhonetic}{变心}{bian4xin1}{8,4}{⼜、⼼}
  \definition{v.+compl.}{deixar de ser fiel}
\end{EntryWithPhonetic}

\begin{EntryWithPhonetic}{变形}{bian4 xing2}{8,7}{⼜、⼺}[HSK 6]
  \definition{v.+compl.}{deformar; ficar fora de forma | transformar; transformar-se em outras formas}
\end{EntryWithPhonetic}

\begin{EntryWithPhonetic}{变性}{bian4xing4}{8,8}{⼜、⼼}
  \definition{s.}{desnaturação | transexual}
  \definition{v.}{desnaturar | mudar de sexo}
\end{EntryWithPhonetic}

\begin{EntryWithPhonetic}{变异}{bian4yi4}{8,6}{⼜、⼶}
  \definition{s.}{variação; mutação; muta; diferenças nas características morfológicas e fisiológicas entre gerações da mesma espécie ou entre indivíduos da mesma geração}
  \definition{v.}{variar; mudar}
\end{EntryWithPhonetic}

\begin{EntryWithPhonetic}{变装}{bian4zhuang1}{8,12}{⼜、⾐}
  \definition{v.}{trocar de roupa | vestir-se | vestir uma fantasia | disfarçar-se ou fantasiar-se de personagem real ou ficcional, \emph{cosplay} | travestir-se}
\end{EntryWithPhonetic}

\begin{EntryWithPhonetic}{便}{bian4}{9}{⼈}[HSK 6]
  \definition{adj.}{prático; conveniente | simples; comum; informal}
  \definition{adv.}{então; apenas no caso de; mesmo significado e uso de 就}
  \definition{conj.}{mesmo que; expressa uma concessão hipotética}
  \definition{s.}{facilidade; conveniência; o momento certo; a oportunidade | fezes ou urina}
  \definition{v.}{aliviar-se; excretar fezes e urina}
  \seeref{便}{pian2}
  \seealsoref{就}{jiu4}
\end{EntryWithPhonetic}

\begin{EntryWithPhonetic}{便利}{bian4li4}{9,7}{⼈、⼑}[HSK 5]
  \definition{adj.}{fácil; conveniente}
  \definition{s.}{facilidade; conveniência; coisas ou condições convenientes}
  \definition{v.}{facilitar; fornecer ajuda para que os outros se sintam confortáveis}
\end{EntryWithPhonetic}

\begin{EntryWithPhonetic}{便是}{bian4 shi4}{9,9}{⼈、⽇}[HSK 6]
  \definition{adv.}{exatamente; precisamente; para expressar afirmação ou ênfase}
  \definition{conj.}{mesmo; mesmo que; usado para introduzir um caso extremo hipotético, enfatizando que o mesmo resultado ocorreria em circunstâncias tão extremas, sem mencionar circunstâncias normais; você também pode usar 即便是}
  \definition{part.}{usada no final de uma frase para expressar afirmação}
  \seealsoref{即便是}{ji2bian4 shi4}
\end{EntryWithPhonetic}

\begin{EntryWithPhonetic}{便条}{bian4tiao2}{9,7}{⼈、⽊}[HSK 5]
  \definition[张,个]{s.}{nota ou mensagem informal; geralmente uma mensagem ou notificação}
\end{EntryWithPhonetic}

\begin{EntryWithPhonetic}{便宜}{bian4yi2}{9,8}{⼈、⼧}
  \definition{adj.}{prático; conveniente; adequado}
  \seeref{便宜}{pian2yi5}
\end{EntryWithPhonetic}

\begin{EntryWithPhonetic}{便于}{bian4yu2}{9,3}{⼈、⼆}[HSK 5]
  \definition{v.}{ser fácil para; ser conveniente para (algo ou fazer algo)}
\end{EntryWithPhonetic}

\begin{EntryWithPhonetic}{遍}{bian4}{12}{⾡}[HSK 2]
  \definition{adv.}{por toda parte; em toda parte; em todos os lugares}
  \definition{clas.}{usado para a repetição de ações de leitura, fala ou escrita}
\end{EntryWithPhonetic}

\begin{EntryWithPhonetic}{遍地}{bian4 di4}{12,6}{⾡、⼟}[HSK 6]
  \definition{adv.}{em todos os lugares; em toda parte; por toda parte}
\end{EntryWithPhonetic}

\begin{EntryWithPhonetic}{辩}{bian4}{16}{⾟}
  \definition{v.}{argumentar; disputar; debater}
\end{EntryWithPhonetic}

\begin{EntryWithPhonetic}{辩论}{bian4lun4}{16,6}{⾟、⾔}[HSK 4]
  \definition{v.}{debater; obter um entendimento unificado ou correto, ambos os lados usam linguagem, palavras etc. para explicar seus pontos de vista, apontar os erros ou as contradições do outro lado}
\end{EntryWithPhonetic}

\begin{EntryWithPhonetic}{辫}{bian4}{17}{⾟}
  \definition{s.}{trança; rabo de cavalo | para coisas como uma trança}
\end{EntryWithPhonetic}

\begin{EntryWithPhonetic}{辫子}{bian4zi5}{17,3}{⾟、⼦}
  \definition[根,条]{s.}{trança | um erro ou falha que pode ser explorado por um oponente | alça}
\end{EntryWithPhonetic}

\begin{EntryWithPhonetic}{边}{bian5}{5}{⾡}
  \definition{suf.}{sufixo de uma palavra de localidade (lado); indica posição e direção, usado após palavras que indicam direção, como 上, 下, 前, 后, 左, 右}
  \seeref{边}{bian1}
\end{EntryWithPhonetic}

\begin{EntryWithPhonetic}{标}{biao1}{9}{⽊}
  \definition{clas.}{usado para equipes (o numeral é limitado a um, 一, o que é comum no chinês moderno)}
  \definition[个]{s.}{copa da árvore (significado original) | marca; sinal | padrão; cota | sinal externo; sintoma | prêmio; troféu | oferta; licitação comercial pública | a ponta de uma árvore | aparência externa; ramos ou superfícies | partes aéreas das plantas | rótulo; etiqueta; identificação; sinal | regimento na Dinastia Qing; uma das organizações militares no final da Dinastia Qing}
  \definition{v.}{colocar uma marca, etiqueta ou rótulo em; rotular | agrupar; formar equipe | marcar; expressar com palavras ou outras coisas |}
\end{EntryWithPhonetic}

\begin{EntryWithPhonetic}{标题}{biao1ti2}{9,15}{⽊、⾴}[HSK 3]
  \definition[个,条,篇]{s.}{título; manchete; cabeçalho; resumo conciso do conteúdo da obra}
\end{EntryWithPhonetic}

\begin{EntryWithPhonetic}{标志}{biao1zhi4}{9,7}{⽊、⼼}[HSK 4]
  \definition[个,种]{s.}{sinal; marca; logotipo; símbolo; emblema; marcações que caracterizam um objeto para facilitar a identificação}
  \definition{v.}{marcar; indicar; simbolizar; identificar}
\end{EntryWithPhonetic}

\begin{EntryWithPhonetic}{标准}{biao1zhun3}{9,10}{⽊、⼎}[HSK 3]
  \definition{adj.}{padrão (que serve como ou está em conformidade com um padrão); em conformidade com os documentos e princípios regulamentares}
  \definition[个,条,项,种]{s.}{padrão; critério; critérios de avaliação das coisas}
\end{EntryWithPhonetic}

\begin{EntryWithPhonetic}{髟}{biao1}{10}{⾽}[Kangxi 190]
  \definition{adj.}{(de cabelo) solto, caído}
\end{EntryWithPhonetic}

\begin{EntryWithPhonetic}{彪}{biao1}{11}{⾌}
  \definition*{s.}{Sobrenome Biao}
  \definition{adj.}{semelhante a um tigre (metáfora para estatura alta)}
  \definition{s.}{tigre jovem}
\end{EntryWithPhonetic}

\begin{EntryWithPhonetic}{镖}{biao1}{16}{⾦}
  \definition{s.}{dardo | arma de arremesso | mercadorias enviadas sob a proteção de uma escolta armada}
\end{EntryWithPhonetic}

\begin{EntryWithPhonetic}{表}{biao3}{8}{⾐}[HSK 2]
  \definition*{s.}{Sobrenome Biao}
  \definition{s.}{exterior; superfície; externo | a relação entre os filhos ou netos de um irmão e uma irmã ou de irmãs | modelo; exemplo; padrão | memorial a um imperador; um tipo de petição da antiguidade, frequentemente usado para expressar intenções; mais tarde, também usado para expressar opiniões sobre eventos importantes | formulário; lista; gráfico; tabela | medidor; instrumento para medir uma determinada quantidade | relógio; um dispositivo para medir o tempo, menor que um relógio, que geralmente pode ser carregado no bolso | medidor de luz solar; antiga vara de madeira para medir o tempo através da sombra do sol | coluna usada antigamente para marcação}
  \definition{v.}{mostrar; expressar; expressar ideias, pensamentos, sentimentos, etc. | administrar medicamentos para aliviar o resfriado; na medicina tradicional chinesa refere-se ao uso de medicamentos para dissipar o frio e o vento que afetam o corpo}
\end{EntryWithPhonetic}

\begin{EntryWithPhonetic}{表白}{biao3bai2}{8,5}{⾐、⽩}
  \definition{s.}{declaração | confissão}
  \definition{v.}{confessar a si mesmo | expressar | revelar pensamentos ou sentimentos de alguém}
\end{EntryWithPhonetic}

\begin{EntryWithPhonetic}{表达}{biao3da2}{8,6}{⾐、⾡}[HSK 3]
  \definition{v.}{entregar; expressar; mostrar; manifestar; transmitir; comunicar; refere-se ao processo de transmitir pensamentos, sentimentos ou opiniões pessoais a outras pessoas por meio de linguagem, texto, ações, etc.}
\end{EntryWithPhonetic}

\begin{EntryWithPhonetic}{表格}{biao3ge2}{8,10}{⾐、⽊}[HSK 3]
  \definition[张,份,个]{s.}{tabela; formulário}
\end{EntryWithPhonetic}

\begin{EntryWithPhonetic}{表面}{biao3mian4}{8,9}{⾐、⾯}[HSK 3]
  \definition{s.}{superfície; face; exterior; aparência | aparência; superficialidade | mostrador (placa); mostrador do relógio | aparência; a aparência externa das coisas ou a parte não essencial delas}
\end{EntryWithPhonetic}

\begin{EntryWithPhonetic}{表面上}{biao3 mian4 shang4}{8,9,3}{⾐、⾯、⼀}[HSK 6]
  \definition{adj.}{superficial; ostensivo; aparente}
\end{EntryWithPhonetic}

\begin{EntryWithPhonetic}{表明}{biao3ming2}{8,8}{⾐、⽇}[HSK 3]
  \definition{v.}{indicar; demonstrar; expressar; marcar; expressar claramente; expressar de forma clara}
\end{EntryWithPhonetic}

\begin{EntryWithPhonetic}{表情}{biao3qing2}{8,11}{⾐、⼼}[HSK 4]
  \definition[个,种,幅]{s.}{expressão; expressão facial; expressão de pensamentos e sentimentos internos por meio de mudanças faciais ou de gestos}
  \definition{v.}{expressar pensamentos e sentimentos internos por meio de mudanças faciais ou de gestos}
\end{EntryWithPhonetic}

\begin{EntryWithPhonetic}{表示}{biao3shi4}{8,5}{⾐、⽰}[HSK 2]
  \definition{s.}{expressão; indicação}
  \definition{v.}{mostrar; expressar; indicar | significar | expressar pensamentos e sentimentos através de palavras, ações ou expressões faciais}
\end{EntryWithPhonetic}

\begin{EntryWithPhonetic}{表现}{biao3xian4}{8,8}{⾐、⾒}[HSK 3]
  \definition[个,种,份]{s.}{desempenho; expressão; manifestação; comportamento; as ideias, o estilo, as qualidades, o nível ou as capacidades demonstrados em ação.}
  \definition{v.}{mostrar; expressar; exibir; manifestar; descrever; demonstrar algum tipo de pensamento, espírito, qualidade, sentimento ou habilidade, etc. | exibir-se; demonstrar de forma inadequada e intencional alguma habilidade, ponto forte ou vantagem.}
\end{EntryWithPhonetic}

\begin{EntryWithPhonetic}{表演}{biao3yan3}{8,14}{⾐、⽔}[HSK 3]
  \definition[场]{s.}{performance; exposição; refere-se às atividades expressas pelos atores por meio da linguagem, voz, expressões faciais, instrumentos musicais ou movimentos}
  \definition{v.}{atuar; representar; interpretar | demonstrar; fazer demonstrações | fingir; agir de forma afetada; metáfora para fingir deliberadamente uma determinada atitude para enganar alguém}
\end{EntryWithPhonetic}

\begin{EntryWithPhonetic}{表演赛}{biao3yan3sai4}{8,14,14}{⾐、⽔、⾙}
  \definition{s.}{partida de exibição; jogo de exibição; uma competição realizada para celebração, comemoração, demonstração, publicidade, etc.}
\end{EntryWithPhonetic}

\begin{EntryWithPhonetic}{表演特技}{biao3yan3 te4ji4}{8,14,10,7}{⾐、⽔、⽜、⼿}
  \definition{s.}{acrobacia | pirueta | façanha}
\end{EntryWithPhonetic}

\begin{EntryWithPhonetic}{表演艺术}{biao3yan3 yi4shu4}{8,14,4,5}{⾐、⽔、⾋、⽊}
  \definition{s.}{arte performática}
\end{EntryWithPhonetic}

\begin{EntryWithPhonetic}{表演游戏}{biao3yan3 you2xi4}{8,14,12,6}{⾐、⽔、⽔、⼽}
  \definition{s.}{exibição dramática}
\end{EntryWithPhonetic}

\begin{EntryWithPhonetic}{表演者}{biao3yan3 zhe3}{8,14,8}{⾐、⽔、⽼}
  \definition{s.}{artista; intérprete}
\end{EntryWithPhonetic}

\begin{EntryWithPhonetic}{表扬}{biao3yang2}{8,6}{⾐、⼿}[HSK 4]
  \definition{v.}{elogiar; louvar; elogiar publicamente as pessoas boas e as boas ações}
\end{EntryWithPhonetic}

\begin{EntryWithPhonetic}{表扬信}{biao3yang2 xin4}{8,6,9}{⾐、⼿、⼈}
  \definition{s.}{carta de elogio; depoimento}
\end{EntryWithPhonetic}

\begin{EntryWithPhonetic}{别}{bie2}{7}{⼑}[HSK 1,4]
  \definition*{s.}{Sobrenome Bie}
  \definition{adv.}{não; nada de (pedir a alguém para não fazer); é melhor não | talvez, usado em conjunto com a palavra 是 para indicar especulação}
  \definition{pron.}{outro; algum outro}
  \definition{s.}{distinção; diferença | classificação}
  \definition{v.}{deixar; partir; separar | diferenciar; distinguir; encontrar aspectos diferentes | fixar objetos com pinos | girar; virar | aderir; colar; preder}
  \seeref{别}{bie4}
  \seealsoref{是}{shi4}
\end{EntryWithPhonetic}

\begin{EntryWithPhonetic}{别不过}{bie2 bu2guo4}{7,4,6}{⼑、⼀、⾡}
  \definition{expr.}{Não, mas}
\end{EntryWithPhonetic}

\begin{EntryWithPhonetic}{别的}{bie2 de5}{7,8}{⼑、⽩}[HSK 1]
  \definition{pron.}{outro; o resto}
\end{EntryWithPhonetic}

\begin{EntryWithPhonetic}{别人}{bie2 ren2}{7,2}{⼑、⼈}[HSK 1]
  \definition{pron.}{outros; outras pessoas}
  \definition{s.}{outros; pessoas; outras pessoas; refere-se a alguém diferente de si mesmo}
\end{EntryWithPhonetic}

\begin{EntryWithPhonetic}{别说}{bie2shuo1}{7,9}{⼑、⾔}
  \definition{v.}{não falar de | não mencionar}
\end{EntryWithPhonetic}

\begin{EntryWithPhonetic}{别}{bie4}{7}{⼑}
  \definition{v.}{fazer com que alguém mude seus hábitos, opiniões, etc. | mudar a opinião de alguém (usado principalmente em 别不过)}
  \seeref{别}{bie2}
  \seealsoref{别不过}{bie2 bu2guo4}
\end{EntryWithPhonetic}

\begin{EntryWithPhonetic}{宾}{bin1}{10}{⼧}
  \definition*{s.}{Sobrenome Bin}
  \definition[个,位,名,些]{s.}{convidado}
\end{EntryWithPhonetic}

\begin{EntryWithPhonetic}{宾馆}{bin1guan3}{10,11}{⼧、⾷}[HSK 5]
  \definition[家,个,座]{s.}{hotel; acomodações públicas para hóspedes}
\end{EntryWithPhonetic}

\begin{EntryWithPhonetic}{冰}{bing1}{6}{⼎}[HSK 4]
  \definition[块,层,些]{s.}{gelo; água em estado sólido |  algo parecido com gelo | (gíria) metanfetamina}
  \definition{v.}{colocar gelo; colocar gelo ao redor; colocar no gelo; resfriar objetos com gelo ou água fria | sentir frio}
\end{EntryWithPhonetic}

\begin{EntryWithPhonetic}{冰糕}{bing1gao1}{6,16}{⼎、⽶}
  \definition{s.}{sorvete | picolé}
\end{EntryWithPhonetic}

\begin{EntryWithPhonetic}{冰棍}{bing1gun4}{6,12}{⼎、⽊}
  \definition[根]{s.}{picolé}
\end{EntryWithPhonetic}

\begin{EntryWithPhonetic}{冰激凌}{bing1ji1ling2}{6,16,10}{⼎、⽔、⼎}
  \definition{s.}{sorvete}
\end{EntryWithPhonetic}

\begin{EntryWithPhonetic}{冰球}{bing1qiu2}{6,11}{⼎、⽟}
  \definition[个]{s.}{hóquei no gelo | disco; a ``bola'' usada no hóquei no gelo}
\end{EntryWithPhonetic}

\begin{EntryWithPhonetic}{冰天雪地}{bing1tian1-xue3di4}{6,4,11,6}{⼎、⼤、⾬、⼟}
  \definition{expr.}{um mundo de gelo e neve}
\end{EntryWithPhonetic}

\begin{EntryWithPhonetic}{冰箱}{bing1xiang1}{6,15}{⼎、⾋}[HSK 4]
  \definition[台,个]{s.}{geladeira; freezer; refrigerador; aparelhos para congelar alimentos ou medicamentos com gelo para mantê-los frios}
\end{EntryWithPhonetic}

\begin{EntryWithPhonetic}{冰雪}{bing1 xue3}{6,11}{⼎、⾬}[HSK 4]
  \definition{adj.}{puro como gelo e neve; descreve uma pessoa como pura}
  \definition[片,场]{s.}{gelo e neve}
\end{EntryWithPhonetic}

\begin{EntryWithPhonetic}{兵}{bing1}{7}{⼋}[HSK 4]
  \definition[个,种]{s.}{armas; armamentos | soldado; pessoal militar | exército; tropas | soldado raso; membro mais jovem do exército | assuntos militares (estratégia) | peão, uma das peças do xadrez chinês}
\end{EntryWithPhonetic}

\begin{EntryWithPhonetic}{兵器}{bing1qi4}{7,16}{⼋、⼝}
  \definition{s.}{armas | armamento}
\end{EntryWithPhonetic}

\begin{EntryWithPhonetic}{屏}{bing1}{9}{⼫}
  \definition{s.}{antigamente, referia-se à pequena parede de tela em frente ao portão de um antigo palácio; no chinês moderno, também é usado como uma palavra humilde para expressar o significado de 惶恐}
  \seeref{屏}{bing3}
  \seeref{屏}{ping2}
  \seealsoref{惶恐}{huang2kong3}
\end{EntryWithPhonetic}

\begin{EntryWithPhonetic}{屏}{bing3}{9}{⼫}
  \definition*{s.}{Sobrenome Bing}
  \definition{v.}{prender (a respiração); conter a respiração | rejeitar; livrar-se de; remover; pôr (colocar) de lado; abandonar; descartar}
  \seeref{屏}{bing1}
  \seeref{屏}{ping2}
\end{EntryWithPhonetic}

\begin{EntryWithPhonetic}{饼}{bing3}{9}{⾷}[HSK 5]
  \definition[张]{s.}{um bolo redondo e plano; massa assada ou cozida no vapor | algo que tem o formato de um bolo; semelhante a uma torta}
\end{EntryWithPhonetic}

\begin{EntryWithPhonetic}{饼干}{bing3gan1}{9,3}{⾷、⼲}[HSK 5]
  \definition[块,片,包,盒,袋]{s.}{biscoito; bolacha; \emph{cookie}; alimentos, pedaços pequenos e finos cozidos em farinha com açúcar, ovos, leite, etc.}
\end{EntryWithPhonetic}

\begin{EntryWithPhonetic}{并}{bing4}{6}{⼲}[HSK 3,4]
  \definition{adv.}{lado a lado; igualmente; simultaneamente | (usado para reforçar uma negação) na verdade; definitivamente | mesmo assim | (usado para reforçar uma negação) na verdade; de forma alguma | todos; indica o conjunto completo, equivalente a 全部}
  \definition{conj.}{e; além disso}
  \definition{v.}{combinar; fundir; incorporar | ficar (ou colocar) lado a lado | estar paralelo a | anexar; juntar}
  \seealsoref{全部}{quan2bu4}
\end{EntryWithPhonetic}

\begin{EntryWithPhonetic}{并排}{bing4pai2}{6,11}{⼲、⼿}
  \definition{adv.}{lado a lado}
\end{EntryWithPhonetic}

\begin{EntryWithPhonetic}{并且}{bing4qie3}{6,5}{⼲、⼀}[HSK 3]
  \definition{conj.}{e; bem como; usado entre verbos, adjetivos ou frases paralelas para indicar que várias ações são realizadas ao mesmo tempo ou que propriedades existem ao mesmo tempo | além disso; além do mais; ademais; usado na segunda metade de uma frase complexa para expressar um significado adicional}
\end{EntryWithPhonetic}

\begin{EntryWithPhonetic}{幷}{bing4}{8}{⼲}
  \variantof{并}
\end{EntryWithPhonetic}

\begin{EntryWithPhonetic}{倂}{bing4}{10}{⼈}
  \variantof{并}
\end{EntryWithPhonetic}

\begin{EntryWithPhonetic}{病}{bing4}{10}{⽧}[HSK 1]
  \definition[种]{s.}{doença; enfermidade | doença; males | falha; defeito; desvantagem; erro}
  \definition{v.}{adoecer; ficar doente | ferir; causar danos a | angustiar; desaprovar}
\end{EntryWithPhonetic}

\begin{EntryWithPhonetic}{病毒}{bing4du2}{10,9}{⽧、⽏}[HSK 5]
  \definition[种,株,类]{s.}{vírus; patógenos que são menores que os germes e visíveis somente com um microscópio eletrônico | Computação: vírus de computador}
\end{EntryWithPhonetic}

\begin{EntryWithPhonetic}{病房}{bing4 fang2}{10,8}{⽧、⼾}[HSK 6]
  \definition[个,间]{s.}{enfermaria de um hospital; quartos onde ficam os pacientes em hospitais e onde vivem em casas de repouso}
\end{EntryWithPhonetic}

\begin{EntryWithPhonetic}{病情}{bing4 qing2}{10,11}{⽧、⼼}[HSK 6]
  \definition{s.}{estado de uma doença; condição do paciente; mudanças na doença}
\end{EntryWithPhonetic}

\begin{EntryWithPhonetic}{病人}{bing4 ren2}{10,2}{⽧、⼈}[HSK 1]
  \definition[个,位]{s.}{doente; paciente; pessoas doentes; pessoas em tratamento}
\end{EntryWithPhonetic}

\begin{EntryWithPhonetic}{拨}{bo1}{8}{⼿}
  \definition{clas.}{usado para agrupar pessoas; grupo; lote}
  \definition{v.}{mover (mexer) com a mão, o pé, o bastão, etc.; usar as mãos, os pés ou os bastões para mover objetos | atribuir; alocar; reservar | virar-se; inverter a marcha | dedilhar (uma corda de violão) com os dedos ou com um instrumento | chamar (alguém)}
\end{EntryWithPhonetic}

\begin{EntryWithPhonetic}{拨打}{bo1 da3}{8,5}{⼿、⼿}[HSK 6]
  \definition{v.}{ligar; discar; de acordo com o número da chamada, discar o número no telefone ou pressionar as teclas numéricas para fazer uma chamada}
\end{EntryWithPhonetic}

\begin{EntryWithPhonetic}{拨转}{bo1zhuan3}{8,8}{⼿、⾞}
  \definition{v.}{transferir (fundos, etc.) | virar | dar a volta}
\end{EntryWithPhonetic}

\begin{EntryWithPhonetic}{波}{bo1}{8}{⽔}
  \definition*{s.}{Polônia, abreviação de 波兰 | Sobrenome Bo}
  \definition{s.}{ondas, a superfície irregular da água em rios, lagos e oceanos | onda, o processo de propagação da vibração | mudanças inesperadas; uma reviravolta inesperada nos acontecimentos; metáfora para mudanças inesperadas nas coisas | olhos; metáfora do olhar errante}
  \seealsoref{波兰}{bo1lan2}
\end{EntryWithPhonetic}

\begin{EntryWithPhonetic}{波动}{bo1 dong4}{8,6}{⽔、⼒}[HSK 6]
  \definition{s.}{ondulação; flutuação; movimento de onda}
  \definition{v.}{ondular; flutuar}
\end{EntryWithPhonetic}

\begin{EntryWithPhonetic}{波兰}{bo1lan2}{8,5}{⽔、⼋}
  \definition*{s.}{Polônia}
\end{EntryWithPhonetic}

\begin{EntryWithPhonetic}{波浪}{bo1lang4}{8,10}{⽔、⽔}[HSK 6]
  \definition{s.}{onda; a superfície irregular da água nos rios, lagos e oceanos, geralmente se refere a águas menores e mais bonitas, frequentemente usada na linguagem falada}
\end{EntryWithPhonetic}

\begin{EntryWithPhonetic}{波音}{bo1yin1}{8,9}{⽔、⾳}
  \definition*{s.}{Boeing (empresa aeroespacial)}
  \definition{s.}{mordente (música)}
\end{EntryWithPhonetic}

\begin{EntryWithPhonetic}{玻}{bo1}{9}{⽟}
  \definition{s.}{vidro}
\end{EntryWithPhonetic}

\begin{EntryWithPhonetic}{玻璃}{bo1li5}{9,14}{⽟、⽟}[HSK 5]
  \definition[张,块]{s.}{vidro; corpo duro, quebradiço e transparente, geralmente feito de areia, calcário, carbonato de sódio, etc. | \emph{nylon}; plástico; refere-se a determinados plásticos que se assemelham ao vidro.}
\end{EntryWithPhonetic}

\begin{EntryWithPhonetic}{般}{bo1}{10}{⾈}
  \definition{s.}{utilizado em 般若}
  \seealsoref{般若}{bo1re3}
\end{EntryWithPhonetic}

\begin{EntryWithPhonetic}{般若}{bo1re3}{10,8}{⾈、⾋}
  \definition*{s.}{Prajña (sânscrito), \emph{insight} sobre a verdadeira natureza da realidade}
  \definition{s.}{budismo: sabedoria}
\end{EntryWithPhonetic}

\begin{EntryWithPhonetic}{啵}{bo1}{11}{⼝}
  \definition{part.}{denotando pedido, comando, etc.; o uso é semelhante ao de 吧, que é mais comum no vernáculo antigo}
  \definition{v.aux.}{indicando uma sugestão, pedido ou comando suave | indicando consentimento ou aprovação | em uma pergunta tendenciosa que pede a confirmação de uma suposição | indicando alguma dúvida na mente do falante | marcando uma pausa após suposições como alternativas}
  \seeref{啵}{bo5}
  \seealsoref{吧}{ba5}
\end{EntryWithPhonetic}

\begin{EntryWithPhonetic}{菠}{bo1}{11}{⾋}
  \definition{s.}{espinafre}
\end{EntryWithPhonetic}

\begin{EntryWithPhonetic}{菠菜}{bo1cai4}{11,11}{⾋、⾋}
  \definition[棵]{s.}{espinafre}
\end{EntryWithPhonetic}

\begin{EntryWithPhonetic}{播}{bo1}{15}{⼿}[HSK 6]
  \definition{v.}{espalhar; transmitir | semear | mover-se; migrar; ir para o exílio}
\end{EntryWithPhonetic}

\begin{EntryWithPhonetic}{播出}{bo1 chu1}{15,5}{⼿、⼐}[HSK 3]
  \definition{v.}{radiodifundir; transmitir; estar no ar; transmitir via rádio e televisão}
\end{EntryWithPhonetic}

\begin{EntryWithPhonetic}{播放}{bo1fang4}{15,8}{⼿、⽅}[HSK 3]
  \definition{v.}{ir ao ar; transmitir por rádio | mostrar; exibir; transmitir (um programa de TV)}
\end{EntryWithPhonetic}

\begin{EntryWithPhonetic}{播音}{bo1yin1}{15,9}{⼿、⾳}
  \definition{s.}{transmissão}
  \definition{v.+compl.}{transmitir}
\end{EntryWithPhonetic}

\begin{EntryWithPhonetic}{蕃}{bo1}{15}{⾋}
  \definition[种]{s.}{estrangeiros}
  \seeref{蕃}{fan1}
  \seeref{蕃}{fan2}
\end{EntryWithPhonetic}

\begin{EntryWithPhonetic}{柏}{bo2}{9}{⽊}
  \definition{s.}{cipreste | usado para transcrever nomes}[柏林,德国城市名。===Berlim, uma cidade alemã.]
  \seeref{柏}{bai3}
  \seeref{柏}{bo4}
\end{EntryWithPhonetic}

\begin{EntryWithPhonetic}{柏林}{bo2lin2}{9,8}{⽊、⽊}
  \definition*{s.}{Berlim, capital da Alemanha}
\end{EntryWithPhonetic}

\begin{EntryWithPhonetic}{脖}{bo2}{11}{⾁}
  \definition[个]{s.}{pescoço | em forma de pescoço | parte semelhante ao pescoço}
\end{EntryWithPhonetic}

\begin{EntryWithPhonetic}{脖子}{bo2zi5}{11,3}{⾁、⼦}
  \definition[个]{s.}{pescoço}
\end{EntryWithPhonetic}

\begin{EntryWithPhonetic}{博}{bo2}{12}{⼗}
  \definition*{s.}{Sobrenome Bo}
  \definition{adj.}{rico; abundante | erudito; bem informado | solto; grande | grande}
  \definition{s.}{doutor em filosofia; doutorado}
  \definition{v.}{ter um amplo conhecimento de; ser bem lido | ganhar; vencer | jogar}
\end{EntryWithPhonetic}

\begin{EntryWithPhonetic}{博客}{bo2 ke4}{12,9}{⼗、⼧}[HSK 5]
  \definition[个]{s.}{\emph{blog}; página da Web ou site gerenciado por um indivíduo, geralmente composto por postagens organizadas da mais recente para a mais antiga | blogueiro; \emph{blogger}; pessoas que possuem ou escrevem \emph{blogs}}
\end{EntryWithPhonetic}

\begin{EntryWithPhonetic}{博览会}{bo2lan3hui4}{12,9,6}{⼗、⾒、⼈}[HSK 5]
  \definition[次,届]{s.}{exposição; feira internacional; exposições de produtos em grande escala}
\end{EntryWithPhonetic}

\begin{EntryWithPhonetic}{博士}{bo2shi4}{12,3}{⼗、⼠}[HSK 5]
  \definition[位,名,个,些]{s.}{doutorado; grau de doutor; nível mais alto de um diploma; também, uma pessoa que obteve esse diploma | doutor; antigo título honorífico para uma pessoa que é habilidosa em um determinado ofício ou especializada em uma determinada ocupação | doutor; autoridades que ensinavam as escrituras na China nos tempos antigos}
\end{EntryWithPhonetic}

\begin{EntryWithPhonetic}{博文}{bo2wen2}{12,4}{⼗、⽂}
  \definition{s.}{artigo em um blog}
  \definition{v.}{escrever um artigo em um blog}
\end{EntryWithPhonetic}

\begin{EntryWithPhonetic}{博物馆}{bo2wu4guan3}{12,8,11}{⼗、⽜、⾷}[HSK 5]
  \definition[座,个]{s.}{museu; locais para coleta, armazenamento, pesquisa, exibição e exposição de relíquias culturais ou espécimes relacionados à história, cultura, arte, ciências naturais, ciência e tecnologia, etc.}
\end{EntryWithPhonetic}

\begin{EntryWithPhonetic}{博主}{bo2zhu3}{12,5}{⼗、⼂}
  \definition{s.}{blogueiro}
\end{EntryWithPhonetic}

\begin{EntryWithPhonetic}{薄}{bo2}{16}{⾋}
  \definition*{s.}{Sobrenome Bo}
  \definition{adj.}{pequeno; leve; magro | mau; cruel; mesquinho | frívolo; fútil; não solene | fraco; frágil}
  \definition{v.}{desprezar; tratar com desprezo; menosprezar | aproximar-se}
  \seeref{薄}{bao2}
  \seeref{薄}{bo4}
\end{EntryWithPhonetic}

\begin{EntryWithPhonetic}{薄弱}{bo2ruo4}{16,10}{⾋、⼸}[HSK 5]
  \definition{adj.}{fraco; frágil; não é firme; não é sólido}
\end{EntryWithPhonetic}

\begin{EntryWithPhonetic}{柏}{bo4}{9}{⽊}
  \definition{s.}{cedro; cipreste amarelo}
  \seeref{柏}{bai3}
  \seeref{柏}{bo2}
\end{EntryWithPhonetic}

\begin{EntryWithPhonetic}{薄}{bo4}{16}{⾋}
  \definition{s.}{menta; uma erva perene com aroma refrescante nos caules e folhas}
  \seeref{薄}{bao2}
  \seeref{薄}{bo2}
\end{EntryWithPhonetic}

\begin{EntryWithPhonetic}{啵}{bo5}{11}{⼝}
  \definition{part.}{partícula gramaticalmente equivalente a 吧}
  \seeref{啵}{bo1}
  \seealsoref{吧}{ba5}
\end{EntryWithPhonetic}

\begin{EntryWithPhonetic}{不}{bu2}[(antes de quarto tom)]{4}{⼀}[HSK 1]
  \seeref{不}{bu4}
  \seeref{不}{bu5}
\end{EntryWithPhonetic}

\begin{EntryWithPhonetic}{不必}{bu2 bi4}{4,5}{⼀、⼼}[HSK 3]
  \definition{adv.}{não precisa; não tem que; indica que não é necessário em termos de razão ou emoção}
\end{EntryWithPhonetic}

\begin{EntryWithPhonetic}{不便}{bu2 bian4}{4,9}{⼀、⼈}[HSK 6]
  \definition{adj.}{inconveniente; inapropriado | não ter dinheiro em mãos; estar com pouco dinheiro}
  \definition{v.}{inadequado fazer algo; indica que fazer algo é inapropriado ou inconveniente}
\end{EntryWithPhonetic}

\begin{EntryWithPhonetic}{不错}{bu2 cuo4}{4,13}{⼀、⾦}[HSK 2]
  \definition{adj.}{certo; correto | nada mal; muito bom}
\end{EntryWithPhonetic}

\begin{EntryWithPhonetic}{不大}{bu2 da4}{4,3}{⼀、⼤}[HSK 1]
  \definition{adv.}{não muito (indicando um grau baixo); não demasiado | não com frequência; raramente; dificilmente}
\end{EntryWithPhonetic}

\begin{EntryWithPhonetic}{不大离}{bu2da4li2}{4,3,10}{⼀、⼤、⼇}
  \definition{adj.}{bem perto | quase certo | nada mal}
\end{EntryWithPhonetic}

\begin{EntryWithPhonetic}{不但}{bu2 dan4}{4,7}{⼀、⼈}[HSK 2]
  \definition{conj.}{não só\dots mas também; usado na primeira parte de uma frase composta que expressa progressão, a segunda parte geralmente contém conjunções como 而且,  并且 ou advérbios como 也, 还 que correspondem à primeira parte}
  \seealsoref{并且}{bing4qie3}
  \seealsoref{而且}{er2 qie3}
  \seealsoref{还}{hai2}
  \seealsoref{也}{ye3}
\end{EntryWithPhonetic}

\begin{EntryWithPhonetic}{不但……而且……}{bu2 dan4 er2qie3}{4,7,6,5}{⼀、⼈、⽽、⼀}[HSK 2]
  \definition{conj.}{não só\dots mas também\dots}
\end{EntryWithPhonetic}

\begin{EntryWithPhonetic}{不到}{bu2dao4}{4,8}{⼀、⼑}
  \definition{adj.}{insuficiente}
  \definition{adv.}{menos que}
  \definition{v.}{não chegar}
\end{EntryWithPhonetic}

\begin{EntryWithPhonetic}{不断}{bu2duan4}{4,11}{⼀、⽄}[HSK 3]
  \definition{adv.}{incessantemente; ininterruptamente; continuamente; constantemente}
  \definition{v.}{continuar; enfatiza a continuação da ação}
\end{EntryWithPhonetic}

\begin{EntryWithPhonetic}{不对}{bu2 dui4}{4,5}{⼀、⼨}[HSK 1]
  \definition{adj.}{incorreto; errado | anormal; anômalo; estranho | desarmonia; incompatibilidade; discórdia}
\end{EntryWithPhonetic}

\begin{EntryWithPhonetic}{不够}{bu2 gou4}{4,11}{⼀、⼣}[HSK 2]
  \definition{adv.}{insuficiente; indica que não atingiu o nível esperado}
  \definition{v.}{não ser suficiente; indica que é inferior ao exigido em quantidade ou grau}
\end{EntryWithPhonetic}

\begin{EntryWithPhonetic}{不顾}{bu2gu4}{4,10}{⼀、⾴}[HSK 5]
  \definition{v.}{não considerar; desconsiderar | desconsiderar; não levar em consideração; ignorar; não se preocupar com}
\end{EntryWithPhonetic}

\begin{EntryWithPhonetic}{不过}{bu2guo4}{4,6}{⼀、⾡}[HSK 2]
  \definition{adv.}{apenas; meramente; nada mais do que; indica que não excede um determinado limite, equivalente a 仅 ou 只 | como intensificador após certos adjetivos}
  \definition{conj.}{mas; no entanto; apenas; usado no início da segunda parte da frase, indica o contrário do sentido anterior e modifica ou complementa o significado anterior}
\end{EntryWithPhonetic}

\begin{EntryWithPhonetic}{不计其数}{bu2 ji4 qi2 shu4}{4,4,8,13}{⼀、⾔、⼋、⽁}
  \definition{expr.}{seu número não pode ser contado; incontáveis; inumeráveis}
\end{EntryWithPhonetic}

\begin{EntryWithPhonetic}{不见}{bu2 jian4}{4,4}{⼀、⾒}[HSK 6]
  \definition{v.}{não ver; não conhecer; não encontrar | estar desaparecido; desaparecer; não consiguir encontrar algo}
\end{EntryWithPhonetic}

\begin{EntryWithPhonetic}{不客气}{bu2 ke4 qi5}{4,9,4}{⼀、⼧、⽓}[HSK 1]
  \definition{adj.}{rude; indelicado; duro | franco; sincero; direto}
  \definition{expr.}{de modo algum; não mencione isso; de nada}
  \definition{v.}{dizer palavras ou fazer gestos indelicados}
\end{EntryWithPhonetic}

\begin{EntryWithPhonetic}{不利}{bu2 li4}{4,7}{⼀、⼑}[HSK 5]
  \definition{adj.}{desfavorável; desvantajoso; nocivo; prejudicial | malsucedido}
\end{EntryWithPhonetic}

\begin{EntryWithPhonetic}{不料}{bu2liao4}{4,10}{⼀、⽃}[HSK 6]
  \definition{conj.}{inesperadamente; para surpresa de alguém}
\end{EntryWithPhonetic}

\begin{EntryWithPhonetic}{不论}{bu2 lun4}{4,6}{⼀、⾔}[HSK 3]
  \definition{conj.}{não importa (o que, quem, como, etc.); se \dots ou \dots; significa que as condições ou situações são diferentes, mas os resultados permanecem os mesmos; geralmente é seguido por palavras paralelas ou pronomes interrogativos; geralmente é seguido por advérbios como 都 e 总}
  \definition{v.}{não discutir nem argumentar; não discutir; não debater}
  \seealsoref{都}{dou1}
  \seealsoref{总}{zong3}
\end{EntryWithPhonetic}

\begin{EntryWithPhonetic}{不论……都……}{bu2lun4 dou1}{4,6,10}{⼀、⾔、⾢}
  \definition{conj.}{não apenas\dots, (o que, quem, como, etc.), \dots}
\end{EntryWithPhonetic}

\begin{EntryWithPhonetic}{不论……也……}{bu2lun4 ye3}{4,6,3}{⼀、⾔、⼄}
  \definition{conj.}{não apenas\dots, (o que, quem, como, etc.), \dots}
\end{EntryWithPhonetic}

\begin{EntryWithPhonetic}{不耐烦}{bu2nai4fan2}{4,9,10}{⼀、⽽、⽕}[HSK 5]
  \definition{adj.}{impaciente; significa não ser capaz de suportar coisas tediosas ou que causam distração}
\end{EntryWithPhonetic}

\begin{EntryWithPhonetic}{不日}{bu2ri4}{4,4}{⼀、⽇}
  \definition{adv.}{em alguns dias}
\end{EntryWithPhonetic}

\begin{EntryWithPhonetic}{不是话}{bu2shi4hua4}{4,9,8}{⼀、⽇、⾔}
  \definition{expr.}{inacreditável; além das palavras; (palavras) não fazem sentido}
  \seealsoref{不像话}{bu2xiang4hua4}
  \seealsoref{不成话}{bu4cheng2hua4}
\end{EntryWithPhonetic}

\begin{EntryWithPhonetic}{不太}{bu2 tai4}{4,4}{⼀、⼤}[HSK 2]
  \definition{adv.}{não exatamente | não muito bom}
\end{EntryWithPhonetic}

\begin{EntryWithPhonetic}{不像话}{bu2xiang4hua4}{4,13,8}{⼀、⼈、⾔}
  \definition{expr.}{absurdo; sem sentido; irracional; uma determinada prática ou afirmação não está de acordo com o senso comum ou a razão e parece irracional | chocante; ultrajante; uma ação, palavra ou situação tão extrema que não pode ser aceita ou tolerada}
  \seealsoref{不是话}{bu2shi4hua4}
  \seealsoref{不成话}{bu4cheng2hua4}
\end{EntryWithPhonetic}

\begin{EntryWithPhonetic}{不幸}{bu2 xing4}{4,8}{⼀、⼲}[HSK 5]
  \definition{adj.}{triste; infeliz; lamentável; azarado | infeliz; indica o mais indesejável (que aconteceu)}
  \definition[些]{s.}{morte; desastre; infortúnio; adversidade; calamidade}
\end{EntryWithPhonetic}

\begin{EntryWithPhonetic}{不要}{bu2 yao4}{4,9}{⼀、⾑}[HSK 2]
  \definition{adv.}{nada de (pedir a alguém para não fazer) | não; expressa proibição e dissuasão}
\end{EntryWithPhonetic}

\begin{EntryWithPhonetic}{不要紧}{bu2yao4jin3}{4,9,10}{⼀、⾑、⽷}[HSK 4]
  \definition{adj.}{não é sério; não importa; não importa; não é um problema; nenhum obstáculo; nenhum problema; parece estar tudo bem; à primeira vista, não parece haver nenhum obstáculo}
\end{EntryWithPhonetic}

\begin{EntryWithPhonetic}{不易}{bu2 yi4}{4,8}{⼀、⽇}[HSK 5]
  \definition{adj.}{não é fácil; difícil | imutável}
  \definition{v.}{não é fácil fazer algo | não pode ser alterado}
\end{EntryWithPhonetic}

\begin{EntryWithPhonetic}{不用}{bu2 yong4}{4,5}{⼀、⽤}[HSK 1]
  \definition{v.}{não precisar; não ter necessidade; indicar que, na verdade, não é necessário}
\end{EntryWithPhonetic}

\begin{EntryWithPhonetic}{不再}{bu2 zai4}{4,6}{⼀、⼌}[HSK 6]
  \definition{adv.}{não mais; não repita uma segunda vez}
  \definition{v.}{ter ido embora; não retornar; não aparecer; não existir mais}
\end{EntryWithPhonetic}

\begin{EntryWithPhonetic}{不在乎}{bu2 zai4 hu1}{4,6,5}{⼀、⼟、⼃}[HSK 4]
  \definition{v.}{não se importar; não dar a mínima; não dar atenção}
\end{EntryWithPhonetic}

\begin{EntryWithPhonetic}{不至于}{bu2 zhi4 yu2}{4,6,3}{⼀、⾄、⼆}[HSK 6]
  \definition{adv.}{não pode ir tão longe a ponto de; não tanto\dots. a ponto de\dots; não a ponto de}
\end{EntryWithPhonetic}

\begin{EntryWithPhonetic}{不注意}{bu2zhu4yi4}{4,8,13}{⼀、⽔、⼼}
  \definition{adj.}{impensado | distraído}
  \definition{s.}{descuido | distração}
\end{EntryWithPhonetic}

\begin{EntryWithPhonetic}{补}{bu3}{7}{⾐}[HSK 3]
  \definition*{s.}{Sobrenome Bu}
  \definition{s.}{ajuda; uso; benefício; utilidade}
  \definition{v.}{reparar; consertar; remendar; adicionar materiais, consertar coisas quebradas | abastecer; encher; repor; adicionar suplemento; complementar; completar; preencher | nutrir}
\end{EntryWithPhonetic}

\begin{EntryWithPhonetic}{补偿}{bu3chang2}{7,11}{⾐、⼈}[HSK 5]
  \definition{v.}{compensar (perda, consumo); compensar (deficiências, diferenças)}
\end{EntryWithPhonetic}

\begin{EntryWithPhonetic}{补充}{bu3chong1}{7,6}{⾐、⼉}[HSK 3]
  \definition{adj.}{adicional | suplementar}
  \definition{v.}{reabastecer; suplementar; complementar; aumentar uma parte quando houver insuficiência ou perda}
\end{EntryWithPhonetic}

\begin{EntryWithPhonetic}{补考}{bu3 kao3}{7,6}{⾐、⽼}[HSK 6]
  \definition{v.}{repetir ou refazer um exame}
\end{EntryWithPhonetic}

\begin{EntryWithPhonetic}{补课}{bu3 ke4}{7,10}{⾐、⾔}[HSK 6]
  \definition{v.+compl.}{compensar uma aula perdida; compensar cursos perdidos | refazer; fazer algo de novo; metáfora para refazer algo que não foi bem feito}
\end{EntryWithPhonetic}

\begin{EntryWithPhonetic}{补贴}{bu3tie1}{7,9}{⾐、⾙}[HSK 5]
  \definition[笔,项,种,份]{s.}{subsídio; ajuda de custo; custos de indenização ou assistência concedida a empresas ou indivíduos pelo estado ou governo}
  \definition{v.}{subsidiar; compensar a falta de dinheiro ou coisas; refere-se principalmente à compensação financeira ou ajuda dada pelo estado ou governo a empresas ou indivíduos}
\end{EntryWithPhonetic}

\begin{EntryWithPhonetic}{补习}{bu3 xi2}{7,3}{⾐、⼄}[HSK 6]
  \definition{clas.}{treinar; ter aulas depois da escola ou do trabalho; estudar depois da aula ou no seu tempo livre para compensar a falta de conhecimento}
\end{EntryWithPhonetic}

\begin{EntryWithPhonetic}{补助}{bu3 zhu4}{7,7}{⾐、⼒}[HSK 6]
  \definition{s.}{subsídio; mesada}
  \definition{v.}{ajudar financeiramente; subsidiar}
\end{EntryWithPhonetic}

\begin{EntryWithPhonetic}{捕}{bu3}{10}{⼿}[HSK 6]
  \definition{v.}{pegar; apreender; prender}
\end{EntryWithPhonetic}

\begin{EntryWithPhonetic}{不}{bu4}{4}{⼀}[HSK 1]
  \definition{adv.}{(antes de verbos, adjetivos e outros advérbios; nunca antes do verbo 有) não; não vai; não quer | em algumas expressões educadas, significa que não é necessário fazer isso, o que equivale a 不用 ou 不要 | | (entre um verbo e seu complemento) não pode | usado com 就 para indicar escolha}
  \definition{part.}{no final da frase para indicar uma pergunta; (usar sozinho ou com uma partícula nas respostas) não}
  \definition{pref.}{(antes de certos substantivos para formar um adjetivo) un-; in-}
  \seeref{不}{bu2}
  \seeref{不}{bu5}
  \seealsoref{不要}{bu2 yao4}
  \seealsoref{不用}{bu2 yong4}
  \seealsoref{就}{jiu4}
  \seealsoref{有}{you3}
\end{EntryWithPhonetic}

\begin{EntryWithPhonetic}{不安}{bu4'an1}{4,6}{⼀、⼧}[HSK 3]
  \definition{adj.}{(humor) inquieto; (ambiente, etc.)  instável; intranquilo; perturbado; sem paz | desculpe; frases de cortesia, expressões de desculpas, equivalentes a 不好意思}
  \seealsoref{不好意思}{bu4 hao3 yi4 si5}
\end{EntryWithPhonetic}

\begin{EntryWithPhonetic}{不曾}{bu4 ceng2}{4,12}{⼀、⽈}[HSK 5]
  \definition{adv.}{nunca (ter feito algo); indica que não aconteceu (negação de 曾经)}
  \seealsoref{曾经}{ceng2jing1}
\end{EntryWithPhonetic}

\begin{EntryWithPhonetic}{不成}{bu4 cheng2}{4,6}{⼀、⼽}[HSK 6]
  \definition{adj.}{não é bom; não funciona; impraticável}
  \definition{part.}{usada no final de uma frase para expressar especulação ou tom contraintuitivo, geralmente precedido por palavras como 嘛 ou 莫非}
  \definition{v.}{não ser permitido; não ser permissível; ser impossível}
  \seealsoref{莫非}{mo4fei1}
  \seealsoref{难道}{nan2dao4}
\end{EntryWithPhonetic}

\begin{EntryWithPhonetic}{不成话}{bu4cheng2hua4}{4,6,8}{⼀、⼽、⾔}
  \definition{expr.}{irracional | chocante; ultrajante; inapropriado}
  \seealsoref{不是话}{bu2shi4hua4}
  \seealsoref{不像话}{bu2xiang4hua4}
\end{EntryWithPhonetic}

\begin{EntryWithPhonetic}{不得不}{bu4de2bu4}{4,11,4}{⼀、⼻、⼀}[HSK 3]
  \definition{adv.}{ter que; não ter outra escolha a não ser; como obrigação ou necessidade}
\end{EntryWithPhonetic}

\begin{EntryWithPhonetic}{不得了}{bu4de2liao3}{4,11,2}{⼀、⼻、⼅}[HSK 5]
  \definition{adj.}{terrível; horrível; extremamente sério; indica uma situação grave}
  \definition{adv.}{muito; extremamente; excessivamente; indica um grau profundo}
\end{EntryWithPhonetic}

\begin{EntryWithPhonetic}{不敢当}{bu4gan3dang1}{4,11,6}{⼀、⽁、⼹}[HSK 5]
  \definition{expr.}{Eu realmente não mereço isso.; Eu não sou digno de tais elogios.; Não estou à altura da honra.; Você me lisonjeia.; palavra de humildade, para mostrar que você não pode pagar (hospitalidade, elogios, etc.)}
\end{EntryWithPhonetic}

\begin{EntryWithPhonetic}{不公}{bu4gong1}{4,4}{⼀、⼋}
  \definition{adj.}{injusto}
\end{EntryWithPhonetic}

\begin{EntryWithPhonetic}{不管}{bu4guan3}{4,14}{⼀、⽵}[HSK 4]
  \definition{conj.}{não importa (o que, como, etc.); independentemente de; indica que, embora as condições ou circunstâncias tenham mudado, o resultado permanece o mesmo; 不管 deve ser seguido por algo incerto}
  \seealsoref{不管……都……}{bu4guan3 dou1}
  \seealsoref{不管……也……}{bu4guan3 ye3}
\end{EntryWithPhonetic}

\begin{EntryWithPhonetic}{不管……都……}{bu4guan3 dou1}{4,14,10}{⼀、⽵、⾢}
  \definition{conj.}{não apenas\dots, (o que, quem, como, etc.), \dots}
\end{EntryWithPhonetic}

\begin{EntryWithPhonetic}{不管……也……}{bu4guan3 ye3}{4,14,3}{⼀、⽵、⼄}
  \definition{conj.}{não apenas\dots, (o que, quem, como, etc.), \dots}
\end{EntryWithPhonetic}

\begin{EntryWithPhonetic}{不光}{bu4 guang1}{4,6}{⼀、⼉}[HSK 3]
  \definition{adv.}{não é o único; não apenas; não só; indica que excede uma determinada quantidade ou faixa}
  \definition{conj.}{não somente; não só}
\end{EntryWithPhonetic}

\begin{EntryWithPhonetic}{不好意思}{bu4 hao3 yi4 si5}{4,6,13,9}{⼀、⼥、⼼、⼼}[HSK 2]
  \definition{adj.}{envergonhado; desconfortável; constrangido; sem jeito}
  \definition{interj.}{com licença; peço desculpas; desculpe-me}
  \definition{v.}{achar constrangedor (fazer algo) | pedir desculpas (por incomodar alguém) | sentir-se envergonhado | achar algo embaraçoso}
\end{EntryWithPhonetic}

\begin{EntryWithPhonetic}{不禁}{bu4jin1}{4,13}{⼀、⽰}[HSK 6]
  \definition{adv.}{não pode evitar (fazer algo); não pode se abster de; incapaz de conter (produzir certas emoções, realizar certas ações)}
\end{EntryWithPhonetic}

\begin{EntryWithPhonetic}{不仅}{bu4jin3}{4,4}{⼀、⼈}[HSK 3]
  \definition{adv.}{não apenas (em número, quantidade ou extensão); costuma-se dizer 不仅仅}
  \definition{conj.}{não somente}
  \seealsoref{不仅仅}{bu4 jin3 jin3}
\end{EntryWithPhonetic}

\begin{EntryWithPhonetic}{不仅仅}{bu4 jin3 jin3}{4,4,4}{⼀、⼈、⼈}[HSK 6]
  \definition{adv.}{não só; não apenas}
\end{EntryWithPhonetic}

\begin{EntryWithPhonetic}{不久}{bu4 jiu3}{4,3}{⼀、⼃}[HSK 2]
  \definition{adv.}{em breve; dentro em breve; num futuro próximo | logo depois; pouco tempo depois | não muito tempo (antes ou depois de algo)}
\end{EntryWithPhonetic}

\begin{EntryWithPhonetic}{不可避免}{bu4ke3bi4mian3}{4,5,16,7}{⼀、⼝、⾌、⼉}
  \definition{adj./adv.}{inevitável}
\end{EntryWithPhonetic}

\begin{EntryWithPhonetic}{不良}{bu4 liang2}{4,7}{⼀、⾉}[HSK 5]
  \definition{adj.}{ruim; prejudicial; nocivo; insalubre}
\end{EntryWithPhonetic}

\begin{EntryWithPhonetic}{不满}{bu4 man3}{4,13}{⼀、⽔}[HSK 2]
  \definition{adj.}{ressentido; insatisfeito; descontente}
  \definition{v.}{estar descontente com; insatisfação ou descontentamento com alguém ou alguma coisa |ser menor que; quantidade ou tempo insuficientes ou inadequados}
\end{EntryWithPhonetic}

\begin{EntryWithPhonetic}{不免}{bu4mian3}{4,7}{⼀、⼉}[HSK 5]
  \definition{adv.}{inevitavelmente; inexoravelmente}
\end{EntryWithPhonetic}

\begin{EntryWithPhonetic}{不能不}{bu4 neng2 bu4}{4,10,4}{⼀、⾁、⼀}[HSK 5]
  \definition{adv.}{tem que; não pode, mas; necessariamente; definitivamente}
\end{EntryWithPhonetic}

\begin{EntryWithPhonetic}{不然}{bu4ran2}{4,12}{⼀、⽕}[HSK 4]
  \definition{adj.}{não é assim; não é o caso}
  \definition{conj.}{se não; caso contrário; indica outra consequência ou circunstância que teria ocorrido se não fosse}
\end{EntryWithPhonetic}

\begin{EntryWithPhonetic}{不如}{bu4ru2}{4,6}{⼀、⼥}[HSK 2]
  \definition{conj.}{em vez de; melhor do que; seria melhor; preferiria; seria melhor; usado no início da segunda parte da frase, indica uma escolha feita após comparação (geralmente em correspondência com o termo 与其 no texto anterior)}
  \definition{v.}{ser inferior a; não ser igual a; não ser tão bom quanto;  não poder fazer melhor que}
  \seealsoref{与其}{yu3qi2}
\end{EntryWithPhonetic}

\begin{EntryWithPhonetic}{不少}{bu4 shao3}{4,4}{⼀、⼩}[HSK 2]
  \definition{adj.}{muitos; bastante; não poucos; indica uma quantidade considerável, equivalente a muitos ou bastante}
\end{EntryWithPhonetic}

\begin{EntryWithPhonetic}{不时}{bu4shi2}{4,7}{⼀、⽇}[HSK 5]
  \definition{adv.}{frequentemente; de tempos em tempos | a qualquer momento}
\end{EntryWithPhonetic}

\begin{EntryWithPhonetic}{不是……而是}{bu4shi4 er2 shi4}{4,9,6,9}{⼀、⽇、⽽、⽇}
  \definition{conj.}{não somente\dots mas também\dots, expressam um relacionamento mais profundo e avançado em significado, mas as orações antes e depois são consistentes em expressar significados negativos e afirmativos, entretanto, a primeira metade da frase expressa negação, e a segunda metade expressa afirmação, e o significado das orações anteriores e seguintes não pode ser de um nível mais alto}
\end{EntryWithPhonetic}

\begin{EntryWithPhonetic}{不停}{bu4 ting2}{4,11}{⼀、⼈}[HSK 5]
  \definition{adv.}{sem parar; sem interrupção; continuamente}
\end{EntryWithPhonetic}

\begin{EntryWithPhonetic}{不通}{bu4 tong1}{4,10}{⼀、⾡}[HSK 6]
  \definition{adj.}{sem sentido; ilógico; agramatical | usado para se referir a coisas abstratas}
  \definition{v.}{obstruir; bloquear; estar obstruído; estar bloqueado; ser intransitável | não saber; não entender; não poder aceitar}
\end{EntryWithPhonetic}

\begin{EntryWithPhonetic}{不同}{bu4 tong2}{4,6}{⼀、⼝}[HSK 2]
  \definition{adj.}{diferente; distinto; não semelhante;}
\end{EntryWithPhonetic}

\begin{EntryWithPhonetic}{不行}{bu4 xing2}{4,6}{⼀、⾏}[HSK 2]
  \definition{adj.}{não funciona; não é bom; falta de capacidade e habilidade; nível baixo}
  \definition{adv.}{profundamente; terrivelmente; extremamente; expressa um grau muito profundo; incrível (usado após o caractere 得 como complemento)}
  \definition{v.}{não servir; não ser permitido; estar fora de questão | estar à beira da morte}
  \seealsoref{得}{de5}
\end{EntryWithPhonetic}

\begin{EntryWithPhonetic}{不许}{bu4 xu3}{4,6}{⼀、⾔}[HSK 5]
  \definition{v.}{não permitir; ser proibido; proibir firmemente | não pode (usado em perguntas retóricas)}
\end{EntryWithPhonetic}

\begin{EntryWithPhonetic}{不一定}{bu4 yi2 ding4}{4,1,8}{⼀、⼀、⼧}[HSK 2]
  \definition{adv.}{talvez; incerto; não tenho certeza; não necessariamente assim; refere-se a algo que não pode ser determinado}
\end{EntryWithPhonetic}

\begin{EntryWithPhonetic}{不一会儿}{bu4 yi2 hui4r5}{4,1,6,2}{⼀、⼀、⼈、⼉}[HSK 2]
  \definition{expr.}{em um momento; em pouco tempo; em breve; depois de algum tempo}
\end{EntryWithPhonetic}

\begin{EntryWithPhonetic}{不怎么}{bu4 zen3 me5}{4,9,3}{⼀、⼼、⼃}[HSK 6]
  \definition{adv.}{não muito; não particularmente; não exatamente}
\end{EntryWithPhonetic}

\begin{EntryWithPhonetic}{不怎么样}{bu4 zen3 me5 yang4}{4,9,3,10}{⼀、⼼、⼃、⽊}[HSK 6]
  \definition{adj.}{não muito bom; não particularmente bom | muito indiferente; mais ou menos}
\end{EntryWithPhonetic}

\begin{EntryWithPhonetic}{不值}{bu4 zhi2}{4,10}{⼀、⼈}[HSK 6]
  \definition{v.}{não valer a pena}
\end{EntryWithPhonetic}

\begin{EntryWithPhonetic}{不止}{bu4zhi3}{4,4}{⼀、⽌}[HSK 5]
  \definition{adv.}{mais do que; não limitado a; indica mais do que esse valor ou intervalo}
  \definition{v.}{exceder; superar; não ser possível interromper a ação}
\end{EntryWithPhonetic}

\begin{EntryWithPhonetic}{不足}{bu4zu2}{4,7}{⼀、⾜}[HSK 5]
  \definition{adj.}{não o bastante; inadequado; insuficiente}
  \definition{s.}{deficiência; inadequação; desvantagens, não é bom o suficiente}
  \definition{v.}{não exceder um determinado número | não valer a pena; ser inferior; não merecer | não pode; não deveria}
\end{EntryWithPhonetic}

\begin{EntryWithPhonetic}{布}{bu4}{5}{⼱}[HSK 3]
  \definition*{s.}{Sobrenome Bu}
  \definition[块,幅,匹]{s.}{tecido; tecido de algodão; algodão, linho ou fibras sintéticas tecidas, que podem ser utilizadas como material para confecção de roupas ou outros objetos | uma moeda antiga | algo parecido com um pano}
  \definition{v.}{declarar; anunciar; publicar; proclamar | divulgar; espalhar por toda parte; difundir amplamente | implantar; dispor; organizar}
\end{EntryWithPhonetic}

\begin{EntryWithPhonetic}{布谷鸟}{bu4gu3niao3}{5,7,5}{⼱、⾕、⿃}
  \definition{s.}{cuco (pássaro)}
  \seealsoref{杜鹃}{du4juan1}
  \seealsoref{杜鹃鸟}{du4juan1niao3}
  \seealsoref{杜宇}{du4yu3}
\end{EntryWithPhonetic}

\begin{EntryWithPhonetic}{布满}{bu4 man3}{5,13}{⼱、⽔}[HSK 6]
  \definition{v.}{abundar em; estar cheio de; espalhar-se e preencher um certo espaço}
\end{EntryWithPhonetic}

\begin{EntryWithPhonetic}{布署}{bu4shu3}{5,13}{⼱、⽹}
  \variantof{部署}
\end{EntryWithPhonetic}

\begin{EntryWithPhonetic}{布置}{bu4zhi4}{5,13}{⼱、⽹}[HSK 4]
  \definition{v.}{arrumar; organizar; decorar; colocar adequadamente objetos ou paisagismo, conforme necessário | designar; tomar providências para; dar instruções sobre; organizar trabalho, atividades, etc.}
\end{EntryWithPhonetic}

\begin{EntryWithPhonetic}{步}{bu4}{7}{⽌}[HSK 3]
  \definition*{s.}{Geralmente em nomes de lugares | Sobrenome Bu}[盐步===Yanbu, na província de Guangdong]
  \definition{clas.}{uma unidade antiga para medida de comprimento, equivalente a cinco 尺}
  \definition{s.}{passo; ritmo | etapa; passo | condição; situação; estado | cais; píer | porto; cidade portuária | (geralmente em nomes de lugares)}
  \definition{v.}{caminhar; ir a pé | seguir os passos de alguém | (dialeto) medir com passos | seguir; acompanhar | medir a distância com os passos}
  \seealsoref{尺}{chi3}
\end{EntryWithPhonetic}

\begin{EntryWithPhonetic}{步行}{bu4 xing2}{7,6}{⽌、⾏}[HSK 4]
  \definition{v.}{caminhar; ir a pé; andar a pé (diferente de andar de carro, a cavalo, etc.)}
\end{EntryWithPhonetic}

\begin{EntryWithPhonetic}{部}{bu4}{10}{⾢}[HSK 3]
  \definition*{s.}{Sobrenome Bu}
  \definition{clas.}{usado para obras de literatura, livros, filmes, etc.}
  \definition[根]{s.}{parte; seção | unidade; ministério; departamento; conselho | sede; matriz; quartel general | tropas; forças | divisão; região}
  \definition{v.}{comandar; liderar}
\end{EntryWithPhonetic}

\begin{EntryWithPhonetic}{部队}{bu4 dui4}{10,4}{⾢、⾩}[HSK 6]
  \definition[支,个]{s.}{militar; exército; forças armadas | tropas; refere-se a uma parte do exército}
\end{EntryWithPhonetic}

\begin{EntryWithPhonetic}{部分}{bu4fen5}{10,4}{⾢、⼑}[HSK 2]
  \definition[个,些,快,份]{s.}{parte; seção; porção; parte do todo; alguns indivíduos dentro do todo | ramo; parte separada de um sistema ou entidade}
\end{EntryWithPhonetic}

\begin{EntryWithPhonetic}{部门}{bu4men2}{10,3}{⾢、⾨}[HSK 3]
  \definition[个]{s.}{departamento; ramo; classe; seção; partes ou unidades que compõem um todo}
\end{EntryWithPhonetic}

\begin{EntryWithPhonetic}{部属}{bu4shu3}{10,12}{⾢、⼫}
  \definition{s.}{afiliado a um ministério | subordinado | tropas sob comando de alguém}
\end{EntryWithPhonetic}

\begin{EntryWithPhonetic}{部署}{bu4shu3}{10,13}{⾢、⽹}
  \definition{s.}{implantação}
  \definition{v.}{implantar}
\end{EntryWithPhonetic}

\begin{EntryWithPhonetic}{部位}{bu4wei4}{10,7}{⾢、⼈}[HSK 5]
  \definition{s.}{lugar; posição (usado principalmente para o corpo humano)}
\end{EntryWithPhonetic}

\begin{EntryWithPhonetic}{部下}{bu4xia4}{10,3}{⾢、⼀}
  \definition{s.}{subordinado | tropas sob comando de alguém}
\end{EntryWithPhonetic}

\begin{EntryWithPhonetic}{部长}{bu4 zhang3}{10,4}{⾢、⾧}[HSK 3]
  \definition[个,位,名]{s.}{ministro; chefe de departamento; um alto funcionário do estado encarregado pelo chefe de estado ou chefe executivo do governo da gestão das atividades governamentais de um departamento | chefe de seção; líder tribal}
\end{EntryWithPhonetic}

\begin{EntryWithPhonetic}{部族}{bu4zu2}{10,11}{⾢、⽅}
  \definition{adj.}{tribal}
  \definition{s.}{tribo}
\end{EntryWithPhonetic}

\begin{EntryWithPhonetic}{不}{bu5}{4}{⼀}[HSK 1]
  \definition{adv.}{não (em expressões \{v.\} + 不 + \{v.\})}
  \seeref{不}{bu2}
  \seeref{不}{bu4}
\end{EntryWithPhonetic}

%%%%% EOF %%%%%


 %%%
%%% C
%%%
\section*{C}
\addcontentsline{toc}{section}{C}
\begin{multicols}{2}

\entry{菜}{n.}{cai4}{hortaliça; verdura; prato}
\entry{菜单}{n.}{cai4dan1}{menu; ementa; cardápio}

\entry{草地}{n.}{cao3di4}{relva; pastagem}

\entry{参加}{v.}{can1jia1}{juntar; participar}

\entry{餐厅}{n.}{can1ting1}{cantina; sala de jantar}

\entry{厕所}{n.}{ce4suo3}{sanitário; toilette}

\entry{磁带}{n.}{ci2dai4}{cassete|\fbox{盘}}
\entry{磁盘}{n.}{ci2pan2}{disquete}

\entry{词典}{n.}{ci2dian3}{dicionário|\fbox{本}}

\entry{茶}{n.}{cha2}{chá}

\entry{长}{adj.}{chang2}{comprido; longo}
\entry{长成}{n.}{chang2cheng2}{Grande Muralha}

\entry{常常}{adv.}{chang2chang2}{frequentemente}

\entry{炒}{v.}{chao3}{saltear}

\entry{车}{n.}{che1}{veículo; viatura}
\entry{车牌}{n.}{che1pai2}{matrícula; placa de carro}
\entry{车站}{n.}{che1zhan4}{estação; paragem}

\entry{衬衫}{n.}{chen4shan1}{camisa|\fbox{件}}

\entry{成都}{n.}{cheng2du1}{Chengdu}

\entry{城市}{n.}{cheng2shi4}{cidade}

\entry{橙色}{n.}{cheng2se4}{cor de laranja}
\entry{橙汁}{n.}{cheng2zhi1}{suco de laranja}

\entry{惩罚}{v.}{cheng2fa2}{punir; penalizar}
\entry{惩处}{v.}{cheng2chu3}{punir; penalizar}

\entry{吃}{v.}{chi1}{comer}

\entry{迟到}{v.}{chi1dao4}{chegar atrasado; tardar}

\entry{憧憬}{v.}{chong1jing3}{ansiar por; esperar por}

\entry{宠物}{n.}{chong3wu4}{animal de estimação}

\entry{酬劳}{n.}{chou2lao2}{recompensa}

\entry{出}{v.d.}{chu1}{sair}
\entry{出去}{v.d.}{chu1qu0}{sair; ir para fora}
\entry{出口}{n.}{chu1kou3}{exportação}
\entry{出口}{v.}{chu1kou3}{exportar}
\entry{出站}{n.}{chu1zhan4}{saída da estação}
\entry{出租汽车}{n.}{chu1zu1qi4che1}{táxi|\fbox{辆}}

\entry{穿}{v.}{chuan1}{vestir}

\entry{船}{v.}{chuan2}{barco; navio}

\entry{传真}{n.}{chuan2zhen1}{fax; facsímile}

\entry{床}{n.}{chuang2}{cama|\fbox{张}}

\entry{春天}{n.}{chun1tian1}{primavera}

\entry{绰号}{n.}{chuo4hao4}{apelido}

\entry{聪明}{adj.}{cong1ming2}{inteligente; brilhante; esperto}
\entry{聪慧}{adj.}{cong1hui4}{inteligente; brilhante}

\entry{从}{prep.}{cong2}{de; desde; a partir de}

\entry{醋}{n.}{cu4}{vinagre}

\entry{错}{adj.}{cuo4}{errado; enganado}

\end{multicols}

 \section*{D}
\addcontentsline{toc}{section}{D}
\begin{multicols}{2}
%%%
%%% D
%%%
% \entry{打}{v.}{jogar}
% \entry{打电话}{v.}{ligar; dar um telefonema}
% \entry{打算}{v.}{pretender}
% \entry{大}{adj.}{grande}
% \entry{大概}{adv.}{aproximadamente; por volta de}
% \entry{大海}{n.}{mar}
% \entry{大家}{pron.}{todos; todas}
% \entry{大学}{n.}{universidade}
% \entry{\xpinyin{大}{Da4}洋洲}{n.}{Oceania}
% \entry{带}{v.}{levar}
% \entry{戴}{v.}{usar; vestir}
% \entry{担心}{v.}{preocupar-se}
% \entry{蛋糕}{n.}{bolo}
% \entry{当然}{adv.}{claro; certamente; com certeza}
% \entry{到}{v.}{chegar}
% \entry{得}{v.}{ganhar; obter}
% \entry{\xpinyin{德}{De2}国}{n.}{Alemanha}
% \entry{的}{part.}{partícula utilizada em possessivos; partícula utilizada entre adjetivos e substantivos, opcional se substantivo possui apenas um caracter}
% \entry{地方}{n.}{lugar; local; sítio}
% \entry{等}{v.}{esperar}
% \entry{第}{num.}{prefixo para expressar números ordinais}
% \entry{弟\xpinyin{弟}{di0}}{n.}{irmão mais novo}
% \entry{\xpinyin{地}{di4}图}{n.}{mapa}
% \entry{点(钟)}{p.c.}{hora}
% \entry{(一)\xpinyin{点}{dianr3}\xpinyin{儿}{}}{p.c.}{um pouco}
% \entry{(商)店}{n.}{loja}
% \entry{电话}{n.}{telefone}
% \entry{电脑}{n.}{computador}
% \entry{电视}{n.}{televisor}
% \entry{电影}{n.}{cinema; filme}
% \entry{电子}{n.}{eletrônico; eletrônica)}
% \entry{电子邮件}{n.}{correio eletrônico; e-mail}
% \entry{东}{n.}{leste}
% \entry{\xpinyin{东}{Dong1}方}{n.}{Oriente}
% \entry{东天}{n.}{inverno}
% \entry{东\xpinyin{西}{xi0}}{n.}{coisa}
% \entry{都}{adv.}{todo; toda; todos; todas}
% \entry{读}{v.}{ler}
% \entry{度}{v.}{passar}
% \entry{锻炼}{v.}{fazer exercício físico}
% \entry{对}{adj.}{correto; sim}
% \entry{对不起}{}{desculpar; pedir desculpa; perdão}
% \entry{多}{adj.}{muito; muita; muitos; muitas}
% \entry{多大}{interr.}{quantos anos; que idade}
% \entry{多少}{interr.}{quanto; quanta; quantos; quantas}
% \entry{打算}{v./n.}{pensar; planear; plano}
% \entry{打球}{v.}{jogar à bola; jogar (futebol; basquetebol; handbol; etc)}
% \entry{冰球}{n.}{hóquei em gelo}
\end{multicols}

 %%%
%%% E
%%%
\section*{E}
\addcontentsline{toc}{section}{E}
\begin{multicols}{2}
\entry{俄罗斯}{n.}{É\ luo2si1}{Rússia}
\entry{儿媳}{n.}{er2xi2}{esposa do filho}
\entry{儿子}{n.}{er2zi0}{filho}
\entry{二}{num.}{er4}{dois}
\end{multicols}

 %%%
%%% F
%%%
\section*{F}\addcontentsline{toc}{section}{F}

%%%%%%%%%% 发 %%%%%%%%%%
\subsection*{发}

\begin{EntryWithPhonetic}{发}{fa1}{5}{⼜}[HSK 2]
  \definition*{s.}{Sobrenome: Fa}
  \definition{clas.}{bala, usada para munições e cartuchos}
  \definition{v.}{distribuir; enviar; entregar | emitir; disparar; lançar; descarregar | produzir; gerar; criar (dar origem a) | proferir; emitir; expressar | expandir; desenvolver | prosperar; prosperidade graças à aquisição de bens materiais | crescer ou expandir quando fermentado ou embebido | difundir; dispersar; espalhar | expor; descobrir; revelar | transformar-se; tornar-se; entrar em um determinado estado | demonstrar seus sentimentos; expressar (sentimentos) | sentir; ter um sentimento | começar; estabelecer | fazer com que se faça; iniciar um empreendimento; começar a agir; provocar uma ação}
  \seeref{fa4}
\end{EntryWithPhonetic}

\begin{EntryWithPhonetic}{发表}{fa1biao3}{5,8}{⼜、⾐}[HSK 3]
  \definition{v.}{publicar; entregar; emitir; expressar; anunciar; expressar (opiniões) ou divulgar (assuntos) ao público, verbalmente ou por escrito | publicar em jornais (artigos, etc.)}
\end{EntryWithPhonetic}

\begin{EntryWithPhonetic}{发病}{fa1 bing4}{5,10}{⼜、⽧}[HSK 6]
  \definition{v.}{(de uma doença) avanço | patogênese; morbidade | surto (de uma doença)}
\end{EntryWithPhonetic}

\begin{EntryWithPhonetic}{发布}{fa1bu4}{5,5}{⼜、⼱}[HSK 5]
  \definition{v.}{emitir; publicar; liberar; anunciar; fazer ordens públicas, anúncios, notícias, etc.}
\end{EntryWithPhonetic}

\begin{EntryWithPhonetic}{发布会}{fa1bu4hui4}{5,5,6}{⼜、⼱、⼈}[HSK 7-9]
  \definition[次]{s.}{coletiva de imprensa; um formato de conferência usado para divulgar notícias ou responder perguntas da mídia e do público | \emph{briefing}; atividades de exposição para promover novos produtos, etc.}
\end{EntryWithPhonetic}

\begin{EntryWithPhonetic}{发财}{fa1/cai2}{5,7}{⼜、⾙}[HSK 7-9]
  \definition{v.+compl.}{ficar rico; fazer fortuna; acumular fortuna; ganhar muito dinheiro ou propriedades | trabalhar; conseguir um emprego; uma maneira educada de perguntar a alguém onde ele trabalha é dizer onde ele fez fortuna}
\end{EntryWithPhonetic}

\begin{EntryWithPhonetic}{发愁}{fa1/chou2}{5,13}{⼜、⼼}[HSK 7-9]
  \definition{v.+compl.}{preocupar-se; ficar ansioso; ficar triste; sentir-se deprimido por não ter ideias ou soluções}
\end{EntryWithPhonetic}

\begin{EntryWithPhonetic}{发出}{fa1 chu1}{5,5}{⼜、⼐}[HSK 3]
  \definition{v.}{fazer; produzir; deixar sair; ocorrer (som, dúvida, etc.) | emitir; anunciar; publicar; divulgar (ordens, instruções) | enviar (mercadorias, cartas, etc.); partir (veículos, etc.) | emitir; exalar (cheiro, calor, etc.)}
\end{EntryWithPhonetic}

\begin{EntryWithPhonetic}{发达}{fa1da2}{5,6}{⼜、⾡}[HSK 3]
  \definition{adj.}{desenvolvido; florescente; (coisas) Já estão bem desenvolvidas; (negócios) prosperam}
  \definition{v.}{desenvolver; promover; florescer; a pessoa tem um bom desempenho profissional e é muito bem-sucedida}
\end{EntryWithPhonetic}

\begin{EntryWithPhonetic}{发电}{fa1 dian4}{5,5}{⼜、⽥}[HSK 6]
  \definition{s.}{geração de energia elétrica; produção de eletricidade; fornecimento de energia}
  \definition{v.}{gerar eletricidade (ou energia elétrica) | enviar um telegrama}
\end{EntryWithPhonetic}

\begin{EntryWithPhonetic}{发电机}{fa1dian4ji1}{5,5,6}{⼜、⽥、⽊}[HSK 7-9]
  \definition{s.}{gerador; dínamo | alternador; gerador elétrico}
\end{EntryWithPhonetic}

\begin{EntryWithPhonetic}{发动}{fa1dong4}{5,6}{⼜、⼒}[HSK 3]
  \definition{v.}{iniciar; começar; lançar | chamar à ação; mobilizar; despertar | ligar o motor; dar a partida; dar o pontapé inicial (motor de combustão interna) | estimular; colocar em ação}
\end{EntryWithPhonetic}

\begin{EntryWithPhonetic}{发动机}{fa1dong4ji1}{5,6,6}{⼜、⼒、⽊}
  \definition[台]{s.}{motor}
\end{EntryWithPhonetic}

\begin{EntryWithPhonetic}{发抖}{fa1dou3}{5,7}{⼜、⼿}[HSK 7-9]
  \definition{v.}{tremer; sacudir; estremecer; tremer devido ao medo, raiva ou frio}
\end{EntryWithPhonetic}

\begin{EntryWithPhonetic}{发放}{fa1 fang4}{5,8}{⼜、⽅}[HSK 6]
  \definition{v.}{conceder; estender; fornecer; (governo, organização) distribuir dinheiro ou suprimentos para os necessitados | emitir; enviar}
\end{EntryWithPhonetic}

\begin{EntryWithPhonetic}{发奋图强}{fa1fen4-tu2qiang2}{5,8,8,12}{⼜、⼤、⼞、⼸}[HSK 7-9]
  \definition{expr.}{fazer um esforço para se tornar forte (expressão idiomática); determinado a fazer melhor | arregaçar as mangas}
\end{EntryWithPhonetic}

\begin{EntryWithPhonetic}{发光}{fa1/guang1}{5,6}{⼜、⼉}[HSK 7-9]
  \definition{s.}{luminescência}
  \definition{v.+compl.}{emitir luz; brilhar; ser luminoso; cintilar}
\end{EntryWithPhonetic}

\begin{EntryWithPhonetic}{发挥}{fa1hui1}{5,9}{⼜、⼿}[HSK 4]
  \definition{v.}{colocar em jogo; dar jogo a; dar espaço a; dar rédea solta a; revelar a natureza ou a capacidade interior | expressar; desenvolver (uma ideia, um tema, etc.); elaborar; fazer valer o ponto ou o motivo}
\end{EntryWithPhonetic}

\begin{EntryWithPhonetic}{发火}{fa1/huo3}{5,4}{⼜、⽕}[HSK 7-9]
  \definition{v.+compl.}{ficar com raiva; explodir; ficar furioso; perder a paciência | detonar; explodir | inflamar; pegar fogo; acender; começar a queimar}
\end{EntryWithPhonetic}

\begin{EntryWithPhonetic}{发酵}{fa1/jiao4}{5,14}{⼜、⾣}[HSK 7-9]
  \definition{s.}{fermentação; zimólise; compostos orgânicos complexos são decompostos em substâncias mais simples sob a ação de microrganismos}
  \definition{v.+compl.}{fermentar}
\end{EntryWithPhonetic}

\begin{EntryWithPhonetic}{发觉}{fa1jue2}{5,9}{⼜、⾒}[HSK 5]
  \definition{v.}{vir a saber; estar ciente (de); perceber; tornar-se consciente | encontrar; detectar; perceber; descobrir}
\end{EntryWithPhonetic}

\begin{EntryWithPhonetic}{发掘}{fa1jue2}{5,11}{⼜、⼿}[HSK 7-9]
  \definition{v.}{explorar; escavar; desenterrar}
\end{EntryWithPhonetic}

\begin{EntryWithPhonetic}{发愣}{fa1/leng4}{5,12}{⼜、⼼}[HSK 7-9]
  \definition{v.+compl.}{olhar fixamente; estar em transe (ou atordoado)}
\end{EntryWithPhonetic}

\begin{EntryWithPhonetic}{发明}{fa1ming2}{5,8}{⼜、⽇}[HSK 3]
  \definition[个,项,种]{s.}{invenção; novos produtos ou métodos inventados}
  \definition{v.}{inventar; pesquisa que cria (novos produtos ou novos métodos) | expor; explicar; explicação criativa}
\end{EntryWithPhonetic}

\begin{EntryWithPhonetic}{发明者}{fa1ming2zhe3}{5,8,8}{⼜、⽇、⽼}
  \definition{s.}{inventor}
\end{EntryWithPhonetic}

\begin{EntryWithPhonetic}{发怒}{fa1 nu4}{5,9}{⼜、⼼}[HSK 6]
  \definition{v.}{ficar com raiva; explodir; perder a paciência | entrar em fúria | entrar em fúria (paixão)}
\end{EntryWithPhonetic}

\begin{EntryWithPhonetic}{发脾气}{fa1 pi2qi5}{5,12,4}{⼜、⾁、⽓}[HSK 7-9]
  \definition{v.}{ficar com raiva; perder a paciência; ficar furioso; fazer barulho ou xingar porque as coisas não saem do seu jeito}
\end{EntryWithPhonetic}

\begin{EntryWithPhonetic}{发票}{fa1piao4}{5,11}{⼜、⽰}[HSK 4]
  \definition[张,种]{s.}{conta; recibo; fatura; recibos emitidos por lojas ou outros escritórios de cobrança}
\end{EntryWithPhonetic}

\begin{EntryWithPhonetic}{发起}{fa1 qi3}{5,10}{⼜、⾛}[HSK 6]
  \definition{s.}{iniciador; patrocinador}
  \definition{v.}{iniciar; patrocinar; começar; lançar}
\end{EntryWithPhonetic}

\begin{EntryWithPhonetic}{发起人}{fa1qi3ren2}{5,10,2}{⼜、⾛、⼈}[HSK 7-9]
  \definition{s.}{iniciador; patrocinador | membro fundador; originadores; autores | propositor}
\end{EntryWithPhonetic}

\begin{EntryWithPhonetic}{发热}{fa1/re4}{5,10}{⼜、⽕}[HSK 7-9]
  \definition{pref.}{piro-}
  \definition{s.}{ebulição; febre; calor; pirexia}
  \definition{v.+compl.}{emitir calor; gerar calor; aquecer; esquentar | ter febre | ser cabeça quente}
\end{EntryWithPhonetic}

\begin{EntryWithPhonetic}{发烧}{fa1shao1}{5,10}{⼜、⽕}[HSK 4]
  \definition{v.}{ter febre; a temperatura corporal normal de uma pessoa é de cerca de 37ºC; se exceder 37,5ºC, é febre}
\end{EntryWithPhonetic}

\begin{EntryWithPhonetic}{发射}{fa1she4}{5,10}{⼜、⼨}[HSK 5]
  \definition{v.}{subir; disparar; lançar; irradiar; projetar; descarregar; enviar algo (como uma bala, um projétil, um satélite, etc.) de um dispositivo em uma velocidade muito alta}
\end{EntryWithPhonetic}

\begin{EntryWithPhonetic}{发生}{fa1sheng1}{5,5}{⼜、⽣}[HSK 3]
  \definition{v.}{ocorrer; acontecer; tomar lugar; surgir algo que não existia antes}
\end{EntryWithPhonetic}

\begin{EntryWithPhonetic}{发誓}{fa1/shi4}{5,14}{⼜、⾔}[HSK 7-9]
  \definition{v.+compl.}{jurar; prometer; fazer um juramento; expressar solenemente a resolução e a promessa de fazer o que foi acordado ou dito}
\end{EntryWithPhonetic}

\begin{EntryWithPhonetic}{发送}{fa1 song4}{5,9}{⼜、⾡}[HSK 3]
  \definition{v.}{enviar; despachar | transmitir (rádio)}
\end{EntryWithPhonetic}

\begin{EntryWithPhonetic}{发现}{fa1xian4}{5,8}{⼜、⾒}[HSK 2]
  \definition[个,项]{s.}{descoberta; achado}
  \definition{v.}{encontrar; descobrir; detectar; identificar; através de pesquisa, exploração, etc., ver ou encontrar coisas ou leis que os antepassados não viram | descobrir; perceber; perceber; notar; estar ciente de}
\end{EntryWithPhonetic}

\begin{EntryWithPhonetic}{发现者}{fa1xian4 zhe3}{5,8,8}{⼜、⾒、⽼}
  \definition{s.}{descobridor}
\end{EntryWithPhonetic}

\begin{EntryWithPhonetic}{发泄}{fa1xie4}{5,8}{⼜、⽔}[HSK 7-9]
  \definition{v.}{soltar; abreviar; dar vazão a; desabafar emoções ou desejos}
\end{EntryWithPhonetic}

\begin{EntryWithPhonetic}{发行}{fa1xing2}{5,6}{⼜、⾏}[HSK 5]
  \definition{v.}{emitir; liberar; publicar; emitir ou vender de publicações recém-impressas, moeda, selos, etc.}
\end{EntryWithPhonetic}

\begin{EntryWithPhonetic}{发言}{fa1/yan2}{5,7}{⼜、⾔}[HSK 3]
  \definition[个]{s.}{discurso; declaração; palestra; opiniões publicadas}
  \definition{v.+compl.}{falar; fazer uma declaração (discurso); expressar opinião (geralmente em reuniões)}
\end{EntryWithPhonetic}

\begin{EntryWithPhonetic}{发言人}{fa1 yan2 ren2}{5,7,2}{⼜、⾔、⼈}[HSK 6]
  \definition{s.}{porta-voz}
\end{EntryWithPhonetic}

\begin{EntryWithPhonetic}{发炎}{fa1yan2}{5,8}{⼜、⽕}[HSK 6]
  \definition{s.}{inflamação}
  \definition{v.}{irritar; inflamar; reação complexa de organismos a fatores patogênicos, como microrganismos, substâncias químicas e estímulos físicos; os sintomas sistêmicos incluem aumento da temperatura corporal, alterações na composição do sangue, vermelhidão local, inchaço, febre, dor, etc.}
\end{EntryWithPhonetic}

\begin{EntryWithPhonetic}{发扬}{fa1yang2}{5,6}{⼜、⼿}[HSK 7-9]
  \definition{v.}{desenvolver; continuar; levar adiante; desenvolver e promover (boas práticas, tradições, etc.) | aproveitar ao máximo; fazer uso total de; exercer ou mostrar (algum poder, habilidade, etc.) tanto quanto possível}
\end{EntryWithPhonetic}

\begin{EntryWithPhonetic}{发扬光大}{fa1yang2-guang1da4}{5,6,6,3}{⼜、⼿、⼉、⼤}[HSK 7-9]
  \definition{expr.}{levar adiante; desenvolver; aprimorar; fomentar e aprimorar; dar pleno uso a; dar maior escopo a; levar a um maior nível de desenvolvimento; desenvolver para um estágio mais alto; espalhar e florescer; ``O desenvolvimento e a promoção tornam-no cada vez mais grandioso.''}
\end{EntryWithPhonetic}

\begin{EntryWithPhonetic}{发音}{fa1yin1}{5,9}{⼜、⾳}
  \definition{s.}{pronúncia}
  \definition{v.}{pronunciar}
\end{EntryWithPhonetic}

\begin{EntryWithPhonetic}{发育}{fa1yu4}{5,8}{⼜、⾁}[HSK 7-9]
  \definition{s.}{crescimento}
  \definition{v.}{crescer; desenvolver; a estrutura e a função dos organismos evoluem do simples para o complexo ou do imaturo para o maduro}
\end{EntryWithPhonetic}

\begin{EntryWithPhonetic}{发源地}{fa1yuan2di4}{5,13,6}{⼜、⽔、⼟}[HSK 7-9]
  \definition{s.}{fonte; berço; lar; terra natal; lugar de origem; terra de origem | local de nascimento}[青藏高原是藏族文化的发源地。===O Planalto Qinghai-Tibete é o berço da cultura tibetana.]
\end{EntryWithPhonetic}

\begin{EntryWithPhonetic}{发展}{fa1zhan3}{5,10}{⼜、⼫}[HSK 3]
  \definition{v.}{crescer; expandir; avançar; desenvolver; a mudança das coisas de pequeno para grande, de simples para complexo, de inferior para superior | recrutar; admitir expandir (organização, escala, etc.)}
\end{EntryWithPhonetic}

\begin{EntryWithPhonetic}{发作}{fa1zuo4}{5,7}{⼜、⼈}[HSK 7-9]
  \definition{v.}{sair; mostrar efeito; a doença no corpo se manifesta repentinamente ou o álcool ou as drogas fazem efeito | explodir; ter um ataque de raiva; perder a paciência porque está muito zangado ou insatisfeito}
\end{EntryWithPhonetic}

%%%%%%%%%% 罚 %%%%%%%%%%
\subsection*{罚}

\begin{EntryWithPhonetic}{罚}{fa2}{9}{⽹}[HSK 5]
  \definition{s.}{punição; penalidade}
  \definition{v.}{punir; penalizar; multar; confiscar}
\end{EntryWithPhonetic}

\begin{EntryWithPhonetic}{罚款}{fa2/kuan3}{9,12}{⽹、⽋}[HSK 5]
  \definition[笔,次,宗]{s.}{multa; penalidade; refere-se ao dinheiro pago por uma pessoa ou entidade de acordo com as disposições de um delito ou violação de contrato ou contrato}
  \definition{v.+compl.}{multar; penalizar; exigir, de acordo com os regulamentos, uma determinada quantia de dinheiro de uma pessoa ou entidade que tenha violado a lei ou descumprido um regulamento ou contrato}
\end{EntryWithPhonetic}

%%%%%%%%%% 阀 %%%%%%%%%%
\subsection*{阀}

\begin{EntryWithPhonetic}{阀}{fa2}{9}{⾨}
  \definition[个]{s.}{casa estabelecida ou grupo de poder; uma pessoa ou família poderosa; refere-se a uma pessoa ou família que tem uma influência dominante em uma determinada área | válvula (mecânica)}
\end{EntryWithPhonetic}

\begin{EntryWithPhonetic}{阀门}{fa2men2}{9,3}{⾨、⾨}[HSK 7-9]
  \definition{s.}{válvula (mecânica); dispositivos para controlar o fluxo de água e ar em máquinas e tubulações}
\end{EntryWithPhonetic}

%%%%%%%%%% 筏 %%%%%%%%%%
\subsection*{筏}

\begin{EntryWithPhonetic}{筏}{fa2}{12}{⽵}
  \definition[条]{s.}{jangada (de troncos, bambus, etc.)}
\end{EntryWithPhonetic}

%%%%%%%%%% 法 %%%%%%%%%%
\subsection*{法}

\begin{EntryWithPhonetic}{法}{fa3}{8}{⽔}[HSK 4]
  \definition*{s.}{Doutrina budista; o dharma | França, abreviação de 法国 | Sobrenome: Fa}
  \definition{adj.}{(usado após advérbios negativos) legal; cumpridor da lei}
  \definition{clas.}{F; Farad, medida de capacitância}
  \definition{s.}{lei; termo geral para regras de comportamento estabelecidas ou endossadas pelo Estado | maneira; método; modo; meios | padrão; modelo | artes mágicas; feitiço}
  \definition{v.}{seguir; imitar; aprender (os pontos fortes dos outros) |}
  \seealsoref{法国}{fa3guo2}
\end{EntryWithPhonetic}

\begin{EntryWithPhonetic}{法官}{fa3 guan1}{8,8}{⽔、⼧}[HSK 4]
  \definition[位,名,个,些]{s.}{juiz; justiça; termo genérico para um membro do judiciário em um tribunal de justiça}
\end{EntryWithPhonetic}

\begin{EntryWithPhonetic}{法规}{fa3 gui1}{8,8}{⽔、⾒}[HSK 5]
  \definition[部,项,条,套,个]{s.}{lei e regulamento; estatuto; termo geral para leis, decretos, regulamentos, regras, estatutos, etc.}
\end{EntryWithPhonetic}

\begin{EntryWithPhonetic}{法国}{fa3guo2}{8,8}{⽔、⼞}
  \definition*{s.}{França}
\end{EntryWithPhonetic}

\begin{EntryWithPhonetic}{法国人}{fa3guo2ren2}{8,8,2}{⽔、⼞、⼈}
  \definition{s.}{francês | pessoa ou povo da França}
\end{EntryWithPhonetic}

\begin{EntryWithPhonetic}{法律}{fa3lv4}{8,9}{⽔、⼻}[HSK 4]
  \definition[项,条,套,个]{s.}{lei; estatuto; regras de conduta formuladas pelo legislativo e cuja aplicação é garantida pelo poder estatal}
\end{EntryWithPhonetic}

\begin{EntryWithPhonetic}{法庭}{fa3 ting2}{8,9}{⽔、⼴}[HSK 6]
  \definition{s.}{corte; tribunal | tribunal; um órgão estatal que exerce o poder judicial de forma independente}
\end{EntryWithPhonetic}

\begin{EntryWithPhonetic}{法网}{fa3wang3}{8,6}{⽔、⽹}
  \definition*{s.}{Torneio de Roland Garros (French Open), torneio de tênis}
\end{EntryWithPhonetic}

\begin{EntryWithPhonetic}{法文}{fa3wen2}{8,4}{⽔、⽂}
  \definition[份]{s.}{françês, língua francesa}
\end{EntryWithPhonetic}

\begin{EntryWithPhonetic}{法语}{fa3 yu3}{8,9}{⽔、⾔}[HSK 6]
  \definition[种,门,句,段]{s.}{françês, língua francesa}
\end{EntryWithPhonetic}

\begin{EntryWithPhonetic}{法院}{fa3yuan4}{8,9}{⽔、⾩}[HSK 4]
  \definition[所,座]{s.}{tribunal; corte; órgãos estatais que exercem poder judicial independente}
\end{EntryWithPhonetic}

\begin{EntryWithPhonetic}{法制}{fa3 zhi4}{8,8}{⽔、⼑}[HSK 5]
  \definition{s.}{legalidade; instituições jurídicas; sistema jurídico}
\end{EntryWithPhonetic}

%%%%%%%%%% 发 %%%%%%%%%%
\subsection*{发}

\begin{EntryWithPhonetic}{发}{fa4}{5}{⼜}
  \definition*{s.}{Sobrenome: Fa}
  \definition[件]{s.}{cabelo}
  \seeref{fa1}
\end{EntryWithPhonetic}

\begin{EntryWithPhonetic}{发型}{fa4xing2}{5,9}{⼜、⼟}[HSK 7-9]
  \definition[个,种]{s.}{penteado}
\end{EntryWithPhonetic}

\begin{EntryWithPhonetic}{发簪}{fa4zan1}{5,18}{⼜、⽵}
  \definition{s.}{grampo de cabelo}
\end{EntryWithPhonetic}

%%%%%%%%%% 帆 %%%%%%%%%%
\subsection*{帆}

\begin{EntryWithPhonetic}{帆}{fan1}{6}{⼱}[HSK 7-9]
  \definition{s.}{vela (de barco) | Literário: barco à vela; veleiro}
\end{EntryWithPhonetic}

\begin{EntryWithPhonetic}{帆船}{fan1chuan2}{6,11}{⼱、⾈}[HSK 7-9]
  \definition[艘,条]{s.}{veleiro; barco a vela; um navio que usa velas para se impulsionar com a ajuda do vento}
\end{EntryWithPhonetic}

%%%%%%%%%% 番 %%%%%%%%%%
\subsection*{番}

\begin{EntryWithPhonetic}{番}{fan1}{12}{⽥}[HSK 6]
  \definition{adj.}{estrangeiro; de tribos estrangeiras; estrangeiro ou alienígena}
  \definition{clas.}{usado para o número de vezes que uma ação é executada, equivalente a 回 ou 次 | usado para o tipo de coisas, equivalente a 种}
  \definition{s.}{estrangeiro; de tribos estrangeiras; (velho) refere-se a países estrangeiros ou raças estrangeiras | tomate; batata-doce | aborígenes; nativos; povos indígenas}
  \definition{v.}{revezar; rotacionar; substituir}
  \seealsoref{次}{ci4}
  \seealsoref{回}{hui2}
  \seealsoref{种}{zhong3}
\end{EntryWithPhonetic}

\begin{EntryWithPhonetic}{番茄}{fan1 qie2}{12,8}{⽥、⾋}[HSK 6]
  \definition[个,斤,磅,公斤]{s.}{tomate | tomateiro}
\end{EntryWithPhonetic}

%%%%%%%%%% 蕃 %%%%%%%%%%
\subsection*{蕃}

\begin{EntryWithPhonetic}{蕃}{fan1}{15}{⾋}
  \definition[种]{s.}{estrangeiros; aborígenes}
  \seeref{bo1}
  \seeref{fan2}
\end{EntryWithPhonetic}

\begin{EntryWithPhonetic}{蕃茄}{fan1 qie2}{15,8}{⾋、⾋}
  \variantof{番茄}
\end{EntryWithPhonetic}

%%%%%%%%%% 翻 %%%%%%%%%%
\subsection*{翻}

\begin{EntryWithPhonetic}{翻}{fan1}{18}{⽻}[HSK 4]
  \definition{v.}{virar; dar a volta; inverter; mudar de posição; torcer; reverter | vasculhar; procurar; pesquisar; mover objetos para localizar algo | reverter; retrair; retirar | passar por cima; ultrapassar; cruzar | multiplicar | traduzir; decodificar | romper-se; cair; desentender-se com alguém}
\end{EntryWithPhonetic}

\begin{EntryWithPhonetic}{翻番}{fan1/fan1}{18,12}{⽻、⽥}[HSK 7-9]
  \definition{v.+compl.}{aumentar em um número especificado de vezes; dobrar}
\end{EntryWithPhonetic}

\begin{EntryWithPhonetic}{翻过}{fan1guo4}{18,6}{⽻、⾡}
  \definition{v.}{virar |  transformar}
\end{EntryWithPhonetic}

\begin{EntryWithPhonetic}{翻来覆去}{fan1lai2-fu4qu4}{18,7,18,5}{⽻、⽊、⾑、⼛}[HSK 7-9]
  \definition{expr.}{jogar de um lado para o outro; tocar a mesma corda; repetir várias vezes; virar e se virar; dizer repetidamente; jogar inquieto de um lado para o outro; virar de um lado para o outro}
\end{EntryWithPhonetic}

\begin{EntryWithPhonetic}{翻脸}{fan1/lian3}{18,11}{⽻、⾁}
  \definition{v.+compl.}{brigar com alguém | tornar-se hostil}
\end{EntryWithPhonetic}

\begin{EntryWithPhonetic}{翻天覆地}{fan1tian1-fu4di4}{18,4,18,6}{⽻、⼤、⾑、⼟}[HSK 7-9]
  \definition{expr.}{virar o mundo de cabeça para baixo; uma mudança tremenda; abalar a terra; marcar época; virar o céu e a terra; sacudir o próprio chão (mundo); virar o mundo de cabeça para baixo; mudanças titânicas; ``Céu e terra virados de cabeça para baixo.''}
\end{EntryWithPhonetic}

\begin{EntryWithPhonetic}{翻译}{fan1yi4}{18,7}{⽻、⾔}[HSK 4]
  \definition[个,位,名]{s.}{tradutor; intérprete; pessoas que fazem trabalhos de tradução}
  \definition{v.}{traduzir; interpretar; colocar o significado de palavras de um idioma em palavras de outro idioma (expressão idiomática); expressar um significado em outro idioma}
\end{EntryWithPhonetic}

%%%%%%%%%% 凡 %%%%%%%%%%
\subsection*{凡}

\begin{EntryWithPhonetic}{凡}{fan2}{3}{⼏}[HSK 7-9]
  \definition*{s.}{Sobrenome: Fan}
  \definition{adj.}{comum; ordinário}
  \definition{adv.}{Literário: qualquer; todos; todo | Literário: em tudo; completamente}
  \definition{s.}{este mundo mortal; a terra | o mundo secular; refere-se ao mundo humano | uma nota da escala em Gongchepu (工尺谱), correspondente a 4 na notação musical numerada | Literário: ideia geral; esboço}
  \seealsoref{工尺谱}{gong1 che3 pu3}
\end{EntryWithPhonetic}

\begin{EntryWithPhonetic}{凡是}{fan2shi4}{3,9}{⼏、⽇}[HSK 6]
  \definition{adv.}{todos; qualquer; cada; resumir tudo dentro de um determinado âmbito}
\end{EntryWithPhonetic}

%%%%%%%%%% 烦 %%%%%%%%%%
\subsection*{烦}

\begin{EntryWithPhonetic}{烦}{fan2}{10}{⽕}[HSK 4]
  \definition{adj.}{redundante e confuso | supérfluo e confuso; muito bagunçado}
  \definition{v.}{aborrecer | irritar; incomodar; estar cansado de; ficar irritado | incomodar; solicitar}
\end{EntryWithPhonetic}

\begin{EntryWithPhonetic}{烦闷}{fan2men4}{10,7}{⽕、⾨}[HSK 7-9]
  \definition{adj.}{infeliz; deprimido; mal-humorado | desconfortável}
\end{EntryWithPhonetic}

\begin{EntryWithPhonetic}{烦恼}{fan2nao3}{10,9}{⽕、⼼}[HSK 7-9]
  \definition{adj.}{irritado; preocupado; incomodado}
  \definition[个,种,些]{s.}{aborrecimento; coisas que te incomodam}
\end{EntryWithPhonetic}

\begin{EntryWithPhonetic}{烦躁}{fan2zao4}{10,20}{⽕、⾜}[HSK 7-9]
  \definition{adj.}{inquieto; agitado; irritável}
\end{EntryWithPhonetic}

%%%%%%%%%% 蕃 %%%%%%%%%%
\subsection*{蕃}

\begin{EntryWithPhonetic}{蕃}{fan2}{15}{⾋}
  \definition{adj.}{exuberante; próspero}
  \definition{v.}{multiplicar; proliferar}
  \seeref{bo1}
  \seeref{fan1}
\end{EntryWithPhonetic}

%%%%%%%%%% 繁 %%%%%%%%%%
\subsection*{繁}

\begin{EntryWithPhonetic}{繁}{fan2}{17}{⽷}
  \definition{adj.}{em grande número; numerosos; múltiplos (oposto a 简) | em grande número; numerosos; complexos; complicado}
  \definition{v.}{propagar; multiplicar}
  \seealsoref{简}{jian3}
\end{EntryWithPhonetic}

\begin{EntryWithPhonetic}{繁花}{fan2hua1}{17,7}{⽷、⾋}
  \definition{s.}{Literário: flores desabrochadas; flores de cores diferentes; flores exuberantes; uma massa de flores}
\end{EntryWithPhonetic}

\begin{EntryWithPhonetic}{繁华}{fan2hua2}{17,6}{⽷、⼗}[HSK 7-9]
  \definition{adj.}{ocupado; agitado; próspero; florescente; (cidade, mercado de rua) movimentado e próspero}
\end{EntryWithPhonetic}

\begin{EntryWithPhonetic}{繁忙}{fan2mang2}{17,6}{⽷、⼼}[HSK 7-9]
  \definition{adj.}{ocupado; agitado; muita coisa para fazer, pouco tempo livre}
\end{EntryWithPhonetic}

\begin{EntryWithPhonetic}{繁荣}{fan2rong2}{17,9}{⽷、⾋}[HSK 5]
  \definition{adj.}{florescente; próspero}
  \definition{v.}{promover; prosperar}
\end{EntryWithPhonetic}

\begin{EntryWithPhonetic}{繁体字}{fan2ti3zi4}{17,7,6}{⽷、⼈、⼦}[HSK 7-9]
  \definition{s.}{forma complexa tradicional de um caractere chinês simplificado; caracteres chineses com mais traços que foram substituídos por caracteres simplificados; oposto a 简体字}
  \seealsoref{简体字}{jian3ti3zi4}
\end{EntryWithPhonetic}

\begin{EntryWithPhonetic}{繁殖}{fan2zhi2}{17,12}{⽷、⽍}[HSK 6]
  \definition{v.}{criar; reproduzir; propagar; multiplicar; os organismos produzem novos indivíduos}
\end{EntryWithPhonetic}

\begin{EntryWithPhonetic}{繁重}{fan2zhong4}{17,9}{⽷、⾥}[HSK 7-9]
  \definition{adj.}{pesado; oneroso; árduo; penoso; (trabalho, tarefas) muitas e pesadas}[我每天都有繁重的工作。===Tenho uma carga de trabalho pesada todos os dias.]
\end{EntryWithPhonetic}

%%%%%%%%%% 反 %%%%%%%%%%
\subsection*{反}

\begin{EntryWithPhonetic}{反}{fan3}{4}{⼜}[HSK 4]
  \definition{adj.}{oposto; contrário; invertido}
  \definition{adv.}{pelo contrário; inversamente}
  \definition{v.}{inverter o lado; de cabeça para baixo; na direção oposta | virar; converter | retornar | opor-se; combater; voltar-se contra | rebelar-se; revoltar-se | inferir; deduzir; raciocinar por analogia}
\end{EntryWithPhonetic}

\begin{EntryWithPhonetic}{反驳}{fan3bo2}{4,7}{⼜、⾺}[HSK 7-9]
  \definition{v.}{refutar; replicar; apresentar suas próprias razões para refutar teorias ou opiniões de outras pessoas que diferem das suas}
\end{EntryWithPhonetic}

\begin{EntryWithPhonetic}{反差}{fan3cha1}{4,9}{⼜、⼯}[HSK 7-9]
  \definition{s.}{contraste; o contraste de cores da foto ou do cenário é muito diferente, como o contraste entre o preto e o branco | discrepância; o contraste entre o bem e o mal, o alto e o baixo, etc. de pessoas ou coisas}
\end{EntryWithPhonetic}

\begin{EntryWithPhonetic}{反常}{fan3chang2}{4,11}{⼜、⼱}[HSK 7-9]
  \definition{adj.}{incomum; anormal; diferente da situação normal}
\end{EntryWithPhonetic}

\begin{EntryWithPhonetic}{反倒}{fan3dao4}{4,10}{⼜、⼈}[HSK 7-9]
  \definition{adv.}{em vez disso; pelo contrário}
  \definition{conj.}{em vez disso; pelo contrário; frequentemente acompanhadas por várias palavras que expressam negação}
\end{EntryWithPhonetic}

\begin{EntryWithPhonetic}{反对}{fan3dui4}{4,5}{⼜、⼨}[HSK 3]
  \definition{v.}{lutar; opor-se; objetar a; ser contra; discordar}
\end{EntryWithPhonetic}

\begin{EntryWithPhonetic}{反对党}{fan3dui4dang3}{4,5,10}{⼜、⼨、⼉}
  \definition{s.}{partido de oposição}
\end{EntryWithPhonetic}

\begin{EntryWithPhonetic}{反对派}{fan3dui4pai4}{4,5,9}{⼜、⼨、⽔}
  \definition{s.}{facção de oposição}
\end{EntryWithPhonetic}

\begin{EntryWithPhonetic}{反对票}{fan3dui4piao4}{4,5,11}{⼜、⼨、⽰}
  \definition{s.}{voto dissidente}
\end{EntryWithPhonetic}

\begin{EntryWithPhonetic}{反而}{fan3'er2}{4,6}{⼜、⽽}[HSK 4]
  \definition{adv.}{em vez disso; ao contrário de; contrário ao significado da frase anterior ou inesperado, desempenha o papel de uma reviravolta em uma frase}
\end{EntryWithPhonetic}

\begin{EntryWithPhonetic}{反复}{fan3fu4}{4,9}{⼜、⼢}[HSK 3]
  \definition{adv.}{repetidamente; de ​​novo e de novo; várias vezes}
  \definition{s.}{reversão; recaída; a situação anterior se repetiu}
  \definition{v.}{recuar; cortar e mudar; virar de cabeça para baixo; arrepender-se; aparecer várias vezes (usado principalmente em situações ruins)}
\end{EntryWithPhonetic}

\begin{EntryWithPhonetic}{反感}{fan3gan3}{4,13}{⼜、⼼}[HSK 7-9]
  \definition{adj.}{avesso; enojado; desgostoso; insatisfeito}
  \definition{s.}{antipatia; aversão; oposição ou insatisfação}
\end{EntryWithPhonetic}

\begin{EntryWithPhonetic}{反过来}{fan3 guo4lai2}{4,6,7}{⼜、⾡、⽊}[HSK 7-9]
  \definition{adv.}{inversamente; na ordem inversa; em direção oposta; indica uma reversão ou mudança de uma ação ou estado}
  \definition{conj.}{vice-versa; inversamente; na ordem inversa; ao contrário; usado para orientar relacionamentos de transição, como condições e causas}
  \definition{v.}{virar; virar para frente e para trás}
\end{EntryWithPhonetic}

\begin{EntryWithPhonetic}{反击}{fan3ji1}{4,5}{⼜、⼐}[HSK 7-9]
  \definition{v.}{revidar; responder fogo; contra-atacar}
\end{EntryWithPhonetic}

\begin{EntryWithPhonetic}{反抗}{fan3kang4}{4,7}{⼜、⼿}[HSK 6]
  \definition{s.}{resistência}
  \definition{v.}{revoltar-se; resistir; opor-se com ação}
\end{EntryWithPhonetic}

\begin{EntryWithPhonetic}{反馈}{fan3kui4}{4,12}{⼜、⾶}[HSK 7-9]
  \definition{s.}{\emph{feedback}; resposta; uma resposta ou reação a algo, informação, etc.}
  \definition{v.}{dar \emph{feedback}; enviar informações de volta; (informações, \emph{feedback}, etc.) retornar ao local de onde foi enviado}
\end{EntryWithPhonetic}

\begin{EntryWithPhonetic}{反面}{fan3mian4}{4,9}{⼜、⾯}[HSK 7-9]
  \definition{adj.}{oposto; negativo; ruim}
  \definition{s.}{costas; lado reverso; lado avesso; o lado de um objeto oposto à frente | o reverso de um estado de coisas, um problema, etc.; o outro lado de uma questão, problema, etc.}
\end{EntryWithPhonetic}

\begin{EntryWithPhonetic}{反思}{fan3si1}{4,9}{⼜、⼼}[HSK 7-9]
  \definition{v.}{refletir; introspectar; refletir sobre o passado e tirar lições dele}
\end{EntryWithPhonetic}

\begin{EntryWithPhonetic}{反弹}{fan3tan2}{4,11}{⼜、⼸}[HSK 7-9]
  \definition{s.}{rebote}
  \definition{v.}{recuperar; um objeto elástico retorna à sua forma original após ser deformado por uma força externa | rebater; saltar de volta; ressurgir; metáfora para recuperação de preço ou mercado | rebotar; quicar; ricochetear; um objeto em movimento salta na direção oposta quando encontra um obstáculo}
\end{EntryWithPhonetic}

\begin{EntryWithPhonetic}{反问}{fan3wen4}{4,6}{⼜、⾨}[HSK 6]
  \definition{v.}{fazer uma pergunta em resposta; responder a uma pergunta com outra pergunta | fazer uma pergunta retórica (uma pergunta com significado negativo)}
\end{EntryWithPhonetic}

\begin{EntryWithPhonetic}{反响}{fan3 xiang3}{4,9}{⼜、⼝}[HSK 6]
  \definition{s.}{eco; reverberação; repercusão}
\end{EntryWithPhonetic}

\begin{EntryWithPhonetic}{反省}{fan3xing3}{4,9}{⼜、⽬}[HSK 7-9]
  \definition{v.}{refletir sobre si mesmo; envolver-se em introspecção e autoexame; refletir sobre seus pensamentos e ações e examinar quaisquer erros; examinar a consciência; questionar-se; sondar a alma}
\end{EntryWithPhonetic}

\begin{EntryWithPhonetic}{反应}{fan3ying4}{4,7}{⼜、⼴}[HSK 3]
  \definition[个]{s.}{reação; resposta; opiniões, atitudes ou ações causadas pelo acontecimento}
  \definition{v.}{reagir; responder; atividade correspondente causada pela estimulação do organismo}
\end{EntryWithPhonetic}

\begin{EntryWithPhonetic}{反映}{fan3ying4}{4,9}{⼜、⽇}[HSK 4]
  \definition{s.}{reflexão; opiniões sobre pessoas ou situações}
  \definition{v.}{refletir; espelhar; figurativamente, trazer à tona a essência de uma questão objetiva (expressão idiomática); expressar a essência de algo objetivamente | relatar; tornar conhecido; informar às autoridades superiores | refletir; espelhar; a imagem de um objeto aparece invertida em outro objeto}
\end{EntryWithPhonetic}

\begin{EntryWithPhonetic}{反正}{fan3zheng4}{4,5}{⼜、⽌}[HSK 3]
  \definition{adv.}{de qualquer forma; de qualquer maneira; embora as circunstâncias sejam diferentes, o resultado é o mesmo | tudo igual; em qualquer caso; tom de voz que expressa afirmação categórica}
\end{EntryWithPhonetic}

%%%%%%%%%% 返 %%%%%%%%%%
\subsection*{返}

\begin{EntryWithPhonetic}{返}{fan3}{7}{⾡}
  \definition{v.}{retornar; vir ou voltar}
\end{EntryWithPhonetic}

\begin{EntryWithPhonetic}{返还}{fan3huan2}{7,7}{⾡、⾡}[HSK 7-9]
  \definition{s.}{remessa | restituição | devolução de algo ao seu dono original}
\end{EntryWithPhonetic}

\begin{EntryWithPhonetic}{返回}{fan3 hui2}{7,6}{⾡、⼞}[HSK 5]
  \definition{v.}{retornar; ir (voltar); reverter; recorrer; retroceder; voltar para (o lugar original)}
\end{EntryWithPhonetic}

%%%%%%%%%% 犯 %%%%%%%%%%
\subsection*{犯}

\begin{EntryWithPhonetic}{犯}{fan4}{5}{⽝}[HSK 6]
  \definition{s.}{criminoso}
  \definition{v.}{ofender; violar; ir contra | atacar; violar; trabalhar contra | fazer; ocorrer | voltar a; ter uma recorrência de; recair; retornar a (velhos hábitos)}
\end{EntryWithPhonetic}

\begin{EntryWithPhonetic}{犯愁}{fan4/chou2}{5,13}{⽝、⼼}[HSK 7-9]
  \definition{v.+compl.}{preocupar-se; estar ansioso}
\end{EntryWithPhonetic}

\begin{EntryWithPhonetic}{犯法}{fan4fa3}{5,8}{⽝、⽔}
  \definition{v.}{violar (quebrar) a lei}
\end{EntryWithPhonetic}

\begin{EntryWithPhonetic}{犯规}{fan4 gui1}{5,8}{⽝、⾒}[HSK 6]
  \definition{v.}{quebrar as regras; violar regras | Esporte: cometer uma falta contra}
\end{EntryWithPhonetic}

\begin{EntryWithPhonetic}{犯罪}{fan4/zui4}{5,13}{⽝、⽹}[HSK 6]
  \definition{v.+compl.}{cometer  um crime}
\end{EntryWithPhonetic}

%%%%%%%%%% 泛 %%%%%%%%%%
\subsection*{泛}

\begin{EntryWithPhonetic}{泛}{fan4}{7}{⽔}
  \definition{adj.}{extenso; amplo; geral; inespecífico | superficial; raso | amarelo (ficar amarelo)}
  \definition{v.}{Literário: flutuar; derivar | afastar-se; espalhar-se; transbordar | transbordar; inundar | emergir; sair}
\end{EntryWithPhonetic}

\begin{EntryWithPhonetic}{泛滥}{fan4lan4}{7,13}{⽔、⽔}[HSK 7-9]
  \definition{v.}{fluir; transbordar; inundar | espalhar sem controle; metáfora para coisas ruins se tornarem populares sem restrições}
\end{EntryWithPhonetic}

%%%%%%%%%% 饭 %%%%%%%%%%
\subsection*{饭}

\begin{EntryWithPhonetic}{饭}{fan4}{7}{⾷}[HSK 1]
  \definition{s.}{(empréstimo linguístico) fã, devoto}
  \definition[顿,份,碗,口,锅]{s.}{cereais cozidos; grãos cozidos | refeição; alimentos consumidos diariamente em horários regulares | trabalho; meio de subsistência; meio de vida}
\end{EntryWithPhonetic}

\begin{EntryWithPhonetic}{饭店}{fan4dian4}{7,8}{⾷、⼴}[HSK 1]
  \definition[家,个]{s.}{restaurante | hotel; hotel grande e bem equipado}
\end{EntryWithPhonetic}

\begin{EntryWithPhonetic}{饭馆}{fan4 guan3}{7,11}{⾷、⾷}[HSK 2]
  \definition[家,个]{s.}{restaurante; lanchonete}
\end{EntryWithPhonetic}

\begin{EntryWithPhonetic}{饭碗}{fan4wan3}{7,13}{⾷、⽯}[HSK 7-9]
  \definition[个,只]{s.}{tigela de arroz | Coloquial: emprego; meio de subsistência | Figurativo: meio de subsistência; maneira de ganhar a vida}
\end{EntryWithPhonetic}

%%%%%%%%%% 贩 %%%%%%%%%%
\subsection*{贩}

\begin{EntryWithPhonetic}{贩}{fan4}{8}{⾙}
  \definition[个]{s.}{comerciante; mascate; negociante; vendedor ambulante}
  \definition{v.}{(comerciantes) comprar para revender}
\end{EntryWithPhonetic}

\begin{EntryWithPhonetic}{贩卖}{fan4mai4}{8,8}{⾙、⼗}[HSK 7-9]
  \definition{v.}{vender; traficar}
\end{EntryWithPhonetic}

%%%%%%%%%% 范 %%%%%%%%%%
\subsection*{范}

\begin{EntryWithPhonetic}{范}{fan4}{9}{⾋}
  \definition*{s.}{Sobrenome: Fan}
  \definition{s.}{padrão; molde; matriz | modelo; exemplo; modelo a seguir | limites; escopo | restrição; limite}
\end{EntryWithPhonetic}

\begin{EntryWithPhonetic}{范成大}{fan4 cheng2da4}{9,6,3}{⾋、⼽、⼤}
  \definition*{s.}{Fan Chengda (1126–1193), de nome de cortesia Zhineng 致能 e também Youyuan 幼元, autodenominou-se Cishan Jushi 此山 居士em seus primeiros anos e Shihu Jushi 石湖 居士em seus últimos anos; natural do Condado de Wu, Suzhou (atual Cidade de Suzhou, Província de Jiangsu), foi um oficial, poeta e escritor durante a Dinastia Song do Sul; seu nome póstumo foi Wenmu 文穆}
\end{EntryWithPhonetic}

\begin{EntryWithPhonetic}{范畴}{fan4chou2}{9,12}{⾋、⽥}[HSK 7-9]
  \definition{s.}{tipo; domínio; escopo; alcance; categoria}
\end{EntryWithPhonetic}

\begin{EntryWithPhonetic}{范围}{fan4wei2}{9,7}{⾋、⼞}[HSK 3]
  \definition[个]{s.}{escopo; limite; alcance}
  \definition{v.}{estabelecer limites para; limitar o escopo de}
\end{EntryWithPhonetic}

%%%%%%%%%% 方 %%%%%%%%%%
\subsection*{方}

\begin{EntryWithPhonetic}{方}{fang1}{4}{⽅}[HSK 4][Kangxi 70]
  \definition*{s.}{Alquimia, 方术 | Sobrenome: Fang}
  \definition{adj.}{reto; honesto; imparcial}
  \definition{adv.}{exatamente quando; no momento em que}
  \definition{clas.}{usado para coisas quadradas | quadrado ou cúbico (geralmente metro quadrado ou cúbico)}
  \definition[个,张]{s.}{quadrado; um quadrado ou sólido com seis faces quadradas | matemática: potência; o número de vezes que uma quantidade deve ser multiplicada por si mesma | direção | lado; festa | lugar; região; localidade | maneira; método; solução | prescrição | lei; regra}
  \seealsoref{方术}{fang1 shu4}
\end{EntryWithPhonetic}

\begin{EntryWithPhonetic}{方案}{fang1'an4}{4,10}{⽅、⽊}[HSK 4]
  \definition[个,些,种]{s.}{plano; esquema; programa; planos específicos para tratar de um determinado problema | o esquema criado pelo governo; medidas ou regulamentações formuladas e implementadas pelo governo ou autoridades relevantes}
\end{EntryWithPhonetic}

\begin{EntryWithPhonetic}{方便}{fang1bian4}{4,9}{⽅、⼈}[HSK 2]
  \definition{adj.}{conveniente; sem complicações; sem dificuldades; muito fácil| adequado; condições ou circunstâncias adequadas}
  \definition{s.}{conveniência}
  \definition{v.}{ir ao banheiro; uma maneira delicada de dizer ``ir ao banheiro'' | facilitar; tornar algo conveniente para alguém; facilitar a realização de tarefas ou o alcance de objetivos | ter dinheiro sobrando}
\end{EntryWithPhonetic}

\begin{EntryWithPhonetic}{方便面}{fang1 bian4 mian4}{4,9,9}{⽅、⼈、⾯}[HSK 2]
  \definition[袋,包,碗,桶]{s.}{macarrão instantâneo}
\end{EntryWithPhonetic}

\begin{EntryWithPhonetic}{方法}{fang1fa3}{4,8}{⽅、⽔}[HSK 2]
  \definition[种,个,套,类]{s.}{método; meio; maneira; sobre os meios e procedimentos para resolver questões relacionadas com o pensamento, a fala e as ações, etc.}
\end{EntryWithPhonetic}

\begin{EntryWithPhonetic}{方方面面}{fang1fang1mian4mian4}{4,4,9,9}{⽅、⽅、⾯、⾯}[HSK 7-9]
  \definition{expr.}{todos os aspectos; todos os lados | multifacetado}
\end{EntryWithPhonetic}

\begin{EntryWithPhonetic}{方面}{fang1mian4}{4,9}{⽅、⾯}[HSK 2]
  \definition[个,种]{s.}{lado; campo; aspecto; respeito}
\end{EntryWithPhonetic}

\begin{EntryWithPhonetic}{方片}{fang1 pian4}{4,4}{⽅、⽚}
  \definition{s.}{ouros ♦ (em jogos de cartas)}
  \seealsoref{黑桃}{hei1 tao2}
  \seealsoref{红心}{hong2 xin1}
  \seealsoref{梅花}{mei2 hua1}
\end{EntryWithPhonetic}

\begin{EntryWithPhonetic}{方式}{fang1shi4}{4,6}{⽅、⼷}[HSK 3]
  \definition[种,个]{s.}{maneira; método}
\end{EntryWithPhonetic}

\begin{EntryWithPhonetic}{方术}{fang1 shu4}{4,5}{⽅、⽊}
  \definition{s.}{artes de cura, adivinhação, horóscopo etc. | Arcaico: artes sobrenaturais}
\end{EntryWithPhonetic}

\begin{EntryWithPhonetic}{方向}{fang1xiang4}{4,6}{⽅、⼝}[HSK 2]
  \definition[个,种]{s.}{direção; orientação; referindo-se a leste, sul, oeste, norte, sudeste, sudoeste, nordeste, noroeste, etc. | objetivo; meta; finalidade}
\end{EntryWithPhonetic}

\begin{EntryWithPhonetic}{方向盘}{fang1xiang4pan2}{4,6,11}{⽅、⼝、⽫}[HSK 7-9]
  \definition[个]{s.}{volante; um dispositivo em forma de roda para controlar a direção de viagem de navios, carros, etc.}
\end{EntryWithPhonetic}

\begin{EntryWithPhonetic}{方言}{fang1yan2}{4,7}{⽅、⾔}[HSK 7-9]
  \definition*{s.}{O primeiro Dicionário de Dialeto Chinês, editado por Yang Xiong, 扬雄, no século I, contendo mais de 9.000 caracteres}
  \definition[口]{s.}{dialeto; um ramo regional de uma língua, formado durante sua evolução, que difere da língua padrão e é usado apenas em uma determinada área}
  \seealsoref{扬雄}{yang2xiong2}
\end{EntryWithPhonetic}

\begin{EntryWithPhonetic}{方针}{fang1zhen1}{4,7}{⽅、⾦}[HSK 4]
  \definition[个,项]{s.}{política; diretriz; princípio orientador; orientação da direção e das metas de um empreendimento}
\end{EntryWithPhonetic}

%%%%%%%%%% 防 %%%%%%%%%%
\subsection*{防}

\begin{EntryWithPhonetic}{防}{fang2}{6}{⾩}[HSK 3]
  \definition*{s.}{Sobrenome: Fang}
  \definition{s.}{defesa | dique; aterro | barragem; represa; estrutura para conter a água}
  \definition{v.}{proteger contra; prevenir contra; tomar precauções contra | defender-se contra}
\end{EntryWithPhonetic}

\begin{EntryWithPhonetic}{防盗}{fang2dao4}{6,11}{⾩、⽫}[HSK 7-9]
  \definition{v.}{proteger-se contra roubos; tomar precauções contra ladrões; impedir que bandidos roubem}
\end{EntryWithPhonetic}

\begin{EntryWithPhonetic}{防盗门}{fang2dao4men2}{6,11,3}{⾩、⽫、⾨}[HSK 7-9]
  \definition{s.}{porta de segurança; equipado com trava antirroubo, ela resiste à abertura anormal sob certas condições e por um período determinado | porta à prova de roubo}
\end{EntryWithPhonetic}

\begin{EntryWithPhonetic}{防范}{fang2 fan4}{6,9}{⾩、⾋}[HSK 6]
  \definition{v.}{vigiar; estar em guarda; ficar de olho}
\end{EntryWithPhonetic}

\begin{EntryWithPhonetic}{防护}{fang2hu4}{6,7}{⾩、⼿}[HSK 7-9]
  \definition{s.}{abrigo; proteção; meios ou medidas para proteger pessoas, coisas, o meio ambiente, etc. de adoecer de receber danos ou destruição}
  \definition{v.}{abrigar; proteger; proteger pessoas, coisas e o meio ambiente de doenças, danos ou destruição}
\end{EntryWithPhonetic}

\begin{EntryWithPhonetic}{防火墙}{fang2huo3qiang2}{6,4,14}{⾩、⽕、⼟}[HSK 7-9]
  \definition[个,堵]{s.}{\emph{firewall} (Internet)}
\end{EntryWithPhonetic}

\begin{EntryWithPhonetic}{防晒}{fang2shai4}{6,10}{⾩、⽇}
  \definition{s.}{protetor solar}
\end{EntryWithPhonetic}

\begin{EntryWithPhonetic}{防守}{fang2shou3}{6,6}{⾩、⼧}[HSK 6]
  \definition{v.}{defender; guardar}
\end{EntryWithPhonetic}

\begin{EntryWithPhonetic}{防卫}{fang2wei4}{6,3}{⾩、⼙}[HSK 7-9]
  \definition{v.}{defender}
\end{EntryWithPhonetic}

\begin{EntryWithPhonetic}{防汛}{fang2xun4}{6,6}{⾩、⽔}[HSK 7-9]
  \definition{s.}{prevenção ou controle de inundações | controle de enchentes}
\end{EntryWithPhonetic}

\begin{EntryWithPhonetic}{防疫}{fang2yi4}{6,9}{⾩、⽧}[HSK 7-9]
  \definition{s.}{prevenção de epidemias | prevenção de doenças | proteção contra epidemias}
\end{EntryWithPhonetic}

\begin{EntryWithPhonetic}{防御}{fang2yu4}{6,12}{⾩、⼻}[HSK 7-9]
  \definition{v.}{guardar; defender; resistir ao ataque do inimigo}
\end{EntryWithPhonetic}

\begin{EntryWithPhonetic}{防止}{fang2zhi3}{6,4}{⾩、⽌}[HSK 3]
  \definition{v.}{evitar; prevenir; prevenir; proteger contra; preparar-se com antecedência para evitar que coisas ruins aconteçam}
\end{EntryWithPhonetic}

\begin{EntryWithPhonetic}{防治}{fang2zhi4}{6,8}{⾩、⽔}[HSK 5]
  \definition{s.}{tratamento preventivo; prevenção e cura; profilaxia e tratamento}
\end{EntryWithPhonetic}

%%%%%%%%%% 妨 %%%%%%%%%%
\subsection*{妨}

\begin{EntryWithPhonetic}{妨}{fang2}{7}{⼥}
  \definition{v.}{dificultar; entravar; impedir; obstruir | (no negativo ou interrogativo) prejudicar | interferir com}
\end{EntryWithPhonetic}

\begin{EntryWithPhonetic}{妨碍}{fang2'ai4}{7,13}{⼥、⽯}[HSK 7-9]
  \definition{v.}{dificultar; entravar; impedir; obstruir; impedir que as coisas corram bem}
\end{EntryWithPhonetic}

\begin{EntryWithPhonetic}{妨害}{fang2hai4}{7,10}{⼥、⼧}[HSK 7-9]
  \definition{v.}{prejudicar; pôr em risco; ser prejudicial a; causar dano; ferir}
\end{EntryWithPhonetic}

%%%%%%%%%% 房 %%%%%%%%%%
\subsection*{房}

\begin{EntryWithPhonetic}{房}{fang2}{8}{⼾}
  \definition*{s.}{Fang, a quarta das vinte e oito constelações nas quais a esfera celeste foi dividida, consistindo de quatro estrelas quase em linha reta em Escorpião | Sobrenome: Fang}
  \definition[幢,个,间]{s.}{casa; edifício | sala; quarto; câmara | estrutura semelhante a uma casa | um ramo de uma família extensa | loja; estoque | local de trabalho do artesão; oficina; moinho}
\end{EntryWithPhonetic}

\begin{EntryWithPhonetic}{房地产}{fang2di4chan3}{8,6,6}{⼾、⼟、⼇}[HSK 7-9]
  \definition{s.}{imóveis; um termo geral para imóveis e terrenos}
\end{EntryWithPhonetic}

\begin{EntryWithPhonetic}{房东}{fang2dong1}{8,5}{⼾、⼀}[HSK 3]
  \definition[个,位,名]{s.}{dono;  proprietário; senhorio; pessoas que alugam ou emprestam imóveis (para os 房客 )}
  \seealsoref{房客}{fang2ke4}
\end{EntryWithPhonetic}

\begin{EntryWithPhonetic}{房价}{fang2 jia4}{8,6}{⼾、⼈}[HSK 6]
  \definition{s.}{custo de moradia; tarifa de quarto | preço da casa}
\end{EntryWithPhonetic}

\begin{EntryWithPhonetic}{房间}{fang2jian1}{8,7}{⼾、⾨}[HSK 1]
  \definition[个,间,套]{s.}{sala; câmara; escritório; apartamento; divisões internas da casa}
\end{EntryWithPhonetic}

\begin{EntryWithPhonetic}{房客}{fang2ke4}{8,9}{⼾、⼧}[HSK 3]
  \definition{s.}{inquilino (de um quarto ou casa); hóspede (oposto a 房东) | inquilino; hóspede; pessoas que alugam ou emprestam imóveis para moradia (para o 房东)}
  \seealsoref{房东}{fang2dong1}
\end{EntryWithPhonetic}

\begin{EntryWithPhonetic}{房屋}{fang2 wu1}{8,9}{⼾、⼫}[HSK 3]
  \definition[间,所,套]{s.}{casas; habitação; edifícios}
\end{EntryWithPhonetic}

\begin{EntryWithPhonetic}{房主}{fang2zhu3}{8,5}{⼾、⼂}
  \definition{s.}{proprietário | dono de um imóvel}
\end{EntryWithPhonetic}

\begin{EntryWithPhonetic}{房子}{fang2 zi5}{8,3}{⼾、⼦}[HSK 1]
  \definition[栋,幢,座,套,间]{s.}{casa; edifício; prédio}
\end{EntryWithPhonetic}

\begin{EntryWithPhonetic}{房租}{fang2 zu1}{8,10}{⼾、⽲}[HSK 3]
  \definition[笔]{s.}{aluguel}
\end{EntryWithPhonetic}

%%%%%%%%%% 仿 %%%%%%%%%%
\subsection*{仿}

\begin{EntryWithPhonetic}{仿}{fang3}{6}{⼈}[HSK 7-9]
  \definition{adv.}{semelhante; como}
  \definition{s.}{caracteres escritos segundo um modelo de caligrafia | cartas modeladas a partir de uma cópia; palavras escritas de acordo com o modelo}
  \definition{v.}{imitar; copiar | assemelhar-se; ser como}
\end{EntryWithPhonetic}

\begin{EntryWithPhonetic}{仿佛}{fang3fu2}{6,7}{⼈、⼈}[HSK 6]
  \definition{adv.}{parece que; como se}
  \definition{v.}{ser como; parecer}
\end{EntryWithPhonetic}

\begin{EntryWithPhonetic}{仿制}{fang3zhi4}{6,8}{⼈、⼑}[HSK 7-9]
  \definition{v.}{copiar; imitar; ser modelado em}
\end{EntryWithPhonetic}

%%%%%%%%%% 访 %%%%%%%%%%
\subsection*{访}

\begin{EntryWithPhonetic}{访}{fang3}{6}{⾔}
  \definition{v.}{visitar; fazer uma visita; ligar para | procurar por meio de investigação ou busca; tentar obter; obter uma entrevista | entrevistar | investigar; procurar por meio de investigação (pesquisar)}
\end{EntryWithPhonetic}

\begin{EntryWithPhonetic}{访谈}{fang3tan2}{6,10}{⾔、⾔}[HSK 7-9]
  \definition{v.}{entrevistar; conversar; visitar e conversar}
\end{EntryWithPhonetic}

\begin{EntryWithPhonetic}{访问}{fang3wen4}{6,6}{⾔、⾨}[HSK 3]
  \definition{v.}{visitar; ligar; entrevistar; visitar e conversar com um objetivo específico | visitar um \emph{site}}
\end{EntryWithPhonetic}

%%%%%%%%%% 纺 %%%%%%%%%%
\subsection*{纺}

\begin{EntryWithPhonetic}{纺}{fang3}{7}{⽷}
  \definition{s.}{tecido de seda fina | um pano de seda fino}
  \definition{v.}{girar | enrolar}
\end{EntryWithPhonetic}

\begin{EntryWithPhonetic}{纺织}{fang3zhi1}{7,8}{⽷、⽷}[HSK 7-9]
  \definition{v.}{tecer; fiar; fiar algodão, linho, seda, lã e outras fibras em fios ou linhas e tecê-los em tecidos, cetim, lã, etc.}
\end{EntryWithPhonetic}

%%%%%%%%%% 放 %%%%%%%%%%
\subsection*{放}

\begin{EntryWithPhonetic}{放}{fang4}{8}{⽅}[HSK 1]
  \definition{v.}{deixar ir; libertar; soltar | ceder; deixar-se levar | levar para se alimentar; pastar | soltar; liberar (ou expelir) | exibir (um filme, etc.); reproduzir (um disco, etc.) | acender; inflamar | emprestar (dinheiro) com juros | tornar maior ou mais longo; soltar; abaixar | moderar (a atitude ou o comportamento de alguém) | (de flores) florescer; abrir | colocar; posicionar; deitar | fazer com que algo (ou alguém) caia no chão | deixar de lado; guardar (para uso futuro); conservar | (seguido por 着\dots 不\dots) permitir que algo permaneça (por fazer, por pegar, por usar, etc.) | adicionar; colocar | colocar em pastagem; soltar para caçar | deixar de lado; suspender; interromper | remover; aliviar; livrar-se; proteger; libertar | deixar-se levar; sem restrições; libertino | mandar embora; tirar o prisioneiro da prisão e deportá-lo para uma região remota | distribuir; emitir; lançar | atear fogo | expandir; ampliar; prolongar | reajustar-se até certo ponto; controlar suas ações, adotar uma determinada atitude, atingir um certo equilíbrio | derrubar}
\end{EntryWithPhonetic}

\begin{EntryWithPhonetic}{放鞭炮}{fang4bian1pao4}{8,18,9}{⽅、⾰、⽕}
  \definition{s.}{um conjunto de bombinhas ou traques}
\end{EntryWithPhonetic}

\begin{EntryWithPhonetic}{放出}{fang4chu1}{8,5}{⽅、⼐}
  \definition{v.}{liberar | libertar}
\end{EntryWithPhonetic}

\begin{EntryWithPhonetic}{放大}{fang4da4}{8,3}{⽅、⼤}[HSK 5]
  \definition{v.}{amplificar; magnificar; aumentar; ampliar; aumentar o tamanho de imagens, textos, sons, etc.}
\end{EntryWithPhonetic}

\begin{EntryWithPhonetic}{放到}{fang4 dao4}{8,8}{⽅、⼑}[HSK 3]
  \definition{v.}{colocar em; meter}
\end{EntryWithPhonetic}

\begin{EntryWithPhonetic}{放电}{fang4dian4}{8,5}{⽅、⽥}
  \definition{s.}{descarga elétrica}
\end{EntryWithPhonetic}

\begin{EntryWithPhonetic}{放飞}{fang4fei1}{8,3}{⽅、⾶}
  \definition{s.}{deixar voar}
\end{EntryWithPhonetic}

\begin{EntryWithPhonetic}{放过}{fang4guo4}{8,6}{⽅、⾡}[HSK 7-9]
  \definition{v.}{deixar escapar; perder}
\end{EntryWithPhonetic}

\begin{EntryWithPhonetic}{放假}{fang4/jia4}{8,11}{⽅、⼈}[HSK 1]
  \definition{v.}{tirar férias (ou feriado); ter um dia de folga}
  \definition{v.+compl.}{tirar férias (ou feriado); começar as férias; ter um dia de folga; estar de férias (feriado)}
\end{EntryWithPhonetic}

\begin{EntryWithPhonetic}{放弃}{fang4qi4}{8,7}{⽅、⼶}[HSK 5]
  \definition{v.}{desistir, abandonar; descartar (direitos originais, reivindicações, opiniões, etc.)}
\end{EntryWithPhonetic}

\begin{EntryWithPhonetic}{放弃权利}{fang4qi4 quan2li4}{8,7,6,7}{⽅、⼶、⽊、⼑}
  \definition{s.}{renúncia}
\end{EntryWithPhonetic}

\begin{EntryWithPhonetic}{放弃者}{fang4qi4zhe3}{8,7,8}{⽅、⼶、⽼}
  \definition{s.}{desistente}
\end{EntryWithPhonetic}

\begin{EntryWithPhonetic}{放任}{fang4ren4}{8,6}{⽅、⼈}
  \definition{v.}{ignorar | saciar-se | deixar sozinho}
\end{EntryWithPhonetic}

\begin{EntryWithPhonetic}{放水}{fang4/shui3}{8,4}{⽅、⽔}[HSK 7-9]
  \definition{v.+compl.}{ligar a água; deixar a água fluir, geralmente significa abrir a fonte de água ou fornecer uma determinada vazão de água | (reservatório, etc.) retirar água; drenar água de reservatórios, lagoas, etc. para irrigação ou outros fins | (em uma competição, etc.) facilitar as coisas para alguém; perder um jogo intencionalmente; deixar deliberadamente o adversário vencer facilmente durante uma partida}
\end{EntryWithPhonetic}

\begin{EntryWithPhonetic}{放肆}{fang4si4}{8,13}{⽅、⾀}[HSK 7-9]
  \definition{adj.}{desenfreado; devasso; atrevido; descontrolado; descreve agir de forma imprudente e sem escrúpulos}
\end{EntryWithPhonetic}

\begin{EntryWithPhonetic}{放松}{fang4song1}{8,8}{⽅、⽊}[HSK 4]
  \definition{v.}{relaxar; afrouxar; soltar; desprender}
\end{EntryWithPhonetic}

\begin{EntryWithPhonetic}{放下}{fang4 xia4}{8,3}{⽅、⼀}[HSK 2]
  \definition{v.}{deitar-se; colocar no chão| deixar ir; soltar; desistir; largar | colocar; acomodar; depositar}
\end{EntryWithPhonetic}

\begin{EntryWithPhonetic}{放心}{fang4xin1}{8,4}{⽅、⼼}[HSK 2]
  \definition{adj.}{despreocupado}
  \definition{v.}{confiar; ter confiança em alguém; sentir-se aliviado; ficar tranquilo; ficar com a consciência tranquila}
\end{EntryWithPhonetic}

\begin{EntryWithPhonetic}{放学}{fang4/xue2}{8,8}{⽅、⼦}[HSK 1]
  \definition{v.+compl.}{encerrar; sair da escola; as aulas terminaram; a escola acabou (por hoje); voltar para casa depois de um dia ou meio dia de aula}
\end{EntryWithPhonetic}

\begin{EntryWithPhonetic}{放养}{fang4yang3}{8,9}{⽅、⼋}
  \definition{v.}{criar (gado, peixes, culturas, etc.) | crescer | criar}
\end{EntryWithPhonetic}

\begin{EntryWithPhonetic}{放映}{fang4ying4}{8,9}{⽅、⽇}[HSK 7-9]
  \definition{v.}{mostrar (um filme); exibir; projetar; usar um dispositivo de luz forte para iluminar a imagem de uma foto ou filme em uma tela ou parede}
\end{EntryWithPhonetic}

\begin{EntryWithPhonetic}{放置}{fang4zhi4}{8,13}{⽅、⽹}[HSK 7-9]
  \definition{v.}{colocar; deitar; deixar de lado}
\end{EntryWithPhonetic}

\begin{EntryWithPhonetic}{放纵}{fang4zong4}{8,7}{⽅、⽷}[HSK 7-9]
  \definition{adj.}{grosseiro; inculto; autoindulgente; indisciplinado}
  \definition{v.}{satisfazer; ser conivente com; bajular; deixar alguém fazer o que quer}
\end{EntryWithPhonetic}

\begin{EntryWithPhonetic}{放走}{fang4zou3}{8,7}{⽅、⾛}
  \definition{v.}{permitir (uma pessoa ou um animal) ir | liberar | libertar}
\end{EntryWithPhonetic}

%%%%%%%%%% 飞 %%%%%%%%%%
\subsection*{飞}

\begin{EntryWithPhonetic}{飞}{fei1}{3}{⾶}[HSK 1][Kangxi 183]
  \definition{adj.}{inesperado; acidental; surgido do nada}
  \definition{adv.}{rapidamente; velozmente}
  \definition{s.}{roda livre de uma bicicleta}
  \definition{v.}{voar; esvoaçar; (pássaros, insetos, etc.) voar pelo ar batendo as asas | voar; utilizar máquinas motorizadas para se deslocar no ar | voar; (objetos naturais) flutuar ou esvoaçar no ar | volatilizar; evaporar; um gás se dissipar no ar | ir muito rapidamente; movimentar-se rapidamente, como se estivesse voando}
\end{EntryWithPhonetic}

\begin{EntryWithPhonetic}{飞船}{fei1 chuan2}{3,11}{⾶、⾈}[HSK 6]
  \definition{s.}{nave espacial; espaçonave | dirigível; aerobarco}
\end{EntryWithPhonetic}

\begin{EntryWithPhonetic}{飞弹}{fei1dan4}{3,11}{⾶、⼸}
  \definition{s.}{míssil (guiado) | bala perdida; bala disparada aleatoriamente | avião de mísseis; bomba voadora; bombas com equipamento de voo automático | dardo}
\end{EntryWithPhonetic}

\begin{EntryWithPhonetic}{飞碟}{fei1die2}{3,14}{⾶、⽯}
  \definition{s.}{disco-voador, OVNI, \emph{UFO} | \emph{frisbee}}
\end{EntryWithPhonetic}

\begin{EntryWithPhonetic}{飞机}{fei1ji1}{3,6}{⾶、⽊}[HSK 1]
  \definition[架,个]{s.}{avião; aeronave; aroplano}
\end{EntryWithPhonetic}

\begin{EntryWithPhonetic}{飞机票}{fei1ji1 piao4}{3,6,11}{⾶、⽊、⽰}
  \definition[张]{s.}{bilhete de avião; documento emitido mediante pagamento de passagem aérea, que autoriza o titular a viajar}
  \seealsoref{机票}{ji1 piao4}
\end{EntryWithPhonetic}

\begin{EntryWithPhonetic}{飞速}{fei1su4}{3,10}{⾶、⾡}[HSK 7-9]
  \definition{adj.}{em velocidade máxima}
\end{EntryWithPhonetic}

\begin{EntryWithPhonetic}{飞往}{fei1wang3}{3,8}{⾶、⼻}[HSK 7-9]
  \definition{v.}{voar para}[飞机将径直飞往圣保罗。===O avião voará diretamente para São Paulo.]
\end{EntryWithPhonetic}

\begin{EntryWithPhonetic}{飞翔}{fei1xiang2}{3,12}{⾶、⽻}[HSK 7-9]
  \definition{v.}{pairar; voar; circular no ar; voar em círculos}
\end{EntryWithPhonetic}

\begin{EntryWithPhonetic}{飞行}{fei1 xing2}{3,6}{⾶、⾏}[HSK 3]
  \definition{s.}{voo; aviação}
  \definition{v.}{voar; fazer um voo; (aviões, foguetes, etc.) voar no ar}
\end{EntryWithPhonetic}

\begin{EntryWithPhonetic}{飞行员}{fei1 xing2 yuan2}{3,6,7}{⾶、⾏、⼝}[HSK 6]
  \definition[名,班]{s.}{piloto; aviador; pilotos de aeronaves}
\end{EntryWithPhonetic}

\begin{EntryWithPhonetic}{飞跃}{fei1yue4}{3,11}{⾶、⾜}[HSK 7-9]
  \definition{v.}{saltar; crescer rapidamente}
\end{EntryWithPhonetic}

%%%%%%%%%% 非 %%%%%%%%%%
\subsection*{非}

\begin{EntryWithPhonetic}{非}{fei1}{8}{⾮}[HSK 4][Kangxi 175]
  \definition*{s.}{África, abreviação de 非洲 | Sobrenome: Fei}
  \definition{adv.}{Em resposta a 不, indica necessidade (deve)}
  \definition{pref.}{indicando negatividade ou exclusão}
  \definition{s.}{engano; erro}
  \definition{v.}{opor-se a; culpar; censurar | não estar em conformidade com; ser contrário a | não ser | ter que; simplesmente precisar (fazer algo)}
  \seealsoref{不}{bu4}
  \seealsoref{非洲}{fei1zhou1}
\end{EntryWithPhonetic}

\begin{EntryWithPhonetic}{非常}{fei1chang2}{8,11}{⾮、⼱}[HSK 1]
  \definition{adj.}{extraordinário; incomum; especial}
  \definition{adv.}{muito; extremamente; altamente}
\end{EntryWithPhonetic}

\begin{EntryWithPhonetic}{非得}{fei1dei3}{8,11}{⾮、⼻}[HSK 7-9]
  \definition{adv.}{(geralmente usado comcomitantemente com 不 ou 才) tem que; deve}[你非得服从命令不可。===Você deve obedecer às ordens.]
  \seealsoref{不}{bu4}
  \seealsoref{才}{cai2}
\end{EntryWithPhonetic}

\begin{EntryWithPhonetic}{非法}{fei1fa3}{8,8}{⾮、⽔}[HSK 7-9]
  \definition{adj.}{ilegal; ilícito; fora da lei}
\end{EntryWithPhonetic}

\begin{EntryWithPhonetic}{非凡}{fei1fan2}{8,3}{⾮、⼏}[HSK 7-9]
  \definition{adj.}{excepcional; extraordinário; incomum; mais do que o normal}
\end{EntryWithPhonetic}

\begin{EntryWithPhonetic}{非金属}{fei1jin4shu3}{8,8,12}{⾮、⾦、⼫}[HSK 7-9]
  \definition{s.}{Química: não metal; metalóide; com exceção do bromo, os elementos que geralmente não têm brilho metálico nem ductilidade e não conduzem facilmente eletricidade ou calor são gases ou sólidos à temperatura ambiente, como oxigênio, enxofre, nitrogênio, fósforo, etc.}
\end{EntryWithPhonetic}

\begin{EntryWithPhonetic}{非洲}{fei1zhou1}{8,9}{⾮、⽔}
  \definition*{s.}{África}
\end{EntryWithPhonetic}

\begin{EntryWithPhonetic}{非洲人}{fei1zhou1ren2}{8,9,2}{⾮、⽔、⼈}
  \definition{s.}{africano | pessoa ou povo da África}
\end{EntryWithPhonetic}

%%%%%%%%%% 绯 %%%%%%%%%%
\subsection*{绯}

\begin{EntryWithPhonetic}{绯}{fei1}{11}{⽷}
  \definition{adj.}{escarlate; vermelho; vermelho escuro; vermelho profundo}
\end{EntryWithPhonetic}

\begin{EntryWithPhonetic}{绯闻}{fei1wen2}{11,9}{⽷、⾨}[HSK 7-9]
  \definition{s.}{boato/fofoca sobre escândalos sexuais | escândalo sexual; rumores sobre relacionamentos entre homens e mulheres}
\end{EntryWithPhonetic}

%%%%%%%%%% 肥 %%%%%%%%%%
\subsection*{肥}

\begin{EntryWithPhonetic}{肥}{fei2}{8}{⾁}[HSK 4]
  \definition{adj.}{gordo; gorduroso; contém muita gordura (o oposto de 瘦, geralmente não usado para descrever pessoas) | fértil; rico | solto; largo; folgado; (roupas, etc.) largas (em oposição a 瘦) | lucrativo; rendendo bons lucros}
  \definition{s.}{fertilizante; esterco}
  \definition{v.}{fertilizar; tornar fértil ou obeso | enriquecer com renda ilegal, ilícita}
  \seealsoref{瘦}{shou4}
\end{EntryWithPhonetic}

\begin{EntryWithPhonetic}{肥料}{fei2liao4}{8,10}{⾁、⽃}[HSK 7-9]
  \definition[种,袋,把]{s.}{esterco; fertilizante}
\end{EntryWithPhonetic}

\begin{EntryWithPhonetic}{肥胖}{fei2pang4}{8,9}{⾁、⾁}[HSK 7-9]
  \definition{adj.}{gordo; obeso; corpulento; excesso de gordura corporal}
\end{EntryWithPhonetic}

\begin{EntryWithPhonetic}{肥沃}{fei2wo4}{8,7}{⾁、⽔}[HSK 7-9]
  \definition{adj.}{fértil; rico (de solo); (terra) contém mais nutrientes e água adequados para o crescimento das plantas}
\end{EntryWithPhonetic}

\begin{EntryWithPhonetic}{肥皂}{fei2zao4}{8,7}{⾁、⽩}[HSK 7-9]
  \definition[块,条]{s.}{sabão; produtos químicos usados ​​para limpeza}
\end{EntryWithPhonetic}

%%%%%%%%%% 诽 %%%%%%%%%%
\subsection*{诽}

\begin{EntryWithPhonetic}{诽}{fei3}{10}{⾔}
  \definition{v.}{calúnia}
\end{EntryWithPhonetic}

\begin{EntryWithPhonetic}{诽谤}{fei3bang4}{10,12}{⾔、⾔}[HSK 7-9]
  \definition{v.}{difamar; caluniar; falar mal; espalhar boatos e caluniar os outros; falar mal dos outros e prejudicar sua reputação}
\end{EntryWithPhonetic}

%%%%%%%%%% 废 %%%%%%%%%%
\subsection*{废}

\begin{EntryWithPhonetic}{废}{fei4}{8}{⼴}[HSK 7-9]
  \definition{adj.}{desperdíçado; inútil; fora de uso; inválido; tendo perdido sua função original | Literário: incapacitado; mutilado; aleijado; desabilitado}
  \definition{v.}{desistir; abandonar; abolir; revogar  | Coloquial: punir; bater em alguém | descartar; abandonar}
\end{EntryWithPhonetic}

\begin{EntryWithPhonetic}{废除}{fei4chu2}{8,9}{⼴、⾩}[HSK 7-9]
  \definition{v.}{revogar; anular; cancelar; abolir (uma lei, sistema, tratado, etc.)}
\end{EntryWithPhonetic}

\begin{EntryWithPhonetic}{废话}{fei4hua4}{8,8}{⼴、⾔}[HSK 7-9]
  \definition{s.}{lixo; absurdo; palavras supérfluas; palavras redundantes e inúteis}
  \definition{v.}{falar bobagens; conversa fiada}
\end{EntryWithPhonetic}

\begin{EntryWithPhonetic}{废品}{fei4pin3}{8,9}{⼴、⼝}[HSK 7-9]
  \definition[件,吨,批,堆]{s.}{produto residual; rejeito; descarte; produtos não qualificados; produto descartado; sucata; refugo; material rejeitado}
\end{EntryWithPhonetic}

\begin{EntryWithPhonetic}{废寝忘食}{fei4qin3-wang4shi2}{8,13,7,9}{⼴、⼧、⼼、⾷}[HSK 7-9]
  \definition{expr.}{esquecer de comer e dormir; estar totalmente absorvido em}
\end{EntryWithPhonetic}

\begin{EntryWithPhonetic}{废物}{fei4wu4}{8,8}{⼴、⽜}[HSK 7-9]
  \definition{s.}{lixo; material residual; coisas que perderam seu valor de uso original}
  \seeref{fei4wu5}
\end{EntryWithPhonetic}

\begin{EntryWithPhonetic}{废物}{fei4wu5}{8,8}{⼴、⽜}
  \definition{s.}{pessoa inútil; imprestável (insulto); uma metáfora para uma pessoa inútil (palavrão)}
  \seeref{fei4wu4}
\end{EntryWithPhonetic}

\begin{EntryWithPhonetic}{废墟}{fei4xu1}{8,14}{⼴、⼟}[HSK 7-9]
  \definition[片,堆,个]{s.}{ruínas; terreno baldio; um lugar como uma cidade ou vila que ficou deserta e desolada após ser destruída ou sofrer um desastre natural}
\end{EntryWithPhonetic}

%%%%%%%%%% 沸 %%%%%%%%%%
\subsection*{沸}

\begin{EntryWithPhonetic}{沸}{fei4}{8}{⽔}
  \definition{adj.}{fervente; borbulhante; em ebulição}
  \definition{v.}{ferver | borbulhar}
\end{EntryWithPhonetic}

\begin{EntryWithPhonetic}{沸沸扬扬}{fei4fei4yang2yang2}{8,8,6,6}{⽔、⽔、⼿、⼿}[HSK 7-9]
  \definition{expr.}{levantar uma confusão de críticas sobre; borbulhando de barulho; discutir animadamente; dar origem a muita discussão; em um rebuliço; ``Tão barulhento quanto água fervente.'', frequentemente usado para descrever muita discussão}
\end{EntryWithPhonetic}

\begin{EntryWithPhonetic}{沸腾}{fei4teng2}{8,13}{⽔、⾁}[HSK 7-9]
  \definition{v.}{ferver; vaporizar | fervilhar de excitação; uma metáfora para alto astral ou vozes barulhentas}
\end{EntryWithPhonetic}

%%%%%%%%%% 狒 %%%%%%%%%%
\subsection*{狒}

\begin{EntryWithPhonetic}{狒}{fei4}{8}{⽝}
  \definition{s.}{babuíno (uma espécie de macaco)}
\end{EntryWithPhonetic}

\begin{EntryWithPhonetic}{狒狒}{fei4fei4}{8,8}{⽝、⽝}
  \definition{s.}{babuíno}
\end{EntryWithPhonetic}

%%%%%%%%%% 肺 %%%%%%%%%%
\subsection*{肺}

\begin{EntryWithPhonetic}{肺}{fei4}{8}{⾁}[HSK 6]
  \definition[叶]{s.}{pulmão | pulmões; órgãos respiratórios de humanos e animais superiores}
\end{EntryWithPhonetic}

%%%%%%%%%% 费 %%%%%%%%%%
\subsection*{费}

\begin{EntryWithPhonetic}{费}{fei4}{9}{⾙}[HSK 3]
  \definition*{s.}{Sobrenome: Fei}
  \definition{s.}{taxa; despesa; encargo}
  \definition{v.}{custar; gastar; despender | ser desperdiçador; consumir em excesso; gastar algo muito rapidamente; consumo excessivo (oposto a 省)}
  \seealsoref{省}{sheng3}
\end{EntryWithPhonetic}

\begin{EntryWithPhonetic}{费劲}{fei4/jin4}{9,7}{⾙、⼒}[HSK 7-9]
  \definition{adj.}{extenuante}
  \definition{v.+compl.}{ser extenuante; precisar ou usar grande esforço; gastar energia; fazer coisas ou falar com cuidado; trabalhar duro}
\end{EntryWithPhonetic}

\begin{EntryWithPhonetic}{费用}{fei4 yong4}{9,5}{⾙、⽤}[HSK 3]
  \definition[笔,个]{s.}{custo; despesa; desembolso}
\end{EntryWithPhonetic}

%%%%%%%%%% 分 %%%%%%%%%%
\subsection*{分}

\begin{EntryWithPhonetic}{分}{fen1}{4}{⼑}[HSK 1,2]
  \definition{adj.}{filial (de uma organização)}
  \definition{clas.}{parte ou subdivisão | fração | um décimo (de certas unidades) | unidade de comprimento equivalente a 0,33cm | unidade de área (=66,666 metros quadrados) | unidade de peso (=1/2 grama) | minuto (unidade de tempo) | minuto (unidade de medida angular) | 0,01 yuan (unidade de dinheiro) | taxa de juros | marca; ponto; unidade de contagem para avaliação de notas, etc.}
  \definition{s.}{fração}
  \definition{v.}{separar; dividir; partir; dividir algo inteiro em várias partes ou separar coisas que estão ligadas entre si | atribuir; designar; distribuir | distinguir; diferenciar; diferenciar um do outro}
  \seeref{fen4}
\end{EntryWithPhonetic}

\begin{EntryWithPhonetic}{分辨}{fen1bian4}{4,16}{⼑、⾟}[HSK 7-9]
  \definition{s.}{resolução; descreve o grau em que imagens, exibições, sensores, etc. podem distinguir claramente os detalhes}
  \definition{v.}{distinguir; diferenciar}
\end{EntryWithPhonetic}

\begin{EntryWithPhonetic}{分别}{fen1bie2}{4,7}{⼑、⼑}[HSK 3]
  \definition{adv.}{diferentemente; de maneiras diferentes; expressar de maneiras diferentes | separadamente; individualmente; respectivamente}
  \definition{s.}{diferença; pontos diferentes}
  \definition{v.}{partir; deixar um ao outro; não estar mais junto | distinguir; diferenciar}
\end{EntryWithPhonetic}

\begin{EntryWithPhonetic}{分布}{fen1bu4}{4,5}{⼑、⼱}[HSK 4]
  \definition{v.}{espalhar; distribuir; dispersar (em uma determinada área)}
\end{EntryWithPhonetic}

\begin{EntryWithPhonetic}{分成}{fen1 cheng2}{4,6}{⼑、⼽}[HSK 5]
  \definition{v.}{dividir em; separar em; dividir dinheiro, bens, etc. de acordo com a porcentagem}
\end{EntryWithPhonetic}

\begin{EntryWithPhonetic}{分寸}{fen1cun5}{4,3}{⼑、⼨}[HSK 7-9]
  \definition{s.}{limites adequados para a fala ou ação; senso de propriedade (ou proporção)}
\end{EntryWithPhonetic}

\begin{EntryWithPhonetic}{分担}{fen1dan1}{4,8}{⼑、⼿}[HSK 7-9]
  \definition{v.}{contribuir; compartilhar a responsabilidade por; participar}
\end{EntryWithPhonetic}

\begin{EntryWithPhonetic}{分割}{fen1ge1}{4,12}{⼑、⼑}[HSK 7-9]
  \definition{v.}{cortar; separar; quebrar; separar uma coisa inteira ou conectada}
\end{EntryWithPhonetic}

\begin{EntryWithPhonetic}{分工}{fen1 gong1}{4,3}{⼑、⼯}[HSK 6]
  \definition[种]{s.}{divisão do trabalho}
  \definition{v.}{dividir o trabalho; envolver-se em várias tarefas diferentes, mas complementares}
\end{EntryWithPhonetic}

\begin{EntryWithPhonetic}{分公司}{fen1gong1si1}{4,4,5}{⼑、⼋、⼝}
  \definition{s.}{sucursal | filial de companhia}
\end{EntryWithPhonetic}

\begin{EntryWithPhonetic}{分红}{fen1/hong2}{4,6}{⼑、⽷}[HSK 7-9]
  \definition{v.+compl.}{receber dividendos; distribuir bônus; obter dividendos extras (lucros)}
\end{EntryWithPhonetic}

\begin{EntryWithPhonetic}{分化}{fen1hua4}{4,4}{⼑、⼔}[HSK 7-9]
  \definition{s.}{diferenciação}
  \definition{v.}{dividir-se; fragmentar; romper; separar-se; separar | Biologia: (células ou tecidos) diferenciar}
\end{EntryWithPhonetic}

\begin{EntryWithPhonetic}{分解}{fen1jie3}{4,13}{⼑、⾓}[HSK 5]
  \definition{v.}{quebrar; separar em partes; dividir um todo em seus componentes | resolver; decompor | Química: transformar uma substância em duas ou mais substâncias por meio de uma reação química | desintegrar-se; dividir-se; desunir uma organização | mediar; fazer a paz; resolver conflitos e disputas | explicar; defender-se}
\end{EntryWithPhonetic}

\begin{EntryWithPhonetic}{分开}{fen1/kai1}{4,4}{⼑、⼶}[HSK 2]
  \definition{v.+compl.}{separar; dividir; desacoplar; desembalar; romper; desfolhar; decolar; romper; distribuir; separar de (em); dividir\dots de\dots | separar; fazer com que uma pessoa ou algo deixe de estar junto com outra pessoa ou coisa}
\end{EntryWithPhonetic}

\begin{EntryWithPhonetic}{分类}{fen1/lei4}{4,9}{⼑、⽶}[HSK 5]
  \definition{v.+compl.}{ordenar; classificar; categorizar; classificar as coisas de acordo com sua natureza e características}
\end{EntryWithPhonetic}

\begin{EntryWithPhonetic}{分离}{fen1 li2}{4,10}{⼑、⼇}[HSK 5]
  \definition{v.}{cortar; separar (de coisas) | separar; sair; separar (de pessoas); partir (em uma longa viagem)}
\end{EntryWithPhonetic}

\begin{EntryWithPhonetic}{分量}{fen1liang4}{4,12}{⼑、⾥}
  \definition{s.}{componente vetorial}
  \seeref{fen4liang4}
  \seeref{fen4liang5}
\end{EntryWithPhonetic}

\begin{EntryWithPhonetic}{分裂}{fen1lie4}{4,12}{⼑、⾐}[HSK 6]
  \definition{s.}{fissão; divisão}
  \definition{v.}{dividir; separar; romper}
\end{EntryWithPhonetic}

\begin{EntryWithPhonetic}{分泌}{fen1mi4}{4,8}{⼑、⽔}[HSK 7-9]
  \definition{v.}{secretar; excretar}
\end{EntryWithPhonetic}

\begin{EntryWithPhonetic}{分明}{fen1ming2}{4,8}{⼑、⽇}[HSK 7-9]
  \definition{adj.}{claro; distinto; óbvio}
  \definition{adv.}{claramente; evidentemente; obviamente; os fatos são claros, óbvios e inquestionáveis}
\end{EntryWithPhonetic}

\begin{EntryWithPhonetic}{分配}{fen1pei4}{4,10}{⼑、⾣}[HSK 3]
  \definition{v.}{atribuir; dispor; organizar o trabalho, as tarefas, os recursos, o tempo, etc. | atribuir; compartilhar; distribuir dinheiro ou bens às pessoas envolvidas de acordo com um determinado plano, padrão ou regulamento}
\end{EntryWithPhonetic}

\begin{EntryWithPhonetic}{分歧}{fen1qi2}{4,8}{⼑、⽌}[HSK 7-9]
  \definition[点,个]{s.}{diferença; divergência; diferenças de pontos de vista, opiniões, etc.}
\end{EntryWithPhonetic}

\begin{EntryWithPhonetic}{分散}{fen1san4}{4,12}{⼑、⽁}[HSK 4]
  \definition{adj.}{espalhado; disperso; desviado; fragmentado; sem foco}
  \definition{v.}{dispersar; espalhar; descentralizar | separar-se; desunir-se}
\end{EntryWithPhonetic}

\begin{EntryWithPhonetic}{分手}{fen1/shou3}{4,4}{⼑、⼿}[HSK 4]
  \definition{v.+compl.}{separar; romper; terminar um relacionamento ou um casal | separar-se (de uma empresa); dizer adeus; despedir-se da família, dos amigos, etc.}
\end{EntryWithPhonetic}

\begin{EntryWithPhonetic}{分数}{fen1 shu4}{4,13}{⼑、⽁}[HSK 2]
  \definition[个]{s.}{fração; número fracionário | nota; classificação; ponto; pontuação registrada ao avaliar o resultado ou a vitória/derrota}
\end{EntryWithPhonetic}

\begin{EntryWithPhonetic}{分为}{fen1 wei2}{4,4}{⼑、⼂}[HSK 4]
  \definition{v.}{subdividir; dividir algo em}
\end{EntryWithPhonetic}

\begin{EntryWithPhonetic}{分析}{fen1xi1}{4,8}{⼑、⽊}[HSK 5]
  \definition{v.}{analisar; dividir uma coisa, um fenômeno, um conceito em componentes mais simples e descobrir as propriedades essenciais desses componentes e a relação entre eles (em oposição à 综合)}
  \seealsoref{综合}{zong1he2}
\end{EntryWithPhonetic}

\begin{EntryWithPhonetic}{分享}{fen1 xiang3}{4,8}{⼑、⼇}[HSK 5]
  \definition{v.}{compartilhar; partilhar}
\end{EntryWithPhonetic}

\begin{EntryWithPhonetic}{分赃}{fen1/zang1}{4,10}{⼑、⾙}[HSK 7-9]
  \definition{v.+compl.}{dividir ganhos ilícitos; compartilhar o saque}
\end{EntryWithPhonetic}

\begin{EntryWithPhonetic}{分之}{fen1 zhi1}{4,3}{⼑、⼂}[HSK 4]
  \definition{expr.}{indicando uma fração; formatação e leitura de frações, ou seja, partes de um total}[顾客减少了三分之一。===O número de clientes caiu em um terço.]
\end{EntryWithPhonetic}

\begin{EntryWithPhonetic}{分支}{fen1zhi1}{4,4}{⼑、⽀}[HSK 7-9]
  \definition{s.}{subdivisão; filial; afiliada; uma parte separada de um sistema ou corpo}
\end{EntryWithPhonetic}

\begin{EntryWithPhonetic}{分钟}{fen1zhong1}{4,9}{⼑、⾦}[HSK 2]
  \definition{clas.}{minuto (usado em intervalos de tempo); 60 segundos}
\end{EntryWithPhonetic}

\begin{EntryWithPhonetic}{分子}{fen1zi3}{4,3}{⼑、⼦}
  \definition{s.}{molécula | (matemática) numerador de uma fração}
  \seeref{fen4zi3}
\end{EntryWithPhonetic}

\begin{EntryWithPhonetic}{分组}{fen1 zu3}{4,8}{⼑、⽷}[HSK 3]
  \definition{v.}{dividir em grupos}
\end{EntryWithPhonetic}

%%%%%%%%%% 吩 %%%%%%%%%%
\subsection*{吩}

\begin{EntryWithPhonetic}{吩}{fen1}{7}{⼝}
  \definition{v.}{deixar instruções; instruir | ordenar; mandar}
\end{EntryWithPhonetic}

\begin{EntryWithPhonetic}{吩咐}{fen1fu4}{7,8}{⼝、⼝}[HSK 7-9]
  \definition{v.}{dizer; instruir; comandar; dizer a alguém para lembrar o que deve ou não ser feito}
\end{EntryWithPhonetic}

%%%%%%%%%% 纷 %%%%%%%%%%
\subsection*{纷}

\begin{EntryWithPhonetic}{纷}{fen1}{7}{⽷}
  \definition[场]{adj.}{confuso; emaranhado; desordenado | muitos e variados; profusos; numerosos}
\end{EntryWithPhonetic}

\begin{EntryWithPhonetic}{纷纷}{fen1fen1}{7,7}{⽷、⽷}[HSK 4]
  \definition{adj.}{numeroso e confuso; muitos e desordenados}
  \definition{adv.}{um após o outro; em sucessão; em rápida sucessão}
\end{EntryWithPhonetic}

%%%%%%%%%% 芬 %%%%%%%%%%
\subsection*{芬}

\begin{EntryWithPhonetic}{芬}{fen1}{7}{⾋}
  \definition*{s.}{Sobrenome: Fen}
  \definition{adj.}{perfumado}
  \definition[阵,股]{s.}{cheiro doce; fragrância; aroma}
\end{EntryWithPhonetic}

\begin{EntryWithPhonetic}{芬芳}{fen1fang1}{7,7}{⾋、⾋}[HSK 7-9]
  \definition{adj.}{perfumado; com cheiro doce}
  \definition{s.}{fragrância; aroma; cheiro doce}
\end{EntryWithPhonetic}

%%%%%%%%%% 氛 %%%%%%%%%%
\subsection*{氛}

\begin{EntryWithPhonetic}{氛}{fen1}{8}{⽓}
  \definition{s.}{atmosfera; gás}
\end{EntryWithPhonetic}

\begin{EntryWithPhonetic}{氛围}{fen1wei2}{8,7}{⽓、⼞}[HSK 7-9]
  \definition[种,片,股]{s.}{atmosfera; a atmosfera e o humor ao redor}
\end{EntryWithPhonetic}

%%%%%%%%%% 坟 %%%%%%%%%%
\subsection*{坟}

\begin{EntryWithPhonetic}{坟}{fen2}{7}{⼟}[HSK 7-9]
  \definition[座,片,个]{s.}{sepultura; túmulo; cova}
\end{EntryWithPhonetic}

\begin{EntryWithPhonetic}{坟墓}{fen2mu4}{7,13}{⼟、⼟}[HSK 7-9]
  \definition[座,片,个]{s.}{sepultura; túmulo; a cova funerária e a sepultura acima dela}
\end{EntryWithPhonetic}

%%%%%%%%%% 焚 %%%%%%%%%%
\subsection*{焚}

\begin{EntryWithPhonetic}{焚}{fen2}{12}{⽕}
  \definition{v.}{queimar}
\end{EntryWithPhonetic}

\begin{EntryWithPhonetic}{焚烧}{fen2shao1}{12,10}{⽕、⽕}[HSK 7-9]
  \definition{v.}{queimar; incendiar; colocar fogo}
\end{EntryWithPhonetic}

\begin{EntryWithPhonetic}{焚香}{fen2xiang1}{12,9}{⽕、⾹}
  \definition{v.}{queimar incenso}
\end{EntryWithPhonetic}

%%%%%%%%%% 粉 %%%%%%%%%%
\subsection*{粉}

\begin{EntryWithPhonetic}{粉}{fen3}{10}{⽶}[HSK 7-9]
  \definition{adj.}{branco | rosa}
  \definition{s.}{pó | cosméticos em pó | farinha de trigo | macarrão ou outro alimento feito de feijão, arroz, batata, amido de batata-doce, etc. | macarrão de arroz}
  \definition{v.}{virar pó | Dialeto: caiar}
\end{EntryWithPhonetic}

\begin{EntryWithPhonetic}{粉色}{fen3 se4}{10,6}{⽶、⾊}
  \definition{s.}{cor-de-rosa}
\end{EntryWithPhonetic}

\begin{EntryWithPhonetic}{粉丝}{fen3si1}{10,5}{⽶、⼀}[HSK 7-9]
  \definition{s.}{(empréstimo linguístico) fã | entusiasta de alguém ou alguma coisa}
  \definition[个,群,位,名,些,批]{s.}{aletria de amido de feijão ou batata; aletria chinesa; macarrão de celofane ou macarrão de vidro (transparente) | Empréstimo linguístico: fã; refere-se a uma pessoa que é obcecada ou adora uma celebridade}
\end{EntryWithPhonetic}

\begin{EntryWithPhonetic}{粉碎}{fen3sui4}{10,13}{⽶、⽯}[HSK 7-9]
  \definition{adj.}{pulverizado; quebrado em pedaços; descreve algo que está muito quebrado, quebrado em partículas muito pequenas}
  \definition{v.}{esmagar; transformar as coisas em partículas muito pequenas | esmagar; quebrar; estilhaçar; fazer com que a outra parte falhe ou seja completamente destruída}
\end{EntryWithPhonetic}

%%%%%%%%%% 分 %%%%%%%%%%
\subsection*{分}

\begin{EntryWithPhonetic}{分}{fen4}{4}{⼑}[HSK 2]
  \definition{s.}{componente | o que está dentro dos deveres ou direitos de alguém; limites das responsabilidades e direitos | afeto; sentimento de amizade}
  \definition{v.}{pensar; esperar; estimar}
  \seeref{fen1}
\end{EntryWithPhonetic}

\begin{EntryWithPhonetic}{分量}{fen4liang4}{4,12}{⼑、⾥}[HSK 7-9]
  \definition{s.}{tamanho da porção (comida)}[这道菜的分量太少了。===A porção deste prato era muito pequena.]
  \seeref{fen1liang4}
  \seeref{fen4liang5}
\end{EntryWithPhonetic}

\begin{EntryWithPhonetic}{分量}{fen4liang5}{4,12}{⼑、⾥}
  \definition{s.}{quantidade; peso; medida | Figurativo: peso (importância, prestígio, autoridade etc.); (de material escrito) densidade}
  \seeref{fen1liang4}
  \seeref{fen4liang4}
\end{EntryWithPhonetic}

\begin{EntryWithPhonetic}{分外}{fen4wai4}{4,5}{⼑、⼣}[HSK 7-9]
  \definition{adj.}{não é tarefa (dever) de alguém; além do dever de alguém; fora do escopo do dever de alguém}[他对分外的工作总是抢着干。===Ele está sempre disposto e ansioso para fazer trabalho extra.]
  \definition{adv.}{particularmente; especialmente; excepcionalmente}[雨后初晴的天空分外明朗。===O céu depois da chuva está excepcionalmente claro.]
\end{EntryWithPhonetic}

\begin{EntryWithPhonetic}{分子}{fen4zi3}{4,3}{⼑、⼦}
  \definition{s.}{membros de uma classe ou grupo | elementos políticos (como intelectuais ou extremistas)}
  \seeref{fen1zi3}
\end{EntryWithPhonetic}

%%%%%%%%%% 份 %%%%%%%%%%
\subsection*{份}

\begin{EntryWithPhonetic}{份}{fen4}{6}{⼈}
  \definition{clas.}{usado para emparelhar itens em grupos | usado para jornais, documentos, etc. | usado para partes de um todo | usado para aparência, estado, etc.}
  \definition{s.}{porção; parte | a unidade de divisão; usado após 省, 县, 年, 月,  indica a unidade de divisão | grau; extensão de algo}
  \seealsoref{年}{nian2}
  \seealsoref{省}{sheng3}
  \seealsoref{县}{xian4}
  \seealsoref{月}{yue4}
\end{EntryWithPhonetic}

\begin{EntryWithPhonetic}{份额}{fen4'e2}{6,15}{⼈、⾴}[HSK 7-9]
  \definition{s.}{quota; quinhão; porção; a proporção ou percentagem do todo}
\end{EntryWithPhonetic}

\begin{EntryWithPhonetic}{份量}{fen4liang5}{6,12}{⼈、⾥}
  \variantof{分量}
\end{EntryWithPhonetic}

%%%%%%%%%% 奋 %%%%%%%%%%
\subsection*{奋}

\begin{EntryWithPhonetic}{奋}{fen4}{8}{⼤}
  \definition{adv.}{energicamente; com força e espírito}
  \definition{v.}{esforçar-se; agir vigorosamente; preparar-se | levantar | aplicar energia; resolver; animar-se | acenar; sacudir; levantar}
\end{EntryWithPhonetic}

\begin{EntryWithPhonetic}{奋斗}{fen4dou4}{8,4}{⼤、⽃}[HSK 4]
  \definition{v.}{lutar; esforçar-se; batalhar; trabalhar duro para atingir um determinado objetivo}
\end{EntryWithPhonetic}

\begin{EntryWithPhonetic}{奋力}{fen4li4}{8,2}{⼤、⼒}[HSK 7-9]
  \definition{v.}{fazer tudo o que puder; não poupar esforços}
\end{EntryWithPhonetic}

\begin{EntryWithPhonetic}{奋勇}{fen4yong3}{8,9}{⼤、⼒}[HSK 7-9]
  \definition{v.}{reunir toda a coragem e energia; criar coragem}
\end{EntryWithPhonetic}

\begin{EntryWithPhonetic}{奋战}{fen4zhan4}{8,9}{⼤、⼽}
  \definition{v.}{lutar bravamente | trabalhar duro}
\end{EntryWithPhonetic}

%%%%%%%%%% 愤 %%%%%%%%%%
\subsection*{愤}

\begin{EntryWithPhonetic}{愤}{fen4}{12}{⼼}
  \definition{s.}{raiva; indignação; ressentimento; exasperação}
  \definition{v.}{ressentir-se; ficar indignado; ficar com raiva}
\end{EntryWithPhonetic}

\begin{EntryWithPhonetic}{愤怒}{fen4nu4}{12,9}{⼼、⼼}[HSK 6]
  \definition{adj.}{zangado; enraivecido; iracundo; furioso; emocionalmente agitado por extrema insatisfação}
\end{EntryWithPhonetic}

\begin{EntryWithPhonetic}{愤世嫉俗}{fen4shi4ji2su2}{12,5,13,9}{⼼、⼀、⼥、⼈}
  \definition{v.}{ser cínico | ser amargurado}
\end{EntryWithPhonetic}

%%%%%%%%%% 粪 %%%%%%%%%%
\subsection*{粪}

\begin{EntryWithPhonetic}{粪}{fen4}{12}{⽶}[HSK 7-9]
  \definition[缸,桶]{s.}{excremento; fezes; esterco}
  \definition{v.}{Literário: aplicar esterco; fertilizar | Literário: limpar; remover; eliminar; acabar com}
\end{EntryWithPhonetic}

\begin{EntryWithPhonetic}{粪便}{fen4bian4}{12,9}{⽶、⼈}[HSK 7-9]
  \definition{s.}{excremento; fezes; esterco; excrementos e urina}
\end{EntryWithPhonetic}

%%%%%%%%%% 丰 %%%%%%%%%%
\subsection*{丰}

\begin{EntryWithPhonetic}{丰}{feng1}{4}{⼁}
  \definition*{s.}{Sobrenome: Feng}
  \definition[阵,丝]{adj.}{cheio; rico; abundante | ótimo | bonito; de boa aparência; cheio e redondo}
\end{EntryWithPhonetic}

\begin{EntryWithPhonetic}{丰富}{feng1fu4}{4,12}{⼁、⼧}[HSK 3]
  \definition{adj.}{rico; abundante; pleno; (riqueza material, conhecimento, experiência, etc.) variedade ou quantidade}
  \definition{v.}{enriquecer}
\end{EntryWithPhonetic}

\begin{EntryWithPhonetic}{丰富多彩}{feng1fu4-duo1cai3}{4,12,6,11}{⼁、⼧、⼣、⼺}[HSK 7-9]
  \definition{expr.}{rica e colorida; variada e colorida; abundante e diversa; descreve uma grande variedade de tipos e cores}
\end{EntryWithPhonetic}

\begin{EntryWithPhonetic}{丰厚}{feng1hou4}{4,9}{⼁、⼚}[HSK 7-9]
  \definition{adj.}{rico e generoso; (renda, presentes, etc.) grande em quantidade e alto em valor | grosso; abundante e espesso}
\end{EntryWithPhonetic}

\begin{EntryWithPhonetic}{丰满}{feng1man3}{4,13}{⼁、⽔}[HSK 7-9]
  \definition{adj.}{(corpo ou partes do corpo) roliço; adulto; cheio e redondo; bem desenvolvido; roliço e bem proporcionado | abundante; adequado | completo; cheio}
\end{EntryWithPhonetic}

\begin{EntryWithPhonetic}{丰盛}{feng1sheng4}{4,11}{⼁、⽫}[HSK 7-9]
  \definition{adj.}{rico; abundante; substancial; suntuoso; descreve comida que é abundante e boa}[他吃了一顿丰盛的早餐。===Ele tomou um café da manhã substancial.]
\end{EntryWithPhonetic}

\begin{EntryWithPhonetic}{丰收}{feng1shou1}{4,6}{⼁、⽁}[HSK 5]
  \definition{v.}{ter uma boa colheita; obter uma colheita boa e abundante; obter bons resultados}
\end{EntryWithPhonetic}

\begin{EntryWithPhonetic}{丰硕}{feng1shuo4}{4,11}{⼁、⽯}[HSK 7-9]
  \definition{adj.}{rico; frutífero; abundante e substancial; usado principalmente para coisas abstratas}[她的研究成果非常丰硕。===Os resultados de sua pesquisa são muito frutíferos.]
\end{EntryWithPhonetic}

%%%%%%%%%% 风 %%%%%%%%%%
\subsection*{风}

\begin{EntryWithPhonetic}{风}{feng1}{4}{⾵}[HSK 1][Kangxi 182]
  \definition*{s.}{Sobrenome: Feng}
  \definition{adj.}{lendário; sem fundamento concreto | rápido; veloz | promíscuo; libertino; sedutor}
  \definition[阵,丝]{s.}{vento; fluxo de ar | prática; ambiente; costume | cena; vista | notícias; fofocas; rumores | comportamento; maneira; estilo | canção folclórica | certas doenças}
  \definition{v.}{colocar para secar ou arejar; secar ao vento}
\end{EntryWithPhonetic}

\begin{EntryWithPhonetic}{风暴}{feng1bao4}{4,15}{⾵、⽇}[HSK 6]
  \definition{s.}{tempestade; vendaval; um termo geral para perturbações violentas na atmosfera e mudanças drásticas no clima, como tempestades de areia, tornados, ciclones tropicais, etc. | tempestade; comoção violenta; uma metáfora para um evento tão poderoso que abala toda a sociedade}
\end{EntryWithPhonetic}

\begin{EntryWithPhonetic}{风波}{feng1bo1}{4,8}{⾵、⽔}[HSK 7-9]
  \definition[场]{s.}{Meteorologia: vento e ondas --- perturbação | Literário: tempestade | perturbação; onda de vento; uma tempestade em copo d'água | tempestade; crise; perturbação}
\end{EntryWithPhonetic}

\begin{EntryWithPhonetic}{风采}{feng1cai3}{4,8}{⾵、⾤}[HSK 7-9]
  \definition{s.}{graça; elegância; comportamento; carisma | estilo; graça; talento literário}
\end{EntryWithPhonetic}

\begin{EntryWithPhonetic}{风餐露宿}{feng1can1-lu4su4}{4,16,21,11}{⾵、⾷、⾬、⼧}[HSK 7-9]
  \definition{expr.}{comer ao vento e dormir no orvalho --- suportar as dificuldades de uma jornada árdua | o sol brilha no chão; para descrever as dificuldades de viajar ou viver ao ar livre, também é dito ``dormir ao ar livre''}
\end{EntryWithPhonetic}

\begin{EntryWithPhonetic}{风度}{feng1du4}{4,9}{⾵、⼴}[HSK 5]
  \definition{s.}{postura; comportamento; porte; conduta; atitude}
\end{EntryWithPhonetic}

\begin{EntryWithPhonetic}{风范}{feng1fan4}{4,9}{⾵、⾋}[HSK 7-9]
  \definition{s.}{comportamento; porte; postura | estilo; maneira; ar | modelo; protótipo}
\end{EntryWithPhonetic}

\begin{EntryWithPhonetic}{风风雨雨}{feng1feng1yu3yu3}{4,4,8,8}{⾵、⾵、⾬、⾬}[HSK 7-9]
  \definition{expr.}{altos e baixos | dificuldades e sofrimentos | fofocas infundadas}
\end{EntryWithPhonetic}

\begin{EntryWithPhonetic}{风格}{feng1ge2}{4,10}{⾵、⽊}[HSK 4]
  \definition{s.}{modo; estilo; maneira; caráter | características das criações literárias de diferentes épocas, povos, escolas ou indivíduos em termos de conteúdo ideológico e técnicas artísticas}
\end{EntryWithPhonetic}

\begin{EntryWithPhonetic}{风光}{feng1guang1}{4,6}{⾵、⼉}[HSK 5]
  \definition{s.}{cena; vista; paisagens naturais e humanas}
\end{EntryWithPhonetic}

\begin{EntryWithPhonetic}{风和日丽}{feng1he2-ri4li4}{4,8,4,7}{⾵、⼝、⽇、⼀}[HSK 7-9]
  \definition{expr.}{vento moderado, sol bonito; tempo bom e ensolarado; sol brilhante e uma brisa suave; clima quente e ensolarado; descreve o clima ensolarado e ameno (usado principalmente na primavera)}
\end{EntryWithPhonetic}

\begin{EntryWithPhonetic}{风景}{feng1jing3}{4,12}{⾵、⽇}[HSK 4]
  \definition[种,处,道]{s.}{cenário; paisagem; cenários e vistas que podem ser apreciados, inclui paisagens, flores, árvores, edifícios e determinados fenômenos naturais}
\end{EntryWithPhonetic}

\begin{EntryWithPhonetic}{风浪}{feng1lang4}{4,10}{⾵、⽔}[HSK 7-9]
  \definition{s.}{tempestade; ondas tempestuosas; vento e ondas na água | dificuldades; sofrimento; uma metáfora para uma experiência difícil ou perigosa}
\end{EntryWithPhonetic}

\begin{EntryWithPhonetic}{风力}{feng1li4}{4,2}{⾵、⼒}[HSK 7-9]
  \definition{s.}{força do vento; o poder do vento | energia eólica; alimentado pelo vento}
\end{EntryWithPhonetic}

\begin{EntryWithPhonetic}{风流}{feng1liu2}{4,10}{⾵、⽔}[HSK 7-9]
  \definition{adj.}{refinado e saboroso | descontrolado em espírito e comportamento | romântico; amoroso; licencioso | distinto e admirável | meritório e talentoso | talentoso e romântico; talentoso em letras e não convencional no estilo de vida | dissoluto; solto}
\end{EntryWithPhonetic}

\begin{EntryWithPhonetic}{风貌}{feng1mao4}{4,14}{⾵、⾘}[HSK 7-9]
  \definition{s.}{estilo e características; o estilo e a aparência das coisas | vista; cena; cenário | aparência e porte elegantes; o comportamento e a aparência de uma pessoa}
\end{EntryWithPhonetic}

\begin{EntryWithPhonetic}{风气}{feng1qi4}{4,4}{⾵、⽓}[HSK 7-9]
  \definition[种]{s.}{ethos; atmosfera; humor geral; prática comum; um \emph{hobby} ou hábito popular na sociedade ou em um grupo}
  \seealsoref{风尚}{feng1shang4}
\end{EntryWithPhonetic}

\begin{EntryWithPhonetic}{风情}{feng1qing2}{4,11}{⾵、⼼}[HSK 7-9]
  \definition{s.}{charme; bom temperamento; graça; expressão | charme; panorama; gosto elegante | charme; sentimentos amorosos; pornografia | costumes e práticas locais | a condição do vento; informações sobre direção e velocidade do vento | sensação; a sensação do ambiente circundante em um determinado ambiente}
\end{EntryWithPhonetic}

\begin{EntryWithPhonetic}{风趣}{feng1qu4}{4,15}{⾵、⾛}[HSK 7-9]
  \definition{adj.}{espirituoso; humorístico; (fala, texto, etc.) humorístico e interessante}
  \definition{s.}{sagacidade; humor; refere-se ao humor e ao gosto humorísticos e interessantes}
\end{EntryWithPhonetic}

\begin{EntryWithPhonetic}{风沙}{feng1sha1}{4,7}{⾵、⽔}[HSK 7-9]
  \definition{s.}{areia soprada pelo vento; areia levada pelo vento}
\end{EntryWithPhonetic}

\begin{EntryWithPhonetic}{风扇}{feng1shan4}{4,10}{⾵、⼾}
  \definition{s.}{ventilador elétrico}
\end{EntryWithPhonetic}

\begin{EntryWithPhonetic}{风尚}{feng1shang4}{4,8}{⾵、⼩}[HSK 7-9]
  \definition{s.}{costume (ou prática, hábito) predominante; as tendências e costumes sociais predominantes durante um determinado período}
  \seealsoref{风气}{feng1qi4}
\end{EntryWithPhonetic}

\begin{EntryWithPhonetic}{风水}{feng1shui3}{4,4}{⾵、⽔}[HSK 7-9]
  \definition*{s.}{Feng Shui}
  \definition{s.}{geomancia; refere-se à localização geográfica de locais residenciais, cemitérios, etc., como a direção dos veios de terra, montanhas e rios}
\end{EntryWithPhonetic}

\begin{EntryWithPhonetic}{风俗}{feng1su2}{4,9}{⾵、⼈}[HSK 4]
  \definition[种,个,些]{s.}{costumes; a soma de costumes sociais, maneiras, hábitos, etc., desenvolvidos ao longo do tempo}
\end{EntryWithPhonetic}

\begin{EntryWithPhonetic}{风味}{feng1wei4}{4,8}{⾵、⼝}[HSK 7-9]
  \definition{s.}{sabor especial; cor (ou sabor) local; características das coisas (referindo-se principalmente às características locais)}
\end{EntryWithPhonetic}

\begin{EntryWithPhonetic}{风险}{feng1xian3}{4,9}{⾵、⾩}[HSK 3]
  \definition[个,种]{s.}{risco; perigo; ameaça; riscos possíveis}
\end{EntryWithPhonetic}

\begin{EntryWithPhonetic}{风雨}{feng1yu3}{4,8}{⾵、⾬}[HSK 7-9]
  \definition{s.}{o clima; vento e chuva | tribulações; estresse e tempestade; provações e dificuldades; uma metáfora para uma situação difícil e dura}
\end{EntryWithPhonetic}

\begin{EntryWithPhonetic}{风云}{feng1yun2}{4,4}{⾵、⼆}[HSK 7-9]
  \definition{s.}{vento e nuvem; uma situação tempestuosa ou instável}
\end{EntryWithPhonetic}

\begin{EntryWithPhonetic}{风筝}{feng1zheng5}{4,12}{⾵、⽵}[HSK 7-9]
  \definition[个,只]{s.}{pipa; papagaio; pandorga; é feito de tiras de bambu amarradas em um esqueleto em forma de pássaros, insetos, peixes, dragões, etc., coberto com papel ou seda, e flutua no ar com a ajuda do vento, as pessoas puxam a longa linha amarrada a ele para controlá-lo}
\end{EntryWithPhonetic}

%%%%%%%%%% 枫 %%%%%%%%%%
\subsection*{枫}

\begin{EntryWithPhonetic}{枫}{feng1}{8}{⽊}
  \definition[棵]{s.}{goma doce chinesa | bordo; \emph{maple}}
\end{EntryWithPhonetic}

\begin{EntryWithPhonetic}{枫叶}{feng1ye4}{8,5}{⽊、⼝}
  \definition{s.}{folha de bordo (maple, tipo de árvore)}
\end{EntryWithPhonetic}

%%%%%%%%%% 封 %%%%%%%%%%
\subsection*{封}

\begin{EntryWithPhonetic}{封}{feng1}{9}{⼨}[HSK 2,5]
  \definition*{s.}{Sobrenome: Feng}
  \definition{clas.}{usado para objetos selados, especialmente cartas}
  \definition{s.}{feudalismo | embalagem; envelope | pacote}
  \definition{v.}{conferir (um título, território, etc.) a | selar | acender uma fogueira | fechar}
\end{EntryWithPhonetic}

\begin{EntryWithPhonetic}{封闭}{feng1bi4}{9,6}{⼨、⾨}[HSK 4]
  \definition{adj.}{fechado; aqueles que não têm contato com o mundo exterior; aqueles que são muito conservadores (em seu pensamento) e não se comunicam com os outros}
  \definition{v.}{selar; fechar; lacrar; vedar; de modo a impedir a passagem, o uso ou a abertura}
\end{EntryWithPhonetic}

\begin{EntryWithPhonetic}{封底}{feng1di3}{9,8}{⼨、⼴}
  \definition{s.}{contracapa de um livro}
\end{EntryWithPhonetic}

\begin{EntryWithPhonetic}{封顶}{feng1ding3}{9,8}{⼨、⾴}[HSK 7-9]
  \definition{v.}{parar de crescer (de broto ou galho de planta) | Figurativo: impor um teto (sobre preços, salários, bônus, etc.) | cobrir (um edifício, etc.); cobrir o telhado (finalizar um projeto de construção); colocar um telhado (em um edifício); completar; encerrar | Figurativo: atingir o ponto mais alto (de crescimento, lucro, taxas de juros)}
\end{EntryWithPhonetic}

\begin{EntryWithPhonetic}{封冻}{feng1dong4}{9,7}{⼨、⼎}
  \definition{v.}{congelar (água ou terra)}
\end{EntryWithPhonetic}

\begin{EntryWithPhonetic}{封盖}{feng1gai4}{9,11}{⼨、⽫}
  \definition{s.}{arremesso bloqueado (basquete) | boné | capa | selo}
  \definition{v.}{(no basquete) bloquear (um arremesso) | cobrir}
\end{EntryWithPhonetic}

\begin{EntryWithPhonetic}{封建}{feng1jian4}{9,8}{⼨、⼵}[HSK 7-9]
  \definition{adj.}{feudal}
  \definition{s.}{feudalismo; sistema de feudo; o sistema político de feudos e estabelecimento de estados vassalos; esse sistema foi implementado pela primeira vez durante a Dinastia Zhou Ocidental | ideologia feudal; pensamento feudal}
\end{EntryWithPhonetic}

\begin{EntryWithPhonetic}{封口}{feng1kou3}{9,3}{⼨、⼝}
  \definition{v.}{selar | fechar | curar (uma ferida) | manter os lábios selados}
\end{EntryWithPhonetic}

\begin{EntryWithPhonetic}{封面}{feng1mian4}{9,9}{⼨、⾯}[HSK 7-9]
  \definition{s.}{capa (de uma publicação) ; capa frontal; sobrecapa}
\end{EntryWithPhonetic}

\begin{EntryWithPhonetic}{封锁}{feng1suo3}{9,12}{⼨、⾦}[HSK 7-9]
  \definition{v.}{bloquear; selar; tomar medidas militares, etc.) impedir a passagem}
\end{EntryWithPhonetic}

\begin{EntryWithPhonetic}{封印}{feng1yin4}{9,5}{⼨、⼙}
  \definition{s.}{selo (em envelopes)}
\end{EntryWithPhonetic}

\begin{EntryWithPhonetic}{封斋}{feng1zhai1}{9,10}{⼨、⽂}
  \definition*{s.}{Ramadã (Islã)}
\end{EntryWithPhonetic}

%%%%%%%%%% 疯 %%%%%%%%%%
\subsection*{疯}

\begin{EntryWithPhonetic}{疯}{feng1}{9}{⽧}[HSK 5]
  \definition{adj.}{louco; insano | tolo; leviano | (de uma planta, safra de grãos, etc.) esguia; refere-se ao crescimento vigoroso das plantações, mas sem frutos | com todas as forças; fazer o máximo possível}
  \definition{v.}{jogar sem restrições}
\end{EntryWithPhonetic}

\begin{EntryWithPhonetic}{疯狂}{feng1kuang2}{9,7}{⽧、⽝}[HSK 5]
  \definition{adj.}{louco; insano; frenético; desenfreado}
\end{EntryWithPhonetic}

\begin{EntryWithPhonetic}{疯子}{feng1zi5}{9,3}{⽧、⼦}[HSK 7-9]
  \definition[个,些,种]{s.}{maníaco; lunático; louco; pessoas com doenças mentais graves}[别把我当成疯子!===Não me trate como um louco!]
\end{EntryWithPhonetic}

%%%%%%%%%% 峰 %%%%%%%%%%
\subsection*{峰}

\begin{EntryWithPhonetic}{峰}{feng1}{10}{⼭}
  \definition{clas.}{usado para camelos}
  \definition{s.}{pico; cume; o pico proeminente de uma montanha | coisa parecida com um pico; coisas em forma de montanhas}
\end{EntryWithPhonetic}

\begin{EntryWithPhonetic}{峰回路转}{feng1hui2-lu4zhuan3}{10,6,13,8}{⼭、⼞、⾜、⾞}[HSK 7-9]
  \definition{expr.}{cume em meio a elevações circundantes e estradas sinuosas;  (estrada de montanha) torcendo e virando; a estrada da montanha serpenteia em torno de cada novo pico | boa (ou nova) reviravolta nos acontecimentos; uma oportunidade surgiu inesperadamente; as coisas tomaram um novo rumo}
\end{EntryWithPhonetic}

\begin{EntryWithPhonetic}{峰会}{feng1 hui4}{10,6}{⼭、⼈}[HSK 6]
  \definition{s.}{cúpula; reunião de cúpula}
\end{EntryWithPhonetic}

%%%%%%%%%% 蜂 %%%%%%%%%%
\subsection*{蜂}

\begin{EntryWithPhonetic}{蜂}{feng1}{13}{⾍}
  \definition{adv.}{em enxames}
  \definition[只,群,窝]{s.}{vespa | abelha}
\end{EntryWithPhonetic}

\begin{EntryWithPhonetic}{蜂蜜}{feng1mi4}{13,14}{⾍、⾍}[HSK 7-9]
  \definition[杯,瓶,罐,斤,碗,勺]{s.}{mel (de abelhas)}
\end{EntryWithPhonetic}

%%%%%%%%%% 逢 %%%%%%%%%%
\subsection*{逢}

\begin{EntryWithPhonetic}{逢}{feng2}{10}{⾡}[HSK 7-9]
  \definition*{s.}{Sobrenome: Feng}
  \definition{v.}{encontrar; vir até; encontrar-se por acaso}
\end{EntryWithPhonetic}

%%%%%%%%%% 缝 %%%%%%%%%%
\subsection*{缝}

\begin{EntryWithPhonetic}{缝}{feng2}{13}{⽷}[HSK 7-9]
  \definition{v.}{costurar; dar um ponto}
  \seeref{feng4}
\end{EntryWithPhonetic}

\begin{EntryWithPhonetic}{缝合}{feng2he2}{13,6}{⽷、⼝}[HSK 7-9]
  \definition{v.}{suturar; costurar uma ferida com agulhas e linhas especiais}
\end{EntryWithPhonetic}

\begin{EntryWithPhonetic}{缝纫}{feng2ren4}{13,6}{⽷、⽷}
  \definition{v.}{costurar}
\end{EntryWithPhonetic}

\begin{EntryWithPhonetic}{缝纫机}{feng2ren4ji1}{13,6,6}{⽷、⽷、⽊}
  \definition[架]{s.}{máquina de costura}
\end{EntryWithPhonetic}

%%%%%%%%%% 讽 %%%%%%%%%%
\subsection*{讽}

\begin{EntryWithPhonetic}{讽}{feng3}{6}{⾔}
  \definition{v.}{satirizar; zombar | Literário: cantar; entoar}
\end{EntryWithPhonetic}

\begin{EntryWithPhonetic}{讽刺}{feng3ci4}{6,8}{⾔、⼑}[HSK 7-9]
  \definition{adj.}{irônico; satírico; sarcástico}
  \definition{v.}{satirizar; ridicularizar; usar metáforas, exageros, ironia e outras expressões para expor, criticar ou ridicularizar}
\end{EntryWithPhonetic}

%%%%%%%%%% 凤 %%%%%%%%%%
\subsection*{凤}

\begin{EntryWithPhonetic}{凤}{feng4}{4}{⼏}
  \definition*{s.}{Sobrenome: Feng}
  \definition[只]{s.}{Mitologia: fênix}
\end{EntryWithPhonetic}

\begin{EntryWithPhonetic}{凤凰}{feng4huang2}{4,11}{⼏、⼏}[HSK 7-9]
  \definition[只,个,对]{s.}{fênix; o rei dos pássaros nas lendas antigas, com belas penas, o macho é chamado de 凤 e a fêmea é chamada de 凰, frequentemente usado para simbolizar auspiciosidade}
  \seealsoref{凤}{feng4}
  \seealsoref{凰}{huang2}
\end{EntryWithPhonetic}

%%%%%%%%%% 奉 %%%%%%%%%%
\subsection*{奉}

\begin{EntryWithPhonetic}{奉}{feng4}{8}{⼤}
  \definition*{s.}{Sobrenome: Feng}
  \definition{v.}{Literário: dedicar ou presentear com respeito | receber (pedidos, instruções, etc.) | Literário: estimar; reverenciar | Litrário: acreditar em  | esperar; atender; servir}
\end{EntryWithPhonetic}

\begin{EntryWithPhonetic}{奉献}{feng4xian4}{8,13}{⼤、⽝}[HSK 6]
  \definition{v.}{dedicar; oferecer como tributo; apresentar com todo respeito; entregar respeitosamente}
\end{EntryWithPhonetic}

%%%%%%%%%% 缝 %%%%%%%%%%
\subsection*{缝}

\begin{EntryWithPhonetic}{缝}{feng4}{13}{⽷}[HSK 7-9]
  \definition[道]{s.}{costura | fenda; rachadura; fissura; brecha; fresta}
  \seeref{feng2}
\end{EntryWithPhonetic}

%%%%%%%%%% 佛 %%%%%%%%%%
\subsection*{佛}

\begin{EntryWithPhonetic}{佛}{fo2}{7}{⼈}[HSK 6]
  \definition*{s.}{Buda, abreviação de 佛陀 | Budismo}
  \definition{s.}{imagem de Buda | budista | nome de Buda; escritura budista | uma pessoa que alcançou a perfeição na prática espiritual; budista real | estátua do Buda}
  \seeref{fu2}
  \seealsoref{佛陀}{fo2tuo2}
\end{EntryWithPhonetic}

\begin{EntryWithPhonetic}{佛教}{fo2 jiao4}{7,11}{⼈、⽁}[HSK 6]
  \definition*{s.}{Budismo; uma das principais religiões do mundo, diz-se que foi fundada por Sakyamuni, um príncipe do antigo reino indiano de Kapilavastu (no atual Nepal), no século VI ou V a.C.; foi amplamente difundida em muitos países asiáticos e introduzida na China no final da Dinastia Han Ocidental}
\end{EntryWithPhonetic}

\begin{EntryWithPhonetic}{佛陀}{fo2tuo2}{7,7}{⼈、⾩}
  \definition{s.}{Buda, um título para Sakyamuni ou uma pessoa que atingiu a iluminação | Buda, uma pessoa que atingiu a Budeidade, ou especificamente Siddhartha Gautama}
\end{EntryWithPhonetic}

%%%%%%%%%% 否 %%%%%%%%%%
\subsection*{否}

\begin{EntryWithPhonetic}{否}{fou3}{7}{⼝}
  \definition{adv.}{não; expressa discordância, equivalente à palavra falada 不 | usado no final de uma pergunta para indicar investigação | 是否, 能否 e 可否 que significa respectivamente 是不是, 能不能 e 可不可}
  \definition{v.}{negar}
  \seeref{pi3}
  \seealsoref{不}{bu4}
  \seealsoref{可}{ke3}
  \seealsoref{能}{neng2}
  \seealsoref{是}{shi4}
\end{EntryWithPhonetic}

\begin{EntryWithPhonetic}{否定}{fou3ding4}{7,8}{⼝、⼧}[HSK 3]
  \definition{adj.}{negativo; contrário}
  \definition{v.}{rejeitar; negar a existência ou a autenticidade de algo}
\end{EntryWithPhonetic}

\begin{EntryWithPhonetic}{否决}{fou3jue2}{7,6}{⼝、⼎}[HSK 7-9]
  \definition{v.}{rejeitar; votar contra; vetar; anular}
\end{EntryWithPhonetic}

\begin{EntryWithPhonetic}{否认}{fou3ren4}{7,4}{⼝、⾔}[HSK 3]
  \definition{v.}{negar; repudiar; não reconhecer}
\end{EntryWithPhonetic}

\begin{EntryWithPhonetic}{否则}{fou3ze2}{7,6}{⼝、⼑}[HSK 4]
  \definition{conj.}{senão; se não; ou então; se não for isso}
\end{EntryWithPhonetic}

%%%%%%%%%% 夫 %%%%%%%%%%
\subsection*{夫}

\begin{EntryWithPhonetic}{夫}{fu1}{4}{⼤}
  \definition{s.}{marido | homem | (velho) alguém que faz algum tipo de trabalho manual | (velho) uma pessoa que serviu em trabalho forçado}
  \seeref{fu2}
\end{EntryWithPhonetic}

\begin{EntryWithPhonetic}{夫妇}{fu1fu4}{4,6}{⼤、⼥}[HSK 4]
  \definition[对]{s.}{casal; marido e mulher}
\end{EntryWithPhonetic}

\begin{EntryWithPhonetic}{夫妻}{fu1qi1}{4,8}{⼤、⼥}[HSK 4]
  \definition[对]{s.}{casal; marido e mulher}
\end{EntryWithPhonetic}

\begin{EntryWithPhonetic}{夫人}{fu1ren2}{4,2}{⼤、⼈}[HSK 4]
  \definition[位,名,个]{s.}{senhora; \emph{lady}; madame; na antiguidade, as esposas dos senhores feudais eram chamadas de ``madame'' e, nas dinastias Ming e Qing, as esposas dos oficiais de primeiro e segundo escalão eram chamadas de ``madame'', que mais tarde foi usada para homenagear as esposas das pessoas em geral e agora é usada principalmente em ocasiões diplomáticas}
\end{EntryWithPhonetic}

%%%%%%%%%% 孵 %%%%%%%%%%
\subsection*{孵}

\begin{EntryWithPhonetic}{孵}{fu1}{14}{⼦}
  \definition{v.}{chocar; incubar; (pássaros) sentar em ovos}
\end{EntryWithPhonetic}

\begin{EntryWithPhonetic}{孵化}{fu1hua4}{14,4}{⼦、⼔}[HSK 7-9]
  \definition{v.}{chocar; incubar | incubar; metaforicamente, cultivar e desenvolver coisas novas (agora se refere principalmente ao suporte a empresas de alta tecnologia recém-criadas)}
\end{EntryWithPhonetic}

%%%%%%%%%% 敷 %%%%%%%%%%
\subsection*{敷}

\begin{EntryWithPhonetic}{敷}{fu1}{15}{⽁}[HSK 7-9]
  \definition*{s.}{Sobrenome: Fu}
  \definition{v.}{aplicar (pó, pomada, etc.) | espalhar; dispor | ser suficiente para | espalhar-se}
\end{EntryWithPhonetic}

%%%%%%%%%% 夫 %%%%%%%%%%
\subsection*{夫}

\begin{EntryWithPhonetic}{夫}{fu2}{4}{⼤}
  \definition{part.}{usado no início de uma frase | usado no final de uma frase ou em uma pausa no meio de uma frase para expressar uma exclamação}
  \definition{pron.}{isto; isso; aqueles; estes | ele}
  \seeref{fu1}
\end{EntryWithPhonetic}

%%%%%%%%%% 佛 %%%%%%%%%%
\subsection*{佛}

\begin{EntryWithPhonetic}{佛}{fu2}{7}{⼈}
  \definition{adv.}{aparentemente}
  \definition{s.}{ornamento da cabeça (feminino)}
  \seeref{fo2}
\end{EntryWithPhonetic}

%%%%%%%%%% 扶 %%%%%%%%%%
\subsection*{扶}

\begin{EntryWithPhonetic}{扶}{fu2}{7}{⼿}[HSK 5]
  \definition*{s.}{Sobrenome: Fu}
  \definition{v.}{segurar; apoiar com a mão; segurar algo com o apoio das mãos para que ninguém, objeto ou pessoa caia | dar apoio a; ajudar uma pessoa deitada ou caída a se levantar com as mãos; endireitar um objeto caído com as mãos | ajudar; tirar de baixo}
\end{EntryWithPhonetic}

\begin{EntryWithPhonetic}{扶持}{fu2chi2}{7,9}{⼿、⼿}[HSK 7-9]
  \definition{v.}{apoiar com a mão; colocar uma mão em alguém para apoio; apoiar | apoiar; dar ajuda a; ajudar a sustentar}
\end{EntryWithPhonetic}

\begin{EntryWithPhonetic}{扶梯}{fu2ti1}{7,11}{⼿、⽊}
  \definition{s.}{escada rolante}
\end{EntryWithPhonetic}

%%%%%%%%%% 服 %%%%%%%%%%
\subsection*{服}

\begin{EntryWithPhonetic}{服}{fu2}{8}{⽉}[HSK 6]
  \definition*{s.}{Sobrenome: Fu}
  \definition{s.}{roupas | vestuário de luto; refere-se a roupas de luto}
  \definition{v.}{vestir (roupas) | tomar (remédio) | envolver-se em; servir | obedecer; ser convencido | convencer; persuadir | adaptar-se; acostumar-se a}
  \seeref{fu4}
\end{EntryWithPhonetic}

\begin{EntryWithPhonetic}{服从}{fu2cong2}{8,4}{⽉、⼈}[HSK 5]
  \definition{v.}{obedecer; submeter-se a; estar subordinado a}
\end{EntryWithPhonetic}

\begin{EntryWithPhonetic}{服饰}{fu2shi4}{8,8}{⽉、⾷}[HSK 7-9]
  \definition[套]{s.}{roupa; vestido; traje; vestimenta e adorno pessoal}
\end{EntryWithPhonetic}

\begin{EntryWithPhonetic}{服务}{fu2 wu4}{8,5}{⽉、⼒}[HSK 2]
  \definition{v.}{prestar serviço a; estar a serviço de; servir; trabalhar para o benefício coletivo (ou de outras pessoas) ou para uma causa específica | trabalhar; servir}
\end{EntryWithPhonetic}

\begin{EntryWithPhonetic}{服务器}{fu2wu4qi4}{8,5,16}{⽉、⼒、⼝}[HSK 7-9]
  \definition[个,台]{s.}{Computção: servidor; um dispositivo dedicado que fornece serviços aos usuários em uma rede eletrônica de computadores}
\end{EntryWithPhonetic}

\begin{EntryWithPhonetic}{服务员}{fu2wu4yuan2}{8,5,7}{⽉、⼒、⼝}
  \definition{s.}{atendente | garçom | garçonete | pessoal de atendimento ao cliente}
\end{EntryWithPhonetic}

\begin{EntryWithPhonetic}{服用}{fu2yong4}{8,5}{⽉、⽤}[HSK 7-9]
  \definition{v.}{tomar (remédio)}[他已开始服用这种药。===Ele começou a tomar o remédio.]
\end{EntryWithPhonetic}

\begin{EntryWithPhonetic}{服装}{fu2zhuang1}{8,12}{⽉、⾐}[HSK 3]
  \definition[套,件,身]{s.}{roupas; vestuário; trajes; termo genérico para roupas, sapatos e chapéus, geralmente referido especificamente a roupas}
\end{EntryWithPhonetic}

%%%%%%%%%% 俘 %%%%%%%%%%
\subsection*{俘}

\begin{EntryWithPhonetic}{俘}{fu2}{9}{⼈}
  \definition{s.}{prisioneiro de guerra; cativo}
  \definition{v.}{capturar; fazer prisioneiro | fazer prisioneiro de guerra}
\end{EntryWithPhonetic}

\begin{EntryWithPhonetic}{俘获}{fu2huo4}{9,10}{⼈、⾋}[HSK 7-9]
  \definition{s.}{Física: captura; aprisionamento}[中子俘获===Captura de nêutrons]
  \definition{v.}{capturar; apreender}[军队成功俘获敌方指挥官。===O exército conseguiu capturar o comandante inimigo.]
\end{EntryWithPhonetic}

\begin{EntryWithPhonetic}{俘虏}{fu2lu3}{9,8}{⼈、⾌}[HSK 7-9]
  \definition[个,名,批,群]{s.}{cativo; prisioneiro de guerra; inimigos capturados durante a batalha}
  \definition{v.}{capturar (um inimigo) durante o combate; fazer prisioneiro; capturar}
\end{EntryWithPhonetic}

%%%%%%%%%% 浮 %%%%%%%%%%
\subsection*{浮}

\begin{EntryWithPhonetic}{浮}{fu2}{10}{⽔}[HSK 6]
  \definition*{s.}{Sobrenome: Fu}
  \definition{adj.}{superficial; na superfície | móvel; removível | temporário; provisório | superficial e frívolo; volátil; impetuoso | oco; vazio; inflado | excessivo; excedente}
  \definition{v.}{flutuar (oposto a 沉) | (dialeto) nadar | flutuar; derivar; flutuar na superfície do líquido}
  \seealsoref{沉}{chen2}
\end{EntryWithPhonetic}

\begin{EntryWithPhonetic}{浮力}{fu2li4}{10,2}{⽔、⼒}[HSK 7-9]
  \definition{s.}{flutuabilidade; a força de empuxo, a força ascendente exercida sobre um objeto em um fluido, é igual ao peso do fluido deslocado pelo objeto}
\end{EntryWithPhonetic}

\begin{EntryWithPhonetic}{浮图}{fu2tu2}{10,8}{⽔、⼞}
  \definition*{s.}{Termo alternativo para 佛陀}
  \variantof{浮屠}
  \seealsoref{佛陀}{fo2tuo2}
\end{EntryWithPhonetic}

\begin{EntryWithPhonetic}{浮屠}{fu2tu2}{10,11}{⽔、⼫}
  \definition*{s.}{Buda | Templo (Stupa) Budista (transliteração de Pali Thuo)}
\end{EntryWithPhonetic}

\begin{EntryWithPhonetic}{浮现}{fu2xian4}{10,8}{⽔、⾒}[HSK 7-9]
  \definition{v.}{(experiência passada) ressurgir; vir à mente | aparecer; apresentar; revelar}
\end{EntryWithPhonetic}

\begin{EntryWithPhonetic}{浮躁}{fu2zao4}{10,20}{⽔、⾜}[HSK 7-9]
  \definition{adj.}{inquieto; impetuoso; impaciente; frívolo e impaciente}
\end{EntryWithPhonetic}

%%%%%%%%%% 符 %%%%%%%%%%
\subsection*{符}

\begin{EntryWithPhonetic}{符}{fu2}{11}{⽵}
  \definition*{s.}{Sobrenome: Fu}
  \definition[个]{s.}{registro emitido por um governante para generais, enviados, etc., como credenciais na China antiga | símbolo; emblema | figuras mágicas desenhadas por sacerdotes taoístas para invocar ou expulsar espíritos e trazer boa ou má sorte | marca; sinal}
  \definition{v.}{(usado com 相 xiāng ou 不) coincidir com; concordar com | encaixar bem; combinar com; em conformidade com}
  \seealsoref{不}{bu4}
  \seealsoref{相}{xiang1}
\end{EntryWithPhonetic}

\begin{EntryWithPhonetic}{符号}{fu2hao4}{11,5}{⽵、⼝}[HSK 4]
  \definition[个]{s.}{marca; símbolo; sinais que marcam as coisas | insígnia; emblema; um símbolo usado no corpo para indicar posição, \emph{status}, etc.}
\end{EntryWithPhonetic}

\begin{EntryWithPhonetic}{符合}{fu2he2}{11,6}{⽵、⼝}[HSK 4]
  \definition{v.}{conformar-se com, estar de acordo com, estar em conformidade com}
\end{EntryWithPhonetic}

%%%%%%%%%% 幅 %%%%%%%%%%
\subsection*{幅}

\begin{EntryWithPhonetic}{幅}{fu2}{12}{⼱}[HSK 5]
  \definition{clas.}{usado para tecidos, telas de lã, pinturas, etc.}
  \definition{s.}{largura do tecido, seda, tweed, etc. | tamanho; largura; geralmente se refere à largura}
\end{EntryWithPhonetic}

\begin{EntryWithPhonetic}{幅度}{fu2du4}{12,9}{⼱、⼴}[HSK 5]
  \definition{s.}{alcance; escopo; extensão; largura; largura da propagação de um objeto que vibra ou balança, uma metáfora para a magnitude de uma mudança em algo}
\end{EntryWithPhonetic}

%%%%%%%%%% 福 %%%%%%%%%%
\subsection*{福}

\begin{EntryWithPhonetic}{福}{fu2}{13}{⽰}[HSK 3]
  \definition*{s.}{Província de Fujian | Sobrenome: Fu}
  \definition{s.}{benção; felicidade; boa sorte; boa fortuna; sorte (oposto de 祸)}
  \definition{v.}{(de uma mulher) fazer uma reverência; antigamente, as mulheres faziam a reverência 万福 (colocando as duas mãos na cintura do mesmo lado e dobrando ligeiramente os joelhos)}
  \seealsoref{祸}{huo4}
  \seealsoref{万福}{wan4fu2}
\end{EntryWithPhonetic}

\begin{EntryWithPhonetic}{福克斯}{fu2ke4si1}{13,7,12}{⽰、⼗、⽄}
  \definition*{s.}{Fox (empresa de mídia) | Focus (automóvel fabricado pela Ford)}
\end{EntryWithPhonetic}

\begin{EntryWithPhonetic}{福利}{fu2li4}{13,7}{⽰、⼑}[HSK 5]
  \definition[项,种]{s.}{bem-estar; benefícios materiais}
  \definition{v.}{melhorar suas condições de vida; facilitar a vida}
\end{EntryWithPhonetic}

\begin{EntryWithPhonetic}{福气}{fu2qi5}{13,4}{⽰、⽓}[HSK 7-9]
  \definition{s.}{bênção; boa sorte; refere-se ao destino de desfrutar de uma vida feliz}
\end{EntryWithPhonetic}

\begin{EntryWithPhonetic}{福泽}{fu2ze2}{13,8}{⽰、⽔}
  \definition{s.}{boa sorte}
\end{EntryWithPhonetic}

%%%%%%%%%% 辐 %%%%%%%%%%
\subsection*{辐}

\begin{EntryWithPhonetic}{辐}{fu2}{13}{⾞}
  \definition{s.}{raio (de uma roda); a conexão entre o cubo e o aro de uma roda}
\end{EntryWithPhonetic}

\begin{EntryWithPhonetic}{辐射}{fu2she4}{13,10}{⾞、⼨}[HSK 7-9]
  \definition{s.}{radiação;  uma forma de propagação de calor que se irradia da fonte de calor em linha reta para a área circundante; a propagação de ondas eletromagnéticas, como luz e ondas de rádio}
  \definition{v.}{irradiar}
\end{EntryWithPhonetic}

%%%%%%%%%% 抚 %%%%%%%%%%
\subsection*{抚}

\begin{EntryWithPhonetic}{抚}{fu3}{7}{⼿}
  \definition{v.}{confortar; consolar | nutrir; fomentar | Literário: acariciar | proteger; promover; criar | o mesmo que 拊}
  \seealsoref{拊}{fu3}
\end{EntryWithPhonetic}

\begin{EntryWithPhonetic}{抚摸}{fu3mo1}{7,13}{⼿、⼿}[HSK 7-9]
  \definition{v.}{acariciar; afagar; amimar}
\end{EntryWithPhonetic}

\begin{EntryWithPhonetic}{抚恤}{fu3xu4}{7,9}{⼿、⼼}[HSK 7-9]
  \definition{v.}{(estado ou organização) fornecer conforto e assistência material às famílias de pessoal ferido ou incapacitado no cumprimento do dever, ou daqueles que morreram de doença ou morreram no cumprimento do dever}
\end{EntryWithPhonetic}

\begin{EntryWithPhonetic}{抚养}{fu3yang3}{7,9}{⼿、⼋}[HSK 7-9]
  \definition{v.}{criar; cuidar; proporcionar às crianças as condições de vida necessárias para que possam crescer com saúde}
\end{EntryWithPhonetic}

\begin{EntryWithPhonetic}{抚养费}{fu3yang3fei4}{7,9,9}{⼿、⼋、⾙}[HSK 7-9]
  \definition{s.}{pensão alimentícia (após o divórcio) | pagamento pela educação dos filhos (como após o divórcio)}
\end{EntryWithPhonetic}

%%%%%%%%%% 拊 %%%%%%%%%%
\subsection*{拊}

\begin{EntryWithPhonetic}{拊}{fu3}{8}{⼿}
  \definition{v.}{Literário: bater palmas; esbofetear; golpear}
\end{EntryWithPhonetic}

%%%%%%%%%% 斧 %%%%%%%%%%
\subsection*{斧}

\begin{EntryWithPhonetic}{斧}{fu3}{8}{⽄}
  \definition[把,只]{s.}{machado; machadinha | machado de batalha (um tipo de arma usada na China antiga)}
\end{EntryWithPhonetic}

\begin{EntryWithPhonetic}{斧子}{fu3zi5}{8,3}{⽄、⼦}[HSK 7-9]
  \definition[把,个]{s.}{machado; machadinha}
\end{EntryWithPhonetic}

%%%%%%%%%% 俯 %%%%%%%%%%
\subsection*{俯}

\begin{EntryWithPhonetic}{俯}{fu3}{10}{⼈}
  \definition{v.}{curvar (a cabeça), oposto a 仰 | inclinar-se | Obsoleto: (em documentos ou cartas oficiais) condescender com | curvar-se; fazer uma reverência}
  \seealsoref{仰}{yang3}
\end{EntryWithPhonetic}

\begin{EntryWithPhonetic}{俯首}{fu3shou3}{10,9}{⼈、⾸}[HSK 7-9]
  \definition{v.}{abaixar a cabeça; curvar-se; inclinar-se}
\end{EntryWithPhonetic}

%%%%%%%%%% 辅 %%%%%%%%%%
\subsection*{辅}

\begin{EntryWithPhonetic}{辅}{fu3}{11}{⾞}
  \definition*{s.}{Sobrenome: Fu}
  \definition{adj.}{subsidiário}
  \definition{s.}{barras laterais do carrinho atuando como proteção da roda; duas barras retas de madeira são adicionadas na parte externa da roda para prender o cubo | maçã do rosto | assistente oficial; títulos oficiais antigos | (literário) território que circunda a capital}
  \definition{v.}{auxiliar; complementar; suplementar | ajudar}
\end{EntryWithPhonetic}

\begin{EntryWithPhonetic}{辅导}{fu3dao3}{11,6}{⾞、⼨}[HSK 7-9]
  \definition{v.}{orientar no estudo ou treinamento; treinar; guiar; dar aulas particulares}
\end{EntryWithPhonetic}

\begin{EntryWithPhonetic}{辅助}{fu3zhu4}{11,7}{⾞、⼒}[HSK 5]
  \definition{adj.}{auxiliar; suplementar; complementar}
  \definition{v.}{auxiliar; ajudar; colocar os outros em primeiro lugar e dar-lhes alguma ajuda externa}
\end{EntryWithPhonetic}

%%%%%%%%%% 腐 %%%%%%%%%%
\subsection*{腐}

\begin{EntryWithPhonetic}{腐}{fu3}{14}{⾁}
  \definition{adj.}{podre; obsoleto; corrupto | corroído; pútrido}
  \definition{s.}{tofu}
  \definition{v.}{apodrecer; corroer; estragar; decair}
\end{EntryWithPhonetic}

\begin{EntryWithPhonetic}{腐败}{fu3bai4}{14,8}{⾁、⾒}[HSK 7-9]
  \definition{adj.}{(ideias) corrupto; decadente; (pensamento) obsoleto; (comportamento) degenerado | (sistema, organização, instituição, medida, etc.) corrupto}
  \definition{s.}{deterioração; podridão}
  \definition{v.}{apodrecer; decair}
\end{EntryWithPhonetic}

\begin{EntryWithPhonetic}{腐化}{fu3hua4}{14,4}{⾁、⼔}[HSK 7-9]
  \definition{adj.}{degenerado; corrupto, dissoluto ou depravado; desmoralizado; decadente}
  \definition{v.}{decompor; apodrecer; tornar-se pútrido | quebrar; corroer}
\end{EntryWithPhonetic}

\begin{EntryWithPhonetic}{腐烂}{fu3lan4}{14,9}{⾁、⽕}[HSK 7-9]
  \definition{adj.}{corrupto; depravado | (pensamentos) obsoletos; (comportamento) degenerado}
  \definition{v.}{apodrecer; decompor; tornar-se pútrido}
\end{EntryWithPhonetic}

\begin{EntryWithPhonetic}{腐蚀}{fu3shi2}{14,9}{⾁、⾷}[HSK 7-9]
  \definition{v.}{corroer; destruir gradualmente um objeto por meio de reações químicas | corroer; corromper (pensamentos e comportamentos)}
\end{EntryWithPhonetic}

\begin{EntryWithPhonetic}{腐朽}{fu3xiu3}{14,6}{⾁、⽊}[HSK 7-9]
  \definition{adj.}{decaído; decadente; degenerado; uma metáfora para as ideias ultrapassadas das pessoas ou para a moral social corrupta}
  \definition{v.}{apodrecer; decair; apodrecimento e deterioração da madeira e outros materiais fibrosos}
\end{EntryWithPhonetic}

%%%%%%%%%% 父 %%%%%%%%%%
\subsection*{父}

\begin{EntryWithPhonetic}{父}{fu4}{4}{⽗}[Kangxi 88]
  \definition{s.}{pai | um homem mais velho na família ou parentes | fundador; uma pessoa que inventa ou inicia algo}
\end{EntryWithPhonetic}

\begin{EntryWithPhonetic}{父母}{fu4 mu3}{4,5}{⽗、⽏}[HSK 3]
  \definition{s.}{pai e mãe; pais}
\end{EntryWithPhonetic}

\begin{EntryWithPhonetic}{父母亲}{fu4mu3qin1}{4,5,9}{⽗、⽏、⼇}
  \definition{s.}{os pais; pai e mãe}
\end{EntryWithPhonetic}

\begin{EntryWithPhonetic}{父女}{fu4 nv3}{4,3}{⽗、⼥}[HSK 6]
  \definition{s.}{pai e filha}
\end{EntryWithPhonetic}

\begin{EntryWithPhonetic}{父亲}{fu4qin1}{4,9}{⽗、⼇}[HSK 3]
  \definition[个,位,名]{s.}{pai; homem com filhos; pai dos filhos}
\end{EntryWithPhonetic}

\begin{EntryWithPhonetic}{父子}{fu4 zi3}{4,3}{⽗、⼦}[HSK 6]
  \definition{s.}{pai e filho}
\end{EntryWithPhonetic}

%%%%%%%%%% 讣 %%%%%%%%%%
\subsection*{讣}

\begin{EntryWithPhonetic}{讣}{fu4}{4}{⾔}
  \definition[个,则]{s.}{obituário}
  \definition{v.}{anunciar a morte de alguém | relatar um luto}
\end{EntryWithPhonetic}

%%%%%%%%%% 付 %%%%%%%%%%
\subsection*{付}

\begin{EntryWithPhonetic}{付}{fu4}{5}{⼈}[HSK 3]
  \definition*{s.}{Sobrenome: Fu}
  \definition{clas.}{usado para pares ou conjuntos de coisas | usado para expressões faciais}
  \definition{v.}{comprometer-se com; entregar (ou transferir) para | pagar; refere-se especificamente a dar dinheiro}
\end{EntryWithPhonetic}

\begin{EntryWithPhonetic}{付出}{fu4 chu1}{5,5}{⼈、⼐}[HSK 4]
  \definition{v.}{pagar; gastar; entregar (dinheiro, consideração, etc.)}
\end{EntryWithPhonetic}

\begin{EntryWithPhonetic}{付费}{fu4fei4}{5,9}{⼈、⾙}[HSK 7-9]
  \definition{v.}{pagar}
\end{EntryWithPhonetic}

\begin{EntryWithPhonetic}{付款}{fu4kuan3}{5,12}{⼈、⽋}[HSK 7-9]
  \definition[次]{s.}{pagamento}
  \definition{v.}{pagar uma quantia em dinheiro}
\end{EntryWithPhonetic}

%%%%%%%%%% 妇 %%%%%%%%%%
\subsection*{妇}

\begin{EntryWithPhonetic}{妇}{fu4}{6}{⼥}
  \definition{s.}{mulher | mulher casada | esposa}
\end{EntryWithPhonetic}

\begin{EntryWithPhonetic}{妇女}{fu4nv3}{6,3}{⼥、⼥}[HSK 6]
  \definition[个,位,群,名,帮]{s.}{mulher; mulheres; um termo geral para mulheres adultas}
\end{EntryWithPhonetic}

%%%%%%%%%% 负 %%%%%%%%%%
\subsection*{负}

\begin{EntryWithPhonetic}{负}{fu4}{6}{⾙}[HSK 6]
  \definition{adj.}{negativo; menor que zero | negativo; referindo-se ao que recebe elétrons (oposto a 正)}
  \definition{v.}{carregar; transportar nas costas ou nos ombros | suportar; assumir; encarar | confiar em; contar com; depender | sofrer | aproveitar; desfrutar | ter dívidas | trair; violar | perder; ser derrotado}
  \seealsoref{正}{zheng4}
\end{EntryWithPhonetic}

\begin{EntryWithPhonetic}{负担}{fu4dan1}{6,8}{⾙、⼿}[HSK 4]
  \definition{s.}{carga; fardo; frete; ônus; pressão ou responsabilidade, despesas, etc.}
  \definition{v.}{carregar; carregar (um fardo); assumir (responsabilidade, trabalho, despesas, etc.)}
\end{EntryWithPhonetic}

\begin{EntryWithPhonetic}{负面}{fu4mian4}{6,9}{⾙、⾯}[HSK 7-9]
  \definition{adj.}{ruim; negativo; prejudicial; desvantajoso}
  \definition{s.}{lado reverso; o negativo; refere-se aos aspectos ou partes ruins, negativas, prejudiciais ou desfavoráveis}
\end{EntryWithPhonetic}

\begin{EntryWithPhonetic}{负有}{fu4you3}{6,6}{⾙、⽉}[HSK 7-9]
  \definition{v.}{ser responsável por}
\end{EntryWithPhonetic}

\begin{EntryWithPhonetic}{负责}{fu4ze2}{6,8}{⾙、⾙}[HSK 3]
  \definition{adj.}{consciencioso; ser sério e responsável}
  \definition{v.}{ser responsável por; estar encarregado de; assumir responsabilidades}
\end{EntryWithPhonetic}

\begin{EntryWithPhonetic}{负责人}{fu4 ze2 ren2}{6,8,2}{⾙、⾙、⼈}[HSK 5]
  \definition[位]{s.}{pessoa responsável; pessoa encarregada; pessoas com responsabilidades de liderança}
\end{EntryWithPhonetic}

%%%%%%%%%% 附 %%%%%%%%%%
\subsection*{附}

\begin{EntryWithPhonetic}{附}{fu4}{7}{⾩}[HSK 7-9]
  \definition*{s.}{Sobrenome: Fu}
  \definition{v.}{adicionar; anexar; incluir | chegar perto de; estar perto de | depender de; confiar em; cumprir com | concordar com; anexar a; aderir a; cumprir com; depender de}
\end{EntryWithPhonetic}

\begin{EntryWithPhonetic}{附带}{fu4dai4}{7,9}{⾩、⼱}[HSK 7-9]
  \definition{adj.}{subsidiário; suplementar; incidental}
  \definition{adv.}{de passagem; a propósito; incidentalmente}
  \definition{v.}{anexar}
\end{EntryWithPhonetic}

\begin{EntryWithPhonetic}{附和}{fu4he4}{7,8}{⾩、⼝}[HSK 7-9]
  \definition{v.}{ecoar; entrar na conversa com; repetir o que os outros dizem (alguém disse)}
\end{EntryWithPhonetic}

\begin{EntryWithPhonetic}{附加}{fu4jia1}{7,5}{⾩、⼒}[HSK 7-9]
  \definition{adj.}{adicional; anexado; extra}
  \definition{v.}{adicionar; anexar; exceder a quantidade ou intervalo prescrito}
\end{EntryWithPhonetic}

\begin{EntryWithPhonetic}{附件}{fu4jian4}{7,6}{⾩、⼈}[HSK 5]
  \definition*{s.}{\emph{Adnexa Uteri}, refere-se à genitália interna feminina que não seja o útero, as trompas de falópio e os ovários}
  \definition{s.}{apêndice; documentos que acompanham o documento principal | acessório; anexo; peças ou sobressalentes que não sejam peças principais de máquinas e equipamentos | anexo; documentos ou itens relevantes emitidos com o documento}
\end{EntryWithPhonetic}

\begin{EntryWithPhonetic}{附近}{fu4jin4}{7,7}{⾩、⾡}[HSK 4]
  \definition{adj.}{perto; vizinho}
  \definition{s.}{vizinhança; bairro}
\end{EntryWithPhonetic}

\begin{EntryWithPhonetic}{附属}{fu4shu3}{7,12}{⾩、⼫}[HSK 7-9]
  \definition{adj.}{anexado; afiliado; dependente ou pertencente a uma instituição}
  \definition{v.}{afiliar-se; filiar-se; inscrever-se; agregar-se}
\end{EntryWithPhonetic}

%%%%%%%%%% 服 %%%%%%%%%%
\subsection*{服}

\begin{EntryWithPhonetic}{服}{fu4}{8}{⽉}
  \definition{clas.}{usado para remédio: dose; usado na medicina tradicional chinesa}
  \seeref{fu2}
\end{EntryWithPhonetic}

%%%%%%%%%% 复 %%%%%%%%%%
\subsection*{复}

\begin{EntryWithPhonetic}{复}{fu4}{9}{⼢}
  \definition*{s.}{Sobrenome: Fu}
  \definition{adj.}{composto; complexo; nem um único; dois ou mais}
  \definition{adv.}{de novo; novamente; indica o reaparecimento de uma situação, equivalente a 再}
  \definition{s.}{jaqueta; roupas forradas}
  \definition{v.}{virar; virar-se | responder; retornar | recuperar; retornar a; restaurar | vingar | duplicar; repetir}
  \seealsoref{再}{zai4}
\end{EntryWithPhonetic}

\begin{EntryWithPhonetic}{复查}{fu4cha2}{9,9}{⼢、⽊}[HSK 7-9]
  \definition{v.}{verificar novamente; reexaminar; revisar}
\end{EntryWithPhonetic}

\begin{EntryWithPhonetic}{复发}{fu4fa1}{9,5}{⼢、⼜}[HSK 7-9]
  \definition{v.}{ter uma recaída; recorrer | reaparecer; recrudescer | recorrer (de uma doença) | recair (em um antigo estado ruim)}
\end{EntryWithPhonetic}

\begin{EntryWithPhonetic}{复合}{fu4he2}{9,6}{⼢、⼝}[HSK 7-9]
  \definition{v.}{compor; tornar complexo; combinar; juntar | recombinar; juntar; metáfora para estarem juntos novamente depois de estarem separados}
\end{EntryWithPhonetic}

\begin{EntryWithPhonetic}{复活}{fu4huo2}{9,9}{⼢、⽔}[HSK 7-9]
  \definition{s.}{ressurreição (cristianismo)}
  \definition{v.}{reviver; voltar à vida; morrer e voltar à vida, frequentemente usado como metáfora}
\end{EntryWithPhonetic}

\begin{EntryWithPhonetic}{复活节}{fu4huo2jie2}{9,9,5}{⼢、⽔、⾋}
  \definition*{s.}{Páscoa; festival cristão que comemora a ressurreição de Jesus ocorre no primeiro domingo após a primeira lua cheia após o equinócio da primavera}
\end{EntryWithPhonetic}

\begin{EntryWithPhonetic}{复刻}{fu4ke4}{9,8}{⼢、⼑}
  \definition{v.}{reimprimir (um trabalho que esteve fora do catálogo) | reeditar (um disco de vinil, um CD, etc.) | replicar | recriar | (empréstimo linguístico) (computação) \emph{fork}}
\end{EntryWithPhonetic}

\begin{EntryWithPhonetic}{复苏}{fu4 su1}{9,7}{⼢、⾋}[HSK 6]
  \definition{s.}{recuperação}
  \definition{v.}{reviver; recuperar; ressuscitar; voltar à vida}
\end{EntryWithPhonetic}

\begin{EntryWithPhonetic}{复习}{fu4xi2}{9,3}{⼢、⼄}[HSK 2]
  \definition{s.}{revisão}
  \definition{v.}{revisar; corrigir (lições, etc.); repetir o que já aprendeu para consolidar o conhecimento}
\end{EntryWithPhonetic}

\begin{EntryWithPhonetic}{复兴}{fu4xing1}{9,6}{⼢、⼋}[HSK 7-9]
  \definition{v.}{reviver; rejuvenescer | reviver; desenvolver-se e tornar-se mais forte}
\end{EntryWithPhonetic}

\begin{EntryWithPhonetic}{复印}{fu4yin4}{9,5}{⼢、⼙}[HSK 3]
  \definition{v.}{fotografar; fotocopiar; duplicar; sem passar pelo processo de impressão, obter uma cópia diretamente do original (geralmente referindo-se à cópia feita com uma copiadora)}
\end{EntryWithPhonetic}

\begin{EntryWithPhonetic}{复元}{fu4/yuan2}{9,4}{⼢、⼉}
  \variantof{复原}
\end{EntryWithPhonetic}

\begin{EntryWithPhonetic}{复原}{fu4/yuan2}{9,10}{⼢、⼚}[HSK 7-9]
  \definition{s.}{reconversão; recuperação; redefinição; reabilitação; restauração; recura; analepsia; analepse}
  \definition{v.+compl.}{recuperar-se de uma doença; ter a saúde restaurada | restaurar; reabilitar}
\end{EntryWithPhonetic}

\begin{EntryWithPhonetic}{复杂}{fu4za2}{9,6}{⼢、⽊}[HSK 3]
  \definition{adj.}{complexo; complicado; em oposição a 单纯 e 简单}
  \seealsoref{单纯}{dan1chun2}
  \seealsoref{简单}{jian3dan1}
\end{EntryWithPhonetic}

\begin{EntryWithPhonetic}{复制}{fu4zhi4}{9,8}{⼢、⼑}[HSK 4]
  \definition{v.}{copiar; duplicar; reproduzir; fazer uma cópia de; fazer uma cópia do original ou reproduzi-lo, reimprimi-lo ou copiá-lo em sua forma original (geralmente referindo-se a relíquias culturais ou obras de arte)}
\end{EntryWithPhonetic}

%%%%%%%%%% 赴 %%%%%%%%%%
\subsection*{赴}

\begin{EntryWithPhonetic}{赴}{fu4}{9}{⾛}[HSK 7-9]
  \definition{v.}{ir para | comparecer}
\end{EntryWithPhonetic}

%%%%%%%%%% 副 %%%%%%%%%%
\subsection*{副}

\begin{EntryWithPhonetic}{副}{fu4}{11}{⼑}[HSK 6]
  \definition{adj.}{segundo em exercício; deputado; auxiliar | subsidiário; incidental; secundário}
  \definition{clas.}{usado para conjuntos completos de itens; usado para \emph{kits} | usado para expressões faciais | usado para som ou voz}
  \definition{pref.}{vice-}
  \definition{s.}{assistente; ajudante; auxiliar; posição auxiliar; pessoa que ocupa uma posição auxiliar}
  \definition{v.}{ajustar; corresponder a; conformar-se a}
\end{EntryWithPhonetic}

\begin{EntryWithPhonetic}{副研}{fu4yan2}{11,9}{⼑、⽯}
  \definition{s.}{pesquisador adjunto}
\end{EntryWithPhonetic}

\begin{EntryWithPhonetic}{副作用}{fu4zuo4yong4}{11,7,5}{⼑、⼈、⽤}[HSK 7-9]
  \definition{s.}{efeito colateral; efeitos adversos além dos efeitos principais}
\end{EntryWithPhonetic}

%%%%%%%%%% 富 %%%%%%%%%%
\subsection*{富}

\begin{EntryWithPhonetic}{富}{fu4}{12}{⼧}[HSK 3]
  \definition*{s.}{Sobrenome: Fu}
  \definition{adj.}{rico; abastado; abundante; refere-se a ter muito dinheiro (oposto de 贫) | rico; abundante}
  \definition{v.}{tornar-se rico; enriquecer}
  \seealsoref{贫}{pin2}
\end{EntryWithPhonetic}

\begin{EntryWithPhonetic}{富含}{fu4han2}{12,7}{⼧、⼝}[HSK 7-9]
  \definition{v.}{ser rico em}[橘子富含维生素。===As laranjas são ricas em vitaminas.]
\end{EntryWithPhonetic}

\begin{EntryWithPhonetic}{富豪}{fu4hao2}{12,14}{⼧、⾗}[HSK 7-9]
  \definition{s.}{rico; pessoa com muito dinheiro e grande poder}
\end{EntryWithPhonetic}

\begin{EntryWithPhonetic}{富强}{fu4qiang2}{12,12}{⼧、⼸}[HSK 7-9]
  \definition{adj.}{próspero e forte; próspero e poderoso; rico e poderoso; (país) rico e poderoso}
\end{EntryWithPhonetic}

\begin{EntryWithPhonetic}{富人}{fu4 ren2}{12,2}{⼧、⼈}[HSK 6]
  \definition{s.}{os ricos; os abastados}
\end{EntryWithPhonetic}

\begin{EntryWithPhonetic}{富翁}{fu4weng1}{12,10}{⼧、⽺}[HSK 7-9]
  \definition[个,名,位]{s.}{homem rico; homem de riqueza; pessoas que possuem muitas propriedades}
\end{EntryWithPhonetic}

\begin{EntryWithPhonetic}{富有}{fu4 you3}{12,6}{⼧、⽉}[HSK 6]
  \definition{adj.}{rico; abastado; possuir uma grande quantidade de propriedades | rico em espírito; metáfora para uma vida espiritual rica}
  \definition{v.}{ser rico ou abundante em; principalmente referindo-se a coisas abstratas com significados positivos que são suficientes}
\end{EntryWithPhonetic}

\begin{EntryWithPhonetic}{富裕}{fu4yu4}{12,12}{⼧、⾐}[HSK 7-9]
  \definition{adj.}{rico; próspero; abastado; em boas condições econômicas e com dinheiro suficiente}
\end{EntryWithPhonetic}

\begin{EntryWithPhonetic}{富足}{fu4zu2}{12,7}{⼧、⾜}[HSK 7-9]
  \definition{adj.}{rico; pleno; abundante}
\end{EntryWithPhonetic}

%%%%%%%%%% 赋 %%%%%%%%%%
\subsection*{赋}

\begin{EntryWithPhonetic}{赋}{fu4}{12}{⾙}
  \definition{s.}{dotação (natural) | Obsoleto: imposto territorial | fu, estilo antigo, uma forma literária complexa que combina elementos de poesia e prosa, muito cultivada desde a época Han até o período das Seis Dinastias}
  \definition{v.}{compor (versos); escrever poemas e letras | Literário: conceder; entregar; dotar com}
\end{EntryWithPhonetic}

\begin{EntryWithPhonetic}{赋予}{fu4yu3}{12,4}{⾙、⼅}[HSK 7-9]
  \definition{v.}{conceder; dotar (uma tarefa importante, missão, etc.); dar a alguém uma tarefa, responsabilidade, direito, autoridade, etc. | investir com (significado, característica, etc.); dar a algo uma cor, significado, importância, etc.}
\end{EntryWithPhonetic}

%%%%%%%%%% 腹 %%%%%%%%%%
\subsection*{腹}

\begin{EntryWithPhonetic}{腹}{fu4}{13}{⾁}
  \definition*{s.}{Sobrenome: Fu}
  \definition[个]{s.}{barriga (do corpo); abdômen; estômago | barriga (de uma garrafa, etc.) | coração; mente | parte vazia e saliente no meio de um recipiente ou vaso}
\end{EntryWithPhonetic}

\begin{EntryWithPhonetic}{腹部}{fu4bu4}{13,10}{⾁、⾢}[HSK 7-9]
  \definition{s.}{abdômen; estômago; barriga}
\end{EntryWithPhonetic}

\begin{EntryWithPhonetic}{腹泻}{fu4xie4}{13,8}{⾁、⽔}[HSK 7-9]
  \definition{s.}{diarreia; refere-se ao aumento da frequência de fezes aquosas, com pus ou com sangue, acompanhadas de dor abdominal, causadas por infecção intestinal ou disfunção digestiva}
\end{EntryWithPhonetic}

%%%%%%%%%% 覆 %%%%%%%%%%
\subsection*{覆}

\begin{EntryWithPhonetic}{覆}{fu4}{18}{⾑}
  \definition{v.}{cobrir; encapar | derrubar; perturbar; virar de cabeça para baixo}
\end{EntryWithPhonetic}

\begin{EntryWithPhonetic}{覆盖}{fu4gai4}{18,11}{⾑、⽫}[HSK 7-9]
  \definition{s.}{vegetação; cobertura vegetal; refere-se às plantas que cobrem o solo}
  \definition{v.}{cobrir}
\end{EntryWithPhonetic}

\begin{EntryWithPhonetic}{覆盆子}{fu4pen2zi5}{18,9,3}{⾑、⽫、⼦}
  \definition{s.}{framboesa}
\end{EntryWithPhonetic}

%%%%% EOF %%%%%


 %%%
%%% G
%%%

\section*{G}\addcontentsline{toc}{section}{G}

\begin{EntryWithPhonetic}{夹}{ga1}{6}{⼤}
  \definition{s.}{axila; sovaco; atualmente, costuma-se escrever 胳肢窝}
  \seeref{jia1}
  \seeref{jia2}
  \seealsoref{胳肢窝}{ga1 zhi1 wo1}
\end{EntryWithPhonetic}

\begin{EntryWithPhonetic}{胳}{ga1}{10}{⾁}
  \definition{s.}{usado em 胳肢窝}
  \seeref{ge1}
  \seeref{ge2}
  \seealsoref{胳肢窝}{ga1 zhi1 wo1}
\end{EntryWithPhonetic}

\begin{EntryWithPhonetic}{胳肢窝}{ga1 zhi1 wo1}{10,8,12}{⾁、⾁、⽳}
  \definition{s.}{axila; sovaco; também escrito 夹肢窝}
  \seealsoref{夹肢窝}{jia1 zhi1 wo1}
\end{EntryWithPhonetic}

\begin{EntryWithPhonetic}{该}{gai1}{8}{⾔}[HSK 2]
  \definition{adj.}{completo; integral; abrangente; inclusivo; o mesmo que 赅}
  \definition{pron.}{isto; aquilo; o referido; o acima mencionado; indica a pessoa ou coisa mencionada acima, equivalente a 此, 这个, etc.}
  \definition{v.}{deveria ser; deveria ser assim | caber a alguém; ser a vez (ou dever) de alguém fazer algo | merecer; servir a alguém de direito; indica que algo deve ser feito | dever | deve; provavelmente irá; muito provavelmente; pode ser razoavelmente ou naturalmente esperado que; expressa uma conclusão lógica ou provável com base na razão ou na experiência}
  \definition{v.aux.}{usado em frases exclamativas, tem a função de reforçar o tom}
  \seealsoref{此}{ci3}
  \seealsoref{赅}{gai1}
  \seealsoref{这个}{zhe4ge5}
\end{EntryWithPhonetic}

\begin{EntryWithPhonetic}{赅}{gai1}{10}{⾙}
  \definition*{s.}{Sobrenome Gai}
  \definition{adj.}{completo; integral; abrangente; inclusivo}
\end{EntryWithPhonetic}

\begin{EntryWithPhonetic}{改}{gai3}{7}{⽁}[HSK 2]
  \definition{v.}{mudar; converter; transformar; alterar; substituir | alterar; revisar; aperfeiçoar; modificar | corrigir; retificar; remediar; consertar}
\end{EntryWithPhonetic}

\begin{EntryWithPhonetic}{改变}{gai3bian4}{7,8}{⽁、⼜}[HSK 2]
  \definition{v.}{mudar; alterar; transformar; converter; moldar; modificar | causar mudanças; alterar}
\end{EntryWithPhonetic}

\begin{EntryWithPhonetic}{改革}{gai3ge2}{7,9}{⽁、⾰}[HSK 5]
  \definition[项,次,种]{s.}{reforma; reformação; iniciativas para aprimorar a inovação}
  \definition{v.}{reformar; transformar as antigas partes irracionais das coisas em novas que possam ser adaptadas à situação objetiva}
\end{EntryWithPhonetic}

\begin{EntryWithPhonetic}{改进}{gai3jin4}{7,7}{⽁、⾡}[HSK 3]
  \definition[个,些]{s.}{melhoria}
  \definition{v.}{aprimorar; aperfeiçoar; melhorar; tornar melhor; mudar a situação antiga para melhorar | modificar (mudança mecânica)}
\end{EntryWithPhonetic}

\begin{EntryWithPhonetic}{改良}{gai3liang2}{7,7}{⽁、⾉}
  \definition{v.}{melhorar (algo) | reformar (um sistema)}
\end{EntryWithPhonetic}

\begin{EntryWithPhonetic}{改善}{gai3shan4}{7,12}{⽁、⼝}[HSK 4]
  \definition{v.}{melhorar; amenizar; mudar a situação original para torná-la melhor}
\end{EntryWithPhonetic}

\begin{EntryWithPhonetic}{改善关系}{gai3shan4guan1xi5}{7,12,6,7}{⽁、⼝、⼋、⽷}
  \definition{v.}{melhorar a relação}
\end{EntryWithPhonetic}

\begin{EntryWithPhonetic}{改善通讯}{gai3shan4tong1xun4}{7,12,10,5}{⽁、⼝、⾡、⾔}
  \definition{v.}{melhorar a comunicação}
\end{EntryWithPhonetic}

\begin{EntryWithPhonetic}{改造}{gai3 zao4}{7,10}{⽁、⾡}[HSK 3]
  \definition{v.}{transformar; renovar; modificar o original para melhor se adequar às necessidades; usado principalmente para coisas específicas | remodelar; mudar radicalmente o que é velho e ruim; criar algo novo e bom, para se adaptar às novas circunstâncias e necessidades; usado principalmente para coisas abstratas}
\end{EntryWithPhonetic}

\begin{EntryWithPhonetic}{改正}{gai3 zheng4}{7,5}{⽁、⽌}[HSK 4]
  \definition{v.}{corrigir; emendar; mudar o errado para o correto}
\end{EntryWithPhonetic}

\begin{EntryWithPhonetic}{改装}{gai3 zhuang1}{7,12}{⽁、⾐}[HSK 6]
  \definition{v.}{mudar de traje ou vestido | reembalar | reequipar; reaparelhar | modificar; alterar o dispositivo original}
\end{EntryWithPhonetic}

\begin{EntryWithPhonetic}{芥}{gai4}{7}{⾋}
  \definition{s.}{mostarda}
  \seeref{jie4}
  \seealsoref{芥蓝}{gai4lan2}
\end{EntryWithPhonetic}

\begin{EntryWithPhonetic}{芥兰}{gai4lan2}{7,5}{⾋、⼋}
  \variantof{芥蓝}
\end{EntryWithPhonetic}

\begin{EntryWithPhonetic}{芥蓝}{gai4lan2}{7,13}{⾋、⾋}
  \definition{s.}{brócolis chinês; couve chinesa; mostarda}
  \seealsoref{格兰菜}{ge2lan2cai4}
\end{EntryWithPhonetic}

\begin{EntryWithPhonetic}{盖}{gai4}{11}{⽫}[HSK 4]
  \definition*{s.}{Sobrenome Gai}
  \definition{adj.}{excelente; soberbo; fantástico}
  \definition{adv.}{cerca de; ao redor; aproximadamente; expressa um julgamento especulativo sobre algo, ou uma explicação da causa, o que é equivalente a 大概 ou 原来}
  \definition{conj.}{para; porque; dando continuidade à frase anterior, afirmando a razão ou causa, com tom incerto}
  \definition{s.}{tampa; capa; cobertura; algo que cobre ou sela a parte superior de um objeto | carapaça; concha (de tartaruga, caranguejo, etc.); ossos em formato de crânio em certas partes do corpo humano; as conchas nas costas de certos animais | dossel; capota; toldo | nivelador (uma ferramenta agrícola usada para nivelar terras)}
  \definition{v.}{cobrir; proteger; colocar uma capa em; colocar uma tampa em um objeto | selar; afixar um selo em | superar; sobressair; sobrepujar; ultrapassar | construir; colocar para cima | esconder; ocultar; encobrir | nivelar o terreno com um nivelador (ferramenta agrícola)}
  \seeref{ge3}
  \seealsoref{大概}{da4gai4}
  \seealsoref{原来}{yuan2lai2}
\end{EntryWithPhonetic}

\begin{EntryWithPhonetic}{概}{gai4}{13}{⽊}
  \definition{adj.}{geral; aproximado}
  \definition{adv.}{sem exceção; categoricamente}
  \definition{s.}{ideia principal; esboço geral | maneira de se portar e conduzir; comportamento}
  \definition{v.}{generalizar; exemplificar; tipificar}
\end{EntryWithPhonetic}

\begin{EntryWithPhonetic}{概括}{gai4kuo4}{13,9}{⽊、⼿}[HSK 4]
  \definition{adj.}{genérico; simples e claro, captando o conteúdo principal}
  \definition{s.}{generalização}
  \definition{v.}{generalizar; resumir}
\end{EntryWithPhonetic}

\begin{EntryWithPhonetic}{概念}{gai4nian4}{13,8}{⽊、⼼}[HSK 3]
  \definition[个,种,项]{s.}{ideia; noção; conceito; concepção; uma forma de pensamento que resume as características comuns de algo em uma palavra}
\end{EntryWithPhonetic}

\begin{EntryWithPhonetic}{干}{gan1}{3}{⼲}[HSK 1][Kangxi 51]
  \definition*{s.}{Sobrenome Gan}
  \definition{adj.}{seco (oposto a 湿) | vazio; oco; seco | sem substância; vazio | de parentesco nominal; (parentes) não ligados por laços sanguíneos | sem água; (água) esgotada; completamente vazia | assumido como parente nominal; relação familiar reconhecida por adoção | rude; grosseiro; mal-educado; descreve alguém que fala de forma muito direta e rude (sem delicadeza).}
  \definition{adv.}{em vão; fútil; sem propósito; para nada; sem resultado | apenas; sem nada mais | inutilmente; sem uso, sem aproveitamento | superficialmente; significa que não há conteúdo, apenas forma}
  \definition{s.}{(arcaico) escudo | margem; ribeira; margem das águas | alimentos desidratados | abreviação para os dez troncos celestiais}
  \definition{v.}{ofender; afrontar | ter a ver com; estar relacionado com; estar implicado em; interferir com | (antiquado) buscar (cargo público, remuneração, etc.) | (dialeto) deixar alguém de fora; tratar alguém com indiferença; desprezar | assediar; perturbar; criar confusão; causar estragos; bagunçar | solicitar; procurar; buscar (cargo, salário, etc.) | beber até o fim | tratar com indiferença; ignorar}
  \seeref{gan4}
  \seealsoref{干儿}{gan1 er2}
  \seealsoref{干儿}{gan1r5}
  \seealsoref{湿}{shi1}
\end{EntryWithPhonetic}

\begin{EntryWithPhonetic}{干杯}{gan1/bei1}{3,8}{⼲、⽊}[HSK 2]
  \definition{interj.}{Saúde!}
  \definition{v.+compl.}{fazer um brinde;  brindar até a última gota}
\end{EntryWithPhonetic}

\begin{EntryWithPhonetic}{干脆}{gan1cui4}{3,10}{⼲、⾁}[HSK 5]
  \definition{adj.}{claro; direto; (falar, fazer coisas) sem hesitação; atitude clara}
  \definition{adv.}{justamente; diretamente; sem maiores considerações}
\end{EntryWithPhonetic}

\begin{EntryWithPhonetic}{干儿}{gan1 er2}{3,2}{⼲、⼉}
  \definition{s.}{filho adotivo (adoção tradicional, ou seja, sem implicações legais)}
  \seeref{gan1r5}
\end{EntryWithPhonetic}

\begin{EntryWithPhonetic}{干净}{gan1jing4}{3,8}{⼲、⼎}[HSK 1]
  \definition{adj.}{limpo; limpo e arrumado; sem poeira, impurezas, etc. |}
  \definition{adv.}{completely; totally; sem deixar nada para trás}
\end{EntryWithPhonetic}

\begin{EntryWithPhonetic}{干你屁事}{gan1 ni3 pi4shi4}{3,7,7,8}{⼲、⼈、⼫、⼅}
  \definition{interj.}{Foda-se!}
\end{EntryWithPhonetic}

\begin{EntryWithPhonetic}{干儿}{gan1r5}{3,2}{⼲、⼉}
  \definition{s.}{alimentos secos, desidratados}
  \seeref{gan1 er2}
\end{EntryWithPhonetic}

\begin{EntryWithPhonetic}{干扰}{gan1rao3}{3,7}{⼲、⼿}[HSK 5]
  \definition{v.}{perturbar; incomodar | interferir; interromper o funcionamento adequado de equipamentos eletrônicos com sinais eletrônicos dispersos}
\end{EntryWithPhonetic}

\begin{EntryWithPhonetic}{干涉}{gan1she4}{3,10}{⼲、⽔}[HSK 6]
  \definition{s.}{interferência; refere-se ao ato ou comportamento de interferir nos assuntos dos outros}
  \definition{v.}{interferir; intervir; intrometer-se; pedir ou impedir algo geralmente significa interferir quando não se deve}
\end{EntryWithPhonetic}

\begin{EntryWithPhonetic}{干与}{gan1yu4}{3,3}{⼲、⼀}
  \variantof{干预}
\end{EntryWithPhonetic}

\begin{EntryWithPhonetic}{干预}{gan1yu4}{3,10}{⼲、⾴}[HSK 5]
  \definition{s.}{intromissão; intervenção}
  \definition{v.}{intrometer-se; intervir; interpor-se;}
\end{EntryWithPhonetic}

\begin{EntryWithPhonetic}{甘}{gan1}{5}{⽢}[Kangxi 99]
  \definition*{s.}{Província de Gansu, abreviação de 甘肃 | Sobrenome Gan}
  \definition{adj.}{doce; agradável; satisfatório}
  \definition{v.}{estar disposto a; estar contente ou satisfeito com}
  \seealsoref{甘肃}{gan1su4}
\end{EntryWithPhonetic}

\begin{EntryWithPhonetic}{甘薯}{gan1shu3}{5,16}{⽢、⾋}
  \definition{s.}{batata doce}
\end{EntryWithPhonetic}

\begin{EntryWithPhonetic}{甘肃}{gan1su4}{5,8}{⽢、⾀}
  \definition*{s.}{Província de Gansu}
\end{EntryWithPhonetic}

\begin{EntryWithPhonetic}{甘心}{gan1xin1}{5,4}{⽢、⼼}
  \definition{v.}{estar disposto a | resignar-se a}
\end{EntryWithPhonetic}

\begin{EntryWithPhonetic}{杆}{gan1}{7}{⽊}[HSK 6]
  \definition{s.}{poste; pólo; mastro}
  \seeref{gan3}
\end{EntryWithPhonetic}

\begin{EntryWithPhonetic}{肝}{gan1}{7}{⾁}[HSK 6]
  \definition[个]{s.}{fígado; um dos órgãos digestivos dos humanos e dos animais superiores}
\end{EntryWithPhonetic}

\begin{EntryWithPhonetic}{杆}{gan3}{7}{⽊}
  \definition{clas.}{usado para objetos semelhantes a hastes}
  \definition{s.}{eixo; braço | haste; barra; poste; a parte longa e fina de um objeto, semelhante a um bastão}
  \seeref{gan1}
\end{EntryWithPhonetic}

\begin{EntryWithPhonetic}{赶}{gan3}{10}{⾛}[HSK 3]
  \definition*{s.}{Sobrenome Gan}
  \definition{prep.}{por; até; até que; até quando; introduzir o momento em que algo aconteceu, indicando que se espera até um determinado momento}
  \definition{v.}{ultrapassar; alcançar | perseguir; correr atrás; tentar alcançar; dar uma corrida; acelerar ou intensificar  | dirigir; conduzir | expulsar; afugentar; afastar | encontrar; deparar-se com; esbarrar em; acontecer; encontrar-se em (uma situação); aproveitar-se de (uma oportunidade) | ir para; participar (atividades com horário marcado)}
\end{EntryWithPhonetic}

\begin{EntryWithPhonetic}{赶不上}{gan3 bu5 shang4}{10,4,3}{⾛、⼀、⼀}[HSK 6]
  \definition{v.}{ficar para trás; ser incapaz de alcançar; não conseguir alcançar; não conseguir acompanhar | ser tarde demais (para fazer algo); (não) existir tempo suficiente (para fazer algo) |  deixar de ter; ser incapaz de encontrar ou ter a chance de encontrar; não encontrar; não encontrar (boa oportunidade) | não poder ser comparado a}
\end{EntryWithPhonetic}

\begin{EntryWithPhonetic}{赶到}{gan3 dao4}{10,8}{⾛、⼑}[HSK 3]
  \definition{v.}{correr (para algum lugar); apressar-se}
\end{EntryWithPhonetic}

\begin{EntryWithPhonetic}{赶赴}{gan3fu4}{10,9}{⾛、⾛}
  \definition{v.}{apressar}
\end{EntryWithPhonetic}

\begin{EntryWithPhonetic}{赶集}{gan3ji2}{10,12}{⾛、⾫}
  \definition{v.}{ir a uma feira | ir ao mercado}
\end{EntryWithPhonetic}

\begin{EntryWithPhonetic}{赶脚}{gan3jiao3}{10,11}{⾛、⾁}
  \definition{v.}{transportar mercadorias para ganhar a vida (especialmente de burro) | trabalhar como carroceiro ou porteiro}
\end{EntryWithPhonetic}

\begin{EntryWithPhonetic}{赶紧}{gan3jin3}{10,10}{⾛、⽷}[HSK 3]
  \definition{adv.}{apressadamente; precipitadamente; às pressas; significa agir imediatamente, sem demora}
\end{EntryWithPhonetic}

\begin{EntryWithPhonetic}{赶快}{gan3kuai4}{10,7}{⾛、⼼}[HSK 3]
  \definition{adv.}{rapidamente; imediatamente; aproveite o momento e acelere o ritmo}
\end{EntryWithPhonetic}

\begin{EntryWithPhonetic}{赶路}{gan3lu4}{10,13}{⾛、⾜}
  \definition{v.}{apressar a jornada | apressar-se}
\end{EntryWithPhonetic}

\begin{EntryWithPhonetic}{赶忙}{gan3 mang2}{10,6}{⾛、⼼}[HSK 6]
  \definition{adv.}{imediatamente; com pressa; às pressas; rapidamente}
\end{EntryWithPhonetic}

\begin{EntryWithPhonetic}{赶跑}{gan3pao3}{10,12}{⾛、⾜}
  \definition{v.}{afastar | forçar a saída | repelir}
\end{EntryWithPhonetic}

\begin{EntryWithPhonetic}{赶上}{gan3 shang4}{10,3}{⾛、⼀}[HSK 6]
  \definition{v.}{alcançar; manter o ritmo com; acompanhar alguém ou o padrão do planejador | chegar a tempo para; ter tempo suficiente; não ser tarde demais | encontrar; topar com; cruzar com; encontrar-se com; acontecer de encontrar; encontrar algo, em um determinado momento ou oportunidade}
\end{EntryWithPhonetic}

\begin{EntryWithPhonetic}{赶早}{gan3zao3}{10,6}{⾛、⽇}
  \definition{adv.}{o mais breve possível | na primeira oportunidade | antes que seja tarde | quanto antes melhor}
\end{EntryWithPhonetic}

\begin{EntryWithPhonetic}{赶走}{gan3zou3}{10,7}{⾛、⾛}
  \definition{v.}{expulsar | voltar atrás}
\end{EntryWithPhonetic}

\begin{EntryWithPhonetic}{敢}{gan3}{11}{⽁}[HSK 3]
  \definition{adj.}{ousado; corajoso; audacioso; valente}
  \definition{adv.}{talvez; provavelmente}
  \definition{v.}{ser ousado o suficiente; atrever-se | ter confiança em; ter certeza; estar certo | aventurar-se; ter coragem de fazer algo | ser ousado; arriscar-se}
\end{EntryWithPhonetic}

\begin{EntryWithPhonetic}{敢情}{gan3qing5}{11,11}{⽁、⼼}
  \definition{adv.}{claro | acontece que\dots}
\end{EntryWithPhonetic}

\begin{EntryWithPhonetic}{敢于}{gan3 yu2}{11,3}{⽁、⼆}[HSK 6]
  \definition{v.}{ousar; ser ousado em; ter determinação; ter coragem (para fazer ou se esforçar para fazer)}
\end{EntryWithPhonetic}

\begin{EntryWithPhonetic}{感}{gan3}{13}{⼼}
  \definition{s.}{sentido; sensação; sentimento; impressão | emoção; sentimento}
  \definition{v.}{sentir; perceber; estar ciente | mover; tocar; afetar | ser grato; ser agradecido | ser afetado (pelo frio); pegar um resfriado | (fotografia) sensibilizar | ser grato; apreciar | ser afetado}
\end{EntryWithPhonetic}

\begin{EntryWithPhonetic}{感到}{gan3 dao4}{13,8}{⼼、⼑}[HSK 2]
  \definition{v.}{sentir; achar; perceber}
\end{EntryWithPhonetic}

\begin{EntryWithPhonetic}{感动}{gan3dong4}{13,6}{⼼、⼒}[HSK 2]
  \definition{v.}{mover (alguém) | tocar (alguém emocionalmente)}
\end{EntryWithPhonetic}

\begin{EntryWithPhonetic}{感觉}{gan3jue2}{13,9}{⼼、⾒}[HSK 2]
  \definition[个]{s.}{sentimento; sensação; percepção sensorial;}
  \definition{v.}{sentir; perceber; tomar consciência; sentir no coração, acreditar}
\end{EntryWithPhonetic}

\begin{EntryWithPhonetic}{感冒}{gan3mao4}{13,9}{⼼、⽇}[HSK 3]
  \definition{adj.}{interessado em}
  \definition[场,次]{s.}{resfriado; gripe comum; \emph{influenza}; doença infecciosa causada por um vírus, que tende a causar sintomas como garganta seca, congestão nasal, tosse, espirros, dor de cabeça e febre quando o corpo está excessivamente cansado, resfriado ou com a imunidade enfraquecida}
  \definition{v.}{pegar (ter) um resfriado}
\end{EntryWithPhonetic}

\begin{EntryWithPhonetic}{感情}{gan3qing2}{13,11}{⼼、⼼}[HSK 3]
  \definition[份,个,种]{s.}{emoção; sentimento; reações psicológicas como amor, ódio, alegria, raiva, tristeza e felicidade, geradas por estímulos externos | amor; afeto; apego; preocupação e afeição por pessoas ou coisas}
\end{EntryWithPhonetic}

\begin{EntryWithPhonetic}{感染}{gan3ran3}{13,9}{⼼、⽊}
  \definition{s.}{infecção}
  \definition{v.}{infectar | (figurativo) influenciar}
\end{EntryWithPhonetic}

\begin{EntryWithPhonetic}{感人}{gan3 ren2}{13,2}{⼼、⼈}[HSK 6]
  \definition{adj.}{comovente; tocante}
\end{EntryWithPhonetic}

\begin{EntryWithPhonetic}{感受}{gan3shou4}{13,8}{⼼、⼜}[HSK 3]
  \definition{s.}{percepção ; compreenção; sentimento; experiência; influência do contato com o mundo exterior}
  \definition{v.}{sentir; sentir (através dos sentidos); experimentar; ser afetado}
\end{EntryWithPhonetic}

\begin{EntryWithPhonetic}{感想}{gan3xiang3}{13,13}{⼼、⼼}[HSK 5]
  \definition[个,条]{s.}{pensamentos; impressões; reflexões; resposta do pensamento decorrente da exposição ao mundo exterior}
\end{EntryWithPhonetic}

\begin{EntryWithPhonetic}{感谢}{gan3xie4}{13,12}{⼼、⾔}[HSK 2]
  \definition{v.}{agradecer; ser grato; expressar gratidão com palavras ou ações}
\end{EntryWithPhonetic}

\begin{EntryWithPhonetic}{感兴趣}{gan3xing4qu4}{13,6,15}{⼼、⼋、⾛}[HSK 4]
  \definition{v.}{estar interessado}
  \seealsoref{对……感兴趣}{dui4 gan3xing4qu4}
\end{EntryWithPhonetic}

\begin{EntryWithPhonetic}{橄}{gan3}{15}{⽊}
  \definition*{s.}{Sobrenome Gan}
\end{EntryWithPhonetic}

\begin{EntryWithPhonetic}{橄榄球}{gan3lan3qiu2}{15,13,11}{⽊、⽊、⽟}
  \definition{s.}{futebol jogado com bola oval (rúgbi, futebol americano, regras australianas, etc.)}
\end{EntryWithPhonetic}

\begin{EntryWithPhonetic}{干}{gan4}{3}{⼲}[HSK 1][Kangxi 51]
  \definition{adj.}{capaz; competente; habilidoso}
  \definition{s.}{tronco; parte principal; corpo principal ou parte importante de algo | habilidade; capacidade; competência}
  \definition{v.}{fazer; trabalhar; cuidar; fazer coisas | ocupar o cargo de; estar envolvido em; assumir, exercer | lutar; golpear; esforçar-se}
  \seeref{gan1}
\end{EntryWithPhonetic}

\begin{EntryWithPhonetic}{干活}{gan4/huo2}{3,9}{⼲、⽔}
  \definition{v.+compl.}{trabalhar | trabalhar em um emprego}
\end{EntryWithPhonetic}

\begin{EntryWithPhonetic}{干活儿}{gan4huo2r5}{3,9,2}{⼲、⽔、⼉}[HSK 2]
  \definition{v.}{trabalhar; gastar energia física ou mental para fazer algo, especialmente trabalho árduo ou esforçado.}
\end{EntryWithPhonetic}

\begin{EntryWithPhonetic}{干吗}{gan4 ma2}{3,6}{⼲、⼝}[HSK 3]
  \definition{pron.}{por que?}
  \definition{v.}{o que fazer?}
\end{EntryWithPhonetic}

\begin{EntryWithPhonetic}{干什么}{gan4 shen2 me5}{3,4,3}{⼲、⼈、⼃}[HSK 1]
  \definition{adv.}{o que fazer; o que ele está fazendo?; o que você está fazendo?; perguntar a razão ou o objetivo}
\end{EntryWithPhonetic}

\begin{EntryWithPhonetic}{刚}{gang1}{6}{⼑}[HSK 2]
  \definition*{s.}{Sobrenome Gang}
  \definition{adj.}{duro; firme; rígido; forte; (personalidade, atitude) forte; (vontade) determinada}
  \definition{adv.}{apenas; exatamente; justamente | apenas; apenas por pouco; significa atingir um certo nível com dificuldade | apenas; há pouco tempo; indica que a ação ou situação ocorreu há pouco tempo | assim que; somente neste momento; aconteceu que; use a palavra 就 para indicar que duas coisas estão intimamente relacionadas}
  \seealsoref{就}{jiu4}
\end{EntryWithPhonetic}

\begin{EntryWithPhonetic}{刚才}{gang1cai2}{6,3}{⼑、⼿}[HSK 2]
  \definition{s.}{agora mesmo; há pouco; há pouco tempo; referindo-se ao período recente que acabou de passar}
\end{EntryWithPhonetic}

\begin{EntryWithPhonetic}{刚刚}{gang1 gang5}{6,6}{⼑、⼑}[HSK 2]
  \definition{adv.}{apenas; somente; exatamente; refere-se a algo que é adequado em termos de grau, quantidade, tempo, etc., nem mais nem menos, nem cedo nem tarde, atingindo um estado satisfatório ou que atende exatamente às necessidades | agora mesmo; há pouco; há um momento atrás; referindo-se a um período de tempo muito curto no passado}
\end{EntryWithPhonetic}

\begin{EntryWithPhonetic}{刚好}{gang1 hao3}{6,6}{⼑、⼥}[HSK 6]
  \definition{adj.}{apropriado; na medida certa}
  \definition{adv.}{apenas; acontece que; por acaso}
\end{EntryWithPhonetic}

\begin{EntryWithPhonetic}{扛}{gang1}{6}{⼿}
  \definition{v.}{levantar com as duas mãos | carregar alguma coisa juntos (duas ou mais pessoas)}
  \seeref{kang2}
\end{EntryWithPhonetic}

\begin{EntryWithPhonetic}{杠}{gang1}{7}{⽊}
  \definition{s.}{pequena ponte | mastro de bandeira}
  \seeref{gang4}
\end{EntryWithPhonetic}

\begin{EntryWithPhonetic}{钢}{gang1}{9}{⾦}
  \definition[吨,块,根]{s.}{aço; liga de ferro e carbono}
\end{EntryWithPhonetic}

\begin{EntryWithPhonetic}{钢笔}{gang1 bi3}{9,10}{⾦、⽵}[HSK 5]
  \definition[支,杆]{s.}{caneta-tinteiro; canetas com ponta metálica}
\end{EntryWithPhonetic}

\begin{EntryWithPhonetic}{钢琴}{gang1qin2}{9,12}{⾦、⽟}[HSK 5]
  \definition[架,台]{s.}{piano}
\end{EntryWithPhonetic}

\begin{EntryWithPhonetic}{钢丝}{gang1si1}{9,5}{⾦、⼀}
  \definition{s.}{cabo de aço | corda bamba}
\end{EntryWithPhonetic}

\begin{EntryWithPhonetic}{岗}{gang3}{7}{⼭}
  \definition{s.}{outeiro; monte | crista; vergão (no rosto, pele, etc.) | sentinela; posto | trabalho | batida policial}
\end{EntryWithPhonetic}

\begin{EntryWithPhonetic}{岗位}{gang3 wei4}{7,7}{⼭、⼈}[HSK 6]
  \definition[个,类]{s.}{posto; estação; originalmente se refere ao local guardado pelos militares e pela polícia, agora se refere a uma posição geral}
\end{EntryWithPhonetic}

\begin{EntryWithPhonetic}{港}{gang3}{12}{⽔}
  \definition*{s.}{Sobrenome Gang}
  \definition{s.}{Hong Kong, abreviação de 香港 | porto}
  \seealsoref{香港}{xiang1gang3}
\end{EntryWithPhonetic}

\begin{EntryWithPhonetic}{港口}{gang3kou3}{12,3}{⽔、⼝}[HSK 6]
  \definition[个,座]{s.}{porto; locais com certas condições naturais e instalações portuárias para atracação de navios, embarque e desembarque de passageiros e coleta e distribuição de cargas}
\end{EntryWithPhonetic}

\begin{EntryWithPhonetic}{杠}{gang4}{7}{⽊}
  \definition{s.}{vara grossa | (esportes) barra | peça sobressalente em forma de haste; peça sobressalente em forma de haste usada para máquinas-ferramentas | varas robustas usadas para carregar um caixão | (em um texto) linha grossa desenhada ao lado ou abaixo das palavras como uma marca | (coloquial) padrão; critério}
  \definition{v.}{marcar com uma linha grossa | afiar (faca, navalha, etc.)}
  \seeref{gang1}
\end{EntryWithPhonetic}

\begin{EntryWithPhonetic}{高}{gao1}{10}{⾼}[HSK 1][Kangxi 189]
  \definition*{s.}{Sobrenome Gao}
  \definition{adj.}{alto; elevado; grande distância de baixo para cima; longe do chão | barulhento | sofisticado; caro; de preço elevado; acima do valor real ou do preço de mercado | acima da média; de alto nível ou grau; acima do padrão geral ou da média; de nível superior}
  \definition{s.}{altura; altitude}
\end{EntryWithPhonetic}

\begin{EntryWithPhonetic}{高层}{gao1 ceng2}{10,7}{⾼、⼫}[HSK 6]
  \definition{adj.}{(de um edifício) arranha-céu | (de posição oficial) alto nível}
  \definition{s.}{nível superior; piso, camada, etc. | arranha-céus; um prédio de apartamentos alto}
\end{EntryWithPhonetic}

\begin{EntryWithPhonetic}{高潮}{gao1chao2}{10,15}{⾼、⽔}[HSK 4]
  \definition[个,场]{s.}{maré alta; o nível mais alto da maré em um ciclo de maré | pico; aumento; maré alta; uma metáfora para o estágio mais próspero de desenvolvimento das coisas (diferente de 低潮) | (ficção, drama e filmes) clímax}
  \seealsoref{低潮}{di1chao2}
\end{EntryWithPhonetic}

\begin{EntryWithPhonetic}{高大}{gao1 da4}{10,3}{⾼、⼤}[HSK 5]
  \definition{adj.}{alto e grande; alto | elevado; sublime; nobre}
\end{EntryWithPhonetic}

\begin{EntryWithPhonetic}{高档}{gao1dang4}{10,10}{⾼、⽊}[HSK 6]
  \definition{adj.}{grau superior; alta qualidade; alo grau; qualidade superior; boa qualidade, preço alto (produto)}
\end{EntryWithPhonetic}

\begin{EntryWithPhonetic}{高等}{gao1 deng3}{10,12}{⾼、⽵}[HSK 6]
  \definition{adj.}{superior; avançado (oposto a 低等) | alto nível}
  \seealsoref{低等}{di1 deng3}
\end{EntryWithPhonetic}

\begin{EntryWithPhonetic}{高度}{gao1 du4}{10,9}{⾼、⼴}[HSK 5]
  \definition{adj.}{alto; elevado; avançado; alto grau | alta concentração; intenso}
  \definition[个]{s.}{altura; altitude; elevação; distância de baixo para cima; o grau e o nível em que as coisas se desenvolveram}
\end{EntryWithPhonetic}

\begin{EntryWithPhonetic}{高尔夫}{gao1'er3fu1}{10,5,4}{⾼、⼩、⼤}
  \definition{s.}{(empréstimo linguístico) \emph{golf}}
\end{EntryWithPhonetic}

\begin{EntryWithPhonetic}{高峰}{gao1feng1}{10,10}{⾼、⼭}[HSK 6]
  \definition[个,座]{s.}{cume; pináculo; pico da montanha | pico (de atividade, qualidade ou realização); uma metáfora para o ponto mais alto no desenvolvimento das coisas | cúpula; principais líderes; uma metáfora para o mais alto nível de liderança}
\end{EntryWithPhonetic}

\begin{EntryWithPhonetic}{高跟鞋}{gao1 gen1 xie2}{10,13,15}{⾼、⾜、⾰}[HSK 5]
  \definition[双]{s.}{salto alto; sapatos de salto alto; sapato feminino com salto mais alto e mais distante do chão}
\end{EntryWithPhonetic}

\begin{EntryWithPhonetic}{高级}{gao1ji2}{10,6}{⾼、⽷}[HSK 2]
  \definition{adj.}{sênior; de alto escalão; de alto nível; elevado; excelente; superior; estágio avançado | e alta qualidade; de primeira qualidade; avançado}
\end{EntryWithPhonetic}

\begin{EntryWithPhonetic}{高技术}{gao1 ji4 shu4}{10,7,5}{⾼、⼿、⽊}
  \definition{s.}{alta tecnologia; \emph{hight tech}}
  \seealsoref{高科技}{gao1 ke1 ji4}
\end{EntryWithPhonetic}

\begin{EntryWithPhonetic}{高价}{gao1 jia4}{10,6}{⾼、⼈}[HSK 4]
  \definition{s.}{preço alto; bilhete caro; custo elevado; dispendioso}
\end{EntryWithPhonetic}

\begin{EntryWithPhonetic}{高考}{gao1 kao3}{10,6}{⾼、⽼}[HSK 6]
  \definition[次,回,场]{s.}{vestibular; exame de admissão em instituições de ensino superior}
\end{EntryWithPhonetic}

\begin{EntryWithPhonetic}{高科技}{gao1 ke1 ji4}{10,9,7}{⾼、⽲、⼿}[HSK 6]
  \definition[种,类]{s.}{alta tecnologia; \emph{high tech}}
  \seealsoref{高技术}{gao1 ji4 shu4}
\end{EntryWithPhonetic}

\begin{EntryWithPhonetic}{高楼}{gao1lou2}{10,13}{⾼、⽊}
  \definition[座]{s.}{edifício alto | edifício de muitos andares | arranha-céu}
\end{EntryWithPhonetic}

\begin{EntryWithPhonetic}{高尚}{gao1shang4}{10,8}{⾼、⼩}[HSK 4]
  \definition{adj.}{nobre; elevado; descreve um alto padrão moral e uma boa qualidade de pensamento | significativo e não de mau gosto}
\end{EntryWithPhonetic}

\begin{EntryWithPhonetic}{高手}{gao1 shou3}{10,4}{⾼、⼿}[HSK 6]
  \definition[位,个,名,些,群]{s.}{ás; mestre; especialista; \emph{expert}; uma pessoa com habilidades excepcionais}
\end{EntryWithPhonetic}

\begin{EntryWithPhonetic}{高速}{gao1 su4}{10,10}{⾼、⾡}[HSK 3]
  \definition{adj.}{alta velocidade; veloz; rápido}
  \definition[条]{s.}{autoestrada; via expressa; rodovia}
\end{EntryWithPhonetic}

\begin{EntryWithPhonetic}{高速公路}{gao1su4gong1lu4}{10,10,4,13}{⾼、⾡、⼋、⾜}[HSK 3]
  \definition[条]{s.}{via expressa; rodovia; autoestrada; as rodovias destinadas exclusivamente ao tráfego de veículos em alta velocidade são retas e, quando cruzam outras vias, utilizam cruzamentos em nível}
\end{EntryWithPhonetic}

\begin{EntryWithPhonetic}{高铁}{gao1 tie3}{10,10}{⾼、⾦}[HSK 4]
  \definition{s.}{trem de alta velocidade; trem bala}
\end{EntryWithPhonetic}

\begin{EntryWithPhonetic}{高温}{gao1 wen1}{10,12}{⾼、⽔}[HSK 5]
  \definition{s.}{alta temperatura (oposto a 低温); temperatura elevada; hipertermia; megatemperatura; inferno}
  \seealsoref{低温}{di1 wen1}
\end{EntryWithPhonetic}

\begin{EntryWithPhonetic}{高效}{gao1xiao4}{10,10}{⾼、⽁}
  \definition{adj.}{eficiente | altamente eficaz}
\end{EntryWithPhonetic}

\begin{EntryWithPhonetic}{高兴}{gao1xing4}{10,6}{⾼、⼋}[HSK 1]
  \definition{adj.}{contente; feliz; exultante; alegre; satisfeito; animado}
  \definition{v.}{estar contente; estar feliz; estar animado; estar de bom humor; fazer algo com alegria; gostar}
\end{EntryWithPhonetic}

\begin{EntryWithPhonetic}{高于}{gao1 yu2}{10,3}{⾼、⼆}[HSK 5]
  \definition{v.}{ser mais alto do que; sobrepujar}
\end{EntryWithPhonetic}

\begin{EntryWithPhonetic}{高原}{gao1 yuan2}{10,10}{⾼、⼚}[HSK 5]
  \definition[片]{s.}{platô; terras altas; planalto | planalto continental}
\end{EntryWithPhonetic}

\begin{EntryWithPhonetic}{高中}{gao1 zhong1}{10,4}{⾼、⼁}[HSK 2]
  \definition[所,个]{s.}{ensino médio; escola secundária de ensino médio}
\end{EntryWithPhonetic}

\begin{EntryWithPhonetic}{糕}{gao1}{16}{⽶}
  \definition{s.}{bolo; alimentos feitos de farinha de arroz, farinha de trigo, etc.}
\end{EntryWithPhonetic}

\begin{EntryWithPhonetic}{糕点}{gao1dian3}{16,9}{⽶、⽕}
  \definition{s.}{bolos | pastéis}
\end{EntryWithPhonetic}

\begin{EntryWithPhonetic}{糕点店}{gao1dian3 dian4}{16,9,8}{⽶、⽕、⼴}
  \definition{s.}{confeitaria}
\end{EntryWithPhonetic}

\begin{EntryWithPhonetic}{糕点师}{gao1dian3 shi1}{16,9,6}{⽶、⽕、⼱}
  \definition{s.}{confeiteiro}
\end{EntryWithPhonetic}

\begin{EntryWithPhonetic}{搞}{gao3}{13}{⼿}[HSK 5]
  \definition{v.}{fazer; realizar; estar envolvido em; engajar-se em um estudo, fazer algo em relação a, etc. | fazer; produzir; gerar; trabalhar | iniciar; estabelecer; organizar; configurar | consertar (mudar) alguém; fazer alguém sofrer | obter; assegurar; agarrar |  (seguido de um complemento) fazer com que se torne; produzir um determinado efeito ou resultado}
\end{EntryWithPhonetic}

\begin{EntryWithPhonetic}{搞错}{gao3cuo4}{13,13}{⼿、⾦}
  \definition{v.}{cometer um erro}
\end{EntryWithPhonetic}

\begin{EntryWithPhonetic}{搞定}{gao3ding4}{13,8}{⼿、⼧}
  \definition{v.}{consertar | resolver}
\end{EntryWithPhonetic}

\begin{EntryWithPhonetic}{搞鬼}{gao3gui3}{13,9}{⼿、⿁}
  \definition{v.}{fazer travessuras | fazer truques}
\end{EntryWithPhonetic}

\begin{EntryWithPhonetic}{搞好}{gao3 hao3}{13,6}{⼿、⼥}[HSK 5]
  \definition{v.}{fazer um bom trabalho; fazer bem; suar; tornar submisso, tornar útil, por meio de solicitações e presentes amigáveis; amolecer}
\end{EntryWithPhonetic}

\begin{EntryWithPhonetic}{搞混}{gao3hun4}{13,11}{⼿、⽔}
  \definition{v.}{confundir}
\end{EntryWithPhonetic}

\begin{EntryWithPhonetic}{搞乱}{gao3luan4}{13,7}{⼿、⼄}
  \definition{v.}{estragar | confundir | bagunçar}
\end{EntryWithPhonetic}

\begin{EntryWithPhonetic}{搞钱}{gao3qian2}{13,10}{⼿、⾦}
  \definition{v.}{fazer dinheiro | acumular dinheiro}
\end{EntryWithPhonetic}

\begin{EntryWithPhonetic}{搞通}{gao3tong1}{13,10}{⼿、⾡}
  \definition{v.}{entender algo}
\end{EntryWithPhonetic}

\begin{EntryWithPhonetic}{搞笑}{gao3xiao4}{13,10}{⼿、⽵}
  \definition{adj.}{engraçado | hilário}
  \definition{v.}{fazer as pessoas rirem}
\end{EntryWithPhonetic}

\begin{EntryWithPhonetic}{稿}{gao3}{15}{⽲}
  \definition[篇]{s.}{(significado original) talo de grão; palha | rascunho; esboço; manuscrito | texto original}
\end{EntryWithPhonetic}

\begin{EntryWithPhonetic}{稿纸}{gao3zhi3}{15,7}{⽲、⽷}
  \definition{s.}{rascunho | manuscrito}
\end{EntryWithPhonetic}

\begin{EntryWithPhonetic}{稿子}{gao3 zi5}{15,3}{⽲、⼦}[HSK 6]
  \definition[篇,份,堆,叠]{s.}{rascunho; esboço; rascunhos de poemas, ensaios, desenhos, etc. | rascunho; manuscrito; poemas escritos | ideia; plano; plano preliminar ou conceito de trabalho}
\end{EntryWithPhonetic}

\begin{EntryWithPhonetic}{告}{gao4}{7}{⼝}
  \definition*{s.}{Sobrenome Gao}
  \definition{s.}{anúncio; declaração; notificação}
  \definition{v.}{informar; contar; notificar; explicar aos outros | acusar; processar; relatar | pedir; requisitar; solicitar | dar a conhecer; mostrar | anunciar; declarar; proclamar}
\end{EntryWithPhonetic}

\begin{EntryWithPhonetic}{告别}{gao4/bie2}{7,7}{⼝、⼑}[HSK 3]
  \definition{v.+compl.}{dizer adeus a; expressar a outros, por meio de palavras, que está prestes a partir | deixar; sair; partir de | prestar as últimas homenagens ao falecido}
\end{EntryWithPhonetic}

\begin{EntryWithPhonetic}{告急}{gao4ji2}{7,9}{⼝、⼼}
  \definition{v.}{estar em estado de emergência | relatar uma emergência | solicitar assistência de emergência}
\end{EntryWithPhonetic}

\begin{EntryWithPhonetic}{告诉}{gao4su4}{7,7}{⼝、⾔}
  \definition{v.}{dizer; informar (dar a conhecer); dizer aos outros, para que todos saibam}
  \seeref{gao4su5}
\end{EntryWithPhonetic}

\begin{EntryWithPhonetic}{告诉}{gao4su5}{7,7}{⼝、⾔}[HSK 1]
  \definition{v.}{dizer; informar (dar a conhecer)}
  \seeref{gao4su4}
\end{EntryWithPhonetic}

\begin{EntryWithPhonetic}{哥}{ge1}{10}{⼝}[HSK 1]
  \definition[个,位,名,些]{s.}{irmão mais velho | forma de se dirigir a um parente masculino mais velho de sua geração | irmão; termo amigável para se dirigir a conhecidos mais velhos do sexo masculino}
  \seealsoref{哥哥}{ge1 ge5}
\end{EntryWithPhonetic}

\begin{EntryWithPhonetic}{哥哥}{ge1 ge5}{10,10}{⼝、⼝}[HSK 1]
  \definition[个,位]{s.}{irmão mais velho | primo}
\end{EntryWithPhonetic}

\begin{EntryWithPhonetic}{哥们}{ge1men5}{10,5}{⼝、⼈}
  \definition{expr.}{\emph{Brothers!}}
  \definition{s.}{(coloquial) cara | irmão (forma diminuta de tratamento entre homens)}
\end{EntryWithPhonetic}

\begin{EntryWithPhonetic}{哥斯拉}{ge1si1la1}{10,12,8}{⼝、⽄、⼿}
  \definition*{s.}{Godzilla}
  \seealsoref{酷斯拉}{ku4si1la1}
\end{EntryWithPhonetic}

\begin{EntryWithPhonetic}{胳}{ge1}{10}{⾁}
  \definition{s.}{axila; sovaco}
  \seeref{ga1}
  \seeref{ge2}
\end{EntryWithPhonetic}

\begin{EntryWithPhonetic}{鸽}{ge1}{11}{⿃}
  \definition[只]{s.}{pombo}[和平鸽。===Pomba da Paz.]
\end{EntryWithPhonetic}

\begin{EntryWithPhonetic}{鸽子}{ge1zi5}{11,3}{⿃、⼦}
  \definition{s.}{pombo}
\end{EntryWithPhonetic}

\begin{EntryWithPhonetic}{搁}{ge1}{12}{⼿}
  \definition{v.}{pôr; colocar | colocar à parte; deixar para trás; deixar para mais tarde| deixar de lado}
  \seeref{ge2}
\end{EntryWithPhonetic}

\begin{EntryWithPhonetic}{搁浅}{ge1qian3}{12,8}{⼿、⽔}
  \definition{v.}{ficar encalhado (navio) | encalhar | (figurativo) encontrar dificuldades e parar}
\end{EntryWithPhonetic}

\begin{EntryWithPhonetic}{歌}{ge1}{14}{⽋}[HSK 1]
  \definition[首,支,段]{s.}{canção; poesia cantável}
  \definition{v.}{cantar; entoar | louvar; exaltar; cantar louvores a}
\end{EntryWithPhonetic}

\begin{EntryWithPhonetic}{歌唱}{ge1 chang4}{14,11}{⽋、⼝}[HSK 6]
  \definition{v.}{cantar | cantar em louvor de; louvor através de cânticos, recitações, etc.}
\end{EntryWithPhonetic}

\begin{EntryWithPhonetic}{歌词}{ge1 ci2}{14,7}{⽋、⾔}[HSK 6]
  \definition{s.}{letra da música; libreto}
\end{EntryWithPhonetic}

\begin{EntryWithPhonetic}{歌迷}{ge1 mi2}{14,9}{⽋、⾡}
  \definition{s.}{fã de um cantor; pessoas que gostam de ouvir música ou cantar e ficam fascinadas por isso}
\end{EntryWithPhonetic}

\begin{EntryWithPhonetic}{歌曲}{ge1 qu3}{14,6}{⽋、⽈}[HSK 5]
  \definition[首,支]{s.}{música; obra para as pessoas cantarem, uma combinação de poesia e música}
\end{EntryWithPhonetic}

\begin{EntryWithPhonetic}{歌声}{ge1 sheng1}{14,7}{⽋、⼠}[HSK 3]
  \definition{s.}{canto; voz cantada; som do canto}
\end{EntryWithPhonetic}

\begin{EntryWithPhonetic}{歌手}{ge1 shou3}{14,4}{⽋、⼿}[HSK 3]
  \definition[个,位,名]{s.}{cantor; vocalista; pessoa com talento para cantar}
\end{EntryWithPhonetic}

\begin{EntryWithPhonetic}{歌星}{ge1 xing1}{14,9}{⽋、⽇}[HSK 6]
  \definition[位,名]{s.}{cantor famoso; estrela da música}
\end{EntryWithPhonetic}

\begin{EntryWithPhonetic}{阁}{ge2}{9}{⾨}
  \definition{s.}{pavilhão (geralmente de dois andares) | gabinete (de um governo) | (datado) quarto da mulher; \emph{boudoir} | prateleira}
\end{EntryWithPhonetic}

\begin{EntryWithPhonetic}{阁下}{ge2xia4}{9,3}{⾨、⼀}
  \definition{pron.}{Sua Excelência | Sua Majestade | \emph{Sire}}
\end{EntryWithPhonetic}

\begin{EntryWithPhonetic}{革}{ge2}{9}{⾰}[Kangxi 177]
  \definition*{s.}{Sobrenome Ge}
  \definition{s.}{couro; pele; peles de animais depiladas e processadas}
  \definition{v.}{mudar; transformar; reformar | demitir; remover do cargo; expulsar}
\end{EntryWithPhonetic}

\begin{EntryWithPhonetic}{革新}{ge2 xin1}{9,13}{⾰、⽄}[HSK 6]
  \definition{v.}{inovar; renovar; livrar-se do velho e criar o novo}
\end{EntryWithPhonetic}

\begin{EntryWithPhonetic}{格}{ge2}{10}{⽊}
  \definition*{s.}{Sobrenome Ge}
  \definition{s.}{quadrados formados por linhas cruzadas; quadriculado; grade | divisão (horizontal ou não); treliça | padrão; forma; formato; estilo | caso; as categorias morfológicas de substantivos, pronomes e adjetivos em algumas línguas}
  \definition{v.}{resistir; dificultar; obstruir; impedir | estudar cuidadosamente; investigar | lutar; bater}
\end{EntryWithPhonetic}

\begin{EntryWithPhonetic}{格兰菜}{ge2lan2cai4}{10,5,11}{⽊、⼋、⾋}
  \definition{s.}{brócolis chinês | couve chinesa | mostarda}
  \seealsoref{芥蓝}{gai4lan2}
\end{EntryWithPhonetic}

\begin{EntryWithPhonetic}{格外}{ge2wai4}{10,5}{⽊、⼣}[HSK 4]
  \definition{adv.}{especialmente; particularmente; ainda mais; indica mais do que a média | adicionalmente; indica adicional ou extra}
\end{EntryWithPhonetic}

\begin{EntryWithPhonetic}{胳}{ge2}{10}{⾁}
  \definition{v.}{usado em 胳肢}
  \seeref{ga1}
  \seeref{ge1}
  \seealsoref{胳肢}{ge2zhi5}
\end{EntryWithPhonetic}

\begin{EntryWithPhonetic}{胳肢}{ge2zhi5}{10,8}{⾁、⾁}
  \definition{v.}{(dialeto) fazer cócegas}
\end{EntryWithPhonetic}

\begin{EntryWithPhonetic}{鬲}{ge2}{10}{⿀}[Kangxi 193]
  \definition{s.}{um antigo utensílio de cozinha semelhante a um caldeirão; uma grande panela de barro | utilizado em nomes geográficos ou pessoais}
  \seeref{li4}
\end{EntryWithPhonetic}

\begin{EntryWithPhonetic}{搁}{ge2}{12}{⼿}
  \definition{v.}{suportar; resistir}
  \seeref{ge1}
\end{EntryWithPhonetic}

\begin{EntryWithPhonetic}{隔}{ge2}{12}{⾩}[HSK 4]
  \definition{adj.}{seguinte; vizinho}
  \definition{v.}{separar; cortar; dividir; particionar | estar a uma distância de, após ou em um intervalo de | ficar de pé ou deitar entre}
\end{EntryWithPhonetic}

\begin{EntryWithPhonetic}{隔壁}{ge2bi4}{12,16}{⾩、⼟}[HSK 5]
  \definition{s.}{vizinho; casas ou pessoas vizinhas | septo; distante (socialmente distante) | anteparo; partição}
\end{EntryWithPhonetic}

\begin{EntryWithPhonetic}{隔开}{ge2 kai1}{12,4}{⾩、⼶}[HSK 4]
  \definition{v.}{separar; manter separado; barricar; separar completamente duas pessoas (ou coisas) ou duas partes de uma coisa que estão intimamente unidas}
\end{EntryWithPhonetic}

\begin{EntryWithPhonetic}{个}{ge3}{3}{⼈}
  \definition{pron.}{usado em 自个儿}
  \seeref{ge4}
  \seealsoref{自个儿}{zi4ge3r5}
\end{EntryWithPhonetic}

\begin{EntryWithPhonetic}{盖}{ge3}{11}{⽫}
  \definition*{s.}{Sobrenome Ge}
  \seeref{gai4}
\end{EntryWithPhonetic}

\begin{EntryWithPhonetic}{个}{ge4}{3}{⼈}[HSK 1]
  \definition{adj.}{individual}
  \definition{clas.}{usado antes de substantivos que não têm palavras de medida específicas | usado na frente do divisor; usado na frente do número aproximado | usado após verbos com objeto direto |  usado entre verbos e complementos}
  \definition{part.}{usado após pronomes demonstrativos | adicionado após certas palavras de tempo}
  \seeref{ge3}
\end{EntryWithPhonetic}

\begin{EntryWithPhonetic}{个别}{ge4bie2}{3,7}{⼈、⼑}[HSK 4]
  \definition{adj.}{muito poucos; excepcionais}
  \definition{adv.}{separadamente; individualmente; isoladamente}
\end{EntryWithPhonetic}

\begin{EntryWithPhonetic}{个儿}{ge4r5}{3,2}{⼈、⼉}[HSK 5]
  \definition{s.}{tamanho; altura; estatura; tamanho do corpo ou do objeto | pessoas ou coisas consideradas isoladamente; referir-se a uma pessoa ou coisa individualmente}
\end{EntryWithPhonetic}

\begin{EntryWithPhonetic}{个人}{ge4ren2}{3,2}{⼈、⼈}[HSK 3]
  \definition{pron.}{pessoal; si mesmo}
  \definition[个]{s.}{indivíduo; pessoa}
\end{EntryWithPhonetic}

\begin{EntryWithPhonetic}{个体}{ge4ti3}{3,7}{⼈、⼈}[HSK 4]
  \definition[个,位]{s.}{uma única pessoa ou organismo}
\end{EntryWithPhonetic}

\begin{EntryWithPhonetic}{个性}{ge4xing4}{3,8}{⼈、⼼}[HSK 3]
  \definition[种,点儿]{s.}{individualidade; personalidade; caráter individual; as características relativamente fixas de uma pessoa, formadas sob determinadas condições sociais e influências educacionais | propriedade específica; caráter específico; a propriedade ou característica especial que distingue uma coisa de outras coisas}
\end{EntryWithPhonetic}

\begin{EntryWithPhonetic}{个子}{ge4zi5}{3,3}{⼈、⼦}[HSK 2]
  \definition[个,种,些]{s.}{altura; estatura; refere-se ao tamanho do corpo humano e também ao tamanho do corpo dos animais}
\end{EntryWithPhonetic}

\begin{EntryWithPhonetic}{各}{ge4}{6}{⼝}[HSK 3]
  \definition{adv.}{de várias maneiras; de diversas formas; respectivamente; indica que algo é feito separadamente ou que possui uma determinada característica separadamente}
  \definition{pron.}{todo; todos; cada; refere-se a todos os indivíduos dentro de um determinado intervalo, equivalente a 每个}
  \seealsoref{每个}{mei3ge4}
\end{EntryWithPhonetic}

\begin{EntryWithPhonetic}{各地}{ge4 di4}{6,6}{⼝、⼟}[HSK 3]
  \definition{s.}{em todos os lugares; em vários locais}
\end{EntryWithPhonetic}

\begin{EntryWithPhonetic}{各个}{ge4 ge4}{6,3}{⼝、⼈}[HSK 4]
  \definition{adv./pron.}{cada | um a um; um após o outro}
\end{EntryWithPhonetic}

\begin{EntryWithPhonetic}{各位}{ge4 wei4}{6,7}{⼝、⼈}[HSK 3]
  \definition{pron.}{todos; toda a gente; todo mundo | cada um}
\end{EntryWithPhonetic}

\begin{EntryWithPhonetic}{各种}{ge4 zhong3}{6,9}{⼝、⽲}[HSK 3]
  \definition{adv.}{todos os tipos; vários tipos}
\end{EntryWithPhonetic}

\begin{EntryWithPhonetic}{各自}{ge4zi4}{6,6}{⼝、⾃}[HSK 3]
  \definition{pron.}{por si mesmo; por conta própria; cada um por si | cada um; indica cada uma das partes envolvidas}
\end{EntryWithPhonetic}

\begin{EntryWithPhonetic}{给}{gei3}{9}{⽷}[HSK 1]
  \definition{prep.}{por; expressa significado passivo; tem o mesmo significado que 被, 叫; pode ser seguido pelo agente da ação; o agente da ação também pode não aparecer na frase | para; a; seguido por quem se beneficia da ação; igual a 为 | em direção a; seguido pelo destinatário da ação; o mesmo que 向 | indica transmissão}
  \definition{v.}{dar; conceder; fazer com que a outra parte obtenha algo | passar; pagar; indicar que a outra pessoa faça algo | deixar; permitir que alguém faça algo; autorizar alguém a fazer algo}
  \definition{v.aux.}{usado antes de verbos predicativos que expressam passividade, disposição, etc., para reforçar o tom}
  \seeref{ji3}
  \seealsoref{被}{bei4}
  \seealsoref{叫}{jiao4}
  \seealsoref{为}{wei4}
  \seealsoref{向}{xiang4}
\end{EntryWithPhonetic}

\begin{EntryWithPhonetic}{给……打电话}{gei3 da3 dian4 hua4}{9,5,5,8}{⽷、⼿、⽥、⾔}
  \definition{expr.}{dar um telefonema para alguém}
  \seealsoref{打电话}{da3 dian4 hua4}
\end{EntryWithPhonetic}

\begin{EntryWithPhonetic}{根}{gen1}{10}{⽊}[HSK 4]
  \definition*{s.}{Sobrenome Gen}
  \definition{adv.}{completamente; minuciosamente; radicalmente}
  \definition{clas.}{usado para objetos finos, alongados}
  \definition{s.}{raiz (de uma planta) | descendentes; posteridade; analogia com as gerações futuras | raiz (abreviação de raiz quadrada) | radical (química, refere-se a radicais carregados) | base; pé; raiz; parte inferior, base ou parte de um objeto que está presa a outra coisa | a parte de baixo das coisas; fonte; a origem  das coisas | base; fundamento}
\end{EntryWithPhonetic}

\begin{EntryWithPhonetic}{根本}{gen1ben3}{10,5}{⽊、⽊}[HSK 3]
  \definition{adj.}{básico; essencial; fundamental; importante; decisivo}
  \definition{adv.}{nunca; simplesmente; de forma alguma | radicalmente; completamente; nunca (mais usado em negativas)}
  \definition[个]{s.}{base; raiz; fundação; a origem, a base ou a parte mais importante das coisas}
\end{EntryWithPhonetic}

\begin{EntryWithPhonetic}{根据}{gen1ju4}{10,11}{⽊、⼿}[HSK 4]
  \definition{prep.}{com base em; de acordo com; à luz de}
  \definition[个]{s.}{base; fundamentos; razão; fundo; alicerce}
  \definition{v.}{basear; usar algo como premissa para uma conclusão ou como base para uma ação verbal}
\end{EntryWithPhonetic}

\begin{EntryWithPhonetic}{跟}{gen1}{13}{⾜}[HSK 1]
  \definition{conj.}{e; expressa uma relação de união; 和}
  \definition{prep.}{com; Introduzir objetos relacionados à mesma ação, equivalente a 同 | para; em direção a | de; introduzir o objeto de comparação; equivalente a 从, 由 | como; objetos que causam comparações e semelhanças}
  \definition[个]{s.}{calcanhar; parte posterior do pé ou parte posterior do sapato ou meia | base (de um objeto)}
  \definition{v.}{seguir; acompanhar; seguir imediatamente na mesma direção | (de uma mulher) estar casada com; casar-se com alguém}
  \seealsoref{从}{cong2}
  \seealsoref{和}{he2}
  \seealsoref{同}{tong2}
  \seealsoref{由}{you2}
\end{EntryWithPhonetic}

\begin{EntryWithPhonetic}{跟前}{gen1qian2}{13,9}{⾜、⼑}[HSK 5]
  \definition{s.}{próximo; perto de; na frente de; (na ou para) a presença de alguém | o tempo imediatamente anterior a algum evento; tempo que se aproxima}
  \seeref{gen1qian5}
\end{EntryWithPhonetic}

\begin{EntryWithPhonetic}{跟前}{gen1qian5}{13,9}{⾜、⼑}
  \definition{v.}{(filhos de alguém) viver com alguém (exclusivamente com relação à presença ou ausência de crianças)}
  \seeref{gen1qian2}
\end{EntryWithPhonetic}

\begin{EntryWithPhonetic}{跟随}{gen1sui2}{13,11}{⾜、⾩}[HSK 5]
  \definition{s.}{seguidor; usado para se referir a alguém que seguiu}
  \definition{v.}{seguir; ir atrás; acompanhar}
\end{EntryWithPhonetic}

\begin{EntryWithPhonetic}{更}{geng1}{7}{⽈}
  \definition*{s.}{Sobrenome Geng}
  \definition{clas.}{um dos cinco períodos de duas horas em que a noite era anteriormente dividida; vigília; antigamente, a noite era dividida em cinco turnos, cada um com aproximadamente duas horas de duração}
  \definition{v.}{alterar; substituir | experimentar}
  \seeref{geng4}
\end{EntryWithPhonetic}

\begin{EntryWithPhonetic}{更换}{geng1 huan4}{7,10}{⽈、⼿}[HSK 5]
  \definition{v.}{alterar; mudar; substituir; comutar}
\end{EntryWithPhonetic}

\begin{EntryWithPhonetic}{更新}{geng1xin1}{7,13}{⽈、⽄}[HSK 5]
  \definition{v.}{renovar; atualizar; substituir; remover o antigo e substituir pelo novo}
\end{EntryWithPhonetic}

\begin{EntryWithPhonetic}{耕}{geng1}{10}{⽾}
  \definition{v.}{arar; cultivar | trabalhar; fazer | ganhar a vida}
\end{EntryWithPhonetic}

\begin{EntryWithPhonetic}{更}{geng4}{7}{⽈}[HSK 2]
  \definition{adv.}{mais; ainda mais | além disso; além do mais; ainda mais}
  \seeref{geng1}
\end{EntryWithPhonetic}

\begin{EntryWithPhonetic}{更加}{geng4 jia1}{7,5}{⽈、⼒}[HSK 3]
  \definition{adv.}{mais; ainda mais; em maior grau; indica um nível mais profundo ou um aumento ou diminuição quantitativa adicional}
\end{EntryWithPhonetic}

\begin{EntryWithPhonetic}{更是}{geng4 shi4}{7,9}{⽈、⽇}[HSK 6]
  \definition{adv.}{ainda mais (assim)}
\end{EntryWithPhonetic}

\begin{EntryWithPhonetic}{工}{gong1}{3}{⼯}[Kangxi 48]
  \definition*{s.}{Sobrenome Gong}
  \definition{adj.}{fino; requintado; delicado}
  \definition{s.}{trabalhador; operário; artesão | trabalho; labor; trabalho produtivo | projeto; construção; refere-se à engenharia | indústria; refere-se à indústria | homem-dia; a quantidade de trabalho que um trabalhador faz em um dia | uma nota da escala em Gongchepu (工尺谱), correspondente a 3 na notação musical numerada | engenheiro; refere-se a engenheiros}
  \definition{v.}{ser versado em; ser bom em | trabalhar em; agora geralmente escrito como 功}
  \seealsoref{功}{gong1}
  \seealsoref{工尺谱}{gong1 che3 pu3}
\end{EntryWithPhonetic}

\begin{EntryWithPhonetic}{工厂}{gong1chang3}{3,2}{⼯、⼚}[HSK 3]
  \definition[个,家,座,间]{s.}{fábrica; moinho; planta; unidades que realizam atividades de produção industrial diretamente, geralmente incluindo diferentes oficinas}
\end{EntryWithPhonetic}

\begin{EntryWithPhonetic}{工尺谱}{gong1 che3 pu3}{3,4,14}{⼯、⼫、⾔}
  \definition*{s.}{Gongchepu, notação musical tradicional chinesa}
  \definition{s.}{notação musical tradicional chinesa que usa caracteres chineses para representar notas musicais}
\end{EntryWithPhonetic}

\begin{EntryWithPhonetic}{工程}{gong1 cheng2}{3,12}{⼯、⽲}[HSK 4]
  \definition[个,项]{s.}{projeto; programa; trabalhos que utilizam equipamentos grandes e complexos, como projetos de reconstrução urbana e projetos de cestas de alimentos, etc. | engenharia; departamentos de produção e manufatura usam equipamentos grandes e complexos para realizar seu trabalho}
\end{EntryWithPhonetic}

\begin{EntryWithPhonetic}{工程师}{gong1cheng2shi1}{3,12,6}{⼯、⽲、⼱}[HSK 3]
  \definition[个,位,名,些]{s.}{engenheiro; um dos cargos técnicos é o de especialista capaz de realizar de forma independente o projeto e a execução de uma tarefa técnica específica}
\end{EntryWithPhonetic}

\begin{EntryWithPhonetic}{工夫}{gong1 fu1}{3,4}{⼯、⼤}
  \definition[个]{s.}{tempo | tempo livre; lazer}
  \seeref{gong1 fu5}
\end{EntryWithPhonetic}

\begin{EntryWithPhonetic}{工夫}{gong1 fu5}{3,4}{⼯、⼤}[HSK 3]
  \definition[个]{s.}{(um período de) tempo; o tempo ou energia gastos para realizar uma tarefa | tempo livre}
  \seeref{gong1 fu1}
\end{EntryWithPhonetic}

\begin{EntryWithPhonetic}{工具}{gong1ju4}{3,8}{⼯、⼋}[HSK 3]
  \definition[个,件,套]{s.}{ferramenta; ferramentas utilizadas na produção| ferramenta; meio; instrumento; (metáfora) algo ou meio utilizado para atingir um determinado objetivo}
\end{EntryWithPhonetic}

\begin{EntryWithPhonetic}{工龄}{gong1ling2}{3,13}{⼯、⿒}
  \definition{s.}{tempo de serviço | senioridade}
\end{EntryWithPhonetic}

\begin{EntryWithPhonetic}{工人}{gong1ren2}{3,2}{⼯、⼈}[HSK 1]
  \definition[个,名]{s.}{trabalhador; operário; mão de obra; trabalhadores braçais que vivem do salário}
\end{EntryWithPhonetic}

\begin{EntryWithPhonetic}{工商}{gong1 shang1}{3,11}{⼯、⼝}[HSK 6]
  \definition{s.}{indústria e comércio; um termo combinado para indústria e comércio}
\end{EntryWithPhonetic}

\begin{EntryWithPhonetic}{工业}{gong1ye4}{3,5}{⼯、⼀}[HSK 3]
  \definition{s.}{indústria; utilização de recursos naturais; fabricação de meios de produção; meios de subsistência; ou processamento de produtos agrícolas, produtos semiacabados, etc.}
\end{EntryWithPhonetic}

\begin{EntryWithPhonetic}{工艺}{gong1 yi4}{3,4}{⼯、⾋}[HSK 5]
  \definition{s.}{técnica; tecnologia; arte industrial; técnicas ou métodos de fabricação e processamento de produtos | artesanato; arte artesanal}
\end{EntryWithPhonetic}

\begin{EntryWithPhonetic}{工艺品}{gong1 yi4 pin3}{3,4,9}{⼯、⾋、⼝}[HSK 5]
  \definition[个,件]{s.}{trabalho manual; artesanato; habilidade manual; artigo artesanal; itens delicados produzidos com técnicas artesanais. Por exemplo, esculturas em jade, esmaltes Jingtailan, bordados, etc.}
\end{EntryWithPhonetic}

\begin{EntryWithPhonetic}{工资}{gong1zi1}{3,10}{⼯、⾙}[HSK 3]
  \definition[份,笔,月,天]{s.}{pagamento; salário; remuneração; vencimentos; o pagamento em dinheiro ou em espécie feito ao trabalhador como remuneração pelo trabalho realizado}
\end{EntryWithPhonetic}

\begin{EntryWithPhonetic}{工作}{gong1zuo4}{3,7}{⼯、⼈}[HSK 1]
  \definition[份,个,分,项]{s.}{trabalho; emprego | dever; tarefa; negócio}
  \definition{v.}{trabalhar; operar (uma máquina); envolver-se em trabalho físico ou intelectual, também se refere de maneira geral a máquinas e ferramentas operadas por pessoas para realizar funções produtivas}
\end{EntryWithPhonetic}

\begin{EntryWithPhonetic}{工作日}{gong1 zuo4 ri4}{3,7,4}{⼯、⼈、⽇}[HSK 5]
  \definition{s.}{dia de trabalho; dia útil; dias em que você deveria estar trabalhando de acordo com as regras | horas de trabalho por dia; horas do dia para fazer o trabalho necessário}
\end{EntryWithPhonetic}

\begin{EntryWithPhonetic}{公}{gong1}{4}{⼋}[HSK 6]
  \definition*{s.}{Sobrenome Gong}
  \definition{adj.}{público; estatal; coletivo (oposto a 私) | comum; geral | do mundo; internacional; universal; métrico | imparcial; justo; equitativo |  (de um animal) masculino (oposto a 母)}
  \definition{s.}{assuntos públicos; negócios oficiais (ou deveres) (oposto a 私) | autoridade; coletivo | duque | títulos respeitosos para homens idosos; uma saudação respeitosa | marido}
  \definition{v.}{tornar público; divulgar; abrir a todos; exibir}
  \seealsoref{母}{mu3}
  \seealsoref{私}{si1}
\end{EntryWithPhonetic}

\begin{EntryWithPhonetic}{公安}{gong1 an1}{4,6}{⼋、⼧}[HSK 6]
  \definition[名,位]{s.}{segurança pública; a segurança e estabilidade dos direitos dos cidadãos, da propriedade da segurança pública e da ordem social | agente de segurança pública; pessoal que mantém a segurança pública}
\end{EntryWithPhonetic}

\begin{EntryWithPhonetic}{公布}{gong1bu4}{4,5}{⼋、⼱}[HSK 3]
  \definition{v.}{(leis, decretos, comunicados e avisos de órgãos governamentais) promulgar; anunciar; publicar; tornar público; divulgar publicamente}
\end{EntryWithPhonetic}

\begin{EntryWithPhonetic}{公车}{gong1che1}{4,4}{⼋、⾞}
  \definition{s.}{ônibus, abreviação de公共汽车 | carro pertencente a uma organização e usado por seus membros (carro do governo, carro de polícia, carro da empresa etc.), abreviação de 公务用车}
  \seealsoref{公共}{gong1 gong4}
  \seealsoref{公共汽车}{gong1gong4qi4che1}
  \seealsoref{公务用车}{gong1wu4yong4che1}
\end{EntryWithPhonetic}

\begin{EntryWithPhonetic}{公告}{gong1gao4}{4,7}{⼋、⼝}[HSK 5]
  \definition[张,份,项]{s.}{anúncio; notificação de assuntos importantes ao público em geral pelo governo ou por um órgão importante}
  \definition{v.}{anunciar; o governo ou órgão governamental informa publicamente às pessoas algo importante}
\end{EntryWithPhonetic}

\begin{EntryWithPhonetic}{公共}{gong1 gong4}{4,6}{⼋、⼋}[HSK 3]
  \definition{adj.}{público; comum; comunal; comunitário; pertencente à sociedade}
  \definition[辆]{s.}{ônibus}
  \seealsoref{公车}{gong1che1}
  \seealsoref{公共汽车}{gong1gong4qi4che1}
\end{EntryWithPhonetic}

\begin{EntryWithPhonetic}{公共汽车}{gong1gong4qi4che1}{4,6,7,4}{⼋、⼋、⽔、⾞}[HSK 2]
  \definition[辆,个]{s.}{ônibus}
  \seealsoref{公共}{gong1 gong4}
  \seealsoref{公车}{gong1che1}
\end{EntryWithPhonetic}

\begin{EntryWithPhonetic}{公鸡}{gong1 ji1}{4,7}{⼋、⿃}[HSK 6]
  \definition{s.}{galo; frango macho}
\end{EntryWithPhonetic}

\begin{EntryWithPhonetic}{公交车}{gong1 jiao1 che1}{4,6,4}{⼋、⼇、⾞}[HSK 2]
  \definition[辆]{s.}{ônibus urbano; veículo de transporte público}
\end{EntryWithPhonetic}

\begin{EntryWithPhonetic}{公斤}{gong1jin1}{4,4}{⼋、⽄}[HSK 2]
  \definition{clas.}{quilograma (kg)}
\end{EntryWithPhonetic}

\begin{EntryWithPhonetic}{公开}{gong1kai1}{4,4}{⼋、⼶}[HSK 3]
  \definition{adj.}{aberto; público; não oculto; exposto ao público}
  \definition{v.}{tornar público}
\end{EntryWithPhonetic}

\begin{EntryWithPhonetic}{公克}{gong1ke4}{4,7}{⼋、⼗}
  \definition{s.}{grama (medida de peso)}
\end{EntryWithPhonetic}

\begin{EntryWithPhonetic}{公里}{gong1li3}{4,7}{⼋、⾥}[HSK 2]
  \definition{s.}{quilômetro (km)}
\end{EntryWithPhonetic}

\begin{EntryWithPhonetic}{公路}{gong1 lu4}{4,13}{⼋、⾜}[HSK 2]
  \definition[条,段]{s.}{rodovia; via de acesso; via de tráfego; estrada; estrada principal;}
\end{EntryWithPhonetic}

\begin{EntryWithPhonetic}{公民}{gong1min2}{4,5}{⼋、⽒}[HSK 3]
  \definition[个,位]{s.}{cidadão; civil; pessoa que possui a nacionalidade de um país, goza dos direitos e cumpre as obrigações previstos na Constituição e nas demais leis desse país}
\end{EntryWithPhonetic}

\begin{EntryWithPhonetic}{公平}{gong1ping2}{4,5}{⼋、⼲}[HSK 2]
  \definition{adj.}{justo; imparcial; equitativo; equidade}
\end{EntryWithPhonetic}

\begin{EntryWithPhonetic}{公认}{gong1ren4}{4,4}{⼋、⾔}[HSK 5]
  \definition{v.}{(geralmente) reconhecer; (universalmente) aceitar}
\end{EntryWithPhonetic}

\begin{EntryWithPhonetic}{公式}{gong1shi4}{4,6}{⼋、⼷}[HSK 5]
  \definition[个,些,种]{s.}{fórmula; expressão}
\end{EntryWithPhonetic}

\begin{EntryWithPhonetic}{公司}{gong1si1}{4,5}{⼋、⼝}[HSK 2]
  \definition[个,家]{s.}{empresa; companhia; corporação; uma organização industrial e comercial que opera na produção de produtos, circulação de mercadorias ou certos empreendimentos de construção, etc.}
\end{EntryWithPhonetic}

\begin{EntryWithPhonetic}{公司治理}{gong1si1zhi4li3}{4,5,8,11}{⼋、⼝、⽔、⽟}
  \definition{s.}{governança corporativa}
\end{EntryWithPhonetic}

\begin{EntryWithPhonetic}{公务用车}{gong1wu4yong4che1}{4,5,5,4}{⼋、⼒、⽤、⾞}
  \definition{s.}{veículos oficiais}
\end{EntryWithPhonetic}

\begin{EntryWithPhonetic}{公务员}{gong1 wu4 yuan2}{4,5,7}{⼋、⼒、⼝}[HSK 3]
  \definition[个,位,名,些]{s.}{funcionário público; funcionário de órgãos governamentais}
\end{EntryWithPhonetic}

\begin{EntryWithPhonetic}{公用电话}{gong1yong4dian4hua4}{4,5,5,8}{⼋、⽤、⽥、⾔}
  \definition[部]{s.}{telefone público}
\end{EntryWithPhonetic}

\begin{EntryWithPhonetic}{公寓}{gong1yu4}{4,12}{⼋、⼧}
  \definition[套]{s.}{prédio de apartamentos | pensão}
\end{EntryWithPhonetic}

\begin{EntryWithPhonetic}{公元}{gong1yuan2}{4,4}{⼋、⼉}[HSK 4]
  \definition{s.}{D.C. (Depois de~Cristo); a era cristã; um método internacionalmente aceito de registro de datas, o ano lendário do nascimento de Jesus é 1 d.C., também conhecido como o primeiro ano da Era Comum, e é denotado por D.C.}
  \seealsoref{前}{qian2}
\end{EntryWithPhonetic}

\begin{EntryWithPhonetic}{公园}{gong1yuan2}{4,7}{⼋、⼞}[HSK 2]
  \definition[个,座]{s.}{parque; jardim público; os jardins abertos ao público para passeios e descanso geralmente ficam nas cidades, têm muitas flores, árvores e, em alguns casos, lagos}
\end{EntryWithPhonetic}

\begin{EntryWithPhonetic}{公正}{gong1zheng4}{4,5}{⼋、⽌}[HSK 5]
  \definition{adj.}{justo; equitativo; imparcial; de mente justa; equidade e integridade sem favoritismo}
\end{EntryWithPhonetic}

\begin{EntryWithPhonetic}{公众}{gong1 zhong4}{4,6}{⼋、⼈}[HSK 6]
  \definition[对]{s.}{o público; as massas; refere-se à maioria das pessoas na sociedade}
\end{EntryWithPhonetic}

\begin{EntryWithPhonetic}{公主}{gong1zhu3}{4,5}{⼋、⼂}[HSK 6]
  \definition[个,位,名,些]{s.}{princesa; a filha do monarca}
\end{EntryWithPhonetic}

\begin{EntryWithPhonetic}{功}{gong1}{5}{⼒}
  \definition*{s.}{Sobrenome Gong}
  \definition[次,大]{s.}{mérito; façanha; serviço meritório (ação) | resultado; eficácia; realização | habilidade; habilidade técnica; tecnologia e qualificação técnica | trabalho; uma força faz com que um objeto se desloque uma certa distância na direção da força}
\end{EntryWithPhonetic}

\begin{EntryWithPhonetic}{功臣}{gong1chen2}{5,6}{⼒、⾂}
  \definition{s.}{oficial meritório | pessoa que presta serviço excepcional, herói | (fig.) alguém que desempenha um papel vital}
\end{EntryWithPhonetic}

\begin{EntryWithPhonetic}{功夫}{gong1fu5}{5,4}{⼒、⼤}[HSK 3]
  \definition*{s.}{Gongfu (Kung Fu), arte marcial}
  \definition[番]{s.}{habilidade; destreza; conhecimento | luta acrobática; habilidade em artes marciais | esforço; tempo e energia}
\end{EntryWithPhonetic}

\begin{EntryWithPhonetic}{功课}{gong1 ke4}{5,10}{⼒、⾔}[HSK 3]
  \definition[份,门]{s.}{trabalho escolar; dever de casa; refere-se aos trabalhos de casa atribuídos pelos professores aos alunos| tarefa; lições; lição escolar | preparações; preparação necessária antes de fazer algo}
\end{EntryWithPhonetic}

\begin{EntryWithPhonetic}{功能}{gong1neng2}{5,10}{⼒、⾁}[HSK 3]
  \definition[种,项]{s.}{função; os efeitos positivos produzidos por coisas ou métodos}
\end{EntryWithPhonetic}

\begin{EntryWithPhonetic}{攻}{gong1}{7}{⽁}
  \definition*{s.}{Sobrenome Gong}
  \definition{v.}{atacar; assaltar; tomar a ofensiva | acusar; cobrar | estudar; trabalhar em; especializar-se em}
\end{EntryWithPhonetic}

\begin{EntryWithPhonetic}{攻击}{gong1ji1}{7,5}{⽁、⼐}[HSK 6]
  \definition{v.}{atacar; assaltar; lançar uma ofensiva | difamar; caluniar; acusar; atacar (verbalmente)}
\end{EntryWithPhonetic}

\begin{EntryWithPhonetic}{供}{gong1}{8}{⼈}
  \definition*{s.}{Sobrenome Gong}
  \definition{v.}{fornecer; alimentar |  fornecer algo (para uso ou conveniência de); fornecer algumas condições de exploração à outra parte}
  \seeref{gong4}
\end{EntryWithPhonetic}

\begin{EntryWithPhonetic}{供给}{gong1ji3}{8,9}{⼈、⽷}[HSK 6]
  \definition{s.}{fornecer; prover; fornecer produção e necessidades de vida, dinheiro, etc. para aqueles que precisam}
\end{EntryWithPhonetic}

\begin{EntryWithPhonetic}{供应}{gong1 ying4}{8,7}{⼈、⼴}[HSK 4]
  \definition{v.}{fornecer; prover de}
\end{EntryWithPhonetic}

\begin{EntryWithPhonetic}{宫}{gong1}{9}{⼧}[HSK 6]
  \definition*{s.}{Sobrenome Gong}
  \definition[座]{s.}{palácio imperial; palácio; casas onde o imperador, a imperatriz, o príncipe, etc. vivem | morada de seres sobrenaturais; palácio; paraíso; casas onde vivem os deuses na mitologia | templo (usado em um nome de templo) | local para atividades culturais e recreativas; um edifício para atividades culturais e recreativas; casas para fins culturais e de entretenimento | útero | uma nota da antiga escala chinesa de cinco tons, correspondente a 1 na notação musical numerada}
\end{EntryWithPhonetic}

\begin{EntryWithPhonetic}{巩}{gong3}{6}{⼯}
  \definition*{s.}{Sobrenome Gong}
  \definition{s.}{seguro | sólido}
  \definition{v.}{consolidar}
\end{EntryWithPhonetic}

\begin{EntryWithPhonetic}{巩固}{gong3gu4}{6,8}{⼯、⼞}[HSK 6]
  \definition{adj.}{sólido; estável; consolidado; não facilmente abalado (usado principalmente para coisas abstratas)}
  \definition{v.}{consolidar}
\end{EntryWithPhonetic}

\begin{EntryWithPhonetic}{共}{gong4}{6}{⼋}[HSK 4]
  \definition*{s.}{Partido Comunista, abreviação de 共产党 | Sobrenome Gong}
  \definition{adj.}{conjunto; mútuo; geral; comum; o mesmo para todos}
  \definition{adv.}{juntos; juntamente; conjuntamente | em sua totalidade; em todos}
  \definition{v.}{compartilhar com; empreender ou realizar em conjunto}
  \seealsoref{共产党}{gong4chan3dang3}
\end{EntryWithPhonetic}

\begin{EntryWithPhonetic}{共产}{gong4chan3}{6,6}{⼋、⼇}
  \definition{adj.}{comunista}
  \definition{s.}{comunismo}
\end{EntryWithPhonetic}

\begin{EntryWithPhonetic}{共产党}{gong4chan3dang3}{6,6,10}{⼋、⼇、⼉}
  \definition*{s.}{Partido Comunista}
\end{EntryWithPhonetic}

\begin{EntryWithPhonetic}{共计}{gong4ji4}{6,4}{⼋、⾔}[HSK 5]
  \definition{s.}{total; total geral; agregado; montante}
  \definition{v.}{contar até; somar até; totalizar}
\end{EntryWithPhonetic}

\begin{EntryWithPhonetic}{共同}{gong4tong2}{6,6}{⼋、⼝}[HSK 3]
  \definition{adj.}{comum; compartilhado; colaborativo; todos têm}
  \definition{adv.}{juntos; conjuntamente; todos juntos (fazemos)}
\end{EntryWithPhonetic}

\begin{EntryWithPhonetic}{共同体}{gong4tong2ti3}{6,6,7}{⼋、⼝、⼈}
  \definition{s.}{comunidade}
\end{EntryWithPhonetic}

\begin{EntryWithPhonetic}{共享}{gong4 xiang3}{6,8}{⼋、⼇}[HSK 5]
  \definition{v.}{compartilhar; desfrutar juntos; aproveitar as coisas boas juntos}
\end{EntryWithPhonetic}

\begin{EntryWithPhonetic}{共有}{gong4 you3}{6,6}{⼋、⽉}[HSK 3]
  \definition{v.}{compartilhar; possuir (por todos); possuir ou desfrutar em conjunto}
\end{EntryWithPhonetic}

\begin{EntryWithPhonetic}{贡}{gong4}{7}{⾙}
  \definition*{s.}{Sobrenome Gong}
  \definition[个,批]{s.}{tributo}
  \definition{v.}{recomendar uma pessoa adequada à corte imperial; recomendar talentos à corte na era feudal | prestar homenagem; pagar tributos (à corte imperial)}
\end{EntryWithPhonetic}

\begin{EntryWithPhonetic}{贡献}{gong4xian4}{7,13}{⾙、⽝}[HSK 6]
  \definition[份]{s.}{contribuição; boas ações feitas para o país ou para o público}
  \definition{v.}{dedicar; contribuir; contribuir com materiais, força, experiência, etc. para o país ou para o público}
\end{EntryWithPhonetic}

\begin{EntryWithPhonetic}{供}{gong4}{8}{⼈}
  \definition{s.}{oferendas | confissão}
  \definition{v.}{depositar (oferendas) | confessar}
  \seeref{gong1}
\end{EntryWithPhonetic}

\begin{EntryWithPhonetic}{勾}{gou1}{4}{⼓}
  \definition*{s.}{Sobrenome Gou}
  \definition{s.}{nos tempos antigos, referia-se ao lado mais curto de um triângulo retângulo isósceles}
  \definition{v.}{cancelar; riscar; marcar | delinear; desenhar | preencher as juntas da alvenaria com argamassa ou cimento; apontar | adicionar algo para engrossar; engrossar | induzir; evocar; trazer à mente | conspirar com; unir-se a | excluir; apagar | seduzir; atrair}
  \variantof{钩}
  \seeref{gou4}
\end{EntryWithPhonetic}

\begin{EntryWithPhonetic}{沟}{gou1}{7}{⽔}[HSK 5]
  \definition[条,道,段]{s.}{canal; vala; sarjeta; trincheira; cursos d'água ou fortificações escavados | ranhura; sulco raso; uma depressão que se assemelha a uma vala | ravina; barranco; cursos d'água}
\end{EntryWithPhonetic}

\begin{EntryWithPhonetic}{沟通}{gou1tong1}{7,10}{⽔、⾡}[HSK 5]
  \definition{v.}{comunicar; comunicar-se para entender as ideias, opiniões, etc. | conectar; ligar; estabelecer um paralelo entre os dois}
\end{EntryWithPhonetic}

\begin{EntryWithPhonetic}{钩}{gou1}{9}{⾦}
  \definition*{s.}{Sobrenome Gou}
  \definition[只,个]{s.}{gancho | traço de gancho em caracteres chineses | marca de verificação; visto; \emph{tick}; \emph{check mark} | marca em forma de gancho | uma espada em forma de gancho | forma falada do numeral 9 em certas ocasiões}
  \definition{v.}{prender com um gancho; enganchar | fazer crochê | costurar com pontos grandes | costurar com pontos longos}
\end{EntryWithPhonetic}

\begin{EntryWithPhonetic}{狗}{gou3}{8}{⽝}[HSK 2]
  \definition[条,只,群]{s.}{cão; cachorro | palavrão usado para se referir a pessoas más ou seus capangas}
\end{EntryWithPhonetic}

\begin{EntryWithPhonetic}{勾}{gou4}{4}{⼓}
  \definition*{s.}{Sobrenome Gou}
  \definition{s.}{usado em 勾当}
  \seeref{gou1}
  \seealsoref{勾当}{gou4dang4}
\end{EntryWithPhonetic}

\begin{EntryWithPhonetic}{勾当}{gou4dang4}{4,6}{⼓、⼹}
  \definition{s.}{(depreciativo, obscuro) negócio; acordo; esquema; atividade}
\end{EntryWithPhonetic}

\begin{EntryWithPhonetic}{句}{gou4}{5}{⼝}
  \variantof{勾}
  \seeref{ju4}
\end{EntryWithPhonetic}

\begin{EntryWithPhonetic}{构}{gou4}{8}{⽊}
  \definition{s.}{composição literária}
  \definition{v.}{construir | formar | compor}
  \variantof{够}
\end{EntryWithPhonetic}

\begin{EntryWithPhonetic}{构成}{gou4cheng2}{8,6}{⽊、⼽}[HSK 4]
  \definition{s.}{parte; componente; composição; estrutura}
  \definition{v.}{formar; compor; constituir; compor; encaixar muitas partes para formar um todo | consistir; causar; formar (principalmente em termos jurídicos)}
\end{EntryWithPhonetic}

\begin{EntryWithPhonetic}{构建}{gou4 jian4}{8,8}{⽊、⼵}[HSK 6]
  \definition{v.}{estabelecer (usado principalmente para coisas abstratas); montar; instalar}
\end{EntryWithPhonetic}

\begin{EntryWithPhonetic}{构造}{gou4 zao4}{8,10}{⽊、⾡}[HSK 4]
  \definition[种]{s.}{estrutura; construção; disposição, organização e inter-relação dos componentes}
  \definition{v.}{formar; construir}
\end{EntryWithPhonetic}

\begin{EntryWithPhonetic}{诟}{gou4}{8}{⾔}
  \definition*{s.}{Sobrenome Gou}
  \definition{s.}{vergonha; humilhação}
  \definition{v.}{insultar; xingar; falar de forma abusiva}
\end{EntryWithPhonetic}

\begin{EntryWithPhonetic}{诟骂}{gou4ma4}{8,9}{⾔、⾺}
  \definition{v.}{abusar verbalmente | insultar | criticar}
\end{EntryWithPhonetic}

\begin{EntryWithPhonetic}{购}{gou4}{8}{⾙}
  \definition{v.}{comprar}
\end{EntryWithPhonetic}

\begin{EntryWithPhonetic}{购买}{gou4 mai3}{8,6}{⾙、⼄}[HSK 4]
  \definition{v.}{comprar; adquirir; usar dinheiro para obter itens}
\end{EntryWithPhonetic}

\begin{EntryWithPhonetic}{购物}{gou4wu4}{8,8}{⾙、⽜}[HSK 4]
  \definition{s.}{compras; itens comprados; \emph{shopping}}
  \definition{v.}{ir às compras; fazer compras}
\end{EntryWithPhonetic}

\begin{EntryWithPhonetic}{够}{gou4}{11}{⼣}[HSK 2]
  \definition{adj.}{suficiente; adequado; apropriado; atingir e ultrapassar um determinado limite, difícil de suportar}
  \definition{adv.}{suficientemente; o suficiente (para atingir um determinado nível); indica que atingiu um determinado padrão ou nível elevado}
  \definition{v.}{alcançar (algo, esticando-se); (usando membros, etc.) esticar-se para alcançar ou tocar em locais de difícil acesso | atingir (um padrão ou nível); satisfazer ou atingir a quantidade, os padrões, etc. necessários}
\end{EntryWithPhonetic}

\begin{EntryWithPhonetic}{够本}{gou4ben3}{11,5}{⼣、⽊}
  \definition{v.}{empatar | fazer valer o dinheiro}
\end{EntryWithPhonetic}

\begin{EntryWithPhonetic}{够不着}{gou4bu5zhao2}{11,4,11}{⼣、⼀、⽬}
  \definition{v.}{ser incapaz de alcançar}
\end{EntryWithPhonetic}

\begin{EntryWithPhonetic}{够得着}{gou4de5zhao2}{11,11,11}{⼣、⼻、⽬}
  \definition{v.}{estar à altura | alcançar}
\end{EntryWithPhonetic}

\begin{EntryWithPhonetic}{够格}{gou4ge2}{11,10}{⼣、⽊}
  \definition{adj.}{apto | qualificado | apresentável}
\end{EntryWithPhonetic}

\begin{EntryWithPhonetic}{够朋友}{gou4peng2you5}{11,8,4}{⼣、⽉、⼜}
  \definition{v.}{ser um amigo verdadeiro}
\end{EntryWithPhonetic}

\begin{EntryWithPhonetic}{够呛}{gou4qiang4}{11,7}{⼣、⼝}
  \definition{adj.}{suficiente | terrível | insuportável | improvável}
\end{EntryWithPhonetic}

\begin{EntryWithPhonetic}{够戗}{gou4qiang4}{11,8}{⼣、⼽}
  \variantof{够呛}
\end{EntryWithPhonetic}

\begin{EntryWithPhonetic}{够味}{gou4wei4}{11,8}{⼣、⼝}
  \definition{adj.}{excelente | na medida}
\end{EntryWithPhonetic}

\begin{EntryWithPhonetic}{彀}{gou4}{13}{⼸}
  \definition{adj.}{suficiente; adequado}
  \definition{v.}{puxar um arco ao máximo}
\end{EntryWithPhonetic}

\begin{EntryWithPhonetic}{估}{gu1}{7}{⼈}
  \definition{v.}{estimar; avaliar; aferir}
  \seeref{gu4}
\end{EntryWithPhonetic}

\begin{EntryWithPhonetic}{估计}{gu1ji4}{7,4}{⼈、⾔}[HSK 5]
  \definition{v.}{fazer contas; estimar; calcular; julgar a natureza, quantidade, mudança, etc. de uma coisa em uma determinada situação | parecer; parecer como se; aparentar; fazer inferências aproximadas sobre a natureza, a quantidade e a mudança das coisas com base em determinadas circunstâncias}
\end{EntryWithPhonetic}

\begin{EntryWithPhonetic}{姑}{gu1}{8}{⼥}
  \definition{adv.}{provisoriamente; por enquanto}
  \definition[个,位,名,些]{s.}{irmã do pai; tia | irmã do marido; cunhada | mãe do marido; sogra | freira; mulher que exerce uma ocupação religiosa | a irmã do pai de alguém | mulheres jovens (no campo)}
\end{EntryWithPhonetic}

\begin{EntryWithPhonetic}{姑姑}{gu1gu5}{8,8}{⼥、⼥}[HSK 6]
  \definition[个,位,名]{s.}{tia; tia paterna}
\end{EntryWithPhonetic}

\begin{EntryWithPhonetic}{姑娘}{gu1niang5}{8,10}{⼥、⼥}[HSK 3]
  \definition[位,名,个,些]{s.}{menina; jovem senhora; mulher solteira | filha}
\end{EntryWithPhonetic}

\begin{EntryWithPhonetic}{姑且}{gu1qie3}{8,5}{⼥、⼀}
  \definition{adv.}{provisoriamente | por enquanto}
\end{EntryWithPhonetic}

\begin{EntryWithPhonetic}{孤}{gu1}{8}{⼦}
  \definition*{s.}{Sobrenome Gu}
  \definition{adj.}{sozinho; solitário; isolado}
  \definition{pron.}{eu; meu humilde eu (usado por príncipes feudais); título autoproclamado dos príncipes feudais}
  \definition[个,名,位]{s.}{órfão}
\end{EntryWithPhonetic}

\begin{EntryWithPhonetic}{孤独}{gu1du2}{8,9}{⼦、⽝}[HSK 6]
  \definition{adj.}{sozinho; solitário}
\end{EntryWithPhonetic}

\begin{EntryWithPhonetic}{孤儿}{gu1 er2}{8,2}{⼦、⼉}[HSK 6]
  \definition[个,名,位]{s.}{órfão; criança sem pais; crianças que perderam os pais}
\end{EntryWithPhonetic}

\begin{EntryWithPhonetic}{古}{gu3}{5}{⼝}[HSK 3]
  \definition*{s.}{Cuba, abreviação de 古巴 | Sobrenome Gu}
  \definition{adj.}{antigo; milenar; ancestral; secular | simples e sincero | velho | arcaico}
  \definition{pref.}{(distante no tempo; antigo; primitivo) paleo-; arqueo-}
  \definition{s.}{tempos antigos (oposto a 今) | antiguidade; ancestralidade | livros ou ortodoxias dos sábios antigos, a tradição do Tao | uma forma de poesia pré-Tang}
  \seealsoref{古巴}{gu3ba1}
  \seealsoref{今}{jin1}
\end{EntryWithPhonetic}

\begin{EntryWithPhonetic}{古巴}{gu3ba1}{5,4}{⼝、⼰}
  \definition*{s.}{Cuba}
\end{EntryWithPhonetic}

\begin{EntryWithPhonetic}{古城}{gu3cheng2}{5,9}{⼝、⼟}
  \definition{s.}{cidade antiga}
\end{EntryWithPhonetic}

\begin{EntryWithPhonetic}{古代}{gu3dai4}{5,5}{⼝、⼈}[HSK 3]
  \definition{s.}{tempos antigos; o passado é um período muito distante do presente (diferentemente de 近代 e 现代); na periodização histórica chinesa, geralmente se refere ao período anterior a meados do século XIX | sociedade antiga; sociedade primitiva; refere-se especificamente à era da sociedade escravista (às vezes também inclui a era comunal primitiva) |antigamente; tempos antigos; no passado}
  \seealsoref{近代}{jin4dai4}
  \seealsoref{现代}{xian4dai4}
\end{EntryWithPhonetic}

\begin{EntryWithPhonetic}{古典}{gu3dian3}{5,8}{⼝、⼋}[HSK 6]
  \definition{adj.}{clássico; descreve uma obra ou coisa como tendo características tradicionais ou exemplares}
  \definition{s.}{os clássicos}
\end{EntryWithPhonetic}

\begin{EntryWithPhonetic}{古老}{gu3 lao3}{5,6}{⼝、⽼}[HSK 5]
  \definition{adj.}{antigo; antiquado; histórico}
\end{EntryWithPhonetic}

\begin{EntryWithPhonetic}{古人}{gu3ren2}{5,2}{⼝、⼈}
  \definition{s.}{pessoas dos tempos antigos | os antigos | espécies humanas extintas, como \emph{Homo erectus} ou \emph{Homo neanderthalensis} | (literário) pessoa falecida}
\end{EntryWithPhonetic}

\begin{EntryWithPhonetic}{古铜色}{gu3tong2 se4}{5,11,6}{⼝、⾦、⾊}
  \definition{s.}{cor bronze}
\end{EntryWithPhonetic}

\begin{EntryWithPhonetic}{古装}{gu3 zhuang1}{5,12}{⼝、⾐}
  \definition[套]{s.}{traje antigo; roupas tradicionais; roupas de estilo antigo}
\end{EntryWithPhonetic}

\begin{EntryWithPhonetic}{谷}{gu3}{7}{⾕}[Kangxi 150]
  \definition*{s.}{Sobrenome Gu}
  \definition{adj.}{bom; gentil}
  \definition{s.}{vale; ravina; desfiladeiro; garganta; faixa estreita de terra com uma saída no meio de duas colinas ou dois platôs | arroz não descascado | salário de funcionário (na época feudal) | calha; cocho; canal | fossa sob o cerebelo (anatomia); valécula | dificuldade; dilema}
  \definition{v.}{criar (filhos) | crescer}
\end{EntryWithPhonetic}

\begin{EntryWithPhonetic}{股}{gu3}{8}{⾁}[HSK 6]
  \definition*{s.}{Sobrenome Gu}
  \definition{clas.}{usado para coisas em tiras, longas e estreitas | usado para gás, odor, força, etc. | Pejorativo: usado para um grupo de pessoas}
  \definition{s.}{coxa; ancas | seção (de um escritório, empresa, etc.); unidades organizacionais em agências governamentais, empresas e grupos | fio; camada | uma das várias partes iguais de propriedade | ação; \emph{stock}; ação do capital social; uma parte igual de fundos ou propriedade | a perna mais longa de um triângulo retângulo}
\end{EntryWithPhonetic}

\begin{EntryWithPhonetic}{股东}{gu3dong1}{8,5}{⾁、⼀}[HSK 6]
  \definition[个,位,名,家]{s.}{acionista de uma sociedade anônima com direito a participar e votar nas assembleias gerais; refere-se também a investidores em outras empresas industriais e comerciais administradas por sociedades}
\end{EntryWithPhonetic}

\begin{EntryWithPhonetic}{股票}{gu3piao4}{8,11}{⾁、⽰}[HSK 6]
  \definition[只,股]{s.}{ação; quotas; certificado de ações; título de capital; capital social; títulos utilizados para representar ações}
\end{EntryWithPhonetic}

\begin{EntryWithPhonetic}{骨}{gu3}{9}{⾻}[Kangxi 188]
  \definition*{s.}{Sobrenome Gu}
  \definition[根,块]{s.}{osso | esqueleto; estrutura | caráter; espírito | cadáver; corpo}
\end{EntryWithPhonetic}

\begin{EntryWithPhonetic}{骨头}{gu3tou5}{9,5}{⾻、⼤}[HSK 4]
  \definition[根,块]{s.}{osso; tecidos mais duros no corpo de uma pessoa ou de alguns animais que sustentam o corpo ou protegem os órgãos do corpo | caráter de uma pessoa; refere-se à qualidade do caráter de uma pessoa}
\end{EntryWithPhonetic}

\begin{EntryWithPhonetic}{鼓}{gu3}{13}{⿎}[HSK 5][Kangxi 207]
  \definition*{s.}{Sobrenome Gu}
  \definition{adj.}{abaulado; inchado; saliente; protuberante}
  \definition{clas.}{unidades antigas de cronometragem noturna; vigílias da noite}
  \definition[个,架,面,张]{s.}{tambor; instrumento de percussão | coisas semelhantes a tambores; formato, som e função semelhantes aos de um tambor}
  \definition{v.}{soar; bater; golpear; fazer um objeto soar | ventilar; soprar com fole | agitar; despertar; ativar; incitar; revigorar | bater asas | aumentar; fazer beicinho}
\end{EntryWithPhonetic}

\begin{EntryWithPhonetic}{鼓励}{gu3li4}{13,7}{⿎、⼒}[HSK 5]
  \definition{v.}{incitar; encorajar; provocar e incentivar}
\end{EntryWithPhonetic}

\begin{EntryWithPhonetic}{鼓掌}{gu3/zhang3}{13,12}{⿎、⼿}[HSK 5]
  \definition{v.+compl.}{aplaudir; bater palmas, principalmente para expressar felicidade, aprovação ou boas-vindas}
\end{EntryWithPhonetic}

\begin{EntryWithPhonetic}{估}{gu4}{7}{⼈}
  \definition{adj.}{velho | roupas de segunda mão}
  \seeref{gu1}
\end{EntryWithPhonetic}

\begin{EntryWithPhonetic}{固}{gu4}{8}{⼞}
  \definition*{s.}{Sobrenome Gu}
  \definition{adj.}{sólido; firme; forte | duro; sólido | mal informado; superficial; ignorante}
  \definition{adv.}{firmemente; resolutamente | originalmente; em primeiro lugar | certamente; reconhecidamente; seguramente}
  \definition{conj.}{usado da mesma forma que 固然}
  \definition{v.}{solidificar; consolidar; fortalecer | defender; proteger}
  \seealsoref{固然}{gu4ran2}
\end{EntryWithPhonetic}

\begin{EntryWithPhonetic}{固定}{gu4ding4}{8,8}{⼞、⼧}[HSK 4]
  \definition{adj.}{fixo; regular; inalterado ou imóvel}
  \definition{v.}{consertar; tornar fixo, não mover novamente; colocar as coisas em ordem, não mudá-las novamente}
\end{EntryWithPhonetic}

\begin{EntryWithPhonetic}{固然}{gu4ran2}{8,12}{⼞、⽕}
  \definition{conj.}{usado para introduzir uma cláusula adversativa admitindo primeiro um certo fato | admitir um fato sem negar outro}
\end{EntryWithPhonetic}

\begin{EntryWithPhonetic}{故}{gu4}{9}{⽁}
  \definition*{s.}{Sobrenome Gu}
  \definition{adj.}{velho; antigo}
  \definition{adv.}{propositalmente; intencionalmente; deliberadamente}
  \definition{conj.}{assim; portanto; consequentemente; pelo contrário}
  \definition{s.}{evento; incidente; acontecimento; acidente | causa; razão | amigo; conhecido | o velho; refere-se a coisas antigas e passadas}
  \definition{v.}{morrer}
\end{EntryWithPhonetic}

\begin{EntryWithPhonetic}{故宫}{gu4gong1}{9,9}{⽁、⼧}
  \definition*{s.}{Palácio Imperial | Cidade Proibida}
\end{EntryWithPhonetic}

\begin{EntryWithPhonetic}{故事}{gu4shi5}{9,8}{⽁、⼅}[HSK 2]
  \definition[个,段,篇,则]{s.}{história; conto; coisas reais ou fictícias usadas como objeto de narrativa, com coerência, atraentes e capazes de emocionar as pessoas | enredo; trama; enredo que consegue mostrar a personalidade dos personagens e refletir a ideia central da obra literária}
\end{EntryWithPhonetic}

\begin{EntryWithPhonetic}{故乡}{gu4xiang1}{9,3}{⽁、⼄}[HSK 3]
  \definition[个]{s.}{cidade natal; terra natal; local de nascimento ou onde viveu por muito tempo}
\end{EntryWithPhonetic}

\begin{EntryWithPhonetic}{故意}{gu4yi4}{9,13}{⽁、⼼}[HSK 2]
  \definition{adv.}{deliberadamente; intencionalmente; não é por descuido, mas sim conscientemente (geralmente coisas que não se devem fazer ou que não são necessárias)}
  \definition{s.}{intenção; um tipo de mentalidade, uma pessoa sabe claramente que seus atos podem causar danos a outras pessoas ou trazer consequências negativas para a sociedade, mas mesmo assim não faz nada para impedir isso}
\end{EntryWithPhonetic}

\begin{EntryWithPhonetic}{故障}{gu4zhang4}{9,13}{⽁、⾩}[HSK 6]
  \definition[出]{s.}{problema; falha; parada; mau funcionamento; avaria; situações em que máquinas, instrumentos, etc. não podem funcionar normalmente devido a problemas}
\end{EntryWithPhonetic}

\begin{EntryWithPhonetic}{顾}{gu4}{10}{⾴}[HSK 6]
  \definition*{s.}{Sobrenome Gu}
  \definition{adv.}{em vez disso; pelo contrário; indica o oposto, equivalente a 却 ou 反而}
  \definition{conj.}{mas; no entanto}
  \definition{v.}{olhar para trás; olhar para; virar-se e olhar para | cuidar de; atender a; levar em conta ou consideração | visitar; chamar | sentir pena de}
  \seealsoref{反而}{fan3'er2}
  \seealsoref{却}{que4}
\end{EntryWithPhonetic}

\begin{EntryWithPhonetic}{顾客}{gu4ke4}{10,9}{⾴、⼧}[HSK 2]
  \definition[个,位,名,些]{s.}{cliente; comprador; consumidor; paciente}
\end{EntryWithPhonetic}

\begin{EntryWithPhonetic}{顾问}{gu4wen4}{10,6}{⾴、⾨}[HSK 5]
  \definition[个,位,名]{s.}{conselheiro; consultor; assessor; pessoas com conhecimento especializado ou experiência contratadas para prestar consultoria a organizações ou indivíduos}
\end{EntryWithPhonetic}

\begin{EntryWithPhonetic}{瓜}{gua1}{5}{⽠}[HSK 4][Kangxi 97]
  \definition*{s.}{Sobrenome Gua}
  \definition[个]{s.}{qualquer tipo de melão ou cabaça | companheiro (termo depreciativo para uma pessoa)}
  \definition{v.}{fofocar}
\end{EntryWithPhonetic}

\begin{EntryWithPhonetic}{刮}{gua1}{8}{⼑}[HSK 6]
  \definition{v.}{barbear; raspar; depilar | untar com (pasta, etc.)  | extorquir; pilhar; adquirir avidamente (propriedade) por vários meios | (do vento) soprar}
\end{EntryWithPhonetic}

\begin{EntryWithPhonetic}{刮风}{gua1/feng1}{8,4}{⼑、⾵}
  \definition{v.+compl.}{ventar | fazer vento}
\end{EntryWithPhonetic}

\begin{EntryWithPhonetic}{寡}{gua3}{14}{⼧}
  \definition{adj.}{poucos; escassos (oposto a 众, 多)  | insípido; sem sabor | pouco; escasso | insípido; sem graça}
  \definition{pron.}{eu; título autoproclamado de um antigo monarca}
  \definition{s.}{viúva | viuvez; a natureza ou estado de uma mulher viúva que vive sozinha}
  \seealsoref{多}{duo1}
  \seealsoref{众}{zhong4}
\end{EntryWithPhonetic}

\begin{EntryWithPhonetic}{挂}{gua4}{9}{⼿}[HSK 3]
  \definition{clas.}{usado principalmente para coisas que vêm em conjuntos ou séries}
  \definition{v.}{pendurar; colocar; suspender; usando cordas, ganchos, pregos e outros itens para prender objetos em um ou mais pontos específicos | interromper chamada (telefônica) | colocar alguém em contato com; ligar; telefonar; refere-se a ligar o telefone, bem como a fazer uma chamada | falhar; fracassar | colocar em registro; registrarpegar carona; ser pego | preocupar-se com | ser revestido com; ser coberto com | estar pendente; deixar algo sem solução}
\end{EntryWithPhonetic}

\begin{EntryWithPhonetic}{挂号}{gua4/hao4}{9,5}{⼿、⼝}
  \definition{v.+compl.}{registrar-se (em um hospital, etc.) | enviar através de carta registrada}
\end{EntryWithPhonetic}

\begin{EntryWithPhonetic}{挂号信}{gua4hao4xin4}{9,5,9}{⼿、⼝、⼈}
  \definition{s.}{carta registrada}
\end{EntryWithPhonetic}

\begin{EntryWithPhonetic}{乖}{guai1}{8}{⼃}
  \definition{adj.}{(de uma criança) bem comportado; bom; obediente | inteligente; astuto; esperto | (de caráter, comportamento, etc.) estranho; anormal; irracional}
  \definition{v.}{perverter; ser contrário à razão; ir contra | (de caráter, comportamento, etc.) ser anormal; ser estranho}
\end{EntryWithPhonetic}

\begin{EntryWithPhonetic}{乖乖}{guai1guai1}{8,8}{⼃、⼃}
  \definition{adj.}{bem-comportado (criança) | obediente}
  \seeref{guai1guai5}
\end{EntryWithPhonetic}

\begin{EntryWithPhonetic}{乖乖}{guai1guai5}{8,8}{⼃、⼃}
  \definition{expr.}{Graças a Deus! | Oh meu Deus!}
  \seeref{guai1guai1}
\end{EntryWithPhonetic}

\begin{EntryWithPhonetic}{拐}{guai3}{8}{⼿}[HSK 6]
  \definition[支,根,副]{s.}{muleta; bengala; uma bengala com uma barra horizontal na parte superior, usada por pessoas com doenças ou deficiências nos membros inferiores para ajudá-las a caminhar |
sete; forma falada do numeral 七 | esquina; curva; canto}
  \definition{v.}{virar; girar; mudar de direção enquanto se move | enganar | mudar; transformar | mancar}
  \seealsoref{七}{qi1}
\end{EntryWithPhonetic}

\begin{EntryWithPhonetic}{怪}{guai4}{8}{⼼}[HSK 4,5]
  \definition*{s.}{Sobrenome Guai}
  \definition{adj.}{estranho; esquisito; desconcertante | peculiar; excêntrico; pitoresco; monstruoso; anormal; incomum}
  \definition{adv.}{bastante; muito}
  \definition{s.}{monstro; demônio | diabo; ser maligno}
  \definition{v.}{culpar | achar algo estranho; maravilhar-se com; ficar surpreso | repreender; culpar; reclamar}
\end{EntryWithPhonetic}

\begin{EntryWithPhonetic}{怪癖}{guai4pi3}{8,18}{⼼、⽧}
  \definition{adj.}{peculiar}
  \definition{s.}{excentricidade | peculiaridade | hobby estranho}
\end{EntryWithPhonetic}

\begin{EntryWithPhonetic}{怪兽}{guai4shou4}{8,11}{⼼、⼋}
  \definition{s.}{animal raro | animal mítico | monstro}
\end{EntryWithPhonetic}

\begin{EntryWithPhonetic}{关}{guan1}{6}{⼋}[HSK 1,4]
  \definition*{s.}{Sobrenome Guan}
  \definition{s.}{passagem; ponto de controle | alfândega; escritórios de cobrança de impostos para exportação e importação de mercadorias | ponto de inflexão ou barreira; ponto de virada ou dificuldade | momento crítico; mecanismo}
  \definition{v.}{fechar; encerrar; amarrar algo | fechar; trancar | encerrar; sair do mercado; falir | conceder ou sacar o pagamento de um salário | desligar | envolver; preocupar-se; conectar-se}
\end{EntryWithPhonetic}

\begin{EntryWithPhonetic}{关爱}{guan1 ai4}{6,10}{⼋、⽖}[HSK 6]
  \definition{v.}{cuidar; cuidar e amar}
\end{EntryWithPhonetic}

\begin{EntryWithPhonetic}{关闭}{guan1bi4}{6,6}{⼋、⾨}[HSK 4]
  \definition{v.}{fechar | (empresa) falir}
\end{EntryWithPhonetic}

\begin{EntryWithPhonetic}{关怀}{guan1huai2}{6,7}{⼋、⼼}[HSK 5]
  \definition{v.}{mostrar cuidado amoroso por; mostrar solicitude por; cuidar, amar, apoiar ou ajudar os fracos ou grupos em dificuldade | geralmente usado para superiores para subordinados, anciãos para juniores ou organizações para indivíduos}
\end{EntryWithPhonetic}

\begin{EntryWithPhonetic}{关机}{guan1 ji1}{6,6}{⼋、⽊}[HSK 2]
  \definition{v.}{encerrar; terminar; refere-se especificamente à conclusão das filmagens de um filme ou série de TV | desligar; desligar a fonte de alimentação; parar o funcionamento da máquina}
\end{EntryWithPhonetic}

\begin{EntryWithPhonetic}{关键}{guan1jian4}{6,13}{⼋、⾦}[HSK 5]
  \definition{adj.}{crucial; decisivo; importante; que pode determinar o curso e o resultado dos eventos}
  \definition[个,点,些]{s.}{chave; ponto crucial; aspectos ou condições mais importantes que determinam o desenvolvimento e o resultado de algo}
\end{EntryWithPhonetic}

\begin{EntryWithPhonetic}{关联}{guan1 lian2}{6,12}{⼋、⽿}[HSK 6]
  \definition{s.}{conexão; inter-relação; a conexão entre as coisas}
  \definition{v.}{estar relacionado; estar conectado; as coisas estão envolvidas e influenciam umas às outras}
\end{EntryWithPhonetic}

\begin{EntryWithPhonetic}{关上}{guan1 shang4}{6,3}{⼋、⼀}[HSK 1]
  \definition{v.}{fechar (uma porta); fechar um objeto | desligar (luz, equipamento elétrico etc.); parar ou encerrar (uma atividade, situação, etc.)}
\end{EntryWithPhonetic}

\begin{EntryWithPhonetic}{关系}{guan1xi5}{6,7}{⼋、⽷}[HSK 3]
  \definition[个,种]{s.}{relações; conexões; relacionamento; a interligação entre pessoas ou coisas | consequência; impacto; significado a influência ou importância de algo; algo digno de nota (geralmente usado com 没有, 有). | causa; razão (geralmente usado com 由于 ou 因为); refere-se genericamente a causas, condições, etc. | credenciais que mostram filiação a uma organização; documento que comprova a existência de algum tipo de relação organizacional}
  \definition{v.}{preocupar; afetar; ter influência sobre; ter a ver com}
  \seealsoref{没有}{mei2 you3}
  \seealsoref{因为}{yin1wei4}
  \seealsoref{由于}{you2yu2}
  \seealsoref{有}{you3}
\end{EntryWithPhonetic}

\begin{EntryWithPhonetic}{关心}{guan1xin1}{6,4}{⼋、⼼}[HSK 2]
  \definition{v.}{cuidar; preocupar-se com; manifestar interesse por; demonstrar solicitude por; (colocar uma pessoa ou coisa) sempre no coração; valorizar e cuidar}
\end{EntryWithPhonetic}

\begin{EntryWithPhonetic}{关于}{guan1yu2}{6,3}{⼋、⼆}[HSK 4]
  \definition{prep.}{sobre; relativo a; pertencente a; uma questão de; com relação a}
\end{EntryWithPhonetic}

\begin{EntryWithPhonetic}{关注}{guan1 zhu4}{6,8}{⼋、⽔}[HSK 3]
  \definition{v.}{prestar atenção em; seguir algo de perto; seguir (nas redes sociais)}
\end{EntryWithPhonetic}

\begin{EntryWithPhonetic}{观}{guan1}{6}{⾒}
  \definition*{s.}{Templo taoísta; ``Koon''}
  \definition{s.}{visão; vista | perspectiva; visão; conceito | aparência; perspectiva | alcance de visão | noção; ideia; conhecimento ou visão das coisas | ponto de vista; postura; uma visão de uma coisa}
  \definition{v.}{olhar para; assistir; observar | contemplar}
  \seeref{guan4}
\end{EntryWithPhonetic}

\begin{EntryWithPhonetic}{观察}{guan1cha2}{6,14}{⾒、⼧}[HSK 3]
  \definition{v.}{assistir; pesquisar; observar; examinar cuidadosamente coisas ou fenômenos}
\end{EntryWithPhonetic}

\begin{EntryWithPhonetic}{观点}{guan1dian3}{6,9}{⾒、⽕}[HSK 2]
  \definition[个,种]{s.}{ponto de vista; perspectiva; a visão ou atitude que se tem sobre algo a partir de uma determinada posição ou perspectiva | ponto de vista; perspectiva; a posição ou perspectiva adotada ao analisar uma questão}
\end{EntryWithPhonetic}

\begin{EntryWithPhonetic}{观光}{guan1guang1}{6,6}{⾒、⼉}[HSK 6]
  \definition{v.}{visitar; passear; fazer turismo; fazer um passeio em um país ou lugar estrangeiro}
\end{EntryWithPhonetic}

\begin{EntryWithPhonetic}{观看}{guan1 kan4}{6,9}{⾒、⽬}[HSK 3]
  \definition{v.}{assistir; ver propositadamente; observar}
\end{EntryWithPhonetic}

\begin{EntryWithPhonetic}{观念}{guan1nian4}{6,8}{⾒、⼼}[HSK 3]
  \definition[种,个]{s.}{ideia; conceito; consciência ideológica}
\end{EntryWithPhonetic}

\begin{EntryWithPhonetic}{观众}{guan1zhong4}{6,6}{⾒、⼈}[HSK 3]
  \definition[位,名,批,个]{s.}{espectador; público; audiência; pessoas que assistem a espetáculos ou competições}
\end{EntryWithPhonetic}

\begin{EntryWithPhonetic}{官}{guan1}{8}{⼧}[HSK 4]
  \definition*{s.}{Sobrenome Guan}
  \definition{adj.}{propriedade do governo; pertencente ao governo ou ao público | público}
  \definition[个,位,名,些]{s.}{funcionário do governo; oficial; servidor público; titular de cargo; funcionário público nomeado acima de um determinado nível | órgão (parte do tecido do corpo)}
\end{EntryWithPhonetic}

\begin{EntryWithPhonetic}{官方}{guan1fang1}{8,4}{⼧、⽅}[HSK 4]
  \definition{s.}{autoridade; (do ou pelo) governo | oficial (de uma organização ou instituição)}
\end{EntryWithPhonetic}

\begin{EntryWithPhonetic}{官桂}{guan1gui4}{8,10}{⼧、⽊}
  \definition{s.}{canela}
  \seealsoref{肉桂}{rou4gui4}
\end{EntryWithPhonetic}

\begin{EntryWithPhonetic}{官司}{guan1 si5}{8,5}{⼧、⼝}[HSK 6]
  \definition[场,个]{s.}{ação judicial}
\end{EntryWithPhonetic}

\begin{EntryWithPhonetic}{冠}{guan1}{9}{⼍}
  \definition{s.}{chapéu | corona; coroa; copa | crista}
  \seeref{guan4}
\end{EntryWithPhonetic}

\begin{EntryWithPhonetic}{棺}{guan1}{12}{⽊}
  \definition[副]{s.}{caixão; esquife; ataúde}
\end{EntryWithPhonetic}

\begin{EntryWithPhonetic}{管}{guan3}{14}{⽵}[HSK 3]
  \definition*{s.}{Guan, um estado da dinastia Zhou | Sobrenome Guan}
  \definition{adj.}{estreito; restrito; limitado; pequeno}
  \definition{clas.}{usado para objetos cilíndricos longos e finos}
  \definition{conj.}{não importa (quem, o quê, como, etc.)}
  \definition{prep.}{função semelhante a 把, usada especificamente em conjunto com 叫}
  \definition[根,条,排]{s.}{cano; tubo | instrumento musical de sopro | válvula; tubo | duto; canal; vasos}
  \definition{v.}{administrar; dirigir; controlar; cuidar; ser responsável por | ter jurisdição sobre; administrar | disciplinar (crianças ou alunos) | preocupar-se com; importar-se com; incomodar-se com; intervir | fornecer; garantir | supervisionar | governar | submeter alguém a disciplina | assumir; arcar com | incomodar; interferir | assegurar; garantir}
  \seealsoref{把}{ba3}
  \seealsoref{叫}{jiao4}
\end{EntryWithPhonetic}

\begin{EntryWithPhonetic}{管道}{guan3 dao4}{14,12}{⽵、⾡}[HSK 6]
  \definition[根,千米,公里]{s.}{oleoduto; canal; túnel; tubulação; um tubo feito de metal ou outro material usado para transportar ou descarregar fluidos (como vapor, gás, óleo, água, etc.) | caminho; canal; abordagem}
\end{EntryWithPhonetic}

\begin{EntryWithPhonetic}{管家}{guan3jia1}{14,10}{⽵、⼧}
  \definition{s.}{mordomo | governanta}
  \definition{v.}{administrar uma casa}
\end{EntryWithPhonetic}

\begin{EntryWithPhonetic}{管……叫……}{guan3 jiao4}{14,5}{⽵、⼝}
  \definition{expr.}{chamar alguém (ou algo) de alguém (ou algo)}
\end{EntryWithPhonetic}

\begin{EntryWithPhonetic}{管理}{guan3li3}{14,11}{⽵、⽟}[HSK 3]
  \definition{v.}{gerenciar; executar; administrar; governar; estar encarregado de; responsável por garantir o bom andamento de uma determinada tarefa | controlar; gerenciar; fazer com que pessoas e animais obedeçam ou se comportem de maneira ordeira | cuidar; zelar por; proteger; cuidar, organizar coisas}
\end{EntryWithPhonetic}

\begin{EntryWithPhonetic}{观}{guan4}{6}{⾒}
  \definition*{s.}{Sobrenome Guan}
  \definition{s.}{mosteiro taoísta | torre de vigia do portão do palácio | plataforma}
  \seeref{guan1}
\end{EntryWithPhonetic}

\begin{EntryWithPhonetic}{冠}{guan4}{9}{⼍}
  \definition*{s.}{Sobrenome Guan}
  \definition{s.}{primeiro lugar; o melhor; classificado em primeiro lugar}
  \definition{v.}{colocar um chapéu (boné) | preceder com (por); coroar com; adicionar um nome ou texto na frente}
  \seeref{guan1}
\end{EntryWithPhonetic}

\begin{EntryWithPhonetic}{冠军}{guan4jun1}{9,6}{⼍、⼍}[HSK 5]
  \definition[位,名,项,个]{s.}{campeão; medalhista de ouro; primeiro lugar em esportes e outras competições}
\end{EntryWithPhonetic}

\begin{EntryWithPhonetic}{光}{guang1}{6}{⼉}[HSK 3]
  \definition*{s.}{Sobrenome Guang}
  \definition{adj.}{suave; liso; brilhante | esgotado; sem nada sobrando | brilhante}
  \definition{adv.}{somente; sozinho; meramente}
  \definition{s.}{luz; raio | cenário; paisagem | honra; glória; brilho | claridade | favor; graça | momento | corpo celeste; referindo-se especificamente a corpos celestes, como o sol, a lua e as estrelas}
  \definition{v.}{glorificar; recuperar; reconquistar | estar nu; expor}
\end{EntryWithPhonetic}

\begin{EntryWithPhonetic}{光辉}{guang1hui1}{6,12}{⼉、⾞}[HSK 6]
  \definition{adj.}{brilhante; magnífico; glorioso}
  \definition{s.}{esplendor; brilho; glória | chama; brilho; halo; labareda; fulguração; lustre}
\end{EntryWithPhonetic}

\begin{EntryWithPhonetic}{光临}{guang1lin2}{6,9}{⼉、⼁}[HSK 4]
  \definition{v.}{honrar com sua presença, uma palavra de honra, usada para dizer que um convidado chegou}
\end{EntryWithPhonetic}

\begin{EntryWithPhonetic}{光明}{guang1ming2}{6,8}{⼉、⽇}[HSK 3]
  \definition{adj.}{brilhante; luminoso | sincero; ingênuo; metáfora da justiça e da esperança | justo; honesto; franco}
  \definition{s.}{luz}
\end{EntryWithPhonetic}

\begin{EntryWithPhonetic}{光盘}{guang1pan2}{6,11}{⼉、⽫}[HSK 4]
  \definition[张,套,片]{s.}{CD; disco compacto; um disco circular feito de plástico rígido composto que usa um laser para registrar e ler informações}
\end{EntryWithPhonetic}

\begin{EntryWithPhonetic}{光槃}{guang1pan2}{6,14}{⼉、⽊}
  \variantof{光盘}
\end{EntryWithPhonetic}

\begin{EntryWithPhonetic}{光荣}{guang1rong2}{6,9}{⼉、⾋}[HSK 5]
  \definition{adj.}{honroso; honrado; glorioso; por fazer algo que é benéfico para o país ou para a coletividade e que é considerado por todos como digno de respeito ou elogio}
  \definition{s.}{honra; glória; crédito; sentimento de honra decorrente do fato de ser respeitado ou elogiado por fazer algo importante ou grandioso}
\end{EntryWithPhonetic}

\begin{EntryWithPhonetic}{光污染}{guang1 wu1ran3}{6,6,9}{⼉、⽔、⽊}
  \definition{s.}{poluição luminosa}
\end{EntryWithPhonetic}

\begin{EntryWithPhonetic}{光线}{guang1 xian4}{6,8}{⼉、⽷}[HSK 5]
  \definition[条,道]{s.}{luz; feixe luminoso; raio de luz}
\end{EntryWithPhonetic}

\begin{EntryWithPhonetic}{广}{guang3}{3}{⼴}[HSK 5][Kangxi 53]
  \definition*{s.}{Sobrenome Guang}
  \definition{adj.}{largo; vasto; amplo; extenso (oposto a 狭) | numeroso | comum; universal}
  \definition{s.}{Guangdong, 广东, e Guangxi, 广州}
  \definition{v.}{expandir; espalhar; ampliar}
  \seeref{an1}
  \seeref{yan3}
  \seealsoref{广东}{guang3dong1}
  \seealsoref{广州}{guang3zhou1}
  \seealsoref{狭}{xia2}
\end{EntryWithPhonetic}

\begin{EntryWithPhonetic}{广播}{guang3bo1}{3,15}{⼴、⼿}[HSK 3]
  \definition[个,次,段,则,条]{s.}{programa de rádio; transmissão (de rádio); refere-se a programas transmitidos por estações de rádio ou televisão a cabo}
  \definition{v.}{transmitir; estar no ar | espalhar-se amplamente; ser conhecido em toda parte; divulgar amplamente}
\end{EntryWithPhonetic}

\begin{EntryWithPhonetic}{广场}{guang3chang3}{3,6}{⼴、⼟}[HSK 2]
  \definition{s.}{praça; praça pública; esplanada; área ampla, especificamente uma área ampla na cidade}
\end{EntryWithPhonetic}

\begin{EntryWithPhonetic}{广场舞}{guang3chang3wu3}{3,6,14}{⼴、⼟、⾇}
  \definition{s.}{quadrilha, uma rotina de exercícios tocada com música em quadrados públicos, parques e praças, popular especialmente entre mulheres de meia-idade e aposentados na China}
\end{EntryWithPhonetic}

\begin{EntryWithPhonetic}{广大}{guang3da4}{3,3}{⼴、⼤}[HSK 3]
  \definition{adj.}{muito difundido; enorme (alcance, escala) | (uma área ou espaço) vasto; extenso; em grande escala; amplo (área, espaço) | numeroso; muitos (número de pessoas)}
\end{EntryWithPhonetic}

\begin{EntryWithPhonetic}{广东}{guang3dong1}{3,5}{⼴、⼀}
  \definition*{s.}{Província de Guangdong}
  \seealsoref{粤}{yue4}
\end{EntryWithPhonetic}

\begin{EntryWithPhonetic}{广泛}{guang3fan4}{3,7}{⼴、⽔}[HSK 5]
  \definition{adj.}{amplo; extenso; de grande alcance; disseminado; escopo e cobertura amplos}
\end{EntryWithPhonetic}

\begin{EntryWithPhonetic}{广告}{guang3gao4}{3,7}{⼴、⼝}[HSK 2]
  \definition[则,条,段,项,个]{s.}{anúncio; propaganda; uma forma de divulgação ao público de produtos, serviços ou programas culturais e esportivos, geralmente realizada por meio de jornais, televisão, rádio, cartazes, etc.}
  \definition{v.}{anunciar; a ação ou ato de promover ou divulgar algo}
\end{EntryWithPhonetic}

\begin{EntryWithPhonetic}{广阔}{guang3kuo4}{3,12}{⼴、⾨}[HSK 6]
  \definition{adj.}{vasto; largo; amplo}
\end{EntryWithPhonetic}

\begin{EntryWithPhonetic}{广州}{guang3zhou1}{3,6}{⼴、⼮}
  \definition*{s.}{Guangzhou, antigamente Cantão; Capital da Província de Guangdong}
\end{EntryWithPhonetic}

\begin{EntryWithPhonetic}{逛}{guang4}{10}{⾡}[HSK 4]
  \definition{v.}{perambular; passear; vaguear}
\end{EntryWithPhonetic}

\begin{EntryWithPhonetic}{归}{gui1}{5}{⼹}[HSK 4]
  \definition*{s.}{Sobrenome Gui}
  \definition{s.}{divisão no ábaco com divisor de um dígito}
  \definition{v.}{retornar; voltar para; voltar (ou ir) | devolver algo a; dar de volta a | convergir; juntar-se | encarregar alguém de algo | atribuir a; pertencer a}
  \definition{v.aux.}{usado entre dois verbos idênticos, indicando que a ação não levou ao resultado correspondente}
\end{EntryWithPhonetic}

\begin{EntryWithPhonetic}{龟}{gui1}{7}{⿔}[Kangxi 213]
  \definition[只]{s.}{tartaruga; cágado}
\end{EntryWithPhonetic}

\begin{EntryWithPhonetic}{龟速}{gui1su4}{7,10}{⿔、⾡}
  \definition{adv.}{tão lento quanto uma tartaruga}
\end{EntryWithPhonetic}

\begin{EntryWithPhonetic}{规}{gui1}{8}{⾒}
  \definition*{s.}{Sobrenome Gui}
  \definition[个,种]{s.}{bússola | regulamentação; regra | (mecânica) medidor | compasso; ferramenta para desenhar círculos}
  \definition{v.}{admoestar; aconselhar; advertir | planejar; fazer planos}
\end{EntryWithPhonetic}

\begin{EntryWithPhonetic}{规定}{gui1ding4}{8,8}{⾒、⼧}[HSK 3]
  \definition[个,条,项,款]{s.}{regra; regulamento; estipulação; tomar decisões sobre a forma, o método, a quantidade ou a qualidade de algo}
  \definition{v.}{estipular; prover; prescrever; estabelecer requisitos ou restrições em termos de métodos, qualidade, quantidade, tempo, etc.}
\end{EntryWithPhonetic}

\begin{EntryWithPhonetic}{规范}{gui1fan4}{8,9}{⾒、⾋}[HSK 3]
  \definition{adj.}{regular; normal; padrão; que atende às especificações; em conformidade com as normas}
  \definition{s.}{norma; padrão; diretriz}
  \definition{v.}{regular; padronizar; tornar conforme as normas}
\end{EntryWithPhonetic}

\begin{EntryWithPhonetic}{规划}{gui1hua4}{8,6}{⾒、⼑}[HSK 5]
  \definition[个,项]{s.}{plano; projeto; planejamento; programa; programação; esquematização; plano de desenvolvimento de longo prazo mais abrangente}
  \definition{v.}{planejar; programar}
\end{EntryWithPhonetic}

\begin{EntryWithPhonetic}{规律}{gui1lv4}{8,9}{⾒、⼻}[HSK 4]
  \definition{adj.}{estável; regular; coisas, comportamentos, fenômenos, etc. que ocorrem em um determinado momento}
  \definition{s.}{lei; padrão regular; conexão essencial e recorrente entre as coisas}
\end{EntryWithPhonetic}

\begin{EntryWithPhonetic}{规模}{gui1mo2}{8,14}{⾒、⽊}[HSK 4]
  \definition[个,种]{s.}{escala; escopo; dimensões; padrão, forma ou escopo (de um empreendimento, instituição, projeto, movimento, etc.)}
\end{EntryWithPhonetic}

\begin{EntryWithPhonetic}{规则}{gui1ze2}{8,6}{⾒、⼑}[HSK 4]
  \definition{adj.}{ordenado; regular; descreve a forma, estrutura, arranjo, etc., que se conformam a uma determinada maneira organizada}
  \definition{s.}{regra; regulamento; sistema ou código de conduta prescrito para observância comum | lei; norma}
\end{EntryWithPhonetic}

\begin{EntryWithPhonetic}{轨}{gui3}{6}{⾞}
  \definition{s.}{trilho; pista | curso; caminho | ordem; regulamento; regra | rotina; metaforicamente falando, métodos, regras, ordem, etc.}
  \definition{v.}{seguir | Literário: cumprir; aderir a}
\end{EntryWithPhonetic}

\begin{EntryWithPhonetic}{轨道}{gui3dao4}{6,12}{⾞、⾡}[HSK 6]
  \definition[条]{s.}{trilha; uma rota pavimentada com trilhos de aço para trens, bondes, etc. | órbita; trajetória; corpos celestes e objetos têm trajetórias de movimento regulares | caminho; curso; maneira adequada de fazer as coisas; curso adequado; uma metáfora para o desenvolvimento normal das coisas ou as normas e procedimentos que as pessoas devem seguir}
\end{EntryWithPhonetic}

\begin{EntryWithPhonetic}{鬼}{gui3}{9}{⿁}[HSK 5][Kangxi 194]
  \definition*{s.}{Gui, uma das mansões lunares | Gui, a vigésima terceira das vinte e oito constelações em que a esfera celeste foi dividida, consistindo de quatro estrelas em Câncer | Sobrenome Gui}
  \definition{adj.}{evasivo; furtivo; sub-reptício; ardiloso; enganoso, malicioso; obscuro | terrível; ruim; severo; vil | esperto; astuto; inteligente}
  \definition{s.}{espírito; fantasma; aparição; refere-se à alma de uma pessoa após a morte | usado para formar um termo de abuso para caráter ignóbil; refere-se a pessoas que têm maus hábitos ou cujo comportamento é repugnante | companheiro; pessoa que é considerada divertida}
\end{EntryWithPhonetic}

\begin{EntryWithPhonetic}{鬼怪}{gui3guai4}{9,8}{⿁、⼼}
  \definition{s.}{\emph{hobgoblin} | bicho-papão | fantasma}
\end{EntryWithPhonetic}

\begin{EntryWithPhonetic}{鬼火}{gui3huo3}{9,4}{⿁、⽕}
  \definition{s.}{fogo-fátuo | boitatá | fogo corredor | fogo de santelmo}
\end{EntryWithPhonetic}

\begin{EntryWithPhonetic}{柜}{gui4}{8}{⽊}
  \definition{s.}{baú; armário; gabinete | loja; balcão}
  \seeref{ju3}
\end{EntryWithPhonetic}

\begin{EntryWithPhonetic}{柜子}{gui4 zi5}{8,3}{⽊、⼦}[HSK 5]
  \definition[个]{s.}{gabinete; armário; dispositivo para guardar roupas, documentos, livros, etc.}
\end{EntryWithPhonetic}

\begin{EntryWithPhonetic}{贵}{gui4}{9}{⾙}[HSK 1]
  \definition*{s.}{Província de Guizhou, abreviação de 贵州 | Sobrenome Gui}
  \definition{adj.}{caro; dispendioso (oposto de 贱) | altamente valorizado; valioso | de alta patente; nobre (oposto de 贱) | caro; preço ou valor elevado (em oposição a 贱) | digno de ser valorizado ou apreciado | nobre; honrado; posição social elevada}
  \definition{pron.}{honrado: Seu}
  \seealsoref{贵州}{gui4zhou1}
  \seealsoref{贱}{jian4}
\end{EntryWithPhonetic}

\begin{EntryWithPhonetic}{贵姓}{gui4xing4}{9,8}{⾙、⼥}
  \definition{expr.}{qual seu sobrenome?}
\end{EntryWithPhonetic}

\begin{EntryWithPhonetic}{贵州}{gui4zhou1}{9,6}{⾙、⼮}
  \definition*{s.}{Província de Guizhou}
\end{EntryWithPhonetic}

\begin{EntryWithPhonetic}{跪}{gui4}{13}{⾜}[HSK 6]
  \definition{v.}{ajoelhar-se; dobrar os joelhos de modo que um ou ambos os joelhos toquem o chão}
\end{EntryWithPhonetic}

\begin{EntryWithPhonetic}{跪拜}{gui4bai4}{13,9}{⾜、⼿}
  \definition{v.}{prostrar-se | ajoelhar-se e adorar}
\end{EntryWithPhonetic}

\begin{EntryWithPhonetic}{滚}{gun3}{13}{⽔}[HSK 5]
  \definition*{s.}{Sobrenome Gun}
  \definition{adj.}{rolante | fervente | precipitado; torrencial}
  \definition{adv.}{muito; em um grau elevado}
  \definition{v.}{rolar; girar; virar | escapar; fugir; ir embora | ferver | amarrar; aparar; fazer bainha}
\end{EntryWithPhonetic}

\begin{EntryWithPhonetic}{滚滚}{gun3 gun3}{13,13}{⽔、⽔}
  \definition{adj.}{ondulante | rolando continuamente}
  \definition{v.}{rolar; ondular; fluir}
\end{EntryWithPhonetic}

\begin{EntryWithPhonetic}{滚轮}{gun3lun2}{13,8}{⽔、⾞}
  \definition{s.}{pneu | dial rotativo | roda de rolagem (\emph{scroll})  (mouse de computador)}
\end{EntryWithPhonetic}

\begin{EntryWithPhonetic}{过}{guo1}{6}{⾡}
  \definition*{s.}{Sobrenome Guo}
  \seeref{guo4}
  \seeref{guo5}
\end{EntryWithPhonetic}

\begin{EntryWithPhonetic}{锅}{guo1}{12}{⾦}[HSK 5]
  \definition[口,个,只]{s.}{panela; frigideira; utensílios de cozinha, redondos e côncavos, feitos principalmente de ferro, alumínio, etc. | parte que se parece com um pote em alguns objetos}
\end{EntryWithPhonetic}

\begin{EntryWithPhonetic}{囯}{guo2}{7}{⼞}
  \definition*{s.}{Sobrenome Guo}
  \definition{adj.}{do estado; nacional | do nosso país; Chinês | do país}
  \definition{s.}{país; nação; estado | o melhor da nação | o melhor; o mais bonito do país}
  \variantof{国}
\end{EntryWithPhonetic}

\begin{EntryWithPhonetic}{国}{guo2}{8}{⼞}[HSK 1]
  \definition*{s.}{Sobrenome Guo}
  \definition{adj.}{nacional; do estado; representante do país | o melhor de um país}
  \definition[个]{s.}{estado; nação; país}
\end{EntryWithPhonetic}

\begin{EntryWithPhonetic}{国宾馆}{guo2bin1guan3}{8,10,11}{⼞、⼧、⾷}
  \definition{s.}{pousada estadual}
\end{EntryWithPhonetic}

\begin{EntryWithPhonetic}{国产}{guo2 chan3}{8,6}{⼞、⼇}[HSK 6]
  \definition{adj.}{doméstico; feito na China; produzido internamente, especificamente na China}
\end{EntryWithPhonetic}

\begin{EntryWithPhonetic}{国歌}{guo2 ge1}{8,14}{⼞、⽋}[HSK 6]
  \definition[首,支]{s.}{hino nacional; o hino nacional da China, oficialmente designado pelo estado como a música que representa o país, é "Marcha dos Voluntários"}
\end{EntryWithPhonetic}

\begin{EntryWithPhonetic}{国会}{guo2 hui4}{8,6}{⼞、⼈}[HSK 6]
  \definition{s.}{parlamento; congresso}
\end{EntryWithPhonetic}

\begin{EntryWithPhonetic}{国籍}{guo2ji2}{8,20}{⼞、⽵}[HSK 5]
  \definition[个]{s.}{nacionalidade; cidadania; refere-se à identidade de um indivíduo como pertencente a um Estado | identidade nacional (de um avião, navio, etc.)}
\end{EntryWithPhonetic}

\begin{EntryWithPhonetic}{国际}{guo2ji4}{8,7}{⼞、⾩}[HSK 2]
  \definition{adj.}{internacional; entre países; entre nações}
  \definition{s.}{internacional; o mundo; entre nações; entre países de todo o mundo}
\end{EntryWithPhonetic}

\begin{EntryWithPhonetic}{国际儿童节}{guo2ji4 er2tong2jie2}{8,7,2,12,5}{⼞、⾩、⼉、⽴、⾋}
  \definition*{s.}{Dia Internacional das Crianças (1~de~junho)}
\end{EntryWithPhonetic}

\begin{EntryWithPhonetic}{国际妇女节}{guo2ji4 fu4nv3jie2}{8,7,6,3,5}{⼞、⾩、⼥、⼥、⾋}
  \definition*{s.}{Dia Internacional das Mulheres (8~de~março)}
\end{EntryWithPhonetic}

\begin{EntryWithPhonetic}{国际劳动节}{guo2ji4 lao2dong4 jie2}{8,7,7,6,5}{⼞、⾩、⼒、⼒、⾋}
  \definition*{s.}{Dia Internacional dos Trabalhadores (1~de~maio)}
\end{EntryWithPhonetic}

\begin{EntryWithPhonetic}{国家}{guo2jia1}{8,10}{⼞、⼧}[HSK 1]
  \definition[个]{s.}{país; estado; nação; um lugar reconhecido internacionalmente e com soberania independente, incluindo as pessoas e as instituições administrativas desse lugar}
\end{EntryWithPhonetic}

\begin{EntryWithPhonetic}{国民}{guo2 min2}{8,5}{⼞、⽒}[HSK 5]
  \definition{adj.}{nacional}
  \definition[个]{s.}{membro de uma nação; povo de uma nação}
\end{EntryWithPhonetic}

\begin{EntryWithPhonetic}{国内}{guo2 nei4}{8,4}{⼞、⼌}[HSK 3]
  \definition{s.}{interno (a um país); doméstico; lar; dentro de um determinado país}
\end{EntryWithPhonetic}

\begin{EntryWithPhonetic}{国旗}{guo2qi2}{8,14}{⼞、⽅}
  \definition[面]{s.}{bandeira (de um país)}
\end{EntryWithPhonetic}

\begin{EntryWithPhonetic}{国庆}{guo2 qing4}{8,6}{⼞、⼴}[HSK 3]
  \definition*{s.}{Dia Nacional, o dia em que um país comemora sua independência ou fundação}
\end{EntryWithPhonetic}

\begin{EntryWithPhonetic}{国庆节}{guo2qing4jie2}{8,6,5}{⼞、⼴、⾋}
  \definition*{s.}{Dia Nacional (1~de~outubro)}
\end{EntryWithPhonetic}

\begin{EntryWithPhonetic}{国人}{guo2ren2}{8,2}{⼞、⼈}
  \definition{s.}{compatriota}
\end{EntryWithPhonetic}

\begin{EntryWithPhonetic}{国外}{guo2 wai4}{8,5}{⼞、⼣}[HSK 1]
  \definition{adj.}{externo; no exterior; fora do país; outros lugares fora do país; geralmente chamados de exterior;  exterior não é o mesmo que estrangeiro}
\end{EntryWithPhonetic}

\begin{EntryWithPhonetic}{国王}{guo2wang2}{8,4}{⼞、⽟}[HSK 6]
  \definition[位,名,个,些]{s.}{rei; soberanos; o governante supremo de algumas monarquias antigas; nos tempos modernos, refere-se ao chefe de estado de algumas monarquias}
\end{EntryWithPhonetic}

\begin{EntryWithPhonetic}{国语}{guo2yu3}{8,9}{⼞、⾔}
  \definition*{s.}{Língua Chinesa (Mandarim), enfatizando sua natureza nacional}
\end{EntryWithPhonetic}

\begin{EntryWithPhonetic}{果}{guo3}{8}{⽊}
  \definition*{s.}{Sobrenome Guo}
  \definition{adj.}{resoluto; determinado; sem exitação}
  \definition{adv.}{realmente; como esperado; com certeza; isso significa que as coisas são consistentes com as expectativas, equivalente a 果然}
  \definition{conj.}{se realmente; se de fato}
  \definition[个,些,种]{s.}{fruta; fruto da planta | resultado; consequência; o resultado final de um assunto (em oposição à 因)}
  \seealsoref{果然}{guo3ran2}
  \seealsoref{因}{yin1}
\end{EntryWithPhonetic}

\begin{EntryWithPhonetic}{果酱}{guo3 jiang4}{8,13}{⽊、⾣}[HSK 6]
  \definition{s.}{geléia | compota ou doce (de frutas); fruta em conserva}
\end{EntryWithPhonetic}

\begin{EntryWithPhonetic}{果然}{guo3ran2}{8,12}{⽊、⽕}[HSK 3]
  \definition{adv.}{realmente; como esperado; com certeza; indica que os fatos correspondem ao que foi dito ou esperado}
  \definition{conj.}{se realmente; se de fato; suponha que os fatos correspondam ao que foi dito ou esperado}
\end{EntryWithPhonetic}

\begin{EntryWithPhonetic}{果实}{guo3shi2}{8,8}{⽊、⼧}[HSK 4]
  \definition[种]{s.}{fruta; o órgão que se desenvolve a partir do ovário ou com outras partes da flor após a fertilização da flor | ganhos; frutos;  uma metáfora para conquista ou recompensa por trabalho árduo}
\end{EntryWithPhonetic}

\begin{EntryWithPhonetic}{果树}{guo3 shu4}{8,9}{⽊、⽊}[HSK 6]
  \definition[棵,个,片]{s.}{árvore frutífera; árvores cujos frutos são principalmente comestíveis, como pessegueiros e macieiras}
\end{EntryWithPhonetic}

\begin{EntryWithPhonetic}{果汁}{guo3zhi1}{8,5}{⽊、⽔}[HSK 3]
  \definition[杯,瓶,种]{s.}{suco; suco de frutas frescas; também se refere a bebidas feitas com suco de frutas frescas}
\end{EntryWithPhonetic}

\begin{EntryWithPhonetic}{果子}{guo3zi5}{8,3}{⽊、⼦}
  \definition{s.}{fruta}
\end{EntryWithPhonetic}

\begin{EntryWithPhonetic}{过}{guo4}{6}{⾡}[HSK 1,2]
  \definition{adv.}{excessivamente; em excesso}
  \definition{clas.}{tempo; número de vezes usado para a ação}
  \definition{s.}{falha; erro; demérito; equívoco; negligência; (oposto a 功)}
  \definition{v.}{cruzar; passar; mudar-se de um lugar para outro; passar por | exceder; ir além; ultrapassar; usado após um adjetivo, significa ``mais do que'' | gastar (tempo); passar (tempo); exceder (um determinado limite ou limite) | celebrar; comemorar | mudar; transferir; transferir de um lado para o outro | passar por um processo; passar por; submeter a (algum tipo de tratamento) | visitar; fazer uma visita | falecer; morrer | infectar; ser contagioso; espalhar | exceder; ir além; usado após o verbo com o sufixo 得, significa ``superar'' ou ``passar'' | viver | revisar; examinar; usar os olhos para ver ou a mente para lembrar}
  \seeref{guo1}
  \seeref{guo5}
  \seealsoref{得}{de5}
  \seealsoref{功}{gong1}
\end{EntryWithPhonetic}

\begin{EntryWithPhonetic}{过不惯}{guo4 bu5 guan4}{6,4,11}{⾡、⼀、⼼}
  \definition{v.}{não se acostumar; não se habituar}
  \seealsoref{过惯}{guo4guan4}
\end{EntryWithPhonetic}

\begin{EntryWithPhonetic}{过程}{guo4cheng2}{6,12}{⾡、⽲}[HSK 3]
  \definition[个,段]{s.}{curso dos eventos; processo; o processo pelo qual as coisas acontecem ou se desenvolvem.}
\end{EntryWithPhonetic}

\begin{EntryWithPhonetic}{过度}{guo4du4}{6,9}{⾡、⼴}[HSK 5]
  \definition{adj.}{excessivo; acima do limite; além do limite; além do que é apropriado}
\end{EntryWithPhonetic}

\begin{EntryWithPhonetic}{过渡}{guo4du4}{6,12}{⾡、⽔}[HSK 6]
  \definition{v.}{fazer a transição; estar em transição; estar em fase de transição; mudar de um estágio para outro | atravessar; cruzar}
\end{EntryWithPhonetic}

\begin{EntryWithPhonetic}{过分}{guo4fen4}{6,4}{⾡、⼑}[HSK 4]
  \definition{adj.}{excessivo; muito longe; demais; falar ou agir além dos limites ou graus adequados}
  \definition{adv.}{excessivamente; indevidamente; muito mesmo}
\end{EntryWithPhonetic}

\begin{EntryWithPhonetic}{过关}{guo4/guan1}{6,6}{⾡、⼋}
  \definition{v.+compl.}{passar uma barreira | passar por uma provação | passar em um teste | atingir um padrão | passar pela alfândega}
\end{EntryWithPhonetic}

\begin{EntryWithPhonetic}{过惯}{guo4guan4}{6,11}{⾡、⼼}
  \definition{v.}{estar acostumado (a um certo estilo de vida, etc.)}
  \seealsoref{过不惯}{guo4 bu5 guan4}
\end{EntryWithPhonetic}

\begin{EntryWithPhonetic}{过后}{guo4 hou4}{6,6}{⾡、⼝}[HSK 6]
  \definition[期]{s.}{depois; mais tarde}
\end{EntryWithPhonetic}

\begin{EntryWithPhonetic}{过节}{guo4/jie2}{6,5}{⾡、⾋}
  \definition{v.+compl.}{celebrar festividades | comemorar um festival}
\end{EntryWithPhonetic}

\begin{EntryWithPhonetic}{过来}{guo4 lai2}{6,7}{⾡、⽊}[HSK 2]
  \definition{v.}{vir até aqui | ser capaz de cuidar de | lidar com | administrar}
\end{EntryWithPhonetic}

\begin{EntryWithPhonetic}{过敏}{guo4min3}{6,11}{⾡、⽁}[HSK 5]
  \definition{adj.}{sensível; excessivamente sensível; resposta acima do normal; ceticismo excessivo}
  \definition{v.}{ser alérgico a}
\end{EntryWithPhonetic}

\begin{EntryWithPhonetic}{过年}{guo4/nian2}{6,6}{⾡、⼲}[HSK 2]
  \definition{v.+compl.}{comemorar o Ano Novo; comemorar o Festival da Primavera; passar o Ano Novo; passar o Festival da Primavera; realizar atividades comemorativas durante o Ano Novo ou o Festival da Primavera}
\end{EntryWithPhonetic}

\begin{EntryWithPhonetic}{过期}{guo4/qi1}{6,12}{⾡、⽉}
  \definition{v.+compl.}{exceder a data | passar a data | expirar (passar a data de expiração)}
\end{EntryWithPhonetic}

\begin{EntryWithPhonetic}{过去}{guo4 qu4}{6,5}{⾡、⼛}[HSK 2,3]
  \definition{adv.}{(no) passado}
  \definition{s.}{o passado; refere-se a um período anterior; também se refere a coisas anteriores}
  \definition{v.}{atravessar; passar; sair do local onde o interlocutor se encontra e deslocar-se para outro local | acabar; passar; ficar para trás; indica que já passou por uma determinada fase | passar; indica que um determinado período ou situação já não existe mais | falecer | ir lá | passar por}
\end{EntryWithPhonetic}

\begin{EntryWithPhonetic}{过时}{guo4 shi2}{6,7}{⾡、⽇}[HSK 6]
  \definition{adj.}{fora de moda; obsoleto; antiquado; desatualizado; o que era popular no passado não é mais popular}
  \definition{v.}{passar do tempo marcado; estar após o tempo estipulado}
\end{EntryWithPhonetic}

\begin{EntryWithPhonetic}{过瘾}{guo4/yin3}{6,16}{⾡、⽧}
  \definition{adj.}{gratificante | imensamente agradável | satisfatório}
  \definition{v.+compl.}{satisfazer um desejo | se divertir com algo}
\end{EntryWithPhonetic}

\begin{EntryWithPhonetic}{过于}{guo4yu2}{6,3}{⾡、⼆}[HSK 5]
  \definition{adv.}{demais; indevidamente; excessivamente; advérbios de grau ou quantidade excessiva}
\end{EntryWithPhonetic}

\begin{EntryWithPhonetic}{过}{guo5}{6}{⾡}
  \definition{part.}{usado depois de um verbo para indicar conclusão | usado depois de um verbo para indicar que uma ação ou mudança ocorreu | usado depois de um adjetivo para indicar que algo já teve uma certa qualidade ou estado e para compará-lo com o presente}
  \seeref{guo1}
  \seeref{guo4}
\end{EntryWithPhonetic}

%%%%% EOF %%%%%


 %%%
%%% H
%%%

\section*{H}\addcontentsline{toc}{section}{H}

\begin{EntryWithPhonetic}{哈}{ha1}{9}{⼝}
  \definition{interj.}{Onomatopéia: ha; descreve o riso, usado principalmente em duplicata | indica orgulho ou satisfação, frequentemente usado de forma duplicada}
  \definition{v.}{soprar; expirar (com a boca aberta) | dobrar}
  \seeref{ha3}
  \seealsoref{哈哈}{ha1 ha1}
\end{EntryWithPhonetic}

\begin{EntryWithPhonetic}{哈哈}{ha1 ha1}{9,9}{⼝、⼝}[HSK 3]
  \definition{expr.}{(onomatopéia)  ha ha; o som de uma gargalhada}
\end{EntryWithPhonetic}

\begin{EntryWithPhonetic}{哈马斯}{ha1ma3si1}{9,3,12}{⼝、⾺、⽄}
  \definition*{s.}{Hamas (Grupo Palestino)}
\end{EntryWithPhonetic}

\begin{EntryWithPhonetic}{哈}{ha3}{9}{⼝}
  \definition*{s.}{Sobrenome Ha}
  \definition{v.}{repreender}
  \seeref{ha1}
\end{EntryWithPhonetic}

\begin{EntryWithPhonetic}{咳}{hai1}{9}{⼝}
  \definition{interj.}{expressa tristeza, arrependimento ou espanto}
  \seeref{ke2}
\end{EntryWithPhonetic}

\begin{EntryWithPhonetic}{还}{hai2}{7}{⾡}[HSK 1]
  \definition{adv.}{ainda; indica que a ação ou estado permanece inalterado, equivalente a 仍然 | também; além disso; em adição; indica que há um aumento ou suplemento além do escopo já indicado | ainda mais; usado com 比 para indicar que as características e o grau das coisas comparadas aumentaram, o que é equivalente a 更加 razoavelmente; medianamente; usado antes de um adjetivo, indica que algo atinge apenas o nível mínimo exigido | mesmo; usado na primeira parte da frase como complemento, e na segunda parte como conclusão, equivalente a 尚且 | que expressa realização ou descoberta; expressa surpresa por algo que não se esperava, mas que acabou acontecendo | tão cedo quanto; por um curto período de tempo; indica que já era assim há muito tempo | para dar ênfase; para reforçar o tom}
  \seeref{huan2}
  \seealsoref{比}{bi3}
  \seealsoref{更加}{geng4 jia1}
  \seealsoref{仍然}{reng2ran2}
  \seealsoref{尚且}{shang4 qie3}
\end{EntryWithPhonetic}

\begin{EntryWithPhonetic}{还是}{hai2shi5}{7,9}{⾡、⽇}[HSK 1]
  \definition{adv.}{ainda; ainda assim; não é a continuação de um determinado estado, fenômeno ou ação; o resultado é o mesmo de antes, sem mudanças  |que expressa uma preferência por uma alternativa; expressa comparação ou escolha feita após consideração cuidadosa, frequentemente usado para fazer sugestões | que expressa realização ou descoberta; indica que o resultado final foi inesperado}
  \definition{conj.}{ou (somente para frases interrogativas); indica várias opções, geralmente usado em perguntas | tudo; se; não importa; independentemente de; significa que, independentemente das mudanças que ocorram, o resultado permanecerá o mesmo}
\end{EntryWithPhonetic}

\begin{EntryWithPhonetic}{还有}{hai2 you3}{7,6}{⾡、⽉}[HSK 1]
  \definition{adv.}{também; ainda; além disso; então novamente; enfatizar as partes complementares, excedentes ou não mencionadas além do que já é conhecido}
\end{EntryWithPhonetic}

\begin{EntryWithPhonetic}{孩}{hai2}{9}{⼦}
  \definition[个]{s.}{criança}
\end{EntryWithPhonetic}

\begin{EntryWithPhonetic}{孩子}{hai2 zi5}{9,3}{⼦、⼦}[HSK 1]
  \definition[个]{s.}{criança; crianças; pessoas com idade entre alguns anos ou na adolescência, geralmente com menos de 14 anos | crianças; filho ou filha}
\end{EntryWithPhonetic}

\begin{EntryWithPhonetic}{海}{hai3}{10}{⽔}[HSK 2]
  \definition*{s.}{Sobrenome Hai}
  \definition{adj.}{extragrande; de grande capacidade; descreve capacidade, tom de voz, etc.}
  \definition{adv.}{aleatoriamente; sem rumo; sem limites; sem restrições}
  \definition[片]{s.}{mar; grande lago; a parte do oceano próxima à costa, alguns grandes lagos também são chamados de mar | grande número de pessoas ou coisas reunidas; metáfora para muitas coisas semelhantes que formam um grande conjunto}
\end{EntryWithPhonetic}

\begin{EntryWithPhonetic}{海报}{hai3 bao4}{10,7}{⽔、⼿}[HSK 6]
  \definition[张,份,幅]{s.}{pôster; cartaz; cartazes anunciando apresentações culturais, exibições de filmes ou competições esportivas, etc.}
\end{EntryWithPhonetic}

\begin{EntryWithPhonetic}{海边}{hai3 bian1}{10,5}{⽔、⾡}[HSK 2]
  \definition{s.}{praia; costa; litoral; orla marítima; a parte marginal do oceano e as grandes áreas de água salgada cercadas por terra firme, onde a terra e a água se encontram, formam a costa}
\end{EntryWithPhonetic}

\begin{EntryWithPhonetic}{海底}{hai3 di3}{10,8}{⽔、⼴}[HSK 6]
  \definition{s.}{fundo do mar; fundo do oceano; solo oceânico}
\end{EntryWithPhonetic}

\begin{EntryWithPhonetic}{海风}{hai3feng1}{10,4}{⽔、⾵}
  \definition{s.}{brisa do mar | vento que vem do mar}
\end{EntryWithPhonetic}

\begin{EntryWithPhonetic}{海关}{hai3guan1}{10,6}{⽔、⼋}[HSK 3]
  \definition[个]{s.}{alfândega; órgão administrativo nacional, sua principal função é supervisionar e inspecionar os bens e meios de transporte que entram e saem do país, cobrar impostos alfandegários e reprimir o contrabando}
\end{EntryWithPhonetic}

\begin{EntryWithPhonetic}{海军}{hai3 jun1}{10,6}{⽔、⼍}[HSK 6]
  \definition[支,名,位,个]{s.}{marinha; o exército que luta no mar geralmente é composto por navios de superfície, submarinos, aviação naval, fuzileiros navais e outros ramos, além de diversas forças profissionais}
\end{EntryWithPhonetic}

\begin{EntryWithPhonetic}{海浪}{hai3 lang4}{10,10}{⽔、⽔}[HSK 6]
  \definition{s.}{ondas do mar}
\end{EntryWithPhonetic}

\begin{EntryWithPhonetic}{海里}{hai3li3}{10,7}{⽔、⾥}
  \definition{s.}{milha náutica}
\end{EntryWithPhonetic}

\begin{EntryWithPhonetic}{海绵}{hai3mian2}{10,11}{⽔、⽷}
  \definition{s.}{(zoologia) esponja do mar | esponja (feita de poliéster ou celulose, etc.) | espuma de borracha}
\end{EntryWithPhonetic}

\begin{EntryWithPhonetic}{海鸥}{hai3'ou1}{10,9}{⽔、⿃}
  \definition{s.}{gaivota}
\end{EntryWithPhonetic}

\begin{EntryWithPhonetic}{海水}{hai3 shui3}{10,4}{⽔、⽔}[HSK 4]
  \definition[把]{s.}{água do mar; salmoura}
\end{EntryWithPhonetic}

\begin{EntryWithPhonetic}{海棠}{hai3tang2}{10,12}{⽔、⽊}
  \definition{s.}{begônia}
\end{EntryWithPhonetic}

\begin{EntryWithPhonetic}{海外}{hai3 wai4}{10,5}{⽔、⼣}[HSK 6]
  \definition[次]{s.}{fora das fronteiras nacionais; no exterior}
\end{EntryWithPhonetic}

\begin{EntryWithPhonetic}{海湾}{hai3 wan1}{10,12}{⽔、⽔}[HSK 6]
  \definition{s.}{baía; golfo | lago}
\end{EntryWithPhonetic}

\begin{EntryWithPhonetic}{海鲜}{hai3xian1}{10,14}{⽔、⿂}[HSK 4]
  \definition[种,份,桌,批,些]{s.}{frutos do mar; mariscos; peixes marinhos frescos, camarões, etc., para consumo |}
\end{EntryWithPhonetic}

\begin{EntryWithPhonetic}{海洋}{hai3yang2}{10,9}{⽔、⽔}[HSK 6]
  \definition[片,个]{s.}{mar; oceano; um termo geral para os mares e oceanos que formam uma entidade contínua na superfície da Terra; também pode ser usado para descrever um grande número de coisas semelhantes}
\end{EntryWithPhonetic}

\begin{EntryWithPhonetic}{害}{hai4}{10}{⼧}[HSK 5]
  \definition{adj.}{prejudicial; destrutivo; injurioso; nocivo}
  \definition{s.}{mal; maldade; dano; calamidade}
  \definition{v.}{prejudicar; fazer mal a; causar problemas a | matar; assassinar | sofrer de; contrair (uma doença) | sentir-se (envergonhado, com medo, etc.); despertar (um sentimento ou uma emoção)}
\end{EntryWithPhonetic}

\begin{EntryWithPhonetic}{害怕}{hai4pa4}{10,8}{⼧、⼼}[HSK 3]
  \definition{v.}{estar assustado; ter medo; encontrar dificuldades, perigos, etc., e sentir-se inquieto ou nervoso}
\end{EntryWithPhonetic}

\begin{EntryWithPhonetic}{害羞}{hai4xiu1}{10,10}{⼧、⽺}
  \definition{adj.}{tímido | envergonhado}
\end{EntryWithPhonetic}

\begin{EntryWithPhonetic}{汗}{han2}{6}{⽔}
  \definition*{s.}{Abreviação de Khan}[他是成吉思汗。===Ele é Genghis Khan.]
  \seeref{han4}
\end{EntryWithPhonetic}

\begin{EntryWithPhonetic}{含}{han2}{7}{⼝}[HSK 4]
  \definition{v.}{manter na boca (sem engolir ou cuspir) | conter; incluir | cuidar; acalentar; abrigar}
\end{EntryWithPhonetic}

\begin{EntryWithPhonetic}{含金量}{han2jin1liang4}{7,8,12}{⼝、⾦、⾥}
  \definition{adj.}{conteúdo de ouro | (fig.) valioso}
\end{EntryWithPhonetic}

\begin{EntryWithPhonetic}{含量}{han2 liang4}{7,12}{⼝、⾥}[HSK 4]
  \definition{s.}{conteúdo; a quantidade de um componente contido em uma substância}
\end{EntryWithPhonetic}

\begin{EntryWithPhonetic}{含义}{han2yi4}{7,3}{⼝、⼂}[HSK 4]
  \definition[个,种,层]{s.}{sentido; mensagem; significado; implicação; o significado contido em (palavras, frases, sentenças e discursos)}
\end{EntryWithPhonetic}

\begin{EntryWithPhonetic}{含有}{han2 you3}{7,6}{⼝、⽉}[HSK 4]
  \definition{v.}{conter; ter; incluir}
\end{EntryWithPhonetic}

\begin{EntryWithPhonetic}{函}{han2}{8}{⼐}
  \definition*{s.}{Sobrenome Han}
  \definition[封]{s.}{caixa; envelope; capa | carta}
\end{EntryWithPhonetic}

\begin{EntryWithPhonetic}{函数}{han2shu4}{8,13}{⼐、⽁}
  \definition{s.}{função (matemática)}
\end{EntryWithPhonetic}

\begin{EntryWithPhonetic}{涵}{han2}{11}{⽔}
  \definition{s.}{bueiro; galeria}
  \definition{v.}{conter; incorporar}
\end{EntryWithPhonetic}

\begin{EntryWithPhonetic}{寒}{han2}{12}{⼧}
  \definition*{s.}{Sobrenome Han}
  \definition{adj.}{frio | pobre; necessitado | (autodepreciativo) meu/minha humilde\dots | assustado; medroso | com medo; tremendo (de medo) | humilde}
  \definition{s.}{estação fria; inverno (oposto a 暑) | (medicina chinesa) sintomas causados por fatores frios}
  \seealsoref{暑}{shu3}
\end{EntryWithPhonetic}

\begin{EntryWithPhonetic}{寒假}{han2jia4}{12,11}{⼧、⼈}[HSK 4]
  \definition[个,段]{s.}{férias de inverno (feriados); férias escolares no meio do inverno, em janeiro e fevereiro (na China)}
\end{EntryWithPhonetic}

\begin{EntryWithPhonetic}{寒冷}{han2 leng3}{12,7}{⼧、⼎}[HSK 4]
  \definition[度,阵,股]{adj.}{frio; frígido; gélido; gelado}
\end{EntryWithPhonetic}

\begin{EntryWithPhonetic}{韩}{han2}{12}{⾱}
  \definition*{s.}{Um estado durante o Período dos Estados Combatentes nas atuais províncias centrais de Henan e sudeste de Shanxi | O nome de um estado feudal durante a dinastia Zhou, localizado no que hoje é o nordeste de Hejin, província de Shanxi | Coreia do Sul, abreviação de 韩国; República da Coreia (RC) | Sobrenome Han}
  \seealsoref{韩国}{han2guo2}
\end{EntryWithPhonetic}

\begin{EntryWithPhonetic}{韩国}{han2guo2}{12,8}{⾱、⼞}
  \definition*{s.}{Coréia do Sul; República da Coreia}
\end{EntryWithPhonetic}

\begin{EntryWithPhonetic}{韩国人}{han2guo2ren2}{12,8,2}{⾱、⼞、⼈}
  \definition{s.}{coreano | pessoa ou povo da Coréia}
\end{EntryWithPhonetic}

\begin{EntryWithPhonetic}{厂}{han3}{2}{⼚}[Kangxi 27]
  \definition[家,间]{s.}{radical ``penhasco'' em caracteres chineses (radical Kangxi 27)}
  \seeref{an1}
  \seeref{chang3}
\end{EntryWithPhonetic}

\begin{EntryWithPhonetic}{喊}{han3}{12}{⼝}[HSK 2]
  \definition{v.}{gritar; clamar; berrar | chamar (uma pessoa) | chamar; dirigir-se a}
\end{EntryWithPhonetic}

\begin{EntryWithPhonetic}{汉}{han4}{5}{⽔}
  \definition*{s.}{Dinastia Han (206 a.C.-220 d.C.)  | Astronomia: A Via Láctea | Sobrenome Han}
  \definition{s.}{grupo étnico Han | chinês (língua) | homem}
\end{EntryWithPhonetic}

\begin{EntryWithPhonetic}{汉堡包}{han4bao3bao1}{5,12,5}{⽔、⼟、⼓}
  \definition[个]{s.}{hambúrguer}
\end{EntryWithPhonetic}

\begin{EntryWithPhonetic}{汉堡王}{han4bao3wang2}{5,12,4}{⽔、⼟、⽟}
  \definition*{s.}{Burguer King, restaurante de \emph{fast-food}}
\end{EntryWithPhonetic}

\begin{EntryWithPhonetic}{汉服}{han4fu2}{5,8}{⽔、⽉}
  \definition{s.}{vestido chinês tradicional Han}
\end{EntryWithPhonetic}

\begin{EntryWithPhonetic}{汉葡词典}{han4-pu2 ci2dian3}{5,12,7,8}{⽔、⾋、⾔、⼋}
  \definition[部,本]{s.}{dicionário chinês-português}
  \seealsoref{葡汉词典}{pu2-han4 ci2dian3}
\end{EntryWithPhonetic}

\begin{EntryWithPhonetic}{汉语}{han4yu3}{5,9}{⽔、⾔}[HSK 1]
  \definition[门]{s.}{língua chinesa, mandarim}
\end{EntryWithPhonetic}

\begin{EntryWithPhonetic}{汉字}{han4 zi4}{5,6}{⽔、⼦}[HSK 1]
  \definition[个]{s.}{caractere chinês; ideograma chinês; sinograma; com pouquíssimas exceções, os caracteres chineses representam uma sílaba cada um}
\end{EntryWithPhonetic}

\begin{EntryWithPhonetic}{汗}{han4}{6}{⽔}[HSK 5]
  \definition{s.}{suor; transpiração; perspiração}
  \seeref{han2}
\end{EntryWithPhonetic}

\begin{EntryWithPhonetic}{汗水}{han4shui3}{6,4}{⽔、⽔}
  \definition*{s.}{Rio Han (Hanshui)}
  \definition{s.}{suor | transpiração}
\end{EntryWithPhonetic}

\begin{EntryWithPhonetic}{汗腺}{han4xian4}{6,13}{⽔、⾁}
  \definition{s.}{glândula sudorípara}
\end{EntryWithPhonetic}

\begin{EntryWithPhonetic}{汗液}{han4ye4}{6,11}{⽔、⽔}
  \definition{s.}{suor}
\end{EntryWithPhonetic}

\begin{EntryWithPhonetic}{焊}{han4}{11}{⽕}
  \definition{v.}{soldar; usar metal fundido para reparar objetos de metal ou conectar peças de metal}
\end{EntryWithPhonetic}

\begin{EntryWithPhonetic}{撼}{han4}{16}{⼿}
  \definition{v.}{agitar; sacudir}
\end{EntryWithPhonetic}

\begin{EntryWithPhonetic}{行}{hang2}{6}{⾏}[HSK 3][Kangxi 144]
  \definition{adj.}{temporário; improvisado | capaz; competente}
  \definition{adv.}{logo; em breve}
  \definition{clas.}{linha; fileira; coisas usadas para formar filas, linhas}
  \definition{s.}{comportamento; conduta | linha; fileira | empresa comercial; certas instituições comerciais | comércio; profissão; ramo de atividade | especialista; conhecedor; refere-se ao conhecimento e experiência em um determinado setor}
  \definition{v.}{ir; caminhar; viajar | estar atualizado; circular | fazer; executar; realizar | (antes de um verbo dissílabo, indicando a realização de alguma ação) | ficar bem; vai dar certo | (remédio) fazer efeito | classificar (entre irmãos e irmãs por ordem de idade)}
  \seeref{heng2}
  \seeref{xing2}
\end{EntryWithPhonetic}

\begin{EntryWithPhonetic}{行业}{hang2ye4}{6,5}{⾏、⼀}[HSK 4]
  \definition[种,个]{s.}{comércio; indústria; setor; profissão; categorias em negócios e indústria referem-se a ocupações em geral}
\end{EntryWithPhonetic}

\begin{EntryWithPhonetic}{航}{hang2}{10}{⾈}
  \definition*{s.}{Sobrenome Hang}
  \definition[趟]{s.}{barco; navio}
  \definition{v.}{navegar (por água ou ar) | velejar}
\end{EntryWithPhonetic}

\begin{EntryWithPhonetic}{航班}{hang2ban1}{10,10}{⾈、⽟}[HSK 4]
  \definition[个,次]{s.}{número do voo; voo programado; o horário de um navio ou avião de passageiros}
\end{EntryWithPhonetic}

\begin{EntryWithPhonetic}{航空}{hang2kong1}{10,8}{⾈、⽳}[HSK 4]
  \definition{s.}{viagem; aviação; refere-se ao voo de uma aeronave no ar}
\end{EntryWithPhonetic}

\begin{EntryWithPhonetic}{航天员}{hang2tian1yuan2}{10,4,7}{⾈、⼤、⼝}
  \definition{s.}{astronauta}
\end{EntryWithPhonetic}

\begin{EntryWithPhonetic}{号}{hao2}{5}{⼝}
  \definition{v.}{uivar; gritar; gritar em voz alta e prolongada | lamentar; chorar alto | uivar; (vento) assobiar, assoviar}
  \seeref{hao4}
\end{EntryWithPhonetic}

\begin{EntryWithPhonetic}{蚝}{hao2}{10}{⾍}
  \definition[只]{s.}{ostra}
\end{EntryWithPhonetic}

\begin{EntryWithPhonetic}{毫}{hao2}{11}{⽊}
  \definition{adv.}{nem um pouco; absolutamente nenhum; completamente sem}
  \definition{clas.}{hao, uma unidade de comprimento igual a um milésimo de polegada ou 1/30 de milímetro | hao, uma unidade de peso igual a um milésimo de um centavo ou 0,005 grama |
uma fração minúscula; uma parte muito pequena}
  \definition{pref.}{mili-, usado com a unidade de uma quantidade física para representar um milésimo dessa quantidade}
  \definition{s.}{cabelo longo e fino | pincel de escrita | uma das duas ou três alças de uma balança para pendurar na mão do usuário | cerda; uma corda de mão em uma balança ou equilíbrio | fio de cabelo}
\end{EntryWithPhonetic}

\begin{EntryWithPhonetic}{毫不费力}{hao2bu2fei4li4}{11,4,9,2}{⽊、⼀、⾙、⼒}
  \definition{adj.}{sem esforço | não gastando o menor esforço}
\end{EntryWithPhonetic}

\begin{EntryWithPhonetic}{毫米}{hao2mi3}{11,6}{⽊、⽶}[HSK 4]
  \definition{clas.}{milímetro; unidade legal de medida de comprimento, 1 mm equivale a 0,1 cm}
\end{EntryWithPhonetic}

\begin{EntryWithPhonetic}{毫升}{hao2 sheng1}{11,4}{⽊、⼗}[HSK 4]
  \definition{clas.}{mililitro; unidade de volume, milésimo de um litro (ml)}
\end{EntryWithPhonetic}

\begin{EntryWithPhonetic}{豪}{hao2}{14}{⾗}
  \definition*{s.}{Sobrenome Hao}
  \definition{adj.}{direto; irrestrito; ousado | despótico; intimidador | rico e poderoso}
  \definition{s.}{pessoa com poderes ou dons extraordinários}
\end{EntryWithPhonetic}

\begin{EntryWithPhonetic}{豪华}{hao2hua2}{14,6}{⾗、⼗}
  \definition{adj.}{luxuoso}
\end{EntryWithPhonetic}

\begin{EntryWithPhonetic}{好}{hao3}{6}{⼥}[HSK 1,2,4]
  \definition{adj.}{bom; ótimo; agradável; vantajoso; satisfatório | amigável; gentil; amistoso; amável | saudável; bem | pronto; concluído; usado após um verbo para indicar conclusão ou perfeição | fácil (de fazer); conveniente; responsável (por)}
  \definition{adv.}{muito; bastante; tão; usado na frente de uma palavra de quantidade ou uma palavra de tempo para indicar muito ou por muito tempo | em que medida; como; usado antes de adjetivos e verbos para indicar profundidade e com exclamação}
  \definition{interj.}{O.K.; tudo bem; aprovação, acordo ou encerramento | (no início de uma frase ou oração) expressa concordância (ou desaprovação, surpresa, etc.)}
  \definition{prep.}{de modo a; para que}
  \definition{s.}{referindo-se a palavras de elogio ou aplauso | saudações; cumprimentos}
  \definition{suf.}{sufixo que indica conclusão ou prontidão | depois de um pronome significa ``olá''}
  \definition{v.}{deve; precisa; tem que; deveria | apaixonar-se}
  \seeref{hao4}
\end{EntryWithPhonetic}

\begin{EntryWithPhonetic}{好(不)容易}{hao3 bu4 rong2 yi4}{6,4,10,8}{⼥、⼀、⼧、⽇}[HSK 6]
  \definition{adv.}{com grande dificuldade; muito difícil}
  \definition{v.}{ter dificuldade (em fazer algo)}
\end{EntryWithPhonetic}

\begin{EntryWithPhonetic}{好吃}{hao3chi1}{6,6}{⼥、⼝}[HSK 1]
  \definition{adj.}{bom; saboroso; delicioso; descreve o sabor agradável de algo, que as pessoas gostam de comer}
  \seeref{hao4chi1}
\end{EntryWithPhonetic}

\begin{EntryWithPhonetic}{好处}{hao3chu4}{6,5}{⼥、⼡}[HSK 2]
  \definition[个]{s.}{bom; benefício; vantagem; fatores favoráveis a pessoas ou coisas | ganho; lucro; algo que não se deveria receber, dado por outra pessoa ou obtido através de uma oportunidade; geralmente tem conotação pejorativa}
\end{EntryWithPhonetic}

\begin{EntryWithPhonetic}{好多}{hao3 duo1}{6,6}{⼥、⼣}[HSK 2]
  \definition{adj.}{muitos; uma boa quantidade; uma grande quantidade; uma quantidade enorme}
  \definition{pron.}{quantos?; quanto?; frequentemente usado para perguntar sobre quantidade}
\end{EntryWithPhonetic}

\begin{EntryWithPhonetic}{好汉}{hao3han4}{6,5}{⼥、⽔}
  \definition[条]{s.}{herói | pessoa forte e corajosa}
\end{EntryWithPhonetic}

\begin{EntryWithPhonetic}{好好}{hao3 hao3}{6,6}{⼥、⼥}[HSK 3]
  \definition{adj.}{realmente bom/bem; em perfeitas condições; quando tudo está bem}
  \definition{adv.}{diretamente; seriamente; cuidadosamente; com todo o empenho; ao máximo}
\end{EntryWithPhonetic}

\begin{EntryWithPhonetic}{好久}{hao3jiu3}{6,3}{⼥、⼃}[HSK 2]
  \definition{adv.}{por muito tempo | por eras (no passado)}
\end{EntryWithPhonetic}

\begin{EntryWithPhonetic}{好看}{hao3 kan4}{6,9}{⼥、⽬}[HSK 1]
  \definition{adj.}{de boa aparência; agradável; bonito | interessante; descreve o enredo ou conteúdo de filmes, romances, performances, etc., como sendo cativante, agradável ou apreciável}
\end{EntryWithPhonetic}

\begin{EntryWithPhonetic}{好人}{hao3 ren2}{6,2}{⼥、⼈}[HSK 2]
  \definition[个,位,名]{s.}{pessoa boa (ou excelente) (oposto de 坏人) | pessoa saudável | pessoa gentil que tenta se dar bem com todos (muitas vezes em detrimento dos princípios)}
  \seealsoref{坏人}{huai4 ren2}
\end{EntryWithPhonetic}

\begin{EntryWithPhonetic}{好生}{hao3sheng1}{6,5}{⼥、⽣}
  \definition{adv.}{bastante; extremamente | cuidadosamente; apropriadamente}
\end{EntryWithPhonetic}

\begin{EntryWithPhonetic}{好事}{hao3 shi4}{6,8}{⼥、⼅}[HSK 2]
  \definition[个,件]{s.}{boa ação; gentileza | (antigo) obra de caridade | acontecimento feliz; evento festivo}
  \seeref{hao4 shi4}
\end{EntryWithPhonetic}

\begin{EntryWithPhonetic}{好似}{hao3 si4}{6,6}{⼥、⼈}[HSK 6]
  \definition{v.}{parecer; ser como}
\end{EntryWithPhonetic}

\begin{EntryWithPhonetic}{好听}{hao3 ting1}{6,7}{⼥、⼝}[HSK 1]
  \definition{adj.}{agradável de ouvir (de som ou voz) | bom; palatável; satisfatório (de palavras)  | decente; honrado (de ações, etc.); descreve uma coisa que parece prestigiosa | interessante; descreve palavras, histórias e outras coisas interessantes}
\end{EntryWithPhonetic}

\begin{EntryWithPhonetic}{好玩儿}{hao3 wan2r5}{6,8,2}{⼥、⽟、⼉}[HSK 1]
  \definition{adj.}{divertido; interessante; capaz de despertar interesse}
\end{EntryWithPhonetic}

\begin{EntryWithPhonetic}{好象}{hao3xiang4}{6,11}{⼥、⾗}
  \variantof{好像}
\end{EntryWithPhonetic}

\begin{EntryWithPhonetic}{好像}{hao3xiang4}{6,13}{⼥、⼈}[HSK 2]
  \definition{adv.}{como se; um pouco parecido; como se fosse}
  \definition{v.}{parecer; ser como; parecer-se com}
\end{EntryWithPhonetic}

\begin{EntryWithPhonetic}{好心}{hao3xin1}{6,4}{⼥、⼼}
  \definition{s.}{bondade | boas intenções}
\end{EntryWithPhonetic}

\begin{EntryWithPhonetic}{好学}{hao3xue2}{6,8}{⼥、⼦}
  \definition{adj.}{fácil de aprender}
  \seeref{hao4xue2}
\end{EntryWithPhonetic}

\begin{EntryWithPhonetic}{好用}{hao3yong4}{6,5}{⼥、⽤}
  \definition{adj.}{fácil de usar | adequado ao uso}
\end{EntryWithPhonetic}

\begin{EntryWithPhonetic}{好友}{hao3you3}{6,4}{⼥、⼜}[HSK 4]
  \definition[位,名,个,些]{s.}{bom amigo; amigo próximo}
\end{EntryWithPhonetic}

\begin{EntryWithPhonetic}{好运}{hao3 yun4}{6,7}{⼥、⾡}[HSK 5]
  \definition{s.}{boa sorte, fortuna ou oportunidade}
\end{EntryWithPhonetic}

\begin{EntryWithPhonetic}{好转}{hao3 zhuan3}{6,8}{⼥、⾞}[HSK 6]
  \definition{v.}{melhorar; dar uma guinada para melhor; tomar um rumo favorável}
\end{EntryWithPhonetic}

\begin{EntryWithPhonetic}{号}{hao4}{5}{⼝}[HSK 1]
  \definition{clas.}{usado para o número de pessoas |  tipo; espécie; classificação | usado para pessoas ou negócios; número de vezes utilizado para transações}
  \definition[把]{s.}{nome | nome presumido; nome alternativo; pseudônimo; apelido | casa de negócios; loja | marca; sinal; sinalização | número | data | ordem; no exército, as ordens são transmitidas verbalmente ou por meio de clarins | qualquer instrumento de sopro e latão; trombeta usada no exército ou em bandas | qualquer coisa usada como buzina | chamada de corneta; qualquer chamada feita em uma corneta; usar um apito para emitir um som com um significado específico | pessoa em uma condição especial; pessoas que se encontram em uma situação especial}
  \definition{suf.}{sufixo de navio}
  \definition{v.}{marcar; fazer uma marca | sentir; colocar a mão no pulso do paciente e avaliar a situação através do fluxo sanguíneo}
  \seeref{hao2}
\end{EntryWithPhonetic}

\begin{EntryWithPhonetic}{号角}{hao4jiao3}{5,7}{⼝、⾓}
  \definition{s.}{corneta | trombeta}
\end{EntryWithPhonetic}

\begin{EntryWithPhonetic}{号码}{hao4ma3}{5,8}{⼝、⽯}[HSK 4]
  \definition[个,组,串]{s.}{número}
\end{EntryWithPhonetic}

\begin{EntryWithPhonetic}{号召}{hao4zhao4}{5,5}{⼝、⼝}[HSK 5]
  \definition{s.}{chamado; apelo; desejo ou pedido solene (de um governo, partido político, organização etc.) para que as massas façam algo}
  \definition{v.}{chamar;  (governo, partido político, organização, etc.) fazer um pedido solene às massas para que façam algo, na esperança de que todos se esforcem para alcançá-lo}
\end{EntryWithPhonetic}

\begin{EntryWithPhonetic}{好}{hao4}{6}{⼥}
  \definition*{s.}{Sobrenome Hao}
  \definition{adv.}{algo que acontece com frequência, que é fácil de acontecer}
  \definition{v.}{gostar; amar; ter afeição por}
  \seeref{hao3}
\end{EntryWithPhonetic}

\begin{EntryWithPhonetic}{好吃}{hao4chi1}{6,6}{⼥、⼝}
  \definition{v.}{ser guloso; gostar de comer boa comida}
  \seeref{hao3chi1}
\end{EntryWithPhonetic}

\begin{EntryWithPhonetic}{好奇}{hao4qi2}{6,8}{⼥、⼤}[HSK 3]
  \definition{adj.}{curioso; curiosidade e interesse por coisas não conhecidas}
  \definition{s.}{curiosidade}
  \definition{v.}{ser ou estar curioso}
\end{EntryWithPhonetic}

\begin{EntryWithPhonetic}{好事}{hao4 shi4}{6,8}{⼥、⼅}
  \definition[个,件]{s.}{intrometido; gostar de se meter na vida dos outros}
  \seeref{hao3 shi4}
\end{EntryWithPhonetic}

\begin{EntryWithPhonetic}{好学}{hao4xue2}{6,8}{⼥、⼦}[HSK 6]
  \definition[个]{s.}{apaixonado para aprender; estudioso; erudito}
  \seeref{hao3xue2}
\end{EntryWithPhonetic}

\begin{EntryWithPhonetic}{呵}{he1}{8}{⼝}
  \definition{interj.}{Meu Deus!| Ah!; Oh!}
  \definition{v.}{expirar (com a boca aberta) | repreender}
  \seeref{a1}
\end{EntryWithPhonetic}

\begin{EntryWithPhonetic}{欱}{he1}{10}{⽋}
  \definition{v.}{beber | beber bebida alcoólica}
  \variantof{喝}
\end{EntryWithPhonetic}

\begin{EntryWithPhonetic}{喝}{he1}{12}{⼝}[HSK 1]
  \definition{interj.}{Meu Deus!; Oh!; Ah!; Uau!}
  \definition{s.}{bebida; especificamente, vinho}
  \definition{v.}{beber; engolir líquidos ou alimentos líquidos | beber bebida alcoólica; referência específica ao consumo de álcool}
  \seeref{he4}
\end{EntryWithPhonetic}

\begin{EntryWithPhonetic}{喝醉}{he1zui4}{12,15}{⼝、⾣}
  \definition{v.}{ficar bêbado}
\end{EntryWithPhonetic}

\begin{EntryWithPhonetic}{合}{he2}{6}{⼝}[HSK 3]
  \definition{adj.}{todo; completo; inteiro}
  \definition{clas.}{usado para rodadas | 100ml | medida para grãos secos igual a um décimo de 升, ou um centésimo de 斗}
  \definition{s.}{casamento; união matrimonial | (astronomia) conjunção | nota da escala em Gongchepu (工尺谱), correspondente ao 5 na notação musical numerada}
  \definition{v.}{fechar | juntar; combinar (oposto de 分) | adequar-se; concordar; conformar-se a | ser igual a; somar | ser adequado}
  \seealsoref{斗}{dou4}
  \seealsoref{分}{fen1}
  \seealsoref{工尺谱}{gong1 che3 pu3}
  \seealsoref{升}{sheng1}
\end{EntryWithPhonetic}

\begin{EntryWithPhonetic}{合并}{he2bing4}{6,6}{⼝、⼲}[HSK 5]
  \definition{v.}{fundir; amalgamar; combinar várias coisas em uma coisa só | (doença) ser complicada por outra doença; uma doença levar a outra, ataques simultâneos (de várias doenças)}
\end{EntryWithPhonetic}

\begin{EntryWithPhonetic}{合成}{he2cheng2}{6,6}{⼝、⼽}[HSK 5]
  \definition{s.}{compor; integrar; combinar; misturar | Química: sintetizar, reação química para transformar uma substância com uma composição simples em uma substância com uma composição complexa}
\end{EntryWithPhonetic}

\begin{EntryWithPhonetic}{合法}{he2fa3}{6,8}{⼝、⽔}[HSK 3]
  \definition{adj.}{legal; legítimo; lícito;  justo; válido; em conformidade com as disposições legais}
\end{EntryWithPhonetic}

\begin{EntryWithPhonetic}{合格}{he2ge2}{6,10}{⼝、⽊}[HSK 3]
  \definition{adj.}{qualificado; dentro dos padrões; em conformidade com os requisitos ou normas}
\end{EntryWithPhonetic}

\begin{EntryWithPhonetic}{合理}{he2li3}{6,11}{⼝、⽟}[HSK 3]
  \definition{adj.}{racional; razoável; equitativo; razoável ou lógico}
\end{EntryWithPhonetic}

\begin{EntryWithPhonetic}{合适}{he2shi4}{6,9}{⼝、⾡}[HSK 2]
  \definition{adj.}{correto; adequado; apropriado; conveniente; em conformidade com a realidade ou com os requisitos objetivos}
\end{EntryWithPhonetic}

\begin{EntryWithPhonetic}{合同}{he2tong5}{6,6}{⼝、⼝}[HSK 4]
  \definition[个,份]{s.}{contrato; acordo; uma disposição para observância mútua por duas ou mais partes na condução de um assunto com o objetivo de determinar seus respectivos direitos e obrigações.}
\end{EntryWithPhonetic}

\begin{EntryWithPhonetic}{合宪性}{he2xian4xing4}{6,9,8}{⼝、⼧、⼼}
  \definition{s.}{constitucionalismo}
\end{EntryWithPhonetic}

\begin{EntryWithPhonetic}{合约}{he2 yue1}{6,6}{⼝、⽷}[HSK 6]
  \definition[份]{s.}{contrato; geralmente se refere a contratos com cláusulas mais simples}
\end{EntryWithPhonetic}

\begin{EntryWithPhonetic}{合资}{he2zi1}{6,10}{⼝、⾙}
  \definition{s.}{\emph{joint-venture} com capitais mistos}
\end{EntryWithPhonetic}

\begin{EntryWithPhonetic}{合作}{he2zuo4}{6,7}{⼝、⼈}[HSK 3]
  \definition{v.}{cooperar; colaborar; trabalhar em conjunto; trabalhar em conjunto para realizar algo ou concluir uma tarefa}
\end{EntryWithPhonetic}

\begin{EntryWithPhonetic}{何}{he2}{7}{⼈}
  \definition*{s.}{Sobrenome He}
  \definition{adv.}{enfatiza um alto grau de intensidade, equivalente a 多么}
  \definition{pron.}{O que?; Qual?; em nome de pessoas ou coisas, equivalente a 什么 | Onde?; em nome do lugar, equivalente a 哪里 | Por que?; Como?; a razão, é equivalente a 为什么 e 怎么}
  \seealsoref{多么}{duo1me5}
  \seealsoref{哪里}{na3 li3}
  \seealsoref{岂}{qi3}
  \seealsoref{什么}{shen2me5}
  \seealsoref{为什么}{wei4shen2me5}
  \seealsoref{怎}{zen3}
  \seealsoref{怎么}{zen3me5}
\end{EntryWithPhonetic}

\begin{EntryWithPhonetic}{何不}{he2bu4}{7,4}{⼈、⼀}
  \definition{adv.}{por que não?; use o tom interrogativo para expressar "deveria" ou "pode"}
\end{EntryWithPhonetic}

\begin{EntryWithPhonetic}{何故}{he2gu4}{7,9}{⼈、⽁}
  \definition{adv.}{qual razão?; por que? | para quê? | qual é o motivo?}
\end{EntryWithPhonetic}

\begin{EntryWithPhonetic}{何况}{he2kuang4}{7,7}{⼈、⼎}
  \definition{conj.}{além disso | muito menos}
\end{EntryWithPhonetic}

\begin{EntryWithPhonetic}{和}{he2}{8}{⼝}[HSK 1]
  \definition*{s.}{Sobrenome He}
  \definition{adj.}{gentil; suave; amável | harmonioso; em boas condições}
  \definition{conj.}{e (somente para palavras); unidos com}
  \definition{prep.}{relacionado com | para; com; indica correlação; comparação, etc.}
  \definition{s.}{soma; soma total | japonês; refere-se ao Japão}
  \definition{v.}{disputar; reconciliar; acabar com a guerra ou a disputa | empatar; (próxima edição ou torneio) sem vencedor}
  \seeref{he4}
  \seeref{hu2}
  \seeref{huo2}
  \seeref{huo4}
\end{EntryWithPhonetic}

\begin{EntryWithPhonetic}{和平}{he2ping2}{8,5}{⼝、⼲}[HSK 3]
  \definition{adj.}{pacífico; moderado; não violento | pacífico; tranquilo; sereno}
  \definition{s.}{paz ;ausência de guerra}
\end{EntryWithPhonetic}

\begin{EntryWithPhonetic}{和平共处}{he2ping2gong4chu3}{8,5,6,5}{⼝、⼲、⼋、⼡}
  \definition{s.}{coexistência pacífica de nações, sociedades, etc.}
\end{EntryWithPhonetic}

\begin{EntryWithPhonetic}{和谐}{he2xie2}{8,11}{⼝、⾔}[HSK 6]
  \definition{adj.}{harmonioso; sem conflitos | em perfeita harmonia; ajuste adequado e simétrico}
  \definition{v.}{(eufemismo) censurar}
\end{EntryWithPhonetic}

\begin{EntryWithPhonetic}{河}{he2}{8}{⽔}[HSK 2]
  \definition*{s.}{Astronomia: o sistema da Via Láctea | O Rio Amarelo; O Rio Huanghe | Sobrenome He}
  \definition[条,道]{s.}{rio; refere-se a grandes cursos de água}
\end{EntryWithPhonetic}

\begin{EntryWithPhonetic}{河蚌}{he2bang4}{8,10}{⽔、⾍}
  \definition{s.}{mexilhões | bivalves cultivados em rios e lagos}
\end{EntryWithPhonetic}

\begin{EntryWithPhonetic}{河南}{he2nan2}{8,9}{⽔、⼗}
  \definition*{s.}{Província de Henan}
\end{EntryWithPhonetic}

\begin{EntryWithPhonetic}{核}{he2}{10}{⽊}
  \definition{s.}{poço; pedra; caroço | núcleo | núcleo atômico}
  \definition{v.}{examinar; verificar}
\end{EntryWithPhonetic}

\begin{EntryWithPhonetic}{核心}{he2xin1}{10,4}{⽊、⼼}[HSK 6]
  \definition[个]{s.}{núcleo; elite; coração; centro; parte principal (em termos de relacionamento entre as coisas)}
\end{EntryWithPhonetic}

\begin{EntryWithPhonetic}{荷}{he2}{10}{⾋}
  \definition*{s.}{Países Baixos; Holanda, abreviação de 荷兰 | Sobrenome He}
  \definition{s.}{lótus}
  \seeref{he4}
  \seealsoref{荷兰}{he2lan2}
\end{EntryWithPhonetic}

\begin{EntryWithPhonetic}{荷花}{he2hua1}{10,7}{⾋、⾋}
  \definition{s.}{lótus}
\end{EntryWithPhonetic}

\begin{EntryWithPhonetic}{荷兰}{he2lan2}{10,5}{⾋、⼋}
  \definition*{s.}{Países Baixos; Holanda}
\end{EntryWithPhonetic}

\begin{EntryWithPhonetic}{盒}{he2}{11}{⽫}[HSK 5]
  \definition{clas.}{caixa (de pequena dimensão)}
  \definition[个]{s.}{caixa; estojo; recipiente; receptáculo}
\end{EntryWithPhonetic}

\begin{EntryWithPhonetic}{盒饭}{he2 fan4}{11,7}{⽫、⾷}[HSK 5]
  \definition[份]{s.}{refeição embalada; marmita; \emph{fast-food} vendida em caixas}
\end{EntryWithPhonetic}

\begin{EntryWithPhonetic}{盒子}{he2zi5}{11,3}{⽫、⼦}[HSK 5]
  \definition[个,只,堆]{s.}{caixa; recipiente que têm tampas na parte superior e podem conter coisas dentro, geralmente é pequeno e plano}
\end{EntryWithPhonetic}

\begin{EntryWithPhonetic}{和}{he4}{8}{⼝}
  \definition{v.}{compor um poema em resposta (ao poema de alguém) usando a mesma sequência de rimas | juntar-se à cantoria; cantar junto com outros em harmonia}
  \seeref{he2}
  \seeref{hu2}
  \seeref{huo2}
  \seeref{huo4}
\end{EntryWithPhonetic}

\begin{EntryWithPhonetic}{贺}{he4}{9}{⾙}
  \definition*{s.}{Sobrenome He}
  \definition{v.}{parabenizar; congratular | celebrar; comemorar}
\end{EntryWithPhonetic}

\begin{EntryWithPhonetic}{贺卡}{he4 ka3}{9,5}{⾙、⼘}[HSK 5]
  \definition[张]{s.}{cartão de felicitações; pedaço de papel para parabenizar amigos e parentes em seu casamento, aniversário ou festivais, geralmente impresso com palavras e desenhos de felicitações}
\end{EntryWithPhonetic}

\begin{EntryWithPhonetic}{荷}{he4}{10}{⾋}
  \definition{s.}{fardo; responsabilidade}
  \definition{v.}{carregar no ombro ou nas costas | aceitar um favor, frequentemente usado em cartas para expressar cortesia}
  \seeref{he2}
\end{EntryWithPhonetic}

\begin{EntryWithPhonetic}{喝}{he4}{12}{⼝}
  \definition{v.}{gritar bem alto}
  \seeref{he1}
\end{EntryWithPhonetic}

\begin{EntryWithPhonetic}{喝彩}{he4cai3}{12,11}{⼝、⼺}
  \definition{s.}{aclamar | torcer}
\end{EntryWithPhonetic}

\begin{EntryWithPhonetic}{褐}{he4}{14}{⾐}
  \definition{adj.}{marrom; castanho; pardo}
  \definition{s.}{pano de cânhamo grosso}
\end{EntryWithPhonetic}

\begin{EntryWithPhonetic}{褐色}{he4 se4}{14,6}{⾐、⾊}
  \definition{s.}{cor marrom}
\end{EntryWithPhonetic}

\begin{EntryWithPhonetic}{鹤}{he4}{15}{⿃}
  \definition*{s.}{Sobrenome He}
  \definition[只]{s.}{grou (ave)}
\end{EntryWithPhonetic}

\begin{EntryWithPhonetic}{黑}{hei1}{12}{⿊}[HSK 2][Kangxi 203]
  \definition*{s.}{Província de Heilongjiang, abreviação de 黑龙江 | Sobrenome Hei}
  \definition{adj.}{preto; cor semelhante à do carvão | escuro | obscuro; secreto | perverso; sinistro; ruim; cruel | reacionário}
  \definition{s.}{noite}
  \definition{v.}{fazer algo ilegalmente ou de forma desonesta; enganar; desviar dinheiro ilegalmente | invadir (uma rede, sites, computador, etc.)}
  \seealsoref{黑龙江}{hei1long2jiang1}
\end{EntryWithPhonetic}

\begin{EntryWithPhonetic}{黑暗}{hei1 an4}{12,13}{⿊、⽇}[HSK 4]
  \definition{adj.}{escuro; sombrio; sem luz | maligno; corrupto; reacionário}
\end{EntryWithPhonetic}

\begin{EntryWithPhonetic}{黑板}{hei1ban3}{12,8}{⿊、⽊}[HSK 2]
  \definition[块,个]{s.}{quadro negro; quadro de giz; uma placa, na qual se pode escrever com giz}
\end{EntryWithPhonetic}

\begin{EntryWithPhonetic}{黑客}{hei1ke4}{12,9}{⿊、⼧}
  \definition{s.}{(empréstimo linguístico) (computação) \emph{hacker}}
\end{EntryWithPhonetic}

\begin{EntryWithPhonetic}{黑龙江}{hei1long2jiang1}{12,5,6}{⿊、⿓、⽔}
  \definition*{s.}{Província de Heilongjiang | Rio Heilong Jiang;  Rio Amur (na Rússia)}
\end{EntryWithPhonetic}

\begin{EntryWithPhonetic}{黑色}{hei1 se4}{12,6}{⿊、⾊}[HSK 2]
  \definition{adj.}{metafórico: suspeito, ilegal}
  \definition{s.}{cor preta}
\end{EntryWithPhonetic}

\begin{EntryWithPhonetic}{黑桃}{hei1 tao2}{12,10}{⿊、⽊}
  \definition{s.}{espadas ♠ (em jogos de cartas)}
  \seealsoref{方片}{fang1 pian4}
  \seealsoref{红心}{hong2 xin1}
  \seealsoref{梅花}{mei2 hua1}
\end{EntryWithPhonetic}

\begin{EntryWithPhonetic}{黑夜}{hei1 ye4}{12,8}{⿊、⼣}[HSK 6]
  \definition[个]{s.}{noite ; uma noite muito escura sem luz}
\end{EntryWithPhonetic}

\begin{EntryWithPhonetic}{很}{hen3}{9}{⼻}[HSK 1]
  \definition{adv.}{muito; bastante; terrivelmente; indica um grau bastante elevado; definitivo; o mais alto}
\end{EntryWithPhonetic}

\begin{EntryWithPhonetic}{很难说}{hen3 nan2 shuo1}{9,10,9}{⼻、⾫、⾔}[HSK 6]
  \definition{adj.}{difícil dizer}
\end{EntryWithPhonetic}

\begin{EntryWithPhonetic}{狠}{hen3}{9}{⽝}[HSK 6]
  \definition{adj.}{impiedoso; implacável; feroz | firme; resoluto; severo; determinado}
  \definition{adv.}{muito; bastante; bastante | também, frequentemente usado antes de um adjetivo sem intensificar seu significado, ou seja, como um elemento sintático sem sentido}
  \definition{v.}{endurecer (o coração); suprimir (os próprios sentimentos)}
  \variantof{很}
  \seealsoref{很}{hen3}
\end{EntryWithPhonetic}

\begin{EntryWithPhonetic}{恨}{hen4}{9}{⼼}[HSK 5]
  \definition{s.}{ódio; resentimento}
  \definition{v.}{odiar; ressentir-se}
\end{EntryWithPhonetic}

\begin{EntryWithPhonetic}{行}{heng2}{6}{⾏}[Kangxi 144]
  \definition{s.}{usado em 道行}
  \seeref{hang2}
  \seeref{xing2}
  \seealsoref{道行}{dao4 heng2}
\end{EntryWithPhonetic}

\begin{EntryWithPhonetic}{恒}{heng2}{9}{⼼}
  \definition*{s.}{Sobrenome Heng}
  \definition{adj.}{permanente; duradouro | usual; comum; constante | usual; frequente; constante}
  \definition{s.}{perseverança; constância}
\end{EntryWithPhonetic}

\begin{EntryWithPhonetic}{恒星系}{heng2xing1xi4}{9,9,7}{⼼、⽇、⽷}
  \definition{s.}{sistema estelar | galáxia}
\end{EntryWithPhonetic}

\begin{EntryWithPhonetic}{横}{heng2}{15}{⽊}[HSK 6]
  \definition{adj.}{horizontal; transversal; paralelo ao plano horizontal (oposto de 竖 e 直) | em ângulo reto com; direção esquerda-direita (em oposição à 竖, 直 ou 纵) | e leste a oeste ou de oeste a leste; direção leste-oeste (oposta a 纵) | desenfreado; turbulento | violento; feroz; irracional}
  \definition{adv.}{de qualquer forma; em qualquer caso | provavelmente; muito provavelmente}
  \definition{s.}{traço horizontal (em caracteres chineses)}
  \definition{v.}{deitar-se transversalmente; estar de lado | colocar algo transversalmente (ou horizontalmente)}
  \seealsoref{竖}{shu4}
  \seealsoref{直}{zhi2}
  \seealsoref{纵}{zong4}
\end{EntryWithPhonetic}

\begin{EntryWithPhonetic}{横竖}{heng2shu5}{15,9}{⽊、⽴}
  \definition{adv.}{de qualquer maneira | independentemente (linguagem falada)}
\end{EntryWithPhonetic}

\begin{EntryWithPhonetic}{衡}{heng2}{16}{⾏}
  \definition*{s.}{Sobrenome Heng}
  \definition[个]{s.}{braço graduado de uma balança | balança; aparelho de pesagem}
  \definition{v.}{pesar; medir; julgar}
\end{EntryWithPhonetic}

\begin{EntryWithPhonetic}{衡量}{heng2 liang2}{16,12}{⾏、⾥}[HSK 6]
  \definition{v.}{pesar; medir; comparar; avaliar | considerar; pensar sobre; deliberar}
\end{EntryWithPhonetic}

\begin{EntryWithPhonetic}{轰}{hong1}{8}{⾞}
  \definition{interj.}{(onomatopéia) Bum!; estrondo; refere-se aos ruídos altos feitos por trovões, fogo de artilharia, etc.}
  \definition{v.}{retumbar; bombardear; explodir | espantar; expulsar}
\end{EntryWithPhonetic}

\begin{EntryWithPhonetic}{轰鸣}{hong1ming2}{8,8}{⾞、⿃}
  \definition{s.}{bum (som de explosão) | estrondo}
\end{EntryWithPhonetic}

\begin{EntryWithPhonetic}{轰炸机}{hong1zha4ji1}{8,9,6}{⾞、⽕、⽊}
  \definition{s.}{bombardeiro (aeronave)}
\end{EntryWithPhonetic}

\begin{EntryWithPhonetic}{哄}{hong1}{9}{⼝}
  \definition{interj.}{(onomatopéia) gargalhadas ou alvoroço}
  \definition{s.}{rugido; clamor; comoção}
  \seeref{hong3}
  \seeref{hong4}
\end{EntryWithPhonetic}

\begin{EntryWithPhonetic}{弘}{hong2}{5}{⼸}
  \definition{adj.}{grande; grandioso; magnífico}
  \definition{s.}{Sobrenome Hong}
  \definition{v.}{ampliar; expandir | promover}
\end{EntryWithPhonetic}

\begin{EntryWithPhonetic}{红}{hong2}{6}{⽷}[HSK 2]
  \definition*{s.}{Sobrenome Hong}
  \definition{adj.}{vermelho | popular; bem-sucedido; símbolo de sucesso ou valorização | vermelho; revolucionário; símbolo da revolução | festivo; símbolo de alegria}
  \definition{s.}{tecido vermelho, bandeirinhas, etc. usados em ocasiões festivas | bônus; dividendo}
\end{EntryWithPhonetic}

\begin{EntryWithPhonetic}{红包}{hong2 bao1}{6,5}{⽷、⼓}[HSK 4]
  \definition[个]{s.}{saco de papel vermelho ou envelope contendo dinheiro como presente, gorjeta ou bônus | suborno; propina}
\end{EntryWithPhonetic}

\begin{EntryWithPhonetic}{红宝石}{hong2bao3shi2}{6,8,5}{⽷、⼧、⽯}
  \definition{s.}{rubi}
\end{EntryWithPhonetic}

\begin{EntryWithPhonetic}{红茶}{hong2 cha2}{6,9}{⽷、⾋}[HSK 3]
  \definition[杯,壶,斤,种]{s.}{chá preto; chá acabado produzido através de fermentação completa}
\end{EntryWithPhonetic}

\begin{EntryWithPhonetic}{红酒}{hong2 jiu3}{6,10}{⽷、⾣}[HSK 3]
  \definition[瓶,杯,壶,斤,箱]{s.}{vinho tinto}
\end{EntryWithPhonetic}

\begin{EntryWithPhonetic}{红绿灯}{hong2lv4deng1}{6,11,6}{⽷、⽷、⽕}
  \definition[个]{s.}{semáforo | sinal de trânsito}
\end{EntryWithPhonetic}

\begin{EntryWithPhonetic}{红色}{hong2 se4}{6,6}{⽷、⾊}[HSK 2]
  \definition{adj.}{vermelho; revolucionário; símbolo da revolução ou da consciência política elevada}
  \definition{s.}{cor vermelha}
\end{EntryWithPhonetic}

\begin{EntryWithPhonetic}{红烧}{hong2shao1}{6,10}{⽷、⽕}
  \definition{s.}{guisado em molho de soja (prato)}
\end{EntryWithPhonetic}

\begin{EntryWithPhonetic}{红薯}{hong2shu3}{6,16}{⽷、⾋}
  \definition{s.}{batata doce}
\end{EntryWithPhonetic}

\begin{EntryWithPhonetic}{红线}{hong2xian4}{6,8}{⽷、⽷}
  \definition{s.}{linha vermelha}
\end{EntryWithPhonetic}

\begin{EntryWithPhonetic}{红心}{hong2 xin1}{6,4}{⽷、⼼}
  \definition{s.}{coração vermelho, um coração leal à causa da revolução proletária | alvo | coração ♥ (em jogos de cartas) | red, heart-shaped symbol}
  \seealsoref{方片}{fang1 pian4}
  \seealsoref{黑桃}{hei1 tao2}
  \seealsoref{梅花}{mei2 hua1}
\end{EntryWithPhonetic}

\begin{EntryWithPhonetic}{宏}{hong2}{7}{⼧}
  \definition*{s.}{Sobrenome Hong}
  \definition{adj.}{grande; grandioso; magnífico}
  \definition{v.}{divulgar algo; promover algo; atualmente, geralmente é escrito como 弘}
  \seealsoref{弘}{hong2}
\end{EntryWithPhonetic}

\begin{EntryWithPhonetic}{宏大}{hong2 da4}{7,3}{⼧、⼤}[HSK 6]
  \definition{adj.}{grande; ótimo | imenso; vasto}
\end{EntryWithPhonetic}

\begin{EntryWithPhonetic}{洪}{hong2}{9}{⽔}
  \definition*{s.}{Sobrenome Hong}
  \definition{adj.}{alto; vasto | grande; grandioso}
  \definition[场]{s.}{enchente; inundação}
\end{EntryWithPhonetic}

\begin{EntryWithPhonetic}{洪水}{hong2shui3}{9,4}{⽔、⽔}[HSK 6]
  \definition[场]{s.}{dilúvio; inundação; enchente; um aumento repentino em um rio causado por chuva forte ou derretimento de neve}
\end{EntryWithPhonetic}

\begin{EntryWithPhonetic}{哄}{hong3}{9}{⼝}
  \definition{v.}{brincar; enganar; tapear | persuadir; agradar os outros com palavras ou ações, especialmente observando ou cuidando de crianças}
  \seeref{hong1}
  \seeref{hong4}
\end{EntryWithPhonetic}

\begin{EntryWithPhonetic}{哄}{hong4}{9}{⼝}
  \definition{s.}{comoção; tumulto}
  \seeref{hong1}
  \seeref{hong3}
\end{EntryWithPhonetic}

\begin{EntryWithPhonetic}{猴}{hou2}{12}{⽝}[HSK 5]
  \definition{adj.}{esperto; inteligente; perspicaz | travesso (menino)}
  \definition[只,群]{s.}{macaco}
\end{EntryWithPhonetic}

\begin{EntryWithPhonetic}{猴子}{hou2zi5}{12,3}{⽝、⼦}
  \definition[只]{s.}{macaco}
\end{EntryWithPhonetic}

\begin{EntryWithPhonetic}{后}{hou4}{6}{⼝}[HSK 1]
  \definition*{s.}{Sobrenome Hou}
  \definition{s.}{atrás; traseiro; a direção oposta àquela para a qual a pessoa está voltada; a direção oposta àquela para a qual a parte de trás de uma casa está voltada (o oposto de 前)  | depois; mais tarde no tempo; futuro (em oposição a 先 ou 前) | último | posteridade; descendência | rainha; imperatriz | governante; soberano; monarca antigo}
  \seealsoref{前}{qian2}
  \seealsoref{先}{xian1}
\end{EntryWithPhonetic}

\begin{EntryWithPhonetic}{后边}{hou4 bian5}{6,5}{⼝、⾡}[HSK 1]
  \definition{adv.}{costas; traseira; atrás}
\end{EntryWithPhonetic}

\begin{EntryWithPhonetic}{后方}{hou4 fang1}{6,4}{⼝、⽅}
  \definition{s.}{traseira; retaguarda (oposto à 前线 e 前方) | na parte de trás; na parte traseira}
  \seealsoref{前方}{qian2 fang1}
  \seealsoref{前线}{qian2 xian4}
\end{EntryWithPhonetic}

\begin{EntryWithPhonetic}{后果}{hou4guo3}{6,8}{⼝、⽊}[HSK 3]
  \definition{s.}{consequência; resultado (geralmente negativo)}
\end{EntryWithPhonetic}

\begin{EntryWithPhonetic}{后悔}{hou4hui3}{6,10}{⼝、⼼}[HSK 5]
  \definition{v.}{lamentar; ter remorso; arrepender-se}
\end{EntryWithPhonetic}

\begin{EntryWithPhonetic}{后来}{hou4lai2}{6,7}{⼝、⽊}[HSK 2]
  \definition{adv.}{mais tarde; depois; refere-se a um período posterior a um determinado momento no passado}
\end{EntryWithPhonetic}

\begin{EntryWithPhonetic}{后面}{hou4mian4}{6,9}{⼝、⾯}
  \definition{adv.}{parte de trás; retaguarda; atrás; a parte posterior do espaço ou localização | mais tarde; depois; no futuro; a parte posterior de um artigo ou discurso em relação ao que está sendo narrado no momento}
\end{EntryWithPhonetic}

\begin{EntryWithPhonetic}{后年}{hou4nian2}{6,6}{⼝、⼲}[HSK 3]
  \definition{s.}{daqui a dois anos; no ano seguinte ao próximo ano}
\end{EntryWithPhonetic}

\begin{EntryWithPhonetic}{后天}{hou4 tian1}{6,4}{⼝、⼤}[HSK 1]
  \definition{s.}{depois de amanhã; período em que uma pessoa ou animal vive e cresce sozinho após deixar o útero materno (em oposição a 先天)}
  \seealsoref{先天}{xian1tian1}
\end{EntryWithPhonetic}

\begin{EntryWithPhonetic}{后头}{hou4 tou5}{6,5}{⼝、⼤}[HSK 4]
  \definition{adv.}{posteriormente; atrás; mais tarde}
  \definition{s.}{a parte de trás; a parte traseira}
\end{EntryWithPhonetic}

\begin{EntryWithPhonetic}{厚}{hou4}{9}{⼚}[HSK 4]
  \definition*{s.}{Sobrenome Hou}
  \definition{adj.}{espesso; grosso (oposto a 薄) | profundo | gentil; magnânimo | grande; generoso | rico ou forte em sabor}
  \definition[米,厘米]{s.}{espessura | profundidade}
  \definition{v.}{favorecer; enfatizar}
  \seealsoref{薄}{bao2}
\end{EntryWithPhonetic}

\begin{EntryWithPhonetic}{呼}{hu1}{8}{⼝}
  \definition*{s.}{Sobrenome Hu}
  \definition{s.}{Onomatopéia: descreve o som do vento}
  \definition{v.}{expirar | gritar; clamar | chamar; ligar; ligar para alguém}
\end{EntryWithPhonetic}

\begin{EntryWithPhonetic}{呼啦啦}{hu1 la1 la1}{8,11,11}{⼝、⼝、⼝}
  \definition{s.}{Onomatopéia: som de bater asas}
\end{EntryWithPhonetic}

\begin{EntryWithPhonetic}{呼吸}{hu1xi1}{8,6}{⼝、⼝}[HSK 4]
  \definition{s.}{um suspiro; metáfora para um período de tempo muito curto}
  \definition{v.}{respirar}
\end{EntryWithPhonetic}

\begin{EntryWithPhonetic}{呼啸}{hu1xiao4}{8,11}{⼝、⼝}
  \definition{v.}{assobiar}
\end{EntryWithPhonetic}

\begin{EntryWithPhonetic}{忽}{hu1}{8}{⼼}
  \definition*{s.}{Sobrenome Hu}
  \definition{adv.}{agora\dots, agora\dots | de repente; subitamente}[天气忽冷忽热。===O clima está frio em um minuto e quente no outro.]
  \definition{v.}{negligenciar; ignorar; não prestar atenção; não levar a sério}
\end{EntryWithPhonetic}

\begin{EntryWithPhonetic}{忽略}{hu1lve4}{8,11}{⼼、⽥}[HSK 6]
  \definition{v.}{negligenciar; ignorar; não perceber}
\end{EntryWithPhonetic}

\begin{EntryWithPhonetic}{忽然}{hu1ran2}{8,12}{⼼、⽕}[HSK 2]
  \definition{adv.}{repentinamente; de repente; sem aviso prévio; significa que algo aconteceu de forma rápida e inesperada}
\end{EntryWithPhonetic}

\begin{EntryWithPhonetic}{忽视}{hu1shi4}{8,8}{⼼、⾒}[HSK 4]
  \definition{v.}{ignorar; negligenciar; menosprezar; desprezar; dar de ombros}
\end{EntryWithPhonetic}

\begin{EntryWithPhonetic}{和}{hu2}{8}{⼝}
  \definition{v.}{completar um conjunto de Mahjong, 麻将, ou cartas de baralho}
  \seeref{he2}
  \seeref{he4}
  \seeref{huo2}
  \seeref{huo4}
  \seealsoref{麻将}{ma2jiang4}
\end{EntryWithPhonetic}

\begin{EntryWithPhonetic}{胡}{hu2}{9}{⾁}
  \definition*{s.}{Sobrenome Hu}
  \definition{adj.}{introduzidos de nacionalidades do norte e do oeste ou do exterior | nos tempos antigos, o termo "Oriente e Ocidente" se referia às minorias étnicas do norte e do oeste, e também, de modo geral, às pessoas do exterior}
  \definition{adv.}{imprudentemente; desenfreadamente; escandalosamente; sem lei, ordem ou razão}
  \definition{pron.}{Por que?; palavras interrogativas: 为什么, 何故}
  \definition{s.}{nos tempos antigos, geralmente se referia às minorias étnicas do norte e do oeste | violino chinês | barba; bigode}
  \seealsoref{何故}{he2gu4}
  \seealsoref{为什么}{wei4shen2me5}
\end{EntryWithPhonetic}

\begin{EntryWithPhonetic}{胡萝卜}{hu2luo2bo5}{9,11,2}{⾁、⾋、⼘}
  \definition{s.}{cenoura}
\end{EntryWithPhonetic}

\begin{EntryWithPhonetic}{胡琴}{hu2qin2}{9,12}{⾁、⽟}
  \definition{s.}{huqin, um termo geral para certos instrumentos de arco de duas cordas, como 二胡, 京胡, etc. | família de violinos chineses de duas cordas, com caixa de ressonância de madeira revestida de pele de cobra e arco de bambu com corda de crina de cavalo}
  \seealsoref{二胡}{er4hu2}
  \seealsoref{京胡}{jing1hu2}
\end{EntryWithPhonetic}

\begin{EntryWithPhonetic}{胡同儿}{hu2 tong4r5}{9,6,2}{⾁、⼝、⼉}[HSK 5]
  \definition{s.}{beco}
\end{EntryWithPhonetic}

\begin{EntryWithPhonetic}{胡同}{hu2tong5}{9,6}{⾁、⼝}
  \definition[条,个]{s.}{beco; rua pequena}
\end{EntryWithPhonetic}

\begin{EntryWithPhonetic}{胡子}{hu2 zi5}{9,3}{⾁、⼦}[HSK 5]
  \definition[团,根,个,撮]{s.}{barba; bigode | bandido; salteador}
\end{EntryWithPhonetic}

\begin{EntryWithPhonetic}{壶}{hu2}{10}{⼠}[HSK 6]
  \definition*{s.}{Sobrenome Hu}
  \definition[个,把]{s.}{chaleira; panela | garrafa; frasco; recipiente para líquidos}
\end{EntryWithPhonetic}

\begin{EntryWithPhonetic}{斛}{hu2}{11}{⽃}
  \definition*{s.}{Sobrenome Hu}
  \definition{s.}{Arcaico: uma medida seca usada antigamente, originalmente igual a 10 dou (斗), mais tarde 5 dou}
  \seealsoref{斗}{dou4}
\end{EntryWithPhonetic}

\begin{EntryWithPhonetic}{湖}{hu2}{12}{⽔}[HSK 2]
  \definition*{s.}{Huzhou, abreviação de 湖州 | Um nome que se refere às províncias de Hunan, 湖南,  e Hubei, 湖北}
  \definition[个,片]{s.}{lago}
  \seealsoref{湖北}{hu2bei3}
  \seealsoref{湖南}{hu2nan2}
  \seealsoref{湖州}{hu2zhou1}
\end{EntryWithPhonetic}

\begin{EntryWithPhonetic}{湖北}{hu2bei3}{12,5}{⽔、⼔}
  \definition*{s.}{Província de Hubei (Hupeh), na China central}
\end{EntryWithPhonetic}

\begin{EntryWithPhonetic}{湖南}{hu2nan2}{12,9}{⽔、⼗}
  \definition*{s.}{Província de Hunan}
\end{EntryWithPhonetic}

\begin{EntryWithPhonetic}{湖州}{hu2zhou1}{12,6}{⽔、⼮}
  \definition*{s.}{Cidade de Huzhou, em Zhejiang}
\end{EntryWithPhonetic}

\begin{EntryWithPhonetic}{葫}{hu2}{12}{⾋}
  \definition{s.}{cabaça}
\end{EntryWithPhonetic}

\begin{EntryWithPhonetic}{葫芦}{hu2lu5}{12,7}{⾋、⾋}
  \definition{adj.}{confuso}
  \definition{s.}{cabaça | termo genérico para bloco e equipamento (ou partes dele)}
\end{EntryWithPhonetic}

\begin{EntryWithPhonetic}{糊}{hu2}{15}{⽶}
  \definition{adj.}{queimado}
  \definition{s.}{mingau; pasta; papa}
  \definition{v.}{colar com pasta; colar | (de comida) ser queimado}
  \seeref{hu4}
\end{EntryWithPhonetic}

\begin{EntryWithPhonetic}{糊里糊涂}{hu2 li5 hu2tu5}{15,7,15,10}{⽶、⾥、⽶、⽔}
  \definition{adj.}{desnorteado | perturbado}
\end{EntryWithPhonetic}

\begin{EntryWithPhonetic}{蝴}{hu2}{15}{⾍}
  \definition[对]{s.}{borboleta}
\end{EntryWithPhonetic}

\begin{EntryWithPhonetic}{蝴蝶}{hu2die2}{15,15}{⾍、⾍}
  \definition[只]{s.}{borboleta}
\end{EntryWithPhonetic}

\begin{EntryWithPhonetic}{虎}{hu3}{8}{⾌}[HSK 5]
  \definition*{s.}{Sobrenome Hu}
  \definition{adj.}{corajoso; bravo; valente; vigoroso}
  \definition[只]{s.}{tigre}
  \definition{v.}{blefar; o mesmo que 唬 | parecer feroz; mostrar a aparência feroz de alguém}
  \seealsoref{唬}{hu3}
  \seealsoref{老虎}{lao3hu3}
\end{EntryWithPhonetic}

\begin{EntryWithPhonetic}{虎虎}{hu3hu3}{8,8}{⾌、⾌}
  \definition{adj.}{formidável | forte | vigoroso}
\end{EntryWithPhonetic}

\begin{EntryWithPhonetic}{虎口}{hu3kou3}{8,3}{⾌、⼝}
  \definition{s.}{lugar perigoso | cova do tigre}
\end{EntryWithPhonetic}

\begin{EntryWithPhonetic}{虎鼬}{hu3you4}{8,18}{⾌、⿏}
  \definition{s.}{doninha}
\end{EntryWithPhonetic}

\begin{EntryWithPhonetic}{唬}{hu3}{11}{⼝}
  \definition{v.}{blefar, exagerar para assustar ou confundir}
\end{EntryWithPhonetic}

\begin{EntryWithPhonetic}{互}{hu4}{4}{⼆}
  \definition{adv.}{mutuamente; um ao outro}
  \definition{pron.}{um ao outro; mútuo}
\end{EntryWithPhonetic}

\begin{EntryWithPhonetic}{互动}{hu4 dong4}{4,6}{⼆、⼒}[HSK 6]
  \definition{v.}{interagir; participar juntos; promover uns aos outros}
\end{EntryWithPhonetic}

\begin{EntryWithPhonetic}{互利}{hu4li4}{4,7}{⼆、⼑}
  \definition{s.}{benefício mútuo}
\end{EntryWithPhonetic}

\begin{EntryWithPhonetic}{互联网}{hu4 lian2 wang3}{4,12,6}{⼆、⽿、⽹}[HSK 3]
  \definition{s.}{\emph{Internet}; uma enorme rede conectando computadores e redes de computadores}
  \seealsoref{网际网路}{wang3 ji4 wang3 lu4}
  \seealsoref{网际网络}{wang3 ji4 wang3 luo4}
  \seealsoref{网路}{wang3 lu4}
\end{EntryWithPhonetic}

\begin{EntryWithPhonetic}{互相}{hu4xiang1}{4,9}{⼆、⽬}[HSK 3]
  \definition{adv.}{mutuamente; um ao outro; expressa uma relação de igualdade entre as partes}
\end{EntryWithPhonetic}

\begin{EntryWithPhonetic}{户}{hu4}{4}{⼾}[HSK 4][Kangxi 63]
  \definition*{s.}{Sobrenome Hu}
  \definition[个]{s.}{porta com um painel; porta | domicílio; residência; família | status familiar | conta (banco)}
\end{EntryWithPhonetic}

\begin{EntryWithPhonetic}{户外}{hu4 wai4}{4,5}{⼾、⼣}[HSK 6]
  \definition{s.}{ao ar livre; espaço aberto ao ar livre}
\end{EntryWithPhonetic}

\begin{EntryWithPhonetic}{护}{hu4}{7}{⼿}[HSK 6]
  \definition{v.}{proteger; defender | blindar; ser parcial; proteger-se da censura}
\end{EntryWithPhonetic}

\begin{EntryWithPhonetic}{护士}{hu4shi5}{7,3}{⼿、⼠}[HSK 4]
  \definition[名,位]{s.}{enfermeiro; pessoas especializadas em enfermagem em hospitais ou instituições epidemiológicas}
\end{EntryWithPhonetic}

\begin{EntryWithPhonetic}{护照}{hu4zhao4}{7,13}{⼿、⽕}[HSK 2]
  \definition[本,个]{s.}{passaporte; documento emitido pela autoridade competente do país para comprovar a nacionalidade e a identidade dos cidadãos que viajam para o exterior}
\end{EntryWithPhonetic}

\begin{EntryWithPhonetic}{糊}{hu4}{15}{⽶}
  \definition{s.}{pasta; comida que parece mingau}
  \seeref{hu2}
\end{EntryWithPhonetic}

\begin{EntryWithPhonetic}{化}{hua1}{4}{⼔}
  \variantof{花}
  \seeref{hua4}
\end{EntryWithPhonetic}

\begin{EntryWithPhonetic}{花}{hua1}{7}{⾋}[HSK 1,2,4]
  \definition*{s.}{Sobrenome Hua}
  \definition{adj.}{multicolorido; colorido | embaçado; obscuro; deslumbrado e confuso | extravagante; florido; vistoso}
  \definition[朵,支,束,把,盆,簇]{s.}{flor; órgãos de reprodução sexual de plantas com sementes | flor; planta ornamental |  qualquer coisa que se assemelhe a uma flor | fogos de artifício | padrão; design; design decorativo | flor; metáfora para a essência de uma causa | prostituta; cortesã; referindo-se a prostitutas ou a assuntos relacionados a prostitutas | algodão | varíola | ferimento; ferida; lesões traumáticas sofridas em combate}
  \definition{v.}{gastar; despender; consumir}
\end{EntryWithPhonetic}

\begin{EntryWithPhonetic}{花茶}{hua1cha2}{7,9}{⾋、⾋}
  \definition[杯,壶]{s.}{chá perfumado}
\end{EntryWithPhonetic}

\begin{EntryWithPhonetic}{花店}{hua1dian4}{7,8}{⾋、⼴}
  \definition{s.}{floricultura}
\end{EntryWithPhonetic}

\begin{EntryWithPhonetic}{花费}{hua1 fei4}{7,9}{⾋、⾙}[HSK 6]
  \definition[笔]{s.}{dinheiro gasto; despesas | custo; gastos; desembolso | despesa}
  \definition{v.}{gastar (tempo ou dinheiro)}
\end{EntryWithPhonetic}

\begin{EntryWithPhonetic}{花脸}{hua1lian3}{7,11}{⾋、⾁}
  \definition*{s.}{Hualian, personagem do rosto florido, um nome popular para 净 (assim chamado devido à elaborada pintura facial)}
  \seealsoref{净}{jing4}
\end{EntryWithPhonetic}

\begin{EntryWithPhonetic}{花瓶}{hua1 ping2}{7,10}{⾋、⽡}[HSK 6]
  \definition[个,对]{s.}{vaso de flores; vaso usado para arranjos florais colocado em ambientes internos como decoração | Figurativo: um ornamento; mulher empregada não por sua habilidade, mas por sua aparência; uma metáfora para uma pessoa ou coisa que é usada apenas para exibição e não tem uso prático}
\end{EntryWithPhonetic}

\begin{EntryWithPhonetic}{花儿}{hua1r5}{7,2}{⾋、⼉}
  \definition[朵,支,束,把,盆,簇]{s.}{flor}
\end{EntryWithPhonetic}

\begin{EntryWithPhonetic}{花生}{hua1sheng1}{7,5}{⾋、⽣}[HSK 6]
  \definition[把,颗,粒,袋]{s.}{amendoim}
  \seealsoref{落花生}{luo4 hua1 sheng1}
\end{EntryWithPhonetic}

\begin{EntryWithPhonetic}{花样游泳}{hua1yang4you2yong3}{7,10,12,8}{⾋、⽊、⽔、⽔}
  \definition{s.}{nado sincronizado}
\end{EntryWithPhonetic}

\begin{EntryWithPhonetic}{花椰菜}{hua1ye1cai4}{7,12,11}{⾋、⽊、⾋}
  \definition{s.}{couve-flor}
\end{EntryWithPhonetic}

\begin{EntryWithPhonetic}{花园}{hua1 yuan2}{7,7}{⾋、⼞}[HSK 2]
  \definition[个,座]{s.}{jardim; um local onde se plantam flores e árvores para passear e descansar}
\end{EntryWithPhonetic}

\begin{EntryWithPhonetic}{哗}{hua1}{9}{⼝}
  \definition{s.}{(onomatopéia) sons de impacto, batida, fluxo de água, etc.}
  \seeref{hua2}
\end{EntryWithPhonetic}

\begin{EntryWithPhonetic}{哗啦啦}{hua1la1 la5}{9,11,11}{⼝、⼝、⼝}
  \definition{s.}{(onomatopéia) som de colisão, batida}
\end{EntryWithPhonetic}

\begin{EntryWithPhonetic}{划}{hua2}{6}{⼑}[HSK 4]
  \definition{adj.}{rentável; vale (o esforço); compensa (fazer alguma coisa)}
  \definition{v.}{remar | ser vantajoso para alguém; ser uma pechincha | arranhar; cortar a superfície de; cortar em outra coisa com um objeto pontiagudo | arranhar; golpear;  esfregar uma coisa ou varrer sobre outra}
  \seeref{hua4}
\end{EntryWithPhonetic}

\begin{EntryWithPhonetic}{划船}{hua2 chuan2}{6,11}{⼑、⾈}[HSK 3]
  \definition[次,回]{s.}{remo (ato de remar); passeios de barco; a atividade ou esporte de “remar um barco com remos”}
  \definition{v.}{remar um barco; a ação ou comportamento de mover um barco na água usando remos}
\end{EntryWithPhonetic}

\begin{EntryWithPhonetic}{划艇}{hua2ting3}{6,12}{⼑、⾈}
  \definition{s.}{barco a remo}
\end{EntryWithPhonetic}

\begin{EntryWithPhonetic}{华}{hua2}{6}{⼗}
  \definition*{s.}{China; refere-se à China (anteriormente conhecida como Huaxia, 华夏, mais tarde chamada de Zhonghua, 中华, ou simplesmente Hua, 华)}
  \definition{adj.}{esplêndido; magnífico | próspero; florescente | chamativo; extravagante; vaidoso | grisalho}
  \definition{s.}{corona; um halo colorido ao redor do sol ou da lua causado pela difração da luz através das nuvens | creme; melhor parte; a melhor parte das coisas | chinês; refere-se à nacionalidade Han (língua e escrita) | vezes; anos; refere-se a (bons) momentos | elixir; essência líquida; substâncias formadas pela sedimentação de minerais na água de nascente | Seu, palavra honorífica, usada para se referir a coisas relacionadas à outra pessoa}
  \seeref{hua4}
  \seealsoref{华夏}{hua2xia4}
  \seealsoref{中华}{zhong1 hua2}
\end{EntryWithPhonetic}

\begin{EntryWithPhonetic}{华人}{hua2 ren2}{6,2}{⼗、⼈}[HSK 3]
  \definition[名,位,个]{s.}{Chinês; chinês étnico | chineses no exterior; refere-se a cidadãos estrangeiros de ascendência chinesa que obtiveram a nacionalidade do país em que residem}
\end{EntryWithPhonetic}

\begin{EntryWithPhonetic}{华盛顿}{hua2sheng4dun4}{6,11,10}{⼗、⽫、⾴}
  \definition*{s.}{Washington}
\end{EntryWithPhonetic}

\begin{EntryWithPhonetic}{华氏}{hua2shi4}{6,4}{⼗、⽒}
  \definition{s.}{graus Fahrenheit (°F)}
\end{EntryWithPhonetic}

\begin{EntryWithPhonetic}{华夏}{hua2xia4}{6,10}{⼗、⼢}
  \definition*{s.}{Huaxia, nome antigo da China | Catai}
\end{EntryWithPhonetic}

\begin{EntryWithPhonetic}{华裔}{hua2yi4}{6,13}{⼗、⾐}
  \definition{s.}{descendente de chinês}
\end{EntryWithPhonetic}

\begin{EntryWithPhonetic}{华语}{hua2 yu3}{6,9}{⼗、⾔}[HSK 5]
  \definition*{s.}{Chinês (idioma)}
\end{EntryWithPhonetic}

\begin{EntryWithPhonetic}{哗}{hua2}{9}{⼝}
  \definition{v.}{ser barulhento; fazer alvoroço}
  \seeref{hua1}
\end{EntryWithPhonetic}

\begin{EntryWithPhonetic}{滑}{hua2}{12}{⽔}[HSK 5]
  \definition*{s.}{Sobrenome Hua}
  \definition{adj.}{escorregadio; liso; objetos com superfícies lisas e baixo atrito | astuto; ardiloso; escorregadio}
  \definition{v.}{escorregar; deslizar | se atrapalhar; se safar de algo}
\end{EntryWithPhonetic}

\begin{EntryWithPhonetic}{滑雪}{hua2/xue3}{12,11}{⽔、⾬}
  \definition{v.+compl.}{esquiar | praticar esqui}
\end{EntryWithPhonetic}

\begin{EntryWithPhonetic}{化}{hua4}{4}{⼔}[HSK 3]
  \definition*{s.}{Sobrenome Hua}
  \definition{s.}{química | cultura; costumes e tradições}
  \definition{suf.}{modernizar; modernização; anexado a componentes nominais ou adjetivos para formar verbos, indicando a transformação em algum estado ou qualidade}
  \definition{v.}{mudar; converter; transformar; causasr mudanças | converter; influenciar; influenciar e induzir as pessoas com palavras e ações, levando-as a mudar | derreter; dissolver; fundir | digerir | queimar; reduzir a cinzas | (monge, taoísta) morrer | (de monges budistas ou sacerdotes taoístas) pedir esmolas; arrecadar bens, alimentos, etc.}
  \seeref{hua1}
\end{EntryWithPhonetic}

\begin{EntryWithPhonetic}{化合}{hua4he2}{4,6}{⼔、⼝}
  \definition{s.}{combinação química}
  \definition{s.}{Química: combinar; sintentizar}
\end{EntryWithPhonetic}

\begin{EntryWithPhonetic}{化解}{hua4 jie3}{4,13}{⼔、⾓}[HSK 6]
  \definition{v.}{resolver; eliminar; dissolver; neutralizar}
\end{EntryWithPhonetic}

\begin{EntryWithPhonetic}{化石}{hua4shi2}{4,5}{⼔、⽯}[HSK 5]
  \definition{s.}{fóssil; restos, relíquias ou vestígios de organismos antigos enterrados no solo e transformados em objetos semelhantes a pedras}
\end{EntryWithPhonetic}

\begin{EntryWithPhonetic}{化学}{hua4xue2}{4,8}{⼔、⼦}
  \definition[门]{s.}{química; a ciência que estuda a composição, estrutura, propriedades e leis de mudança da matéria | celuloide}
\end{EntryWithPhonetic}

\begin{EntryWithPhonetic}{划}{hua4}{6}{⼑}[HSK 4]
  \definition*{s.}{Sobrenome Hua}
  \definition{s.}{traço de um caracter chinês}
  \definition{v.}{delimitar; diferenciar; delinear | transferir; ceder | planejar; programar | desenhar; marcar; delinear; fazer linhas ou escrever como marcadores com uma caneta ou objeto semelhante a uma caneta}
  \seeref{hua2}
\end{EntryWithPhonetic}

\begin{EntryWithPhonetic}{划分}{hua4fen1}{6,4}{⼑、⼑}[HSK 5]
  \definition{v.}{dividir; particionar; reparticionar | diferenciar; encontrar aspectos diferentes}
\end{EntryWithPhonetic}

\begin{EntryWithPhonetic}{华}{hua4}{6}{⼗}
  \definition*{s.}{Huashan Mountain (na província de Shaanxi) | Sobrenome Hua}
  \seeref{hua2}
\end{EntryWithPhonetic}

\begin{EntryWithPhonetic}{画}{hua4}{8}{⽥}[HSK 2]
  \definition*{s.}{Sobrenome Hua}
  \definition{clas.}{traços (de um caractere chinês)}
  \definition[张,幅]{s.}{desenho; pintura; imagem; figura desenhada | traço horizontal (em caracteres chineses)}
  \definition{v.}{desenhar; pintar | desenhar; marcar; assinar}
  \seealsoref{划}{hua4}
\end{EntryWithPhonetic}

\begin{EntryWithPhonetic}{画地为牢}{hua4di4wei2lao2}{8,6,4,7}{⽥、⼟、⼂、⼧}
  \definition{expr.}{desenhar um círculo no chão para servir como uma prisão; restringir as atividades de alguém a uma área ou esfera designada; limitar; restringir | (literário) ser confinado dentro de um círculo desenhado no chão | (figurativo) limitar-se a uma gama restrita de atividades}
\end{EntryWithPhonetic}

\begin{EntryWithPhonetic}{画家}{hua4 jia1}{8,10}{⽥、⼧}[HSK 2]
  \definition[个,位,名,些]{s.}{pintor; pessoa com talento para pintura}
\end{EntryWithPhonetic}

\begin{EntryWithPhonetic}{画面}{hua4 mian4}{8,9}{⽥、⾯}[HSK 5]
  \definition[个,幅,帧]{s.}{quadro; aparência geral de uma imagem; imagem apresentada no quadro, na tela, etc.}
\end{EntryWithPhonetic}

\begin{EntryWithPhonetic}{画儿}{hua4r5}{8,2}{⽥、⼉}[HSK 2]
  \definition[幅,张]{s.}{imagem; desenho; pintura; obra de arte pintada}
\end{EntryWithPhonetic}

\begin{EntryWithPhonetic}{话}{hua4}{8}{⾔}[HSK 1]
  \definition[句,段,番,种]{s.}{palavra; conversa; a voz que expressa os pensamentos quando falada, ou os caracteres que registram essa voz}
  \definition{v.}{falar sobre; falar a respeito}
\end{EntryWithPhonetic}

\begin{EntryWithPhonetic}{话剧}{hua4 ju4}{8,10}{⾔、⼑}[HSK 3]
  \definition[场,幕,部,出,台]{s.}{drama moderno; peça de teatro; peça teatral representada através de diálogos e ações}
\end{EntryWithPhonetic}

\begin{EntryWithPhonetic}{话题}{hua4ti2}{8,15}{⾔、⾴}[HSK 3]
  \definition[个,种,项]{s.}{assunto de uma palestra; tópico de uma conversa; o foco da conversa}
\end{EntryWithPhonetic}

\begin{EntryWithPhonetic}{怀}{huai2}{7}{⼼}
  \definition*{s.}{Sobrenome Huai}
  \definition{s.}{seio; peito | mente}
  \definition{v.}{manter em mente; estimar; abrigar | sentir falta; pensar em; ansiar por | conceber (uma criança)}
\end{EntryWithPhonetic}

\begin{EntryWithPhonetic}{怀旧}{huai2jiu4}{7,5}{⼼、⽇}
  \definition{s.}{nostalgia}
  \definition{v.}{sentir-se nostálgico}
\end{EntryWithPhonetic}

\begin{EntryWithPhonetic}{怀念}{huai2nian4}{7,8}{⼼、⼼}[HSK 4]
  \definition{v.}{pensar em; valorizar a memória de}
\end{EntryWithPhonetic}

\begin{EntryWithPhonetic}{怀疑}{huai2yi2}{7,14}{⼼、⽦}[HSK 4]
  \definition{v.}{duvidar; suspeitar | supor}
\end{EntryWithPhonetic}

\begin{EntryWithPhonetic}{坏}{huai4}{7}{⼟}[HSK 1]
  \definition{adj.}{ruim; prejudicial; insatisfatório; péssimo | mal; extremamente; indica um grau profundo, geralmente usado após verbos ou adjetivos que expressam estado psicológico | podre; estragado; impróprio; prejudicial ao uso}
  \definition[种]{s.}{ideia maligna; truque sujo; péssima ideia}
  \definition{v.}{estragar; destruir; corromper}
\end{EntryWithPhonetic}

\begin{EntryWithPhonetic}{坏处}{huai4 chu4}{7,5}{⼟、⼡}[HSK 2]
  \definition[个]{s.}{dano; prejuízo; desvantagem; fatores prejudiciais a pessoas ou coisas}
\end{EntryWithPhonetic}

\begin{EntryWithPhonetic}{坏蛋}{huai4dan4}{7,11}{⼟、⾍}
  \definition{s.}{bastardo | canalha | pessoa má}
\end{EntryWithPhonetic}

\begin{EntryWithPhonetic}{坏人}{huai4 ren2}{7,2}{⼟、⼈}[HSK 2]
  \definition[个,种]{s.}{malfeitor; canalha; pessoa má; pessoa de má qualidade; pessoa que faz coisas ruins}
\end{EntryWithPhonetic}

\begin{EntryWithPhonetic}{欢}{huan1}{6}{⽋}
  \definition*{s.}{Sobrenome Huan}
  \definition{adj.}{alegre; feliz; jubilante | vigoroso; energético; em pleno andamento; com grande impulso}
  \definition{s.}{amante; querida; um apelido usado por mulheres nos tempos antigos para se referir aos seus amantes; agora, geralmente se refere a alguém de quem você gosta ou com quem tem um relacionamento romântico}
\end{EntryWithPhonetic}

\begin{EntryWithPhonetic}{欢快}{huan1kuai4}{6,7}{⽋、⼼}
  \definition{adj.}{feliz e sem ansiedade | vívido}
\end{EntryWithPhonetic}

\begin{EntryWithPhonetic}{欢乐}{huan1le4}{6,5}{⽋、⼃}[HSK 3]
  \definition{adj.}{feliz; alegre; felicidade (geralmente coletiva)}
\end{EntryWithPhonetic}

\begin{EntryWithPhonetic}{欢迎}{huan1ying2}{6,7}{⽋、⾡}[HSK 2]
  \definition{adj.}{bem-vindo}
  \definition{v.}{dar as boas-vindas; cumprimentar; receber com alegria | dar as boas-vindas; receber favoravelmente (bem)}
\end{EntryWithPhonetic}

\begin{EntryWithPhonetic}{还}{huan2}{7}{⾡}[HSK 1]
  \definition*{s.}{Sobrenome Huan}
  \definition{v.}{voltar; retornar; voltar ao lugar original ou restaurar o estado original | retribuir; devolver; reembolsar; devolver o dinheiro ou os bens emprestados ao seu proprietário | dar ou fazer algo em troca; retribuir as ações dos outros}
  \seeref{hai2}
\end{EntryWithPhonetic}

\begin{EntryWithPhonetic}{环}{huan2}{8}{⽟}[HSK 3]
  \definition*{s.}{Sobrenome Huan}
  \definition{clas.}{usado para anéis}
  \definition[个,串]{s.}{anel; arco | elo; \emph{link}; passo; etapa | anel; objeto em forma de círculo | arredores}
  \definition{v.}{cercar; rodear; circular; circundar}
\end{EntryWithPhonetic}

\begin{EntryWithPhonetic}{环保}{huan2 bao3}{8,9}{⽟、⼈}[HSK 3]
  \definition{adj.}{ecológico; benefício para o meio ambiente; não prejudica o meio ambiente}
  \definition{s.}{proteção ambiental}
\end{EntryWithPhonetic}

\begin{EntryWithPhonetic}{环节}{huan2jie2}{8,5}{⽟、⾋}[HSK 5]
  \definition[个]{s.}{\emph{link}; ligação; vínculo; uma das muitas coisas que estão inter-relacionadas | segmento; estrutura anelar de alguns animais inferiores}
\end{EntryWithPhonetic}

\begin{EntryWithPhonetic}{环境}{huan2jing4}{8,14}{⽟、⼟}[HSK 3]
  \definition[个]{s.}{ambiente; os arredores | arredores; circunstâncias; condições políticas, econômicas, culturais, etc., dentro de um determinado âmbito}
\end{EntryWithPhonetic}

\begin{EntryWithPhonetic}{环境卫生}{huan2jing4wei4sheng1}{8,14,3,5}{⽟、⼟、⼙、⽣}
  \definition{s.}{saneamento ambiental}
  \seealsoref{环卫}{huan2wei4}
\end{EntryWithPhonetic}

\begin{EntryWithPhonetic}{环卫}{huan2wei4}{8,3}{⽟、⼙}
  \definition{s.}{limpeza pública; saneamento ambiental; saneamento geral; abreviação de 环境卫生 | Arcaico: guardas imperiais; guardas}
  \seealsoref{环境卫生}{huan2jing4wei4sheng1}
\end{EntryWithPhonetic}

\begin{EntryWithPhonetic}{缓}{huan3}{12}{⽶}
  \definition{adj.}{lento; sem pressa | sem tensão; relaxado}
  \definition{v.}{atrasar; adiar; protelar | recuperar; reviver; voltar a si}
\end{EntryWithPhonetic}

\begin{EntryWithPhonetic}{缓解}{huan3jie3}{12,13}{⽶、⾓}[HSK 4]
  \definition{v.}{facilitar; aliviar; atenuar; amenizar; reduzir}
\end{EntryWithPhonetic}

\begin{EntryWithPhonetic}{幻}{huan4}{4}{⼳}
  \definition{adj.}{irreal; imaginário; ilusório | mágico; mutável}
  \definition{v.}{mudar magicamente}
\end{EntryWithPhonetic}

\begin{EntryWithPhonetic}{幻觉}{huan4jue2}{4,9}{⼳、⾒}
  \definition{s.}{ilusão | alucinação}
\end{EntryWithPhonetic}

\begin{EntryWithPhonetic}{幻想}{huan4xiang3}{4,13}{⼳、⼼}[HSK 6]
  \definition[个,种]{s.}{fantasia; visão; arco-íris; ilusão; fruto da imaginação de alguém; imaginar algo que é difícil ou impossível de alcançar}
  \definition{v.}{imaginar; fantasiar; imaginar coisas que ainda não foram realizadas com base em ideais e desejos sociais ou pessoais}
\end{EntryWithPhonetic}

\begin{EntryWithPhonetic}{换}{huan4}{10}{⼿}[HSK 2]
  \definition{v.}{negociar; trocar; permutar; dar algo a alguém e, ao mesmo tempo, obter algo dele em troca | mudar; transformar; substituir | trocar dinheiro (câmbio) | transfundir (sangue) | transplantar (um órgão)}
\end{EntryWithPhonetic}

\begin{EntryWithPhonetic}{换钱}{huan4/qian2}{10,10}{⼿、⾦}
  \definition{v.+compl.}{trocar dinheiro (em pequenas valores ou em outra moeda) | trocar (mercadorias) por dinheiro | vender}
\end{EntryWithPhonetic}

\begin{EntryWithPhonetic}{患}{huan4}{11}{⼼}
  \definition*{s.}{Sobrenome Huan}
  \definition{s.}{perigo; problema; desastre; flagelo | preocupação; ansiedade}
  \definition{v.}{contrair (doença); sofrer de}
\end{EntryWithPhonetic}

\begin{EntryWithPhonetic}{患者}{huan4zhe3}{11,8}{⼼、⽼}[HSK 6]
  \definition[个,位,名]{s.}{paciente; sofredor; pessoas com certas doenças}
\end{EntryWithPhonetic}

\begin{EntryWithPhonetic}{荒}{huang1}{9}{⾋}
  \definition*{s.}{Sobrenome Huang}
  \definition{adj.}{(terra) não utilizada; não cultivada | desolado; estéril | irracional; delirante; fantástico; absurdo | incerto; duvidoso | dissoluto; autoindulgente | grosseiramente processado; bruto}
  \definition[片,块]{s.}{terra devastada; terra inculta; deserto | fome; quebra de safra | escassez | lixo; restos | terra selvagem (floresta)}
  \definition{v.}{(coloquial) negligenciar; estar fora de prática}
\end{EntryWithPhonetic}

\begin{EntryWithPhonetic}{荒芜}{huang1wu2}{9,7}{⾋、⾋}
  \definition{adj.}{estéril}
\end{EntryWithPhonetic}

\begin{EntryWithPhonetic}{慌}{huang1}{12}{⼼}[HSK 5]
  \definition{adj.}{agitado; perturbado; confuso; que inspira terror}
  \definition{v.}{estar em estado de pânico; ficar com medo; ficar nervoso | estar com pressa}
\end{EntryWithPhonetic}

\begin{EntryWithPhonetic}{慌忙}{huang1 mang2}{12,6}{⼼、⼼}[HSK 5]
  \definition{adj.}{apressado; afobado; com muita pressa}
  \definition{adv.}{apressadamente}
\end{EntryWithPhonetic}

\begin{EntryWithPhonetic}{皇}{huang2}{9}{⽩}
  \definition*{s.}{Sobrenome Huang}
  \definition{adj.}{grandioso; magnífico}
  \definition{s.}{imperador, o governante supremo de uma dinastia feudal após a Dinastia Qin; soberano}
\end{EntryWithPhonetic}

\begin{EntryWithPhonetic}{皇帝}{huang2di4}{9,9}{⽩、⼱}[HSK 6]
  \definition[个,位,任]{s.}{imperador; o título do mais alto governante feudal na China começou com o título de Imperador Qin Shi Huang}
\end{EntryWithPhonetic}

\begin{EntryWithPhonetic}{黄}{huang2}{11}{⿈}[HSK 2][Kangxi 201]
  \definition*{s.}{Rio Huanghe, abreviação de 黄河 | Refere-se ao Imperador Amarelo, um imperador da mitologia chinesa antiga | Sobrenome Huang ou Hwang}
  \definition{adj.}{amarelo | obsceno; indecente; pornográfico; símbolo de corrupção e decadência, referindo-se especificamente à pornografia}
  \definition{s.}{gema; ovas de caranguejo; refere-se a certas coisas de cor amarela}
  \definition{v.}{fracassar; dar errado}
  \seealsoref{黄河}{huang2he2}
\end{EntryWithPhonetic}

\begin{EntryWithPhonetic}{黄瓜}{huang2 gua1}{11,5}{⿈、⽠}[HSK 4]
  \definition[根,棵,株,条]{s.}{pepino}
\end{EntryWithPhonetic}

\begin{EntryWithPhonetic}{黄河}{huang2he2}{11,8}{⿈、⽔}
  \definition*{s.}{Rio Amarelo | Rio Huang He}
\end{EntryWithPhonetic}

\begin{EntryWithPhonetic}{黄昏}{huang2hun1}{11,8}{⿈、⽇}
  \definition{s.}{anoitecer}
\end{EntryWithPhonetic}

\begin{EntryWithPhonetic}{黄金}{huang2jin1}{11,8}{⿈、⾦}[HSK 4]
  \definition{adj.}{de primeira qualidade; dourado;}
  \definition[块,克,两]{s.}{ouro; \emph{aurum}; um tipo de metal, de cor amarela, mais precioso, abreviação de 金, frequentemente falado como 金子}
  \seealsoref{金}{jin1}
  \seealsoref{金子}{jin1zi5}
\end{EntryWithPhonetic}

\begin{EntryWithPhonetic}{黄色}{huang2 se4}{11,6}{⿈、⾊}[HSK 2]
  \definition{adj.}{decadente; obsceno; erótico; pornográfico; símbolo de corrupção e decadência, referindo-se especificamente à pornografia}
  \definition[种]{s.}{cor amarela}
\end{EntryWithPhonetic}

\begin{EntryWithPhonetic}{黄油}{huang2you2}{11,8}{⿈、⽔}
  \definition[盒]{s.}{manteiga}
\end{EntryWithPhonetic}

\begin{EntryWithPhonetic}{惶}{huang2}{12}{⼼}
  \definition{adj.}{cheio de medo; assustado}
  \definition{s.}{medo; pânico}
  \definition{v.}{temer}
\end{EntryWithPhonetic}

\begin{EntryWithPhonetic}{惶恐}{huang2kong3}{12,10}{⼼、⼼}
  \definition{adj.}{aterrorizado; em pânico; petrificado | inquieto; apreensivo}
\end{EntryWithPhonetic}

\begin{EntryWithPhonetic}{谎}{huang3}{11}{⾔}
  \definition[句]{s.}{mentira; falsidade}
  \definition{v.}{contar uma mentira; mentir}
\end{EntryWithPhonetic}

\begin{EntryWithPhonetic}{谎话}{huang3hua4}{11,8}{⾔、⾔}
  \definition{s.}{mentira}
\end{EntryWithPhonetic}

\begin{EntryWithPhonetic}{灰}{hui1}{6}{⽕}
  \definition{adj.}{cinza (cor) | desanimado; desencorajado; deprimido}
  \definition[把,堆]{s.}{cinzas; pó que sobra após a queima de um objeto | pó; poeira; substância em pó | cal; argamassa (de cal)}
\end{EntryWithPhonetic}

\begin{EntryWithPhonetic}{灰色}{hui1 se4}{6,6}{⽕、⾊}[HSK 5]
  \definition{adj.}{obscuro; ambíguo | sombrio; pessimista}
  \definition[种]{s.}{cor cinza; acinzentado}
\end{EntryWithPhonetic}

\begin{EntryWithPhonetic}{恢}{hui1}{9}{⼼}
  \definition{adj.}{extenso; vasto | grande; ótimo}
  \definition{v.}{recuperar; restaurar; restabelecer}
\end{EntryWithPhonetic}

\begin{EntryWithPhonetic}{恢复}{hui1fu4}{9,9}{⼼、⼢}[HSK 5]
  \definition{v.}{retomar; renovar; restaurar; voltar a | reviver; recuperar; reencontrar | restaurar; restabelecer; reabilitar; regenerar; ressurgir; restabelecer alguém em; recuperar o que foi perdido}
\end{EntryWithPhonetic}

\begin{EntryWithPhonetic}{挥}{hui1}{9}{⼿}
  \definition{v.}{acenar; empunhar; socar | limpar lágrimas, suor, etc. com as mãos | comandar (um exército) | espalhar; dispersar | afastar-se; livrar-se de}
\end{EntryWithPhonetic}

\begin{EntryWithPhonetic}{挥汗如雨}{hui1han4ru2yu3}{9,6,6,8}{⼿、⽔、⼥、⾬}
  \definition{s.}{suor derramado}
  \definition{v.}{pingar com suor}
\end{EntryWithPhonetic}

\begin{EntryWithPhonetic}{囘}{hui2}{5}{⼞}
  \variantof{回}
\end{EntryWithPhonetic}

\begin{EntryWithPhonetic}{回}{hui2}{6}{⼞}[HSK 1,2]
  \definition*{s.}{Sobrenome Hui}
  \definition*{s.}{Etnia Hui (mulçumanos chineses)}
  \definition{clas.}{usado para coisas, ações, número de vezes |  um trecho de um conto; um capítulo de um romance em capítulos | seção ou capítulo (de um livro clássico)}
  \definition{v.}{circular; enrolar | retornar; voltar; voltar ao lugar de origem | dar meia-volta | responder; contestar | relatar; reportar; responder}
\end{EntryWithPhonetic}

\begin{EntryWithPhonetic}{回报}{hui2bao4}{6,7}{⼞、⼿}[HSK 5]
  \definition{s.}{recompensa; pagamento; benefícios recebidos como resultado de assistência, esforço ou afeto | retornos; benefícios recebidos por meio de investimentos}
  \definition{v.}{pagar de volta; beneficiar pessoas ou organizações que os ajudaram ou cuidaram deles de alguma forma}
\end{EntryWithPhonetic}

\begin{EntryWithPhonetic}{回避}{hui2bi4}{6,16}{⼞、⾌}
  \definition{v.}{fugir (de um problema); em direito, refere-se especificamente à não participação nos procedimentos de um caso de um oficial de justiça, etc., que tenha interesse no caso ou nas partes do caso | esquivar-se; evadir-se; evitar (encontrar alguém)}
\end{EntryWithPhonetic}

\begin{EntryWithPhonetic}{回答}{hui2da2}{6,12}{⼞、⽵}[HSK 1]
  \definition[个]{s.}{resposta}
  \definition{v.}{responder; explicar a questão; expressar opinião sobre a solicitação}
\end{EntryWithPhonetic}

\begin{EntryWithPhonetic}{回到}{hui2 dao4}{6,8}{⼞、⼑}[HSK 1]
  \definition{v.}{retornar para; voltar e chegar (ao lugar onde estava originalmente); (após uma mudança nas circunstâncias) retornar ao estado original}
\end{EntryWithPhonetic}

\begin{EntryWithPhonetic}{回复}{hui2 fu4}{6,9}{⼞、⼢}[HSK 4]
  \definition{v.}{responder (a uma carta) | retornar ao estado normal; restaurar algo ao seu estado original}
\end{EntryWithPhonetic}

\begin{EntryWithPhonetic}{回顾}{hui2gu4}{6,10}{⼞、⾴}[HSK 5]
  \definition{v.}{olhar para trás | revisar; fazer uma retrospectiva; olhar para trás, pensar no passado}
\end{EntryWithPhonetic}

\begin{EntryWithPhonetic}{回国}{hui2 guo2}{6,8}{⼞、⼞}[HSK 2]
  \definition{v.}{regressar ao seu país (terra natal); referindo-se a voltar do exterior}
\end{EntryWithPhonetic}

\begin{EntryWithPhonetic}{回家}{hui2 jia1}{6,10}{⼞、⼧}[HSK 1]
  \definition{v.}{ir (voltar) para casa; estar em casa; voltar para casa}
\end{EntryWithPhonetic}

\begin{EntryWithPhonetic}{回来}{hui2 lai5}{6,7}{⼞、⽊}[HSK 1]
  \definition{v.}{voltar; regressar (para a minha localização) | retornar; usado após um verbo, significa ``vir ao lugar original''}
\end{EntryWithPhonetic}

\begin{EntryWithPhonetic}{回去}{hui2 qu4}{6,5}{⼞、⼛}[HSK 1]
  \definition{v.}{retornar; voltar; estar de volta ; (a partir da minha localização)}
\end{EntryWithPhonetic}

\begin{EntryWithPhonetic}{回收}{hui2shou1}{6,6}{⼞、⽁}[HSK 5]
  \definition{v.}{reciclar; reciclar itens (geralmente resíduos ou produtos antigos) para reutilização | recuperar; recolher; recuperar o que foi emitido ou demitido}
\end{EntryWithPhonetic}

\begin{EntryWithPhonetic}{回头}{hui2 tou2}{6,5}{⼞、⼤}[HSK 5]
  \definition{adv.}{mais tarde; depois de um tempo}
  \definition{conj.}{ou então; usado no início da segunda metade de uma frase para indicar o que acontecerá se você não fizer o que fez na primeira metade da frase}
  \definition{v.}{dar a meia-volta; virar a cabeça; virar a cabeça para trás | retornar; voltar | arrepender-se; corrigir seu caminho; reconhecer e corrigir erros}
\end{EntryWithPhonetic}

\begin{EntryWithPhonetic}{回信}{hui2/xin4}{6,9}{⼞、⼈}[HSK 5]
  \definition[封]{s.}{uma carta em resposta; uma mensagem verbal em resposta}
  \definition{v.+compl.}{escrever em resposta; escrever de volta; responder uma carta; responder verbalmente uma mensagem}
\end{EntryWithPhonetic}

\begin{EntryWithPhonetic}{回旋}{hui2xuan2}{6,11}{⼞、⽅}
  \definition{v.}{circular | rodar | dar a volta}
\end{EntryWithPhonetic}

\begin{EntryWithPhonetic}{回忆}{hui2yi4}{6,4}{⼞、⼼}[HSK 5]
  \definition[个,段]{s.}{memória; lembrança de eventos ou experiências passadas}
  \definition{v.}{lembrar; recordar}
\end{EntryWithPhonetic}

\begin{EntryWithPhonetic}{回应}{hui2 ying4}{6,7}{⼞、⼴}[HSK 6]
  \definition{v.}{responder}
\end{EntryWithPhonetic}

\begin{EntryWithPhonetic}{廻}{hui2}{8}{⼵}
  \variantof{回}
\end{EntryWithPhonetic}

\begin{EntryWithPhonetic}{毁}{hui3}{13}{⽎}[HSK 6]
  \definition{v.}{destruir; arruinar; danificar | (dialeto)  transformar, remodelar um item antigo em outra coisa, geralmente roupas | queimar | difamar; caluniar}
\end{EntryWithPhonetic}

\begin{EntryWithPhonetic}{汇}{hui4}{5}{⽔}[HSK 4]
  \definition{s.}{montagem; coleção; coisas coletadas}
  \definition{v.}{convergir | reunir; coletar | remeter | trocar (câmbio de moedas)}
\end{EntryWithPhonetic}

\begin{EntryWithPhonetic}{汇报}{hui4bao4}{5,7}{⽔、⼿}[HSK 4]
  \definition[份,次]{s.}{relatório; referindo-se ao conteúdo de declarações escritas ou orais feitas a um superior ou pessoa relevante para apresentar uma situação ou refletir um problema}
  \definition{v.}{relatar; fazer um relato de}
\end{EntryWithPhonetic}

\begin{EntryWithPhonetic}{汇款}{hui4/kuan3}{5,12}{⽔、⽋}[HSK 5]
  \definition[笔,个]{s.}{remessa; dinheiro enviado ou recebido}
  \definition{v.+compl.}{remeter dinheiro; fazer uma remessa; enviar dinheiro}
\end{EntryWithPhonetic}

\begin{EntryWithPhonetic}{汇率}{hui4lv4}{5,11}{⽔、⽞}[HSK 4]
  \definition[个,种]{s.}{taxa de câmbio; relação entre a moeda de um país e a de outro}
\end{EntryWithPhonetic}

\begin{EntryWithPhonetic}{会}{hui4}{6}{⼈}[HSK 1,2]
  \definition{adv.}{um momento}
  \definition{clas.}{momento; um curto período de tempo}
  \definition{s.}{reunião; festa; conferência; reunião com um objetivo específico | reunião; reunião no trabalho | feira do templo; festival religioso | associação; sociedade; sindicato; certas organizações públicas | oportunidade; ocasião; momento oportuno | cidade principal; capital; cidade central}
  \definition{suf.}{união; grupo; associação}
  \definition{v.}{ser provável que; ter certeza de; indica que é possível realizar (é possível responder à pergunta separadamente) |  poder; ser capaz de; significa saber como fazer ou ter a capacidade de fazer (geralmente se refere a coisas que precisam ser aprendidas) | saber; compreender; entender | encontrar; ver | reunir-se; reunir; agregar; juntar | destacar-se em; ser bom em; ser hábil em; indica proficiência | pagar (ou custear) contas}
  \seeref{kuai4}
\end{EntryWithPhonetic}

\begin{EntryWithPhonetic}{会见}{hui4 jian4}{6,4}{⼈、⾒}[HSK 6]
  \definition{v.}{entrevistar; encontrar-se com (especialmente um visitante estrangeiro)}
\end{EntryWithPhonetic}

\begin{EntryWithPhonetic}{会首}{hui4shou3}{6,9}{⼈、⾸}
  \definition{s.}{chefe de uma sociedade | patrocinador de uma organização}
\end{EntryWithPhonetic}

\begin{EntryWithPhonetic}{会谈}{hui4 tan2}{6,10}{⼈、⾔}[HSK 5]
  \definition{v.}{manter conversações; comumente usado em assuntos internacionais ou atividades diplomáticas}
\end{EntryWithPhonetic}

\begin{EntryWithPhonetic}{会议}{hui4yi4}{6,5}{⼈、⾔}[HSK 3]
  \definition[次,届,个,场]{s.}{reunião; conferência; reunião organizada pela organização relevante para ouvir opiniões, discutir questões e distribuir tarefas | conselho; congresso; um órgão ou organização permanente que discute e trata frequentemente assuntos importantes}
\end{EntryWithPhonetic}

\begin{EntryWithPhonetic}{会员}{hui4 yuan2}{6,7}{⼈、⼝}[HSK 3]
  \definition[位,名,个,些]{s.}{membro; associado; membros de certos grupos ou organizações}
\end{EntryWithPhonetic}

\begin{EntryWithPhonetic}{会长}{hui4 zhang3}{6,4}{⼈、⾧}[HSK 6]
  \definition[位,名,个,些]{s.}{presidente de uma associação ou sociedade | presidente de um clube, comitê etc.}
\end{EntryWithPhonetic}

\begin{EntryWithPhonetic}{绘}{hui4}{9}{⽷}
  \definition{v.}{pintar; desenhar}
\end{EntryWithPhonetic}

\begin{EntryWithPhonetic}{绘画}{hui4 hua4}{9,8}{⽷、⽥}[HSK 6]
  \definition{s.}{desenho; pintura}
  \definition{v.}{desenhar; pintar}
\end{EntryWithPhonetic}

\begin{EntryWithPhonetic}{昏}{hun1}{8}{⽇}
  \definition*{s.}{Sobrenome Hun}
  \definition{adj.}{escuro; fraco; embaçado | confuso; embaraçado; inconsciente}
  \definition{s.}{crepúsculo; tarde}
  \definition{v.}{perder a consciência; desmaiar}
\end{EntryWithPhonetic}

\begin{EntryWithPhonetic}{荤}{hun1}{9}{⾋}[Kangxi 9]
  \definition{adj.}{obsceno; lascivo; vulgar}
  \definition{s.}{carne ou peixe (oposto a 素) | Budismo: vegetais picantes proibidos aos vegetarianos budistas, como cebola, alho-poró, alho, etc. | alimentos não vegetarianos (carne, peixe etc.) | vegetais com cheiro forte (alho etc.)}
  \seealsoref{素}{su4}
\end{EntryWithPhonetic}

\begin{EntryWithPhonetic}{婚}{hun1}{11}{⼥}
  \definition{s.}{casamento}
  \definition{v.}{casar}
\end{EntryWithPhonetic}

\begin{EntryWithPhonetic}{婚礼}{hun1li3}{11,5}{⼥、⽰}[HSK 4]
  \definition[场]{s.}{casamento; núpcias; cerimônia de casamento}
\end{EntryWithPhonetic}

\begin{EntryWithPhonetic}{魂}{hun2}{13}{⿁}
  \definition[个]{s.}{alma | humor; espírito | espírito elevado de uma nação}
\end{EntryWithPhonetic}

\begin{EntryWithPhonetic}{混}{hun4}{11}{⽔}[HSK 6]
  \definition{adj.}{confuso; imundo; turvo; lamacento; impuro}
  \definition{adv.}{de forma imprudente; irresponsável; irrefletidamente}
  \definition{v.}{misturar; confundir; misturar verdadeiro e falso | passar por; esgueirar-se | vagar à deriva; arrastar-se; sobreviver de maneira superficial; contentar-se com | se dar bem com alguém}
\end{EntryWithPhonetic}

\begin{EntryWithPhonetic}{混饭}{hun4/fan4}{11,7}{⽔、⾷}
  \definition{v.+compl.}{trabalhar para viver}
\end{EntryWithPhonetic}

\begin{EntryWithPhonetic}{混合}{hun4he2}{11,6}{⽔、⼝}[HSK 6]
  \definition{s.}{híbrido; composto; refere-se a duas ou mais substâncias misturadas sem reação química, mas ainda mantendo suas respectivas propriedades (diferente de 化合)}
  \definition{v.}{misturar; mixar; misturar-se}
  \seealsoref{化合}{hua4he2}
\end{EntryWithPhonetic}

\begin{EntryWithPhonetic}{混乱}{hun4luan4}{11,7}{⽔、⼄}[HSK 6]
  \definition{adj.}{caótico; confuso; desordenado; desorganizado; fora de ordem}
  \definition[片]{s.}{caos; confusão}
\end{EntryWithPhonetic}

\begin{EntryWithPhonetic}{和}{huo2}{8}{⼝}
  \definition{v.}{combinar uma substância em pó (farinha, gesso, etc.) com água; adicionar líquido ao pó e mexer ou amassar até ficar pegajoso ou espesso}
  \seeref{he2}
  \seeref{he4}
  \seeref{hu2}
  \seeref{huo4}
\end{EntryWithPhonetic}

\begin{EntryWithPhonetic}{活}{huo2}{9}{⽔}[HSK 3]
  \definition{adj.}{vivo; vivendo; indica que (alguma ação) foi realizada enquanto a pessoa ainda estava viva | vívido; animado; ativo | móvel; em movimento; ativo}
  \definition{adv.}{exatamente; simplesmente; expressa um grau elevado, equivalente a 真正 ou 简直}
  \definition{s.}{emprego; meios de subsistência; trabalho (geralmente refere-se a trabalho físico) | produto; algo fabricado}
  \definition{v.}{viver; ter vida; sobreviver (em oposição a 死) | salvar (a vida de uma pessoa); fazer sobreviver; manter a vida}
  \seealsoref{简直}{jian3zhi2}
  \seealsoref{死}{si3}
  \seealsoref{真正}{zhen1zheng4}
\end{EntryWithPhonetic}

\begin{EntryWithPhonetic}{活动}{huo2dong4}{9,6}{⽔、⼒}[HSK 2]
  \definition{adj.}{móvel; flexível para alterações ou mudanças}
  \definition[些,个,种,类,次]{s.}{atividade; ação tomada com o objetivo de alcançar um determinado objetivo}
  \definition{v.}{fazer exercício; movimentar-se | usar influência pessoal; usar meios irregulares | mover-se}
\end{EntryWithPhonetic}

\begin{EntryWithPhonetic}{活力}{huo2li4}{9,2}{⽔、⼒}[HSK 5]
  \definition{s.}{vigor; vitalidade; energia; muito forte, geralmente usado para descrever pessoas, cidades, empresas, economias, etc.}
\end{EntryWithPhonetic}

\begin{EntryWithPhonetic}{活路}{huo2lu4}{9,13}{⽔、⾜}
  \definition{s.}{maneira de sobreviver | meio de subsistência}
  \seeref{huo2lu5}
\end{EntryWithPhonetic}

\begin{EntryWithPhonetic}{活路}{huo2lu5}{9,13}{⽔、⾜}
  \definition{s.}{labor | trabalho físico}
  \seeref{huo2lu4}
\end{EntryWithPhonetic}

\begin{EntryWithPhonetic}{活泼}{huo2po1}{9,8}{⽔、⽔}[HSK 5]
  \definition{adj.}{vívido; ativo; animado; brilhante; vivaz; cheio de vida | Química: reativo; significa que a substância é ativa e reage facilmente com outras substâncias}
\end{EntryWithPhonetic}

\begin{EntryWithPhonetic}{活跃}{huo2yue4}{9,11}{⽔、⾜}[HSK 6]
  \definition{adj.}{ativo; dinâmico; pensamentos, ações ou atividades positivas; ocorrências frequentes | rápido; ativo; dinâmico}
  \definition{v.}{animar; tornar ativo | ser ativo}
\end{EntryWithPhonetic}

\begin{EntryWithPhonetic}{活着}{huo2zhe5}{9,11}{⽔、⽬}
  \definition{adj.}{vivo}
\end{EntryWithPhonetic}

\begin{EntryWithPhonetic}{火}{huo3}{4}{⽕}[HSK 3,4][Kangxi 86]
  \definition*{s.}{Sobrenome Huo}
  \definition{adj.}{ardente; flamejante; vermelho como fogo | efervescente; próspero}
  \definition{adv.}{urgentemente}
  \definition{clas.}{usado para unidades militares (antigo)}
  \definition[场,把,团,堆]{s.}{fogo; a luz e as chamas emitidas pela combustão de um objeto | fúria; metáfora para emoções agitadas, irritadas ou raivosas | calor interno (uma das seis causas de doenças) | armas de fogo e munições | a ação de lutar}
  \definition{v.}{ficar com raiva; perder a paciência}
\end{EntryWithPhonetic}

\begin{EntryWithPhonetic}{火柴}{huo3chai2}{4,10}{⽕、⽊}[HSK 5]
  \definition[根,盒,包]{s.}{fósforo (palito de fósforo); fósforo de segurança; iniciador de fogo feito de uma tira fina de madeira mergulhada em um composto de fósforo ou enxofre}
\end{EntryWithPhonetic}

\begin{EntryWithPhonetic}{火车}{huo3 che1}{4,4}{⽕、⾞}[HSK 1]
  \definition[个,列,节,班,趟]{s.}{trem; comboio}
\end{EntryWithPhonetic}

\begin{EntryWithPhonetic}{火车司机}{huo3che1 si1ji1}{4,4,5,6}{⽕、⾞、⼝、⽊}
  \definition{s.}{maquinista de trem}
\end{EntryWithPhonetic}

\begin{EntryWithPhonetic}{火海}{huo3hai3}{4,10}{⽕、⽔}
  \definition{s.}{um mar de chamas}
\end{EntryWithPhonetic}

\begin{EntryWithPhonetic}{火箭}{huo3jian4}{4,15}{⽕、⽵}[HSK 6]
  \definition[个,艘,发,枚]{s.}{foguete; uma aeronave de alta velocidade que utiliza força de reação para se impulsionar para a frente; é usado para lançar satélites, naves espaciais, etc.; também pode ser equipado com uma ogiva para fabricar um míssil}
\end{EntryWithPhonetic}

\begin{EntryWithPhonetic}{火腿}{huo3 tui3}{4,13}{⽕、⾁}[HSK 5]
  \definition[道,个]{s.}{presunto; as pernas de porco marinadas mais famosas são produzidas em Jinhua, na província de Zhejiang, e em Xuanwei, na província de Yunnan.}
\end{EntryWithPhonetic}

\begin{EntryWithPhonetic}{火灾}{huo3 zai1}{4,7}{⽕、⽕}[HSK 5]
  \definition[起,场]{s.}{fogo (como um desastre); conflagração; desastres causados por incêndios}
\end{EntryWithPhonetic}

\begin{EntryWithPhonetic}{伙}{huo3}{6}{⼈}[HSK 4]
  \definition{clas.}{grupo; multidão; banda}
  \definition{s.}{iguaria; alimentação; refeições | parceiro; companheiro | coletivo de colegas}
  \definition{v.}{combinar; unir}
\end{EntryWithPhonetic}

\begin{EntryWithPhonetic}{伙伴}{huo3ban4}{6,7}{⼈、⼈}[HSK 4]
  \definition[个,位,群]{s.}{parceiro; companheiro; antigo sistema militar de dez pessoas para uma fogueira, o chefe da fogueira, uma pessoa encarregada de cozinhar, com a fogueira é chamado de parceiro da fogueira, agora se refere à participação comum em uma determinada organização ou engajada em certas atividades}
\end{EntryWithPhonetic}

\begin{EntryWithPhonetic}{和}{huo4}{8}{⼝}
  \definition{clas.}{usado para enxágues de roupas | usado para fervuras de ervas medicinais}
  \definition{v.}{misturar (ingredientes); misturar pós ou grãos; misturar com água para obter uma consistência mais líquida}
  \seeref{he2}
  \seeref{he4}
  \seeref{hu2}
  \seeref{huo2}
\end{EntryWithPhonetic}

\begin{EntryWithPhonetic}{或}{huo4}{8}{⼽}[HSK 2]
  \definition{adv.}{talvez; possivelmente; provavelmente | (geralmente na forma negativa) um pouco; ligeiramente}
  \definition{conj.}{ou (indicando escolha); ou\dots ou\dots}
  \definition{pron.}{alguém; algumas pessoas; refere-se a alguém ou algo, equivalente a 有人 ou 有的}
  \seealsoref{有的}{you3 de5}
  \seealsoref{有人}{you3 ren2}
\end{EntryWithPhonetic}

\begin{EntryWithPhonetic}{或是}{huo4 shi4}{8,9}{⼽、⽇}[HSK 5]
  \definition{adv.}{um ou outro; o outro}
  \definition{conj.}{ou; às vezes, é apenas uma de duas coisas}
\end{EntryWithPhonetic}

\begin{EntryWithPhonetic}{或许}{huo4xu3}{8,6}{⼽、⾔}[HSK 4]
  \definition{adv.}{talvez; possivelmente; receio; não tenho certeza}
\end{EntryWithPhonetic}

\begin{EntryWithPhonetic}{或者}{huo4zhe3}{8,8}{⼽、⽼}[HSK 2]
  \definition{adv.}{talvez; possivelmente}
  \definition{conj.}{ou (usado em expressões afirmativas); ou\dots ou\dots; usado em frases narrativas para indicar uma relação de escolha | ou (usado para indicar equação); indica relação de equivalência, indicando que os objetos anterior e posterior são iguais}
\end{EntryWithPhonetic}

\begin{EntryWithPhonetic}{货}{huo4}{8}{⾙}[HSK 4]
  \definition{s.}{dinheiro; moeda | bens; mercadorias; \emph{commodity} | refere-se a uma pessoa com um certo mau caráter (usado como um insulto) | riqueza; fortuna; um termo geral para dinheiro, joias, tecidos, etc.}
  \definition{v.}{vender}
\end{EntryWithPhonetic}

\begin{EntryWithPhonetic}{货车}{huo4che1}{8,4}{⾙、⾞}
  \definition{s.}{caminhão | van | vagão de carga}
\end{EntryWithPhonetic}

\begin{EntryWithPhonetic}{获}{huo4}{10}{⾋}[HSK 4]
  \definition*{s.}{Sobrenome Huo}
  \definition{v.}{capturar; pegar | obter; ganhar; colher | colher; ceifar}
\end{EntryWithPhonetic}

\begin{EntryWithPhonetic}{获得}{huo4de2}{10,11}{⾋、⼻}[HSK 4]
  \definition{v.}{adquirir; ganhar; obter; alcançar}
\end{EntryWithPhonetic}

\begin{EntryWithPhonetic}{获奖}{huo4 jiang3}{10,9}{⾋、⼤}[HSK 4]
  \definition{v.}{ganhar prêmio; ser recompensado; ganhar um prêmio; receber um prêmio}
\end{EntryWithPhonetic}

\begin{EntryWithPhonetic}{获取}{huo4 qu3}{10,8}{⾋、⼜}[HSK 4]
  \definition{v.}{adquirir; obter; ganhar; colher}
\end{EntryWithPhonetic}

\begin{EntryWithPhonetic}{祸}{huo4}{11}{⽰}
  \definition[场]{s.}{infortúnio; desastre; calamidade (oposto de 福) | desgraça; catástrofe}
  \definition{v.}{trazer desastre; arruinar | causar problemas}
  \seealsoref{福}{fu2}
\end{EntryWithPhonetic}

\begin{EntryWithPhonetic}{惑}{huo4}{12}{⼼}
  \definition{v.}{ficar confuso; ficar perplexo | iludir; enganar; confundir}
\end{EntryWithPhonetic}

\begin{EntryWithPhonetic}{惑星}{huo4xing1}{12,9}{⼼、⽇}
  \definition{s.}{planeta}
  \seealsoref{行星}{xing2xing1}
\end{EntryWithPhonetic}

%%%%% EOF %%%%%


%%%%%%%%%%%%%%%%%%%%%%%%%%%%%%% Não existem palavras com pinyin iniciado em "I"
 %%%
%%% J
%%%

\section*{J}\addcontentsline{toc}{section}{J}

\begin{EntryWithPhonetic}{几}{ji1}{2}{⼏}[Kangxi 16]
  \definition{adv.}{quase; praticamente}
  \definition{s.}{uma mesa pequena}
  \seeref{ji3}
\end{EntryWithPhonetic}

\begin{EntryWithPhonetic}{几乎}{ji1hu1}{2,5}{⼏、⼃}[HSK 4]
  \definition{adv.}{quase; praticamente; próximo a | perto de; quase; à beira de}
\end{EntryWithPhonetic}

\begin{EntryWithPhonetic}{几率}{ji1lv4}{2,11}{⼏、⽞}
  \definition{s.}{probabilidade; um evento pode ou não ocorrer sob as mesmas condições, a grandeza que indica a possibilidade de ocorrência é chamada de probabilidade}
\end{EntryWithPhonetic}

\begin{EntryWithPhonetic}{机}{ji1}{6}{⽊}
  \definition*{s.}{Sobrenome Ji}
  \definition{adj.}{flexível; perspicaz; destreza; agilidade}
  \definition[台]{s.}{máquina; motor | avião; aeronave; aeroplano; refere-se especificamente a aeronaves | ponto crucial; os fatores-chave para a ocorrência e mudança das coisas | chance; ocasião; oportunidade; um momento crítico ou oportuno para o desenvolvimento e mudança das coisas | organismo; funções vitais dos organismos | besta; mecanismo de disparo de flechas de madeira em uma besta antiga | assuntos importantes; assuntos extremamente importantes e confidenciais | ideia; intenção}
\end{EntryWithPhonetic}

\begin{EntryWithPhonetic}{机场}{ji1chang3}{6,6}{⽊、⼟}[HSK 1]
  \definition[个,家,处,座]{s.}{aeródromo; campo de aviação; aeroporto; campo de voo}
\end{EntryWithPhonetic}

\begin{EntryWithPhonetic}{机动车}{ji1 dong4 che1}{6,6,4}{⽊、⼒、⾞}[HSK 6]
  \definition{s.}{veículo motorizado (oposto a 人力车) | veículo automotor; automóvel de passageiros: veículo comercial concebido e tecnicamente adequado para o transporte de passageiros e respetiva bagagem, incluindo o banco do condutor}
  \seealsoref{人力车}{ren2 li4 che1}
\end{EntryWithPhonetic}

\begin{EntryWithPhonetic}{机构}{ji1gou4}{6,8}{⽊、⽊}[HSK 4]
  \definition[所]{s.}{órgão; organização; instituição; instalações; aparelhamento; configuração | mecanismo; funcionamento interno de uma máquina ou unidade | estrutura interna de uma organização}
\end{EntryWithPhonetic}

\begin{EntryWithPhonetic}{机关}{ji1 guan1}{6,6}{⽊、⼋}[HSK 6]
  \definition{adj.}{operado por máquina | controlado mecanicamente}
  \definition[个]{s.}{engrenagem; mecanismo; Antigo: refere-se a certos dispositivos controlados mecanicamente; também se refere às peças de frenagem de dispositivos mecânicos | escritório; órgão; corpo; instituição | esquema; maquinação; estratagema; um plano cuidadoso e inteligente}
\end{EntryWithPhonetic}

\begin{EntryWithPhonetic}{机会}{ji1hui4}{6,6}{⽊、⼈}[HSK 2]
  \definition[个,次,种,些]{s.}{chance; oportunidade; momento favorável raro}
\end{EntryWithPhonetic}

\begin{EntryWithPhonetic}{机甲}{ji1jia3}{6,5}{⽊、⽥}
  \definition{s.}{\emph{mecha} (robôs operados pelo homem em mangá japonês)}
\end{EntryWithPhonetic}

\begin{EntryWithPhonetic}{机票}{ji1 piao4}{6,11}{⽊、⽰}[HSK 1]
  \definition[张]{s.}{passagem aérea; passagem de avião}
  \seealsoref{飞机票}{fei1ji1 piao4}
\end{EntryWithPhonetic}

\begin{EntryWithPhonetic}{机器}{ji1qi4}{6,16}{⽊、⼝}[HSK 3]
  \definition[台,部,个]{s.}{máquina; maquinário; motor; dispositivos e máquinas que são montados a partir de peças, podem funcionar, transformar energia ou produzir trabalho útil podem ser usados como ferramentas de produção, reduzindo a intensidade do trabalho humano e aumentando a produtividade | aparato; sistema político e econômico}
\end{EntryWithPhonetic}

\begin{EntryWithPhonetic}{机器人}{ji1 qi4 ren2}{6,16,2}{⽊、⼝、⼈}[HSK 5]
  \definition[个,些]{s.}{androide; golem | pessoa mecânica | robô}
\end{EntryWithPhonetic}

\begin{EntryWithPhonetic}{机械}{ji1xie4}{6,11}{⽊、⽊}[HSK 6]
  \definition{adj.}{rígido; mecânico; inflexível; uma metáfora para uma abordagem rígida e imutável}
  \definition[台,部,个]{s.}{máquina; maquinário; mecanismo; vários dispositivos compostos por princípios mecânicos}
\end{EntryWithPhonetic}

\begin{EntryWithPhonetic}{机遇}{ji1yu4}{6,12}{⽊、⾡}[HSK 4]
  \definition[个]{s.}{chance; oportunidade; circunstâncias favoráveis}
\end{EntryWithPhonetic}

\begin{EntryWithPhonetic}{机制}{ji1 zhi4}{6,8}{⽊、⼑}[HSK 5]
  \definition{s.}{mecanismo; processado por máquina; feito por máquina}
\end{EntryWithPhonetic}

\begin{EntryWithPhonetic}{肌}{ji1}{6}{⾁}
  \definition[块,片]{s.}{músculo; carne | pele;}
\end{EntryWithPhonetic}

\begin{EntryWithPhonetic}{肌肉}{ji1rou4}{6,6}{⾁、⾁}[HSK 5]
  \definition[身,块]{s.}{músculo; um dos tecidos básicos dos músculos humanos e de alguns animais, composto principalmente de células musculares fibrosas, pode se contrair, é o movimento do corpo e o corpo de digestão, respiração, circulação, excreção e outros processos fisiológicos da fonte de energia; pode ser dividido em três tipos: músculo liso, músculo esquelético e músculo cardíaco}
\end{EntryWithPhonetic}

\begin{EntryWithPhonetic}{鸡}{ji1}{7}{⿃}[HSK 2]
  \definition*{s.}{Sobrenome Ji}
  \definition[只]{s.}{galo, galinha, frango | palavra ofensiva para uma mulher que ganha dinheiro fazendo sexo com homens}
\end{EntryWithPhonetic}

\begin{EntryWithPhonetic}{鸡蛋}{ji1dan4}{7,11}{⿃、⾍}[HSK 1]
  \definition[个,枚,筐,箱,打]{s.}{ovo de galinha}
\end{EntryWithPhonetic}

\begin{EntryWithPhonetic}{积}{ji1}{10}{⽲}
  \definition{adj.}{de longa data; pendente há muito tempo | antiquíssimo; acumulado ao longo de um longo período de tempo}
  \definition{s.}{(medicina chinesa) indigestão (em bebês e crianças) | (matemática)  abreviação de produto, 乘积}
  \definition{v.}{acumular; juntar; amontoar; reunir; coletar}
  \seealsoref{乘积}{cheng2ji1}
\end{EntryWithPhonetic}

\begin{EntryWithPhonetic}{积极}{ji1ji2}{10,7}{⽲、⽊}[HSK 3]
  \definition{adj.}{ativo; descreve uma atitude proativa e esforçada | positivo; que tem um efeito positivo e ajuda no desenvolvimento das coisas}
\end{EntryWithPhonetic}

\begin{EntryWithPhonetic}{积累}{ji1lei3}{10,11}{⽲、⽷}[HSK 4]
  \definition{s.}{acúmulo; acumulação}
  \definition{v.}{acumular}
\end{EntryWithPhonetic}

\begin{EntryWithPhonetic}{积木}{ji1mu4}{10,4}{⽲、⽊}
  \definition{s.}{blocos de montar (brinquedo)}
\end{EntryWithPhonetic}

\begin{EntryWithPhonetic}{基}{ji1}{11}{⼟}
  \definition{adj.}{chave; básico; primário; cardinal; fundamental}
  \definition{s.}{base; fundação | base; grupo; radical; (química) uma parte dos átomos contidos na molécula de um composto, quando considerada como uma unidade, é chamada de base}
\end{EntryWithPhonetic}

\begin{EntryWithPhonetic}{基本}{ji1ben3}{11,5}{⼟、⽊}[HSK 3]
  \definition{adj.}{básico; fundamental; elementar | principal}
  \definition{adv.}{basicamente; em geral; no geral; em termos gerais}
  \definition{s.}{fundação}
\end{EntryWithPhonetic}

\begin{EntryWithPhonetic}{基本法}{ji1ben3fa3}{11,5,8}{⼟、⽊、⽔}
  \definition{s.}{lei básica (constituição)}
\end{EntryWithPhonetic}

\begin{EntryWithPhonetic}{基本功}{ji1ben3gong1}{11,5,5}{⼟、⽊、⼒}
  \definition{s.}{habilidades | fundamentos básicos}
\end{EntryWithPhonetic}

\begin{EntryWithPhonetic}{基本上}{ji1 ben3 shang4}{11,5,3}{⼟、⽊、⼀}[HSK 3]
  \definition{adv.}{basicamente; principalmente | em geral; de modo geral}
\end{EntryWithPhonetic}

\begin{EntryWithPhonetic}{基础}{ji1chu3}{11,10}{⼟、⽯}[HSK 3]
  \definition[个,种,点,层]{s.}{base; fundamento; fundação; a essência ou o ponto de partida do desenvolvimento das coisas | básico; fundamental; refere-se às condições mínimas | fundação do edifício; base do edifício}
\end{EntryWithPhonetic}

\begin{EntryWithPhonetic}{基地}{ji1di4}{11,6}{⼟、⼟}[HSK 5]
  \definition{s.}{base; como base para alguns negócios | base; um local dedicado à realização de um negócio}
\end{EntryWithPhonetic}

\begin{EntryWithPhonetic}{基督教}{ji1 du1 jiao4}{11,13,11}{⼟、⽬、⽁}[HSK 6]
  \definition*{s.}{Cristianismo; A Religião Cristã | Cristão}
\end{EntryWithPhonetic}

\begin{EntryWithPhonetic}{基金}{ji1jin1}{11,8}{⼟、⾦}[HSK 5]
  \definition[只,笔]{s.}{fundo; fundos reservados ou destinados ao estabelecimento ou desenvolvimento de uma empresa}
\end{EntryWithPhonetic}

\begin{EntryWithPhonetic}{基因}{ji1yin1}{11,6}{⼟、⼞}
  \definition{s.}{gene}
\end{EntryWithPhonetic}

\begin{EntryWithPhonetic}{激}{ji1}{16}{⽔}
  \definition*{s.}{Sobrenome Ji}
  \definition{adj.}{afiado; feroz; violento | vívido}
  \definition{adv.}{bruscamente; ferozmente; violentamente}
  \definition{s.}{o impacto de ondas fortes contra a costa}
  \definition{v.}{bater; avançar; correr | despertar; estimular; incitar; excitar | ficar doente por se molhar | esfriar (colocando água gelada, etc.)}
\end{EntryWithPhonetic}

\begin{EntryWithPhonetic}{激动}{ji1dong4}{16,6}{⽔、⼒}[HSK 4]
  \definition{adj.}{animado; entusiasmado; empolgado}
  \definition{v.}{agitar; excitar; tornar fortes os sentimentos de alguém}
\end{EntryWithPhonetic}

\begin{EntryWithPhonetic}{激烈}{ji1lie4}{16,10}{⽔、⽕}[HSK 4]
  \definition{adj.}{agudo; afiado; feroz; violento; intenso}
\end{EntryWithPhonetic}

\begin{EntryWithPhonetic}{激情}{ji1qing2}{16,11}{⽔、⼼}[HSK 6]
  \definition{s.}{paixão; emoções fortes e explosivas, como êxtase, raiva, etc.}
\end{EntryWithPhonetic}

\begin{EntryWithPhonetic}{鷄}{ji1}{21}{⿃}
  \variantof{鸡}
\end{EntryWithPhonetic}

\begin{EntryWithPhonetic}{及}{ji2}{3}{⼃}
  \definition*{s.}{Sobrenome Ji}
  \definition{conj.}{e; bem como; conectando substantivos paralelos ou frases nominais}
  \definition{v.}{alcançar; chegar até | ser comparável a; alcançar (geralmente usado em termos negativos) | chegar a tempo para | estender-se a; cuidar de; envolver | dar}
\end{EntryWithPhonetic}

\begin{EntryWithPhonetic}{及格}{ji2/ge2}{3,10}{⼃、⽊}[HSK 4]
  \definition{v.+compl.}{passar; passar em um teste, exame, etc.}
\end{EntryWithPhonetic}

\begin{EntryWithPhonetic}{及时}{ji2shi2}{3,7}{⼃、⽇}[HSK 3]
  \definition{adj.}{oportuno; na hora certa; adequado; na ocasião certa}
  \definition{adv.}{prontamente; sem demora; imediatamente}
\end{EntryWithPhonetic}

\begin{EntryWithPhonetic}{吉}{ji2}{6}{⼝}
  \definition*{s.}{Província de Jilin, abreviação de 吉林 | Sobrenome Ji}
  \definition{adj.}{sortudo; propício; auspicioso (oposto de 凶)}
  \seealsoref{吉林}{ji2lin2}
  \seealsoref{凶}{xiong1}
\end{EntryWithPhonetic}

\begin{EntryWithPhonetic}{吉利}{ji2 li4}{6,7}{⼝、⼑}[HSK 6]
  \definition{adj.}{sortudo; auspicioso; propício}
\end{EntryWithPhonetic}

\begin{EntryWithPhonetic}{吉林}{ji2lin2}{6,8}{⼝、⽊}
  \definition*{s.}{Província de Jilin}
\end{EntryWithPhonetic}

\begin{EntryWithPhonetic}{吉他}{ji2ta1}{6,5}{⼝、⼈}
  \definition[把]{s.}{Empréstimo linguístico: guitarra, violão}
\end{EntryWithPhonetic}

\begin{EntryWithPhonetic}{吉祥}{ji2xiang2}{6,10}{⼝、⽰}[HSK 6]
  \definition{adj.}{sortudo; auspicioso; propício}
  \definition[个,种]{s.}{sorte; auspiciosidade; propiciação; um sinal ou símbolo de boa sorte ou fortuna}
\end{EntryWithPhonetic}

\begin{EntryWithPhonetic}{级}{ji2}{6}{⽷}[HSK 2]
  \definition{clas.}{usado para degraus, escadas, pisos de torres, etc.}
  \definition[个,种]{s.}{nível; classificação; grau; classe | série; turma; qualquer uma das divisões anuais de um curso escolar | degrau}
\end{EntryWithPhonetic}

\begin{EntryWithPhonetic}{即}{ji2}{7}{⼙}
  \definition{adv.}{no presente; no futuro imediato | prontamente; imediatamente}
  \definition{conj.}{e | mesmo; mesmo que}[她瞟了一眼睡着的孩子,随即匆匆离开了。===Ela olhou para a criança adormecida e então saiu correndo. | 即使下雨我也去。===Eu irei mesmo que chova.]
  \definition{v.}{aproximar-se; alcançar; estar perto | assumir; ascender a; aceitar; começar a se envolver em | ser motivado pela ocasião | estar perto | Literário: ser; significar}
\end{EntryWithPhonetic}

\begin{EntryWithPhonetic}{即便}{ji2bian4}{7,9}{⼙、⼈}
  \definition{conj.}{mesmo se/embora}
\end{EntryWithPhonetic}

\begin{EntryWithPhonetic}{即便是}{ji2bian4 shi4}{7,9,9}{⼙、⼈、⽇}
  \definition{conj.}{mesmo que seja}
\end{EntryWithPhonetic}

\begin{EntryWithPhonetic}{即或}{ji2huo4}{7,8}{⼙、⼽}
  \definition{conj.}{mesmo se/embora}
\end{EntryWithPhonetic}

\begin{EntryWithPhonetic}{即将}{ji2jiang1}{7,9}{⼙、⼨}[HSK 4]
  \definition{adv.}{em breve; estar prestes a; estar a ponto de}
\end{EntryWithPhonetic}

\begin{EntryWithPhonetic}{即若}{ji2ruo4}{7,8}{⼙、⾋}
  \definition{conj.}{mesmo se/embora}
\end{EntryWithPhonetic}

\begin{EntryWithPhonetic}{即使}{ji2shi3}{7,8}{⼙、⼈}[HSK 5]
  \definition{conj.}{mesmo; mesmo que; mesmo se; apesar de; expressando uma concessão hipotética}
\end{EntryWithPhonetic}

\begin{EntryWithPhonetic}{即是}{ji2shi4}{7,9}{⼙、⽇}
  \definition{conj.}{aquilo é}
\end{EntryWithPhonetic}

\begin{EntryWithPhonetic}{极}{ji2}{7}{⽊}[HSK 4]
  \definition*{s.}{Sobrenome Ji}
  \definition{adj.}{máximo; extremo; final; supremo}
  \definition{adv.}{extremamente; excessivamente}
  \definition{s.}{o ponto máximo, mais alto; extremo; ápice; ponto culminante | pólo; as extremidades norte e sul da Terra; as extremidades de um ímã; a extremidade de uma fonte de alimentação ou de um aparelho elétrico onde a corrente entra ou sai do aparelho}
  \definition{v.}{chegar ao fim de; levar a extremos | Literário: fazer o máximo possível}
\end{EntryWithPhonetic}

\begin{EntryWithPhonetic}{极端}{ji2duan1}{7,14}{⽊、⽴}[HSK 6]
  \definition{adj.}{extremo; absoluto; sem quaisquer restrições}
  \definition{adv.}{excessivamente; extremamente; alto grau de expressão}
  \definition{s.}{extremo; extremidade; o auge do desenvolvimento}
\end{EntryWithPhonetic}

\begin{EntryWithPhonetic}{……极了}{ji2le5}{7,2}{⽊、⼅}[HSK 3]
  \definition{expr.}{extremamente; alto grau de expressão}
\end{EntryWithPhonetic}

\begin{EntryWithPhonetic}{极其}{ji2qi2}{7,8}{⽊、⼋}[HSK 4]
  \definition{adv.}{mais; extremamente; excessivamente}
\end{EntryWithPhonetic}

\begin{EntryWithPhonetic}{急}{ji2}{9}{⼼}[HSK 2]
  \definition{adj.}{impaciente; ansioso | irritado; aborrecido; incomodado | rápido e intenso (em oposição a 缓); veloz | urgente; premente}
  \definition{s.}{urgência; emergência; assunto urgente e grave}
  \definition{v.}{preocupar; deixar ansioso | estar ansioso para ajudar; tratar os problemas dos outros como se fossem urgentes e ajudar a resolvê-los imediatamente}
  \seealsoref{缓}{huan3}
\end{EntryWithPhonetic}

\begin{EntryWithPhonetic}{急救}{ji2 jiu4}{9,11}{⼼、⽁}[HSK 6]
  \definition{s.}{primeiros socorros; tratamento médico de emergência (para pessoas gravemente doentes ou gravemente feridas)}
  \definition{v.}{prestar primeiros socorros; dar tratamento de emergência}
\end{EntryWithPhonetic}

\begin{EntryWithPhonetic}{急忙}{ji2mang2}{9,6}{⼼、⼼}[HSK 4]
  \definition{adv.}{apressadamente; com pressa}
\end{EntryWithPhonetic}

\begin{EntryWithPhonetic}{疾}{ji2}{10}{⽧}
  \definition*{s.}{Sobrenome Ji}
  \definition{s.}{doença; enfermidade; moléstia; padecimento | sofrimento; dor; dificuldade; mazela}
\end{EntryWithPhonetic}

\begin{EntryWithPhonetic}{疾病}{ji2bing4}{10,10}{⽧、⽧}[HSK 6]
  \definition[种]{s.}{doença; enfermidade; termo geral para doença}
\end{EntryWithPhonetic}

\begin{EntryWithPhonetic}{集}{ji2}{12}{⾫}[HSK 6]
  \definition*{s.}{Sobrenome Ji}
  \definition{clas.}{parte; volume}
  \definition[个,本]{s.}{mercado; feira rural | coleção; conjunto; antologia | (matemática) conjunto}
  \definition{v.}{reunir; coletar; montar}
\end{EntryWithPhonetic}

\begin{EntryWithPhonetic}{集合}{ji2he2}{12,6}{⾫、⼝}[HSK 4]
  \definition{s.}{conjunto; montagem; coleção; agregação}
  \definition{v.}{reunir-se; juntar-se | reunir, juntar, convocar}
\end{EntryWithPhonetic}

\begin{EntryWithPhonetic}{集体}{ji2ti3}{12,7}{⾫、⼈}[HSK 3]
  \definition{s.}{coletivo; comunidade; grupo; equipe; organizações ou grupos em que muitas pessoas trabalham, estudam e vivem juntas}
\end{EntryWithPhonetic}

\begin{EntryWithPhonetic}{集团}{ji2tuan2}{12,6}{⾫、⼞}[HSK 5]
  \definition[个,家,些]{s.}{anel; bloco; grupo; panelinha; círculo; grupo organizado para agir em conjunto com um determinado objetivo | grupo; entidade econômica com uma direção de negócios especializada, liderada por uma grande empresa com forte poder econômico e alta visibilidade, e formada pela combinação ou fusão de empresas relacionadas}
\end{EntryWithPhonetic}

\begin{EntryWithPhonetic}{集中}{ji2zhong1}{12,4}{⾫、⼁}[HSK 3]
  \definition{adj.}{centralizado; concentrado}
  \definition{v.}{concentrar; centralizar; focar; acumular; reunir (oposto de 分散) | reunir pessoas, coisas, forças, etc. dispersas; resumir opiniões, experiências, etc.}
  \seealsoref{分散}{fen1san4}
\end{EntryWithPhonetic}

\begin{EntryWithPhonetic}{嫉}{ji2}{13}{⼥}
  \definition{v.}{invejar | odiar | ter ciúmes; ter inveja}
\end{EntryWithPhonetic}

\begin{EntryWithPhonetic}{嫉妒}{ji2du4}{13,7}{⼥、⼥}
  \definition{v.}{estar com ciúmes de | invejar}
\end{EntryWithPhonetic}

\begin{EntryWithPhonetic}{几}{ji3}{2}{⼏}[HSK 1][Kangxi 16]
  \definition{adv.}{quanto?, usado para perguntar sobre quantidade e tempo}
  \definition{num.}{alguns; vários; poucos; indica um número indeterminado maior que um e menor que dez}
  \seeref{ji1}
\end{EntryWithPhonetic}

\begin{EntryWithPhonetic}{几何}{ji3he2}{2,7}{⼏、⼈}
  \definition{s.}{geometria}
\end{EntryWithPhonetic}

\begin{EntryWithPhonetic}{纪}{ji3}{6}{⽷}
  \definition*{s.}{Sobrenome Ji}
  \definition{s.}{disciplina | um período de doze anos (na China antiga); um período de anos | (geologia) subdivisão de uma era geológica; período}
  \definition{v.}{colocar por escrito; registrar; mesmo significado de 记, usado principalmente em 记录, 纪年, 纪元, 纪传, etc. | classificar (fios de seda)}
  \seeref{ji4}
  \seealsoref{记}{ji4}
  \seealsoref{纪传}{ji4 zhuan4}
  \seealsoref{记录}{ji4lu4}
  \seealsoref{纪年}{ji4nian2}
  \seealsoref{纪元}{ji4yuan2}
\end{EntryWithPhonetic}

\begin{EntryWithPhonetic}{挤}{ji3}{9}{⼿}[HSK 5]
  \definition{adj.}{lotado; congestionado; descreve um grande número de pessoas ou coisas e muito pouco espaço}
  \definition{v.}{empacotar; amontoar; aglomerar | sacudir; empurrar contra; empurrar alguém ou algo para longe com seu corpo com toda a força que puder| pressionar; apertar; expulsar por pressão}
\end{EntryWithPhonetic}

\begin{EntryWithPhonetic}{给}{ji3}{9}{⽷}
  \definition{adj.}{abundante; próspero; bem provido para}
  \definition{v.}{fornecer; prover}
  \seeref{gei3}
\end{EntryWithPhonetic}

\begin{EntryWithPhonetic}{给予}{ji3yu3}{9,4}{⽷、⼅}[HSK 6]
  \definition{v.}{dar; conceder; dar em troca}
\end{EntryWithPhonetic}

\begin{EntryWithPhonetic}{计}{ji4}{4}{⾔}
  \definition*{s.}{Sobrenome Ji}
  \definition{s.}{medidor; aferidor; indicador; um instrumento para medir ou calcular graus, tempo, etc. | ideia; ardil; estratagema; plano}
  \definition{v.}{contar; calcular; numerar | planejar; traçar; imaginar}
\end{EntryWithPhonetic}

\begin{EntryWithPhonetic}{计划}{ji4hua4}{4,6}{⾔、⼑}[HSK 2]
  \definition[个,项]{s.}{plano; projeto; programa; trabalho, ações, conteúdo e etapas previamente definidos}
  \definition{v.}{planejar; traçar um plano}
\end{EntryWithPhonetic}

\begin{EntryWithPhonetic}{计算}{ji4suan4}{4,14}{⾔、⽵}[HSK 3]
  \definition{v.}{contar; calcular; computar; enumerar; encontrar a variável desconhecida | planejar; considerar | conspirar secretamente contra os outros; planejar secretamente prejudicar os outros}
\end{EntryWithPhonetic}

\begin{EntryWithPhonetic}{计算机}{ji4 suan4 ji1}{4,14,6}{⾔、⽵、⽊}[HSK 2]
  \definition[部,台]{s.}{computador; calculadora; máquinas capazes de realizar cálculos matemáticos são feitas com dispositivos mecânicos, como calculadoras manuais, outras são feitas com componentes eletrônicos, como computadores eletrônicos}
\end{EntryWithPhonetic}

\begin{EntryWithPhonetic}{计算机程序}{ji4suan4ji1 cheng2xu4}{4,14,6,12,7}{⾔、⽵、⽊、⽲、⼴}
  \definition{s.}{programa de computador}
\end{EntryWithPhonetic}

\begin{EntryWithPhonetic}{记}{ji4}{5}{⾔}[HSK 1]
  \definition*{s.}{Sobrenome Ji}
  \definition{clas.}{tapas, palmadas, bofetadas, etc.; usado para indicar o número de vezes que uma determinada ação é realizada}
  \definition{s.}{assinatura; bloco de notas; livro ou artigo que registra fatos | insígnia; indicação; \& comercial; símbolo | marca de nascença; manchas escuras presentes na pele desde o nascimento}
  \definition{v.}{lembrar; ter em mente; guardar na memória; manter a imagem na mente | escrever (anotar); registrar; inscrever}
\end{EntryWithPhonetic}

\begin{EntryWithPhonetic}{记得}{ji4de5}{5,11}{⾔、⼻}[HSK 1]
  \definition{v.}{lembrar; recordar; lembrar-se; não esquecer | manter algo em mente; (informal) não se esquecer de fazer algo, usado para lembrar}
\end{EntryWithPhonetic}

\begin{EntryWithPhonetic}{记录}{ji4lu4}{5,8}{⾔、⼹}[HSK 3]
  \definition[份,名,位,个]{s.}{notas; registro | anotador; registrador; a pessoa que faz registros}
  \definition{v.}{tomar notas; registrar; escrever o que ouviu ou o que aconteceu; gravar o som ou a imagem com um gravador ou uma câmera de vídeo e transformar em algum tipo de obra}
\end{EntryWithPhonetic}

\begin{EntryWithPhonetic}{记性}{ji4xing5}{5,8}{⾔、⼼}
  \definition{s.}{memória (habilidade em reter informações)}
\end{EntryWithPhonetic}

\begin{EntryWithPhonetic}{记忆}{ji4yi4}{5,4}{⾔、⼼}[HSK 5]
  \definition[段]{s.}{memória; manter em sua mente uma imagem do passado}
  \definition{v.}{recordar; lembrar; lembrar-se ou recordar alguém ou algo do passado}
\end{EntryWithPhonetic}

\begin{EntryWithPhonetic}{记载}{ji4zai3}{5,10}{⾔、⾞}[HSK 4]
  \definition[段,种,条]{s.}{registro; conta; artigos e materiais que registram eventos}
  \definition{v.}{registrar; colocar por escrito}
\end{EntryWithPhonetic}

\begin{EntryWithPhonetic}{记者}{ji4zhe3}{5,8}{⾔、⽼}[HSK 3]
  \definition[群,名,位]{s.}{repórter; correspondente; jornalista; profissionais dedicados a entrevistar e reportar notícias para a mídia}
\end{EntryWithPhonetic}

\begin{EntryWithPhonetic}{记住}{ji4 zhu5}{5,7}{⾔、⼈}[HSK 1]
  \definition{v.}{lembrar; aprender de cor; ter em mente; guardar na memória}
\end{EntryWithPhonetic}

\begin{EntryWithPhonetic}{纪}{ji4}{6}{⽷}
  \definition*{s.}{Sobrenome Ji}
  \definition{s.}{disciplina | idade; época | (geologia) período | um período de doze anos (na China antiga); um período de anos | (geologia) subdivisão de uma era geológica}
  \definition{v.}{colocar por escrito; registrar | registrar, mesmo significado de 记, usado principalmente em 记录, 纪年, 纪元, 纪传, etc. | classificar (fios de seda)}
  \seeref{ji3}
  \seealsoref{记}{ji4}
  \seealsoref{纪传}{ji4 zhuan4}
  \seealsoref{记录}{ji4lu4}
  \seealsoref{纪年}{ji4nian2}
  \seealsoref{纪元}{ji4yuan2}
\end{EntryWithPhonetic}

\begin{EntryWithPhonetic}{纪录}{ji4lu4}{6,8}{⽷、⼹}[HSK 3]
  \definition[项,个]{s.}{recorde (esportes); o número mais alto ou mais baixo registrado em um determinado período de tempo}
\end{EntryWithPhonetic}

\begin{EntryWithPhonetic}{纪律}{ji4lv4}{6,9}{⽷、⼻}[HSK 4]
  \definition{s.}{disciplina; código de conduta que cada membro da vida coletiva deve observar}
\end{EntryWithPhonetic}

\begin{EntryWithPhonetic}{纪年}{ji4nian2}{6,6}{⽷、⼲}
  \definition{s.}{cronologia; uma maneira de numerar os anos | registro cronológico de eventos; anais; um dos gêneros de livros históricos é organizar fatos históricos em ordem cronológica}
\end{EntryWithPhonetic}

\begin{EntryWithPhonetic}{纪念}{ji4nian4}{6,8}{⽷、⼼}[HSK 3]
  \definition[个,次]{s.}{lembrança; recordação; usado para representar uma lembrança (objeto)}
  \definition{v.}{comemorar; expressar saudade por pessoas ou coisas através de objetos ou ações}
\end{EntryWithPhonetic}

\begin{EntryWithPhonetic}{纪元}{ji4yuan2}{6,4}{⽷、⼉}
  \definition{s.}{o início de uma era (por exemplo, o reinado de um imperador) | época; era}
\end{EntryWithPhonetic}

\begin{EntryWithPhonetic}{纪传}{ji4 zhuan4}{6,6}{⽷、⼈}
  \definition{s.}{crônica; biografia}
\end{EntryWithPhonetic}

\begin{EntryWithPhonetic}{纪传体}{ji4 zhuan4 ti3}{6,6,7}{⽷、⼈、⼈}
  \definition{s.}{história apresentada em uma série de biografias | gênero histórico baseado em biografia}
\end{EntryWithPhonetic}

\begin{EntryWithPhonetic}{技}{ji4}{7}{⼿}
  \definition[门,项]{s.}{destreza; habilidade; estratagema | técnica; tecnologia}
\end{EntryWithPhonetic}

\begin{EntryWithPhonetic}{技俩}{ji4liang3}{7,9}{⼿、⼈}
  \definition{s.}{truque | estratagema | ardil | esquema | estratégia | tática}
\end{EntryWithPhonetic}

\begin{EntryWithPhonetic}{技能}{ji4 neng2}{7,10}{⼿、⾁}[HSK 5]
  \definition[种,项]{s.}{habilidade técnica; domínio de uma habilidade ou técnica; capacidade de adquirir e aplicar conhecimento}
\end{EntryWithPhonetic}

\begin{EntryWithPhonetic}{技巧}{ji4qiao3}{7,5}{⼿、⼯}[HSK 4]
  \definition[个,些]{s.}{habilidade; técnica; habilidades engenhosas expressas em artes, artesanato, esportes, etc.}
\end{EntryWithPhonetic}

\begin{EntryWithPhonetic}{技术}{ji4shu4}{7,5}{⼿、⽊}[HSK 3]
  \definition[种,门,项]{s.}{habilidade; técnica; tecnologia; a experiência e o conhecimento acumulados pelo ser humano no processo de utilização e transformação da natureza, e refletidos no trabalho produtivo, também se referem, de maneira geral, a outras habilidades operacionais}
\end{EntryWithPhonetic}

\begin{EntryWithPhonetic}{系}{ji4}{7}{⽷}
  \definition{v.}{amarrar; prender; abotoar; dar um nó}
  \seeref{xi4}
\end{EntryWithPhonetic}

\begin{EntryWithPhonetic}{季}{ji4}{8}{⼦}[HSK 4]
  \definition*{s.}{Sobrenome Ji}
  \definition{s.}{estação; o ano é dividido em quatro estações, primavera, verão, outono e inverno, e uma estação dura três meses | temporada | o fim de uma era | o último mês de uma temporada | o quarto ou mais novo entre irmãos; último na ordem de precedência}
\end{EntryWithPhonetic}

\begin{EntryWithPhonetic}{季度}{ji4du4}{8,9}{⼦、⼴}[HSK 4]
  \definition[个]{s.}{trimestre; período de tempo trimestral}
\end{EntryWithPhonetic}

\begin{EntryWithPhonetic}{季节}{ji4jie2}{8,5}{⼦、⾋}[HSK 4]
  \definition[个]{s.}{estação (clima); um período característico do ano}
\end{EntryWithPhonetic}

\begin{EntryWithPhonetic}{既}{ji4}{9}{⽆}[HSK 4]
  \definition*{s.}{Sobrenome Ji}
  \definition{adv.}{já}
  \definition{conj.}{desde; como; agora que | assim como; e também; ambos\dots e\dots; usado em conjunto com advérbios como 且, 又, 也 para indicar uma combinação de ambas as situações}
  \seealsoref{且}{qie3}
  \seealsoref{也}{ye3}
  \seealsoref{又}{you4}
\end{EntryWithPhonetic}

\begin{EntryWithPhonetic}{既不……又不……}{ji4bu4 you4bu4}{9,4,2,4}{⽆、⼀、⼜、⼀}
  \definition{conj.}{nem\dots nem\dots}
\end{EntryWithPhonetic}

\begin{EntryWithPhonetic}{既然}{ji4ran2}{9,12}{⽆、⽕}[HSK 4]
  \definition{conj.}{como; desde; agora que; usado na primeira metade de uma frase, muitas vezes repetido na segunda metade pelos advérbios 就, 也, 还 para indicar que a premissa é primeiro declarada e depois inferida}
  \seealsoref{还}{hai2}
  \seealsoref{就}{jiu4}
  \seealsoref{也}{ye3}
\end{EntryWithPhonetic}

\begin{EntryWithPhonetic}{既又}{ji4you4}{9,2}{⽆、⼜}
  \definition{conj.}{desde | como | agora isso | os dois e | assim como}
\end{EntryWithPhonetic}

\begin{EntryWithPhonetic}{继}{ji4}{10}{⽷}
  \definition{adv.}{então; depois}
  \definition{s.}{filhos; prole}
  \definition{v.}{continuar; ter sucesso; seguir}
\end{EntryWithPhonetic}

\begin{EntryWithPhonetic}{继承}{ji4cheng2}{10,8}{⽷、⼿}[HSK 5]
  \definition{v.}{herdar (o patrimônio de uma pessoa falecida, etc.) de acordo com a lei | continuar; geralmente se refere à aceitação do estilo, da cultura, do conhecimento, etc., daqueles que nos precederam | continuar; os descendentes continuam o trabalho deixado por seus antecessores.}
\end{EntryWithPhonetic}

\begin{EntryWithPhonetic}{继续}{ji4xu4}{10,11}{⽷、⽷}[HSK 3]
  \definition{s.}{continuação}
  \definition{v.}{continuar; prosseguir | prosseguir; continuar; seguir em frente (com); (atividades, eventos, etc.) continuar após uma pausa ou um determinado período de tempo}
\end{EntryWithPhonetic}

\begin{EntryWithPhonetic}{寂}{ji4}{11}{⼧}
  \definition{adj.}{quieto; parado; silencioso | solitário}
\end{EntryWithPhonetic}

\begin{EntryWithPhonetic}{寂寥}{ji4liao2}{11,14}{⼧、⼧}
  \definition{s.}{solidão | vasto e vazio | quieto e desolado (literário)}
\end{EntryWithPhonetic}

\begin{EntryWithPhonetic}{寂寞}{ji4mo4}{11,13}{⼧、⼧}
  \definition{adj.}{sozinho | solitário | (de um lugar) silencioso}
\end{EntryWithPhonetic}

\begin{EntryWithPhonetic}{寄}{ji4}{11}{⼧}[HSK 4]
  \definition*{s.}{Sobrenome Ji}
  \definition{adj.}{adotado; fomentado; promovido}
  \definition{v.}{enviar; postar; remeter | confiar; depositar; colocar | depender de; apegar-se a}
\end{EntryWithPhonetic}

\begin{EntryWithPhonetic}{寄存}{ji4cun2}{11,6}{⼧、⼦}
  \definition{v.}{depositar | deixar algo com alguém | armazenar}
\end{EntryWithPhonetic}

\begin{EntryWithPhonetic}{寄递}{ji4di4}{11,10}{⼧、⾡}
  \definition{s.}{entrega de correspondência}
\end{EntryWithPhonetic}

\begin{EntryWithPhonetic}{寄放}{ji4fang4}{11,8}{⼧、⽅}
  \definition{v.}{deixar algo com alguém}
\end{EntryWithPhonetic}

\begin{EntryWithPhonetic}{寄居}{ji4ju1}{11,8}{⼧、⼫}
  \definition{s.}{morar longe de casa}
\end{EntryWithPhonetic}

\begin{EntryWithPhonetic}{寄卖}{ji4mai4}{11,8}{⼧、⼗}
  \definition{v.}{consignar para venda}
\end{EntryWithPhonetic}

\begin{EntryWithPhonetic}{寄生}{ji4sheng1}{11,5}{⼧、⽣}
  \definition{s.}{parasita | parasitismo}
  \definition{v.}{viver tirando vantagem dos outros | viver dentro ou sobre outro organismo como um parasita}
\end{EntryWithPhonetic}

\begin{EntryWithPhonetic}{寄生生活}{ji4sheng1sheng1huo2}{11,5,5,9}{⼧、⽣、⽣、⽔}
  \definition{s.}{parasitismo | vida parasitária}
\end{EntryWithPhonetic}

\begin{EntryWithPhonetic}{寄售}{ji4shou4}{11,11}{⼧、⼝}
  \definition{v.}{venda em consignação}
\end{EntryWithPhonetic}

\begin{EntryWithPhonetic}{寄送}{ji4song4}{11,9}{⼧、⾡}
  \definition{v.}{enviar | transmitir}
\end{EntryWithPhonetic}

\begin{EntryWithPhonetic}{寄宿}{ji4su4}{11,11}{⼧、⼧}
  \definition{s.}{embarque}
  \definition{v.}{embarcar}
\end{EntryWithPhonetic}

\begin{EntryWithPhonetic}{寄托}{ji4tuo1}{11,6}{⼧、⼿}
  \definition{v.}{investir (sua esperança, energia, etc.) em algo | confiar (a alguém) | colocar (a esperança, a energia, etc.) em}
\end{EntryWithPhonetic}

\begin{EntryWithPhonetic}{寄望}{ji4wang4}{11,11}{⼧、⽉}
  \definition{v.}{depositar esperanças em}
\end{EntryWithPhonetic}

\begin{EntryWithPhonetic}{寄养}{ji4yang3}{11,9}{⼧、⼋}
  \definition{v.}{embarcar | promover | colocar sob os cuidados de alguém (uma criança, animal de estimação, etc.)}
\end{EntryWithPhonetic}

\begin{EntryWithPhonetic}{寄予}{ji4yu3}{11,4}{⼧、⼅}
  \definition{v.}{expressar | colocar (esperança, importância, etc.) em | mostrar}
\end{EntryWithPhonetic}

\begin{EntryWithPhonetic}{旣}{ji4}{11}{⽆}
  \variantof{既}
\end{EntryWithPhonetic}

\begin{EntryWithPhonetic}{加}{jia1}{5}{⼒}[HSK 2]
  \definition*{s.}{Canadá, abreviação de 加拿大 | Sobrenome Jia}
  \definition{v.}{adicionar; somar | aumentar; incrementar; aumentar a quantidade ou o grau em relação ao original | inserir; adicionar; anexar; adicionar o que não existe; colocar no lugar | acrescentar; indica a realização de uma determinada ação | colocar uma coisa em cima da outra | impor ou aplicar algo a outra pessoa; atribuir um determinado comportamento a outra pessoa}
  \seealsoref{加拿大}{jia1na2da4}
\end{EntryWithPhonetic}

\begin{EntryWithPhonetic}{加班}{jia1/ban1}{5,10}{⼒、⽟}[HSK 4]
  \definition{v.+compl.}{fazer horas extras; trabalhar horas extras; aumentar o horário de trabalho ou os turnos além do limite de tempo prescrito}
\end{EntryWithPhonetic}

\begin{EntryWithPhonetic}{加工}{jia1gong1}{5,3}{⼒、⼯}[HSK 3]
  \definition{s.}{processo | trabalho (de uma máquina)}
  \definition{v.}{processar; realizar diversos trabalhos em matérias-primas e produtos semiacabados (como alterar dimensões, formas, propriedades, aumentar a precisão, pureza, etc.) para que atendam aos requisitos especificados | melhorar; polir; refere-se a todos os tipos de trabalho que tornam o produto final mais perfeito e refinado}
\end{EntryWithPhonetic}

\begin{EntryWithPhonetic}{加快}{jia1 kuai4}{5,7}{⼒、⼼}[HSK 3]
  \definition{v.}{acelerar; aumentar a velocidade; agilizar}
\end{EntryWithPhonetic}

\begin{EntryWithPhonetic}{加盟}{jia1 meng2}{5,13}{⼒、⽫}[HSK 6]
  \definition{v.}{aliar-se a; filiar-se a um sindicato; juntar-se a um grupo ou organização}
\end{EntryWithPhonetic}

\begin{EntryWithPhonetic}{加拿大}{jia1na2da4}{5,10,3}{⼒、⼿、⼤}
  \definition{s.}{Canadá}
\end{EntryWithPhonetic}

\begin{EntryWithPhonetic}{加拿大人}{jia1na2da4ren2}{5,10,3,2}{⼒、⼿、⼤、⼈}
  \definition{s.}{canadense | pessoa ou povo do Canadá}
\end{EntryWithPhonetic}

\begin{EntryWithPhonetic}{加强}{jia1 qiang2}{5,12}{⼒、⼸}[HSK 3]
  \definition{v.}{fortalecer; engrandecer; reforçar; tornar mais forte ou mais eficaz}
\end{EntryWithPhonetic}

\begin{EntryWithPhonetic}{加热}{jia1 re4}{5,10}{⼒、⽕}[HSK 5]
  \definition{v.}{aquecer; esquentar; aumentar a temperatura de um objeto}
\end{EntryWithPhonetic}

\begin{EntryWithPhonetic}{加入}{jia1ru4}{5,2}{⼒、⼊}[HSK 4]
  \definition{v.}{juntar-se; unir-se; aderir a; tornar-se um membro de uma organização, grupo | adicionar; colocar em}
\end{EntryWithPhonetic}

\begin{EntryWithPhonetic}{加上}{jia1 shang4}{5,3}{⼒、⼀}[HSK 5]
  \definition{conj.}{além disso; em adição}
  \definition{v.}{adicionar; acrescentar; dar; aumentar}
\end{EntryWithPhonetic}

\begin{EntryWithPhonetic}{加速}{jia1 su4}{5,10}{⼒、⾡}[HSK 5]
  \definition{v.}{acelerar; agilizar}
\end{EntryWithPhonetic}

\begin{EntryWithPhonetic}{加速度}{jia1su4du4}{5,10,9}{⼒、⾡、⼴}
  \definition{s.}{aceleração}
\end{EntryWithPhonetic}

\begin{EntryWithPhonetic}{加以}{jia1 yi3}{5,4}{⼒、⼈}[HSK 5]
  \definition{conj.}{além disso; em adição; indica outras razões ou condições}
  \definition{v.aux.}{usado na frente de palavras dissilábicas para indicar como um objeto mencionado deve ser tratado ou descartado | usado antes de um verbo polifônico ou de um substantivo formado a partir de um verbo para indicar como tratar ou lidar com o que foi mencionado anteriormente}
\end{EntryWithPhonetic}

\begin{EntryWithPhonetic}{加油}{jia1/you2}{5,8}{⼒、⽔}[HSK 2]
  \definition{v.+compl.}{abastecer com óleo; reabastecer; adicionar combustível ou óleo lubrificante | fazer um esforço extra; dar o máximo; (Vamos lá!) metáfora para se esforçar ainda mais}
\end{EntryWithPhonetic}

\begin{EntryWithPhonetic}{加油工}{jia1 you2 gong1}{5,8,3}{⼒、⽔、⼯}[HSK 6]
  \definition{s.}{frentista}
\end{EntryWithPhonetic}

\begin{EntryWithPhonetic}{加油站}{jia1you2zhan4}{5,8,10}{⼒、⽔、⽴}[HSK 4]
  \definition[个,座,家]{s.}{posto de gasolina; posto de combustível; postos de abastecimento para venda a varejo de gasolina e óleo para carros e outros veículos motorizados}
\end{EntryWithPhonetic}

\begin{EntryWithPhonetic}{夹}{jia1}{6}{⼤}[HSK 5]
  \definition{s.}{clipe, grampo, pasta, etc.}
  \definition{v.}{colocar no meio; pressionar de ambos os lados; aplicar força ou ação ao mesmo objeto de ambos os lados ao mesmo tempo | misturar; mesclar; intercalar}
  \seeref{ga1}
  \seeref{jia2}
\end{EntryWithPhonetic}

\begin{EntryWithPhonetic}{夹杂}{jia1 za2}{6,6}{⼤、⽊}
  \definition{v.}{ser misturado com; estar carregado de; adicionar (algo mais)}
\end{EntryWithPhonetic}

\begin{EntryWithPhonetic}{夹肢窝}{jia1 zhi1 wo1}{6,8,12}{⼤、⾁、⽳}
  \definition{s.}{axila; sovaco; também escrito como 胳肢窝}
  \seealsoref{胳肢窝}{ga1 zhi1 wo1}
\end{EntryWithPhonetic}

\begin{EntryWithPhonetic}{夹子}{jia1 zi5}{6,3}{⼤、⼦}
  \definition[个,堆,盒]{s.}{pasta; carteira; algo para guardar dinheiro, papel, etc. | clipe; grampo; pasta; pinça; ferramentas para prender coisas}
\end{EntryWithPhonetic}

\begin{EntryWithPhonetic}{茄}{jia1}{8}{⾋}
  \definition{s.}{caracter fonético usado em empréstimos linguísticos para o som "jia", embora 夹 seja mais comum}
  \seeref{qie2}
  \seealsoref{夹}{jia1}
\end{EntryWithPhonetic}

\begin{EntryWithPhonetic}{家}{jia1}{10}{⼧}[HSK 1,2]
  \definition*{s.}{Sobrenome Jia}
  \definition{adj.}{domado; domesticado; criado; alimentado | interno}
  \definition{clas.}{usado para famílias ou estabelecimentos comerciais; para uso doméstico; lojas; fábricas, etc.}
  \definition{pron.}{Educado: meu (irmã, tio, etc.)}
  \definition[个]{s.}{família; domicílio; clã | lar; casa; residência da família | pessoa ou família envolvida em um determinado comércio; pessoas que trabalham em determinada profissão ou que possuem determinada identidade | especialista em um determinado campo; pessoa que possui conhecimentos especializados ou se dedica a atividades específicas | escola de pensamento; rscola acadêmica | (em cartas de baralho, mah-jong etc.) festa; lado; refere-se a jogar xadrez ou cartas, em que uma das partes joga contra a outra | nacionalidade; referindo-se à etnia | membros da família; parentes; pessoas ou famílias com quem você tem algum tipo de relação | membro do mesmo clã; pessoas com o mesmo sobrenome}
  \definition{suf.}{sufixo substantivo para designar um especialista em alguma atividade, como um músico ou revolucionário, para designar uma profissão como em -eiro, -ista, por exemplo 科学家}
  \seealsoref{科学家}{ke1xue2jia1}
\end{EntryWithPhonetic}

\begin{EntryWithPhonetic}{家电}{jia1 dian4}{10,5}{⼧、⽥}[HSK 6]
  \definition[件,台]{s.}{eletrodomésticos, abreviação de 家用电器}
  \seealsoref{家用电器}{jia1yong4 dian4qi4}
\end{EntryWithPhonetic}

\begin{EntryWithPhonetic}{家伙}{jia1huo5}{10,6}{⼧、⼈}
  \definition[些,个,群,帮]{s.}{ferramenta; utensílio; arma; refere-se a ferramentas ou armas | cara; companheiro; refere-se a pessoas (com desprezo ou humor)  | gado; animal doméstico}
\end{EntryWithPhonetic}

\begin{EntryWithPhonetic}{家具}{jia1ju4}{10,8}{⼧、⼋}[HSK 3]
  \definition[件,套,些,个]{s.}{móveis; mobiliário de casa; utensílios domésticos, incluem principalmente camas, mesas, cadeiras, armários, etc.}
\end{EntryWithPhonetic}

\begin{EntryWithPhonetic}{家俱}{jia1ju4}{10,10}{⼧、⼈}
  \definition{s.}{mobília}
\end{EntryWithPhonetic}

\begin{EntryWithPhonetic}{家里}{jia1 li3}{10,7}{⼧、⾥}[HSK 1]
  \definition{s.}{(em) casa; (em sua) família | esposa}
\end{EntryWithPhonetic}

\begin{EntryWithPhonetic}{家人}{jia1 ren2}{10,2}{⼧、⼈}[HSK 1]
  \definition{s.}{família (de alguém); membro da família; os membros de uma família}
\end{EntryWithPhonetic}

\begin{EntryWithPhonetic}{家属}{jia1shu3}{10,12}{⼧、⼫}[HSK 3]
  \definition{s.}{membros da família; dependentes (familiares); os membros da família que não sejam o próprio chefe da família, ou seja, os membros da família que não sejam o próprio trabalhador}
\end{EntryWithPhonetic}

\begin{EntryWithPhonetic}{家庭}{jia1ting2}{10,9}{⼧、⼴}[HSK 2]
  \definition[个,户]{s.}{família}
\end{EntryWithPhonetic}

\begin{EntryWithPhonetic}{家务}{jia1wu4}{10,5}{⼧、⼒}[HSK 4]
  \definition[堆,次,件]{s.}{trabalho doméstico; tarefas domésticas}
\end{EntryWithPhonetic}

\begin{EntryWithPhonetic}{家乡}{jia1xiang1}{10,3}{⼧、⼄}[HSK 3]
  \definition[片,座]{s.}{cidade natal; o lugar onde sua família vive há gerações}
\end{EntryWithPhonetic}

\begin{EntryWithPhonetic}{家用电器}{jia1yong4 dian4qi4}{10,5,5,16}{⼧、⽤、⽥、⼝}
  \definition{s.}{eletrodoméstico; refere-se a diversos aparelhos elétricos utilizados na vida doméstica e coletiva}
\end{EntryWithPhonetic}

\begin{EntryWithPhonetic}{家园}{jia1 yuan2}{10,7}{⼧、⼞}[HSK 6]
  \definition{s.}{casa; terra natal; um jardim em casa, geralmente referindo-se à cidade natal ou à família}
\end{EntryWithPhonetic}

\begin{EntryWithPhonetic}{家长}{jia1 zhang3}{10,4}{⼧、⾧}[HSK 2]
  \definition[位,名,个]{s.}{pais; patriarca; tutor; guardião; refere-se aos pais ou outros responsáveis legais}
\end{EntryWithPhonetic}

\begin{EntryWithPhonetic}{傢}{jia1}{12}{⼈}
  \definition{s.}{usado em 家伙  e 家俱}
  \variantof{家}
  \seealsoref{傢伙}{jia1huo5}
  \seealsoref{家俱}{jia1ju4}
\end{EntryWithPhonetic}

\begin{EntryWithPhonetic}{傢伙}{jia1huo5}{12,6}{⼈、⼈}
  \variantof{家伙}
\end{EntryWithPhonetic}

\begin{EntryWithPhonetic}{傢俱}{jia1ju4}{12,10}{⼈、⼈}
  \variantof{家俱}
\end{EntryWithPhonetic}

\begin{EntryWithPhonetic}{嘉}{jia1}{14}{⼝}
  \definition*{s.}{Sobrenome Jia}
  \definition{adj.}{bom; ótimo | auspicioso | excelente}
  \definition{v.}{elogiar; recomendar}
  \definition{v.}{elogiar}
\end{EntryWithPhonetic}

\begin{EntryWithPhonetic}{嘉宾}{jia1bin1}{14,10}{⼝、⼧}[HSK 6]
  \definition[个,位,名,些]{s.}{convidado}
\end{EntryWithPhonetic}

\begin{EntryWithPhonetic}{嘉年华}{jia1nian2hua2}{14,6,6}{⼝、⼲、⼗}
  \definition{s.}{(empréstimo linguístico) carnaval}
\end{EntryWithPhonetic}

\begin{EntryWithPhonetic}{夹}{jia2}{6}{⼤}
  \definition{adj.}{forrado; com camada dupla; duas camadas (roupas, colchas, etc.) | pinçado; voz deliberadamente engraçada}
  \seeref{ga1}
  \seeref{jia1}
\end{EntryWithPhonetic}

\begin{EntryWithPhonetic}{甲}{jia3}{5}{⽥}[HSK 5]
  \definition*{s.}{Sobrenome Jia}
  \definition{s.}{alfa; primeiro lugar; o primeiro dos caules celestiais, geralmente usado para indicar o primeiro em ordem ou classificação | concha; carapaça; crustáceos | unha; crostas queratinosas nos dedos das mãos e dos pés | armadura; equipamento de proteção feito de metal | Datado: unidade de administração civil composta por 10 residências | uma palavra substituta para uma pessoa ou coisa indefinida; usado como pronome}
  \definition{v.}{ocupar o primeiro lugar; ser melhor do que}
\end{EntryWithPhonetic}

\begin{EntryWithPhonetic}{甲骨文}{jia3gu3wen2}{5,9,4}{⽥、⾻、⽂}
  \definition{s.}{escrituras de oráculos | inscrições em ossos de oráculos (forma original de escritura chinesa)}
\end{EntryWithPhonetic}

\begin{EntryWithPhonetic}{假}{jia3}{11}{⼈}[HSK 2]
  \definition{adj.}{falso; artificial}
  \definition{conj.}{se; caso; no caso de; conecta frases, expressa relação hipotética, geralmente usada com 如, 若 e 使, equivalente a 如果}
  \definition[个,天]{s.}{falsificação; coisas falsas, irreais ou forjadas}
  \definition{v.}{emprestar | valer-se de; aproveitar; utilizar | supor; presumir; pressupor}
  \seeref{jia4}
  \seealsoref{如}{ru2}
  \seealsoref{如果}{ru2guo3}
  \seealsoref{若}{ruo4}
  \seealsoref{使}{shi3}
\end{EntryWithPhonetic}

\begin{EntryWithPhonetic}{假的}{jia3de5}{11,8}{⼈、⽩}
  \definition{adj.}{falso | substituto | simulado}
\end{EntryWithPhonetic}

\begin{EntryWithPhonetic}{假如}{jia3ru2}{11,6}{⼈、⼥}[HSK 4]
  \definition{conj.}{se; supondo; no caso}
\end{EntryWithPhonetic}

\begin{EntryWithPhonetic}{假声}{jia3sheng1}{11,7}{⼈、⼠}
  \definition{s.}{falsete}
  \seealsoref{真声}{zhen1sheng1}
\end{EntryWithPhonetic}

\begin{EntryWithPhonetic}{假使}{jia3shi3}{11,8}{⼈、⼈}
  \definition{conj.}{se | supondo | em caso}
\end{EntryWithPhonetic}

\begin{EntryWithPhonetic}{假证件}{jia3zheng4jian4}{11,7,6}{⼈、⾔、⼈}
  \definition{s.}{documentos falsos}
\end{EntryWithPhonetic}

\begin{EntryWithPhonetic}{价}{jia4}{6}{⼈}[HSK 5]
  \definition{s.}{preço | valor; (figurativo) valores (éticos, culturais etc.) | Química: valência}
\end{EntryWithPhonetic}

\begin{EntryWithPhonetic}{价格}{jia4ge2}{6,10}{⼈、⽊}[HSK 3]
  \definition[个,种]{s.}{preço; tarifa; o valor monetário da mercadoria}
\end{EntryWithPhonetic}

\begin{EntryWithPhonetic}{价钱}{jia4 qian2}{6,10}{⼈、⾦}[HSK 3]
  \definition[个,种,笔]{s.}{preço}
\end{EntryWithPhonetic}

\begin{EntryWithPhonetic}{价值}{jia4zhi2}{6,10}{⼈、⼈}[HSK 3]
  \definition{s.}{valor; o trabalho social necessário condensado nos produtos | valor; importância; efeitos positivos}
\end{EntryWithPhonetic}

\begin{EntryWithPhonetic}{驾}{jia4}{8}{⾺}
  \definition*{s.}{Sobrenome Jia}
  \definition{s.}{carruagem do imperador; refere-se especificamente ao carro do imperador, referindo-se ao imperador | referindo-se a um veículo, usado como um termo respeitoso para uma pessoa}
  \definition{v.}{atrelar; puxar (uma carroça, etc.) | dirigir (um veículo); pilotar (um avião); velejar (um barco) | montar; cavalgar}
\end{EntryWithPhonetic}

\begin{EntryWithPhonetic}{驾驶}{jia4shi3}{8,8}{⾺、⾺}[HSK 5]
  \definition{v.}{dirigir; pilotar; conduzir; guiar; operar (um carro, navio, avião, trator, etc.) para fazê-lo mover}
\end{EntryWithPhonetic}

\begin{EntryWithPhonetic}{驾照}{jia4 zhao4}{8,13}{⾺、⽕}[HSK 5]
  \definition[本,张]{s.}{carteira de motorista}
\end{EntryWithPhonetic}

\begin{EntryWithPhonetic}{架}{jia4}{9}{⽊}[HSK 3]
  \definition{clas.}{usado para coisas com pilares ou componentes mecânicos | quadrado (usado para montanhas)}
  \definition{s.}{estrutura; organização do corpo humano ou das coisas | prateleira; estante; suporte; componentes que sustentam objetos ou utensílios para colocar objetos, etc.}
  \definition{v.}{colocar para cima; erigir | brigar; discutir | resistir; repelir; afastar | sequestrar; levar alguém à força}
\end{EntryWithPhonetic}

\begin{EntryWithPhonetic}{架式}{jia4shi5}{9,6}{⽊、⼷}
  \variantof{架势}
\end{EntryWithPhonetic}

\begin{EntryWithPhonetic}{架势}{jia4shi5}{9,8}{⽊、⼒}
  \definition{s.}{postura | atitude | posição (sobre um assunto, etc.)}
\end{EntryWithPhonetic}

\begin{EntryWithPhonetic}{假}{jia4}{11}{⼈}
  \definition[个,天]{s.}{feriado; férias; período de suspensão temporária do trabalho ou dos estudos, legal ou aprovado | licença; afastamento temporário; período de licença temporária do trabalho ou dos estudos, após aprovação}
  \seeref{jia3}
\end{EntryWithPhonetic}

\begin{EntryWithPhonetic}{假期}{jia4 qi1}{11,12}{⼈、⽉}[HSK 2]
  \definition[个,段,次,种]{s.}{férias; feriados; período de licença}
\end{EntryWithPhonetic}

\begin{EntryWithPhonetic}{假日}{jia4 ri4}{11,4}{⼈、⽇}[HSK 6]
  \definition[节]{s.}{feriado; dia de folga}
\end{EntryWithPhonetic}

\begin{EntryWithPhonetic}{奸}{jian1}{6}{⼥}
  \definition{adj.}{perverso; maligno; traiçoeiro; malicioso}
  \definition{s.}{traidor; espião | pessoa perversa; pessoa traiçoeira | relações sexuais ilícitas; comportamento sexual impróprio}
  \definition{v.}{ter relações sexuais ilícitas}
\end{EntryWithPhonetic}

\begin{EntryWithPhonetic}{奸夫}{jian1fu1}{6,4}{⼥、⼤}
  \definition{s.}{homem adúltero}
\end{EntryWithPhonetic}

\begin{EntryWithPhonetic}{尖}{jian1}{6}{⼩}[HSK 6]
  \definition{adj.}{pontiagudo; afilado; agudo | agudo; estridente; penetrante | mesquinho; pão-duro | mordaz; cáustico}
  \definition{s.}{ponto; ponta; topo | o melhor do seu tipo; a melhor escolha; a nata da safra; uma pessoa ou coisa notável}
  \definition{v.}{tornar (a voz, etc.) aguda; estridente}
\end{EntryWithPhonetic}

\begin{EntryWithPhonetic}{坚}{jian1}{7}{⼟}
  \definition*{s.}{Sobrenome Jian}
  \definition{adj.}{duro; firme; sólido; forte | firme; resoluto; constante}
  \definition{s.}{fortaleza; fortificação; um ponto fortemente fortificado; coisas sólidas, principalmente referindo-se a posições | armadura}
\end{EntryWithPhonetic}

\begin{EntryWithPhonetic}{坚持}{jian1chi2}{7,9}{⼟、⼿}[HSK 3]
  \definition{v.}{persistir em; perseverar em; defender; insistir em; manter-se fiel a; aderir a; persistir com determinação e não desistir quando se depara com dificuldades | aderir a; insistir em; não alterar (os princípios, opiniões, pontos de vista originais, etc.)}
\end{EntryWithPhonetic}

\begin{EntryWithPhonetic}{坚定}{jian1ding4}{7,8}{⼟、⼧}[HSK 5]
  \definition{adj.}{firme; inabalável; inamovível; (posição, opinião, vontade, etc.) firme e estável, inabalável}
  \definition{v.}{fortalecer}
\end{EntryWithPhonetic}

\begin{EntryWithPhonetic}{坚固}{jian1gu4}{7,8}{⼟、⼞}[HSK 4]
  \definition{adj.}{firme; sólido; robusto; forte; durável; firmemente unidos e inquebráveis}
\end{EntryWithPhonetic}

\begin{EntryWithPhonetic}{坚决}{jian1jue2}{7,6}{⼟、⼎}[HSK 3]
  \definition{adj.}{firme; resoluto; (atitude, opinião, ação, etc.) determinado e inabalável}
\end{EntryWithPhonetic}

\begin{EntryWithPhonetic}{坚强}{jian1qiang2}{7,12}{⼟、⼸}[HSK 3]
  \definition{adj.}{forte; firme; convicto; (qualidades humanas, personalidade, determinação, etc.) firme e forte, não vacila diante das dificuldades}
  \definition{v.}{fortalecer; tornar forte; é a qualidade, a determinação, etc., que não vacilam}
\end{EntryWithPhonetic}

\begin{EntryWithPhonetic}{坚守}{jian1shou3}{7,6}{⼟、⼧}
  \definition{v.}{agarrar-se}
\end{EntryWithPhonetic}

\begin{EntryWithPhonetic}{间}{jian1}{7}{⾨}[HSK 1]
  \definition{clas.}{a menor unidade de uma casa; a menor unidade habitacional; cômodo}
  \definition{s.}{espaço entre duas partes  | (em um) tempo ou espaço definido | sala; quarto | uma seção de uma sala ou o espaço lateral entre dois pares de pilares | com um tempo ou espaço definido}
  \seeref{jian4}
\end{EntryWithPhonetic}

\begin{EntryWithPhonetic}{浅}{jian1}{8}{⽔}
  \definition{adj.}{murmurando, fluindo suavemente, gorgolejando suavemente}
  \definition{s.}{Onomatopéia: som de água em movimento}
  \seeref{qian3}
\end{EntryWithPhonetic}

\begin{EntryWithPhonetic}{肩}{jian1}{8}{⾁}[HSK 5]
  \definition*{s.}{Sobrenome Jian}
  \definition{s.}{ombro; torso}
  \definition{v.}{assumir; empreender; carregar; suportar; suportar um fardo}
\end{EntryWithPhonetic}

\begin{EntryWithPhonetic}{肩膀}{jian1bang3}{8,14}{⾁、⾁}
  \definition{s.}{ombro}
\end{EntryWithPhonetic}

\begin{EntryWithPhonetic}{艰}{jian1}{8}{⾉}
  \definition{adj.}{difícil; duro}
\end{EntryWithPhonetic}

\begin{EntryWithPhonetic}{艰苦}{jian1ku3}{8,8}{⾉、⾋}[HSK 5]
  \definition{adj.}{duro; resistente; árduo; difícil; condições de trabalho ou de vida ruins que tornam as pessoas miseráveis}
\end{EntryWithPhonetic}

\begin{EntryWithPhonetic}{艰难}{jian1nan2}{8,10}{⾉、⾫}[HSK 5]
  \definition{adj.}{duro; árduo; difícil}
\end{EntryWithPhonetic}

\begin{EntryWithPhonetic}{兼}{jian1}{10}{⼋}
  \definition{conj.}{e (ocupando dois ou mais cargos (oficiais) ao mesmo tempo)}
\end{EntryWithPhonetic}

\begin{EntryWithPhonetic}{监}{jian1}{10}{⽫}
  \definition{s.}{prisão; cadeia}
  \definition{v.}{supervisionar; inspecionar; observar}
\end{EntryWithPhonetic}

\begin{EntryWithPhonetic}{监测}{jian1 ce4}{10,9}{⽫、⽔}[HSK 6]
  \definition{v.}{monitorar; supervisionar e testar}
\end{EntryWithPhonetic}

\begin{EntryWithPhonetic}{监督}{jian1du1}{10,13}{⽫、⽬}[HSK 6]
  \definition[个,位,名]{s.}{monitoramento; supervisão; pessoas que supervisionam}
  \definition{v.}{controlar; supervisionar; superintender; monitorar e supervisionar de perto}
\end{EntryWithPhonetic}

\begin{EntryWithPhonetic}{监狱}{jian1yu4}{10,9}{⽫、⽝}
  \definition{s.}{prisão}
\end{EntryWithPhonetic}

\begin{EntryWithPhonetic}{渐}{jian1}{11}{⽔}
  \definition{v.}{encharcar; ficar saturado com | fluir para}
  \seeref{jian4}
\end{EntryWithPhonetic}

\begin{EntryWithPhonetic}{煎}{jian1}{13}{⽕}
  \definition{v.}{fritar | refogar}
\end{EntryWithPhonetic}

\begin{EntryWithPhonetic}{煎饼}{jian1bing3}{13,9}{⽕、⾷}
  \definition[张]{s.}{jianbing, crepe chinês | panqueca}
\end{EntryWithPhonetic}

\begin{EntryWithPhonetic}{煎蛋}{jian1dan4}{13,11}{⽕、⾍}
  \definition{s.}{ovos fritos}
\end{EntryWithPhonetic}

\begin{EntryWithPhonetic}{俭}{jian3}{9}{⼈}
  \definition*{s.}{Sobrenome Jian}
  \definition{adj.}{econômico; frugal | querendo; faltando; curto}
\end{EntryWithPhonetic}

\begin{EntryWithPhonetic}{俭省}{jian3sheng3}{9,9}{⼈、⽬}
  \definition{adj.}{econômico}
\end{EntryWithPhonetic}

\begin{EntryWithPhonetic}{柬}{jian3}{9}{⽊}
  \definition*{s.}{Sobrenome Jian}
  \definition[张,封]{s.}{cartão; nota; carta; um termo geral para cartas, cartões de visita, postagens, etc.}
\end{EntryWithPhonetic}

\begin{EntryWithPhonetic}{柬埔寨}{jian3pu3zhai4}{9,10,14}{⽊、⼟、⼧}
  \definition*{s.}{Camboja}
\end{EntryWithPhonetic}

\begin{EntryWithPhonetic}{捡}{jian3}{10}{⼿}[HSK 6]
  \definition{v.}{coletar; reunir; apanhar; pegar coisas do chão}
\end{EntryWithPhonetic}

\begin{EntryWithPhonetic}{减}{jian3}{11}{⼎}[HSK 4]
  \definition*{s.}{Sobrenome Jian}
  \definition{v.}{subtrair; remover uma parte da quantidade original | reduzir; diminuir; cortar}
\end{EntryWithPhonetic}

\begin{EntryWithPhonetic}{减肥}{jian3/fei2}{11,8}{⼎、⾁}[HSK 4]
  \definition{v.+compl.}{perder peso; dieta, exercícios, medicamentos, massagem, cirurgia, etc., para reduzir o excesso de gordura corporal, de modo que o grau de obesidade seja reduzido}
\end{EntryWithPhonetic}

\begin{EntryWithPhonetic}{减轻}{jian3 qing1}{11,9}{⼎、⾞}[HSK 5]
  \definition{v.}{aliviar; remeter; clarear; facilitar; mitigar}
\end{EntryWithPhonetic}

\begin{EntryWithPhonetic}{减少}{jian3shao3}{11,4}{⼎、⼩}[HSK 4]
  \definition{v.}{cair; reduzir; diminuir; subtrair uma parte}
\end{EntryWithPhonetic}

\begin{EntryWithPhonetic}{剪}{jian3}{11}{⼑}[HSK 5]
  \definition[把]{s.}{tesouras; tesouras de poda; cortadores | pinças; tenazes}
  \definition{v.}{cortar; aparar; tosquiar; cortar (com uma tesoura) | exterminar; eliminar; acabar com}
\end{EntryWithPhonetic}

\begin{EntryWithPhonetic}{剪刀}{jian3dao1}{11,2}{⼑、⼑}[HSK 5]
  \definition[把,个]{s.}{tesoura; tesoura de jardim; instrumento de ferro para cortar tecido, papel, barbante, etc., com duas lâminas interligadas que podem ser abertas e fechadas}
\end{EntryWithPhonetic}

\begin{EntryWithPhonetic}{剪子}{jian3 zi5}{11,3}{⼑、⼦}[HSK 5]
  \definition[把]{s.}{tesouras; tesouras de podar; tosquiadeiras}
\end{EntryWithPhonetic}

\begin{EntryWithPhonetic}{检}{jian3}{11}{⽊}
  \definition*{s.}{Sobrenome Jian}
  \definition{v.}{verificar; inspecionar; examinar | conter-se; ter cuidado na conduta}
\end{EntryWithPhonetic}

\begin{EntryWithPhonetic}{检测}{jian3 ce4}{11,9}{⽊、⽔}[HSK 4]
  \definition{v.}{testar; detectar; verificar}
\end{EntryWithPhonetic}

\begin{EntryWithPhonetic}{检查}{jian3cha2}{11,9}{⽊、⽊}[HSK 2]
  \definition[份,个,次]{s.}{autocrítica; reconhecer e criticar os próprios erros verbais ou escritos}
  \definition{v.}{verificar; inspecionar; examinar; verificar cuidadosamente para descobrir o problema | criticar a si mesmo; identificar seus pontos fracos e erros, e criticar seu próprio comportamento}
\end{EntryWithPhonetic}

\begin{EntryWithPhonetic}{检验}{jian3yan4}{11,10}{⽊、⾺}[HSK 5]
  \definition{v.}{testar; examinar; inspecionar}
\end{EntryWithPhonetic}

\begin{EntryWithPhonetic}{简}{jian3}{13}{⽵}
  \definition*{s.}{Sobrenome Jian}
  \definition{adj.}{simples; simplificado; breve (oposto a 繁) | breve; em resumo; em poucas palavras}
  \definition{s.}{Arcaico: tiras de bambu (para escrever) | carta; correspondência}
  \definition{v.}{simplificar | (literário) selecionar; escolher}
  \seealsoref{繁}{fan2}
\end{EntryWithPhonetic}

\begin{EntryWithPhonetic}{简单}{jian3dan1}{13,8}{⽵、⼗}[HSK 3]
  \definition{adj.}{simples; descomplicado; estrutura simples; poucas complicações; fácil de entender, usar ou lidar | comum; lugar-comum; (experiência, capacidade, etc.) comum (usado principalmente em frases negativas) | casual; simplificado; precipitado; pouco cuidadoso}
\end{EntryWithPhonetic}

\begin{EntryWithPhonetic}{简介}{jian3 jie4}{13,4}{⽵、⼈}[HSK 6]
  \definition{s.}{breve introdução; sinopse; relato resumido}
  \definition{v.}{fazer um breve relato de (algo)}
\end{EntryWithPhonetic}

\begin{EntryWithPhonetic}{简历}{jian3li4}{13,4}{⽵、⼚}[HSK 4]
  \definition[个,份]{s.}{currículo; \emph{curriculum vitae} (CV); notas biográficas}
\end{EntryWithPhonetic}

\begin{EntryWithPhonetic}{简直}{jian3zhi2}{13,8}{⽵、⽬}[HSK 3]
  \definition{adv.}{simplesmente; de forma alguma; praticamente; significa “exatamente assim” (tom exagerado)}
\end{EntryWithPhonetic}

\begin{EntryWithPhonetic}{见}{jian4}{4}{⾒}[HSK 1][Kangxi 147]
  \definition*{s.}{Sobrenome Jian}
  \definition{part.}{usado antes de um verbo para indicar voz passiva ou para expressar como isso me afeta}
  \definition{s.}{visão; ideia; opinião sobre algo; ponto de vista}
  \definition{v.}{ver; avistar | encontrar-se com; ser exposto a | parecer ser; mostrar evidência de | ver; referir-se a; indicar a fonte ou o local onde deve ser consultado | ver; encontrar; convocar}
  \seeref{xian4}
\end{EntryWithPhonetic}

\begin{EntryWithPhonetic}{见到}{jian4 dao4}{4,8}{⾒、⼑}[HSK 2]
  \definition{v.}{ver | encontrar; esbarrar; deparar-se com}
\end{EntryWithPhonetic}

\begin{EntryWithPhonetic}{见过}{jian4 guo4}{4,6}{⾒、⾡}[HSK 2]
  \definition{s.}{visto (ver); já viu alguém ou algo; indica um momento no passado; alguém já viu ou encontrou um determinado objeto}
\end{EntryWithPhonetic}

\begin{EntryWithPhonetic}{见面}{jian4/mian4}{4,9}{⾒、⾯}[HSK 1]
  \definition{v.+compl.}{encontrar-se com alguém;  ver um ao outro; ver alguém face-a-face}
\end{EntryWithPhonetic}

\begin{EntryWithPhonetic}{件}{jian4}{6}{⼈}[HSK 2]
  \definition*{s.}{Sobrenome Jian}
  \definition{clas.}{item; peça; artigo; usado para coisas individuais}
  \definition{s.}{refere-se a coisas que podem ser contadas uma a uma | papel; carta; documento; correspondência}
\end{EntryWithPhonetic}

\begin{EntryWithPhonetic}{间}{jian4}{7}{⾨}
  \definition{s.}{espaço entre as duas partes; abertura; lacuna}
  \definition{v.}{separar | semear a discórdia | desbastar (mudas); podar; remover ou arrancar as mudas em excesso}
  \seeref{jian1}
\end{EntryWithPhonetic}

\begin{EntryWithPhonetic}{间或}{jian4huo4}{7,8}{⾨、⼽}
  \definition{adv.}{às vezes | ocasionalmente | de vez em quando}
\end{EntryWithPhonetic}

\begin{EntryWithPhonetic}{间接}{jian4jie1}{7,11}{⾨、⼿}[HSK 5]
  \definition{adj.}{indireto; de segunda mão; em oposição a 直接}
  \seealsoref{直接}{zhi2jie1}
\end{EntryWithPhonetic}

\begin{EntryWithPhonetic}{建}{jian4}{8}{⼵}[HSK 3]
  \definition*{s.}{Província de Fujian | Rio Jian Jiang (na província de Fujian) | Sobrenome Jian}
  \definition{v.}{construir; construir; erigir | estabelecer; configurar; fundar | propor; defender; apresentar (suas próprias opiniões)}
\end{EntryWithPhonetic}

\begin{EntryWithPhonetic}{建成}{jian4 cheng2}{8,6}{⼵、⼽}[HSK 3]
  \definition{v.}{terminar a construção}
\end{EntryWithPhonetic}

\begin{EntryWithPhonetic}{建立}{jian4li4}{8,5}{⼵、⽴}[HSK 3]
  \definition{v.}{estabelecer; construir; começar a construir | vir a ser; começar a surgir; começar a se formar}
\end{EntryWithPhonetic}

\begin{EntryWithPhonetic}{建立者}{jian4li4zhe3}{8,5,8}{⼵、⽴、⽼}
  \definition{s.}{fundador; construtor}
\end{EntryWithPhonetic}

\begin{EntryWithPhonetic}{建设}{jian4she4}{8,6}{⼵、⾔}[HSK 3]
  \definition{s.}{reconstrução; desenvolvimento; trabalhos relacionados com a construção}
  \definition{v.}{construir; edificar; (Estado ou coletividade) criar novos empreendimentos ou aumento de novas instalações}
\end{EntryWithPhonetic}

\begin{EntryWithPhonetic}{建设性}{jian4she4xing4}{8,6,8}{⼵、⾔、⼼}
  \definition{adj.}{construtivo}
  \definition{s.}{construtividade}
\end{EntryWithPhonetic}

\begin{EntryWithPhonetic}{建设者}{jian4she4zhe3}{8,6,8}{⼵、⾔、⽼}
  \definition{s.}{construtor}
\end{EntryWithPhonetic}

\begin{EntryWithPhonetic}{建议}{jian4yi4}{8,5}{⼵、⾔}[HSK 3]
  \definition[个,点,条]{s.}{proposta; sugestão; recomendação; para que alguém ou alguma coisa evolua para melhor, para o coletivo; pontos de vista e opiniões apresentados pelos líderes, etc.}
  \definition{v.}{propor; sugerir; recomendar; em relação a determinada pessoa ou situação, apresentar seus pontos de vista e opiniões ao coletivo, aos líderes ou a indivíduos, para que as coisas evoluam para melhor}
\end{EntryWithPhonetic}

\begin{EntryWithPhonetic}{建造}{jian4 zao4}{8,10}{⼵、⾡}[HSK 5]
  \definition{v.}{construir; edificar}
\end{EntryWithPhonetic}

\begin{EntryWithPhonetic}{建筑}{jian4zhu4}{8,12}{⼵、⽵}[HSK 5]
  \definition[座,幢,排]{s.}{construção; estrutura; edifício; prédio}
  \definition{v.}{construir; erguer; edificar; construir casas, estradas, pontes, etc.}
\end{EntryWithPhonetic}

\begin{EntryWithPhonetic}{剑}{jian4}{9}{⼑}[HSK 6]
  \definition[把,口]{s.}{espada; sabre; florete}
\end{EntryWithPhonetic}

\begin{EntryWithPhonetic}{剑客}{jian4ke4}{9,9}{⼑、⼧}
  \definition{s.}{espada | esgrimista, espadachim}
\end{EntryWithPhonetic}

\begin{EntryWithPhonetic}{贱}{jian4}{9}{⾙}
  \definition*{s.}{Sobrenome Jian}
  \definition{adj.}{baixo preço; barato (oposto a 贵) | humilde (oposto a 贵) | baixo; básico; desprezível | humilde; baixa posição social}
  \definition{pron.}{meu (autodepreciativo)}
  \seealsoref{贵}{gui4}
\end{EntryWithPhonetic}

\begin{EntryWithPhonetic}{健}{jian4}{10}{⼈}
  \definition{adj.}{forte; saudável; bem definido | ser forte em; ser bom em; apresentar um grau superior à média em determinado aspecto}
  \definition{v.}{fortalecer; endurecer; revigorar}
\end{EntryWithPhonetic}

\begin{EntryWithPhonetic}{健康}{jian4kang1}{10,11}{⼈、⼴}[HSK 2]
  \definition{adj.}{em forma; saudável; descreve que a pessoa está em ótimo estado físico ou mental, sem nenhum problema | sudável; tudo está normal, sem problemas | saudável; livre de doenças; bom para a saúde}
  \definition{s.}{saúde; físico; estado de saúde}
\end{EntryWithPhonetic}

\begin{EntryWithPhonetic}{健全}{jian4quan2}{10,6}{⼈、⼊}[HSK 5]
  \definition{adj.}{saudável; íntegro; capaz; apto; robusto e sem mácula | sólido; completo; perfeito}
  \definition{v.}{aperfeiçoar; melhorar; fortalecer; reforçar}
\end{EntryWithPhonetic}

\begin{EntryWithPhonetic}{健身}{jian4/shen1}{10,7}{⼈、⾝}[HSK 4]
  \definition{s.}{exercício físico | \emph{fitness}}
  \definition{v.+compl.}{exercitar-se; manter a forma; praticar um esporte, especialmente a ginástica, inclusive em aparelhos, para desenvolver força, flexibilidade, aumentar a resistência, melhorar a coordenação e o controle de todas as partes do corpo}
\end{EntryWithPhonetic}

\begin{EntryWithPhonetic}{渐}{jian4}{11}{⽔}
  \definition{adv.}{gradualmente; por graus}
  \seeref{jian1}
\end{EntryWithPhonetic}

\begin{EntryWithPhonetic}{渐渐}{jian4 jian4}{11,11}{⽔、⽔}[HSK 4]
  \definition{adv.}{gradualmente; pouco a pouco; passo a passo; indica um aumento ou diminuição gradual em grau ou quantidade}
\end{EntryWithPhonetic}

\begin{EntryWithPhonetic}{鉴}{jian4}{13}{⾦}[HSK 6]
  \definition*{s.}{Sobrenome Jian}
  \definition{expr.}{uma expressão idiomática antiga usada para escrever cartas, depois da saudação inicial para pedir que alguém leia a carta}
  \definition{s.}{espelho (feito de bronze ou latão); espelho de bronze antigo | advertência; lição objetiva}
  \definition{v.}{Literário: refletir; espelhar | inspecionar; examinar; escrutinar; olhar cuidadosamente}
\end{EntryWithPhonetic}

\begin{EntryWithPhonetic}{鉴定}{jian4ding4}{13,8}{⾦、⼧}[HSK 6]
  \definition{s.}{avaliação dos pontos fortes e fracos de uma pessoa; avaliação de pessoas ou coisas}
  \definition{v.}{avaliar; identificar; autenticar; determinar; identificar e determinar (a autenticidade e a qualidade das coisas) | conduzir uma avaliação; avaliar o desempenho de uma pessoa ao longo de um determinado período de tempo}
\end{EntryWithPhonetic}

\begin{EntryWithPhonetic}{键}{jian4}{13}{⾦}[HSK 5]
  \definition[个]{s.}{chave | tecla (de uma máquina de escrever, piano, etc.) | Química: ligação | Literário: ferrolho (de uma porta) | pino (para máquinas)  | etapa crucial}
\end{EntryWithPhonetic}

\begin{EntryWithPhonetic}{键盘}{jian4pan2}{13,11}{⾦、⽫}[HSK 5]
  \definition[台,个]{s.}{teclado; cravo; painel de teclas}
\end{EntryWithPhonetic}

\begin{EntryWithPhonetic}{箭}{jian4}{15}{⽵}[HSK 6]
  \definition[支]{s.}{seta | distância percorrida por uma flecha}
\end{EntryWithPhonetic}

\begin{EntryWithPhonetic}{江}{jiang1}{6}{⽔}[HSK 4]
  \definition*{s.}{Rio Changjiang | Sobrenome Jiang}
  \definition[条,道]{s.}{rio grande}
\end{EntryWithPhonetic}

\begin{EntryWithPhonetic}{江南水乡}{jiang1nan2shui3xiang1}{6,9,4,3}{⽔、⼗、⽔、⼄}
  \definition*{s.}{Vila Aquática de Jiangnan | Cidades Aquáticas}
\end{EntryWithPhonetic}

\begin{EntryWithPhonetic}{江水}{jiang1shui3}{6,4}{⽔、⽔}
  \definition{s.}{água do rio}
\end{EntryWithPhonetic}

\begin{EntryWithPhonetic}{江苏}{jiang1su1}{6,7}{⽔、⾋}
  \definition*{s.}{Província de Jiangsu}
\end{EntryWithPhonetic}

\begin{EntryWithPhonetic}{江西}{jiang1xi1}{6,6}{⽔、⾑}
  \definition*{s.}{Jiangxi}
\end{EntryWithPhonetic}

\begin{EntryWithPhonetic}{姜}{jiang1}{9}{⼥}
  \definition*{s.}{Sobrenome Jiang}
  \definition[磅,斤,两]{s.}{gengibre; rizoma de gengibre}
\end{EntryWithPhonetic}

\begin{EntryWithPhonetic}{将}{jiang1}{9}{⼨}[HSK 5]
  \definition*{s.}{Sobrenome Jiang}
  \definition{adv.}{estar indo para; parcialmente\dots parcialmente\dots}
  \definition{part.}{expressar uma direção, como 进来, 出去; usado no meio de verbos e complementos que indicam tendência, como 进来, 出去, etc.}
  \definition{prep.}{com; por meio de; por | usado da mesma forma que 把}
  \definition{v.}{fazer algo; lidar com (um assunto) | dar um cheque-mate | cuidar (da saúde) | incitar alguém a agir; desafiar; estimular | segurar; pegar | colocar; tirar | levar; trazer | dar suporte; dar apoio}
  \seeref{jiang4}
  \seeref{qiang1}
  \seealsoref{把}{ba3}
  \seealsoref{出去}{chu1 qu4}
  \seealsoref{进来}{jin4 lai2}
\end{EntryWithPhonetic}

\begin{EntryWithPhonetic}{将近}{jiang1jin4}{9,7}{⼨、⾡}[HSK 3]
  \definition{adv.}{quase}
\end{EntryWithPhonetic}

\begin{EntryWithPhonetic}{将军}{jiang1/jun1}{9,6}{⼨、⼍}[HSK 6]
  \definition[位,名]{s.}{general; geralmente se refere a generais seniores}
  \definition{v.+compl.}{dar xeque-mate; atacar o general ou rei do oponente no xadrez; colocar alguém em grandes apuros; metáfora para dar a alguém um problema difícil ou dificultar a tarefa para essa pessoa}
\end{EntryWithPhonetic}

\begin{EntryWithPhonetic}{将来}{jiang1lai2}{9,7}{⼨、⽊}[HSK 3]
  \definition[个]{s.}{no futuro (geralmente se refere a um período mais longo)}
\end{EntryWithPhonetic}

\begin{EntryWithPhonetic}{将要}{jiang1 yao4}{9,9}{⼨、⾑}[HSK 5]
  \definition{adv.}{irá; deverá; estará prestes a; irá a; indica que um ato ou situação ocorre logo em seguida}
\end{EntryWithPhonetic}

\begin{EntryWithPhonetic}{讲}{jiang3}{6}{⾔}[HSK 2]
  \definition[种]{s.}{palestra; discurso}
  \definition{v.}{contar; falar | explicar; transmitir oralmente; esclarecer | negociar; barganhar | ser exigente com; valorizar; dar importância}
\end{EntryWithPhonetic}

\begin{EntryWithPhonetic}{讲话}{jiang3 hua4}{6,8}{⾔、⾔}[HSK 2]
  \definition[个]{s.}{discurso; palestra | guia; introdução}
  \definition{v.}{falar; conversar; dirigir-se a alguém | criticar}
\end{EntryWithPhonetic}

\begin{EntryWithPhonetic}{讲究}{jiang3jiu5}{6,7}{⾔、⽳}[HSK 4]
  \definition{adj.}{requintado; elegante; de bom gosto; exigente com a vida e com outros aspectos, buscando alto nível, qualidade e detalhes}
  \definition{s.}{estudo cuidadoso; algo que merece atenção; elementos e aspectos que merecem atenção especial}
  \definition{v.}{dar ênfase a; ser específico sobre; prestar atenção a}
\end{EntryWithPhonetic}

\begin{EntryWithPhonetic}{讲课}{jiang3 ke4}{6,10}{⾔、⾔}[HSK 6]
  \definition{v.}{ensinar; dar palestras; proferir uma palestra | dar uma lição (palestra)}
\end{EntryWithPhonetic}

\begin{EntryWithPhonetic}{讲述}{jiang3shu4}{6,8}{⾔、⾡}
  \definition{v.}{falar sobre | narrar | descrever}
\end{EntryWithPhonetic}

\begin{EntryWithPhonetic}{讲座}{jiang3zuo4}{6,10}{⾔、⼴}[HSK 4]
  \definition[场,次]{s.}{palestra; um curso de palestras; a forma de instrução usada para ensinar um determinado assunto ou tópico, geralmente por meio de palestras ao vivo, seriados de rádio ou televisão ou seriados de jornal.}
\end{EntryWithPhonetic}

\begin{EntryWithPhonetic}{奖}{jiang3}{9}{⼤}[HSK 4]
  \definition[个,次]{s.}{prêmio; recompensa | elogio; loa}
  \definition{v.}{elogiar; recompensar; recomendar; incentivar}
\end{EntryWithPhonetic}

\begin{EntryWithPhonetic}{奖金}{jiang3jin1}{9,8}{⼤、⾦}[HSK 4]
  \definition[个,笔]{s.}{bônus; recompensa; prêmio; prêmio em dinheiro; dinheiro de recompensa, dinheiro dado às pessoas para incentivá-las ou elogiá-las por terem se saído bem em alguma coisa}
\end{EntryWithPhonetic}

\begin{EntryWithPhonetic}{奖励}{jiang3li4}{9,7}{⼤、⼒}[HSK 5]
  \definition{s.}{prêmio; recompensa; dinheiro ou honras dadas em troca de elogios ou incentivos}
  \definition{v.}{recompensar; incentivar; encorajar}
\end{EntryWithPhonetic}

\begin{EntryWithPhonetic}{奖学金}{jiang3 xue2 jin1}{9,8,8}{⼤、⼦、⾦}[HSK 4]
  \definition[笔]{s.}{bolsa de estudos; exposição; prêmios concedidos por escolas, organizações ou indivíduos a alunos com bom desempenho acadêmico}
\end{EntryWithPhonetic}

\begin{EntryWithPhonetic}{匠}{jiang4}{6}{⼕}
  \definition*{s.}{Sobrenome Jiang}
  \definition{s.}{artesão | pessoa de realizações notáveis ​​em um campo específico; mestre}
\end{EntryWithPhonetic}

\begin{EntryWithPhonetic}{降}{jiang4}{8}{⾩}[HSK 4]
  \definition*{s.}{Sobrenome Jiang}
  \definition{v.}{cair; descer; quedar-se (oposto de 升 ) | diminuir; reduzir; cair; abaixar | nascer}
  \seealsoref{升}{sheng1}
\end{EntryWithPhonetic}

\begin{EntryWithPhonetic}{降低}{jiang4di1}{8,7}{⾩、⼈}[HSK 4]
  \definition{v.}{reduzir; cortar; diminuir; rebaixar; cair; abaixar}
\end{EntryWithPhonetic}

\begin{EntryWithPhonetic}{降价}{jiang4 jia4}{8,6}{⾩、⼈}[HSK 4]
  \definition{v.}{ficar mais barato; cortar o preço; reduzir o preço}
\end{EntryWithPhonetic}

\begin{EntryWithPhonetic}{降落}{jiang4luo4}{8,12}{⾩、⾋}[HSK 4]
  \definition{v.}{aterrissar; descer; descer do céu}
\end{EntryWithPhonetic}

\begin{EntryWithPhonetic}{降温}{jiang4 wen1}{8,12}{⾩、⽔}[HSK 4]
  \definition{v.}{baixar a temperatura (como em uma oficina);  recusar | cair a temperatura | esfriar; resfriar; metáfora para um declínio no entusiasmo ou uma diminuição no ímpeto de algo}
\end{EntryWithPhonetic}

\begin{EntryWithPhonetic}{将}{jiang4}{9}{⼨}
  \definition{s.}{general; nome do posto; abaixo de marechal de campo; acima de coronel}
  \definition{v.}{comandar; liderar}
  \seeref{jiang1}
  \seeref{qiang1}
\end{EntryWithPhonetic}

\begin{EntryWithPhonetic}{强}{jiang4}{12}{⼸}
  \definition{adj.}{teimoso; inflexível}
  \seeref{qiang2}
  \seeref{qiang3}
\end{EntryWithPhonetic}

\begin{EntryWithPhonetic}{酱}{jiang4}{13}{⾣}[HSK 6]
  \definition{adj.}{marinado em molho de soja; cozido em molho de soja}
  \definition{s.}{molho espesso feito de soja, farinha, etc. | molho; pasta; geleia | um condimento pastoso feito de feijão, trigo fermentados e sal}
  \definition{v.}{cozinhar ou conservar em molho de soja}
\end{EntryWithPhonetic}

\begin{EntryWithPhonetic}{酱油}{jiang4you2}{13,8}{⾣、⽔}[HSK 6]
  \definition[袋,瓶,壶,桶]{s.}{molho de soja}
\end{EntryWithPhonetic}

\begin{EntryWithPhonetic}{犟}{jiang4}{16}{⽜}
  \variantof{强}
\end{EntryWithPhonetic}

\begin{EntryWithPhonetic}{交}{jiao1}{6}{⼇}[HSK 2]
  \definition*{s.}{Sobrenome Jiao}
  \definition{adv.}{mutuamente; recíprocamente; um ao outro | juntos; simultaneamente}
  \definition{s.}{amigo; conhecido; amizade; relacionamento | transação comercial; negócio; barganha | queda}
  \definition{v.}{entregar | (de lugares ou períodos de tempo) cruzar; encontrar; unir | chegar (a uma determinada hora ou estação); estabelecer-se; vir | cruzar; intersectar | associar-se a | ter relações sexuais | acasalar; reproduzir-se | transferir as coisas para as partes interessadas | unir (lugares ou períodos de tempo)}
\end{EntryWithPhonetic}

\begin{EntryWithPhonetic}{交班}{jiao1ban1}{6,10}{⼇、⽟}
  \definition{v.}{passar para o próximo turno de trabalho}
\end{EntryWithPhonetic}

\begin{EntryWithPhonetic}{交杯酒}{jiao1bei1jiu3}{6,8,10}{⼇、⽊、⾣}
  \definition{s.}{copo de vinho nupcial}
\end{EntryWithPhonetic}

\begin{EntryWithPhonetic}{交叉}{jiao1cha1}{6,3}{⼇、⼜}
  \definition{v.}{cruzar | sobrepor}
\end{EntryWithPhonetic}

\begin{EntryWithPhonetic}{交叉点}{jiao1cha1dian3}{6,3,9}{⼇、⼜、⽕}
  \definition{s.}{encruzilhada | cruzamento | junção}
\end{EntryWithPhonetic}

\begin{EntryWithPhonetic}{交叉口}{jiao1cha1kou3}{6,3,3}{⼇、⼜、⼝}
  \definition{s.}{intersecção (rodovia)}
\end{EntryWithPhonetic}

\begin{EntryWithPhonetic}{交代}{jiao1dai4}{6,5}{⼇、⼈}[HSK 5]
  \definition{v.}{contar; entregar | ordenar; insistir; contar aos outros sobre suas intenções, instruções | contar; admitir}
\end{EntryWithPhonetic}

\begin{EntryWithPhonetic}{交叠}{jiao1die2}{6,13}{⼇、⼜}
  \definition{s.}{sobreposição}
\end{EntryWithPhonetic}

\begin{EntryWithPhonetic}{交费}{jiao1 fei4}{6,9}{⼇、⾙}[HSK 3]
  \definition{v.}{pagar taxas ou impostos; pagar uma taxa ou imposto}
\end{EntryWithPhonetic}

\begin{EntryWithPhonetic}{交给}{jiao1 gei3}{6,9}{⼇、⽷}[HSK 2]
  \definition{v.}{entregar para | dar para}
\end{EntryWithPhonetic}

\begin{EntryWithPhonetic}{交媾}{jiao1gou4}{6,13}{⼇、⼥}
  \definition{v.}{copular | ter relações sexuais}
\end{EntryWithPhonetic}

\begin{EntryWithPhonetic}{交换}{jiao1huan4}{6,10}{⼇、⼿}[HSK 4]
  \definition{v.}{trocar; permutar; comutar; intercambiar}
\end{EntryWithPhonetic}

\begin{EntryWithPhonetic}{交际}{jiao1ji4}{6,7}{⼇、⾩}[HSK 4]
  \definition{s.}{contato; comunicação; relações sociais; contato interpessoal, socialização}
\end{EntryWithPhonetic}

\begin{EntryWithPhonetic}{交界}{jiao1jie4}{6,9}{⼇、⽥}
  \definition{s.}{fronteira comum | limite comum | interface}
\end{EntryWithPhonetic}

\begin{EntryWithPhonetic}{交警}{jiao1 jing3}{6,19}{⼇、⾔}[HSK 3]
  \definition{s.}{policial de trânsito, abreviação de 交通警察}
  \seealsoref{交通警察}{jiao1tong1 jing3cha2}
\end{EntryWithPhonetic}

\begin{EntryWithPhonetic}{交流}{jiao1liu2}{6,10}{⼇、⽔}[HSK 3]
  \definition{v.}{trocar; interagir; comunicar-se; compartilhar o que cada um tem com o outro}
\end{EntryWithPhonetic}

\begin{EntryWithPhonetic}{交朋友}{jiao1 peng2 you3}{6,8,4}{⼇、⽉、⼜}[HSK 2]
  \definition{v.}{fazer amizade com alguém; fazer amigos}
\end{EntryWithPhonetic}

\begin{EntryWithPhonetic}{交通}{jiao1tong1}{6,10}{⼇、⾡}[HSK 2]
  \definition{s.}{tráfego | ligação; conexão | transporte; termo genérico para todos os tipos de transporte, como ferroviário e rodoviário}
  \definition{v.}{conspirar; fazer amizades; conchavar | estar conectado; estar ligado; estar vinculado | associar-se a; conspirar com}
\end{EntryWithPhonetic}

\begin{EntryWithPhonetic}{交通警察}{jiao1tong1 jing3cha2}{6,10,19,14}{⼇、⾡、⾔、⼧}
  \definition{s.}{policial de trânsito}
  \seealsoref{交警}{jiao1 jing3}
\end{EntryWithPhonetic}

\begin{EntryWithPhonetic}{交往}{jiao1wang3}{6,8}{⼇、⼻}[HSK 3]
  \definition{v.}{estar em contato com; associar-se a; interagir}
\end{EntryWithPhonetic}

\begin{EntryWithPhonetic}{交响}{jiao1xiang3}{6,9}{⼇、⼝}
  \definition{s.}{sinfonia}
\end{EntryWithPhonetic}

\begin{EntryWithPhonetic}{交易}{jiao1yi4}{6,8}{⼇、⽇}[HSK 3]
  \definition[笔,桩,个,场]{s.}{negócio; comércio; transação comercial; transação; atividades de compra e venda de mercadorias}
  \definition{v.}{negociar; comprar e vender mercadorias}
\end{EntryWithPhonetic}

\begin{EntryWithPhonetic}{交运}{jiao1yun4}{6,7}{⼇、⾡}
  \definition{v.}{despachar (bagagem em um aeroporto, etc.) | entregar para transporte}
\end{EntryWithPhonetic}

\begin{EntryWithPhonetic}{郊}{jiao1}{8}{⾢}
  \definition*{s.}{Sobrenome Jiao}
  \definition{s.}{subúrbios; periferias; áreas ao redor da cidade}
\end{EntryWithPhonetic}

\begin{EntryWithPhonetic}{郊区}{jiao1 qu1}{8,4}{⾢、⼖}[HSK 5]
  \definition[个,片,块]{s.}{subúrbios; arredores; periferia; área ao redor da cidade que está administrativamente sob a jurisdição da cidade}
\end{EntryWithPhonetic}

\begin{EntryWithPhonetic}{骄}{jiao1}{9}{⾺}
  \definition{adj.}{orgulhoso; arrogante; vaidoso | Literário: feroz; intenso; forte; violento}
\end{EntryWithPhonetic}

\begin{EntryWithPhonetic}{骄傲}{jiao1'ao4}{9,12}{⾺、⼈}[HSK 6]
  \definition{adj.}{arrogante; vaidoso; orgulhoso}
  \definition{s.}{orgulho; pessoas ou coisas das quais se orgulhar}
\end{EntryWithPhonetic}

\begin{EntryWithPhonetic}{胶}{jiao1}{10}{⾁}
  \definition*{s.}{Sobrenome Jiao}
  \definition{adj.}{pegajoso; viscoso; grudento}
  \definition{s.}{cola; goma; adesivo | borracha | gel; colóide}
  \definition{v.}{colar com cola | colar; grudar}
\end{EntryWithPhonetic}

\begin{EntryWithPhonetic}{胶带}{jiao1 dai4}{10,9}{⾁、⼱}[HSK 5]
  \definition[卷,条,段]{s.}{fita de embalagem transparente; fita adesiva | fita magnética de plástico; fita de gravação | fita emborrachada; cinta de borracha}
\end{EntryWithPhonetic}

\begin{EntryWithPhonetic}{胶卷}{jiao1juan3}{10,8}{⾁、⼙}
  \definition{s.}{filme | rolo de filme}
\end{EntryWithPhonetic}

\begin{EntryWithPhonetic}{胶水}{jiao1shui3}{10,4}{⾁、⽔}[HSK 5]
  \definition[瓶]{s.}{cola; mucilagem; cola líquida}
\end{EntryWithPhonetic}

\begin{EntryWithPhonetic}{教}{jiao1}{11}{⽁}[HSK 1]
  \definition*{s.}{Sobrenome Jiao}
  \definition{prep.}{em uma frase passiva para introduzir o executor da ação}
  \definition{s.}{religião | professor; referência à educação ou aos professores}
  \definition{v.}{ensinar; instruir |  pedir; ordenar; dizer | permitir; possibilitar}
  \seeref{jiao4}
\end{EntryWithPhonetic}

\begin{EntryWithPhonetic}{教会}{jiao1hui4}{11,6}{⽁、⼈}
  \definition{v.}{mostrar | ensinar}
  \seeref{jiao4hui4}
\end{EntryWithPhonetic}

\begin{EntryWithPhonetic}{焦}{jiao1}{12}{⽕}
  \definition*{s.}{Sobrenome Jiao}
  \definition{adj.}{queimado; chamuscado; carbonizado | preocupado; ansioso}
  \definition{clas.}{J; Joule, abreviação}
  \definition{pref.}{(química) piro-}
  \definition{s.}{Metalurgia: coque}
\end{EntryWithPhonetic}

\begin{EntryWithPhonetic}{焦点}{jiao1dian3}{12,9}{⽕、⽕}[HSK 6]
  \definition{s.}{foco; ponto focal; Matemática: refere-se a um ponto que tem uma relação especial com uma elipse, hipérbole, parábola, etc. | foco; ponto focal; Óptica: refere-se à intersecção de feixes de luz paralelos após serem refratados por uma lente ou refletidos por um espelho curvo | foco; questão central; metaforicamente, uma coisa ou princípio que chama a atenção para o foco}
\end{EntryWithPhonetic}

\begin{EntryWithPhonetic}{焦虑}{jiao1lv4}{12,10}{⽕、⾌}
  \definition{adj.}{ansioso | preocupado | apreensivo}
\end{EntryWithPhonetic}

\begin{EntryWithPhonetic}{角}{jiao3}{7}{⾓}[HSK 2][Kangxi 148]
  \definition*{s.}{Jiao, uma das mansões lunares}
  \definition{clas.}{uma unidade monetária fracionária na China (=1/10 de um yuan ou 10 fen)}
  \definition[个,只,对]{s.}{chifre; o objeto duro que cresce na cabeça de bovinos, ovinos, veados, etc. | buzina; corneta; instrumentos musicais tocados no exército antigo | algo com a forma de um chifre | cabo; promontório; península | esquina; canto; a junção entre duas arestas de um objeto | ângulo}
  \seeref{jue2}
\end{EntryWithPhonetic}

\begin{EntryWithPhonetic}{角度}{jiao3du4}{7,9}{⾓、⼴}[HSK 2]
  \definition[个,种]{s.}{perspectiva; ponto de vista; o ponto de partida para ver as coisas | ângulo; o tamanho do ângulo; normalmente expresso em graus ou radianos}
\end{EntryWithPhonetic}

\begin{EntryWithPhonetic}{饺}{jiao3}{9}{⾷}
  \definition[盘,碗,顿,个]{s.}{bolinho de massa; \emph{dumpling}}
\end{EntryWithPhonetic}

\begin{EntryWithPhonetic}{饺子}{jiao3zi5}{9,3}{⾷、⼦}[HSK 2]
  \definition[个,盘,碗,锅]{s.}{jiaozi; bolinho chinês; bolinho de massa}
\end{EntryWithPhonetic}

\begin{EntryWithPhonetic}{脚}{jiao3}{11}{⾁}[HSK 2]
  \definition{clas.}{usado para chutes}
  \definition[只,双]{s.}{pé; a parte inferior das pernas de pessoas ou animais, que entra em contato com o solo | base; pé; a parte inferior do objeto | antigamente, referia-se ao trabalho físico de transporte de cargas | resíduos; sobras}
\end{EntryWithPhonetic}

\begin{EntryWithPhonetic}{脚步}{jiao3 bu4}{11,7}{⾁、⽌}[HSK 5]
  \definition{s.}{pé; passo; pisada; refere-se ao movimento das pernas ao caminhar | ritmo; passo; distância entre os pés dianteiros e traseiros ao caminhar}
\end{EntryWithPhonetic}

\begin{EntryWithPhonetic}{脚印}{jiao3 yin4}{11,5}{⾁、⼙}[HSK 6]
  \definition{s.}{trilha; pegada; marca de pé; os rastros deixados pelos passos}
\end{EntryWithPhonetic}

\begin{EntryWithPhonetic}{叫}{jiao4}{5}{⼝}[HSK 1,3]
  \definition{adj.}{macho (animal)}
  \definition{prep.}{usado em frases passivas; introduz o agente da ação; equivalente a 被 | combinado com 看, 说; usado para expressar suas ideias e pontos de vista}
  \definition{v.}{chorar; gritar; berrar | nomear; chamar | chamar; chamar a atenção | cumprimentar; saudar; dizer olá | pedir; ordenar; licitar | permitir; concordar com algo; concordar em fazer algo | contratar; encomendar; comprar o que você precisa}
  \seealsoref{被}{bei4}
  \seealsoref{看}{kan4}
  \seealsoref{说}{shuo1}
\end{EntryWithPhonetic}

\begin{EntryWithPhonetic}{叫作}{jiao4 zuo4}{5,7}{⼝、⼈}[HSK 2]
  \definition{v.}{ser chamado de; ser conhecido como}
\end{EntryWithPhonetic}

\begin{EntryWithPhonetic}{觉}{jiao4}{9}{⾒}[HSK 6]
  \definition[个]{s.}{sono; o processo desde adormecer até acordar}
  \seeref{jue2}
\end{EntryWithPhonetic}

\begin{EntryWithPhonetic}{校}{jiao4}{10}{⽊}
  \definition{v.}{verificar | comparar | revisar}
  \seeref{xiao4}
\end{EntryWithPhonetic}

\begin{EntryWithPhonetic}{较}{jiao4}{10}{⾞}[HSK 3]
  \definition{adj.}{claro; óbvio; evidente}
  \definition{adv.}{comparativamente; relativamente; razoavelmente; bastante; bastante}
  \definition{prep.}{usado para comparar características e graus; introduzir o objeto de comparação; equivalente a 比}
  \definition{v.}{comparar | disputar}
  \seealsoref{比}{bi3}
\end{EntryWithPhonetic}

\begin{EntryWithPhonetic}{敎}{jiao4}{11}{⽁}
  \variantof{教}
\end{EntryWithPhonetic}

\begin{EntryWithPhonetic}{教}{jiao4}{11}{⽁}
  \definition*{s.}{Sobrenome Jiao}
  \definition{prep.}{em uma frase passiva para apresentar o autor da ação}
  \definition{s.}{religião | educação; professor}
  \definition{v.}{ensinar; instruir | perguntar; ordenar; contar | permitir; permitir}
  \seeref{jiao1}
\end{EntryWithPhonetic}

\begin{EntryWithPhonetic}{教材}{jiao4cai2}{11,7}{⽁、⽊}[HSK 3]
  \definition[本,套]{s.}{livro didático; materiais didáticos, incluindo livros didáticos, apostilas, materiais de referência, vídeos, imagens, etc.}
\end{EntryWithPhonetic}

\begin{EntryWithPhonetic}{教导}{jiao4dao3}{11,6}{⽁、⼨}
  \definition{s.}{instrução | orientação | ensino}
  \definition{v.}{instruir | orientar | ensinar}
\end{EntryWithPhonetic}

\begin{EntryWithPhonetic}{教官}{jiao4guan1}{11,8}{⽁、⼧}
  \definition{s.}{instrutor militar}
\end{EntryWithPhonetic}

\begin{EntryWithPhonetic}{教会}{jiao4hui4}{11,6}{⽁、⼈}
  \definition{s.}{igreja cristã}
  \seeref{jiao1hui4}
\end{EntryWithPhonetic}

\begin{EntryWithPhonetic}{教练}{jiao4lian4}{11,8}{⽁、⽷}[HSK 3]
  \definition[个,位,名]{s.}{instrutor; treinador (esportes); pessoas que trabalham como treinadores}
  \definition{v.}{treinar; treinar outras pessoas para dominarem uma determinada técnica (como esportes, dirigir carros, pilotar aviões, etc.)}
\end{EntryWithPhonetic}

\begin{EntryWithPhonetic}{教师}{jiao4 shi1}{11,6}{⽁、⼱}[HSK 2]
  \definition[个,位,名]{s.}{professor; professor de escola}
\end{EntryWithPhonetic}

\begin{EntryWithPhonetic}{教室}{jiao4shi4}{11,9}{⽁、⼧}[HSK 2]
  \definition[间]{s.}{sala de aula}
\end{EntryWithPhonetic}

\begin{EntryWithPhonetic}{教授}{jiao4shou4}{11,11}{⽁、⼿}[HSK 4]
  \definition[个,位,名]{s.}{professor (universitário); o professor com a classificação mais alta em uma universidade}
  \definition{v.}{ensinar; instruir; dar aulas; dar palestras}
\end{EntryWithPhonetic}

\begin{EntryWithPhonetic}{教堂}{jiao4tang2}{11,11}{⽁、⼟}[HSK 6]
  \definition[座,所,间]{s.}{igreja; capela; catedral; casa de deus; um lugar onde os cristãos realizam cerimônias religiosas}
\end{EntryWithPhonetic}

\begin{EntryWithPhonetic}{教学}{jiao4 xue2}{11,8}{⽁、⼦}[HSK 2]
  \definition[个,门]{s.}{ensino; educação; o processo de transmissão de conhecimentos e habilidades}
\end{EntryWithPhonetic}

\begin{EntryWithPhonetic}{教学楼}{jiao4 xue2 lou2}{11,8,13}{⽁、⼦、⽊}[HSK 1]
  \definition{s.}{prédio da escola; bloco de ensino; edifícios utilizados para atividades educacionais, geralmente incluindo salas de aula, laboratórios, auditórios, etc.}
\end{EntryWithPhonetic}

\begin{EntryWithPhonetic}{教训}{jiao4xun4}{11,5}{⽁、⾔}[HSK 4]
  \definition[个,次,番,顿]{s.}{moral; lição}
  \definition{v.}{repreender; educar; ensinar uma lição a alguém; dar uma bronca em alguém; dar um sermão em alguém (por ter cometido um erro, etc.)}
\end{EntryWithPhonetic}

\begin{EntryWithPhonetic}{教育}{jiao4yu4}{11,8}{⽁、⾁}[HSK 2]
  \definition{s.}{educação; refere-se a atividades sociais cujo objetivo direto é influenciar o desenvolvimento físico e mental das pessoas; refere-se principalmente ao processo de formação dos alunos nas escolas}
  \definition{v.}{ensinar; educar; inspirar, fazer compreender a razão}
\end{EntryWithPhonetic}

\begin{EntryWithPhonetic}{教育部}{jiao4 yu4 bu4}{11,8,10}{⽁、⾁、⾢}[HSK 6]
  \definition*{s.}{Ministério da Educação}
\end{EntryWithPhonetic}

\begin{EntryWithPhonetic}{教长}{jiao4zhang3}{11,4}{⽁、⾧}
  \definition{s.}{imã (Islã) | mulá}
\end{EntryWithPhonetic}

\begin{EntryWithPhonetic}{节}{jie1}{5}{⾋}
  \definition{adj.}{momento crucial; momento crítico; momento decisivo; metáfora para algo importante, decisivo ou oportuno}
  \seeref{jie2}
\end{EntryWithPhonetic}

\begin{EntryWithPhonetic}{阶}{jie1}{6}{⾩}
  \definition{s.}{degrau; escada; escadaria | classificação | escala | ordem | estágio}
\end{EntryWithPhonetic}

\begin{EntryWithPhonetic}{阶段}{jie1duan4}{6,9}{⾩、⽎}[HSK 4]
  \definition[个,段]{s.}{estágio; fase; período; bancada; gradação}
\end{EntryWithPhonetic}

\begin{EntryWithPhonetic}{皆}{jie1}{9}{⽩}
  \definition{adv.}{todos | em todos os casos}
\end{EntryWithPhonetic}

\begin{EntryWithPhonetic}{结}{jie1}{9}{⽷}
  \definition{v.}{dar (frutos); formar (sementes); produzir frutos ou sementes (uma planta)}
  \seeref{jie2}
\end{EntryWithPhonetic}

\begin{EntryWithPhonetic}{结果}{jie1guo3}{9,8}{⽷、⽊}
  \definition{v.}{dar frutos}
  \seeref{jie2guo3}
\end{EntryWithPhonetic}

\begin{EntryWithPhonetic}{结实}{jie1shi5}{9,8}{⽷、⼧}[HSK 3]
  \definition{adj.}{sólido; resistente; durável | forte; resistente; robusto}
\end{EntryWithPhonetic}

\begin{EntryWithPhonetic}{接}{jie1}{11}{⼿}[HSK 2]
  \definition*{s.}{Sobrenome Jie}
  \definition{v.}{entrar em contato com; aproximar-se de | conectar; unir; juntar | continuar; prosseguir | assumir o controle; assumir o trabalho de outra pessoa e continuar a fazê-lo | pegar; agarrar; segurar ou sustentar com as mãos | receber; aceitar | encontrar; dar as boas-vindas}
\end{EntryWithPhonetic}

\begin{EntryWithPhonetic}{接班人}{jie1ban1ren2}{11,10,2}{⼿、⽟、⼈}
  \definition{s.}{sucessor}
\end{EntryWithPhonetic}

\begin{EntryWithPhonetic}{接触}{jie1chu4}{11,13}{⼿、⾓}[HSK 5]
  \definition{v.}{entrar em contato com | entrar em contato; tocar; interagir | engajar; o termo militar refere-se a fogo cruzado}
\end{EntryWithPhonetic}

\begin{EntryWithPhonetic}{接待}{jie1dai4}{11,9}{⼿、⼻}[HSK 3]
  \definition{v.}{receber (alguém); acolher; recepcionar; receber com cordialidade e generosidade}
\end{EntryWithPhonetic}

\begin{EntryWithPhonetic}{接到}{jie1 dao4}{11,8}{⼿、⼑}[HSK 2]
  \definition{v.}{receber (carta, etc.)}
\end{EntryWithPhonetic}

\begin{EntryWithPhonetic}{接(电话)}{jie1(dian4hua4)}{11,5,8}{⼿、⽥、⾔}
  \definition{v.}{atender (o telefone) | receber (uma ligação telefônica)}
\end{EntryWithPhonetic}

\begin{EntryWithPhonetic}{接近}{jie1jin4}{11,7}{⼿、⾡}[HSK 3]
  \definition{adj.}{perto; próximo; a diferença entre os dois é mínima}
  \definition{v.}{estar perto de; aproximar; aproximar-se}
\end{EntryWithPhonetic}

\begin{EntryWithPhonetic}{接连}{jie1lian2}{11,7}{⼿、⾡}[HSK 5]
  \definition{adv.}{no final; em sucessão; em uma fileira; um após o outro; seguindo o anterior; continuando}
\end{EntryWithPhonetic}

\begin{EntryWithPhonetic}{接收}{jie1 shou1}{11,6}{⼿、⽁}[HSK 6]
  \definition{v.}{aceitar; receber | assumir; expropriar; tomar posse (de uma instituição, propriedade, etc.) de acordo com a lei | admitir; aceitar; absorver}
\end{EntryWithPhonetic}

\begin{EntryWithPhonetic}{接受}{jie1shou4}{11,8}{⼿、⼜}[HSK 2]
  \definition{v.}{aceitar; não recusar (o que os outros oferecem) | concordar; não recusar (opiniões/sugestões/críticas/convites de outras pessoas, etc.)}
\end{EntryWithPhonetic}

\begin{EntryWithPhonetic}{接下来}{jie1 xia4 lai2}{11,3,7}{⼿、⼀、⽊}[HSK 2]
  \definition{expr.}{próximo; seguinte; indica uma sequência temporal subsequente}
\end{EntryWithPhonetic}

\begin{EntryWithPhonetic}{接着}{jie1zhe5}{11,11}{⼿、⽬}[HSK 2]
  \definition{adv.}{por sua vez; um após o outro; sucessivamente; conectado (à frase anterior); imediatamente após (a ação anterior)}
  \definition{v.}{seguir; prosseguir; continuar; seguir em frente; ficar ao lado | pegar com as mãos; apanhar}
\end{EntryWithPhonetic}

\begin{EntryWithPhonetic}{揭}{jie1}{12}{⼿}[HSK 6]
  \definition*{s.}{Sobrenome Jie}
  \definition{v.}{rasgar; arrancar; tirar | descobrir; levantar (a tampa, etc.) | expor; mostrar; trazer à luz | (literário) levantar; içar}
\end{EntryWithPhonetic}

\begin{EntryWithPhonetic}{街}{jie1}{12}{⾏}[HSK 2]
  \definition[条]{s.}{rua; avenida com prédios dos dois lados | mercado; feira rural}
\end{EntryWithPhonetic}

\begin{EntryWithPhonetic}{街道}{jie1dao4}{12,12}{⾏、⾡}[HSK 4]
  \definition[条]{s.}{caminho; rua; estrada; via pública com casas em ambos os lados, relativamente larga | escritório do subdistrito; tipo de organização responsável por gerenciar determinados aspectos da rua}
\end{EntryWithPhonetic}

\begin{EntryWithPhonetic}{街头}{jie1 tou2}{12,5}{⾏、⼤}[HSK 6]
  \definition{s.}{rua; esquina da rua}
\end{EntryWithPhonetic}

\begin{EntryWithPhonetic}{街舞}{jie1wu3}{12,14}{⾏、⾇}
  \definition{s.}{dança de rua, \emph{street dance} (por exemplo, \emph{breakdance})}
\end{EntryWithPhonetic}

\begin{EntryWithPhonetic}{节}{jie2}{5}{⾋}[HSK 2]
  \definition*{s.}{Sobrenome Jie}
  \definition{clas.}{nó (kn), velocidade de um barco | para seções, comprimentos}
  \definition[个]{s.}{junta; botão; nó; geralmente se refere à parte da grama ou caule da grama onde as folhas crescem ou à parte onde os galhos e troncos das plantas são conectados | parte; divisão; um trecho de algo interligado; uma parte do todo | festival; feriado; dia memorável; um período de tempo ou um dia com características específicas | item; assunto | castidade; integridade ética e moral | articulação; o local onde os ossos humanos ou animais se conectam | etiqueta; cerimonial | batida; ritmo | registro; documento utilizado na antiguidade para comprovar a identidade | estação do ano | sílaba}
  \definition{v.}{economizar; conservar; poupar | resumir; extrair; retirar uma parte do todo | controlar; restringir; moderar}
  \seeref{jie1}
\end{EntryWithPhonetic}

\begin{EntryWithPhonetic}{节假日}{jie2 jia4 ri4}{5,11,4}{⾋、⼈、⽇}[HSK 6]
  \definition[个]{s.}{feriados; festivais e feriados}
\end{EntryWithPhonetic}

\begin{EntryWithPhonetic}{节目}{jie2mu4}{5,5}{⾋、⽬}[HSK 2]
  \definition[个,场,项,台]{s.}{programa; item (em um programa); programas artísticos ou projetos transmitidos por rádios e televisões}
\end{EntryWithPhonetic}

\begin{EntryWithPhonetic}{节能}{jie2 neng2}{5,10}{⾋、⾁}[HSK 6]
  \definition{v.}{economizar no consumo de energia; conservar energia}
\end{EntryWithPhonetic}

\begin{EntryWithPhonetic}{节日}{jie2ri4}{5,4}{⾋、⽇}[HSK 2]
  \definition[个,种,类]{s.}{festival; feriado; dia de comemoração tradicional; dia comemorativo estabelecido por lei}
\end{EntryWithPhonetic}

\begin{EntryWithPhonetic}{节省}{jie2sheng3}{5,9}{⾋、⽬}[HSK 4]
  \definition{adj.}{econômico; parcimonioso}
  \definition{v.}{economizar; conservar; usar com moderação; reduzir; eliminar ou minimizar o esgotamento de itens potencialmente esgotáveis}
\end{EntryWithPhonetic}

\begin{EntryWithPhonetic}{节约}{jie2yue1}{5,6}{⾋、⽷}[HSK 3]
  \definition{adj.}{econômico; sem luxo}
  \definition{v.}{guardar; economizar; usar com moderação; economizar gastos desnecessários}
\end{EntryWithPhonetic}

\begin{EntryWithPhonetic}{节奏}{jie2zou4}{5,9}{⾋、⼤}[HSK 6]
  \definition[个,种]{s.}{ritmo; o fenômeno da alternância regular de comprimento, força e fraqueza das notas na música | padrão regular; uma metáfora para um processo de ajuste adequado com tensão e relaxamento}
\end{EntryWithPhonetic}

\begin{EntryWithPhonetic}{杰}{jie2}{8}{⽊}
  \definition{adj.}{notável; proeminente; fora do comum}
  \definition[位,名,个,些]{s.}{pessoa excepcional; herói; uma pessoa com talentos excepcionais}
\end{EntryWithPhonetic}

\begin{EntryWithPhonetic}{杰出}{jie2chu1}{8,5}{⽊、⼐}[HSK 6]
  \definition{adj.}{notável; proeminente; (talento, realização) excepcional}
\end{EntryWithPhonetic}

\begin{EntryWithPhonetic}{拮}{jie2}{9}{⼿}
  \definition{adj.}{trabalhoso | sem dinheiro | antagônico | trabalhando duro | pressionado}
\end{EntryWithPhonetic}

\begin{EntryWithPhonetic}{拮据}{jie2ju1}{9,11}{⼿、⼿}
  \definition{adj.}{em circunstâncias difíceis; sem dinheiro; em dificuldades}
\end{EntryWithPhonetic}

\begin{EntryWithPhonetic}{结}{jie2}{9}{⽷}[HSK 4]
  \definition*{s.}{Sobrenome Jie}
  \definition{s.}{nó | declaração juramentada; garantia por escrito; documento que, antigamente, significava um reconhecimento de encerramento ou uma garantia de responsabilidade}
  \definition{v.}{amarrar; tricotar; dar nó; tecer | formar; forjar; cimentar; solidificar | resolver; concluir | combinar; formar um relacionamento}
  \seeref{jie1}
\end{EntryWithPhonetic}

\begin{EntryWithPhonetic}{结构}{jie2gou4}{9,8}{⽷、⽊}[HSK 4]
  \definition[个]{s.}{estrutura; composição; construção; formação; constituição; tecido; forma; sistematização; mecânica; organização | arquitetura; estrutura; construção; construção de partes de edifícios com suporte de carga ou com carga externa | Geologia: textura}[这些矿物质具有致密结构。===Esses minerais têm uma estrutura densa.]
\end{EntryWithPhonetic}

\begin{EntryWithPhonetic}{结果}{jie2guo3}{9,8}{⽷、⽊}[HSK 2]
  \definition{conj.}{como resultado | no final}
  \definition{s.}{resultado | conclusão | consequência}
  \definition{v.}{despachar | matar}
  \seeref{jie1guo3}
\end{EntryWithPhonetic}

\begin{EntryWithPhonetic}{结合}{jie2he2}{9,6}{⽷、⼝}[HSK 3]
  \definition{v.}{ligar; unir; combinar; integrar; formar uma relação estreita entre pessoas ou coisas | casar-se; unir-se em matrimônio; referir-se especificamente a casais}
\end{EntryWithPhonetic}

\begin{EntryWithPhonetic}{结婚}{jie2/hun1}{9,11}{⽷、⼥}[HSK 3]
  \definition{v.+compl.}{casar; casar-se; casar-se bem;}
\end{EntryWithPhonetic}

\begin{EntryWithPhonetic}{结婚礼服}{jie2hun1 li3 fu2}{9,11,5,8}{⽷、⼥、⽰、⽉}
  \definition{s.}{vestido de casamento}
\end{EntryWithPhonetic}

\begin{EntryWithPhonetic}{结局}{jie2ju2}{9,7}{⽷、⼫}
  \definition{s.}{conclusão | fim | final}
\end{EntryWithPhonetic}

\begin{EntryWithPhonetic}{结论}{jie2lun4}{9,6}{⽷、⾔}[HSK 4]
  \definition[个]{s.}{conclusão; palavra final sobre uma pessoa ou coisa após investigação e pesquisa | veredito; julgamento deduzido de premissas também é chamado de conclusão}
\end{EntryWithPhonetic}

\begin{EntryWithPhonetic}{结社自由}{jie2she4zi4you2}{9,7,6,5}{⽷、⽰、⾃、⽥}
  \definition{s.}{(constitucional) liberdade de associação}
\end{EntryWithPhonetic}

\begin{EntryWithPhonetic}{结束}{jie2shu4}{9,7}{⽷、⽊}[HSK 3]
  \definition{v.}{finalizar; fechar; terminar; concluir; encerrar; desenvolver ou avançar até a fase final, sem continuidade}
\end{EntryWithPhonetic}

\begin{EntryWithPhonetic}{结束辩论}{jie2shu4 bian4 lun4}{9,7,16,6}{⽷、⽊、⾟、⾔}
  \definition{s.}{debate de encerramento}
\end{EntryWithPhonetic}

\begin{EntryWithPhonetic}{结束工作}{jie2shu4gong1zuo4}{9,7,3,7}{⽷、⽊、⼯、⼈}
  \definition{s.}{trabalho final}
  \definition{v.}{terminar o trabalho}
\end{EntryWithPhonetic}

\begin{EntryWithPhonetic}{结束剂}{jie2shu4 ji4}{9,7,8}{⽷、⽊、⼑}
  \definition{s.}{finalizador}
\end{EntryWithPhonetic}

\begin{EntryWithPhonetic}{结束区}{jie2shu4 qu1}{9,7,4}{⽷、⽊、⼖}
  \definition{s.}{zona final}
\end{EntryWithPhonetic}

\begin{EntryWithPhonetic}{结束文本}{jie2shu4 wen2ben3}{9,7,4,5}{⽷、⽊、⽂、⽊}
  \definition{s.}{texto final}
\end{EntryWithPhonetic}

\begin{EntryWithPhonetic}{结束语}{jie2shu4yu3}{9,7,9}{⽷、⽊、⾔}
  \definition{s.}{conclusões finais | considerações finais}
\end{EntryWithPhonetic}

\begin{EntryWithPhonetic}{捷}{jie2}{11}{⼿}
  \definition*{s.}{Sobrenome Jie}
  \definition{adj.}{rápido; ágil}
  \definition{s.}{vitória; triunfo; sucesso}
\end{EntryWithPhonetic}

\begin{EntryWithPhonetic}{捷径}{jie2jing4}{11,8}{⼿、⼻}
  \definition{s.}{atalho}
\end{EntryWithPhonetic}

\begin{EntryWithPhonetic}{截}{jie2}{14}{⼽}
  \definition{clas.}{seção; pedaço; comprimento}
  \definition{prep.}{por (um tempo especificado); até}
  \definition{v.}{cortar; romper | parar; verificar; interromper; interceptar}
\end{EntryWithPhonetic}

\begin{EntryWithPhonetic}{截止}{jie2zhi3}{14,4}{⼽、⽌}[HSK 6]
  \definition{adv.}{até (um certo limite de tempo); por (um tempo especificado)}
\end{EntryWithPhonetic}

\begin{EntryWithPhonetic}{截至}{jie2zhi4}{14,6}{⼽、⾄}[HSK 6]
  \definition{adv.}{a partir de; até (um certo limite de tempo); por (um tempo especificado)}
\end{EntryWithPhonetic}

\begin{EntryWithPhonetic}{姐}{jie3}{8}{⼥}[HSK 1]
  \definition[个,位]{s.}{irmã mais velha; irmã | termo genérico para mulheres jovens | mulheres da mesma geração que são mais velhas do que você (geralmente não inclui aquelas que podem ser chamadas de cunhadas) | um título respeitoso para mulheres jovens profissionais em determinados cargos}
  \seealsoref{姐姐}{jie3 jie5}
\end{EntryWithPhonetic}

\begin{EntryWithPhonetic}{姐夫}{jie3fu5}{8,4}{⼥、⼤}
  \definition{s.}{marido da irmã mais velha}
\end{EntryWithPhonetic}

\begin{EntryWithPhonetic}{姐姐}{jie3 jie5}{8,8}{⼥、⼥}[HSK 1]
  \definition[个]{s.}{irmã mais velha}
\end{EntryWithPhonetic}

\begin{EntryWithPhonetic}{姐妹}{jie3 mei4}{8,8}{⼥、⼥}[HSK 4]
  \definition[个]{s.}{irmãs}
\end{EntryWithPhonetic}

\begin{EntryWithPhonetic}{解}{jie3}{13}{⾓}[HSK 6]
  \definition{s.}{solução; o valor de uma variável desconhecida em uma equação algébrica}
  \definition{v.}{dividir; separar | desfazer; desatar; abrir algo que esteja amarrado ou encadernado | acalmar; dissipar; dispensar; eliminar | resolver; explicar; interpretar | entender; compreender | aliviar-se (excreção de urina e fezes) | dissolver; desintegrar | (cálculo analítico) resolver; solucionar}
\end{EntryWithPhonetic}

\begin{EntryWithPhonetic}{解除}{jie3chu2}{13,9}{⾓、⾩}[HSK 5]
  \definition{v.}{remover; aliviar; livrar-se de; eliminar}
\end{EntryWithPhonetic}

\begin{EntryWithPhonetic}{解放}{jie3fang4}{13,8}{⾓、⽅}[HSK 5]
  \definition*{s.}{Libertação (que significou o fim do domínio do regime reacionário Kuomintang em 1949 e ao estabelecimento da República Popular da China)}
  \definition{v.}{libertar; emancipar; eliminar as restrições para permitir o desenvolvimento da liberdade}
\end{EntryWithPhonetic}

\begin{EntryWithPhonetic}{解雇}{jie3gu4}{13,12}{⾓、⾫}
  \definition{v.}{demitir}
\end{EntryWithPhonetic}

\begin{EntryWithPhonetic}{解救}{jie3jiu4}{13,11}{⾓、⽁}
  \definition{v.}{resgatar | ajudar a sair de dificuldades | salvar a situação}
\end{EntryWithPhonetic}

\begin{EntryWithPhonetic}{解决}{jie3jue2}{13,6}{⾓、⼎}[HSK 3]
  \definition{v.}{solucionar; resolver; liquidar; resolver problemas com resultados | acabar com; descartar; eliminar (o inimigo)}
\end{EntryWithPhonetic}

\begin{EntryWithPhonetic}{解开}{jie3 kai1}{13,4}{⾓、⼶}[HSK 3]
  \definition{v.}{desatar; desamarrar; desabotoar; desamarrar ou desfazer nós}
\end{EntryWithPhonetic}

\begin{EntryWithPhonetic}{解释}{jie3shi4}{13,12}{⾓、⾤}[HSK 4]
  \definition{v.}{explicar; expor; interpretar | analisar; explicaro significado, razões, justificativas, etc.}
\end{EntryWithPhonetic}

\begin{EntryWithPhonetic}{解说}{jie3 shuo1}{13,9}{⾓、⾔}[HSK 6]
  \definition{v.}{narrar; comentar; fazer um comentário; explicar oralmente}
\end{EntryWithPhonetic}

\begin{EntryWithPhonetic}{解压}{jie3ya1}{13,6}{⾓、⼚}
  \definition{v.}{aliviar o estresse | (computação) descomprimir}
\end{EntryWithPhonetic}

\begin{EntryWithPhonetic}{介}{jie4}{4}{⼈}
  \definition*{s.}{Sobrenome Jie}
  \definition{adj.}{direto; honesto e franco; correto}
  \definition{s.}{armadura | concha (crustáceos e criaturas aquáticas) | preposição}
  \definition{v.}{estar situado entre; interpor | levar a sério; levar em conta; ter em mente}
\end{EntryWithPhonetic}

\begin{EntryWithPhonetic}{介绍}{jie4shao4}{4,8}{⼈、⽷}[HSK 1]
  \definition{s.}{introdução; apresentação}
  \definition{v.}{introduzir; apresentar | recomendar; sugerir | dar a conhecer; informar}
\end{EntryWithPhonetic}

\begin{EntryWithPhonetic}{戒}{jie4}{7}{⼽}[HSK 5]
  \definition[个,枚]{s.}{advertência; exortação | disciplina monástica budista; preceitos budistas | anel (dedo)}
  \definition{v.}{proteger-se contra; estar preparado; estar atento | advertir; exortar; admoestar | abandonar; parar; desistir; desistir (de um hábito ruim)}
\end{EntryWithPhonetic}

\begin{EntryWithPhonetic}{芥}{jie4}{7}{⾋}
  \definition{s.}{mostarda}
  \seeref{gai4}
\end{EntryWithPhonetic}

\begin{EntryWithPhonetic}{芥兰}{jie4lan2}{7,5}{⾋、⼋}
  \definition{s.}{couve}
\end{EntryWithPhonetic}

\begin{EntryWithPhonetic}{届}{jie4}{8}{⼫}[HSK 5]
  \definition{clas.}{sessões (de uma conferência); anos (de graduação); quantificador, ligeiramente equivalente a 次, usado para reuniões regulares ou turmas de formandos, etc.}
  \definition{v.}{vencer o prazo}
  \seealsoref{次}{ci4}
\end{EntryWithPhonetic}

\begin{EntryWithPhonetic}{界}{jie4}{9}{⽥}[HSK 6]
  \definition{s.}{fronteira; limite | escopo; extensão | círculos | divisão primária; reino | era geológica | (matemática) limite | mundo; faixa dividida por ocupação, emprego ou gênero, etc. | grupo}
\end{EntryWithPhonetic}

\begin{EntryWithPhonetic}{界碑}{jie4bei1}{9,13}{⽥、⽯}
  \definition{s.}{marco de fronteira}
\end{EntryWithPhonetic}

\begin{EntryWithPhonetic}{借}{jie4}{10}{⼈}[HSK 2]
  \definition{adv.}{por meio de}
  \definition{v.}{emprestar | pedir emprestado | usar como pretexto | aproveitar; tirar proveito (de uma oportunidade, etc.)}
\end{EntryWithPhonetic}

\begin{EntryWithPhonetic}{借鉴}{jie4jian4}{10,13}{⼈、⾦}[HSK 6]
  \definition{s.}{tirar lições de; aproveitar a experiência de; ganhar experiência e lições com o passado ou com as experiências de outras pessoas}
\end{EntryWithPhonetic}

\begin{EntryWithPhonetic}{借书证}{jie4shu1zheng4}{10,4,7}{⼈、⼄、⾔}
  \definition{s.}{cartão de biblioteca | (literalmente) cartão para pedir emprestado livros}
\end{EntryWithPhonetic}

\begin{EntryWithPhonetic}{今}{jin1}{4}{⼈}
  \definition*{s.}{Sobrenome Jin}
  \definition{s.}{agora; o presente | moderno (em oposição a 古) | de hoje; deste ano | isso; isto}
  \seealsoref{古}{gu3}
\end{EntryWithPhonetic}

\begin{EntryWithPhonetic}{今后}{jin1 hou4}{4,6}{⼈、⼝}[HSK 2]
  \definition{s.}{a partir de agora; doravante; no futuro; desde o momento em que falamos}
\end{EntryWithPhonetic}

\begin{EntryWithPhonetic}{今年}{jin1 nian2}{4,6}{⼈、⼲}[HSK 1]
  \definition{adv.}{este ano}
\end{EntryWithPhonetic}

\begin{EntryWithPhonetic}{今日}{jin1 ri4}{4,4}{⼈、⽇}[HSK 5]
  \definition{s.}{hoje}
\end{EntryWithPhonetic}

\begin{EntryWithPhonetic}{今天}{jin1tian1}{4,4}{⼈、⼤}[HSK 1]
  \definition{adv.}{hoje; neste dia | agora; o momento ou a época atual}
\end{EntryWithPhonetic}

\begin{EntryWithPhonetic}{斤}{jin1}{4}{⽄}[HSK 2][Kangxi 69]
  \definition{clas.}{uma unidade de peso (=500 gramas)}
  \definition{s.}{machado; cutelo; ferramentas antigas para cortar árvores}
\end{EntryWithPhonetic}

\begin{EntryWithPhonetic}{金}{jin1}{8}{⾦}[HSK 3][Kangxi 167]
  \definition*{s.}{Dinastia Jin (1115-1234) | Sobrenome Jin}
  \definition{adj.}{dourado | altamente respeitado; precioso. metáfora de nobreza}
  \definition[锭,块]{s.}{ouro | metal | dinheiro | instrumento antigo de percussão de metal}
\end{EntryWithPhonetic}

\begin{EntryWithPhonetic}{金额}{jin1 e2}{8,15}{⾦、⾴}[HSK 6]
  \definition[份,笔]{s.}{quantidade de dinheiro; soma de dinheiro}
\end{EntryWithPhonetic}

\begin{EntryWithPhonetic}{金刚石}{jin1gang1shi2}{8,6,5}{⾦、⼑、⽯}
  \definition{s.}{diamante, também chamado de 钻石}[金刚石比什么金属都硬。===O diamante é mais duro que qualquer metal.]
  \seealsoref{钻石}{zuan4shi2}
\end{EntryWithPhonetic}

\begin{EntryWithPhonetic}{金牌}{jin1 pai2}{8,12}{⾦、⽚}[HSK 3]
  \definition[枚]{s.}{medalha de ouro; refere-se à medalha conquistada pelo campeão em uma competição esportiva | ficha de ouro; placa de ouro usada como símbolo}
\end{EntryWithPhonetic}

\begin{EntryWithPhonetic}{金钱}{jin1 qian2}{8,10}{⾦、⾦}[HSK 6]
  \definition[沓,笔,堆]{s.}{dinheiro; moeda}
\end{EntryWithPhonetic}

\begin{EntryWithPhonetic}{金融}{jin1rong2}{8,16}{⾦、⿀}[HSK 6]
  \definition{s.}{finanças; serviços bancários; refere-se a atividades econômicas como a emissão, circulação e retirada de moeda, a concessão e retirada de empréstimos, o depósito e retirada de depósitos e transações de câmbio}
\end{EntryWithPhonetic}

\begin{EntryWithPhonetic}{金色}{jin1 se4}{8,6}{⾦、⾊}
  \definition{s.}{cor ouro; dourado}
\end{EntryWithPhonetic}

\begin{EntryWithPhonetic}{金子}{jin1zi5}{8,3}{⾦、⼦}
  \definition{s.}{ouro; elemento metálico, símbolo Au (aurum) amarelo-avermelhado, macio, dúctil, quimicamente estável é um metal precioso, usado para fabricar dinheiro, ornamentos etc.}
\end{EntryWithPhonetic}

\begin{EntryWithPhonetic}{矜}{jin1}{9}{⽭}
  \definition{adj.}{presunçoso; vaidoso | contido; reservado; determinado}
  \definition{v.}{ter pena; simpatizar com; compadecer-se}
\end{EntryWithPhonetic}

\begin{EntryWithPhonetic}{仅}{jin3}{4}{⼈}[HSK 3]
  \definition{adv.}{somente; meramente; por muito pouco}
\end{EntryWithPhonetic}

\begin{EntryWithPhonetic}{仅此而已}{jin3ci3'er2yi3}{4,6,6,3}{⼈、⽌、⽽、⼰}
  \definition{adv.}{apenas isso e nada mais | isso é tudo}
\end{EntryWithPhonetic}

\begin{EntryWithPhonetic}{仅仅}{jin3 jin3}{4,4}{⼈、⼈}[HSK 3]
  \definition{adv.}{somente; meramente; por muito pouco; indica que está limitado a um determinado âmbito}
\end{EntryWithPhonetic}

\begin{EntryWithPhonetic}{尽}{jin3}{6}{⼫}
  \definition{adv.}{na maior extensão possível | na extremidade mais distante de | usado antes de palavras que indicam direção, o mesmo que 最 | de agora em diante}
  \definition{prep.}{dentro dos limites de}
  \definition{v.}{dar prioridade a | deixar que certas pessoas ou coisas tenham precedência}
  \seeref{jin4}
  \seealsoref{最}{zui4}
\end{EntryWithPhonetic}

\begin{EntryWithPhonetic}{尽管}{jin3guan3}{6,14}{⼫、⽵}[HSK 5]
  \definition{adv.}{justo; livremente; faça o que quiser, não se preocupe, não há restrições de movimento ou comportamento}
  \definition{conj.}{no entanto; embora; apesar de ; normalmente usado no início de uma frase anterior para introduzir um fato, seguido de 但是, etc. para introduzir um resultado que o fato não deveria ter; às vezes, também pode ser usado no início de uma frase posterior.}
  \seealsoref{但是}{dan4 shi4}
\end{EntryWithPhonetic}

\begin{EntryWithPhonetic}{尽可能}{jin3 ke3 neng2}{6,5,10}{⼫、⼝、⾁}[HSK 5]
  \definition{adv.}{na medida do possível; com o melhor de sua capacidade; tentar fazer algo, atingir um determinado nível ou extensão}
\end{EntryWithPhonetic}

\begin{EntryWithPhonetic}{尽快}{jin3kuai4}{6,7}{⼫、⼼}[HSK 4]
  \definition{adv.}{com toda a velocidade; o mais rápido possível; o mais breve possível}
\end{EntryWithPhonetic}

\begin{EntryWithPhonetic}{尽量}{jin3liang4}{6,12}{⼫、⾥}[HSK 3]
  \definition{adv.}{tanto quanto possível; da melhor maneira possível}
\end{EntryWithPhonetic}

\begin{EntryWithPhonetic}{紧}{jin3}{10}{⽷}[HSK 3]
  \definition{adj.}{tenso; apertado; o estado em que um objeto se encontra após ser submetido a uma grande força de tração ou pressão.| seguro; firme | cerrado; apertado | urgente; premente; tenso | rigoroso; rígido; severo | difícil; sem dinheiro}
  \definition{v.}{apertar; tornar mais apertado}
\end{EntryWithPhonetic}

\begin{EntryWithPhonetic}{紧急}{jin3ji2}{10,9}{⽷、⼼}[HSK 3]
  \definition{adj./adj.}{urgente; premente; crítico}
\end{EntryWithPhonetic}

\begin{EntryWithPhonetic}{紧紧}{jin3 jin3}{10,10}{⽷、⽷}[HSK 5]
  \definition{adv.}{firmemente; estreitamente; apertadamente; prestar muita atenção (em algo)}
\end{EntryWithPhonetic}

\begin{EntryWithPhonetic}{紧密}{jin3 mi4}{10,11}{⽷、⼧}[HSK 4]
  \definition{adj.}{próximos; inseparáveis | incessante; rápido e intenso}
\end{EntryWithPhonetic}

\begin{EntryWithPhonetic}{紧张}{jin3zhang1}{10,7}{⽷、⼸}[HSK 3]
  \definition{adj.}{nervoso; tenso; mentalmente em estado de alerta, excitado e inquieto | apertado; em falta; o que está disponível não satisfaz os requisitos| tenso; intenso; intenso ou urgente, causando tensão mental}
\end{EntryWithPhonetic}

\begin{EntryWithPhonetic}{锦}{jin3}{13}{⾦}
  \definition*{s.}{Sobrenome Jin}
  \definition{adj.}{brilhante e bonito (cores brilhantes e lindas)}
  \definition[块]{s.}{brocado; tecidos de seda com padrões coloridos}
\end{EntryWithPhonetic}

\begin{EntryWithPhonetic}{锦上添花}{jin3 shang4 tian1 hua1}{13,3,11,7}{⾦、⼀、⽔、⾋}
  \definition{expr.}{adicionar flores ao brocado --- tornar o que é bom ainda melhor; melhorar | dourando o lírio}
\end{EntryWithPhonetic}

\begin{EntryWithPhonetic}{尽}{jin4}{6}{⼫}[HSK 6]
  \definition*{s.}{Sobrenome Jin}
  \definition{adj.}{exausto; acabado | ao máximo; ao limite | tudo; exaustivo}
  \definition{v.}{esgotar | tentar o seu melhor; fazer o melhor uso possível | morrer; falecer | terminar | chegar ao fim ao máximo; alcançar extremos}
  \seeref{jin3}
\end{EntryWithPhonetic}

\begin{EntryWithPhonetic}{尽力}{jin4/li4}{6,2}{⼫、⼒}[HSK 4]
  \definition{v.+compl.}{esforçar-se ao máximo; esforçar-se ao máximo; usar toda a sua força; fazer algo com seu melhor esforço}
\end{EntryWithPhonetic}

\begin{EntryWithPhonetic}{近}{jin4}{7}{⾡}[HSK 2]
  \definition{adj.}{próximo; perto; distância espacial ou temporal curta (oposto de 远) | íntimo; intimamente relacionado; relação estreita | fácil de entender}
  \seealsoref{远}{yuan3}
\end{EntryWithPhonetic}

\begin{EntryWithPhonetic}{近代}{jin4dai4}{7,5}{⾡、⼈}[HSK 4]
  \definition{s.}{tempos modernos; era passada relativamente próxima à era moderna, geralmente referida na história chinesa como 1840 a 1919 | o tempo ou era do capitalismo}
\end{EntryWithPhonetic}

\begin{EntryWithPhonetic}{近来}{jin4lai2}{7,7}{⾡、⽊}[HSK 5]
  \definition{adv.}{ultimamente; recentemente; de ​​tarde; refere-se a um período de tempo entre o passado imediato e o presente}
\end{EntryWithPhonetic}

\begin{EntryWithPhonetic}{近期}{jin4 qi1}{7,12}{⾡、⽉}[HSK 3]
  \definition{adv.}{num futuro próximo; brevemente}
\end{EntryWithPhonetic}

\begin{EntryWithPhonetic}{近日}{jin4 ri4}{7,4}{⾡、⽇}[HSK 6]
  \definition{s.}{recentemente; nos últimos dias; apontando para o passado | nos próximos dias; refere-se ao futuro}
\end{EntryWithPhonetic}

\begin{EntryWithPhonetic}{近视}{jin4 shi4}{7,8}{⾡、⾒}[HSK 6]
  \definition{adj.}{miopia; uma deficiência visual em que a visão próxima é clara, mas a visão distante é turva | míope (figurativo); metáfora para miopia}
\end{EntryWithPhonetic}

\begin{EntryWithPhonetic}{进}{jin4}{7}{⾡}[HSK 1]
  \definition*{s.}{Sobrenome Jin}
  \definition{clas.}{para seções em um edifício ou complexo residencial; qualquer uma das várias fileiras de casas em um complexo residencial de estilo antigo}
  \definition{s.}{(matemática) base de um sistema numérico}
  \definition{v.}{avançar; ir adiante; seguir em frente; (oposto a 退) | entrar; entrar em; entrar ou sair; (oposto a 出) | receber | comer; tomar; beber | submeter; apresentar | marcar um gol}
  \definition{v.aux.}{usado após um verbo, significa ``para dentro''}
  \seealsoref{出}{chu1}
  \seealsoref{退}{tui4}
\end{EntryWithPhonetic}

\begin{EntryWithPhonetic}{进步}{jin4bu4}{7,7}{⾡、⽌}[HSK 3]
  \definition{adj.}{progressivo; adequado às tendências da época; que impulsiona o desenvolvimento social (em oposição a 落后)}
  \definition{v.}{avançar; progredir; melhorar}
  \seealsoref{落后}{luo4hou4}
\end{EntryWithPhonetic}

\begin{EntryWithPhonetic}{进出口}{jin4chu1kou3}{7,5,3}{⾡、⼐、⼝}
  \definition{s.}{importação e exportação}
  \definition{v.}{importar e exportar}
\end{EntryWithPhonetic}

\begin{EntryWithPhonetic}{进攻}{jin4gong1}{7,7}{⾡、⽁}[HSK 6]
  \definition{s.}{ofensiva}
  \definition{v.}{atacar; assaltar; tomar a ofensiva (oposto à 防守)}
  \seealsoref{防守}{fang2shou3}
\end{EntryWithPhonetic}

\begin{EntryWithPhonetic}{进化}{jin4hua4}{7,4}{⾡、⼔}[HSK 5]
  \definition[个]{s.}{evolução; os organismos se desenvolvem e evoluem do simples para o complexo e de níveis baixos para altos}
  \definition{v.}{evoluir; um termo geral usado para descrever uma mudança gradual para melhor}
\end{EntryWithPhonetic}

\begin{EntryWithPhonetic}{进口}{jin4/kou3}{7,3}{⾡、⼝}[HSK 4]
  \definition{adj.}{importado}
  \definition{s.}{importação; entrada de um edifício ou local, também chamada de 入口}
  \definition{v.+compl.}{importar; comprar ou transportar mercadorias de outro país ou região | entrar no porto; navegar em direção a um porto}
  \seealsoref{入口}{ru4kou3}
\end{EntryWithPhonetic}

\begin{EntryWithPhonetic}{进来}{jin4 lai2}{7,7}{⾡、⽊}[HSK 1]
  \definition{v.}{entrar (para a minha localização)}
\end{EntryWithPhonetic}

\begin{EntryWithPhonetic}{进去}{jin4 qu4}{7,5}{⾡、⼛}[HSK 1]
  \definition{v.}{entrar (a partir da minha localização)}
  \definition{v.aux.}{usado depois de um verbo, significa ``ir para dentro''; para um determinado intervalo ou período de tempo}
\end{EntryWithPhonetic}

\begin{EntryWithPhonetic}{进入}{jin4 ru4}{7,2}{⾡、⼊}[HSK 2]
  \definition{v.}{entrar; entrar em}
\end{EntryWithPhonetic}

\begin{EntryWithPhonetic}{进行}{jin4xing2}{7,6}{⾡、⾏}[HSK 2]
  \definition{v.}{continuar; estar em andamento; estar em progresso | fazer; conduzir; realizar; executar | marchar; avançar; prosseguir; estar em marcha}
\end{EntryWithPhonetic}

\begin{EntryWithPhonetic}{进行编程}{jin4xing2bian1cheng2}{7,6,12,12}{⾡、⾏、⽷、⽲}
  \definition{s.}{programa de computador executável}
\end{EntryWithPhonetic}

\begin{EntryWithPhonetic}{进一步}{jin4 yi2 bu4}{7,1,7}{⾡、⼀、⽌}[HSK 3]
  \definition{adv.}{mais; dar um passo adiante; avançar um passo; indica que as coisas estão progredindo em um nível mais alto do que antes}
\end{EntryWithPhonetic}

\begin{EntryWithPhonetic}{进展}{jin4zhan3}{7,10}{⾡、⼫}[HSK 3]
  \definition{v.}{fazer progresso; progredir; avançar no desenvolvimento}
\end{EntryWithPhonetic}

\begin{EntryWithPhonetic}{禁}{jin4}{13}{⽰}
  \definition*{s.}{Sobrenome Jin}
  \definition{s.}{um tabu; assuntos não permitidos por lei ou costume | área proibida | residência real; o lugar onde o imperador viveu nos tempos antigos}
  \definition{v.}{proibir; banir | aprisionar; deter}
\end{EntryWithPhonetic}

\begin{EntryWithPhonetic}{禁止}{jin4zhi3}{13,4}{⽰、⽌}[HSK 4]
  \definition{v.}{banir; proibir; interditar}
\end{EntryWithPhonetic}

\begin{EntryWithPhonetic}{京}{jing1}{8}{⼇}
  \definition*{s.}{Pequim (Beijing), abreviação de 北京 | Sobrenome Jing}
  \definition{num.}{dez milhões (um numeral antigo); 10.000.000; 1000.0000}
  \definition{s.}{capital de um país}
  \seealsoref{北京}{bei3 jing1}
\end{EntryWithPhonetic}

\begin{EntryWithPhonetic}{京二胡}{jing1'er4hu2}{8,2,9}{⼇、⼆、⾁}
  \definition{s.}{um tipo de violino chinês semelhante ao 二胡 de duas cordas, usado principalmente para acompanhamento do canto da ópera de Pequim | também chamado de 京胡 | jing'erhu, um violino de duas cordas, intermediário em tamanho e tom entre o 京胡 e o 二胡, usado para acompanhar a ópera chinesa}
  \seealsoref{二胡}{er4hu2}
  \seealsoref{京胡}{jing1hu2}
\end{EntryWithPhonetic}

\begin{EntryWithPhonetic}{京胡}{jing1hu2}{8,9}{⼇、⾁}
  \definition{s.}{jinghu, um instrumento de arco de duas cordas com registro agudo; violino da ópera de Pequim | também chamado de 京二胡 | jinghu, um 二胡 (violino de duas cordas) menor e mais agudo, usado para acompanhar a ópera chinesa}
  \seealsoref{二胡}{er4hu2}
  \seealsoref{胡琴}{hu2qin2}
  \seealsoref{京二胡}{jing1'er4hu2}
\end{EntryWithPhonetic}

\begin{EntryWithPhonetic}{京剧}{jing1ju4}{8,10}{⼇、⼑}[HSK 3]
  \definition*[场,段]{s.}{Ópera de Pequim}
\end{EntryWithPhonetic}

\begin{EntryWithPhonetic}{经}{jing1}{8}{⽷}
  \definition*{s.}{Sobrenome Jing}
  \definition{adj.}{constante; regular}
  \definition{prep.}{como resultado de; depois; através de}
  \definition{s.}{urdidura, os fios longitudinais de um tecido (oposto a 纬) | Medicina chinesa: canais principais e colaterais | Geografia: longitude (oposto a 纬) | escritura; sutra; cânone; clássico | menstruação}
  \definition{v.}{Literário: gerenciar; lidar com; envolver-se em | enforcar-se | suportar; ficar de pé; aguentar; resistir | passar por; sofrer; experimentar}
  \seealsoref{纬}{wei3}
\end{EntryWithPhonetic}

\begin{EntryWithPhonetic}{经常}{jing1chang2}{8,11}{⽷、⼱}[HSK 2]
  \definition{adj.}{habitual; cotidiano; diário; do dia a dia}
  \definition{adv.}{frequentemente; regularmente; constantemente; com frequência; indica que a ação ocorre repetidamente}
\end{EntryWithPhonetic}

\begin{EntryWithPhonetic}{经典}{jing1dian3}{8,8}{⽷、⼋}[HSK 4]
  \definition{adj.}{clássico; (escritos ou obras, etc.) que são típicos, autorizados}
  \definition{s.}{clássicos; escritos tradicionais e valiosos; os livros mais importantes e fundamentais da religião | escrituras; escritos de doutrinas religiosas}
\end{EntryWithPhonetic}

\begin{EntryWithPhonetic}{经费}{jing1fei4}{8,9}{⽷、⾙}[HSK 5]
  \definition[笔]{s.}{fundos; desembolso; gastos | despesas; gastos}
\end{EntryWithPhonetic}

\begin{EntryWithPhonetic}{经过}{jing1guo4}{8,6}{⽷、⾡}[HSK 2]
  \definition{prep.}{depois; através; como resultado de; passar por uma atividade ou evento que traz novas mudanças para pessoas ou coisas}
  \definition[个,段,番]{s.}{processo; curso; experiência}
  \definition{v.}{passar; atravessar; passar por; através de (local, tempo, ação, etc.)}
\end{EntryWithPhonetic}

\begin{EntryWithPhonetic}{经济}{jing1ji4}{8,9}{⽷、⽔}[HSK 3]
  \definition{adj.}{econômico;  parcimonioso; descreve algo que custa pouco e rende muito; preço acessível}
  \definition{s.}{economia; a soma das relações de produção social em um determinado período histórico|econômico; de valor industrial ou econômico; refere-se à economia nacional; também se refere a um determinado setor da economia nacional | economia; refere-se às atividades econômicas, incluindo produção, circulação, distribuição e consumo, bem como atividades ou processos financeiros, de seguros, etc. | renda; situação financeira; refere-se à situação financeira de uma pessoa}
  \definition{v.}{governar o país e beneficiar o povo}
\end{EntryWithPhonetic}

\begin{EntryWithPhonetic}{经理}{jing1li3}{8,11}{⽷、⽟}[HSK 2]
  \definition[个,位,名]{s.}{gerente; diretor; pessoas responsáveis pela gestão e administração de empresas ou corporações}
\end{EntryWithPhonetic}

\begin{EntryWithPhonetic}{经历}{jing1li4}{8,4}{⽷、⼚}[HSK 3]
  \definition[个,次,段,种]{s.}{experiência; coisas que você viu, fez ou sofreu pessoalmente}
  \definition{v.}{passar por; atravessar; ter visto, feito ou sofrido pessoalmente}
\end{EntryWithPhonetic}

\begin{EntryWithPhonetic}{经验}{jing1yan4}{8,10}{⽷、⾺}[HSK 3]
  \definition[个,次,种]{s.}{experiência; conhecimento ou habilidades adquiridos através da prática}
  \definition{v.}{experimentar; passar por; ter visto, feito ou sofrido pessoalmente}
\end{EntryWithPhonetic}

\begin{EntryWithPhonetic}{经营}{jing1ying2}{8,11}{⽷、⾋}[HSK 3]
  \definition{v.}{executar; gerenciar; operar; envolver-se em; planejar e gerenciar (empresas, etc.) | gerenciar; refere-se a planos e organizações em geral}
\end{EntryWithPhonetic}

\begin{EntryWithPhonetic}{惊}{jing1}{11}{⼼}
  \definition{v.}{assustar; ficar assustado; ficar nervoso devido a estímulo repentino; ficar com medo | surpreender; chocar; alarmar}
\end{EntryWithPhonetic}

\begin{EntryWithPhonetic}{惊呆}{jing1dai1}{11,7}{⼼、⼝}
  \definition{adj.}{estupefato | chocado}
\end{EntryWithPhonetic}

\begin{EntryWithPhonetic}{惊人}{jing1 ren2}{11,2}{⼼、⼈}[HSK 6]
  \definition{adj.}{surpreso; espantado; atônito; surpreendente}
\end{EntryWithPhonetic}

\begin{EntryWithPhonetic}{惊喜}{jing1 xi3}{11,12}{⼼、⼝}[HSK 6]
  \definition{s.}{boa surpresa; agradavelmente surpreso}
\end{EntryWithPhonetic}

\begin{EntryWithPhonetic}{精}{jing1}{14}{⽶}[HSK 6]
  \definition{adv.}{muito; extremamente; antes de certos adjetivos, significa 十分 ou 非常}
  \definition{s.}{refinado; escolhido; escolha; purificado ou selecionado | perfeito; excelente; melhor | fino (em oposição a 粗); preciso; meticuloso | inteligente; astuto; esperto | habilidoso; versado; proficiente | extrato; essência; essência refinada ou selecionada; extraída | energia; espírito | semente; esperma; sêmen | \emph{goblin}; espírito; elfo; demônio}
  \seealsoref{粗}{cu1}
  \seealsoref{非常}{fei1chang2}
  \seealsoref{十分}{shi2fen1}
\end{EntryWithPhonetic}

\begin{EntryWithPhonetic}{精彩}{jing1cai3}{14,11}{⽶、⼺}[HSK 3]
  \definition{adj.}{brilhante; esplêndido; maravilhoso}
\end{EntryWithPhonetic}

\begin{EntryWithPhonetic}{精力}{jing1li4}{14,2}{⽶、⼒}[HSK 4]
  \definition[些]{s.}{energia; vigor; força mental e física}
\end{EntryWithPhonetic}

\begin{EntryWithPhonetic}{精灵}{jing1ling2}{14,7}{⽶、⽕}
  \definition{s.}{espírito | fada | elfo | duende | gênio}
\end{EntryWithPhonetic}

\begin{EntryWithPhonetic}{精美}{jing1 mei3}{14,9}{⽶、⽺}[HSK 6]
  \definition{adj.}{elegante; requintado}
\end{EntryWithPhonetic}

\begin{EntryWithPhonetic}{精品}{jing1pin3}{14,9}{⽶、⼝}[HSK 6]
  \definition[个]{s.}{belas obras (de arte); objetos de arte | produtos de qualidade; artigos de excelente qualidade; produto \emph{premium}}
\end{EntryWithPhonetic}

\begin{EntryWithPhonetic}{精神}{jing1shen2}{14,9}{⽶、⽰}[HSK 3]
  \definition[种,个,类,股]{s.}{espírito; mente; estado mental; refere-se à consciência, às atividades mentais e ao estado psicológico geral de uma pessoa | substância; espírito; essência; propósito; significado principal}
  \seeref{jing1shen5}
\end{EntryWithPhonetic}

\begin{EntryWithPhonetic}{精神}{jing1shen5}{14,9}{⽶、⽰}[HSK 3]
  \definition{adj.}{animado; espirituoso; vigoroso; descreve uma pessoa como cheia de energia | muito bonito; boa aparência, bom físico}
  \definition[种,个,类,股]{s.}{impulso; vigor; vitalidade}
  \seeref{jing1shen2}
\end{EntryWithPhonetic}

\begin{EntryWithPhonetic}{精致}{jing1zhi4}{14,10}{⽶、⾄}
  \definition{adj.}{delicado | exótico | refinado}
\end{EntryWithPhonetic}

\begin{EntryWithPhonetic}{鲸}{jing1}{16}{⿂}
  \definition[头,只,条]{s.}{baleia; cetáceo}
\end{EntryWithPhonetic}

\begin{EntryWithPhonetic}{鲸鲨}{jing1sha1}{16,15}{⿂、⿂}
  \definition{s.}{tubarão baleia}
\end{EntryWithPhonetic}

\begin{EntryWithPhonetic}{鲸鱼}{jing1yu2}{16,8}{⿂、⿂}
  \definition{s.}{baleia}
\end{EntryWithPhonetic}

\begin{EntryWithPhonetic}{井}{jing3}{4}{⼆}[HSK 6]
  \definition*{s.}{Jing, uma das mansões lunares | Sobrenome Jing}
  \definition{adj.}{limpo; organizado}
  \definition[口]{s.}{poço; um buraco profundo cavado no chão para tirar água | algo em forma de poço | vila natal ou cidade natal}
\end{EntryWithPhonetic}

\begin{EntryWithPhonetic}{景}{jing3}{12}{⽇}[HSK 6]
  \definition*{s.}{Sobrenome Jing}
  \definition{adj.}{grandioso; elevado; grande}
  \definition{s.}{vista; cenário; cena | situação; condição | cenário (de uma peça ou filme) | cena (de uma peça)}
  \definition{v.}{admirar; reverenciar; respeitar}
\end{EntryWithPhonetic}

\begin{EntryWithPhonetic}{景点}{jing3 dian3}{12,9}{⽇、⽕}[HSK 6]
  \definition[个,处]{s.}{local cênico; atração turística; um lugar onde se concentram as atrações turísticas, incluindo atrações naturais e culturais}
\end{EntryWithPhonetic}

\begin{EntryWithPhonetic}{景色}{jing3se4}{12,6}{⽇、⾊}[HSK 3]
  \definition[片,幅,道,处]{s.}{vista; cena; cenário; paisagem}
\end{EntryWithPhonetic}

\begin{EntryWithPhonetic}{景象}{jing3 xiang4}{12,11}{⽇、⾗}[HSK 5]
  \definition[个,种]{s.}{cena; visão; vista; quadro}
\end{EntryWithPhonetic}

\begin{EntryWithPhonetic}{警}{jing3}{19}{⾔}
  \definition{s.}{policial}
  \definition{v.}{alertar | avisar}
\end{EntryWithPhonetic}

\begin{EntryWithPhonetic}{警察}{jing3cha2}{19,14}{⾔、⼧}[HSK 3]
  \definition[个,位,名,群,队]{s.}{polícia; policial; oficial de polícia; as forças armadas que mantêm a segurança social do país são uma parte importante do aparato estatal; também se refere aos membros dessas forças armadas}
\end{EntryWithPhonetic}

\begin{EntryWithPhonetic}{警告}{jing3gao4}{19,7}{⾔、⼝}[HSK 5]
  \definition[个]{s.}{advertência (como medida disciplinar); uma forma de punição}
  \definition{v.}{avisar; advertir; admoestar}
\end{EntryWithPhonetic}

\begin{EntryWithPhonetic}{警官}{jing3guan1}{19,8}{⾔、⼧}
  \definition{s.}{polícia | policial}
\end{EntryWithPhonetic}

\begin{EntryWithPhonetic}{净}{jing4}{8}{⼎}[HSK 6]
  \definition{adj.}{limpo | (depois de um verbo) terminado; sem nada sobrando | líquido | vazio; oco; nu}
  \definition{adv.}{todo; o tempo todo | somente; meramente; nada além de | inteiramente; indica puro e nada mais}
  \definition{s.}{o “rosto pintado”, comumente conhecido como Hualian, 花脸, um tipo de personagem da ópera de Pequim, etc.}
  \definition{v.}{tornar limpo | limpar; lavar; esfregar para limpar}
  \seealsoref{花脸}{hua1lian3}
\end{EntryWithPhonetic}

\begin{EntryWithPhonetic}{竞}{jing4}{10}{⽴}
  \definition{adj.}{forte; poderoso}
  \definition{v.}{competir; contender; disputar | contestar}
\end{EntryWithPhonetic}

\begin{EntryWithPhonetic}{竞赛}{jing4sai4}{10,14}{⽴、⾙}[HSK 5]
  \definition[个]{s.}{concurso; competição; partida; corrida}
  \definition{v.}{correr; competir; competir uns com os outros por superioridade; em esportes, produção e outras atividades, para comparar competência, habilidade etc., usado principalmente na linguagem falada}
\end{EntryWithPhonetic}

\begin{EntryWithPhonetic}{竞争}{jing4zheng1}{10,6}{⽴、⼑}[HSK 5]
  \definition{v.}{competir; disputar; lutar; entre duas ou mais partes; em prol de seus próprios interesses; lutar pela vitória por meio de uma disputa de sua própria força contra outra}
\end{EntryWithPhonetic}

\begin{EntryWithPhonetic}{竟}{jing4}{11}{⾳}
  \definition{adj.}{todo; por toda parte; do começo ao fim}
  \definition{adv.}{no final; eventualmente | na verdade; inesperadamente; significa algo inesperado, equivalente a 居然}
  \definition{v.}{terminar; completar | investigar}
  \seealsoref{居然}{ju1ran2}
\end{EntryWithPhonetic}

\begin{EntryWithPhonetic}{竟然}{jing4ran2}{11,12}{⾳、⽕}[HSK 4]
  \definition{adv.}{de fato; inesperadamente; para surpresa de alguém; chegar ao ponto de; indica que algo é um pouco inesperado}
\end{EntryWithPhonetic}

\begin{EntryWithPhonetic}{敬}{jing4}{12}{⽁}
  \definition*{s.}{Sobrenome Jing}
  \definition{adj.}{respeitoso; reverente}
  \definition{adv.}{respeitosamente}
  \definition{v.}{respeitar; honrar; estimar | oferecer educadamente | envolver-se em; dedicar-se a}
\end{EntryWithPhonetic}

\begin{EntryWithPhonetic}{敬礼}{jing4li3}{12,5}{⽁、⽰}
  \definition{s.}{saudação}
  \definition{v.}{saudar}
\end{EntryWithPhonetic}

\begin{EntryWithPhonetic}{静}{jing4}{14}{⾭}[HSK 3]
  \definition*{s.}{Sobrenome Jing}
  \definition{adj.}{tranquilo;  sossegado; calmo; imóvel | silencioso; quieto; sem emitir nenhum som | calmo, sereno; serenidade; (interior) paz}
  \definition{v.}{acalmar-se; aquietar-se; tranquilizar (o coração)}
\end{EntryWithPhonetic}

\begin{EntryWithPhonetic}{镜}{jing4}{16}{⾦}
  \definition*{s.}{Sobrenome Jing}
  \definition[面,副]{s.}{espelho | lente; vidro; dispositivos para auxiliar a visão ou conduzir experimentos ópticos}
  \definition{v.}{espelhar | perceber | usar como referência}
\end{EntryWithPhonetic}

\begin{EntryWithPhonetic}{镜头}{jing4tou2}{16,5}{⾦、⼤}[HSK 4]
  \definition[个,组]{s.}{lente de câmera; objetiva; combinação de várias lentes, usada para formar uma imagem | foto; cena}
\end{EntryWithPhonetic}

\begin{EntryWithPhonetic}{镜子}{jing4zi5}{16,3}{⾦、⼦}[HSK 4]
  \definition[面,个]{s.}{espelho; instrumento de reflexão de imagem liso e plano, antigamente esmerilhado a partir de um disco grosso de cobre fundido, atualmente feito de vidro plano revestido de prata ou alumínio | óculos; óculos de grau}
\end{EntryWithPhonetic}

\begin{EntryWithPhonetic}{纠}{jiu1}{5}{⽷}
  \definition*{s.}{Sobrenome Jiu}
  \definition{v.}{emaranhar | reunir-se | corrigir; retificar | supervisionar; superintender}
\end{EntryWithPhonetic}

\begin{EntryWithPhonetic}{纠纷}{jiu1fen1}{5,7}{⽷、⽷}[HSK 6]
  \definition[个,次]{s.}{questão; disputa; existem contradições ou conflitos de interesse entre as duas partes que precisam ser resolvidos}
\end{EntryWithPhonetic}

\begin{EntryWithPhonetic}{纠葛}{jiu1ge2}{5,12}{⽷、⾋}
  \definition{s.}{emaranhado | disputa}
\end{EntryWithPhonetic}

\begin{EntryWithPhonetic}{纠正}{jiu1zheng4}{5,5}{⽷、⽌}[HSK 6]
  \definition{v.}{fazer certo; corrigir (deficiências ou erros em pensamentos, ações, métodos, etc.)}
\end{EntryWithPhonetic}

\begin{EntryWithPhonetic}{究}{jiu1}{7}{⽳}
  \definition{adv.}{na verdade; realmente; afinal}
  \definition{v.}{estudar cuidadosamente; aprofundar; investigar; rastrear}
\end{EntryWithPhonetic}

\begin{EntryWithPhonetic}{究竟}{jiu1jing4}{7,11}{⽳、⾳}[HSK 4]
  \definition{adv.}{de fato; exatamente; usado em frases interrogativas para buscar | afinal de contas, no final; ênfase em fatos ou motivos}
  \definition{s.}{resultado; desfecho; a causa, o efeito ou a história completa do que aconteceu}
\end{EntryWithPhonetic}

\begin{EntryWithPhonetic}{九}{jiu3}{2}{⼄}[HSK 1]
  \definition*{s.}{Sobrenome Jiu}
  \definition{adj.}{muitos; numerosos; indica várias vezes ou a maioria das vezes}
  \definition{num.}{nove; 9}
  \definition{s.}{cada um dos nove períodos de nove dias começando no dia seguinte ao solstício de inverno}
\end{EntryWithPhonetic}

\begin{EntryWithPhonetic}{久}{jiu3}{3}{⼃}[HSK 3]
  \definition{adj.}{por muito tempo; longo período de tempo | duração de tempo especificada}
\end{EntryWithPhonetic}

\begin{EntryWithPhonetic}{韭}{jiu3}{9}{⾲}[Kangxi 179]
  \definition{s.}{alho de flor perfumada; cebolinha chinesa}
\end{EntryWithPhonetic}

\begin{EntryWithPhonetic}{韭菜}{jiu3cai4}{9,11}{⾲、⾋}
  \definition{s.}{cebolinha chinesa | (figurativo) investidores de varejo que perdem seu dinheiro para operadores mais experientes (ou seja, são ``colhidos'' como cebolinhas)}
\end{EntryWithPhonetic}

\begin{EntryWithPhonetic}{酒}{jiu3}{10}{⾣}[HSK 2]
  \definition*{s.}{Sobrenome Jiu}
  \definition[口,杯,瓶,罐,桶,缸]{s.}{bebida alcoólica; vinho; licor; bebidas destiladas}
\end{EntryWithPhonetic}

\begin{EntryWithPhonetic}{酒吧}{jiu3ba1}{10,7}{⾣、⼝}[HSK 4]
  \definition[家,间]{s.}{bar; \emph{pub}; um local onde são vendidas bebidas alcoólicas e onde as pessoas podem beber e conversar, referindo-se principalmente a um restaurante ou hotel de estilo ocidental especializado na venda de bebidas alcoólicas.}
\end{EntryWithPhonetic}

\begin{EntryWithPhonetic}{酒店}{jiu3 dian4}{10,8}{⾣、⼴}[HSK 2]
  \definition[家,个]{s.}{hotel; Estabelecimento comercial que oferece hospedagem e alimentação aos hóspedes | restaurante}
\end{EntryWithPhonetic}

\begin{EntryWithPhonetic}{酒馆}{jiu3guan3}{10,11}{⾣、⾷}
  \definition{s.}{bar | taverna | adega}
\end{EntryWithPhonetic}

\begin{EntryWithPhonetic}{酒鬼}{jiu3gui3}{10,9}{⾣、⿁}[HSK 5]
  \definition[个]{s.}{bebedor de vinho; beberrão; ébrio | alcoólatra}
\end{EntryWithPhonetic}

\begin{EntryWithPhonetic}{酒水}{jiu3 shui3}{10,4}{⾣、⽔}[HSK 6]
  \definition{s.}{bebidas; bebidas e álcool | Dialeto: festa; banquete}
\end{EntryWithPhonetic}

\begin{EntryWithPhonetic}{旧}{jiu4}{5}{⽇}[HSK 3]
  \definition{adj.}{passado; antigo; velho; ultrapassado (em oposição a 新)| usado; desgastado; velho; descolorido ou deformado devido ao uso prolongado ou ao tempo | antigo; único; que já existiu; anterior}
  \definition{s.}{velha amizade; velho amigo}
  \seealsoref{新}{xin1}
\end{EntryWithPhonetic}

\begin{EntryWithPhonetic}{救}{jiu4}{11}{⽁}[HSK 3]
  \definition*{s.}{Sobrenome Jiu}
  \definition{v.}{resgatar; salvar | salvar de; aliviar (angústia, etc.) | resgatar; livrar alguém de um desastre ou perigo | ajudar; aliviar; socorrer; livrar pessoas e coisas de desastres e perigos}
\end{EntryWithPhonetic}

\begin{EntryWithPhonetic}{救出}{jiu4chu1}{11,5}{⽁、⼐}
  \definition{v.}{resgatar | tirar do perigo}
\end{EntryWithPhonetic}

\begin{EntryWithPhonetic}{救护车}{jiu4hu4che1}{11,7,4}{⽁、⼿、⾞}
  \definition[辆]{s.}{ambulância}
\end{EntryWithPhonetic}

\begin{EntryWithPhonetic}{救命}{jiu4/ming4}{11,8}{⽁、⼝}[HSK 6]
  \definition{interj.}{Socorro!; Salve-me!}
  \definition{v.+compl.}{ajudar; salvar a vida de alguém}
\end{EntryWithPhonetic}

\begin{EntryWithPhonetic}{救援}{jiu4 yuan2}{11,12}{⽁、⼿}[HSK 6]
  \definition{v.}{resgatar; socorrer; vir em auxílio de alguém (resgate)}
\end{EntryWithPhonetic}

\begin{EntryWithPhonetic}{救灾}{jiu4 zai1}{11,7}{⽁、⽕}[HSK 5]
  \definition{v.}{ajudar as vítimas de desastres, aliviar o desastre; resgatar pessoas afetadas por desastres; recuperar danos causados por desastres}
\end{EntryWithPhonetic}

\begin{EntryWithPhonetic}{就}{jiu4}{12}{⼪}[HSK 1]
  \definition{adv.}{de imediato; imediatamente; indica que algo ocorrerá em breve | tão cedo quanto; já; há muito tempo; indica que a ação ocorreu há muito tempo | assim que; logo depois; indica que os eventos se sucedem imediatamente | nesse caso; então; indica que, sob determinadas condições, ocorre naturalmente um determinado resultado | exatamente; precisamente; indica que é exatamente assim | apenas; meramente; somente | tantos quanto; enfatiza a quantidade | apenas; simplesmente; reforço da afirmação | colocado entre dois componentes idênticos, significa tolerância ou indiferença}
  \definition{prep.}{tirar proveito de alguém (algo); expressa condições, oportunidades, etc., equivalente a 趁 | quando se trata de alguém (algo); relativo a; com relação a; sobre; objeto ou escopo da introdução da ação |no local; introduz o local próximo ao qual a ação ocorreu}
  \definition{v.}{ser comido com; ir com; pratos, frutas, etc., acompanhados de alimentos básicos ou bebidas alcoólicas | aproximar-se; mover-se em direção a | ir para; assumir; empreender; envolver-se em; entrar em | realizar; fazer | tirar proveito de; acomodar-se a; adequar-se; encaixar-se | assumir; começar a entrar ou a exercer | seguir; acompanhar}
  \seealsoref{趁}{chen4}
\end{EntryWithPhonetic}

\begin{EntryWithPhonetic}{就是}{jiu4 shi4}{12,9}{⼪、⽇}[HSK 3]
  \definition{adv.}{exatamente; precisamente; expressar concordância com a afirmação da outra pessoa ou confirmar que a afirmação da outra pessoa está correta | apenas; simplesmente; expressa afirmação, determinação ou ênfase, o significado específico deve ser determinado com base no contexto anterior ou posterior | usado para indicar escolha}
  \definition{conj.}{ainda que; mesmo que se reconheça que essa situação é verdadeira, a situação posterior não mudará}
  \definition{part.}{usado no final de uma frase para expressar afirmação}
\end{EntryWithPhonetic}

\begin{EntryWithPhonetic}{就是说}{jiu4 shi4 shuo1}{12,9,9}{⼪、⽇、⾔}[HSK 6]
  \definition{interj.}{ou seja; isto é; em outras palavras; é frequentemente usado como uma interjeição em uma frase para indicar que as palavras seguintes são uma explicação ou esclarecimento das anteriores}
\end{EntryWithPhonetic}

\begin{EntryWithPhonetic}{就算}{jiu4 suan4}{12,14}{⼪、⽵}[HSK 6]
  \definition{conj.}{mesmo que; concedido que; expressam uma relação hipotética e concessiva, frequentemente usadas com 也, equivalente a 即使}
  \seealsoref{即使}{ji2shi3}
  \seealsoref{也}{ye3}
\end{EntryWithPhonetic}

\begin{EntryWithPhonetic}{就要}{jiu4 yao4}{12,9}{⼪、⾑}[HSK 2]
  \definition{adv.}{estar prestes a; estar indo para; estar no ponto de}
\end{EntryWithPhonetic}

\begin{EntryWithPhonetic}{就业}{jiu4/ye4}{12,5}{⼪、⼀}[HSK 3]
  \definition{v.+compl.}{conseguir um emprego; obter emprego; assumir uma ocupação; começar a trabalhar}
\end{EntryWithPhonetic}

\begin{EntryWithPhonetic}{就职}{jiu4zhi2}{12,11}{⼪、⽿}
  \definition{v.}{assumir o cargo | assumir um posto}
\end{EntryWithPhonetic}

\begin{EntryWithPhonetic}{车}{ju1}{4}{⾞}[Kangxi 159]
  \definition{s.}{torre; castelo; carruagem, uma das peças do xadrez chinês}
  \seeref{che1}
\end{EntryWithPhonetic}

\begin{EntryWithPhonetic}{居}{ju1}{8}{⼫}
  \definition*{s.}{Sobrenome Ju}
  \definition{s.}{residência; casa | restaurante (em nomes de restaurantes)}
  \definition{v.}{residir; morar; viver | ocupar uma determinada posição; ocupar (um lugar); estar (em uma determinada posição) | reivindicar; afirmar | armazenar; guardar | ficar parado; estar parado}
\end{EntryWithPhonetic}

\begin{EntryWithPhonetic}{居民}{ju1min2}{8,5}{⼫、⽒}[HSK 4]
  \definition[个,户,位]{s.}{residente; habitante; pessoas que estão fixas em um único lugar}
\end{EntryWithPhonetic}

\begin{EntryWithPhonetic}{居然}{ju1ran2}{8,12}{⼫、⽕}[HSK 5]
  \definition{adv.}{inesperadamente; para surpresa de alguém; além da expectativa (expressão idiomática)}
  \definition{v.}{ir tão longe a ponto de; ter a impudência de; ter o descaramento de}
\end{EntryWithPhonetic}

\begin{EntryWithPhonetic}{居住}{ju1zhu4}{8,7}{⼫、⼈}[HSK 4]
  \definition{v.}{viver; residir; morar; habitar}
\end{EntryWithPhonetic}

\begin{EntryWithPhonetic}{据}{ju1}{11}{⼿}
  \definition{part.}{elemento formador de palavras}
  \seeref{ju4}
  \seealsoref{拮据}{jie2ju1}
\end{EntryWithPhonetic}

\begin{EntryWithPhonetic}{局}{ju2}{7}{⼫}[HSK 4,6]
  \definition{adj.}{limitado; confinado}
  \definition{clas.}{\emph{set}; jogo; turno}
  \definition{s.}{tabuleiro de xadrez | situação; estado de coisas | generosidade de espírito; extensão da tolerância de alguém | festa; reunião; refere-se a certas reuniões | ardil; armadilha | parte; porção; papel | escritório; agência; agências governamentais divididas por negócios | significa ``loja'' em nomes de lojas | departamento; agência; nomes de certas entidades empresariais | escritório; usado como nome de uma instituição ou outro local de negócios}
\end{EntryWithPhonetic}

\begin{EntryWithPhonetic}{局面}{ju2mian4}{7,9}{⼫、⾯}[HSK 5]
  \definition[种]{s.}{aspecto; fase; situação; o estado das coisas em um período de tempo, em sua maior parte abstraído | escopo; escala}
\end{EntryWithPhonetic}

\begin{EntryWithPhonetic}{局长}{ju2 zhang3}{7,4}{⼫、⾧}[HSK 5]
  \definition[位,名,个,些]{s.}{comissário; diretor; principais chefes de gabinete do governo}
\end{EntryWithPhonetic}

\begin{EntryWithPhonetic}{橘}{ju2}{16}{⽊}
  \definition[只,棵]{s.}{tangerina}
\end{EntryWithPhonetic}

\begin{EntryWithPhonetic}{橘子汁}{ju2zi5zhi1}{16,3,5}{⽊、⼦、⽔}
  \definition[瓶,杯,罐,盒]{s.}{suco de laranja}
  \seealsoref{橙汁}{cheng2zhi1}
  \seealsoref{柳橙汁}{liu3cheng2zhi1}
\end{EntryWithPhonetic}

\begin{EntryWithPhonetic}{柜}{ju3}{8}{⽊}
  \definition{s.}{faia; salgueiro}
  \seeref{gui4}
\end{EntryWithPhonetic}

\begin{EntryWithPhonetic}{举}{ju3}{9}{⼂}[HSK 2]
  \definition*{s.}{Sobrenome Ju}
  \definition{adj.}{inteiro; completo}
  \definition{s.}{ato; ação; movimento; comportamento | (nas dinastias Ming e Qing) candidato aprovado nos exames imperiais a nível provincial}
  \definition{v.}{levantar; erguer; sustentar | começar; iniciar; surgir | eleger; escolher; recomendar; selecionar | citar; enumerar; propor; revelar}
\end{EntryWithPhonetic}

\begin{EntryWithPhonetic}{举办}{ju3ban4}{9,4}{⼂、⼒}[HSK 3]
  \definition{v.}{conduzir; organizar; realizar}
\end{EntryWithPhonetic}

\begin{EntryWithPhonetic}{举动}{ju3dong4}{9,6}{⼂、⼒}[HSK 5]
  \definition{s.}{ato; atividade; movimento; ação}
\end{EntryWithPhonetic}

\begin{EntryWithPhonetic}{举手}{ju3 shou3}{9,4}{⼂、⼿}[HSK 2]
  \definition{v.}{levantar a mão ou as mãos; levantar a mão para sinalizar ou responder a uma pergunta}
\end{EntryWithPhonetic}

\begin{EntryWithPhonetic}{举行}{ju3xing2}{9,6}{⼂、⾏}[HSK 2]
  \definition{v.}{realizar (uma reunião, cerimônia, etc.); realizar (atividades formais ou solenes)}
\end{EntryWithPhonetic}

\begin{EntryWithPhonetic}{巨}{ju4}{4}{⼯}
  \definition*{s.}{Sobrenome Ju}
  \definition{adj.}{enorme; tremendo; gigantesco}
\end{EntryWithPhonetic}

\begin{EntryWithPhonetic}{巨大}{ju4da4}{4,3}{⼯、⼤}[HSK 4]
  \definition{adj.}{enorme; tremendo; enorme; gigantesco; imenso}
\end{EntryWithPhonetic}

\begin{EntryWithPhonetic}{句}{ju4}{5}{⼝}[HSK 2]
  \definition{clas.}{para sentenças, frases ou linhas de versos}
  \definition{s.}{frase; sentença}
  \seeref{gou4}
\end{EntryWithPhonetic}

\begin{EntryWithPhonetic}{句子}{ju4zi5}{5,3}{⼝、⼦}[HSK 2]
  \definition[个,句]{s.}{sentença; uma unidade linguística composta por palavras ou frases que expressa um significado completo}
\end{EntryWithPhonetic}

\begin{EntryWithPhonetic}{拒}{ju4}{7}{⼿}
  \definition{v.}{resistir; repelir | recusar; rejeitar}
\end{EntryWithPhonetic}

\begin{EntryWithPhonetic}{拒绝}{ju4jue2}{7,9}{⼿、⽷}[HSK 5]
  \definition{v.}{recusar; rejeitar; declinar; não aceitar (pedidos, sugestões ou presentes)}
\end{EntryWithPhonetic}

\begin{EntryWithPhonetic}{足}{ju4}{7}{⾜}[Kangxi 157]
  \definition{adj.}{excessivo}
  \seeref{zu2}
\end{EntryWithPhonetic}

\begin{EntryWithPhonetic}{具}{ju4}{8}{⼋}
  \definition*{s.}{Sobrenome Ju}
  \definition{clas.}{(literário) usado para caixões, cadáveres e certos objetos}
  \definition{s.}{utensílio; ferramenta; implemento | capacidade; habilidade}
  \definition{v.}{possuir; ter | fornecer; prover | declarar; enumerar}
\end{EntryWithPhonetic}

\begin{EntryWithPhonetic}{具备}{ju4bei4}{8,8}{⼋、⼡}[HSK 4]
  \definition{v.}{ter; possuir; ser provido de}
\end{EntryWithPhonetic}

\begin{EntryWithPhonetic}{具体}{ju4ti3}{8,7}{⼋、⼈}[HSK 3]
  \definition{adj.}{específico; particular | concreto; específico; mais detalhado; muito detalhado; muito claro | concreto; real; não é abstrato, tem uma forma definida; pode ser visto ou sentido}
  \definition{v.}{incorporar; objetivar; combinar teorias, princípios, padrões, etc. com pessoas ou coisas específicas}
\end{EntryWithPhonetic}

\begin{EntryWithPhonetic}{具有}{ju4 you3}{8,6}{⼋、⽉}[HSK 3]
  \definition{v.}{ter; possuir; ser provido de}
\end{EntryWithPhonetic}

\begin{EntryWithPhonetic}{俱}{ju4}{10}{⼈}
  \definition{adv.}{(literário) tudo; completamente; inteiramente}
\end{EntryWithPhonetic}

\begin{EntryWithPhonetic}{俱乐部}{ju4le4bu4}{10,5,10}{⼈、⼃、⾢}[HSK 5]
  \definition[个,家,间]{s.}{clube; grupos e locais para atividades sociais, políticas, literárias, recreativas e outras}
\end{EntryWithPhonetic}

\begin{EntryWithPhonetic}{剧}{ju4}{10}{⼑}[HSK 6]
  \definition*{s.}{Sobrenome Ju}
  \definition{adj.}{agudo; grave; intenso; violento}
  \definition[部,个,种]{s.}{obra teatral; drama; peça; ópera}
\end{EntryWithPhonetic}

\begin{EntryWithPhonetic}{剧本}{ju4ben3}{10,5}{⼑、⽊}[HSK 5]
  \definition[部,个]{s.}{cenário; roteiro (para drama, filme, etc.); gênero de obra literária que consiste em diálogos entre personagens (às vezes cantados) e indicações de palco}
\end{EntryWithPhonetic}

\begin{EntryWithPhonetic}{剧场}{ju4 chang3}{10,6}{⼑、⼟}[HSK 3]
  \definition[个,坐]{s.}{teatro; local para apresentações teatrais, musicais, etc.}
\end{EntryWithPhonetic}

\begin{EntryWithPhonetic}{据}{ju4}{11}{⼿}[HSK 6]
  \definition*{s.}{Sobrenome Ju}
  \definition{prep.}{de acordo com; com base em}
  \definition{s.}{evidência; certificado; prova}
  \definition{v.}{ocupar; apreender | confiar em; depender de}
  \seeref{ju1}
\end{EntryWithPhonetic}

\begin{EntryWithPhonetic}{据说}{ju4shuo1}{11,9}{⼿、⾔}[HSK 3]
  \definition{v.}{é o que dizem; é o que se diz}
\end{EntryWithPhonetic}

\begin{EntryWithPhonetic}{距}{ju4}{11}{⾜}
  \definition{s.}{distância | espora (de um galo, etc.)}
  \definition{v.}{estar separado (longe) de; estar distante de}
\end{EntryWithPhonetic}

\begin{EntryWithPhonetic}{距离}{ju4li2}{11,10}{⾜、⼇}[HSK 4]
  \definition[个,段]{s.}{distância}
  \definition{v.}{estar distante de}
\end{EntryWithPhonetic}

\begin{EntryWithPhonetic}{聚}{ju4}{14}{⽿}[HSK 4]
  \definition*{s.}{Sobrenome Ju}
  \definition{v.}{reunir-se; juntar-se}
\end{EntryWithPhonetic}

\begin{EntryWithPhonetic}{聚会}{ju4hui4}{14,6}{⽿、⼈}[HSK 4]
  \definition[个,次]{s.}{reunião; encontro; confraternização; festa}
  \definition{v.}{encontrar-se; reunir-se}
\end{EntryWithPhonetic}

\begin{EntryWithPhonetic}{聚散}{ju4san4}{14,12}{⽿、⽁}
  \definition{s.}{juntos e separados | agregação e dissipação}
\end{EntryWithPhonetic}

\begin{EntryWithPhonetic}{捐}{juan1}{10}{⼿}[HSK 6]
  \definition{s.}{imposto}
  \definition{v.}{renunciar; abandonar | contribuir; doar; assinar}
\end{EntryWithPhonetic}

\begin{EntryWithPhonetic}{捐款}{juan1/kuan3}{10,12}{⼿、⽋}[HSK 6]
  \definition[笔]{s.}{doação; contribuição (de dinheiro); valor doado}
  \definition{v.+compl.}{doar; contribuir com dinheiro}
\end{EntryWithPhonetic}

\begin{EntryWithPhonetic}{捐赠}{juan1 zeng4}{10,16}{⼿、⾙}[HSK 6]
  \definition{v.}{apresentar; contribuir (como um presente); doar (itens para um país ou grupo)}
\end{EntryWithPhonetic}

\begin{EntryWithPhonetic}{捐助}{juan1 zhu4}{10,7}{⼿、⼒}[HSK 6]
  \definition{v.}{oferecer (assistência financeira ou material); contribuir; doar}
\end{EntryWithPhonetic}

\begin{EntryWithPhonetic}{圈}{juan1}{11}{⼞}
  \definition{v.}{prender aves e animais de criação | prender; colocar na cadeia, prisão | confinar}
  \seeref{juan4}
  \seeref{quan1}
\end{EntryWithPhonetic}

\begin{EntryWithPhonetic}{卷}{juan3}{8}{⼙}[HSK 4]
  \definition{clas.}{usado para pequenas coisas enroladas (maço de papel dinheiro, carretel de filme, etc.) | usado para rolos, carretéis, bobinas, etc.}
  \definition[张]{s.}{rolo; carretel; bobina}
  \definition{v.}{enrolar; dobrar algo em um cilindro ou semicírculo | varrer; carregar; levar junto | envolver-se; participar}
  \seeref{juan4}
\end{EntryWithPhonetic}

\begin{EntryWithPhonetic}{卷}{juan4}{8}{⼙}[HSK 4]
  \definition{clas.}{usado para capítulos, seções ou volumes; fascículos}
  \definition[大,小]{s.}{livro; livros e pinturas que são enrolados para coleção; geralmente se refere a pinturas e caligrafia | papel de exame | arquivo; dossiê}
  \seeref{juan3}
\end{EntryWithPhonetic}

\begin{EntryWithPhonetic}{圈}{juan4}{11}{⼞}
  \definition*{s.}{Sobrenome Juan}
  \definition{s.}{curral; local onde o gado ou as aves são mantidos, geralmente cercado ou murado, alguns com galpões}
  \seeref{juan1}
  \seeref{quan1}
\end{EntryWithPhonetic}

\begin{EntryWithPhonetic}{决}{jue2}{6}{⼎}
  \definition{v.}{decidir; determinar | executar uma pessoa | (de um dique, etc.) romper; desabar}
\end{EntryWithPhonetic}

\begin{EntryWithPhonetic}{决不}{jue2 bu4}{6,4}{⼎、⼀}[HSK 5]
  \definition{adv.}{em hipótese alguma; nunca | definitivamente não; certamente não; sob nenhuma circunstância; de forma alguma}
\end{EntryWithPhonetic}

\begin{EntryWithPhonetic}{决策}{jue2ce4}{6,12}{⼎、⽵}[HSK 6]
  \definition{s.}{decisão política; decisão de importância estratégica; estratégia ou método de decisão}
  \definition{v.}{formular políticas; tomar uma decisão estratégica; decidir sobre uma estratégia ou abordagem}
\end{EntryWithPhonetic}

\begin{EntryWithPhonetic}{决定}{jue2ding4}{6,8}{⼎、⼧}[HSK 3]
  \definition{adj.}{decisivo; as leis objetivas levam as coisas a se desenvolverem e mudarem em determinada direção}
  \definition[项,个]{s.}{decisão; resolução; assuntos decididos}
  \definition{v.}{decidir; determinar; algo se torna a base ou o pré-requisito para outra coisa; desempenha um papel dominante | decidir; resolver; tomar uma decisão; propor uma forma de agir}
\end{EntryWithPhonetic}

\begin{EntryWithPhonetic}{决赛}{jue2sai4}{6,14}{⼎、⾙}[HSK 3]
  \definition[场]{s.}{finais (de uma competição); em competições esportivas, a última partida ou rodada disputada para determinar a classificação}
\end{EntryWithPhonetic}

\begin{EntryWithPhonetic}{决心}{jue2xin1}{6,4}{⼎、⼼}[HSK 3]
  \definition{s.}{resolução; determinação; determinação inabalável}
  \definition{v.}{secidir-se; decidir fazer algo e não vacilar nem mudar de ideia}
\end{EntryWithPhonetic}

\begin{EntryWithPhonetic}{角}{jue2}{7}{⾓}[Kangxi 148]
  \definition*{s.}{Sobrenome Jue}
  \definition[个,只,对]{s.}{papel; parte; personagem | tipo de papel (no drama tradicional chinês); categorias de divisão profissional do trabalho entre atores de ópera | ator ou atriz | uma antiga taça de vinho com três pernas | uma nota da antiga escala chinesa de cinco tons, correspondente a 3 na notação musical numerada}
  \definition{v.}{competir; contender; lutar}
  \seeref{jiao3}
\end{EntryWithPhonetic}

\begin{EntryWithPhonetic}{角色}{jue2se4}{7,6}{⾓、⾊}[HSK 4]
  \definition{s.}{papel; personagem em uma peça; personagem representado por um ator | papel; função; parte; uma metáfora para um certo tipo de pessoas na vida social}
\end{EntryWithPhonetic}

\begin{EntryWithPhonetic}{绝}{jue2}{9}{⽷}[HSK 6]
  \definition{adj.}{exausto; esgotado; acabado | desesperado; sem esperança | único; soberbo; incomparável | não deixar margem de manobra; não fazer concessões; intransigente}
  \definition{adv.}{extremamente; mais | (antes de uma negativa) absolutamente; no mínimo; por qualquer meio; em qualquer conta}
  \definition{s.}{(literário) jueju, um poema de quatro linhas}
  \definition{v.}{cortar; romper | parar de respirar; morrer}
\end{EntryWithPhonetic}

\begin{EntryWithPhonetic}{绝版}{jue2ban3}{9,8}{⽷、⽚}
  \definition{adj.}{esgotado | fora de catálogo}
\end{EntryWithPhonetic}

\begin{EntryWithPhonetic}{绝不}{jue2bu4}{9,4}{⽷、⼀}
  \definition{adv.}{definitivamente não | de forma alguma | sob nenhuma circunstância}
\end{EntryWithPhonetic}

\begin{EntryWithPhonetic}{绝大多数}{jue2 da4 duo1 shu4}{9,3,6,13}{⽷、⼤、⼣、⽁}[HSK 6]
  \definition{expr.}{maioria absoluta | uma maioria esmagadora}
\end{EntryWithPhonetic}

\begin{EntryWithPhonetic}{绝对}{jue2dui4}{9,5}{⽷、⼨}[HSK 3]
  \definition{adj.}{absoluto; sem condições; sem restrições | absoluto; extremo; incompleto; sem margem para negociação ou alteração}
  \definition{adv.}{absolutamente; completamente; com certeza}
\end{EntryWithPhonetic}

\begin{EntryWithPhonetic}{绝望}{jue2/wang4}{9,11}{⽷、⽉}[HSK 5]
  \definition{v.+compl.}{desesperar; desistir de toda esperança; perder toda esperança de}
\end{EntryWithPhonetic}

\begin{EntryWithPhonetic}{绝招}{jue2zhao1}{9,8}{⽷、⼿}
  \definition{s.}{habilidade única | movimento delicado inesperado (como último recurso) | golpe de mestre | golpe final}
\end{EntryWithPhonetic}

\begin{EntryWithPhonetic}{觉}{jue2}{9}{⾒}
  \definition{s.}{sentimento; senso; percepção e discriminação de estímulos externos}
  \definition{v.}{sentir; perceber | acordar | tornar-se consciente; tornar-se desperto; despertar; entender}
  \seeref{jiao4}
\end{EntryWithPhonetic}

\begin{EntryWithPhonetic}{觉得}{jue2de5}{9,11}{⾒、⼻}[HSK 1]
  \definition{v.}{sentir; estar ciente; pressentir; causar uma sensação | pensar; sentir; encontrar; considerar (tom menos assertivo)}
\end{EntryWithPhonetic}

\begin{EntryWithPhonetic}{觉悟}{jue2wu4}{9,10}{⾒、⼼}[HSK 6]
  \definition{s.}{consciência; percepção; compreensão; nível de consciência}
  \definition{v.}{vir a compreender; tornar-se consciente de; tornar-se politicamente desperto; despertar}
\end{EntryWithPhonetic}

\begin{EntryWithPhonetic}{脚}{jue2}{11}{⾁}
  \variantof{角}
\end{EntryWithPhonetic}

\begin{EntryWithPhonetic}{军}{jun1}{6}{⼍}
  \definition*{s.}{Sobrenome Jun}
  \definition{s.}{forças armadas; exército; tropas | exército; contingente; muitas pessoas participando de uma atividade | exército; unidades militares}
\end{EntryWithPhonetic}

\begin{EntryWithPhonetic}{军队}{jun1dui4}{6,4}{⼍、⾩}[HSK 6]
  \definition[支,个]{s.}{forças armadas; exército; tropas}
\end{EntryWithPhonetic}

\begin{EntryWithPhonetic}{军舰}{jun1 jian4}{6,10}{⼍、⾈}[HSK 6]
  \definition[艘,只]{s.}{navio de guerra; embarcação naval | \emph{warcraft}; um termo geral para embarcações militares equipadas com armas e equipamentos que podem executar missões de combate, incluindo principalmente navios de guerra, cruzadores, contratorpedeiros, porta-aviões, submarinos, torpedeiros, etc.}
\end{EntryWithPhonetic}

\begin{EntryWithPhonetic}{军人}{jun1 ren2}{6,2}{⼍、⼈}[HSK 5]
  \definition[名,位,个]{s.}{soldado; militar; pessoal militar; pessoas com status militar; pessoas servindo nas forças armadas}
\end{EntryWithPhonetic}

\begin{EntryWithPhonetic}{军事}{jun1shi4}{6,8}{⼍、⼅}[HSK 6]
  \definition{s.}{militar; assuntos militares; assuntos relativos aos militares e à guerra}
\end{EntryWithPhonetic}

\begin{EntryWithPhonetic}{军装}{jun1zhuang1}{6,12}{⼍、⾐}
  \definition{s.}{uniforme militar}
\end{EntryWithPhonetic}

\begin{EntryWithPhonetic}{君}{jun1}{7}{⼝}
  \definition*{s.}{Sobrenome Jun}
  \definition[个,位,名,些]{s.}{monarca; soberano; governante supremo | (como título) Senhor; Sr. | (literário) (em trato direto) você; senhor | cavalheiro | governante}
\end{EntryWithPhonetic}

\begin{EntryWithPhonetic}{君主立宪制}{jun1zhu3li4xian4zhi4}{7,5,5,9,8}{⼝、⼂、⽴、⼧、⼑}
  \definition{s.}{monarquia constitucional}
\end{EntryWithPhonetic}

%%%%% EOF %%%%%


 %%%
%%% K
%%%

\section*{K}\addcontentsline{toc}{section}{K}

\begin{EntryWithPhonetic}{咖}{ka1}{8}{⼝}
  \definition[杯]{s.}{classe | café | graduação}
\end{EntryWithPhonetic}

\begin{EntryWithPhonetic}{咖啡}{ka1fei1}{8,11}{⼝、⼝}[HSK 3]
  \definition[杯,瓶,罐,壶,包,袋,盒]{s.}{(empréstimo linguístico) café}
\end{EntryWithPhonetic}

\begin{EntryWithPhonetic}{咖啡馆}{ka1fei1guan3}{8,11,11}{⼝、⼝、⾷}
  \definition[家]{s.}{cafeteria}
\end{EntryWithPhonetic}

\begin{EntryWithPhonetic}{咖啡色}{ka1fei1 se4}{8,11,6}{⼝、⼝、⾊}
  \definition{s.}{cor café}
\end{EntryWithPhonetic}

\begin{EntryWithPhonetic}{卡}{ka3}{5}{⼘}[HSK 2]
  \definition{clas.}{calorias (cal)}
  \definition[张,片]{s.}{cartão; documento semelhante a um cartão | cassete; dispositivo tipo compartimento para colocar fitas cassete no gravador | caminhão}
  \seeref{qia3}
\end{EntryWithPhonetic}

\begin{EntryWithPhonetic}{卡车司机}{ka3che1 si1ji1}{5,4,5,6}{⼘、⾞、⼝、⽊}
  \definition{s.}{motorista de caminhão}
\end{EntryWithPhonetic}

\begin{EntryWithPhonetic}{卡片}{ka3pian4}{5,4}{⼘、⽚}
  \definition{s.}{cartão}
\end{EntryWithPhonetic}

\begin{EntryWithPhonetic}{卡片游戏}{ka3pian4 you2xi4}{5,4,12,6}{⼘、⽚、⽔、⼽}
  \definition{s.}{carta de baralho}
\end{EntryWithPhonetic}

\begin{EntryWithPhonetic}{卡通}{ka3tong1}{5,10}{⼘、⾡}
  \definition{s.}{(empréstimo linguístico) \emph{cartoon}}
\end{EntryWithPhonetic}

\begin{EntryWithPhonetic}{开}{kai1}{4}{⼶}[HSK 1]
  \definition*{s.}{Sobrenome Kai}
  \definition{clas.}{divisão do papel de impressão de tamanho padrão (uma parte da folha inteira) | quilate; unidade de cálculo da quantidade de ouro puro contida no ouro}
  \definition{s.}{porcentagem; percentual}
  \definition{v.}{abrir; estar ligado; ligar | recuperar; abrir; fazer uma abertura; escavar; abrir caminho; desbravar | abrir para fora; soltar-se | descongelar (rios); tornar-se navegável | levantar; libertar | iniciar; operar; manobrar | mover; estabelecer | executar; configurar | começar; iniciar | manter | escrever; fazer uma lista de | pagamento (salários, tarifas, etc.) | ferver}
  \definition{v.aux.}{usado após um verbo, indica ampliação ou expansão | usado após um verbo, indica o início e a continuidade}
  \seealsoref{开尔文}{kai1'er3wen2}
\end{EntryWithPhonetic}

\begin{EntryWithPhonetic}{开车}{kai1/che1}{4,4}{⼶、⾞}[HSK 1]
  \definition{v.+compl.}{dirigir um carro, trem, etc. | colocar uma máquina em funcionamento | (de um trem, etc.) partida | dirigir veículos motorizados}
\end{EntryWithPhonetic}

\begin{EntryWithPhonetic}{开创}{kai1 chuang4}{4,6}{⼶、⼑}[HSK 6]
  \definition{v.}{começar; iniciar; fundar; ser pioneiro; estabelecer; criar}
\end{EntryWithPhonetic}

\begin{EntryWithPhonetic}{开尔文}{kai1'er3wen2}{4,5,4}{⼶、⼩、⽂}
  \definition{s.}{Kelvin, temperatura absoluta | K, escala de temperatura}
\end{EntryWithPhonetic}

\begin{EntryWithPhonetic}{开发}{kai1fa1}{4,5}{⼶、⼜}[HSK 3]
  \definition{v.}{explorar; trabalhar com recursos naturais como terras baldias, minas, florestas e energia hidráulica para fins de aproveitamento | tornar acessível; descobrir ou explorar talentos, tecnologias, etc. para aproveitamento}
\end{EntryWithPhonetic}

\begin{EntryWithPhonetic}{开发区}{kai1fa1qu1}{4,5,4}{⼶、⼜、⼖}
  \definition{s.}{zona de desenvolvimento}
\end{EntryWithPhonetic}

\begin{EntryWithPhonetic}{开放}{kai1fang4}{4,8}{⼶、⽅}[HSK 3]
  \definition{adj.}{de mente aberta; sem restrições por convenções; pensamento e ambiente não conservadores, disposto a aceitar coisas novas e novas ideias; personalidade alegre}
  \definition{v.}{florescer | abrir (para o público); levantar bloqueios, proibições, restrições, etc. | diminuir uma proibição, restrição, etc. (de política); (economia) reduzir as restrições políticas, com justificativas específicas}
\end{EntryWithPhonetic}

\begin{EntryWithPhonetic}{开关}{kai1 guan1}{4,6}{⼶、⼋}[HSK 6]
  \definition[个,种,些]{s.}{interruptor; um dispositivo que conecta e desconecta o circuito de um dispositivo elétrico | registro; um dispositivo instalado em uma tubulação de fluido para controlar o fluxo}
\end{EntryWithPhonetic}

\begin{EntryWithPhonetic}{开花}{kai1hua1}{4,7}{⼶、⾋}[HSK 4]
  \definition{v.}{florescer; desabrochar; estar em flor; entrar em flor;  metáfora para um coração feliz ou um rosto sorridente | explodir; quebrar; dividir | sentir-se feliz ou sorrir alegremente | (experiência) espalhar-se; (empreendimento) surgir; surgir | (cabeça) ser ferido e sangrar profusamente}
\end{EntryWithPhonetic}

\begin{EntryWithPhonetic}{开会}{kai1/hui4}{4,6}{⼶、⼈}[HSK 1]
  \definition{v.+compl.}{realizar uma reunião; ter uma reunião; participar de uma reunião (conferência)}
\end{EntryWithPhonetic}

\begin{EntryWithPhonetic}{开机}{kai1 ji1}{4,6}{⼶、⽊}[HSK 2]
  \definition{v.}{começar a filmar um filme ou programa de TV; refere-se ao início das filmagens (de filmes, séries de TV, etc.) | ligar uma máquina}
\end{EntryWithPhonetic}

\begin{EntryWithPhonetic}{开口}{kai1kou3}{4,3}{⼶、⼝}
  \definition{v.}{abrir a boca de alguém | começar a falar}
\end{EntryWithPhonetic}

\begin{EntryWithPhonetic}{开幕}{kai1 mu4}{4,13}{⼶、⼱}[HSK 5]
  \definition{v.}{começar a apresentação; iniciar o espetáculo; levantar das cortinas | abrir; inaugurar; iniciar (uma conferência, exposição, etc.)}
\end{EntryWithPhonetic}

\begin{EntryWithPhonetic}{开幕式}{kai1mu4shi4}{4,13,6}{⼶、⼱、⼷}[HSK 5]
  \definition[场,次,届]{s.}{cerimônia de abertura; cerimônias e apresentações antes de eventos esportivos ou grandes eventos}
\end{EntryWithPhonetic}

\begin{EntryWithPhonetic}{开启}{kai1qi3}{4,7}{⼶、⼝}
  \definition{v.}{abrir | iniciar | (computação) ativar}
\end{EntryWithPhonetic}

\begin{EntryWithPhonetic}{开设}{kai1 she4}{4,6}{⼶、⾔}[HSK 6]
  \definition{v.}{montar; estabelecer; abrir (uma loja, fábrica, etc.); estabelecer novas instituições ou campos | oferecer (um curso na faculdade, etc.)}
\end{EntryWithPhonetic}

\begin{EntryWithPhonetic}{开始}{kai1shi3}{4,8}{⼶、⼥}[HSK 3]
  \definition[个]{s.}{começo; início; estágio inicial}
  \definition{v.}{começar; iniciar; começar a fazer algo}
\end{EntryWithPhonetic}

\begin{EntryWithPhonetic}{开水}{kai1shui3}{4,4}{⼶、⽔}[HSK 4]
  \definition[杯,瓶]{s.}{água fervida; água fervente}
\end{EntryWithPhonetic}

\begin{EntryWithPhonetic}{开锁}{kai1suo3}{4,12}{⼶、⾦}
  \definition{v.}{desbloquear | destravar}
\end{EntryWithPhonetic}

\begin{EntryWithPhonetic}{开通}{kai1 tong1}{4,10}{⼶、⾡}[HSK 6]
  \definition{v.}{limpar; dragar; remover obstáculos de; abrir o canal; desbloquear}
  \seeref{kai1 tong5}
\end{EntryWithPhonetic}

\begin{EntryWithPhonetic}{开通}{kai1 tong5}{4,10}{⼶、⾡}
  \definition{adj.}{liberal; mente aberta; mente moderna; mente liberal; sábio e sensato; não conservador ou teimoso}
  \seeref{kai1 tong1}
\end{EntryWithPhonetic}

\begin{EntryWithPhonetic}{开头}{kai1/tou2}{4,5}{⼶、⼤}[HSK 6]
  \definition{s.}{início; começo; o momento ou estágio do início; antecedente no tempo}
  \definition{v.+compl.}{começar, iniciar; a primeira ocorrência de um evento, ação, fenômeno, etc. | pôr-se a pé; começar}
\end{EntryWithPhonetic}

\begin{EntryWithPhonetic}{开玩笑}{kai1wan2xiao4}{4,8,10}{⼶、⽟、⽵}[HSK 1]
  \definition{v.}{fazer (ou brincar, fazer) uma piada; gracejar; zombar de; provocar; fazer uma brincadeira; zombar de alguém | tratar casualmente; dar pouca importância a; considerar como um assunto insignificante; insignificante | fazer uma brincadeira; pregar uma peça; brincar; em tom de brincadeira}
\end{EntryWithPhonetic}

\begin{EntryWithPhonetic}{开心}{kai1xin1}{4,4}{⼶、⼼}[HSK 2]
  \definition{adj.}{feliz; alegre; exultante; encantado}
  \definition{v.}{provocar; brincar; tirar sarro de alguém; zombar; divertir-se}
\end{EntryWithPhonetic}

\begin{EntryWithPhonetic}{开学}{kai1 xue2}{4,8}{⼶、⼦}[HSK 2]
  \definition{v.}{iniciar as aulas; iniciar o semestre; começar as aulas}
\end{EntryWithPhonetic}

\begin{EntryWithPhonetic}{开业}{kai1 ye4}{4,5}{⼶、⼀}[HSK 3]
  \definition[场]{v.}{iniciar um negócio; abrir para negócios | abrir um consultório particular}
\end{EntryWithPhonetic}

\begin{EntryWithPhonetic}{开夜车}{kai1/ye4che1}{4,8,4}{⼶、⼣、⾞}[HSK 6]
  \definition{v.+compl.}{trabalhar até tarde da noite; ficar acordado até tarde da noite estudando ou trabalhando para cumprir prazos | (literalmente) ``conduzir carro à noite''}
\end{EntryWithPhonetic}

\begin{EntryWithPhonetic}{开展}{kai1zhan3}{4,10}{⼶、⼫}[HSK 3]
  \definition{v.}{lançar; desenvolver | abrir; inaugurar}
\end{EntryWithPhonetic}

\begin{EntryWithPhonetic}{看}{kan1}{9}{⽬}
  \definition{v.}{cuidar de; tomar conta de; cuidar de; proteger | manter sob vigilância}
  \seeref{kan4}
\end{EntryWithPhonetic}

\begin{EntryWithPhonetic}{看管}{kan1 guan3}{9,14}{⽬、⽵}[HSK 6]
  \definition{v.}{cuidar; atender | guardar; vigiar; ficar de olho em | assumir o comando; estar no comando}
\end{EntryWithPhonetic}

\begin{EntryWithPhonetic}{砍}{kan3}{9}{⽯}
  \definition{v.}{cortar}
\end{EntryWithPhonetic}

\begin{EntryWithPhonetic}{砍刀}{kan3dao1}{9,2}{⽯、⼑}
  \definition{s.}{facão | machete}
\end{EntryWithPhonetic}

\begin{EntryWithPhonetic}{砍掉}{kan3diao4}{9,11}{⽯、⼿}
  \definition{v.}{amputar}
\end{EntryWithPhonetic}

\begin{EntryWithPhonetic}{砍断}{kan3duan4}{9,11}{⽯、⽄}
  \definition{v.}{cortar}
\end{EntryWithPhonetic}

\begin{EntryWithPhonetic}{砍价}{kan3jia4}{9,6}{⽯、⼈}
  \definition{v.}{barganhar | cortar ou derrubar um preço}
\end{EntryWithPhonetic}

\begin{EntryWithPhonetic}{砍杀}{kan3sha1}{9,6}{⽯、⽊}
  \definition{v.}{atacar com arma branca}
\end{EntryWithPhonetic}

\begin{EntryWithPhonetic}{砍伤}{kan3shang1}{9,6}{⽯、⼈}
  \definition{v.}{ferir com lâmina ou machado}
\end{EntryWithPhonetic}

\begin{EntryWithPhonetic}{砍树}{kan3shu4}{9,9}{⽯、⽊}
  \definition{v.}{derrubar árvores}
\end{EntryWithPhonetic}

\begin{EntryWithPhonetic}{砍死}{kan3si3}{9,6}{⽯、⽍}
  \definition{v.}{matar com um machado}
\end{EntryWithPhonetic}

\begin{EntryWithPhonetic}{砍头}{kan3tou2}{9,5}{⽯、⼤}
  \definition{v.}{decapitar}
\end{EntryWithPhonetic}

\begin{EntryWithPhonetic}{看}{kan4}{9}{⽬}[HSK 1,6]
  \definition{interj.}{Cuidado! (para um perigo)}
  \definition{part.}{tentar, usado depois de outros verbos}
  \definition{v.}{ver; olhar para; observar; fazer contato visual com pessoas ou objetos | pensar; considerar; observar; julgar; observar e analisar | visitar; ver; fazer uma visita | olhar para; considerar; tratar | tratar (um paciente ou uma doença) | cuidar | ficar atento; ficar de olho | depender de; ser dependente de | ler}
  \seeref{kan1}
\end{EntryWithPhonetic}

\begin{EntryWithPhonetic}{看病}{kan4/bing4}{9,10}{⽬、⽧}[HSK 1]
  \definition{v.+compl.}{(de um médico) ver um paciente | (de um paciente) ver (consultar) um médico}
\end{EntryWithPhonetic}

\begin{EntryWithPhonetic}{看不起}{kan4bu5qi3}{9,4,10}{⽬、⼀、⾛}[HSK 4]
  \definition{v.}{desprezar; desdenhar; menosprezar; ter desprezo; olhar de cima para baixo}
\end{EntryWithPhonetic}

\begin{EntryWithPhonetic}{看成}{kan4 cheng2}{9,6}{⽬、⼽}[HSK 5]
  \definition{v.}{ser capaz de ver ou assistir | tomar como; olhar como; considerar como | tratar como; considerar como; pensar como; ter como}
\end{EntryWithPhonetic}

\begin{EntryWithPhonetic}{看出}{kan4 chu1}{9,5}{⽬、⼐}[HSK 5]
  \definition{v.}{decifrar; ver; sondar; encontrar; discernir; perceber | descobrir; estar ciente de}
\end{EntryWithPhonetic}

\begin{EntryWithPhonetic}{看待}{kan4dai4}{9,9}{⽬、⼻}[HSK 5]
  \definition{v.}{tratar; considerar; olhar com atenção; ter uma certa atitude ou visão em relação a alguém ou alguma coisa}
\end{EntryWithPhonetic}

\begin{EntryWithPhonetic}{看淡}{kan4dan4}{9,11}{⽬、⽔}
  \definition{v.}{considerar sem importância | ser indiferente a (fama, riqueza, etc.) | (de uma economia ou mercado) enfraquecer, ficar mais lento, diminuir a velocidade}
\end{EntryWithPhonetic}

\begin{EntryWithPhonetic}{看到}{kan4 dao4}{9,8}{⽬、⼑}[HSK 1]
  \definition{v.}{ver; avistar}
\end{EntryWithPhonetic}

\begin{EntryWithPhonetic}{看得见}{kan4 de5 jian4}{9,11,4}{⽬、⼻、⾒}[HSK 6]
  \definition{adj.}{perceptível; visível; tangível}
\end{EntryWithPhonetic}

\begin{EntryWithPhonetic}{看得起}{kan4 de5 qi3}{9,11,10}{⽬、⼻、⾛}[HSK 6]
  \definition{v.}{ter uma boa opinião sobre; pensar muito (ou muito) sobre}
\end{EntryWithPhonetic}

\begin{EntryWithPhonetic}{看法}{kan4fa3}{9,8}{⽬、⽔}[HSK 2]
  \definition[个,种,点]{s.}{opinião; perspectiva; (ponto de) vista; uma maneira de ver uma coisa | opinião desfavorável (ou crítica) sobre alguém}
\end{EntryWithPhonetic}

\begin{EntryWithPhonetic}{看好}{kan4 hao3}{9,6}{⽬、⼥}[HSK 6]
  \definition{v.}{elogiar; apreciar; encorajar; acreditar que pessoas ou coisas terão uma boa tendência | estar prestes a surgir uma boa tendência}
\end{EntryWithPhonetic}

\begin{EntryWithPhonetic}{看见}{kan4jian4}{9,4}{⽬、⾒}[HSK 1]
  \definition{v.}{ver; avistar; ao olhar, descobrir alguém ou algo}
\end{EntryWithPhonetic}

\begin{EntryWithPhonetic}{看来}{kan4 lai2}{9,7}{⽬、⽊}[HSK 4]
  \definition{adv.}{parecer; parecer como se (ou embora); refere-se a um julgamento aproximado; expressa um julgamento por observação}
  \definition{v.}{ser considerado; na visão de alguém; na opinião de alguém; expressar a ideia aproximada que o locutor tem da situação}
\end{EntryWithPhonetic}

\begin{EntryWithPhonetic}{看起来}{kan4 qi3 lai5}{9,10,7}{⽬、⾛、⽊}[HSK 3]
  \definition{v.}{parecer; aparentar; dar a impressão de (ou como se)}
\end{EntryWithPhonetic}

\begin{EntryWithPhonetic}{看上去}{kan4 shang4 qu4}{9,3,5}{⽬、⼀、⼛}[HSK 3]
  \definition{adv.}{parece que}
\end{EntryWithPhonetic}

\begin{EntryWithPhonetic}{看望}{kan4wang4}{9,11}{⽬、⽉}[HSK 4]
  \definition{v.}{ver; visitar; ligar; dar uma olhada; ir até os pais, idosos, professores ou amigos para cumprimentá-los}
\end{EntryWithPhonetic}

\begin{EntryWithPhonetic}{看作}{kan4 zuo4}{9,7}{⽬、⼈}[HSK 6]
  \definition{v.}{considerar como; olhar como}
\end{EntryWithPhonetic}

\begin{EntryWithPhonetic}{康}{kang1}{11}{⼴}
  \definition*{s.}{Sobrenome Kang}
  \definition{adj.}{saudável |  fácil; pacífico; abundante | amplo; largo | Dialeto: de baixa qualidade; inferior}
  \definition{s.}{bem-estar; saúde | palha; farelo; casca}
  \definition{v.}{(normalmente de um rabanete) tornar-se esponjoso}
\end{EntryWithPhonetic}

\begin{EntryWithPhonetic}{康复}{kang1 fu4}{11,9}{⼴、⼢}[HSK 6]
  \definition{v.}{Saúde: estaurar; recuperar; reabilitar}
\end{EntryWithPhonetic}

\begin{EntryWithPhonetic}{扛}{kang2}{6}{⼿}
  \definition{v.}{carregar objetos nos ombros |  suportar; aguentar | lidar; assumir}
  \seeref{gang1}
\end{EntryWithPhonetic}

\begin{EntryWithPhonetic}{抗}{kang4}{7}{⼿}[HSK 6]
  \definition*{s.}{Sobrenome Kang}
  \definition{pref.}{anti-}
  \definition{v.}{resistir; combater; lutar | recusar; desafiar}
\end{EntryWithPhonetic}

\begin{EntryWithPhonetic}{抗议}{kang4yi4}{7,5}{⼿、⾔}[HSK 6]
  \definition{v.}{protestar; reconsiderar; levantar objeções fortes}
\end{EntryWithPhonetic}

\begin{EntryWithPhonetic}{考}{kao3}{6}{⽼}[HSK 1]
  \definition*{s.}{Sobrenome Kao}
  \definition{adj.}{antigo; velho; com idade avançada}
  \definition{s.}{o pai falecido de alguém}
  \definition{v.}{examinar; dar (fazer) um exame, teste ou questionário | verificar; inspecionar | estudar; verificar; investigar | perguntar; testar; fazer perguntas para que o outro responda, a fim de testar suas habilidades em determinada área}
\end{EntryWithPhonetic}

\begin{EntryWithPhonetic}{考察}{kao3cha2}{6,14}{⽼、⼧}[HSK 4]
  \definition{v.}{inspecionar; investigar; observar e estudar}
\end{EntryWithPhonetic}

\begin{EntryWithPhonetic}{考场}{kao3 chang3}{6,6}{⽼、⼟}[HSK 6]
  \definition{s.}{sala de exames}
\end{EntryWithPhonetic}

\begin{EntryWithPhonetic}{考核}{kao3he2}{6,10}{⽼、⽊}[HSK 5]
  \definition{v.}{examinar; checar; avaliar; avaliar (a proficiência de alguém)}
\end{EntryWithPhonetic}

\begin{EntryWithPhonetic}{考虑}{kao3lv4}{6,10}{⽼、⾌}[HSK 4]
  \definition{v.}{considerar; refletir sobre; levar em conta}
\end{EntryWithPhonetic}

\begin{EntryWithPhonetic}{考生}{kao3 sheng1}{6,5}{⽼、⽣}[HSK 2]
  \definition{s.}{candidato a exame; alunos inscritos para o exame de admissão}
\end{EntryWithPhonetic}

\begin{EntryWithPhonetic}{考试}{kao3/shi4}{6,8}{⽼、⾔}[HSK 1]
  \definition[次]{s.}{teste; exame; prova; atividades realizadas para verificar conhecimentos ou habilidades}
  \definition{v.+compl.}{testar; avaliar; avaliar conhecimentos e habilidades por meio de perguntas escritas ou orais.}
\end{EntryWithPhonetic}

\begin{EntryWithPhonetic}{考题}{kao3 ti2}{6,15}{⽼、⾴}[HSK 6]
  \definition{s.}{questões de exame; prova de exame; tópicos de exame}
\end{EntryWithPhonetic}

\begin{EntryWithPhonetic}{考验}{kao3yan4}{6,10}{⽼、⾺}[HSK 3]
  \definition[场,个,种]{s.}{teste; julgamento; atividade realizada para verificar se as habilidades, ideias, moral e qualidades de uma pessoa atendem aos requisitos}
  \definition{v.}{testar; testar as capacidades, ideias, moral e qualidades de uma pessoa através de situações, ações ou ambientes difíceis, para verificar se elas atendem aos requisitos}
\end{EntryWithPhonetic}

\begin{EntryWithPhonetic}{烤}{kao3}{10}{⽕}
  \definition{v.}{assar | grelhar}
\end{EntryWithPhonetic}

\begin{EntryWithPhonetic}{烤肉}{kao3 rou4}{10,6}{⽕、⾁}[HSK 5]
  \definition[块,串,片,盘]{s.}{churrasco (literalmente carne assada)}
\end{EntryWithPhonetic}

\begin{EntryWithPhonetic}{烤鸭}{kao3ya1}{10,10}{⽕、⿃}[HSK 5]
  \definition[只,盘]{s.}{pato assado; pato recheado e assado em um forno especial após ser abatido}
\end{EntryWithPhonetic}

\begin{EntryWithPhonetic}{靠}{kao4}{15}{⾮}[HSK 2]
  \definition{prep.}{manter (em); aproximar-se (de); ao longo de | por; graças a; com base em; de acordo com}
  \definition{s.}{armadura de palco (feita de seda bordada); armadura usada pelos generais militares antigos nas peças teatrais}
  \definition{v.}{inclinar-se; sentado ou em pé, deixar parte do peso do corpo ser suportado por outra pessoa ou objeto (pessoa) | encostar-se (em); apoiar-se ou levantar-se com a ajuda de alguma coisa | aproximar-se; estar perto de | confiar em; depender de | confiar}
\end{EntryWithPhonetic}

\begin{EntryWithPhonetic}{靠近}{kao4 jin4}{15,7}{⾮、⾡}[HSK 5]
  \definition{adv.}{próximo; perto de; ao lado de}
  \definition{v.}{aproximar-se; chegar perto; avançar em direção a um determinado objetivo de modo que a distância fique cada vez menor}
\end{EntryWithPhonetic}

\begin{EntryWithPhonetic}{科}{ke1}{9}{⽲}[HSK 2]
  \definition*{s.}{Sobrenome Ke}
  \definition{s.}{um ramo de estudo acadêmico ou profissional |uma divisão ou subdivisão de uma unidade administrativa | família | instruções de palco no drama chinês clássico; nos roteiros de peças clássicas, termos usados para indicar as ações dos personagens | nível; classificação; categoria | sessão de exames; refere-se às disciplinas, notas e anos das provas para a seleção de candidatos a cargos públicos militares e civis na antiguidade | tecnológico | assunto | lei; regulamento; decreto | penalidade; pena; punição | treinamento profissional ou formal; curso profissionalizante}
  \definition{v.}{proferir uma sentença (penal)}
\end{EntryWithPhonetic}

\begin{EntryWithPhonetic}{科技}{ke1 ji4}{9,7}{⽲、⼿}[HSK 3]
  \definition{s.}{ciência e tecnologia}
\end{EntryWithPhonetic}

\begin{EntryWithPhonetic}{科学}{ke1xue2}{9,8}{⽲、⼦}[HSK 2]
  \definition{adj.}{científico; em conformidade com as leis da ciência}
  \definition[门,个,种]{s.}{ciência; um conjunto de conhecimentos que reflete as leis objetivas da natureza, da sociedade, do pensamento, etc.}
\end{EntryWithPhonetic}

\begin{EntryWithPhonetic}{科学家}{ke1xue2jia1}{9,8,10}{⽲、⼦、⼧}
  \definition[位,名,个]{s.}{cientista; pessoas com realizações significativas no campo da pesquisa científica}
\end{EntryWithPhonetic}

\begin{EntryWithPhonetic}{科研}{ke1 yan2}{9,9}{⽲、⽯}[HSK 6]
  \definition{s.}{pesquisa científica}
  \definition{v.}{envolver-se em pesquisa científica}
\end{EntryWithPhonetic}

\begin{EntryWithPhonetic}{棵}{ke1}{12}{⽊}[HSK 4]
  \definition{clas.}{usado para plantas, árvores}
\end{EntryWithPhonetic}

\begin{EntryWithPhonetic}{颗}{ke1}{14}{⾴}[HSK 5]
  \definition{clas.}{usado para grãos, pérolas, dentes, corações, satelites, pequenas esferas, etc.}
  \definition{s.}{grão; partícula; pequenas coisas redondas}
\end{EntryWithPhonetic}

\begin{EntryWithPhonetic}{蝌}{ke1}{15}{⾍}
  \definition[只]{s.}{girino}
\end{EntryWithPhonetic}

\begin{EntryWithPhonetic}{蝌蚪}{ke1dou3}{15,10}{⾍、⾍}
  \definition{s.}{girino}
\end{EntryWithPhonetic}

\begin{EntryWithPhonetic}{壳}{ke2}{7}{⼠}
  \definition[层,个]{s.}{casca, concha; significa o mesmo que 壳 | concha; revestimento externo | empresa de fachada}
  \seeref{qiao4}
\end{EntryWithPhonetic}

\begin{EntryWithPhonetic}{咳}{ke2}{9}{⼝}[HSK 5]
  \definition{v.}{tossir}
  \seeref{hai1}
\end{EntryWithPhonetic}

\begin{EntryWithPhonetic}{咳嗽}{ke2sou5}{9,14}{⼝、⼝}
  \definition{v.}{ter tosse | tossir}
\end{EntryWithPhonetic}

\begin{EntryWithPhonetic}{可}{ke3}{5}{⼝}[HSK 5]
  \definition*{s.}{Sobrenome Ke}
  \definition{adv.}{indica ênfase | indica o fortalecimento de perguntas retóricas | indica um tom de questionamento mais forte | sobre; a respeito de}
  \definition{conj.}{mas; ainda}
  \definition{v.}{aprovar; concordar com | poder; permitir; ser capaz de | precisar (fazer); valer a pena (fazer); merecer | ajustar; adequar | estar pronto para; estar disposto a; pretender}
  \seeref{ke4}
\end{EntryWithPhonetic}

\begin{EntryWithPhonetic}{可爱}{ke3'ai4}{5,10}{⼝、⽖}[HSK 2]
  \definition{adj.}{adorável; simpático; encantador | bonitinho; adorável | amado; querido; encantador; cativante; relacionamento próximo, sentimentos profundos | fofo; bonito}
\end{EntryWithPhonetic}

\begin{EntryWithPhonetic}{可编程}{ke3bian1cheng2}{5,12,12}{⼝、⽷、⽲}
  \definition{adj.}{programável}
\end{EntryWithPhonetic}

\begin{EntryWithPhonetic*}{可擦写可编程只读存储器}{ke3 ca1 xie3 ke3 bian1cheng2 zhi1 du2 cun2chu3qi4}{5,17,5,5,12,12,5,10,6,12,16}{⼝、⼿、⼍、⼝、⽷、⽲、⼝、⾔、⼦、⼈、⼝}
  \definition{s.}{EPROM (\emph{erasable programmable read-only memory})}
\end{EntryWithPhonetic*}

\begin{EntryWithPhonetic}{可见}{ke3jian4}{5,4}{⼝、⾒}[HSK 4]
  \definition{adj.}{visível; concebível; algo que é óbvio ou evidente}
  \definition{conj.}{isso mostra; isto prova; é, portanto, claro (ou evidente, óbvio) que}
  \definition{v.}{ser ou estar visível ; ser ou estar claro}
\end{EntryWithPhonetic}

\begin{EntryWithPhonetic}{可卡因}{ke3ka3yin1}{5,5,6}{⼝、⼘、⼞}
  \definition{s.}{(empréstimo linguístico) cocaína}
\end{EntryWithPhonetic}

\begin{EntryWithPhonetic}{可靠}{ke3kao4}{5,15}{⼝、⾮}[HSK 3]
  \definition{adj.}{confiável; digno de confiança | verdadeiro; autêntico; descrever notícias e outras informações como verdadeiras, de modo que as pessoas possam acreditar nelas}
\end{EntryWithPhonetic}

\begin{EntryWithPhonetic}{可口可乐}{ke3kou3ke3le4}{5,3,5,5}{⼝、⼝、⼝、⼃}
  \definition*{s.}{Empréstimo linguístico: Coca-Cola}
\end{EntryWithPhonetic}

\begin{EntryWithPhonetic}{可乐}{ke3 le4}{5,5}{⼝、⼃}[HSK 3]
  \definition*[罐,杯,瓶,听,口]{s.}{\emph{coke}; coca; coca-cola}
  \definition{adj.}{engraçado; divertido; risível}
\end{EntryWithPhonetic}

\begin{EntryWithPhonetic}{可怜}{ke3lian2}{5,8}{⼝、⼼}[HSK 5]
  \definition{adj.}{pobre; lamentável; lastimável | miserável (de quantidade ou qualidade); descreve um número pequeno ou um lugar tão pequeno que não vale a pena falar sobre ele}
  \definition{v.}{ter pena; ter piedade de; ter simpatia por pessoas que tiveram coisas muito ruins acontecendo com elas}
\end{EntryWithPhonetic}

\begin{EntryWithPhonetic}{可能}{ke3neng2}{5,10}{⼝、⾁}[HSK 2]
  \definition{adj.}{possível}
  \definition{adv.}{possivelmente}
  \definition[种]{s.}{possibilidade; tendências ou oportunidades que podem se tornar realidade}
\end{EntryWithPhonetic}

\begin{EntryWithPhonetic}{可怕}{ke3pa4}{5,8}{⼝、⼼}[HSK 2]
  \definition{adj.}{assustador; terrível; hediondo; medonho; horrível; aterrorizante}
  \definition{adv.}{terrivelmente}
\end{EntryWithPhonetic}

\begin{EntryWithPhonetic}{可是}{ke3shi4}{5,9}{⼝、⽇}[HSK 2]
  \definition{adv.}{de fato (usado para dar ênfase), equivalente a 的确}
  \definition{conj.}{mas; no entanto; contudo; conecta frases, expressa uma relação de transição, equivalente a 但是}
  \seealsoref{但是}{dan4 shi4}
  \seealsoref{的确}{di2que4}
\end{EntryWithPhonetic}

\begin{EntryWithPhonetic}{可惜}{ke3xi1}{5,11}{⼝、⼼}[HSK 5]
  \definition{adj.}{é uma pena; é muito ruim; é lamentável}
  \definition{adv.}{infelizmente}
\end{EntryWithPhonetic}

\begin{EntryWithPhonetic}{可以}{ke3yi3}{5,4}{⼝、⼈}[HSK 2]
  \definition{adj.}{aceitável; nada mal; muito bom | impressionante; espantoso; tremendo}
  \definition{v.}{poder; ter condições, capacidade e tempo para fazer algo ou ter alguma utilidade | permitir; poder | valer a pena fazer; considerar que vale a pena, recomendar fazer algo}
\end{EntryWithPhonetic}

\begin{EntryWithPhonetic}{渴}{ke3}{12}{⽔}[HSK 1]
  \definition{adj.}{sedento}
  \definition{adv.}{ansiosamente; metáfora de urgência}
  \definition{v.}{desejar; ansiar por}
\end{EntryWithPhonetic}

\begin{EntryWithPhonetic}{渴望}{ke3wang4}{12,11}{⽔、⽉}[HSK 5]
  \definition{v.}{aspirar; (ter sede, ansiar, desejar) por}
\end{EntryWithPhonetic}

\begin{EntryWithPhonetic}{可}{ke4}{5}{⼝}
  \definition{s.}{governante supremo de uma tribo nômade do norte; Khan (可汗), título do governante supremo dos antigos grupos étnicos xianbei, turco, uigur e mongol}
  \seeref{ke3}
  \seealsoref{可汗}{ke4han2}
\end{EntryWithPhonetic}

\begin{EntryWithPhonetic}{可汗}{ke4han2}{5,6}{⼝、⽔}
  \definition{s.}{khan (empréstimo linguístico); cham}
\end{EntryWithPhonetic}

\begin{EntryWithPhonetic}{克}{ke4}{7}{⼗}[HSK 2]
  \definition*{s.}{Sobrenome Ke}
  \definition{clas.}{g, grama, unidade de peso | unidade tibetana de volume ou medida seca (com capacidade para cerca de 25 斤, de cevada) | unidade tibetana de área de terra equivalente a cerca de 1 亩}
  \definition{v.}{poder; ser capaz de | tolerar; conter; restringir; suprimir| subjugar; capturar; conquistar (uma cidade, etc.) | digerir (alimentos) | reduzir; diminuir | definir um limite de tempo}
  \seealsoref{斤}{jin1}
  \seealsoref{亩}{mu3}
\end{EntryWithPhonetic}

\begin{EntryWithPhonetic}{克服}{ke4fu2}{7,8}{⼗、⽉}[HSK 3]
  \definition{v.}{sobrepujar; superar; conquistar; vencer com força de vontade e determinação (deficiências, erros, fenômenos negativos, condições desfavoráveis, etc.) | aguentar; suportar (dificuldades, inconveniências, etc.)}
\end{EntryWithPhonetic}

\begin{EntryWithPhonetic}{刻}{ke4}{8}{⼑}[HSK 2,5]
  \definition{adj.}{cruel; severo; rude; indelicado | no mais alto grau}
  \definition{clas.}{um quarto (de uma hora, 15min)}
  \definition[件]{s.}{quarto (de hora); momento}
  \definition{v.}{esculpir; inscrever; gravar; talhar com uma faca (padrões, texto, etc.) | definir um limite de tempo | imprimir (CD)}
\end{EntryWithPhonetic}

\begin{EntryWithPhonetic}{刻画}{ke4hua4}{8,8}{⼑、⽥}
  \definition{v.}{retratar | tirar um retrato}
\end{EntryWithPhonetic}

\begin{EntryWithPhonetic}{刻钟}{ke4 zhong1}{8,9}{⼑、⾦}
  \definition{s.}{um quarto de hora}
\end{EntryWithPhonetic}

\begin{EntryWithPhonetic}{客}{ke4}{9}{⼧}
  \definition*{s.}{Sobrenome Ke}
  \definition{adj.}{objetivo; independente da consciência humana | estrangeiro; não desta região, unidade ou indústria}
  \definition{clas.}{porção (de comida, bebida, etc.); em algumas áreas, é usado para vender alimentos e bebidas em porções}
  \definition[个,位,名,些]{s.}{convidado; visitante; aquele que é convidado; aquele que vem visitar (em oposição a 主) | viajante; passageiro | comerciante viajante; refere-se especificamente a comerciantes que transportam mercadorias de um lugar para o outro | cliente; patrono; consumidor | uma pessoa envolvida em alguma atividade específica; pessoas que viajam fazendo algum tipo de atividade}
  \definition{v.}{ser um estranho; estabelecer-se (ou viver) em um lugar estranho; estar longe de casa ou morar no exterior}
  \seealsoref{主}{zhu3}
\end{EntryWithPhonetic}

\begin{EntryWithPhonetic}{客车}{ke4 che1}{9,4}{⼧、⾞}[HSK 6]
  \definition[辆,列,次,趟]{s.}{ônibus; veículo de passageiros; veículos que transportam passageiros em ferrovias e estradas}
\end{EntryWithPhonetic}

\begin{EntryWithPhonetic}{客观}{ke4guan1}{9,6}{⼧、⾒}[HSK 3]
  \definition{adj.}{objetivo; justo e razoável; imparcial; com base na situação real, sem preconceitos pessoais}
  \definition{s.}{objetivo; existe fora da consciência, sem depender da consciência subjetiva}
\end{EntryWithPhonetic}

\begin{EntryWithPhonetic}{客户}{ke4hu4}{9,4}{⼧、⼾}[HSK 5]
  \definition[位,个,家,批]{s.}{cliente; consumidor}
\end{EntryWithPhonetic}

\begin{EntryWithPhonetic}{客气}{ke4qi5}{9,4}{⼧、⽓}[HSK 5]
  \definition{adj.}{educado; modesto; cortês}
  \definition{v.}{ser educado; ser cortês; fazer comentários educados ou agir educadamente}
\end{EntryWithPhonetic}

\begin{EntryWithPhonetic}{客人}{ke4ren2}{9,2}{⼧、⼈}[HSK 2]
  \definition[位,个,桌,拨,批]{s.}{visitante; convidado | cliente; passageiro; hóspede; viajante}
\end{EntryWithPhonetic}

\begin{EntryWithPhonetic}{客厅}{ke4ting1}{9,4}{⼧、⼚}[HSK 5]
  \definition[间,个]{s.}{sala de estar; sala de visitas; sala para receber convidados}
\end{EntryWithPhonetic}

\begin{EntryWithPhonetic}{课}{ke4}{10}{⾔}[HSK 1]
  \definition{clas.}{aula; lição; unidade de tempo de ensino; parágrafo do material didático}
  \definition[门,节]{s.}{classe; aula; ensino por etapas planejado | disciplina; curso | imposto; antiga referência a impostos | seção; departamentos de escritório criados no antigo governo}
  \definition{v.}{cobrar; impor; taxar}
\end{EntryWithPhonetic}

\begin{EntryWithPhonetic}{课本}{ke4 ben3}{10,5}{⾔、⽊}[HSK 1]
  \definition[本]{s.}{livro didático; livro-texto}
\end{EntryWithPhonetic}

\begin{EntryWithPhonetic}{课程}{ke4cheng2}{10,12}{⾔、⽲}[HSK 3]
  \definition[个,堂,节,门]{s.}{curso; currículo; as disciplinas e o programa letivo da escola}
\end{EntryWithPhonetic}

\begin{EntryWithPhonetic}{课堂}{ke4 tang2}{10,11}{⾔、⼟}[HSK 2]
  \definition[间,节,个]{s.}{sala de aula; local onde se realizam as aulas; local onde se realizam as atividades de ensino}
\end{EntryWithPhonetic}

\begin{EntryWithPhonetic}{课题}{ke4ti2}{10,15}{⾔、⾴}[HSK 5]
  \definition[组]{s.}{uma questão para estudo ou discussão; principais questões a serem pesquisadas ou discutidas, ou assuntos importantes que precisam ser resolvidos com urgência | tarefa; problema; questões a serem resolvidas}
\end{EntryWithPhonetic}

\begin{EntryWithPhonetic}{课文}{ke4 wen2}{10,4}{⾔、⽂}[HSK 1]
  \definition[篇,段]{s.}{texto (de uma lição); texto principal do livro didático (diferente das notas de rodapé, exercícios, etc.)}
\end{EntryWithPhonetic}

\begin{EntryWithPhonetic}{肯}{ken3}{8}{⾁}[HSK 6]
  \definition{s.}{carne presa ao osso}
  \definition{v.}{concordar; consentir}
  \definition{v.aux.}{estar disposto a; estar pronto para; para expressar vontade subjetiva; vontade de aceitar}
\end{EntryWithPhonetic}

\begin{EntryWithPhonetic}{肯定}{ken3ding4}{8,8}{⾁、⼧}[HSK 5]
  \definition{adj.}{certo; definitivo; positivo; afirmativo | positivo; afirmativo; aceitável}
  \definition{adv.}{certamente; definitivamente; sem dúvida; sem dúvida alguma}
  \definition{v.}{afirmar; aprovar; confirmar; considerar positivo; reconhecer a existência de algo ou sua autenticidade ou racionalidade (em oposição à 否定)}
  \seealsoref{否定}{fou3ding4}
\end{EntryWithPhonetic}

\begin{EntryWithPhonetic}{坑}{keng1}{7}{⼟}
  \definition[个]{s.}{poço; buraco; cavidade | poço; túnel; caverna subterrânea}
  \definition{v.}{enredar; enganar; trapacear | nos tempos antigos, significava enterrar as pessoas vivas}
\end{EntryWithPhonetic}

\begin{EntryWithPhonetic}{坑人}{keng1/ren2}{7,2}{⼟、⼈}
  \definition{v.+compl.}{trapacear alguém}
\end{EntryWithPhonetic}

\begin{EntryWithPhonetic}{空}{kong1}{8}{⽳}[HSK 3]
  \definition*{s.}{Sobrenome Kong}
  \definition{adj.}{vazio; oco; nulo; não inclui nada; não contém nada ou não tem conteúdo; irrealista}
  \definition{adv.}{por nada; em vão; sem efeito}
  \definition{s.}{céu; ar | vazio; vazio do mundo dos sentidos}
  \seeref{kong4}
\end{EntryWithPhonetic}

\begin{EntryWithPhonetic}{空间}{kong1jian1}{8,7}{⽳、⾨}[HSK 4]
  \definition[个]{s.}{espaço; recinto; cômodo; espaço em branco; interespaço}
\end{EntryWithPhonetic}

\begin{EntryWithPhonetic}{空间站}{kong1jian1zhan4}{8,7,10}{⽳、⾨、⽴}
  \definition{s.}{estação espacial}
\end{EntryWithPhonetic}

\begin{EntryWithPhonetic}{空姐}{kong1jie3}{8,8}{⽳、⼥}
  \definition[名,位,个]{s.}{aeromoça; comissária de bordo; abreviação de 空中小姐}
  \seealsoref{空中小姐}{kong1zhong1xiao3jie3}
\end{EntryWithPhonetic}

\begin{EntryWithPhonetic}{空军}{kong1 jun1}{8,6}{⽳、⼍}[HSK 6]
  \definition[名,位,个,支]{s.}{força aérea; um exército que luta no ar, geralmente composto por várias unidades de aviação e unidades terrestres da força aérea}
\end{EntryWithPhonetic}

\begin{EntryWithPhonetic}{空气}{kong1qi4}{8,4}{⽳、⽓}[HSK 2]
  \definition[缕,股,份,阵]{s.}{ar; gases que compõe a atmosfera terrestre | atmosfera}
\end{EntryWithPhonetic}

\begin{EntryWithPhonetic}{空调}{kong1tiao2}{8,10}{⽳、⾔}[HSK 3]
  \definition[台,个]{s.}{ar-condicionado;  condicionador de ar}
\end{EntryWithPhonetic}

\begin{EntryWithPhonetic}{空心菜}{kong1xin1cai4}{8,4,11}{⽳、⼼、⾋}
  \definition{s.}{espinafre aquático | \emph{ong choy} | repolho do pântano | convolvulus aquático | glória-da-manhã aquática}
  \seealsoref{蕹菜}{weng4cai4}
\end{EntryWithPhonetic}

\begin{EntryWithPhonetic}{空中}{kong1 zhong1}{8,4}{⽳、⼁}[HSK 5]
  \definition{adj.}{aéreo; aerotransportado; refere-se à transmissão de sinais de rádio}
  \definition{s.}{no céu; no ar}
\end{EntryWithPhonetic}

\begin{EntryWithPhonetic}{空中小姐}{kong1zhong1xiao3jie3}{8,4,3,8}{⽳、⼁、⼩、⼥}
  \definition{s.}{aeromoça}
\end{EntryWithPhonetic}

\begin{EntryWithPhonetic}{孔}{kong3}{4}{⼦}
  \definition*{s.}{Abreviação de Confúcio, 孔子 | Sobrenome Kong}
  \definition{adj.}{Clássico: muito; bastante; razoavelmente}
  \definition{adj.}{aberto; desimpedido; claro; desobstruído}
  \definition{clas.}{usado para habitações em cavernas}
  \definition[个,排]{s.}{buraco; abertura; poro}
  \seealsoref{孔子}{kong3zi3}
\end{EntryWithPhonetic}

\begin{EntryWithPhonetic}{孔夫子}{kong3fu1zi3}{4,4,3}{⼦、⼤、⼦}
  \definition*{s.}{Confúcio (551-479 aC), pensador e filósofo social chinês}
  \seealsoref{孔子}{kong3zi3}
\end{EntryWithPhonetic}

\begin{EntryWithPhonetic}{孔雀}{kong3que4}{4,11}{⼦、⾫}
  \definition{s.}{pavão}
\end{EntryWithPhonetic}

\begin{EntryWithPhonetic}{孔子}{kong3zi3}{4,3}{⼦、⼦}
  \definition*{s.}{Confúcio (551-479 aC), pensador e filósofo social chinês}
  \seealsoref{孔夫子}{kong3fu1zi3}
\end{EntryWithPhonetic}

\begin{EntryWithPhonetic}{孔子学院}{kong3zi3 xue2yuan4}{4,3,8,9}{⼦、⼦、⼦、⾩}
  \definition*{s.}{Instituto Confúcio, organização estabelecida internacionalmente pela República Popular da China, que promove a língua e a cultura chinesas}
\end{EntryWithPhonetic}

\begin{EntryWithPhonetic}{恐}{kong3}{10}{⼼}
  \definition{adv.}{talvez; provavelmente}
  \definition{v.}{temer; recear; ter medo de | ameaçar; aterrorizar; intimidar}
\end{EntryWithPhonetic}

\begin{EntryWithPhonetic}{恐怖主义}{kong3bu4zhu3yi4}{10,8,5,3}{⼼、⼼、⼂、⼂}
  \definition{adj.}{terrorista}
  \definition{s.}{terrorismo}
\end{EntryWithPhonetic}

\begin{EntryWithPhonetic}{恐龙}{kong3long2}{10,5}{⼼、⿓}
  \definition[头,只]{s.}{dinossauro}
\end{EntryWithPhonetic}

\begin{EntryWithPhonetic}{恐怕}{kong3pa4}{10,8}{⼼、⼼}[HSK 3]
  \definition{adv.}{talvez; provavelmente; pode ser; expressa suposição; estimativa. | por medo de; expressar estimativa e preocupação}
  \definition{v.}{ter medo de; temer; recear}
\end{EntryWithPhonetic}

\begin{EntryWithPhonetic}{空}{kong4}{8}{⽳}[HSK 4]
  \definition*{s.}{Sobrenome Kong}
  \definition{adj.}{vazio; oco; nulo; que não contém nada; que não tem nada ou nenhum conteúdo; impraticável}
  \definition{adv.}{para nada; em vão; sem efeito}
  \definition{s.}{céu; ar | vazio; ausência do mundo dos sentidos}
  \seeref{kong1}
\end{EntryWithPhonetic}

\begin{EntryWithPhonetic}{空儿}{kong4r5}{8,2}{⽳、⼉}[HSK 3]
  \definition[个]{s.}{tempo livre; sem horário específico | sala; espaço (não utilizado); área ainda não utilizada}
  \definition{v.}{ter tempo livre}
\end{EntryWithPhonetic}

\begin{EntryWithPhonetic}{控}{kong4}{11}{⼿}
  \definition{v.}{acusar; cobrar | controlar; dominar | manter (parte do corpo em uma determinada posição) sem apoio | virar (um recipiente) de cabeça para baixo para deixar o líquido escorrer}
\end{EntryWithPhonetic}

\begin{EntryWithPhonetic}{控制}{kong4zhi4}{11,8}{⼿、⼑}[HSK 5]
  \definition{v.}{controlar; restringir; dominar; fazer com que não ultrapasse um determinado limite | controlar; dominar; comandar; ocupar, fazer com que não se perca}
\end{EntryWithPhonetic}

\begin{EntryWithPhonetic}{口}{kou3}{3}{⼝}[HSK 1][Kangxi 30]
  \definition*{s.}{Sobrenome Kou}
  \definition{clas.}{usado para coisas com bocas (pessoas, animais domésticos, canhões, etc.) | usado para mordidas ou bocados | usado para idiomas}
  \definition{s.}{boca | borda; boca; o espaço externo ao recipiente | saída; entrada; local de entrada e saída | o gosto de alguém | corte; buraco; ferida |  a borda de uma faca; lâminas de facas, espadas, tesouras, etc. | a idade de um animal de tração | seção; departamento; sistema integrado de departamentos relacionados | conversa, discurso; pronunciamento; referência à fala | um portão da Grande Muralha (frequentemente usado em nomes de lugares)}
\end{EntryWithPhonetic}

\begin{EntryWithPhonetic}{口吃}{kou3chi1}{3,6}{⼝、⼝}
  \definition{s.}{gagueira; espasmofemia; balbucinato; mogilalia; battarismo; battarismo; iscnofonia; pselismo; o fenômeno de repetir palavras ou interromper frases ao falar é um defeito habitual de linguagem comumente conhecido como gagueira}
\end{EntryWithPhonetic}

\begin{EntryWithPhonetic}{口吃病}{kou3chi1 bing4}{3,6,10}{⼝、⼝、⽧}
  \definition{s.}{doença da gagueira}
\end{EntryWithPhonetic}

\begin{EntryWithPhonetic}{口袋}{kou3dai4}{3,11}{⼝、⾐}[HSK 4]
  \definition[个,只]{s.}{bolso | saco; sacola; artigos de tecido ou couro}
\end{EntryWithPhonetic}

\begin{EntryWithPhonetic}{口袋妖怪}{kou3dai4 yao1guai4}{3,11,7,8}{⼝、⾐、⼥、⼼}
  \definition*{s.}{Pokémon (franquia de mídia japonesa)}
\end{EntryWithPhonetic}

\begin{EntryWithPhonetic}{口号}{kou3 hao4}{3,5}{⼝、⼝}[HSK 5]
  \definition[个,条,些]{s.}{\emph{slogan}; palavra de ordem; lema}
\end{EntryWithPhonetic}

\begin{EntryWithPhonetic}{口试}{kou3 shi4}{3,8}{⼝、⾔}[HSK 6]
  \definition{s.}{exame oral (ou teste); um tipo de exame que exige que os candidatos respondam a perguntas oralmente (em oposição a 笔试)}
  \definition{v.}{examinar oralmente}
  \seealsoref{笔试}{bi3 shi4}
\end{EntryWithPhonetic}

\begin{EntryWithPhonetic}{口香糖}{kou3xiang1tang2}{3,9,16}{⼝、⾹、⽶}
  \definition{s.}{goma de mascar | chiclete}
\end{EntryWithPhonetic}

\begin{EntryWithPhonetic}{口音}{kou3yin1}{3,9}{⼝、⾳}
  \definition{s.}{sons da fala oral (linguística)}
  \seeref{kou3yin5}
\end{EntryWithPhonetic}

\begin{EntryWithPhonetic}{口音}{kou3yin5}{3,9}{⼝、⾳}
  \definition{s.}{sotaque | voz}
  \seeref{kou3yin1}
\end{EntryWithPhonetic}

\begin{EntryWithPhonetic}{口语}{kou3 yu3}{3,9}{⼝、⾔}[HSK 4]
  \definition[门]{s.}{linguagem oral; linguagem falada; linguagem coloquial; linguagem usada em conversas}
\end{EntryWithPhonetic}

\begin{EntryWithPhonetic}{扣}{kou4}{6}{⼿}[HSK 6]
  \definition*{s.}{Sobrenome Kou}
  \definition{clas.}{giro; volta; uma volta de uma rosca}
  \definition[个,颗,粒]{s.}{nó | fivela; botão | círculo de rosca (em um parafuso)}
  \definition{v.}{fivela; abotoar; amarrar ou prender com um laço ou anel | colocar uma xícara, tigela etc. de cabeça para baixo; cobrir com uma xícara, tigela etc. invertida; colocar a boca do recipiente para baixo | deter; prender; levar sob custódia | cravar; esmagar (a bola); arremessar ou bater (em uma bola) com força de cima para baixo | atracar; deduzir; descontar; subtrair uma parte do valor original | puxar; pressionar | impor; marcar sem fundamento; acusar injustamente; impor ou atribuir (um crime ou má fama) a alguém}
\end{EntryWithPhonetic}

\begin{EntryWithPhonetic}{枯}{ku1}{9}{⽊}
  \definition{adj.}{murcho | (de um poço, rio, etc.) seco | chato; desinteressante | magro e abatido; emaciado}
  \definition[片]{s.}{borra; resíduo}
\end{EntryWithPhonetic}

\begin{EntryWithPhonetic}{枯木}{ku1mu4}{9,4}{⽊、⽊}
  \definition{s.}{árvore morta | madeira morta}
\end{EntryWithPhonetic}

\begin{EntryWithPhonetic}{哭}{ku1}{10}{⼝}[HSK 2]
  \definition{v.}{chorar; soluçar; lamentar-se; chorar de dor ou emoção}
\end{EntryWithPhonetic}

\begin{EntryWithPhonetic}{哭墙}{ku1qiang2}{10,14}{⼝、⼟}
  \definition*{s.}{Muro das Lamentações (Jerusalém)}
\end{EntryWithPhonetic}

\begin{EntryWithPhonetic}{苦}{ku3}{8}{⾋}[HSK 4]
  \definition{adj.}{amargo; descreve um sabor parecido com o de melão amargo ou raiz de coptis (em oposição a 甘 ou 甜) | difícil; doloroso; sofrido}
  \definition{adv.}{meticulosamente; diligentemente; pacientemente}
  \definition{v.}{causar sofrimento a alguém; dificultar a vida de alguém; causar dor; tornar desconfortável | sofrer de; ser incomodado por; sentir-se angustiado com uma situação | estar desgastado; cortar demais; descrever a superação de um certo nível em algum aspecto}
  \seealsoref{甘}{gan1}
  \seealsoref{甜}{tian2}
\end{EntryWithPhonetic}

\begin{EntryWithPhonetic}{苦瓜}{ku3gua1}{8,5}{⾋、⽠}
  \definition{s.}{melão amargo (cabaça amarga, pêra bálsamo, maçã bálsamo, pepino amargo)}
\end{EntryWithPhonetic}

\begin{EntryWithPhonetic}{库}{ku4}{7}{⼴}[HSK 5]
  \definition{s.}{depósito; tesouraria; armazém; almoxarifado; edifícios e equipamentos para armazenamento de mercadorias | Computação: banco de dados}
\end{EntryWithPhonetic}

\begin{EntryWithPhonetic}{裤}{ku4}{12}{⾐}
  \definition[条]{s.}{calças}
\end{EntryWithPhonetic}

\begin{EntryWithPhonetic}{裤子}{ku4zi5}{12,3}{⾐、⼦}[HSK 3]
  \definition[条]{s.}{calças; calções; roupas usadas abaixo da cintura, com cós, virilha e duas pernas}
\end{EntryWithPhonetic}

\begin{EntryWithPhonetic}{酷}{ku4}{14}{⾣}[HSK 6]
  \definition{adj.}{cruel; opressivo | feroz; escaldante | brutal | \emph{cool} (empréstimo linguístico); legal; excelente; moderno; ótimo | elegante e sóbrio; gracioso e severo}
  \definition{adv.}{muito; extremamente}
\end{EntryWithPhonetic}

\begin{EntryWithPhonetic}{酷斯拉}{ku4si1la1}{14,12,8}{⾣、⽄、⼿}
  \definition*{s.}{Godzilla. do Japonês Gojira, ゴジラ}
  \seealsoref{哥斯拉}{ge1si1la1}
\end{EntryWithPhonetic}

\begin{EntryWithPhonetic}{跨}{kua4}{13}{⾜}[HSK 6]
  \definition{adj.}{localizado ao lado de; anexo a}
  \definition{v.}{dar um passo; andar a passos largos | disputar; ficar de pernas abertas | atravessar; ir além (dos limites de uma certa quantidade, tempo, região, etc.)}
\end{EntryWithPhonetic}

\begin{EntryWithPhonetic}{会}{kuai4}{6}{⼈}
  \definition[个,场,次]{s.}{contabilidade}
  \definition{v.}{computar; calcular; equilibrar uma conta}
  \seeref{hui4}
\end{EntryWithPhonetic}

\begin{EntryWithPhonetic}{会计}{kuai4ji4}{6,4}{⼈、⾔}[HSK 4]
  \definition[个,位,名]{s.}{contabilidade | contador; contabilista; guarda-livros; pessoal que trabalha como contador}
\end{EntryWithPhonetic}

\begin{EntryWithPhonetic}{块}{kuai4}{7}{⼟}[HSK 1]
  \definition{clas.}{usado para coisas em pedaços | usado para coisas em pedaços ou em algumas formas de folhas | usado para moedas de prata ou notas de papel equivalentes a 圆}
  \definition{s.}{pedaço; pedaço (de terra); peça; algo que forma um pedaço ou massa}
  \seealsoref{圆}{yuan2}
\end{EntryWithPhonetic}

\begin{EntryWithPhonetic}{快}{kuai4}{7}{⼼}[HSK 1]
  \definition*{s.}{Sobrenome Kuai}
  \definition{adj.}{rápido; veloz (oposto a 慢) | apressado | perspicaz; ágil; inteligente; de ​​mente rápida | (de uma faca, espada, etc.) afiado (oposto a 钝) | direto; franco; sem rodeios | satisfeito; feliz; gratificado | rápido; veloz; alta velocidade; tempo de execução curto | satisfeito; feliz; contente | engenhoso; ágil | afiado; facas, tesouras, machados e outros objetos afiados | sincero}
  \definition{adv.}{em breve; antes de muito tempo; estar prestes a | rapidamente}
  \definition{s.}{policial; polícia | (antigo) oficial encarregado de efetuar prisões}
  \seealsoref{钝}{dun4}
  \seealsoref{慢}{man4}
\end{EntryWithPhonetic}

\begin{EntryWithPhonetic}{快餐}{kuai4 can1}{7,16}{⼼、⾷}[HSK 2]
  \definition[份,顿]{s.}{pedido (comida) rápido; \emph{fast food}; refere-se a refeições simples preparadas com antecedência e que podem ser servidas rapidamente}
\end{EntryWithPhonetic}

\begin{EntryWithPhonetic}{快车}{kuai4 che1}{7,4}{⼼、⾞}[HSK 6]
  \definition{s.}{trem ou ônibus expresso (em oposição a 慢车); um trem ou ônibus com menos paradas e tempos de viagem mais curtos (usado principalmente para transporte de passageiros)}
  \seealsoref{慢车}{man4 che1}
\end{EntryWithPhonetic}

\begin{EntryWithPhonetic}{快递}{kuai4 di4}{7,10}{⼼、⾡}[HSK 4]
  \definition[个,件,批]{s.}{correio rápido; entrega expressa; entrega rápida}
  \definition{v.}{entregar (serviço de entrega rápida por transportadoras especializadas)}
\end{EntryWithPhonetic}

\begin{EntryWithPhonetic}{快点儿}{kuai4 dian3r5}{7,9,2}{⼼、⽕、⼉}[HSK 2]
  \definition{v.}{apressar-se}
\end{EntryWithPhonetic}

\begin{EntryWithPhonetic}{快活}{kuai4huo5}{7,9}{⼼、⽔}[HSK 5]
  \definition{adj.}{feliz; alegre; contente; animado}
\end{EntryWithPhonetic}

\begin{EntryWithPhonetic}{快乐}{kuai4le4}{7,5}{⼼、⼃}[HSK 2]
  \definition{adj.}{feliz; alegre; animado; prazeiroso}
  \definition{s.}{felicidade | alegria}
\end{EntryWithPhonetic}

\begin{EntryWithPhonetic}{快速}{kuai4 su4}{7,10}{⼼、⾡}[HSK 3]
  \definition{adj.}{rápido; veloz; de alta velocidade; descreve o tempo curto gasto para caminhar, fazer algo, etc.}
\end{EntryWithPhonetic}

\begin{EntryWithPhonetic}{快要}{kuai4 yao4}{7,9}{⼼、⾑}[HSK 2]
  \definition{adv.}{estar prestes a; estar indo para; estar à beira de; em breve; em pouco tempo; indica que a situação está prestes a ocorrer}
\end{EntryWithPhonetic}

\begin{EntryWithPhonetic}{筷}{kuai4}{13}{⽵}
  \definition[双,根,个]{s.}{pauzinhos para comer}
\end{EntryWithPhonetic}

\begin{EntryWithPhonetic}{筷子}{kuai4zi5}{13,3}{⽵、⼦}[HSK 2]
  \definition[根,双,副,把,对]{s.}{pauzinhos; \emph{chopsticks}; dois bastôes finos feitos de bambu, madeira, metal ou outro material, usados para segurar comida ou outros objetos}
\end{EntryWithPhonetic}

\begin{EntryWithPhonetic}{宽}{kuan1}{10}{⼧}[HSK 4]
  \definition*{s.}{Sobrenome Kuan}
  \definition{adj.}{largo; amplo; espaçoso; grandes distâncias horizontais (em oposição a 窄) | leniente; generoso; indulgente | bem de vida; rico; confortável}
  \definition[米]{s.}{largura; amplitude}[桌子有一米宽。===A mesa tem um metro de largura.]
  \definition{v.}{relaxar; aliviar}
  \seealsoref{窄}{zhai3}
\end{EntryWithPhonetic}

\begin{EntryWithPhonetic}{宽度}{kuan1 du4}{10,9}{⼧、⼴}[HSK 5]
  \definition{s.}{largura; amplitude; duração; o grau de largura e estreiteza; a distância horizontal (no caso de um retângulo, a distância entre os dois lados mais longos)}
\end{EntryWithPhonetic}

\begin{EntryWithPhonetic}{宽广}{kuan1 guang3}{10,3}{⼧、⼴}[HSK 4]
  \definition{adj.}{vasto; amplo; espaçoso; extenso}
\end{EntryWithPhonetic}

\begin{EntryWithPhonetic}{宽阔}{kuan1 kuo4}{10,12}{⼧、⾨}[HSK 6]
  \definition{adj.}{amplo; largo; espaçoso | tolerante; mente aberta; descreve uma mente alegre e ampla}
\end{EntryWithPhonetic}

\begin{EntryWithPhonetic}{宽影片}{kuan1ying3pian4}{10,15,4}{⼧、⼺、⽚}
  \definition{s.}{filme \emph{widescreen}}
\end{EntryWithPhonetic}

\begin{EntryWithPhonetic}{款}{kuan3}{12}{⽋}
  \definition{clas.}{para versões ou modelos (de um produto)}
  \definition[笔,个]{s.}{montante de dinheiro | fundos | parágrafo | seção}
\end{EntryWithPhonetic}

\begin{EntryWithPhonetic}{窾}{kuan3}{17}{⽳}
  \definition{adj.}{oco}
  \definition{s.}{rachadura; cavidade | (onomatopéia) água batendo na rocha}
  \definition{v.}{escavar um buraco}
  \seeref{cuan4}
\end{EntryWithPhonetic}

\begin{EntryWithPhonetic}{狂}{kuang2}{7}{⽝}[HSK 5]
  \definition*{s.}{Sobrenome Kuang}
  \definition{adj.}{louco; maluco | violento; selvagem | selvagem; delirante; furioso; desenfreado; desinibido; sem restrições | arrogante; autoritário}
\end{EntryWithPhonetic}

\begin{EntryWithPhonetic}{狂欢节}{kuang2huan1 jie2}{7,6,5}{⽝、⽋、⾋}
  \definition*{s.}{Carnaval}
\end{EntryWithPhonetic}

\begin{EntryWithPhonetic}{况}{kuang4}{7}{⼎}
  \definition*{s.}{Sobrenome Kuang}
  \definition{conj.}{além disso | mesmo; muito menos; sem mencionar}
  \definition{s.}{condição; situação}
  \definition{v.}{comparar}
\end{EntryWithPhonetic}

\begin{EntryWithPhonetic}{况且}{kuang4qie3}{7,5}{⼎、⼀}
  \definition{conj.}{além disso; além do mais; orações de conexão para expressar uma relação progressiva}
\end{EntryWithPhonetic}

\begin{EntryWithPhonetic}{旷}{kuang4}{7}{⽇}
  \definition*{s.}{Sobrenome Kuang}
  \definition{adj.}{vasto; espaçoso | livre de preocupações e ideias mesquinhas | folgado}
  \definition{v.}{negligenciar ou desperdiçar | estar ausente de | desperdiçar; abandonar; negligenciar}
\end{EntryWithPhonetic}

\begin{EntryWithPhonetic}{旷野}{kuang4ye3}{7,11}{⽇、⾥}
  \definition{s.}{região selvagem}
\end{EntryWithPhonetic}

\begin{EntryWithPhonetic}{矿}{kuang4}{8}{⽯}[HSK 6]
  \definition[个,座]{s.}{depósito de minério | minério | mina}
\end{EntryWithPhonetic}

\begin{EntryWithPhonetic}{矿泉水}{kuang4quan2shui3}{8,9,4}{⽯、⽔、⽔}[HSK 4]
  \definition[瓶,杯,口]{s.}{água mineral de nascente}
\end{EntryWithPhonetic}

\begin{EntryWithPhonetic}{亏}{kui1}{3}{⼆}[HSK 5]
  \definition{adv.}{felizmente; por sorte; graças a | contrariamente, expressando sarcasmo}
  \definition{s.}{prejuízo; perda; déficit | perda; dano; ferida}
  \definition{v.}{perder dinheiro, etc.; ter um déficit; ter prejuízo | ter falta de; ser deficiente; carecer de | tratar injustamente; causar prejuízo; trair a confiança}
\end{EntryWithPhonetic}

\begin{EntryWithPhonetic}{葵}{kui2}{12}{⾋}
  \definition*{s.}{Sobrenome Kui}
  \definition[朵]{s.}{certas ervas de flores grandes}
\end{EntryWithPhonetic}

\begin{EntryWithPhonetic}{葵花}{kui2hua1}{12,7}{⾋、⾋}
  \definition{s.}{girassol (flor)}
\end{EntryWithPhonetic}

\begin{EntryWithPhonetic}{困}{kun4}{7}{⼞}[HSK 3]
  \definition{adj.}{cansado; exausto; fatigado | difícil; complicado; difícil e penoso; pobre e miserável | sonolento; com sono; cansado, com vontade de dormir}
  \definition{v.}{ficar encalhado; estar em apuros; preso em dificuldades e sofrimentos ou limitado por circunstâncias e condições que não pode escapar | cercar; envolver; imobilizar; controlar dentro de um determinado limite | dormir}
\end{EntryWithPhonetic}

\begin{EntryWithPhonetic}{困难}{kun4nan5}{7,10}{⼞、⾫}[HSK 3]
  \definition{adj.}{dificuldades financeiras; circunstâncias difíceis | complicado; complexo; difícil; árduo; a situação é complexa e há muitos obstáculos}
  \definition[种]{s.}{dificuldade; situação difícil; problemas ou situações difíceis de resolver no trabalho e na vida}
\end{EntryWithPhonetic}

\begin{EntryWithPhonetic}{困扰}{kun4 rao3}{7,7}{⼞、⼿}[HSK 5]
  \definition{v.}{perturbar; deixar perplexo; perseguir}
\end{EntryWithPhonetic}

\begin{EntryWithPhonetic}{扩}{kuo4}{6}{⼿}
  \definition{v.}{expandir; ampliar; estender; alargar}
\end{EntryWithPhonetic}

\begin{EntryWithPhonetic}{扩大}{kuo4da4}{6,3}{⼿、⼤}[HSK 4]
  \definition{v.}{ampliar; expandir; estender; alargar}
\end{EntryWithPhonetic}

\begin{EntryWithPhonetic}{扩展}{kuo4 zhan3}{6,10}{⼿、⼫}[HSK 4]
  \definition{v.}{esticar; expandir; estender; espalhar}
\end{EntryWithPhonetic}

\begin{EntryWithPhonetic}{括}{kuo4}{9}{⼿}
  \definition{v.}{unir (músculos, etc.); contrair | incluir | adicionar colchetes a | amarrar; empacotar}
\end{EntryWithPhonetic}

\begin{EntryWithPhonetic}{括号}{kuo4 hao4}{9,5}{⼿、⼝}[HSK 4]
  \definition{s.}{chaves, colchetes e parênteses (em fórmulas aritméticas ou algébricas, os símbolos que indicam a combinação e a ordem de vários números ou termos) | colchetes e parênteses usados como um tipo de sinal de pontuação para mostrar a parte explicativa de uma passagem em um texto}
\end{EntryWithPhonetic}

\begin{EntryWithPhonetic}{阔}{kuo4}{12}{⾨}[HSK 6]
  \definition{adj.}{amplo; amplo; vasto | rico | longo, no sentido de ``há muito tempo'' | vazio; impraticável}
\end{EntryWithPhonetic}

%%%%% EOF %%%%%


 %%%
%%% L
%%%

\section*{L}\addcontentsline{toc}{section}{L}

\begin{EntryWithPhonetic}{垃}{la1}{8}{⼟}
  \definition[堆]{s.}{lixo}
\end{EntryWithPhonetic}

\begin{EntryWithPhonetic}{垃圾}{la1 ji1}{8,6}{⼟、⼟}[HSK 4]
  \definition{adj.}{lixo; inútil, ruim ou prejudicial}
  \definition[袋,桶,堆,车,片]{s.}{entulho; lixo; refugo; rejeito; resíduo; coisa inútil que é jogada fora; metáfora para alguém ou algo que perdeu seu valor ou serve a um propósito ruim}
\end{EntryWithPhonetic}

\begin{EntryWithPhonetic}{垃圾车}{la1ji1che1}{8,6,4}{⼟、⼟、⾞}
  \definition{s.}{caminhão de lixo}
\end{EntryWithPhonetic}

\begin{EntryWithPhonetic}{垃圾电邮}{la1ji1 dian4you2}{8,6,5,7}{⼟、⼟、⽥、⾢}
  \definition{s.}{\emph{e-mail} de \emph{spam}}
  \seealsoref{垃圾邮件}{la1ji1 you2jian4}
\end{EntryWithPhonetic}

\begin{EntryWithPhonetic}{垃圾堆}{la1ji1dui1}{8,6,11}{⼟、⼟、⼟}
  \definition{s.}{depósito de lixo}
\end{EntryWithPhonetic}

\begin{EntryWithPhonetic}{垃圾工}{la1ji1gong1}{8,6,3}{⼟、⼟、⼯}
  \definition{s.}{lixeiro | gari}
\end{EntryWithPhonetic}

\begin{EntryWithPhonetic}{垃圾食品}{la1ji1shi2pin3}{8,6,9,9}{⼟、⼟、⾷、⼝}
  \definition{s.}{\emph{junk food}}
\end{EntryWithPhonetic}

\begin{EntryWithPhonetic}{垃圾筒}{la1ji1tong3}{8,6,12}{⼟、⼟、⽵}
  \definition{s.}{cesto de lixo}
\end{EntryWithPhonetic}

\begin{EntryWithPhonetic}{垃圾箱}{la1ji1xiang1}{8,6,15}{⼟、⼟、⾋}
  \definition{s.}{cesto de lixo}
\end{EntryWithPhonetic}

\begin{EntryWithPhonetic}{垃圾邮件}{la1ji1 you2jian4}{8,6,7,6}{⼟、⼟、⾢、⼈}
  \definition{s.}{\emph{spam}, \emph{e-mail} não solicitado}
  \seealsoref{垃圾电邮}{la1ji1 dian4you2}
\end{EntryWithPhonetic}

\begin{EntryWithPhonetic}{拉}{la1}{8}{⼿}[HSK 2]
  \definition{s.}{abreviação de América Latina, 拉丁美洲}
  \definition{v.}{puxar; arrastar; rebocar | transportar por veículo; rebocar | arrastar (ou puxar) para fora | mover (tropas para um lugar) | dar uma mãozinha; ajudar | arrastar para dentro; implicar; envolver | criar (criança) | atrair; conquistar; solicitar; angariar votos | bater-papo | organizar; preparar | ter dívidas; estar endividado | pressionar; recrutar à força | (no tênis, tênis de mesa, etc.) levantar (a bola) | tocar (certos instrumentos musicais); puxar uma parte do instrumento para que ele emita som | prolongar; espaçar | envolver-se em | (coloquial) esvaziar os intestinos | levantar, uma das técnicas do tênis de mesa | destruir; esmagar; quebrar}
  \seeref{la4}
  \seealsoref{拉丁美洲}{la1ding1 mei3zhou1}
\end{EntryWithPhonetic}

\begin{EntryWithPhonetic}{拉布布}{la1bu4bu4}{8,5,5}{⼿、⼱、⼱}
  \definition*{s.}{Labubu}
\end{EntryWithPhonetic}

\begin{EntryWithPhonetic}{拉丁美洲}{la1ding1 mei3zhou1}{8,2,9,9}{⼿、⼀、⽺、⽔}
  \definition*{s.}{América Latina, nome coletivo dos países da América Central e do Sul, devido ao fato de a maioria de seus habitantes ser descendente de povos latinos e de a língua falada ser do grupo latino}
\end{EntryWithPhonetic}

\begin{EntryWithPhonetic}{拉开}{la1 kai1}{8,4}{⼿、⼶}[HSK 4]
  \definition{v.}{puxar para abrir; recuar| ampliar; espaçar; distanciar; afastar; separar}
\end{EntryWithPhonetic}

\begin{EntryWithPhonetic}{拉拉队}{la1la1dui4}{8,8,4}{⼿、⼿、⾩}
  \definition{s.}{claque | torcida}
\end{EntryWithPhonetic}

\begin{EntryWithPhonetic}{拉萨}{la1sa4}{8,11}{⼿、⾋}
  \definition*{s.}{Lhasa, capital da Região Autônoma do Tibete, 西藏自治区}
  \seealsoref{西藏自治区}{xi1zang4 zi4zhi4qu1}
\end{EntryWithPhonetic}

\begin{EntryWithPhonetic}{啦}{la1}{11}{⼝}
  \definition{s.}{(onomatoméia) som de canto, aplausos etc.; usado para palavras como 呼啦啦, 哗啦啦, 哩哩啦啦, etc.}
  \seeref{la5}
  \seealsoref{呼啦啦}{hu1 la1 la1}
  \seealsoref{哗啦啦}{hua1la1 la5}
  \seealsoref{哩哩啦啦}{li1 li1 la1 la1}
\end{EntryWithPhonetic}

\begin{EntryWithPhonetic}{拉}{la4}{8}{⼿}
  \definition{s.}{usado em 拉拉蛄 \dpy{la4la4gu3}}
  \seeref{la1}
  \seealsoref{拉拉蛄}{la4la4gu3}
\end{EntryWithPhonetic}

\begin{EntryWithPhonetic}{拉拉蛄}{la4la4gu3}{8,8,11}{⼿、⼿、⾍}
  \variantof{蝲蝲蛄}
\end{EntryWithPhonetic}

\begin{EntryWithPhonetic}{落}{la4}{12}{⾋}[HSK 5]
  \definition{v.}{deixar de fora; estar ausente | deixar para trás; esquecer de trazer; deixar algo em algum lugar e esquecer de levar| ficar para trás (ou cair); não conseguir acompanhar}
  \seeref{lao4}
  \seeref{luo4}
\end{EntryWithPhonetic}

\begin{EntryWithPhonetic}{蜡}{la4}{14}{⾍}
  \definition{s.}{cera; óleos produzidos por animais, minerais ou plantas | vela}
\end{EntryWithPhonetic}

\begin{EntryWithPhonetic}{蜡烛}{la4zhu2}{14,10}{⾍、⽕}
  \definition[根,支]{s.}{vela | círio | peça, geralmente de cera, que possui um pavio e se utiliza para iluminar}
\end{EntryWithPhonetic}

\begin{EntryWithPhonetic}{辣}{la4}{14}{⾟}[HSK 4]
  \definition{adj.}{apimentado; picante; pungente; quente | cruel; implacável; venenoso; vicioso}
  \definition{v.}{queimar; picar; formigar; ter uma irritação picante (boca, nariz ou olhos)}
\end{EntryWithPhonetic}

\begin{EntryWithPhonetic}{蝲}{la4}{15}{⾍}
  \definition{s.}{lagostim de água doce}
  \seealsoref{蝲蛄}{la4gu3}
\end{EntryWithPhonetic}

\begin{EntryWithPhonetic}{蝲蛄}{la4gu3}{15,11}{⾍、⾍}
  \definition{s.}{lagostim; lagostim de água doce}
\end{EntryWithPhonetic}

\begin{EntryWithPhonetic}{蝲蝲蛄}{la4la4gu3}{15,15,11}{⾍、⾍、⾍}
  \definition{s.}{grilo toupeira}
\end{EntryWithPhonetic}

\begin{EntryWithPhonetic}{啦}{la5}{11}{⼝}[HSK 6]
  \definition{part.}{uma palavra composta de 了 e 啊, que tem o significado de ambos}
  \seeref{la1}
  \seealsoref{啊}{a5}
  \seealsoref{了}{le5}
\end{EntryWithPhonetic}

\begin{EntryWithPhonetic}{来}{lai2}{7}{⽊}[HSK 1]
  \definition*{s.}{Sobrenome Lai}
  \definition{part.}{usado após uma palavra numérica ou de quantidade; indica uma quantidade aproximada | usado depois de numerais como 一, 二, 三; para listar razões ou fatos, etc.}
  \definition{s.}{usado após uma expressão de tempo para indicar uma duração que vai do passado ao presente}
  \definition{v.}{vir; chegar; de outro lugar para o lugar onde o interlocutor se encontra | aparecer; acontecer; vir; (problemas, coisas, etc.) ocorrerem; surgirem | substitui um verbo com significado específico, indicando a realização de uma ação específica | estar indo para; usado antes de outro verbo, indica que algo será feito | vir para fazer algo; usado após outro verbo, indica que se vai fazer algo | usado para indicar um propósito; expressar o objetivo, fazer algo usando o método, a atitude ou a direção anteriores | usado com 得 ou 不 para indicar possibilidade, capacidade ou hábito}
  \seealsoref{不}{bu4}
  \seealsoref{得}{de5}
\end{EntryWithPhonetic}

\begin{EntryWithPhonetic}{来不及}{lai2bu5ji2}{7,4,3}{⽊、⼀、⼃}[HSK 4]
  \definition{v.}{ser tarde demais; não ter tempo; não ter tempo suficiente (para fazer algo); não ser possível participar ou se atualizar devido a restrições de tempo}
\end{EntryWithPhonetic}

\begin{EntryWithPhonetic}{来到}{lai2 dao4}{7,8}{⽊、⼑}[HSK 1]
  \definition{v.}{chegar; vir}
\end{EntryWithPhonetic}

\begin{EntryWithPhonetic}{来得及}{lai2de5ji2}{7,11,3}{⽊、⼻、⼃}[HSK 4]
  \definition{v.}{ainda ter tempo; ser capaz de fazer isso; ser capaz de fazer algo a tempo; ainda ter tempo de cuidar ou de colocar em dia}
\end{EntryWithPhonetic}

\begin{EntryWithPhonetic}{来往}{lai2 wang3}{7,8}{⽊、⼻}[HSK 6]
  \definition{s.}{negociação; contato com alguém; interações sociais}
  \definition{v.}{ir e vir | ter negócios com alguém}
\end{EntryWithPhonetic}

\begin{EntryWithPhonetic}{来信}{lai2 xin4}{7,9}{⽊、⼈}[HSK 5]
  \definition[封]{s.}{sua carta; carta recebida; carta ao interlocutor}
  \definition{v.}{enviar uma carta para aqui; enviar uma carta para o remetente}
\end{EntryWithPhonetic}

\begin{EntryWithPhonetic}{来源}{lai2yuan2}{7,13}{⽊、⽔}[HSK 4]
  \definition{s.}{origem; causa; fonte; tabula rasa (ou seja, o lugar de onde as coisas vêm)}
  \definition{v.}{originar-se; surgir; ter origem; (algo) originar (seguido de 于)}
  \seealsoref{于}{yu2}
\end{EntryWithPhonetic}

\begin{EntryWithPhonetic}{来自}{lai2zi4}{7,6}{⽊、⾃}[HSK 2]
  \definition{v.}{vir de (um local) | \emph{From:} (cabeçalho de \emph{e -mail})}
\end{EntryWithPhonetic}

\begin{EntryWithPhonetic}{赖}{lai4}{13}{⾙}[HSK 6]
  \definition*{s.}{Sobrenome Lai}
  \definition{adj.}{ruim; pobre; não é bom}
  \definition{v.}{confiar em; depender de | permanecer em um lugar; prolongar a permanência de alguém em um lugar; ficar e recusar-se a sair | negar o próprio erro ou responsabilidade; voltar atrás na palavra; repudiar; negar; não admitir culpa; não assumir responsabilidade | colocar a culpa nos outros; incriminar falsamente (acusar); acusar alguém de algo errado; acusar alguém falsamente | culpar}
\end{EntryWithPhonetic}

\begin{EntryWithPhonetic}{兰}{lan2}{5}{⼋}
  \definition*{s.}{Sobrenome Lan}
  \definition{s.}{orquídea | lírio magnólia}
\end{EntryWithPhonetic}

\begin{EntryWithPhonetic}{兰花}{lan2hua1}{5,7}{⼋、⾋}
  \definition{s.}{orquídea}
\end{EntryWithPhonetic}

\begin{EntryWithPhonetic}{兰州}{lan2zhou1}{5,6}{⼋、⼮}
  \definition*{s.}{Lanzhou. capital da província de Gansu, 甘肃}
  \seealsoref{甘肃}{gan1su4}
\end{EntryWithPhonetic}

\begin{EntryWithPhonetic}{栏}{lan2}{9}{⽊}
  \definition{s.}{cerca; corrimão; balaustrada | curral; galpão; celeiro; chiqueiro | coluna (de uma página ou tabela, ou de um jornal) | quadro (de avisos); prancha; tabuleiro | Esporte: obstáculo}
\end{EntryWithPhonetic}

\begin{EntryWithPhonetic}{栏目}{lan2mu4}{9,5}{⽊、⽬}[HSK 6]
  \definition[个,档]{s.}{coluna; programa; seções nomeadas de jornais, revistas, etc. divididas de acordo com a natureza de seu conteúdo}
\end{EntryWithPhonetic}

\begin{EntryWithPhonetic}{蓝}{lan2}{13}{⾋}[HSK 2]
  \definition*{s.}{Sobrenome Lan}
  \definition{adj.}{azul}
  \definition{s.}{planta índigo; anil | plantas azuis; refere-se a certas plantas que podem ser usadas como corante azul ou certas plantas cujas folhas são azul-esverdeadas}
\end{EntryWithPhonetic}

\begin{EntryWithPhonetic}{蓝领}{lan2 ling3}{13,11}{⾋、⾴}[HSK 6]
  \definition[名,位,个]{s.}{trabalhador braçal}
\end{EntryWithPhonetic}

\begin{EntryWithPhonetic}{蓝色}{lan2 se4}{13,6}{⾋、⾊}[HSK 2]
  \definition[抹,片,缕,团,块]{s.}{cor azul}
\end{EntryWithPhonetic}

\begin{EntryWithPhonetic}{篮}{lan2}{16}{⽵}
  \definition[个]{s.}{cesto | o anel de ferro e a rede na cesta de basquete}
\end{EntryWithPhonetic}

\begin{EntryWithPhonetic}{篮球}{lan2qiu2}{16,11}{⽵、⽟}[HSK 2]
  \definition[个,只]{s.}{basquetebol | bola de basquete; refere-se à bola utilizada no basquetebol}
\end{EntryWithPhonetic}

\begin{EntryWithPhonetic}{懒}{lan3}{16}{⼼}[HSK 6]
  \definition{adj.}{indolente; preguiçoso (oposto de 勤) | lento; lânguido | ocioso; preguiçoso}
  \seealsoref{勤}{qin2}
\end{EntryWithPhonetic}

\begin{EntryWithPhonetic}{懒虫}{lan3chong2}{16,6}{⼼、⾍}
  \definition{s.}{desleixado ocioso | (insulto) sujeito preguiçoso}
\end{EntryWithPhonetic}

\begin{EntryWithPhonetic}{懒怠}{lan3dai4}{16,9}{⼼、⼼}
  \definition{s.}{preguiça}
\end{EntryWithPhonetic}

\begin{EntryWithPhonetic}{懒得}{lan3de5}{16,11}{⼼、⼻}
  \definition{adv.}{demasiado preguiçoso}
  \definition{v.}{não sentir vontade (de fazer algo)}
\end{EntryWithPhonetic}

\begin{EntryWithPhonetic}{懒惰}{lan3duo4}{16,12}{⼼、⼼}
  \definition{adj.}{preguiçoso}
\end{EntryWithPhonetic}

\begin{EntryWithPhonetic}{懒鬼}{lan3gui3}{16,9}{⼼、⿁}
  \definition{s.}{cara preguiçoso}
\end{EntryWithPhonetic}

\begin{EntryWithPhonetic}{懒汉}{lan3han4}{16,5}{⼼、⽔}
  \definition{s.}{sujeito ocioso | vagabundo | preguiçosos}
\end{EntryWithPhonetic}

\begin{EntryWithPhonetic}{懒人}{lan3ren2}{16,2}{⼼、⼈}
  \definition{s.}{pessoa preguiçosa}
\end{EntryWithPhonetic}

\begin{EntryWithPhonetic}{懒散}{lan3san3}{16,12}{⼼、⽁}
  \definition{adj.}{inativo | indolente | preguiçoso | negligente}
\end{EntryWithPhonetic}

\begin{EntryWithPhonetic}{懒腰}{lan3yao1}{16,13}{⼼、⾁}
  \definition[个]{s.}{alongamento (do corpo)}
\end{EntryWithPhonetic}

\begin{EntryWithPhonetic}{烂}{lan4}{9}{⽕}[HSK 5]
  \definition{adj.}{macio; pastoso; amassado | podre; deteriorado | quebrado; esfarrapado; gasto | desorganizado; indigno}
  \definition{adv.}{totalmente; extremamente; completamente; expressa um grau muito profundo}
  \definition{v.}{apodrecer; infeccionar; decompor-se}
\end{EntryWithPhonetic}

\begin{EntryWithPhonetic}{廊}{lang2}{11}{⼴}
  \definition[个]{s.}{varanda; corredor}
\end{EntryWithPhonetic}

\begin{EntryWithPhonetic}{廊坊}{lang2fang2}{11,7}{⼴、⼟}
  \definition*{s.}{Cidade de Langfang em Hebei}
\end{EntryWithPhonetic}

\begin{EntryWithPhonetic}{朗}{lang3}{10}{⽉}
  \definition*{s.}{Sobrenome Lang}
  \definition{adj.}{claro; brilhante | alto e claro (som)}
\end{EntryWithPhonetic}

\begin{EntryWithPhonetic}{朗读}{lang3du2}{10,10}{⽉、⾔}[HSK 5]
  \definition{v.}{ler em voz alta; recitar com voz clara e alta}
\end{EntryWithPhonetic}

\begin{EntryWithPhonetic}{浪}{lang4}{10}{⽔}
  \definition*{s.}{Sobrenome Lang}
  \definition{adj.}{desenfreado; perdulário}
  \definition{adv.}{livremente}
  \definition[朵,阵,波]{s.}{onda; vagalhão; rebentação | algo ondulatório | coisas ondulando como ondas}
  \definition{v.}{passear; divagar}
\end{EntryWithPhonetic}

\begin{EntryWithPhonetic}{浪费}{lang4fei4}{10,9}{⽔、⾙}[HSK 3]
  \definition{adj.}{desperdiçado; extravagante; não econômico}
  \definition{v.}{desperdiçar; dissipar; esbanjar; ser extravagante; uso excessivo ou inadequado de bens, recursos humanos, tempo, etc.}
\end{EntryWithPhonetic}

\begin{EntryWithPhonetic}{浪花}{lang4hua1}{10,7}{⽔、⾋}
  \definition[朵]{s.}{\emph{spray} | \emph{spray} do oceano | (figurativo) acontecimentos de sua vida}
\end{EntryWithPhonetic}

\begin{EntryWithPhonetic}{浪漫}{lang4man4}{10,14}{⽔、⽔}[HSK 5]
  \definition{adj.}{romântico; poético | não convencional; boêmio; abandonado; libertino; devasso; comportar-se de maneira descuidada e descuidada (geralmente se referindo a relacionamentos entre pessoas) | irrealista; impraticável}
\end{EntryWithPhonetic}

\begin{EntryWithPhonetic}{捞}{lao1}{10}{⼿}
  \definition{v.}{pescar | dragar}
\end{EntryWithPhonetic}

\begin{EntryWithPhonetic}{劳}{lao2}{7}{⼒}
  \definition*{s.}{Sobrenome Lao}
  \definition{adj.}{difícil; cansativo; cansado}
  \definition{s.}{fadiga; trabalho árduo | ação meritória; serviço; conquistas | trabalhador | mérito | trabalhador braçal}
  \definition{v.}{trabalho; labor | esforço; exercício intenso | (pedir um favor a alguém, também 有劳) colocar alguém no trabalho de | expressar apreço (ao executor de uma tarefa); recompensar | colocar alguém no trabalho de; incomodar alguém com algo | trazer presentes para}
  \seealsoref{有劳}{you3lao2}
\end{EntryWithPhonetic}

\begin{EntryWithPhonetic}{劳动}{lao2dong4}{7,6}{⼒、⼒}[HSK 5]
  \definition[次]{s.}{trabalho; mão de obra; atividades intelectuais ou físicas que podem criar valor | trabalho físico; trabalho manual; referindo-se especificamente ao trabalho físico}
  \definition{v.}{realizar trabalho físico}
\end{EntryWithPhonetic}

\begin{EntryWithPhonetic}{劳工同事}{lao2gong1 tong2shi4}{7,3,6,8}{⼒、⼯、⼝、⼅}
  \definition{s.}{colaborador | colega de trabalho}
\end{EntryWithPhonetic}

\begin{EntryWithPhonetic}{牢}{lao2}{7}{⼧}[HSK 6]
  \definition*{s.}{Sobrenome Lao}
  \definition{adj.}{firme; durável}
  \definition{s.}{prisão; cadeia | (cercado para animais) curral; baia; galinheiro; estábulo; estrebaria; cocheira | (arcaico) animal de sacrifício}
\end{EntryWithPhonetic}

\begin{EntryWithPhonetic}{老}{lao3}{6}{⽼}[HSK 1,2][Kangxi 125]
  \definition*{s.}{Sobrenome Lao}
  \definition{adj.}{velho; envelhecido; idade avançada | antigo; de longa data; existe há muito tempo | antigo; desatualizado; obsoleto; ultrapassado  | antigo; tradicional; original | coberto de vegetação; os vegetais cresceram além do período ideal para serem consumidos | resistente; endurecido; alimentos muito cozidos | escuro; profundo; (sobre cores) | último nascido; o mais novo | veterano; experiente; sofisticado}
  \definition{adv.}{longo; por muito tempo | sempre (fazendo algo) | muito}
  \definition{pref.}{usado para designar pessoas, ordem de classificação, certos nomes de animais e plantas}
  \definition{s.}{idosos; pessoas mais velhas | ancião; sênior; um título respeitoso para pessoas mais velhas}
  \definition{v.}{envelhecer | morrer; referindo-se à morte de um idoso}
\end{EntryWithPhonetic}

\begin{EntryWithPhonetic}{老百姓}{lao3bai3xing4}{6,6,8}{⽼、⽩、⼥}[HSK 3]
  \definition[些]{s.}{povo; civis; pessoas comuns; residentes (em contraste com militares e funcionários públicos)}
\end{EntryWithPhonetic}

\begin{EntryWithPhonetic}{老板}{lao3ban3}{6,8}{⽼、⽊}[HSK 3]
  \definition[个,位]{s.}{chefe; dono; líder; refere-se ao gerente de uma empresa comercial ou industrial | antigo título honorífico dado a atores famosos de ópera ou atores que também eram diretores de companhias de ópera}
\end{EntryWithPhonetic}

\begin{EntryWithPhonetic}{老兵}{lao3bing1}{6,7}{⽼、⼋}
  \definition{s.}{velho soldado | veterano de guerra | veterano (alguém que tem muita experiência em algum domínio)}
\end{EntryWithPhonetic}

\begin{EntryWithPhonetic}{老公}{lao3 gong1}{6,4}{⽼、⼋}[HSK 4]
  \definition[个,位,名]{s.}{marido; esposo}
\end{EntryWithPhonetic}

\begin{EntryWithPhonetic}{老虎}{lao3hu3}{6,8}{⽼、⾌}
  \definition[只]{s.}{tigre}
  \seealsoref{虎}{hu3}
\end{EntryWithPhonetic}

\begin{EntryWithPhonetic}{老家}{lao3 jia1}{6,10}{⽼、⼧}[HSK 4]
  \definition{s.}{cidade natal; local de origem | lugar nativo; refere-se às gerações anteriores da família ou ao local onde a pessoa nasceu ou viveu}
\end{EntryWithPhonetic}

\begin{EntryWithPhonetic}{老年}{lao3 nian2}{6,6}{⽼、⼲}[HSK 2]
  \definition[个]{s.}{idoso; velhice; idade acima de 60 ou 70 anos}
\end{EntryWithPhonetic}

\begin{EntryWithPhonetic}{老朋友}{lao3 peng2 you3}{6,8,4}{⽼、⽉、⼜}[HSK 2]
  \definition[个,位,名]{s.}{velho amigo; refere-se a amigos que conhecemos há muito tempo e com quem temos uma relação íntima}
\end{EntryWithPhonetic}

\begin{EntryWithPhonetic}{老婆}{lao3po2}{6,11}{⽼、⼥}[HSK 4]
  \definition[个,位,名]{s.}{esposa}
\end{EntryWithPhonetic}

\begin{EntryWithPhonetic}{老人}{lao3 ren2}{6,2}{⽼、⼈}[HSK 1]
  \definition[位]{s.}{homem ou mulher de idade avançada; o idoso; o velho}
\end{EntryWithPhonetic}

\begin{EntryWithPhonetic}{老人家}{lao3 ren2 jia1}{6,2,10}{⽼、⼈、⼧}
  \definition[位,名,个]{s.}{avô; avó; pessoa idosa venerável; um título respeitoso para os idosos | maneira de chamar o pai ou a mãe idosos na frente dos outros; referir-se aos próprios pais ou aos pais, professores, etc. de outras pessoas}
\end{EntryWithPhonetic}

\begin{EntryWithPhonetic}{老师}{lao3shi1}{6,6}{⽼、⼱}[HSK 1]
  \definition[个,位]{s.}{professor; título honorífico para professores; refere-se, de maneira geral, a pessoas que transmitem cultura e tecnologia ou que são dignas de admiração em termos de ideias, moralidade e conhecimentos profissionais}
\end{EntryWithPhonetic}

\begin{EntryWithPhonetic}{老是}{lao3 shi4}{6,9}{⽼、⽇}[HSK 2]
  \definition{adv.}{sempre; indica que a ação continua ou que o estado permanece inalterado, equivalente a 一直}
  \seealsoref{一直}{yi4zhi2}
\end{EntryWithPhonetic}

\begin{EntryWithPhonetic}{老实}{lao3shi5}{6,8}{⽼、⼧}[HSK 4]
  \definition{adj.}{franco; sincero; honesto | bom; bem-comportado | ingênuo; simplório; meio bobo; facilmente enganado; eufemismo para pouco inteligente}
\end{EntryWithPhonetic}

\begin{EntryWithPhonetic}{老太太}{lao3 tai4 tai5}{6,4,4}{⽼、⼤、⼤}[HSK 3]
  \definition[位,名,个]{s.}{velha senhora; (em tratamento direto)Venerável Senhora; uma maneira respeitosa de chamar uma senhora idosa; título honorífico para mulheres idosas | (forma de tratamento) sua velha mãe; minha velha mãe, avó ou sogra; referindo-se à própria mãe, à mãe do outro ou à mãe de outra pessoa, à sogra ou à sogra política}
\end{EntryWithPhonetic}

\begin{EntryWithPhonetic}{老头儿}{lao3 tou2r5}{6,5,2}{⽼、⼤、⼉}[HSK 3]
  \definition{s.}{(coloquial) (com um tom de intimidade) velho; velho amigo}
  \seealsoref{老头子}{lao3 tou2zi5}
\end{EntryWithPhonetic}

\begin{EntryWithPhonetic}{老头子}{lao3 tou2zi5}{6,5,3}{⽼、⼤、⼦}
  \definition{s.}{velho antiquado (ou velho rabugento) | (referindo-se ao marido idoso) meu velho | chefe de uma sociedade secreta | (coloquial) velho; velho rabugento}
  \seealsoref{老头儿}{lao3 tou2r5}
\end{EntryWithPhonetic}

\begin{EntryWithPhonetic}{老乡}{lao3 xiang1}{6,3}{⽼、⼄}[HSK 6]
  \definition[个,位]{s.}{conterrâneo; conterrâneo | uma maneira de chamar um fazendeiro cujo nome você não conhece}
\end{EntryWithPhonetic}

\begin{EntryWithPhonetic}{落}{lao4}{12}{⾋}
  \definition{v.}{cair; cair de uma altura elevada | se abaixar; descer; ir para baixo | permanecer; fazer uma parada; deixar para trás | obter; ter; receber}
  \seeref{la4}
  \seeref{luo4}
\end{EntryWithPhonetic}

\begin{EntryWithPhonetic}{乐}{le4}{5}{⼃}[HSK 3]
  \definition*{s.}{Sobrenome Le}
  \definition{adj.}{feliz; contente; rejubilante; animado; bem disposto}
  \definition{s.}{prazer; diversão; felicidade}
  \definition{v.}{desfrutar; ficar feliz em; amar; encontrar prazer em | rir; divertir-se}
  \seeref{yue4}
\end{EntryWithPhonetic}

\begin{EntryWithPhonetic}{乐高}{le4gao1}{5,10}{⼃、⾼}
  \definition*{s.}{Lego (brinquedo)}
\end{EntryWithPhonetic}

\begin{EntryWithPhonetic}{乐观}{le4guan1}{5,6}{⼃、⾒}[HSK 3]
  \definition{adj.}{esperançoso; otimista; confiante; espírito alegre, confiante no futuro (oposto a 悲观)}
  \seealsoref{悲观}{bei1guan1}
\end{EntryWithPhonetic}

\begin{EntryWithPhonetic}{乐趣}{le4qu4}{5,15}{⼃、⾛}[HSK 4]
  \definition[个,种,些,点]{s.}{alegria; deleite; prazer; implicação de fazer alguém se sentir feliz; um humor de preferência}
\end{EntryWithPhonetic}

\begin{EntryWithPhonetic}{乐园}{le4yuan2}{5,7}{⼃、⼞}
  \definition{s.}{paraíso}
\end{EntryWithPhonetic}

\begin{EntryWithPhonetic}{了}{le5}{2}{⼅}[HSK 1,3]
  \definition{part.}{usada após verbos ou adjetivos para indicar a conclusão de uma ação, em um momento no passado ou antes do início de outra ação, ou uma ação esperada ou presumida | usada para indicar uma mudança de situação ou estado, seja real ou prevista | comandos ou solicitações em resposta a uma situação alterada; usada para xpressar urgência ou dissuadir | usada para indicar que algo chegou ao extremo; usada no final da frase ou em pausas no meio da frase, para expressar um tom de exclamação}
  \seeref{liao3}
\end{EntryWithPhonetic}

\begin{EntryWithPhonetic}{累}{lei2}{11}{⽷}
  \definition*{s.}{Sobrenome Lei}
  \definition{adj.}{incômodo; complicado}
  \definition{s.}{corda; cordão | touro na época de acasalamento}
  \definition{v.}{amarrar; prender; atar | copular}
  \seeref{lei3}
  \seeref{lei4}
\end{EntryWithPhonetic}

\begin{EntryWithPhonetic}{雷}{lei2}{13}{⾬}
  \definition*{s.}{Sobrenome Lei}
  \definition[声,个,颗]{s.}{trovão | (militar) mina}
\end{EntryWithPhonetic}

\begin{EntryWithPhonetic}{雷电}{lei2dian4}{13,5}{⾬、⽥}
  \definition{s.}{trovão e relâmpago; raio}
\end{EntryWithPhonetic}

\begin{EntryWithPhonetic}{雷亚尔}{lei2ya4'er3}{13,6,5}{⾬、⼆、⼩}
  \definition*{s.}{Real Brasileiro}
\end{EntryWithPhonetic}

\begin{EntryWithPhonetic}{累}{lei3}{11}{⽷}
  \definition*{s.}{Sobrenome Lei}
  \definition{adj.}{em andamento; repetido; contínuo}
  \definition{v.}{acumular; empilhar; colocar em cima de outro | envolver; implicar | construir empilhando tijolos, pedras, terra, etc.}
  \seeref{lei2}
  \seeref{lei4}
\end{EntryWithPhonetic}

\begin{EntryWithPhonetic}{絫}{lei3}{12}{⽷}
  \variantof{累}
\end{EntryWithPhonetic}

\begin{EntryWithPhonetic}{泪}{lei4}{8}{⽔}[HSK 4]
  \definition[滴,行]{s.}{lágrima | algo semelhante a uma lágrima}
\end{EntryWithPhonetic}

\begin{EntryWithPhonetic}{泪水}{lei4 shui3}{8,4}{⽔、⽔}[HSK 4]
  \definition[滴,行]{s.}{lágrima}
\end{EntryWithPhonetic}

\begin{EntryWithPhonetic}{类}{lei4}{9}{⽶}[HSK 3]
  \definition*{s.}{Sobrenome Lei}
  \definition{clas.}{tipo; espécie; categoria usada para pessoas ou coisas}
  \definition{s.}{classe; categoria; tipo; variedade; a combinação de muitas coisas semelhantes ou iguais}
  \definition{v.}{assemelhar-se a; ser semelhante a}
\end{EntryWithPhonetic}

\begin{EntryWithPhonetic}{类似}{lei4si4}{9,6}{⽶、⼈}[HSK 3]
  \definition{adj.}{semelhante; análogo}
\end{EntryWithPhonetic}

\begin{EntryWithPhonetic}{类型}{lei4xing2}{9,9}{⽶、⼟}[HSK 4]
  \definition[种,个]{s.}{tipo; espécie; categoria; tipos formados por coisas com características comuns}
\end{EntryWithPhonetic}

\begin{EntryWithPhonetic}{累}{lei4}{11}{⽷}[HSK 1]
  \definition{adj.}{cansado; exausto; fatigado}
  \definition{v.}{cansar; desgastar; fatigar; esgotar | labutar; trabalhar duro}
  \seeref{lei2}
  \seeref{lei3}
\end{EntryWithPhonetic}

\begin{EntryWithPhonetic}{冷}{leng3}{7}{⼎}[HSK 1]
  \definition*{s.}{Sobrenome Leng}
  \definition{adj.}{frio; baixa temperatura; sensação de frio | gelado; frio por natureza; sem entusiasmo; sem gentileza | desolado; pouco frequentado; quieto; sem agitação | negligenciado; indesejável; ignorado por todos | raro; estranho; incomum | feito em segredo; filmado de forma escondida; lançado secretamente}
  \definition{v.}{esfriar; resfriar | esfriar; congelar; tornar-se indiferente, apático | ignorar}
\end{EntryWithPhonetic}

\begin{EntryWithPhonetic}{冷静}{leng3jing4}{7,14}{⼎、⾭}[HSK 4]
  \definition{adj.}{calmo; descreve uma pessoa que consegue ficar atenta em uma situação importante ou de emergência e não toma decisões aleatórias por causa de seus sentimentos no momento | (lugar) tranquilo; quieto; deserto}
\end{EntryWithPhonetic}

\begin{EntryWithPhonetic}{冷门}{leng3men2}{7,3}{⼎、⾨}
  \definition{s.}{uma profissão, ofício ou ramo de aprendizagem que recebe pouca atenção | um vencedor inesperado; azarão}
\end{EntryWithPhonetic}

\begin{EntryWithPhonetic}{冷气}{leng3 qi4}{7,4}{⼎、⽓}[HSK 6]
  \definition[股,阵]{s.}{ar frio (ou fresco); correntes de ar frio | ar condicionado; ar resfriado por equipamento de refrigeração | ar condicionado; equipamentos de ar condicionado}
\end{EntryWithPhonetic}

\begin{EntryWithPhonetic}{冷水}{leng3 shui3}{7,4}{⼎、⽔}[HSK 6]
  \definition[杯,瓶]{s.}{água fria | água não fervida}
\end{EntryWithPhonetic}

\begin{EntryWithPhonetic}{哩哩啦啦}{li1 li1 la1 la1}{10,10,11,11}{⼝、⼝、⼝、⼝}
  \definition{adj.}{espalhado; disperso; disseminado; difuso; esporádico; aqui e ali}
\end{EntryWithPhonetic}

\begin{EntryWithPhonetic}{厘}{li2}{9}{⼚}
  \definition*{s.}{Sobrenome Li}
  \definition{clas.}{li, uma unidade tradicional de comprimento, igual a 0,001 chi (市尺), e equivalente a 0,333 milímetro ou 0,013 polegada | li, uma unidade tradicional de peso, igual a 0,0001 jin (市斤), e equivalente a 5 centigramas ou 0,771 grãos | li, uma unidade tradicional de área, igual a 0,01 mu (市亩), e equivalente a 0,667 metro quadrado ou 0,797 jarda quadrada | li, unidade monetária chinesa, igual a 0,1 fen ou 0,001 yuan | li, unidade de taxa de juros, igual a 0,1\% de juros mensais ou 1\% de juros anuais | quantidade muito pequena; fração; o mínimo}
  \definition{v.}{regular; retificar | administrar}
  \seealsoref{市尺}{shi4 chi3}
  \seealsoref{市斤}{shi4jin1}
  \seealsoref{市亩}{shi4mu3}
\end{EntryWithPhonetic}

\begin{EntryWithPhonetic}{厘米}{li2mi3}{9,6}{⼚、⽶}[HSK 4]
  \definition{clas.}{centímetro; unidade de comprimento, símbolo cm, 1 metro é igual a 100 centímetros}
\end{EntryWithPhonetic}

\begin{EntryWithPhonetic}{离}{li2}{10}{⼇}[HSK 2]
  \definition*{s.}{Um dos Oito Diagramas | Sobrenome Li}
  \definition{prep.}{(ser longe) de\dots até\dots}
  \definition{v.}{partir; separar-se; afastar-se; estar longe de | prescindir; dispensar; ser independente de | mudar de; desviar-se de | mudar de; desviar-se de; trair; ser incompatível}
\end{EntryWithPhonetic}

\begin{EntryWithPhonetic}{离不开}{li2 bu4 kai1}{10,4,4}{⼇、⼀、⼶}[HSK 4]
  \definition{v.}{não pode prescindir; ser inseparável de; não ser capaz de se separar ou deixar uma pessoa, coisa ou circunstância}
\end{EntryWithPhonetic}

\begin{EntryWithPhonetic}{离婚}{li2/hun1}{10,11}{⼇、⼥}[HSK 3]
  \definition{v.+compl.}{divórciar; romper um casamento; obter o divórcio}
\end{EntryWithPhonetic}

\begin{EntryWithPhonetic}{离开}{li2kai1}{10,4}{⼇、⼶}[HSK 2]
  \definition{v.}{deixar; partir; desviar-se; separar-se das pessoas, dos lugares e das coisas}
\end{EntryWithPhonetic}

\begin{EntryWithPhonetic}{梨}{li2}{11}{⽊}[HSK 5]
  \definition*{s.}{Sobrenome Li}
  \definition[个,只,斤,棵,种]{s.}{perira; árvore de pera | pera}
\end{EntryWithPhonetic}

\begin{EntryWithPhonetic}{黎}{li2}{15}{⿉}
  \definition*{s.}{Etnia Li, uma das minorias nacionais da província de Hainan | Sobrenome Li}
  \definition{adj.}{Literário: preto; escuro | Literário: numeroso}
  \definition{s.}{multidão; as massas; a população}
\end{EntryWithPhonetic}

\begin{EntryWithPhonetic}{礼}{li3}{5}{⽰}[HSK 5]
  \definition*{s.}{Sobrenome Li}
  \definition[份]{s.}{observâncias cerimoniais em geral; cerimônia; rito | cortesia; etiqueta; boas maneiras | presente; oferta}
\end{EntryWithPhonetic}

\begin{EntryWithPhonetic}{礼拜}{li3 bai4}{5,9}{⽰、⼿}[HSK 5]
  \definition[个]{s.}{dia da semana; usado em conjunto com 一, 二, 三, 四, 五, 六, 日(或天, indica um dia específico da semana | semana; referência à semana | domingo}
  \definition{v.}{prestar homenagem aos deuses que veneram; rezar; orar}
\end{EntryWithPhonetic}

\begin{EntryWithPhonetic}{礼节}{li3jie2}{5,5}{⽰、⾋}
  \definition{s.}{protocolo | cerimônia | etiqueta}
\end{EntryWithPhonetic}

\begin{EntryWithPhonetic}{礼貌}{li3mao4}{5,14}{⽰、⾘}[HSK 5]
  \definition{adj.}{educado; descreve uma pessoa que fala e age respeitando os outros, sem arrogância, de acordo com as exigências das relações sociais}
  \definition{s.}{cortesia; educação; boas maneiras}
\end{EntryWithPhonetic}

\begin{EntryWithPhonetic}{礼让}{li3rang4}{5,5}{⽰、⾔}
  \definition{s.}{cortesia}
  \definition{v.}{mostrar consideração por (outros) | ceder a (outro veículo, etc.)}
\end{EntryWithPhonetic}

\begin{EntryWithPhonetic}{礼堂}{li3 tang2}{5,11}{⽰、⼟}[HSK 6]
  \definition[个,座,处]{s.}{auditórios; salão de assembleias; um salão para reuniões ou cerimônias}
\end{EntryWithPhonetic}

\begin{EntryWithPhonetic}{礼物}{li3wu4}{5,8}{⽰、⽜}[HSK 2]
  \definition[份,件,个,分,些]{s.}{presente; lembrança; itens oferecidos como forma de respeito ou celebração, referindo-se de maneira geral a itens oferecidos como presente}
\end{EntryWithPhonetic}

\begin{EntryWithPhonetic}{李}{li3}{7}{⽊}
  \definition*{s.}{Sobrenome Li}
  \definition[棵]{s.}{ameixa | ameixeira}
\end{EntryWithPhonetic}

\begin{EntryWithPhonetic}{李四}{li3si4}{7,5}{⽊、⼞}
  \definition*{s.}{Li Si | Zé Ninguém | Nome para uma pessoa não especificada, 2 de 3}
  \seealsoref{王五}{wang2wu3}
  \seealsoref{张三}{zhang1san1}
\end{EntryWithPhonetic}

\begin{EntryWithPhonetic}{李子}{li3zi5}{7,3}{⽊、⼦}
  \definition[个]{s.}{ameixa}
\end{EntryWithPhonetic}

\begin{EntryWithPhonetic}{里}{li3}{7}{⾥}[HSK 1][Kangxi 166]
  \definition*{s.}{Sobrenome Li}
  \definition{clas.}{li, uma unidade chinesa de comprimento (= 1/2 quilômetro)}
  \definition{s.}{forro; revestimento; interior; parte de trás do tecido | interno; dentro; no interior | vizinhança; vizinhos | cidade natal; local de origem}
\end{EntryWithPhonetic}

\begin{EntryWithPhonetic}{里边}{li3 bian5}{7,5}{⾥、⾡}[HSK 1]
  \definition{s.}{em; dentro; no interior}
\end{EntryWithPhonetic}

\begin{EntryWithPhonetic}{里面}{li3 mian4}{7,9}{⾥、⾯}[HSK 3]
  \definition{s.}{dentro; interior}
\end{EntryWithPhonetic}

\begin{EntryWithPhonetic}{里斯本}{li3si1ben3}{7,12,5}{⾥、⽄、⽊}
  \definition*{s.}{Lisboa}
\end{EntryWithPhonetic}

\begin{EntryWithPhonetic}{里斯本大学}{li3si1ben3 da4xue2}{7,12,5,3,8}{⾥、⽄、⽊、⼤、⼦}
  \definition*{s.}{Universidade de Lisboa}
\end{EntryWithPhonetic}

\begin{EntryWithPhonetic}{里头}{li3 tou5}{7,5}{⾥、⼤}[HSK 2]
  \definition{s.}{dentro}
\end{EntryWithPhonetic}

\begin{EntryWithPhonetic}{哩}{li3}{10}{⼝}
  \definition{clas.}{milha (unidade de comprimento igual a 1.609,344 m)}
  \seeref{li5}
\end{EntryWithPhonetic}

\begin{EntryWithPhonetic}{理}{li3}{11}{⽟}[HSK 6]
  \definition*{s.}{Sobrenome Li}
  \definition{s.}{textura; grão (em madeira, pele, etc.) | ordem; sequência | razão; lógica; verdade | ciências naturais (especialmente física)}
  \definition{v.}{gerenciar; executar | colocar em ordem; arrumar | (geralmente no negativo) prestar atenção a; fazer um gesto ou falar com | tratar | colocar em ordem; limpar | tomar conhecimento de; prestar atenção a; expressar uma atitude; expressar uma opinião}
\end{EntryWithPhonetic}

\begin{EntryWithPhonetic}{理财}{li3 cai2}{11,7}{⽟、⾙}[HSK 6]
  \definition{v.}{administrar questões financeiras; conduzir transações financeiras; administrar propriedade; ser responsável pelo trabalho financeiro}
\end{EntryWithPhonetic}

\begin{EntryWithPhonetic}{理发}{li3/fa4}{11,5}{⽟、⼜}[HSK 3]
  \definition{v.+compl.}{cortar e aparar o cabelo; ter (dar) um corte de cabelo}
\end{EntryWithPhonetic}

\begin{EntryWithPhonetic}{理解}{li3jie3}{11,13}{⽟、⾓}[HSK 3]
  \definition{v.}{entender; compreender; compreender o significado por trás de algo através da reflexão e do aprendizado | entender com empatia; achar que os outros não conseguem fazer determinada coisa e demonstrar compaixão, perdão e não crítica}
\end{EntryWithPhonetic}

\begin{EntryWithPhonetic}{理论}{li3lun4}{11,6}{⽟、⾔}[HSK 3]
  \definition[套,个]{s.}{teoria; uma série de conclusões tiradas pelas pessoas sobre atividades naturais ou sociais}
  \definition{v.}{argumentar; raciocinar com alguém; discutir com outras pessoas sobre quem está certo ou errado}
\end{EntryWithPhonetic}

\begin{EntryWithPhonetic}{理想}{li3xiang3}{11,13}{⽟、⼼}[HSK 2]
  \definition{adj.}{ideal; perfeito | conforme o desejado; satisfatório}
  \definition{adv.}{idealmente}
  \definition[个,种]{s.}{ideal; sonho; aspiração}
\end{EntryWithPhonetic}

\begin{EntryWithPhonetic}{理由}{li3you2}{11,5}{⽟、⽥}[HSK 3]
  \definition[个,条,种,堆]{s.}{razão; justificativa; fundamento; a razão pela qual as coisas são feitas desta ou daquela maneira}
\end{EntryWithPhonetic}

\begin{EntryWithPhonetic}{理智}{li3zhi4}{11,12}{⽟、⽇}[HSK 6]
  \definition{adj.}{racional; sensato; cabeça fria; sóbrio; calmo}
  \definition{s.}{sentido; razão; intelecto; a capacidade de distinguir o certo do errado, analisar e julgar e controlar as emoções e o comportamento de acordo}
\end{EntryWithPhonetic}

\begin{EntryWithPhonetic}{力}{li4}{2}{⼒}[HSK 3][Kangxi 19]
  \definition*{s.}{Sobrenome Li}
  \definition{adj.}{forte; eficiente; capaz | forte; poderoso; referência geral à função das coisas}
  \definition{adv.}{energicamente; arduamente; vigorosamente; com todo o esforço; com toda a dedicação}
  \definition{s.}{força; energia; poder; (física) refere-se à ação de alterar o estado de movimento ou a forma de um objeto |poder; força; habilidade; capacidade; funções dos órgãos do corpo humano | força física; resistência física}
\end{EntryWithPhonetic}

\begin{EntryWithPhonetic}{力量}{li4liang5}{2,12}{⼒、⾥}[HSK 3]
  \definition[出]{s.}{força física; força espiritual | habilidade; capacidade | eficácia; efeito | força (pessoa ou grupo que tem muito poder ou influência); referência a uma pessoa ou grupo que pode desempenhar um papel importante}
\end{EntryWithPhonetic}

\begin{EntryWithPhonetic}{力气}{li4qi5}{2,4}{⼒、⽓}[HSK 4]
  \definition[把]{s.}{força física; eficiência muscular; força | esforço}
\end{EntryWithPhonetic}

\begin{EntryWithPhonetic}{历}{li4}{4}{⼚}
  \definition{adj.}{todas as anteriores (ocasiões, sessões, etc.)}
  \definition{adv.}{por toda parte; um por um}
  \definition{s.}{experiência; registro | almanaque; anuário; calendário}
  \definition{v.}{passar por; sofrer; experimentar | passar através; atravessar}
\end{EntryWithPhonetic}

\begin{EntryWithPhonetic}{历史}{li4shi3}{4,5}{⼚、⼝}[HSK 4]
  \definition[段]{s.}{história; registro do passado; processo de desenvolvimento da natureza e da sociedade humana; processo de desenvolvimento de uma coisa ou pessoa | história; eventos passados; experiência | história; refere-se ao tema da história}
\end{EntryWithPhonetic}

\begin{EntryWithPhonetic}{厉}{li4}{5}{⼚}
  \definition*{s.}{Sobrenome Li}
  \definition{adj.}{rigoroso; estrito | severo; sombrio; sério}
\end{EntryWithPhonetic}

\begin{EntryWithPhonetic}{厉害}{li4hai5}{5,10}{⼚、⼧}[HSK 5]
  \definition{adj.}{feroz; severo; descreve uma situação como sendo muito grave | severo; duro; descreve uma pessoa que é exigente com os outros, muito severa, muitas vezes deixando os outros um pouco assustados | incrível; talentoso; impressionante; usado para avaliar a capacidade de uma pessoa ou algo que ela fez que é notável | aterrorizante; assustador; descreve animais ferozes e assustadores}
\end{EntryWithPhonetic}

\begin{EntryWithPhonetic}{立}{li4}{5}{⽴}[HSK 5][Kangxi 117]
  \definition{adj.}{ereto; vertical; na vertical}
  \definition{adv.}{imediatamente; instantaneamente}
  \definition{v.}{ficar em pé, com os pés no chão ou apoiados em algum objeto; o objeto deve estar na vertical | erguer; colocar (ou levantar) algo; colocar em pé | encontrar; criar; elaborar; formular; estabelecer | configurar; fundar; estabelecer | viver; existir | ascender ao trono; antigamente, referia-se à ascensão ao trono de um monarca | nomear; designar; antigamente, significava estabelecer uma determinada posição ou status}
\end{EntryWithPhonetic}

\begin{EntryWithPhonetic}{立场}{li4chang3}{5,6}{⽴、⼟}[HSK 5]
  \definition[个]{s.}{posição; postura; a posição e a atitude adotadas ao reconhecer e lidar com os problemas | ponto de vista; refere-se especificamente à atitude de reconhecer e lidar com questões a partir dos interesses de uma determinada classe, ou seja, a posição de classe}
\end{EntryWithPhonetic}

\begin{EntryWithPhonetic}{立法}{li4fa3}{5,8}{⽴、⽔}
  \definition{s.}{legislação}
  \definition{v.}{promulgar leis | legislar}
\end{EntryWithPhonetic}

\begin{EntryWithPhonetic}{立即}{li4ji2}{5,7}{⽴、⼙}[HSK 4]
  \definition{adv.}{prontamente; imediatamente; de imediato}
\end{EntryWithPhonetic}

\begin{EntryWithPhonetic}{立刻}{li4ke4}{5,8}{⽴、⼑}[HSK 3]
  \definition{adv.}{imediatamente; de ​​uma vez; indica que algo acontecerá imediatamente após um determinado momento}
\end{EntryWithPhonetic}

\begin{EntryWithPhonetic}{利}{li4}{7}{⼑}[HSK 6]
  \definition*{s.}{Sobrenome Li}
  \definition{adj.}{afiado; cortante | favorável; conveniente; sem dificuldades; sem ou com poucas dificuldades}
  \definition{s.}{benefício; vantagem | lucro; ganhos; juros}
  \definition{v.}{beneficiar; tornar vantajoso}
\end{EntryWithPhonetic}

\begin{EntryWithPhonetic}{利润}{li4run4}{7,10}{⼑、⽔}[HSK 5]
  \definition[笔,份]{s.}{lucro; o dinheiro ganho com atividades comerciais e industriais}
\end{EntryWithPhonetic}

\begin{EntryWithPhonetic}{利息}{li4xi1}{7,10}{⼑、⼼}[HSK 4]
  \definition{s.}{acréscimo; juros; dinheiro recebido além do valor principal como resultado de depósitos ou empréstimos (diferenciado de 本金)}
  \seealsoref{本金}{ben3 jin1}
\end{EntryWithPhonetic}

\begin{EntryWithPhonetic}{利益}{li4yi4}{7,10}{⼑、⽫}[HSK 4]
  \definition[个,种]{s.}{ganho; lucro; juros; benefício}
\end{EntryWithPhonetic}

\begin{EntryWithPhonetic}{利用}{li4yong4}{7,5}{⼑、⽤}[HSK 3]
  \definition{v.}{usar; utilizar; fazer uso de; fazer com que algo ou alguém funcione bem| explorar; tirar vantagem de; usar meios para fazer com que pessoas ou coisas sirvam aos seus interesses}
\end{EntryWithPhonetic}

\begin{EntryWithPhonetic}{例}{li4}{8}{⼈}
  \definition{adj.}{regular; rotineiro}
  \definition{s.}{exemplo; instância | precedente | caso; instância | regras; estatutos; regulamentos}
  \definition{v.}{analogizar}
\end{EntryWithPhonetic}

\begin{EntryWithPhonetic}{例如}{li4ru2}{8,6}{⼈、⼥}[HSK 2]
  \definition{conj.}{por exemplo; tal como; como por exemplo; colocado antes do exemplo, indica que o exemplo vem a seguir}
\end{EntryWithPhonetic}

\begin{EntryWithPhonetic}{例外}{li4wai4}{8,5}{⼈、⼣}[HSK 5]
  \definition[个,种]{s.}{exceção; situações que não se enquadram nas regras gerais ou nas leis comuns}
  \definition{v.}{ser excepcional; ser uma exceção}
\end{EntryWithPhonetic}

\begin{EntryWithPhonetic}{例子}{li4 zi5}{8,3}{⼈、⼦}[HSK 2]
  \definition[个]{s.}{exemplo; algo usado para ajudar a explicar ou provar uma determinada situação ou afirmação}
\end{EntryWithPhonetic}

\begin{EntryWithPhonetic}{隶}{li4}{8}{⾪}[Kangxi 171]
  \definition*{s.}{Sobrenome Li}
  \definition{s.}{escravo; pessoa em servidão; pessoas escravizadas | Arcaico: corredor de cargo governamental na China feudal | um dos estilos antigos da caligrafia chinesa}
  \definition{v.}{estar subordinado a; estar afiliado a (ou com)}
\end{EntryWithPhonetic}

\begin{EntryWithPhonetic}{荔}{li4}{9}{⾋}
  \definition[颗]{s.}{lichia | (arcaico) uma espécie de grama semelhante à taboa}
\end{EntryWithPhonetic}

\begin{EntryWithPhonetic}{荔枝}{li4zhi1}{9,8}{⾋、⽊}
  \definition{s.}{lichia}
\end{EntryWithPhonetic}

\begin{EntryWithPhonetic}{鬲}{li4}{10}{⿀}[Kangxi 193]
  \definition{s.}{recipiente de cerâmica antigo com três pernas usado para cozinhar, com marcas de cordão na parte externa e pernas ocas}
  \seeref{ge2}
\end{EntryWithPhonetic}

\begin{EntryWithPhonetic}{詈}{li4}{12}{⾔}
  \definition{v.}{xingar; usar linguagem severa}
\end{EntryWithPhonetic}

\begin{EntryWithPhonetic}{詈骂}{li4ma4}{12,9}{⾔、⾺}
  \definition{v.}{xingar | abusar}
\end{EntryWithPhonetic}

\begin{EntryWithPhonetic}{哩}{li5}{10}{⼝}
  \definition{part.}{(dialeto) final modal semelhante a 呢 ou 啦, usado em um tom definido, mas um tanto exagerado}
  \seeref{li3}
  \seealsoref{啦}{la5}
  \seealsoref{呢}{ne5}
\end{EntryWithPhonetic}

\begin{EntryWithPhonetic}{俩}{lia3}{9}{⼈}[HSK 4]
  \definition{num.}{ambos; dois; contração de 两个 | alguns; vários; refere-se a um pequeno número}
\end{EntryWithPhonetic}

\begin{EntryWithPhonetic}{俩钱}{lia3qian2}{9,10}{⼈、⾦}
  \definition{s.}{uma pequena quantia de dinheiro}
\end{EntryWithPhonetic}

\begin{EntryWithPhonetic}{连}{lian2}{7}{⾡}[HSK 3]
  \definition*{s.}{Sobrenome Lian}
  \definition{adv.}{em sucessão; um após o outro; repetidamente}
  \definition{prep.}{incluindo; incluido | até mesmo}
  \definition[个]{s.}{companhia; unidades organizacionais das forças armadas}
  \definition{v.}{ligar; juntar; conectar | envolver-se (em problemas); implicar; incriminar | costurar; coser}
\end{EntryWithPhonetic}

\begin{EntryWithPhonetic}{连接}{lian2 jie1}{7,11}{⾡、⼿}[HSK 5]
  \definition[条]{s.}{conexão}
  \definition{v.}{ligar; unir; relacionar, conectar; anexar}
\end{EntryWithPhonetic}

\begin{EntryWithPhonetic}{连忙}{lian2mang2}{7,6}{⾡、⼼}[HSK 3]
  \definition{adv.}{imediatamente; de imediato; com pressa; apressadamente}
\end{EntryWithPhonetic}

\begin{EntryWithPhonetic}{连锁反应}{lian2suo3fan3ying4}{7,12,4,7}{⾡、⾦、⼜、⼴}
  \definition{s.}{reação em cadeia}
\end{EntryWithPhonetic}

\begin{EntryWithPhonetic}{连续}{lian2xu4}{7,11}{⾡、⽷}[HSK 3]
  \definition{adv.}{continuamente; sucessivamente; em uma fileira; um após o outro}
\end{EntryWithPhonetic}

\begin{EntryWithPhonetic}{连续剧}{lian2 xu4 ju4}{7,11,10}{⾡、⽷、⼑}[HSK 3]
  \definition[部,集]{s.}{série; novela; drama dividido em vários episódios, transmitido continuamente pela rádio ou televisão, com enredo contínuo}
\end{EntryWithPhonetic}

\begin{EntryWithPhonetic}{帘}{lian2}{8}{⼱}
  \definition[块,个]{s.}{bandeira em mastro sobre adega; bandeira como placa de loja | cortina; tela de bambu ou tecido; objetos para cobrir portas e janelas}
\end{EntryWithPhonetic}

\begin{EntryWithPhonetic}{莲}{lian2}{10}{⾋}
  \definition*{s.}{Sobrenome Lian}
  \definition[粒]{s.}{lótus}
\end{EntryWithPhonetic}

\begin{EntryWithPhonetic}{莲花}{lian2hua1}{10,7}{⾋、⾋}
  \definition{s.}{flor de lótus | lírio aquático}
\end{EntryWithPhonetic}

\begin{EntryWithPhonetic}{莲藕}{lian2'ou3}{10,18}{⾋、⾋}
  \definition{s.}{raiz de Lotus}
\end{EntryWithPhonetic}

\begin{EntryWithPhonetic}{联}{lian2}{12}{⽿}
  \definition{s.}{dísticos (antitéticos)}
  \definition{v.}{aliar-se a; unir-se; juntar-se a}
\end{EntryWithPhonetic}

\begin{EntryWithPhonetic}{联合}{lian2he2}{12,6}{⽿、⼝}[HSK 3]
  \definition{adj.}{conjunto; unido; federal; combinado}
  \definition{s.}{aliado; união; aliança; conectar-se ou unir-se para agir em conjunto}
\end{EntryWithPhonetic}

\begin{EntryWithPhonetic}{联合国}{lian2 he2 guo2}{12,6,8}{⽿、⼝、⼞}[HSK 3]
  \definition*{s.}{Nações Unidas; Organização internacional fundada em 1945, após o fim da Segunda Guerra Mundial, com sede em Nova Iorque, Estados Unidos ; as suas principais instituições são a Assembleia Geral, o Conselho de Segurança, o Conselho Econômico e Social e o Secretariado; de acordo com a Carta das Nações Unidas, os seus principais objetivos são manter a paz e a segurança internacionais, desenvolver relações amigáveis entre os países e promover a cooperação internacional nas áreas econômica e cultural}
\end{EntryWithPhonetic}

\begin{EntryWithPhonetic}{联合会}{lian2he2hui4}{12,6,6}{⽿、⼝、⼈}
  \definition{s.}{federação}
\end{EntryWithPhonetic}

\begin{EntryWithPhonetic}{联络}{lian2luo4}{12,9}{⽿、⽷}[HSK 5]
  \definition{v.}{entrar em contato; comunicar-se; entrar em contato com}
\end{EntryWithPhonetic}

\begin{EntryWithPhonetic}{联盟}{lian2meng2}{12,13}{⽿、⽫}[HSK 6]
  \definition{s.}{aliança; coalizão; liga; união}
\end{EntryWithPhonetic}

\begin{EntryWithPhonetic}{联赛}{lian2 sai4}{12,14}{⽿、⾙}[HSK 6]
  \definition{s.}{jogos da liga | liga (esportiva) | torneio da liga}
\end{EntryWithPhonetic}

\begin{EntryWithPhonetic}{联手}{lian2 shou3}{12,4}{⽿、⼿}[HSK 6]
  \definition{v.}{dar as mãos; cooperar | Literário: dar as mãos | agir em conjunto}
\end{EntryWithPhonetic}

\begin{EntryWithPhonetic}{联系}{lian2xi4}{12,7}{⽿、⽷}[HSK 3]
  \definition[个,种,层]{s.}{relacionamento; relacionamento entre duas coisas}
  \definition{v.}{entrar em contato; contatar; comunicar-se com alguém por telefone, e-mail ou carta | agendar; entrar em contato com; estabelecer algum tipo de relação com a outra parte | relacionar; combinar; integrar}
\end{EntryWithPhonetic}

\begin{EntryWithPhonetic}{联想}{lian2xiang3}{12,13}{⽿、⼼}[HSK 5]
  \definition*{s.}{Lenovo (empresa)}
  \definition{v.}{associar-se a; estabelecer uma conexão mental; lembrar-se de algo; lembrar-se de outras pessoas ou coisas relacionadas devido a alguém ou algo; evocar outros conceitos relacionados devido a um determinado conceito}
\end{EntryWithPhonetic}

\begin{EntryWithPhonetic}{脸}{lian3}{11}{⾁}[HSK 2]
  \definition[张,个]{s.}{rosto (de pessoas ou animais); a parte frontal da cabeça, da testa ao queixo | parte frontal de algo | cara; autoestima; aparência | rosto; expressões faciais}
\end{EntryWithPhonetic}

\begin{EntryWithPhonetic}{脸盆}{lian3 pen2}{11,9}{⾁、⽫}[HSK 5]
  \definition[个]{s.}{lavatório; bacia para lavar as mãos e o rosto}
\end{EntryWithPhonetic}

\begin{EntryWithPhonetic}{脸色}{lian3 se4}{11,6}{⾁、⾊}[HSK 5]
  \definition{s.}{aparência; tez; cor da pele | aparência; expressão facial | (indicando a condição física de alguém) aparência; cor}
\end{EntryWithPhonetic}

\begin{EntryWithPhonetic}{练}{lian4}{8}{⽷}[HSK 2]
  \definition*{s.}{Sobrenome Lian}
  \definition{adj.}{habilidoso; experiente; bem treinado}
  \definition{s.}{seda branca}
  \definition{v.}{tratar, amaciar e branquear a seda por meio de fervura; cozinhar seda crua ou tecidos de seda crua | treinar; praticar; exercitar}
\end{EntryWithPhonetic}

\begin{EntryWithPhonetic}{练习}{lian4xi2}{8,3}{⽷、⼄}[HSK 2]
  \definition[项,次]{s.}{exercício (em livros); tarefas ou exercícios organizados para consolidar os resultados da aprendizagem}
  \definition{v.}{praticar; exercitar; repitir várias vezes até ficar bem treinado}
\end{EntryWithPhonetic}

\begin{EntryWithPhonetic}{炼}{lian4}{9}{⽕}
  \definition{v.}{fundir; refinar | temperar (um metal) com fogo | pesar a palavra; procurar a frase certa; polir | trabalhar; tornar uma substância pura ou resistente por aquecimento, etc. | polir; fazer as palavras bonitas e concisas}
\end{EntryWithPhonetic}

\begin{EntryWithPhonetic}{恋}{lian4}{10}{⼼}
  \definition*{s.}{Sobrenome Lian}
  \definition{v.}{amor (romântico) | ansiar por; sentir-se apegado a | amar; apaixonar-se por | não querendo se separar de; sentir sua falta para sempre; não suportar ficar separado}
\end{EntryWithPhonetic}

\begin{EntryWithPhonetic}{恋爱}{lian4'ai4}{10,10}{⼼、⽖}[HSK 5]
  \definition[个,场,段]{s.}{namoro; afeto; amor romântico; ações que demonstram o amor mútuo}
  \definition{v.}{amar; estar apaixonado}
\end{EntryWithPhonetic}

\begin{EntryWithPhonetic}{良}{liang2}{7}{⾉}
  \definition*{s.}{Sobrenome Liang}
  \definition{adj.}{bom; ótimo; agradável}
  \definition{adv.}{muito; muito mesmo; de fato}
  \definition{s.}{boas pessoas; pessoas gentis; talentos excepcionais}
\end{EntryWithPhonetic}

\begin{EntryWithPhonetic}{良好}{liang2hao3}{7,6}{⾉、⼥}[HSK 4]
  \definition{adj.}{bom; ótimo; bem; satisfatório}
\end{EntryWithPhonetic}

\begin{EntryWithPhonetic}{良田}{liang2tian2}{7,5}{⾉、⽥}
  \definition{s.}{terra agrícola boa | terra fértil}
\end{EntryWithPhonetic}

\begin{EntryWithPhonetic}{良心}{liang2xin1}{7,4}{⾉、⼼}
  \definition{s.}{consciência}
\end{EntryWithPhonetic}

\begin{EntryWithPhonetic}{凉}{liang2}{10}{⼎}[HSK 2]
  \definition{adj.}{frio; gelado; ligeiramente fria (menos do que 冷) | sombrio; desolado; sem animação | desanimado; desapontado | usado para prevenir o calor e manter a temperatura amena; para proteção contra o calor}
  \definition{s.}{frio; refere-se a um ambiente fresco ou a uma brisa fresca}
  \seeref{liang4}
  \seealsoref{冷}{leng3}
\end{EntryWithPhonetic}

\begin{EntryWithPhonetic}{凉快}{liang2kuai5}{10,7}{⼎、⼼}[HSK 2]
  \definition{adj.}{fresco; refrescante}
  \definition{v.}{refrescar; refrescar-se; deixar o corpo fresco e revigorado}
\end{EntryWithPhonetic}

\begin{EntryWithPhonetic}{凉水}{liang2 shui3}{10,4}{⼎、⽔}[HSK 3]
  \definition{s.}{água fria; água não aquecida | água não fervida}
\end{EntryWithPhonetic}

\begin{EntryWithPhonetic}{凉鞋}{liang2 xie2}{10,15}{⼎、⾰}[HSK 6]
  \definition[双,只]{s.}{sandália; alpargata; alpercata; alparca ; sapatos de verão com cabedal ventilado}
\end{EntryWithPhonetic}

\begin{EntryWithPhonetic}{量}{liang2}{12}{⾥}[HSK 4]
  \definition{v.}{medir | estimar; dimensionar}
  \seeref{liang4}
\end{EntryWithPhonetic}

\begin{EntryWithPhonetic}{粮}{liang2}{13}{⽶}
  \definition[斤,粒]{s.}{grãos; alimentos; provisões | imposto sobre grãos | nutrição | imposto agrícola; grãos como imposto agrícola}
\end{EntryWithPhonetic}

\begin{EntryWithPhonetic}{粮食}{liang2shi5}{13,9}{⽶、⾷}[HSK 4]
  \definition[种,吨,袋,颗,粒]{s.}{alimentos; grãos; termo geral para os vários tipos de arroz, feijão, etc. que podem ser consumidos}
\end{EntryWithPhonetic}

\begin{EntryWithPhonetic}{两}{liang3}{7}{⼀}[HSK 1,2]
  \definition*{s.}{Sobrenome Liang}
  \definition{clas.}{liang, uma unidade de peso (=50 gramas)}
  \definition{num.}{dois (sempre usado antes de classificadores) | poucos; alguns; indica um número indeterminado}
  \definition{s.}{ambos (lados); qualquer (lado)}
\end{EntryWithPhonetic}

\begin{EntryWithPhonetic}{两岸}{liang3 an4}{7,8}{⼀、⼭}[HSK 5]
  \definition{s.}{ambos os lados; ambas as margens; ambas as costas; entre os dois lados do estreito; bilateral}
\end{EntryWithPhonetic}

\begin{EntryWithPhonetic}{两边}{liang3 bian1}{7,5}{⼀、⾡}[HSK 4]
  \definition{s.}{ambos os lados; ambas as direções; ambos os lugares | ambas as partes; ambos os lados}
\end{EntryWithPhonetic}

\begin{EntryWithPhonetic}{两侧}{liang3 ce4}{7,8}{⼀、⼈}[HSK 6]
  \definition{s.}{dois flancos; dois (ambos) lados; ambos}
\end{EntryWithPhonetic}

\begin{EntryWithPhonetic}{两码事}{liang3ma3shi4}{7,8,8}{⼀、⽯、⼅}
  \definition{expr.}{duas coisas completamente diferentes; dois assuntos diferentes}
\end{EntryWithPhonetic}

\begin{EntryWithPhonetic}{两手}{liang3 shou3}{7,4}{⼀、⼿}[HSK 6]
  \definition{s.}{ambas as mãos | ambos os aspectos; táticas duplas | Coloquial: habilidade; capacidade}
\end{EntryWithPhonetic}

\begin{EntryWithPhonetic}{亮}{liang4}{9}{⼇}[HSK 2]
  \definition*{s.}{Sobrenome Lian}
  \definition{adj.}{brilhante; claro | alto e claro; retumbante | esclarecido; aberto e claro}
  \definition{s.}{luz}
  \definition{v.}{iluminar; clarear; brilhar | elevar a voz; ressoar; tornar o som mais alto | revelar; mostrar; aparecer; exibir}
\end{EntryWithPhonetic}

\begin{EntryWithPhonetic}{凉}{liang4}{10}{⼎}
  \definition{v.}{deixar algo esfriar; deixar um objeto quente descansar por um tempo para que a temperatura diminua}
  \seeref{liang2}
\end{EntryWithPhonetic}

\begin{EntryWithPhonetic}{辆}{liang4}{11}{⾞}[HSK 2]
  \definition{clas.}{usado para automóveis, veículos, etc.}
\end{EntryWithPhonetic}

\begin{EntryWithPhonetic}{量}{liang4}{12}{⾥}
  \definition{s.}{instrumento de medida; antigamente, o termo se referia a objetos como baldes e litros, que medem o volume | capacidade (para tolerância ou ingestão de alimentos ou bebidas); refere-se ao limite do que pode ser acomodado | quantidade; valor; volume; número}
  \definition{v.}{estimar; medir; pesar}
  \seeref{liang2}
\end{EntryWithPhonetic}

\begin{EntryWithPhonetic}{疗}{liao2}{7}{⽧}
  \definition{v.}{tratar; curar | recuperar}
\end{EntryWithPhonetic}

\begin{EntryWithPhonetic}{疗养}{liao2 yang3}{7,9}{⽧、⼋}[HSK 4]
  \definition{v.}{recuperar; convalescer; tratar pessoas com doenças crônicas ou debilitantes em instituições médicas especializadas com foco na recuperação}
\end{EntryWithPhonetic}

\begin{EntryWithPhonetic}{聊}{liao2}{11}{⽿}[HSK 6]
  \definition*{s.}{Sobrenome Liao}
  \definition{adv.}{apenas; meramente; provisoriamente; por enquanto | um pouco; ligeiramente}
  \definition{v.}{tagarelar; conversar; bater papo | confiar (ou depender, recorrer) a}
\end{EntryWithPhonetic}

\begin{EntryWithPhonetic}{聊天}{liao2/tian1}{11,4}{⽿、⼤}
  \definition{v.+compl.}{papear | bater papo}
\end{EntryWithPhonetic}

\begin{EntryWithPhonetic}{聊天儿}{liao2/tian1r5}{11,4,2}{⽿、⼤、⼉}[HSK 6]
  \definition{v.+compl.}{conversar; fofocar; bater papo; duas ou mais pessoas conversando sem um tópico ou propósito específico}
\end{EntryWithPhonetic}

\begin{EntryWithPhonetic}{了}{liao3}{2}{⼅}
  \definition*{s.}{Sobrenome Liao}
  \definition{adv.}{inteiramente; um pouco; totalmente (mais usado em negativas)}
  \definition{v.}{terminar; concluir; encerrar; cumprir; eliminar; resolver | compreender; saber; perceber; saber claramente | expressar possibilidade ou impossibilidade; usado com 得 ou 不 após o verbo, indica possibilidade ou impossibilidade}
  \seeref{le5}
  \seealsoref{不}{bu4}
  \seealsoref{得}{de5}
\end{EntryWithPhonetic}

\begin{EntryWithPhonetic}{了不起}{liao3bu5qi3}{2,4,10}{⼅、⼀、⾛}[HSK 4]
  \definition{adj.}{incrível; fantástico; extraordinário | sério; grave}
\end{EntryWithPhonetic}

\begin{EntryWithPhonetic}{了解}{liao3jie3}{2,13}{⼅、⾓}[HSK 4]
  \definition{v.}{entender; compreender | investigar; indagar sobre}
\end{EntryWithPhonetic}

\begin{EntryWithPhonetic}{料}{liao4}{10}{⽃}[HSK 6]
  \definition{clas.}{usado na medicina tradicional chinesa para preparar pílulas | unidade usada para calcular um pedaço de madeira, é a seção transversal em ambas as extremidades, que é de 1 pé (quadrado) com 7 pés de comprimento}
  \definition{s.}{material; coisa | (grão) alimento; forragem | artigos de vidro; vidros coloridos opacos | (para pílulas de medicina chinesa) prescrição}
  \definition{v.}{supor; esperar; antecipar | gerenciar; cuidar de | prever}
\end{EntryWithPhonetic}

\begin{EntryWithPhonetic}{列}{lie4}{6}{⼑}[HSK 4]
  \definition*{s.}{Sobrenome Lie}
  \definition{clas.}{usado para coisas em linhas e colunas}
  \definition{pron.}{cada um e todos; cada; muito}
  \definition{s.}{linha; arquivo; classificação (oposto a 行) | classificação; escopo | ranque | tipo}
  \definition{v.}{organizar; alinhar; colocar em ordem | listar; inserir em uma lista; classificar | formar uma linha}
  \seealsoref{行}{hang2}
\end{EntryWithPhonetic}

\begin{EntryWithPhonetic}{列车}{lie4che1}{6,4}{⼑、⾞}[HSK 4]
  \definition[列,班,趟,辆,节]{s.}{trem; trem em uma composição contínua, puxado por uma locomotiva e equipado com uma tripulação e marcações prescritas; geralmente um trem de passageiros}
\end{EntryWithPhonetic}

\begin{EntryWithPhonetic}{列入}{lie4 ru4}{6,2}{⼑、⼊}[HSK 4]
  \definition{v.}{listar; entrar em uma lista; ser incluído em | incluir em uma lista; juntar-se; registrar-se}
\end{EntryWithPhonetic}

\begin{EntryWithPhonetic}{列为}{lie4 wei2}{6,4}{⼑、⼂}[HSK 4]
  \definition{v.}{ser classificado como; ser listado como}
\end{EntryWithPhonetic}

\begin{EntryWithPhonetic}{烈}{lie4}{10}{⽕}
  \definition*{s.}{Sobrenome Lie}
  \definition{adj.}{forte; violento; intenso; feroz | justo; severo | firme; convicto; rigoroso}
  \definition{s.}{pessoa que morreu por uma causa justa | conquistas; façanhas | mártir sacrificando-se por uma causa justa}
\end{EntryWithPhonetic}

\begin{EntryWithPhonetic}{烈士}{lie4shi4}{10,3}{⽕、⼠}
  \definition{s.}{mártir}
\end{EntryWithPhonetic}

\begin{EntryWithPhonetic}{猎}{lie4}{11}{⽝}
  \definition[个]{s.}{traje de caça}
  \definition{v.}{caçar | procurar; perseguir}
\end{EntryWithPhonetic}

\begin{EntryWithPhonetic}{猎物}{lie4wu4}{11,8}{⽝、⽜}
  \definition{s.}{presa (vítima de um predador)}
\end{EntryWithPhonetic}

\begin{EntryWithPhonetic}{裂}{lie4}{12}{⾐}[HSK 6]
  \definition{s.}{entalhe; incisão; entalhes grandes e profundos nas bordas das folhas ou corolas | brecha; lacuna; rachadura; refere-se à rachadura ou divisão que aparece na superfície ou no interior de um objeto}
  \definition{v.}{dividir; rachar; rasgar | (figurativo) quebrar; esmagar; arruinar}
\end{EntryWithPhonetic}

\begin{EntryWithPhonetic}{邻}{lin2}{7}{⾢}
  \definition{adj.}{vizinho; perto; adjacente; perto; próximo}
  \definition{s.}{vizinho | bairro; vizinhança}
\end{EntryWithPhonetic}

\begin{EntryWithPhonetic}{邻居}{lin2ju1}{7,8}{⾢、⼫}[HSK 5]
  \definition[个,位,名,家]{s.}{vizinho; pessoas ou famílias que moram muito perto}
\end{EntryWithPhonetic}

\begin{EntryWithPhonetic}{临}{lin2}{9}{⼁}
  \definition*{s.}{Sobrenome Lin}
  \definition{adv.}{pouco antes; prestes a; no ponto de; indica que uma ação está prestes a ocorrer}
  \definition{v.}{encarar; enfrentar; aproximar-se | chegar; estar presente | copiar (um modelo de caligrafia ou pintura); traçar sobre as palavras ou figuras | olhar de cima para baixo | ir de cima para baixo}
\end{EntryWithPhonetic}

\begin{EntryWithPhonetic}{临近}{lin2jin4}{9,7}{⼁、⾡}
  \definition{v.}{aproximar-se; estar perto de}
\end{EntryWithPhonetic}

\begin{EntryWithPhonetic}{临时}{lin2shi2}{9,7}{⼁、⽇}[HSK 4]
  \definition{adj.}{temporário; provisório; por um breve período}
  \definition{adv.}{no momento em que algo acontece (quando as coisas dão errado)}
\end{EntryWithPhonetic}

\begin{EntryWithPhonetic}{淋}{lin2}{11}{⽔}
  \definition{v.}{borrifar | pingar | derramar | encharcar}
  \seeref{lin4}
\end{EntryWithPhonetic}

\begin{EntryWithPhonetic}{淋}{lin4}{11}{⽔}
  \definition{s.}{gonorréia}
  \definition{v.}{filtrar | coar | drenar}
  \seeref{lin2}
\end{EntryWithPhonetic}

\begin{EntryWithPhonetic}{令}{ling2}{5}{⼈}
  \definition*{s.}{Antigo nome geográfico, na região atual de Linyi, província de Shanxi | Sobrenome Ling}
  \seeref{ling3}
  \seeref{ling4}
\end{EntryWithPhonetic}

\begin{EntryWithPhonetic}{灵}{ling2}{7}{⽕}
  \definition*{s.}{Sobrenome Ling}
  \definition{adj.}{rápido; inteligente; afiado | eficaz; efetivo | flexível; hábil}
  \definition{s.}{espírito; alma | inteligência; mente | fada; duende; elfo | restos mortais do falecido; esquife | carro funerário; caixão ou algo relacionado aos mortos}
\end{EntryWithPhonetic}

\begin{EntryWithPhonetic}{灵感}{ling2gan3}{7,13}{⽕、⼼}
  \definition{s.}{inspiração | explosão de criatividade em empreendimento científico ou artístico}
\end{EntryWithPhonetic}

\begin{EntryWithPhonetic}{灵魂}{ling2hun2}{7,13}{⽕、⿁}
  \definition{s.}{alma | espírito}
\end{EntryWithPhonetic}

\begin{EntryWithPhonetic}{灵活}{ling2huo2}{7,9}{⽕、⽔}[HSK 6]
  \definition[种,点,些]{adj.}{ágil; rápido; ligeiro; descreve a capacidade de fazer rapidamente mudanças apropriadas com base na situação ao lidar com as coisas | flexível; elástico; descreve reações rápidas, como movimentos e funções cerebrais}
\end{EntryWithPhonetic}

\begin{EntryWithPhonetic}{铃}{ling2}{10}{⾦}[HSK 5]
  \definition[串,个]{s.}{sino; instrumento musical feito de metal | objetos em forma de sino | cápsula; botão; broto}
\end{EntryWithPhonetic}

\begin{EntryWithPhonetic}{铃声}{ling2 sheng1}{10,7}{⾦、⼠}[HSK 5]
  \definition{s.}{o tilintar de sinos; o som de um sino tocando}
\end{EntryWithPhonetic}

\begin{EntryWithPhonetic}{陵}{ling2}{10}{⾩}
  \definition*{s.}{Sobrenome Ling}
  \definition{s.}{colina; monte | túmulo imperial; mausoléu}
  \definition{v.}{(literário) intimidar; violar}
\end{EntryWithPhonetic}

\begin{EntryWithPhonetic}{陵园}{ling2yuan2}{10,7}{⾩、⼞}
  \definition{s.}{cemitério}
\end{EntryWithPhonetic}

\begin{EntryWithPhonetic}{菱}{ling2}{11}{⾋}
  \definition{s.}{maruca; caltrop aquático; castanha d'água}
\end{EntryWithPhonetic}

\begin{EntryWithPhonetic}{菱角}{ling2jiao5}{11,7}{⾋、⾓}
  \definition{s.}{castanha d'água}
\end{EntryWithPhonetic}

\begin{EntryWithPhonetic}{零}{ling2}{13}{⾬}[HSK 1]
  \definition*{s.}{Sobrenome Ling}
  \definition{adj.}{ímpar; dispersos; fragmentados (em oposição a 整)}
  \definition{num.}{zero; 0; também grafado como 〇; representa um número menor que qualquer número positivo e maior que qualquer número negativo; representa a ausência de quantidade | zero grau no termômetro | usado para indicar qualidade, comprimento, tempo, idade, etc. Entre dois dígitos, indica que a quantidade da unidade mais alta é acompanhada pela quantidade da unidade mais baixa | sinal de zero (0); nulo; espaço em branco para indicar números em caracteres chineses maiúsculos}
  \definition{s.}{fragmento; fração; lote ímpar; um número fracionário que não é suficiente para uma determinada unidade; um ponto decimal diferente de um inteiro}
  \definition{v.}{(de chuva, lágrimas, etc.) cair | murchar e cair}
  \seealsoref{整}{zheng3}
\end{EntryWithPhonetic}

\begin{EntryWithPhonetic}{零散}{ling2san3}{13,12}{⾬、⽁}
  \definition{adj.}{espalhado; disperso}
\end{EntryWithPhonetic}

\begin{EntryWithPhonetic}{零食}{ling2shi2}{13,9}{⾬、⾷}[HSK 4]
  \definition[包,袋,盒,箱,堆]{s.}{lanches; refrescos; petiscos entre as refeições; alimentação esporádica, além das refeições normais}
\end{EntryWithPhonetic}

\begin{EntryWithPhonetic}{零下}{ling2 xia4}{13,3}{⾬、⼀}[HSK 2]
  \definition{s.}{abaixo de zero; negativo}
\end{EntryWithPhonetic}

\begin{EntryWithPhonetic}{令}{ling3}{5}{⼈}
  \definition{clas.}{resma (de papel); unidade de medida de papel: 500 folhas inteiras de papel original produzidas mecanicamente equivalem a 1 resma}
  \seeref{ling2}
  \seeref{ling4}
\end{EntryWithPhonetic}

\begin{EntryWithPhonetic}{岭}{ling3}{8}{⼭}
  \definition{s.}{cordilheira}
\end{EntryWithPhonetic}

\begin{EntryWithPhonetic}{领}{ling3}{11}{⾴}[HSK 3]
  \definition{clas.}{usado para roupas, mantos, esteiras, tapetes, telas, etc.}
  \definition{s.}{pescoço; gargalo | gola; colarinho; faixa de pescoço | esboço; ponto principal; essência}
  \definition{v.}{conduzir; guiar; orientar | possuir; ser o possuidor de; ter jurisdição sobre | obter; conseguir; receber (o que foi distribuído) | aceitar; tomar |entender; compreender (o significado)}
\end{EntryWithPhonetic}

\begin{EntryWithPhonetic}{领带}{ling3 dai4}{11,9}{⾴、⼱}[HSK 5]
  \definition[条]{s.}{colar; gargantilha; gravata}
\end{EntryWithPhonetic}

\begin{EntryWithPhonetic}{领导}{ling3dao3}{11,6}{⾴、⼨}[HSK 3]
  \definition[个,位,名,些]{s.}{líder; liderança; pessoa que ocupa uma posição de liderança}
  \definition{v.}{liderar; exercer liderança; (elogio) liderar, gerenciar outras pessoas;  trabalhar com outras pessoas ou avançar em direção a um objetivo}
\end{EntryWithPhonetic}

\begin{EntryWithPhonetic}{领情}{ling3/qing2}{11,11}{⾴、⼼}
  \definition{v.+compl.}{sentir-se grato a alguém}
\end{EntryWithPhonetic}

\begin{EntryWithPhonetic}{领取}{ling3 qu3}{11,8}{⾴、⼜}[HSK 6]
  \definition{v.}{sacar; receber; obter; receber o que lhe é enviado}
\end{EntryWithPhonetic}

\begin{EntryWithPhonetic}{领先}{ling3xian1}{11,6}{⾴、⼉}[HSK 3]
  \definition{v.}{liderar; assumir a liderança; estar na liderança; (velocidade, desempenho, etc.) superar pessoas ou coisas semelhantes, estar na vanguarda}
\end{EntryWithPhonetic}

\begin{EntryWithPhonetic}{领袖}{ling3xiu4}{11,10}{⾴、⾐}[HSK 6]
  \definition[个,位,名]{s.}{líder de estados, grupos políticos, organizações de massa, etc.}
\end{EntryWithPhonetic}

\begin{EntryWithPhonetic}{令}{ling4}{5}{⼈}[HSK 5]
  \definition{adj.}{bom; excelente | termos de cortesia usados para se referir aos familiares e parentes da outra pessoa}
  \definition{s.}{ordem; decreto; comando; ordem emitida pela autoridade superior | um título oficial; administradores de certos departamentos governamentais na antiguidade | temporada; estação; clima e fenologia de uma determinada estação | poema-canção; letra curta}
  \definition{v.}{ordenar; comandar | fazer com que alguém; fazer com que; permitir que}
  \seeref{ling2}
  \seeref{ling3}
\end{EntryWithPhonetic}

\begin{EntryWithPhonetic}{令人}{ling4ren2}{5,2}{⼈、⼈}
  \definition{v.}{causar alguém (a fazer alguma coisa) | fazer alguém ficar zangado, encantado, etc.}
\end{EntryWithPhonetic}

\begin{EntryWithPhonetic}{另}{ling4}{5}{⼝}[HSK 6]
  \definition*{s.}{Sobrenome Ling}
  \definition{adv.}{além disso; indica que está fora do escopo da declaração | no lugar de; em vez de}
  \definition{pron.}{(com substantivos) outro; diferente; refere-se a pessoas ou coisas fora do escopo do que é dito}
\end{EntryWithPhonetic}

\begin{EntryWithPhonetic}{另外}{ling4wai4}{5,5}{⼝、⼣}[HSK 3]
  \definition{adv.}{além disso; em adição; ademais; além do mais; além de que; além do que já foi dito}
  \definition{conj.}{além disso; usada entre duas ou mais frases, indica algo além do que foi mencionado anteriormente}
  \definition{pron.}{outro; além das pessoas ou coisas mencionadas anteriormente}
\end{EntryWithPhonetic}

\begin{EntryWithPhonetic}{另一方面}{ling4 yi4 fang1 mian4}{5,1,4,9}{⼝、⼀、⽅、⾯}[HSK 3]
  \definition{adv./conj.}{outro aspecto | por outro lado; por sua vez; em contrapartida}
\end{EntryWithPhonetic}

\begin{EntryWithPhonetic}{刘}{liu2}{6}{⼑}
  \definition*{s.}{Sobrenome Liu}
  \definition{s.}{Clássico: um tipo de machado de batalha}
  \definition{v.}{matar; massacrar}
\end{EntryWithPhonetic}

\begin{EntryWithPhonetic}{流}{liu2}{10}{⽔}[HSK 2]
  \definition*{s.}{Sobrenome Liu}
  \definition{adj.}{fluente; tão suave quanto a água corrente}
  \definition{clas.}{lúmen; abreviação de lumens, 流明}
  \definition[名,个]{s.}{corrente de água | corrente; algo que se assemelha a um fluxo de água | razão; taxa; classe; grau; ramificação; facção; hierarquia}
  \definition{v.}{(de líquido) fluir | vaguear; vagar; mover-se de um lugar para outro; movimentar-se sem direção fixa | espalhar; circular; transmitir; divulgar | degenerar; mudar para pior; tender (aspecto negativo) | banir; enviar para o exílio | correr (ou fluir) como líquido; refere-se à parte do rio após deixar sua nascente (em contraste com a 源)}
  \seealsoref{流明}{liu2ming2}
  \seealsoref{源}{yuan2}
\end{EntryWithPhonetic}

\begin{EntryWithPhonetic}{流传}{liu2chuan2}{10,6}{⽔、⼈}[HSK 4]
  \definition[间]{v.}{espalhar; circular; passar adiante}
\end{EntryWithPhonetic}

\begin{EntryWithPhonetic}{流动}{liu2 dong4}{10,6}{⽔、⼒}[HSK 5]
  \definition{v.}{(água, ar, etc.) fluir; correr; circular | ir de um lugar para outro; estar em movimento; ser móvel (oposto a 固定)}
  \seealsoref{固定}{gu4ding4}
\end{EntryWithPhonetic}

\begin{EntryWithPhonetic}{流感}{liu2 gan3}{10,13}{⽔、⼼}[HSK 6]
  \definition{s.}{gripe; influenza; abreviação de 流行性感冒}
  \seealsoref{流行性感冒}{liu2xing2 xing4 gan3mao4}
\end{EntryWithPhonetic}

\begin{EntryWithPhonetic}{流利}{liu2li4}{10,7}{⽔、⼑}[HSK 2]
  \definition{adj.}{fluente; suave; lúcido; falar e escrever com fluência e clareza | com fluência; sem dificuldades}
\end{EntryWithPhonetic}

\begin{EntryWithPhonetic}{流明}{liu2ming2}{10,8}{⽔、⽇}
  \definition{s.}{(empréstimo linguístico) lúmen (unidade de fluxo luminoso)}
\end{EntryWithPhonetic}

\begin{EntryWithPhonetic}{流水}{liu2shui3}{10,4}{⽔、⽔}
  \definition{s.}{água corrente | (negócio) rotatividade}
\end{EntryWithPhonetic}

\begin{EntryWithPhonetic}{流通}{liu2tong1}{10,10}{⽔、⾡}[HSK 5]
  \definition{v.}{(ar, dinheiro, mercadorias, etc.) fluir; circular}
\end{EntryWithPhonetic}

\begin{EntryWithPhonetic}{流星}{liu2xing1}{10,9}{⽔、⽇}
  \definition{s.}{meteoro | estrela cadente}
\end{EntryWithPhonetic}

\begin{EntryWithPhonetic}{流行}{liu2xing2}{10,6}{⽔、⾏}[HSK 2]
  \definition{adj.}{popular; na moda; muito popular}
  \definition{v.}{ser popular; prevalecer; espalhar-se amplamente; divulgar amplamente}
\end{EntryWithPhonetic}

\begin{EntryWithPhonetic}{流行性感冒}{liu2xing2 xing4 gan3mao4}{10,6,8,13,9}{⽔、⾏、⼼、⼼、⽇}
  \definition{s.}{gripe muito forte; influenza}
\end{EntryWithPhonetic}

\begin{EntryWithPhonetic}{留}{liu2}{10}{⽥}[HSK 2]
  \definition*{s.}{Sobrenome Liu}
  \definition{v.}{ficar; permanecer; parar em um determinado local ou posição; não se afastar | estudar no exterior (geralmente seguido pelo nome de um país com uma sílaba) | pedir a alguém para ficar; manter alguém onde está | concentrar-se em; concentrar a atenção em algo | manter; guardar; reservar; não joger fora | acumular; deixar crescer | aceitar; receber | transmitir (legado); deixar para trás}
\end{EntryWithPhonetic}

\begin{EntryWithPhonetic}{留神}{liu2/shen2}{10,9}{⽥、⽰}
  \definition{v.+compl.}{tomar cuidado | prestar atenção | manter os olhos abertos}
\end{EntryWithPhonetic}

\begin{EntryWithPhonetic}{留下}{liu2 xia4}{10,3}{⽥、⼀}[HSK 2]
  \definition{v.}{deixar; parar em algum lugar}
\end{EntryWithPhonetic}

\begin{EntryWithPhonetic}{留学}{liu2xue2}{10,8}{⽥、⼦}[HSK 3]
  \definition{v.}{estudar no exterior; permanecer no estrangeiro para estudar ou pesquisar}
\end{EntryWithPhonetic}

\begin{EntryWithPhonetic}{留学生}{liu2 xue2 sheng1}{10,8,5}{⽥、⼦、⽣}[HSK 2]
  \definition[个,位,名,批,帮]{s.}{estudante estrangeiro; estudante que retornou; estudante que estuda no exterior}
\end{EntryWithPhonetic}

\begin{EntryWithPhonetic}{留言}{liu2 yan2}{10,7}{⽥、⾔}[HSK 6]
  \definition[条]{s.}{mensagem; recado}
  \definition{v.}{deixar uma mensagem; deixar seus comentários}
\end{EntryWithPhonetic}

\begin{EntryWithPhonetic}{柳}{liu3}{9}{⽊}
  \definition*{s.}{Liu, a vigésima quarta das vinte e oito constelações, consistindo de oito estrelas em Hydra | Liu, uma das mansões lunares | Sobrenome Liu}
  \definition[棵]{s.}{salgueiro}
\end{EntryWithPhonetic}

\begin{EntryWithPhonetic}{柳橙汁}{liu3cheng2zhi1}{9,16,5}{⽊、⽊、⽔}
  \definition[瓶,杯,罐,盒]{s.}{suco de laranja}
  \seealsoref{橙汁}{cheng2zhi1}
  \seealsoref{橘子汁}{ju2zi5zhi1}
\end{EntryWithPhonetic}

\begin{EntryWithPhonetic}{六}{liu4}{4}{⼋}[HSK 1]
  \definition*{s.}{Sobrenome Liu}
  \definition{num.}{seis; 6}
  \definition{s.}{símbolo musical utilizado na partitura da música tradicional chinesa, representando o primeiro grau da escala musical, equivalente ao ``5'' da notação musical simplificada}
\end{EntryWithPhonetic}

\begin{EntryWithPhonetic}{陆}{liu4}{7}{⾩}
  \definition{num.}{seis, usado para o numeral 六 em cheques, etc. para evitar erros ou alterações}
  \seeref{lu4}
  \seealsoref{六}{liu4}
\end{EntryWithPhonetic}

\begin{EntryWithPhonetic}{遛}{liu4}{13}{⾡}
  \definition{v.}{passear | andar (um animal) | caminhar conduzindo um animal doméstico}
\end{EntryWithPhonetic}

\begin{EntryWithPhonetic}{遛狗}{liu4/gou3}{13,8}{⾡、⽝}
  \definition{v.+compl.}{passear com um cachorro}
\end{EntryWithPhonetic}

\begin{EntryWithPhonetic}{龙}{long2}{5}{⿓}[HSK 3][Kangxi 212]
  \definition*{s.}{Sobrenome Long}
  \definition[条]{s.}{dragão; animal mítico e sobrenatural, com chifres, escamas, garras e bigodes, capaz de voar e mergulhar na água, provocar nuvens e chuva | dinossauro; um enorme réptil extinto; referência a certos répteis gigantes da antiguidade | do imperador; dragão como símbolo do imperador; usado na era feudal como símbolo do imperador; também se refere a coisas pertencentes ao imperador | em forma de dragão; com um desenho de dragão; refere-se a certos objetos que formam uma sequência semelhante a um dragão ou decorados com motivos de dragões}
\end{EntryWithPhonetic}

\begin{EntryWithPhonetic}{龙山}{long2shan1}{5,3}{⿓、⼭}
  \definition*{s.}{Longshan}
\end{EntryWithPhonetic}

\begin{EntryWithPhonetic}{龙虾}{long2xia1}{5,9}{⿓、⾍}
  \definition{s.}{lagosta}
\end{EntryWithPhonetic}

\begin{EntryWithPhonetic}{笼}{long2}{11}{⽵}
  \definition{s.}{armação fechada de bambu, arame, etc. | jaula | gaiola}
  \seeref{long3}
\end{EntryWithPhonetic}

\begin{EntryWithPhonetic}{笼子}{long2zi5}{11,3}{⽵、⼦}
  \definition{s.}{jaula | cesta | gaiola | recipiente}
  \seeref{long3zi5}
\end{EntryWithPhonetic}

\begin{EntryWithPhonetic}{笼}{long3}{11}{⽵}
  \definition{v.}{envolver | cobrir}
  \seeref{long2}
\end{EntryWithPhonetic}

\begin{EntryWithPhonetic}{笼子}{long3zi5}{11,3}{⽵、⼦}
  \definition{s.}{caixa grande | porta-malas}
  \seeref{long2zi5}
\end{EntryWithPhonetic}

\begin{EntryWithPhonetic}{弄}{long4}{7}{⼶}
  \definition{s.}{rua estreita; beco; viela; travessa}
  \seeref{nong4}
\end{EntryWithPhonetic}

\begin{EntryWithPhonetic}{楼}{lou2}{13}{⽊}[HSK 1]
  \definition*{s.}{Sobrenome Lou}
  \definition{clas.}{andar, piso}
  \definition[层,座,栋]{s.}{um prédio com muitos andares | piso; andar | superestrutura; uma estrutura com um convés superior; um andar adicional construído sobre uma casa ou outro edifício | nome usado para certas lojas ou locais de entretenimento | arco ornamental; certas construções decorativas altas com passagens por baixo}
\end{EntryWithPhonetic}

\begin{EntryWithPhonetic}{楼道}{lou2 dao4}{13,12}{⽊、⾡}[HSK 6]
  \definition[个]{s.}{corredor; passagem | passagem (em edifício de vários andares)}
\end{EntryWithPhonetic}

\begin{EntryWithPhonetic}{楼房}{lou2 fang2}{13,8}{⽊、⼾}[HSK 6]
  \definition[栋,幢,座,套,层]{s.}{um edifício de dois ou mais andares}
\end{EntryWithPhonetic}

\begin{EntryWithPhonetic}{楼上}{lou2 shang4}{13,3}{⽊、⼀}[HSK 1]
  \definition{s.}{no andar de cima | autor anterior em um tópico do fórum; em plataformas como fóruns na internet, refere-se à pessoa que se manifesta antes de você.}
\end{EntryWithPhonetic}

\begin{EntryWithPhonetic}{楼梯}{lou2 ti1}{13,11}{⽊、⽊}[HSK 4]
  \definition[个,层,段,阶]{s.}{escada; escadaria; degraus no meio de dois andares para permitir que as pessoas subam ou desçam as escadas}
\end{EntryWithPhonetic}

\begin{EntryWithPhonetic}{楼下}{lou2 xia4}{13,3}{⽊、⼀}[HSK 1]
  \definition{s.}{no andar de baixo}
\end{EntryWithPhonetic}

\begin{EntryWithPhonetic}{漏}{lou4}{14}{⽔}[HSK 5]
  \definition{s.}{relógio de água; ampulheta | falha; ponto fraco | gonorreia; a medicina tradicional chinesa refere-se a certas doenças que causam secreção de pus, sangue e muco | unidade de tempo medida por um relógio de água durante a noite}
  \definition{v.}{(líquido, gás, etc.) pingar; vazar; escorrer; cair (de um buraco ou fenda) | vazar; deixar escapar; divulgar | perder; deixar de fora por engano | vazar; o objeto tem poros e pode vazar coisas | há uma fuga de ar}
\end{EntryWithPhonetic}

\begin{EntryWithPhonetic}{漏电}{lou4dian4}{14,5}{⽔、⽥}
  \definition{v.}{vazar eletricidade}
\end{EntryWithPhonetic}

\begin{EntryWithPhonetic}{漏洞}{lou4 dong4}{14,9}{⽔、⽔}[HSK 5]
  \definition[个,点]{s.}{vazamento; rachadura; lacunas ou buracos desnecessários que permitem que coisas vazem | falha; defeito; lacuna; (fala, ação, método, etc.) imperfeições}
\end{EntryWithPhonetic}

\begin{EntryWithPhonetic}{露}{lou4}{21}{⾬}[HSK 6]
  \definition{v.}{mostrar; apresentar (uma certa emoção ou olhar no rosto) | mostrar; aparentar; fazer algo visível; as pessoas podem ver}
  \seeref{lu4}
\end{EntryWithPhonetic}

\begin{EntryWithPhonetic}{卢}{lu2}{5}{⼘}
  \definition*{s.}{Luxemburgo, abreviação de 卢森堡 | Sobrenome Lu}
  \definition{s.}{Aarcaico: preta (cor)}
  \seealsoref{卢森堡}{lu2sen1bao3}
\end{EntryWithPhonetic}

\begin{EntryWithPhonetic}{卢森堡}{lu2sen1bao3}{5,12,12}{⼘、⽊、⼟}
  \definition*{s.}{Luxemburgo}
\end{EntryWithPhonetic}

\begin{EntryWithPhonetic}{卢旺达}{lu2wang4da2}{5,8,6}{⼘、⽇、⾡}
  \definition*{s.}{Ruanda}
\end{EntryWithPhonetic}

\begin{EntryWithPhonetic}{芦}{lu2}{7}{⾋}
  \definition*{s.}{Sobrenome Lu}
  \definition{s.}{junco}
\end{EntryWithPhonetic}

\begin{EntryWithPhonetic}{芦笋}{lu2sun3}{7,10}{⾋、⽵}
  \definition{s.}{aspargos}
\end{EntryWithPhonetic}

\begin{EntryWithPhonetic}{陆}{lu4}{7}{⾩}
  \definition*{s.}{Sobrenome Lu}
  \definition[个]{s.}{terra; terreno | rota terrestre; por terra}
  \seeref{liu4}
\end{EntryWithPhonetic}

\begin{EntryWithPhonetic}{陆地}{lu4di4}{7,6}{⾩、⼟}[HSK 4]
  \definition[块,片]{s.}{terra; terra seca (em oposição ao mar); superfície da Terra, excluindo os oceanos (e, às vezes, rios e lagos)}
\end{EntryWithPhonetic}

\begin{EntryWithPhonetic}{陆军}{lu4 jun1}{7,6}{⾩、⼍}[HSK 6]
  \definition{s.}{força terrestre; exército}
\end{EntryWithPhonetic}

\begin{EntryWithPhonetic}{陆路}{lu4lu4}{7,13}{⾩、⾜}
  \definition{s.}{rota terrestre}
\end{EntryWithPhonetic}

\begin{EntryWithPhonetic}{陆续}{lu4xu4}{7,11}{⾩、⽷}[HSK 4]
  \definition{adv.}{sucessivamente; um após o outro; intermitentemente}
\end{EntryWithPhonetic}

\begin{EntryWithPhonetic}{录}{lu4}{8}{⼹}[HSK 3]
  \definition{s.}{registro; cadastro; coleção; seleções}
  \definition{v.}{copiar; gravar; escrever; copiar; registrar | contratar; selecionar; empregar; adotar ou nomear | gravar em fita magnética}
\end{EntryWithPhonetic}

\begin{EntryWithPhonetic}{录取}{lu4qu3}{8,8}{⼹、⼜}[HSK 4]
  \definition{v.}{aceitar; admitir; recrutar; entrar; matricular (os aprovados no exame)}
\end{EntryWithPhonetic}

\begin{EntryWithPhonetic}{录像}{lu4/xiang4}{8,13}{⼹、⼈}[HSK 6]
  \definition[段,个,些,盘]{s.}{vídeo; gravação; fita de vídeo; imagens gravadas com celulares, câmeras, etc.}
  \definition{v.+compl.}{gravar bídeo; gravar em fita de vídeo | usar celulares, câmeras e outros dispositivos para salvar registros de vídeo}
\end{EntryWithPhonetic}

\begin{EntryWithPhonetic}{录像带}{lu4xiang4dai4}{8,13,9}{⼹、⼈、⼱}
  \definition[盘]{s.}{video-cassete}
\end{EntryWithPhonetic}

\begin{EntryWithPhonetic}{录像机}{lu4xiang4ji1}{8,13,6}{⼹、⼈、⽊}
  \definition[台]{s.}{gravador de vídeo | VCR}
\end{EntryWithPhonetic}

\begin{EntryWithPhonetic}{录音}{lu4/yin1}{8,9}{⼹、⾳}[HSK 3]
  \definition[段,个]{s.}{gravação de som; som gravado com equipamento especializado}
  \definition{v.+compl.}{gravar; converter o som em sinal elétrico e, em seguida, gravá-lo por meios mecânicos, ópticos ou eletromagnéticos}
\end{EntryWithPhonetic}

\begin{EntryWithPhonetic}{录音机}{lu4 yin1 ji1}{8,9,6}{⼹、⾳、⽊}[HSK 6]
  \definition[台]{s.}{gravador de som; máquina de gravação (de fita)}
\end{EntryWithPhonetic}

\begin{EntryWithPhonetic}{鹿}{lu4}{11}{⿅}[Kangxi 198]
  \definition*{s.}{Sobrenome Lu}
  \definition[只,头,群]{s.}{cervo | veado}
\end{EntryWithPhonetic}

\begin{EntryWithPhonetic}{路}{lu4}{13}{⾜}[HSK 1]
  \definition*{s.}{Sobrenome Lu}
  \definition{clas.}{tipo; classe | linha; coluna; usado para um grupo de pessoas ou uma equipe; para organizar em ordem}
  \definition[条]{s.}{estrada; caminho; via | viagem; jornada; distância | maneira; meios | sequência; linha; lógica | região; distrito | rota | classe; classificação; grau | linha; fileira}
\end{EntryWithPhonetic}

\begin{EntryWithPhonetic}{路边}{lu4 bian1}{13,5}{⾜、⾡}[HSK 2]
  \definition{s.}{calçada; beira da estrada; margem da rua}
\end{EntryWithPhonetic}

\begin{EntryWithPhonetic}{路过}{lu4 guo4}{13,6}{⾜、⾡}[HSK 6]
  \definition{v.}{passar por (algum lugar); atravessar}
\end{EntryWithPhonetic}

\begin{EntryWithPhonetic}{路口}{lu4 kou3}{13,3}{⾜、⼝}[HSK 1]
  \definition[个]{s.}{cruzamento; intersecção; onde as estradas se encontram}
\end{EntryWithPhonetic}

\begin{EntryWithPhonetic}{路上}{lu4 shang5}{13,3}{⾜、⼀}[HSK 1]
  \definition{s.}{na estrada | a caminho; na rota; em processo de mudança de um lugar para outro}
\end{EntryWithPhonetic}

\begin{EntryWithPhonetic}{路线}{lu4 xian4}{13,8}{⾜、⽷}[HSK 3]
  \definition[条]{s.}{rota; caminho; linha; a estrada percorrida de um lugar a outro | linha; diretriz (de política, ideologia, campo de trabalho); a via fundamental a seguir em termos ideológicos, políticos ou profissionais}
\end{EntryWithPhonetic}

\begin{EntryWithPhonetic}{露}{lu4}{21}{⾬}[HSK 6]
  \definition{adj.}{fora de uma casa, tenda, etc., sem cobertura}
  \definition{s.}{orvalho; gotas de água condensadas | xarope; suco de fruta; bebida destilada de flores, folhas ou frutos}
  \definition{v.}{revelar; expor; mostrar; trair}
  \seeref{lou4}
\end{EntryWithPhonetic}

\begin{EntryWithPhonetic}{露珠}{lu4zhu1}{21,10}{⾬、⽟}
  \definition{s.}{orvalho}
\end{EntryWithPhonetic}

\begin{EntryWithPhonetic}{乱}{luan4}{7}{⼄}[HSK 3]
  \definition{adj.}{em desordem; em confusão; em desarrumação; sem ordem nem organização | em um estado mental confuso | (de uma sociedade) turbulento; agitado | (de relações sexuais) impróprio; promíscuo}
  \definition{adv.}{aleatoriamente; arbitrariamente; indiscriminadamente; sem restrições; à vontade}
  \definition{s.}{motim; agitação; tumulto; revolta; guerra; calamidade}
  \definition{v.}{confundir; embaralhar; misturar; causar desordem}
\end{EntryWithPhonetic}

\begin{EntryWithPhonetic}{伦}{lun2}{6}{⼈}
  \definition*{s.}{Sobrenome Lun}
  \definition{s.}{relações humanas (especialmente como concebidas pela ética feudal) | lógica; ordem | par; correspondência; (mesma) classe | ética; relações humanas | sequência lógica; ordem | o mesmo tipo; semelhante; igual}
\end{EntryWithPhonetic}

\begin{EntryWithPhonetic}{伦敦}{lun2dun1}{6,12}{⼈、⽁}
  \definition*{s.}{Londres}
\end{EntryWithPhonetic}

\begin{EntryWithPhonetic}{论}{lun2}{6}{⾔}
  \definition*{s.}{Os Analectos de Confúcio, registro dos ditos e feitos de Confúcio e seus discípulos}
  \seeref{lun4}
\end{EntryWithPhonetic}

\begin{EntryWithPhonetic}{轮}{lun2}{8}{⾞}[HSK 4]
  \definition{clas.}{usado para sol vermelho, lua brilhante, etc. | usado para rodadas | doze anos de idade (os doze ramos terrestres são usados para lembrar o gênero humano e cada doze anos de idade é um ciclo)}
  \definition{s.}{roda | anel; disco; objeto semelhante a uma roda | navio a vapor; barco a vapor}
  \definition{v.}{revezar; substituir um ao outro em sequência (para fazer algo)}
\end{EntryWithPhonetic}

\begin{EntryWithPhonetic}{轮船}{lun2chuan2}{8,11}{⾞、⾈}[HSK 4]
  \definition[艘,班]{s.}{vapor; navio a vapor; barco a vapor}
\end{EntryWithPhonetic}

\begin{EntryWithPhonetic}{轮回}{lun2hui2}{8,6}{⾞、⼞}
  \definition[个]{s.}{reencarnação (Budismo) | ciclo}
  \definition{v.}{reencarnar}
\end{EntryWithPhonetic}

\begin{EntryWithPhonetic}{轮椅}{lun2 yi3}{8,12}{⾞、⽊}[HSK 4]
  \definition{s.}{cadeira de rodas; dispositivo de assento especialmente projetado com rodas para pessoas com dificuldade de locomoção, que pode ser acionado por um disco de roda ou manivela operados manualmente}
\end{EntryWithPhonetic}

\begin{EntryWithPhonetic}{轮子}{lun2 zi5}{8,3}{⾞、⼦}[HSK 4]
  \definition[个,只]{s.}{roda; peças circulares de veículos ou máquinas com capacidade de rotação}
\end{EntryWithPhonetic}

\begin{EntryWithPhonetic}{论}{lun4}{6}{⾔}
  \definition*{s.}{Sobrenome Lun}
  \definition{prep.}{por (uma certa unidade de medida) | de acordo com (um certo sistema ou princípio)}
  \definition{s.}{visão; opinião; declaração | (frequentemente em títulos) dissertação; ensaio; tratado | teoria; doutrina | ideia; palavras ou artigos que analisam e explicam coisas}
  \definition{v.}{discutir; falar sobre; discursar sobre; comentar | mencionar; considerar; falar de | decidir sobre; determinar | decidir sobre a natureza da culpa; punir | argumentar; analisar e explicar coisas | considerar; ponderar; medir; avaliar}
  \seeref{lun2}
\end{EntryWithPhonetic}

\begin{EntryWithPhonetic}{论文}{lun4wen2}{6,4}{⾔、⽂}[HSK 4]
  \definition[篇]{s.}{tese; redação; artigo; artigo que discute ou examina uma questão}
\end{EntryWithPhonetic}

\begin{EntryWithPhonetic}{罗}{luo2}{8}{⽹}
  \definition*{s.}{Sobrenome Luo}
  \definition{clas.}{uma grosa; uma bruta; doze dúzias; 144 unidades}
  \definition{s.}{uma rede para capturar pássaros | peneira; tela | uma espécie de gaze de seda}
  \definition{v.}{pegar pássaros com uma rede | espalhar; exibir; mostrar | coletar; reunir; recrutar | peneirar}
\end{EntryWithPhonetic}

\begin{EntryWithPhonetic}{逻}{luo2}{11}{⾡}
  \definition{s.}{patrulha | (literário) a beira de um riacho de montanha}
  \definition{v.}{patrulhar; fazer rondas}
\end{EntryWithPhonetic}

\begin{EntryWithPhonetic}{逻辑}{luo2ji5}{11,13}{⾡、⾞}[HSK 5]
  \definition[套,条,种]{s.}{lógica; lei objetiva; a objetividade das leis que regem o desenvolvimento das coisas | lógica; razão; regras para o pensamento | lógica como ciência do raciocínio, do pensamento; disciplina que estuda a lógica}
\end{EntryWithPhonetic}

\begin{EntryWithPhonetic}{螺}{luo2}{17}{⾍}
  \definition{s.}{concha em espiral | caracol | búzio}
\end{EntryWithPhonetic}

\begin{EntryWithPhonetic}{螺丝}{luo2si1}{17,5}{⾍、⼀}
  \definition{s.}{parafuso}
\end{EntryWithPhonetic}

\begin{EntryWithPhonetic}{骆}{luo4}{9}{⾺}
  \definition*{s.}{Sobrenome Luo}
  \definition[只]{s.}{Arcaico: um cavalo branco com crina preta, mencionado em antigos livros chineses}
\end{EntryWithPhonetic}

\begin{EntryWithPhonetic}{骆驼}{luo4tuo5}{9,8}{⾺、⾺}
  \definition[头,只,匹]{s.}{camelo | coloquial: cabeça-dura, idiota}
\end{EntryWithPhonetic}

\begin{EntryWithPhonetic}{落}{luo4}{12}{⾋}[HSK 4]
  \definition*{s.}{Sobrenome Luo}
  \definition{s.}{paradeiro; lugar para ficar; local de descanso | assentamento; local de reunião | parte curta; área pequena; refere-se a um pequeno lugar ou área}
  \definition{v.}{cair; cair de uma altura elevada | se abaixar; descer; ir para baixo | abaixar; deixar cair (ou descer); fazer descer | afundar; declinar; descer | ficar para trás; ficar para trás ou ficar de fora | permanecer; fazer uma parada; deixar para trás | cair sobre; repousar com | obter; ter; receber | anotar; escrever no papel | cair em; entrar em; ficar preso}
  \seeref{la4}
  \seeref{lao4}
\end{EntryWithPhonetic}

\begin{EntryWithPhonetic}{落后}{luo4hou4}{12,6}{⾋、⼝}[HSK 3]
  \definition{adj.}{atrasado; trabalho em atraso, nível de desenvolvimento ou grau de reconhecimento (em oposição a 进步)}
  \definition{v.}{ficar para trás; ficar atrasado; ficar para trás em relação aos outros durante o avanço ou o progresso do trabalho}
  \seealsoref{进步}{jin4bu4}
\end{EntryWithPhonetic}

\begin{EntryWithPhonetic}{落花生}{luo4 hua1 sheng1}{12,7,5}{⾋、⾋、⽣}
  \definition{s.}{amendoim | noz de macaco}
\end{EntryWithPhonetic}

\begin{EntryWithPhonetic}{落日}{luo4ri4}{12,4}{⾋、⽇}
  \definition{s.}{pôr do sol}
\end{EntryWithPhonetic}

\begin{EntryWithPhonetic}{落实}{luo4shi2}{12,8}{⾋、⼧}[HSK 5]
  \definition{adj.}{sentimento de tranquilidade; (humor) estável; seguro}
  \definition{v.}{implementar; ser praticável; tornar os planos, políticas, medidas, etc. específicos e compreensíveis, de modo a que possam ser realizados | implementar; colocar em prática; pôr em prática significa que os planos, políticas e medidas são específicos e claros, e podem ser realizados}
\end{EntryWithPhonetic}

\begin{EntryWithPhonetic}{落汤鸡}{luo4tang1ji1}{12,6,7}{⾋、⽔、⿃}
  \definition{s.}{uma pessoa que parece encharcada e acamada| sofrimento profundo}
\end{EntryWithPhonetic}

\begin{EntryWithPhonetic}{驴}{lv2}{7}{⾺}
  \definition[头,只]{s.}{burro; asno; jumento; jegue}
\end{EntryWithPhonetic}

\begin{EntryWithPhonetic}{旅}{lv3}{10}{⽅}
  \definition{adv.}{juntos; conjuntamente}
  \definition[个]{s.}{brigada; unidade organizacional militar, abaixo do nível de divisão e acima do nível de regimento ou batalhão | força; tropas; geralmente se refere aos militares | viajante; passageiro; turista | viagem; jornada | pessoas}
  \definition{v.}{viajar; ficar longe de casa; ir para longe; morar longe de casa}
\end{EntryWithPhonetic}

\begin{EntryWithPhonetic}{旅程}{lv3cheng2}{10,12}{⽅、⽲}
  \definition{s.}{jornada | viagem}
\end{EntryWithPhonetic}

\begin{EntryWithPhonetic}{旅店}{lv3 dian5}{10,8}{⽅、⼴}[HSK 6]
  \definition[家,个]{s.}{pousada; albergue; hotel}
\end{EntryWithPhonetic}

\begin{EntryWithPhonetic}{旅馆}{lv3 guan3}{10,11}{⽅、⾷}[HSK 3]
  \definition[家,个,所]{s.}{pousada; hotel; local comercial destinado ao alojamento de viajantes}
\end{EntryWithPhonetic}

\begin{EntryWithPhonetic}{旅客}{lv3 ke4}{10,9}{⽅、⼧}[HSK 2]
  \definition[名,位,个,些]{s.}{viajante; passageiro; as agências de transporte e turismo referem-se às pessoas que viajam}
\end{EntryWithPhonetic}

\begin{EntryWithPhonetic}{旅行}{lv3xing2}{10,6}{⽅、⾏}[HSK 2]
  \definition{v.}{viajar; passear; para tratar de assuntos ou passear, ir de um lugar para outro (geralmente se refere a distâncias longas)}
\end{EntryWithPhonetic}

\begin{EntryWithPhonetic}{旅行社}{lv3 xing2 she4}{10,6,7}{⽅、⾏、⽰}[HSK 3]
  \definition[家]{s.}{agência de viagens; agência especializada em serviços relacionados a viagens, que providencia hospedagem, transporte e outros serviços para viajantes}
\end{EntryWithPhonetic}

\begin{EntryWithPhonetic}{旅游}{lv3you2}{10,12}{⽅、⽔}[HSK 2]
  \definition{v.}{viajar para outros lugares para passear e fazer turismo}
\end{EntryWithPhonetic}

\begin{EntryWithPhonetic}{屡}{lv3}{12}{⼫}
  \definition{adv.}{uma e outra vez; repetidamente | frequentemente}
\end{EntryWithPhonetic}

\begin{EntryWithPhonetic}{屡次}{lv3ci4}{12,6}{⼫、⽋}
  \definition{adv.}{repetidamente | uma e outra vez | muitas vezes}
\end{EntryWithPhonetic}

\begin{EntryWithPhonetic}{律}{lv4}{9}{⼻}
  \definition*{s.}{Sobrenome Lü}
  \definition{s.}{lei; regra; estatuto; regulamento}
  \definition{v.}{restringir; disciplinar; manter sob controle}
\end{EntryWithPhonetic}

\begin{EntryWithPhonetic}{律师}{lv4shi1}{9,6}{⼻、⼱}[HSK 4]
  \definition[名,个,位]{s.}{advogado; procurador; profissionais encarregados pelas partes ou nomeados pelo tribunal para auxiliar as partes no litígio, para comparecer ao tribunal para defesa e para tratar de assuntos jurídicos relacionados, de acordo com a lei}
\end{EntryWithPhonetic}

\begin{EntryWithPhonetic}{率}{lv4}{11}{⽞}
  \definition{s.}{taxa; razão; proporção; a relação proporcional entre duas grandezas relacionadas}
  \seeref{shuai4}
\end{EntryWithPhonetic}

\begin{EntryWithPhonetic}{绿}{lv4}{11}{⽷}[HSK 2]
  \definition*{s.}{Sobrenome Lü}
  \definition{adj.}{verde}
  \definition{v.}{tornar-se verde; ficar verde}
\end{EntryWithPhonetic}

\begin{EntryWithPhonetic}{绿茶}{lv4 cha2}{11,9}{⽷、⾋}[HSK 3]
  \definition{s.}{chá verde; chá produzido apenas através dos processos de maturação, enrolamento (ou sem enrolamento) e secagem, sem passar por fermentação, com cor verde-claro}
\end{EntryWithPhonetic}

\begin{EntryWithPhonetic}{绿豆}{lv4dou4}{11,7}{⽷、⾖}
  \definition{s.}{vagens}
\end{EntryWithPhonetic}

\begin{EntryWithPhonetic}{绿豆芽}{lv4dou4 ya2}{11,7,7}{⽷、⾖、⾋}
  \definition{s.}{broto de feijão verde}
\end{EntryWithPhonetic}

\begin{EntryWithPhonetic}{绿化}{lv4 hua4}{11,4}{⽷、⼔}[HSK 6]
  \definition{v.}{tornar verde plantando árvores, flores, etc.; arborizar; reflorestar; plantar árvores, flores e plantas para embelezar o ambiente ou prevenir a erosão do solo}
\end{EntryWithPhonetic}

\begin{EntryWithPhonetic}{绿色}{lv4 se4}{11,6}{⽷、⾊}[HSK 2]
  \definition{adj.}{verde; ecológico; sem poluição; em conformidade com os requisitos ambientais}
  \definition{s.}{cor verde}
\end{EntryWithPhonetic}

\begin{EntryWithPhonetic}{略}{lve4}{11}{⽥}
  \definition{adv.}{ligeiramente | marginalmente | aproximadamente}
\end{EntryWithPhonetic}

\begin{EntryWithPhonetic}{略微}{lve4wei1}{11,13}{⽥、⼻}
  \definition{adv.}{ligeiramente | marginalmente | aproximadamente}
\end{EntryWithPhonetic}

%%%%% EOF %%%%%


 %%%
%%% M
%%%

\section*{M}\addcontentsline{toc}{section}{M}

\begin{EntryWithPhonetic}{妈}{ma1}{6}{⼥}[HSK 1]
  \definition[个,位]{s.}{mãe; mamãe | uma forma de tratamento para uma mulher casada uma geração mais velha | (antigo) uma forma de tratamento para uma empregada doméstica de meia-idade ou velha}
  \seealsoref{妈妈}{ma1 ma5}
\end{EntryWithPhonetic}

\begin{EntryWithPhonetic}{妈妈}{ma1 ma5}{6,6}{⼥、⼥}[HSK 1]
  \definition[个,位]{s.}{mamãe; mãe | uma forma de chamar uma mulher de meia-idade; títulos de respeito para mulheres mais velhas}
\end{EntryWithPhonetic}

\begin{EntryWithPhonetic}{抹}{ma1}{8}{⼿}
  \definition{v.}{esfregar; limpar | deslizar algo para fora; tirar}
  \seeref{抹}{mo3}
  \seeref{抹}{mo4}
\end{EntryWithPhonetic}

\begin{EntryWithPhonetic}{蚂}{ma1}{9}{⾍}
  \definition{part.}{caracter formador de palavras}
  \definition[只]{s.}{libélula}
  \seeref{蚂}{ma3}
  \seeref{蚂}{ma4}
\end{EntryWithPhonetic}

\begin{EntryWithPhonetic}{吗}{ma2}{6}{⼝}
  \definition{adv.}{(coloquial) que?}
  \seeref{吗}{ma3}
  \seeref{吗}{ma5}
\end{EntryWithPhonetic}

\begin{EntryWithPhonetic}{麻}{ma2}{11}{⿇}[Kangxi 200]
  \definition*{s.}{Sobrenome Ma}
  \definition{adj.}{áspero; grosseiro | marcado; manchado | espinhas; manchas ásperas; cicatrizes deixadas após a varíola}
  \definition[棵,株]{s.}{nome geral para cânhamo, linho, etc. | fibra de cânhamo, linho, etc. para têxteis | sésamo; gergelim | marcas de varíola; um rosto com marcas de varíola}
  \definition{v.}{anestesiar | corromper (a mente de alguém); envenenar}
\end{EntryWithPhonetic}

\begin{EntryWithPhonetic}{麻烦}{ma2fan5}{11,10}{⿇、⽕}[HSK 3]
  \definition{adj.}{incômodo; inconveniente; complicado; trabalhoso; burocrático | incômodo; inconveniente; (a situação) é confusa e complicada}
  \definition[个,些,点,堆]{s.}{problema; inconveniência; assuntos complicados e difíceis de resolver}
  \definition{v.}{incomodar; perturbar; incomodar alguém; irritar; aborrecer; causar incômodo ou sobrecarregar outras pessoas}
\end{EntryWithPhonetic}

\begin{EntryWithPhonetic}{麻将}{ma2jiang4}{11,9}{⿇、⼨}
  \definition*[副]{s.}{Mahjong}
\end{EntryWithPhonetic}

\begin{EntryWithPhonetic}{麻辣豆腐}{ma2la4 dou4fu5}{11,14,7,14}{⿇、⾟、⾖、⾁}
  \definition{s.}{tofú guisado em molho picante (prato)}
\end{EntryWithPhonetic}

\begin{EntryWithPhonetic}{马}{ma3}{3}{⾺}[HSK 3][Kangxi 187]
  \definition*{s.}{Sobrenome Ma}
  \definition{adj.}{grande; extenso; amplo}
  \definition[匹,头,只,群]{s.}{cavalo | a peça do cavalo no xadrez chinês}
\end{EntryWithPhonetic}

\begin{EntryWithPhonetic}{马车}{ma3 che1}{3,4}{⾺、⾞}[HSK 6]
  \definition[辆]{s.}{carruagem (puxada por cavalo); carroça; charrete}
\end{EntryWithPhonetic}

\begin{EntryWithPhonetic}{马耳他}{ma3'er3ta1}{3,6,5}{⾺、⽿、⼈}
  \definition*{s.}{Malta}
\end{EntryWithPhonetic}

\begin{EntryWithPhonetic}{马克思列宁主义}{ma3ke4si1 lie4ning2 zhu3yi4}{3,7,9,6,5,5,3}{⾺、⼗、⼼、⼑、⼧、⼂、⼂}
  \definition*{s.}{Marxismo-Leninismo}
\end{EntryWithPhonetic}

\begin{EntryWithPhonetic}{马路}{ma3lu4}{3,13}{⾺、⾜}[HSK 1]
  \definition[条]{s.}{estrada; rua; avenida; estradas largas e planas para o tráfego de carros e cavalos nas cidades ou nos subúrbios}
\end{EntryWithPhonetic}

\begin{EntryWithPhonetic}{马马虎虎}{ma3ma3hu3hu3}{3,3,8,8}{⾺、⾺、⾌、⾌}
  \definition{adj.}{descuidado | casual | tolerável | vago | mais ou menos}
\end{EntryWithPhonetic}

\begin{EntryWithPhonetic}{马上}{ma3shang4}{3,3}{⾺、⼀}[HSK 1]
  \definition{adv.}{imediatamente; de uma só vez; em um piscar de olhos | em breve; em um futuro próximo; em um curto espaço de tempo}
\end{EntryWithPhonetic}

\begin{EntryWithPhonetic}{马尾}{ma3wei3}{3,7}{⾺、⼫}
  \definition{s.}{(penteado) rabo de cavalo | cauda de cavalo}
\end{EntryWithPhonetic}

\begin{EntryWithPhonetic}{吗}{ma3}{6}{⼝}
  \definition{s.}{usada em 吗啡, morfina}
  \seealsoref{吗啡}{ma3fei1}
\end{EntryWithPhonetic}

\begin{EntryWithPhonetic}{吗啡}{ma3fei1}{6,11}{⼝、⼝}
  \definition{s.}{morfina (empréstimo linguístico)}
\end{EntryWithPhonetic}

\begin{EntryWithPhonetic}{码}{ma3}{8}{⽯}
  \definition{clas.}{refere-se a um assunto específico ou a uma categoria de assuntos; refere-se a uma coisa ou a uma classe de coisas | jarda; unidade de comprimento britânica e americana}
  \definition{s.}{um sinal ou objeto que indica número; código; símbolos ou ferramentas que indicam números}
  \definition{v.}{empilhar; acumular}
\end{EntryWithPhonetic}

\begin{EntryWithPhonetic}{码头}{ma3tou2}{8,5}{⽯、⼤}[HSK 5]
  \definition[个,座]{s.}{doca; cais; píer; molhe; edifícios à beira-mar ou à beira do rio destinados exclusivamente à atracação de embarcações, embarque e desembarque de passageiros e carga e descarga de mercadorias | cidade portuária; centro comercial e de transportes; refere-se a uma cidade comercial com transporte terrestre e marítimo bem desenvolvido.}
\end{EntryWithPhonetic}

\begin{EntryWithPhonetic}{蚂}{ma3}{9}{⾍}
  \definition{part.}{caracter formador de palavras}
  \seeref{蚂}{ma1}
  \seeref{蚂}{ma4}
\end{EntryWithPhonetic}

\begin{EntryWithPhonetic}{蚂蚁}{ma3yi3}{9,9}{⾍、⾍}
  \definition{s.}{formiga}
\end{EntryWithPhonetic}

\begin{EntryWithPhonetic}{蚂}{ma4}{9}{⾍}
  \definition{part.}{caracter formador de palavras}
  \seeref{蚂}{ma1}
  \seeref{蚂}{ma3}
\end{EntryWithPhonetic}

\begin{EntryWithPhonetic}{骂}{ma4}{9}{⾺}[HSK 5]
  \definition{v.}{abusar; xingar; insultar; insultar alguém com palavras grosseiras ou maliciosas | repreender; censurar; condenar}
\end{EntryWithPhonetic}

\begin{EntryWithPhonetic}{骂街}{ma4jie1}{9,12}{⾺、⾏}
  \definition{v.}{gritar abusos na rua}
\end{EntryWithPhonetic}

\begin{EntryWithPhonetic}{骂名}{ma4ming2}{9,6}{⾺、⼝}
  \definition{s.}{infâmia}
\end{EntryWithPhonetic}

\begin{EntryWithPhonetic}{吗}{ma5}{6}{⼝}[HSK 1]
  \definition{part.}{usado no final de uma pergunta | como uma pausa em uma frase antes de introduzir o ponto principal | usado no final de uma pergunta retórica}
  \seeref{吗}{ma2}
  \seeref{吗}{ma3}
\end{EntryWithPhonetic}

\begin{EntryWithPhonetic}{嘛}{ma5}{14}{⼝}[HSK 6]
  \definition{part.}{usado no final de uma declaração para expressar que é claro que é verdade que é óbvio | usado no final de uma frase imperativa para expressar expectativa ou dissuasão | usado em uma frase para indicar uma pausa e chamar a atenção da outra pessoa}
\end{EntryWithPhonetic}

\begin{EntryWithPhonetic}{埋}{mai2}{10}{⼟}[HSK 6]
  \definition{v.}{cobrir (com terra, neve, etc.); enterrar | esconder | enterrar (uma pessoa morta)}
  \seeref{埋}{man2}
\end{EntryWithPhonetic}

\begin{EntryWithPhonetic}{埋伏}{mai2fu2}{10,6}{⼟、⼈}
  \definition{s.}{emboscada}
  \definition{v.}{emboscar}
\end{EntryWithPhonetic}

\begin{EntryWithPhonetic}{买}{mai3}{6}{⼄}[HSK 1]
  \definition*{s.}{Sobrenome Mai}
  \definition{v.}{comprar; adquirir | comprar; subornar; usar dinheiro ou outros meios para angariar apoio| pedir; obter; trocar dinheiro por coisas}
\end{EntryWithPhonetic}

\begin{EntryWithPhonetic}{买东西}{mai3 dong1xi5}{6,5,6}{⼄、⼀、⾑}
  \definition{v.}{fazer compras; comprar bens ou serviços}
\end{EntryWithPhonetic}

\begin{EntryWithPhonetic}{买卖}{mai3 mai4}{6,8}{⼄、⼗}[HSK 5]
  \definition[笔,桩,宗,家]{s.}{negócio; compra e venda; transação | Privado: loja; armazém}
\end{EntryWithPhonetic}

\begin{EntryWithPhonetic}{麦}{mai4}{7}{⿆}[Kangxi 199]
  \definition*{s.}{Sobrenome Mai}
  \definition[袋,筐,车]{s.}{um termo geral para trigo, cevada, etc.}
\end{EntryWithPhonetic}

\begin{EntryWithPhonetic}{麦当劳}{mai4dang1lao2}{7,6,7}{⿆、⼹、⼒}
  \definition*{s.}{McDonald's, restaurante de \emph{fast-food}}
  \seealsoref{麦当劳叔叔}{mai4dang1lao2 shu1shu5}
\end{EntryWithPhonetic}

\begin{EntryWithPhonetic}{麦当劳叔叔}{mai4dang1lao2 shu1shu5}{7,6,7,8,8}{⿆、⼹、⼒、⼜、⼜}
  \definition*{s.}{Ronald McDonald}
  \seealsoref{麦当劳}{mai4dang1lao2}
\end{EntryWithPhonetic}

\begin{EntryWithPhonetic}{麦淇淋}{mai4qi2lin2}{7,11,11}{⿆、⽔、⽔}
  \definition{s.}{Empréstimo linguístico: margarina}
\end{EntryWithPhonetic}

\begin{EntryWithPhonetic}{卖}{mai4}{8}{⼗}[HSK 2]
  \definition*{s.}{Sobrenome Mai}
  \definition{clas.}{um prato (nos tempos antigos); antigamente, os restaurantes chamavam cada prato vendido de 一卖 (uma porção)}
  \definition{v.}{vender (oposto de 买) | trair (o próprio país ou amigos); alcançar objetivos pessoais à custa dos interesses do país, da nação e dos outros | não poupar esforços; esforçar-se ao máximo; tentar fazer o máximo possível | mostrar-se intencionalmente; exibir-se | vender o próprio trabalho; trabalhar em troca de dinheiro}
  \seealsoref{买}{mai3}
\end{EntryWithPhonetic}

\begin{EntryWithPhonetic}{埋}{man2}{10}{⼟}
  \definition{part.}{caracter formador de palavras}
  \seeref{埋}{mai2}
\end{EntryWithPhonetic}

\begin{EntryWithPhonetic}{谩}{man2}{13}{⾔}
  \definition{v.}{enganar; ludibriar; iludir}
  \seeref{谩}{man4}
\end{EntryWithPhonetic}

\begin{EntryWithPhonetic}{蔓}{man2}{14}{⾋}
  \definition{s.}{couve-chinesa | nabo}
  \seeref{蔓}{man4}
  \seeref{蔓}{wan4}
\end{EntryWithPhonetic}

\begin{EntryWithPhonetic}{馒}{man2}{14}{⾷}
  \definition{s.}{pão cozido no vapor}
\end{EntryWithPhonetic}

\begin{EntryWithPhonetic}{馒头}{man2tou5}{14,5}{⾷、⼤}[HSK 6]
  \definition[个,锅,屉,筐]{s.}{pão cozido no vapor; um alimento cozido no vapor feito de farinha fermentada, geralmente redondo na parte superior e plano na parte inferior, sem recheio}
\end{EntryWithPhonetic}

\begin{EntryWithPhonetic}{满}{man3}{13}{⽔}[HSK 2]
  \definition*{s.}{Etnia Manchu | Sobrenome Man}
  \definition{adj.}{cheio; repleto; lotado; totalmente cheio; atingindo o limite da capacidade | tudo; inteiro; completo | presunçoso; complacente; orgulhoso}
  \definition{adv.}{muito; um tanto; bastante | completamente; inteiramente; perfeitamente}
  \definition{v.}{encher | sentir-se satisfeito; sentir que já é o suficiente | expirar; atingir o limite; atingir um determinado prazo ou limite}
\end{EntryWithPhonetic}

\begin{EntryWithPhonetic}{满分}{man3fen1}{13,4}{⽔、⼑}
  \definition{s.}{pontuação completa}
\end{EntryWithPhonetic}

\begin{EntryWithPhonetic}{满满}{man3man3}{13,13}{⽔、⽔}
  \definition{adj.}{completo | densamente empacotado}
\end{EntryWithPhonetic}

\begin{EntryWithPhonetic}{满意}{man3yi4}{13,13}{⽔、⼼}[HSK 2]
  \definition{adj.}{satisfeito; contente; gratificado}
  \definition{v.}{estar satisfeito; sentir-se contente; satisfazer os seus desejos; estar de acordo com os seus desejos}
\end{EntryWithPhonetic}

\begin{EntryWithPhonetic}{满足}{man3zu2}{13,7}{⽔、⾜}[HSK 3]
  \definition{v.}{estar satisfeito; contentar-se; sentir-se satisfeito | satisfazer}
\end{EntryWithPhonetic}

\begin{EntryWithPhonetic}{谩}{man4}{13}{⾔}
  \definition{v.}{ser desrespeitoso | caluniar | desconsiderar}
  \seeref{谩}{man2}
\end{EntryWithPhonetic}

\begin{EntryWithPhonetic}{谩骂}{man4ma4}{13,9}{⾔、⾺}
  \definition{v.}{ridicularizar | abusar}
\end{EntryWithPhonetic}

\begin{EntryWithPhonetic}{慢}{man4}{14}{⼼}[HSK 1]
  \definition*{s.}{Sobrenome Man}
  \definition{adj.}{lento; devagar; baixa velocidade; longa duração (em oposição a 快) | rude; arrogante; sem educação com as pessoas | frouxo; lento}
  \definition{adv.}{lentamente}
  \seealsoref{快}{kuai4}
\end{EntryWithPhonetic}

\begin{EntryWithPhonetic}{慢车}{man4 che1}{14,4}{⼼、⾞}[HSK 6]
  \definition{s.}{trem lento com muitas paradas (oposto a 快车) | ônibus ou trem local; parada do trem}
  \seealsoref{快车}{kuai4 che1}
\end{EntryWithPhonetic}

\begin{EntryWithPhonetic}{慢动作}{man4dong4zuo4}{14,6,7}{⼼、⼒、⼈}
  \definition{s.}{(cinema) câmera lenta}
\end{EntryWithPhonetic}

\begin{EntryWithPhonetic}{慢慢}{man4 man4}{14,14}{⼼、⼼}[HSK 3]
  \definition{adv.}{lentamente; vagarosamente; gradualmente | lentamente; vagarosamente; gradualmente; depois de um longo período de tempo}
\end{EntryWithPhonetic}

\begin{EntryWithPhonetic}{漫}{man4}{14}{⽔}
  \definition{adj.}{livre; irrestrito; casual | longo; extenso | em todos os lugares; por toda parte | aleatório; irrestrito; livre}
  \definition{adv.}{não}
  \definition{v.}{transbordar; inundar | estar em todo lugar}
\end{EntryWithPhonetic}

\begin{EntryWithPhonetic}{漫长}{man4chang2}{14,4}{⽔、⾧}[HSK 5]
  \definition{adj.}{muito longo; interminável; (tempo, espaço) dura muito tempo}
\end{EntryWithPhonetic}

\begin{EntryWithPhonetic}{漫画}{man4hua4}{14,8}{⽔、⽥}[HSK 5]
  \definition[幅,本,张,套]{s.}{desenho animado; caricatura; \emph{cartoon}}
\end{EntryWithPhonetic}

\begin{EntryWithPhonetic}{漫骂}{man4ma4}{14,9}{⽔、⾺}
  \variantof{谩骂}
\end{EntryWithPhonetic}

\begin{EntryWithPhonetic}{蔓}{man4}{14}{⾋}
  \definition{s.}{uma videira com gavinhas; caule fino que não consegue ficar em pé}
  \definition{v.}{rastejar; espalhar; estender}
  \seeref{蔓}{man2}
  \seeref{蔓}{wan4}
\end{EntryWithPhonetic}

\begin{EntryWithPhonetic}{蔓草}{man4cao3}{14,9}{⾋、⾋}
  \definition{s.}{videira | trepadeira}
\end{EntryWithPhonetic}

\begin{EntryWithPhonetic}{忙}{mang2}{6}{⼼}[HSK 1]
  \definition*{s.}{Sobrenome Mang}
  \definition{adj.}{ocupado; movimentado; totalmente ocupado; muitas coisas para fazer, sem tempo livre (oposto de 闲) | imperativo; ansioso; urgente}
  \definition{v.}{apressar-se; agitar-se; fazer algo com urgência e constantemente | trabalhar; fazer}
  \seealsoref{闲}{xian2}
\end{EntryWithPhonetic}

\begin{EntryWithPhonetic}{盲}{mang2}{8}{⽬}
  \definition{adj.}{cego | incapaz de distinguir coisas | totalmente incompetente}
  \definition{adv.}{cegamente}
\end{EntryWithPhonetic}

\begin{EntryWithPhonetic}{盲目}{mang2mu4}{8,5}{⽬、⽬}
  \definition{adj.}{ignorante | sem compreensão}
  \definition{adv.}{cegamente}
  \definition{s.}{cego}
\end{EntryWithPhonetic}

\begin{EntryWithPhonetic}{盲人}{mang2 ren2}{8,2}{⽬、⼈}[HSK 6]
  \definition[个,位,名]{s.}{cego; pessoa cega; pessoas com deficiência visual}
\end{EntryWithPhonetic}

\begin{EntryWithPhonetic}{猫}{mao1}{11}{⽝}[HSK 2]
  \definition*[只,种,群,窝,个]{s.}{gato |  Empréstimo linguístico: MODEM}
  \definition{v.}{esconder-se; entrar em esconderijo | inclinar-se para a frente; curvar-se}
  \seeref{猫}{mao2}
\end{EntryWithPhonetic}

\begin{EntryWithPhonetic}{猫熊}{mao1xiong2}{11,14}{⽝、⽕}
  \definition[把,只]{s.}{panda gigante}
  \seealsoref{熊猫}{xiong2mao1}
\end{EntryWithPhonetic}

\begin{EntryWithPhonetic}{毛}{mao2}{4}{⽑}[HSK 1,3][Kangxi 82]
  \definition*{s.}{Sobrenome Mao}
  \definition{adj.}{bruto; semiacabado | grosseiro | pequeno | fino | descuidado; rude; precipitado | assustado; nervoso; em pânico | impetuoso | rústico; sem acabamento | impuro | (de moeda) que não vale mais seu valor nominal; depreciado}
  \definition{clas.}{mao, uma unidade fracionária de dinheiro na China; dez centavos; uma peça de dez centavos}
  \definition[根]{s.}{(de um animal, planta, etc.) cabelo; pena; penugem | (de humanos) cabelo; barba | planta; colheita | lã | mofo; bolor}
  \definition{v.}{depreciar; desvalorizar; refere-se à desvalorização da moeda | (de cavalos, gado, etc.) assustar-se; sentir medo}
\end{EntryWithPhonetic}

\begin{EntryWithPhonetic}{毛笔}{mao2 bi3}{4,10}{⽑、⽵}[HSK 5]
  \definition[支,枝,根,管]{s.}{pincel para escrever; pincel chinês; canetas feitas com pelos de coelho, carneiro, doninha, etc., são materiais tradicionais utilizados para escrever caracteres chineses e pintar pinturas tradicionais chinesas}
\end{EntryWithPhonetic}

\begin{EntryWithPhonetic}{毛病}{mao2bing4}{4,10}{⽑、⽧}[HSK 3]
  \definition[个,点,种,些]{s.}{doença ou deficiência; condição de saúde precária ou deficiência física | problema; fracasso; onde o produto está com defeito ou não funciona corretamente | mau hábito; deficiência; falhas no comportamento humano}
\end{EntryWithPhonetic}

\begin{EntryWithPhonetic}{毛巾}{mao2jin1}{4,3}{⽑、⼱}[HSK 4]
  \definition[条,块]{s.}{toalha; toalha de banho}
\end{EntryWithPhonetic}

\begin{EntryWithPhonetic}{毛衣}{mao2 yi1}{4,6}{⽑、⾐}[HSK 4]
  \definition[件,个]{s.}{suéter; blusa feita de lã}
\end{EntryWithPhonetic}

\begin{EntryWithPhonetic}{矛}{mao2}{5}{⽭}[Kangxi 110]
  \definition{s.}{Arcaico: lança; lanceta}
\end{EntryWithPhonetic}

\begin{EntryWithPhonetic}{矛盾}{mao2dun4}{5,9}{⽭、⽬}[HSK 5]
  \definition{adj.}{contraditório; descreve pessoas ou coisas que se opõem ou se repelem mutuamente}
  \definition[对,个,种]{s.}{problema; contradição; discrepância; inconsistência | disputas e conflitos; relacionamento de oposição entre as duas partes devido a diferenças de opinião ou abordagem}
  \definition{v.}{opor-se; entrar em conflito; contradizer; nesta situação, apenas uma das opções está correta ou é verdadeira; não é possível que ambas estejam corretas ao mesmo tempo}
\end{EntryWithPhonetic}

\begin{EntryWithPhonetic}{牦}{mao2}{8}{⽜}
  \definition[头]{s.}{iaque; boi da Tartária}
\end{EntryWithPhonetic}

\begin{EntryWithPhonetic}{牦牛}{mao2niu2}{8,4}{⽜、⽜}
  \definition{s.}{iaque}
\end{EntryWithPhonetic}

\begin{EntryWithPhonetic}{茅}{mao2}{8}{⾋}
  \definition*{s.}{Sobrenome Mao}
  \definition[座]{s.}{capim-cogon | planta semelhante ao capim-cogon (como palha)}
\end{EntryWithPhonetic}

\begin{EntryWithPhonetic}{茅厕}{mao2ce4}{8,8}{⾋、⼚}
  \definition{s.}{(dialeto) latrina}
\end{EntryWithPhonetic}

\begin{EntryWithPhonetic}{猫}{mao2}{11}{⽝}
  \definition{v.}{utilizado em 猫腰 \dpy{mao2yao1}}
  \seeref{猫}{mao1}
  \seealsoref{猫腰}{mao2yao1}
\end{EntryWithPhonetic}

\begin{EntryWithPhonetic}{猫腰}{mao2yao1}{11,13}{⽝、⾁}
  \definition{v.}{curvar-se}
\end{EntryWithPhonetic}

\begin{EntryWithPhonetic}{冒}{mao4}{9}{⽇}[HSK 5]
  \definition*{s.}{Sobrenome Mao}
  \definition{adv.}{com ousadia; precipitadamente | fingidamente; falsamente; fraudulentamente}
  \definition{v.}{emitir; liberar; enviar (para cima) | arriscar; ser corajoso}
\end{EntryWithPhonetic}

\begin{EntryWithPhonetic}{冒险}{mao4xian3}{9,9}{⽇、⾩}
  \definition{adj.}{corajoso}
  \definition{s.}{risco | aventura}
  \definition{v.+compl.}{correr risco | arriscar-se | aventurar-se em}
\end{EntryWithPhonetic}

\begin{EntryWithPhonetic}{贸}{mao4}{9}{⾙}
  \definition*{s.}{Sobrenome Mao}
  \definition{s.}{comércio; negociação}
\end{EntryWithPhonetic}

\begin{EntryWithPhonetic}{贸易}{mao4yi4}{9,8}{⾙、⽇}[HSK 5]
  \definition[笔,宗,项,个]{s.}{comércio; troca; negócios; refere-se a atividades comerciais, como a troca de mercadorias}
  \definition{v.}{fazer uma transação comercial}
\end{EntryWithPhonetic}

\begin{EntryWithPhonetic}{帽}{mao4}{12}{⼱}
  \definition[个,顶]{s.}{chapéu; boné | capa; uma coisa que cobre um objeto e tem a função ou formato de um chapéu | elmo; capacete}
\end{EntryWithPhonetic}

\begin{EntryWithPhonetic}{帽子}{mao4zi5}{12,3}{⼱、⼦}[HSK 4]
  \definition[顶,个,种]{s.}{boné; chapéu; capacete | etiqueta; rótulo; marca}
\end{EntryWithPhonetic}

\begin{EntryWithPhonetic}{没}{mei2}{7}{⽔}[HSK 1]
  \definition{adv.}{não; nunca; negar que uma ação ou situação tenha ocorrido, com o significado de 不曾}
  \definition{pref.}{não (prefixo negativo para verbos, traduzido para outras línguas com verbos no pretérito)}
  \definition{v.}{não possuir; não ter | não existe; não há | ninguém; usado antes de 谁, 什么, 哪个, significa 全都不 | não ser tão bom quanto; ser inferior a; não chega a; não é tão bom quanto | menor que; insuficiente}
  \seeref{没}{mo4}
  \seealsoref{不曾}{bu4 ceng2}
  \seealsoref{哪个}{na3ge5}
  \seealsoref{全都不}{quan2dou1 bu4}
  \seealsoref{谁}{shei2}
  \seealsoref{什么}{shen2me5}
\end{EntryWithPhonetic}

\begin{EntryWithPhonetic}{没错}{mei2 cuo4}{7,13}{⽔、⾦}[HSK 4]
  \definition{adv.}{está certo; é isso mesmo; não há como errar}
\end{EntryWithPhonetic}

\begin{EntryWithPhonetic}{没法儿}{mei2 fa3r5}{7,8,2}{⽔、⽔、⼉}[HSK 4]
  \definition{adv.}{não pode; sem chance}
\end{EntryWithPhonetic}

\begin{EntryWithPhonetic}{没关系}{mei2guan1xi5}{7,6,7}{⽔、⼋、⽷}[HSK 1]
  \definition{v.}{está tudo bem; não é nada; não importa; não se preocupe}
  \seealsoref{没有关系}{mei2you3guan1xi5}
\end{EntryWithPhonetic}

\begin{EntryWithPhonetic}{没了}{mei2le5}{7,2}{⽔、⼅}
  \definition{v.}{estar morto | deixar de existir}
\end{EntryWithPhonetic}

\begin{EntryWithPhonetic}{没什么}{mei2 shen2 me5}{7,4,3}{⽔、⼈、⼃}[HSK 1]
  \definition{expr.}{não é nada; está tudo bem; não importa}
\end{EntryWithPhonetic}

\begin{EntryWithPhonetic}{没事儿}{mei2 shi4r5}{7,8,2}{⽔、⼅、⼉}[HSK 1]
  \definition{expr.}{fora de perigo; nada sério | não importa; não é nada; está tudo bem; não importa | está tudo bem; sem problemas; não se preocupe com isso; não é grande coisa; não há nada errado}
  \definition{v.}{não ter nada para fazer; ser livre; estar perdido | estar desempregado; estar sem trabalho | não ter responsabilidade}
\end{EntryWithPhonetic}

\begin{EntryWithPhonetic}{没想到}{mei2 xiang3 dao4}{7,13,8}{⽔、⼼、⼑}[HSK 4]
  \definition{expr.}{não esperava; inesperado}
\end{EntryWithPhonetic}

\begin{EntryWithPhonetic}{没用}{mei2 yong4}{7,5}{⽔、⽤}[HSK 3]
  \definition{adj.}{inútil; imprestável; sem valor; sem préstimo; vão; que não serve para nada}
\end{EntryWithPhonetic}

\begin{EntryWithPhonetic}{没有}{mei2 you3}{7,6}{⽔、⽉}[HSK 1]
  \definition{adv.}{ainda não; (usado com o pretérito) não; ação ou estado negativo ocorreu}
  \definition{v.}{não há; não tem; não existe}
\end{EntryWithPhonetic}

\begin{EntryWithPhonetic}{没有次序}{mei2you3 ci4xu4}{7,6,6,7}{⽔、⽉、⽋、⼴}
  \definition{adj.}{sem ordem; nenhuma ordem}
\end{EntryWithPhonetic}

\begin{EntryWithPhonetic}{没有关系}{mei2you3guan1xi5}{7,6,6,7}{⽔、⽉、⼋、⽷}
  \definition{expr.}{Está tudo bem; sem problemas}
  \seealsoref{没关系}{mei2guan1xi5}
\end{EntryWithPhonetic}

\begin{EntryWithPhonetic}{没有哪一种东西}{mei2you3 na3 yi4 zhong3 dong1xi1}{7,6,9,1,9,5,6}{⽔、⽉、⼝、⼀、⽲、⼀、⾑}
  \definition{pron.}{nada; não existe tal coisa}
\end{EntryWithPhonetic}

\begin{EntryWithPhonetic}{没有谁}{mei2you3 shei2}{7,6,10}{⽔、⽉、⾔}
  \definition{pron.}{ninguém}
\end{EntryWithPhonetic}

\begin{EntryWithPhonetic}{没有意思}{mei2you3yi4si5}{7,6,13,9}{⽔、⽉、⼼、⼼}
  \definition{adj.}{tedioso | chato | sem interesse}
\end{EntryWithPhonetic}

\begin{EntryWithPhonetic}{眉}{mei2}{9}{⽬}
  \definition{s.}{sobrancelha | margem superior}
\end{EntryWithPhonetic}

\begin{EntryWithPhonetic}{眉毛}{mei2mao5}{9,4}{⽬、⽑}
  \definition[根]{s.}{sobrancelha}
\end{EntryWithPhonetic}

\begin{EntryWithPhonetic}{眉头}{mei2tou2}{9,5}{⽬、⼤}
  \definition{s.}{testa}
\end{EntryWithPhonetic}

\begin{EntryWithPhonetic}{梅}{mei2}{11}{⽊}
  \definition*{s.}{Sobrenome Mei}
  \definition{s.}{ameixa | flor de ameixa | ameixeira | estação chuvosa}
\end{EntryWithPhonetic}

\begin{EntryWithPhonetic}{梅花}{mei2 hua1}{11,7}{⽊、⾋}[HSK 6]
  \definition[朵,枝,片,瓣,束,株]{s.}{paus ♣ (um naipe em jogos de cartas) | flor de ameixa | doçura-de-inverno; refere-se especificamente à flor-de-inverno ; também se refere a algo que se parece com esta flor}
  \seealsoref{方片}{fang1 pian4}
  \seealsoref{黑桃}{hei1 tao2}
  \seealsoref{红心}{hong2 xin1}
\end{EntryWithPhonetic}

\begin{EntryWithPhonetic}{梅赛德斯-奔驰}{mei2sai4de2si1-ben1chi2}{11,14,15,12,8,6}{⽊、⾙、⼻、⽄、⼤、⾺}
  \definition*{s.}{Mercedes-Benz}
\end{EntryWithPhonetic}

\begin{EntryWithPhonetic}{媒}{mei2}{12}{⼥}
  \definition{s.}{casamenteiro; intermediário | intermediário; médio}
  \definition{v.}{fazer uma combinação}
\end{EntryWithPhonetic}

\begin{EntryWithPhonetic}{媒体}{mei2ti3}{12,7}{⼥、⼈}[HSK 3]
  \definition[家,个,种]{s.}{mídia; mídia de massa; vários meios de comunicação e transmissão de informações, como televisão, rádio, jornais, etc.}
\end{EntryWithPhonetic}

\begin{EntryWithPhonetic}{煤}{mei2}{13}{⽕}[HSK 5]
  \definition[块,吨,斤,堆]{s.}{carvão; carvão vegetal; minério sólido preto}
\end{EntryWithPhonetic}

\begin{EntryWithPhonetic}{煤气}{mei2 qi4}{13,4}{⽕、⽓}[HSK 5]
  \definition[罐,瓶]{s.}{gás; gás de carvão; gás obtido a partir do processamento do carvão não tem cor nem odor, é tóxico e pode ser queimado ou utilizado como matéria-prima na indústria química | envenenamento por monóxido de carbono}
\end{EntryWithPhonetic}

\begin{EntryWithPhonetic}{每}{mei3}{7}{⽏}[HSK 3]
  \definition{adv.}{cada um; cada qual; indica qualquer uma das repetições ou um conjunto de repetições de um movimento}
  \definition{pron.}{cada; cada um; cada qual; refere-se a qualquer indivíduo do grupo, enfatizando as semelhanças entre os indivíduos}
\end{EntryWithPhonetic}

\begin{EntryWithPhonetic}{每次}{mei3ci4}{7,6}{⽏、⽋}
  \definition{adv.}{toda vez | cada vez}
\end{EntryWithPhonetic}

\begin{EntryWithPhonetic}{每个}{mei3ge4}{7,3}{⽏、⼈}
  \definition{pron.}{cada; cada um}
\end{EntryWithPhonetic}

\begin{EntryWithPhonetic}{每个人}{mei3ge5ren2}{7,3,2}{⽏、⼈、⼈}
  \definition{pron.}{todo mundo | todos}
\end{EntryWithPhonetic}

\begin{EntryWithPhonetic}{每天}{mei3tian1}{7,4}{⽏、⼤}
  \definition{adv.}{todo dia | cada dia}
\end{EntryWithPhonetic}

\begin{EntryWithPhonetic}{美}{mei3}{9}{⽺}[HSK 3]
  \definition*{s.}{Abreviatura de América, 美洲 | Abreviatura de Estados Unidos da América, 美国 | As Américas, 美洲}
  \definition{adj.}{belo; bonito (oposto de 丑) | muito satisfatório; bom; agradável}
  \definition{s.}{beleza (oposto de 丑)}
  \definition{v.}{embelezar; tornar mais bonito | estar satisfeito consigo mesmo; orgulhar-se; sentir-se presunçoso}
  \seealsoref{丑}{chou3}
  \seealsoref{美国}{mei3guo2}
  \seealsoref{美洲}{mei3zhou1}
\end{EntryWithPhonetic}

\begin{EntryWithPhonetic}{美国}{mei3guo2}{9,8}{⽺、⼞}
  \definition*{s.}{Estados Unidos da América}
\end{EntryWithPhonetic}

\begin{EntryWithPhonetic}{美国人}{mei3guo2ren2}{9,8,2}{⽺、⼞、⼈}
  \definition{s.}{americano | pessoa ou povo dos Estados Unidos da América}
\end{EntryWithPhonetic}

\begin{EntryWithPhonetic}{美好}{mei3 hao3}{9,6}{⽺、⼥}[HSK 3]
  \definition{adj.}{bem; feliz; glorioso; descreve a vida, os desejos, etc. como sendo muito bons e satisfatórios}
\end{EntryWithPhonetic}

\begin{EntryWithPhonetic}{美甲}{mei3jia3}{9,5}{⽺、⽥}
  \definition{s.}{manicure e/ou pedicure}
\end{EntryWithPhonetic}

\begin{EntryWithPhonetic}{美金}{mei3 jin1}{9,8}{⽺、⾦}[HSK 4]
  \definition{s.}{USD; dólar americano: a moeda local dos Estados Unidos}
\end{EntryWithPhonetic}

\begin{EntryWithPhonetic}{美丽}{mei3li4}{9,7}{⽺、⼀}[HSK 3]
  \definition{adj.}{bonito; lindo; capaz de proporcionar uma sensação de beleza}
\end{EntryWithPhonetic}

\begin{EntryWithPhonetic}{美女}{mei3 nv3}{9,3}{⽺、⼥}[HSK 4]
  \definition[个,位,名,些]{s.}{beldade; mulher bonita; uma jovem linda}
\end{EntryWithPhonetic}

\begin{EntryWithPhonetic}{美容}{mei3 rong2}{9,10}{⽺、⼧}[HSK 6]
  \definition{v.}{embelezar; melhorar a aparência de alguém; deixar seu rosto bonito retocando, cuidando, etc.}
\end{EntryWithPhonetic}

\begin{EntryWithPhonetic}{美食}{mei3 shi2}{9,9}{⽺、⾷}[HSK 3]
  \definition[种,道,桌]{s.}{iguaria; (gastronomia) comida saborosa}
\end{EntryWithPhonetic}

\begin{EntryWithPhonetic}{美术}{mei3shu4}{9,5}{⽺、⽊}[HSK 3]
  \definition[种]{s.}{arte; artes plásticas: arte que ocupa um determinado espaço, compõe imagens estéticas e permite que as pessoas apreciem visualmente, incluindo pintura, escultura, arquitetura, etc. | pintura; pintura tradicional chinesa}
\end{EntryWithPhonetic}

\begin{EntryWithPhonetic}{美味}{mei3wei4}{9,8}{⽺、⼝}
  \definition{adj.}{delicioso}
  \definition{s.}{comida deliciosa | delicadeza (\emph{delicacy})}
\end{EntryWithPhonetic}

\begin{EntryWithPhonetic}{美学}{mei3xue2}{9,8}{⽺、⼦}
  \definition{s.}{estética; a ciência que estuda as leis e os princípios gerais da beleza na natureza, na sociedade e na arte explora principalmente a natureza da beleza, a relação entre arte e realidade e as leis gerais da criação artística}
\end{EntryWithPhonetic}

\begin{EntryWithPhonetic}{美元}{mei3yuan2}{9,4}{⽺、⼉}[HSK 3]
  \definition*[元,笔,沓]{s.}{Dólar Americano; a moeda dos Estados Unidos}
\end{EntryWithPhonetic}

\begin{EntryWithPhonetic}{美洲}{mei3zhou1}{9,9}{⽺、⽔}
  \definition*{s.}{América (incluindo Norte, Central e Sul)}
\end{EntryWithPhonetic}

\begin{EntryWithPhonetic}{美洲人}{mei3zhou1ren2}{9,9,2}{⽺、⽔、⼈}
  \definition{s.}{americano | pessoa ou povo do continente Americano}
\end{EntryWithPhonetic}

\begin{EntryWithPhonetic}{妹}{mei4}{8}{⼥}[HSK 1]
  \definition*{s.}{Sobrenome Mei}
  \definition[个]{s.}{irmã mais nova | parente do sexo feminino da mesma geração | jovem garota; jovem mulher ou menina}
  \seealsoref{妹妹}{mei4 mei5}
\end{EntryWithPhonetic}

\begin{EntryWithPhonetic}{妹夫}{mei4fu5}{8,4}{⼥、⼤}
  \definition{s.}{marido da irmã mais nova}
\end{EntryWithPhonetic}

\begin{EntryWithPhonetic}{妹妹}{mei4 mei5}{8,8}{⼥、⼥}[HSK 1]
  \definition[个]{s.}{irmã mais nova}
\end{EntryWithPhonetic}

\begin{EntryWithPhonetic}{魅}{mei4}{14}{⿁}
  \definition{s.}{espírito maligno; demônio | \emph{goblin}; trasgo; gnomo; duende maléfico}
  \definition{v.}{atormentar; cativar}
\end{EntryWithPhonetic}

\begin{EntryWithPhonetic}{魅力}{mei4li4}{14,2}{⿁、⼒}
  \definition{s.}{charme | fascínio | glamour | carisma}
\end{EntryWithPhonetic}

\begin{EntryWithPhonetic}{闷}{men1}{7}{⾨}
  \definition{adj.}{abafado; fechado; sufocante; baixa pressão de ar ou má circulação de ar | abafado; som baixo ou opaco}
  \definition{v.}{cubrir bem; fazer algo hermético | ficar sem fala; parar de falar | fechar a si mesmo ou alguém dentro de casa; ficar em casa e não sair}
  \seeref{闷}{men4}
\end{EntryWithPhonetic}

\begin{EntryWithPhonetic}{闷热}{men1re4}{7,10}{⾨、⽕}
  \definition{adj.}{abafado | quente e abafado | sufocantemente quente | quente e sensual}
\end{EntryWithPhonetic}

\begin{EntryWithPhonetic}{门}{men2}{3}{⾨}[HSK 1][Kangxi 169]
  \definition*{s.}{Sobrenome Men}
  \definition{clas.}{para equipamentos de artilharia (por exemplo: canhões) | para trabalhos escolares, ciência e tecnologia, etc. | para idiomas | para casamentos | para parentes}
  \definition[个,把,道,扇]{s.}{entradas e saídas de edifícios, veículos, navios, aviões, etc. | válvula; interruptor; algo que funciona como um interruptor ou como uma porta | habilidade; método; acesso; maneira de fazer algo | família; ramo de uma família ou clã | seita (religiosa); escola (de pensamento); faculdades acadêmicas, ideológicas ou religiosas | classe; categoria; ramo de estudo; refere-se à categoria geral de coisas | filo; segundo nível da classificação biológica | (computador) \emph{gate}; porta (lógica) | porta; portão; entrada; refere-se a uma porta que pode ser aberta e fechada, instalada na entrada e saída | qualquer abertura; partes de objetos que podem ser abertas e fechadas | orifício no corpo humano; refere-se especificamente aos orifícios do corpo humano | estudar com o mesmo professor; refere-se especificamente ao professor ou mestre | posição em um jogo de apostas (em relação ao local onde se senta ou onde se faz uma aposta)}
\end{EntryWithPhonetic}

\begin{EntryWithPhonetic}{门口}{men2 kou3}{3,3}{⾨、⼝}[HSK 1]
  \definition[个]{s.}{porta; portão; entrada; porta de entrada}
\end{EntryWithPhonetic}

\begin{EntryWithPhonetic}{门票}{men2 piao4}{3,11}{⾨、⽰}[HSK 1]
  \definition{s.}{bilhete de entrada; bilhete de admissão; ingressos para locais de turismo, entretenimento, etc.}
\end{EntryWithPhonetic}

\begin{EntryWithPhonetic}{门诊}{men2 zhen3}{3,7}{⾨、⾔}[HSK 5]
  \definition{s.}{(no hospital) clínica ambulatorial; seção para pacientes ambulatoriais; local onde os médicos atendem pacientes que não estão internados no hospital}
\end{EntryWithPhonetic}

\begin{EntryWithPhonetic}{闷}{men4}{7}{⾨}
  \definition{adj.}{entediado; deprimido; irritado; desanimado | hermeticamente fechado; selado | triste e silencioso; chateado | hermético}
  \definition{s.}{desânimo}
  \seeref{闷}{men1}
\end{EntryWithPhonetic}

\begin{EntryWithPhonetic}{们}{men5}{5}{⼈}[HSK 1]
  \definition{suf.}{usado após pronomes ou substantivos que se referem a pessoas para indicar pluralidade}
\end{EntryWithPhonetic}

\begin{EntryWithPhonetic}{蒙}{meng1}{13}{⾋}[HSK 6]
  \definition{adj.}{inconsciente; sem sentido;  em coma | confuso; perplexo}
  \definition{v.}{enganar; enganar; trapacear; iludir; trair | fazer um palpite ousado; dar um palpite ousado; arriscar-se}
  \seeref{蒙}{meng2}
  \seeref{蒙}{meng3}
\end{EntryWithPhonetic}

\begin{EntryWithPhonetic}{蒙}{meng2}{13}{⾋}[HSK 6]
  \definition*{s.}{Sobrenome Meng}
  \definition{adj.}{ignorância; analfabetismo; falta de instrução | nebuloso; aparência pequena e pouco clara, como chuva ou neblina}
  \definition{s.}{aberto; inicial}
  \definition{v.}{cobrir; espalhar | receber apoio | receber; encontrar-se com; encontrar-se; palavras respeitosas; expressam os benefícios recebidos de outros | sofrer; incorrer}
  \seeref{蒙}{meng1}
  \seeref{蒙}{meng3}
\end{EntryWithPhonetic}

\begin{EntryWithPhonetic}{蒙面}{meng2mian4}{13,9}{⾋、⾯}
  \definition{adj.}{descarado | desavergonhado | mascarado}
  \definition{v.}{cobrir o rosto | usar uma máscara}
\end{EntryWithPhonetic}

\begin{EntryWithPhonetic}{猛}{meng3}{11}{⽝}[HSK 6]
  \definition*{s.}{Sobrenome Meng}
  \definition{adj.}{feroz; violento | enérgico; vigoroso | valente}
  \definition{adv.}{de repente; abruptamente | vigorosamente; com força repentina | (coloquial) ao contentamento do coração; de todo o coração | ferozmente; violentamente}
\end{EntryWithPhonetic}

\begin{EntryWithPhonetic}{猛然}{meng3ran2}{11,12}{⽝、⽕}
  \definition{adv.}{de repente; abruptamente; indica ação repentina e rápida}
\end{EntryWithPhonetic}

\begin{EntryWithPhonetic}{蒙}{meng3}{13}{⾋}
  \definition{s.}{grupo étnico mongol; mongol}
  \seeref{蒙}{meng1}
  \seeref{蒙}{meng2}
\end{EntryWithPhonetic}

\begin{EntryWithPhonetic}{懵}{meng3}{18}{⼼}
  \definition{adj.}{confuso; ignorante; irracional | inconsciente; entorpecido}
\end{EntryWithPhonetic}

\begin{EntryWithPhonetic}{懵懂}{meng3dong3}{18,15}{⼼、⼼}
  \definition{adj.}{confuso | ignorante}
\end{EntryWithPhonetic}

\begin{EntryWithPhonetic}{梦}{meng4}{11}{⼣}[HSK 4]
  \definition*{s.}{Sobrenome Meng}
  \definition[个,场]{s.}{sonho; atividade de representação no cérebro durante o sono}
  \definition{v.}{sonhar; ter um sonho}
\end{EntryWithPhonetic}

\begin{EntryWithPhonetic}{梦见}{meng4 jian4}{11,4}{⼣、⾒}[HSK 4]
  \definition{v.}{sonhar; sonhar com; ver em um sonho}
\end{EntryWithPhonetic}

\begin{EntryWithPhonetic}{梦想}{meng4xiang3}{11,13}{⼣、⼼}[HSK 4]
  \definition[个,种,些,番]{s.}{sonho; esperança vã; sonho irreal; divagação; um desejo ou ideia que você espera particularmente realizar}
  \definition{v.}{sonhar; desejar sinceramente; ansiar}
\end{EntryWithPhonetic}

\begin{EntryWithPhonetic}{眯}{mi1}{11}{⽬}
  \definition{v.}{estreitar os olhos | esmagar | (dialeto) tirar uma soneca}
  \seeref{眯}{mi2}
\end{EntryWithPhonetic}

\begin{EntryWithPhonetic}{迷}{mi2}{9}{⾡}[HSK 3]
  \definition[个]{s.}{fã; entusiasta; aficionado; pessoa que gosta excessivamente de algo}
  \definition{v.}{estar confuso; perder o rumo; se perder-se; perda da capacidade de discernimento e julgamento | ficar fascinado por; entregar-se a; ficar encantado com (por); ser louco por | confundir; desorientar; fascinar; encantar; tornar indistinto; deixar encantado e fascinado}
\end{EntryWithPhonetic}

\begin{EntryWithPhonetic}{迷宫}{mi2gong1}{9,9}{⾡、⼧}
  \definition{s.}{labirinto}
\end{EntryWithPhonetic}

\begin{EntryWithPhonetic}{迷恋}{mi2lian4}{9,10}{⾡、⼼}
  \definition{adj.}{obcecado}
  \definition{v.}{estar/ser apaixonado por | ficar encantado por | estar/ser obcecado por}
\end{EntryWithPhonetic}

\begin{EntryWithPhonetic}{迷路}{mi2lu4}{9,13}{⾡、⾜}
  \definition{s.}{labirinto | ouvido interno}
  \definition{v.+compl.}{perder o caminho | perder-se | seguir pelo caminho errado | não conseguir encontrar o caminho}
\end{EntryWithPhonetic}

\begin{EntryWithPhonetic}{迷你}{mi2ni3}{9,7}{⾡、⼈}
  \definition{adj.}{(empréstimo linguístico) mini, como em minissaia ou \emph{Mini Cooper}}
\end{EntryWithPhonetic}

\begin{EntryWithPhonetic}{迷人}{mi2ren2}{9,2}{⾡、⼈}[HSK 5]
  \definition{adj.}{encantador; fascinante; sedutor; hipnotizante}
  \definition{v.}{confundir; intrigar; enganar}
\end{EntryWithPhonetic}

\begin{EntryWithPhonetic}{迷信}{mi2xin4}{9,9}{⾡、⼈}[HSK 5]
  \definition{s.}{superstição; crença supersticiosa | fé cega; adoração cega}
  \definition{v.}{ter fé cega em; ter um fetiche de}
\end{EntryWithPhonetic}

\begin{EntryWithPhonetic}{眯}{mi2}{11}{⽬}
  \definition{v.}{cegar (como com poeira)}
  \seeref{眯}{mi1}
\end{EntryWithPhonetic}

\begin{EntryWithPhonetic}{米}{mi3}{6}{⽶}[HSK 2,3][Kangxi 119]
  \definition*{s.}{Sobrenome Mi}
  \definition{clas.}{m, metro; unidade principal de comprimento do sistema métrico}
  \definition[粒,斤]{s.}{arroz | sementes descascadas; refere-se a sementes comestíveis descascadas ou sem casca | qualquer coisa que se assemelhe a um grão de arroz}
\end{EntryWithPhonetic}

\begin{EntryWithPhonetic}{米饭}{mi3fan4}{6,7}{⽶、⾷}[HSK 1]
  \definition{s.}{arroz (cozido)}
\end{EntryWithPhonetic}

\begin{EntryWithPhonetic}{秘}{mi4}{10}{⽲}
  \definition{adj.}{secreto; misterioso | raro; raramente visto; estranho}
  \definition{adv.}{secretamente; privadamente}
  \definition{s.}{secretário}
  \definition{v.}{manter algo em segredo; esconder algo; guardar segredos | bloquear; obstruir; ter dificuldade para defecar}
  \seeref{秘}{bi4}
\end{EntryWithPhonetic}

\begin{EntryWithPhonetic}{秘密}{mi4mi4}{10,11}{⽲、⼧}[HSK 4]
  \definition{adj.}{secreto}
  \definition[个,条,些]{s.}{segredo; algo secreto; coisas que você não quer que as pessoas saibam}
\end{EntryWithPhonetic}

\begin{EntryWithPhonetic}{秘书}{mi4shu1}{10,4}{⽲、⼄}[HSK 4]
  \definition[个,位,名]{s.}{o cargo de secretário; funções de secretariado | secretário; pessoas encarregadas da correspondência e que auxiliam o chefe do órgão ou departamento na condução diária de seu trabalho}
\end{EntryWithPhonetic}

\begin{EntryWithPhonetic}{秘书长}{mi4 shu1 zhang3}{10,4,4}{⽲、⼄、⾧}[HSK 6]
  \definition{s.}{secretário-geral}
\end{EntryWithPhonetic}

\begin{EntryWithPhonetic}{密}{mi4}{11}{⼧}[HSK 4]
  \definition*{s.}{Sobrenome Mi}
  \definition{adj.}{fechado; denso; espesso | íntimo; próximo; afetuoso | delicado; fino; cuidadoso; meticuloso}
  \definition{adv.}{secretamente}
  \definition{s.}{segredo | densidade | senha; \emph{password}}
\end{EntryWithPhonetic}

\begin{EntryWithPhonetic}{密码}{mi4ma3}{11,8}{⼧、⽯}[HSK 4]
  \definition[个,种]{s.}{código; senha; um código secreto especialmente formulado usado entre as partes acordadas (diferente do 明码)}
  \seealsoref{明码}{ming2ma3}
\end{EntryWithPhonetic}

\begin{EntryWithPhonetic}{密切}{mi4qie4}{11,4}{⼧、⼑}[HSK 4]
  \definition{adj.}{próximo; íntimo; relacionamento próximo}
  \definition{adv.}{cuidadosamente; atentamente; intimamente}
  \definition{v.}{tornar-se próximo; tornar-se íntimo; conectar-se}
\end{EntryWithPhonetic}

\begin{EntryWithPhonetic}{蜜}{mi4}{14}{⾍}
  \definition{adj.}{melado; doce}
  \definition{s.}{mel | semelhante ao mel | coisas parecidas com mel; melaço}
\end{EntryWithPhonetic}

\begin{EntryWithPhonetic}{蜜桃}{mi4tao2}{14,10}{⾍、⽊}
  \definition{s.}{pêssego suculento}
\end{EntryWithPhonetic}

\begin{EntryWithPhonetic}{棉}{mian2}{12}{⽊}
  \definition{adj.}{almofadado com algodão; acolchoado}
  \definition[些,种,类]{s.}{termo genérico para algodão ou paina | algodão | material semelhante ao algodão | acolchoado ou estofado de algodão}
\end{EntryWithPhonetic}

\begin{EntryWithPhonetic}{免}{mian3}{7}{⼉}
  \definition*{s.}{Sobrenome Mian}
  \definition{v.}{desculpar alguém de algo; isentar; dispensar; renunciar | remover do cargo; demitir | evitar; desviar; escapar | não deveria ser permitido; não precisar fazer algo | remover; livrar-se de | isentar; dispensar | não permitir}
\end{EntryWithPhonetic}

\begin{EntryWithPhonetic}{免得}{mian3de5}{7,11}{⼉、⼻}[HSK 6]
  \definition{conj.}{de modo a não; para evitar; para que não; indica evitar uma situação que não é desejável e é frequentemente usado no início da oração seguinte}
\end{EntryWithPhonetic}

\begin{EntryWithPhonetic}{免费}{mian3fei4}{7,9}{⼉、⾙}[HSK 4]
  \definition{v.+compl.}{isentar de taxas; tonar grátis}
\end{EntryWithPhonetic}

\begin{EntryWithPhonetic}{免税}{mian3shui4}{7,12}{⼉、⽲}
  \definition{adj.}{isento de impostos (tributação)}
  \definition{s.}{livre de impostos | isenção de impostos}
  \definition{v.+compl.}{isentar impostos}
\end{EntryWithPhonetic}

\begin{EntryWithPhonetic}{靣}{mian4}{8}{⼀}[Kangxi 176]
  \variantof{面}
\end{EntryWithPhonetic}

\begin{EntryWithPhonetic}{面}{mian4}{9}{⾯}[HSK 2][Kangxi 176]
  \definition*{s.}{Sobrenome Mian}
  \definition{adj.}{macio e farinhento; descreve algo que é muito macio ao comer | superficial}
  \definition{adv.}{diretamente; pessoalmente; na frente de alguém; cara a cara}
  \definition{clas.}{usado para objetos planos | usado para indicar o número de vezes que as pessoas se encontram}
  \definition[斤,两,碗]{s.}{face; parte frontal da cabeça; rosto | topo; superfície | capa; exterior; a parte externa de um objeto ou a face frontal de um tecido (em oposição à 里) | (matemática) superfície | cara; sentimento; emoção | geral; área total; abrangente; toda a região | lado; aspecto | escopo; escala; extensão; alcance; âmbito | farinha; farinha de trigo | pó; algo em pó | macarrão; \emph{noodle}}
  \definition{suf.}{sufixo para localização ou direção; anexado ao final de palavras que indicam localização, equivalente a 边}
  \definition{v.}{encarar algo | encontrar; revelar-se}
  \seealsoref{边}{bian1}
  \seealsoref{里}{li3}
\end{EntryWithPhonetic}

\begin{EntryWithPhonetic}{面包}{mian4bao1}{9,5}{⾯、⼓}[HSK 1]
  \definition[个,片,袋,块]{s.}{pão}[我买八个面包了。===Comprei oito pães. | 他在吃两片面包。===Ele está comendo duas fatias de pão. | 我在家里带了一袋面包。===Trouxe um saco de pão para casa. | 我拿了一块面包。===Peguei um pedaço de pão.]
\end{EntryWithPhonetic}

\begin{EntryWithPhonetic}{面对}{mian4dui4}{9,5}{⾯、⼨}[HSK 3]
  \definition{v.}{enfrentar; defrontar; olhar para (uma pessoa ou um objeto específico) | confrontar (problema); problemas, dificuldades e outras questões que precisam ser resolvidas e que merecem atenção}
\end{EntryWithPhonetic}

\begin{EntryWithPhonetic}{面对面}{mian4 dui4 mian4}{9,5,9}{⾯、⼨、⾯}[HSK 6]
  \definition*{expr.}{frente a frente; cara a cara; vis-à-vis}
\end{EntryWithPhonetic}

\begin{EntryWithPhonetic}{面对面吃面}{mian4dui4mian4 chi1 mian4}{9,5,9,6,9}{⾯、⼨、⾯、⼝、⾯}
  \definition{expr.}{Comer macarrão cara a cara; indica que o seu estado atual, ou algumas das posições em que você está, ou algumas das coisas que você fez são muito claras}
\end{EntryWithPhonetic}

\begin{EntryWithPhonetic}{面积}{mian4ji1}{9,10}{⾯、⽲}[HSK 3]
  \definition{s.}{área (de um andar, pedaço de terreno, etc.); área de uma superfície; o tamanho de uma superfície plana ou da superfície de um objeto}
\end{EntryWithPhonetic}

\begin{EntryWithPhonetic}{面临}{mian4lin2}{9,9}{⾯、⼁}[HSK 4]
  \definition{v.}{ser confrontado com; encontrar (uma situação) na frente de}
\end{EntryWithPhonetic}

\begin{EntryWithPhonetic}{面貌}{mian4mao4}{9,14}{⾯、⾘}[HSK 5]
  \definition[种,个]{s.}{rosto; traços faciais; formato do rosto; aparência | aparência; aspecto; aparência (das coisas)}
\end{EntryWithPhonetic}

\begin{EntryWithPhonetic}{面前}{mian4 qian2}{9,9}{⾯、⼑}[HSK 2]
  \definition{s.}{antes; na frente de; diante de}
\end{EntryWithPhonetic}

\begin{EntryWithPhonetic}{面试}{mian4 shi4}{9,8}{⾯、⾔}[HSK 4]
  \definition{v.}{entrevistar (é realizado na forma de perguntas e respostas orais presenciais)}
\end{EntryWithPhonetic}

\begin{EntryWithPhonetic}{面条}{mian4tiao2}{9,7}{⾯、⽊}
  \definition{s.}{macarrão | espaguete}
\end{EntryWithPhonetic}

\begin{EntryWithPhonetic}{面条儿}{mian4 tiao2r5}{9,7,2}{⾯、⽊、⼉}[HSK 1]
  \definition{s.}{macarrão; \emph{noodles}}
\end{EntryWithPhonetic}

\begin{EntryWithPhonetic}{面团}{mian4tuan2}{9,6}{⾯、⼞}
  \definition{s.}{massa | pasta}
\end{EntryWithPhonetic}

\begin{EntryWithPhonetic}{面向}{mian4 xiang4}{9,6}{⾯、⼝}[HSK 6]
  \definition{v.}{virar o rosto para; virar na direção de; defrontar; voltado para algum lugar | estar orientado para as necessidades de; atender a; principalmente para um certo tipo de pessoas}
\end{EntryWithPhonetic}

\begin{EntryWithPhonetic}{面子}{mian4zi5}{9,3}{⾯、⼦}[HSK 5]
  \definition{s.}{face; exterior; parte externa; superfície do objeto | imagem; reputação; prestígio; decência; vaidade superficial | sentimentos; sensibilidades | pó}
\end{EntryWithPhonetic}

\begin{EntryWithPhonetic}{糆}{mian4}{15}{⽶}
  \variantof{面}
\end{EntryWithPhonetic}

\begin{EntryWithPhonetic}{麫}{mian4}{15}{⿆}
  \variantof{面}
\end{EntryWithPhonetic}

\begin{EntryWithPhonetic}{描}{miao2}{11}{⼿}
  \definition{v.}{traçar; copiar | retocar; retocar | traçar um desenho | retratar | esboçar}
\end{EntryWithPhonetic}

\begin{EntryWithPhonetic}{描述}{miao2 shu4}{11,8}{⼿、⾡}[HSK 4]
  \definition[段,种]{s.}{descrição; trecho que descreve um evento ou uma cena}
  \definition{v.}{descrever; representar}
\end{EntryWithPhonetic}

\begin{EntryWithPhonetic}{描写}{miao2xie3}{11,5}{⼿、⼍}[HSK 4]
  \definition{v.}{representar; retratar; descrever; usar a linguagem e as palavras para transmitir uma imagem concreta de uma pessoa, evento ou situação}
\end{EntryWithPhonetic}

\begin{EntryWithPhonetic}{秒}{miao3}{9}{⽲}[HSK 5]
  \definition{adv.}{instantaneamente}
  \definition{s.}{segundo (unidade de tempo) | segundo (unidade de medida angular)}
\end{EntryWithPhonetic}

\begin{EntryWithPhonetic}{妙}{miao4}{7}{⼥}[HSK 6]
  \definition*{s.}{Sobrenome Miao}
  \definition{adj.}{maravilhoso; excelente; bom | engenhoso; esperto; sutil | extraordinário | requintado; mágico; engenhoso; misterioso}
\end{EntryWithPhonetic}

\begin{EntryWithPhonetic}{妙招}{miao4zhao1}{7,8}{⼥、⼿}
  \definition{adj.}{escorregadio}
  \definition{s.}{movimento inteligente | maneira inteligente de fazer algo}
\end{EntryWithPhonetic}

\begin{EntryWithPhonetic}{灭}{mie4}{5}{⽕}[HSK 6]
  \definition{v.}{extinguir-se | extinguir; apagar; desligar | afogar; inundar; submergir | perecer; destruir | exterminar; apagar; acabar com; tornar inexistente}
\end{EntryWithPhonetic}

\begin{EntryWithPhonetic}{灭火}{mie4huo3}{5,4}{⽕、⽕}
  \definition{s.}{combate a incêndios}
  \definition{v.}{extinguir um incêndio}
\end{EntryWithPhonetic}

\begin{EntryWithPhonetic}{民}{min2}{5}{⽒}
  \definition*{s.}{Sobrenome Min}
  \definition{adj.}{folclórico ; civil (não militar)}
  \definition{s.}{pessoa | membro de um grupo étnico | uma pessoa de uma determinada ocupação | do povo; folclore | civil; cidadão | o povo | um membro de uma nacionalidade}
\end{EntryWithPhonetic}

\begin{EntryWithPhonetic}{民歌}{min2 ge1}{5,14}{⽒、⽋}[HSK 6]
  \definition[支,首]{s.}{canção folclórica; os nomes dos autores das canções transmitidas oralmente são muitas vezes desconhecidos}
\end{EntryWithPhonetic}

\begin{EntryWithPhonetic}{民工}{min2 gong1}{5,3}{⽒、⼯}[HSK 6]
  \definition{s.}{trabalhador trabalhando em um projeto público | trabalhador temporário alistado em um projeto público | agricultor que trabalha em empregos temporários na cidade | trabalhador migrante}
\end{EntryWithPhonetic}

\begin{EntryWithPhonetic}{民间}{min2jian1}{5,7}{⽒、⾨}[HSK 3]
  \definition{s.}{entre o povo | não governamental; de pessoa para pessoa}
\end{EntryWithPhonetic}

\begin{EntryWithPhonetic}{民警}{min2 jing3}{5,19}{⽒、⾔}[HSK 6]
  \definition{s.}{polícia; policial}
\end{EntryWithPhonetic}

\begin{EntryWithPhonetic}{民意}{min2 yi4}{5,13}{⽒、⼼}[HSK 6]
  \definition{s.}{vontade do povo; vontade popular | opinião pública}
\end{EntryWithPhonetic}

\begin{EntryWithPhonetic}{民众}{min2zhong4}{5,6}{⽒、⼈}
  \definition{s.}{a população | as massas | as pessoas comuns}
\end{EntryWithPhonetic}

\begin{EntryWithPhonetic}{民主}{min2zhu3}{5,5}{⽒、⼂}[HSK 6]
  \definition{adj.}{democrático; em consonância com os princípios democráticos}
  \definition[个]{s.}{democracia; direitos democráticos; refere-se ao direito do povo de participar da vida política e dos assuntos do Estado e de expressar livremente suas opiniões}
\end{EntryWithPhonetic}

\begin{EntryWithPhonetic}{民族}{min2zu2}{5,11}{⽒、⽅}[HSK 3]
  \definition[个]{s.}{nação; uma comunidade estável formada ao longo da história pela humanidade, com uma língua comum, uma região comum, uma vida econômica comum e uma mentalidade comum expressa em uma cultura comum | grupo étnico; refere-se, de maneira geral, às comunidades formadas ao longo da história por pessoas em diferentes estágios de desenvolvimento social}
\end{EntryWithPhonetic}

\begin{EntryWithPhonetic}{敏}{min3}{11}{⽁}
  \definition*{s.}{Sobrenome Min}
  \definition{adj.}{rápido; ágil | perspicaz; inteligente; rápido | inteligente; esperto}
\end{EntryWithPhonetic}

\begin{EntryWithPhonetic}{敏感}{min3gan3}{11,13}{⽁、⼼}[HSK 5]
  \definition{adj.}{sensível; descreve pessoas ou animais que rapidamente percebem mudanças ou estímulos externos | reativo; sensível; fácil de causar reações intensas}
\end{EntryWithPhonetic}

\begin{EntryWithPhonetic}{名}{ming2}{6}{⼝}[HSK 2]
  \definition*{s.}{Sobrenome Ming}
  \definition{adj.}{notável; famoso; conhecido; renomado}
  \definition{clas.}{usado para pessoas | usado para classificação por ordem}
  \definition{s.}{nome; denominação | desculpa; pretexto | fama; reputação}
  \definition{v.}{nome próprio (é) | expressar; descrever | possuir; tomar; ter}
\end{EntryWithPhonetic}

\begin{EntryWithPhonetic}{名称}{ming2 cheng1}{6,10}{⼝、⽲}[HSK 2]
  \definition[个,种]{s.}{nomes, apelidos e formas de se referir a pessoas ou coisas}
\end{EntryWithPhonetic}

\begin{EntryWithPhonetic}{名单}{ming2 dan1}{6,8}{⼝、⼗}[HSK 2]
  \definition[个,份]{s.}{lista com nomes de pessoas ou nomes de organizações}
\end{EntryWithPhonetic}

\begin{EntryWithPhonetic}{名额}{ming2'e2}{6,15}{⼝、⾴}[HSK 6]
  \definition[个]{s.}{cota de pessoas; número de pessoas designadas ou permitidas; número necessário de pessoal}
\end{EntryWithPhonetic}

\begin{EntryWithPhonetic}{名牌儿}{ming2 pai2r5}{6,12,2}{⼝、⽚、⼉}[HSK 4]
  \definition{s.}{marca famosa}
\end{EntryWithPhonetic}

\begin{EntryWithPhonetic}{名片}{ming2pian4}{6,4}{⼝、⽚}[HSK 4]
  \definition[张,盒,叠]{s.}{cartão de visita; um pedaço de papel retangular com o nome, o cargo, o endereço etc. impressos}
\end{EntryWithPhonetic}

\begin{EntryWithPhonetic}{名人}{ming2 ren2}{6,2}{⼝、⼈}[HSK 4]
  \definition[位,个]{s.}{celebridade; pessoa famosa}
\end{EntryWithPhonetic}

\begin{EntryWithPhonetic}{名胜}{ming2 sheng4}{6,9}{⼝、⾁}[HSK 6]
  \definition[处,个]{s.}{pontos turísticos; atrações famosas; lugares famosos com locais históricos ou belas paisagens}
\end{EntryWithPhonetic}

\begin{EntryWithPhonetic}{名义}{ming2 yi4}{6,3}{⼝、⼂}[HSK 6]
  \definition{s.}{nominal; em nome (geralmente seguido por 上); um nome ou título usado como base para fazer algo}[有人盗用我名义申请贷款。===Alguém solicitou um empréstimo em meu nome. | 他们只是名义上的夫妻。===Eles são marido e mulher apenas no nome.]
  \seealsoref{上}{shang4}
\end{EntryWithPhonetic}

\begin{EntryWithPhonetic}{名誉}{ming2yu4}{6,13}{⼝、⾔}[HSK 6]
  \definition{adj.}{1. honorário; nominal (geralmente se refere ao nome de um presente, com um sentido de respeito)}[他是学校的名誉教授。===Ele é professor honorário da escola.]
  \definition{s.}{fama; reputação; honra}[名誉才是最神圣的。===Reputação é a coisa mais sagrada. | 我用自己的名誉发誓。===Juro pela minha honra.]
\end{EntryWithPhonetic}

\begin{EntryWithPhonetic}{名字}{ming2zi5}{6,6}{⼝、⼦}[HSK 1]
  \definition[个]{s.}{nome; nome próprio | nome (de uma coisa)}
\end{EntryWithPhonetic}

\begin{EntryWithPhonetic}{明}{ming2}{8}{⽇}
  \definition*{s.}{Dinastia Ming (1368-1644) | Sobrenome Ming}
  \definition{adj.}{claro; brilhante; brilhante | claro; distinto; de fácil entendimento | aberto; evidente; explícito; exposto | de ​​olhos aguçados; boa visão; visão nítida | honesto}
  \definition{adv.}{claramente; definitivamente; aparentemente; de fato}
  \definition{s.}{imediatamente a seguir no tempo; ao lado deste ano e hoje; visão}
  \definition{v.}{mostrar; revelar; tornar conhecido; deixar claro | entender; compreender}
\end{EntryWithPhonetic}

\begin{EntryWithPhonetic}{明白}{ming2bai5}{8,5}{⽇、⽩}[HSK 1]
  \definition{adj.}{claro; óbvio; evidente; inequívoco | sensato; razoável | aberto; franco; inequívoco; explícito}
  \definition{v.}{entender; compreender; saber}
\end{EntryWithPhonetic}

\begin{EntryWithPhonetic}{明亮}{ming2 liang4}{8,9}{⽇、⼇}[HSK 5]
  \definition{adj.}{claro; bem iluminado | brilhante; resplandecente | claro; simples; compreensível}
\end{EntryWithPhonetic}

\begin{EntryWithPhonetic}{明码}{ming2ma3}{8,8}{⽇、⽯}
  \definition{s.}{código simples, em claro (oposto a 密码) | preço claramente marcado}
  \seealsoref{密码}{mi4ma3}
\end{EntryWithPhonetic}

\begin{EntryWithPhonetic}{明明}{ming2ming2}{8,8}{⽇、⽇}[HSK 5]
  \definition{adv.}{obviamente; claramente; sem dúvida; indica que o fenômeno ou princípio é evidente}
\end{EntryWithPhonetic}

\begin{EntryWithPhonetic}{明年}{ming2 nian2}{8,6}{⽇、⼲}[HSK 1]
  \definition{s.}{próximo ano}
\end{EntryWithPhonetic}

\begin{EntryWithPhonetic}{明确}{ming2que4}{8,12}{⽇、⽯}[HSK 3]
  \definition{adj.}{claro; definido; específico}
  \definition{v.}{deixar claro; tornar definitivo; tornar um ponto de vista, uma tarefa, etc. claro, compreensível e definitivo}
\end{EntryWithPhonetic}

\begin{EntryWithPhonetic}{明日}{ming2 ri4}{8,4}{⽇、⽇}[HSK 6]
  \definition{s.}{amanhã}
  \seealsoref{明天}{ming2tian1}
\end{EntryWithPhonetic}

\begin{EntryWithPhonetic}{明天}{ming2tian1}{8,4}{⽇、⼤}[HSK 1]
  \definition{s.}{amanhã | futuro próximo}
\end{EntryWithPhonetic}

\begin{EntryWithPhonetic}{明显}{ming2xian3}{8,9}{⽇、⽇}[HSK 3]
  \definition{adj.}{claro; óbvio; distinto; claramente visível}
\end{EntryWithPhonetic}

\begin{EntryWithPhonetic}{明星}{ming2xing1}{8,9}{⽇、⽇}[HSK 2]
  \definition[个,位,颗,名]{s.}{estrela; ator, atleta, cantor famosos, etc. | talento de ponta; profissional de destaque; também é usado como metáfora para pessoas ou grupos que se destacam pelo seu bom desempenho ou excelência | estrela brilhante; estrela resplandecente; referindo-se a estrelas muito brilhantes}
\end{EntryWithPhonetic}

\begin{EntryWithPhonetic}{明珠}{ming2zhu1}{8,10}{⽇、⽟}
  \definition{s.}{pérola | jóia (de grande valor)}
\end{EntryWithPhonetic}

\begin{EntryWithPhonetic}{鸣}{ming2}{8}{⿃}
  \definition{v.}{chorar (pássaros, animais e insetos) | fazer um som | dar voz (gratidão, queixas, etc.)}
\end{EntryWithPhonetic}

\begin{EntryWithPhonetic}{命}{ming4}{8}{⼝}[HSK 6]
  \definition[条]{s.}{vida | sorte; destino; fado | ordem; comando; instrução | atribuição de um nome, título etc.}
  \definition{v.}{ordenar; nomear | atribuir (um nome etc.)}
\end{EntryWithPhonetic}

\begin{EntryWithPhonetic}{命令}{ming4ling4}{8,5}{⼝、⼈}[HSK 5]
  \definition[条,项,道,个]{s.}{ordem; comando; instruções emitidas pelos superiores aos subordinados}
  \definition{v.}{ordenar; comandar}
\end{EntryWithPhonetic}

\begin{EntryWithPhonetic}{命运}{ming4yun4}{8,7}{⼝、⾡}[HSK 3]
  \definition[个]{s.}{tendência de desenvolvimento; tendência de futuro; metáfora para a direção e tendência do desenvolvimento e das mudanças | destino; sina; sorte; refere-se à vida e à morte, à riqueza e à pobreza e a todas as experiências da vida}
\end{EntryWithPhonetic}

\begin{EntryWithPhonetic}{摸}{mo1}{13}{⼿}[HSK 4]
  \definition{v.}{sentir; acariciar; tocar; tocar (um objeto) levemente com a mão e depois removê-lo ou mover a mão suavemente sobre a superfície do objeto | sentir para; tatear para; sentir algo com as mãos | descobrir; sentir; sondar; explorar; tentar fazer ou entender | sentir o caminho; tatear no escuro; andar por estradas que você não consegue reconhecer | furtar; roubar}
\end{EntryWithPhonetic}

\begin{EntryWithPhonetic}{模}{mo2}{14}{⽊}
  \definition{s.}{padrão | modelo; exemplo | modelo (pessoa) | exame simulado | módulo}
  \definition{v.}{imitar | copiar; emular}
  \seeref{模}{mu2}
\end{EntryWithPhonetic}

\begin{EntryWithPhonetic}{模范}{mo2fan4}{14,9}{⽊、⾋}[HSK 5]
  \definition{adj.}{exemplar}
  \definition{s.}{modelo; exemplo excelente; pessoa exemplar; coisa exemplar; pessoas ou coisas exemplares que servem de modelo}
\end{EntryWithPhonetic}

\begin{EntryWithPhonetic}{模仿}{mo2fang3}{14,6}{⽊、⼈}[HSK 5]
  \definition{v.}{copiar; imitar; aprender a fazer algo seguindo um modelo pronto}
\end{EntryWithPhonetic}

\begin{EntryWithPhonetic}{模糊}{mo2hu5}{14,15}{⽊、⽶}[HSK 5]
  \definition{adj.}{vago; confuso; indistinto}
  \definition{v.}{confundir; desorientar}
\end{EntryWithPhonetic}

\begin{EntryWithPhonetic}{模式}{mo2shi4}{14,6}{⽊、⼷}[HSK 5]
  \definition{s.}{modelo; modo; padrão; a forma padrão de algo ou o modelo padrão que as pessoas podem seguir}
\end{EntryWithPhonetic}

\begin{EntryWithPhonetic}{模特儿}{mo2 te4r5}{14,10,2}{⽊、⽜、⼉}[HSK 4]
  \definition[个,名,位]{s.}{modelo (pessoa que posa para um fotógrafo ou pintor ou escultor); objeto de representação ou referência usado por artistas para esboços e esculturas, como o corpo humano, objetos, modelos etc.; também se refere aos arquétipos que os estudiosos da literatura usam para retratar seus personagens | modelo (uma pessoa que usa roupas para exibir modas); pessoa ou manequim usado para exibir estilos de roupas}
\end{EntryWithPhonetic}

\begin{EntryWithPhonetic}{模型}{mo2xing2}{14,9}{⽊、⼟}[HSK 4]
  \definition[个]{s.}{modelo; padrão; itens feitos em escala com base em objetos ou desenhos | molde; padrão; molde para fundir máquinas, objetos, etc.}
\end{EntryWithPhonetic}

\begin{EntryWithPhonetic}{膜}{mo2}{14}{⾁}[HSK 6]
  \definition[张]{s.}{membrana | filme; revestimento fino}
\end{EntryWithPhonetic}

\begin{EntryWithPhonetic}{膜拜}{mo2bai4}{14,9}{⾁、⼿}
  \definition{v.}{ajoelhar-se e se curvar com as mãos unidas no nível da testa | ter ou mostrar sentimentos fortes de respeito e admiração por um deus}
\end{EntryWithPhonetic}

\begin{EntryWithPhonetic}{摩}{mo2}{15}{⼿}
  \definition{v.}{esfregar; raspar; tocar | refletir; estudar | afagar}
\end{EntryWithPhonetic}

\begin{EntryWithPhonetic}{摩擦}{mo2ca1}{15,17}{⼿、⼿}[HSK 5]
  \definition{s.}{atrito; desacordo; conflito (entre duas partes); a ação de impedir o movimento relativo entre dois objetos em contato, produzida na superfície de contato | atrito; metáfora para o conflito entre as duas partes}
  \definition{v.}{esfregar}
\end{EntryWithPhonetic}

\begin{EntryWithPhonetic}{摩托}{mo2 tuo1}{15,6}{⼿、⼿}[HSK 5]
  \definition[辆]{s.}{Empréstimo linguístico: motor; motor de combustão interna | Empréstimo linguístico: motocicleta, abreviação de 摩托车}
  \seealsoref{摩托车}{mo2tuo1che1}
\end{EntryWithPhonetic}

\begin{EntryWithPhonetic}{摩托车}{mo2tuo1che1}{15,6,4}{⼿、⼿、⾞}
  \definition[辆,部]{s.}{(empréstimo linguístico) motocicleta}
\end{EntryWithPhonetic}

\begin{EntryWithPhonetic}{磨}{mo2}{16}{⽯}[HSK 6]
  \definition{v.}{esfregar; desgastar | moer; refletir; polir | desgastar; esgotar; cansar; exaurir | incomodar; causar problemas | destruir; obliterar; extinguir-se | ficar ocioso; perder tempo; perder tempo; procrastinar}
  \seeref{磨}{mo4}
\end{EntryWithPhonetic}

\begin{EntryWithPhonetic}{磨菇}{mo2gu5}{16,11}{⽯、⾋}
  \variantof{蘑菇}
\end{EntryWithPhonetic}

\begin{EntryWithPhonetic}{蘑}{mo2}{19}{⾋}
  \definition{s.}{cogumelo}
\end{EntryWithPhonetic}

\begin{EntryWithPhonetic}{蘑菇}{mo2gu5}{19,11}{⾋、⾋}
  \definition{s.}{cogumelo}
  \definition{v.}{mandriar | embromar | amofinar | incomodar alguém com solicitações ou interrupções frequentes ou persistentes}
\end{EntryWithPhonetic}

\begin{EntryWithPhonetic}{魔}{mo2}{20}{⿁}
  \definition{adj.}{místico; misterioso; mágico}
  \definition{s.}{espírito maligno; demônio; diabo; monstro | mágico; místico}
\end{EntryWithPhonetic}

\begin{EntryWithPhonetic}{魔术}{mo2shu4}{20,5}{⿁、⽊}
  \definition{s.}{magia}
\end{EntryWithPhonetic}

\begin{EntryWithPhonetic}{魔头}{mo2tou2}{20,5}{⿁、⼤}
  \definition{s.}{monstro | diabo}
\end{EntryWithPhonetic}

\begin{EntryWithPhonetic}{抹}{mo3}{8}{⼿}
  \definition{v.}{colocar; aplicar; untar; engessar | limpar | anular; apagar | (para nuvem, etc.) irradiar; raiar; riscar; traçar | riscar; cancelar; marcar; remover; excluir}
  \seeref{抹}{ma1}
  \seeref{抹}{mo3}
\end{EntryWithPhonetic}

\begin{EntryWithPhonetic}{抹泪}{mo3lei4}{8,8}{⼿、⽔}
  \definition{v.}{limpar as lágrimas | (figurativo) derramar lágrimas}
\end{EntryWithPhonetic}

\begin{EntryWithPhonetic}{末}{mo4}{5}{⽊}[HSK 4]
  \definition{adj.}{último; final}
  \definition{s.}{ponta; terminal; extremidade; o final de algo | não essenciais; detalhes secundários | fim; final | pó; poeira | um papel na ópera tradicional}
\end{EntryWithPhonetic}

\begin{EntryWithPhonetic}{没}{mo4}{7}{⽔}
  \definition{adj.}{último; final}
  \definition{v.}{afundar na água; submergir | transbordar; subir além; exceder ou ultrapassar | esconder-se; desaparecer; sumir; ocultar-se | confiscar; expropriar | morrer}
  \variantof{没}
\end{EntryWithPhonetic}

\begin{EntryWithPhonetic}{没收}{mo4 shou1}{7,6}{⽔、⽁}[HSK 6]
  \definition{v.}{confiscar; expropriar; os bens e pertences de pessoas ou grupos que violem leis ou proibições serão tornados propriedade pública, de acordo com a lei}
\end{EntryWithPhonetic}

\begin{EntryWithPhonetic}{抹}{mo4}{8}{⼿}
  \definition{v.}{rebocar; engessar; alisar a massa ou o gesso com uma espátula | virar; contornar; dar uma volta de perto}
  \seeref{抹}{ma1}
  \seeref{抹}{mo3}
\end{EntryWithPhonetic}

\begin{EntryWithPhonetic}{莫}{mo4}{10}{⾋}
  \definition*{s.}{Sobrenome Mo}
  \definition{adv.}{não, frequentemente usado em frases imperativas | não; não pode | pode ser que; não pode ser que; é possível que}
  \definition{pron.}{nenhum; nada; ninguém; significa 没有谁 ou 没有哪一种东西}
  \seealsoref{没有哪一种东西}{mei2you3 na3 yi4 zhong3 dong1xi1}
  \seealsoref{没有谁}{mei2you3 shei2}
\end{EntryWithPhonetic}

\begin{EntryWithPhonetic}{莫非}{mo4fei1}{10,8}{⾋、⾮}
  \definition{expr.}{Não é mesmo?; é frequentemente usado com 不成}
  \definition{v.}{pode ser que; é possível que}
  \seealsoref{不成}{bu4 cheng2}
\end{EntryWithPhonetic}

\begin{EntryWithPhonetic}{莫名其妙}{mo4ming2qi2miao4}{10,6,8,7}{⾋、⼝、⼋、⼥}
  \definition{adj.}{desconcertante | bizzaro | inexplicável | perplexo}
\end{EntryWithPhonetic}

\begin{EntryWithPhonetic}{墨}{mo4}{15}{⿊}
  \definition*{s.}{Escola Moísta; Moísmo | México, abreviação de 墨西哥}
  \definition{adj.}{preto; escuro como breu | corrupto | escuro}
  \definition{s.}{tinta chinesa; bastão de tinta | pigmento; tinta | caligrafia ou pintura | aprendizagem; alfabetização | marcador de linha de carpinteiro; marcador de tinta | tatuar o rosto (um castigo); uma punição na China antiga | corrupção; peculato; fraude}
  \seealsoref{墨西哥}{mo4xi1ge1}
\end{EntryWithPhonetic}

\begin{EntryWithPhonetic}{墨镜}{mo4jing4}{15,16}{⿊、⾦}
  \definition[只,双,副]{s.}{óculos escuros}
\end{EntryWithPhonetic}

\begin{EntryWithPhonetic}{墨水}{mo4 shui3}{15,4}{⿊、⽔}[HSK 6]
  \definition[瓶]{s.}{tinta chinesa preparada; tinta (para caneta-tinteiro) | aprendizagem; alfabetização; uma metáfora para o conhecimento ou a capacidade de ler e escrever}
\end{EntryWithPhonetic}

\begin{EntryWithPhonetic}{墨西哥}{mo4xi1ge1}{15,6,10}{⿊、⾑、⼝}
  \definition*{s.}{México; Planalto no México}
\end{EntryWithPhonetic}

\begin{EntryWithPhonetic}{磨}{mo4}{16}{⽯}
  \definition[盘]{s.}{mó (pedra pesada e redonda para moinho)}
  \definition{v.}{moer; esfarelar; triturar | virar; inverter a marcha}
\end{EntryWithPhonetic}

\begin{EntryWithPhonetic}{默}{mo4}{16}{⿊}
  \definition*{s.}{Sobrenome Mo}
  \definition{adj.}{taciturno; reservado | silencioso}
  \definition{v.}{escrever de memória}
\end{EntryWithPhonetic}

\begin{EntryWithPhonetic}{默默}{mo4mo4}{16,16}{⿊、⿊}[HSK 4]
  \definition{adj.}{mudo; quieto; silencioso}
  \definition{adv.}{silenciosamente}
\end{EntryWithPhonetic}

\begin{EntryWithPhonetic}{默契}{mo4qi4}{16,9}{⿊、⼤}
  \definition{adj.}{(de membros da equipe) bem coordenados}
  \definition{s.}{entendimento tácito | entendimento mútuo | conectado em um nível mútuo profundo | (de membros da equipe) bem coordenados}
\end{EntryWithPhonetic}

\begin{EntryWithPhonetic}{某}{mou3}{9}{⽊}[HSK 3]
  \definition{pron.}{alguém ou algo indefinido; refere-se a pessoas ou coisas incertas | referindo-se a si mesmo; em vez do seu próprio nome | alguns; certos; refere-se a uma pessoa ou coisa específica cujo nome não se sabe ou não se pode revelar | tal e tal; substituir o nome de outra pessoa (geralmente com um tom rude)}
\end{EntryWithPhonetic}

\begin{EntryWithPhonetic}{模}{mu2}{14}{⽊}
  \definition*{s.}{Sobrenome Mu}
  \definition{s.}{molde; padrão; matriz}
  \seeref{模}{mo2}
\end{EntryWithPhonetic}

\begin{EntryWithPhonetic}{模具}{mu2ju4}{14,8}{⽊、⼋}
  \definition{s.}{molde | matriz | padrão}
\end{EntryWithPhonetic}

\begin{EntryWithPhonetic}{模样}{mu2yang4}{14,10}{⽊、⽊}[HSK 5]
  \definition[副,种]{s.}{aparência; a aparência ou o estilo de vestir de uma pessoa | indicando uma estimativa aproximada de tempo ou idade; expressão de estimativas relativas a tempo, idade, etc. | tendência; situação; inclinação}
\end{EntryWithPhonetic}

\begin{EntryWithPhonetic}{母}{mu3}{5}{⽏}[HSK 6][Kangxi 80]
  \definition*{s.}{Sobrenome Mu}
  \definition{adj.}{fêmea}
  \definition[位,名,个,些]{s.}{mãe | fêmea (animal) (oposto a 公) | origem; pais | parentes idosas; geralmente se refere a mulheres idosas | côncavo | fonte; algo que tem a capacidade ou função de produzir outras coisas}
  \seealsoref{公}{gong1}
\end{EntryWithPhonetic}

\begin{EntryWithPhonetic}{母鸡}{mu3ji1}{5,7}{⽏、⿃}[HSK 6]
  \definition{s.}{galinha}
\end{EntryWithPhonetic}

\begin{EntryWithPhonetic}{母女}{mu3 nv3}{5,3}{⽏、⼥}[HSK 6]
  \definition{s.}{mãe e filha}
\end{EntryWithPhonetic}

\begin{EntryWithPhonetic}{母亲}{mu3qin1}{5,9}{⽏、⼇}[HSK 3]
  \definition[位,名,个,些]{s.}{mãe}
\end{EntryWithPhonetic}

\begin{EntryWithPhonetic}{母语}{mu3yu3}{5,9}{⽏、⾔}
  \definition{s.}{língua materna | língua nativa}
\end{EntryWithPhonetic}

\begin{EntryWithPhonetic}{母子}{mu3 zi3}{5,3}{⽏、⼦}[HSK 6]
  \definition{s.}{mãe e filho}
\end{EntryWithPhonetic}

\begin{EntryWithPhonetic}{亩}{mu3}{7}{⼇}
  \definition{clas.}{usado para campos | unidade de área igual a um décimo quinto de um hectare}
\end{EntryWithPhonetic}

\begin{EntryWithPhonetic}{木}{mu4}{4}{⽊}[Kangxi 75]
  \definition{adj.}{de madeira; feito de madeira | estúpido; de raciocínio lento; atordoado; lento para reagir | simplório; chato | 3. entorpecido; de madeira; dormência localizada ou perda de sensibilidade}
  \definition{s.}{árvore | madeira; madeiramento | caixão}
\end{EntryWithPhonetic}

\begin{EntryWithPhonetic}{木偶}{mu4'ou3}{4,11}{⽊、⼈}
  \definition{s.}{fantoche, marionete}
\end{EntryWithPhonetic}

\begin{EntryWithPhonetic}{木头}{mu4tou5}{4,5}{⽊、⼤}[HSK 3]
  \definition[根,块,堆,截]{s.}{tronco; madeira; lenha; denominação genérica para madeira e materiais de madeira}
\end{EntryWithPhonetic}

\begin{EntryWithPhonetic}{目}{mu4}{5}{⽬}[Kangxi 109]
  \definition*{s.}{Sobrenome Mu}
  \definition{s.}{olho | item | (biologia) ordem | lista de coisas; catálogo; sumário | buraco em uma rede; malha (abertura)  | (de documentos, teses, etc.) nome; título | ponto; ponto de território, um termo do Go; refere-se à intersecção das linhas verticais e horizontais no tabuleiro, uma intersecção é chamada de 一目, \dpy{yi2 mu4}}
  \definition{v.}{(literário) olhar; considerar}
\end{EntryWithPhonetic}

\begin{EntryWithPhonetic}{目标}{mu4biao1}{5,9}{⽬、⽊}[HSK 3]
  \definition[个,项]{s.}{alvo; objetivo; objeto de tiro, ataque ou busca| objetivo; meta; destino; a situação ou padrão que se deseja alcançar}
\end{EntryWithPhonetic}

\begin{EntryWithPhonetic}{目的}{mu4di4}{5,8}{⽬、⽩}[HSK 2]
  \definition[个,些,种]{s.}{objetivo; meta; alvo; finalidade; propósito; o lugar ou situação que se deseja alcançar; o resultado que se deseja obter; o centro do alvo}
\end{EntryWithPhonetic}

\begin{EntryWithPhonetic}{目光}{mu4guang1}{5,6}{⽬、⼉}[HSK 5]
  \definition[道,束,种]{s.}{olhar fixo; a expressão e atitude reveladas pelos olhos | visão; vista; percepção visual; a linha imaginária formada entre os olhos e o objeto quando se olha para ele | perspicácia (capacidade de observar e reconhecer coisas); conhecimento adquirido através do contato com as coisas, capacidade de observar as coisas}
\end{EntryWithPhonetic}

\begin{EntryWithPhonetic}{目前}{mu4qian2}{5,9}{⽬、⼑}[HSK 3]
  \definition{adv.}{agora; recentemente; no momento; no presente}
\end{EntryWithPhonetic}

\begin{EntryWithPhonetic}{墓}{mu4}{13}{⼟}
  \definition[座,个,号]{s.}{sepultura; túmulo; mausoléu}
\end{EntryWithPhonetic}

\begin{EntryWithPhonetic}{幕}{mu4}{13}{⼱}
  \definition{s.}{cortina ou tela | dossel ou tenda | quartel de um general | ato (de uma peça)}
\end{EntryWithPhonetic}

%%%%% EOF %%%%%


 %%%
%%% N
%%%

\section*{N}\addcontentsline{toc}{section}{N}

\begin{EntryWithPhonetic}{那}{na1}{6}{⾢}
  \definition*{s.}{Sobrenome Na}
  \seeref{na3}
  \seeref{na4}
  \seeref{ne4}
  \seeref{nei4}
  \seeref{nuo2}
\end{EntryWithPhonetic}

\begin{EntryWithPhonetic}{拿}{na2}{10}{⼿}[HSK 1]
  \definition{part.}{usado da mesma forma que 把: para marcar o seguinte substantivo seguinte como objeto direto}
  \definition{prep.}{ferramentas, materiais, métodos, etc. utilizados para a introdução | os objetos que estão sendo manipulados para introdução}
  \definition{v.}{segurar; pegar; pegar ou mover objetos com as mãos ou de outra forma | apreender; capturar; prender; usar força bruta para capturar | ter certeza de; ser capaz de fazer; ter uma compreensão firme de | tornar as coisas difíceis para alguém; colocar alguém em uma situação difícil; obstruir; chantagear; coagir; causar dificuldades intencionalmente | fingir ou fazer (algum tipo de postura ou aparência) | ter certeza de; tomar uma decisão | obter; ganhar; receber}
\end{EntryWithPhonetic}

\begin{EntryWithPhonetic}{拿出}{na2 chu1}{10,5}{⼿、⼐}[HSK 2]
  \definition{v.}{apresentar (evidências) | fornecer | apresentar (uma proposta) | oferecer; servir | retirar; tirar}
\end{EntryWithPhonetic}

\begin{EntryWithPhonetic}{拿到}{na2 dao4}{10,8}{⼿、⼑}[HSK 2]
  \definition{v.}{pegar; obter, conseguir}
\end{EntryWithPhonetic}

\begin{EntryWithPhonetic}{拿走}{na2 zou3}{10,7}{⼿、⾛}[HSK 6]
  \definition{v.}{tirar; remover}
\end{EntryWithPhonetic}

\begin{EntryWithPhonetic}{那}{na3}{6}{⾢}
  \definition{adv.}{expressa negação em perguntas retóricas}
  \definition{pron.}{qual? | qualquer que seja; qualquer que; para expressar incerteza em uma declaração | variante de 哪}
  \seeref{na1}
  \seeref{na4}
  \seeref{ne4}
  \seeref{nei4}
  \seeref{nuo2}
  \seealsoref{哪}{na3}
\end{EntryWithPhonetic}

\begin{EntryWithPhonetic}{哪}{na3}{9}{⼝}[HSK 1,4]
  \definition{adv.}{para expressar uma pergunta retórica, indicando que é impossível}
  \definition{pron.}{qual?; o que?; expressa a necessidade de determinar um entre várias pessoas ou coisas | qualquer; ser usado em um sentido geral | qual?; o que?; (usado sozinho, o mesmo que 什么, frequentemente usado de forma intercambiável com 什么) | qualquer; qualquer que seja; refere-se a qualquer um, geralmente seguido por 都 ou 也, ou usando dois 哪 antes e depois | qual (indica algo incerto)}
  \seeref{na5}
  \seeref{nei3}
  \seealsoref{都}{dou1}
  \seealsoref{什么}{shen2me5}
  \seealsoref{也}{ye3}
\end{EntryWithPhonetic}

\begin{EntryWithPhonetic}{哪个}{na3ge5}{9,3}{⼝、⼈}
  \definition{pron.}{qual deles (pergunta sobre o objeto) | quem (perguntar a alguém ou indicar qualquer pessoa)}
\end{EntryWithPhonetic}

\begin{EntryWithPhonetic}{哪国人}{na3 guo2ren2}{9,8,2}{⼝、⼞、⼈}
  \definition{expr.}{de qual país?}
\end{EntryWithPhonetic}

\begin{EntryWithPhonetic}{哪里}{na3 li3}{9,7}{⼝、⾥}[HSK 1]
  \definition{adv.}{usado em perguntas retóricas para expressar um significado negativo}
  \definition{pron.}{onde?; em que lugar? | onde quer que seja; em qualquer lugar | usado como uma resposta educada a um elogio}
\end{EntryWithPhonetic}

\begin{EntryWithPhonetic}{哪怕}{na3pa4}{9,8}{⼝、⼼}[HSK 4]
  \definition{conj.}{mesmo; mesmo se; mesmo que; não importa o quão}
\end{EntryWithPhonetic}

\begin{EntryWithPhonetic}{哪儿}{na3r5}{9,2}{⼝、⼉}[HSK 1]
  \definition{adv.}{usado para perguntas retóricas, indicando negação}
  \definition{pron.}{onde? | onde quer que seja; em qualquer lugar | usado como uma resposta educada a um elogio}
\end{EntryWithPhonetic}

\begin{EntryWithPhonetic}{哪些}{na3xie1}{9,8}{⼝、⼆}[HSK 1]
  \definition{pron.}{quais?}
\end{EntryWithPhonetic}

\begin{EntryWithPhonetic}{那}{na4}{6}{⾢}[HSK 1,2]
  \definition{conj.}{então; nessa situação; nesse caso; o mesmo que 那么}
  \definition{pron.}{aquele; aquilo; indica pessoas ou coisas distantes | aquele; aquilo; expressa muitas coisas, sem se referir especificamente a uma pessoa ou coisa, e é frequentemente usado em conjunto com 这}
  \seeref{na1}
  \seeref{na3}
  \seeref{ne4}
  \seeref{nei4}
  \seeref{nuo2}
  \seealsoref{那么}{na4 me5}
  \seealsoref{这}{zhe4}
\end{EntryWithPhonetic}

\begin{EntryWithPhonetic}{那边}{na4 bian5}{6,5}{⾢、⾡}[HSK 1]
  \definition{pron.}{ali; acolá; aquele lado}
\end{EntryWithPhonetic}

\begin{EntryWithPhonetic}{那个}{na4ge5}{6,3}{⾢、⼈}
  \definition{pron.}{aquele | usado antes de verbos e adjetivos para indicar exagero | para substituir o discurso direto inconveniente}
\end{EntryWithPhonetic}

\begin{EntryWithPhonetic}{那会儿}{na4 hui4r5}{6,6,2}{⾢、⼈、⼉}[HSK 2]
  \definition{pron.}{então; naquela época; refere-se ao passado ou ao futuro}
\end{EntryWithPhonetic}

\begin{EntryWithPhonetic}{那里}{na4 li3}{6,7}{⾢、⾥}[HSK 1]
  \definition{pron./s.}{lá; ali; aquele lugar; indica um lugar distante}
\end{EntryWithPhonetic}

\begin{EntryWithPhonetic}{那么}{na4 me5}{6,3}{⾢、⼃}[HSK 2]
  \definition{conj.}{então; nesse caso; afirmar o resultado esperado ou fazer um julgamento}
  \definition{pron.}{assim; dessa maneira; indica a natureza, o estado, a forma, o grau, etc. | assim; sobre; colocado antes do numeral, indica uma estimativa}
\end{EntryWithPhonetic}

\begin{EntryWithPhonetic}{那麽}{na4 me5}{6,14}{⾢、⿇}
  \variantof{那么}
\end{EntryWithPhonetic}

\begin{EntryWithPhonetic}{那儿}{na4r5}{6,2}{⾢、⼉}[HSK 1]
  \definition{pron.}{lá; ali; naquele lugar | então; naquela época (usado após 打, 从 e 由)}
  \seealsoref{从}{cong2}
  \seealsoref{打}{da3}
  \seealsoref{由}{you2}
\end{EntryWithPhonetic}

\begin{EntryWithPhonetic}{那时}{na4 shi2}{6,7}{⾢、⽇}[HSK 2]
  \definition{pron.}{então; naquela época; naqueles dias; geralmente se refere a um período de tempo distante do presente}
  \seealsoref{那时候}{na4 shi2 hou5}
\end{EntryWithPhonetic}

\begin{EntryWithPhonetic}{那时候}{na4 shi2 hou5}{6,7,10}{⾢、⽇、⼈}[HSK 2]
  \definition{adv.}{naquela hora; em algum momento no passado}
  \seealsoref{那时}{na4 shi2}
\end{EntryWithPhonetic}

\begin{EntryWithPhonetic}{那些}{na4 xie1}{6,8}{⾢、⼆}[HSK 1]
  \definition{pron.}{aqueles; indica duas ou mais pessoas ou coisas}
\end{EntryWithPhonetic}

\begin{EntryWithPhonetic}{那样}{na4 yang4}{6,10}{⾢、⽊}[HSK 2]
  \definition{pron.}{assim; tal; desse tipo; desse gênero; dessa natureza; desse tipo; indica a natureza, o estado, a maneira, o grau ou refere-se a uma ação ou situação específica}
\end{EntryWithPhonetic}

\begin{EntryWithPhonetic}{那咱}{na4 zan5}{6,9}{⾢、⼝}
  \definition{s.}{(informal) naquela época; então | (antigo) naquela época}
\end{EntryWithPhonetic}

\begin{EntryWithPhonetic}{哪}{na5}{9}{⼝}
  \definition{part.}{usado depois de uma palavra com a terminação -n, é equivalente a 啊}
  \seeref{na3}
  \seeref{nei3}
  \seealsoref{啊}{a5}
\end{EntryWithPhonetic}

\begin{EntryWithPhonetic}{奶}{nai3}{5}{⼥}[HSK 1]
  \definition{adj.}{bebê; infância; infantil}
  \definition[杯,滴,瓶,只,桶]{s.}{seios; mama | leite; produtos lácteos}
  \definition{v.}{amamentar; mamar}
\end{EntryWithPhonetic}

\begin{EntryWithPhonetic}{奶茶}{nai3 cha2}{5,9}{⼥、⾋}[HSK 3]
  \definition[杯]{s.}{chá com leite; chá com leite de vaca ou de ovelha}
\end{EntryWithPhonetic}

\begin{EntryWithPhonetic}{奶粉}{nai3 fen3}{5,10}{⼥、⽶}[HSK 6]
  \definition[袋,桶,罐,勺]{s.}{leite em pó}
\end{EntryWithPhonetic}

\begin{EntryWithPhonetic}{奶奶}{nai3nai5}{5,5}{⼥、⼥}[HSK 1]
  \definition[位]{s.}{avó (paterna) | vovó; avó; mulheres mais velhas | jovem senhora da casa}
\end{EntryWithPhonetic}

\begin{EntryWithPhonetic}{奶牛}{nai3 niu2}{5,4}{⼥、⽜}[HSK 6]
  \definition{s.}{vaca leiteira (ou leiteira); vaca}
\end{EntryWithPhonetic}

\begin{EntryWithPhonetic}{耐}{nai4}{9}{⽽}
  \definition{v.}{ser capaz de suportar; aguentar}
\end{EntryWithPhonetic}

\begin{EntryWithPhonetic}{耐心}{nai4xin1}{9,4}{⽽、⼼}[HSK 5]
  \definition{adj.}{paciente}
  \definition[些]{s.}{paciência; uma pessoa que não se importa com problemas e é paciente}
\end{EntryWithPhonetic}

\begin{EntryWithPhonetic}{男}{nan2}{7}{⽥}[HSK 1]
  \definition{adj.}{homem; macho; masculino (em oposição a 女)}
  \definition[个,位]{s.}{filho; menino | homem | barão (o mais baixo de cinco ordens de nobreza)}
  \seealsoref{女}{nv3}
\end{EntryWithPhonetic}

\begin{EntryWithPhonetic}{男孩儿}{nan2hai2r5}{7,9,2}{⽥、⼦、⼉}[HSK 1]
  \definition{s.}{menino; rapaz}
\end{EntryWithPhonetic}

\begin{EntryWithPhonetic}{男女}{nan2 nv3}{7,3}{⽥、⼥}[HSK 4]
  \definition{s.}{homens e mulheres; masculino e feminino}
\end{EntryWithPhonetic}

\begin{EntryWithPhonetic}{男朋友}{nan2 peng2 you5}{7,8,4}{⽥、⽉、⼜}[HSK 1]
  \definition{s.}{namorado}
\end{EntryWithPhonetic}

\begin{EntryWithPhonetic}{男人}{nan2 ren2}{7,2}{⽥、⼈}[HSK 1]
  \definition[个]{s.}{homem adulto; macho; cavalheiro | marido}
\end{EntryWithPhonetic}

\begin{EntryWithPhonetic}{男生}{nan2 sheng1}{7,5}{⽥、⽣}[HSK 1]
  \definition[个]{s.}{menino; estudante; estudante do sexo masculino; aluno do sexo masculino}
\end{EntryWithPhonetic}

\begin{EntryWithPhonetic}{男士}{nan2 shi4}{7,3}{⽥、⼠}[HSK 4]
  \definition{s.}{cavalheiro; \emph{gentleman}}
\end{EntryWithPhonetic}

\begin{EntryWithPhonetic}{男性}{nan2 xing4}{7,8}{⽥、⼼}[HSK 5]
  \definition{s.}{masculino; homem; masculinidade; em oposição a 女性}
  \seealsoref{女性}{nv3 xing4}
\end{EntryWithPhonetic}

\begin{EntryWithPhonetic}{男子}{nan2zi3}{7,3}{⽥、⼦}[HSK 3]
  \definition[个,位]{s.}{uma pessoa do sexo masculino; um homem}
\end{EntryWithPhonetic}

\begin{EntryWithPhonetic}{南}{nan2}{9}{⼗}[HSK 1]
  \definition*{s.}{Sobrenome Nan}
  \definition{s.}{sul; uma das quatro direções básicas, o lado direito quando se está de frente para o sol pela manhã (oposto ao 北) | especificamente no sul da China}
  \seealsoref{北}{bei3}
\end{EntryWithPhonetic}

\begin{EntryWithPhonetic}{南北}{nan2 bei3}{9,5}{⼗、⼔}[HSK 5]
  \definition{s.}{(território) norte e sul | (distância) de norte a sul}
\end{EntryWithPhonetic}

\begin{EntryWithPhonetic}{南边}{nan2 bian5}{9,5}{⼗、⾡}[HSK 1]
  \definition{s.}{sul; lado sul}
\end{EntryWithPhonetic}

\begin{EntryWithPhonetic}{南部}{nan2 bu4}{9,10}{⼗、⾢}[HSK 3]
  \definition{s.}{parte sul; sul | a parte sul}
\end{EntryWithPhonetic}

\begin{EntryWithPhonetic}{南方}{nan2 fang1}{9,4}{⼗、⽅}[HSK 2]
  \definition{s.}{sul; indica a direção sul | o sul; a região sul}
\end{EntryWithPhonetic}

\begin{EntryWithPhonetic}{南极}{nan2ji2}{9,7}{⼗、⽊}[HSK 5]
  \definition*{s.}{Polo Sul; Polo Antártico | Polo sul magnético}
  \definition{s.}{polo sul magnético}
\end{EntryWithPhonetic}

\begin{EntryWithPhonetic}{南京}{nan2jing1}{9,8}{⼗、⼇}
  \definition*{s.}{Nanquim, capital da província de Jiangsu, 江苏}
  \seealsoref{江苏}{jiang1su1}
\end{EntryWithPhonetic}

\begin{EntryWithPhonetic}{南面}{nan2mian4}{9,9}{⼗、⾯}
  \definition{s.}{sul | lado sul}
\end{EntryWithPhonetic}

\begin{EntryWithPhonetic}{难}{nan2}{10}{⾫}[HSK 1]
  \definition{adj.}{difícil; duro; problemático (oposto a 易) | dificilmente possível; inevitável | ruim; desagradável | problemático; improvável}
  \definition{s.}{dificuldade}
  \definition{v.}{colocar alguém em uma situação difícil}
  \seeref{nan4}
  \seealsoref{易}{yi4}
\end{EntryWithPhonetic}

\begin{EntryWithPhonetic}{难道}{nan2dao4}{10,12}{⾫、⾡}[HSK 3]
  \definition{adv.}{certamente não significa que\dots?; é possível que\dots?; não me diga\dots; poderia ser que\dots?; usado em frases interrogativas para reforçar o tom interrogativo; frequentemente usado com palavras como "吗" e "不成".}
  \seealsoref{不成}{bu4 cheng2}
  \seealsoref{吗}{ma5}
\end{EntryWithPhonetic}

\begin{EntryWithPhonetic}{难得}{nan2de2}{10,11}{⾫、⼻}[HSK 5]
  \definition{adj.}{raro; difícil de encontrar; difícil de obter ou realizar, indicando que é valioso}
  \definition{adv.}{raramente; com pouca frequência}
\end{EntryWithPhonetic}

\begin{EntryWithPhonetic}{难度}{nan2 du4}{10,9}{⾫、⼴}[HSK 3]
  \definition{s.}{dificuldade; grau de dificuldade}
\end{EntryWithPhonetic}

\begin{EntryWithPhonetic}{难过}{nan2guo4}{10,6}{⾫、⾡}[HSK 2]
  \definition{adj.}{triste; ruim; psicologicamente desconfortável | difícil; árduo}
\end{EntryWithPhonetic}

\begin{EntryWithPhonetic}{难看}{nan2 kan4}{10,9}{⾫、⽬}[HSK 2]
  \definition{adj.}{feio; desagradável à vista | vergonhoso; embaraçoso; desonroso; sem glória; sem dignidade}
\end{EntryWithPhonetic}

\begin{EntryWithPhonetic}{难受}{nan2shou4}{10,8}{⾫、⼜}[HSK 2]
  \definition{adj.}{sentir dor; sentir-se mal; sentir-se desconfortável | sentir-se mal; sentir-se infeliz; de mau humor; triste}
\end{EntryWithPhonetic}

\begin{EntryWithPhonetic}{难题}{nan2 ti2}{10,15}{⾫、⾴}[HSK 2]
  \definition[个,道]{s.}{desafio; problema difícil; questão difícil; questões difíceis de responder ou resolver}
\end{EntryWithPhonetic}

\begin{EntryWithPhonetic}{难听}{nan2 ting1}{10,7}{⾫、⼝}[HSK 2]
  \definition{adj.}{desagradável de ouvir | ofensivo; grosseiro; vulgar e desagradável | escandaloso; indigno}
\end{EntryWithPhonetic}

\begin{EntryWithPhonetic}{难忘}{nan2 wang4}{10,7}{⾫、⼼}[HSK 6]
  \definition{adj.}{memorável; inesquecível}
\end{EntryWithPhonetic}

\begin{EntryWithPhonetic}{难以}{nan2 yi3}{10,4}{⾫、⼈}[HSK 5]
  \definition{adj.}{difícil; complicado}
\end{EntryWithPhonetic}

\begin{EntryWithPhonetic}{难}{nan4}{10}{⾫}
  \definition{s.}{catástrofe; calamidade; desastre; adversidade; grande infortúnio}
  \definition{v.}{acusar; culpar}
  \seeref{nan2}
\end{EntryWithPhonetic}

\begin{EntryWithPhonetic}{难免}{nan4mian3}{10,7}{⾫、⼉}[HSK 4]
  \definition{adj.}{inevitável; difícil de evitar}
\end{EntryWithPhonetic}

\begin{EntryWithPhonetic}{孬}{nao1}{10}{⼥}
  \definition{adj.}{ruim | covarde | Dialeto: não (é) bom (contração de 不 + 好)}
  \seealsoref{不}{bu4}
  \seealsoref{好}{hao3}
\end{EntryWithPhonetic}

\begin{EntryWithPhonetic}{呶}{nao2}{8}{⼝}
  \definition{interj.}{(onomatopéia) ruído alto e contínuo}
  \definition{v.}{(literário) gritar; clamar; falar ruidosamente}
  \seealsoref{努}{nu3}
\end{EntryWithPhonetic}

\begin{EntryWithPhonetic}{脑}{nao3}{10}{⾁}
  \definition{s.}{(fisiologia) cérebro | tofu;  substância branca semelhante ao cérebro ou à medula espinhal cerebral | cabeça | a essência de um objeto}
\end{EntryWithPhonetic}

\begin{EntryWithPhonetic}{脑袋}{nao3dai5}{10,11}{⾁、⾐}[HSK 4]
  \definition[颗,个]{s.}{cabeça; a parte mais alta do corpo humano ou a parte mais alta de um animal que contém órgãos como a boca, o nariz, os olhos etc. | mente; cérebro; capacidade de pensar, lembrar, etc.}
\end{EntryWithPhonetic}

\begin{EntryWithPhonetic}{脑瓜}{nao3gua1}{10,5}{⾁、⽠}
  \definition{s.}{crânio | cérebro | cabeça | mente | mentalidade | ideia}
  \seealsoref{脑瓜子}{nao3gua1zi5}
\end{EntryWithPhonetic}

\begin{EntryWithPhonetic}{脑瓜子}{nao3gua1zi5}{10,5,3}{⾁、⽠、⼦}
  \definition{s.}{Coloquial: crânio; cérebro; cabeça; mente; mentalidade; ideia}
  \seealsoref{脑瓜}{nao3gua1}
\end{EntryWithPhonetic}

\begin{EntryWithPhonetic}{脑子}{nao3 zi5}{10,3}{⾁、⼦}[HSK 5]
  \definition[个]{s.}{cérebro | mente; cabeça; cérebro; inteligência; poder mental; refere-se à capacidade de pensar, memorizar, raciocinar, etc.; inteligência}
\end{EntryWithPhonetic}

\begin{EntryWithPhonetic}{闹}{nao4}{8}{⾾}[HSK 4]
  \definition{adj.}{barulhento}
  \definition{v.}{fazer barulho; provocar problemas | dar vazão (à sua raiva, ressentimento, etc.) | sofrer de; ser incomodado por; ocorrer (um desastre ou coisa ruim) | fazer;  entrar em ação | agitar; perturbar | brincar; fazer bagunça}
\end{EntryWithPhonetic}

\begin{EntryWithPhonetic}{闹钟}{nao4 zhong1}{8,9}{⾾、⾦}[HSK 4]
  \definition[个,台,只,款]{s.}{despertador; relógios capazes de tocar alarmes em horários predeterminados}
\end{EntryWithPhonetic}

\begin{EntryWithPhonetic}{那}{ne4}{6}{⾢}
  \definition{conj.}{então; nesse caso; o mesmo que 那么}
  \definition{pron.}{aquele; aquilo; pronúncia coloquial de 那 (\dpy{na4})}
  \seeref{na1}
  \seeref{na3}
  \seeref{na4}
  \seeref{nei4}
  \seeref{nuo2}
  \seealsoref{那么}{na4 me5}
\end{EntryWithPhonetic}

\begin{EntryWithPhonetic}{呢}{ne5}{8}{⼝}[HSK 1]
  \definition{part.}{usada no final de frases interrogativas (especificamente perguntas, perguntas de escolha e perguntas retóricas) para indicar um tom interrogativo | usada no final de uma frase declarativa, indica que uma ação ou situação está em andamento | usada em frases para indicar uma pausa (muitas vezes em pares) | usada no final de uma frase declarativa para confirmar um fato e convencer o interlocutor (com um tom de indicação e exagero)}
  \seeref{ni2}
\end{EntryWithPhonetic}

\begin{EntryWithPhonetic}{哪}{nei3}{9}{⼝}
  \definition{part.}{qual? (interrogativo, seguido de classificador ou numeral-classificador)}
  \seeref{na3}
  \seeref{na5}
\end{EntryWithPhonetic}

\begin{EntryWithPhonetic}{内}{nei4}{4}{⼌}[HSK 3]
  \definition*{s.}{Sobrenome Nei}
  \definition{s.}{dentro; interior; parte interna ou lateral (oposto de 外) |  (antes de um substantivo ou verbo na formação de uma palavra composta) interno | (depois de um substantivo para indicar lugar, tempo, escopo ou limites) dentro; em | coração; mente | esposa ou parentes da esposa}
  \seealsoref{外}{wai4}
\end{EntryWithPhonetic}

\begin{EntryWithPhonetic}{内部}{nei4bu4}{4,10}{⼌、⾢}[HSK 4]
  \definition{s.}{interior; dentro; interno; dentro de um determinado intervalo}
\end{EntryWithPhonetic}

\begin{EntryWithPhonetic}{内存}{nei4cun2}{4,6}{⼌、⼦}
  \definition{s.}{armazenamento interno | memória do computador | RAM (\emph{random access memory})}
  \seealsoref{随机存取存储器}{sui2ji1cun2qu3cun2chu3qi4}
  \seealsoref{随机存取记忆体}{sui2ji1cun2qu3ji4yi4ti3}
\end{EntryWithPhonetic}

\begin{EntryWithPhonetic}{内地}{nei4 di4}{4,6}{⼌、⼟}[HSK 6]
  \definition{s.}{interior; sertão | China continental (RPC excluindo Hong Kong e Macau, mas incluindo ilhas como Hainan) | Japão (usado em Taiwan durante a colonização japonesa)}
\end{EntryWithPhonetic}

\begin{EntryWithPhonetic}{内科}{nei4ke1}{4,9}{⼌、⽲}[HSK 4]
  \definition{s.}{medicina geral; clínica geral; clínica médica}
\end{EntryWithPhonetic}

\begin{EntryWithPhonetic}{内燃机}{nei4ran2ji1}{4,16,6}{⼌、⽕、⽊}
  \definition{s.}{motor de combustão interna}
\end{EntryWithPhonetic}

\begin{EntryWithPhonetic}{内容}{nei4rong2}{4,10}{⼌、⼧}[HSK 3]
  \definition[份,个,项]{s.}{conteúdo; substância; a substância ou significado contido em algo}
\end{EntryWithPhonetic}

\begin{EntryWithPhonetic}{内外}{nei4 wai4}{4,5}{⼌、⼣}[HSK 6]
  \definition{s.}{dentro e fora; nacional e estrangeiro; interno e externo | ao redor; aproximadamente; número aproximado de exibição}
\end{EntryWithPhonetic}

\begin{EntryWithPhonetic}{内心}{nei4 xin1}{4,4}{⼌、⼼}[HSK 3]
  \definition{s.}{coração; interior; íntimo do ser}
\end{EntryWithPhonetic}

\begin{EntryWithPhonetic}{内省}{nei4xing3}{4,9}{⼌、⽬}
  \definition{s.}{introspecção}
  \definition{v.}{refletir sobre si mesmo}
\end{EntryWithPhonetic}

\begin{EntryWithPhonetic}{内衣}{nei4 yi1}{4,6}{⼌、⾐}[HSK 6]
  \definition[件,个]{s.}{roupa íntima}
\end{EntryWithPhonetic}

\begin{EntryWithPhonetic}{内在}{nei4zai4}{4,6}{⼌、⼟}[HSK 5]
  \definition{adj.}{intrínseco; algo que existe em si mesmo, mas que não pode ser descoberto através da observação direta | interno; imanente; difícil de perceber}
\end{EntryWithPhonetic}

\begin{EntryWithPhonetic}{内资}{nei4 zi1}{4,10}{⼌、⾙}
  \definition{s.}{capital nacional; financiamento interno; investimento de fontes nacionais (em oposição a 外资)}
  \seealsoref{外资}{wai4 zi1}
\end{EntryWithPhonetic}

\begin{EntryWithPhonetic}{那}{nei4}{6}{⾢}
  \definition{conj.}{então; o mesmo que 那么}
  \definition{pron.}{aquele; aquilo; A pronúncia coloquial de 那 (\dpy{na4})}
  \seeref{na1}
  \seeref{na3}
  \seeref{na4}
  \seeref{ne4}
  \seeref{nuo2}
  \seealsoref{那么}{na4 me5}
\end{EntryWithPhonetic}

\begin{EntryWithPhonetic}{能}{neng2}{10}{⾁}[HSK 1]
  \definition*{s.}{Sobrenome Neng}
  \definition{adv.}{talvez}
  \definition{s.}{habilidade; capacidade; competência | potência; energia; em física, refere-se à energia}
  \definition{v.}{poder fazer; ser capaz de | ser possível | entre 不 \dots 不 para expressar obrigação, certeza ou grande probabilidade | poder; ter permissão para | ser bom em fazer algo | permitir}
\end{EntryWithPhonetic}

\begin{EntryWithPhonetic}{能不能}{neng2 bu4 neng2}{10,4,10}{⾁、⼀、⾁}[HSK 3]
  \definition{adv.}{pode ou não pode\dots?}
\end{EntryWithPhonetic}

\begin{EntryWithPhonetic}{能否}{neng2 fou3}{10,7}{⾁、⼝}[HSK 6]
  \definition{adv.}{é possível; se ou não; pode ou não pode; Você consegue?; expressa dúvida, frequentemente usado em perguntas de sim ou não}
\end{EntryWithPhonetic}

\begin{EntryWithPhonetic}{能干}{neng2gan4}{10,3}{⾁、⼲}[HSK 4]
  \definition{adj.}{apto; capaz; competente}
\end{EntryWithPhonetic}

\begin{EntryWithPhonetic}{能够}{neng2 gou4}{10,11}{⾁、⼣}[HSK 2]
  \definition{v.}{poder; ser capaz de; indica que possui uma determinada capacidade ou que atingiu um determinado nível de eficiência | poder; ser capaz de; indica que algo é permitido sob certas condições ou por motivos razoáveis}
\end{EntryWithPhonetic}

\begin{EntryWithPhonetic}{能力}{neng2li4}{10,2}{⾁、⼒}[HSK 3]
  \definition[个,种]{s.}{habilidade; capacidade; aptidão; as condições subjetivas para ser competente para uma tarefa}
\end{EntryWithPhonetic}

\begin{EntryWithPhonetic}{能量}{neng2liang4}{10,12}{⾁、⾥}[HSK 5]
  \definition[种]{s.}{energia; quantidade de energia; Uma grandeza física que mede a capacidade da matéria de realizar trabalho | capacidade; competências; capacidade e papel que uma pessoa pode desempenhar}
\end{EntryWithPhonetic}

\begin{EntryWithPhonetic}{能上能下}{neng2shang4neng2xia4}{10,3,10,3}{⾁、⼀、⾁、⼀}
  \definition{s.}{pronto para aceitar qualquer trabalho, alto ou baixo}
\end{EntryWithPhonetic}

\begin{EntryWithPhonetic}{呢}{ni2}{8}{⼝}
  \definition{s.}{(tecido feito de) lã; tecido de lã (para roupas pesadas); tecido de lã pesada; revestimento ou roupa de lã}
  \seeref{ne5}
\end{EntryWithPhonetic}

\begin{EntryWithPhonetic}{泥}{ni2}{8}{⽔}[HSK 6]
  \definition*{s.}{Sobrenome Ni}
  \definition{s.}{lama; atoleiro | pasta ou polpa; amassado | qualquer matéria pastosa; purê de vegetais ou frutas}
  \seeref{ni4}
\end{EntryWithPhonetic}

\begin{EntryWithPhonetic}{泥潭}{ni2tan2}{8,15}{⽔、⽔}
  \definition{s.}{atoleiro | lamaçal | charco | pântano}
\end{EntryWithPhonetic}

\begin{EntryWithPhonetic}{你}{ni3}{7}{⼈}[HSK 1]
  \definition{pron.}{você (segunda pessoa do singular); refere-se à pessoa com quem se está conversando | (referindo-se a qualquer pessoa) você; um; qualquer um | com 我 ou 你 em estruturas paralelas para indicar várias ou muitas pessoas se comportando da mesma maneira}
  \seealsoref{您}{nin2}
  \seealsoref{我}{wo3}
\end{EntryWithPhonetic}

\begin{EntryWithPhonetic}{你的}{ni3 de5}{7,8}{⼈、⽩}
  \definition{pron.}{seu}
\end{EntryWithPhonetic}

\begin{EntryWithPhonetic}{你好}{ni3hao3}{7,6}{⼈、⼥}
  \definition{interj.}{Olá! | Oi!}
\end{EntryWithPhonetic}

\begin{EntryWithPhonetic}{你们}{ni3men5}{7,5}{⼈、⼈}[HSK 1]
  \definition{pron.}{você (segunda pessoa do plural); refere-se a mais de uma pessoa ou a várias pessoas, incluindo a outra parte}
\end{EntryWithPhonetic}

\begin{EntryWithPhonetic}{你们的}{ni3men5 de5}{7,5,8}{⼈、⼈、⽩}
  \definition{pron.}{vossos}
\end{EntryWithPhonetic}

\begin{EntryWithPhonetic}{伲}{ni4}{7}{⼈}
  \definition{pron.}{Dialeto: eu; nós; meu; nosso}
  \seealsoref{你}{ni3}
\end{EntryWithPhonetic}

\begin{EntryWithPhonetic}{泥}{ni4}{8}{⽔}
  \definition{adj.}{fanático; teimoso; obstinado; cabeçudo}
  \definition{v.}{cobrir ou rebocar com gesso, massa de vidraceiro, etc.}
  \seeref{ni2}
\end{EntryWithPhonetic}

\begin{EntryWithPhonetic}{逆}{ni4}{9}{⾡}
  \definition{adj.}{contrário (oposto a 顺) ; contra; oposto; inverso | traidor; rebelde}
  \definition{adv.}{antecipadamente; com antecedência}
  \definition{s.}{traidor; rebelde}
  \definition{v.}{ir contra; opor-se; desobedecer; resistir; desafiar (oposto a 顺) | (literário) saudar; cumprimentar}
  \seealsoref{顺}{shun4}
\end{EntryWithPhonetic}

\begin{EntryWithPhonetic}{逆境}{ni4jing4}{9,14}{⾡、⼟}
  \definition[对]{s.}{adversidade; tribulação; circunstâncias adversas; circunstâncias desfavoráveis}
\end{EntryWithPhonetic}

\begin{EntryWithPhonetic}{年}{nian2}{6}{⼲}[HSK 1]
  \definition*{s.}{Sobrenome Nian}
  \definition{clas.}{ano; usado para calcular o número de anos}
  \definition{s.}{ano | idade | um período (época) da história | colheita anual | Ano Novo | artigos para o dia de Ano Novo | um período da vida de uma pessoa; fases da vida humana divididas por idade}
\end{EntryWithPhonetic}

\begin{EntryWithPhonetic}{年初}{nian2 chu1}{6,7}{⼲、⾐}[HSK 3]
  \definition{s.}{o começo do ano; os primeiros dias do ano}
\end{EntryWithPhonetic}

\begin{EntryWithPhonetic}{年代}{nian2dai4}{6,5}{⼲、⼈}[HSK 3]
  \definition[个]{s.}{idade; anos; tempo; um período de tempo com características distintas na história | uma década de um século; período de dez anos}
\end{EntryWithPhonetic}

\begin{EntryWithPhonetic}{年底}{nian2 di3}{6,8}{⼲、⼴}[HSK 3]
  \definition[个]{s.}{fim de ano; o fim do ano; geralmente os últimos dias de dezembro ou o fim do ano}
\end{EntryWithPhonetic}

\begin{EntryWithPhonetic}{年度}{nian2du4}{6,9}{⼲、⼴}[HSK 5]
  \definition{s.}{ano; de acordo com a natureza e as necessidades de um negócio, há um prazo de doze meses com data de início e término definidas}
\end{EntryWithPhonetic}

\begin{EntryWithPhonetic}{年货}{nian2huo4}{6,8}{⼲、⾙}
  \definition{s.}{mercadorias vendidas no Ano Novo Chinês}
\end{EntryWithPhonetic}

\begin{EntryWithPhonetic}{年级}{nian2ji2}{6,6}{⼲、⽷}[HSK 2]
  \definition[个]{s.}{série; ano; níveis divididos de acordo com o tempo de estudo dos alunos na escola}
\end{EntryWithPhonetic}

\begin{EntryWithPhonetic}{年纪}{nian2ji4}{6,6}{⼲、⽷}[HSK 3]
  \definition[把,个]{s.}{idade (de uma pessoa)}
\end{EntryWithPhonetic}

\begin{EntryWithPhonetic}{年龄}{nian2ling2}{6,13}{⼲、⿒}[HSK 5]
  \definition[个,段]{s.}{idade; animais, plantas e outros seres vivos vivem e crescem no mundo durante um determinado número de anos}
\end{EntryWithPhonetic}

\begin{EntryWithPhonetic}{年前}{nian2 qian2}{6,9}{⼲、⼑}[HSK 5]
  \definition{s.}{(pouco) antes da virada do ano | antes do final do ano | antes do ano novo}
\end{EntryWithPhonetic}

\begin{EntryWithPhonetic}{年轻}{nian2qing1}{6,9}{⼲、⾞}[HSK 2]
  \definition{adj.}{jovem; não muito velho (geralmente se refere a pessoas entre 10 e 20 anos)}
\end{EntryWithPhonetic}

\begin{EntryWithPhonetic}{碾}{nian3}{15}{⽯}
  \definition[台,个]{s.}{rolo e mó; rolo de pedra | rolo compressor}
  \definition{v.}{moer ou descascar com um rolo; esmagar | (literário) cortar e polir (jade, vidro, etc.) | achatar | pisar; pisotear, 轧}
  \seealsoref{辗}{zhan3}
\end{EntryWithPhonetic}

\begin{EntryWithPhonetic}{碾碎}{nian3sui4}{15,13}{⽯、⽯}
  \definition{v.}{pulverizar | esmagar}
\end{EntryWithPhonetic}

\begin{EntryWithPhonetic}{廿}{nian4}{4}{⼶}
  \definition{num.}{(dialeto) vinte; 20}
\end{EntryWithPhonetic}

\begin{EntryWithPhonetic}{念}{nian4}{8}{⼼}[HSK 3]
  \definition*{s.}{Sobrenome Nian}
  \definition{num.}{vinte; 20; capitalização do número 廿}
  \definition{s.}{ideia; pensamento; pensamentos ou intenções internas}
  \definition{v.}{ler em voz alta | estudar; frequentar a escola | considerar; levar em conta | sentir falta; pensar em; pensar sobre; pensar frequentemente sobre}
  \seealsoref{廿}{nian4}
\end{EntryWithPhonetic}

\begin{EntryWithPhonetic}{鸟}{niao3}{5}{⿃}[HSK 2][Kangxi 196]
  \definition*{s.}{Sobrenome Niao}
  \definition[只,群]{s.}{pássaro; ave}
  \seeref{diao3}
\end{EntryWithPhonetic}

\begin{EntryWithPhonetic}{鸟儿}{niao3r5}{5,2}{⿃、⼉}
  \definition[只]{s.}{pássaro | ave}
\end{EntryWithPhonetic}

\begin{EntryWithPhonetic}{尿}{niao4}{7}{⼫}
  \definition[泡]{s.}{urina}
  \definition{v.}{urinar}
  \seeref{sui1}
\end{EntryWithPhonetic}

\begin{EntryWithPhonetic}{您}{nin2}{11}{⼼}[HSK 1]
  \definition{pron.}{você; a forma de tratamento respeitosa da segunda pessoa do singular 你}
  \seealsoref{你}{ni3}
\end{EntryWithPhonetic}

\begin{EntryWithPhonetic}{宁}{ning2}{5}{⼧}
  \definition*{s.}{Região Autônoma de Ningxia Hui, abreviação de 宁夏回族自治区 | outro nome para Nanquim, 南京 | Sobrenome Ning}
  \definition{adj.}{calmo, pacífico, sereno | saudável}
  \definition{v.}{Literário: fazer uma visita (aos pais ou aos mais velhos); | Literário: pacificar; apaziguar}
  \seeref{ning4}
  \seealsoref{南京}{nan2jing1}
  \seealsoref{宁夏回族自治区}{ning2xia4 hui2zu2 zi4zhi4qu1}
\end{EntryWithPhonetic}

\begin{EntryWithPhonetic}{宁静}{ning2 jing4}{5,14}{⼧、⾭}[HSK 4]
  \definition{adj.}{calmo; tranquilo; pacífico}
\end{EntryWithPhonetic}

\begin{EntryWithPhonetic}{宁夏回族自治区}{ning2xia4 hui2zu2 zi4zhi4qu1}{5,10,6,11,6,8,4}{⼧、⼢、⼞、⽅、⾃、⽔、⼖}
  \definition*{s.}{Região Autônoma de Ningxia Hui}
\end{EntryWithPhonetic}

\begin{EntryWithPhonetic}{拧}{ning2}{8}{⼿}
  \definition{v.}{torcer | beliscar; torcer a pele com os dedos e virá-la com força}
  \seeref{ning3}
  \seeref{ning4}
\end{EntryWithPhonetic}

\begin{EntryWithPhonetic}{柠}{ning2}{9}{⽊}
  \definition{s.}{limão}
\end{EntryWithPhonetic}

\begin{EntryWithPhonetic}{柠檬}{ning2meng2}{9,17}{⽊、⽊}
  \definition[个,片,只]{s.}{limão}
\end{EntryWithPhonetic}

\begin{EntryWithPhonetic}{拧}{ning3}{8}{⼿}
  \definition{adj.}{errado; equivocado; de cabeça para baixo; oposto}
  \definition{v.}{torcer; parafusar | divergir; discordar; estar em desacordo}
  \seeref{ning2}
  \seeref{ning4}
\end{EntryWithPhonetic}

\begin{EntryWithPhonetic}{拧开}{ning3kai1}{8,4}{⼿、⼶}
  \definition{v.}{desaparafusar | desatarrachar | torcer (uma tampa) | abrir (uma torneira) | ligar (girando um botão) | girar (maçaneta da porta)}
\end{EntryWithPhonetic}

\begin{EntryWithPhonetic}{宁}{ning4}{5}{⼧}
  \definition{conj.}{mais\dots do que\dots, melhor\dots do que\dots}
  \seeref{ning2}
\end{EntryWithPhonetic}

\begin{EntryWithPhonetic}{宁可}{ning4ke3}{5,5}{⼧、⼝}
  \definition{conj.}{mais\dots do que\dots | melhor\dots do que\dots}
\end{EntryWithPhonetic}

\begin{EntryWithPhonetic}{宁可……也不……}{ning4ke3 ye3bu4}{5,5,3,4}{⼧、⼝、⼄、⼀}
  \definition{conj.}{preferiria\dots do que\dots}
\end{EntryWithPhonetic}

\begin{EntryWithPhonetic}{宁可……也要……}{ning4ke3 ye3yao4}{5,5,3,9}{⼧、⼝、⼄、⾑}
  \definition{conj.}{mesmo que tenhamos que\dots nós iremos\dots}
\end{EntryWithPhonetic}

\begin{EntryWithPhonetic}{宁肯}{ning4ken3}{5,8}{⼧、⾁}
  \definition{conj.}{mais\dots do que\dots, melhor\dots do que\dots}
\end{EntryWithPhonetic}

\begin{EntryWithPhonetic}{宁愿}{ning4yuan4}{5,14}{⼧、⽕}
  \definition{conj.}{mais\dots do que\dots, melhor\dots do que\dots}
\end{EntryWithPhonetic}

\begin{EntryWithPhonetic}{拧}{ning4}{8}{⼿}
  \definition{adj.}{teimoso}
  \seeref{ning2}
  \seeref{ning3}
\end{EntryWithPhonetic}

\begin{EntryWithPhonetic}{牛}{niu2}{4}{⽜}[HSK 3,5][Kangxi 93]
  \definition*{s.}{Sobrenome Niu}
  \definition{adj.}{muito capaz ou bom; descreve pessoas ou coisas como sendo muito capazes, muito competentes | teimoso; arrogante; descreve uma pessoa que é muito orgulhosa ou muito insistente em suas opiniões, difícil de mudar}
  \definition{clas.}{Newton (medida física de força)}
  \definition[头]{s.}{gado; boi | niu (nona das vinte e oito constelações em que a esfera celeste foi dividida, consistindo de seis estrelas, três em Áries e três em Sagitário)}
\end{EntryWithPhonetic}

\begin{EntryWithPhonetic}{牛顿}{niu2dun4}{4,10}{⽜、⾴}
  \definition*{s.}{Newton (nome) | N; Newton, unidade de força do SI}
\end{EntryWithPhonetic}

\begin{EntryWithPhonetic}{牛郎织女}{niu2 lang2 zhi1nv3}{4,8,8,3}{⽜、⾢、⽷、⼥}
  \definition*{s.}{Vaqueiro e Tecelã (personagens de contos folclóricos) | Altair e Vega (estrelas)}[我们这些牛郎织女都恨透了那条无情的“天河”。===Nós, o Vaqueiro e a Tecelã, odiamos a implacável ``Via Láctea''.]
  \definition{s.}{marido e mulher que vivem longe um do outro}
\end{EntryWithPhonetic}

\begin{EntryWithPhonetic}{牛奶}{niu2nai3}{4,5}{⽜、⼥}[HSK 1]
  \definition[杯,袋,瓶,盒,箱,桶]{s.}{leite}
\end{EntryWithPhonetic}

\begin{EntryWithPhonetic}{牛人}{niu2ren2}{4,2}{⽜、⼈}
  \definition{s.}{(coloquial) o cara | verdadeiro especialista | \emph{badass}}
\end{EntryWithPhonetic}

\begin{EntryWithPhonetic}{牛肉}{niu2rou4}{4,6}{⽜、⾁}
  \definition{s.}{carne de vaca | bife}
\end{EntryWithPhonetic}

\begin{EntryWithPhonetic}{牛仔裤}{niu2zai3ku4}{4,5,12}{⽜、⼈、⾐}[HSK 5]
  \definition[条]{s.}{calças jeans; calças geralmente feitas de tecido jeans azul grosso}
\end{EntryWithPhonetic}

\begin{EntryWithPhonetic}{扭}{niu3}{7}{⼿}[HSK 6]
  \definition{v.}{virar-se; girar | torcer; girar | torcer; luxar | rolar; balançar (ao caminhar) | agarrar; pegar;  lutar com}
\end{EntryWithPhonetic}

\begin{EntryWithPhonetic}{农}{nong2}{6}{⼍}
  \definition*{s.}{Sobrenome Nong}
  \definition{s.}{agricultura; criação de animais | camponês; fazendeiro}
\end{EntryWithPhonetic}

\begin{EntryWithPhonetic}{农产品}{nong2 chan3 pin3}{6,6,9}{⼍、⼇、⼝}[HSK 5]
  \definition[批]{s.}{produtos agrícolas}
\end{EntryWithPhonetic}

\begin{EntryWithPhonetic}{农村}{nong2cun1}{6,7}{⼍、⽊}[HSK 3]
  \definition{s.}{aldeia; campo; área rural; locais onde vivem os trabalhadores principalmente dedicados à produção agrícola}
\end{EntryWithPhonetic}

\begin{EntryWithPhonetic}{农民}{nong2min2}{6,5}{⼍、⽒}[HSK 3]
  \definition[个,位,名,些]{s.}{fazendeiro; camponês; campesinato; trabalhadores que participam da produção agrícola há muito tempo}
\end{EntryWithPhonetic}

\begin{EntryWithPhonetic}{农业}{nong2ye4}{6,5}{⼍、⼀}[HSK 3]
  \definition{s.}{agricultura}
\end{EntryWithPhonetic}

\begin{EntryWithPhonetic}{浓}{nong2}{9}{⽔}[HSK 4]
  \definition{adj.}{denso; espesso; concentrado; um líquido ou gás que contém mais de um determinado ingrediente | grande; forte; profundo (de grau ou extensão) | profundo; (algumas cores) escuro}
\end{EntryWithPhonetic}

\begin{EntryWithPhonetic}{弄}{nong4}{7}{⼶}[HSK 2]
  \definition{v.}{fazer, realizar; tratar; organizar | obter; buscar; tentar conseguir; encontrar uma maneira de conseguir | brincar com; enganar | pregar uma peça; brincar; manipular | mexer com; perturbar}
  \seeref{long4}
\end{EntryWithPhonetic}

\begin{EntryWithPhonetic}{努}{nu3}{7}{⼒}
  \definition{v.}{(coloquial) aplicar (a força de alguém); exercer (o esforço de alguém) | (dialeto) machucar-se por esforço excessivo | projetar-se; inchar | aplicar (força); exercer (esforço); usar}
  \seealsoref{呶}{nao2}
\end{EntryWithPhonetic}

\begin{EntryWithPhonetic}{努力}{nu3li4}{7,2}{⼒、⼒}[HSK 2]
  \definition{adj.}{extenuante; árduo | diligente; trabalhador; quem faz as coisas com o máximo de capacidade ou esforço possível}
  \definition{s.}{esforço; tentativa; fazer o melhor possível}
  \definition{v.}{fazer grandes esforços; esforçar-se; empenhar-se | esforçar-se; usar toda a força possível}
\end{EntryWithPhonetic}

\begin{EntryWithPhonetic}{怒}{nu4}{9}{⼼}
  \definition{adj.}{zangado; furioso | feroz; forte; descreve um forte impulso}
  \definition{adv.}{com força; vigorosamente; dinamicamente | com raiva}
  \definition{s.}{raiva; fúria}
  \definition{v.}{enfurecer-se; ficar com raiva}
\end{EntryWithPhonetic}

\begin{EntryWithPhonetic}{怒放}{nu4fang4}{9,8}{⼼、⽅}
  \definition{v.}{florescer em plena floração}
\end{EntryWithPhonetic}

\begin{EntryWithPhonetic}{怒骂}{nu4ma4}{9,9}{⼼、⾺}
  \definition{v.}{praguejar de raiva}
\end{EntryWithPhonetic}

\begin{EntryWithPhonetic}{暖}{nuan3}{13}{⽇}[HSK 5]
  \definition{adj.}{caloroso; cordial}
  \definition{v.}{aquecer; esquentar; aquecer algo ou aquecer o corpo}
\end{EntryWithPhonetic}

\begin{EntryWithPhonetic}{暖和}{nuan3huo5}{13,8}{⽇、⼝}[HSK 3]
  \definition{adj.}{morno; nem frio nem quente}
  \definition{v.}{aquecer; esquentar}
\end{EntryWithPhonetic}

\begin{EntryWithPhonetic}{暖气}{nuan3qi4}{13,4}{⽇、⽓}[HSK 4]
  \definition[个,种]{s.}{aquecedor; aquecimento; aquecimento central}
\end{EntryWithPhonetic}

\begin{EntryWithPhonetic}{那}{nuo2}{6}{⾢}
  \definition*{s.}{Sobrenome Nuo}
  \seeref{na1}
  \seeref{na3}
  \seeref{na4}
  \seeref{ne4}
  \seeref{nei4}
\end{EntryWithPhonetic}

\begin{EntryWithPhonetic}{诺}{nuo4}{10}{⾔}
  \definition*{s.}{Sobrenome Nuo}
  \definition{interj.}{Sim!}
  \definition{v.}{prometer}
\end{EntryWithPhonetic}

\begin{EntryWithPhonetic}{诺贝尔奖}{nuo4bei4'er3 jiang3}{10,4,5,9}{⾔、⾙、⼩、⼤}
  \definition*{s.}{Prêmio Nobel}
\end{EntryWithPhonetic}

\begin{EntryWithPhonetic}{诺奖}{nuo4jiang3}{10,9}{⾔、⼤}
  \definition*{s.}{Prêmio Nobel, abreviação de 诺贝尔奖}
  \seealsoref{诺贝尔奖}{nuo4bei4'er3 jiang3}
\end{EntryWithPhonetic}

\begin{EntryWithPhonetic}{女}{nv3}{3}{⼥}[HSK 1][Kangxi 38]
  \definition{adj.}{mulher; feminino (em oposição a 男) | fêmea (de certos animais)}
  \definition{s.}{menina; filha | nü, uma das mansões lunares | mulher}
  \seealsoref{男}{nan2}
\end{EntryWithPhonetic}

\begin{EntryWithPhonetic}{女儿}{nv3'er2}{3,2}{⼥、⼉}[HSK 1]
  \definition[个]{s.}{menina; filha}
  \seealsoref{儿子}{er2zi5}
\end{EntryWithPhonetic}

\begin{EntryWithPhonetic}{女孩}{nv3hai2}{3,9}{⼥、⼦}
  \definition{s.}{menina | garota}
\end{EntryWithPhonetic}

\begin{EntryWithPhonetic}{女孩儿}{nv3 hai2r5}{3,9,2}{⼥、⼦、⼉}[HSK 1]
  \definition{s.}{garota; menina; atualmente também se refere a mulher adolescente | filha}
\end{EntryWithPhonetic}

\begin{EntryWithPhonetic}{女朋友}{nv3 peng2 you5}{3,8,4}{⼥、⽉、⼜}[HSK 1]
  \definition{s.}{namorada}
\end{EntryWithPhonetic}

\begin{EntryWithPhonetic}{女人}{nv3 ren2}{3,2}{⼥、⼈}[HSK 1]
  \definition[个,位]{s.}{mulher adulta}
\end{EntryWithPhonetic}

\begin{EntryWithPhonetic}{女生}{nv3 sheng1}{3,5}{⼥、⽣}[HSK 1]
  \definition[个]{s.}{estudante; aluna; estudante do sexo feminino | menina; jovem mulher}
\end{EntryWithPhonetic}

\begin{EntryWithPhonetic}{女士}{nv3shi4}{3,3}{⼥、⼠}[HSK 4]
  \definition{pron.}{Sra.; Senhorita; Senhora; título honorífico para mulheres (agora usado em contextos diplomáticos)}
  \definition[位,名,个,些]{s.}{senhora; madame}
\end{EntryWithPhonetic}

\begin{EntryWithPhonetic}{女王}{nv3wang2}{3,4}{⼥、⽟}
  \definition{s.}{rainha}
\end{EntryWithPhonetic}

\begin{EntryWithPhonetic}{女性}{nv3 xing4}{3,8}{⼥、⼼}[HSK 5]
  \definition[个,位,名]{s.}{mulher; feminino; feminilidade; em oposição a 男性}
  \seealsoref{男性}{nan2 xing4}
\end{EntryWithPhonetic}

\begin{EntryWithPhonetic}{女婿}{nv3xu5}{3,12}{⼥、⼥}
  \definition{s.}{marido da filha}
\end{EntryWithPhonetic}

\begin{EntryWithPhonetic}{女子}{nv3 zi3}{3,3}{⼥、⼦}[HSK 3]
  \definition[位,名,个]{s.}{mulher; feminino; pessoa do sexo feminino}
\end{EntryWithPhonetic}

%%%%% EOF %%%%%


 %%%
%%% O
%%%

\section*{O}\addcontentsline{toc}{section}{O}

\begin{EntryWithPhonetic}{喔}{o1}{12}{⼝}
  \definition{interj.}{Oh!, Entendi!, usado para indicar realização, compreensão}
\end{EntryWithPhonetic}

\begin{EntryWithPhonetic}{哦}{o2}{10}{⼝}
  \definition{interj.}{Oh! (indicando dúvida ou surpresa)}
  \seeref{e2}
  \seeref{o4}
  \seeref{o5}
\end{EntryWithPhonetic}

\begin{EntryWithPhonetic}{哦}{o4}{10}{⼝}
  \definition{interj.}{Oh! (indicando que acabou de aprender algo)}
  \seeref{e2}
  \seeref{o2}
  \seeref{o5}
\end{EntryWithPhonetic}

\begin{EntryWithPhonetic}{哦}{o5}{10}{⼝}
  \definition{part.}{usada no final da frase para indicar que uma pessoa está afirmando um fato que a outra pessoa não sabe | usada no final da frase para transmitir informalidade, calor, simpatia ou intimidade}
  \seeref{e2}
  \seeref{o2}
  \seeref{o4}
\end{EntryWithPhonetic}

\begin{EntryWithPhonetic}{区}{ou1}{4}{⼖}
  \definition*{s.}{Sobrenome Ou}
  \seeref{qu1}
\end{EntryWithPhonetic}

\begin{EntryWithPhonetic}{欧}{ou1}{8}{⽋}
  \definition*{s.}{Europa, abreviação de 欧洲 | Sobrenome Ou}
  \seealsoref{欧洲}{ou1zhou1}
\end{EntryWithPhonetic}

\begin{EntryWithPhonetic}{欧盟}{ou1meng2}{8,13}{⽋、⽫}
  \definition*{s.}{União Europeia (EU)}
\end{EntryWithPhonetic}

\begin{EntryWithPhonetic}{欧洲}{ou1zhou1}{8,9}{⽋、⽔}
  \definition*{s.}{Europa}
\end{EntryWithPhonetic}

\begin{EntryWithPhonetic}{欧洲共同体}{ou1zhou1 gong4tong2ti3}{8,9,6,6,7}{⽋、⽔、⼋、⼝、⼈}
  \definition*{s.}{Comunidade Europeia}
\end{EntryWithPhonetic}

\begin{EntryWithPhonetic}{欧洲人}{ou1zhou1ren2}{8,9,2}{⽋、⽔、⼈}
  \definition{s.}{europeu | pessoa ou povo da Europa}
\end{EntryWithPhonetic}

\begin{EntryWithPhonetic}{偶}{ou3}{11}{⼈}
  \definition{adv.}{por acaso; por acidente; de vez em quando; ocasionalmente | par; número par; pareado (em oposição a 奇)}
  \definition{s.}{imagem; ídolo; figuras feitas de madeira, barro, etc. | companheiro; cônjuge; parceiro; refere-se a um casal ou a um dos casais}
  \seealsoref{奇}{qi2}
\end{EntryWithPhonetic}

\begin{EntryWithPhonetic}{偶尔}{ou3'er3}{11,5}{⼈、⼩}[HSK 5]
  \definition{adj.}{ocasional}
  \definition{adv.}{ocasionalmente; de vez em quando; às vezes}
\end{EntryWithPhonetic}

\begin{EntryWithPhonetic}{偶然}{ou3ran2}{11,12}{⼈、⽕}[HSK 5]
  \definition{adj.}{acidental; ocasional}
  \definition{adv.}{por acaso; acidentalmente; sem querer; inesperadamente | ocasionalmente; de vez em quando; às vezes}
\end{EntryWithPhonetic}

\begin{EntryWithPhonetic}{偶像}{ou3xiang4}{11,13}{⼈、⼈}[HSK 5]
  \definition[位,个,名]{s.}{ídolo; pessoa amada pelas pessoas; refere-se a uma pessoa que é apreciada por todos e que, em certos aspectos, é digna de admiração e respeito}
\end{EntryWithPhonetic}

%%%%% EOF %%%%%


 %%%
%%% P
%%%
\section*{P}\addcontentsline{toc}{section}{P}\addcontentsline{loh}{figure}{\#\#\#\#\#\#\#\# P}

%%%%%%%%%% 趴 %%%%%%%%%%
\subsection*{趴}\addcontentsline{loh}{figure}{趴 \dpy{pa1}}

\begin{EntryWithPhonetic}{趴}{pa1}{9}{⾜}[HSK 7-9]
  \definition{v.}{deitar-se de bruços; arriar; espreguiçar-se | curvar-se; apoiar-se em; inclinar-se para a frente apoiando-se em um objeto}
\end{EntryWithPhonetic}

%%%%%%%%%% 扒 %%%%%%%%%%
\subsection*{扒}\addcontentsline{loh}{figure}{扒 \dpy{pa2}}

\begin{EntryWithPhonetic}{扒}{pa2}{5}{⼿}
  \definition{v.}{reunir; juntar; reunir ou espalhar coisas com as mãos ou com um ancinho | roubar; furtar | arranhar; coçar com as mãos | cozinhar; refogar; cozinhar os alimentos em fogo baixo}
  \seeref{ba1}
\end{EntryWithPhonetic}

\begin{EntryWithPhonetic}{扒犁}{pa2li2}{5,11}{⼿、⽜}
  \definition{s.}{Dialeto: trenó; arado}
  \seealsoref{爬犁}{pa2li2}
\end{EntryWithPhonetic}

%%%%%%%%%% 爬 %%%%%%%%%%
\subsection*{爬}\addcontentsline{loh}{figure}{爬 \dpy{pa2}}

\begin{EntryWithPhonetic}{爬}{pa2}{8}{⽖}[HSK 2]
  \definition{v.}{rastejar; arrastar-se; engatinhar | escalar; trepar; subir com dificuldade | sentar-se; levantar-se; levantar-se da posição deitada ou sentada}
\end{EntryWithPhonetic}

\begin{EntryWithPhonetic}{爬杆}{pa2gan1}{8,7}{⽖、⽊}
  \definition{s.}{escalada em poste}
  \definition{v.}{escalar um poste}
\end{EntryWithPhonetic}

\begin{EntryWithPhonetic}{爬竿}{pa2gan1}{8,9}{⽖、⽵}
  \definition{s.}{poste de escalada | escalada em poste (como ginástica ou ato de circo)}
\end{EntryWithPhonetic}

\begin{EntryWithPhonetic}{爬犁}{pa2li2}{8,11}{⽖、⽜}
  \definition{s.}{trenó}
  \seealsoref{扒犁}{pa2li2}
\end{EntryWithPhonetic}

\begin{EntryWithPhonetic}{爬墙}{pa2qiang2}{8,14}{⽖、⼟}
  \definition{v.}{escalar uma parede}
\end{EntryWithPhonetic}

\begin{EntryWithPhonetic}{爬山}{pa2/shan1}{8,3}{⽖、⼭}[HSK 2]
  \definition{v.+compl.}{escalar uma montanha;}
\end{EntryWithPhonetic}

\begin{EntryWithPhonetic}{爬上}{pa2shang4}{8,3}{⽖、⼀}
  \definition{v.}{escalar}
\end{EntryWithPhonetic}

\begin{EntryWithPhonetic}{爬升}{pa2sheng1}{8,4}{⽖、⼗}
  \definition{v.}{ascender | ganhar promoção | subir (números de vendas, etc.) | aumentar}
\end{EntryWithPhonetic}

\begin{EntryWithPhonetic}{爬梳}{pa2shu1}{8,11}{⽖、⽊}
  \definition{v.}{vasculhar (documentos históricos, etc.) | desvendar}
\end{EntryWithPhonetic}

\begin{EntryWithPhonetic}{爬行}{pa2xing2}{8,6}{⽖、⾏}
  \definition{v.}{rastejar | arrastar | engatinhar}
\end{EntryWithPhonetic}

%%%%%%%%%% 怕 %%%%%%%%%%
\subsection*{怕}\addcontentsline{loh}{figure}{怕 \dpy{pa4}}

\begin{EntryWithPhonetic}{怕}{pa4}{8}{⼼}[HSK 2]
  \definition{adv.}{(expressando suposição, julgamento, estimativa, etc.) talvez; suponho; receio (que)}
  \definition{adv.}{por medo; talvez; suponho}
  \definition{v.}{temer; ter medo; recear; sentir medo, ficar nervoso | estar preocupado com; estar preocupado por (ou sobre); ter medo de que algo possa acontecer | ser afetado por; não conseguir suportar; não aguentar mais}
\end{EntryWithPhonetic}

%%%%%%%%%% 拍 %%%%%%%%%%
\subsection*{拍}\addcontentsline{loh}{figure}{拍 \dpy{pai1}}

\begin{EntryWithPhonetic}{拍}{pai1}{8}{⼿}[HSK 3]
  \definition[个,副,对]{s.}{bastão; raquete | batida; tempo; (música) uma unidade para medir a duração de uma nota musical}
  \definition{v.}{tirar (uma foto); usar uma câmera para capturar imagens de pessoas e objetos em filme | dar um tapinha; bater suavemente com as mãos ou ferramentas | bater asas | bater (ondas do mar) | enviar (um telegrama, etc.) | bajular}
\end{EntryWithPhonetic}

\begin{EntryWithPhonetic}{拍板}{pai1/ban3}{8,8}{⼿、⽊}[HSK 7-9]
  \definition{s.}{aplausos}
  \definition{v.+compl.}{marcar o tempo com palmas | bater o martelo | ter a palavra final; dar o veredicto final; tomar a decisão final}
\end{EntryWithPhonetic}

\begin{EntryWithPhonetic}{拍马}{pai1ma3}{8,3}{⼿、⾺}
  \definition{v.}{instigar um cavalo dando tapinhas em seu traseiro | lisonjear | bajular}
  \seealsoref{拍马屁}{pai1ma3pi4}
\end{EntryWithPhonetic}

\begin{EntryWithPhonetic}{拍马屁}{pai1ma3pi4}{8,3,7}{⼿、⾺、⼫}
  \definition{s.}{puxa-saco | bajulador}
  \definition{v.}{puxar o saco | bajular}
  \seealsoref{拍马}{pai1ma3}
\end{EntryWithPhonetic}

\begin{EntryWithPhonetic}{拍卖}{pai1mai4}{8,8}{⼿、⼗}[HSK 7-9]
  \definition{s.}{leilão; uma forma pública de venda de produtos onde todos oferecem publicamente o seu preço, e o item é vendido para quem oferecer o preço mais alto}
  \definition{v.}{leiloar; realizar atividades de leilão | vender mercadorias a preços reduzidos; baixar o preço para vender as mercadorias rapidamente}
\end{EntryWithPhonetic}

\begin{EntryWithPhonetic}{拍摄}{pai1 she4}{8,13}{⼿、⼿}[HSK 5]
  \definition{s.}{fotografar; tirar (uma foto); usar uma câmera fotográfica para capturar imagens de pessoas e objetos}
\end{EntryWithPhonetic}

\begin{EntryWithPhonetic}{拍戏}{pai1/xi4}{8,6}{⼿、⼽}[HSK 7-9]
  \definition{v.+compl.}{fazer um filme ou peça de televisão; filmar uma cena | filmar}
\end{EntryWithPhonetic}

\begin{EntryWithPhonetic}{拍照}{pai1/zhao4}{8,13}{⼿、⽕}[HSK 4]
  \definition{v.+compl.}{fotografar; tirar uma foto}
\end{EntryWithPhonetic}

%%%%%%%%%% 徘 %%%%%%%%%%
\subsection*{徘}\addcontentsline{loh}{figure}{徘 \dpy{pai2}}

\begin{EntryWithPhonetic}{徘}{pai2}{11}{⼻}
  \definition{adj.}{irresoluto; indeciso}
  \definition{v.}{vagar}
\end{EntryWithPhonetic}

\begin{EntryWithPhonetic}{徘徊}{pai2huai2}{11,9}{⼻、⼻}[HSK 7-9]
  \definition{v.}{andar de um lado para o outro no mesmo lugar | Figurativo: vacilar; hesitar; uma metáfora para hesitação e indecisão | flutuar; essa metáfora descreve as coisas mudando para cima e para baixo dentro de uma determinada faixa}
\end{EntryWithPhonetic}

%%%%%%%%%% 排 %%%%%%%%%%
\subsection*{排}\addcontentsline{loh}{figure}{排 \dpy{pai2}}

\begin{EntryWithPhonetic}{排}{pai2}{11}{⼿}[HSK 2,3]
  \definition{clas.}{usado para linhas, filas; coisas usadas para formar filas}
  \definition{s.}{linha; fileira; fileiras horizontais | pelotão; unidade militar, abaixo do nível de companhia, acima do nível de pelotão | jangada; balsa; um meio de transporte aquático feito de bambu e madeira unidos lado a lado; também se refere a bambu e madeira amarrados em fileiras para facilitar o transporte aquático | torta; bolo de carne; bolinho assado; comida cozida no vapor}
  \definition{v.}{organizar; alinhar; colocar em ordem; posicionar ou organizar em uma determinada ordem; ordenar | ensaiar | ejetar; excluir; dispensar; remover; eliminar | empurrar o obstáculo para fora do caminho}
\end{EntryWithPhonetic}

\begin{EntryWithPhonetic}{排斥}{pai2chi4}{11,5}{⼿、⽄}[HSK 7-9]
  \definition{v.}{repelir; rejeitar; excluir; fazer com que (uma pessoa ou coisa) se afaste do seu próprio grupo}
\end{EntryWithPhonetic}

\begin{EntryWithPhonetic}{排除}{pai2chu2}{11,9}{⼿、⾩}[HSK 5]
  \definition{v.}{remover; superar; excluir; eliminar; livrar-se de}
\end{EntryWithPhonetic}

\begin{EntryWithPhonetic}{排队}{pai2/dui4}{11,4}{⼿、⾩}[HSK 2]
  \definition{v.+compl.}{formar uma fila; alinhar-se; enfileirar-se; organizar em sequência | listar; classificar}
\end{EntryWithPhonetic}

\begin{EntryWithPhonetic}{排放}{pai2fang4}{11,8}{⼿、⽅}[HSK 7-9]
  \definition{v.}{colocar (as coisas) em ordem adequada | emitir; descarregar (gases de escape, águas residuais, etc.); deixar sair; drenar}
\end{EntryWithPhonetic}

\begin{EntryWithPhonetic}{排行榜}{pai2 hang2 bang3}{11,6,14}{⼿、⾏、⽊}[HSK 6]
  \definition{s.}{lista; classificação; lista de classificação; (de registros) os gráficos; uma lista em uma determinada ordem publicada com base em certos resultados estatísticos}
\end{EntryWithPhonetic}

\begin{EntryWithPhonetic}{排挤}{pai2ji3}{11,9}{⼿、⼿}
  \definition{v.}{ostracizar; afastar; expulsar; espremer; excluir; marginalizar; usar o poder ou os meios para fazer com que aqueles que lhe são desfavoráveis ​​percam seu status ou seus interesses}
\end{EntryWithPhonetic}

\begin{EntryWithPhonetic}{排练}{pai2lian4}{11,8}{⼿、⽷}[HSK 7-9]
  \definition{v.}{ensaiar; ensaiar ou praticar uma determinada cerimônia ou apresentação}
\end{EntryWithPhonetic}

\begin{EntryWithPhonetic}{排列}{pai2lie4}{11,6}{⼿、⼑}[HSK 4]
  \definition{v.}{classificar; colocar; variar; organizar; pôr em ordem}
\end{EntryWithPhonetic}

\begin{EntryWithPhonetic}{排名}{pai2 ming2}{11,6}{⼿、⼝}[HSK 3]
  \definition{s.}{classificação; resultado; organizado de acordo com determinados critérios}
\end{EntryWithPhonetic}

\begin{EntryWithPhonetic}{排球}{pai2 qiu2}{11,11}{⼿、⽟}[HSK 2]
  \definition[场,只,个]{s.}{voleibol; bola de voleibol}
\end{EntryWithPhonetic}

\begin{EntryWithPhonetic}{排水}{pai2shui3}{11,4}{⼿、⽔}
  \definition{v.}{drenar}
\end{EntryWithPhonetic}

%%%%%%%%%% 牌 %%%%%%%%%%
\subsection*{牌}\addcontentsline{loh}{figure}{牌 \dpy{pai2}}

\begin{EntryWithPhonetic}{牌}{pai2}{12}{⽚}[HSK 4]
  \definition[块,副,张,个,种]{s.}{placa; tabuleta; quadro; placar | marca; marca registrada; marca comercial; \emph{trademark} | cartas, dominó, etc. | a tonalidade de uma música}
\end{EntryWithPhonetic}

\begin{EntryWithPhonetic}{牌照}{pai2zhao4}{12,13}{⽚、⽕}[HSK 7-9]
  \definition{s.}{placa de matrícula; certificado de licenciamento; certificado de registro de veículo ou licença comercial emitida pelo departamento administrativo competente}
\end{EntryWithPhonetic}

\begin{EntryWithPhonetic}{牌子}{pai2 zi5}{12,3}{⽚、⼦}[HSK 3]
  \definition[个,种,块]{s.}{sinal; placa; placas feitas de madeira ou outros materiais, geralmente com texto nelas | marca; marca registrada; um nome especial dado por uma empresa ao seu próprio produto}
\end{EntryWithPhonetic}

%%%%%%%%%% 派 %%%%%%%%%%
\subsection*{派}\addcontentsline{loh}{figure}{派 \dpy{pai4}}

\begin{EntryWithPhonetic}{派}{pai4}{9}{⽔}[HSK 3]
  \definition{adj.}{elegante; bonito; imponente}
  \definition{clas.}{usado para grupos, escolas de pensamento ou arte, etc. | usado para um discursos, situações, cenas, etc.}
  \definition[个,块,种]{s.}{panelinha; facção; pessoas com ideias, visões e estilos semelhantes | torta; um alimento recheado comumente consumido pelos ocidentais, geralmente doce | maneira e ar; estilo ou comportamento | afluente; braço de rio}
  \definition{v.}{enviar; despachar; arranjar ou ordenar que uma pessoa faça algo; providenciar transporte | alocar; repartir; distribuir}
\end{EntryWithPhonetic}

\begin{EntryWithPhonetic}{派别}{pai4bie2}{9,7}{⽔、⼑}[HSK 7-9]
  \definition{s.}{grupo; seita; escola; facção | categorias; panelinha}
\end{EntryWithPhonetic}

\begin{EntryWithPhonetic}{派出}{pai4 chu1}{9,5}{⽔、⼐}[HSK 6]
  \definition{v.}{despachar; expedi | enviar}
\end{EntryWithPhonetic}

\begin{EntryWithPhonetic}{派遣}{pai4qian3}{9,13}{⽔、⾡}[HSK 7-9]
  \definition{v.}{despachar; enviar alguém em missão (governo, organização, etc.)}
\end{EntryWithPhonetic}

%%%%%%%%%% 扳 %%%%%%%%%%
\subsection*{扳}\addcontentsline{loh}{figure}{扳 \dpy{pan1}}

\begin{EntryWithPhonetic}{扳}{pan1}{7}{⼿}
  \definition{v.}{segurar; agarrar; puxar; escalar | confiar em; buscar ajuda; associar-se a pessoas de status superior; refere-se a formar um relacionamento ou estabelecer um relacionamento com alguém de alto \emph{status} | envolver; relacionar-se com}
  \seeref{ban1}
\end{EntryWithPhonetic}

%%%%%%%%%% 攀 %%%%%%%%%%
\subsection*{攀}\addcontentsline{loh}{figure}{攀 \dpy{pan1}}

\begin{EntryWithPhonetic}{攀}{pan1}{19}{⼿}[HSK 7-9]
  \definition{v.}{escalar; escalar | buscar conexões em altos cargos | envolver; implicar | agarrar; agarrar-se; segurar-se a}
\end{EntryWithPhonetic}

\begin{EntryWithPhonetic}{攀爬}{pan1pa2}{19,8}{⼿、⽖}
  \definition{v.}{escalar; escalada em rocha; refere-se ao movimento em uma determinada direção usando apenas as mãos e os pés, com o mínimo uso de ferramentas}
\end{EntryWithPhonetic}

\begin{EntryWithPhonetic}{攀升}{pan1sheng1}{19,4}{⼿、⼗}[HSK 7-9]
  \definition{v.}{subir para um ponto mais alto | (preços, quantidade, etc.) subir; aumentar; escalar | subir}
\end{EntryWithPhonetic}

\begin{EntryWithPhonetic}{攀岩}{pan1yan2}{19,8}{⼿、⼭}
  \definition{s.}{escalada em rocha; isso se refere a esse tipo de esporte}
  \definition{v.}{escalar uma parede rochosa íngreme com equipamento mínimo}
\end{EntryWithPhonetic}

%%%%%%%%%% 爿 %%%%%%%%%%
\subsection*{爿}\addcontentsline{loh}{figure}{爿 \dpy{pan2}}

\begin{EntryWithPhonetic}{爿}{pan2}{4}{⽙}[Kangxi 90]
  \definition{clas.}{usado para faixas de terra ou bambu, lojas, fábricas etc.}
\end{EntryWithPhonetic}

%%%%%%%%%% 胖 %%%%%%%%%%
\subsection*{胖}\addcontentsline{loh}{figure}{胖 \dpy{pan2}}

\begin{EntryWithPhonetic}{胖}{pan2}{9}{⾁}
  \definition{adj.}{saudável}
  \seeref{pang4}
\end{EntryWithPhonetic}

%%%%%%%%%% 般 %%%%%%%%%%
\subsection*{般}\addcontentsline{loh}{figure}{般 \dpy{pan2}}

\begin{EntryWithPhonetic}{般}{pan2}{10}{⾈}
  \definition{adj.}{feliz; bem-aventurado}
  \seeref{ban1}
  \seeref{bo1}
\end{EntryWithPhonetic}

\begin{EntryWithPhonetic}{般乐}{pan2le4}{10,5}{⾈、⼃}
  \definition{v.}{jogar | divertir-se}
\end{EntryWithPhonetic}

%%%%%%%%%% 盘 %%%%%%%%%%
\subsection*{盘}\addcontentsline{loh}{figure}{盘 \dpy{pan2}}

\begin{EntryWithPhonetic}{盘}{pan2}{11}{⽫}[HSK 4,7-9]
  \definition*{s.}{Sobrenome: Pan}
  \definition{clas.}{usado para pratos, pedras de moer, etc. | usado para jogos de xadrez e de bola | usado para as coisas que estão entrelaçadas, emaranhadas}
  \definition[套,只]{s.}{bandeja; tabuleiro | recipiente plano e raso, como uma bandeja, prato, travessa etc.  | preço atual; cotação de mercado; refere-se ao preço básico pelo qual as commodities são negociadas}
  \definition{v.}{enrolar; torcer; enrolar (para cima); entrelaçar; cercar | construir (assentando tijolos, pedras, etc.) | checar; examinar; interrogar; verificar um por um ou repetidamente (quantidade, situação, etc.) | transferir a propriedade de; passar para outra pessoa | carregar; transportar}
\end{EntryWithPhonetic}

\begin{EntryWithPhonetic}{盘算}{pan2suan5}{11,14}{⽫、⽵}[HSK 7-9]
  \definition{v.}{calcular; determinar; planejar | considerar e ponderar; premeditar; deliberar}
\end{EntryWithPhonetic}

\begin{EntryWithPhonetic}{盘子}{pan2zi5}{11,3}{⽫、⼦}[HSK 4]
  \definition[个,叠,套,只]{s.}{prato; utensílio de fundo raso para guardar objetos, maior do que um pires, geralmente de formato redondo | situação de mercado; taxa de mercado; transação comercial}
\end{EntryWithPhonetic}

%%%%%%%%%% 槃 %%%%%%%%%%
\subsection*{槃}\addcontentsline{loh}{figure}{槃 \dpy{pan2}}

\begin{EntryWithPhonetic}{槃}{pan2}{14}{⽊}
  \variantof{盘}
\end{EntryWithPhonetic}

%%%%%%%%%% 判 %%%%%%%%%%
\subsection*{判}\addcontentsline{loh}{figure}{判 \dpy{pan4}}

\begin{EntryWithPhonetic}{判}{pan4}{7}{⼑}[HSK 6]
  \definition{adv.}{obviamente há uma diferença}
  \definition{v.}{distinguir; discriminar; separar | julgar; decidir; avaliar | sentenciar; condenar}
\end{EntryWithPhonetic}

\begin{EntryWithPhonetic}{判处}{pan4chu3}{7,5}{⼑、⼡}[HSK 7-9]
  \definition{s.}{sentença; condenação; o julgamento e a sentença, pelo tribunal, daqueles que violam a lei penal}
  \definition{v.}{sentenciar (alguém) a; condenar (alguém) a}
\end{EntryWithPhonetic}

\begin{EntryWithPhonetic}{判定}{pan4ding4}{7,8}{⼑、⼧}[HSK 7-9]
  \definition{v.}{julgar; decidir; determinar; a análise leva a uma conclusão ou a uma decisão}
\end{EntryWithPhonetic}

\begin{EntryWithPhonetic}{判断}{pan4duan4}{7,11}{⼑、⽄}[HSK 3]
  \definition[个,项]{s.}{julgamento; conclusões tiradas após reflexão e análise}
  \definition{v.}{julgar; decidir}
\end{EntryWithPhonetic}

\begin{EntryWithPhonetic}{判决}{pan4jue2}{7,6}{⼑、⼎}[HSK 7-9]
  \definition[个]{s.}{decisão judicial; julgamento}
  \definition{v.}{proferir um veredicto; adjudicar; condenar; emitir um julgamento; pronunciar (julgamento)}
\end{EntryWithPhonetic}

%%%%%%%%%% 叛 %%%%%%%%%%
\subsection*{叛}\addcontentsline{loh}{figure}{叛 \dpy{pan4}}

\begin{EntryWithPhonetic}{叛}{pan4}{9}{⼜}
  \definition{adj.}{rebelde}
  \definition{s.}{rebelião}
  \definition{v.}{trair | rebelar-se | revoltar-se}
\end{EntryWithPhonetic}

\begin{EntryWithPhonetic}{叛逆}{pan4ni4}{9,9}{⼜、⾡}[HSK 7-9]
  \definition{s.}{rebelde; pessoas que traem}
  \definition{v.}{rebelar-se/revoltar-se contra; trair}
\end{EntryWithPhonetic}

%%%%%%%%%% 盼 %%%%%%%%%%
\subsection*{盼}\addcontentsline{loh}{figure}{盼 \dpy{pan4}}

\begin{EntryWithPhonetic}{盼}{pan4}{9}{⽬}[HSK 7-9]
  \definition*{s.}{Sobrenome: Pan}
  \definition{adj.}{(olhos) com preto e branco fortemente contrastados; olhos claros}
  \definition{v.}{olhar | esperar por; ansiar por | sentir falta de ; continuar pensando sobre}
\end{EntryWithPhonetic}

\begin{EntryWithPhonetic}{盼望}{pan4wang4}{9,11}{⽬、⽉}[HSK 6]
  \definition{v.}{esperar por; ansiar por; esperar que algo aconteça em breve}
\end{EntryWithPhonetic}

%%%%%%%%%% 乓 %%%%%%%%%%
\subsection*{乓}\addcontentsline{loh}{figure}{乓 \dpy{pang1}}

\begin{EntryWithPhonetic}{乓}{pang1}{6}{⼃}
  \definition{interj.}{(onomatopéia) barulho repentino feito por tiros, uma porta batendo, coisas quebrando, etc.; estrondo; estouro; batida; colisão}
\end{EntryWithPhonetic}

%%%%%%%%%% 庞 %%%%%%%%%%
\subsection*{庞}\addcontentsline{loh}{figure}{庞 \dpy{pang2}}

\begin{EntryWithPhonetic}{庞}{pang2}{8}{⼴}
  \definition*{s.}{Sobrenome: Pang}
  \definition{adj.}{enorme | inúmeros e desordenados; numerosos e desorganizados}
  \definition{s.}{molde do rosto de alguém | rosto; placa frontal}
\end{EntryWithPhonetic}

\begin{EntryWithPhonetic}{庞大}{pang2da4}{8,3}{⼴、⼤}[HSK 7-9]
  \definition{adj.}{enorme; colossal; gigantesco; imenso; (em termos de forma, estrutura, quantidade, etc.) é muito grande; excessivamente grande}
\end{EntryWithPhonetic}

%%%%%%%%%% 旁 %%%%%%%%%%
\subsection*{旁}\addcontentsline{loh}{figure}{旁 \dpy{pang2}}

\begin{EntryWithPhonetic}{旁}{pang2}{10}{⽅}[HSK 5]
  \definition{adj.}{outro | abundante; abrangente}
  \definition{s.}{lado | radical lateral de um caractere chinês}
\end{EntryWithPhonetic}

\begin{EntryWithPhonetic}{旁边}{pang2bian1}{10,5}{⽅、⾡}[HSK 1]
  \definition{s.}{junto a; próximo de; ao lado}
\end{EntryWithPhonetic}

\begin{EntryWithPhonetic}{旁观}{pang2guan1}{10,6}{⽅、⾒}[HSK 7-9]
  \definition{v.}{observar; ser um espectador | observar de fora}
\end{EntryWithPhonetic}

%%%%%%%%%% 磅 %%%%%%%%%%
\subsection*{磅}\addcontentsline{loh}{figure}{磅 \dpy{pang2}}

\begin{EntryWithPhonetic}{磅}{pang2}{15}{⽯}
  \definition{adj.}{majestoso; abundante; cheio de energia; magnífico}
  \seeref{bang4}
\end{EntryWithPhonetic}

%%%%%%%%%% 胖 %%%%%%%%%%
\subsection*{胖}\addcontentsline{loh}{figure}{胖 \dpy{pang4}}

\begin{EntryWithPhonetic}{胖}{pang4}{9}{⾁}[HSK 3]
  \definition{adj.}{gordo; robusto; rechonchudo; (corpo humano) com muita gordura ou carne (em oposição a 瘦)}
  \seeref{pan2}
  \seealsoref{瘦}{shou4}
\end{EntryWithPhonetic}

\begin{EntryWithPhonetic}{胖子}{pang4 zi5}{9,3}{⾁、⼦}[HSK 4]
  \definition[个]{s.}{obeso; gordo; pessoa gorda}
\end{EntryWithPhonetic}

%%%%%%%%%% 抛 %%%%%%%%%%
\subsection*{抛}\addcontentsline{loh}{figure}{抛 \dpy{pao1}}

\begin{EntryWithPhonetic}{抛}{pao1}{7}{⼿}[HSK 7-9]
  \definition{v.}{atirar; lançar; arremessar | deixar para trás; descartar; abandonar | vender por um preço inferior ao real; vender em excesso; vender em grande quantidade | mostrar; expor}
\end{EntryWithPhonetic}

\begin{EntryWithPhonetic}{抛开}{pao1kai1}{7,4}{⼿、⼶}[HSK 7-9]
  \definition{v.}{livrar-se de; revogar; afastar-se | jogar fora}
\end{EntryWithPhonetic}

\begin{EntryWithPhonetic}{抛弃}{pao1qi4}{7,7}{⼿、⼶}[HSK 7-9]
  \definition{v.}{abandonar; deixar de lado; renunciar; jogar fora}
\end{EntryWithPhonetic}

%%%%%%%%%% 泡 %%%%%%%%%%
\subsection*{泡}\addcontentsline{loh}{figure}{泡 \dpy{pao1}}

\begin{EntryWithPhonetic}{泡}{pao1}{8}{⽔}
  \definition{adj.}{esponjoso; oco e macio; não duro}
  \definition{clas.}{usado para fezes e urina}
  \definition[串,个]{s.}{algo fofo e macio | pequeno lago}
  \seeref{pao4}
\end{EntryWithPhonetic}

%%%%%%%%%% 刨 %%%%%%%%%%
\subsection*{刨}\addcontentsline{loh}{figure}{刨 \dpy{pao2}}

\begin{EntryWithPhonetic}{刨}{pao2}{7}{⼑}[HSK 7-9]
  \definition{v.}{cavar; escavar | cortar; remover; remover ou subtrair de algo que já existe}
  \seeref{bao4}
\end{EntryWithPhonetic}

%%%%%%%%%% 炮 %%%%%%%%%%
\subsection*{炮}\addcontentsline{loh}{figure}{炮 \dpy{pao2}}

\begin{EntryWithPhonetic}{炮}{pao2}{9}{⽕}
  \definition{v.}{(medicina tradicional chinesa) preparar a medicina chinesa assando-a em uma panela de ferro quente até dourar e estalar}
  \seeref{bao1}
  \seeref{pao4}
\end{EntryWithPhonetic}

%%%%%%%%%% 跑 %%%%%%%%%%
\subsection*{跑}\addcontentsline{loh}{figure}{跑 \dpy{pao2}}

\begin{EntryWithPhonetic}{跑}{pao2}{12}{⾜}
  \definition{v.}{(de animais) bater com a pata (no chão); (de animais) escavar o solo com suas garras ou cascos}
  \seeref{pao3}
\end{EntryWithPhonetic}

\begin{EntryWithPhonetic}{跑}{pao3}{12}{⾜}[HSK 1]
  \definition{v.}{correr; pessoas ou animais que se movem rapidamente para a frente com as pernas e os pés | caminhar; passear | fugir; escapar | correr de um lado para outro; fazer rondas; correr atrás de algo | de um líquido ou gás) vazar; evaporar | (como complemento de um verbo) fora; longe | participar de uma corrida}
  \seeref{pao2}
\end{EntryWithPhonetic}

\begin{EntryWithPhonetic}{跑步}{pao3/bu4}{12,7}{⾜、⽌}[HSK 3]
  \definition{v.+compl.}{correr; trotar}
\end{EntryWithPhonetic}

\begin{EntryWithPhonetic}{跑车}{pao3che1}{12,4}{⾜、⾞}[HSK 7-9]
  \definition{s.}{bicicleta de corrida | \emph{roadster}; carro de corrida | carrinho para transportar toras em uma floresta}
  \definition{v.}{Coloquial: (condutores de trem) estar em serviço | (vagões de carvão em uma mina) deslizar acidentalmente para baixo (desgovernado) | Dialeto: dirigir um veículo de transporte | trabalhar em um trem; atendente de trem trabalhando no trem | escorregar acidentalmente para baixo; isso se refere a um acidente em um poço inclinado de mina, onde o cabo de aço se rompe repentinamente durante o içamento ou o guincho escorrega por outros motivos}
\end{EntryWithPhonetic}

\begin{EntryWithPhonetic}{跑道}{pao3dao4}{12,12}{⾜、⾡}[HSK 7-9]
  \definition[条]{s.}{pista de decolagem; a pista de taxiagem utilizada pelas aeronaves durante a decolagem e o pouso | pista; pista de atletismo; as linhas brancas desenhadas na pista são usadas para corridas de corrida ou ciclismo}
\end{EntryWithPhonetic}

\begin{EntryWithPhonetic}{跑调}{pao3diao4}{12,10}{⾜、⾔}
  \definition{v.}{(coloquial) estar fora do tom ou desafinado (enquanto canta)}
\end{EntryWithPhonetic}

\begin{EntryWithPhonetic}{跑掉}{pao3diao4}{12,11}{⾜、⼿}
  \definition{v.}{fugir}
\end{EntryWithPhonetic}

\begin{EntryWithPhonetic}{跑肚}{pao3du4}{12,7}{⾜、⾁}
  \definition{v.}{(coloquial) ter diarréia}
\end{EntryWithPhonetic}

\begin{EntryWithPhonetic}{跑酷}{pao3ku4}{12,14}{⾜、⾣}
  \definition*{s.}{Eempréstimo linguístico: Parkour}
\end{EntryWithPhonetic}

\begin{EntryWithPhonetic}{跑龙套}{pao3 long2tao4}{12,5,10}{⾜、⿓、⼤}[HSK 7-9]
  \definition{v.}{Teatro: interpretar um papel secundário | desempenhar um papel secundário; não ser ninguém | desempenhar um papel pequeno}
\end{EntryWithPhonetic}

\begin{EntryWithPhonetic}{跑马}{pao3ma3}{12,3}{⾜、⾺}
  \definition{s.}{corrida de cavalos}
  \definition{v.}{andar a cavalo em ritmo acelerado}
\end{EntryWithPhonetic}

\begin{EntryWithPhonetic}{跑题}{pao3ti2}{12,15}{⾜、⾴}
  \definition{v.}{divagar | fugir do assunto | tergiversar}
\end{EntryWithPhonetic}

\begin{EntryWithPhonetic}{跑腿}{pao3tui3}{12,13}{⾜、⾁}
  \definition{v.}{realizar tarefas}
\end{EntryWithPhonetic}

%%%%%%%%%% 泡 %%%%%%%%%%
\subsection*{泡}\addcontentsline{loh}{figure}{泡 \dpy{pao4}}

\begin{EntryWithPhonetic}{泡}{pao4}{8}{⽔}[HSK 6]
  \definition[串,个]{s.}{bolha | algo em forma de bolha}
  \definition{v.}{mergulhar; encharcar | despejar água fervente em (chá, sopa instantânea, etc.) | enrolar; demorar-se; ficar por aí | (coloquial) (de um homem) brincar no campo; brincar com uma mulher | perder tempo; matar o tempo deliberadamente}
  \seeref{pao1}
\end{EntryWithPhonetic}

\begin{EntryWithPhonetic}{泡沫}{pao4mo4}{8,8}{⽔、⽔}[HSK 7-9]
  \definition{s.}{espuma; pequenas bolhas se aglomeraram na superfície do líquido | ilusão; bolha econômica; essa metáfora descreve a prosperidade superficial e o florescimento de algo que, na realidade, é vazio e irreal}
\end{EntryWithPhonetic}

%%%%%%%%%% 炮 %%%%%%%%%%
\subsection*{炮}\addcontentsline{loh}{figure}{炮 \dpy{pao4}}

\begin{EntryWithPhonetic}{炮}{pao4}{9}{⽕}[HSK 6]
  \definition{s.}{arma grande; canhão; peça de artilharia | fogo de artifício | buraco de explosão cheio de dinamite | canhão, uma das peças do xadrez chinês}
  \seeref{bao1}
  \seeref{pao2}
\end{EntryWithPhonetic}

\begin{EntryWithPhonetic}{炮车}{pao4che1}{9,4}{⽕、⾞}
  \definition{s.}{veículo de artilharia; tanque de guerra}
\end{EntryWithPhonetic}

%%%%%%%%%% 胚 %%%%%%%%%%
\subsection*{胚}\addcontentsline{loh}{figure}{胚 \dpy{pei1}}

\begin{EntryWithPhonetic}{胚}{pei1}{9}{⾁}
  \definition{s.}{embrião}
\end{EntryWithPhonetic}

\begin{EntryWithPhonetic}{胚胎}{pei1tai1}{9,9}{⾁、⾁}[HSK 7-9]
  \definition{s.}{embrião}[子宫里的胚胎。===Um embrião no útero.]
\end{EntryWithPhonetic}

%%%%%%%%%% 陪 %%%%%%%%%%
\subsection*{陪}\addcontentsline{loh}{figure}{陪 \dpy{pei2}}

\begin{EntryWithPhonetic}{陪}{pei2}{10}{⾩}[HSK 5]
  \definition{v.}{servir; acompanhar; cuidar; fazer companhia a alguém | auxiliar; ajudar}
\end{EntryWithPhonetic}

\begin{EntryWithPhonetic}{陪伴}{pei2ban4}{10,7}{⾩、⼈}[HSK 7-9]
  \definition{v.}{acompanhar; fazer companhia a alguém}
\end{EntryWithPhonetic}

\begin{EntryWithPhonetic}{陪同}{pei2 tong2}{10,6}{⾩、⼝}[HSK 6]
  \definition{v.}{acompanhar; acompanhar alguém para fazer uma atividade ou trabalhar junto}
\end{EntryWithPhonetic}

\begin{EntryWithPhonetic}{陪葬}{pei2zang4}{10,12}{⾩、⾋}[HSK 7-9]
  \definition{v.}{ser enterrado com ou ao lado de uma pessoa falecida (do cônjuge ou companheiro(a) do falecido, ou de objetos funerários) | Obsoleto: (esposa, concubina ou escrava) ser enterrada viva com os mortos | (estatuetas ou objetos) enterrar com os mortos | (esposa ou concubina após a sua morte) ser enterrada com o marido ou junto ao seu túmulo}
\end{EntryWithPhonetic}

%%%%%%%%%% 培 %%%%%%%%%%
\subsection*{培}\addcontentsline{loh}{figure}{培 \dpy{pei2}}

\begin{EntryWithPhonetic}{培}{pei2}{11}{⼟}
  \definition{v.}{aterrar com terra; aterrar | fomentar; treinar | cultivar; crescer e desenvolver-se propositalmente}
\end{EntryWithPhonetic}

\begin{EntryWithPhonetic}{培训}{pei2xun4}{11,5}{⼟、⾔}[HSK 4]
  \definition{v.}{treinar (trabalhadores técnicos, quadros profissionais, etc.)}
\end{EntryWithPhonetic}

\begin{EntryWithPhonetic}{培训班}{pei2 xun4 ban1}{11,5,10}{⼟、⾔、⽟}[HSK 4]
  \definition{s.}{aula de treinamento; curso de treinamento}
\end{EntryWithPhonetic}

\begin{EntryWithPhonetic}{培养}{pei2yang3}{11,9}{⼟、⼋}[HSK 4]
  \definition{v.}{cultivar (plantas, microorganismos) | promover; treinar ou desenvolver; educar e treinar para um determinado propósito durante um longo período de tempo; fazer crescer | progredir gradualmente; desenvolver ou cultivar gradualmente (hábito, qualidade, caráter, emoção, estilo, interesse, habilidade, etc.)}
\end{EntryWithPhonetic}

\begin{EntryWithPhonetic}{培育}{pei2yu4}{11,8}{⼟、⾁}[HSK 4]
  \definition{v.}{criar; fomentar; educar; procriar; nutrir; cultivar}
\end{EntryWithPhonetic}

%%%%%%%%%% 赔 %%%%%%%%%%
\subsection*{赔}\addcontentsline{loh}{figure}{赔 \dpy{pei2}}

\begin{EntryWithPhonetic}{赔}{pei2}{12}{⾙}[HSK 5]
  \definition{v.}{compensar; pagar por; indenizar | sofrer uma perda; fazer negócios e perder dinheiro | desculpar-se | suportar uma perda}
\end{EntryWithPhonetic}

\begin{EntryWithPhonetic}{赔偿}{pei2chang2}{12,11}{⾙、⼈}[HSK 5]
  \definition{v.}{indenizar; compensar; pagar por; indenizar outras pessoas ou grupos por perdas causadas por suas próprias ações}
\end{EntryWithPhonetic}

\begin{EntryWithPhonetic}{赔钱}{pei2/qian2}{12,10}{⾙、⾦}[HSK 7-9]
  \definition{v.+compl.}{perder dinheiro | compensar; compensar com dinheiro os prejuízos causados a terceiros}
\end{EntryWithPhonetic}

%%%%%%%%%% 佩 %%%%%%%%%%
\subsection*{佩}\addcontentsline{loh}{figure}{佩 \dpy{pei4}}

\begin{EntryWithPhonetic}{佩}{pei4}{8}{⼈}
  \definition{s.}{um ornamento usado como pingente amarrados em cintos nos tempos antigos}
  \definition{v.}{vestir (na cintura, etc.) | (arcaico) admirar | (arcaico) usar, especialmente uma pistola ou espada, na cintura}
\end{EntryWithPhonetic}

\begin{EntryWithPhonetic}{佩服}{pei4fu2}{8,8}{⼈、⽉}[HSK 7-9]
  \definition{v.}{admirar; respeitar; dar os parabéns a alguém; ter uma alta opinião de alguém; considerar respeitáveis ​​e adoráveis}
\end{EntryWithPhonetic}

%%%%%%%%%% 配 %%%%%%%%%%
\subsection*{配}\addcontentsline{loh}{figure}{配 \dpy{pei4}}

\begin{EntryWithPhonetic}{配}{pei4}{10}{⾣}[HSK 3]
  \definition{adj.}{adequado; bem combinado}
  \definition{s.}{cônjuge (geralmente referindo-se a uma esposa)}
  \definition{v.}{unir-se em matrimônio | (animais) acasalar; copular | compor; combinar; mesclar; amalgamar; misturar | distribuir de forma planejada; repartir | encontrar algo para encaixar ou substituir outra coisa; compensar as partes faltantes de acordo com certos padrões | combinar; harmonizar com; estar em harmonia com | exilar; banir; nos tempos antigos, referia-se ao exílio de criminosos}
  \definition{v.aux.}{adequar-se a; merecer; ser qualificado; ser digno de}
\end{EntryWithPhonetic}

\begin{EntryWithPhonetic}{配备}{pei4bei4}{10,8}{⾣、⼡}[HSK 5]
  \definition{s.}{equipamento; material; conjunto completo de utensílios, etc.}
  \definition{v.}{fornecer; alocar; equipar; distribuir conforme necessário | posicionar; dispor (tropas, etc.)}
\end{EntryWithPhonetic}

\begin{EntryWithPhonetic}{配合}{pei4he2}{10,6}{⾣、⼝}[HSK 3]
  \definition{v.}{cooperar; coordenar; todas as partes trabalham juntas para concluir tarefas comuns}
\end{EntryWithPhonetic}

\begin{EntryWithPhonetic}{配件}{pei4jian4}{10,6}{⾣、⼈}[HSK 7-9]
  \definition{s.}{peças; acessórios; complementos; peças ou componentes usados ​​para montar máquinas | peças de substituição; peças ou componentes que são reinstalados após terem sido danificados}
\end{EntryWithPhonetic}

\begin{EntryWithPhonetic}{配偶}{pei4'ou3}{10,11}{⾣、⼈}[HSK 7-9]
  \definition{s.}{cônjuge; refere-se ao marido ou à esposa (frequentemente usado em documentos legais)}
\end{EntryWithPhonetic}

\begin{EntryWithPhonetic}{配送}{pei4song4}{10,9}{⾣、⾡}[HSK 7-9]
  \definition{s.}{1. entrega; distribuição; \emph{delivery}}
\end{EntryWithPhonetic}

\begin{EntryWithPhonetic}{配套}{pei4/tao4}{10,10}{⾣、⼤}[HSK 5]
  \definition{v.+compl.}{formar um conjunto ou sistema completo; combinar vários elementos relacionados em um conjunto completo}
\end{EntryWithPhonetic}

\begin{EntryWithPhonetic}{配音}{pei4/yin1}{10,9}{⾣、⾳}[HSK 7-9]
  \definition{s.}{dublagem (de um filme, etc.); locução}[这部电影有配音版本。===Este filme possui uma versão dublada.]
  \definition{v.+compl.}{dublar; narrar}
\end{EntryWithPhonetic}

\begin{EntryWithPhonetic}{配置}{pei4 zhi4}{10,13}{⾣、⽹}[HSK 6]
  \definition{s.}{configuração; refere-se especificamente à seleção e combinação de software e hardware em várias partes de computadores, carros, etc.}
  \definition{v.}{implantar; alocar; dispor (tropas, etc.); equipar e configurar}
\end{EntryWithPhonetic}

%%%%%%%%%% 喷 %%%%%%%%%%
\subsection*{喷}\addcontentsline{loh}{figure}{喷 \dpy{pen1}}

\begin{EntryWithPhonetic}{喷}{pen1}{12}{⼝}[HSK 5]
  \definition{v.}{jorrar; esguichar; expelir sob pressão | borrifar; espalhar; pulverizar}
  \seeref{pen4}
\end{EntryWithPhonetic}

\begin{EntryWithPhonetic}{喷泉}{pen1quan2}{12,9}{⼝、⽔}[HSK 7-9]
  \definition[个,处,注]{s.}{fonte; fonte que jorra água}
\end{EntryWithPhonetic}

%%%%%%%%%% 盆 %%%%%%%%%%
\subsection*{盆}\addcontentsline{loh}{figure}{盆 \dpy{pen2}}

\begin{EntryWithPhonetic}{盆}{pen2}{9}{⽫}[HSK 5]
  \definition*{s.}{Sobrenome: Pen}
  \definition{s.}{bacia; banheira; panela; utensílios para guardar ou lavar coisas}
\end{EntryWithPhonetic}

\begin{EntryWithPhonetic}{盆友}{pen2you3}{9,4}{⽫、⼜}
  \definition{s.}{Gíria da \emph{Internet}: amigo (trocadilho com 朋友)}
  \seealsoref{朋友}{peng2you5}
\end{EntryWithPhonetic}

%%%%%%%%%% 喷 %%%%%%%%%%
\subsection*{喷}\addcontentsline{loh}{figure}{喷 \dpy{pen4}}

\begin{EntryWithPhonetic}{喷}{pen4}{12}{⼝}
  \definition{s.}{na época; tempo no mercado; época em que frutas, peixes e camarões são comercializados em grande quantidade | colheita; número de vezes que floresceu e frutificou; número de vezes que foi colhido na maturação}
  \seeref{pen1}
\end{EntryWithPhonetic}

%%%%%%%%%% 抨 %%%%%%%%%%
\subsection*{抨}\addcontentsline{loh}{figure}{抨 \dpy{peng1}}

\begin{EntryWithPhonetic}{抨}{peng1}{8}{⼿}
  \definition{s.}{Literário: \emph{impeachment}; censura}
  \definition{v.}{atacar; criticar; destituir; censurar}
\end{EntryWithPhonetic}

\begin{EntryWithPhonetic}{抨击}{peng1ji1}{8,5}{⼿、⼐}[HSK 7-9]
  \definition{v.}{atacar (falando ou escrevendo); bombardear (com palavras); criticar}[他把电视采访作为一个机会,向反对党进行猛烈抨击。===Ele aproveitou a entrevista na televisão para lançar um ataque feroz contra o partido da oposição.]
\end{EntryWithPhonetic}

%%%%%%%%%% 朋 %%%%%%%%%%
\subsection*{朋}\addcontentsline{loh}{figure}{朋 \dpy{peng2}}

\begin{EntryWithPhonetic}{朋}{peng2}{8}{⽉}
  \definition*{s.}{Sobrenome: Peng}
  \definition{s.}{amigo}
  \definition{v.}{(literário) rivalizar; igualar; comparar | (literário) reunir-se em grupo; juntar-se em grupo}
\end{EntryWithPhonetic}

\begin{EntryWithPhonetic}{朋友}{peng2you5}{8,4}{⽉、⼜}[HSK 1]
  \definition[个,位,帮,群]{s.}{amigo; pessoas que têm um bom relacionamento, uma boa relação, se entendem e se ajudam mutuamente | namorado; namorada}
\end{EntryWithPhonetic}

%%%%%%%%%% 膨 %%%%%%%%%%
\subsection*{膨}\addcontentsline{loh}{figure}{膨 \dpy{peng2}}

\begin{EntryWithPhonetic}{膨}{peng2}{16}{⾁}
  \definition{v.}{inchar; inflar | expandir; aumentar o comprimento ou o volume de um objeto}
\end{EntryWithPhonetic}

\begin{EntryWithPhonetic}{膨胀}{peng2zhang4}{16,8}{⾁、⾁}
  \definition{v.}{expandir | inflar | inchar}
\end{EntryWithPhonetic}

%%%%%%%%%% 碰 %%%%%%%%%%
\subsection*{碰}\addcontentsline{loh}{figure}{碰 \dpy{peng4}}

\begin{EntryWithPhonetic}{碰}{peng4}{13}{⽯}[HSK 2]
  \definition{v.}{tocar; bater; esbarrar | encontrar; esbarrar | arriscar; tentar | tentar a sorte | reunir-se para discutir; ter uma reunião curta}
\end{EntryWithPhonetic}

\begin{EntryWithPhonetic}{碰到}{peng4 dao4}{13,8}{⽯、⼑}[HSK 2]
  \definition{v.}{encontrar (com); esbarrar; cruzar}
\end{EntryWithPhonetic}

\begin{EntryWithPhonetic}{碰见}{peng4 jian4}{13,4}{⽯、⾒}[HSK 2]
  \definition{v.}{encontrar; encontrar-se; sem combinar, encontrar-se por acaso}
\end{EntryWithPhonetic}

\begin{EntryWithPhonetic}{碰头}{peng4/tou2}{13,5}{⽯、⼤}
  \definition{s.}{colisão | conflito}
  \definition{v.}{colidir}
  \definition{v.+compl.}{conhecer e discutir | juntar ideias | ver-se}
\end{EntryWithPhonetic}

\begin{EntryWithPhonetic}{碰运气}{peng4yun4qi5}{13,7,4}{⽯、⾡、⽓}
  \definition{v.}{deixar algo ao acaso | tentar a sorte}
\end{EntryWithPhonetic}

%%%%%%%%%% 批 %%%%%%%%%%
\subsection*{批}\addcontentsline{loh}{figure}{批 \dpy{pi1}}

\begin{EntryWithPhonetic}{批}{pi1}{7}{⼿}[HSK 4]
  \definition{adj.}{(compra ou venda) atacado; a granel; em grandes quantidades}
  \definition{clas.}{usado para mercadorias a granel, grande número de pessoas}
  \definition{s.}{fibras de algodão, linho, etc., prontas para serem estiradas e torcidas | anotação; comentário}
  \definition{v.}{escrever comentários ou críticas sobre documentos subordinados, textos de outras pessoas, tarefas etc. | refutar; criticar | dar um tapa}
\end{EntryWithPhonetic}

\begin{EntryWithPhonetic}{批发}{pi1fa1}{7,5}{⼿、⼜}
  \definition{v.}{verder no atacado; vender mercadorias a granel; comprar e vender mercadorias a granel}
\end{EntryWithPhonetic}

\begin{EntryWithPhonetic}{批评}{pi1ping2}{7,7}{⼿、⾔}[HSK 3]
  \definition{v.}{criticar; comentar sobre deficiências e erros | criticar; apontar vantagens e desvantagens; comentar sobre o que é bom e o que é ruim}
\end{EntryWithPhonetic}

\begin{EntryWithPhonetic}{批准}{pi1zhun3}{7,10}{⼿、⼎}[HSK 3]
  \definition{v.}{aprovar}
\end{EntryWithPhonetic}

%%%%%%%%%% 披 %%%%%%%%%%
\subsection*{披}\addcontentsline{loh}{figure}{披 \dpy{pi1}}

\begin{EntryWithPhonetic}{披}{pi1}{8}{⼿}[HSK 5]
  \definition{v.}{colocar sobre os ombros; enrolar em volta; cobrir ou colocar sobre os ombros | abrir; desenrolar; espalhar | abrir-se; rachar}
\end{EntryWithPhonetic}

%%%%%%%%%% 皮 %%%%%%%%%%
\subsection*{皮}\addcontentsline{loh}{figure}{皮 \dpy{pi2}}

\begin{EntryWithPhonetic}{皮}{pi2}{5}{⽪}[HSK 3][Kangxi 107]
  \definition*{s.}{Sobrenome: Pi}
  \definition{adj.}{macios e encharcados; não mais crocantes | malandro; travesso | apático; endurecido; indiferente devido a repetidas repreensões | pegajoso; tenaz; resiliente}
  \definition{pref.}{pico- (um trilhonésimo)}
  \definition[层,块,张,个]{s.}{pele; casca; uma camada de tecido na superfície dos organismos animais e vegetais | pele; couro; couro processado | capa; embalagem; a camada externa que envolve algo | superfície do objeto | folha; peça larga e plana (de algum material fino) | borracha}
\end{EntryWithPhonetic}

\begin{EntryWithPhonetic}{皮包}{pi2 bao1}{5,5}{⽪、⼓}[HSK 3]
  \definition[个,只,款]{s.}{bolsa; pasta; portfólio; bolsas de couro}
\end{EntryWithPhonetic}

\begin{EntryWithPhonetic}{皮肤}{pi2fu1}{5,8}{⽪、⾁}[HSK 5]
  \definition{adj.}{superficial}
  \definition[种,块,片,层]{s.}{pele; couro; derme}
\end{EntryWithPhonetic}

\begin{EntryWithPhonetic}{皮卡}{pi2ka3}{5,5}{⽪、⼘}
  \definition{s.}{(empréstimo linguístico) \emph{pick-up} | caminhonete}
\end{EntryWithPhonetic}

\begin{EntryWithPhonetic}{皮卡丘}{pi2ka3qiu1}{5,5,5}{⽪、⼘、⼀}
  \definition*{s.}{Pikachu (Pokémon, 口袋妖怪)}
  \seealsoref{口袋妖怪}{kou3dai4 yao1guai4}
\end{EntryWithPhonetic}

\begin{EntryWithPhonetic}{皮球}{pi2 qiu2}{5,11}{⽪、⽟}[HSK 6]
  \definition{s.}{bola (feita de borracha, couro etc.)}
\end{EntryWithPhonetic}

\begin{EntryWithPhonetic}{皮下}{pi2xia4}{5,3}{⽪、⼀}
  \definition{adj.}{(injeção) subcutâneo | sob a pele}
\end{EntryWithPhonetic}

\begin{EntryWithPhonetic}{皮鞋}{pi2xie2}{5,15}{⽪、⾰}[HSK 5]
  \definition[双,只,款]{s.}{sapatos feitos de couro}
\end{EntryWithPhonetic}

%%%%%%%%%% 啤 %%%%%%%%%%
\subsection*{啤}\addcontentsline{loh}{figure}{啤 \dpy{pi2}}

\begin{EntryWithPhonetic}{啤}{pi2}{11}{⼝}
  \definition{s.}{cerveja}
\end{EntryWithPhonetic}

\begin{EntryWithPhonetic}{啤酒}{pi2jiu3}{11,10}{⼝、⾣}[HSK 3]
  \definition[杯,瓶,罐,桶,缸]{s.}{(empréstimo linguístico) cerveja; uma bebida de baixo teor alcoólico feita de malte de cevada e lúpulo, com espuma e aroma especial}
\end{EntryWithPhonetic}

\begin{EntryWithPhonetic}{啤酒馆}{pi2jiu3guan3}{11,10,11}{⼝、⾣、⾷}
  \definition{s.}{cervejaria}
\end{EntryWithPhonetic}

%%%%%%%%%% 脾 %%%%%%%%%%
\subsection*{脾}\addcontentsline{loh}{figure}{脾 \dpy{pi2}}

\begin{EntryWithPhonetic}{脾}{pi2}{12}{⾁}
  \definition{s.}{baço}
\end{EntryWithPhonetic}

\begin{EntryWithPhonetic}{脾气}{pi2qi5}{12,4}{⾁、⽓}[HSK 5]
  \definition[股]{s.}{temperamento; disposição; referindo-se ao caráter de uma pessoa | mau humor; temperamento irascível}
\end{EntryWithPhonetic}

%%%%%%%%%% 匹 %%%%%%%%%%
\subsection*{匹}\addcontentsline{loh}{figure}{匹 \dpy{pi3}}

\begin{EntryWithPhonetic}{匹}{pi3}{4}{⼖}[HSK 5]
  \definition{adj.}{solitário}
  \definition{clas.}{usado para cavalos, mulas, etc. | usado para rolos inteiros de seda ou tecido}
  \definition{v.}{ser igual a; ser compatível com}
\end{EntryWithPhonetic}

%%%%%%%%%% 否 %%%%%%%%%%
\subsection*{否}\addcontentsline{loh}{figure}{否 \dpy{pi3}}

\begin{EntryWithPhonetic}{否}{pi3}{7}{⼝}
  \definition{adj.}{ruim; maligno; perverso}
  \definition{v.}{censurar}
  \seeref{fou3}
\end{EntryWithPhonetic}

%%%%%%%%%% 屁 %%%%%%%%%%
\subsection*{屁}\addcontentsline{loh}{figure}{屁 \dpy{pi4}}

\begin{EntryWithPhonetic}{屁}{pi4}{7}{⼫}
  \definition{s.}{vento (ou gás) (dos intestinos); peido | (vulgar) bobagem; merda; lixo | quadril; bunda}
\end{EntryWithPhonetic}

\begin{EntryWithPhonetic}{屁股}{pi4gu5}{7,8}{⼫、⾁}
  \definition{s.}{nádega | quadris}
\end{EntryWithPhonetic}

\begin{EntryWithPhonetic}{屁话}{pi4hua4}{7,8}{⼫、⾔}
  \definition{s.}{absurdo | tolice | besteira}
\end{EntryWithPhonetic}

%%%%%%%%%% 譬 %%%%%%%%%%
\subsection*{譬}\addcontentsline{loh}{figure}{譬 \dpy{pi4}}

\begin{EntryWithPhonetic}{譬}{pi4}{20}{⾔}
  \definition{s.}{exemplo; analogia; metáfora}
  \definition{v.}{dar um exemplo; fazer uma analogia}
\end{EntryWithPhonetic}

\begin{EntryWithPhonetic}{譬如}{pi4ru2}{20,6}{⾔、⼥}
  \definition{conj.}{por exemplo | como}
\end{EntryWithPhonetic}

%%%%%%%%%% 片 %%%%%%%%%%
\subsection*{片}\addcontentsline{loh}{figure}{片 \dpy{pian1}}

\begin{EntryWithPhonetic}{片}{pian1}{4}{⽚}[Kangxi 91]
  \definition{s.}{película; filme; refere-se a filmes com imagens, paisagens ou imagens gravadas com som}
  \seeref{pian4}
\end{EntryWithPhonetic}

\begin{EntryWithPhonetic}{片儿}{pian1r5}{4,2}{⽚、⼉}
  \definition{s.}{folha | película; filme}
\end{EntryWithPhonetic}

%%%%%%%%%% 扁 %%%%%%%%%%
\subsection*{扁}\addcontentsline{loh}{figure}{扁 \dpy{pian1}}

\begin{EntryWithPhonetic}{扁}{pian1}{9}{⼾}
  \definition{adj.}{pequeno | fora do caminho; remoto}
  \seeref{bian3}
\end{EntryWithPhonetic}

\begin{EntryWithPhonetic}{扁舟}{pian1 zhou1}{9,6}{⼾、⾈}
  \definition[叶,艘]{s.}{pequeno barco; esquife}
\end{EntryWithPhonetic}

%%%%%%%%%% 偏 %%%%%%%%%%
\subsection*{偏}\addcontentsline{loh}{figure}{偏 \dpy{pian1}}

\begin{EntryWithPhonetic}{偏}{pian1}{11}{⼈}[HSK 6]
  \definition{adj.}{parcial; preconceituoso; injusto; focando apenas em um lado | torto; inclinado (oposto de 正) | não dominante; auxiliar | remoto; periférico; longe do centro; incomum}
  \definition{adv.}{intencionalmente; insistentemente; persistentemente; indica ir intencionalmente contra o senso comum ou a solicitação de outra pessoa}
  \definition{expr.}{uma expressão educada para indicar que alguém já tomou chá ou comeu}
  \definition{v.}{divergir; não ser igual a; ser diferente de; exceder ou ficar aquém dos padrões normais | desviar-se; afastar-se; sair na direção certa}
  \seealsoref{正}{zheng4}
\end{EntryWithPhonetic}

\begin{EntryWithPhonetic}{偏偏}{pian1pian1}{11,11}{⼈、⼈}
  \definition{adv.}{voluntariamente | insistentemente | persistentemente | ao contrário da expectativa | infelizmente (indicando que alguma coisa aconteceu ao contrário do que se esperava) | teimosamente (indicando que algo é o oposto ao que seria normal ou razoável) | precisamente (indicando que alguém ou um grupo é escolhido)}
\end{EntryWithPhonetic}

%%%%%%%%%% 篇 %%%%%%%%%%
\subsection*{篇}\addcontentsline{loh}{figure}{篇 \dpy{pian1}}

\begin{EntryWithPhonetic}{篇}{pian1}{15}{⽵}[HSK 2]
  \definition*{s.}{Sobrenome: Pian}
  \definition{clas.}{usado para folhas de papel, páginas de livros, artigos, etc.}
  \definition{s.}{um pedaço de escrita | folha (de papel, etc.) | (para papel, folhas de livros, artigos, etc.) folha; página; pedaço}
\end{EntryWithPhonetic}

%%%%%%%%%% 便 %%%%%%%%%%
\subsection*{便}\addcontentsline{loh}{figure}{便 \dpy{pian2}}

\begin{EntryWithPhonetic}{便}{pian2}{9}{⼈}
  \definition*{s.}{Sobrenome: Pian}
  \definition{adj.}{silencioso e confortável}
  \seeref{bian4}
\end{EntryWithPhonetic}

\begin{EntryWithPhonetic}{便宜}{pian2yi5}{9,8}{⼈、⼧}[HSK 2]
  \definition{adj.}{barato; acessível}
  \definition[个,份,件]{s.}{vantagem em algum aspecto | ganho; lucro; vantagem; benefício indevido}
  \definition{v.}{deixar alguém escapar impune; obter algum benefício}
  \seeref{bian4yi2}
\end{EntryWithPhonetic}

%%%%%%%%%% 片 %%%%%%%%%%
\subsection*{片}\addcontentsline{loh}{figure}{片 \dpy{pian4}}

\begin{EntryWithPhonetic}{片}{pian4}{4}{⽚}[HSK 2][Kangxi 91]
  \definition*{s.}{Sobrenome: Pian}
  \definition{adj.}{breve; parcial; incompleto; fragmentário; esporádico; breve | unilateral}
  \definition{clas.}{usado para coisas em forma de lâminas | usado para terrenos ou superfícies aquáticas com a mesma paisagem e que estão conectados entre si | usado para paisagens, clima, sons, linguagem, intenções, etc. (usado em conjunto com o numeral 一)}
  \definition{s.}{plano, fatia; floco; pedaço fino; algo plano e fino | seção; parte de uma grande área; uma pequena parte do todo ou uma área menor dividida dentro de uma área maior | filme; peça de TV; referência ao filme}
  \definition{v.}{fatiar; cortar em fatias; cortar em fatias finas com uma faca | abrir; cortar; separar}
  \seeref{pian1}
  \seealsoref{一}{yi1}
\end{EntryWithPhonetic}

\begin{EntryWithPhonetic}{片面}{pian4mian4}{4,9}{⽚、⾯}[HSK 4]
  \definition{adj.}{unilateral; tendencioso para um lado (em oposição a 全面)}
  \seealsoref{全面}{quan2mian4}
\end{EntryWithPhonetic}

%%%%%%%%%% 骗 %%%%%%%%%%
\subsection*{骗}\addcontentsline{loh}{figure}{骗 \dpy{pian4}}

\begin{EntryWithPhonetic}{骗}{pian4}{12}{⾺}[HSK 5]
  \definition{v.}{enganar; trapacear; iludir; ludibriar; usar mentiras ou meios fraudulentos para fazer alguém acreditar ou ser enganado | enganar; fraudar | montar (um cavalo); balançar (ou saltar) para a sela}
\end{EntryWithPhonetic}

\begin{EntryWithPhonetic}{骗子}{pian4 zi5}{12,3}{⾺、⼦}[HSK 5]
  \definition[个]{s.}{trapaceiro; vigarista; fraudador; impostor; golpista; pessoa que obtém bens de forma fraudulenta}
\end{EntryWithPhonetic}

%%%%%%%%%% 漂 %%%%%%%%%%
\subsection*{漂}\addcontentsline{loh}{figure}{漂 \dpy{piao1}}

\begin{EntryWithPhonetic}{漂}{piao1}{14}{⽔}
  \definition{v.}{flutuar | estar a deriva}
  \seeref{piao3}
  \seeref{piao4}
\end{EntryWithPhonetic}

\begin{EntryWithPhonetic}{漂流}{piao1liu2}{14,10}{⽔、⽔}
  \definition{s.}{\emph{rafting}}
  \definition{v.}{ser levado pela correnteza | flutuar ao longo ou sobre}
\end{EntryWithPhonetic}

%%%%%%%%%% 飘 %%%%%%%%%%
\subsection*{飘}\addcontentsline{loh}{figure}{飘 \dpy{piao1}}

\begin{EntryWithPhonetic}{飘}{piao1}{15}{⾵}
  \definition{adj.}{complacente | frívolo | fraco | instável | bambo | cambaleante}
  \definition{v.}{flutuar (no ar) | esvoaçar | tremular}
\end{EntryWithPhonetic}

%%%%%%%%%% 漂 %%%%%%%%%%
\subsection*{漂}\addcontentsline{loh}{figure}{漂 \dpy{piao3}}

\begin{EntryWithPhonetic}{漂}{piao3}{14}{⽔}
  \definition{v.}{alvejar | branquear}
  \seeref{piao1}
  \seeref{piao4}
\end{EntryWithPhonetic}

%%%%%%%%%% 票 %%%%%%%%%%
\subsection*{票}\addcontentsline{loh}{figure}{票 \dpy{piao4}}

\begin{EntryWithPhonetic}{票}{piao4}{11}{⽰}[HSK 1]
  \definition{clas.}{para grupos, lotes, transações comerciais}
  \definition[张]{s.}{bilhete; passagem; ingresso | cédula | nota bancária; conta | pessoa mantida em cativeiro por sequestradores para obter resgate; refém | apresentação amadora (de ópera de Pequim, etc.); peças teatrais amadoras}
  \definition{v.}{atuar como amador (na ópera de Pequim)}
\end{EntryWithPhonetic}

\begin{EntryWithPhonetic}{票价}{piao4 jia4}{11,6}{⽰、⼈}[HSK 3]
  \definition[个]{s.}{o preço de um ingresso; taxa de admissão; taxa de entrada}
\end{EntryWithPhonetic}

%%%%%%%%%% 漂 %%%%%%%%%%
\subsection*{漂}\addcontentsline{loh}{figure}{漂 \dpy{piao4}}

\begin{EntryWithPhonetic}{漂}{piao4}{14}{⽔}
  \definition{adj.}{bonita; usado em 漂亮}
  \definition{v.}{falhar; terminar em fracasso}[这笔投资的钱全都漂了。===Todo o dinheiro desse investimento foi perdido.]
  \seeref{piao1}
  \seeref{piao3}
  \seealsoref{漂亮}{piao4liang5}
\end{EntryWithPhonetic}

\begin{EntryWithPhonetic}{漂亮}{piao4liang5}{14,9}{⽔、⼇}[HSK 2]
  \definition{adj.}{bonito; lindo; atraente; de boa aparência; esteticamente agradável | excelente; notável | não pode ser utilizado para descrever homens}
\end{EntryWithPhonetic}

%%%%%%%%%% 拼 %%%%%%%%%%
\subsection*{拼}\addcontentsline{loh}{figure}{拼 \dpy{pin1}}

\begin{EntryWithPhonetic}{拼}{pin1}{9}{⼿}[HSK 5]
  \definition{v.}{montar; juntar as peças | dar tudo de si no trabalho; estar disposto a arriscar a vida (em lutas, no trabalho, etc.); fazer tudo o que for preciso; arriscar tudo}
\end{EntryWithPhonetic}

\begin{EntryWithPhonetic}{拼命}{pin1/ming4}{9,8}{⼿、⼝}
  \definition{adv.}{com toda a força | desesperadamente}
  \definition{v.+compl.}{arriscar a vida de alguém | desafiar a morte | colocar-se em uma luta desesperada | fazer algo desesperadamente | exercer a maior força}
\end{EntryWithPhonetic}

\begin{EntryWithPhonetic}{拼音}{pin1yin1}{9,9}{⼿、⾳}
  \definition{s.}{escrita fonética | pinyin (romanização chinesa)}
\end{EntryWithPhonetic}

%%%%%%%%%% 贫 %%%%%%%%%%
\subsection*{贫}\addcontentsline{loh}{figure}{贫 \dpy{pin2}}

\begin{EntryWithPhonetic}{贫}{pin2}{8}{⾙}
  \definition{adj.}{pobre; empobrecido | inadequado; deficiente; insuficiente | tagarela; loquaz; falante; chato e irritante}
\end{EntryWithPhonetic}

\begin{EntryWithPhonetic}{贫困}{pin2kun4}{8,7}{⾙、⼞}[HSK 6]
  \definition{adj.}{pobre; indigente; necessitado; empobrecido; assolado pela pobreza; em circunstâncias difíceis}
\end{EntryWithPhonetic}

\begin{EntryWithPhonetic}{贫民窟}{pin2min2ku1}{8,5,13}{⾙、⽒、⽳}
  \definition{s.}{favela}
\end{EntryWithPhonetic}

%%%%%%%%%% 频 %%%%%%%%%%
\subsection*{频}\addcontentsline{loh}{figure}{频 \dpy{pin2}}

\begin{EntryWithPhonetic}{频}{pin2}{13}{⾴}
  \definition*{s.}{Sobrenome: Pin}
  \definition{adj.}{frequente}
  \definition{adv.}{frequentemente; repetidamente}
  \definition{s.}{Física: frequência; o número de vezes que um objeto vibra por segundo}
\end{EntryWithPhonetic}

\begin{EntryWithPhonetic}{频道}{pin2dao4}{13,12}{⾴、⾡}[HSK 5]
  \definition[个]{s.}{canal; canal de frequência; televisão e rádio, os sinais de som e imagem ocupam um determinado canal de frequência}
\end{EntryWithPhonetic}

\begin{EntryWithPhonetic}{频繁}{pin2fan2}{13,17}{⾴、⽷}[HSK 5]
  \definition{adj.}{frequentemente}
  \definition{adj.}{frequente}
\end{EntryWithPhonetic}

%%%%%%%%%% 品 %%%%%%%%%%
\subsection*{品}\addcontentsline{loh}{figure}{品 \dpy{pin3}}

\begin{EntryWithPhonetic}{品}{pin3}{9}{⼝}[HSK 5]
  \definition*{s.}{Sobrenome: Pin}
  \definition{s.}{artigo; produto | grau; classe; classificação; nível | caráter; qualidade | classificação; os graus dos funcionários públicos antigos, num total de nove graus}
  \definition{v.}{provar; saborear; degustar algo com discernimento | soprar; tocar (instrumentos de sopro) | avaliar; distinguir o bom do ruim}
\end{EntryWithPhonetic}

\begin{EntryWithPhonetic}{品德}{pin3de2}{9,15}{⼝、⼻}
  \definition{s.}{caráter moral | moralidade}
\end{EntryWithPhonetic}

\begin{EntryWithPhonetic}{品牌}{pin3 pai2}{9,12}{⼝、⽚}[HSK 6]
  \definition[个,种]{s.}{marca registrada; nome de marca}
\end{EntryWithPhonetic}

\begin{EntryWithPhonetic}{品质}{pin3zhi4}{9,8}{⼝、⾙}[HSK 4]
  \definition[个,种]{s.}{qualidade; caráter; natureza do pensamento, da compreensão, do caráter, etc., conforme expresso no comportamento, no estilo, etc. | qualidade (de produtos, mercadorias, etc.)}
\end{EntryWithPhonetic}

\begin{EntryWithPhonetic}{品种}{pin3zhong3}{9,9}{⼝、⽲}[HSK 5]
  \definition[个,些]{s.}{raça; linhagem; variedade; refere-se a um grupo de organismos com características genéticas comuns, formados por meio da seleção e cultivo artificiais de culturas, gado, aves, etc. | variedade; sortimento; referência geral ao tipo de item}
\end{EntryWithPhonetic}

%%%%%%%%%% 牝 %%%%%%%%%%
\subsection*{牝}\addcontentsline{loh}{figure}{牝 \dpy{pin4}}

\begin{EntryWithPhonetic}{牝}{pin4}{6}{⽜}
  \definition{adj.}{(de certas aves e animais) fêmea (oposto de 牡)}
  \definition{s.}{fêmea (de algumas aves e animais)}
  \seealsoref{牡}{mu3}
\end{EntryWithPhonetic}

%%%%%%%%%% 聘 %%%%%%%%%%
\subsection*{聘}\addcontentsline{loh}{figure}{聘 \dpy{pin4}}

\begin{EntryWithPhonetic}{聘}{pin4}{13}{⽿}
  \definition{v.}{contratar | noivar | (de uma menina) casar ou ser casada}
\end{EntryWithPhonetic}

\begin{EntryWithPhonetic}{聘请}{pin4 qing3}{13,10}{⽿、⾔}[HSK 6]
  \definition{v.}{convidar; empregar; envolver; chamar; contratar alguém para assumir uma posição}
\end{EntryWithPhonetic}

%%%%%%%%%% 乒 %%%%%%%%%%
\subsection*{乒}\addcontentsline{loh}{figure}{乒 \dpy{ping1}}

\begin{EntryWithPhonetic}{乒}{ping1}{6}{⼃}
  \definition{interj.}{(onomatopéia) estalo; estouro; estrondo | (onomatopéia)  ``ping''}
  \definition{s.}{(abreviação) tênis de mesa; pingue-pongue | (abreviação) bola de tênis de mesa; bola de pingue-pongue}
\end{EntryWithPhonetic}

\begin{EntryWithPhonetic}{乒乓球}{ping1pang1qiu2}{6,6,11}{⼃、⼃、⽟}
  \definition[个]{s.}{tênis de mesa |ping-pong}
\end{EntryWithPhonetic}

%%%%%%%%%% 平 %%%%%%%%%%
\subsection*{平}\addcontentsline{loh}{figure}{平 \dpy{ping2}}

\begin{EntryWithPhonetic}{平}{ping2}{5}{⼲}[HSK 2]
  \definition*{s.}{Sobrenome: Ping}
  \definition{adj.}{plano; nivelado; uniforme; liso | igual; justo | mesma pontuação; empatado | médio; comum | silencioso; tranquilo | no mesmo nível; altura igual; sem diferença | imparcial; médio; equitativo | calmo; estável; tranquilo | comum;  frequente}
  \definition{s.}{no mesmo nível; em pé de igualdade com; igual | tom nivelado, um dos quatro tons do chinês clássico}
  \definition{v.}{tornar nivelado ou uniforme; nivelar | reprimir; suprimir | acalmar; tornar pacífico; silenciar (acalmar); conter a raiva | estar no mesmo nível | acalmar; amenizar; controlar a raiva}
\end{EntryWithPhonetic}

\begin{EntryWithPhonetic}{平安}{ping2'an1}{5,6}{⼲、⼧}[HSK 2]
  \definition{s.}{seguro; bem; sem contratempos; sem acidentes; são e salvo}
\end{EntryWithPhonetic}

\begin{EntryWithPhonetic}{平常}{ping2chang2}{5,11}{⼲、⼱}[HSK 2]
  \definition{adj.}{comum; normal; ordinário; nada de especial}
  \definition{adv.}{normalmente; geralmente; como regra geral}
\end{EntryWithPhonetic}

\begin{EntryWithPhonetic}{平等}{ping2deng3}{5,12}{⼲、⽵}[HSK 2]
  \definition{adj.}{igual; igualdade; refere-se ao fato de as pessoas gozarem de tratamento igualitário nos aspectos sociais, políticos, econômicos e jurídicos}
\end{EntryWithPhonetic}

\begin{EntryWithPhonetic}{平地}{ping2di4}{5,6}{⼲、⼟}
  \definition{v.}{nivelar a terra | aplanar}
\end{EntryWithPhonetic}

\begin{EntryWithPhonetic}{平凡}{ping2fan2}{5,3}{⼲、⼏}[HSK 6]
  \definition{adj.}{comum; ordinário; normal; não surpreendente}
\end{EntryWithPhonetic}

\begin{EntryWithPhonetic}{平方}{ping2fang1}{5,4}{⼲、⽅}[HSK 4]
  \definition{s.}{Matemática: segunda potência (de uma quantidade); quadrado | metro quadrado (m²)}[那间房有十二平方。===O quarto tem doze metros quadrados.]
\end{EntryWithPhonetic}

\begin{EntryWithPhonetic}{平方米}{ping2 fang1 mi3}{5,4,6}{⼲、⽅、⽶}[HSK 6]
  \definition{s.}{metro quadrado; a unidade legal de medida de área, 1 metro quadrado é igual a 10.000 centímetros quadrados}
\end{EntryWithPhonetic}

\begin{EntryWithPhonetic}{平方市丈}{ping2fang1 shi4 zhang4}{5,4,5,3}{⼲、⽅、⼱、⼀}
  \definition{clas.}{pés quadrados}
\end{EntryWithPhonetic}

\begin{EntryWithPhonetic}{平衡}{ping2 heng2}{5,16}{⼲、⾏}[HSK 6]
  \definition{adj.}{balanceado; equilibrado; os aspectos opostos são iguais ou compensados ​​em quantidade ou qualidade | equilibrado; várias forças atuam sobre um objeto com magnitude igual e direções opostas para manter o objeto estável}
  \definition{v.}{equilibrar; trazer ou manter em equilíbrio; tornar as coisas ou alimentos iguais em quantidade, qualidade ou força}
\end{EntryWithPhonetic}

\begin{EntryWithPhonetic}{平静}{ping2jing4}{5,14}{⼲、⾭}[HSK 4]
  \definition{adj.}{(humor, ambiente, etc.) calmo; quieto; pacífico; tranquilo}
\end{EntryWithPhonetic}

\begin{EntryWithPhonetic}{平均}{ping2jun1}{5,7}{⼲、⼟}[HSK 4]
  \definition{adj.}{igual; médio}
  \definition{s.}{média}
  \definition{v.}{calcular a média de um conjunto de números}
\end{EntryWithPhonetic}

\begin{EntryWithPhonetic}{平时}{ping2shi2}{5,7}{⼲、⽇}[HSK 2]
  \definition{s.}{em tempos normais; em tempos comuns | em tempo de paz; refere-se a períodos normais}
\end{EntryWithPhonetic}

\begin{EntryWithPhonetic}{平台}{ping2 tai2}{5,5}{⼲、⼝}[HSK 6]
  \definition[个]{s.}{casa com telhado plano rebocado | terraço | plataforma móvel; metaforicamente, refere-se às áreas, oportunidades, ambientes, espaços, etc. que fornecem suporte e garantia para algo | plataforma; um sistema em um computador eletrônico que consiste em software e hardware básicos; tal sistema pode suportar a execução de programas aplicativos e softwares aplicativos podem ser desenvolvidos nesse sistema | plataforma; lugar; falando metaforicamente, o mesmo nível ou grau}
\end{EntryWithPhonetic}

\begin{EntryWithPhonetic}{平坦}{ping2tan3}{5,8}{⼲、⼟}[HSK 5]
  \definition{adj.}{plano; uniforme; nivelado; liso; sem elevações ou depressões (referindo-se principalmente ao relevo)}
\end{EntryWithPhonetic}

\begin{EntryWithPhonetic}{平稳}{ping2 wen3}{5,14}{⼲、⽲}[HSK 4]
  \definition{adj.}{firme; estável; suave e constante; sem oscilações ou flutuações}
\end{EntryWithPhonetic}

\begin{EntryWithPhonetic}{平原}{ping2yuan2}{5,10}{⼲、⼚}[HSK 5]
  \definition[片,个]{s.}{campo; planície; terreno plano e extenso}
\end{EntryWithPhonetic}

%%%%%%%%%% 评 %%%%%%%%%%
\subsection*{评}\addcontentsline{loh}{figure}{评 \dpy{ping2}}

\begin{EntryWithPhonetic}{评}{ping2}{7}{⾔}[HSK 6]
  \definition*{s.}{Sobrenome: Ping}
  \definition{v.}{comentar; criticar; revisar | julgar; avaliar}
\end{EntryWithPhonetic}

\begin{EntryWithPhonetic}{评估}{ping2gu1}{7,7}{⾔、⼈}[HSK 5]
  \definition{v.}{estimar; avaliar; apreciar; avaliar e estimar (coisas abstratas)}
\end{EntryWithPhonetic}

\begin{EntryWithPhonetic}{评价}{ping2jia4}{7,6}{⾔、⼈}[HSK 3]
  \definition[个,项,条,份]{s.}{avaliação; apreciação; comentários ou opiniões de pessoas sobre alguém ou algo}
  \definition{v.}{estimar valor; avaliar valor}
\end{EntryWithPhonetic}

\begin{EntryWithPhonetic}{评论}{ping2lun4}{7,6}{⾔、⾔}[HSK 5]
  \definition[篇,些]{s.}{revisão; comentário; artigos ou comentários críticos}
  \definition{v.}{discutir; comentar sobre algo ou alguém}
\end{EntryWithPhonetic}

\begin{EntryWithPhonetic}{评选}{ping2 xuan3}{7,9}{⾔、⾡}[HSK 6]
  \definition{v.}{escolher por meio de avaliação pública; avaliar e eleger}
\end{EntryWithPhonetic}

%%%%%%%%%% 凭 %%%%%%%%%%
\subsection*{凭}\addcontentsline{loh}{figure}{凭 \dpy{ping2}}

\begin{EntryWithPhonetic}{凭}{ping2}{8}{⼏}[HSK 5]
  \definition{conj.}{não importa (o que, como, etc.); conecta frases complexas condicionais para expressar incondicionalidade, equivalente a 任凭 ou 不论}
  \definition{prep.}{introduzir a ação ou o comportamento com base em algo; quando a frase nominal após 凭 é longa, pode-se adicionar 着 após 凭}
  \definition[张]{s.}{prova; evidência}
  \definition{v.}{apoiar-se; encostar-se | confiar em; depender de | basear-se em; tomar como base}
  \seealsoref{不论}{bu2 lun4}
  \seealsoref{任凭}{ren4 ping2}
  \seealsoref{着}{zhe5}
\end{EntryWithPhonetic}

%%%%%%%%%% 苹 %%%%%%%%%%
\subsection*{苹}\addcontentsline{loh}{figure}{苹 \dpy{ping2}}

\begin{EntryWithPhonetic}{苹}{ping2}{8}{⾋}
  \definition[个]{s.}{uma espécie de artemísia | maçã | lentilha-d'água}
\end{EntryWithPhonetic}

\begin{EntryWithPhonetic}{苹果}{ping2guo3}{8,8}{⾋、⽊}[HSK 3]
  \definition[个,斤,筐,箱,棵,种]{s.}{maçã}
\end{EntryWithPhonetic}

%%%%%%%%%% 屏 %%%%%%%%%%
\subsection*{屏}\addcontentsline{loh}{figure}{屏 \dpy{ping2}}

\begin{EntryWithPhonetic}{屏}{ping2}{9}{⼫}
  \definition{s.}{tela | um conjunto de pergaminhos; tiras de tela}
  \definition{v.}{proteger alguém ou algo; resguardar}
  \seeref{bing1}
  \seeref{bing3}
\end{EntryWithPhonetic}

\begin{EntryWithPhonetic}{屏幕}{ping2 mu4}{9,13}{⼫、⼱}[HSK 6]
  \definition[个,块]{s.}{tela; a parte dos computadores, televisores, celulares, etc. que exibe texto, imagens, etc.}
\end{EntryWithPhonetic}

%%%%%%%%%% 瓶 %%%%%%%%%%
\subsection*{瓶}\addcontentsline{loh}{figure}{瓶 \dpy{ping2}}

\begin{EntryWithPhonetic}{瓶}{ping2}{10}{⽡}[HSK 2]
  \definition*{s.}{Sobrenome: Ping}
  \definition{clas.}{usado para coisas que são engarrafadas; quantidade contida em um frasco, vaso, garrafa}
  \definition[个]{s.}{jarra; vaso; frasco; garrafa}
\end{EntryWithPhonetic}

\begin{EntryWithPhonetic}{瓶盖}{ping2gai4}{10,11}{⽡、⽫}
  \definition{s.}{tampa de garrafa}
\end{EntryWithPhonetic}

\begin{EntryWithPhonetic}{瓶装}{ping2zhuang1}{10,12}{⽡、⾐}
  \definition{adj.}{engarrafado}
\end{EntryWithPhonetic}

\begin{EntryWithPhonetic}{瓶子}{ping2zi5}{10,3}{⽡、⼦}[HSK 2]
  \definition[个,只,种]{s.}{garrafa; recipientes com gargalo feitos de cerâmica, vidro, plástico, etc., geralmente em forma cilíndrica}
\end{EntryWithPhonetic}

%%%%%%%%%% 甁 %%%%%%%%%%
\subsection*{甁}\addcontentsline{loh}{figure}{甁 \dpy{ping2}}

\begin{EntryWithPhonetic}{甁}{ping2}{12}{⽡}
  \variantof{瓶}
\end{EntryWithPhonetic}

%%%%%%%%%% 坡 %%%%%%%%%%
\subsection*{坡}\addcontentsline{loh}{figure}{坡 \dpy{po1}}

\begin{EntryWithPhonetic}{坡}{po1}{8}{⼟}[HSK 6]
  \definition{adj.}{inclinado}
  \definition{s.}{declive | encosta}
\end{EntryWithPhonetic}

%%%%%%%%%% 泼 %%%%%%%%%%
\subsection*{泼}\addcontentsline{loh}{figure}{泼 \dpy{po1}}

\begin{EntryWithPhonetic}{泼}{po1}{8}{⽔}[HSK 5]
  \definition{adj.}{rude e irracional; mal-humorado | Dialeto: ousado e vigoroso; ousado e resoluto}
  \definition{v.}{espalhar; salpicar; derramar; derramar ou espalhar o líquido com força para fora}
\end{EntryWithPhonetic}

%%%%%%%%%% 颇 %%%%%%%%%%
\subsection*{颇}\addcontentsline{loh}{figure}{颇 \dpy{po1}}

\begin{EntryWithPhonetic}{颇}{po1}{11}{⽪}
  \definition*{s.}{Sobrenome: Po}
  \definition{adj.}{oblíquo; inclinado para um lado | Literário: tendencioso; incorreto}
  \definition{adv.}{muito; bastante; consideravelmente}
\end{EntryWithPhonetic}

%%%%%%%%%% 迫 %%%%%%%%%%
\subsection*{迫}\addcontentsline{loh}{figure}{迫 \dpy{po4}}

\begin{EntryWithPhonetic}{迫}{po4}{8}{⾡}
  \definition{adj.}{urgente; premente}
  \definition{s.}{morteiro; artilharia}
  \definition{v.}{compelir; forçar; pressionar | aproximar-se; ir em direção a (ou perto de)}
\end{EntryWithPhonetic}

\begin{EntryWithPhonetic}{迫切}{po4qie4}{8,4}{⾡、⼑}[HSK 4]
  \definition{adj.}{urgente; premente; muito ansiosamente, a ponto de ser difícil esperar}
\end{EntryWithPhonetic}

%%%%%%%%%% 破 %%%%%%%%%%
\subsection*{破}\addcontentsline{loh}{figure}{破 \dpy{po4}}

\begin{EntryWithPhonetic}{破}{po4}{10}{⽯}[HSK 3]
  \definition{adj.}{quebrado; danificado; rasgado; desgastado | insignificante; péssimo; medíocre}
  \definition{v.}{quebrar; danificar | dividir; cortar; separar | trocar (dinheiro) | livrar-se de; destruir; romper com | derrotar; capturar (uma cidade, etc.) | gastar dinheiro | revelar a verdade sobre; expor | mudar; romper; quebrar (regras, hábitos, ideias, etc.)}
\end{EntryWithPhonetic}

\begin{EntryWithPhonetic}{破产}{po4/chan3}{10,6}{⽯、⼇}[HSK 4]
  \definition{v.+compl.}{falir; ir à falência; tornar-se insolvente; entrar em liquidação; perder todo o patrimônio | falhar; fracassar; não dar em nada; figura de linguagem (geralmente com uma conotação depreciativa)}
\end{EntryWithPhonetic}

\begin{EntryWithPhonetic}{破坏}{po4huai4}{10,7}{⽯、⼟}[HSK 3]
  \definition{v.}{demolir; naufragar; soçobrar; destruir; obliterar | quebrar; violar (um acordo, regulamento, etc.); não cumprir (disposições legais, regras, acordos, princípios, etc.) | prejudicar; perturbar; sabotar; causar grande dano; causar danos às coisas | reverter; mudar (um sistema social, costume, etc.) completamente ou violentamente | destruir; decompor; danificar o tecido ou a estrutura de um objeto}
\end{EntryWithPhonetic}

\begin{EntryWithPhonetic}{破坏性}{po4huai4xing4}{10,7,8}{⽯、⼟、⼼}
  \definition{adj.}{destrutivo}
  \definition{s.}{poder destrutivo}
\end{EntryWithPhonetic}

%%%%%%%%%% 扑 %%%%%%%%%%
\subsection*{扑}\addcontentsline{loh}{figure}{扑 \dpy{pu1}}

\begin{EntryWithPhonetic}{扑}{pu1}{5}{⼿}[HSK 6]
  \definition{s.}{sopro; refere-se a gases, fragrâncias, cinzas, areia, etc. que se apresentam | espanador}
  \definition{v.}{atacar; lançar-se sobre; correr para frente com toda a sua força e, de repente, jogar todo o seu corpo em um objeto | dedicar; dedicar todas as energias a uma causa; colocar toda a sua energia em (trabalho, carreira, etc.) | bater asas; esvoaçar | inclinar-se}
\end{EntryWithPhonetic}

\begin{EntryWithPhonetic}{扑克}{pu1ke4}{5,7}{⼿、⼗}
  \definition{s.}{(empréstimo linguístico) (jogo) \emph{poker}  | baralho}
\end{EntryWithPhonetic}

%%%%%%%%%% 铺 %%%%%%%%%%
\subsection*{铺}\addcontentsline{loh}{figure}{铺 \dpy{pu1}}

\begin{EntryWithPhonetic}{铺}{pu1}{12}{⾦}[HSK 6]
  \definition{clas.}{usado para kang, etc.; kang, uma plataforma de alvenaria ou de barro em uma extremidade de um cômodo, aquecida no inverno por fogueiras embaixo e coberta com esteiras para dormir}
  \definition{v.}{espalhar; estender; desdobrar | colocar; pavimentar}
  \seeref{pu4}
\end{EntryWithPhonetic}

\begin{EntryWithPhonetic}{铺垫}{pu1dian4}{12,9}{⾦、⼟}
  \definition{s.}{cobre leito | colcha | roupa de cama}
  \definition{v.}{pavimentar}
\end{EntryWithPhonetic}

%%%%%%%%%% 葡 %%%%%%%%%%
\subsection*{葡}\addcontentsline{loh}{figure}{葡 \dpy{pu2}}

\begin{EntryWithPhonetic}{葡}{pu2}{12}{⾋}
  \definition*{s.}{Portugal, abreviação de 葡萄牙}
  \seealsoref{葡萄牙}{pu2tao2ya2}
\end{EntryWithPhonetic}

\begin{EntryWithPhonetic}{葡汉词典}{pu2-han4 ci2dian3}{12,5,7,8}{⾋、⽔、⾔、⼋}
  \definition{s.}{dicionário português-chinês}
  \seealsoref{汉葡词典}{han4-pu2 ci2dian3}
\end{EntryWithPhonetic}

\begin{EntryWithPhonetic}{葡萄酒}{pu2 tao2 jiu3}{12,11,10}{⾋、⾋、⾣}[HSK 5]
  \definition[瓶,杯,口,桶]{s.}{vinho (de uva)}
\end{EntryWithPhonetic}

\begin{EntryWithPhonetic}{葡萄牙}{pu2tao2ya2}{12,11,4}{⾋、⾋、⽛}
  \definition{s.}{Portugal}
\end{EntryWithPhonetic}

\begin{EntryWithPhonetic}{葡萄牙文}{pu2tao2ya2wen2}{12,11,4,4}{⾋、⾋、⽛、⽂}
  \definition{s.}{português, língua portuguesa}
  \seealsoref{葡文}{pu2wen2}
\end{EntryWithPhonetic}

\begin{EntryWithPhonetic}{葡萄牙语}{pu2tao2ya2yu3}{12,11,4,9}{⾋、⾋、⽛、⾔}
  \definition{s.}{português, língua portuguesa}
  \seealsoref{葡语}{pu2yu3}
\end{EntryWithPhonetic}

\begin{EntryWithPhonetic}{葡萄}{pu2tao5}{12,11}{⾋、⾋}[HSK 5]
  \definition[串,颗,粒,棵,种]{s.}{parreira | uva}
\end{EntryWithPhonetic}

\begin{EntryWithPhonetic}{葡文}{pu2wen2}{12,4}{⾋、⽂}
  \definition{s.}{português, língua portuguesa}
  \seealsoref{葡萄牙文}{pu2tao2ya2wen2}
\end{EntryWithPhonetic}

\begin{EntryWithPhonetic}{葡语}{pu2yu3}{12,9}{⾋、⾔}
  \definition{s.}{português, língua portuguesa}
  \seealsoref{葡萄牙语}{pu2tao2ya2yu3}
\end{EntryWithPhonetic}

%%%%%%%%%% 普 %%%%%%%%%%
\subsection*{普}\addcontentsline{loh}{figure}{普 \dpy{pu3}}

\begin{EntryWithPhonetic}{普}{pu3}{12}{⽇}
  \definition*{s.}{Sobrenome: Pu}
  \definition{adj.}{geral; universal}
\end{EntryWithPhonetic}

\begin{EntryWithPhonetic}{普遍}{pu3bian4}{12,12}{⽇、⾡}[HSK 3]
  \definition{adj.}{geral; comum; universal; difundido; a existência é muito ampla; tem semelhança}
\end{EntryWithPhonetic}

\begin{EntryWithPhonetic}{普及}{pu3ji2}{12,3}{⽇、⼃}[HSK 3]
  \definition{adj.}{popular; universal; onipresente; amplamente compreendido, aceito ou utilizado}
  \definition[种]{v.}{popularizar; disseminar; espalhar entre as pessoas; promover amplamente o conhecimento, a educação, a tecnologia, etc. para popularizá-los}
\end{EntryWithPhonetic}

\begin{EntryWithPhonetic}{普通}{pu3 tong1}{12,10}{⽇、⾡}[HSK 2]
  \definition{adj.}{comum; normal; geral; médio; em geral, nada de especial, como a maioria das pessoas ou coisas}
\end{EntryWithPhonetic}

\begin{EntryWithPhonetic}{普通话}{pu3tong1hua4}{12,10,8}{⽇、⾡、⾔}[HSK 2]
  \definition*{s.}{Mandarim (literalmente "linguagem comum") | Putonghua (fala comum da língua chinesa) | Língua oficial da China}
\end{EntryWithPhonetic}

%%%%%%%%%% 堡 %%%%%%%%%%
\subsection*{堡}\addcontentsline{loh}{figure}{堡 \dpy{pu4}}

\begin{EntryWithPhonetic}{堡}{pu4}{12}{⼟}
  \definition{s.}{cidade ou rua (frequentemente usado em nomes de lugares)}
  \seeref{bao3}
  \seeref{bu3}
\end{EntryWithPhonetic}

%%%%%%%%%% 铺 %%%%%%%%%%
\subsection*{铺}\addcontentsline{loh}{figure}{铺 \dpy{pu4}}

\begin{EntryWithPhonetic}{铺}{pu4}{12}{⾦}
  \definition{s.}{pequena loja; depósito | uma cama feita de tábuas de madeira; geralmente se refere a uma cama | estação de correios; antiga estação de correios (usada principalmente em nomes de lugares)}
  \seeref{pu1}
\end{EntryWithPhonetic}

%%%%%%%%%% 瀑 %%%%%%%%%%
\subsection*{瀑}\addcontentsline{loh}{figure}{瀑 \dpy{pu4}}

\begin{EntryWithPhonetic}{瀑}{pu4}{18}{⽔}
  \definition{s.}{cachoeira; catarata}
  \seeref{bao4}
\end{EntryWithPhonetic}

\begin{EntryWithPhonetic}{瀑布}{pu4bu4}{18,5}{⽔、⼱}
  \definition{s.}{queda de água | cachoeira | cascata | catarata}
\end{EntryWithPhonetic}

%%%%%%%%%% 曝 %%%%%%%%%%
\subsection*{曝}\addcontentsline{loh}{figure}{曝 \dpy{pu4}}

\begin{EntryWithPhonetic}{曝}{pu4}{19}{⽇}
  \definition{v.}{expor ao sol}
  \seeref{bao4}
\end{EntryWithPhonetic}

%%%%% EOF %%%%%


 %%%
%%% Q
%%%

\section*{Q}\addcontentsline{toc}{section}{Q}

\begin{EntryWithPhonetic}{七}{qi1}{2}{⼀}[HSK 1]
  \definition*{s.}{Sobrenome Qi}
  \definition{num.}{sete; 7}
  \definition{s.}{antigamente, os mortos eram homenageados a cada sete dias, chamados de 七, até o quadragésimo nono dia, num total de sete 七}
\end{EntryWithPhonetic}

\begin{EntryWithPhonetic}{七夕}{qi1xi1}{2,3}{⼀、⼣}
  \definition*{s.}{Dia dos Namorados Chinês, quando o vaqueiro e a tecelã (牛郎织女) têm permissão para se reunirem anualmente | Festival das Meninas | Festival Duplo Sete, noite do sétimo mês lunar}
  \seealsoref{牛郎织女}{niu2 lang2 zhi1nv3}
\end{EntryWithPhonetic}

\begin{EntryWithPhonetic}{妻}{qi1}{8}{⼥}
  \definition{s.}{esposa}
  \seeref{qi4}
\end{EntryWithPhonetic}

\begin{EntryWithPhonetic}{妻子}{qi1zi3}{8,3}{⼥、⼦}
  \definition[个]{s.}{esposa e filhos; (chinês antigo) refere-se a esposas, filhos e filhas}
  \seeref{qi1zi5}
\end{EntryWithPhonetic}

\begin{EntryWithPhonetic}{妻子}{qi1zi5}{8,3}{⼥、⼦}[HSK 4]
  \definition[个]{s.}{esposa (não é usado como um termo carinhoso)}
  \seeref{qi1zi3}
\end{EntryWithPhonetic}

\begin{EntryWithPhonetic}{期}{qi1}{12}{⽉}[HSK 3]
  \definition{clas.}{questão; número; termo; coisas usadas para parcelamento}
  \definition{s.}{um período de tempo; fase; estágio | horário agendado; data agendada | tempo designado (programado)}
  \definition{v.}{marcar uma consulta | esperar; aguardar | esperar; ter esperança}
\end{EntryWithPhonetic}

\begin{EntryWithPhonetic}{期待}{qi1dai4}{12,9}{⽉、⼻}[HSK 4]
  \definition{v.}{aguardar; esperar; aguardar ansiosamente; ter em mente a realização de um determinado fim ou a ocorrência de uma determinada situação}
\end{EntryWithPhonetic}

\begin{EntryWithPhonetic}{期间}{qi1jian1}{12,7}{⽉、⾨}[HSK 4]
  \definition{s.}{prazo; tempo; período}
\end{EntryWithPhonetic}

\begin{EntryWithPhonetic}{期末}{qi1 mo4}{12,5}{⽉、⽊}[HSK 4]
  \definition{s.}{terminal; final do prazo; fim do período}
\end{EntryWithPhonetic}

\begin{EntryWithPhonetic}{期望}{qi1wang4}{12,11}{⽉、⽉}[HSK 5]
  \definition{s.}{esperança; expectativa}
  \definition{v.}{esperar; ter esperança}
\end{EntryWithPhonetic}

\begin{EntryWithPhonetic}{期限}{qi1xian4}{12,8}{⽉、⾩}[HSK 4]
  \definition{s.}{prazo; limite de tempo; tempo alocado; período de tempo limitado, também o limite final do limite de tempo; \emph{deadline}}
\end{EntryWithPhonetic}

\begin{EntryWithPhonetic}{期中}{qi1 zhong1}{12,4}{⽉、⼁}[HSK 4]
  \definition{adj.}{provisório; interino; intermediário}
\end{EntryWithPhonetic}

\begin{EntryWithPhonetic}{欺}{qi1}{12}{⽋}
  \definition{v.}{enganar; trapacear | intimidar; tirar vantagem de alguém; tirar vantagem da fraqueza de (alguém, etc.)}
\end{EntryWithPhonetic}

\begin{EntryWithPhonetic}{欺负}{qi1fu5}{12,6}{⽋、⾙}[HSK 6]
  \definition{v.}{violar, oprimir ou insultar com meios irracionais; \emph{bully}}
\end{EntryWithPhonetic}

\begin{EntryWithPhonetic}{齐}{qi2}{6}{⿑}[HSK 3][Kangxi 210]
  \definition*{s.}{Qi, um estado da Dinastia Zhou | Dinastia Qi do Sul (479-502), uma das Dinastias do Sul | Dinastia Qi do Norte (550-577), uma das Dinastias do Norte | Sobrenome Qi}
  \definition{adj.}{arrumado; uniforme; regular; comprimento, tamanho, etc. são praticamente iguais; uniformes | semelhante; similar; da mesma forma; de acordo| tudo pronto; todos presentes; completo; perfeito}
  \definition{adv.}{juntos; simultaneamente; ao mesmo tempo}
  \definition{v.}{estar no mesmo nível que; alcançar o mesmo nível | estar nivelado em um ponto ou ao longo de uma linha; tornar consistente; harmonizar}
\end{EntryWithPhonetic}

\begin{EntryWithPhonetic}{齐全}{qi2quan2}{6,6}{⿑、⼊}[HSK 5]
  \definition{adj.}{completo; tudo pronto}
\end{EntryWithPhonetic}

\begin{EntryWithPhonetic}{其}{qi2}{8}{⼋}[HSK 5]
  \definition*{s.}{Sobrenome Qi}
  \definition{adv.}{fazer uma suposição ou uma réplica | expressar comando, ordem}
  \definition{pron.}{dele (dela, deles, delas) | ele, ela, isso, eles; elas | isso; tal | isso (referindo-se a nenhuma pessoa ou coisa específica)}
  \definition{suf.}{sufixo de palavra, anexado ao advérbio}
\end{EntryWithPhonetic}

\begin{EntryWithPhonetic}{其次}{qi2ci4}{8,6}{⼋、⽋}[HSK 3]
  \definition{adj.}{secundário}
  \definition{conj.}{próximo; então; em segundo lugar; mais tarde na ordem}
\end{EntryWithPhonetic}

\begin{EntryWithPhonetic}{其实}{qi2shi2}{8,8}{⼋、⼧}[HSK 3]
  \definition{adv.}{na verdade; na realidade; a primeira parte é a situação aparente, e 其实 é usado para introduzir a situação real}
\end{EntryWithPhonetic}

\begin{EntryWithPhonetic}{其他}{qi2ta1}{8,5}{⼋、⼈}[HSK 2]
  \definition{pron.}{outra pessoa/outra coisa | outras coisas; outras pessoas; em substituição de outras pessoas ou coisas}
\end{EntryWithPhonetic}

\begin{EntryWithPhonetic}{其余}{qi2yu2}{8,7}{⼋、⼈}[HSK 4]
  \definition{pron.}{o resto; os outros; o restante}
\end{EntryWithPhonetic}

\begin{EntryWithPhonetic}{其中}{qi2zhong1}{8,4}{⼋、⼁}[HSK 2]
  \definition{pron.}{dentro; entre (os quais, eles, etc.); em (o qual, ele, etc.); nas pessoas ou coisas mencionadas anteriormente}
\end{EntryWithPhonetic}

\begin{EntryWithPhonetic}{奇}{qi2}{8}{⼤}
  \definition{adj.}{ímpar (número); singular; solteiro; não em pares (ao contrário de 偶)}
  \definition{s.}{lotes ímpares; quantidade fracionária (acima daquela mencionada em um número redondo)}
  \seealsoref{偶}{ou3}
\end{EntryWithPhonetic}

\begin{EntryWithPhonetic}{奇怪}{qi2guai4}{8,8}{⼤、⼼}[HSK 3]
  \definition{adj.}{estranho; diferente do habitual; raramente visto, até um pouco irracional | estranho; esquisito; a descrição é diferente do imaginado e é difícil de entender}
  \definition{v.}{ficar perplexo; maravilhar-se; sentir-se surpreso; sentir-se estranho; sentir-se incompreensível}
\end{EntryWithPhonetic}

\begin{EntryWithPhonetic}{奇迹}{qi2ji4}{8,9}{⼤、⾡}
  \definition[个,种]{s.}{milagre; maravilha; coisas extraordinárias inimagináveis}
\end{EntryWithPhonetic}

\begin{EntryWithPhonetic}{奇妙}{qi2miao4}{8,7}{⼤、⼥}[HSK 6]
  \definition{adj.}{maravilhoso; milagroso; intrigante; muito inteligente e engenhoso (usado principalmente para descrever coisas interessantes e novas)}
\end{EntryWithPhonetic}

\begin{EntryWithPhonetic}{骑}{qi2}{11}{⾺}[HSK 2]
  \definition{s.}{cavalos ou outros animais para montaria | cavalaria; cavaleiro, também se refere genericamente a qualquer pessoa que monta a cavalo}
  \definition{v.}{montar (um animal ou bicicleta); sentar-se na parte de trás de | montar; abranger ambos os lados}
\end{EntryWithPhonetic}

\begin{EntryWithPhonetic}{骑车}{qi2 che1}{11,4}{⾺、⾞}[HSK 2]
  \definition{v.}{andar de bicicleta; pedalar}
\end{EntryWithPhonetic}

\begin{EntryWithPhonetic}{旗}{qi2}{14}{⽅}
  \definition[面]{s.}{bandeira}
\end{EntryWithPhonetic}

\begin{EntryWithPhonetic}{企}{qi3}{6}{⼈}
  \definition{v.}{ficar na ponta dos pés | esperar ansiosamente por algo; ansiar por | planejar um projeto}
\end{EntryWithPhonetic}

\begin{EntryWithPhonetic}{企图}{qi3tu2}{6,8}{⼈、⼞}[HSK 6]
  \definition[种]{s.}{plano; tentativa; intenção (principalmente negativa)}
  \definition{v.}{procurar; tentar; pretender}
\end{EntryWithPhonetic}

\begin{EntryWithPhonetic}{企业}{qi3ye4}{6,5}{⼈、⼀}[HSK 4]
  \definition[家,个]{s.}{empresa; estabelecimento; empreendimento; negócio; setores envolvidos em atividades econômicas como produção, transporte, comércio, etc., como fábricas, minas, ferrovias, empresas comerciais, etc.}
\end{EntryWithPhonetic}

\begin{EntryWithPhonetic}{岂}{qi3}{6}{⼭}
  \definition*{s.}{Sobrenome Qi}
  \definition{adv.}{Litarário: expressa uma pergunta retórica, equivalente a 哪里, 怎么 e 难道}
  \seealsoref{哪里}{na3 li3}
  \seealsoref{难道}{nan2dao4}
  \seealsoref{怎么}{zen3me5}
\end{EntryWithPhonetic}

\begin{EntryWithPhonetic}{岂有此理}{qi3you3ci3li3}{6,6,6,11}{⼭、⽉、⽌、⽟}
  \definition{interj.}{Que exorbitante! | Absurdo! | Como isso pode ser assim? | Ridículo!}
\end{EntryWithPhonetic}

\begin{EntryWithPhonetic}{启}{qi3}{7}{⼝}
  \definition*{s.}{Sobrenome Qi}
  \definition{s.}{nota; carta; um dos antigos estilos literários, uma carta relativamente curta}
  \definition{v.}{abrir | despertar; iluminar | começar; iniciar | declarar; informar}
\end{EntryWithPhonetic}

\begin{EntryWithPhonetic}{启动}{qi3 dong4}{7,6}{⼝、⼒}[HSK 5]
  \definition{v.}{ligar (uma máquina); acionar; ligar máquinas, equipamentos elétricos, etc., para começar a trabalhar | entrar em vigor; começar a vigorar e a ser implementados planos, projetos, documentos jurídicos, etc.}
\end{EntryWithPhonetic}

\begin{EntryWithPhonetic}{启发}{qi3fa1}{7,5}{⼝、⼜}[HSK 5]
  \definition{s.}{iluminação; esclarecimento; fenômenos e princípios que levam as pessoas a refletir e a abrir suas mentes}
  \definition{v.}{despertar; inspirar; esclarecer; orientar, fazer com que compreendam}
\end{EntryWithPhonetic}

\begin{EntryWithPhonetic}{启事}{qi3shi4}{7,8}{⼝、⼅}[HSK 5]
  \definition[个,则,份,张,条]{s.}{aviso; anúncio; texto publicado em jornais ou afixado em paredes com o objetivo de divulgar publicamente algo}
\end{EntryWithPhonetic}

\begin{EntryWithPhonetic}{起}{qi3}{10}{⾛}[HSK 1]
  \definition{clas.}{caso; instância | lote; grupo}
  \definition{prep.}{de; colocado antes de uma palavra de tempo ou lugar, indica um ponto de partida | por; colocado antes de uma palavra de lugar, indica um lugar por onde passou}
  \definition{v.}{levantar-se; ficar de pé| iniciar; lançar; deicar a posição original | subir; ascender | aparecer; levantar; crescer (bolhas, protuberâncias, brotoeja) | puxar para cima; puxar para fora; tirar o que está guardado ou incorporado | crescer; aumentar | esboçar; elaborar | construir; montar; estabelecer | receber (comprovante) | começar; iniciar; combina com 从 e 由; indica quando, onde e quem começou | buscar; pegar; usado após um verbo, indica movimento para cima | indicar se alguém tem força suficiente ou não; usado após um verbo, indica que a força é suficiente ou insuficiente | indicar que a ação envolve alguém ou algo; equivalente a 及 ou 到 | começar; iniciar; usado depois de um verbo, indica o início de uma ação | juntar; implodir; (informal) usado depois de um verbo, para unir coisas ou fechá-las}
  \seealsoref{从}{cong2}
  \seealsoref{到}{dao4}
  \seealsoref{及}{ji2}
  \seealsoref{由}{you2}
\end{EntryWithPhonetic}

\begin{EntryWithPhonetic}{起床}{qi3/chuang2}{10,7}{⾛、⼴}[HSK 1]
  \definition{v.+compl.}{levantar-se; sair da cama; acordar e sair da cama (geralmente pela manhã); levantar-se da posição sentada, deitada ou deitada de bruços, ou sentar-se a partir da posição deitada}
\end{EntryWithPhonetic}

\begin{EntryWithPhonetic}{起到}{qi3 dao4}{10,8}{⾛、⼑}[HSK 5]
  \definition{v.}{ter (um efeito motivador, etc.); desempenhar (um papel estabilizador, etc.)}
\end{EntryWithPhonetic}

\begin{EntryWithPhonetic}{起点}{qi3 dian3}{10,9}{⾛、⽕}[HSK 6]
  \definition[个]{s.}{ponto de partida (para o tempo ou local do início de algo); o lugar ou hora de início | ponto de partida (para o nível ou base de algo feito inicialmente); refere-se especificamente ao ponto de partida designado em um evento de pista}
\end{EntryWithPhonetic}

\begin{EntryWithPhonetic}{起飞}{qi3fei1}{10,3}{⾛、⾶}[HSK 2]
  \definition{v.}{decolar; levantar voo | crescer rapidamente; decolar; disparar; metáfora para o rápido desenvolvimento de negócios, economia, etc.}
\end{EntryWithPhonetic}

\begin{EntryWithPhonetic}{起来}{qi3/lai2}{10,7}{⾛、⽊}[HSK 1]
  \definition{v.+compl.}{levantar-se; passar de posições como deitado, sentado ou ajoelhado para ficar em pé | levantar-se; sair da cama | levantar-se; revoltar-se; rebelar-se; refere-se a ascensão, surgimento, levantamento, etc.}
  \seeref{qi3lai5}
  \seeref{qi5lai2}
\end{EntryWithPhonetic}

\begin{EntryWithPhonetic}{起来}{qi3lai5}{10,7}{⾛、⽊}
  \definition{v.aux.}{usado depois de um verbo para indicar movimento ascendente}
  \seeref{qi3/lai2}
  \seeref{qi5lai2}
\end{EntryWithPhonetic}

\begin{EntryWithPhonetic}{起码}{qi3ma3}{10,8}{⾛、⽯}[HSK 5]
  \definition{adj.}{mínimo; elementar; rudimentar}
  \definition{adv.}{mínimamente; pelo menos;}
\end{EntryWithPhonetic}

\begin{EntryWithPhonetic}{起诉}{qi3 su4}{10,7}{⾛、⾔}[HSK 6]
  \definition{v.}{processar; entrar com uma ação judicial}
\end{EntryWithPhonetic}

\begin{EntryWithPhonetic}{起跳}{qi3tiao4}{10,13}{⾛、⾜}
  \definition{v.}{(atletismo) decolar (no início de um salto) | (de preço, salário, etc.) começar (de um determinado nível)}
\end{EntryWithPhonetic}

\begin{EntryWithPhonetic}{气}{qi4}{4}{⽓}[HSK 2][Kangxi 84]
  \definition*{s.}{Sobrenome Qi}
  \definition[口]{s.}{gás; gás em geral | ar; especificamente, o ar | respiração | clima; refere-se a fenômenos naturais como sol, chuva, frio e calor | cheiro; odor; o cheiro que o nariz sente | ânimo; moral; estado mental | ares; estilo; maneiras; refere-se ao estilo e aos hábitos de uma pessoa | raiva; irritação; aborrecimento; sentimento de irritação | energia vital; energia da vida; na medicina tradicional chinesa refere-se às substâncias sutis que circulam no corpo humano e permitem que os vários órgãos funcionem normalmente | certos sintomas (de doenças); na medicina tradicional chinesa refere-se a um determinado quadro clínico}
  \definition{v.}{ficar com raiva; ficar furioso; ficar irritado | irritar; enfurecer; deixar com raiva | ser intimidado; sofrer injustiça; intimidar}
\end{EntryWithPhonetic}

\begin{EntryWithPhonetic}{气氛}{qi4fen1}{4,8}{⽓、⽓}[HSK 6]
  \definition{s.}{atmosfera; sensação circundante; uma certa emoção ou cena que existe em um determinado ambiente e pode fazer as pessoas sentirem}
\end{EntryWithPhonetic}

\begin{EntryWithPhonetic}{气候}{qi4hou4}{4,10}{⽓、⼈}[HSK 3]
  \definition[种]{s.}{clima; tempo; condições meteorológicas gerais obtidas após muitos anos de observação em uma determinada região, estão relacionadas com correntes de ar, latitude, altitude acima do nível do mar, relevo, etc. | tendência; situação; metáfora do ambiente social, de uma determinada tendência | resultado; influência; conquista; realização; metáfora para algum tipo de resultado, conquista, influência significativa ou potencial de desenvolvimento}
\end{EntryWithPhonetic}

\begin{EntryWithPhonetic}{气球}{qi4qiu2}{4,11}{⽓、⽟}[HSK 4]
  \definition[个,只]{s.}{balão; bolas feitas de borracha, plástico, etc., que podem ser aumentadas soprando ar nelas e podem ser usadas como brinquedos, decorações ou meios de transporte}
\end{EntryWithPhonetic}

\begin{EntryWithPhonetic}{气体}{qi4 ti3}{4,7}{⽓、⼈}[HSK 5]
  \definition[种,瓶,升]{s.}{gás; não têm forma nem volume definidos e podem fluir.; o ar, o oxigênio, o gás metano e outros são gases}
\end{EntryWithPhonetic}

\begin{EntryWithPhonetic}{气温}{qi4 wen1}{4,12}{⽓、⽔}[HSK 2]
  \definition[个]{s.}{temperatura do ar}
\end{EntryWithPhonetic}

\begin{EntryWithPhonetic}{气象}{qi4xiang4}{4,11}{⽓、⾗}[HSK 5]
  \definition[种,派]{s.}{fenômenos meteorológicos; condições e fenômenos atmosféricos, como vento, relâmpagos, trovões, geadas, neve, etc. | meteorologia | situação; atmosfera; cena; circunstância | maneira imponente}
\end{EntryWithPhonetic}

\begin{EntryWithPhonetic}{气质}{qi4zhi4}{4,8}{⽓、⾙}
  \definition{s.}{traços de personalidade, temperamento, disposição | aura, ar, sentimento, \emph{vibe} | refinamento, sofisticação, classe}
\end{EntryWithPhonetic}

\begin{EntryWithPhonetic}{汽}{qi4}{7}{⽔}
  \definition{s.}{vapor | vaporizador}
\end{EntryWithPhonetic}

\begin{EntryWithPhonetic}{汽车}{qi4 che1}{7,4}{⽔、⾞}[HSK 1]
  \definition[辆,种,款]{s.}{automóvel; carro; veículo motorizado; veículo movido a motor de combustão interna, que circula principalmente em rodovias ou ruas, geralmente com quatro ou mais pneus de borracha, usado para transportar pessoas ou mercadorias}
\end{EntryWithPhonetic}

\begin{EntryWithPhonetic}{汽水}{qi4 shui3}{7,4}{⽔、⽔}[HSK 4]
  \definition[罐,杯,瓶,听,口]{s.}{refrigerante; refrigerante gaseificado; bebida refrescante, feita com a pressão de dióxido de carbono para dissolver na água e adicionar açúcar, suco de frutas, especiarias etc.}
\end{EntryWithPhonetic}

\begin{EntryWithPhonetic}{汽油}{qi4you2}{7,8}{⽔、⽔}[HSK 4]
  \definition[桶,升,吨]{s.}{gasolina; mistura líquida de hidrocarbonetos com volatilidade e combustibilidade, que é usada como combustível a partir do fracionamento ou craqueamento do petróleo}
\end{EntryWithPhonetic}

\begin{EntryWithPhonetic}{妻}{qi4}{8}{⼥}
  \definition{v.}{casar uma mulher com (alguém)}
  \seeref{qi1}
\end{EntryWithPhonetic}

\begin{EntryWithPhonetic}{器}{qi4}{16}{⼝}
  \definition[台]{s.}{dispositivo | ferramenta | utensílio}
\end{EntryWithPhonetic}

\begin{EntryWithPhonetic}{器官}{qi4guan1}{16,8}{⼝、⼧}[HSK 4]
  \definition[个,种]{s.}{órgão; aparelho; parte de um organismo que consiste em vários tipos de tecidos celulares que podem desempenhar uma função fisiológica separada}
\end{EntryWithPhonetic}

\begin{EntryWithPhonetic}{起来}{qi5lai2}{10,7}{⾛、⽊}
  \definition{v.}{descrever resultados, retratar comportamentos, transmitir movimento}
  \seeref{qi3/lai2}
  \seeref{qi3lai5}
\end{EntryWithPhonetic}

\begin{EntryWithPhonetic}{卡}{qia3}{5}{⼘}
  \definition*{s.}{Sobrenome Qia}
  \definition[张,片]{s.}{clipe; prendedor; pinça; utensílio para prender objetos | posto de controle; posto de guarda ou posto de controle localizado em vias de comunicação importantes ou em locais com terreno acidentado}
  \definition{v.}{encravar; ficar preso; impedir de se mover | parar; controlar; impedir | pressionar firmemente com a palma da mão}
  \seeref{ka3}
\end{EntryWithPhonetic}

\begin{EntryWithPhonetic}{恰}{qia4}{9}{⼼}
  \definition{adv.}{exatamente | apenas}
\end{EntryWithPhonetic}

\begin{EntryWithPhonetic}{恰当}{qia4dang4}{9,6}{⼼、⼹}[HSK 6]
  \definition{adj.}{adequado; apropriado; conveniente; apropriado; a linguagem ou abordagem é muito apropriada}
\end{EntryWithPhonetic}

\begin{EntryWithPhonetic}{恰到好处}{qia4dao4hao3chu4}{9,8,6,5}{⼼、⼑、⼥、⼡}
  \definition{expr.}{é simplesmente perfeito | é simplesmente correto}
\end{EntryWithPhonetic}

\begin{EntryWithPhonetic}{恰好}{qia4 hao3}{9,6}{⼼、⼥}[HSK 6]
  \definition{adv.}{na medida certa; como a sorte quis}
\end{EntryWithPhonetic}

\begin{EntryWithPhonetic}{恰恰}{qia4 qia4}{9,9}{⼼、⼼}[HSK 6]
  \definition{adv.}{justamente; exatamente; precisamente; bem na hora}
\end{EntryWithPhonetic}

\begin{EntryWithPhonetic}{千}{qian1}{3}{⼗}[HSK 2]
  \definition*{s.}{Sobrenome Qian}
  \definition{num.}{mil; 1.000; 1000 | a grande quantidade de; um grande número de}
\end{EntryWithPhonetic}

\begin{EntryWithPhonetic}{千古}{qian1gu3}{3,5}{⼗、⼝}
  \definition{adv.}{por toda a eternidade | em todas as idades}
  \definition{s.}{eternidade (usada em um dístico elegíaco, coroa de flores, etc., dedicada aos mortos)}
\end{EntryWithPhonetic}

\begin{EntryWithPhonetic}{千克}{qian1 ke4}{3,7}{⼗、⼗}[HSK 2]
  \definition{clas.}{kg; quilo; quilograma; 1 quilograma equivale a 1.000 gramas, ou 2 jin (斤)}
  \seealsoref{斤}{jin1}
\end{EntryWithPhonetic}

\begin{EntryWithPhonetic}{千年}{qian1nian2}{3,6}{⼗、⼲}
  \definition{s.}{milênio}
\end{EntryWithPhonetic}

\begin{EntryWithPhonetic}{千千万万}{qian1qian1wan4wan4}{3,3,3,3}{⼗、⼗、⼀、⼀}
  \definition{num.}{inumerável | números incontáveis | milhares e milhares}
\end{EntryWithPhonetic}

\begin{EntryWithPhonetic}{千万}{qian1wan4}{3,3}{⼗、⼀}[HSK 3]
  \definition{adv.}{(usado para indicar desejos fortes) por todos os meios; sob quaisquer circunstâncias; expressa uma exortação sincera, equivalente a 务必}
  \definition{num.}{dez milhões; 10.000.000; 1000.0000; milhões e milhões; um número aproximado, indicando um grande número}
  \seealsoref{务必}{wu4bi4}
\end{EntryWithPhonetic}

\begin{EntryWithPhonetic}{牵}{qian1}{9}{⽜}[HSK 6]
  \definition{v.}{conduzir (segurando a mão, o cabresto, etc.); puxar | envolver-se | sentir falta; preocupar-se com | controlar; restringir; ser retido; ser constrangido}
\end{EntryWithPhonetic}

\begin{EntryWithPhonetic}{铅}{qian1}{10}{⾦}
  \definition[根,盒]{s.}{chumbo (Pb) | grafite (em um lápis); grafite preta |}
\end{EntryWithPhonetic}

\begin{EntryWithPhonetic}{铅笔}{qian1bi3}{10,10}{⾦、⽵}[HSK 6]
  \definition[支,盒,种,枝,杆]{s.}{lápis; canetas com pontas de grafite ou argila pigmentada}
\end{EntryWithPhonetic}

\begin{EntryWithPhonetic}{谦}{qian1}{12}{⾔}
  \definition*{s.}{Sobrenome Qian}
  \definition{adj.}{modesto}
  \definition{s.}{modéstia}
\end{EntryWithPhonetic}

\begin{EntryWithPhonetic}{谦虚}{qian1xu1}{12,11}{⾔、⾌}[HSK 6]
  \definition{adj.}{modesto; não se orgulhe de suas próprias conquistas e esteja disposto a aceitar críticas e opiniões de outras pessoas}
  \definition{v.}{falar modestamente; quando recebo elogios e cumprimentos de outras pessoas, sinto que não sou tão bom}
\end{EntryWithPhonetic}

\begin{EntryWithPhonetic}{签}{qian1}{13}{⽵}[HSK 5]
  \definition[个,根,支]{s.}{tiras de bambu usadas para adivinhação ou sorteio; pPequenas tiras de bambu ou varas finas com caracteres e símbolos gravados, usadas para adivinhação, jogos de azar ou como fichas para contagem, etc. | etiqueta; adesivo; pequena tira usada como marca | um pedaço fino e pontiagudo de bambu ou madeira; pequeno bastão pontiagudo}
  \definition{v.}{assinar; autografar; escrever o nome, palavras ou fazer marcas em documentos ou recibos | fazer comentários breves em um documento; escrever brevemente (pontos principais ou opiniões) | (em costura) alinhavar; costura grosseira}
\end{EntryWithPhonetic}

\begin{EntryWithPhonetic}{签订}{qian1 ding4}{13,4}{⽵、⾔}[HSK 5]
  \definition{v.}{concluir e assinar (um tratado, etc.)}
\end{EntryWithPhonetic}

\begin{EntryWithPhonetic}{签名}{qian1/ming2}{13,6}{⽵、⼝}[HSK 5]
  \definition[个,次]{s.}{assinatura; autógrafo}
  \definition{v.+compl.}{assinar o próprio nome; autografar; escrever seu nome para indicar concordância, apoio ou homenagem, etc.}
\end{EntryWithPhonetic}

\begin{EntryWithPhonetic}{签约}{qian1 yue1}{13,6}{⽵、⽷}[HSK 5]
  \definition{v.}{assinar um contrato; assinar contratos e tratados, frequentemente utilizado no trabalho e em cooperações comerciais}
\end{EntryWithPhonetic}

\begin{EntryWithPhonetic}{签证}{qian1zheng4}{13,7}{⽵、⾔}[HSK 5]
  \definition[张,个,份]{s.}{visto; visto de entrada em um país}
\end{EntryWithPhonetic}

\begin{EntryWithPhonetic}{签字}{qian1 zi4}{13,6}{⽵、⼦}[HSK 5]
  \definition{v.}{assinar; colocar a assinatura; escrever seu nome à mão em documentos, recibos, etc., para demonstrar responsabilidade}
\end{EntryWithPhonetic}

\begin{EntryWithPhonetic}{前}{qian2}{9}{⼑}[HSK 1]
  \definition*{s.}{Sobrenome Qian}
  \definition{s.}{frente | futuro; perspectiva | atrás; antes; mais cedo do que uma coisa ou um momento | à frente; para a frente; na parte frontal (referindo-se ao espaço, em oposição a 后) | precedente; antes que algo aconteça | antigo; antigamente | topo; primeiro; primeiro na ordem | frente; campo de batalha | A.C. (Antes de~Cristo)}[前293年===293 a.C.]
  \definition{v.}{seguir em frente; ir em frente}
  \seealsoref{公元}{gong1yuan2}
  \seealsoref{后}{hou4}
\end{EntryWithPhonetic}

\begin{EntryWithPhonetic}{前边}{qian2 bian5}{9,5}{⼑、⾡}[HSK 1]
  \definition{adv.}{à frente; na frente}
\end{EntryWithPhonetic}

\begin{EntryWithPhonetic}{前方}{qian2 fang1}{9,4}{⼑、⽅}[HSK 6]
  \definition{s.}{frente; o espaço à frente; a direção voltada para a frente; a frente (em oposição à 后方) | linha de frente; frente de batalha; áreas onde os exércitos de ambos os lados estão se aproximando ou lutando}
  \seealsoref{后方}{hou4 fang1}
\end{EntryWithPhonetic}

\begin{EntryWithPhonetic}{前后}{qian2 hou4}{9,6}{⼑、⼝}[HSK 3]
  \definition{s.}{em volta; sobre; um período de tempo ligeiramente anterior ou posterior a um horário específico| do início ao fim; refere-se ao período de tempo do início ao fim de algo | frente e verso; na frente e atrás de algo}
\end{EntryWithPhonetic}

\begin{EntryWithPhonetic}{前进}{qian2 jin4}{9,7}{⼑、⾡}[HSK 3]
  \definition{v.}{marchar; avançar; para ir em frente; seguir em frente; geralmente se refere ao desenvolvimento futuro}
\end{EntryWithPhonetic}

\begin{EntryWithPhonetic}{前景}{qian2jing3}{9,12}{⼑、⽇}[HSK 5]
  \definition{s.}{primeiro plano (de uma vista, imagem, foto, etc.); as imagens que parecem mais próximas do espectador em pinturas, palcos e telas | vista; perspectiva; prospecto; ponto de vista; situações que podem ocorrer no trabalho, na carreira, etc.}
\end{EntryWithPhonetic}

\begin{EntryWithPhonetic}{前来}{qian2 lai2}{9,7}{⼑、⽊}[HSK 6]
  \definition{v.}{vir; em direção à localização e direção do falante}
\end{EntryWithPhonetic}

\begin{EntryWithPhonetic}{前面}{qian2mian4}{9,9}{⼑、⾯}[HSK 3]
  \definition{s.}{frente; a parte frontal do espaço ou posição | parte anterior; acima; a parte que vem primeiro na ordem; a parte de um artigo ou discurso que precede a narração atual}
\end{EntryWithPhonetic}

\begin{EntryWithPhonetic}{前年}{qian2 nian2}{9,6}{⼑、⼲}[HSK 2]
  \definition{adv.}{há dois anos; dois anos atrás}
\end{EntryWithPhonetic}

\begin{EntryWithPhonetic}{前提}{qian2ti2}{9,12}{⼑、⼿}[HSK 5]
  \definition[个,项]{s.}{premissa; pressuposto | pré-requisito; pressuposição; condições prévias para que algo aconteça ou se desenvolva}
\end{EntryWithPhonetic}

\begin{EntryWithPhonetic}{前天}{qian2 tian1}{9,4}{⼑、⼤}[HSK 1]
  \definition{adv.}{anteontem; dia anterior a ontem}
\end{EntryWithPhonetic}

\begin{EntryWithPhonetic}{前头}{qian2 tou5}{9,5}{⼑、⼤}[HSK 4]
  \definition{s.}{à frente; na frente; adiante}
\end{EntryWithPhonetic}

\begin{EntryWithPhonetic}{前途}{qian2tu2}{9,10}{⼑、⾡}[HSK 4]
  \definition[片,段,种]{s.}{futuro; perspectiva; prospecto; originalmente, refere-se à jornada à frente, mas, metaforicamente, refere-se ao futuro.}
\end{EntryWithPhonetic}

\begin{EntryWithPhonetic}{前往}{qian2 wang3}{9,8}{⼑、⼻}[HSK 3]
  \definition{v.}{ir para; prosseguir para; partir para; ir em frente}
\end{EntryWithPhonetic}

\begin{EntryWithPhonetic}{前线}{qian2 xian4}{9,8}{⼑、⽷}
  \definition{s.}{linha de frente; frente (oposto à 后方) | frente de batalha; a área onde os dois exércitos se aproximam durante uma batalha (em oposição à 后方)}
  \seealsoref{后方}{hou4 fang1}
\end{EntryWithPhonetic}

\begin{EntryWithPhonetic}{钱}{qian2}{10}{⾦}[HSK 1]
  \definition*{s.}{Sobrenome Qian}
  \definition{clas.}{qian, uma unidade de peso (=5 gramas) | qian, uma unidade de peso (um décimo de um tael 两)}
  \definition[笔]{s.}{dinheiro; riqueza; bens | moeda de cobre; dinheiro | objeto em forma de moeda de cobre | fundo; montante | dinheiro guardado ou gasto para algum fim específico (geralmente se refere a quantias significativas de dinheiro que entram e saem de órgãos públicos, organizações, etc.)}
  \seealsoref{两}{liang3}
\end{EntryWithPhonetic}

\begin{EntryWithPhonetic}{钱包}{qian2 bao1}{10,5}{⾦、⼓}[HSK 1]
  \definition[个]{s.}{carteira; bolsa; bolsa de dinheiro}
\end{EntryWithPhonetic}

\begin{EntryWithPhonetic}{潜}{qian2}{15}{⽔}
  \definition*{s.}{Sobrenome Qian}
  \definition{adj.}{latente; oculto}
  \definition{adv.}{furtivamente; secretamente; às escondidas}
  \definition{v.}{ir para debaixo d'água; esconder-se debaixo d'água; mergulhar | esconder | vadear (atravessar) na água | enterrar | fugir de casa}
\end{EntryWithPhonetic}

\begin{EntryWithPhonetic}{潜力}{qian2li4}{15,2}{⽔、⼒}[HSK 6]
  \definition{s.}{potencial; potencialidade; capacidade latente; as habilidades e possibilidades de desenvolvimento que as pessoas e as coisas ainda não demonstraram}
\end{EntryWithPhonetic}

\begin{EntryWithPhonetic}{潜在}{qian2zai4}{15,6}{⽔、⼟}
  \definition{adj.}{oculto | latente}
  \definition{s.}{potencial}
\end{EntryWithPhonetic}

\begin{EntryWithPhonetic}{浅}{qian3}{8}{⽔}[HSK 4]
  \definition{adj.}{raso; superficial;  (em oposição a 深) | fácil; simples; redação, conteúdo, etc. simples e fáceis de entender | superficial; não é profundo em aprendizado, percepção e sabedoria | não próximo; não íntimo; sentimentos não profundos | (cor) claro; pálido;  cor pouco intensa; leve |experiência breve; duração de tempo breve | baixo grau; peso leve; nível baixo}
  \seeref{jian1}
  \seealsoref{深}{shen1}
\end{EntryWithPhonetic}

\begin{EntryWithPhonetic}{欠}{qian4}{4}{⽋}[HSK 5][Kangxi 76]
  \definition{v.}{bocejar | levantar ligeiramente (uma parte do corpo) | estar em dívida; estar atrasado com; não devolver o que pediu emprestado a outra pessoa, ou não dar o que deveria ter dado a outra pessoa | faltar; não ser suficiente}
\end{EntryWithPhonetic}

\begin{EntryWithPhonetic}{抢}{qiang1}{7}{⼿}
  \definition{prep.}{contra; direção relativa inversa}
  \definition{v.}{bater; tocar}
  \seeref{qiang3}
\end{EntryWithPhonetic}

\begin{EntryWithPhonetic}{枪}{qiang1}{8}{⽊}[HSK 5]
  \definition*{s.}{Sobrenome Qiang}
  \definition[把,杆,支,挺]{s.}{lança | arma; rifle; arma de fogo | uma coisa em forma de arma | enxada; ferramenta para cavar a terra}
  \definition{v.}{escrever artigos ou responder perguntas para outras pessoas}
\end{EntryWithPhonetic}

\begin{EntryWithPhonetic}{将}{qiang1}{9}{⼨}
  \definition{v.}{pedir; apelar para}
  \seeref{jiang1}
  \seeref{jiang4}
\end{EntryWithPhonetic}

\begin{EntryWithPhonetic}{强}{qiang2}{12}{⼸}[HSK 3]
  \definition*{s.}{Sobrenome Qiang}
  \definition{adj.}{forte; poderoso  (em oposição a 弱) | melhor; superior | mais; extra; adicional; um pouco mais que; usado após uma fração ou decimal para indicar que é um pouco maior que o número | resoluto; firme | violento | alto padrão}
  \definition{v.}{fortalecer; tornar forte; tornar poderoso}
  \seeref{jiang4}
  \seeref{qiang3}
  \seealsoref{弱}{ruo4}
\end{EntryWithPhonetic}

\begin{EntryWithPhonetic}{强大}{qiang2 da4}{12,3}{⼸、⼤}[HSK 3]
  \definition{adj.}{forte; poderoso; potente; possante; descreve força forte e grande poder}
\end{EntryWithPhonetic}

\begin{EntryWithPhonetic}{强盗}{qiang2 dao4}{12,11}{⼸、⽫}[HSK 6]
  \definition[个,群,伙,帮]{s.}{ladrão; bandido; uma pessoa que usa violência para confiscar a propriedade de outros; também se refere a uma pessoa ou força que se envolve em comportamento semelhante}
\end{EntryWithPhonetic}

\begin{EntryWithPhonetic}{强调}{qiang2diao4}{12,10}{⼸、⾔}[HSK 3]
  \definition{v.}{salientar; sublinhar; enfatizar; dar ênfase a; vincar}
\end{EntryWithPhonetic}

\begin{EntryWithPhonetic}{强度}{qiang2 du4}{12,9}{⼸、⼴}[HSK 5]
  \definition[个,种]{s.}{intensidade; força | magnitude; rigor; avidez}
\end{EntryWithPhonetic}

\begin{EntryWithPhonetic}{强化}{qiang2 hua4}{12,4}{⼸、⼔}[HSK 6]
  \definition{v.}{intensificar; fortalecer; consolidar; tornar mais forte, melhorar sua habilidade e nível}
\end{EntryWithPhonetic}

\begin{EntryWithPhonetic}{强烈}{qiang2lie4}{12,10}{⼸、⽕}[HSK 3]
  \definition{adj.}{muito forte; intenso; poderoso | violento; impetuoso; nível muito alto; atitude muito firme, sem espaço para mudanças | afiado; marcante; mostrado em contraste; muito claro}
\end{EntryWithPhonetic}

\begin{EntryWithPhonetic}{强势}{qiang2 shi4}{12,8}{⼸、⼒}[HSK 6]
  \definition*{adj.}{forte; poderoso; dominante}
  \definition{s.}{momento; ímpeto; grande impulso; forte impulso | força; influência dominante; forças poderosas}
\end{EntryWithPhonetic}

\begin{EntryWithPhonetic}{强壮}{qiang2 zhuang4}{12,6}{⼸、⼠}[HSK 6]
  \definition{s.}{(corpo) forte, poderoso, robusto, resistente}
  \definition{v.}{fortalecer; construir}
\end{EntryWithPhonetic}

\begin{EntryWithPhonetic}{墙}{qiang2}{14}{⼟}[HSK 2]
  \definition[面,堵,道]{s.}{parede; barreira ou perímetro construído com tijolos, pedras, etc. | qualquer coisa com a forma ou função de uma parede; a parte de um objeto que funciona como parede ou divisória}
  \definition{v.}{(gíria) bloquear (um website) (usado geralmente na voz passiva: 被墙)}
\end{EntryWithPhonetic}

\begin{EntryWithPhonetic}{墙壁}{qiang2 bi4}{14,16}{⼟、⼟}[HSK 5]
  \definition[面,堵,道]{s.}{parede; barreira ou perímetro construído com tijolos, pedras ou terra}
\end{EntryWithPhonetic}

\begin{EntryWithPhonetic}{墙纸}{qiang2zhi3}{14,7}{⼟、⽷}
  \definition{s.}{papel de parede}
\end{EntryWithPhonetic}

\begin{EntryWithPhonetic}{抢}{qiang3}{7}{⼿}[HSK 5]
  \definition{v.}{roubar; saquear | agarrar; apanhar; arrebatar | disputar; lutar por; ser o primeiro; competir para ser o primeiro | correr; apressar-se; fazer uma incursão | raspar; arranhar; raspar ou esfregar uma camada da superfície de um objeto}
  \seeref{qiang1}
\end{EntryWithPhonetic}

\begin{EntryWithPhonetic}{抢救}{qiang3jiu4}{7,11}{⼿、⽁}[HSK 5]
  \definition{v.}{salvar; resgatar; prestar de socorro ou assistência rápidos em situações de emergência | salvar; tomar medidas rápidas para evitar ou minimizar perdas iminentes.}
\end{EntryWithPhonetic}

\begin{EntryWithPhonetic}{抢掠}{qiang3lve4}{7,11}{⼿、⼿}
  \definition{s.}{saque | pilhagem}
  \definition{v.}{saquear | pilhar}
\end{EntryWithPhonetic}

\begin{EntryWithPhonetic}{强}{qiang3}{12}{⼸}
  \definition{v.}{fazer um esforço; esforçar-se}
  \seeref{jiang4}
  \seeref{qiang2}
\end{EntryWithPhonetic}

\begin{EntryWithPhonetic}{强迫}{qiang3po4}{12,8}{⼸、⾡}[HSK 5]
  \definition{v.}{impelir; forçar; impor; compelir; aplicar pessão para obedecer}
\end{EntryWithPhonetic}

\begin{EntryWithPhonetic}{悄}{qiao1}{10}{⼼}
  \definition{adj.}{quieto; silencioso}
  \seeref{qiao3}
\end{EntryWithPhonetic}

\begin{EntryWithPhonetic}{悄悄}{qiao1qiao1}{10,10}{⼼、⼼}[HSK 5]
  \definition{adv.}{silenciosamente; em silêncio; aos sussuros; sem som ou em voz baixa; com o mínimo de ruído possível}
\end{EntryWithPhonetic}

\begin{EntryWithPhonetic}{敲}{qiao1}{14}{⽁}[HSK 5]
  \definition{v.}{bater; dar uma pancada; golpear | explorar alguém; cobrar a mais; extorquir; chantagear | lembrar; criticar; alertar; advertir}
\end{EntryWithPhonetic}

\begin{EntryWithPhonetic}{敲门}{qiao1 men2}{14,3}{⽁、⾨}[HSK 5]
  \definition{v.}{bater na porta}
\end{EntryWithPhonetic}

\begin{EntryWithPhonetic}{桥}{qiao2}{10}{⽊}[HSK 3]
  \definition*{s.}{Sobrenome Qiao}
  \definition[座]{s.}{ponte; construção que atravessa a água conectando as duas margens}
\end{EntryWithPhonetic}

\begin{EntryWithPhonetic}{桥梁}{qiao2liang2}{10,11}{⽊、⽊}[HSK 6]
  \definition[座]{s.}{ponte; acesso; uma obra construída na superfície do rio, conectando as duas margens | ponte; metáfora para pessoas ou coisas que podem se comunicar}
\end{EntryWithPhonetic}

\begin{EntryWithPhonetic}{瞧}{qiao2}{17}{⽬}[HSK 5]
  \definition{v.}{ver; olhar | tratar; diagnosticar e tratar | ver; visitar; fazer uma visita}
\end{EntryWithPhonetic}

\begin{EntryWithPhonetic}{巧}{qiao3}{5}{⼯}[HSK 3]
  \definition{adj.}{habilidoso; engenhoso; esperto | oportuno; coincidente; fortuito | astuto; enganoso; enganador; traiçoeiro; ardiloso | (de mão, língua) hábil; loquaz}
  \definition{s.}{(tecnologia, artesanato) habilidade; destreza}
\end{EntryWithPhonetic}

\begin{EntryWithPhonetic}{巧合}{qiao3he2}{5,6}{⼯、⼝}
  \definition{s.}{coincidência; (coisas) coincidentes ou idênticas}
\end{EntryWithPhonetic}

\begin{EntryWithPhonetic}{巧克力}{qiao3ke4li4}{5,7,2}{⼯、⼗、⼒}[HSK 4]
  \definition[块,颗,盒,包]{s.}{Empréstimo linguístico: chocolate; alimentos feitos com cacau em pó como principal matéria-prima, açúcar e especiarias}
\end{EntryWithPhonetic}

\begin{EntryWithPhonetic}{巧妙}{qiao3miao4}{5,7}{⼯、⼥}[HSK 6]
  \definition{adj.}{inteligente; engenhoso; (método ou técnica, etc.) inteligente, além do comum}
\end{EntryWithPhonetic}

\begin{EntryWithPhonetic}{悄}{qiao3}{10}{⼼}
  \definition{adj.}{quieto; silencioso | triste; preocupado; aflito}
  \seeref{qiao1}
\end{EntryWithPhonetic}

\begin{EntryWithPhonetic}{壳}{qiao4}{7}{⼠}
  \definition[层,个]{s.}{Coloquial: concha | invólucro; caixa; carapaça | empresa de fachada (ou corporação) | superfície dura}
  \seeref{ke2}
\end{EntryWithPhonetic}

\begin{EntryWithPhonetic}{切}{qie1}{4}{⼑}[HSK 4]
  \definition{v.}{cortar; fatiar; separar itens com uma faca | cortar ou romper; truncar | Geometria: refere-se a quando uma linha, círculo ou superfície intercepta um círculo, arco ou esfera em apenas um ponto}
  \seeref{qie4}
\end{EntryWithPhonetic}

\begin{EntryWithPhonetic}{切割}{qie1ge1}{4,12}{⼑、⼑}
  \definition{v.}{cortar}
\end{EntryWithPhonetic}

\begin{EntryWithPhonetic}{茄}{qie2}{8}{⾋}
  \definition[只]{s.}{berinjela}
  \seeref{jia1}
\end{EntryWithPhonetic}

\begin{EntryWithPhonetic}{茄子}{qie2 zi5}{8,3}{⾋、⼦}[HSK 6]
  \definition{interj.}{Onomatopéia: ``xis'' fonético (ao ser fotografado), equivale ao ``diga xis''}
  \definition[个,根]{s.}{berinjela (fruto e planta)}
\end{EntryWithPhonetic}

\begin{EntryWithPhonetic}{且}{qie3}{5}{⼀}
  \definition*{s.}{Sobrenome Qie}
  \definition{adv.}{apenas; por enquanto | por um longo tempo}
  \definition{conj.}{mesmo; até; até mesmo; usado na primeira cláusula de uma frase complexa para expressar concessão, equivalente a 尚且 | ambos\dots e\dots; conecta adjetivos ou verbos para expressar relacionamento paralelo, equivalente a 而且 e 又……又……}
  \seealsoref{而且}{er2 qie3}
  \seealsoref{尚且}{shang4 qie3}
  \seealsoref{又……又……}{you4 you4}
\end{EntryWithPhonetic}

\begin{EntryWithPhonetic}{切}{qie4}{4}{⼑}
  \definition{adj.}{ansioso; sério | duro; severo; rude; áspero}
  \definition{adv.}{com certeza; certamente}
  \definition{s.}{limiar; degrau}
  \definition{v.}{ser prático ou realista | ajustar-se ou corresponder | ser próximo ou íntimo | cortar algo em pedaços com uma faca | tomar o pulso (medicina tradicional chinesa)}
  \seeref{qie1}
\end{EntryWithPhonetic}

\begin{EntryWithPhonetic}{切实}{qie4shi2}{4,8}{⼑、⼧}[HSK 6]
  \definition{adj.}{prático; viável; realista}
\end{EntryWithPhonetic}

\begin{EntryWithPhonetic}{亲}{qin1}{9}{⼇}[HSK 3]
  \definition{adj.}{parente próximo; relacionado por sangue; de ​​parentesco consanguíneo; parente consanguíneo mais próximo | querido; próximo; íntimo; relações próximas entre pessoas; sentimentos profundos (em oposição a 疏) | em si mesmo; pessoalmente}
  \definition[位]{s.}{pais; refere-se aos pais; também se refere apenas ao pai ou à mãe | parente; refere-se a pessoas que são relacionadas por sangue ou casamento| casal; casamento; refere-se ao casamento ou relacionamento conjugal | noiva; refere-se especificamente à noiva}
  \definition{v.}{beijar | (de países, partidos, etc.) a favor de; apoiar; estar perto de}
  \seeref{qing4}
  \seealsoref{疏}{shu1}
\end{EntryWithPhonetic}

\begin{EntryWithPhonetic}{亲爱}{qin1'ai4}{9,10}{⼇、⽖}[HSK 4]
  \definition{adj.}{querido; amado; termo carinhoso que expressa intimidade e afeto}
\end{EntryWithPhonetic}

\begin{EntryWithPhonetic}{亲密}{qin1mi4}{9,11}{⼇、⼧}[HSK 4]
  \definition{adj.}{próximo; íntimo; relacionamento afetuoso e próximo}
\end{EntryWithPhonetic}

\begin{EntryWithPhonetic}{亲切}{qin1qie4}{9,4}{⼇、⼑}[HSK 3]
  \definition{adj.}{gentil; cordial; cheio de sinceridade e cuidado, fazendo com que as pessoas se sintam acolhidas e acessíveis | próximo; íntimo; por familiaridade e afeição}
\end{EntryWithPhonetic}

\begin{EntryWithPhonetic}{亲人}{qin1 ren2}{9,2}{⼇、⼈}[HSK 3]
  \definition[个,位]{s.}{um membro da família; os pais, o cônjuge, os filhos, etc.; refere-se a parentes ou cônjuges | queridos; entes queridos; aqueles queridos para alguém; uma metáfora para pessoas que têm um relacionamento próximo e sentimentos profundos}
\end{EntryWithPhonetic}

\begin{EntryWithPhonetic}{亲属}{qin1 shu3}{9,12}{⼇、⼫}[HSK 6]
  \definition{s.}{parentes; cognatos}
\end{EntryWithPhonetic}

\begin{EntryWithPhonetic}{亲眼}{qin1 yan3}{9,11}{⼇、⽬}[HSK 6]
  \definition{adv.}{pessoalmente; com os próprios olhos}
\end{EntryWithPhonetic}

\begin{EntryWithPhonetic}{亲自}{qin1zi4}{9,6}{⼇、⾃}[HSK 3]
  \definition{adv.}{pessoalmente; em pessoa; si mesmo; fazer algo diretamente por si mesmo}
\end{EntryWithPhonetic}

\begin{EntryWithPhonetic}{侵}{qin1}{9}{⼈}
  \definition*{s.}{Sobrenome Qin}
  \definition{prep.}{aproximando-se; aproximar}
  \definition{v.}{invadir; intrometer-se em; infringir | aproximar-se (amanhecer)}
\end{EntryWithPhonetic}

\begin{EntryWithPhonetic}{侵犯}{qin1fan4}{9,5}{⼈、⽝}[HSK 6]
  \definition{v.}{violar; invadir; infringir; interferência ilegal com terceiros e violação de seus direitos | violar; fazer incursões; invadir o território de outro país}
\end{EntryWithPhonetic}

\begin{EntryWithPhonetic}{侵略}{qin1lve4}{9,11}{⼈、⽥}
  \definition{s.}{invasão}
  \definition{v.}{invadir}
\end{EntryWithPhonetic}

\begin{EntryWithPhonetic}{芹}{qin2}{7}{⾋}
  \definition[把,棵]{s.}{aipo | aipo chinês}
\end{EntryWithPhonetic}

\begin{EntryWithPhonetic}{芹菜}{qin2cai4}{7,11}{⾋、⾋}
  \definition{s.}{salsão}
\end{EntryWithPhonetic}

\begin{EntryWithPhonetic}{琴}{qin2}{12}{⽟}[HSK 5]
  \definition*{s.}{Sobrenome Qin}
  \definition[架,台]{s.}{cítara; qin; guqin (um instrumento de cordas dedilhadas com sete cordas, em alguns aspectos semelhante à cítara)  | nome genérico para certos instrumentos musicais}
\end{EntryWithPhonetic}

\begin{EntryWithPhonetic}{琴键}{qin2jian4}{12,13}{⽟、⾦}
  \definition{s.}{tecla de piano}
\end{EntryWithPhonetic}

\begin{EntryWithPhonetic}{禽}{qin2}{12}{⽱}
  \definition*{s.}{Sobrenome Qin}
  \definition[只]{s.}{aves; pássaros | termo genérico para aves e animais}
\end{EntryWithPhonetic}

\begin{EntryWithPhonetic}{勤}{qin2}{13}{⼒}
  \definition*{s.}{Sobrenome Qin}
  \definition{adj.}{diligente; industrial; trabalhador}
  \definition{adv.}{frequentemente}
  \definition{s.}{dever; serviço | presença; trabalhadores que chegam ao trabalho no horário especificado}
\end{EntryWithPhonetic}

\begin{EntryWithPhonetic}{勤奋}{qin2fen4}{13,8}{⼒、⼤}[HSK 5]
  \definition{adj.}{diligente; assíduo; trabalhador; descreve alguém que se esforça continuamente nos estudos ou no trabalho}
\end{EntryWithPhonetic}

\begin{EntryWithPhonetic}{擒}{qin2}{15}{⼿}
  \definition{v.}{capturar; pegar; apreender}
\end{EntryWithPhonetic}

\begin{EntryWithPhonetic}{擒获}{qin2huo4}{15,10}{⼿、⾋}
  \definition{v.}{apreender | capturar}
\end{EntryWithPhonetic}

\begin{EntryWithPhonetic}{青}{qing1}{8}{⾭}[HSK 5][Kangxi 174]
  \definition*{s.}{Província de Qinghai, abreviação de 青海 | Sobrenome Qing}
  \definition{adj.}{azul ou verde | preto | jovens (pessoas)}
  \definition{s.}{grama verde | colheitas jovens (não maduras) | tiras de bambu verde}
  \seealsoref{青海}{qing1hai3}
\end{EntryWithPhonetic}

\begin{EntryWithPhonetic}{青菜}{qing1cai4}{8,11}{⾭、⾋}
  \definition{s.}{verduras}
\end{EntryWithPhonetic}

\begin{EntryWithPhonetic}{青春}{qing1chun1}{8,9}{⾭、⽇}[HSK 4]
  \definition[个]{s.}{juventude; jovialidade}
\end{EntryWithPhonetic}

\begin{EntryWithPhonetic}{青海}{qing1hai3}{8,10}{⾭、⽔}
  \definition*{s.}{Província de Qinghai}
\end{EntryWithPhonetic}

\begin{EntryWithPhonetic}{青椒}{qing1jiao1}{8,12}{⾭、⽊}
  \definition{s.}{pimenta verde}
\end{EntryWithPhonetic}

\begin{EntryWithPhonetic}{青年}{qing1 nian2}{8,6}{⾭、⼲}[HSK 2]
  \definition[个,位,名,些]{s.}{juventude; jovem; refere-se ao período entre os 15 e os 30 anos de idade.}
\end{EntryWithPhonetic}

\begin{EntryWithPhonetic}{青年节}{qing1nian2jie2}{8,6,5}{⾭、⼲、⾋}
  \definition*{s.}{Dia da Juventude (4 de maio)}
\end{EntryWithPhonetic}

\begin{EntryWithPhonetic}{青少年}{qing1shao4nian2}{8,4,6}{⾭、⼩、⼲}[HSK 2]
  \definition[位,名,个,些]{s.}{adolescentes}
\end{EntryWithPhonetic}

\begin{EntryWithPhonetic}{青天}{qing1tian1}{8,4}{⾭、⼤}
  \definition{s.}{céu claro, limpo ou azul}
\end{EntryWithPhonetic}

\begin{EntryWithPhonetic}{青铜}{qing1tong2}{8,11}{⾭、⾦}
  \definition{s.}{bronze (liga de cobre, 銅, e estanho, 锡)}
\end{EntryWithPhonetic}

\begin{EntryWithPhonetic}{青蛙}{qing1wa1}{8,12}{⾭、⾍}
  \definition{adj.}{(gíria velha) cara feio}
  \definition[只]{s.}{sapo}
\end{EntryWithPhonetic}

\begin{EntryWithPhonetic}{青玉米}{qing1yu4mi3}{8,5,6}{⾭、⽟、⽶}
  \definition{s.}{milho verde}
\end{EntryWithPhonetic}

\begin{EntryWithPhonetic}{轻}{qing1}{9}{⾞}[HSK 2]
  \definition{adj.}{de pouco peso; leve (oposto de 重) | (de carga, equipamento, etc.) pequeno; simples | pequeno em número, grau, etc. | não sério; relaxante; leve | sem importância | suave; delicado | levianos, crédulos | leve; peso leve; densidade baixa | leve; descontraído; fácil | imprudente; descuidado | inconstante; frívolo}
  \definition{v.}{menosprezar; subestimar}
  \seealsoref{重}{zhong4}
\end{EntryWithPhonetic}

\begin{EntryWithPhonetic}{轻松}{qing1song1}{9,8}{⾞、⽊}[HSK 4]
  \definition{adj.}{leve; relaxado; livre de fardos; não nervoso; não cansado}
  \definition{v.}{sentir-se livre de fardos; não se sentir nervoso ou cansado}
\end{EntryWithPhonetic}

\begin{EntryWithPhonetic}{轻易}{qing1yi4}{9,8}{⾞、⽇}[HSK 4]
  \definition{adv.}{facilmente; prontamente | facilmente; precipitadamente; indica que uma ação é realizada casualmente, geralmente usado em frases negativas}
\end{EntryWithPhonetic}

\begin{EntryWithPhonetic}{倾}{qing1}{10}{⼈}
  \definition{s.}{desvio; tendência}
  \definition{v.}{inclinar; inclinar-se; dobrar-se | colapsar | virar e despejar; esvaziar | fazer tudo o que puder; usar todos os recursos | sobrecarregar; dominar; dominar | admirar | superar}
\end{EntryWithPhonetic}

\begin{EntryWithPhonetic}{倾城}{qing1cheng2}{10,9}{⼈、⼟}
  \definition{adj.}{sedutora (mulher)}
  \definition{adv.}{de todo o lugar | vindo de todos os lugares}
  \definition{v.}{arruinar e derrubar o estado}
\end{EntryWithPhonetic}

\begin{EntryWithPhonetic}{倾向}{qing1xiang4}{10,6}{⼈、⼝}[HSK 6]
  \definition{s.}{tendência; desvio; inclinação; direção do desenvolvimento}
  \definition{v.}{preferir; estar inclinado a; concordar com uma determinada opinião}
\end{EntryWithPhonetic}

\begin{EntryWithPhonetic}{清}{qing1}{11}{⽔}[HSK 6]
  \definition*{s.}{Dinastia Qing (1644-1911) | Sobrenome Qing}
  \definition{adj.}{claro; não misturado; (líquido ou gasoso) puro e sem mistura (em oposição a 浊) | silencioso; quieto | justo e honesto | distinto; claro; esclarecido | simples; puro, sem qualquer adulteração ou combinação | limpo; puro}
  \definition{v.}{limpar; tornar limpo | resolver; esclarecer; pagar; liquidar | contar; inspecionar}
  \seealsoref{浊}{zhuo2}
\end{EntryWithPhonetic}

\begin{EntryWithPhonetic}{清唱}{qing1chang4}{11,11}{⽔、⼝}
  \definition{v.}{cantar à capela}
\end{EntryWithPhonetic}

\begin{EntryWithPhonetic}{清彻}{qing1che4}{11,7}{⽔、⼻}
  \variantof{清澈}
\end{EntryWithPhonetic}

\begin{EntryWithPhonetic}{清澈}{qing1che4}{11,15}{⽔、⽔}
  \definition{adj.}{claro | límpido}
\end{EntryWithPhonetic}

\begin{EntryWithPhonetic}{清晨}{qing1chen2}{11,11}{⽔、⽇}[HSK 5]
  \definition{s.}{matinal; manhã cedo; geralmente se refere ao período do amanhecer até logo após o nascer do sol}
\end{EntryWithPhonetic}

\begin{EntryWithPhonetic}{清楚}{qing1chu5}{11,13}{⽔、⽊}[HSK 2]
  \definition{adj.}{claro; distinto; compreensível; organizado; fácil de identificar e entender | plenamente consciente de; claro sobre}
  \definition{v.}{ter clareza sobre; compreender; ação que expressa compreensão e conhecimento}
\end{EntryWithPhonetic}

\begin{EntryWithPhonetic}{清洁}{qing1jie2}{11,9}{⽔、⽔}[HSK 6]
  \definition{adj.}{limpo; sem poeira, gordura, etc.}
  \definition{v.}{limpar}
\end{EntryWithPhonetic}

\begin{EntryWithPhonetic}{清洁工}{qing1 jie2 gong1}{11,9,3}{⽔、⽔、⼯}[HSK 6]
  \definition{s.}{coletor de lixo; trabalhador de saneamento; limpador de rua; trabalhadores envolvidos na limpeza do ambiente, remoção de lixo e fezes, etc.}
\end{EntryWithPhonetic}

\begin{EntryWithPhonetic}{清理}{qing1li3}{11,11}{⽔、⽟}[HSK 5]
  \definition{v.}{esclarecer; resolver; verificar; colocar em ordem; organizar tudo e jogar fora o que não for útil}
\end{EntryWithPhonetic}

\begin{EntryWithPhonetic}{清凉}{qing1liang2}{11,10}{⽔、⼎}
  \definition{adj.}{fresco | refrescante | (roupa) ousada, reveladora}
\end{EntryWithPhonetic}

\begin{EntryWithPhonetic}{清明节}{qing1 ming2 jie2}{11,8,5}{⽔、⽇、⾋}[HSK 6]
  \definition*{s.}{Qingming ou Festival do Brilho Puro ou Dia da Varredura de Túmulos, Dia dos Finados (uma das 24~divisões do ano solar no calendário lunar chinês:~dia~4 ou 5~de~abril solar)}
\end{EntryWithPhonetic}

\begin{EntryWithPhonetic}{清爽}{qing1shuang3}{11,11}{⽔、⽘}
  \definition{adj.}{refrescante | relaxado}
\end{EntryWithPhonetic}

\begin{EntryWithPhonetic}{清晰}{qing1xi1}{11,12}{⽔、⽇}
  \definition{adj.}{claro | distinto}
\end{EntryWithPhonetic}

\begin{EntryWithPhonetic}{清洗}{qing1 xi3}{11,9}{⽔、⽔}[HSK 6]
  \definition{v.}{enxaguar; lavar; limpar | purgar; limpar | eliminar}
\end{EntryWithPhonetic}

\begin{EntryWithPhonetic}{清醒}{qing1xing3}{11,16}{⽔、⾣}[HSK 4]
  \definition{adj.}{sóbrio; lúcido}
  \definition{v.}{recuperar a consciência; recuperar-se de um coma}
\end{EntryWithPhonetic}

\begin{EntryWithPhonetic}{蜻}{qing1}{14}{⾍}
  \definition[只]{s.}{libélula, 蜻蜓}
  \seealsoref{蜻蜓}{qing1ting2}
\end{EntryWithPhonetic}

\begin{EntryWithPhonetic}{蜻蜓}{qing1ting2}{14,12}{⾍、⾍}
  \definition{s.}{libélula}
\end{EntryWithPhonetic}

\begin{EntryWithPhonetic}{蜻蝏}{qing1ting2}{14,15}{⾍、⾍}
  \variantof{蜻蜓}
\end{EntryWithPhonetic}

\begin{EntryWithPhonetic}{情}{qing2}{11}{⼼}
  \definition{s.}{sentimento; afeição | amor; paixão | paixão sexual; luxúria | favor; gentileza | situação; circunstâncias; condição | razão; sentido | sensibilidades; sentimentos}
\end{EntryWithPhonetic}

\begin{EntryWithPhonetic}{情感}{qing2 gan3}{11,13}{⼼、⼼}[HSK 3]
  \definition[份]{s.}{emoção; sentimento | afeição; apego; reações psicológicas positivas ou negativas a estímulos externos, como gosto, raiva, tristeza, medo, amor, nojo, etc.}
\end{EntryWithPhonetic}

\begin{EntryWithPhonetic}{情节}{qing2jie2}{11,5}{⼼、⾋}[HSK 5]
  \definition[个,段]{s.}{enredo; trama; desenrolar específico dos acontecimentos | circunstância; detalhes do crime ou erro | enredo; roteiro; refere-se especificamente ao processo de desenvolvimento e evolução dos conflitos e contradições em obras literárias narrativas}
\end{EntryWithPhonetic}

\begin{EntryWithPhonetic}{情景}{qing2jing3}{11,12}{⼼、⽇}[HSK 4]
  \definition[个,幕,种]{s.}{cena; vista; circunstâncias}
\end{EntryWithPhonetic}

\begin{EntryWithPhonetic}{情况}{qing2kuang4}{11,7}{⼼、⼎}[HSK 3]
  \definition[种,个,些]{s.}{condição; situação; circunstâncias; estado das coisas | mudanças notáveis e impactantes}
\end{EntryWithPhonetic}

\begin{EntryWithPhonetic}{情形}{qing2xing2}{11,7}{⼼、⼺}[HSK 5]
  \definition[个,种]{s.}{situação; condição; circunstâncias; estado de coisas; a situação específica das coisas}
\end{EntryWithPhonetic}

\begin{EntryWithPhonetic}{情绪}{qing2xu4}{11,11}{⼼、⽷}[HSK 6]
  \definition[种,片,股,丝]{s.}{mau humor; depressão; um sentimento ruim no coração, especialmente um estado mental desagradável quando se sente injusto | emoção; humor; moral; sentimento; o estado mental de uma pessoa ao longo de um período de tempo}
\end{EntryWithPhonetic}

\begin{EntryWithPhonetic}{晴}{qing2}{12}{⽇}[HSK 2]
  \definition{adj.}{ensolarado; bom; claro; não há nuvens no céu ou há poucas nuvens}
\end{EntryWithPhonetic}

\begin{EntryWithPhonetic}{晴朗}{qing2lang3}{12,10}{⽇、⽉}[HSK 5]
  \definition{adj.}{bom; claro; ensolarado; céu limpo e sem nuvens}
\end{EntryWithPhonetic}

\begin{EntryWithPhonetic}{晴天}{qing2 tian1}{12,4}{⽇、⼤}[HSK 2]
  \definition[个]{s.}{dia ensolarado; tempo sem nuvens ou com poucas nuvens; em meteorologia, refere-se a um tempo em que a cobertura de nuvens no céu é inferior a 10\%}
\end{EntryWithPhonetic}

\begin{EntryWithPhonetic}{请}{qing3}{10}{⾔}[HSK 1]
  \definition*{s.}{Sobrenome Qing}
  \definition{v.}{solicitar; perguntar | convidar; envolver | por favor; uma expressão educada usada quando você quer que alguém faça algo | comprar coisas sagradas para sacrifício, como incenso, velas, cavalos de papel e santuários de Buda; superstição se refere à compra de estátuas de Buda, santuários, etc. | entreter}
\end{EntryWithPhonetic}

\begin{EntryWithPhonetic}{请假}{qing3/jia4}{10,11}{⾔、⼈}[HSK 1]
  \definition{v.+compl.}{pedir licença para sair; solicitar permissão para não trabalhar ou estudar por um determinado período de tempo devido a doença ou outros motivos}
\end{EntryWithPhonetic}

\begin{EntryWithPhonetic}{请假条}{qing3jia4tiao2}{10,11,7}{⾔、⼈、⽊}
  \definition{s.}{pedido de licença de ausência (do trabalho ou da escola)}
\end{EntryWithPhonetic}

\begin{EntryWithPhonetic}{请教}{qing3jiao4}{10,11}{⾔、⽁}[HSK 3]
  \definition{v.}{consultar; pedir conselho}
\end{EntryWithPhonetic}

\begin{EntryWithPhonetic}{请进}{qing3 jin4}{10,7}{⾔、⾡}[HSK 1]
  \definition{v.}{por favor entre; convidar alguém para um espaço ou lugar}
\end{EntryWithPhonetic}

\begin{EntryWithPhonetic}{请客}{qing3/ke4}{10,9}{⾔、⼧}[HSK 2]
  \definition{v.+compl.}{receber convidados; hospedar convidados | oferecer; convidar; pagar a conta; arcar com os custos; convidar alguém para comer, tomar chá, etc.}
\end{EntryWithPhonetic}

\begin{EntryWithPhonetic}{请求}{qing3qiu2}{10,7}{⾔、⽔}[HSK 2]
  \definition[个,次]{s.}{pedido; petição; solicitação; refere-se à exigência apresentada}
  \definition{v.}{pedir; solicitar; requerer; peticionar; fazer uma solicitação e pedir que a outra parte concorde com ela}
\end{EntryWithPhonetic}

\begin{EntryWithPhonetic}{请问}{qing3 wen4}{10,6}{⾔、⾨}[HSK 1]
  \definition{expr.}{Com licença, posso perguntar\dots? (para perguntar por qualquer coisa); uma maneira educada de pedir para alguém responder a uma pergunta}
\end{EntryWithPhonetic}

\begin{EntryWithPhonetic}{请坐}{qing3 zuo4}{10,7}{⾔、⼟}[HSK 1]
  \definition{v.}{por favor, sente-se; convidar outras pessoas para sentar ou descansar}
\end{EntryWithPhonetic}

\begin{EntryWithPhonetic}{庆}{qing4}{6}{⼴}
  \definition*{s.}{Sobrenome Qing}
  \definition{s.}{celebração | ocasião para celebração; um aniversário que vale a pena comemorar}
  \definition{v.}{celebrar; felicitar; comemorar}
\end{EntryWithPhonetic}

\begin{EntryWithPhonetic}{庆祝}{qing4zhu4}{6,9}{⼴、⽰}[HSK 3]
  \definition{v.}{celebrar; comemorar; festejar; realizar atividades para comemorar ou celebrar festivais comuns e eventos felizes}
\end{EntryWithPhonetic}

\begin{EntryWithPhonetic}{亲}{qing4}{9}{⼇}
  \definition{s.}{parentes por afinidade; parentes por casamento}
  \seeref{qin1}
\end{EntryWithPhonetic}

\begin{EntryWithPhonetic}{穷}{qiong2}{7}{⽳}[HSK 4]
  \definition{adj.}{remoto; isolado; de difícil acesso | pobre; atingido pela pobreza | situação difícil, sem saída}
  \definition{adv.}{completamente | extremamente}
  \definition{v.}{exaurir; esgotar; consmir | ir até o fim; perseguir completamente perseguido; sondar profundamente | gastar}
\end{EntryWithPhonetic}

\begin{EntryWithPhonetic}{穷人}{qiong2 ren2}{7,2}{⽳、⼈}[HSK 4]
  \definition[个]{s.}{os pobres; pessoas pobres}
\end{EntryWithPhonetic}

\begin{EntryWithPhonetic}{丘}{qiu1}{5}{⼀}
  \definition*{s.}{Sobrenome Qiu}
  \definition[个]{s.}{monte; outeiro | (literário) sepultura}
\end{EntryWithPhonetic}

\begin{EntryWithPhonetic}{丘陵}{qiu1ling2}{5,10}{⼀、⾩}
  \definition{s.}{colinas}
\end{EntryWithPhonetic}

\begin{EntryWithPhonetic}{秋}{qiu1}{9}{⽲}
  \definition*{s.}{Sobrenome Qiu}
  \definition{s.}{outono | época da colheita; a estação em que as colheitas amadurecem; colheitas maduras no outono | ano; refere-se a um ano | um período de tempo (geralmente conturbado)}
\end{EntryWithPhonetic}

\begin{EntryWithPhonetic}{秋季}{qiu1 ji4}{9,8}{⽲、⼦}[HSK 4]
  \definition[个]{s.}{outono; terceiro trimestre do ano, segundo o costume chinês, refere-se ao período de três meses entre o outono e o inverno, também se refere aos sétimo, oitavo e nono meses do calendário lunar}
\end{EntryWithPhonetic}

\begin{EntryWithPhonetic}{秋天}{qiu1 tian1}{9,4}{⽲、⼤}[HSK 2]
  \definition[个,段,季,番]{s.}{outono}
\end{EntryWithPhonetic}

\begin{EntryWithPhonetic}{仇}{qiu2}{4}{⼈}
  \definition*{s.}{Sobrenome Qiu}
  \definition{s.}{Literário: cônjuge; esposa; companheira}
  \seeref{chou2}
\end{EntryWithPhonetic}

\begin{EntryWithPhonetic}{求}{qiu2}{7}{⽔}[HSK 2]
  \definition*{s.}{Sobrenome Qiu}
  \definition{v.}{implorar; solicitar; suplicar; rogar | lutar por; buscar; investigar | tentar; procurar; tentar obter | demandar}
\end{EntryWithPhonetic}

\begin{EntryWithPhonetic}{求职}{qiu2 zhi2}{7,11}{⽔、⽿}[HSK 6]
  \definition{v.}{procurar emprego; candidatar-se a um emprego; encontrar um emprego}
\end{EntryWithPhonetic}

\begin{EntryWithPhonetic}{球}{qiu2}{11}{⽟}[HSK 1]
  \definition[个,颗,筐]{s.}{esfera; globo; equipamento de jogo antigo, objeto tridimensional circular, feito de couro, recheado com penas, para ser chutado com os pés ou batido com um bastão | qualquer coisa com formato de bola; algo esférico ou quase esférico | bola; refere-se a certos artigos esportivos (geralmente redondos e tridimensionais) | jogo; partida; referência a esportes com bola | o Globo; a Terra; referindo-se especificamente à Terra}
\end{EntryWithPhonetic}

\begin{EntryWithPhonetic}{球场}{qiu2 chang3}{11,6}{⽟、⼟}[HSK 2]
  \definition[个,座]{s.}{quadra; campo; terreno para jogos com bola; campos para a prática de esportes com bola, como basquete, futebol, tênis e vôlei, cuja forma, tamanho e equipamentos variam de acordo com as exigências de cada esporte}
\end{EntryWithPhonetic}

\begin{EntryWithPhonetic}{球队}{qiu2 dui4}{11,4}{⽟、⾩}[HSK 2]
  \definition[个,支]{s.}{equipe (basquete, futebol, etc.); equipe de atletas formada para competições esportivas com bola, como times de basquete, futebol, etc.}
\end{EntryWithPhonetic}

\begin{EntryWithPhonetic}{球迷}{qiu2mi2}{11,9}{⽟、⾡}[HSK 3]
  \definition[个,位,名,些]{s.}{fã (de esportes de bola); pessoas obcecadas por jogar ou assistir jogos de bola}
\end{EntryWithPhonetic}

\begin{EntryWithPhonetic}{球拍}{qiu2 pai1}{11,8}{⽟、⼿}[HSK 6]
  \definition[支]{s.}{(tênis, badminton, etc.) raquete}
\end{EntryWithPhonetic}

\begin{EntryWithPhonetic}{球鞋}{qiu2 xie2}{11,15}{⽟、⾰}[HSK 2]
  \definition[双,只,款]{s.}{tênis de ginástica; tênis de tênis; tênis esportivos}
\end{EntryWithPhonetic}

\begin{EntryWithPhonetic}{球星}{qiu2 xing1}{11,9}{⽟、⽇}[HSK 6]
  \definition[位,名]{s.}{estrela do esporte (esporte com bola)}
\end{EntryWithPhonetic}

\begin{EntryWithPhonetic}{球员}{qiu2 yuan2}{11,7}{⽟、⼝}[HSK 6]
  \definition[名,位,个]{s.}{Esporte: jogador | membro do clube esportivo}
\end{EntryWithPhonetic}

\begin{EntryWithPhonetic}{区}{qu1}{4}{⼖}[HSK 3]
  \definition{s.}{área; distrito; região; zona; uma determinada área em terra, água ou ar | uma divisão administrativa; as divisões administrativas incluem regiões autônomas étnicas de nível provincial, distritos municipais e de condado e distritos de condado; grandes regiões administrativas, regiões, zonas especiais e regiões administrativas especiais}
  \definition{v.}{classificar; subdividir; distinguir}
  \seeref{ou1}
\end{EntryWithPhonetic}

\begin{EntryWithPhonetic}{区别}{qu1bie2}{4,7}{⼖、⼑}[HSK 3]
  \definition[种,个]{s.}{diferença; distinção; discriminação}
  \definition{v.}{distinguir; diferenciar; fazer distinção entre}
\end{EntryWithPhonetic}

\begin{EntryWithPhonetic}{区分}{qu1fen1}{4,4}{⼖、⼑}[HSK 6]
  \definition{v.}{discriminar; diferenciar; distinguir; comparar dois ou mais objetos; reconhecer suas diferenças}
\end{EntryWithPhonetic}

\begin{EntryWithPhonetic}{区域}{qu1yu4}{4,11}{⼖、⼟}[HSK 5]
  \definition[片,块,个]{s.}{área; setor; região; faixa; inclui áreas regionais com condições naturais, culturais, administrativas, etc.}
\end{EntryWithPhonetic}

\begin{EntryWithPhonetic}{曲}{qu1}{6}{⽈}
  \definition*{s.}{Sobrenome Qu}
  \definition{adj.}{dobrado; curvado (oposto a 直) | errado; injustificável | torto}
  \definition{v.}{dobrar | torcer}
  \seealsoref{直}{zhi2}
\end{EntryWithPhonetic}

\begin{EntryWithPhonetic}{曲棍球}{qu1gun4qiu2}{6,12,11}{⽈、⽊、⽟}
  \definition{s.}{hóquei em campo}
\end{EntryWithPhonetic}

\begin{EntryWithPhonetic}{驱}{qu1}{7}{⾺}
  \definition{v.}{dirigir (um cavalo, um carro, etc.) | expulsar; dispersar | correr rápido}
\end{EntryWithPhonetic}

\begin{EntryWithPhonetic}{屈}{qu1}{8}{⼫}
  \definition*{s.}{Sobrenome Qu}
  \definition[个]{s.}{injustiça; tratamento injusto | erro; queixa; injustiça}
  \definition{v.}{dobrar; curvar; encurvar | subjugar; submeter | tratar mal; tratar injustamente (ou deslealmente) | estar errado}
\end{EntryWithPhonetic}

\begin{EntryWithPhonetic}{屈原}{qu1yuan2}{8,10}{⼫、⼚}
  \definition*{s.}{Qu Yuan, poeta, é uma figura histórica famosa na cultura chinesa que viveu durante o Período dos Reinos Combatentes (340-278 a.C.).}
\end{EntryWithPhonetic}

\begin{EntryWithPhonetic}{趋}{qu1}{12}{⾛}
  \definition{v.}{apressar-se | tender para; tender a se tornar | (de um ganso, cobra, etc.) estalar a cabeça e morder as pessoas}
\end{EntryWithPhonetic}

\begin{EntryWithPhonetic}{趋势}{qu1shi4}{12,8}{⾛、⼒}[HSK 4]
  \definition{s.}{tendência; tendência; direção; impulso das coisas que se movem em uma direção ou outra}
\end{EntryWithPhonetic}

\begin{EntryWithPhonetic}{渠}{qu2}{11}{⽊}
  \definition*{s.}{Sobrenome Qu}
  \definition{adj.}{Literário: grande}
  \definition{pron.}{Dialeto: ele; ela}
  \definition[条]{s.}{canal; vala; fosso; trincheira | borda externa da roda | escudo}
\end{EntryWithPhonetic}

\begin{EntryWithPhonetic}{渠道}{qu2dao4}{11,12}{⽊、⾡}[HSK 6]
  \definition[条,个,种]{s.}{vala de irrigação; os cursos de água escavados pelos trabalhadores para drenagem e irrigação | maneira; meio; caminho}
\end{EntryWithPhonetic}

\begin{EntryWithPhonetic}{取}{qu3}{8}{⼜}[HSK 2]
  \definition{v.}{pegar; obter; buscar; pegar de um lugar; pegar nas mãos | visar; procurar; obter; provocar | adotar; assumir; escolher; selecionar}
\end{EntryWithPhonetic}

\begin{EntryWithPhonetic}{取得}{qu3 de2}{8,11}{⼜、⼻}[HSK 2]
  \definition{v.}{ganhar; adquirir; obter; ser o primeiro a conseguir}
\end{EntryWithPhonetic}

\begin{EntryWithPhonetic}{取款}{qu3kuan3}{8,12}{⼜、⽋}[HSK 6]
  \definition{v.}{sacar dinheiro (de um banco); retirar o dinheiro que você depositou (geralmente se refere a retirar dinheiro do banco)}
\end{EntryWithPhonetic}

\begin{EntryWithPhonetic}{取款机}{qu3 kuan3 ji1}{8,12,6}{⼜、⽋、⽊}[HSK 6]
  \definition{s.}{ATM; caixa eletrônico; um caixa eletrônico é uma máquina que pode concluir automaticamente operações bancárias, como saques e consultas de saldo}
\end{EntryWithPhonetic}

\begin{EntryWithPhonetic}{取胜}{qu3sheng4}{8,9}{⼜、⾁}
  \definition{v.}{prevalecer sobre os oponentes | marcar uma vitória}
\end{EntryWithPhonetic}

\begin{EntryWithPhonetic}{取水}{qu3shui3}{8,4}{⼜、⽔}
  \definition{v.}{obter água (de um poço, etc.)}
\end{EntryWithPhonetic}

\begin{EntryWithPhonetic}{取现}{qu3xian4}{8,8}{⼜、⾒}
  \definition{v.}{sacar dinheiro}
\end{EntryWithPhonetic}

\begin{EntryWithPhonetic}{取消}{qu3xiao1}{8,10}{⼜、⽔}[HSK 3]
  \definition{v.}{cancelar; suspender; anular; abolir; revogar; rescindir; tornar o sistema original, regulamentos, qualificações, direitos, etc. inválidos}
\end{EntryWithPhonetic}

\begin{EntryWithPhonetic}{取悦}{qu3yue4}{8,10}{⼜、⼼}
  \definition{v.}{tentar agradar}
\end{EntryWithPhonetic}

\begin{EntryWithPhonetic}{厺}{qu4}{5}{⼤}
  \variantof{去}
\end{EntryWithPhonetic}

\begin{EntryWithPhonetic}{去}{qu4}{5}{⼛}[HSK 1]
  \definition{adj.}{passado; último; refere-se ao tempo passado (um ano)}
  \definition{adv.}{muito; extremamente; usado depois de adjetivos como 大, 多 e 远, significa 极 ou 非常}
  \definition{s.}{tom descendente, um dos quatro tons do chinês clássico e o quarto tom na pronúncia padrão do chinês moderno}
  \definition{v.}{ir; partir; sair | estar separado de | perder | remover; livrar-se de | ir (a algum lugar) para fazer algo; sair do local onde o interlocutor se encontra para outro lugar (oposto a 来) | ir para; estar indo para (fazer algo lá); usado antes de outro verbo para indicar fazer algo | desempenhar o papel de; representar o papel de; interpretar papéis em óperas | enviar; fazer ir; despachar}
  \definition{v.aux.}{usado entre uma frase verbal (ou frase preposicional) e um verbo para indicar que o primeiro é um método ou atitude e o último é um propósito | usado depois de um verbo para indicar que a ação está longe da localização do falante}
  \seealsoref{大}{da4}
  \seealsoref{多}{duo1}
  \seealsoref{非常}{fei1chang2}
  \seealsoref{极}{ji2}
  \seealsoref{来}{lai2}
  \seealsoref{远}{yuan3}
\end{EntryWithPhonetic}

\begin{EntryWithPhonetic}{去掉}{qu4 diao4}{5,11}{⼛、⼿}[HSK 6]
  \definition{v.}{livrar-se de; tirar; acabar com; abandonar; erradicar}
\end{EntryWithPhonetic}

\begin{EntryWithPhonetic}{去年}{qu4nian2}{5,6}{⼛、⼲}[HSK 1]
  \definition{s.}{ano passado}
\end{EntryWithPhonetic}

\begin{EntryWithPhonetic}{去世}{qu4shi4}{5,5}{⼛、⼀}[HSK 3]
  \definition{v.}{(usado apenas para adultos, com conotações solenes) morrer; falecer; deixar este mundo}
\end{EntryWithPhonetic}

\begin{EntryWithPhonetic}{去死}{qu4si3}{5,6}{⼛、⽍}
  \definition{interj.}{Caia morto! | Vá para o Inferno!}
\end{EntryWithPhonetic}

\begin{EntryWithPhonetic}{圈}{quan1}{11}{⼞}[HSK 4]
  \definition[个]{s.}{anel; círculo; refere-se a algo em forma de anel | domínio; grupo; escopo; círculo(s)}
  \definition{v.}{cercar; rodear; circundar | marcar com um círculo}
  \seeref{juan1}
  \seeref{juan4}
\end{EntryWithPhonetic}

\begin{EntryWithPhonetic}{圈粉}{quan1fen3}{11,10}{⼞、⽶}
  \definition{s.}{(neologismo, coloquial) ganhar alguém como fã, obter novos fãs}
\end{EntryWithPhonetic}

\begin{EntryWithPhonetic}{全}{quan2}{6}{⼊}[HSK 2]
  \definition*{s.}{Sobrenome Quan}
  \definition{adj.}{completo; total; inteiro}
  \definition{adv.}{inteiramente; totalmente; completamente; significa 100\%; equivalente a 完全 ou 全然}
  \definition{v.}{manter intacto; tornar perfeito ou completo; completar}
  \seealsoref{全然}{quan2ran2}
  \seealsoref{完全}{wan2quan2}
\end{EntryWithPhonetic}

\begin{EntryWithPhonetic}{全部}{quan2bu4}{6,10}{⼊、⾢}[HSK 2]
  \definition{adv.}{tudo; total; inteiro; completo; aplica-se a todos, sem exceção}
  \definition{s.}{totalidade; total; integridade; a soma de todas as partes; o todo}
\end{EntryWithPhonetic}

\begin{EntryWithPhonetic}{全场}{quan2 chang3}{6,6}{⼊、⼟}[HSK 3]
  \definition{s.}{toda a audiência; todos os presentes; todo o público}
\end{EntryWithPhonetic}

\begin{EntryWithPhonetic}{全称特命全权大使}{quan2cheng1 te4ming4 quan2quan2 da4shi3}{6,10,10,8,6,6,3,8}{⼊、⽲、⽜、⼝、⼊、⽊、⼤、⼈}
  \definition*{s.}{Embaixador Extraordinário e Plenipotenciário}
\end{EntryWithPhonetic}

\begin{EntryWithPhonetic}{全都}{quan2 dou1}{6,10}{⼊、⾢}[HSK 5]
  \definition{adv.}{tudo; todos; sem exceção}
\end{EntryWithPhonetic}

\begin{EntryWithPhonetic}{全都不}{quan2dou1 bu4}{6,10,4}{⼊、⾢、⼀}
  \definition{adj.}{nada; nenhum; nenhum deles; nada disso}
\end{EntryWithPhonetic}

\begin{EntryWithPhonetic}{全国}{quan2 guo2}{6,8}{⼊、⼞}[HSK 2]
  \definition{s.}{toda a nação (ou país); em todo o país; em todo o território nacional | toda a nação; todo o país}
\end{EntryWithPhonetic}

\begin{EntryWithPhonetic}{全家}{quan2 jia1}{6,10}{⼊、⼧}[HSK 2]
  \definition{s.}{toda a família; a família inteira}
\end{EntryWithPhonetic}

\begin{EntryWithPhonetic}{全力}{quan2 li4}{6,2}{⼊、⼒}[HSK 6]
  \definition{s.}{exercendo todos os seus esforços; energia ou força total; toda força ou energia}
\end{EntryWithPhonetic}

\begin{EntryWithPhonetic}{全面}{quan2mian4}{6,9}{⼊、⾯}[HSK 3]
  \definition{adj.}{geral; completo; abrangente; onipotente}
  \definition{s.}{todos os aspectos; cada aspecto}
  \seealsoref{片面}{pian4mian4}
\end{EntryWithPhonetic}

\begin{EntryWithPhonetic}{全年}{quan2 nian2}{6,6}{⼊、⼲}[HSK 2]
  \definition{s.}{ano inteiro | anual; todo ano}
\end{EntryWithPhonetic}

\begin{EntryWithPhonetic}{全球}{quan2 qiu2}{6,11}{⼊、⽟}[HSK 3]
  \definition[门]{s.}{o mundo inteiro; a Terra inteira}
\end{EntryWithPhonetic}

\begin{EntryWithPhonetic}{全然}{quan2ran2}{6,12}{⼊、⽕}
  \definition{adv.}{completamente; inteiramente}
\end{EntryWithPhonetic}

\begin{EntryWithPhonetic}{全身}{quan2 shen1}{6,7}{⼊、⾝}[HSK 2]
  \definition{s.}{corpo inteiro; por todo o corpo; todo o corpo}
\end{EntryWithPhonetic}

\begin{EntryWithPhonetic}{全世界}{quan2 shi4 jie4}{6,5,9}{⼊、⼀、⽥}[HSK 5]
  \definition[种]{s.}{mundo inteiro; mundo todo | em todo o mundo}
\end{EntryWithPhonetic}

\begin{EntryWithPhonetic}{全体}{quan2 ti3}{6,7}{⼊、⼈}[HSK 2]
  \definition{s.}{(frequentemente referido a pessoas) todos; número total; todos | por todo o corpo | todos; inteiro; a soma de todas as partes; a soma de todos os indivíduos (geralmente se refere a pessoas)}
\end{EntryWithPhonetic}

\begin{EntryWithPhonetic}{全新}{quan2 xin1}{6,13}{⼊、⽄}[HSK 6]
  \definition{adj.}{totalmente novo; inteiramente/completamente novo; refere-se a algo completamente novo, especialmente algo que não foi usado}
\end{EntryWithPhonetic}

\begin{EntryWithPhonetic}{全职}{quan2zhi2}{6,11}{⼊、⽿}
  \definition{s.}{período integral | tempo inteiro | (trabalho) \emph{full-time}}
\end{EntryWithPhonetic}

\begin{EntryWithPhonetic}{权}{quan2}{6}{⽊}[HSK 6]
  \definition*{s.}{Sobrenome Quan}
  \definition{adv.}{provisoriamente; por enquanto}
  \definition{s.}{Lliterário: contrapeso; peso deslizante de uma balança romana | poder; autoridade | direito | posição vantajosa | conveniência}
  \definition{v.}{pesar; medir o peso}
\end{EntryWithPhonetic}

\begin{EntryWithPhonetic}{权力}{quan2li4}{6,2}{⽊、⼒}[HSK 6]
  \definition[种]{s.}{poder; autoridade; o poder de liderança no âmbito da responsabilidade | poder; coerção política; o poder coercitivo do status social e político}
\end{EntryWithPhonetic}

\begin{EntryWithPhonetic}{权利}{quan2li4}{6,7}{⽊、⼑}[HSK 4]
  \definition[项,种,个,条,份]{s.}{direito; interesse; os poderes e benefícios (em oposição a 义务) exercidos por um cidadão ou pessoa jurídica de acordo com a lei}
  \seealsoref{义务}{yi4wu4}
\end{EntryWithPhonetic}

\begin{EntryWithPhonetic}{泉}{quan2}{9}{⽔}[HSK 5]
  \definition*{s.}{Sobrenome Quan}
  \definition[股,眼,汪]{s.}{fonte (de água mineral) | a nascente de um rio | termo antigo para moeda}
\end{EntryWithPhonetic}

\begin{EntryWithPhonetic}{拳}{quan2}{10}{⼿}
  \definition*{s.}{Sobrenome Quan}
  \definition[个,记,套]{s.}{punho | boxe; pugilismo}
  \definition{v.}{enrolar}
\end{EntryWithPhonetic}

\begin{EntryWithPhonetic}{拳法}{quan2fa3}{10,8}{⼿、⽔}
  \definition{s.}{boxe | luta}
\end{EntryWithPhonetic}

\begin{EntryWithPhonetic}{拳王}{quan2wang2}{10,4}{⼿、⽟}
  \definition{s.}{pugilista | boxeador}
\end{EntryWithPhonetic}

\begin{EntryWithPhonetic}{犬}{quan3}{4}{⽝}[Kangxi 94]
  \definition{s.}{cachorro}
\end{EntryWithPhonetic}

\begin{EntryWithPhonetic}{劝}{quan4}{4}{⼒}[HSK 5]
  \definition*{s.}{Sobrenome Quan}
  \definition{v.}{insistir; aconselhar; tentar persuadir; persuadir, argumentar para que as pessoas obedeçam | incentivar; encorajar}
\end{EntryWithPhonetic}

\begin{EntryWithPhonetic}{券}{quan4}{8}{⼑}[HSK 6]
  \definition[张]{s.}{certificado; bilhete; ingresso; uma conta ou pedaço de papel que serve como recibo}
\end{EntryWithPhonetic}

\begin{EntryWithPhonetic}{缺}{que1}{10}{⽸}[HSK 3]
  \definition{adj.}{incompleto; imperfeito}
  \definition[种]{s.}{vaga; abertura; falta}
  \definition{v.}{estar com falta de; faltar | estar ausente}
\end{EntryWithPhonetic}

\begin{EntryWithPhonetic}{缺点}{que1dian3}{10,9}{⽸、⽕}[HSK 3]
  \definition[个,些]{s.}{desvantagem; deficiência; inconveniência; ponto fraco; uma deficiência ou imperfeição (em oposição a 优点)}
  \seealsoref{优点}{you1dian3}
\end{EntryWithPhonetic}

\begin{EntryWithPhonetic}{缺乏}{que1fa2}{10,4}{⽸、⼃}[HSK 5]
  \definition{v.}{faltar; estar em falta de; não ter ou não ter totalmente (algo que deveria possuir ou é desejaria possuir)}
\end{EntryWithPhonetic}

\begin{EntryWithPhonetic}{缺勤}{que1/qin2}{10,13}{⽸、⼒}
  \definition{v.+compl.}{ausentar-se do dever (trabalho)}
\end{EntryWithPhonetic}

\begin{EntryWithPhonetic}{缺少}{que1shao3}{10,4}{⽸、⼩}[HSK 3]
  \definition{v.}{falta; estar com falta de; estar em falta de; geralmente se refere à falta de pessoas ou coisas}
\end{EntryWithPhonetic}

\begin{EntryWithPhonetic}{缺陷}{que1xian4}{10,10}{⽸、⾩}[HSK 6]
  \definition[个,处,项]{pron.}{defeito; falha; inconveniência; mancha; um lugar onde uma pessoa ou coisa está incompleta ou tem falhas porque algo está faltando}
\end{EntryWithPhonetic}

\begin{EntryWithPhonetic}{却}{que4}{7}{⼙}[HSK 4]
  \definition{adv.}{mas; contudo; no entanto; enquanto; indica um ponto de virada}
  \definition{v.}{recuar; retroceder | afastar; repelir; desencorajar | declinar; recusar; rejeitar}
  \definition{v.aux.}{usado depois de certos verbos para indicar a conclusão de uma ação, resultado, equivalente a 去 ou 掉}
  \seealsoref{掉}{diao4}
  \seealsoref{去}{qu4}
\end{EntryWithPhonetic}

\begin{EntryWithPhonetic}{却是}{que4 shi4}{7,9}{⼙、⽇}[HSK 6]
  \definition{conj.}{na verdade; no entanto; o fato é\dots; indica um ponto de virada, contrário às suas expectativas anteriores}
\end{EntryWithPhonetic}

\begin{EntryWithPhonetic}{确}{que4}{12}{⽯}
  \definition{adj.}{autenticado | sólido | firme | real | verdadeiro}
\end{EntryWithPhonetic}

\begin{EntryWithPhonetic}{确保}{que4bao3}{12,9}{⽯、⼈}[HSK 3]
  \definition{v.}{assegurar; garantir; manter ou garantir com certeza}
\end{EntryWithPhonetic}

\begin{EntryWithPhonetic}{确定}{que4ding4}{12,8}{⽯、⼧}[HSK 3]
  \definition{adj.}{definido; certo; claro}
  \definition{v.}{firmar; definir; determinar; tomar uma decisão clara e não mudar}
\end{EntryWithPhonetic}

\begin{EntryWithPhonetic}{确立}{que4li4}{12,5}{⽯、⽴}[HSK 5]
  \definition{v.}{estabelecer; criar; construir; estabelecer ou consolidar firmemente}
\end{EntryWithPhonetic}

\begin{EntryWithPhonetic}{确认}{que4ren4}{12,4}{⽯、⾔}[HSK 4]
  \definition{v.}{afirmar; confirmar; reconhecer; confirmar explicitamente (fatos, princípios, etc.)}
\end{EntryWithPhonetic}

\begin{EntryWithPhonetic}{确实}{que4shi2}{12,8}{⽯、⼧}[HSK 3]
  \definition{adj.}{verdadeiro; confiável; autêntico}
  \definition{adv.}{verdadeiramente; realmente; de ​​fato; afirmar a autenticidade de fatos objetivos}
\end{EntryWithPhonetic}

\begin{EntryWithPhonetic}{裙}{qun2}{12}{⾐}
  \definition[条]{s.}{saia | avental | algo como uma saia}
\end{EntryWithPhonetic}

\begin{EntryWithPhonetic}{裙子}{qun2zi5}{12,3}{⾐、⼦}[HSK 3]
  \definition[条,件]{s.}{saia (peça de vestuário); uma vestimenta usada abaixo da cintura}
\end{EntryWithPhonetic}

\begin{EntryWithPhonetic}{群}{qun2}{13}{⽺}[HSK 3]
  \definition*{s.}{Sobrenome Qun}
  \definition{adj.}{em grupos; numerosos}
  \definition{clas.}{usado para grupos de pessoas ou coisas; grupo; rebanho; manada}
  \definition{s.}{multidão; grupo; muitas pessoas ou coisas reunidas | as massas; um grupo de pessoas; refere-se a um grande número de pessoas}
\end{EntryWithPhonetic}

\begin{EntryWithPhonetic}{群山}{qun2shan1}{13,3}{⽺、⼭}
  \definition{s.}{montanhas | uma cadeia de colinas}
\end{EntryWithPhonetic}

\begin{EntryWithPhonetic}{群体}{qun2 ti3}{13,7}{⽺、⼈}[HSK 5]
  \definition[个]{s.}{colônia; um conjunto composto por muitos indivíduos da mesma espécie que estão fisicamente conectados, exemplos incluem corais entre os animais e certas algas entre as plantas | grupos; refere-se, de maneira geral, ao conjunto formado por muitos indivíduos interligados que compartilham características essenciais em comum}
\end{EntryWithPhonetic}

\begin{EntryWithPhonetic}{群众}{qun2zhong4}{13,6}{⽺、⼈}[HSK 5]
  \definition[个,名,位]{s.}{as massas; refere-se ao povo em geral | não filiado; apartidário; refere-se a pessoas que não são membros do Partido Comunista Chinês nem da Liga da Juventude Comunista | alguém que não ocupa uma posição de liderança}
\end{EntryWithPhonetic}

%%%%% EOF %%%%%


 %%%
%%% R
%%%

\section*{R}\addcontentsline{toc}{section}{R}

\begin{EntryWithPhonetic}{儿}{r5}{2}{⼉}[Kangxi 10]
  \definition{suf.}{sufixo diminutivo não silábico | final retroflexo, pronunciado como ``r'' | adicionado a substantivos para expressar pequenez  | adicionado a verbos, adjetivos e classificadores para formar substantivos | adicionado a substantivos para formar substantivos com significados diferentes | sufixos de alguns verbos | anexado após adjetivos duplicados}
  \seeref{er2}
\end{EntryWithPhonetic}

\begin{EntryWithPhonetic}{然}{ran2}{12}{⽕}
  \definition{conj.}{mas | no entanto}
\end{EntryWithPhonetic}

\begin{EntryWithPhonetic}{然而}{ran2'er2}{12,6}{⽕、⽽}[HSK 4]
  \definition{conj.}{ainda; mas; contudo; todavia; usado no início de uma frase para indicar uma transição; para indicar uma transição, geralmente é precedido por uma conjunção como 虽然 para indicar concessão}
  \seealsoref{虽然}{sui1 ran2}
\end{EntryWithPhonetic}

\begin{EntryWithPhonetic}{然后}{ran2hou4}{12,6}{⽕、⼝}[HSK 2]
  \definition{conj.}{então; depois disso; posteriormente; indica que algo segue após uma ação ou situação}
\end{EntryWithPhonetic}

\begin{EntryWithPhonetic}{燃}{ran2}{16}{⽕}
  \definition{v.}{queimar | acender; inflamar}
\end{EntryWithPhonetic}

\begin{EntryWithPhonetic}{燃料}{ran2 liao4}{16,10}{⽕、⽃}[HSK 4]
  \definition[种]{s.}{combustível; carburante; substâncias que podem gerar calor e energia luminosa quando queimadas podem ser divididas em três tipos de acordo com sua forma: combustível sólido (como carvão, carvão vegetal, madeira), combustível líquido (como gasolina, querosene) e combustível gasoso (como gás de carvão, biogás); também se refere a substâncias que podem gerar energia nuclear, como urânio, plutônio, etc.}
\end{EntryWithPhonetic}

\begin{EntryWithPhonetic}{燃烧}{ran2shao1}{16,10}{⽕、⽕}[HSK 4]
  \definition{v.}{queimar; acender | arder; inflamar; ferver; metáfora para as emoções de uma pessoa serem muito fortes, como um fogo ardente}
\end{EntryWithPhonetic}

\begin{EntryWithPhonetic}{染}{ran3}{9}{⽊}[HSK 5]
  \definition*{s.}{Sobrenome Ran}
  \definition{s.}{soja fermentada e temperada em forma de pasta}
  \definition{v.}{tingir; pintar | pegar (uma doença); cair em (um mau hábito, etc.) | sujar; contaminar | pegar (contrair) (uma doença) | adquirir (um mau hábito, etc.); contaminar}
\end{EntryWithPhonetic}

\begin{EntryWithPhonetic}{壤}{rang3}{20}{⼟}
  \definition{s.}{solo | terra | (literário) a terra (em contraste com o céu 天)}
\end{EntryWithPhonetic}

\begin{EntryWithPhonetic}{让}{rang4}{5}{⾔}[HSK 2]
  \definition*{s.}{Sobrenome Rang}
  \definition{prep.}{em uma frase passiva para introduzir o executor da ação | de acordo com; em conformidade com; à luz de; com base em; usado para expressar a opinião subjetiva de alguém}
  \definition{v.}{ceder; recuar; render-se; desistir; admitir | convidar; oferecer | deixar; permitir; fazer | deixar alguém ter algo por um preço justo | ser inferior a; não ser tão bom quanto | ceder; afastar-se | expressar desejos | esquivar-se; evitar; fugir | Usado antes de 我们, indica uma ordem ou sugestão para que todos façam algo juntos}
  \seealsoref{我们}{wo3men5}
\end{EntryWithPhonetic}

\begin{EntryWithPhonetic}{让步}{rang4/bu4}{5,7}{⾔、⽌}
  \definition{v.+compl.}{fazer uma concessão | entregar | desistir | comprometer}
\end{EntryWithPhonetic}

\begin{EntryWithPhonetic}{让座}{rang4 zuo4}{5,10}{⾔、⼴}[HSK 6]
  \definition{v.}{oferecer seu lugar a alguém; ceder seu lugar a alguém | convidar os convidados para se sentarem}
\end{EntryWithPhonetic}

\begin{EntryWithPhonetic}{绕}{rao4}{9}{⽷}[HSK 5]
  \definition*{s.}{Sobrenome Rao}
  \definition{v.}{enrolar; bobinar; rebobinar | mover-se em círculo; girar; revolver | fazer um desvio; contornar; dar a volta | confundir; desorientar}
\end{EntryWithPhonetic}

\begin{EntryWithPhonetic}{热}{re4}{10}{⽕}[HSK 1]
  \definition{adj.}{quente; temperatura elevada | ardente; caloroso; profundamente afetuoso | ansioso; invejoso; descreve inveja e desejo de possuir algo | térmico; altamente radioativo | popular; muito procurado; muito apreciado por muitas pessoas}
  \definition{s.}{calor; energia liberada pelo movimento irregular das moléculas dentro de um objeto | febre; febre alta causada por doença | moda passageira; mania; febre}
  \definition{v.}{aquecer (geralmente se refere a alimentos)}
\end{EntryWithPhonetic}

\begin{EntryWithPhonetic}{热爱}{re4'ai4}{10,10}{⽕、⽖}[HSK 3]
  \definition{v.}{amar ardentemente; amar de coração; ter amor profundo por; amar apaixonadamente}
\end{EntryWithPhonetic}

\begin{EntryWithPhonetic}{热点}{re4 dian3}{10,9}{⽕、⽕}[HSK 6]
  \definition{s.}{ponto de acesso; \emph{hotspot}}
\end{EntryWithPhonetic}

\begin{EntryWithPhonetic}{热泪盈眶}{re4lei4ying2kuang4}{10,8,9,11}{⽕、⽔、⽫、⽬}
  \definition{expr.}{olhos cheios de lágrimas de emoção | extremamente emocionado}
\end{EntryWithPhonetic}

\begin{EntryWithPhonetic}{热量}{re4 liang4}{10,12}{⽕、⾥}[HSK 5]
  \definition{s.}{calor; quantidade de calor; calorias; em física, refere-se à energia transferida entre objetos com temperaturas diferentes, do objeto com temperatura mais alta para o objeto com temperatura mais baixa}
\end{EntryWithPhonetic}

\begin{EntryWithPhonetic}{热烈}{re4lie4}{10,10}{⽕、⽕}[HSK 3]
  \definition{adj.}{caloroso; fervoroso; ardente; entusiasmado; excitado}
\end{EntryWithPhonetic}

\begin{EntryWithPhonetic}{热门}{re4men2}{10,3}{⽕、⾨}[HSK 5]
  \definition{adj.}{popular}
  \definition{s.}{algo que desperta o interesse popular; metáfora para algo que está na moda e recebe a atenção de todos (em contraste com 冷门)}
  \seealsoref{冷门}{leng3men2}
\end{EntryWithPhonetic}

\begin{EntryWithPhonetic}{热闹}{re4nao5}{10,8}{⽕、⾾}[HSK 4]
  \definition{adj.}{animado; agitado; movimentado com barulho e excitação; descreve uma cena animada com uma atmosfera calorosa}
  \definition{s.}{uma vista emocionante; uma cena de agitação e excitação; atmosfera acolhedora}
  \definition{v.}{animar; divertir-se}
\end{EntryWithPhonetic}

\begin{EntryWithPhonetic}{热情}{re4qing2}{10,11}{⽕、⼼}[HSK 2]
  \definition{adj.}{caloroso; fervoroso; entusiasmado; cordial; descreve sentimentos calorosos por alguém}
  \definition{s.}{entusiasmo; ardor; devoção; calor humano; zelo; sentimentos calorosos}
\end{EntryWithPhonetic}

\begin{EntryWithPhonetic}{热水}{re4 shui3}{10,4}{⽕、⽔}[HSK 6]
  \definition{s.}{água quente; água em temperatura mais alta}
\end{EntryWithPhonetic}

\begin{EntryWithPhonetic}{热水器}{re4 shui3 qi4}{10,4,16}{⽕、⽔、⼝}[HSK 6]
  \definition[台]{s.}{aquecedor de água; aparelhos que aquecem água usando eletricidade, gás natural, gás liquefeito de petróleo ou energia solar}
\end{EntryWithPhonetic}

\begin{EntryWithPhonetic}{热线}{re4 xian4}{10,8}{⽕、⽷}[HSK 6]
  \definition[条]{s.}{raio infravermelho | linha direta; \emph{hot line}; uma linha telefônica ou telegráfica direta; uma linha para um ponto de acesso | rota quente (ou movimentada, popular) | raio de calor}
\end{EntryWithPhonetic}

\begin{EntryWithPhonetic}{热心}{re4xin1}{10,4}{⽕、⼼}[HSK 4]
  \definition{adj.}{ardente; sincero; entusiasmado; afetuoso; apaixonado; interessado}
  \definition{v.}{ser entusiasmado com alguma coisa}
\end{EntryWithPhonetic}

\begin{EntryWithPhonetic}{热血沸腾}{re4xue4-fei4teng2}{10,6,8,13}{⽕、⾎、⽔、⾁}
  \definition{expr.}{estar animado; ter o sangue correndo}
\end{EntryWithPhonetic}

\begin{EntryWithPhonetic}{人}{ren2}{2}{⼈}[HSK 1][Kangxi 9]
  \definition*{s.}{Sobrenome Ren}
  \definition[个,名,位]{s.}{homem; pessoa; pessoas; ser humano | todos; cada um; todo mundo | adulto; crescido | uma pessoa envolvida em uma atividade específica | pessoas; outras pessoas | caráter; personalidade; qualidade, caráter ou reputação de uma pessoa | como alguém se sente; estado de saúde de alguém | mão de obra; força de trabalho}
\end{EntryWithPhonetic}

\begin{EntryWithPhonetic}{人才}{ren2cai2}{2,3}{⼈、⼿}[HSK 3]
  \definition{adj.}{aparência bonita, elegante}
  \definition[个,些,位]{s.}{talento; pessoal qualificado; pessoa com capacidade; uma pessoa com capacidade e integridade política; uma pessoa com talentos especiais | aparência bonita; refere-se à aparência; especialmente à aparência bonita}
\end{EntryWithPhonetic}

\begin{EntryWithPhonetic}{人材}{ren2cai2}{2,7}{⼈、⽊}
  \variantof{人才}
\end{EntryWithPhonetic}

\begin{EntryWithPhonetic}{人道}{ren2dao4}{2,12}{⼈、⾡}
  \definition{s.}{solidariedade humana | humanitarismo | humano | a ``maneira humana'', um dos estágios do ciclo de reencarnação (budismo) | relação sexual}
\end{EntryWithPhonetic}

\begin{EntryWithPhonetic}{人工}{ren2gong1}{2,3}{⼈、⼯}[HSK 3]
  \definition{adj.}{feito pelo homem; artificial (oposto a 天然)}
  \definition[个]{s.}{trabalho manual; trabalho feito à mão | mão de obra; homem-dia; uma unidade de cálculo da quantidade de trabalho realizado}
  \seealsoref{天然}{tian1 ran2}
\end{EntryWithPhonetic}

\begin{EntryWithPhonetic}{人工智能}{ren2 gong1 zhi4 neng2}{2,3,12,10}{⼈、⼯、⽇、⾁}
  \definition*{s.}{Inteligência Artificial (IA)}
\end{EntryWithPhonetic}

\begin{EntryWithPhonetic}{人海}{ren2hai3}{2,10}{⼈、⽔}
  \definition{s.}{uma multidão | um mar de pessoas}
\end{EntryWithPhonetic}

\begin{EntryWithPhonetic}{人家}{ren2jia1}{2,10}{⼈、⼧}[HSK 4]
  \definition[户,个]{s.}{lar; família; família do noivo; casa do futuro marido}
  \seeref{ren2jia5}
\end{EntryWithPhonetic}

\begin{EntryWithPhonetic}{人家}{ren2jia5}{2,10}{⼈、⼧}
  \definition{pron.}{outros; uma pessoa ou pessoas diferentes do falante ou ouvinte; refere-se a alguém diferente de si mesmo ou de outra pessoa | certa pessoa ou pessoas (a pessoa ou pessoas mencionadas em um contexto próximo, aproximadamente equivalente ao pronome de terceira pessoa);  refere-se a uma pessoa ou algumas pessoas, com significado semelhante a 他 | eu; mim (usado retoricamente no lugar do primeiro pronome pessoal, muitas vezes expressando descontentamento de forma jocosa; geralmente usado quando se fala com pessoas próximas, para significar 自己, usado principamente por meninas)}
  \seeref{ren2jia1}
  \seealsoref{他}{ta1}
  \seealsoref{自己}{zi4ji3}
\end{EntryWithPhonetic}

\begin{EntryWithPhonetic}{人间}{ren2jian1}{2,7}{⼈、⾨}[HSK 5]
  \definition{s.}{o mundo humano; o Mundo; a Terra}
\end{EntryWithPhonetic}

\begin{EntryWithPhonetic}{人口}{ren2kou3}{2,3}{⼈、⼝}[HSK 2]
  \definition[个,群]{s.}{população; o número total de pessoas que vivem em uma determinada região durante um determinado período de tempo | número de membros da família; o número total de pessoas em uma família | pessoas; público; população; referência geral a pessoas | rumores do povo; referindo-se à opinião pública}
\end{EntryWithPhonetic}

\begin{EntryWithPhonetic}{人类}{ren2lei4}{2,9}{⼈、⽶}[HSK 3]
  \definition[种]{s.}{humano; humanidade; raça humana; um termo geral para pessoas}
\end{EntryWithPhonetic}

\begin{EntryWithPhonetic}{人力}{ren2 li4}{2,2}{⼈、⼒}[HSK 5]
  \definition{s.}{mão de obra; trabalho manual; força de trabalho}
\end{EntryWithPhonetic}

\begin{EntryWithPhonetic}{人力车}{ren2 li4 che1}{2,2,4}{⼈、⼒、⾞}
  \definition{s.}{veículo de duas rodas puxado ou empurrado por um homem (oposto a 兽力车 e 机动车) | Datado: riquixá | uma carroça puxada ou empurrada por humanos}
  \seealsoref{机动车}{ji1 dong4 che1}
  \seealsoref{兽力车}{shou4 li4 che1}
\end{EntryWithPhonetic}

\begin{EntryWithPhonetic}{人们}{ren2 men5}{2,5}{⼈、⼈}[HSK 2]
  \definition{s.}{homens; pessoas; o público; referindo-se a muitas pessoas; todos}
\end{EntryWithPhonetic}

\begin{EntryWithPhonetic}{人民}{ren2 min2}{2,5}{⼈、⽒}[HSK 3]
  \definition[群,批,个,国]{s.}{o povo; refere-se a um certo tipo de pessoas; membros básicos da sociedade com as massas trabalhadoras como o corpo principal}
\end{EntryWithPhonetic}

\begin{EntryWithPhonetic}{人民币}{ren2min2bi4}{2,5,4}{⼈、⽒、⼱}[HSK 3]
  \definition*[块,张,元]{s.}{Renminbi (RMB); Yuan Chinês (CYN); nome da moeda chinesa}
\end{EntryWithPhonetic}

\begin{EntryWithPhonetic}{人权}{ren2quan2}{2,6}{⼈、⽊}[HSK 6]
  \definition{s.}{direitos humanos}[最基本的人权是生存权。===O direito humano mais básico é o direito à vida.]
  \seealsoref{人权法}{ren2quan2fa3}
\end{EntryWithPhonetic}

\begin{EntryWithPhonetic}{人权法}{ren2quan2fa3}{2,6,8}{⼈、⽊、⽔}
  \definition*{s.}{Direitos Humanos}
  \seealsoref{人权}{ren2quan2}
\end{EntryWithPhonetic}

\begin{EntryWithPhonetic}{人群}{ren2 qun2}{2,13}{⼈、⽺}[HSK 3]
  \definition[个,类]{s.}{multidão; ajuntamento; torpel; aglomeração; um grupo de pessoas}
\end{EntryWithPhonetic}

\begin{EntryWithPhonetic}{人生}{ren2sheng1}{2,5}{⼈、⽣}[HSK 3]
  \definition{s.}{vida; sobrevivência e vida humana}
\end{EntryWithPhonetic}

\begin{EntryWithPhonetic}{人士}{ren2shi4}{2,3}{⼈、⼠}[HSK 5]
  \definition{s.}{pessoa; figura; personalidade; figura pública; pessoas com certa influência social}
\end{EntryWithPhonetic}

\begin{EntryWithPhonetic}{人数}{ren2 shu4}{2,13}{⼈、⽁}[HSK 2]
  \definition{s.}{número de pessoas; significa o número total de pessoas, uma quantidade de pessoas; normalmente, usa-se números para fazer estatísticas específicas, mas às vezes também se usa um intervalo aproximado para fazer estimativas}
\end{EntryWithPhonetic}

\begin{EntryWithPhonetic}{人物}{ren2wu4}{2,8}{⼈、⽜}[HSK 5]
  \definition[个,位,名]{s.}{personagem; personagens criados em obras literárias e artísticas | figura; personalidade; homem influente; refere-se a pessoas com grande talento e status; também se refere a pessoas com certas características ou que são representativas em algum aspecto | pintura figurativa; um tipo de pintura tradicional chinesa com personagens como tema}
\end{EntryWithPhonetic}

\begin{EntryWithPhonetic}{人像}{ren2xiang4}{2,13}{⼈、⼈}
  \definition{s.}{``retrato'' de uma pessoa (esboço, foto, escultura, etc.)}
\end{EntryWithPhonetic}

\begin{EntryWithPhonetic}{人行道}{ren2xing2dao4}{2,6,12}{⼈、⾏、⾡}
  \definition{s.}{calçada}
\end{EntryWithPhonetic}

\begin{EntryWithPhonetic}{人鱼}{ren2yu2}{2,8}{⼈、⿂}
  \definition{s.}{sereia | peixe-boi | salamandra gigante}
\end{EntryWithPhonetic}

\begin{EntryWithPhonetic}{人员}{ren2yuan2}{2,7}{⼈、⼝}[HSK 3]
  \definition[个,位,名]{s.}{funcionários ; uma pessoa que ocupa uma determinada posição| pessoal; membros de um grupo}
\end{EntryWithPhonetic}

\begin{EntryWithPhonetic}{忍}{ren3}{7}{⼼}[HSK 5]
  \definition{v.}{suportar; aguentar; tolerar; aturar | ter coragem para; ser insensível o suficiente para; ser capaz de endurecer o coração e fazer coisas que não se devem fazer por uma questão de razão}
\end{EntryWithPhonetic}

\begin{EntryWithPhonetic}{忍不住}{ren3bu5zhu4}{7,4,7}{⼼、⼀、⼈}[HSK 5]
  \definition{v.}{incapaz de suportar; não conseguir evitar fazer algo; não conseguir se controlar}
\end{EntryWithPhonetic}

\begin{EntryWithPhonetic}{忍耐}{ren3nai4}{7,9}{⼼、⽽}
  \definition{s.}{paciência | resistência}
  \definition{v.}{suportar | resistir | exercer paciência}
\end{EntryWithPhonetic}

\begin{EntryWithPhonetic}{忍受}{ren3shou4}{7,8}{⼼、⼜}[HSK 5]
  \definition{v.}{suportar; sofrer; aguentar; tolerar; suportar com dificuldade o sofrimento, as dificuldades e as adversidades da vida}
\end{EntryWithPhonetic}

\begin{EntryWithPhonetic}{认}{ren4}{4}{⾔}[HSK 5]
  \definition{v.}{reconhecer; saber; distinguir; identificar | estabelecer uma determinada relação com; adotar | admitir; reconhecer; assumir | comprometer-se a fazer algo | (frequentemente seguido por 了) aceitar como inevitável; resignar-se}
  \seealsoref{了}{le5}
\end{EntryWithPhonetic}

\begin{EntryWithPhonetic}{认出}{ren4 chu1}{4,5}{⾔、⼐}[HSK 3]
  \definition{v.}{reconhecer; identificar; reconhecer alguém ou algo pela observação ou memória}
\end{EntryWithPhonetic}

\begin{EntryWithPhonetic}{认得}{ren4 de5}{4,11}{⾔、⼻}[HSK 3]
  \definition{v.}{saber; reconhecer; capacidade de confirmar a pessoa ou coisa que você vê}
\end{EntryWithPhonetic}

\begin{EntryWithPhonetic}{认定}{ren4ding4}{4,8}{⾔、⼧}[HSK 5]
  \definition{v.}{afirmar; manter; acreditar firmemente; considerar com certeza | decidir-se por algo; confirmar; chegar a uma conclusão afirmativa}
\end{EntryWithPhonetic}

\begin{EntryWithPhonetic}{认可}{ren4ke3}{4,5}{⾔、⼝}[HSK 3]
  \definition{v.}{aceitar; aprovar; confirmar; dar força legal a | permitir; concordar}
\end{EntryWithPhonetic}

\begin{EntryWithPhonetic}{认识}{ren4shi5}{4,7}{⾔、⾔}[HSK 1]
  \definition[份]{s.}{cognição; conhecimento; compreensão; refere-se à reflexão da mente humana sobre o mundo objetivo}
  \definition{v.}{saber; compreender; reconhecer}
\end{EntryWithPhonetic}

\begin{EntryWithPhonetic}{认同}{ren4 tong2}{4,6}{⾔、⼝}[HSK 6]
  \definition{v.}{identificar; pensar que a outra pessoa tem algo em comum com você | aprovar; reconhecer}
\end{EntryWithPhonetic}

\begin{EntryWithPhonetic}{认为}{ren4wei2}{4,4}{⾔、⼂}[HSK 2]
  \definition{v.}{pensar; considerar; manter; julgar; formar uma opinião sobre uma pessoa ou coisa, fazer um julgamento}
\end{EntryWithPhonetic}

\begin{EntryWithPhonetic}{认真}{ren4zhen1}{4,10}{⾔、⼗}[HSK 1]
  \definition{adj.}{sério; sério e meticuloso}
  \definition{adv.}{seriamente}
  \definition{v.}{levar algo a sério; considerar como verdadeiro; levar a sério}
\end{EntryWithPhonetic}

\begin{EntryWithPhonetic}{任}{ren4}{6}{⼈}[HSK 3]
  \definition{clas.}{usado para o número de mandatos cumpridos em um cargo oficial}
  \definition{conj.}{não importa (como, o que, etc.); orações de conexão, ou usadas antes de pronomes interrogativos, para expressar incondicionalidade, equivalente a 不管 ou 无论}
  \definition{s.}{escritório; posto oficial; cargo | dever; fardo; responsabilidade}
  \definition{v.}{nomear; designar alguém para um cargo | assumir um emprego; assumir um posto; assumir uma posição | deixar; permitir; dar rédea solta a | suportar; empreender | ceder; permitir sem restrições; deixar (alguém) fazer o que quiser}
  \seealsoref{不管}{bu4guan3}
  \seealsoref{无论}{wu2lun4}
\end{EntryWithPhonetic}

\begin{EntryWithPhonetic}{任何}{ren4he2}{6,7}{⼈、⼈}[HSK 3]
  \definition{pron.}{qualquer; qualquer que seja; o que for; não importa o que}
\end{EntryWithPhonetic}

\begin{EntryWithPhonetic}{任凭}{ren4 ping2}{6,8}{⼈、⼏}
  \definition{conj.}{não importa (como, o quê, etc.) | mesmo que; embora}
  \definition{v.}{permitir; deixar (algo como: fazer o que lhe agrada); conforme a conveniência de alguém}
\end{EntryWithPhonetic}

\begin{EntryWithPhonetic}{任务}{ren4wu5}{6,5}{⼈、⼒}[HSK 3]
  \definition[项,个,种,些]{s.}{tarefa; dever; missão; designação; trabalho designado; responsabilidades designadas}
\end{EntryWithPhonetic}

\begin{EntryWithPhonetic}{韧}{ren4}{7}{⾱}
  \definition{adj.}{flexível, mas forte; tenaz; resistente (oposto a 脆) | resistente; macio e forte, não quebra facilmente (ao contrário de 脆)}
  \seealsoref{脆}{cui4}
\end{EntryWithPhonetic}

\begin{EntryWithPhonetic}{扔}{reng1}{5}{⼿}[HSK 5]
  \definition{v.}{arremessar; lançar; atirar; jogar | esquecer; jogar fora; descartar | colocar casualmente; deixar as pessoas ou as coisas de lado, não se importar}
\end{EntryWithPhonetic}

\begin{EntryWithPhonetic}{扔掉}{reng1diao4}{5,11}{⼿、⼿}
  \definition{v.}{jogar fora}
\end{EntryWithPhonetic}

\begin{EntryWithPhonetic}{扔弃}{reng1qi4}{5,7}{⼿、⼶}
  \definition{v.}{abandonar | descartar | jogar fora}
\end{EntryWithPhonetic}

\begin{EntryWithPhonetic}{扔下}{reng1xia4}{5,3}{⼿、⼀}
  \definition{v.}{lançar (uma bomba) | derrubar}
\end{EntryWithPhonetic}

\begin{EntryWithPhonetic}{仍}{reng2}{4}{⼈}[HSK 3]
  \definition{adv.}{ainda; repetidamente; frequentemente; continuamente}
  \definition{v.}{permanecer}
\end{EntryWithPhonetic}

\begin{EntryWithPhonetic}{仍旧}{reng2jiu4}{4,5}{⼈、⽇}[HSK 5]
  \definition{adv.}{ainda; ainda assim; contudo}
  \definition{v.}{permanecer igual; continuar sendo}
\end{EntryWithPhonetic}

\begin{EntryWithPhonetic}{仍然}{reng2ran2}{4,12}{⼈、⽕}[HSK 3]
  \definition{adv.}{ainda; contudo; como antes; indica que a situação continua inalterada ou retorna ao seu estado original}
\end{EntryWithPhonetic}

\begin{EntryWithPhonetic}{日}{ri4}{4}{⽇}[HSK 1][Kangxi 72]
  \definition*{s.}{Japão, abreviação de 日本}
  \definition{clas.}{usado para contar o número de dias}
  \definition{s.}{sol | dia (em oposição a 夜); período diurno | diariamente; todos os dias; a cada dia que passa | um dia específico; dia especial | tempo; refere-se a um período de tempo | dia; uma rotação da Terra}
  \seealsoref{日本}{ri4ben3}
  \seealsoref{夜}{ye4}
\end{EntryWithPhonetic}

\begin{EntryWithPhonetic}{日报}{ri4 bao4}{4,7}{⽇、⼿}[HSK 2]
  \definition[份,种]{s.}{diário; jornais diários; jornal publicado todas as manhãs}
\end{EntryWithPhonetic}

\begin{EntryWithPhonetic}{日本}{ri4ben3}{4,5}{⽇、⽊}
  \definition*{s.}{Japão}
\end{EntryWithPhonetic}

\begin{EntryWithPhonetic}{日本人}{ri4ben3ren2}{4,5,2}{⽇、⽊、⼈}
  \definition{s.}{japonês | pessoa ou povo do Japão}
\end{EntryWithPhonetic}

\begin{EntryWithPhonetic}{日常}{ri4chang2}{4,11}{⽇、⼱}[HSK 3]
  \definition{adj.}{usual; diário; cotidiano; dia a dia; pertencem ao habitual}
\end{EntryWithPhonetic}

\begin{EntryWithPhonetic}{日出}{ri4chu1}{4,5}{⽇、⼐}
  \definition{s.}{nascer do sol}
  \seealsoref{夕阳}{xi1yang2}
\end{EntryWithPhonetic}

\begin{EntryWithPhonetic}{日光灯}{ri4guang1deng1}{4,6,6}{⽇、⼉、⽕}
  \definition{s.}{lâmpada fluorescente}
\end{EntryWithPhonetic}

\begin{EntryWithPhonetic}{日记}{ri4ji4}{4,5}{⽇、⾔}[HSK 4]
  \definition[本,篇,册]{s.}{diário; artigo que registra eventos e pensamentos diários}
\end{EntryWithPhonetic}

\begin{EntryWithPhonetic}{日历}{ri4li4}{4,4}{⽇、⼚}[HSK 4]
  \definition[张,本]{s.}{caledário; livro com o ano, mês, dia, semana, termo solar, aniversário, etc. registrados, um livro por ano, uma página por dia, aberto diariamente}
\end{EntryWithPhonetic}

\begin{EntryWithPhonetic}{日期}{ri4qi1}{4,12}{⽇、⽉}[HSK 1]
  \definition[个,段]{s.}{data; a data ou período específico em que algo aconteceu}
\end{EntryWithPhonetic}

\begin{EntryWithPhonetic}{日心说}{ri4 xin1 shuo1}{4,4,9}{⽇、⼼、⾔}
  \definition{s.}{teoria heliocêntrica | a teoria de que o sol está no centro do universo}
\end{EntryWithPhonetic}

\begin{EntryWithPhonetic}{日夜}{ri4 ye4}{4,8}{⽇、⼣}[HSK 6]
  \definition{s.}{dia e noite; noite e dia; 24 horas por dia}
\end{EntryWithPhonetic}

\begin{EntryWithPhonetic}{日语}{ri4 yu3}{4,9}{⽇、⾔}[HSK 6]
  \definition{s.}{japonês; língua japonesa}
\end{EntryWithPhonetic}

\begin{EntryWithPhonetic}{日子}{ri4zi5}{4,3}{⽇、⼦}[HSK 2]
  \definition[个,段,些,番]{s.}{dia; data; referência a uma data específica | dias; tempo; referência ao número de dias e horas | vida; subsistência; refere-se à vida ou ao sustento}
\end{EntryWithPhonetic}

\begin{EntryWithPhonetic}{荣}{rong2}{9}{⾋}
  \definition*{s.}{Sobrenome Rong}
  \definition{adj.}{próspero; florescente | exuberante | glorioso}
  \definition{s.}{honra; glória (oposto a 辱) | guarda-sol chinês | flor; flor de planta herbácea | beirais virados para cima}
  \definition{v.}{glorificar; luxuriar; crescer abundantemente; florescer | florescer | lançar}
  \seealsoref{辱}{ru3}
\end{EntryWithPhonetic}

\begin{EntryWithPhonetic}{容}{rong2}{10}{⼧}
  \definition*{s.}{Sobrenome Rong}
  \definition{adv.}{talvez; provavelmente; possivelmente}
  \definition{s.}{expressão facial e tez | aparência; o estado ou condição das coisas}
  \definition{v.}{permitir; quando os outros querem fazer algo, deixe-os fazer | tolerar; ser capaz de aceitar pessoas ou coisas com as quais você não está satisfeito | conter (número de pessoas ou coisas que podem ser colocadas em um determinado espaço)}
\end{EntryWithPhonetic}

\begin{EntryWithPhonetic}{容貌}{rong2mao4}{10,14}{⼧、⾘}
  \definition{s.}{aparência | aspecto | características}
\end{EntryWithPhonetic}

\begin{EntryWithPhonetic}{容易}{rong2yi4}{10,8}{⼧、⽇}[HSK 3]
  \definition{adj.}{fácil; simples; sem complicações | provável; passível; inclinado; indica uma alta probabilidade de algo acontecer}
\end{EntryWithPhonetic}

\begin{EntryWithPhonetic}{融}{rong2}{16}{⿀}
  \definition*{s.}{Sobrenome Rong}
  \definition{adj.}{permanente; longo prazo; duradouro | muito brilhante | circulante; corrente}
  \definition{s.}{fogo | plena luz do dia}
  \definition{v.}{derreter; descongelar | misturar; fundir; estar em harmonia | circular (dinheiro, etc.)}
\end{EntryWithPhonetic}

\begin{EntryWithPhonetic}{融合}{rong2 he2}{16,6}{⿀、⼝}[HSK 6]
  \definition{v.}{fundir; mesclar; misturar; combinar várias coisas diferentes em uma}
\end{EntryWithPhonetic}

\begin{EntryWithPhonetic}{融入}{rong2 ru4}{16,2}{⿀、⼊}[HSK 6]
  \definition{v.}{integrar em; juntar-se, integrar-se ao grupo | misturar-se; enfatizar a mistura e a combinação com o ambiente circundante para se tornar harmonioso e consistente | encher com (um certo sentimento); imbuir com (uma certa qualidade); preparar (chá, ervas, etc.); imergir; infundir (drogas, etc.)}
\end{EntryWithPhonetic}

\begin{EntryWithPhonetic}{柔}{rou2}{9}{⽊}
  \definition*{s.}{Sobrenome Rou}
  \definition{adj.}{macio; flexível; maleável | gentil; flexível; brando}
  \definition{v.}{tornar macio; amolecer | apaziguar}
\end{EntryWithPhonetic}

\begin{EntryWithPhonetic}{柔软}{rou2ruan3}{9,8}{⽊、⾞}
  \definition{adj.}{macio | suave}
\end{EntryWithPhonetic}

\begin{EntryWithPhonetic}{揉}{rou2}{12}{⼿}
  \definition{v.}{amassar | massagear | esfregar}
\end{EntryWithPhonetic}

\begin{EntryWithPhonetic}{揉碎}{rou2sui4}{12,13}{⼿、⽯}
  \definition{v.}{esmagar | desintegrar-se em pedaços}
\end{EntryWithPhonetic}

\begin{EntryWithPhonetic}{肉}{rou4}{6}{⾁}[HSK 1][Kangxi 130]
  \definition{adj.}{não crocante; mole | lento (em movimento); preguiçoso | carnal; erótico}
  \definition[块]{s.}{carne (especialmente carne de porco) | carne | polpa (da fruta)}
\end{EntryWithPhonetic}

\begin{EntryWithPhonetic}{肉桂}{rou4gui4}{6,10}{⾁、⽊}
  \definition{s.}{canela (árvore) | casca seca desta árvore; canela (uma especiaria aromática) | canela chinesa; cássia}
  \seealsoref{官桂}{guan1gui4}
\end{EntryWithPhonetic}

\begin{EntryWithPhonetic}{如}{ru2}{6}{⼥}[HSK 6]
  \definition{adv.}{por exemplo; tal como; como}
  \definition{conj.}{se; no caso (de); no caso de; como se; como}
  \definition{prep.}{em conformidade com; de acordo com}
  \definition{v.}{estar em conformidade (ou de acordo) com | (geralmente no negativo) pode ser comparado com; ser comparável a; ser tão bom quanto | superar; exceder | (literário) ir para}
\end{EntryWithPhonetic}

\begin{EntryWithPhonetic}{如此}{ru2 ci3}{6,6}{⼥、⽌}[HSK 5]
  \definition{adv.}{assim; tal; dessa forma; dessa maneira; refere-se a uma situação mencionada anteriormente, equivalente a 这样}
  \seealsoref{这样}{zhe4 yang4}
\end{EntryWithPhonetic}

\begin{EntryWithPhonetic}{如果}{ru2guo3}{6,8}{⼥、⽊}[HSK 2]
  \definition{conj.}{se; no caso de; na eventualidade de; supondo que; para expressar suposições, pode-se usar 要是 na linguagem falada.}
  \seealsoref{要是}{yao4shi5}
\end{EntryWithPhonetic}

\begin{EntryWithPhonetic}{如何}{ru2he2}{6,7}{⼥、⼈}[HSK 3]
  \definition{pron.}{como?; o que?; usado para perguntar como resolver um problema | como?; o que?; usado para perguntar sobre a situação ou obter a opinião de outras pessoas}
\end{EntryWithPhonetic}

\begin{EntryWithPhonetic}{如画}{ru2hua4}{6,8}{⼥、⽥}
  \definition{adj.}{pitoresco}
\end{EntryWithPhonetic}

\begin{EntryWithPhonetic}{如今}{ru2jin1}{6,4}{⼥、⼈}[HSK 4]
  \definition{s.}{agora; hoje em dia; atualmente; no presente}
\end{EntryWithPhonetic}

\begin{EntryWithPhonetic}{如同}{ru2 tong2}{6,6}{⼥、⼝}[HSK 5]
  \definition{v.}{parecer que; usado principalmente em metáforas}
\end{EntryWithPhonetic}

\begin{EntryWithPhonetic}{如下}{ru2 xia4}{6,3}{⼥、⼀}[HSK 5]
  \definition{adv.}{como descrito ou listado abaixo; conforme segue; conforme abaixo}
\end{EntryWithPhonetic}

\begin{EntryWithPhonetic}{如一}{ru2 yi1}{6,1}{⼥、⼀}[HSK 6]
  \definition{adj.}{consistente; coerente}
\end{EntryWithPhonetic}

\begin{EntryWithPhonetic}{儒}{ru2}{16}{⼈}
  \definition*{s.}{Confucionismo; Confucionista | Sobrenome Ru}
  \definition{s.}{(antigo) erudito; homem culto}
\end{EntryWithPhonetic}

\begin{EntryWithPhonetic}{儒教}{ru2jiao4}{16,11}{⼈、⽁}
  \definition*{s.}{Confucionismo}
\end{EntryWithPhonetic}

\begin{EntryWithPhonetic}{乳}{ru3}{8}{⼄}
  \definition{adj.}{recém-nascido (animal); lactente}
  \definition{s.}{mama; peito | leite (em geral) | qualquer líquido semelhante ao leite}
  \definition{v.}{dar à luz}
\end{EntryWithPhonetic}

\begin{EntryWithPhonetic}{乳房}{ru3fang2}{8,8}{⼄、⼾}
  \definition{s.}{seio | mama | úbere}
\end{EntryWithPhonetic}

\begin{EntryWithPhonetic}{乳制品}{ru3 zhi4 pin3}{8,8,9}{⼄、⼑、⼝}[HSK 6]
  \definition{s.}{produtos lácteos}
\end{EntryWithPhonetic}

\begin{EntryWithPhonetic}{辱}{ru3}{10}{⾠}
  \definition*{s.}{Sobrenome Ru}
  \definition{s.}{desgraça; desonra (oposto a 荣)}
  \definition{v.}{trazer desgraça (ou humilhação) para | trazer desgraça; ser uma desgraça para | estar em dívida (com alguém por uma gentileza) | humilhar; insultar}
  \seealsoref{荣}{rong2}
\end{EntryWithPhonetic}

\begin{EntryWithPhonetic}{辱骂}{ru3ma4}{10,9}{⾠、⾺}
  \definition{v.}{insultar | abusar}
\end{EntryWithPhonetic}

\begin{EntryWithPhonetic}{入}{ru4}{2}{⼊}[HSK 6][Kangxi 11]
  \definition{s.}{renda | tom de entrada}
  \definition{v.}{entrar; entrar (oposto a 出) | juntar-se; ser admitido em; tornar-se membro de | conformar-se com; concordar com | alcançar; atingir; entrar em (um certo nível ou estado) | fazer entrar; fazer algo entrar; fazer entrada}
  \seealsoref{出}{chu1}
\end{EntryWithPhonetic}

\begin{EntryWithPhonetic}{入党}{ru4dang3}{2,10}{⼊、⼉}
  \definition{v.}{ingressar em um partido político (especialmente o partido comunista)}
\end{EntryWithPhonetic}

\begin{EntryWithPhonetic}{入境}{ru4/jing4}{2,14}{⼊、⼟}
  \definition{v.+compl.}{entrar em um país; imigrar}
\end{EntryWithPhonetic}

\begin{EntryWithPhonetic}{入口}{ru4kou3}{2,3}{⼊、⼝}[HSK 2]
  \definition[个]{s.}{entrada; entrada em locais, edifícios, estradas, etc., através de portões ou portas}
  \definition{v.}{entrar na boca | importar; mercadorias estrangeiras importadas, às vezes também se refere a mercadorias de outras regiões importadas para esta região}
\end{EntryWithPhonetic}

\begin{EntryWithPhonetic}{入门}{ru4/men2}{2,3}{⼊、⾨}[HSK 5]
  \definition{s.}{(geralmente em títulos de livros) curso básico; manual introdutório | ABC; guia; refere-se a leituras básicas; conhecimentos básicos}
  \definition{v.+compl.}{ultrapassar o limiar; aprender os rudimentos de um assunto | aprender o ABC de; ser introduzido a um assunto; aprender o básico}
\end{EntryWithPhonetic}

\begin{EntryWithPhonetic}{入乡随俗}{ru4xiang1-sui2su2}{2,3,11,9}{⼊、⼄、⾩、⼈}
  \definition{expr.}{Em roma, faça como os romanos!}
\end{EntryWithPhonetic}

\begin{EntryWithPhonetic}{入学}{ru4/xue2}{2,8}{⼊、⼦}[HSK 6]
  \definition{v.+compl.}{(de uma criança) começar a escola; começar a escola primária | entrar em uma escola; matricular-se em uma escola}
\end{EntryWithPhonetic}

\begin{EntryWithPhonetic}{软}{ruan3}{8}{⾞}[HSK 5]
  \definition*{s.}{Sobrenome Ruan}
  \definition{adj.}{macio; flexível; maleável; maleável (oposto de 硬) | suave; brando; delicado | fraco; débil | de baixa qualidade, capacidade, etc. | facilmente movido (ou influenciado) | de maneira suave (ou gentil) | indulgente; tolerante | maleável; flexível | fácil de se emocionar ou abalar}
  \seealsoref{硬}{ying4}
\end{EntryWithPhonetic}

\begin{EntryWithPhonetic}{软件}{ruan3jian4}{8,6}{⾞、⼈}[HSK 5]
  \definition[款,个]{s.}{\emph{software}; programas de computador, procedimentos, regras e quaisquer arquivos, documentos e dados relacionados à operação do sistema de computador}
\end{EntryWithPhonetic}

\begin{EntryWithPhonetic}{锐}{rui4}{12}{⾦}
  \definition*{s.}{Sobrenome Rui}
  \definition{adj.}{afiado; aguçado (oposto a 钝) | agudo; perspicaz | rápido; ágil; veloz}
  \definition{adv.}{rapidamente; de ​​repente}
  \definition{s.}{vigor; espírito de luta | armas afiadas}
  \seealsoref{钝}{dun4}
\end{EntryWithPhonetic}

\begin{EntryWithPhonetic}{若}{ruo4}{8}{⾋}[HSK 6]
  \definition*{s.}{Sobrenome Ruo}
  \definition{adv.}{como se; como se fosse; usado antes do verbo para indicar que o que foi dito é mais ou menos assim, equivalente a 好像}
  \definition{conj.}{se; usado na primeira parte de uma frase composta, expressa uma relação hipotética, equivalente a 如果}
  \definition{pron.}{você; referir-se ao interlocutor como 你 ou 你的}
  \definition{v.}{parecer}
  \seealsoref{好像}{hao3xiang4}
  \seealsoref{你}{ni3}
  \seealsoref{你的}{ni3 de5}
  \seealsoref{如果}{ru2guo3}
\end{EntryWithPhonetic}

\begin{EntryWithPhonetic}{弱}{ruo4}{10}{⼸}[HSK 4]
  \definition{adj.}{fraco; debilitado | jovem | inferior; pior | colocado depois de uma fração ou decimal para indicar que é um pouco menor que esse número (oposto a 强)}
  \definition{v.}{perder (através da morte)}
  \seealsoref{强}{qiang2}
\end{EntryWithPhonetic}

%%%%% EOF %%%%%


 %%%
%%% S
%%%

\section*{S}\addcontentsline{toc}{section}{S}

\begin{EntryWithPhonetic}{撒}{sa1}{15}{⼿}
  \definition{v.}{lançar; deixar ir; deixar sair; liberar | livrar-se de todas as restrições; deixar-se levar; tentar usá-lo ou exibi-lo o máximo possível}
\end{EntryWithPhonetic}

\begin{EntryWithPhonetic}{撒旦}{sa1dan4}{15,5}{⼿、⽇}
  \definition*{s.}{Satã}
\end{EntryWithPhonetic}

\begin{EntryWithPhonetic}{撒旦主义}{sa1dan4 zhu3yi4}{15,5,5,3}{⼿、⽇、⼂、⼂}
  \definition*{s.}{Satanismo}
\end{EntryWithPhonetic}

\begin{EntryWithPhonetic}{撒但}{sa1dan4}{15,7}{⼿、⼈}
  \variantof{撒旦}
\end{EntryWithPhonetic}

\begin{EntryWithPhonetic}{洒}{sa3}{9}{⽔}[HSK 5]
  \definition{adj.}{natural e sem restrições; confortável (sem restrições)}
  \definition{v.}{derramar; espalhar; borrifar; salpicar; fazer com que (água ou outra coisa) caia de forma dispersa | derramar; cair de forma dispersa}
\end{EntryWithPhonetic}

\begin{EntryWithPhonetic}{洒水}{sa3shui3}{9,4}{⽔、⽔}
  \definition{v.}{borrifar}
\end{EntryWithPhonetic}

\begin{EntryWithPhonetic}{飒}{sa4}{9}{⾵}
  \definition{adj.}{(das mulheres) natural e desenfreada; elegante; valente}
  \definition{interj.}{(onomatopéia) farfalhar; sussurrar | (onomatopéia) som do vento}
  \definition{v.}{murchar}
\end{EntryWithPhonetic}

\begin{EntryWithPhonetic}{飒飒}{sa4sa4}{9,9}{⾵、⾵}
  \definition{s.}{o farfalhar | sussurro | murmúrio (do vento nas árvores, o mar, etc.)}
\end{EntryWithPhonetic}

\begin{EntryWithPhonetic}{塞}{sai1}{13}{⼟}[HSK 6]
  \definition{s.}{rolha; plugue}
  \definition{v.}{encher; conectar; preencher; espremer; bloquear | superar (para comparação)}
\end{EntryWithPhonetic}

\begin{EntryWithPhonetic}{赛}{sai4}{14}{⾙}[HSK 6]
  \definition*{s.}{Sobrenome Sai}
  \definition{s.}{jogo; partida; competição | sacrifício; cerimônia de sacrifício; antigamente, sacrifícios eram feitos para agradecer aos deuses por suas dádivas}
  \definition{v.}{ter uma competição (comparando alto e baixo, forte e fraco) | superar; ser comparável a; comparar com}
\end{EntryWithPhonetic}

\begin{EntryWithPhonetic}{赛场}{sai4 chang3}{14,6}{⾙、⼟}[HSK 6]
  \definition{s.}{local de competição; arena; ringue; terreno | campo (para competição de atletismo) | pista de corrida}
\end{EntryWithPhonetic}

\begin{EntryWithPhonetic}{赛车}{sai4che1}{14,4}{⾙、⾞}
  \definition{s.}{corrida de automóvel | corrida de bicicleta | carro de corrida}
\end{EntryWithPhonetic}

\begin{EntryWithPhonetic}{三}{san1}{3}{⼀}[HSK 1]
  \definition*{s.}{Sobrenome San}
  \definition{num.}{três; 3 | muitos; vários; mais de dois; referindo-se a muitos ou à maioria | alguns; poucos; menos; não muitos}
\end{EntryWithPhonetic}

\begin{EntryWithPhonetic}{三角}{san1jiao3}{3,7}{⼀、⾓}
  \definition{s.}{triângulo}
\end{EntryWithPhonetic}

\begin{EntryWithPhonetic}{三角恋爱}{san1jiao3lian4'ai4}{3,7,10,10}{⼀、⾓、⼼、⽖}
  \definition{s.}{triângulo amoroso}
\end{EntryWithPhonetic}

\begin{EntryWithPhonetic}{三轮车}{san1lun2che1}{3,8,4}{⼀、⾞、⾞}
  \definition{s.}{triciclo}
\end{EntryWithPhonetic}

\begin{EntryWithPhonetic}{三明治}{san1 ming2 zhi4}{3,8,8}{⼀、⽇、⽔}[HSK 6]
  \definition[个,些,块]{s.}{Empréstimo linguístico: sanduíche, \emph{sandwich}}
\end{EntryWithPhonetic}

\begin{EntryWithPhonetic}{伞}{san3}{6}{⼈}[HSK 4]
  \definition*{s.}{Sobrenome San}
  \definition[把]{s.}{guarda-chuva; proteção contra chuva ou sol | algo que tem o formato de um guarda-chuva}
\end{EntryWithPhonetic}

\begin{EntryWithPhonetic}{散}{san3}{12}{⽁}[HSK 5]
  \definition{adj.}{disperso; fragmentado; não integrado}
  \definition{s.}{medicamento em forma de pó}
  \definition{v.}{divergir; espalhar-se; separar-se; soltar-se; não se manter unido;  desintegrar}
  \seeref{san4}
\end{EntryWithPhonetic}

\begin{EntryWithPhonetic}{散文}{san3wen2}{12,4}{⽁、⽂}[HSK 5]
  \definition[篇,种]{s.}{ensaio; prosa; gênero literário, na antiguidade, referia-se a textos em prosa, em oposição à poesia e à prosa paralela; atualmente, refere-se a obras literárias que não sejam poesia, teatro ou romance, incluindo ensaios, contos, crônicas, relatos de viagem, etc.}
\end{EntryWithPhonetic}

\begin{EntryWithPhonetic}{散}{san4}{12}{⽁}
  \definition{v.}{quebrar; fragmentar; dispersar | dar; distribuir; disseminar; divulgar | dissipar; deixar sai  | terminar um acordo ou contrato; demitir}
  \seeref{san3}
\end{EntryWithPhonetic}

\begin{EntryWithPhonetic}{散步}{san4/bu4}{12,7}{⽁、⽌}[HSK 3]
  \definition{v.+compl.}{dar uma volta; dar um passeio; dar uma caminhada}
\end{EntryWithPhonetic}

\begin{EntryWithPhonetic}{散心}{san4/xin1}{12,4}{⽁、⼼}
  \definition{v.+compl.}{aliviar o tédio | desfrutar de uma diversão | estar despreocupado}
\end{EntryWithPhonetic}

\begin{EntryWithPhonetic}{丧}{sang1}{8}{⼗}
  \definition{adj.}{decepcionado; deprimido; desanimado}
  \definition{v.}{perder | desanimar; frustrar}
  \seeref{sang4}
\end{EntryWithPhonetic}

\begin{EntryWithPhonetic}{丧钟}{sang1zhong1}{8,9}{⼗、⾦}
  \definition{s.}{sentença de morte}
\end{EntryWithPhonetic}

\begin{EntryWithPhonetic}{桑}{sang1}{10}{⽊}
  \definition*{s.}{Sobrenome Sang}
  \definition[棵]{s.}{amoreira}
\end{EntryWithPhonetic}

\begin{EntryWithPhonetic}{桑巴舞}{sang1ba1wu3}{10,4,14}{⽊、⼰、⾇}
  \definition{s.}{samba}
\end{EntryWithPhonetic}

\begin{EntryWithPhonetic}{桑树}{sang1shu4}{10,9}{⽊、⽊}
  \definition{s.}{amoreira, suas folhas são utilizadas para alimentar bichos-da-seda}
\end{EntryWithPhonetic}

\begin{EntryWithPhonetic}{丧}{sang4}{8}{⼗}
  \definition{adj.}{decepcionado | desanimado}
  \definition{v.}{estar enlutado (do cônjuge etc.) | morrer}
  \seeref{sang1}
\end{EntryWithPhonetic}

\begin{EntryWithPhonetic}{丧失}{sang4shi1}{8,5}{⼗、⼤}[HSK 6]
  \definition{v.}{perder (algo que se tem)}
\end{EntryWithPhonetic}

\begin{EntryWithPhonetic}{骚}{sao1}{12}{⾺}
  \definition*{s.}{Abreviação de Li Sao (Encontrando a Tristeza), um poema do poeta e estadista do século IV a.C. Qu Yuan (屈原)}
  \definition{adj.}{coquete; (de uma mulher) lasciva | masculino (de alguns animais domésticos)}
  \definition{s.}{escritos literários; geralmente se refere à poesia | o cheiro de urina; mau cheiro}
  \definition{v.}{perturbar}
  \seealsoref{屈原}{qu1yuan2}
\end{EntryWithPhonetic}

\begin{EntryWithPhonetic}{骚乱}{sao1luan4}{12,7}{⾺、⼄}
  \definition{s.}{rebelião | perturbação | tumulto}
  \definition{v.}{criar um distúrbio}
\end{EntryWithPhonetic}

\begin{EntryWithPhonetic}{扫}{sao3}{6}{⼿}[HSK 4]
  \definition{v.}{varrer; limpar | passar rapidamente ao longo ou sobre; varrer | juntar tudo | Computação: scanear}
  \seeref{sao4}
\end{EntryWithPhonetic}

\begin{EntryWithPhonetic}{扫兴}{sao3/xing4}{6,6}{⼿、⼋}
  \definition{v.+compl.}{sentir-se decepcionado | entristecer alguém}
\end{EntryWithPhonetic}

\begin{EntryWithPhonetic}{嫂}{sao3}{12}{⼥}
  \definition[个,位,名,些]{s.}{esposa do irmão mais velho; cunhada | irmã (uma forma de tratamento para uma mulher casada, mais ou menos da mesma idade)}
\end{EntryWithPhonetic}

\begin{EntryWithPhonetic}{嫂子}{sao3zi5}{12,3}{⼥、⼦}
  \definition{s.}{esposa do irmão mais velho}
\end{EntryWithPhonetic}

\begin{EntryWithPhonetic}{扫}{sao4}{6}{⼿}
  \definition{s.}{elemento formadore de palavra}
  \seeref{sao3}
  \seealsoref{扫帚}{sao4zhou5}
\end{EntryWithPhonetic}

\begin{EntryWithPhonetic}{扫帚}{sao4zhou5}{6,8}{⼿、⼱}
  \definition[把,个]{s.}{vassoura; ferramenta de varredura feita de varas de bambu, etc., maior que uma vassora}
\end{EntryWithPhonetic}

\begin{EntryWithPhonetic}{色}{se4}{6}{⾊}[HSK 4][Kangxi 139]
  \definition*{s.}{Sobrenome Se}
  \definition[种]{s.}{cor | aparência; semblante; expressão | tipo; gênero; descrição | cena; cenário;  paisagem | qualidade (de metais preciosos, mercadorias, etc.) | aparência feminina; beleza feminina | erotismo; apetite sexual; luxúria; desejo sexual}
  \seeref{shai3}
\end{EntryWithPhonetic}

\begin{EntryWithPhonetic}{色彩}{se4cai3}{6,11}{⾊、⼺}[HSK 4]
  \definition[种,丝]{s.}{cor; matiz; tonalidade | cor; sabor; característica; metáfora para um determinado estado de espírito ou tendência de pensamento}
\end{EntryWithPhonetic}

\begin{EntryWithPhonetic}{色狼}{se4lang2}{6,10}{⾊、⽝}
  \definition*{s.}{Sátiro}
  \definition{adj.}{depravado | tarado}
\end{EntryWithPhonetic}

\begin{EntryWithPhonetic}{森}{sen1}{12}{⽊}
  \definition{adj.}{cheio de árvores | multitudinário; em multidões | escuro; sombrio}
\end{EntryWithPhonetic}

\begin{EntryWithPhonetic}{森林}{sen1lin2}{12,8}{⽊、⽊}[HSK 4]
  \definition[片,座,处]{s.}{floresta; bosque; normalmente, refere-se a uma grande área de árvores em crescimento; na silvicultura, refere-se a um grande número de árvores que crescem em uma área razoavelmente grande de terra, juntamente com os animais e outras plantas}
\end{EntryWithPhonetic}

\begin{EntryWithPhonetic}{僧}{seng1}{14}{⼈}
  \definition*{s.}{Sobrenome Seng}
  \definition[位,名,个]{s.}{monge Budista, abreviação de 僧伽}
  \seealsoref{僧伽}{seng1qie2}
\end{EntryWithPhonetic}

\begin{EntryWithPhonetic}{僧伽}{seng1qie2}{14,7}{⼈、⼈}
  \definition{s.}{sangha ou sanga (Budismo) | a comunidade monástica | monge}
\end{EntryWithPhonetic}

\begin{EntryWithPhonetic}{杀}{sha1}{6}{⽊}[HSK 5]
  \definition{adv.}{em extremo; excessivamente; usado após um verbo, indica grau intenso}
  \definition{v.}{matar; abater; esquartejar | lutar; entrar em batalha | enfraquecer; reduzir; diminuir | decolar; neutralizar}
\end{EntryWithPhonetic}

\begin{EntryWithPhonetic}{杀毒}{sha1 du2}{6,9}{⽊、⽏}[HSK 5]
  \definition{s.}{Computação: antivírus}
  \definition{v.}{esterilizar; desinfetar | Computação: eliminar um vírus}
\end{EntryWithPhonetic}

\begin{EntryWithPhonetic}{杀气}{sha1qi4}{6,4}{⽊、⽓}
  \definition{s.}{espírito assassino | aura de morte}
  \definition{v.}{desabafar a raiva de alguém}
\end{EntryWithPhonetic}

\begin{EntryWithPhonetic}{沙}{sha1}{7}{⽔}
  \definition*{s.}{Sobrenome Sha}
  \definition{adj.}{granulado; em pó | rouco}[我今天感冒了,嗓音有点沙哑。===Estou resfriado hoje e minha voz está um pouco rouca.]
  \definition[车,把,袋,吨]{s.}{areia; cascalho; grânulo; pó}
\end{EntryWithPhonetic}

\begin{EntryWithPhonetic}{沙发}{sha1fa1}{7,5}{⽔、⼜}[HSK 3]
  \definition[套,组,个,张]{s.}{sofá; assentos com molas ou espuma plástica espessa, etc., com apoios de braços em ambos os lados}
\end{EntryWithPhonetic}

\begin{EntryWithPhonetic}{沙漠}{sha1mo4}{7,13}{⽔、⽔}[HSK 5]
  \definition[个,片]{s.}{deserto; superfície totalmente coberta por areia, sem água corrente, clima seco e vegetação escassa}
\end{EntryWithPhonetic}

\begin{EntryWithPhonetic}{沙特}{sha1te4}{7,10}{⽔、⽜}
  \definition*{s.}{Saudita | Arábia Saudita, abreviação de 沙特阿拉伯}
  \seealsoref{沙特阿拉伯}{sha1te4 a1la1bo2}
\end{EntryWithPhonetic}

\begin{EntryWithPhonetic}{沙特阿拉伯}{sha1te4 a1la1bo2}{7,10,7,8,7}{⽔、⽜、⾩、⼿、⼈}
  \definition*{s.}{Arábia Saudita}
\end{EntryWithPhonetic}

\begin{EntryWithPhonetic}{沙鱼}{sha1yu2}{7,8}{⽔、⿂}
  \variantof{鲨鱼}
\end{EntryWithPhonetic}

\begin{EntryWithPhonetic}{沙子}{sha1 zi5}{7,3}{⽔、⼦}[HSK 3]
  \definition[粒,把,堆,袋,车]{s.}{areia; grão; pequenas pedras | \emph{pellets}; grãos pequenos; coisas parecidas com areia}
\end{EntryWithPhonetic}

\begin{EntryWithPhonetic}{刹}{sha1}{8}{⼑}
  \definition{v.}{acionar o(s) freio(s); frear; brecar}
  \seeref{cha4}
\end{EntryWithPhonetic}

\begin{EntryWithPhonetic}{刹多罗}{sha1duo1luo2}{8,6,8}{⼑、⼣、⽹}
  \definition*{s.}{Kshatara, sânscrito ``ksetra''}
\end{EntryWithPhonetic}

\begin{EntryWithPhonetic}{砂}{sha1}{9}{⽯}
  \variantof{沙}
\end{EntryWithPhonetic}

\begin{EntryWithPhonetic}{莎}{sha1}{10}{⾋}
  \definition{s.}{em nomes pessoais e de lugares | cigarra | fonético "sha" usado na transliteração}
  \seeref{suo1}
\end{EntryWithPhonetic}

\begin{EntryWithPhonetic}{莎莎舞}{sha1sha1wu3}{10,10,14}{⾋、⾋、⾇}
  \definition{s.}{salsa (dança)}
\end{EntryWithPhonetic}

\begin{EntryWithPhonetic}{鲨}{sha1}{15}{⿂}
  \definition[只,条]{s.}{tubarão}
\end{EntryWithPhonetic}

\begin{EntryWithPhonetic}{鲨鱼}{sha1yu2}{15,8}{⿂、⿂}
  \definition{s.}{tubarão}
\end{EntryWithPhonetic}

\begin{EntryWithPhonetic}{啥}{sha2}{11}{⼝}
  \definition{pron.}{Dialeto: O que?; equivalente a 什么}
\end{EntryWithPhonetic}

\begin{EntryWithPhonetic}{傻}{sha3}{13}{⼈}[HSK 5]
  \definition{adj.}{estúpido; confuso; burro; idiota; inflexível; (ação ou pensamento) mecânico}
\end{EntryWithPhonetic}

\begin{EntryWithPhonetic}{傻瓜}{sha3gua1}{13,5}{⼈、⽠}
  \definition{adj.}{tolo | burro | simplório | idiota}
  \definition{v.}{enganar | iludir | lograr}
\end{EntryWithPhonetic}

\begin{EntryWithPhonetic}{傻眼}{sha3yan3}{13,11}{⼈、⽬}
  \definition{adj.}{estupefato | atordoado}
\end{EntryWithPhonetic}

\begin{EntryWithPhonetic}{嗄}{sha4}{13}{⼝}
  \definition{adj.}{rouco}
  \seeref{a2}
\end{EntryWithPhonetic}

\begin{EntryWithPhonetic}{色}{shai3}{6}{⾊}[Kangxi 139]
  \definition[4]{s.}{cor; (~儿) tem o mesmo significado que "色", usado em algumas palavras faladas}
  \seeref{se4}
\end{EntryWithPhonetic}

\begin{EntryWithPhonetic}{晒}{shai4}{10}{⽇}[HSK 4]
  \definition{v.}{(sol) brilhar sobre | aquecer-se; secar ao sol; colocar algo sob a luz do sol para secar | ignorar (alguém) | mostrar; divulgar o conteúdo de sua vida privada na Internet}
\end{EntryWithPhonetic}

\begin{EntryWithPhonetic}{晒干}{shai4gan1}{10,3}{⽇、⼲}
  \definition{v.}{secar ao sol}
\end{EntryWithPhonetic}

\begin{EntryWithPhonetic}{山}{shan1}{3}{⼭}[HSK 1][Kangxi 46]
  \definition*{s.}{Sobrenome Shan}
  \definition[座]{s.}{colina; maciço; montanha | qualquer coisa que se assemelhe a uma montanha | arbustos nos quais os bichos-da-seda tecem seus casulos; referindo-se a casulos de bicho-da-seda | eco; metáfora para um som muito alto}
\end{EntryWithPhonetic}

\begin{EntryWithPhonetic}{山顶}{shan1ding3}{3,8}{⼭、⾴}
  \definition{s.}{cume da montanha}
\end{EntryWithPhonetic}

\begin{EntryWithPhonetic}{山东}{shan1dong1}{3,5}{⼭、⼀}
  \definition*{s.}{Província de Shandong (Shantung) no nordeste da China}
\end{EntryWithPhonetic}

\begin{EntryWithPhonetic}{山峰}{shan1 feng1}{3,10}{⼭、⼭}[HSK 6]
  \definition[座,个]{s.}{pico (montanha); topo alto e pontudo da montanha}
\end{EntryWithPhonetic}

\begin{EntryWithPhonetic}{山谷}{shan1 gu3}{3,7}{⼭、⾕}[HSK 6]
  \definition[条,个]{s.}{vale; desfiladeiro; ravina; a área baixa e estreita entre duas montanhas geralmente tem riachos no meio}
\end{EntryWithPhonetic}

\begin{EntryWithPhonetic}{山坡}{shan1 po1}{3,8}{⼭、⼟}[HSK 6]
  \definition[个,座,片]{s.}{encosta; encosta da montanha; a inclinação entre o topo da montanha e o terreno plano}
\end{EntryWithPhonetic}

\begin{EntryWithPhonetic}{山区}{shan1 qu1}{3,4}{⼭、⼖}[HSK 5]
  \definition[片]{s.}{área montanhosa; região montanhosa | colina; serra; montanha | distrito montanhoso}
\end{EntryWithPhonetic}

\begin{EntryWithPhonetic}{山体}{shan1ti3}{3,7}{⼭、⼈}
  \definition{s.}{forma de uma montanha}
\end{EntryWithPhonetic}

\begin{EntryWithPhonetic}{山羊}{shan1yang2}{3,6}{⼭、⽺}
  \definition{s.}{cabra | (ginástica) cavalo de salto de pequeno porte}
\end{EntryWithPhonetic}

\begin{EntryWithPhonetic}{山阴}{shan1yin1}{3,6}{⼭、⾩}
  \definition*{s.}{Condado de Shanyin em Shuozhou, Shanxi}
  \definition{s.}{lado norte (ou sombreado) de uma montanha}
\end{EntryWithPhonetic}

\begin{EntryWithPhonetic}{山寨}{shan1zhai4}{3,14}{⼭、⼧}
  \definition{s.}{fortaleza fortificada da vila | fortaleza da montanha (especialmente de bandidos) | falsificação | imitação | (fig.) pechincha}
\end{EntryWithPhonetic}

\begin{EntryWithPhonetic}{扇}{shan1}{10}{⼾}[HSK 5]
  \definition{s.}{ventilar; agitar um leque para fazer o ar circular | dar um tapa; bater com a palma da mão | bater asas; esvoaçar | incitar; instigar; estimular; agitar}
  \seeref{shan4}
\end{EntryWithPhonetic}

\begin{EntryWithPhonetic}{煽}{shan1}{14}{⽕}
  \definition{v.}{abanar (fogo); agitar um leque ou outra folha | incitar; instigar; agitar | vangloriar-se de; esbanjar prêmios em}
\end{EntryWithPhonetic}

\begin{EntryWithPhonetic}{煽动}{shan1dong4}{14,6}{⽕、⼒}
  \definition{v.}{instigar; incitar; agitar; atiçar | incitar; açoitar; conduzir; chicotear}
\end{EntryWithPhonetic}

\begin{EntryWithPhonetic}{闪}{shan3}{5}{⾨}[HSK 4]
  \definition*{s.}{Sobrenome Shan}
  \definition{s.}{relâmpago}
  \definition{v.}{esquivar-se; desviar; sair do caminho | torcer; distender | surgir de repente | cintilar; brilhar | deixar para trás; abandonar | (corpo) oscilar dramaticamente}
\end{EntryWithPhonetic}

\begin{EntryWithPhonetic}{闪存盘}{shan3cun2pan2}{5,6,11}{⾨、⼦、⽫}
  \definition{s.}{unidade de memória \emph{USB} | cartão de memória}
  \seealsoref{优盘}{you1pan2}
\end{EntryWithPhonetic}

\begin{EntryWithPhonetic}{闪电}{shan3dian4}{5,5}{⾨、⽥}[HSK 4]
  \definition[道]{s.}{relâmpago; descargas elétricas entre nuvens ou entre nuvens e o solo}
  \seealsoref{雷电}{lei2dian4}
\end{EntryWithPhonetic}

\begin{EntryWithPhonetic}{掺}{shan3}{11}{⼿}
  \definition{v.}{misturar; mesclar | conter; reter}
  \seeref{can4}
  \seeref{chan1}
\end{EntryWithPhonetic}

\begin{EntryWithPhonetic}{单}{shan4}{8}{⼗}
  \definition*{s.}{Sobrenome Shan}
  \definition{s.}{material de tecido de largura simples (dupla) | número singular (plural)}
  \seeref{chan2}
  \seeref{dan1}
\end{EntryWithPhonetic}

\begin{EntryWithPhonetic}{扇}{shan4}{10}{⼾}[HSK 5]
  \definition{clas.}{usado para portas, janelas, etc.}
  \definition[把]{s.}{leque | folha; algo em forma de placa ou folha}
  \seeref{shan1}
\end{EntryWithPhonetic}

\begin{EntryWithPhonetic}{扇子}{shan4zi5}{10,3}{⼾、⼦}[HSK 5]
  \definition[把,个]{s.}{leque; abano; abanador; utensílios que produzem vento ao serem agitados}
\end{EntryWithPhonetic}

\begin{EntryWithPhonetic}{善}{shan4}{12}{⼝}
  \definition*{s.}{Sobrenome Shan}
  \definition{adj.}{bom; bem | bom; satisfatório | gentil; amigável | familiar}
  \definition{adv.}{bom; bem}
  \definition{s.}{boa ação; ato benevolente; coisas boas (em oposição a 恶)}
  \definition{v.}{fazer sucesso; fazer bem; fazer acontecer | ser bom em; ser especialista (versado) em | ser apto a}
  \seealsoref{恶}{e4}
\end{EntryWithPhonetic}

\begin{EntryWithPhonetic}{善良}{shan4liang2}{12,7}{⼝、⾉}[HSK 4]
  \definition{adj.}{de bom coração; bom e honesto; de bom coração e cheio de boa vontade}
\end{EntryWithPhonetic}

\begin{EntryWithPhonetic}{善意}{shan4yi4}{12,13}{⼝、⼼}
  \definition{s.}{boa vontade | benevolência | bondade}
\end{EntryWithPhonetic}

\begin{EntryWithPhonetic}{善于}{shan4yu2}{12,3}{⼝、⼆}[HSK 4]
  \definition{adv./v.}{ser bom em; ser hábil em}
\end{EntryWithPhonetic}

\begin{EntryWithPhonetic}{禅}{shan4}{12}{⽰}
  \definition{v.}{abdicar e entregar a coroa a outra pessoa}
  \seeref{chan2}
\end{EntryWithPhonetic}

\begin{EntryWithPhonetic}{擅}{shan4}{16}{⼿}
  \definition{adv.}{sem autorização; arbitrariamente | fazer algo por conta própria}
  \definition{v.}{ser bom em; ser especialista em | arrogar-se a si mesmo; fazer algo por conta própria | reivindicar arbitrariamente; ir além do escopo e ajir arbitrariamente}
\end{EntryWithPhonetic}

\begin{EntryWithPhonetic}{擅自}{shan4zi4}{16,6}{⼿、⾃}
  \definition{adv.}{sem permissão ou autorização | por iniciativa própria}
\end{EntryWithPhonetic}

\begin{EntryWithPhonetic}{伤}{shang1}{6}{⼈}[HSK 3]
  \definition*{s.}{Sobrenome Shang}
  \definition[处]{s.}{ferida; ferimento}
  \definition{v.}{ferir; machucar | ter os sentimentos feridos | estar angustiado | enjoar de algo; desenvolver aversão a algo | ser prejudicial a; entravar}
\end{EntryWithPhonetic}

\begin{EntryWithPhonetic}{伤害}{shang1hai4}{6,10}{⼈、⼧}[HSK 4]
  \definition[种]{v.}{ferir; prejudicar; machucar; magoar; causar danos físicos ou mentais}
\end{EntryWithPhonetic}

\begin{EntryWithPhonetic}{伤口}{shang1 kou3}{6,3}{⼈、⼝}[HSK 6]
  \definition[处]{s.}{corte; ferida; onde a pele, os músculos, etc. são feridos, rompidos ou onde são realizadas aberturas cirúrgicas}
\end{EntryWithPhonetic}

\begin{EntryWithPhonetic}{伤亡}{shang1 wang2}{6,3}{⼈、⼇}[HSK 6]
  \definition{s.}{ferimentos e mortes; feridos e mortos; pessoas feridas e mortas; baixas}
  \definition{v.}{ser ferido e morto}
\end{EntryWithPhonetic}

\begin{EntryWithPhonetic}{伤心}{shang1/xin1}{6,4}{⼈、⼼}[HSK 3]
  \definition{v.+compl.}{estar triste; lamentar; estar com o coração partido; sentir-se triste por causa de infortúnio ou decepção}
\end{EntryWithPhonetic}

\begin{EntryWithPhonetic}{伤员}{shang1 yuan2}{6,7}{⼈、⼝}[HSK 6]
  \definition[名,位,个]{s.}{Exército: pessoal ferido; os feridos}
\end{EntryWithPhonetic}

\begin{EntryWithPhonetic}{汤}{shang1}{6}{⽔}
  \definition{s.}{correnteza forte}
  \seeref{tang1}
\end{EntryWithPhonetic}

\begin{EntryWithPhonetic}{商}{shang1}{11}{⼝}
  \definition*{s.}{Dinastia Shang (1600-1046 a.C.) | Shang, nome da estrela da constelação do coração entre as vinte e oito constelações | Sobrenome Shang}
  \definition{s.}{comércio; negócio; a atividade econômica de compra e venda de mercadorias | comerciante; negociante; comerciante; empresário; pessoas que compram e vendem mercadorias | (matemática) quociente;  o resultado de uma operação de divisão em aritmética | uma nota da antiga escala chinesa de cinco tons, correspondente a 2 na notação musical numerada}
  \definition{v.}{discutir; consultar; trocar ideias}
\end{EntryWithPhonetic}

\begin{EntryWithPhonetic}{商标}{shang1biao1}{11,9}{⼝、⽊}[HSK 5]
  \definition[个]{s.}{marca; marca registrada; \emph{trademark}; marca ou símbolo (desenho, padrão, texto, etc.) gravado ou impresso na superfície ou embalagem de um produto, para diferenciá-lo de outros produtos semelhantes}
\end{EntryWithPhonetic}

\begin{EntryWithPhonetic}{商场}{shang1 chang3}{11,6}{⼝、⼟}[HSK 1]
  \definition[家]{s.}{mercado; shopping center; loja de departamentos; loja de grande área com uma variedade completa de produtos | o mundo dos negócios; referindo-se ao mundo dos negócios | mercado; mercado composto por várias lojas reunidas em um ou vários edifícios interligados}
\end{EntryWithPhonetic}

\begin{EntryWithPhonetic}{商城}{shang1 cheng2}{11,9}{⼝、⼟}[HSK 6]
  \definition{s.}{um mercado; um centro comercial; um \emph{shopping center}; refere-se a um complexo comercial contíguo com um grande espaço de construção}
\end{EntryWithPhonetic}

\begin{EntryWithPhonetic}{商店}{shang1dian4}{11,8}{⼝、⼴}[HSK 1]
  \definition[间,家,个]{s.}{loja; armazém; local de venda de mercadorias em recinto fechado}
\end{EntryWithPhonetic}

\begin{EntryWithPhonetic}{商量}{shang1liang5}{11,12}{⼝、⾥}[HSK 2]
  \definition{v.}{consultar; discutir; conversar sobre; discutir e trocar opiniões}
\end{EntryWithPhonetic}

\begin{EntryWithPhonetic}{商贸}{shang1mao4}{11,9}{⼝、⾙}
  \definition{s.}{comércio}
\end{EntryWithPhonetic}

\begin{EntryWithPhonetic}{商品}{shang1pin3}{11,9}{⼝、⼝}[HSK 3]
  \definition[种,个,件,批]{s.}{bens; mercadoria; \emph{merchande}; os produtos do trabalho produzidos para troca têm a dupla natureza de valor de uso e valor; as mercadorias incorporam diferentes relações de produção em diferentes sistemas sociais}
\end{EntryWithPhonetic}

\begin{EntryWithPhonetic}{商人}{shang1 ren2}{11,2}{⼝、⼈}[HSK 2]
  \definition[位,名]{s.}{comerciante; mercador; empresário; homem de negócios; pessoas que trabalham com a distribuição de mercadorias}
\end{EntryWithPhonetic}

\begin{EntryWithPhonetic}{商务}{shang1wu4}{11,5}{⼝、⼒}[HSK 4]
  \definition[种,类,项]{s.}{negócios; assuntos de negócios; assuntos comerciais}
\end{EntryWithPhonetic}

\begin{EntryWithPhonetic}{商业}{shang1ye4}{11,5}{⼝、⼀}[HSK 3]
  \definition[个,种]{s.}{barganha; negócio; comércio; atividade econômica que circula mercadorias por meio de compra e venda}
\end{EntryWithPhonetic}

\begin{EntryWithPhonetic}{上}{shang3}{3}{⼀}
  \definition{s.}{tom descendente-ascendente; significa o segundo tom dos quatro tons do mandarim, e também se refere ao terceiro tom do mandarim padrão}
  \seeref{shang4}
\end{EntryWithPhonetic}

\begin{EntryWithPhonetic}{上声}{shang3sheng1}{3,7}{⼀、⼠}
  \definition{s.}{tom descendente e ascendente | terceiro tom no mandarim moderno}
\end{EntryWithPhonetic}

\begin{EntryWithPhonetic}{赏}{shang3}{12}{⾙}[HSK 4]
  \definition*{s.}{Sobrenome Shang}
  \definition{s.}{recompensa; prêmio}
  \definition{v.}{conceder (outorgar) uma recompensa; recompensar; premiar | admirar; desfrutar; apreciar; valorizar}
\end{EntryWithPhonetic}

\begin{EntryWithPhonetic}{赏赐}{shang3ci4}{12,12}{⾙、⾙}
  \definition{s.}{recompensa | prêmio}
  \definition{v.}{recompensar | premiar}
\end{EntryWithPhonetic}

\begin{EntryWithPhonetic}{赏心悦目}{shang3xin1yue4mu4}{12,4,10,5}{⾙、⼼、⼼、⽬}
  \definition{expr.}{``Aquece o coração e encanta os olhos.''; achar a paisagem agradável tanto aos olhos quanto à mente}
\end{EntryWithPhonetic}

\begin{EntryWithPhonetic}{上}{shang4}{3}{⼀}[HSK 1]
  \definition{adj.}{mais recente; último; anterior; tempo ou a sequência anterior | superior; mais alto; melhor; indica uma posição elevada em termos de qualidade, nível, etc. | lugar elevado; posição superior (em oposição a 下)}
  \definition{s.}{superior; acima; para cima; um lugar alto ou mais alto do que um determinado local | na superfície de um objeto; usado após um substantivo, indica a superfície de um objeto | indica estar dentro do escopo de algo; usado após um substantivo, indica que algo está dentro do âmbito de determinada coisa | indica um aspecto específico | antigamente, referia-se ao imperador | usado após palavras que indicam idade, equivale a ``\dots 的时候'' | o primeiro nível da escala da música folclórica chinesa, usado como um símbolo de nota na notação musical, equivalente ao '1' na notação simplificada.}
  \definition{v.}{subir; montar; embarcar; entrar | ir para; partir para | estar ocupado (com trabalho, estudos, etc.) em um horário fixo; começar a trabalhar ou estudar na hora marcada, etc. | seguir em frente; prosseguir | encher; abastecer; servir; melhorar; aumentar | aparecer no palco; entrar | colocar algo em posição; ajustar; fixar; montar as duas partes de algo | aplicar; pintar; espalhar | ser registrado; ser publicado (em uma publicação) | atingir; ser suficiente (uma determinada quantidade ou grau) | submeter; enviar; apresentar; submeter à aprovação superior | ventilar; apertar; torcer | trazer; servir; colocar comida, pratos, chá e outras coisas na mesa para os convidados | indicar que uma ação tem um resultado | pesquisar na \emph{Internet} | emaranhar-se; ficar emaranhado; enredar-se}
  \definition{v.aux.}{usado após um verbo para indicar início e continuidade}
  \seeref{shang3}
  \seealsoref{的时候}{de5 shi2hou4}
  \seealsoref{下}{xia4}
\end{EntryWithPhonetic}

\begin{EntryWithPhonetic}{上班}{shang4/ban1}{3,10}{⼀、⽟}[HSK 1]
  \definition{v.+compl.}{ir trabalhar; começar a trabalhar; estar de plantão; ir trabalhar no local de trabalho regular no horário especificado}
\end{EntryWithPhonetic}

\begin{EntryWithPhonetic}{上班族}{shang4 ban1 zu2}{3,10,11}{⼀、⽟、⽅}
  \definition[本]{s.}{trabalhadores de escritório (como grupo social)}
\end{EntryWithPhonetic}

\begin{EntryWithPhonetic}{上边}{shang4 bian5}{3,5}{⼀、⾡}[HSK 1]
  \definition{s.}{topo; acima; sobre; superior}
\end{EntryWithPhonetic}

\begin{EntryWithPhonetic}{上车}{shang4 che1}{3,4}{⼀、⾞}[HSK 1]
  \definition{v.}{entrar; subir (em um ônibus, trem, carro etc.)}
\end{EntryWithPhonetic}

\begin{EntryWithPhonetic}{上次}{shang4 ci4}{3,6}{⼀、⽋}[HSK 1]
  \definition{adv.}{última vez}
\end{EntryWithPhonetic}

\begin{EntryWithPhonetic}{上当}{shang4/dang4}{3,6}{⼀、⼹}[HSK 6]
  \definition{v.+compl.}{ser enganado; ser ludibriado; morder a isca; cair nas mãos de alguém}
\end{EntryWithPhonetic}

\begin{EntryWithPhonetic}{上帝}{shang4 di4}{3,9}{⼀、⼱}[HSK 6]
  \definition*{s.}{Deus; O Deus Supremo no Cristianismo | O Imperador do Céu; um deus na antiga crença chinesa que pode controlar tudo no mundo}
  \definition[个]{s.}{(figurado) cliente; metáfora para consumidores}
\end{EntryWithPhonetic}

\begin{EntryWithPhonetic}{上访}{shang4fang3}{3,6}{⼀、⾔}
  \definition{v.}{buscar uma audiência com superiores (especialmente funcionários do governo) para fazer uma petição por algo}
\end{EntryWithPhonetic}

\begin{EntryWithPhonetic}{上个月}{shang4 ge4 yue4}{3,3,4}{⼀、⼈、⽉}[HSK 4]
  \definition{s.}{mês passado; refere-se à hora de um mês atrás, ou seja, o mês passado mais próximo da hora atual}
\end{EntryWithPhonetic}

\begin{EntryWithPhonetic}{上古}{shang4gu3}{3,5}{⼀、⼝}
  \definition{s.}{o passado distante | tempos antigos | antiguidade}
\end{EntryWithPhonetic}

\begin{EntryWithPhonetic}{上海}{shang4hai3}{3,10}{⼀、⽔}
  \definition*{s.}{Município de Xangai (Shanghai), centro-leste da China}
\end{EntryWithPhonetic}

\begin{EntryWithPhonetic}{上级}{shang4ji2}{3,6}{⼀、⽷}[HSK 5]
  \definition[个,位]{s.}{nível superior; organização ou pessoa em nível superior; organizações ou pessoas de nível superior dentro do mesmo sistema organizacional}
\end{EntryWithPhonetic}

\begin{EntryWithPhonetic}{上课}{shang4/ke4}{3,10}{⼀、⾔}[HSK 1]
  \definition{v.+compl.}{frequentar aulas; ir às aulas; dar uma aula}
\end{EntryWithPhonetic}

\begin{EntryWithPhonetic}{上来}{shang4 lai2}{3,7}{⼀、⽊}[HSK 3]
  \definition{v.}{subir (para a minha localização) | estar no começo; começar; iniciar | surgir; de um lugar baixo para um lugar alto (o interlocutor está em um lugar alto) | usado após o verbo, indica que algo foi concluído com sucesso}
\end{EntryWithPhonetic}

\begin{EntryWithPhonetic}{上楼}{shang4 lou2}{3,13}{⼀、⽊}[HSK 4]
  \definition{v.}{subir as escadas; ir para o andar de cima}
\end{EntryWithPhonetic}

\begin{EntryWithPhonetic}{上门}{shang4 men2}{3,3}{⼀、⾨}[HSK 4]
  \definition{v.}{chamar; visitar; aparecer; ir ou vir para ver alguém; ir até a porta; ir até a casa de alguém | trancar a porta; fechar a porta durante a noite | casar-se e morar com a família da noiva}
\end{EntryWithPhonetic}

\begin{EntryWithPhonetic}{上面}{shang4 mian4}{3,9}{⼀、⾯}[HSK 3]
  \definition{s.}{uma posição mais alta que algo; uma posição acima/acima de algo | superfície do objeto | aspecto | a parte acima mencionada; a parte que vem primeiro na ordem; a parte de um artigo ou discurso que vem antes da presente | autoridades superiores | os mais velhos; a geração mais velha da família}
\end{EntryWithPhonetic}

\begin{EntryWithPhonetic}{上坡路}{shang4po1lu4}{3,8,13}{⼀、⼟、⾜}
  \definition{s.}{aclive | progresso | (fig.) tendência ascendente}
\end{EntryWithPhonetic}

\begin{EntryWithPhonetic}{上去}{shang4 qu4}{3,5}{⼀、⼛}[HSK 3]
  \definition{v.}{subir (a partir da minha localização) | ascender a um lugar (ou estado) considerado mais elevado (ou acima); usado depois de um verbo para indicar movimento, de baixo para cima ou de perto para longe}
\end{EntryWithPhonetic}

\begin{EntryWithPhonetic}{上升}{shang4 sheng1}{3,4}{⼀、⼗}[HSK 3]
  \definition{v.}{elevar; subir; mover-se para cima; mover de baixo para cima; aumentar em nível, grau, quantidade, etc.}
\end{EntryWithPhonetic}

\begin{EntryWithPhonetic}{上市}{shang4 shi4}{3,5}{⼀、⼱}[HSK 6]
  \definition{v.}{listar; abrir o capital; ser listado (na bolsa de valores) | estar na estação; estar (aparecer) no mercado | ir ao mercado (para fazer compras)}
\end{EntryWithPhonetic}

\begin{EntryWithPhonetic}{上台}{shang4 tai2}{3,5}{⼀、⼝}[HSK 6]
  \definition{v.}{aparecer no palco; subir na plataforma; ir para o palco ou pódio | assumir o poder; chegar (subir) ao poder; começar a assumir papéis de liderança ou a ganhar algum tipo de poder}
\end{EntryWithPhonetic}

\begin{EntryWithPhonetic}{上网}{shang4wang3}{3,6}{⼀、⽹}[HSK 1]
  \definition{v.}{conectar-se à \emph{Internet}; acessar a \emph{Internet}; entrar na \emph{Internet}; acessar a rede; refere-se especificamente ao computador do usuário conectado à Internet para pesquisar e consultar informações, etc.}
\end{EntryWithPhonetic}

\begin{EntryWithPhonetic}{上午}{shang4wu3}{3,4}{⼀、⼗}[HSK 1]
  \definition[个]{s.}{manhã; \emph{ante meridiem} (a.m.); geralmente refere-se ao período entre a manhã e o meio-dia}
\end{EntryWithPhonetic}

\begin{EntryWithPhonetic}{上下}{shang4 xia4}{3,3}{⼀、⼀}[HSK 5]
  \definition{adv.}{para cima e para baixo}
  \definition[顶]{s.}{alto e baixo | de cima para baixo; para cima e para baixo | superioridade ou inferioridade relativa | (após números redondos) aproximadamente; mais ou menos; por aí | velhos e jovens; hierarquia em termos de cargo e posição social}
  \definition{v.}{subir ou descer | subir e descer; da alta para a baixa ou da baixa para a alta}
\end{EntryWithPhonetic}

\begin{EntryWithPhonetic}{上学}{shang4 xue2}{3,8}{⼀、⼦}[HSK 1]
  \definition{v.}{ir à escola; frequentar a escola; estar na escola; ir à escola para estudar | começar a escola; começar a estudar no ensino fundamental}
\end{EntryWithPhonetic}

\begin{EntryWithPhonetic}{上询}{shang4 xun2}{3,8}{⼀、⾔}
  \definition{adv.}{primeira dezena do mês}
\end{EntryWithPhonetic}

\begin{EntryWithPhonetic}{上演}{shang4 yan3}{3,14}{⼀、⽔}[HSK 6]
  \definition{s.}{exibição | encenação}
  \definition{v.}{exibir (um filme); encenar (uma peça); atuar; colocar no palco}
\end{EntryWithPhonetic}

\begin{EntryWithPhonetic}{上衣}{shang4 yi1}{3,6}{⼀、⾐}[HSK 3]
  \definition[件]{s.}{jaqueta; roupas para a parte superior do corpo}
\end{EntryWithPhonetic}

\begin{EntryWithPhonetic}{上涨}{shang4 zhang3}{3,10}{⼀、⽔}[HSK 5]
  \definition{v.}{subir; ir para cima; ascender}
\end{EntryWithPhonetic}

\begin{EntryWithPhonetic}{上周}{shang4 zhou1}{3,8}{⼀、⼝}[HSK 2]
  \definition{s.}{semana passada}
\end{EntryWithPhonetic}

\begin{EntryWithPhonetic}{尚}{shang4}{8}{⼩}
  \definition*{s.}{Sobrenome Shang}
  \definition{adv.}{ainda}
  \definition{s.}{costume predominante; refere-se à tendência predominante na sociedade; coisas que geralmente são admiradas pelas pessoas}
  \definition{v.}{valorizar; estimar; dar grande importância a}
\end{EntryWithPhonetic}

\begin{EntryWithPhonetic}{尚且}{shang4 qie3}{8,5}{⼩、⼀}
  \definition{conj.}{nem\dots; muito menos\dots; é usado antes do verbo da primeira oração de uma frase complexa para apresentar alguns exemplos óbvios para comparação, a segunda oração frequentemente usa 何况 ou 更 para ecoar e tirar conclusões inevitáveis ​​sobre exemplos semelhantes com diferentes graus de gravidade}
  \seealsoref{更}{geng4}
  \seealsoref{何况}{he2kuang4}
\end{EntryWithPhonetic}

\begin{EntryWithPhonetic}{尚且……何况……}{shang4qie3 he2kuang4}{8,5,7,7}{⼩、⼀、⼈、⼎}
  \definition{conj.}{ainda que\dots, \dots; além do mais\dots e muito menos\dots}
\end{EntryWithPhonetic}

\begin{EntryWithPhonetic}{烧}{shao1}{10}{⽕}[HSK 4]
  \definition[次]{s.}{febre; temperatura corporal mais alta do que o normal}
  \definition{v.}{queimar; pegar fogo | cozinhar; aquecer; assar | guisar depois de fritar ou fritar depois de guisar | assar; grelhar os ingredientes dos alimentos diretamente sobre o fogo | ter febre; estar com febre | danificar (matar ou murchar) as plantas pelo uso excessivo (ou inadequado) de fertilizantes | tornar-se arrogante ou presunçoso; metáfora de estar em uma boa posição e se deixar levar}
\end{EntryWithPhonetic}

\begin{EntryWithPhonetic}{烧烤}{shao1kao3}{10,10}{⽕、⽕}
  \definition{s.}{churrasco}
  \definition{v.}{assar}
\end{EntryWithPhonetic}

\begin{EntryWithPhonetic}{稍}{shao1}{12}{⽲}[HSK 5]
  \definition{adv.}{ligeiramente; um pouco; um pouquinho}
\end{EntryWithPhonetic}

\begin{EntryWithPhonetic}{稍微}{shao1wei1}{12,13}{⽲、⼻}[HSK 5]
  \definition{adv.}{um pouco; um pouquinho; uma ninharia; indica que a quantidade é pequena ou o grau é superficial}
\end{EntryWithPhonetic}

\begin{EntryWithPhonetic}{勺}{shao2}{3}{⼓}[HSK 6]
  \definition{clas.}{shao; uma unidade tradicional de volume, igual a 0,01 市升, e equivalente a 1 centilitro ou 0,018 \emph{pint}}
  \definition{s.}{colher; concha}
  \seealsoref{市升}{shi4sheng1}
\end{EntryWithPhonetic}

\begin{EntryWithPhonetic}{少}{shao3}{4}{⼩}[HSK 1]
  \definition{adj.}{menos; pouco (oposto a 多); escasso; não atingir a quantidade original ou esperada}
  \definition{adv.}{um momento; um instante; provisoriamente; ligeiramente}
  \definition{v.}{faltar; ser insuficiente | dever | perder; desaparecer; extraviar | parar; desistir}
  \seeref{shao4}
  \seealsoref{多}{duo1}
\end{EntryWithPhonetic}

\begin{EntryWithPhonetic}{少见}{shao3jian4}{4,4}{⼩、⾒}
  \definition{adj.}{raramente visto; infrequente; raro}
  \definition{v.}{(forma de saudação) ``Não te vejo há muito tempo.''; ou ``Tenho te visto muito pouco ultimamente.''--``Estou muito feliz em te ver novamente.'' | difícil de ver | não familiar (para o falante) | ser raro (algo)}
\end{EntryWithPhonetic}

\begin{EntryWithPhonetic}{少数}{shao3 shu4}{4,13}{⼩、⽁}[HSK 2]
  \definition{s.}{número pequeno; poucos; minoria}
\end{EntryWithPhonetic}

\begin{EntryWithPhonetic}{少}{shao4}{4}{⼩}
  \definition*{s.}{Sobrenome Shao}
  \definition{s.}{jovem (em oposição a 老)}
  \definition{s.}{jovem mestre; filho de uma família rica}
  \seeref{shao3}
  \seealsoref{老}{lao3}
\end{EntryWithPhonetic}

\begin{EntryWithPhonetic}{少儿}{shao4 er2}{4,2}{⼩、⼉}[HSK 6]
  \definition{s.}{criança}
\end{EntryWithPhonetic}

\begin{EntryWithPhonetic}{少年}{shao4 nian2}{4,6}{⼩、⼲}[HSK 2]
  \definition[个,名,位]{s.}{adolescente; juventude; atualmente, a faixa etária geralmente referida é de 10 anos ou mais a 18 anos ou mais | menor; jovem; juvenil; refere-se a menores na faixa etária anterior | jovem; adolescente; rapaz}
\end{EntryWithPhonetic}

\begin{EntryWithPhonetic}{召}{shao4}{5}{⼝}
  \definition*{s.}{Sobrenome Shao}
  \definition{s.}{(frequentemente em nomes de lugares mongóis) templo; mosteiro}
  \definition{v.}{convocar; intimar; invocar}
  \seeref{zhao4}
\end{EntryWithPhonetic}

\begin{EntryWithPhonetic}{绍}{shao4}{8}{⽷}
  \definition*{s.}{Shaoxing, abreviação de 绍兴 | Sobrenome Shao}
  \definition{v.}{continuar; herdar}
  \seealsoref{绍兴}{shao4xing1}
\end{EntryWithPhonetic}

\begin{EntryWithPhonetic}{绍兴}{shao4xing1}{8,6}{⽷、⼋}
  \definition*{s.}{Shaoxing, anteriormente conhecida como Kuaiji, é uma cidade de nível de prefeitura na província de Zhejiang, na China; é uma grande cidade localizada na parte centro-norte da província de Zhejiang}
\end{EntryWithPhonetic}

\begin{EntryWithPhonetic}{舌}{she2}{6}{⾆}[Kangxi 135]
  \definition*{s.}{Sobrenome She}
  \definition[片,条]{s.}{língua (de um ser humano ou animal); glossa | algo em forma de língua | língua de sino; badalo}
\end{EntryWithPhonetic}

\begin{EntryWithPhonetic}{舌头}{she2tou5}{6,5}{⾆、⼤}[HSK 6]
  \definition[个]{s.}{língua; órgão que auxilia no paladar, na mastigação e na pronúncia | espião}
\end{EntryWithPhonetic}

\begin{EntryWithPhonetic}{折}{she2}{7}{⼿}
  \definition*{s.}{Sobrenome Zhe}
  \definition{clas.}{um ato de zaju | um parágrafo em um drama da Dinastia Yuan, aproximadamente equivalente a uma cena ou ato em uma ópera moderna}
  \definition[张,个,些]{s.}{abatimento; desconto | os traços dos caracteres chineses têm o formato de 𠃍 e 乚 | pasta; livreto}
  \definition{v.}{estalar; quebrar; fazer quebrar | perder; sofrer a perda de | dobrar; torcer; curvar-se | voltar; mudar de direção; retornar | estar convencido; estar cheio de admiração | equivaler a; converter em}
  \seeref{zhe1}
  \seeref{zhe2}
\end{EntryWithPhonetic}

\begin{EntryWithPhonetic}{蛇}{she2}{11}{⾍}[HSK 5]
  \definition[条]{s.}{cobra; serpente; répteis}
\end{EntryWithPhonetic}

\begin{EntryWithPhonetic}{舍}{she3}{8}{⾆}
  \definition{v.}{abandonar; desistir; descartar; jogar fora | dar esmola; dispensar caridade}
  \seeref{she4}
\end{EntryWithPhonetic}

\begin{EntryWithPhonetic}{舍不得}{she3bu5de5}{8,4,11}{⾆、⼀、⼻}[HSK 5]
  \definition{v.}{não se pode abandonar ou deixar, não se quer usar ou descartar; detestar separar-me ou usar}
\end{EntryWithPhonetic}

\begin{EntryWithPhonetic}{舍得}{she3 de5}{8,11}{⾆、⼻}[HSK 5]
  \definition{v.}{não guardar rancor; estar disposto a abrir mão de algo; estar disposto a gastar dinheiro, tempo, etc.; estar disposto a abrir mão de pessoas, oportunidades, coisas, etc. que são importantes para você}
\end{EntryWithPhonetic}

\begin{EntryWithPhonetic}{设}{she4}{6}{⾔}
  \definition*{s.}{Sobrenome She}
  \definition{conj.}{se; no caso | (matemática) dado; suponha; se}
  \definition{v.}{configurar; estabelecer; encontrar; colocar em prática}
\end{EntryWithPhonetic}

\begin{EntryWithPhonetic}{设备}{she4bei4}{6,8}{⾔、⼡}[HSK 3]
  \definition[台,套]{s.}{instalação; equipamento; montagem; um conjunto de edifícios ou equipamentos necessários para executar uma determinada tarefa ou suprir uma determinada necessidade}
\end{EntryWithPhonetic}

\begin{EntryWithPhonetic}{设计}{she4ji4}{6,4}{⾔、⾔}[HSK 3]
  \definition[份]{s.}{plano; esquema; refere-se a um plano de design ou a um projeto para um plano, etc.}
  \definition{v.}{planejar; projetar; formular métodos, desenhos, etc. com antecedência, de acordo com determinados requisitos de finalidade, antes de iniciar oficialmente um trabalho | arquitetar; idear; tramar; fazer um plano}
\end{EntryWithPhonetic}

\begin{EntryWithPhonetic}{设计师}{she4 ji4 shi1}{6,4,6}{⾔、⾔、⼱}[HSK 6]
  \definition[个,位,名,些]{s.}{planejador de projeto; designer | arquiteto}
\end{EntryWithPhonetic}

\begin{EntryWithPhonetic}{设立}{she4li4}{6,5}{⾔、⽴}[HSK 3]
  \definition{v.}{fundar; estabelecer; começar}
\end{EntryWithPhonetic}

\begin{EntryWithPhonetic}{设施}{she4shi1}{6,9}{⾔、⽅}[HSK 4]
  \definition{s.}{facilidade; instalação; instituições, sistemas, organizações, edifícios, etc., estabelecidos para realizar um trabalho ou atender a uma necessidade}
\end{EntryWithPhonetic}

\begin{EntryWithPhonetic}{设想}{she4xiang3}{6,13}{⾔、⼼}[HSK 5]
  \definition[个,种]{s.}{plano provisório (ou ideia); (item, tipo) refere-se a algo hipotético ou imaginário}
  \definition{v.}{imaginar; prever; conceber; supor | ter consideração por}
\end{EntryWithPhonetic}

\begin{EntryWithPhonetic}{设置}{she4zhi4}{6,13}{⾔、⽹}[HSK 4]
  \definition{v.}{estabelecer; colocar em prática; estabelecer ou criar instituições, empregos, profissões ou códigos, etc. | encaixar; ajustar; instalar; configurar; colocar}
\end{EntryWithPhonetic}

\begin{EntryWithPhonetic}{社}{she4}{7}{⽰}[HSK 5]
  \definition[个,家]{s.}{agência; sociedade; órgão organizado; organização; comunidade | comuna popular | o deus da terra, sacrifícios a ele ou altares para tais sacrifícios; na antiguidade, o deus da terra, o local onde ele era venerado, o dia da veneração e o ritual eram chamados de 社 | agência de notícias |  imprensa}
\end{EntryWithPhonetic}

\begin{EntryWithPhonetic}{社会}{she4hui4}{7,6}{⽰、⼈}[HSK 3]
  \definition[个,种]{s.}{sociedade; em um determinado estágio do desenvolvimento histórico, a relação geral entre as pessoas nas atividades de produção | comunidade; geralmente se refere a um grupo de pessoas que estão conectadas por atividades comuns}
\end{EntryWithPhonetic}

\begin{EntryWithPhonetic}{社区}{she4qu1}{7,4}{⽰、⼖}[HSK 5]
  \definition[个]{s.}{bairro; comunidade residencial; bairros da cidade, divididos de acordo com a localização geográfica | distrito; comunidade (para pessoas da mesma classe social, etc.) ; lugar onde pessoas com características comuns, como classe social, vivem juntas}
\end{EntryWithPhonetic}

\begin{EntryWithPhonetic}{舍}{she4}{8}{⾆}
  \definition*{s.}{Sobrenome She}
  \definition{clas.}{uma unidade antiga de distância igual a 30 li, 里}
  \definition{pron.}{meu, uma palavra humilde usada para se referir aos parentes mais jovens ou de geração inferior}
  \definition{s.}{cabana; casa | minha casa; minha humilde morada | chiqueiro; galpão; curral de gado}
  \seeref{she3}
  \seealsoref{里}{li3}
\end{EntryWithPhonetic}

\begin{EntryWithPhonetic}{射}{she4}{10}{⼨}[HSK 5]
  \definition*{s.}{Sobrenome She}
  \definition{v.}{atirar; disparar | descarregar em jato; jorrar | emitir (luz, calor, etc.) | irradiar | aludir a algo ou alguém; insinuar}
\end{EntryWithPhonetic}

\begin{EntryWithPhonetic}{射击}{she4ji1}{10,5}{⼨、⼐}[HSK 5]
  \definition{s.}{tiro; tiro ao alvo}
  \definition{v.}{disparar; atirar}
\end{EntryWithPhonetic}

\begin{EntryWithPhonetic}{涉}{she4}{10}{⽔}[HSK 6]
  \definition*{s.}{Sobrenome She}
  \definition{v.}{vadear; atravessar ou passar um rio ou um obstáculo | passar por; experimentar | envolver; implicar}
\end{EntryWithPhonetic}

\begin{EntryWithPhonetic}{涉及}{she4ji2}{10,3}{⽔、⼃}[HSK 6]
  \definition{v.}{envolver; relacionar-se com; referir-se a; tocar em}
\end{EntryWithPhonetic}

\begin{EntryWithPhonetic}{摄}{she4}{13}{⼿}
  \definition*{s.}{Sobrenome She}
  \definition{v.}{absorver; assimilar | tirar uma fotografia de; fotografar | conservar (a saúde) | atuar}
\end{EntryWithPhonetic}

\begin{EntryWithPhonetic}{摄氏}{she4shi4}{13,4}{⼿、⽒}
  \definition{s.}{graus Celsius (°C), centígrado}
\end{EntryWithPhonetic}

\begin{EntryWithPhonetic}{摄像}{she4 xiang4}{13,13}{⼿、⼈}[HSK 5]
  \definition{v.}{gravar; filmar; filmar com câmera; fazer uma gravação de vídeo (com uma câmera de vídeo ou TV)}
\end{EntryWithPhonetic}

\begin{EntryWithPhonetic}{摄像机}{she4 xiang4 ji1}{13,13,6}{⼿、⼈、⽊}[HSK 5]
  \definition[个,部,台]{s.}{câmera de vídeo; dispositivo que pode ser usado para converter imagens captadas em sinais de imagem de televisão}
\end{EntryWithPhonetic}

\begin{EntryWithPhonetic}{摄影}{she4ying3}{13,15}{⼿、⼺}[HSK 5]
  \definition{v.}{fotografar; tirar uma foto; tirar fotos ou filmar}
\end{EntryWithPhonetic}

\begin{EntryWithPhonetic}{摄影师}{she4 ying3 shi1}{13,15,6}{⼿、⼺、⼱}[HSK 5]
  \definition[个,名,位]{s.}{fotógrafo; cinegrafista; operador de câmera; técnico de fotografia em estúdio fotográfico}
\end{EntryWithPhonetic}

\begin{EntryWithPhonetic}{谁}{shei2}{10}{⾔}[HSK 1]
  \definition{pron.}{quem? | (em pergunta retórica) quem?; usado em perguntas retóricas, para indicar que não há ninguém | refere-se a pessoas que não têm certeza, incluindo aquelas que não sabem | alguém; qualquer pessoa; indica qualquer pessoa ou qualquer um | repetido em uma frase para se referir a uma pessoa | (repetido em duas frases) quem quer que seja; fazer com que o sujeito e o objeto se refiram a duas pessoas diferentes}
  \seeref{shui2}
\end{EntryWithPhonetic}

\begin{EntryWithPhonetic}{申}{shen1}{5}{⽥}
  \definition*{s.}{O nono dos doze Ramos Terrestres | Outro nome para Xangai, 上海 | Sobrenome Shen}
  \definition{v.}{declarar; explicar; expressar}
  \seealsoref{上海}{shang4hai3}
\end{EntryWithPhonetic}

\begin{EntryWithPhonetic}{申请}{shen1qing3}{5,10}{⽥、⾔}[HSK 4]
  \definition[份,批,项]{s.}{a solicitação para; o requerimento para; um pedido para ser visto pelos superiores ou departamentos relevantes}
  \definition{v.}{solicitar; apresentar uma solicitação; apresentar os motivos e fazer o pedido aos superiores ou aos departamentos competentes}
\end{EntryWithPhonetic}

\begin{EntryWithPhonetic}{伸}{shen1}{7}{⼈}[HSK 5]
  \definition{v.}{alongar; esticar; estender}
\end{EntryWithPhonetic}

\begin{EntryWithPhonetic}{身}{shen1}{7}{⾝}[Kangxi 158]
  \definition*{s.}{Sobrenome Shen}
  \definition{adv.}{eu mesmo; a si mesmo; pessoalmente}
  \definition{s.}{corpo humano ou animal | vida | o caráter moral e a conduta de alguém; cultivo moral | corpo; a parte principal de uma estrutura; o corpo principal ou tronco de um objeto |  uma vida inteira; a vida inteira de alguém | \emph{status} social; identidade}
\end{EntryWithPhonetic}

\begin{EntryWithPhonetic}{身边}{shen1 bian1}{7,5}{⾝、⾡}[HSK 2]
  \definition{adv.}{ao redor; ao lado de alguém; perto do corpo | carregar consigo (transportar); à mão}
\end{EntryWithPhonetic}

\begin{EntryWithPhonetic}{身材}{shen1cai2}{7,7}{⾝、⽊}[HSK 4]
  \definition[种,个,具]{s.}{figura; estatura; altura e peso corporal}
\end{EntryWithPhonetic}

\begin{EntryWithPhonetic}{身份}{shen1fen4}{7,6}{⾝、⼈}[HSK 4]
  \definition[种]{s.}{status; capacidade; identidade; refere-se à origem, ao status e às qualificações de uma pessoa | dignidade; posição honrada; referência especial ao status respeitável}
\end{EntryWithPhonetic}

\begin{EntryWithPhonetic}{身份证}{shen1 fen4 zheng4}{7,6,7}{⾝、⼈、⾔}[HSK 3]
  \definition[张]{s.}{ID; bilhete de identidade; carteira de identidade}
\end{EntryWithPhonetic}

\begin{EntryWithPhonetic}{身高}{shen1 gao1}{7,10}{⾝、⾼}[HSK 4]
  \definition[个,种,段]{s.}{estatura; altura (de uma pessoa)}
\end{EntryWithPhonetic}

\begin{EntryWithPhonetic}{身上}{shen1 shang5}{7,3}{⾝、⼀}[HSK 1]
  \definition{s.}{no corpo de alguém | em um;  com um}
\end{EntryWithPhonetic}

\begin{EntryWithPhonetic}{身体}{shen1ti3}{7,7}{⾝、⼈}[HSK 1]
  \definition[具,个]{s.}{corpo | saúde; saúde das pessoas}
\end{EntryWithPhonetic}

\begin{EntryWithPhonetic}{身体能力}{shen1ti3 neng2li4}{7,7,10,2}{⾝、⼈、⾁、⼒}
  \definition{s.}{habilidade física}
\end{EntryWithPhonetic}

\begin{EntryWithPhonetic}{身体乳}{shen1ti3 ru3}{7,7,8}{⾝、⼈、⼄}
  \definition{s.}{loção corporal}
\end{EntryWithPhonetic}

\begin{EntryWithPhonetic}{身亡}{shen1wang2}{7,3}{⾝、⼇}
  \definition{v.}{morrer}
\end{EntryWithPhonetic}

\begin{EntryWithPhonetic}{深}{shen1}{11}{⽔}[HSK 3]
  \definition*{s.}{Sobrenome Shen}
  \definition{adj.}{profundo | difícil; intenso; profundo | completo; penetrante; intenso; profundo | próximo; íntimo; afeição profunda; relacionamento próximo | escuro; profundo | tardio}
  \definition{adv.}{muito; grandemente; profundamente}
  \definition{s.}{profundidade}
  \seealsoref{浅}{qian3}
\end{EntryWithPhonetic}

\begin{EntryWithPhonetic}{深处}{shen1 chu4}{11,5}{⽔、⼡}[HSK 5]
  \definition{s.}{profundidades; recantos; recessos | profundezas}
\end{EntryWithPhonetic}

\begin{EntryWithPhonetic}{深度}{shen1 du4}{11,9}{⽔、⼴}[HSK 5]
  \definition{adj.}{(em grau ou extensão) profundo; sério; grave}
  \definition{s.}{profundidade; grau de profundidade; | profundidade; rigor; meticulosidade; grau de contato com a essência das coisas | estágio avançado (ou em deterioração) de desenvolvimento; grau de crescimento e desenvolvimento das coisas}
\end{EntryWithPhonetic}

\begin{EntryWithPhonetic}{深厚}{shen1hou4}{11,9}{⽔、⼚}[HSK 4]
  \definition{adj.}{profundo; sentimentos fortes | sólido; profundamente enraizado; fundação sólida}
\end{EntryWithPhonetic}

\begin{EntryWithPhonetic}{深化}{shen1 hua4}{11,4}{⽔、⼔}[HSK 6]
  \definition{v.}{aprofundar; avançar; intensificar; tornar-se mais profundo; tornar mais profundo}
\end{EntryWithPhonetic}

\begin{EntryWithPhonetic}{深刻}{shen1ke4}{11,8}{⽔、⼑}[HSK 3]
  \definition{adj.}{profundo; instenso; chegar à essência de um assunto ou problema}
\end{EntryWithPhonetic}

\begin{EntryWithPhonetic}{深入}{shen1 ru4}{11,2}{⽔、⼊}[HSK 3]
  \definition{adj.}{profundo; completo}
  \definition{v.}{ir fundo em; penetrar em; penetrar o exterior; alcançar o interior ou o centro de algo}
\end{EntryWithPhonetic}

\begin{EntryWithPhonetic}{深深}{shen1 shen1}{11,11}{⽔、⽔}[HSK 6]
  \definition{adj.}{profundo; intenso}
  \definition{adv.}{profundamente; intensamente; descreve um grau profundo ou forte}
\end{EntryWithPhonetic}

\begin{EntryWithPhonetic}{深夜}{shen1ye4}{11,8}{⽔、⼣}
  \definition{adv.}{tarde da noite}
\end{EntryWithPhonetic}

\begin{EntryWithPhonetic}{什}{shen2}{4}{⼈}
  \definition{pron.}{o que; qualquer coisa}
  \seeref{shi2}
  \seealsoref{什么}{shen2me5}
\end{EntryWithPhonetic}

\begin{EntryWithPhonetic}{什么}{shen2me5}{4,3}{⼈、⼃}[HSK 1]
  \definition{pron.}{o que?; expressar dúvida, perguntar sobre o mundo, locais, pessoas ou coisas | usado para se referir a algo indefinido; expressar incerteza | qualquer; todos; refere-se a todas as pessoas ou coisas | dois 什么 são usados juntos, indicando que o primeiro determina o segundo | usado para expressar surpresa ou insatisfação | usado para expressar discordância com o que foi dito; expressar negação | usado antes de elementos paralelos para indicar que a lista é infinita}
\end{EntryWithPhonetic}

\begin{EntryWithPhonetic}{什么时候}{shen2me5shi2hou5}{4,3,7,10}{⼈、⼃、⽇、⼈}
  \definition{adv.}{quando? | a que horas?}
\end{EntryWithPhonetic}

\begin{EntryWithPhonetic}{什么样}{shen2 me5 yang4}{4,3,10}{⼈、⼃、⽊}[HSK 2]
  \definition{pron.}{que tipo?; usado para perguntar sobre a natureza, características ou aparência de algo |  o quê?; de que tipo?; usado para perguntar sobre a situação ou o estado de alguém ou algo}
\end{EntryWithPhonetic}

\begin{EntryWithPhonetic}{神}{shen2}{9}{⽰}[HSK 5]
  \definition*{s.}{Deus | Sobrenome Shen}
  \definition{adj.}{inteligente; esperto | mágico; sobrenatural}
  \definition[个,位,尊,名]{s.}{divindade; deidade | espírito; mente; refere-se ao espírito, energia ou atenção de uma pessoa | olhar; expressão; expressões que refletem o estado interior}
\end{EntryWithPhonetic}

\begin{EntryWithPhonetic}{神话}{shen2hua4}{9,8}{⽰、⾔}[HSK 4]
  \definition[段,篇]{s.}{mito; mitologia; conto de fadas; refere-se a deuses e deusas lendários e histórias de heróis antigos deificados | lorota; refere-se a alegações ridículas e infundadas}
\end{EntryWithPhonetic}

\begin{EntryWithPhonetic}{神经}{shen2jing1}{9,8}{⽰、⽷}[HSK 5]
  \definition{adj.}{excêntrico; estranho; peculiar; descreve anormalidade neurológica}
  \definition[根,条]{s.}{nervo; um tipo de tecido presente no corpo humano ou animal que conecta o cérebro aos órgãos, transmitindo as sensações ao cérebro e as informações do cérebro aos órgãos}
\end{EntryWithPhonetic}

\begin{EntryWithPhonetic}{神经病的}{shen2jing1bing4 de5}{9,8,10,8}{⽰、⽷、⽧、⽩}
  \definition{adj.}{neuropático; neurótico}
\end{EntryWithPhonetic}

\begin{EntryWithPhonetic}{神经病学}{shen2jing1bing4 xue2}{9,8,10,8}{⽰、⽷、⽧、⼦}
  \definition{s.}{neurologia}
\end{EntryWithPhonetic}

\begin{EntryWithPhonetic}{神秘}{shen2mi4}{9,10}{⽰、⽲}[HSK 4]
  \definition{adj.}{místico; misterioso}
\end{EntryWithPhonetic}

\begin{EntryWithPhonetic}{神明}{shen2ming2}{9,8}{⽰、⽇}
  \definition{s.}{divindades | deuses}
\end{EntryWithPhonetic}

\begin{EntryWithPhonetic}{神奇}{shen2qi2}{9,8}{⽰、⼤}[HSK 5]
  \definition{adj.}{mágico; peculiar; místico; milagroso; faz as pessoas se sentirem muito revigoradas; é completamente inesperado e geralmente traz boas influências}
  \definition{adj.}{mágico; peculiar; místico; milagroso; algo que parece muito novo; algo que ninguém imaginaria, mas que geralmente traz bons resultados}
\end{EntryWithPhonetic}

\begin{EntryWithPhonetic}{神器}{shen2qi4}{9,16}{⽰、⼝}
  \definition{s.}{objeto mágico | objeto simbólico do poder imperial | arma fina | ferramenta muito útil}
\end{EntryWithPhonetic}

\begin{EntryWithPhonetic}{神情}{shen2 qing2}{9,11}{⽰、⼼}[HSK 5]
  \definition{s.}{aparência; expressão; atividades internas reveladas no rosto das pessoas}
\end{EntryWithPhonetic}

\begin{EntryWithPhonetic}{神兽}{shen2shou4}{9,11}{⽰、⼋}
  \definition{s.}{animal mitológico | fera}
\end{EntryWithPhonetic}

\begin{EntryWithPhonetic}{审}{shen3}{8}{⼧}[HSK 6]
  \definition*{s.}{Sobrenome Shen}
  \definition{adj.}{cuidadoso; detalhado; completo}
  \definition{adv.}{Literpario: realmente; de ​​fato; como esperado}
  \definition{v.}{examinar; analizar | julgar; interrogar | Literário: saber}
\end{EntryWithPhonetic}

\begin{EntryWithPhonetic}{审查}{shen3cha2}{8,9}{⼧、⽊}[HSK 6]
  \definition{v.}{examinar; investigar; verificar se algo está correto e apropriado (geralmente referindo-se a planos, propostas, escritos, qualificações pessoais, etc.); ler e avaliar (provas ou trabalhos de exame)}
\end{EntryWithPhonetic}

\begin{EntryWithPhonetic}{甚}{shen4}{9}{⽢}
  \definition{adv.}{muito; extremamente}
  \definition{pron.}{o que}
  \definition{v.}{exceder; superar}
  \seealsoref{什么}{shen2me5}
\end{EntryWithPhonetic}

\begin{EntryWithPhonetic}{甚而}{shen4'er2}{9,6}{⽢、⽽}
  \definition{conj.}{(ir) tão longe quanto | tanto que}
\end{EntryWithPhonetic}

\begin{EntryWithPhonetic}{甚或}{shen4huo4}{9,8}{⽢、⼽}
  \definition{conj.}{(ir) tão longe quanto | tanto que}
\end{EntryWithPhonetic}

\begin{EntryWithPhonetic}{甚至}{shen4zhi4}{9,6}{⽢、⾄}[HSK 4]
  \definition{conj.}{e até mesmo; nem mesmo; para apresentar uma situação típica e especial, para enfatizar a profundidade e a seriedade de uma situação}
\end{EntryWithPhonetic}

\begin{EntryWithPhonetic}{升}{sheng1}{4}{⼗}[HSK 3]
  \definition*{s.}{Sobrenome Sheng}
  \definition{clas.}{litro (l)}
  \definition{s.}{sheng, uma unidade de medida seca para grãos (= 1 litro), um décimo de 斗}
  \definition{v.}{elevar; içar; subir; ascender; subir ou subir mais alto (oposto de 降) | promover; melhorar (nível)}
  \seealsoref{斗}{dou4}
  \seealsoref{降}{jiang4}
\end{EntryWithPhonetic}

\begin{EntryWithPhonetic}{升高}{sheng1 gao1}{4,10}{⼗、⾼}[HSK 5]
  \definition{v.}{subir; ascender | promover; elevar; intensificar; potencializar; melhorar}
\end{EntryWithPhonetic}

\begin{EntryWithPhonetic}{升级}{sheng1/ji2}{4,6}{⼗、⽷}[HSK 6]
  \definition{v.+compl.}{atualizar (software) | (guerra) escalar; (tensão) aprofundar | subir um ou mais níveis; passar de uma série ou classe inferior para uma série ou classe superior}
\end{EntryWithPhonetic}

\begin{EntryWithPhonetic}{升起}{sheng1qi3}{4,10}{⼗、⾛}
  \definition{v.}{levantar | içar | subir}
\end{EntryWithPhonetic}

\begin{EntryWithPhonetic}{升学}{sheng1 xue2}{4,8}{⼗、⼦}[HSK 6]
  \definition{v.}{ir para uma universidade, faculdade; entrar em uma universidade, faculdade}
\end{EntryWithPhonetic}

\begin{EntryWithPhonetic}{升值}{sheng1 zhi2}{4,10}{⼗、⼈}[HSK 6]
  \definition{v.}{Economia: reavaliar; apreciar | Figurativo: aumento de valor | valorização; apreciação; aumentar o valor; aumentar os preços}
\end{EntryWithPhonetic}

\begin{EntryWithPhonetic}{生}{sheng1}{5}{⽣}[HSK 2,3][Kangxi 100]
  \definition*{s.}{Sobrenome Sheng}
  \definition{adj.}{vivo; vital | verde; não maduro | cru; não cozido; mal cozido | bruto; não refinado; não processado | estranho; desconhecido; não familiarizado | rígido; mecânico; forçado}
  \definition{adv.}{muito; usado antes de certas palavras que expressam emoções e sentimentos | verdadeiramente; realmente; forçosamente}
  \definition{s.}{vida | meio de subsistência | aluno; estudante | estudioso; antigamente chamados de eruditos | o tipo de personagem masculino na ópera de Pequim, etc.}
  \definition{suf.}{certos sufixos substantivos que se referem a pessoas (学生) | sufixos de certos advérbios (好生)}
  \definition{v.}{dar à luz; ter um filho | nascer | crescer; cultivar | viver; existir; sobreviver | favorecer; gerar; ocorrer | acender (uma fogueira); fazer o combustível queimar}
  \seealsoref{好生}{hao3sheng1}
  \seealsoref{学生}{xue2sheng5}
\end{EntryWithPhonetic}

\begin{EntryWithPhonetic}{生病}{sheng1bing4}{5,10}{⽣、⽧}[HSK 1]
  \definition{v.}{adoecer; ficar doente; ficar mal; contrair uma doença}
\end{EntryWithPhonetic}

\begin{EntryWithPhonetic}{生菜}{sheng1cai4}{5,11}{⽣、⾋}
  \definition{s.}{alface}
\end{EntryWithPhonetic}

\begin{EntryWithPhonetic}{生产}{sheng1chan3}{5,6}{⽣、⼇}[HSK 3]
  \definition{v.}{produzir; fabricar; utilizar ferramentas para mudar o objeto de trabalho e criar meios de produção e meios de subsistência | dar à luz uma criança; ter filhos}
\end{EntryWithPhonetic}

\begin{EntryWithPhonetic}{生成}{sheng1 cheng2}{5,6}{⽣、⼽}[HSK 5]
  \definition{v.}{formar; gerar; produzir | ter por natureza; nascer com}
\end{EntryWithPhonetic}

\begin{EntryWithPhonetic}{生词}{sheng1 ci2}{5,7}{⽣、⾔}[HSK 2]
  \definition[个,组,堆,条]{s.}{nova palavra; palavras que não aprendi, não conheço ou não entendo}
\end{EntryWithPhonetic}

\begin{EntryWithPhonetic}{生存}{sheng1cun2}{5,6}{⽣、⼦}[HSK 3]
  \definition{v.}{viver; sobreviver; subsistir; manter a vida; estar vivo}
\end{EntryWithPhonetic}

\begin{EntryWithPhonetic}{生的}{sheng1de5}{5,8}{⽣、⽩}
  \definition{conj.}{para evitar isso | para que\dots não\dots}
\end{EntryWithPhonetic}

\begin{EntryWithPhonetic}{生动}{sheng1dong4}{5,6}{⽣、⼒}[HSK 3]
  \definition{adj.}{vívido; animado; descreve a linguagem e as formas de expressão como sendo ativas e em movimento}
\end{EntryWithPhonetic}

\begin{EntryWithPhonetic}{生活}{sheng1huo2}{5,9}{⽣、⽔}[HSK 2]
  \definition[个,段,种]{s.}{vida; subsistência; as diversas atividades realizadas por pessoas ou seres vivos para sobreviver e se desenvolver | estilo de vida; condições de vida; situação em termos de vestuário, alimentação, habitação e transporte | trabalho (principalmente nas áreas industrial, agrícola e artesanal)}
  \definition{v.}{viver; realizar várias atividades | sobreviver}
\end{EntryWithPhonetic}

\begin{EntryWithPhonetic}{生活费}{sheng1 huo2 fei4}{5,9,9}{⽣、⽔、⾙}[HSK 6]
  \definition{s.}{subsídio; despesas de subsistência; despesas necessárias para manter a vida diária}
\end{EntryWithPhonetic}

\begin{EntryWithPhonetic}{生活垃圾}{sheng1huo2la1ji1}{5,9,8,6}{⽣、⽔、⼟、⼟}
  \definition{s.}{lixo doméstico}
\end{EntryWithPhonetic}

\begin{EntryWithPhonetic}{生活型}{sheng1huo2 xing2}{5,9,9}{⽣、⽔、⼟}
  \definition{s.}{forma de vida}
\end{EntryWithPhonetic}

\begin{EntryWithPhonetic}{生理}{sheng1li3}{5,11}{⽣、⽟}
  \definition{adj.}{fisiológico}
  \definition{s.}{fisiologia}
\end{EntryWithPhonetic}

\begin{EntryWithPhonetic}{生命}{sheng1ming4}{5,8}{⽣、⼝}[HSK 3]
  \definition{s.}{vida; não envolve apenas a existência e as atividades dos organismos, mas também inclui experiências de vida humana, valores e elementos-chave da sobrevivência e do desenvolvimento de várias coisas}
\end{EntryWithPhonetic}

\begin{EntryWithPhonetic}{生气}{sheng1/qi4}{5,4}{⽣、⽓}[HSK 1]
  \definition{s.}{vitalidade; vigor; energia da vida}
  \definition{v.+compl.}{ficar com raiva; ficar ofendido; ficar zangado; encontrar algo que não é do seu agrado e sentir-se descontente}
\end{EntryWithPhonetic}

\begin{EntryWithPhonetic}{生日}{sheng1ri4}{5,4}{⽣、⽇}[HSK 1]
  \definition[个,次]{s.}{aniversário; dia de nascimento, também se refere ao dia em que se completa um ano de idade a cada ano}
\end{EntryWithPhonetic}

\begin{EntryWithPhonetic}{生态}{sheng1tai4}{5,8}{⽣、⼼}
  \definition{adj.}{ecológico}
  \definition{s.}{ecologia}
\end{EntryWithPhonetic}

\begin{EntryWithPhonetic}{生物}{sheng1wu4}{5,8}{⽣、⽜}
  \definition{adj.}{biológico}
  \definition{s.}{biologia (disciplina) | organismo | ser vivo}
\end{EntryWithPhonetic}

\begin{EntryWithPhonetic}{生意}{sheng1yi4}{5,13}{⽣、⼼}
  \definition[笔,种,次]{s.}{tendência a crescer; vitalidade; vigor; energia}
  \seeref{sheng1yi5}
\end{EntryWithPhonetic}

\begin{EntryWithPhonetic}{生意}{sheng1yi5}{5,13}{⽣、⼼}[HSK 3]
  \definition[笔,种,次]{s.}{comércio, compra e venda; negócios; indústria; colegas do mesmo setor}
  \seeref{sheng1yi4}
\end{EntryWithPhonetic}

\begin{EntryWithPhonetic}{生鱼片}{sheng1yu2pian4}{5,8,4}{⽣、⿂、⽚}
  \definition{s.}{fatias de peixe cru, \emph{sashimi}}
\end{EntryWithPhonetic}

\begin{EntryWithPhonetic}{生长}{sheng1zhang3}{5,4}{⽣、⾧}[HSK 3]
  \definition{v.}{cresçer; sob certas condições de vida, o volume e o peso dos organismos aumentam gradualmente | nascer e crescer}
\end{EntryWithPhonetic}

\begin{EntryWithPhonetic}{声}{sheng1}{7}{⼠}[HSK 5]
  \definition{clas.}{indica o número de vezes que um som é emitido}
  \definition{s.}{som; voz | reputação | consoante inicial (de uma sílaba chinesa) | tom; tom de voz | informação; notícia}
  \definition{v.}{declarar; anunciar; emitir um som}
\end{EntryWithPhonetic}

\begin{EntryWithPhonetic}{声明}{sheng1ming2}{7,8}{⼠、⽇}[HSK 3]
  \definition[项,份]{s.}{declaração}
  \definition{v.}{declarar; anunciar; expressar publicamente a sua atitude ou dizer a verdade}
\end{EntryWithPhonetic}

\begin{EntryWithPhonetic}{声音}{sheng1yin1}{7,9}{⼠、⾳}[HSK 2]
  \definition[个,种]{s.}{som; voz; a percepção auditiva das ondas sonoras}
\end{EntryWithPhonetic}

\begin{EntryWithPhonetic}{绳}{sheng2}{11}{⽷}
  \definition*{s.}{Sobrenome Sheng}
  \definition[根]{s.}{corda; cordão; barbante | a linha no marcador de tinta de carpinteiro}
  \definition{v.}{restringir; corrigir; sancionar | medir | continuar}
\end{EntryWithPhonetic}

\begin{EntryWithPhonetic}{绳子}{sheng2zi5}{11,3}{⽷、⼦}
  \definition[条]{s.}{corda | cordão}
\end{EntryWithPhonetic}

\begin{EntryWithPhonetic}{省}{sheng3}{9}{⽬}[HSK 2]
  \definition*{s.}{Sobrenome Sheng}
  \definition{s.}{província; unidade administrativa, subordinada diretamente ao governo central | capital provincial; refere-se à capital da província, localização da administração provincial | abreviação (de palavras)}
  \definition{v.}{economizar; poupar; reduzir o consumo (em oposição a 费) | omitir; deixar de fora}
  \seeref{xing3}
  \seealsoref{费}{fei4}
\end{EntryWithPhonetic}

\begin{EntryWithPhonetic}{省城}{sheng3cheng2}{9,9}{⽬、⼟}
  \definition{s.}{capital da província}
\end{EntryWithPhonetic}

\begin{EntryWithPhonetic}{省会}{sheng3hui4}{9,6}{⽬、⼈}
  \definition{s.}{capital da província}
\end{EntryWithPhonetic}

\begin{EntryWithPhonetic}{省俭}{sheng3jian3}{9,9}{⽬、⼈}
  \definition{s.}{econômico | frugal}
  \definition{v.}{economizar}
\end{EntryWithPhonetic}

\begin{EntryWithPhonetic}{省力}{sheng3li4}{9,2}{⽬、⼒}
  \definition{v.}{economizar esforço ou trabalho}
\end{EntryWithPhonetic}

\begin{EntryWithPhonetic}{省钱}{sheng3 qian2}{9,10}{⽬、⾦}[HSK 6]
  \definition{adj.}{barato; não caro}
  \definition{v.}{economizar dinheiro}
\end{EntryWithPhonetic}

\begin{EntryWithPhonetic}{省却}{sheng3que4}{9,7}{⽬、⼙}
  \definition{v.}{livrar-se (para economizar espaço) | salvar}
\end{EntryWithPhonetic}

\begin{EntryWithPhonetic}{省心}{sheng3xin1}{9,4}{⽬、⼼}
  \definition{adj.}{despreocupado}
  \definition{v.}{ser poupado de preocupações | despreocupar-se}
\end{EntryWithPhonetic}

\begin{EntryWithPhonetic}{省长}{sheng3zhang3}{9,4}{⽬、⾧}
  \definition[位,任]{s.}{governador; governador de uma província}
\end{EntryWithPhonetic}

\begin{EntryWithPhonetic}{圣}{sheng4}{5}{⼟}
  \definition*{s.}{Sobrenome Sheng}
  \definition{adj.}{santo; sagrado | imperial}
  \definition{s.}{santo; sábio | imperador | o maior mestre de uma determinada arte ou habilidade}
\end{EntryWithPhonetic}

\begin{EntryWithPhonetic}{圣诞节}{sheng4 dan4 jie2}{5,8,5}{⼟、⾔、⾋}[HSK 6]
  \definition*{s.}{Natal; Nascimento de Jesus Cristo em 25 de dezembro}
\end{EntryWithPhonetic}

\begin{EntryWithPhonetic}{圣地}{sheng4di4}{5,6}{⼟、⼟}
  \definition{s.}{terra santa (de uma religião) | lugar sagrado | santuário | cidade santa (como Jerusalém, Meca, etc.) | centro de interesse histórico}
\end{EntryWithPhonetic}

\begin{EntryWithPhonetic}{胜}{sheng4}{9}{⾁}[HSK 3]
  \definition{adj.}{soberbo; maravilhoso; adorável}
  \definition[场]{s.}{vitória; sucesso | penteado de mulher; joias usadas pelas mulheres na antiguidade}
  \definition{v.}{vencer (oposto de 负, 败) | derrotar | (frequentemente seguido por 于, etc.) superar; ser superior a; levar a melhor sobre | vencer; ter sucesso; derrotar o adversário | ultrapassar; ser superior ao outro | suportar; ser capaz de suportar ou aguentar}
  \seealsoref{败}{bai4}
  \seealsoref{负}{fu4}
  \seealsoref{于}{yu2}
\end{EntryWithPhonetic}

\begin{EntryWithPhonetic}{胜负}{sheng4fu4}{9,6}{⾁、⾙}[HSK 5]
  \definition{s.}{vitória ou derrota; sucesso ou fracasso}
\end{EntryWithPhonetic}

\begin{EntryWithPhonetic}{胜利}{sheng4li4}{9,7}{⾁、⼑}[HSK 3]
  \definition{adv.}{com sucesso; triunfantemente; atingir o objetivo previsto}
  \definition{v.}{ganhar; vencer; triunfar; ter sucesso}
\end{EntryWithPhonetic}

\begin{EntryWithPhonetic}{胜算}{sheng4suan4}{9,14}{⾁、⽵}
  \definition{s.}{probabilidade de sucesso | estratégia que garante o sucesso}
  \definition{v.}{ter certeza do sucesso}
\end{EntryWithPhonetic}

\begin{EntryWithPhonetic}{乘}{sheng4}{10}{⽲}
  \definition{clas.}{usado para carruagens de guerra puxada por quatro cavalos}
  \definition{s.}{obras históricas; livros de história geral | antigamente, uma carruagem puxada por quatro cavalos}
  \seeref{cheng2}
\end{EntryWithPhonetic}

\begin{EntryWithPhonetic}{盛}{sheng4}{11}{⽫}
  \definition*{s.}{Sobrenome Sheng}
  \definition{adj.}{florescente; próspero | vigoroso; enérgico | grandioso; magnífico | abundante; profundo | popular; comum; difundido; universal | amplo; generoso; abundante; suficiente | ótimo}
  \definition{adv.}{muito; profundamente}
  \seeref{cheng2}
\end{EntryWithPhonetic}

\begin{EntryWithPhonetic}{盛行}{sheng4xing2}{11,6}{⽫、⾏}[HSK 6]
  \definition{v.}{predominar; estar atual; estar na moda; ser amplamente popular}
\end{EntryWithPhonetic}

\begin{EntryWithPhonetic}{盛宴}{sheng4yan4}{11,10}{⽫、⼧}
  \definition{s.}{celebração}
\end{EntryWithPhonetic}

\begin{EntryWithPhonetic}{剩}{sheng4}{12}{⼑}[HSK 5]
  \definition*{s.}{Sobrenome Sheng}
  \definition{v.}{permanecer; ser deixado (para trás)}
\end{EntryWithPhonetic}

\begin{EntryWithPhonetic}{剩下}{sheng4 xia4}{12,3}{⼑、⼀}[HSK 5]
  \definition{v.}{permanecer; ser deixado (para trás); consumir e utilizar, restando apenas os resíduos}
\end{EntryWithPhonetic}

\begin{EntryWithPhonetic}{失}{shi1}{5}{⼤}
  \definition{s.}{deslize; erro; defeito; acidente}
  \definition{v.}{perder (oposto de 得) | perder; deixar escapar | não agir de acordo com; negligenciar; violar | perder o controle de | errar; cometer um deslize; apresentar defeito em | não consiguir encontrar | não conseguir atingir o objetivo | desviar-se do normal | quebrar (uma promessa); voltar atrás (na palavra dada) | não conseguir obter | se perder}
  \seealsoref{得}{de2}
\end{EntryWithPhonetic}

\begin{EntryWithPhonetic}{失败}{shi1bai4}{5,8}{⼤、⾒}[HSK 4]
  \definition{adj.}{insatisfatório; a maneira como as coisas aconteceram deixou muito a desejar; o resultado final deixou muito a desejar}
  \definition{v.}{perder; ser derrotado; não vencer em uma guerra ou competição | falhar; fracassar; não dar em nada; falhar em atingir um objetivo ou meta desejada (trabalho, carreira, etc.)}
\end{EntryWithPhonetic}

\begin{EntryWithPhonetic}{失落}{shi1luo4}{5,12}{⼤、⾋}
  \definition{s.}{frustração | decepção | perda}
  \definition{v.}{perder (algo) | cair (algo) | sentir uma sensação de perda}
\end{EntryWithPhonetic}

\begin{EntryWithPhonetic}{失眠}{shi1mian2}{5,10}{⼤、⽬}
  \definition{s.}{insônia}
  \definition{v.}{ter insônia}
\end{EntryWithPhonetic}

\begin{EntryWithPhonetic}{失去}{shi1qu4}{5,5}{⼤、⼛}[HSK 3]
  \definition{v.}{perder}
\end{EntryWithPhonetic}

\begin{EntryWithPhonetic}{失望}{shi1wang4}{5,11}{⼤、⽉}[HSK 4]
  \definition{adj.}{desapontado; decepcionado}
  \definition{v.}{ficar desapontado; ficar decepcionado; estar desapontado; sentir-se sem esperança; perder a confiança}
\end{EntryWithPhonetic}

\begin{EntryWithPhonetic}{失误}{shi1wu4}{5,9}{⼤、⾔}[HSK 5]
  \definition[个]{s.}{erro; engano; equívoco; erros causados por negligência ou medidas inadequadas}
  \definition{v.}{cometer um erro; cometer um equívoco}
\end{EntryWithPhonetic}

\begin{EntryWithPhonetic}{失业}{shi1ye4}{5,5}{⼤、⼀}[HSK 4]
  \definition{v.}{não ter emprego; estar desempregado; estar sem trabalho; refere-se àqueles que estão dentro da idade legal para trabalhar, têm capacidade para trabalhar, estão desempregados e querem encontrar um emprego, mas não conseguem; embora se envolvam em certos trabalhos sociais, sua remuneração é menor do que o padrão mínimo de vida urbano local e são considerados desempregados}
\end{EntryWithPhonetic}

\begin{EntryWithPhonetic}{失意}{shi1yi4}{5,13}{⼤、⼼}
  \definition{adj.}{desapontado | frustrado}
\end{EntryWithPhonetic}

\begin{EntryWithPhonetic}{师}{shi1}{6}{⼱}
  \definition*{s.}{Sobrenome Shi}
  \definition[位,名,个]{s.}{professor; tutor; mestre | exemplo; modelo a seguir | título honorífico para um monge budista; (termo de respeito para um monge ou freira) mestre; mãe | do seu mestre ou professor | divisão; tropas; exército}
  \definition{suf.}{pessoa qualificada em determinada profissão}
  \definition{v.}{Literário: imitar; aprender}
\end{EntryWithPhonetic}

\begin{EntryWithPhonetic}{师父}{shi1 fu5}{6,4}{⼱、⽗}[HSK 6]
  \definition[个,位,名,些]{s.}{mestre; mestre trabalhador; um título respeitoso dado por um aprendiz ao seu mestre | um título respeitoso para monges, freiras e sacerdotes taoístas}
\end{EntryWithPhonetic}

\begin{EntryWithPhonetic}{师傅}{shi1fu5}{6,12}{⼱、⼈}[HSK 5]
  \definition[个,位,名]{s.}{mestre; um trabalhador qualificado; título honorífico para pessoas habilidosas | mestre; professor (em certos ofícios); pessoas que ensinam técnicas em áreas como engenharia, comércio e teatro}
\end{EntryWithPhonetic}

\begin{EntryWithPhonetic}{师生}{shi1 sheng1}{6,5}{⼱、⽣}[HSK 6]
  \definition{s.}{mestre e discípulo; professores e alunos; um nome combinado para professores e alunos}
\end{EntryWithPhonetic}

\begin{EntryWithPhonetic}{诗}{shi1}{8}{⾔}[HSK 4]
  \definition[首,句,行]{s.}{poesia; verso; poema; um gênero literário que reflete a vida e expressa emoções por meio de uma linguagem rítmica e rimada}
  \seealsoref{诗经}{shi1jing1}
\end{EntryWithPhonetic}

\begin{EntryWithPhonetic}{诗词}{shi1ci2}{8,7}{⾔、⾔}
  \definition{s.}{verso}
\end{EntryWithPhonetic}

\begin{EntryWithPhonetic}{诗歌}{shi1 ge1}{8,14}{⾔、⽋}[HSK 5]
  \definition[本,首,段]{s.}{poesia; poemas e canções; refere-se a todos os tipos de poesia}
\end{EntryWithPhonetic}

\begin{EntryWithPhonetic}{诗经}{shi1jing1}{8,8}{⾔、⽷}
  \definition*{s.}{Shijing, o Livro das Canções, antiga coleção de poemas chineses e um dos Cinco Clássicos do Confucionismo}
\end{EntryWithPhonetic}

\begin{EntryWithPhonetic}{诗句}{shi1ju4}{8,5}{⾔、⼝}
  \definition[行]{s.}{verso | versículo}
\end{EntryWithPhonetic}

\begin{EntryWithPhonetic}{诗人}{shi1 ren2}{8,2}{⾔、⼈}[HSK 4]
  \definition[个,位,名,些]{s.}{poeta; escritor de poesia}
\end{EntryWithPhonetic}

\begin{EntryWithPhonetic}{诗意}{shi1yi4}{8,13}{⾔、⼼}
  \definition{adj.}{poético}
  \definition{s.}{poesia}
\end{EntryWithPhonetic}

\begin{EntryWithPhonetic}{湿}{shi1}{12}{⽔}[HSK 4]
  \definition{adj.}{molhado; úmido; algo com água ou com muita água dentro}
\end{EntryWithPhonetic}

\begin{EntryWithPhonetic}{十}{shi2}{2}{⼗}[HSK 1][Kangxi 24]
  \definition*{s.}{Sobrenome Shi}
  \definition{num.}{dez; 10 | dezena | completo; no topo; máximo; referindo-se a algo que atingiu o ápice da perfeição ou plenitude | um monte de; indica que há muitos}
\end{EntryWithPhonetic}

\begin{EntryWithPhonetic}{十分}{shi2fen1}{2,4}{⼗、⼑}[HSK 2]
  \definition{adv.}{muito; totalmente; completamente; extremamente; indica um nível muito alto}
\end{EntryWithPhonetic}

\begin{EntryWithPhonetic}{十足}{shi2zu2}{2,7}{⼗、⾜}[HSK 5]
  \definition{adj.}{puro e simples; apenas este componente ou esta característica é muito evidente | 100\%; completo; total; muito satisfatório; muito adequado}
\end{EntryWithPhonetic}

\begin{EntryWithPhonetic}{什}{shi2}{4}{⼈}
  \definition*{s.}{Sobrenome Shi}
  \definition{adj.}{variado; sortido; diverso; vários; misturados}
  \definition{num.}{(em frações ou múltiplos) dez}
  \definition{s.}{várias coisas; artigos diversos}
  \seeref{shen2}
\end{EntryWithPhonetic}

\begin{EntryWithPhonetic}{石}{shi2}{5}{⽯}[Kangxi 112]
  \definition*{s.}{Sobrenome Shi}
  \definition{s.}{pedra; rocha; o material duro que constitui a crosta terrestre é composto por uma coleção de minerais | inscrição em pedra; esculturas em pedra}
  \seeref{dan4}
\end{EntryWithPhonetic}

\begin{EntryWithPhonetic}{石头}{shi2tou5}{5,5}{⽯、⼤}[HSK 3]
  \definition[块,堆,些]{s.}{rocha; pedra; uma substância muito dura que é o principal material da superfície da Terra}
\end{EntryWithPhonetic}

\begin{EntryWithPhonetic}{石油}{shi2you2}{5,8}{⽯、⽔}[HSK 3]
  \definition[桶,吨,升]{s.}{óleo; óleo fóssil; petróleo; um líquido inflamável extraído do solo, geralmente marrom escuro, preto ou verde escuro, do qual gasolina e outras substâncias podem ser obtidas}
\end{EntryWithPhonetic}

\begin{EntryWithPhonetic}{时}{shi2}{7}{⽇}[HSK 3]
  \definition*{s.}{Sobrenome Shi}
  \definition{adj.}{atual; presente | temporário; oportuno}
  \definition{adv.}{de vez em quando; ocasionalmente; de ​​tempos em tempos; equivalente a 常常 ou 经常 | às vezes\dots às vezes\dots; dois caracteres 时 usados juntos são equivalentes a ``有时……有时……'' e ``一会儿……一会儿……''}
  \definition{clas.}{hora, cada uma das 24 partes iguais de um dia e uma noite; também usada como unidade legal de tempo}
  \definition{s.}{dias; tempos; longo período de tempo; refere-se a um período de tempo | tempo; tempo fixo; refere-se ao tempo especificado | hora; hora do dia | temporada | chance; oportunidade; momento oportuno | atual; presente | tempo verbal; uma categoria gramatical que utiliza certas formas gramaticais para indicar o momento em que uma ação ocorre; geralmente é dividida em presente, pretérito e futuro}
  \seealsoref{常常}{chang2 chang2}
  \seealsoref{经常}{jing1chang2}
  \seealsoref{一会儿……一会儿……}{yi1hui4r5 yi1hui4r5}
  \seealsoref{有时……有时……}{you3shi2 you3shi2}
\end{EntryWithPhonetic}

\begin{EntryWithPhonetic}{时差}{shi2cha1}{7,9}{⽇、⼯}
  \definition{s.}{diferença de tempo | \emph{jet lag}}
\end{EntryWithPhonetic}

\begin{EntryWithPhonetic}{时常}{shi2chang2}{7,11}{⽇、⼱}[HSK 5]
  \definition{adv.}{frequentemente; com frequência}
\end{EntryWithPhonetic}

\begin{EntryWithPhonetic}{时代}{shi2dai4}{7,5}{⽇、⼈}[HSK 3]
  \definition[个]{s.}{idade; era; tempos; época; períodos e fases históricas divididas de acordo com condições econômicas, políticas, culturais e outras | um período na vida de alguém; uma fase na vida de uma pessoa}
\end{EntryWithPhonetic}

\begin{EntryWithPhonetic}{时而}{shi2'er2}{7,6}{⽇、⽽}[HSK 6]
  \definition{adv.}{às vezes; de tempos em tempos; indica que algo acontece repetidamente em intervalos irregulares}
\end{EntryWithPhonetic}

\begin{EntryWithPhonetic}{时而……,时而……}{shi2'er2 shi2'er2}{7,6,7,6}{⽇、⽽、⽇、⽽}[HSK 6]
  \definition{adv.}{agora\dots, agora\dots; às vezes\dots, às vezes\dots; usado antes e depois; indica que diferentes fenômenos ou coisas ocorrem alternadamente ou mudam continuamente dentro de um determinado período de tempo}[\underline{时而}下雨,\underline{时而}晴天。===Às vezes chove, às vezes faz sol. | 这个地方\underline{时而}热,\underline{时而}冷。===Este lugar às vezes é quente e às vezes frio.]
\end{EntryWithPhonetic}

\begin{EntryWithPhonetic}{时光}{shi2guang1}{7,6}{⽇、⼉}[HSK 5]
  \definition[台]{s.}{tempo; passagem do tempo | dias; horas; anos; épocas; períodos}
\end{EntryWithPhonetic}

\begin{EntryWithPhonetic}{时候}{shi2hou5}{7,10}{⽇、⼈}[HSK 1]
  \definition[个]{s.}{(um ponto no) tempo; momento; um determinado momento no tempo | (a duração do) tempo; um período de tempo com início e fim}
\end{EntryWithPhonetic}

\begin{EntryWithPhonetic}{时机}{shi2ji1}{7,6}{⽇、⽊}[HSK 5]
  \definition[个]{s.}{oportunidade; momento oportuno}
\end{EntryWithPhonetic}

\begin{EntryWithPhonetic}{时间}{shi2jian1}{7,7}{⽇、⾨}[HSK 1]
  \definition[段]{s.}{tempo; refere-se à forma de existência do movimento da matéria, um sistema contínuo composto pelo passado, presente e futuro | tempo; período (duração); um período de tempo com início e fim | tempo (um ponto); em algum momento do tempo}
\end{EntryWithPhonetic}

\begin{EntryWithPhonetic}{时节}{shi2 jie2}{7,5}{⽇、⾋}[HSK 6]
  \definition{s.}{temporada; um período de tempo em um ano com certas características, geralmente relacionadas à estação ou ao termo solar | época; tempo}
\end{EntryWithPhonetic}

\begin{EntryWithPhonetic}{时刻}{shi2ke4}{7,8}{⽇、⼑}[HSK 3]
  \definition{adv.}{constantemente; sempre; a cada momento; frequentemente}
  \definition[个,段]{s.}{tempo; hora; momento; conjuntura; um ponto no tempo}
\end{EntryWithPhonetic}

\begin{EntryWithPhonetic}{时期}{shi2qi1}{7,12}{⽇、⽉}[HSK 6]
  \definition[个,段]{s.}{um período específico; um período de tempo com uma certa característica}
\end{EntryWithPhonetic}

\begin{EntryWithPhonetic}{时时}{shi2 shi2}{7,7}{⽇、⽇}[HSK 6]
  \definition{adv.}{frequentemente; sempre; constantemente; indica que algo acontece várias vezes dentro de um determinado período de tempo}
\end{EntryWithPhonetic}

\begin{EntryWithPhonetic}{时事}{shi2shi4}{7,8}{⽇、⼅}[HSK 5]
  \definition{s.}{acontecimentos atuais; assuntos atuais; eventos atuais | tendências atuais | como as coisas estão indo | a situação atual}
\end{EntryWithPhonetic}

\begin{EntryWithPhonetic}{时装}{shi2 zhuang1}{7,12}{⽇、⾐}[HSK 6]
  \definition{s.}{vestido da moda; a última moda; os últimos estilos de roupas | roupas contemporâneas (em oposição ao 古装)}
  \seealsoref{古装}{gu3 zhuang1}
\end{EntryWithPhonetic}

\begin{EntryWithPhonetic}{识}{shi2}{7}{⾔}[HSK 6]
  \definition{s.}{percepção; conhecimento}
  \definition{v.}{saber; reconhecer | saber; entender}
  \seeref{zhi4}
\end{EntryWithPhonetic}

\begin{EntryWithPhonetic}{识字}{shi2 zi4}{7,6}{⾔、⼦}[HSK 6]
  \definition{v.}{aprender a ler; tornar-se alfabetizado; reconhecer caracteres}
\end{EntryWithPhonetic}

\begin{EntryWithPhonetic}{实}{shi2}{8}{⼧}
  \definition{adj.}{sólido; cheio por dentro; sem espaços vazios (oposto de 虚) | verdadeiro; real; atual; sincero | forte; eficaz; concreto; real}
  \definition{adv.}{verdadeiramente; realmente; de fato; originalmente}
  \definition{s.}{fato; realidade | semente; fruto}
  \definition{v.}{preencher}
  \seealsoref{虚}{xu1}
\end{EntryWithPhonetic}

\begin{EntryWithPhonetic}{实惠}{shi2hui4}{8,12}{⼧、⼼}[HSK 5]
  \definition{adj.}{sólido; substancial; benefícios práticos}
  \definition{s.}{benefício material; benefícios tangíveis; benefícios reais}
\end{EntryWithPhonetic}

\begin{EntryWithPhonetic}{实际}{shi2ji4}{8,7}{⼧、⾩}[HSK 2]
  \definition{adj.}{real; efetivo; concreto; prático | factual; prático; realista; de acordo com os fatos}
  \definition{s.}{realidade; prática; coisas e situações que existem objetivamente}
\end{EntryWithPhonetic}

\begin{EntryWithPhonetic}{实际上}{shi2 ji4 shang4}{8,7,3}{⼧、⾩、⼀}[HSK 3]
  \definition{adv.}{de fato; na verdade}
\end{EntryWithPhonetic}

\begin{EntryWithPhonetic}{实践}{shi2jian4}{8,12}{⼧、⾜}[HSK 6]
  \definition{s.}{prática; filosoficamente, refere-se às ações conscientes das pessoas para transformar a natureza e a sociedade; as atividades de produção são as atividades práticas mais básicas e também incluem atividades políticas, experimentos científicos, educação cultural, etc.}
  \definition{v.}{praticar; realizar; implementar planos e intenções em ações específicas}
\end{EntryWithPhonetic}

\begin{EntryWithPhonetic}{实力}{shi2li4}{8,2}{⼧、⼒}[HSK 3]
  \definition{s.}{força real; geralmente se refere à força militar e econômica de um país, grupo ou indivíduo, e também se refere à capacidade de um indivíduo ou grupo em uma competição}
\end{EntryWithPhonetic}

\begin{EntryWithPhonetic}{实施}{shi2shi1}{8,9}{⼧、⽅}[HSK 4]
  \definition{v.}{colocar em vigor; implementar (leis, políticas, etc.); executar; trazer (colocar) algo em vigor; fazer cumprir; colocar algo em (prática)}
\end{EntryWithPhonetic}

\begin{EntryWithPhonetic}{实习}{shi2xi2}{8,3}{⼧、⼄}[HSK 2]
  \definition{s.}{estagiário; prática; estágio}
  \definition{v.}{aplicar e testar os conhecimentos teóricos aprendidos no trabalho prático, a fim de exercitar a capacidade profissional}
\end{EntryWithPhonetic}

\begin{EntryWithPhonetic}{实现}{shi2xian4}{8,8}{⼧、⾒}[HSK 2]
  \definition{v.}{alcançar; atingir; realizar; concretizar; tornar (ideais, planos, etc.) realidade}
\end{EntryWithPhonetic}

\begin{EntryWithPhonetic}{实行}{shi2xing2}{8,6}{⼧、⾏}[HSK 3]
  \definition{v.}{praticar; implementar; executar; colocar em prática; realizar (programa, política, plano, etc.) por meio de ação}
\end{EntryWithPhonetic}

\begin{EntryWithPhonetic}{实验}{shi2yan4}{8,10}{⼧、⾺}[HSK 3]
  \definition[个,次]{s.}{teste; experimento; trabalho de laboratório}
  \definition{v.}{testar; experimentar; realizar uma operação ou se envolver em uma atividade para testar uma teoria ou hipótese científica}
\end{EntryWithPhonetic}

\begin{EntryWithPhonetic}{实验室}{shi2 yan4 shi4}{8,10,9}{⼧、⾺、⼧}[HSK 3]
  \definition[个,间]{s.}{laboratório; salas especiais para experimentos científicos}
\end{EntryWithPhonetic}

\begin{EntryWithPhonetic}{实用}{shi2yong4}{8,5}{⼧、⽤}[HSK 4]
  \definition{adj.}{prático; pragmático; funcional; atende aos requisitos reais da aplicação}
  \definition{v.}{colocar em uso prático}
\end{EntryWithPhonetic}

\begin{EntryWithPhonetic}{实在}{shi2zai4}{8,6}{⼧、⼟}[HSK 2]
  \definition{adj.}{honesto; sincero | verdadeiro; honesto; realista; não é falso, não é enganador}
  \definition{adv.}{verdadeiramente; de fato; na verdade; usado para reforçar o tom afirmativo, enfatizando que a situação é realmente assim}
\end{EntryWithPhonetic}

\begin{EntryWithPhonetic}{拾}{shi2}{9}{⼿}[HSK 5]
  \definition{num.}{dez (usado no lugar do numeral 十 em cheques, notas bancárias, etc., para evitar erros ou alterações)}
  \definition{v.}{pegar (do chão); recolher}
\end{EntryWithPhonetic}

\begin{EntryWithPhonetic}{食}{shi2}{9}{⾷}[Kangxi 184]
  \definition{adj.}{para cozinhar; comestível}
  \definition{s.}{refeição; comida; o que as pessoas e os animais comem | alimentação; alimento para animais; ração | eclipse solar; eclipse lunar}
  \definition{v.}{comer}
  \seeref{si4}
\end{EntryWithPhonetic}

\begin{EntryWithPhonetic}{食品}{shi2 pin3}{9,9}{⾷、⼝}[HSK 3]
  \definition[种]{s.}{comida; gêneros alimentícios; provisões; alimentos vendidos em lojas que passaram por algum processamento}
\end{EntryWithPhonetic}

\begin{EntryWithPhonetic}{食堂}{shi2 tang2}{9,11}{⾷、⼟}[HSK 4]
  \definition[个,间]{s.}{cantina; refeitório}
\end{EntryWithPhonetic}

\begin{EntryWithPhonetic}{食物}{shi2wu4}{9,8}{⾷、⽜}[HSK 2]
  \definition[种]{s.}{comida; alimentos; comestíveis}
\end{EntryWithPhonetic}

\begin{EntryWithPhonetic}{食欲}{shi2 yu4}{9,11}{⾷、⽋}[HSK 6]
  \definition{adj.}{apetitoso}
  \definition{s.}{apetite; desejo humano de comer}
\end{EntryWithPhonetic}

\begin{EntryWithPhonetic}{使}{shi3}{8}{⼈}[HSK 3]
  \definition{conj.}{se; supondo; usado como a primeira cláusula de uma frase complexa; indica uma relação hipotética; equivalente a 假如}
  \definition{s.}{enviado; mensageiro; pessoas em uma missão}
  \definition{v.}{enviar; despachar; dizer a alguém para fazer algo | usar; empregar; aplicar | deixar; chamar; habilitar}
  \seealsoref{假如}{jia3ru2}
\end{EntryWithPhonetic}

\begin{EntryWithPhonetic}{使得}{shi3 de5}{8,11}{⼈、⼻}[HSK 5]
  \definition{v.}{ser utilizável; poder ser usado | ser viável; ser exequível; ser possível;  poder fazer | fazer; tornar; causar um determinado resultado (intenção, plano, coisa)}
\end{EntryWithPhonetic}

\begin{EntryWithPhonetic}{使劲}{shi3/jin4}{8,7}{⼈、⼒}[HSK 4]
  \definition{v.+compl.}{colocar energia; exercer toda a sua força | esforçar-se para ajudar; colocar energia para ajudar}
\end{EntryWithPhonetic}

\begin{EntryWithPhonetic}{使用}{shi3yong4}{8,5}{⼈、⽤}[HSK 2]
  \definition{v.}{usar; empregar; aplicar; fazer com que pessoas, equipamentos, fundos, etc. sirvam a um determinado propósito}
\end{EntryWithPhonetic}

\begin{EntryWithPhonetic}{始}{shi3}{8}{⼥}
  \definition*{s.}{Sobrenome Shi}
  \definition{adv.}{somente então; não\dots até}
  \definition{s.}{começo; início}
  \definition{v.}{começar; iniciar}
\end{EntryWithPhonetic}

\begin{EntryWithPhonetic}{始终}{shi3zhong1}{8,8}{⼥、⽷}[HSK 3]
  \definition{adv.}{sempre; o tempo todo; durante todo; do começo ao fim; indica continuidade do início ao fim}
  \definition{s.}{todo o processo do começo ao fim}
\end{EntryWithPhonetic}

\begin{EntryWithPhonetic}{屎}{shi3}{9}{⼫}
  \definition{s.}{fezes | excrementos | (forma ligada) secreção (do ouvido, olho, etc.)}
\end{EntryWithPhonetic}

\begin{EntryWithPhonetic}{士}{shi4}{3}{⼠}[Kangxi 33]
  \definition*{s.}{Sobrenome Shi}
  \definition[位,名,个]{s.}{soldado; militar | oficial não comissionado; primeira classe de soldados | pessoa treinada em uma determinada área; algum tipo de técnico | pessoa (louvável) | bacharel (na China antiga) | classe social, entre os oficiais, 大夫, e o povo comum, 庶民 | estudioso | guarda-costas, uma das peças do xadrez chinês}
  \seealsoref{大夫}{da4fu1}
  \seealsoref{庶民}{shu4min2}
\end{EntryWithPhonetic}

\begin{EntryWithPhonetic}{士兵}{shi4bing1}{3,7}{⼠、⼋}[HSK 4]
  \definition[个,名,位,批,群]{s.}{soldado; militar; termo coletivo para oficiais não comissionados e soldados; os membros mais jovens do exército}
\end{EntryWithPhonetic}

\begin{EntryWithPhonetic}{世}{shi4}{5}{⼀}
  \definition*{s.}{Sobrenome Shi}
  \definition{s.}{vida; tempo de vida; vida humana | geração; geração após geração | idade; era | o mundo; sociedade | (geologia) época, abaixo de ``período''}
\end{EntryWithPhonetic}

\begin{EntryWithPhonetic}{世代}{shi4dai4}{5,5}{⼀、⼈}
  \definition{adv.}{por muitas gerações, eras}
  \definition{s.}{geração | era}
\end{EntryWithPhonetic}

\begin{EntryWithPhonetic}{世纪}{shi4ji4}{5,6}{⼀、⽷}[HSK 3]
  \definition[个,段]{s.}{século; uma unidade para calcular anos, cem anos é um século}
\end{EntryWithPhonetic}

\begin{EntryWithPhonetic}{世界}{shi4jie4}{5,9}{⼀、⽥}[HSK 3]
  \definition[个,片,种]{s.}{mundo; todos os lugares da Terra | a soma da natureza e da sociedade humana; refere-se à soma de toda a existência objetiva na natureza e na sociedade humana | campo; refere-se a uma determinada área ou campo | o universo sem limites; costumava ser um termo budista, mas agora também se refere ao mundo natural ilimitado e à sociedade humana | situação social; a situação ou atmosfera social de um determinado período}
\end{EntryWithPhonetic}

\begin{EntryWithPhonetic}{世界杯}{shi4jie4bei1}{5,9,8}{⼀、⽥、⽊}[HSK 3]
  \definition*{s.}{Copa do Mundo; Troféu da Copa do Mundo}
\end{EntryWithPhonetic}

\begin{EntryWithPhonetic}{世锦赛}{shi4jin3sai4}{5,13,14}{⼀、⾦、⾙}
  \definition*{s.}{Campeonato Mundial}
\end{EntryWithPhonetic}

\begin{EntryWithPhonetic}{市}{shi4}{5}{⼱}[HSK 2]
  \definition{s.}{mercado; lugar onde se concentra o comércio | cidade; município; áreas densamente povoadas, com indústrias, comércio e cultura desenvolvidos | relativo ao sistema tradicional chinês de pesos e medidas; unidades administrativas, incluindo cidades sob jurisdição direta e cidades sob jurisdição provincial (ou autônoma) | unidade padrão de mercado; pertencente ao sistema municipal (unidades de medida) | preço de transação no mercado}
  \definition{v.}{comprar ou vender; fazer transações}
\end{EntryWithPhonetic}

\begin{EntryWithPhonetic}{市场}{shi4chang3}{5,6}{⼱、⼟}[HSK 3]
  \definition[家]{s.}{mercado (também no abstrato); um lugar fixo onde as pessoas compram e vendem coisas juntas | área de \emph{marketing}; região onde o produto é vendido | âmbito de influência (figurado); uma metáfora para o escopo e o grau em que uma determinada ideia ou comportamento é aceito por outros}
\end{EntryWithPhonetic}

\begin{EntryWithPhonetic}{市尺}{shi4 chi3}{5,4}{⼱、⼫}
  \definition{clas.}{chi, uma unidade tradicional de comprimento, equivalente a 0,333 metros ou 1,094 pés}
\end{EntryWithPhonetic}

\begin{EntryWithPhonetic}{市斤}{shi4jin1}{5,4}{⼱、⽄}
  \definition{clas.}{jin, uma unidade tradicional de peso, cada uma contendo 10 liang (市两)  e equivalente a 0,5 quilogramas ou 1,102 libras}
  \seealsoref{市两}{shi4liang3}
\end{EntryWithPhonetic}

\begin{EntryWithPhonetic}{市两}{shi4liang3}{5,7}{⼱、⼀}
  \definition{clas.}{liang, uma unidade tradicional de peso, igual a 0,1 jin (市斤), e equivalente a 50 gramas ou 1,763 onças}
  \seealsoref{市斤}{shi4jin1}
\end{EntryWithPhonetic}

\begin{EntryWithPhonetic}{市民}{shi4 min2}{5,5}{⼱、⽒}[HSK 6]
  \definition[位,名]{s.}{habitantes da cidade; residente da cidade; moradores da cidade | cidadão; refere-se especificamente aos artesãos e comerciantes de pequeno e médio porte nas cidades da sociedade feudal tardia}
\end{EntryWithPhonetic}

\begin{EntryWithPhonetic}{市亩}{shi4mu3}{5,7}{⼱、⼇}
  \definition{clas.}{mu, uma unidade tradicional de área, igual a 60 zhang quadrados (平方市丈) e equivalente a 6,667 ares ou 0,165 acre}
  \seealsoref{平方市丈}{ping2fang1 shi4 zhang4}
\end{EntryWithPhonetic}

\begin{EntryWithPhonetic}{市区}{shi4 qu1}{5,4}{⼱、⼖}[HSK 4]
  \definition[个]{s.}{\emph{downtown}; centro da cidade; distrito urbano; áreas que ficam dentro dos limites da cidade e geralmente têm uma alta concentração de população e estoque de moradias}
\end{EntryWithPhonetic}

\begin{EntryWithPhonetic}{市升}{shi4sheng1}{5,4}{⼱、⼗}
  \definition{clas.}{sheng; uma unidade tradicional de volume, equivalente a 1 litro ou 1,76 \emph{pints} ou 0,22 galão}
\end{EntryWithPhonetic}

\begin{EntryWithPhonetic}{市长}{shi4 zhang3}{5,4}{⼱、⾧}[HSK 2]
  \definition[个,位,名]{s.}{prefeito; chefe administrativo responsável pela administração de uma cidade}
\end{EntryWithPhonetic}

\begin{EntryWithPhonetic}{市中心}{shi4zhong1xin1}{5,4,4}{⼱、⼁、⼼}
  \definition{s.}{centro da cidade}
\end{EntryWithPhonetic}

\begin{EntryWithPhonetic}{示}{shi4}{5}{⽰}[Kangxi 113]
  \definition*{s.}{Sobrenome Shi}
  \definition{s.}{(sua) carta  | missiva; instruções; palavras ou escritos para subordinados ou gerações mais jovens}
  \definition{v.}{mostrar; notificar; instruir | indicar; significar; mostrar ou apontar, fazer conhecido}
\end{EntryWithPhonetic}

\begin{EntryWithPhonetic}{示范}{shi4fan4}{5,9}{⽰、⾋}[HSK 5]
  \definition{v.}{demonstrar; dar o exemplo; criar um modelo que todos possam aprender}
\end{EntryWithPhonetic}

\begin{EntryWithPhonetic}{似}{shi4}{6}{⼈}
  \definition{v.}{ver; parecer}
  \seeref{si4}
\end{EntryWithPhonetic}

\begin{EntryWithPhonetic}{似的}{shi4de5}{6,8}{⼈、⽩}[HSK 4]
  \definition{part.}{como; como\dots como; como se (embora); usada após uma palavra ou frase para indicar uma semelhança com algo ou uma situação | usada para indicar alto grau}
\end{EntryWithPhonetic}

\begin{EntryWithPhonetic}{式}{shi4}{6}{⼷}[HSK 5]
  \definition*{s.}{Sobrenome Shi}
  \definition{s.}{tipo; estilo | forma; padrão | ritual; cerimônia | fórmula; conjunto de símbolos que expressam uma lei natural na ciência natural | humor; modo; categoria gramatical que expressa a atitude subjetiva do falante em relação ao que está sendo dito, como narrativa, imperativa e condicional}
\end{EntryWithPhonetic}

\begin{EntryWithPhonetic}{事}{shi4}{8}{⼅}[HSK 1]
  \definition[件,桩,回]{s.}{assunto; questão; coisa; negócio | problema; acidente | emprego; trabalho | responsabilidade; envolvimento | caso, coisa; o que aconteceu}
  \definition{v.}{servir; atender | estar envolvido em; dedicar-se a}
\end{EntryWithPhonetic}

\begin{EntryWithPhonetic}{事故}{shi4gu4}{8,9}{⼅、⽁}[HSK 3]
  \definition[起,桩,次,场]{s.}{acidente; perdas ou desastres repentinos, muitas vezes relacionados ao transporte, produção, trabalho e segurança pessoal}
\end{EntryWithPhonetic}

\begin{EntryWithPhonetic}{事后}{shi4 hou4}{8,6}{⼅、⼝}[HSK 6]
  \definition{s.}{depois; depois do evento; após o incidente ocorrer ou o problema ser resolvido}
\end{EntryWithPhonetic}

\begin{EntryWithPhonetic}{事件}{shi4jian4}{8,6}{⼅、⼈}[HSK 3]
  \definition[个,件,次]{s.}{evento; incidente; grandes eventos na história ou na sociedade}
\end{EntryWithPhonetic}

\begin{EntryWithPhonetic}{事情}{shi4qing5}{8,11}{⼅、⼼}[HSK 2]
  \definition[件,个,些,种]{s.}{assunto; questão; coisa; negócio | erro; acidente; infortúnio | (coloquial) emprego; trabalho}
\end{EntryWithPhonetic}

\begin{EntryWithPhonetic}{事儿}{shi4r5}{8,2}{⼅、⼉}
  \definition[件,桩]{s.}{o emprego | negócio | afazeres | assunto que precisa ser resolvido | matéria}
\end{EntryWithPhonetic}

\begin{EntryWithPhonetic}{事实}{shi4shi2}{8,8}{⼅、⼧}[HSK 3]
  \definition[个,件]{s.}{mito; lenda; uma narrativa sobre alguém ou algo que foi transmitida oralmente}
  \definition{v.}{dizer; contar; ser dito; contar a história}
\end{EntryWithPhonetic}

\begin{EntryWithPhonetic}{事实上}{shi4 shi2 shang4}{8,8,3}{⼅、⼧、⼀}[HSK 3]
  \definition{adv.}{realmente; de fato; na realidade; na verdade; de fato}
\end{EntryWithPhonetic}

\begin{EntryWithPhonetic}{事物}{shi4wu4}{8,8}{⼅、⽜}[HSK 4]
  \definition[种,类,个]{s.}{coisa; objeto; todos os objetos e fenômenos que existem objetivamente}
\end{EntryWithPhonetic}

\begin{EntryWithPhonetic}{事先}{shi4xian1}{8,6}{⼅、⼉}[HSK 4]
  \definition{adv.}{antes; de antemão; com antecedência; antecipadamente}
\end{EntryWithPhonetic}

\begin{EntryWithPhonetic}{事业}{shi4ye4}{8,5}{⼅、⼀}[HSK 3]
  \definition[个]{s.}{causa; carreira; empreendimento; atividades regulares realizadas por pessoas com um determinado objetivo, escala e sistema que têm impacto no desenvolvimento social | instituição; instalações; unidade de trabalho apoiada financeiramente pelo governo; refere-se especificamente a empresas que não têm rendimentos de produção, são financiadas pelo Estado e não realizam contabilidade económica}
\end{EntryWithPhonetic}

\begin{EntryWithPhonetic}{势}{shi4}{8}{⼒}
  \definition{s.}{poder; força; influência | momentum; tendência | aparência externa de um objeto natural; fenômenos ou situações naturais | situação; estado de coisas; circunstâncias | sinal; gesto | genitais masculinos}
\end{EntryWithPhonetic}

\begin{EntryWithPhonetic}{势力}{shi4li4}{8,2}{⼒、⼒}[HSK 5]
  \definition[股]{s.}{força; poder; influência; forças políticas, econômicas, militares, etc.}
\end{EntryWithPhonetic}

\begin{EntryWithPhonetic}{视}{shi4}{8}{⾒}
  \definition{v.}{olhar para | considerar; olhar para | inspecionar; observar}
\end{EntryWithPhonetic}

\begin{EntryWithPhonetic}{视角}{shi4jiao3}{8,7}{⾒、⾓}
  \definition{s.}{ângulo do qual se observa um objeto | (figurativo) perspectiva, ponto de vista, quadro de referência | (cinematografia) ângulo da câmera | (percepção visual) ângulo visual (o ângulo que um objeto visto subtende no olho) | (fotografia) ângulo de visão}
\end{EntryWithPhonetic}

\begin{EntryWithPhonetic}{视频}{shi4pin2}{8,13}{⾒、⾴}[HSK 5]
  \definition[个,段,条]{s.}{vídeo; videoclipe}
\end{EntryWithPhonetic}

\begin{EntryWithPhonetic}{视为}{shi4 wei2}{8,4}{⾒、⼂}[HSK 5]
  \definition{v.}{considerar; ver como; considerar como; considerar ser; achar que é}
\end{EntryWithPhonetic}

\begin{EntryWithPhonetic}{试}{shi4}{8}{⾔}[HSK 1]
  \definition{s.}{teste; exame; avaliação de conhecimentos ou habilidades através de métodos específicos}
  \definition{v.}{tentar; investigar resultados ou verificar a natureza, não se envolver formalmente (em determinada atividade)}
\end{EntryWithPhonetic}

\begin{EntryWithPhonetic}{试点}{shi4 dian3}{8,9}{⾔、⽕}[HSK 6]
  \definition[个]{s.}{local onde um experimento é conduzido; unidade experimental; local de teste; um lugar para pequenos experimentos}
  \definition{v.}{experimentar; fazer experimentos; realizar testes em pontos selecionados; lançar um projeto piloto}
\end{EntryWithPhonetic}

\begin{EntryWithPhonetic}{试卷}{shi4juan4}{8,8}{⾔、⼙}[HSK 4]
  \definition[分,张]{s.}{folha de teste; folha de exame; papel usado para escrever as respostas nos exames}
\end{EntryWithPhonetic}

\begin{EntryWithPhonetic}{试题}{shi4 ti2}{8,15}{⾔、⾴}[HSK 3]
  \definition[道]{s.}{questões de um exame}
\end{EntryWithPhonetic}

\begin{EntryWithPhonetic}{试图}{shi4tu2}{8,8}{⾔、⼞}[HSK 5]
  \definition{v.}{tentar; pretender, fazer o possível para realizar algo}
\end{EntryWithPhonetic}

\begin{EntryWithPhonetic}{试验}{shi4yan4}{8,10}{⾔、⾺}[HSK 3]
  \definition{v.}{testar; fazer um teste; fazer um experimento; para examinar o efeito ou desempenho de algo, primeiro experimente em um laboratório ou em uma escala menor}
\end{EntryWithPhonetic}

\begin{EntryWithPhonetic}{室}{shi4}{9}{⼧}[HSK 3]
  \definition*{s.}{Shi, a décima terceira das vinte e oito constelações da esfera celeste, composta por duas estrelas em linha reta na constelação de Pégaso | Sobrenome Shi}
  \definition{s.}{sala; quarto; casa | departamento; sala como unidade administrativa ou de trabalho; órgãos públicos, fábricas, escolas e outras unidades de trabalho internas | esposa; familiares ou esposa | família; clã | cavidade; órgão com forma semelhante a uma câmara}
\end{EntryWithPhonetic}

\begin{EntryWithPhonetic}{是}{shi4}{9}{⽇}[HSK 1]
  \definition*{s.}{Sobrenome Shi}
  \definition{adj.}{correto; certo | verdadeiro}
  \definition{adv.}{(expressar afirmação firme) de fato; realmente}
  \definition{pron.}{isso; isto |  todos; qualquer um; usado antes de substantivos, tem o significado de 凡是}
  \definition{s.}{assuntos (importantes); grandes planos}
  \definition{v.}{usado como “ser” antes de substantivos ou pronomes para identificar, descrever ou ampliar o sujeito; indica que duas coisas são iguais, ou que a segunda explica a primeira | usado entre duas palavras idênticas; relacionar duas palavras semelhantes |  (usado antes de substantivos) ser exatamente; ser corretamente; usado antes de substantivos, tem o significado de 适合 | elogiar; justificar | expressar afirmação ou concordância (frequentemente usado sozinho) | usado para escolher perguntas, perguntas sim/não ou perguntas retóricas | (usado no início de uma frase) enfatizar uma determinada parte de uma frase | usado em perguntas sim-não}
  \seealsoref{凡是}{fan2shi4}
  \seealsoref{适合}{shi4he2}
\end{EntryWithPhonetic}

\begin{EntryWithPhonetic}{是不是}{shi4 bu2 shi4}{9,4,9}{⽇、⼀、⽇}[HSK 1]
  \definition{expr.}{sim ou não; é ou não é; se ou não; questões levantadas sobre a confirmação e a negação dos fatos}
\end{EntryWithPhonetic}

\begin{EntryWithPhonetic}{是的}{shi4de5}{9,8}{⽇、⽩}
  \definition{adv.}{sim | está certo}
\end{EntryWithPhonetic}

\begin{EntryWithPhonetic}{是否}{shi4fou3}{9,7}{⽇、⼝}[HSK 4]
  \definition{adv.}{se; se ou não; sim ou não}
\end{EntryWithPhonetic}

\begin{EntryWithPhonetic}{适}{shi4}{9}{⾡}
  \definition*{s.}{Sobrenome Shi}
  \definition{adj.}{confortável; bem | adequado; apropriado | certo; oportuno}
  \definition{v.}{ser apto; ser adequado; ser apropriado | ir; seguir; perseguir | (de uma mulher) casar}
\end{EntryWithPhonetic}

\begin{EntryWithPhonetic}{适当}{shi4 dang4}{9,6}{⾡、⼹}[HSK 6]
  \definition{s.}{adequado; apropriado}
\end{EntryWithPhonetic}

\begin{EntryWithPhonetic}{适合}{shi4he2}{9,6}{⾡、⼝}[HSK 3]
  \definition{v.}{servir; caber; se adequar; atender às necessidades de uma determinada situação ou pessoa}
\end{EntryWithPhonetic}

\begin{EntryWithPhonetic}{适应}{shi4ying4}{9,7}{⾡、⼴}[HSK 3]
  \definition{v.}{ajustar-se; adequar-se; adaptar-se; fazer as alterações correspondentes para se adequar à medida que as condições mudam}
\end{EntryWithPhonetic}

\begin{EntryWithPhonetic}{适用}{shi4 yong4}{9,5}{⾡、⽤}[HSK 3]
  \definition{adj.}{adequado; aplicável}
\end{EntryWithPhonetic}

\begin{EntryWithPhonetic}{收}{shou1}{6}{⽁}[HSK 2]
  \definition{expr.}{aos cuidados de (usado na linha de endereço após o nome)}
  \definition{v.}{recolocar; juntar; reunir e juntar coisas espalhadas ou dispersas | recolher; cobrar | ganhar; obter (benefícios econômicos) | colher; recolher; colher ou cortar frutas, legumes, cereais maduros, etc. | aceitar; receber; acolher | controlar; restringir; restringir, controlar os sentimentos ou ações, para voltar ao estado normal | finalizar; parar; concluir; encerrar | prender; deter; colocar sob custódia}
\end{EntryWithPhonetic}

\begin{EntryWithPhonetic}{收藏}{shou1cang2}{6,17}{⽁、⾋}[HSK 6]
  \definition{v.}{coletar; armazenar; consagrar}
\end{EntryWithPhonetic}

\begin{EntryWithPhonetic}{收到}{shou1 dao4}{6,8}{⽁、⼑}[HSK 2]
  \definition{v.}{conseguir; obter; receber; alcançar}
\end{EntryWithPhonetic}

\begin{EntryWithPhonetic}{收费}{shou1 fei4}{6,9}{⽁、⾙}[HSK 3]
  \definition{v.}{cobrar; cobrar taxas}
\end{EntryWithPhonetic}

\begin{EntryWithPhonetic}{收购}{shou1 gou4}{6,8}{⽁、⾙}[HSK 5]
  \definition{v.}{comprar; adquirir; comprar muito em vários lugares | adquirir uma empresa; obter o controle efetivo de uma empresa por meio de dinheiro, transações de ações, etc.}
\end{EntryWithPhonetic}

\begin{EntryWithPhonetic}{收回}{shou1 hui2}{6,6}{⽁、⼞}[HSK 4]
  \definition{v.}{retomar; recuperar; relembrar; recordar; receber de volta o que foi enviado ou emprestado, ou o dinheiro que foi emprestado ou usado | sacar; retirar; recolher; rescindir; cancelar (uma opinião, ordem, etc.)}
\end{EntryWithPhonetic}

\begin{EntryWithPhonetic}{收获}{shou1huo4}{6,10}{⽁、⾋}[HSK 4]
  \definition[次,番,份]{s.}{resultados; ganhos; metaforicamente falando, conhecimento, experiência, etc. obtidos em estudo ou trabalho; os resultados obtidos por meio de trabalho árduo | colheita; colheita de safras}
  \definition{v.}{colher; juntar as colheitas}
\end{EntryWithPhonetic}

\begin{EntryWithPhonetic}{收集}{shou1 ji2}{6,12}{⽁、⾫}[HSK 5]
  \definition{v.}{coletar; reunir; recolher}
\end{EntryWithPhonetic}

\begin{EntryWithPhonetic}{收据}{shou1ju4}{6,11}{⽁、⼿}
  \definition[张]{s.}{recibo | \emph{voucher}}
\end{EntryWithPhonetic}

\begin{EntryWithPhonetic}{收看}{shou1 kan4}{6,9}{⽁、⽬}[HSK 3]
  \definition{v.}{assistir (a um programa de TV)}
\end{EntryWithPhonetic}

\begin{EntryWithPhonetic}{收敛}{shou1lian3}{6,11}{⽁、⽁}
  \definition{v.}{diminuir | desaparecer | fazer desaparecer | exercer restrição | conter (alegria, arrogância, etc.) | constringir | (matemática) convergir}
\end{EntryWithPhonetic}

\begin{EntryWithPhonetic}{收买}{shou1mai3}{6,6}{⽁、⼄}
  \definition{v.}{subornar | comprar}
\end{EntryWithPhonetic}

\begin{EntryWithPhonetic}{收取}{shou1 qu3}{6,8}{⽁、⼜}[HSK 6]
  \definition{v.}{obter; coletar; receber; aceitar o dinheiro pago pela outra parte}
\end{EntryWithPhonetic}

\begin{EntryWithPhonetic}{收入}{shou1ru4}{6,2}{⽁、⼊}[HSK 2]
  \definition[笔,个]{s.}{renda; salário; dinheiro recebido}
  \definition{v.}{receber dinheiro | coletar; receber}
\end{EntryWithPhonetic}

\begin{EntryWithPhonetic}{收拾}{shou1shi5}{6,9}{⽁、⼿}[HSK 5]
  \definition{v.}{arrumar; empacotar; limpar; organizar, policiar, restaurar a normalidade em situações adversas | consertar; reparar; restaurar algo que está danificado ao seu estado ou função original |  punir; punir alguém, geralmente com medidas mais severas | matar}
\end{EntryWithPhonetic}

\begin{EntryWithPhonetic}{收听}{shou1 ting1}{6,7}{⽁、⼝}[HSK 3]
  \definition{v.}{ouvir (rádio)}
\end{EntryWithPhonetic}

\begin{EntryWithPhonetic}{收养}{shou1 yang3}{6,9}{⽁、⼋}[HSK 6]
  \definition{v.}{acolher e criar; adotar; acolher os filhos dos outros e criá-los como se fossem da sua própria família}
\end{EntryWithPhonetic}

\begin{EntryWithPhonetic}{收益}{shou1yi4}{6,10}{⽁、⽫}[HSK 4]
  \definition{s.}{lucro; renda; benefício; ganhos; vantagens ou benefícios obtidos}
\end{EntryWithPhonetic}

\begin{EntryWithPhonetic}{收音机}{shou1yin1ji1}{6,9,6}{⽁、⾳、⽊}[HSK 3]
  \definition[部,台]{s.}{rádio; sem fio; um termo geral para receptores de rádio}
\end{EntryWithPhonetic}

\begin{EntryWithPhonetic}{手}{shou3}{4}{⼿}[HSK 1][Kangxi 64]
  \definition{adj.}{prático; conveniente}
  \definition{adv.}{pessoalmente | para habilidade ou destreza}
  \definition{clas.}{usado para habilidades e competências | usado para indicar o número de vezes em que algo foi feito}
  \definition[双,只]{s.}{mão | pessoa proficiente em determinada atividade | habilidade; meios; referência a habilidades, técnicas ou meios | uma pessoa que faz ou é boa em determinado trabalho}
  \definition{v.}{ter na mão; segurar}
\end{EntryWithPhonetic}

\begin{EntryWithPhonetic}{手臂}{shou3bi4}{4,17}{⼿、⾁}
  \definition{s.}{braço}
\end{EntryWithPhonetic}

\begin{EntryWithPhonetic}{手边}{shou3bian1}{4,5}{⼿、⾡}
  \definition{adv.}{à mão | na mão}
\end{EntryWithPhonetic}

\begin{EntryWithPhonetic}{手表}{shou3biao3}{4,8}{⼿、⾐}[HSK 2]
  \definition[块,只,个]{s.}{relógio de pulso}
\end{EntryWithPhonetic}

\begin{EntryWithPhonetic}{手段}{shou3 duan4}{4,9}{⼿、⽎}[HSK 5]
  \definition[种,个]{s.}{meios; meio; medida; método; métodos e técnicas utilizados para atingir um determinado objetivo | truque; artifício; métodos inadequados de lidar com as pessoas | habilidade; capacidade; delicadeza; sutileza; técnica}
\end{EntryWithPhonetic}

\begin{EntryWithPhonetic}{手法}{shou3fa3}{4,8}{⼿、⽔}[HSK 5]
  \definition[种,个]{s.}{habilidade; técnica; técnicas de criação (de obras literárias e artísticas) | truque; artifício; artimanha; refere-se a métodos inadequados usados para lidar com as pessoas}
\end{EntryWithPhonetic}

\begin{EntryWithPhonetic}{手工}{shou3gong1}{4,3}{⼿、⼯}[HSK 4]
  \definition{s.}{trabalho manual; trabalho feito à mão | método de operação manual; método manual, sem máquina | remuneração por trabalho manual, braçal; custo de mão de obra braçal}
\end{EntryWithPhonetic}

\begin{EntryWithPhonetic}{手工艺人}{shou3gong1 yi4ren2}{4,3,4,2}{⼿、⼯、⾋、⼈}
  \definition{s.}{artesão}
\end{EntryWithPhonetic}

\begin{EntryWithPhonetic}{手机}{shou3ji1}{4,6}{⼿、⽊}[HSK 1]
  \definition[部,台,个]{s.}{celular; telefone celular; telefone móvel}
\end{EntryWithPhonetic}

\begin{EntryWithPhonetic}{手里}{shou3 li3}{4,7}{⼿、⾥}[HSK 4]
  \definition[个]{s.}{(uma situação está) nas mãos de alguém | em mãos}
\end{EntryWithPhonetic}

\begin{EntryWithPhonetic}{手刹}{shou3sha1}{4,8}{⼿、⼑}
  \definition{s.}{freio de mão}
\end{EntryWithPhonetic}

\begin{EntryWithPhonetic}{手术}{shou3shu4}{4,5}{⼿、⽊}[HSK 4]
  \definition[个,次]{s.}{cirurgia; operação (cirúrgica); método de tratamento no qual o médico usa uma faca, tesoura etc. para fazer uma incisão em uma parte do corpo do paciente}
  \definition{v.}{realizar uma cirurgia}
\end{EntryWithPhonetic}

\begin{EntryWithPhonetic}{手套}{shou3tao4}{4,10}{⼿、⼤}[HSK 4]
  \definition[副,套,双,种]{s.}{luvas; itens usados ​​nas mãos, feitos de algodão, lã, couro, etc., para proteger as mãos ou manter o frio longe}
\end{EntryWithPhonetic}

\begin{EntryWithPhonetic}{手续}{shou3xu4}{4,11}{⼿、⽷}[HSK 3]
  \definition[项]{s.}{processo; formalidade; procedimento; procedimentos realizados de acordo com os regulamentos}
\end{EntryWithPhonetic}

\begin{EntryWithPhonetic}{手续费}{shou3 xu4 fei4}{4,11,9}{⼿、⽷、⾙}[HSK 6]
  \definition{s.}{comissão; corretagem; taxa de serviço; taxas a pagar pelos procedimentos de manuseio}
\end{EntryWithPhonetic}

\begin{EntryWithPhonetic}{手指}{shou3zhi3}{4,9}{⼿、⼿}[HSK 3]
  \definition[个,根,只]{s.}{dedo da mão}
\end{EntryWithPhonetic}

\begin{EntryWithPhonetic}{守}{shou3}{6}{⼧}[HSK 4]
  \definition*{s.}{Sobrenome Shou}
  \definition{adv.}{próximo; perto de; perto de algum lugar em posição, perto de algum lugar}
  \definition{v.}{guardar; defender; estar presente para cuidar; não ir embora | manter vigilância; defender do ataque do oponente em uma luta ou confronto | observar; cumprir; respeitar; fazer as coisas como elas devem ser feitas | manter, observar a integridade; honrar a palavra de alguém; manter a palavra de alguém}
\end{EntryWithPhonetic}

\begin{EntryWithPhonetic}{守门员}{shou3men2yuan2}{6,3,7}{⼧、⾨、⼝}
  \definition{s.}{goleiro}
\end{EntryWithPhonetic}

\begin{EntryWithPhonetic}{首}{shou3}{9}{⾸}[HSK 4,6][Kangxi 185]
  \definition*{s.}{Sobrenome Shou}
  \definition{adj.}{primeiro}
  \definition{adv.}{inicialmente; como o primeiro; em primeiro lugar}
  \definition{clas.}{usado para canções e poemas}
  \definition{s.}{cabeça | cabeça; chefe; líder | capital (cidade)}
  \definition{v.}{apresentar acusações contra alguém}
\end{EntryWithPhonetic}

\begin{EntryWithPhonetic}{首次}{shou3 ci4}{9,6}{⾸、⽋}[HSK 6]
  \definition{s.}{o primeiro; pela primeira vez}
\end{EntryWithPhonetic}

\begin{EntryWithPhonetic}{首都}{shou3du1}{9,10}{⾸、⾢}[HSK 3]
  \definition[个,座]{s.}{capital (cidade); a sede do mais alto poder político do país e o centro político do país}
\end{EntryWithPhonetic}

\begin{EntryWithPhonetic}{首脑}{shou3 nao3}{9,10}{⾸、⾁}[HSK 6]
  \definition[位]{s.}{cabeça; líder; chefe}
\end{EntryWithPhonetic}

\begin{EntryWithPhonetic}{首席}{shou3 xi2}{9,10}{⾸、⼱}[HSK 6]
  \definition{adj.}{chefe; a primeira; a posição mais alta}
  \definition{s.}{assento de honra; o assento mais honroso}
\end{EntryWithPhonetic}

\begin{EntryWithPhonetic}{首席执行官}{shou3xi2 zhi2xing2 guan1}{9,10,6,6,8}{⾸、⼱、⼿、⾏、⼧}
  \definition{s.}{\emph{chief executive officer}, CEO}
\end{EntryWithPhonetic}

\begin{EntryWithPhonetic}{首先}{shou3xian1}{9,6}{⾸、⼉}[HSK 3]
  \definition{adv.}{primeiramente; antes de todos os outros}
  \definition{conj.}{acima de tudo; primeiramente; em primeiro lugar}
\end{EntryWithPhonetic}

\begin{EntryWithPhonetic}{首相}{shou3 xiang4}{9,9}{⾸、⽬}[HSK 6]
  \definition*[个,名,位]{s.}{Primeiro-Ministro (Japão, UK, etc.); o mais alto cargo oficial no gabinete de uma monarquia; o chefe do governo central de alguns países não monárquicos às vezes usa esse nome}
\end{EntryWithPhonetic}

\begin{EntryWithPhonetic}{掱}{shou3}{12}{⼿}
  \variantof{手}
\end{EntryWithPhonetic}

\begin{EntryWithPhonetic}{寿}{shou4}{7}{⼨}
  \definition[个,份]{s.}{vida longa; velhice | vida; idade | aniversário | (eufenismo) funerário; preparado antes da morte | longevidade}
\end{EntryWithPhonetic}

\begin{EntryWithPhonetic}{寿司}{shou4 si1}{7,5}{⼨、⼝}[HSK 5]
  \definition[份]{s.}{\emph{sushi}; iguaria tradicional japonesa}
\end{EntryWithPhonetic}

\begin{EntryWithPhonetic}{受}{shou4}{8}{⼜}[HSK 3]
  \definition{v.}{receber; aceitar | sofrer; ser submetido a | aguentar; suportar; tolerar | ser agradável}
\end{EntryWithPhonetic}

\begin{EntryWithPhonetic}{受不了}{shou4bu5liao3}{8,4,2}{⼜、⼀、⼅}[HSK 4]
  \definition{v.}{ser insuportável; não poder suportar algo; não suportar algo}
\end{EntryWithPhonetic}

\begin{EntryWithPhonetic}{受到}{shou4dao4}{8,8}{⼜、⼑}[HSK 2]
  \definition{v.}{receber; receber itens, mensagens, instruções, etc. fornecidos por outras pessoas}
\end{EntryWithPhonetic}

\begin{EntryWithPhonetic}{受得了}{shou4de5liao3}{8,11,2}{⼜、⼻、⼅}
  \definition{v.}{suportar | aguentar}
\end{EntryWithPhonetic}

\begin{EntryWithPhonetic}{受伤}{shou4shang1}{8,6}{⼜、⼈}[HSK 3]
  \definition{v.}{ser ferido; sofrer uma lesão}
\end{EntryWithPhonetic}

\begin{EntryWithPhonetic}{受限}{shou4xian4}{8,8}{⼜、⾩}
  \definition{v.}{ser limitado | ser restrito | ser constrangido}
\end{EntryWithPhonetic}

\begin{EntryWithPhonetic}{受灾}{shou4 zai1}{8,7}{⼜、⽕}[HSK 5]
  \definition{v.}{ser atingido por um desastre natural (ou calamidade) | ser atingido por uma adversidade natural}
\end{EntryWithPhonetic}

\begin{EntryWithPhonetic}{兽}{shou4}{11}{⼋}
  \definition{adj.}{bestial; brutal}
  \definition{s.}{besta; animal}
\end{EntryWithPhonetic}

\begin{EntryWithPhonetic}{兽力车}{shou4 li4 che1}{11,2,4}{⼋、⼒、⾞}
  \definition{s.}{veículo puxado por animais  (oposto a 人力车) | carruagem; carroça}
  \seealsoref{人力车}{ren2 li4 che1}
\end{EntryWithPhonetic}

\begin{EntryWithPhonetic}{兽行}{shou4xing2}{11,6}{⼋、⾏}
  \definition{s.}{ato brutal; brutalidade | bestialidade}
\end{EntryWithPhonetic}

\begin{EntryWithPhonetic}{售}{shou4}{11}{⼝}
  \definition{v.}{vender | fazer (o plano, truque, etc.) funcionar; continuar (as intrigas) | realizar (intrigas)}
\end{EntryWithPhonetic}

\begin{EntryWithPhonetic}{售货员}{shou4huo4yuan2}{11,8,7}{⼝、⾙、⼝}[HSK 4]
  \definition[名,位]{s.}{vendedor; balconista; assistente de loja; equipe que vende produtos em lojas}
\end{EntryWithPhonetic}

\begin{EntryWithPhonetic}{瘦}{shou4}{14}{⽧}[HSK 5]
  \definition{adj.}{magro; esquelético (oposto de 胖, 肥) | magro (oposto de 肥) | apertado (oposto de 肥) | infértil; pobre | esquelético; pouca gordura; pouca carne (em oposição a 或 ou 肥) | (roupas, sapatos, meias, etc.) apertado (em oposição a 肥) |magra; (carne comestível) com baixo teor de gordura (em oposição a 肥)}
  \definition{v.}{perder peso}
  \seealsoref{肥}{fei2}
  \seealsoref{或}{huo4}
  \seealsoref{胖}{pang4}
\end{EntryWithPhonetic}

\begin{EntryWithPhonetic}{书}{shu1}{4}{⼄}[HSK 1]
  \definition*{s.}{Sobrenome Shu}
  \definition[本,册,部,套,卷]{s.}{livro; obras encadernadas | carta; carta especial | documento | estilo de caligrafia; escrita}
  \definition{v.}{escrever; registrar}
\end{EntryWithPhonetic}

\begin{EntryWithPhonetic}{书包}{shu1 bao1}{4,5}{⼄、⼓}[HSK 1]
  \definition[个,款]{s.}{mochila para guardar livros e materiais escolares}
\end{EntryWithPhonetic}

\begin{EntryWithPhonetic}{书店}{shu1 dian4}{4,8}{⼄、⼴}[HSK 1]
  \definition[个,家]{s.}{livraria; lojas que vendem livros}
\end{EntryWithPhonetic}

\begin{EntryWithPhonetic}{书法}{shu1fa3}{4,8}{⼄、⽔}[HSK 5]
  \definition[幅,卷,种,派]{s.}{caligrafia; arte de escrever caracteres, especialmente arte de escrever caracteres chineses com um pincel}
\end{EntryWithPhonetic}

\begin{EntryWithPhonetic}{书房}{shu1 fang2}{4,8}{⼄、⼾}[HSK 6]
  \definition[间]{s.}{uma biblioteca (em uma residência privada); espaço para leitura e escrita}
\end{EntryWithPhonetic}

\begin{EntryWithPhonetic}{书柜}{shu1 gui4}{4,8}{⼄、⽊}[HSK 5]
  \definition{s.}{estante; armário de livros}
\end{EntryWithPhonetic}

\begin{EntryWithPhonetic}{书记}{shu1ji5}{4,5}{⼄、⾔}
  \definition{s.}{secretário (chefe de um ramo de um partido socialista ou comunista) | atendente | balconista | escriturário}
\end{EntryWithPhonetic}

\begin{EntryWithPhonetic}{书架}{shu1jia4}{4,9}{⼄、⽊}[HSK 3]
  \definition[个,种,套]{s.}{estante de livros}
\end{EntryWithPhonetic}

\begin{EntryWithPhonetic}{书桌}{shu1 zhuo1}{4,10}{⼄、⽊}[HSK 5]
  \definition[个,张]{s.}{escrivaninha; mesa para ler e escrever}
\end{EntryWithPhonetic}

\begin{EntryWithPhonetic}{叔}{shu1}{8}{⼜}
  \definition*{s.}{Sobrenome Shu}
  \definition{s.}{irmão mais novo do pai; tio (por parte de pai)| irmão mais novo do marido | terceiro entre irmãos | tio | uma forma de tratamento para um homem um pouco mais jovem que o pai; tio | terceiro tio (de quatro irmãos) | primo mais novo da mãe}
\end{EntryWithPhonetic}

\begin{EntryWithPhonetic}{叔叔}{shu1shu5}{8,8}{⼜、⼜}
  \definition[个,位,名]{s.}{tio; irmão mais novo do pai | tio, dirigindo-se a um homem da mesma geração que o pai e mais jovem em idade}
\end{EntryWithPhonetic}

\begin{EntryWithPhonetic}{疏}{shu1}{12}{⽦}
  \definition*{s.}{Sobrenome Shu}
  \definition{adj.}{fino; esparso; disperso (oposto a 密) | espalhado; disperso; difuso; a distância entre as coisas é grande; as lacunas entre as partes das coisas são grandes | distante; relacionamento distante; não próximo (de relações familiares ou sociais) | não familiarizado com; desconhecido | escasso; vazio}
  \definition{s.}{memorial; memorial ao trono; um texto em que um ministro na era feudal apresentava seus assuntos ao monarca em detalhes | comentário; anotações mais detalhadas de livros antigos do que 注}
  \definition{v.}{dragar (um rio, etc.) | negligenciar | dispersar; espalhar}
  \seealsoref{密}{mi4}
  \seealsoref{注}{zhu4}
\end{EntryWithPhonetic}

\begin{EntryWithPhonetic}{舒}{shu1}{12}{⾆}
  \definition*{s.}{Sobrenome Shu}
  \definition{adj.}{lento; vagaroso; sem pressa | confortável; relaxado e feliz}
  \definition{v.}{esticar; desdobrar | alongar; relaxar}
\end{EntryWithPhonetic}

\begin{EntryWithPhonetic}{舒服}{shu1fu5}{12,8}{⾆、⽉}[HSK 2]
  \definition{adj.}{confortável; sentir-se relaxado e feliz, tanto física quanto mentalmente}
\end{EntryWithPhonetic}

\begin{EntryWithPhonetic}{舒适}{shu1shi4}{12,9}{⾆、⾡}[HSK 4]
  \definition{adj.}{aconchegante; confortável; acolhedor; cômodo}
\end{EntryWithPhonetic}

\begin{EntryWithPhonetic}{输}{shu1}{13}{⾞}[HSK 3]
  \definition{v.}{transportar; entregar | contribuir com dinheiro; doar | perder; falhar; ser batido; ser derrotado}
\end{EntryWithPhonetic}

\begin{EntryWithPhonetic}{输出}{shu1 chu1}{13,5}{⾞、⼐}[HSK 5]
  \definition{v.}{exportar (de dentro para fora); transportar (de dentro) para fora | exportar; vender ou distribuir no exterior ou fora do país | emitir informações, programas, dados, sinais, etc. a partir de uma máquina; enviar por uma determinada instituição ou dispositivo (energia, sinal, etc.)}
\end{EntryWithPhonetic}

\begin{EntryWithPhonetic}{输入}{shu1ru4}{13,2}{⾞、⼊}[HSK 3]
  \definition{v.}{introduzir; importar; comprar bens, introduzir tecnologia, contratar mão de obra, introduzir capital, etc. | inserir informações, programas, dados, sinais, etc. em uma máquina}
\end{EntryWithPhonetic}

\begin{EntryWithPhonetic}{蔬}{shu1}{15}{⾋}
  \definition{s.}{vegetais}
\end{EntryWithPhonetic}

\begin{EntryWithPhonetic}{蔬菜}{shu1cai4}{15,11}{⾋、⾋}[HSK 5]
  \definition[样,种]{s.}{verduras; legumes; vegetais; ervas que podem ser usadas na culinária}
\end{EntryWithPhonetic}

\begin{EntryWithPhonetic}{熟}{shu2}{15}{⽕}[HSK 2]
  \definition{adj.}{maduro (frutos) | pronto; cozido | processado, fabricado ou exercitado | familiar, bem conhecido; conhecido por ser comum ou frequentemente utilizado | habilidoso;  (trabalho, tecnologia) experiente; não é novato | profundo; sólido}
\end{EntryWithPhonetic}

\begin{EntryWithPhonetic}{熟练}{shu2lian4}{15,8}{⽕、⽷}[HSK 4]
  \definition{adj.}{especializado; proficiente; qualificado; habilidoso}
\end{EntryWithPhonetic}

\begin{EntryWithPhonetic}{熟人}{shu2 ren2}{15,2}{⽕、⼈}[HSK 3]
  \definition[位,名,个,些]{s.}{amigo; conhecido; pessoas que se conhecem há muito tempo; pessoas que são muito familiares}
\end{EntryWithPhonetic}

\begin{EntryWithPhonetic}{熟悉}{shu2xi1}{15,11}{⽕、⼼}[HSK 5]
  \definition{adj.}{familiarizado com; não ser estranho}
  \definition{v.}{estar familiarizado com; saber claramente que | conhecer bem algo ou alguém; compreender e dominar (a situação) através da observação ou da experiência}
\end{EntryWithPhonetic}

\begin{EntryWithPhonetic}{属}{shu3}{12}{⼫}[HSK 3]
  \definition{s.}{categoria | gênero | membros da família; dependentes; familiares; parentes}
  \definition{v.}{estar sob; subordinado a | pertencer a | nascer no ano de (um dos doze animais do zodíaco)}
  \seeref{zhu3}
\end{EntryWithPhonetic}

\begin{EntryWithPhonetic}{属于}{shu3yu2}{12,3}{⼫、⼆}[HSK 3]
  \definition{v.}{pertencer a; fazer parte de; pertencer ou ser propriedade de uma determinada parte}
\end{EntryWithPhonetic}

\begin{EntryWithPhonetic}{暑}{shu3}{12}{⽇}
  \definition{adj.}{calor; clima quente; quente (em oposição a 寒)}
  \definition{s.}{verão}
  \seealsoref{寒}{han2}
\end{EntryWithPhonetic}

\begin{EntryWithPhonetic}{暑假}{shu3 jia4}{12,11}{⽇、⼈}[HSK 4]
  \definition[个]{s.}{férias de verão; feriado de verão; férias escolares de verão, na China, durante o sétimo e o oitavo meses do calendário gregoriano}
\end{EntryWithPhonetic}

\begin{EntryWithPhonetic}{黍}{shu3}{12}{⿉}[Kangxi 202]
  \definition{s.}{painço}
\end{EntryWithPhonetic}

\begin{EntryWithPhonetic}{数}{shu3}{13}{⽁}[HSK 2]
  \definition{v.}{contar (número); contar (número) um a um | ser considerado excepcionalmente (bom, ruim, etc.) | enumerar; listar}
  \seeref{shu4}
  \seeref{shuo4}
\end{EntryWithPhonetic}

\begin{EntryWithPhonetic}{鼠}{shu3}{13}{⿏}[HSK 5][Kangxi 208]
  \definition[只]{s.}{rato; camundongo}
\end{EntryWithPhonetic}

\begin{EntryWithPhonetic}{鼠标}{shu3biao1}{13,9}{⿏、⽊}[HSK 5]
  \definition[个,只]{s.}{\emph{mouse} (de computador); dispositivo de entrada externo para computadores, usado para controlar o movimento do cursor na tela do computador, selecionar objetos de operação, executar vários comandos, etc.}
\end{EntryWithPhonetic}

\begin{EntryWithPhonetic}{薯}{shu3}{16}{⾋}
  \definition{s.}{batata | inhame}
\end{EntryWithPhonetic}

\begin{EntryWithPhonetic}{薯片}{shu3 pian4}{16,4}{⾋、⽚}[HSK 6]
  \definition{s.}{batatas fritas (\emph{chips}); batatas fritas crocantes ; flocos finos feitos de batatas}
\end{EntryWithPhonetic}

\begin{EntryWithPhonetic}{薯条}{shu3 tiao2}{16,7}{⾋、⽊}[HSK 6]
  \definition{s.}{batatas fritas (palito)}
\end{EntryWithPhonetic}

\begin{EntryWithPhonetic}{术}{shu4}{5}{⽊}
  \definition*{s.}{Sobrenome Shu}
  \definition{s.}{arte; habilidade; técnica; tecnologia; acadêmico | método; tática; estratégia}
  \seeref{zhu2}
\end{EntryWithPhonetic}

\begin{EntryWithPhonetic}{术科}{shu4ke1}{5,9}{⽊、⽲}
  \definition{s.}{cursos técnicos oferecidos em treinamento militar ou físico (oposto a 学科)}
  \seealsoref{学科}{xue2 ke1}
\end{EntryWithPhonetic}

\begin{EntryWithPhonetic}{束}{shu4}{7}{⽊}[HSK 3]
  \definition*{s.}{Sobrenome Shu}
  \definition{clas.}{usado para cachos, molhos, feixes, feixes de luz, etc.}
  \definition{s.}{monte; pacote; maço; feixe; cacho; coisas agrupadas ou reunidas em tiras}
  \definition{v.}{atar; amarrar; vincular | controlar; restringir}
\end{EntryWithPhonetic}

\begin{EntryWithPhonetic}{束腰}{shu4yao1}{7,13}{⽊、⾁}
  \definition{s.}{cinto | cinta | cinturão}
\end{EntryWithPhonetic}

\begin{EntryWithPhonetic}{树}{shu4}{9}{⽊}[HSK 1]
  \definition*{s.}{Sobrenome Shu}
  \definition[棵,株]{s.}{árvore; nome comum das plantas lenhosas}
  \definition{v.}{plantar; cultivar | configurar; manter; estabelecer}
\end{EntryWithPhonetic}

\begin{EntryWithPhonetic}{树林}{shu4 lin2}{9,8}{⽊、⽊}[HSK 4]
  \definition[片,座]{s.}{bosque; muitas árvores que crescem em fragmentos, menores que as florestas}
\end{EntryWithPhonetic}

\begin{EntryWithPhonetic}{树莓}{shu4mei2}{9,10}{⽊、⾋}
  \definition{s.}{framboesa}
\end{EntryWithPhonetic}

\begin{EntryWithPhonetic}{树木}{shu4mu4}{9,4}{⽊、⽊}
  \definition{s.}{árvore}
\end{EntryWithPhonetic}

\begin{EntryWithPhonetic}{树叶}{shu4ye4}{9,5}{⽊、⼝}[HSK 4]
  \definition[片,枚,堆]{s.}{folha; folhagem}
\end{EntryWithPhonetic}

\begin{EntryWithPhonetic}{竖}{shu4}{9}{⽴}
  \definition*{s.}{Sobrenome Shu}
  \definition{adj.}{vertical; ereto; perpendicular ao solo}
  \definition{s.}{traço vertical (em caracteres chineses) | empregados domésticos; jovens criados}
  \definition{v.}{colocar em pé; erguer; ficar de pé; colocar o objeto perpendicular ao solo}
\end{EntryWithPhonetic}

\begin{EntryWithPhonetic}{竖向}{shu4xiang4}{9,6}{⽴、⼝}
  \definition{adj.}{vertical}
\end{EntryWithPhonetic}

\begin{EntryWithPhonetic}{庶}{shu4}{11}{⼴}
  \definition*{s.}{Sobrenome Shu}
  \definition{adj.}{multitudinário; numeroso}
  \definition{conj.}{para que; de ​​modo a}
  \definition{s.}{da ou pela concubina (diferentemente da esposa legal); no sistema patriarcal, refere-se ao ramo lateral da família}
\end{EntryWithPhonetic}

\begin{EntryWithPhonetic}{庶民}{shu4min2}{11,5}{⼴、⽒}
  \definition{s.}{(antigo) pessoas comuns | (antigo) plebeu; plebeus | (antigo) a multidão de pessoas comuns (na literatura erudita)}
\end{EntryWithPhonetic}

\begin{EntryWithPhonetic}{数}{shu4}{13}{⽁}
  \definition{num.}{vários; alguns}
  \definition{s.}{número; cifra; figura | número (conceito matemático básico que representa a quantidade de coisas) | número; indica a quantidade de coisas a que se referem os substantivos ou pronomes | destino; sorte}
  \seeref{shu3}
  \seeref{shuo4}
\end{EntryWithPhonetic}

\begin{EntryWithPhonetic}{数据}{shu4ju4}{13,11}{⽁、⼿}[HSK 4]
  \definition[组,个,条]{s.}{dados; valores com base nos quais são realizadas estatísticas, cálculos, pesquisas científicas ou projetos técnicos}
\end{EntryWithPhonetic}

\begin{EntryWithPhonetic}{数量}{shu4liang4}{13,12}{⽁、⾥}[HSK 3]
  \definition[个,种]{s.}{quantidade; quantum; quantia; magnitude; número}
\end{EntryWithPhonetic}

\begin{EntryWithPhonetic}{数码}{shu4ma3}{13,8}{⽁、⽯}[HSK 4]
  \definition{s.}{dígito; numeral; algarismo | número; quantidade (usado principalmente na linguagem falada)}
  \definition{v.}{digitalizar}
\end{EntryWithPhonetic}

\begin{EntryWithPhonetic}{数目}{shu4 mu4}{13,5}{⽁、⽬}[HSK 5]
  \definition{s.}{número; quantidade; quantidade de algo expressa em uma determinada medida padrão (como unidades de medida, etc.)}
\end{EntryWithPhonetic}

\begin{EntryWithPhonetic}{数学}{shu4xue2}{13,8}{⽁、⼦}
  \definition{s.}{matemática; a ciência que estuda as formas espaciais e as relações quantitativas do mundo real, incluindo matemática elementar e matemática superior}
\end{EntryWithPhonetic}

\begin{EntryWithPhonetic}{数字}{shu4zi4}{13,6}{⽁、⼦}[HSK 2]
  \definition{adj.}{digital; usando tecnologia digital}
  \definition[个,串]{s.}{dígito; número; um caractere que representa um número | numeral; símbolos que representam números, como algarismos arábicos, algarismos romanos, etc. | quantidade; montante}
\end{EntryWithPhonetic}

\begin{EntryWithPhonetic}{刷}{shua1}{8}{⼑}[HSK 4]
  \definition{s.}{escova; pincel | (onomatopéia) farfalhar; descreve o som de uma passagem rápida}
  \definition{v.}{escovar; esfregar; remover com uma escova | borrar; colar; aplicar com um pincel | eliminar; remover; limpar | rolar; navegar; visualizar grandes quantidades de informações muito rapidamente em um curto período de tempo online ou em dispositivos móveis | deslizar (passar o cartão magnético)}
  \seeref{shua4}
\end{EntryWithPhonetic}

\begin{EntryWithPhonetic}{刷牙}{shua1ya2}{8,4}{⼑、⽛}[HSK 4]
  \definition{s.}{escovar os dentes}
\end{EntryWithPhonetic}

\begin{EntryWithPhonetic}{刷子}{shua1zi5}{8,3}{⼑、⼦}[HSK 4]
  \definition[把,个]{s.}{escova; escovão; utensílio feito de lã, fio de plástico, fio de metal, etc., para remover sujeira ou aplicar óleo de unção, etc., geralmente longo ou oval, alguns com alças}
\end{EntryWithPhonetic}

\begin{EntryWithPhonetic}{耍}{shua3}{9}{⽽}
  \definition{v.}{brincar com | empunhar | agir (legal, calmo, tranquilo, descolado, etc.) | exibir (uma habilidade, o temperamento de alguém, etc.)}
\end{EntryWithPhonetic}

\begin{EntryWithPhonetic}{耍赖}{shua3lai4}{9,13}{⽽、⾙}
  \definition{v.}{agir descaradamente | recusar -se a reconhecer que alguém perdeu o jogo ou fez uma promessa, etc. | agir como um idiota | agir como se algo nunca tivesse acontecido}
\end{EntryWithPhonetic}

\begin{EntryWithPhonetic}{刷}{shua4}{8}{⼑}
  \definition{adj.}{pálido ou branco-azulado}
  \definition{adv.}{bastante; completamente; extremamente; descreve movimentos ágeis}
  \seeref{shua1}
\end{EntryWithPhonetic}

\begin{EntryWithPhonetic}{摔}{shuai1}{14}{⼿}[HSK 5]
  \definition{v.}{cair; tropeçar; perder o equilíbrio | mergulhar; precipitar-se; cair de uma altura elevada | quebrar; fazer cair e quebrar | lançar; atirar; arremessar; joguar coisas com força e para baixo | bater; golpear; bater com força para que o que está grudado cair}
\end{EntryWithPhonetic}

\begin{EntryWithPhonetic}{摔倒}{shuai1dao3}{14,10}{⼿、⼈}[HSK 5]
  \definition{v.}{cair; tropeçar; perder o equilíbrio e cair}
\end{EntryWithPhonetic}

\begin{EntryWithPhonetic}{帅}{shuai4}{5}{⼱}[HSK 4]
  \definition*{s.}{Sobrenome Shuai}
  \definition{adj.}{bonito; arrojado; elegante; inteligente}
  \definition{interj.}{Legal!}
  \definition[位,名,个,些]{s.}{comandante em chefe; o mais alto comandante do exército | comandante em chefe, a peça principal no xadrez chinês}
\end{EntryWithPhonetic}

\begin{EntryWithPhonetic}{帅哥}{shuai4 ge1}{5,10}{⼱、⼝}[HSK 4]
  \definition[个,位,名,些]{s.}{rapaz bonito; um garoto que é bonito e atraente na aparência}
\end{EntryWithPhonetic}

\begin{EntryWithPhonetic}{率}{shuai4}{11}{⽞}
  \definition*{s.}{Sobrenome Shuai}
  \definition{adj.}{precipitado; não cuidadoso; não cauteloso | franco; direto | elegante; bonito; o mesmo que 帅}
  \definition{adv.}{geralmente; expressa uma estimativa incerta, equivalente a 大约 e 大抵}
  \definition{s.}{modelo; exemplo}
  \definition{v.}{liderar; comandar | obedecer; seguir}
  \seeref{lv4}
  \seealsoref{大抵}{da4di3}
  \seealsoref{大约}{da4yue1}
  \seealsoref{帅}{shuai4}
\end{EntryWithPhonetic}

\begin{EntryWithPhonetic}{率领}{shuai4ling3}{11,11}{⽞、⾴}[HSK 5]
  \definition{v.}{liderar (equipe ou grupo); chefiar; comandar}
\end{EntryWithPhonetic}

\begin{EntryWithPhonetic}{率先}{shuai4 xian1}{11,6}{⽞、⼉}[HSK 4]
  \definition{v.}{tomar a iniciativa de fazer algo; ser o primeiro a fazer algo; assumir a liderança}
\end{EntryWithPhonetic}

\begin{EntryWithPhonetic}{双}{shuang1}{4}{⼜}[HSK 3]
  \definition*{s.}{Sobrenome Shuang}
  \definition{adj.}{dois; gêmeo; par; dual; em oposição a 单 | números pares | duplo; dobro}
  \definition{clas.}{usado para certos membros, órgãos ou coisas pareadas que são bilateralmente simétricas, por exemplo, sapatos, meias, pauzinhos, etc.}
  \seealsoref{单}{dan1}
\end{EntryWithPhonetic}

\begin{EntryWithPhonetic}{双层床}{shuang1ceng2chuang2}{4,7,7}{⼜、⼫、⼴}
  \definition{s.}{beliche}
\end{EntryWithPhonetic}

\begin{EntryWithPhonetic}{双打}{shuang1 da3}{4,5}{⼜、⼿}[HSK 6]
  \definition[场,局,次]{s.}{duplas (em esportes)}
\end{EntryWithPhonetic}

\begin{EntryWithPhonetic}{双方}{shuang1fang1}{4,4}{⼜、⽅}[HSK 3]
  \definition{s.}{ambos os lados; as duas partes; duas pessoas ou dois grupos frente a frente em um determinado relacionamento ou situação}
\end{EntryWithPhonetic}

\begin{EntryWithPhonetic}{双方同意}{shuang1fang1tong2yi4}{4,4,6,13}{⼜、⽅、⼝、⼼}
  \definition{s.}{acordo bilateral}
\end{EntryWithPhonetic}

\begin{EntryWithPhonetic}{双手}{shuang1 shou3}{4,4}{⼜、⼿}[HSK 5]
  \definition{s.}{com as duas mãos; ambas as mãos; par de mãos}
\end{EntryWithPhonetic}

\begin{EntryWithPhonetic}{霜}{shuang1}{17}{⾬}
  \definition{s.}{geada | pó branco ou creme espalhado por uma superfície | glacê | creme de pele}
\end{EntryWithPhonetic}

\begin{EntryWithPhonetic}{爽}{shuang3}{11}{⽘}[HSK 6]
  \definition{adj.}{claro; nítido; brilhante | franco; de coração aberto; direto | relaxado; confortável}
  \definition{v.}{desviar; afastar | tornar confortável; ficar confortável}
\end{EntryWithPhonetic}

\begin{EntryWithPhonetic}{谁}{shui2}{10}{⾔}[HSK 1]
  \seeref{shei2}
\end{EntryWithPhonetic}

\begin{EntryWithPhonetic}{水}{shui3}{4}{⽔}[HSK 1][Kangxi 85]
  \definition*{s.}{Etnia Shui, que vive principalmente em Guizhou | Sobrenome Shui}
  \definition{adj.}{de má qualidade; mal feito; de baixa qualidade e conteúdo}
  \definition{clas.}{usado para número de lavagens}
  \definition[条,杯]{s.}{água | rio | termo geral para rios, lagos, mares, etc.; água | corrente; fluxo de água | um líquido; suco ralo | teor de prata nas moedas | encargos adicionais ou receitas | água, um dos cinco elementos}
\end{EntryWithPhonetic}

\begin{EntryWithPhonetic}{水边}{shui3bian1}{4,5}{⽔、⾡}
  \definition{s.}{beira d'água | beira-mar | costa (de mar, lago ou rio)}
\end{EntryWithPhonetic}

\begin{EntryWithPhonetic}{水波}{shui3bo1}{4,8}{⽔、⽔}
  \definition{s.}{ondulação (na água) | onda}
\end{EntryWithPhonetic}

\begin{EntryWithPhonetic}{水槽}{shui3cao2}{4,15}{⽔、⽊}
  \definition{s.}{pia (de cozinha)}
\end{EntryWithPhonetic}

\begin{EntryWithPhonetic}{水产品}{shui3 chan3 pin3}{4,6,9}{⽔、⼇、⼝}[HSK 5]
  \definition{s.}{produto aquático (peixes, camarões, etc.)}
\end{EntryWithPhonetic}

\begin{EntryWithPhonetic}{水分}{shui3 fen4}{4,4}{⽔、⼑}[HSK 5]
  \definition{s.}{teor de umidade; água contida em um objeto | exagero; metáfora de algo falso}
\end{EntryWithPhonetic}

\begin{EntryWithPhonetic}{水果}{shui3guo3}{4,8}{⽔、⽊}[HSK 1]
  \definition[个]{s.}{fruta; um nome genérico para frutas com alto teor de água que podem ser consumidas, como peras, pêssegos, maçãs, etc.}
\end{EntryWithPhonetic}

\begin{EntryWithPhonetic}{水饺}{shui3jiao3}{4,9}{⽔、⾷}
  \definition{s.}{\emph{dumplings} | pastéis chineses cozidos}
\end{EntryWithPhonetic}

\begin{EntryWithPhonetic}{水库}{shui3 ku4}{4,7}{⽔、⼴}[HSK 5]
  \definition[座]{s.}{reservatório; lago artificial construído pelo homem, que utiliza barragens e outras estruturas para represar a água e regular o fluxo, podendo ser utilizado para armazenamento de água, geração de energia e piscicultura, entre outros fins}
\end{EntryWithPhonetic}

\begin{EntryWithPhonetic}{水灵}{shui3ling2}{4,7}{⽔、⽕}
  \definition{adj.}{cheio de vida (sobre uma pessoa, etc.) | úmido e brilhante (sobre os olhos) | fresco (sobre frutas, etc.) | brilhante | aparência saudável}
\end{EntryWithPhonetic}

\begin{EntryWithPhonetic}{水路}{shui3lu4}{4,13}{⽔、⾜}
  \definition{s.}{hidrovia}
\end{EntryWithPhonetic}

\begin{EntryWithPhonetic}{水泥}{shui3ni2}{4,8}{⽔、⽔}[HSK 6]
  \definition[袋,层]{s.}{cimento; um tipo de material mineral em pó que pode endurecer gradualmente no ar e na água após a mistura com água}
\end{EntryWithPhonetic}

\begin{EntryWithPhonetic}{水培}{shui3pei2}{4,11}{⽔、⼟}
  \definition{v.}{cultivar plantas hidroponicamente}
\end{EntryWithPhonetic}

\begin{EntryWithPhonetic}{水平}{shui3ping2}{4,5}{⽔、⼲}[HSK 2]
  \definition{adj.}{horizontal; nivelado; paralelo à superfície da água}
  \definition{s.}{padrão; nível; o nível alcançado em determinado aspecto}
\end{EntryWithPhonetic}

\begin{EntryWithPhonetic}{水平尺}{shui3ping2chi3}{4,5,4}{⽔、⼲、⼫}
  \definition{s.}{nível espiritual}
\end{EntryWithPhonetic}

\begin{EntryWithPhonetic}{水平度}{shui3ping2 du4}{4,5,9}{⽔、⼲、⼴}
  \definition{s.}{nivelamento}
\end{EntryWithPhonetic}

\begin{EntryWithPhonetic}{水平面}{shui3ping2mian4}{4,5,9}{⽔、⼲、⾯}
  \definition{s.}{plano horizontal | nível-da-água | superfície horizontal}
\end{EntryWithPhonetic}

\begin{EntryWithPhonetic}{水平视差}{shui3ping2 shi4cha1}{4,5,8,9}{⽔、⼲、⾒、⼯}
  \definition{s.}{paralaxe horizontal}
\end{EntryWithPhonetic}

\begin{EntryWithPhonetic}{水平仪}{shui3ping2yi2}{4,5,5}{⽔、⼲、⼈}
  \definition{s.}{nível (dispositivo para determinar horizontal) | nível espiritual | nível de topógrafo}
\end{EntryWithPhonetic}

\begin{EntryWithPhonetic}{水平以下}{shui3ping2 yi3xia4}{4,5,4,3}{⽔、⼲、⼈、⼀}
  \definition{s.}{sub-nível}
\end{EntryWithPhonetic}

\begin{EntryWithPhonetic}{水平轴}{shui3ping2zhou2}{4,5,9}{⽔、⼲、⾞}
  \definition{s.}{eixo horizontal}
\end{EntryWithPhonetic}

\begin{EntryWithPhonetic}{水瓶}{shui3 ping2}{4,10}{⽔、⽡}
  \definition{s.}{garrada de água}
\end{EntryWithPhonetic}

\begin{EntryWithPhonetic}{水豚}{shui3tun2}{4,11}{⽔、⾗}
  \definition{s.}{capivara}
\end{EntryWithPhonetic}

\begin{EntryWithPhonetic}{水污染}{shui3wu1ran3}{4,6,9}{⽔、⽔、⽊}
  \definition{s.}{poluição da água}
\end{EntryWithPhonetic}

\begin{EntryWithPhonetic}{水灾}{shui3zai1}{4,7}{⽔、⽕}[HSK 5]
  \definition[场,次]{s.}{inundação; desastres causados por excesso de chuvas, entre outros motivos}
\end{EntryWithPhonetic}

\begin{EntryWithPhonetic}{说}{shui4}{9}{⾔}
  \definition{v.}{persuadir}
  \seeref{shuo1}
\end{EntryWithPhonetic}

\begin{EntryWithPhonetic}{税}{shui4}{12}{⽲}[HSK 6]
  \definition*{s.}{Sobrenome Shui}
  \definition{s.}{imposto; taxa; tarifa}
\end{EntryWithPhonetic}

\begin{EntryWithPhonetic}{睡}{shui4}{13}{⽬}[HSK 1]
  \definition{v.}{dormir | deitar-se}
\end{EntryWithPhonetic}

\begin{EntryWithPhonetic}{睡觉}{shui4/jiao4}{13,9}{⽬、⾒}[HSK 1]
  \definition{v.+compl.}{dormir; ir para a cama; entrar em estado de sono}
\end{EntryWithPhonetic}

\begin{EntryWithPhonetic}{睡懒觉}{shui4lan3jiao4}{13,16,9}{⽬、⼼、⾒}
  \definition{v.}{levantar-se tarde | passar o tempo a dormir}
\end{EntryWithPhonetic}

\begin{EntryWithPhonetic}{睡眠}{shui4 mian2}{13,10}{⽬、⽬}[HSK 5]
  \definition{s.}{sono; \emph{somnus}; sonolência}
\end{EntryWithPhonetic}

\begin{EntryWithPhonetic}{睡衣}{shui4yi1}{13,6}{⽬、⾐}
  \definition{s.}{pijamas | roupas de dormir}
\end{EntryWithPhonetic}

\begin{EntryWithPhonetic}{睡着}{shui4 zhao2}{13,11}{⽬、⽬}[HSK 4]
  \definition{v.}{dormir; adormecer; cair no sono}
\end{EntryWithPhonetic}

\begin{EntryWithPhonetic}{顺}{shun4}{9}{⾴}[HSK 6]
  \definition{adj.}{(de escritos) legível; claro e bem escrito; organizado | favorável; harmonioso | favorável; bem-sucedido}
  \definition{prep.}{conforme a conveniência de alguém | ao longo; a introdução da rota, situação ou oportunidade que a ação segue pode ser seguida por 着 | com a corrente; na mesma direção |  com; na mesma direção que}
  \definition{v.}{organizar; colocar em ordem; tornar as coisas organizadas ou ordenadas | obedecer; ceder a; agir em submissão a | ser adequado; ser agradável}
  \seealsoref{着}{zhe5}
\end{EntryWithPhonetic}

\begin{EntryWithPhonetic}{顺便}{shun4bian4}{9,9}{⾴、⼈}
  \definition{adv.}{convenientemente | de passagem | sem muito esforço extra}
\end{EntryWithPhonetic}

\begin{EntryWithPhonetic}{顺畅}{shun4chang4}{9,8}{⾴、⽥}
  \definition{adj.}{liso e sem obstáculos | fluente}
\end{EntryWithPhonetic}

\begin{EntryWithPhonetic}{顺从}{shun4cong2}{9,4}{⾴、⼈}
  \definition{v.}{obedecer | submeter-se}
\end{EntryWithPhonetic}

\begin{EntryWithPhonetic}{顺当}{shun4dang5}{9,6}{⾴、⼹}
  \definition{adv.}{suavemente}
\end{EntryWithPhonetic}

\begin{EntryWithPhonetic}{顺耳}{shun4'er3}{9,6}{⾴、⽿}
  \definition{adj.}{agradável ao ouvido}
\end{EntryWithPhonetic}

\begin{EntryWithPhonetic}{顺境}{shun4jing4}{9,14}{⾴、⼟}
  \definition{s.}{circunstâncias favoráveis}
\end{EntryWithPhonetic}

\begin{EntryWithPhonetic}{顺利}{shun4li4}{9,7}{⾴、⼑}[HSK 2]
  \definition{adj.}{sem problemas; com sucesso; sem dificuldades; sem contratempos; sem obstáculos; sem obstáculos ou dificuldades significativas no desempenho das tarefas}
\end{EntryWithPhonetic}

\begin{EntryWithPhonetic}{顺水}{shun4shui3}{9,4}{⾴、⽔}
  \definition{v.}{ir com o fluxo}
\end{EntryWithPhonetic}

\begin{EntryWithPhonetic}{顺心}{shun4xin1}{9,4}{⾴、⼼}
  \definition{adj.}{satisfatório | satisfeito}
\end{EntryWithPhonetic}

\begin{EntryWithPhonetic}{顺序}{shun4xu4}{9,7}{⾴、⼴}[HSK 4]
  \definition{adv.}{por sua vez; na ordem correta; na devida ordem; na ordem adequada; na ordem apropriada}
  \definition[个]{s.}{ordem; sequência; sucessão; subsequência; sequência simples; ordem de prioridade}
\end{EntryWithPhonetic}

\begin{EntryWithPhonetic}{顺叙}{shun4xu4}{9,9}{⾴、⼜}
  \definition{s.}{narrativa cronológica}
\end{EntryWithPhonetic}

\begin{EntryWithPhonetic}{顺延}{shun4yan2}{9,6}{⾴、⼵}
  \definition{v.}{adiar | procrastinar}
\end{EntryWithPhonetic}

\begin{EntryWithPhonetic}{顺眼}{shun4yan3}{9,11}{⾴、⽬}
  \definition{adj.}{agradável aos olhos}
\end{EntryWithPhonetic}

\begin{EntryWithPhonetic}{顺嘴}{shun4zui3}{9,16}{⾴、⼝}
  \definition{v.}{deixar escapar (sem pensar) | ler suavemente (texto) | adequar-se  ao gosto (comida)}
\end{EntryWithPhonetic}

\begin{EntryWithPhonetic}{舜}{shun4}{12}{⾇}
  \definition*{s.}{Shun, o nome de um monarca lendário da China antiga | Sobrenome Shun}
\end{EntryWithPhonetic}

\begin{EntryWithPhonetic}{说}{shuo1}{9}{⾔}[HSK 1]
  \definition{s.}{uma teoria (normalmente o último caractere, como em 日心说, teoria heliocêntrica); ensinamentos; doutrina}
  \definition{v.}{falar; conversar; dizer | explicar | repreender | atuar como casamenteiro | referir-se a; indicar | criticar; aconselhar | fazer uma combinação; conciliar; mediar | discutir; falar sobre; conversar sobre | uma forma de expressão linguística da arte cênica}
  \seeref{shui4}
  \seealsoref{日心说}{ri4 xin1 shuo1}
\end{EntryWithPhonetic}

\begin{EntryWithPhonetic}{说不定}{shuo1bu5ding4}{9,4,8}{⾔、⼀、⼧}[HSK 4]
  \definition{adv.}{talvez; indica uma estimativa, possivelmente, provavelmente}
  \definition{v.}{não ter certeza; não estar certo; ser impreciso}
\end{EntryWithPhonetic}

\begin{EntryWithPhonetic}{说法}{shuo1 fa3}{9,8}{⾔、⽔}[HSK 5]
  \definition[种,个]{s.}{formulação; maneira de dizer uma coisa; formas de expressar opiniões | versão; argumento; declaração; opinião | explicação; acordo; palavras justas; razões ou fundamentos legítimos}
\end{EntryWithPhonetic}

\begin{EntryWithPhonetic}{说服}{shuo1fu2}{9,8}{⾔、⽉}[HSK 4]
  \definition{v.}{persuadir; convencer; convencer a outra parte com palavras bem fundamentadas}
\end{EntryWithPhonetic}

\begin{EntryWithPhonetic}{说好}{shuo1hao3}{9,6}{⾔、⼥}
  \definition{v.}{chegar a um acordo | concluir negociações}
\end{EntryWithPhonetic}

\begin{EntryWithPhonetic}{说话}{shuo1hua4}{9,8}{⾔、⾔}[HSK 1]
  \definition{adv.}{imediatamente; em um minuto; refere-se ao tempo que leva para falar, indicando um período muito curto}
  \definition{v.}{falar; conversar; dizer; expressar o significado através da linguagem | conversar (conversa fiada); bater papo | fofocar; conversar; criticar; censurar}
\end{EntryWithPhonetic}

\begin{EntryWithPhonetic}{说谎}{shuo1/huang3}{9,11}{⾔、⾔}
  \definition{v.+compl.}{mentir | contar uma mentira}
\end{EntryWithPhonetic}

\begin{EntryWithPhonetic}{说理}{shuo1li3}{9,11}{⾔、⽟}
  \definition{v.}{racionalizar | discutir logicamente}
\end{EntryWithPhonetic}

\begin{EntryWithPhonetic}{说明}{shuo1ming2}{9,8}{⾔、⽇}[HSK 2]
  \definition[本,个]{s.}{legenda; instrução; explicação}
  \definition{v.}{mostrar; explicar; ilustrar | indicar; mostrar; provar; demonstrar; usar materiais confiáveis para demonstrar ou determinar a autenticidade de pessoas ou coisas}
\end{EntryWithPhonetic}

\begin{EntryWithPhonetic}{说明书}{shuo1 ming2 shu1}{9,8,4}{⾔、⽇、⼄}[HSK 6]
  \definition[本]{s.}{manual; livro de instruções; descrições textuais da finalidade, especificações, desempenho e uso de itens, bem como enredos de peças e filmes, etc.}
\end{EntryWithPhonetic}

\begin{EntryWithPhonetic}{说实话}{shuo1 shi2 hua4}{9,8,8}{⾔、⼧、⾔}[HSK 6]
  \definition{v.}{falar a verdade; dizer a verdade sobre (os próprios erros ou crimes)}
\end{EntryWithPhonetic}

\begin{EntryWithPhonetic}{说完}{shuo1-wan2}{9,7}{⾔、⼧}
  \definition{expr.}{acabar/terminar palavras}
\end{EntryWithPhonetic}

\begin{EntryWithPhonetic}{硕}{shuo4}{11}{⽯}
  \definition*{s.}{Sobrenome Shuo}
  \definition{adj.}{grande; enorme}
  \definition{s.}{mestrado (MBA)}
\end{EntryWithPhonetic}

\begin{EntryWithPhonetic}{硕士}{shuo4shi4}{11,3}{⽯、⼠}[HSK 5]
  \definition[个,位,名]{s.}{mestrado; um diploma concedido por uma universidade ou faculdade a um aluno após um ou dois anos de estudo adicional após o bacharelado}
\end{EntryWithPhonetic}

\begin{EntryWithPhonetic}{数}{shuo4}{13}{⽁}
  \definition{adv.}{com frequência; repetidamente; indica uma ação frequente, equivalente a 屡次}
  \seeref{shu3}
  \seeref{shu4}
  \seealsoref{屡次}{lv3ci4}
\end{EntryWithPhonetic}

\begin{EntryWithPhonetic}{丝}{si1}{5}{⼀}
  \definition{clas.}{si, uma unidade de peso (=0,0005 gramas) | usado para expressar a aparência ou expressão de uma pessoa | um décimo de milésimo de certas unidades de medida (medida de comprimento) | usado para representar coisas abstratas}
  \definition[些,种,类,跟,缕]{s.}{seda | uma coisa semelhante a um fio; itens semelhantes à seda | cordas; instrumentos de corda}
\end{EntryWithPhonetic}

\begin{EntryWithPhonetic}{司}{si1}{5}{⼝}
  \definition*{s.}{Sobrenome Si}
  \definition{s.}{departamento (sob um ministério); um departamento dentro de uma agência de nível ministerial}
  \definition{v.}{assumir o comando de; atender; administrar; operar; gerenciar}
\end{EntryWithPhonetic}

\begin{EntryWithPhonetic}{司机}{si1ji1}{5,6}{⼝、⽊}[HSK 2]
  \definition[个,名,位]{s.}{motorista; motorista particular; chofer; motoristas de veículos de transporte público, como trens, ônibus e bondes}
\end{EntryWithPhonetic}

\begin{EntryWithPhonetic}{司长}{si1 zhang3}{5,4}{⼝、⾧}[HSK 6]
  \definition[位,名]{s.}{diretor-geral | chefe de gabinete}
\end{EntryWithPhonetic}

\begin{EntryWithPhonetic}{私}{si1}{7}{⽲}
  \definition*{s.}{Sobrenome Si}
  \definition{adj.}{pessoal; privado (oposição a 公) | egoísta (oposto a 公) | secreto; privado | ilícito; ilegal}
  \definition{s.}{interesse privado (ou egoísta); motivo (ou ideia) egoísta (oposição a 公) | contrabando; mercadorias contrabandeadas | propriedade privada | interesses privados; ganho pessoal}
  \seealsoref{公}{gong1}
\end{EntryWithPhonetic}

\begin{EntryWithPhonetic}{私函}{si1han2}{7,8}{⽲、⼐}
  \definition{s.}{carta privada}
\end{EntryWithPhonetic}

\begin{EntryWithPhonetic}{私立}{si1li4}{7,5}{⽲、⽴}
  \definition{s.}{privado; estabelecido privadamente}[这是一所私立学校。===Esta é uma escola particular.]
  \definition{v.}{estabelecer-se ilegalmente}
\end{EntryWithPhonetic}

\begin{EntryWithPhonetic}{私人}{si1ren2}{7,2}{⽲、⼈}[HSK 5]
  \definition{adj.}{privado; pertencente a um indivíduo ou exercido a título individual; não público | pessoal; entre indivíduos}
  \definition[个]{s.}{algo privado; pessoas que se aproximam de você por motivos pessoais ou interesses próprios}
\end{EntryWithPhonetic}

\begin{EntryWithPhonetic}{私人信件}{si1ren2 xin4jian4}{7,2,9,6}{⽲、⼈、⼈、⼈}
  \definition{s.}{carta pessoal}
\end{EntryWithPhonetic}

\begin{EntryWithPhonetic}{私人钥匙}{si1ren2yao4shi5}{7,2,9,11}{⽲、⼈、⾦、⼔}
  \definition{s.}{(criptografia) chave privada}
\end{EntryWithPhonetic}

\begin{EntryWithPhonetic}{私人诊所}{si1ren2 zhen3suo3}{7,2,7,8}{⽲、⼈、⾔、⼾}
  \definition[些]{s.}{clínica privada}
\end{EntryWithPhonetic}

\begin{EntryWithPhonetic}{私生活}{si1sheng1huo2}{7,5,9}{⽲、⽣、⽔}
  \definition{s.}{vida privada}
\end{EntryWithPhonetic}

\begin{EntryWithPhonetic}{私事}{si1shi4}{7,8}{⽲、⼅}
  \definition[件,桩]{s.}{privacidade; assuntos privados; assuntos pessoais (oposto a 公事)}
  \seealsoref{公事}{gong1shi4}
\end{EntryWithPhonetic}

\begin{EntryWithPhonetic}{私自}{si1zi4}{7,6}{⽲、⾃}
  \definition{adj.}{privado | pessoal}
  \definition{adv.}{secretamente | sem aprovação explícita}
\end{EntryWithPhonetic}

\begin{EntryWithPhonetic}{思}{si1}{9}{⼼}
  \definition*{s.}{Sobrenome Si}
  \definition{s.}{pensamento; ideias | pensamentos; emoções; humor}
  \definition{v.}{pensar; considerar; deliberar | pensar em; ansiar por}
\end{EntryWithPhonetic}

\begin{EntryWithPhonetic}{思考}{si1kao3}{9,6}{⼼、⽼}[HSK 4]
  \definition{v.}{pensar; ponderar; considerar; deliberar; envolver-se em atividades de pensamento, como análise, síntese, julgamento, raciocínio e generalização}
\end{EntryWithPhonetic}

\begin{EntryWithPhonetic}{思维}{si1wei2}{9,11}{⼼、⽷}[HSK 5]
  \definition[种]{s.}{pensamento; reflexão; organizar e transformar os materiais obtidos através do conhecimento sensorial para formar conceitos, julgamentos e raciocínios}
  \definition{v.}{pensar}
\end{EntryWithPhonetic}

\begin{EntryWithPhonetic}{思想}{si1xiang3}{9,13}{⼼、⼼}[HSK 3]
  \definition[个,种]{s.}{reflexão; pensamento; ideologia; a existência objetiva é refletida na consciência das pessoas por meio de atividades de pensamento, que pertencem à cognição racional | ideia; pensamento}
\end{EntryWithPhonetic}

\begin{EntryWithPhonetic}{斯}{si1}{12}{⽄}
  \definition*{s.}{Sobrenome Si}
  \definition{adv.}{então; assim}
  \definition{pron.}{isto; aqui}
\end{EntryWithPhonetic}

\begin{EntryWithPhonetic}{斯巴达}{si1ba1da2}{12,4,6}{⽄、⼰、⾡}
  \definition*{s.}{Esparta}
\end{EntryWithPhonetic}

\begin{EntryWithPhonetic}{死}{si3}{6}{⽍}[HSK 3]
  \definition{adj.}{até a morte | implacável; mortal | fixo; rígido; inflexível | intransitável; fechado | (expressando raiva, reclamação, etc., às vezes jocosamente) maldito}
  \definition{adv.}{(frequentemente no negativo) teimosamente; inflexivelmente}
  \definition{v.}{morrer; estar morto (oposto a 生 e 活)}
  \seealsoref{活}{huo2}
  \seealsoref{生}{sheng1}
\end{EntryWithPhonetic}

\begin{EntryWithPhonetic}{死亡}{si3wang2}{6,3}{⽍、⼇}[HSK 6]
  \definition{s.}{morte; condenação; dar o último suspiro; refere-se ao estado de vida desaparecendo |}
  \definition{v.}{morrer; estar morto; perder a vida (em oposição à 生存)}
  \seealsoref{生存}{sheng1cun2}
\end{EntryWithPhonetic}

\begin{EntryWithPhonetic}{四}{si4}{5}{⼞}[HSK 1]
  \definition*{s.}{Sobrenome Si}
  \definition{num.}{quatro; 4}
  \definition{s.}{uma nota da escala em Gongchepu (工尺谱), correspondente ao 6 na notação musical numerada}
  \seealsoref{工尺谱}{gong1 che3 pu3}
\end{EntryWithPhonetic}

\begin{EntryWithPhonetic}{四处}{si4 chu4}{5,5}{⼞、⼡}[HSK 6]
  \definition{adv.}{em volta; ao redor; em todos os lugares; em todas as direções}
\end{EntryWithPhonetic}

\begin{EntryWithPhonetic}{四川}{si4chuan1}{5,3}{⼞、⼮}
  \definition*{s.}{Província de Sichuan}
\end{EntryWithPhonetic}

\begin{EntryWithPhonetic}{四季分明}{si4ji4-fen1ming2}{5,8,4,8}{⼞、⼦、⼑、⽇}
  \definition{expr.}{as quatro estações são muito distintas}
\end{EntryWithPhonetic}

\begin{EntryWithPhonetic}{四季如春}{si4ji4-ru2chun1}{5,8,6,9}{⼞、⼦、⼥、⽇}
  \definition{expr.}{é primavera todo o ano | clima favorável durante todo o ano | quatro estações como a primavera}
\end{EntryWithPhonetic}

\begin{EntryWithPhonetic}{四周}{si4 zhou1}{5,8}{⼞、⼝}[HSK 5]
  \definition{s.}{ao redor; por todos os lados; a parte que circunda o centro}
\end{EntryWithPhonetic}

\begin{EntryWithPhonetic}{似}{si4}{6}{⼈}
  \definition*{s.}{Sobrenome Si}
  \definition{adv.}{parece; como se}
  \definition{v.}{ser semelhante; parecer-se com | parecer; aparecer | exceder}
  \seeref{shi4}
\end{EntryWithPhonetic}

\begin{EntryWithPhonetic}{似曾相识}{si4ceng2xiang1shi2}{6,12,9,7}{⼈、⽈、⽬、⾔}
  \definition{s.}{\emph{déjà vu} (a experiência de ver exatamente a mesma situação pela segunda vez) | situação aparentemente familiar}
\end{EntryWithPhonetic}

\begin{EntryWithPhonetic}{似乎}{si4hu1}{6,5}{⼈、⼃}[HSK 4]
  \definition{adv.}{como se; aparentemente; se parece como}
\end{EntryWithPhonetic}

\begin{EntryWithPhonetic}{寺}{si4}{6}{⼨}[HSK 6]
  \definition*{s.}{Sobrenome Si}
  \definition[座]{s.}{templo | (Islã) mesquita | (datado) ministério; agência governamental na China antiga}
\end{EntryWithPhonetic}

\begin{EntryWithPhonetic}{寺庙}{si4miao4}{6,8}{⼨、⼴}
  \definition{s.}{templo | mosteiro | santuário}
\end{EntryWithPhonetic}

\begin{EntryWithPhonetic}{伺}{si4}{7}{⼈}
  \definition{v.}{aguardar; observar; esperar por}
  \seeref{ci4}
\end{EntryWithPhonetic}

\begin{EntryWithPhonetic}{食}{si4}{9}{⾷}[Kangxi 184]
  \definition{v.}{alimentar; dar comida a}
  \seeref{shi2}
\end{EntryWithPhonetic}

\begin{EntryWithPhonetic}{肆}{si4}{13}{⾀}
  \definition*{s.}{Sobrenome Si}
  \definition{adj.}{desenfreado; sem limites; descuidado; imprudente}
  \definition{num.}{quatro (usado para o numeral 四 em cheques, etc., para evitar erros ou alterações)}
  \definition{s.}{Literário: loja; armazém}
  \seealsoref{四}{si4}
\end{EntryWithPhonetic}

\begin{EntryWithPhonetic}{厕}{si5}{8}{⼚}
  \definition{s.}{componente formador de palavras | latrina; fossa sanitária}
  \seeref{ce4}
  \seealsoref{茅厕}{mao2ce4}
\end{EntryWithPhonetic}

\begin{EntryWithPhonetic}{松}{song1}{8}{⽊}[HSK 4]
  \definition*{s.}{Sobrenome Song}
  \definition{adj.}{solto; frouxo; folgado | leve e crocante; macio | relaxado; confortável}
  \definition[棵]{s.}{pinheiro | fio de carne seca; carne moída seca; alimentos macios ou quebradiços}
  \definition{v.}{afrouxar; relaxar; abrandar | desamarrar; desatar; liberar}
\end{EntryWithPhonetic}

\begin{EntryWithPhonetic}{松木}{song1mu4}{8,4}{⽊、⽊}
  \definition{s.}{pinheiro}
\end{EntryWithPhonetic}

\begin{EntryWithPhonetic}{松树}{song1 shu4}{8,9}{⽊、⽊}[HSK 4]
  \definition[棵]{s.}{pinheiro; conífera comum, geralmente com folhas longas e pontiagudas e cones lenhosos}
\end{EntryWithPhonetic}

\begin{EntryWithPhonetic}{宋}{song4}{7}{⼧}
  \definition*{s.}{Dinastia Song (960-1279) | Song das dinastias do sul (420-479) | Sobrenome Song}
  \definition{clas.}{sone; unidade de intensidade sonora}
\end{EntryWithPhonetic}

\begin{EntryWithPhonetic}{送}{song4}{9}{⾡}[HSK 1]
  \definition*{s.}{Sobrenome Song}
  \definition{v.}{transportar; entregar | dar; dar como presente; presentear | acompanhar; despedir-se de alguém (ao sair); acompanhar a pessoa que está partindo até o destino ou caminhar um trecho com ela | escoltar}
\end{EntryWithPhonetic}

\begin{EntryWithPhonetic}{送到}{song4 dao4}{9,8}{⾡、⼑}[HSK 2]
  \definition{v.}{enviar para (lugar)}
\end{EntryWithPhonetic}

\begin{EntryWithPhonetic}{送给}{song4 gei3}{9,9}{⾡、⽷}[HSK 2]
  \definition{v.}{dar a (alguém ou organização); dar como algo gratuito; dar como presente}
\end{EntryWithPhonetic}

\begin{EntryWithPhonetic}{送礼}{song4 li3}{9,5}{⾡、⽰}[HSK 6]
  \definition{v.}{dar um presente a alguém; presentear alguém com um presente | enviar presentes (para obter favores) | dar um presente; enviar um presente}
\end{EntryWithPhonetic}

\begin{EntryWithPhonetic}{送行}{song4 xing2}{9,6}{⾡、⾏}[HSK 6]
  \definition{v.}{ver alguém partir; ir até o local onde o viajante iniciou sua jornada, despedir-se dele e observar ele partir | dar uma festa de despedida; realizar uma festa de despedida | despedir-se do falecido}
\end{EntryWithPhonetic}

\begin{EntryWithPhonetic}{㮸}{song4}{14}{⽊}
  \variantof{送}
\end{EntryWithPhonetic}

\begin{EntryWithPhonetic}{搜}{sou1}{12}{⼿}[HSK 5]
  \definition{v.}{procurar | pesquisar | coletar; reunir | procurar ou revistar um lugar de forma completa e desordenada}
\end{EntryWithPhonetic}

\begin{EntryWithPhonetic}{搜索}{sou1suo3}{12,10}{⼿、⽷}[HSK 5]
  \definition{v.}{procurar; caçar; explorar; pesquisar cuidadosamente; refere-se especificamente à busca militar para identificar situações suspeitas em determinada região, área marítima ou aérea}
\end{EntryWithPhonetic}

\begin{EntryWithPhonetic}{苏}{su1}{7}{⾋}
  \definition*{s.}{Suzhou, abreviação de 苏州 | Província de Jiangsu, abreviação de 江苏 | União Soviética, abreviação de 苏联 | Sobrenome Su}
  \definition{s.}{perilla planta da família das mentas}
  \definition{v.}{reviver; vir a; acordar}
  \seealsoref{江苏}{jiang1su1}
  \seealsoref{苏联}{su1lian2}
  \seealsoref{苏州}{su1zhou1}
\end{EntryWithPhonetic}

\begin{EntryWithPhonetic}{苏格兰}{su1ge2lan2}{7,10,5}{⾋、⽊、⼋}
  \definition*{s.}{Escócia}
\end{EntryWithPhonetic}

\begin{EntryWithPhonetic}{苏联}{su1lian2}{7,12}{⾋、⽿}
  \definition*{s.}{União das Repúblicas Socialistas Soviéticas (1922-1991)}
\end{EntryWithPhonetic}

\begin{EntryWithPhonetic}{苏州}{su1zhou1}{7,6}{⾋、⼮}
  \definition*{s.}{Suzhou, cidade na Província de Jiangsu}
\end{EntryWithPhonetic}

\begin{EntryWithPhonetic}{素}{su4}{10}{⽷}
  \definition{adj.}{branco; de cor natural | simples; natural; singelo; de cor simples | nativo; original | normal; usual; geral}
  \definition{adv.}{geralmente; sempre; habitualmente}
  \definition{s.}{vegetais, frutas e outros alimentos (em oposição à 荤) | matéria-prima; matéria-prima básico; tecidos de seda naturais e não processados | elemento; os componentes básicos de algo}
  \seealsoref{荤}{hun1}
\end{EntryWithPhonetic}

\begin{EntryWithPhonetic}{素质}{su4zhi4}{10,8}{⽷、⾙}[HSK 6]
  \definition[个,种]{s.}{qualidade; características; caráter; o nível físico, moral, mental, intelectual e cultural de uma pessoa}
\end{EntryWithPhonetic}

\begin{EntryWithPhonetic}{速}{su4}{10}{⾡}
  \definition{adj.}{rápido; veloz}
  \definition{s.}{velocidade}
  \definition{v.aux.}{convidar}
\end{EntryWithPhonetic}

\begin{EntryWithPhonetic}{速度}{su4du4}{10,9}{⾡、⼴}[HSK 3]
  \definition[个,种]{s.}{velocidade; taxa; ritmo; andamento; uma quantidade física que indica a velocidade e a direção do movimento de um objeto, ou seja, a distância que um objeto percorre em uma direção por unidade de tempo | velocidade; rapidez; geralmente se refere ao grau de velocidade}
\end{EntryWithPhonetic}

\begin{EntryWithPhonetic}{宿}{su4}{11}{⼧}
  \definition*{s.}{Sobrenome Su}
  \definition{adj.}{de longa data; antigo; velho | veterano; velho; experiente}
  \definition{v.}{hospedar-se para passar a noite; passar a noite}
  \seeref{xiu3}
  \seeref{xiu4}
\end{EntryWithPhonetic}

\begin{EntryWithPhonetic}{宿舍}{su4she4}{11,8}{⼧、⾆}[HSK 5]
  \definition[间,幢]{s.}{alojamento; dormitório; república; albergue; casas onde escolas, empresas, etc. acomodam seus alunos ou funcionários}
\end{EntryWithPhonetic}

\begin{EntryWithPhonetic}{塑}{su4}{13}{⼟}
  \definition{s.}{plástico; material plástico}
  \definition{v.}{modelo; molde; forma}
\end{EntryWithPhonetic}

\begin{EntryWithPhonetic}{塑料}{su4 liao4}{13,10}{⼟、⽃}[HSK 4]
  \definition[块,种]{s.}{plástico; compostos de polímeros feitos de resinas naturais ou sintéticas como componente principal}
\end{EntryWithPhonetic}

\begin{EntryWithPhonetic}{塑料袋}{su4liao4dai4}{13,10,11}{⼟、⽃、⾐}[HSK 4]
  \definition[个,只]{s.}{saco plástico; sacola de plástico}
\end{EntryWithPhonetic}

\begin{EntryWithPhonetic}{痠}{suan1}{12}{⽧}
  \definition{v.}{doer | estar dolorido}
  \variantof{酸}
\end{EntryWithPhonetic}

\begin{EntryWithPhonetic}{酸}{suan1}{14}{⾣}[HSK 4]
  \definition{adj.}{azedo; ácido | aflito; angustiado; doente do coração | pedante; descreve uma pessoa que finge ser culta e também descreve uma pessoa que é muito inflexível com suas próprias ideias e não está disposta a mudá-las para atender às exigências da época, é usado principalmente para satirizar intelectuais que fingem ser capazes de escrever poemas e artigos | ciumento; invejoso; sentimentos desconfortáveis porque outra pessoa é melhor do que você e, em geral, também apresenta comportamento hostil}
  \definition{s.}{ácido; produto químico que tem um sabor ácido quando misturado com água}
  \definition{v.}{estar dolorido (devido à fadiga ou doença); descreve a sensação de não ter força muscular e um pouco de dor por estar doente ou muito cansado}
\end{EntryWithPhonetic}

\begin{EntryWithPhonetic}{酸辣汤}{suan1la4tang1}{14,14,6}{⾣、⾟、⽔}
  \definition{s.}{sopa avinagrada e picante (prato)}
\end{EntryWithPhonetic}

\begin{EntryWithPhonetic}{酸奶}{suan1 nai3}{14,5}{⾣、⼥}[HSK 4]
  \definition[瓶,杯,盒,袋]{s.}{iogurte; produto lácteo fermentado por bactérias de ácido láctico}
\end{EntryWithPhonetic}

\begin{EntryWithPhonetic}{酸甜苦辣}{suan1 tian2 ku3 la4}{14,11,8,14}{⾣、⽢、⾋、⾟}[HSK 5]
  \definition{expr.}{os altos e baixos da vida; as experiências agridoces da vida; os aspectos doces, azedos, amargos e picantes da vida; refere-se a todos os tipos de sabores, como metáfora para experiências diversas, como felicidade, sofrimento, etc. | azedo, doce, amargo, picante --- alegrias e tristezas da vida}
\end{EntryWithPhonetic}

\begin{EntryWithPhonetic}{算}{suan4}{14}{⽵}[HSK 2]
  \definition{adv.}{finalmente; por fim; no final; significa que, após um longo período de tempo ou muitas dificuldades, finalmente se alcançou o objetivo, equivalente a 总算}
  \definition{v.}{calcular; estimar; computar | contar; incluir | planejar; calcular; projetar | pensar; supor; especular | considerar; considerar como; contar como; reconhecer como | (aritmética) contar; ter peso | deixe estar; deixe passar; seguido por 了: desistir, não se importar mais}
  \seealsoref{了}{le5}
  \seealsoref{总算}{zong3suan4}
\end{EntryWithPhonetic}

\begin{EntryWithPhonetic}{算了}{suan4 le5}{14,2}{⽵、⼅}[HSK 6]
  \definition{part.}{deixe estar; deixe passar; usado no final de uma frase para expressar imperativo, término, etc.}
  \definition{v.}{deixar;  deixe estar; deixe passar; esquecer isso; não querer continuar; é usado para persuadir os outros ou para expressar que posso aceitar a situação atual, para encerrar o assunto ou assunto atual, ou para dizer "esqueça"}
\end{EntryWithPhonetic}

\begin{EntryWithPhonetic}{算命}{suan4ming4}{14,8}{⽵、⼝}
  \definition{s.}{cartomante}
  \definition{v.}{ler a sorte | fazer advinhações}
\end{EntryWithPhonetic}

\begin{EntryWithPhonetic}{算是}{suan4 shi4}{14,9}{⽵、⽇}[HSK 6]
  \definition{adv.}{finalmente; por fim; depois de muito tempo, o objetivo foi finalmente alcançado}
  \definition{v.}{contar como; pensar que; ser considerado}
\end{EntryWithPhonetic}

\begin{EntryWithPhonetic}{尿}{sui1}{7}{⼫}
  \definition{s.}{(coloquial) urina}
  \seeref{niao4}
\end{EntryWithPhonetic}

\begin{EntryWithPhonetic}{虽}{sui1}{9}{⾍}[HSK 6]
  \definition{conj.}{no entanto; embora | mesmo se}
\end{EntryWithPhonetic}

\begin{EntryWithPhonetic}{虽然}{sui1 ran2}{9,12}{⾍、⽕}[HSK 2]
  \definition{conj.}{apesar de; embora (frequentemente usado correlativamente com 可是, 但是, etc); geralmente é usado no início de uma frase para indicar que o fato anterior foi reconhecido, mas não mudará o que acontecerá em seguida}
  \seealsoref{但是}{dan4 shi4}
  \seealsoref{可是}{ke3shi4}
\end{EntryWithPhonetic}

\begin{EntryWithPhonetic}{随}{sui2}{11}{⾩}[HSK 3]
  \definition*{s.}{Sobrenome Sui}
  \definition{adv.}{fazer algo imediatamente assim que ocorre, sem demora ou hesitação; usado antes de dois verbos ou frases verbais para indicar que a última ação segue a anterior}
  \definition{prep.}{junto com (alguma outra ação) | apresentando as condições das quais a ação depende}
  \definition{v.}{seguir; vir (ou ir) junto com | concordar com; adaptar-se a | deixar (alguém fazer o que quiser) | (dialeto) parecer-se com; assemelhar-se a | seguir ou agir de acordo com a condição ou circunstância da qual a ação depende}
\end{EntryWithPhonetic}

\begin{EntryWithPhonetic}{随便}{sui2bian4}{11,9}{⾩、⼈}[HSK 2]
  \definition{adj.}{relaxado; descontraído; sem restrições; sem limitações | aleatório; casual; descuidado; indiferente; distraído, não pensa bem antes de falar ou agir | casual; informal; não dá importância aos detalhes}
  \definition{conj.}{qualquer; qualquer que seja; não importa}
  \definition{v.}{deixar alguém à vontade}
\end{EntryWithPhonetic}

\begin{EntryWithPhonetic}{随处}{sui2chu4}{11,5}{⾩、⼡}
  \definition{adv.}{em qualquer lugar}
\end{EntryWithPhonetic}

\begin{EntryWithPhonetic}{随地}{sui2di4}{11,6}{⾩、⼟}
  \definition{adv.}{qualquer lugar | todo lugar}
\end{EntryWithPhonetic}

\begin{EntryWithPhonetic}{随后}{sui2 hou4}{11,6}{⾩、⼝}[HSK 5]
  \definition{adv.}{logo em seguida; logo depois; indica que segue imediatamente após a ação ou situação anterior (geralmente usado em conjunto com 就)}
  \seealsoref{就}{jiu4}
\end{EntryWithPhonetic}

\begin{EntryWithPhonetic}{随机存取存储器}{sui2ji1cun2qu3cun2chu3qi4}{11,6,6,8,6,12,16}{⾩、⽊、⼦、⼜、⼦、⼈、⼝}
  \definition{s.}{RAM (\emph{random access memory})}
  \seealsoref{内存}{nei4cun2}
  \seealsoref{随机存取记忆体}{sui2ji1cun2qu3ji4yi4ti3}
\end{EntryWithPhonetic}

\begin{EntryWithPhonetic}{随机存取记忆体}{sui2ji1cun2qu3ji4yi4ti3}{11,6,6,8,5,4,7}{⾩、⽊、⼦、⼜、⾔、⼼、⼈}
  \definition{s.}{RAM (\emph{random access memory})}
  \seealsoref{内存}{nei4cun2}
  \seealsoref{随机存取存储器}{sui2ji1cun2qu3cun2chu3qi4}
\end{EntryWithPhonetic}

\begin{EntryWithPhonetic}{随时}{sui2shi2}{11,7}{⾩、⽇}[HSK 2]
  \definition{adv.}{a qualquer momento; em todos os momentos}
\end{EntryWithPhonetic}

\begin{EntryWithPhonetic}{随手}{sui2shou3}{11,4}{⾩、⼿}[HSK 4]
  \definition{adv.}{convenientemente; sem problemas adicionais; casualmente}
\end{EntryWithPhonetic}

\begin{EntryWithPhonetic}{随意}{sui2yi4}{11,13}{⾩、⼼}[HSK 5]
  \definition{adj.}{aleatório; casual; à vontade; como se deseja}
\end{EntryWithPhonetic}

\begin{EntryWithPhonetic}{随着}{sui2zhe5}{11,11}{⾩、⽬}[HSK 5]
  \definition{prep.}{junto com; na esteira de; em sintonia com; usado no início da frase ou antes do verbo, indica as condições necessárias para que uma ação, comportamento ou evento ocorra}
\end{EntryWithPhonetic}

\begin{EntryWithPhonetic}{岁}{sui4}{6}{⼭}[HSK 1]
  \definition{clas.}{usado para anos (de idade)}
  \definition{s.}{ano (literário) | colheita do ano (literário) | idade | tempo (literário) | ano (de idade) | ano (para as colheitas)}
\end{EntryWithPhonetic}

\begin{EntryWithPhonetic}{岁数}{sui4 shu4}{6,13}{⼭、⽁}[HSK 6]
  \definition{s.}{idade; anos; a idade de uma pessoa}
\end{EntryWithPhonetic}

\begin{EntryWithPhonetic}{岁月}{sui4yue4}{6,4}{⼭、⽉}[HSK 5]
  \definition[段,番]{s.}{anos; ano e mês; refere-se a tempo em geral}
\end{EntryWithPhonetic}

\begin{EntryWithPhonetic}{碎}{sui4}{13}{⽯}[HSK 5]
  \definition*{s.}{Sobrenome Sui}
  \definition{adj.}{quebrado; fragmentado | tagarela; falante}
  \definition{v.}{quebrar em pedaços; esmagar}
\end{EntryWithPhonetic}

\begin{EntryWithPhonetic}{隧}{sui4}{14}{⾩}
  \definition{s.}{túnel; passagem subterrânea | estrada | subúrbios; áreas suburbanas}
  \definition{v.}{virar}
\end{EntryWithPhonetic}

\begin{EntryWithPhonetic}{隧道}{sui4dao4}{14,12}{⾩、⾡}
  \definition{s.}{túnel}
\end{EntryWithPhonetic}

\begin{EntryWithPhonetic}{孙}{sun1}{6}{⼦}
  \definition*{s.}{Sobrenome Sun}
  \definition{s.}{neto; neta | gerações abaixo da do neto | parentes pertencentes à geração do neto | segundo crescimento das plantas}
\end{EntryWithPhonetic}

\begin{EntryWithPhonetic}{孙女}{sun1nv3}{6,3}{⼦、⼥}[HSK 4]
  \definition[个]{s.}{filha do filho; neta}
\end{EntryWithPhonetic}

\begin{EntryWithPhonetic}{孙武}{sun1wu3}{6,8}{⼦、⽌}
  \definition*{s.}{Sun Wu, também conhecido por Sun Tzu, 孙子, general, estrategista e filósofo autor do ``Arte da Guerra'', 《孙子兵法》}
  \seealsoref{孙子}{sun1zi3}
  \seealsoref{孙子兵法}{sun1zi3 bing1fa3}
\end{EntryWithPhonetic}

\begin{EntryWithPhonetic}{孙子}{sun1zi3}{6,3}{⼦、⼦}
  \definition*{s.}{Sun Tzu, também conhecido por Sun Wu, 孙武, general, estrategista e filósofo autor do ``Arte da Guerra'', 《孙子兵法》}
  \seeref{sun1zi5}
  \seealsoref{孙武}{sun1wu3}
  \seealsoref{孙子兵法}{sun1zi3 bing1fa3}
\end{EntryWithPhonetic}

\begin{EntryWithPhonetic}{孙子兵法}{sun1zi3 bing1fa3}{6,3,7,8}{⼦、⼦、⼋、⽔}
  \definition*{s.}{``Arte da Guerra'', o antigo clássico chinês sobre estratégia militar, escrito por Sun Tzu, 孫子}
  \seealsoref{孙武}{sun1wu3}
  \seealsoref{孙子}{sun1zi3}
\end{EntryWithPhonetic}

\begin{EntryWithPhonetic}{孙子}{sun1zi5}{6,3}{⼦、⼦}[HSK 4]
  \definition[个]{s.}{filho do filho; neto}
  \seeref{sun1zi3}
\end{EntryWithPhonetic}

\begin{EntryWithPhonetic}{损}{sun3}{10}{⼿}
  \definition{adj.}{sarcástico; cortante; de ​​língua afiada; maldoso; mau; cruel}
  \definition{v.}{diminuir; perder; reduzir | prejudicar; danificar; degradar; destruir; arruinar; destruir o estado original ou fazê-lo perder sua eficácia original | ser sarcástico; ser cáustico; ser cortante; ferir; insultar; usar palavras duras para zombar de alguém}
\end{EntryWithPhonetic}

\begin{EntryWithPhonetic}{损害}{sun3 hai4}{10,10}{⼿、⼧}[HSK 5]
  \definition{v.}{prejudicar; danificar; ferir; causar danos; causar perdas}
\end{EntryWithPhonetic}

\begin{EntryWithPhonetic}{损失}{sun3shi1}{10,5}{⼿、⼤}[HSK 5]
  \definition{s.}{perda; desperdício; algo que se consome ou se perde sem custo algum}
  \definition{v.}{perder; consumir ou perder}
\end{EntryWithPhonetic}

\begin{EntryWithPhonetic}{笋}{sun3}{10}{⽵}
  \definition{s.}{broto de bambu}
\end{EntryWithPhonetic}

\begin{EntryWithPhonetic}{莎}{suo1}{10}{⾋}
  \seeref{sha1}
\end{EntryWithPhonetic}

\begin{EntryWithPhonetic}{缩}{suo1}{14}{⽷}
  \definition*{s.}{Sobrenome Suo}
  \definition{v.}{contrair; encolher | recuar; retirar-se | economizar}
\end{EntryWithPhonetic}

\begin{EntryWithPhonetic}{缩短}{suo1duan3}{14,12}{⽷、⽮}[HSK 4]
  \definition{v.}{encurtar; reduzir; diminuir}
\end{EntryWithPhonetic}

\begin{EntryWithPhonetic}{缩小}{suo1 xiao3}{14,3}{⽷、⼩}[HSK 4]
  \definition{v.}{reduzir, estreitar, encolher;  tornar menor (em oposição a 放大)}
  \seealsoref{放大}{fang4da4}
\end{EntryWithPhonetic}

\begin{EntryWithPhonetic}{缩影卡片}{suo1ying3 ka3pian4}{14,15,5,4}{⽷、⼺、⼘、⽚}
  \definition{s.}{cartão em miniatura}
\end{EntryWithPhonetic}

\begin{EntryWithPhonetic}{所}{suo3}{8}{⼾}[HSK 3,6]
  \definition*{s.}{Sobrenome Suo}
  \definition{clas.}{usado para casas, etc.}
  \definition{part.}{usado com 为 ou 被 para indicar voz passiva | usado antes do verbo para formar um substantivo ou para qualificar um substantivo | usado antes do verbo na estrutura sujeito-predicado usada como complemento, indica que o termo central é o objeto}
  \definition{s.}{lugar | usado como nome de órgãos governamentais ou outros locais de trabalho}
  \seealsoref{被}{bei4}
  \seealsoref{为}{wei4}
\end{EntryWithPhonetic}

\begin{EntryWithPhonetic}{所长}{suo3 chang2}{8,4}{⼾、⾧}
  \definition{s.}{aquilo em que alguém é bom; o ponto forte de alguém; o forte de alguém}
  \seeref{suo3 zhang3}
\end{EntryWithPhonetic}

\begin{EntryWithPhonetic}{所以}{suo3 yi3}{8,4}{⼾、⼈}[HSK 2]
  \definition{conj.}{assim; portanto; como resultado; conecta frases, expressa resultados e costuma corresponder a expressões como 因为 e 由于}
  \definition[个]{s.}{motivo real; causa real; comportamento adequado}
  \seealsoref{因为}{yin1wei4}
  \seealsoref{由于}{you2yu2}
\end{EntryWithPhonetic}

\begin{EntryWithPhonetic}{所有}{suo3you3}{8,6}{⼾、⽉}[HSK 2]
  \definition{adj.}{todo | tudo}
  \definition{adj.}{tudo}
  \definition{s.}{bens; posses;}
  \definition{v.}{possuir; ter}
\end{EntryWithPhonetic}

\begin{EntryWithPhonetic}{所在}{suo3 zai4}{8,6}{⼾、⼟}[HSK 5]
  \definition[个]{s.}{lugar; local; localização | o lugar onde alguém ou algo está}
\end{EntryWithPhonetic}

\begin{EntryWithPhonetic}{所长}{suo3 zhang3}{8,4}{⼾、⾧}[HSK 3]
  \definition{s.}{chefe de um instituto, etc. | superintendente}
  \seeref{suo3 chang2}
\end{EntryWithPhonetic}

\begin{EntryWithPhonetic}{索}{suo3}{10}{⽷}
  \definition*{s.}{Sobrenome Suo}
  \definition{adj.}{completamente sozinho; sozinho | maçante; insípido; sem significado}
  \definition[根]{s.}{corda; cabo; cordão; corrente | uma corda grande}
  \definition{v.}{(literário) pesquisar | exigir; pedir}
\end{EntryWithPhonetic}

\begin{EntryWithPhonetic}{索性}{suo3xing4}{10,8}{⽷、⼼}
  \definition{adv.}{poderia muito bem | simplesmente | apenas}
\end{EntryWithPhonetic}

\begin{EntryWithPhonetic}{锁}{suo3}{12}{⾦}[HSK 5]
  \definition[把]{s.}{fechadura; dispositivo que impede que as pessoas abram facilmente a parte que se abre e fecha | correntes; cadeado e correntes | qualquer coisa com a forma de um cadeado antigo}
  \definition{v.}{trancar; trancar com chave | costurar com ponto fixo | tricotar}
\end{EntryWithPhonetic}

%%%%% EOF %%%%%


 %%%
%%% T
%%%

\section*{T}\addcontentsline{toc}{section}{T}

\begin{EntryWithPhonetic}{T-恤}{t5 xu4}{∅,9}{∅、⼼}
  \definition{s.}{camiseta | pulôver | suéter}
\end{EntryWithPhonetic}

\begin{EntryWithPhonetic}{他}{ta1}{5}{⼈}[HSK 1]
  \definition{pron.}{ele | outro; referindo-se a outro; diferente | usado após o verbo, indica referência vaga | alguém; todos; usado em conjunto com 你, significa qualquer pessoa ou muitas pessoas | em outro lugar; outro lugar}
  \seealsoref{你}{ni3}
  \seealsoref{怹}{tan1}
\end{EntryWithPhonetic}

\begin{EntryWithPhonetic}{他的}{ta1 de5}{5,8}{⼈、⽩}
  \definition{pron.}{dele}
\end{EntryWithPhonetic}

\begin{EntryWithPhonetic}{他妈的}{ta1ma1de5}{5,6,8}{⼈、⼥、⽩}
  \definition{interj.}{Dane-se! | Foda-se!}
\end{EntryWithPhonetic}

\begin{EntryWithPhonetic}{他们}{ta1men5}{5,5}{⼈、⼈}[HSK 1]
  \definition{pron.}{eles}
\end{EntryWithPhonetic}

\begin{EntryWithPhonetic}{他们的}{ta1men5 de5}{5,5,8}{⼈、⼈、⽩}
  \definition{pron.}{deles}
\end{EntryWithPhonetic}

\begin{EntryWithPhonetic}{它}{ta1}{5}{⼧}[HSK 2]
  \definition*{s.}{Sobrenome Ta}
  \definition{pron.}{ele; referência a algo além da pessoa (para objetos inanimados) | ele; usado após o verbo, indica referência vaga}
\end{EntryWithPhonetic}

\begin{EntryWithPhonetic}{它们}{ta1 men5}{5,5}{⼧、⼈}[HSK 2]
  \definition{pron.}{eles; usado para se referir a mais de uma coisa não humana; geralmente se refere a animais, objetos ou conceitos abstratos}
\end{EntryWithPhonetic}

\begin{EntryWithPhonetic}{她}{ta1}{6}{⼥}[HSK 1]
  \definition{pron.}{ela | ela; referir-se a coisas que se ama ou aprecia, como a pátria, a bandeira nacional, etc.}
\end{EntryWithPhonetic}

\begin{EntryWithPhonetic}{她的}{ta1 de5}{6,8}{⼥、⽩}
  \definition{pron.}{dela}
\end{EntryWithPhonetic}

\begin{EntryWithPhonetic}{她们}{ta1men5}{6,5}{⼥、⼈}[HSK 1]
  \definition{pron.}{elas; referindo-se a várias mulheres: em textos escritos, use 她们 quando todas as pessoas forem mulheres e 他们 quando houver homens e mulheres}
  \seealsoref{他们}{ta1men5}
\end{EntryWithPhonetic}

\begin{EntryWithPhonetic}{她们的}{ta1men5 de5}{6,5,8}{⼥、⼈、⽩}
  \definition{pron.}{delas}
\end{EntryWithPhonetic}

\begin{EntryWithPhonetic}{踏}{ta1}{15}{⾜}
  \definition{part.}{Caracter formador de palavras}
  \seeref{踏}{ta4}
\end{EntryWithPhonetic}

\begin{EntryWithPhonetic}{踏实}{ta1shi5}{15,8}{⾜、⼧}[HSK 6]
  \definition{adj.}{confiável; sério; estável e seguro; descreve uma atitude séria em relação ao trabalho ou estudo | à vontade; livre de ansiedade; descreve uma mente ou sentimento estável, sem qualquer preocupação ou ansiedade}
\end{EntryWithPhonetic}

\begin{EntryWithPhonetic}{塔}{ta3}{12}{⼟}[HSK 6]
  \definition*{s.}{Sobrenome Ta}
  \definition[个,座]{s.}{pagode budista; pagode | torre | (química) coluna; torre}[蒸馏塔===torre de destilação]
\end{EntryWithPhonetic}

\begin{EntryWithPhonetic}{踏}{ta4}{15}{⾜}[HSK 6]
  \definition{v.}{por os pés em; pisar em; esmagar com o pé | fazer uma investigação ou levantamento no local}
  \seeref{踏}{ta1}
\end{EntryWithPhonetic}

\begin{EntryWithPhonetic}{踏板}{ta4ban3}{15,8}{⾜、⽊}
  \definition{s.}{pedal (em um carro, em um piano, etc.) |  apoio para os pés | estribo}
\end{EntryWithPhonetic}

\begin{EntryWithPhonetic}{台}{tai2}{5}{⼝}[HSK 3]
  \definition*{s.}{Sobrenome Tai}
  \definition{clas.}{usado para certas máquinas, aparelhos, instrumentos, etc | usado para uma performance completa, como drama, música e dança}
  \definition{s.}{torre | plataforma; palco | suporte; pedestal | qualquer coisa em forma de plataforma ou palco | mesa; escrivaninha | estação de transmissão; refere-se a estações de rádio | um serviço telefônico especial; refere-se à estação telefônica | ``seu'' (um termo respeitoso usado antigamente para se dirigir a alguém) | tufão}
\end{EntryWithPhonetic}

\begin{EntryWithPhonetic}{台灯}{tai2 deng1}{5,6}{⼝、⽕}[HSK 6]
  \definition[个,盏]{s.}{luminária de mesa; luminária de leitura; uma luminária com base para uso sobre uma mesa}
\end{EntryWithPhonetic}

\begin{EntryWithPhonetic}{台风}{tai2feng1}{5,4}{⼝、⾵}[HSK 5]
  \definition[场,阵,级]{s.}{tufão; classificação de um ciclone tropical ocorrido no oeste do Pacífico Norte | postura; presença de palco; comportamento ou estilo que os atores demonstram no palco}
\end{EntryWithPhonetic}

\begin{EntryWithPhonetic}{台阶}{tai2jie1}{5,6}{⼝、⾩}[HSK 4]
  \definition[个,级]{s.}{escada; escadaria | passos; metáfora para uma maneira ou oportunidade de evitar constrangimentos causados ​​por um impasse | nova fase; novo nível; novo patamar; metáfora para novas conquistas ou novos patamares alcançados no estudo ou no trabalho}
\end{EntryWithPhonetic}

\begin{EntryWithPhonetic}{台上}{tai2 shang4}{5,3}{⼝、⼀}[HSK 4]
  \definition{s.}{no palco}
\end{EntryWithPhonetic}

\begin{EntryWithPhonetic}{台下}{tai2xia4}{5,3}{⼝、⼀}
  \definition{s.}{platéia | fora do palco}
\end{EntryWithPhonetic}

\begin{EntryWithPhonetic}{抬}{tai2}{8}{⼿}[HSK 5]
  \definition{clas.}{usado para objetos que precisam ser carregados por pessoas quando transportados (por exemplo, móveis)}
  \definition{v.}{levantar; elevar; puxar para cima | (por duas ou mais pessoas) carregar; transportar; duas ou mais pessoas carregando algo com as mãos ou nos ombros | discutir, debater (geralmente sem sentido ou sem importância)}
\end{EntryWithPhonetic}

\begin{EntryWithPhonetic}{抬杠}{tai2gang4}{8,7}{⼿、⽊}
  \definition{v.+compl.}{discutir pelo prazer em discutir | discutir obstinadamente | brigar}
\end{EntryWithPhonetic}

\begin{EntryWithPhonetic}{抬头}{tai2 tou2}{8,5}{⼿、⼤}[HSK 5]
  \definition{s.}{(em recibos, contas, etc.) nome do comprador ou beneficiário, o local no documento onde o nome do beneficiário ou destinatário é escrito}
  \definition{v.}{levantar a cabeça}
\end{EntryWithPhonetic}

\begin{EntryWithPhonetic}{太}{tai4}{4}{⼤}[HSK 1]
  \definition*{s.}{Sobrenome Tai}
  \definition{adj.}{mais alto; maior; mais distante | maior; extremo | bisavô; mais velho ou mais antigo; o de maior posição social ou hierarquia}
  \definition{adv.}{demais; expressa um grau excessivo (usado principalmente para coisas indesejáveis) | muito; extremamente; excessivamente; indica um grau extremamente elevado | muito; usado após o advérbio negativo 不, enfraquece o grau de negação e contém um tom diplomático}
\end{EntryWithPhonetic}

\begin{EntryWithPhonetic}{太极拳}{tai4ji2quan2}{4,7,10}{⼤、⽊、⼿}
  \definition*{s.}{Tai Chi Chuan, Taiji, T'aichi ou T'aichichuan; forma tradicional de exercício físico ou relaxamento}
\end{EntryWithPhonetic}

\begin{EntryWithPhonetic}{太空}{tai4kong1}{4,8}{⼤、⽳}[HSK 5]
  \definition[把]{s.}{firmamento; espaço sideral; espaço além da atmosfera terrestre; o céu vasto e infinito}
\end{EntryWithPhonetic}

\begin{EntryWithPhonetic}{太平洋}{tai4ping2 yang2}{4,5,9}{⼤、⼲、⽔}
  \definition*{s.}{Oceano Pacífico}
\end{EntryWithPhonetic}

\begin{EntryWithPhonetic}{太太}{tai4tai5}{4,4}{⼤、⼤}[HSK 2]
  \definition[位,名,个,些]{s.}{senhora; madame; títulos para mulheres casadas | esposa; senhora; madame; referir-se à própria esposa ou à esposa de outra pessoa}
\end{EntryWithPhonetic}

\begin{EntryWithPhonetic}{太阳窗}{tai4yang2chuang1}{4,6,12}{⼤、⾩、⽳}
  \definition{s.}{teto solar (de veículos)}
\end{EntryWithPhonetic}

\begin{EntryWithPhonetic}{太阳灯}{tai4yang2deng1}{4,6,6}{⼤、⾩、⽕}
  \definition{s.}{lâmpada solar (com células fotovoltaicas)}
\end{EntryWithPhonetic}

\begin{EntryWithPhonetic}{太阳风}{tai4yang2feng1}{4,6,4}{⼤、⾩、⾵}
  \definition{s.}{vento solar}
\end{EntryWithPhonetic}

\begin{EntryWithPhonetic}{太阳镜}{tai4yang2jing4}{4,6,16}{⼤、⾩、⾦}
  \definition{s.}{óculos de sol}
\end{EntryWithPhonetic}

\begin{EntryWithPhonetic}{太阳能}{tai4 yang2 neng2}{4,6,10}{⼤、⾩、⾁}[HSK 6]
  \definition{s.}{energia solar; a energia de radiação emitida pelo Sol é a energia solar produzida pela reação de fusão dos núcleos de hidrogênio no Sol, é a fonte de luz e calor na Terra}
\end{EntryWithPhonetic}

\begin{EntryWithPhonetic}{太阳日}{tai4yang2ri4}{4,6,4}{⼤、⾩、⽇}
  \definition{s.}{dia solar}
\end{EntryWithPhonetic}

\begin{EntryWithPhonetic}{太阳穴}{tai4yang2xue2}{4,6,5}{⼤、⾩、⽳}
  \definition{s.}{têmpora (nas laterais da cabeça humana)}
\end{EntryWithPhonetic}

\begin{EntryWithPhonetic}{太阳翼}{tai4yang2yi4}{4,6,17}{⼤、⾩、⽻}
  \definition{s.}{painel solar}
\end{EntryWithPhonetic}

\begin{EntryWithPhonetic}{太阳雨}{tai4yang2yu3}{4,6,8}{⼤、⾩、⾬}
  \definition{s.}{banho de sol}
\end{EntryWithPhonetic}

\begin{EntryWithPhonetic}{太阳}{tai4yang5}{4,6}{⼤、⾩}[HSK 2]
  \definition[个,轮,枚,颗,盏]{s.}{o Sol | luz do sol; luz solar}
\end{EntryWithPhonetic}

\begin{EntryWithPhonetic}{态}{tai4}{8}{⼼}
  \definition{s.}{forma; aparência; condição | (física) estado | (linguística) voz}[气态===estado gasoso | 被动态===voz passiva]
\end{EntryWithPhonetic}

\begin{EntryWithPhonetic}{态度}{tai4du5}{8,9}{⼼、⼴}[HSK 2]
  \definition[种,个]{s.}{maneira; comportamento; atitude; comportamento e expressão facial das pessoas | atitude; abordagem; opinião sobre o assunto e medidas tomadas}
\end{EntryWithPhonetic}

\begin{EntryWithPhonetic}{贪}{tan1}{8}{⾙}
  \definition{adj.}{corrupto; venal | ganancioso; avarento; ambicioso}
  \definition{v.}{apropriar-se indevidamente; praticar corrupção; ser corrupto | ter um desejo insaciável por; ter um desejo voraz por | cobiçar; ansiar por; ser ganancioso por}
\end{EntryWithPhonetic}

\begin{EntryWithPhonetic}{贪婪}{tan1lan2}{8,11}{⾙、⼥}
  \definition{adj.}{avaro | ambicioso | voraz | insaciável}
\end{EntryWithPhonetic}

\begin{EntryWithPhonetic}{怹}{tan1}{9}{⼼}
  \definition{pron.}{ele, ela (cortês, em oposição a 他)}
  \seealsoref{他}{ta1}
\end{EntryWithPhonetic}

\begin{EntryWithPhonetic}{谈}{tan2}{10}{⾔}[HSK 3]
  \definition*{s.}{Sobrenome Tan}
  \definition{s.}{o que é dito ou falado; discurso}
  \definition{v.}{falar; bater papo; discutir}
\end{EntryWithPhonetic}

\begin{EntryWithPhonetic}{谈话}{tan2 hua4}{10,8}{⾔、⾔}[HSK 3]
  \definition[次]{s.}{declaração; opiniões (principalmente políticas) expressas na forma de conversas}
  \definition{v.+compl.}{conversar; discutir | falar; refere-se especificamente ao uso da conversa para entender a situação, fazer trabalho ideológico, etc. (usado principalmente por superiores para subordinados)}
\end{EntryWithPhonetic}

\begin{EntryWithPhonetic}{谈恋爱}{tan2lian4'ai4}{10,10,10}{⾔、⼼、⽖}
  \definition{v.}{namorar | apaixonar-se}
\end{EntryWithPhonetic}

\begin{EntryWithPhonetic}{谈判}{tan2pan4}{10,7}{⾔、⼑}[HSK 3]
  \definition{v.}{negociar; manter conversações; para resolver um grande problema, as partes relevantes trocaram opiniões entre si, na esperança de encontrar uma solução com a qual todos pudessem concordar}
\end{EntryWithPhonetic}

\begin{EntryWithPhonetic}{弹}{tan2}{11}{⼸}[HSK 5]
  \definition{v.}{enviar; atirar (como com uma catapulta, etc.); usar a elasticidade de um objeto para lançar outro objeto | afofar; preparar fibras; usar um dispositivo elástico para amolecer as fibras | virar; sacudir | dedilhar; tocar (um instrumento musical de cordas) | acusar; atacar; criticar; relatar | saltar; quicar}
\end{EntryWithPhonetic}

\begin{EntryWithPhonetic}{坦}{tan3}{8}{⼟}
  \definition*{s.}{Sobrenome Tan}
  \definition{adj.}{nivelado; suave; plano | calmo; composto | aberto; sincero; franco}
\end{EntryWithPhonetic}

\begin{EntryWithPhonetic}{坦克}{tan3ke4}{8,7}{⼟、⼗}
  \definition{s.}{(empréstimo linguístico) tanque (veículo militar)}
\end{EntryWithPhonetic}

\begin{EntryWithPhonetic}{叹}{tan4}{5}{⼝}
  \definition{v.}{suspirar | exclamar com admiração; aclamar; louvar |recitar com cadência; entoar cântico; entoar}
\end{EntryWithPhonetic}

\begin{EntryWithPhonetic}{叹气}{tan4qi4}{5,4}{⼝、⽓}[HSK 6]
  \definition{v.}{suspirar; soltar um suspiro; soltar um longo suspiro e fazer um som devido à insatisfação ou desamparo}
\end{EntryWithPhonetic}

\begin{EntryWithPhonetic}{探}{tan4}{11}{⼿}
  \definition[个,位,名]{s.}{batedor; espião; detetive}
  \definition{v.}{tentar descobrir; explorar; soar | explorar; espionar | visitar; fazer uma visita em | se destacar | preocupar-se com; envolver-se em | ver; invocar}
\end{EntryWithPhonetic}

\begin{EntryWithPhonetic}{探亲}{tan4qin1}{11,9}{⼿、⼇}
  \definition{v.+compl.}{ir para casa para visitar a família}
\end{EntryWithPhonetic}

\begin{EntryWithPhonetic}{探索}{tan4suo3}{11,10}{⼿、⽷}[HSK 6]
  \definition{v.}{sondar; explorar; procurar respostas de várias fontes para resolver dúvidas}
\end{EntryWithPhonetic}

\begin{EntryWithPhonetic}{探讨}{tan4tao3}{11,5}{⼿、⾔}[HSK 6]
  \definition{v.}{examinar; indagar; investigar; discutir}
\end{EntryWithPhonetic}

\begin{EntryWithPhonetic}{碳}{tan4}{14}{⽯}
  \definition{s.}{carbono (elemento químico)}
\end{EntryWithPhonetic}

\begin{EntryWithPhonetic}{汤}{tang1}{6}{⽔}[HSK 3]
  \definition*{s.}{Sobrenome Tang}
  \definition[勺,碗,杯,锅]{s.}{água quente; água fervente | fontes termais | água utilizada para ferver algo| sopa; caldo | uma preparação líquida de ervas medicinais; decocção}
  \seeref{汤}{shang1}
\end{EntryWithPhonetic}

\begin{EntryWithPhonetic}{趟}{tang1}{15}{⾛}
  \definition{v.}{atravessar; andar na grama ou onde não haja caminho | usar arados, capinadores, etc. para virar o solo e remover ervas daninhas | vadear; atravessar a vau; caminhar por águas rasas}[我们趟水去那小岛。===Nós vadeamos até a ilha.]
  \seeref{趟}{tang4}
\end{EntryWithPhonetic}

\begin{EntryWithPhonetic}{唐}{tang2}{10}{⼝}
  \definition*{s.}{Dinastia estabelecida pelo Imperador Yao, 尧, no período lendário da história chinesa | Dinastia Tang (618-907) | Dinastia Tang posterior (923-936), uma das cinco dinastias | Sobrenome Tang}
  \definition{adj.}{exagerado; bombástico; orgulhoso | em vão; por nada}
  \seealsoref{尧}{yao2}
\end{EntryWithPhonetic}

\begin{EntryWithPhonetic}{唐人街}{tang2ren2 jie1}{10,2,12}{⼝、⼈、⾏}
  \definition*[条,座]{s.}{Bairro Chinês; Chinatown; refere-se ao mercado de rua onde os chineses do exterior vivem e abrem muitas lojas com características chinesas}
  \seealsoref{中国城}{zhong1guo2cheng2}
\end{EntryWithPhonetic}

\begin{EntryWithPhonetic}{糖}{tang2}{16}{⽶}[HSK 3]
  \definition[包,斤,勺,袋,块]{s.}{açúcar; um tipo de açúcar; um tipo de composto orgânico, que pode ser dividido em três tipos: monossacarídeos, dissacarídeos e polissacarídeos; é a principal substância que produz energia térmica no corpo humano, como glicose, sacarose, lactose, amido, etc. | açúcar; açúcar comestível; termo geral para açúcar | doces; balas | carboidrato; algo doce e calórico}
\end{EntryWithPhonetic}

\begin{EntryWithPhonetic}{糖醋鱼}{tang2cu4yu2}{16,15,8}{⽶、⾣、⿂}
  \definition{s.}{peixe guisado em molho agridoce (prato)}
\end{EntryWithPhonetic}

\begin{EntryWithPhonetic}{倘}{tang3}{10}{⼈}
  \definition{conj.}{se; supondo; no caso}
  \seeref{倘}{chang2}
\end{EntryWithPhonetic}

\begin{EntryWithPhonetic}{倘或}{tang3huo4}{10,8}{⼈、⼽}
  \definition{conj.}{se | supondo que | no caso}
\end{EntryWithPhonetic}

\begin{EntryWithPhonetic}{倘若}{tang3ruo4}{10,8}{⼈、⾋}
  \definition{conj.}{se | supondo que | no caso}
\end{EntryWithPhonetic}

\begin{EntryWithPhonetic}{倘使}{tang3shi3}{10,8}{⼈、⼈}
  \definition{conj.}{se | supondo que | no caso}
\end{EntryWithPhonetic}

\begin{EntryWithPhonetic}{躺}{tang3}{15}{⾝}[HSK 4]
  \definition{v.}{deitar; reclinar; cair no chão ou sobre um objeto}
\end{EntryWithPhonetic}

\begin{EntryWithPhonetic}{趟}{tang4}{15}{⾛}[HSK 6]
  \definition{clas.}{usado para o número de vezes de viagens de ida e volta |  usado para coisas dispostas em fileiras ou tiras | usado para a programação de veículos, navios, etc. que circulam em uma determinada ordem | usado em conjuntos de movimentos de artes marciais}
  \definition{s.}{marcha; procissão; jornada; viagem}
\end{EntryWithPhonetic}

\begin{EntryWithPhonetic}{掏}{tao1}{11}{⼿}[HSK 6]
  \definition{v.}{extrair; retirar; pescar | cavar (um buraco, etc.); escavar; retirar | (coloquial) roubar do bolso de alguém | tirar}
\end{EntryWithPhonetic}

\begin{EntryWithPhonetic}{滔}{tao1}{13}{⽔}
  \definition{adj.}{(de água) transbordando | arrogante | turbulento | largo e longo; grande}
  \definition{v.}{inundar; alagar}
\end{EntryWithPhonetic}

\begin{EntryWithPhonetic}{滔天}{tao1tian1}{13,4}{⽔、⼤}
  \definition{adj.}{(ondas, raiva, desastres, crimes, etc.) imponente, avassalador, imenso}
\end{EntryWithPhonetic}

\begin{EntryWithPhonetic}{逃}{tao2}{9}{⾡}[HSK 5]
  \definition{v.}{fugir; escapar; correr; dar no pé | evadir; esquivar-se; escapar}
\end{EntryWithPhonetic}

\begin{EntryWithPhonetic}{逃跑}{tao2 pao3}{9,12}{⾡、⾜}[HSK 5]
  \definition{v.}{fugir; escapar; correr; partir para fugir de um ambiente ou de coisas que não lhe são favoráveis}
\end{EntryWithPhonetic}

\begin{EntryWithPhonetic}{逃走}{tao2 zou3}{9,7}{⾡、⾛}[HSK 5]
  \definition{v.}{escapar; afastar-se de pessoas, coisas ou lugares que não são bons para você ou que você não gosta}
\end{EntryWithPhonetic}

\begin{EntryWithPhonetic}{桃}{tao2}{10}{⽊}[HSK 5]
  \definition*{s.}{Sobrenome Tao}
  \definition[个,箱,袋,斤,棵,种]{s.}{pêssego | em forma de pêssego | pessegueiro}
\end{EntryWithPhonetic}

\begin{EntryWithPhonetic}{桃花}{tao2 hua1}{10,7}{⽊、⾋}[HSK 5]
  \definition[朵,枝,株]{s.}{Figurativo: caso amoroso | flor de pessegueiro}
\end{EntryWithPhonetic}

\begin{EntryWithPhonetic}{桃树}{tao2 shu4}{10,9}{⽊、⽊}[HSK 5]
  \definition[棵,株]{s.}{pêssego (árvore) | pessegueiro; pêssegos}
\end{EntryWithPhonetic}

\begin{EntryWithPhonetic}{讨}{tao3}{5}{⾔}
  \definition{v.}{enviar forças armadas para suprimir; enviar uma expedição punitiva contra; enviar exército ou despachar tropas para suprimir ou atacar | denunciar; condenar; censurar | exigir; pedir; implorar por | casar (com uma mulher) | incorrer; convidar | discutir; estudar | provocar; cortejar}
\end{EntryWithPhonetic}

\begin{EntryWithPhonetic}{讨论}{tao3lun4}{5,6}{⾔、⾔}[HSK 2]
  \definition{v.}{discutir; conversar sobre; trocar opiniões ou debater as questões levantadas}
\end{EntryWithPhonetic}

\begin{EntryWithPhonetic}{讨生活}{tao3sheng1huo2}{5,5,9}{⾔、⽣、⽔}
  \definition{v.}{ganhar a vida}
\end{EntryWithPhonetic}

\begin{EntryWithPhonetic}{讨厌}{tao3yan4}{5,6}{⾔、⼚}[HSK 5]
  \definition{adj.}{desagradável; repugnante; repulsivo; irritante; incômodo}
  \definition{v.}{odiar; não gostar; sentir repulsa por}
\end{EntryWithPhonetic}

\begin{EntryWithPhonetic}{套}{tao4}{10}{⼤}[HSK 2]
  \definition{clas.}{usado para coisas agrupadas: conjuntos, coleções, séries, etc.}
  \definition{s.}{estojo; capa; bainha | local onde o rio ou a cordilheira faz uma curva (usado principalmente em nomes de lugares) | enchimento de algodão em roupas e edredons | arreios; corda para amarrar animais | nó; laço; um objeto circular feito com corda ou algo semelhante | cortersia; convenção; conversa fiada; métodos repetitivos | armadilha; truque; conspiração}
  \definition{v.}{sobrepor; interligar | deslizar sobre; cobrir por fora | atrelar; engatar; usar um cinto de segurança | copiar; imitar; seguir o modelo de | extrair; induzir a falar; persuadir alguém a revelar um segredo; induzir; provocar | tentar vencer; aproximar-se de; aproximar-se intencionalmente de outras pessoas para algum propósito | fazer a rosca de um parafuso; usar um macho de rosca ou uma chave de rosca para fazer roscas}
\end{EntryWithPhonetic}

\begin{EntryWithPhonetic}{套餐}{tao4 can1}{10,16}{⼤、⾷}[HSK 4]
  \definition{s.}{combo; pacote de produtos; pacote de serviços; metaforicamente, bens ou projetos que são combinados e levados ao mercado | refeição preparada; pacotes de refeições completos}
\end{EntryWithPhonetic}

\begin{EntryWithPhonetic}{套问}{tao4wen4}{10,6}{⼤、⾨}
  \definition{s.}{retórica}
  \definition{v.}{descobrir por meio de questionamento indireto diplomático}
\end{EntryWithPhonetic}

\begin{EntryWithPhonetic}{特}{te4}{10}{⽜}[HSK 6]
  \definition{adj.}{especial; incomum; particular; excepcional; diferente do geral | especial; solteiro; solitário}
  \definition{adv.}{muito; extremamente | especialmente; para um propósito especial |mas; somente}
  \definition{clas.}{TEX; abreviação para unidades de medida como TEX; a unidade de medida TEX indica a espessura de um fio têxtil através do seu peso}
  \definition{s.}{espião; agente secreto}
\end{EntryWithPhonetic}

\begin{EntryWithPhonetic}{特别}{te4bie2}{10,7}{⽜、⼑}[HSK 2]
  \definition{adj.}{especial; particular; fora do comum; diferente dos outros, com características próprias}
  \definition{adv.}{especialmente; particularmente | ainda mais; em particular; frequentemente usado com 是 | especialmente; deliberadamente; para um propósito específico}
  \seealsoref{是}{shi4}
\end{EntryWithPhonetic}

\begin{EntryWithPhonetic}{特别快车}{te4bie2 kuai4che1}{10,7,7,4}{⽜、⼑、⼼、⾞}
  \definition{s.}{trem expresso; expresso; expresso especial; refere-se a trens de passageiros que param em menos estações e têm menor tempo de viagem do que trens expressos diretos}
\end{EntryWithPhonetic}

\begin{EntryWithPhonetic}{特大}{te4 da4}{10,3}{⽜、⼤}[HSK 6]
  \definition{adj.}{especialmente (excepcionalmente) grande; o mais}
\end{EntryWithPhonetic}

\begin{EntryWithPhonetic}{特地}{te4 di4}{10,6}{⽜、⼟}[HSK 6]
  \definition{adv.}{especialmente; propositalmente; para um propósito especial}
\end{EntryWithPhonetic}

\begin{EntryWithPhonetic}{特点}{te4dian3}{10,9}{⽜、⽕}[HSK 2]
  \definition[个,大]{s.}{característica; peculiaridade; traço distintivo; a singularidade de uma pessoa ou coisa}
\end{EntryWithPhonetic}

\begin{EntryWithPhonetic}{特定}{te4ding4}{10,8}{⽜、⼧}[HSK 5]
  \definition{adj.}{específico; especialmente designado | dado; especificado; específico (pessoa, hora, lugar, local, ambiente, etc.)}
\end{EntryWithPhonetic}

\begin{EntryWithPhonetic}{特技}{te4ji4}{10,7}{⽜、⼿}
  \definition{s.}{efeito especial | dublê}
\end{EntryWithPhonetic}

\begin{EntryWithPhonetic}{特价}{te4 jia4}{10,6}{⽜、⼈}[HSK 4]
  \definition{s.}{oferta especial; preço de barganha; preço especial reduzido}
\end{EntryWithPhonetic}

\begin{EntryWithPhonetic}{特快}{te4 kuai4}{10,7}{⽜、⼼}[HSK 6]
  \definition{adj.}{expresso (trem, entrega etc.)}
  \definition{s.}{trem expresso (opp. 普快); abreviação de 特别快车}
  \seealsoref{特别快车}{te4bie2 kuai4che1}
\end{EntryWithPhonetic}

\begin{EntryWithPhonetic}{特色}{te4se4}{10,6}{⽜、⾊}[HSK 3]
  \definition{s.}{característica; característica distintiva; a cor única, estilo, etc. de um objeto}
\end{EntryWithPhonetic}

\begin{EntryWithPhonetic}{特殊}{te4shu1}{10,10}{⽜、⽍}[HSK 4]
  \definition{adj.}{especial; particular; peculiar; excepcional; incomum}
\end{EntryWithPhonetic}

\begin{EntryWithPhonetic}{特性}{te4 xing4}{10,8}{⽜、⼼}[HSK 5]
  \definition[种,个]{s.}{propriedade específica (ou característica) | característica; sabores | propriedade}
\end{EntryWithPhonetic}

\begin{EntryWithPhonetic}{特意}{te4yi4}{10,13}{⽜、⼼}[HSK 6]
  \definition{adv.}{especialmente; para um propósito especial}
\end{EntryWithPhonetic}

\begin{EntryWithPhonetic}{特有}{te4 you3}{10,6}{⽜、⽉}[HSK 5]
  \definition{adj.}{específico; peculiar; característico; único; exclusivo; especial}
\end{EntryWithPhonetic}

\begin{EntryWithPhonetic}{特征}{te4zheng1}{10,8}{⽜、⼻}[HSK 4]
  \definition[个,种]{s.}{característica; aparência ou o fenômeno característico de uma pessoa ou coisa que pode ser visto de fora}
\end{EntryWithPhonetic}

\begin{EntryWithPhonetic}{疼}{teng2}{10}{⽧}[HSK 2]
  \definition{adj.}{dolorido; doído; sensação de extremo desconforto causada por ferimentos, doenças, etc.}
  \definition{v.}{ferir; machucar | adorar; amar profundamente; gostar muito; cuidar}
\end{EntryWithPhonetic}

\begin{EntryWithPhonetic}{疼痛}{teng2 tong4}{10,12}{⽧、⽧}[HSK 6]
  \definition[阵,种]{s.}{dor; sofrimento; ferimento; descreve a sensação de dor causada por lesão ou doença}
\end{EntryWithPhonetic}

\begin{EntryWithPhonetic}{梯}{ti1}{11}{⽊}
  \definition*{s.}{Sobrenome Ti}
  \definition{adj.}{em forma de escada; em socalcos}
  \definition[个]{s.}{escada; degrau; socalco (são plataformas niveladas, semelhantes a degraus, cortadas em encostas de morros para permitir o cultivo agrícola e evitar a erosão do solo)}
\end{EntryWithPhonetic}

\begin{EntryWithPhonetic}{梯恩梯}{ti1'en1ti1}{11,10,11}{⽊、⼼、⽊}
  \definition{s.}{(empréstimo linguístico) TNT, trinitrotolueno}
\end{EntryWithPhonetic}

\begin{EntryWithPhonetic}{踢}{ti1}{15}{⾜}[HSK 6]
  \definition{v.}{chutar | jogar (por exemplo, futebol)}
\end{EntryWithPhonetic}

\begin{EntryWithPhonetic}{踢爆}{ti1bao4}{15,19}{⾜、⽕}
  \definition{v.}{expor | revelar}
\end{EntryWithPhonetic}

\begin{EntryWithPhonetic}{踢蹋舞}{ti1ta4wu3}{15,17,14}{⾜、⾜、⾇}
  \definition{s.}{sapateado | passo de dança}
\end{EntryWithPhonetic}

\begin{EntryWithPhonetic}{提}{ti2}{12}{⼿}[HSK 2]
  \definition*{s.}{Sobrenome Ti}
  \definition{s.}{concha; utensílio para servir óleo ou vinho | traço ascendente (em caracteres chineses)}
  \definition{v.}{carregar (na mão, com o braço para baixo) ; segurar com as mãos para baixo | elevar; levantar; promover | avançar; antecipar uma data; mudar para uma data anterior; adiar o prazo previsto | levantar; apresentar; indicar ou citar | extrair; retirar (tirar) | (prisioneiros) trazer; entregar | mencionar; referir-se a; abordar}
\end{EntryWithPhonetic}

\begin{EntryWithPhonetic}{提倡}{ti2chang4}{12,10}{⼿、⼈}[HSK 5]
  \definition{v.}{promover; incentivar; recomendar; apresentar as vantagens de algo para incentivar as pessoas a usá-lo ou implementá-lo}
\end{EntryWithPhonetic}

\begin{EntryWithPhonetic}{提出}{ti2 chu1}{12,5}{⼿、⼐}[HSK 2]
  \definition{v.}{levantar; propor; apresentar; expressar seus desejos, ideias, sugestões, etc. por meio de palavras ou textos}
\end{EntryWithPhonetic}

\begin{EntryWithPhonetic}{提到}{ti2 dao4}{12,8}{⼿、⼑}[HSK 2]
  \definition{v.}{mencionar; referir-se a; levantar (assunto)}
\end{EntryWithPhonetic}

\begin{EntryWithPhonetic}{提高}{ti2gao1}{12,10}{⼿、⾼}[HSK 2]
  \definition{v.}{elevar; aprimorar; aumentar; melhorar a posição, o nível, a quantidade, a qualidade e outros aspectos em relação ao estado original}
\end{EntryWithPhonetic}

\begin{EntryWithPhonetic}{提供}{ti2gong1}{12,8}{⼿、⼈}[HSK 4]
  \definition{v.}{oferecer; fornecer; suprir; prover; proporcionar}
\end{EntryWithPhonetic}

\begin{EntryWithPhonetic}{提及}{ti2ji2}{12,3}{⼿、⼃}
  \definition{v.}{mencionar | levantar (um assunto) | chamar a atenção de alguém}
\end{EntryWithPhonetic}

\begin{EntryWithPhonetic}{提交}{ti2 jiao1}{12,6}{⼿、⼇}[HSK 6]
  \definition{v.}{referir-se a; submeter (um problema, etc.) a; enviar questões que precisam ser discutidas, decididas ou tratadas para agências ou reuniões relevantes}
\end{EntryWithPhonetic}

\begin{EntryWithPhonetic}{提起}{ti2 qi3}{12,10}{⼿、⾛}[HSK 5]
  \definition{v.}{mencionar; falar sobre; abordar | levantar; despertar; estimular; revigorar; alegrar/animar | iniciar; instituir; propor | levantar; pegar}
\end{EntryWithPhonetic}

\begin{EntryWithPhonetic}{提前}{ti2qian2}{12,9}{⼿、⼑}[HSK 3]
  \definition{adv.}{antecipadamente; faça uma coisa antes de fazer outra}
  \definition{v.}{avançar; adiantar; mudar para uma data anterior; trazer para frente}
\end{EntryWithPhonetic}

\begin{EntryWithPhonetic}{提升}{ti2 sheng1}{12,4}{⼿、⼗}[HSK 6]
  \definition{v.}{promover; avançar; melhorar (posição, grau, qualidade, etc.) | içar; elevar; transportar (minerais, materiais, etc.) para um local mais alto usando um guincho, etc.}
\end{EntryWithPhonetic}

\begin{EntryWithPhonetic}{提示}{ti2shi4}{12,5}{⼿、⽰}[HSK 5]
  \definition[个]{s.}{dica; lembrete; pistas ou informações fornecidas para chamar a atenção, fazer com que a outra pessoa pense ou compreenda}
  \definition{v.}{solicitar; lembrar; indicar; alertar; levantar questões que o outro não tenha pensado ou não tenha imaginado, para chamar a atenção dele}
\end{EntryWithPhonetic}

\begin{EntryWithPhonetic}{提问}{ti2wen4}{12,6}{⼿、⾨}[HSK 3]
  \definition{v.}{\emph{quiz}; fazer uma pergunta; colocar questões para}
\end{EntryWithPhonetic}

\begin{EntryWithPhonetic}{提醒}{ti2xing3}{12,16}{⼿、⾣}[HSK 4]
  \definition{v.+compl.}{alertar; avisar; advertir; lembrar; apontar para ou chamar a atenção para}
\end{EntryWithPhonetic}

\begin{EntryWithPhonetic}{题}{ti2}{15}{⾴}[HSK 2]
  \definition*{s.}{Sobrenome Ti}
  \definition[个,道]{s.}{tópico; título; assunto; problema; frases que indicam o conteúdo de poemas ou discursos | questão; questões que devem ser respondidas durante os exercícios ou exames | antigamente, referia-se à testa}
  \definition{v.}{inscrever; escrever; assinar}
\end{EntryWithPhonetic}

\begin{EntryWithPhonetic}{题材}{ti2cai2}{15,7}{⾴、⽊}[HSK 5]
  \definition{s.}{tema; assunto; material que compõe as obras literárias e artísticas, ou seja, os eventos ou fenômenos da vida descritos concretamente nas obras}
\end{EntryWithPhonetic}

\begin{EntryWithPhonetic}{题目}{ti2mu4}{15,5}{⾴、⽬}[HSK 3]
  \definition[个,道]{s.}{título; assunto; tópico; o título de um poema ou discurso | quebra-cabeça; problema de exercício; questões a serem respondidas em exercícios ou provas}
\end{EntryWithPhonetic}

\begin{EntryWithPhonetic}{体}{ti3}{7}{⼈}
  \definition{s.}{corpo; parte do corpo | substância; objeto; estado de uma substância | estilo; forma | sistema | estilo de caligrafia | tipo de letra; fonte | (linguística) aspecto (de um verbo) | estrutura; a forma escrita do texto; o gênero da obra}
  \definition{v.}{fazer ou vivenciar algo pessoalmente | colocar-se na posição de outro; colocar-se mentalmente na posição do outro; colocar-se no lugar do outro}
\end{EntryWithPhonetic}

\begin{EntryWithPhonetic}{体操}{ti3 cao1}{7,16}{⼈、⼿}[HSK 4]
  \definition{s.}{ginástica; esportes, exercícios ou performances de vários movimentos, sem armas ou com o auxílio de determinados equipamentos}
\end{EntryWithPhonetic}

\begin{EntryWithPhonetic}{体会}{ti3hui4}{7,6}{⼈、⼈}[HSK 3]
  \definition[个,些,种]{s.}{conhecimento; compreensão; experiência pessoal}
  \definition{v.}{perceber; saber (ou aprender) com a experiência}
\end{EntryWithPhonetic}

\begin{EntryWithPhonetic}{体积}{ti3ji1}{7,10}{⼈、⽲}[HSK 5]
  \definition[个]{s.}{volume; quantidade; o tamanho do espaço ocupado pelo objeto}
\end{EntryWithPhonetic}

\begin{EntryWithPhonetic}{体检}{ti3 jian3}{7,11}{⼈、⽊}[HSK 4]
  \definition{v.}{fazer um exame médico}
\end{EntryWithPhonetic}

\begin{EntryWithPhonetic}{体力}{ti3 li4}{7,2}{⼈、⼒}[HSK 5]
  \definition{s.}{força física; vigor físico (ou corporal); a força do corpo humano para sustentar suas próprias atividades}
\end{EntryWithPhonetic}

\begin{EntryWithPhonetic}{体内}{ti3nei4}{7,4}{⼈、⼌}
  \definition{adj.}{dentro do corpo | \emph{in vivo} (versus \emph{in vitro} | interno a}
\end{EntryWithPhonetic}

\begin{EntryWithPhonetic}{体现}{ti3xian4}{7,8}{⼈、⾒}[HSK 3]
  \definition{v.}{refletir; incorporar; encarnar; uma certa qualidade ou fenômeno se manifesta especificamente em uma determinada coisa}
\end{EntryWithPhonetic}

\begin{EntryWithPhonetic}{体验}{ti3yan4}{7,10}{⼈、⾺}[HSK 3]
  \definition[种]{s.}{experiência; a sensação adquirida pela experiência pessoal}
  \definition{v.}{aprender através da prática; aprender através da experiência pessoal; entender as coisas através da experiência pessoal}
\end{EntryWithPhonetic}

\begin{EntryWithPhonetic}{体育}{ti3yu4}{7,8}{⼈、⾁}[HSK 2]
  \definition{s.}{cultura física; treinamento físico; educação cuja principal tarefa é desenvolver a capacidade física e fortalecer a constituição física, alcançada através da participação em várias atividades esportivas | esportes; atividades esportivas; refere-se a esportes}
\end{EntryWithPhonetic}

\begin{EntryWithPhonetic}{体育场}{ti3 yu4 chang3}{7,8,6}{⼈、⾁、⼟}[HSK 2]
  \definition[个,座]{s.}{estádio; campo esportivo; espaço ao ar livre para a prática de exercícios físicos ou competições esportivas}
\end{EntryWithPhonetic}

\begin{EntryWithPhonetic}{体育馆}{ti3 yu4 guan3}{7,8,11}{⼈、⾁、⾷}[HSK 2]
  \definition[个,座,家]{s.}{ginásio; locais esportivos ou competições em ambientes fechados geralmente têm arquibancadas fixas}
\end{EntryWithPhonetic}

\begin{EntryWithPhonetic}{体重}{ti3 zhong4}{7,9}{⼈、⾥}[HSK 4]
  \definition{s.}{peso corporal}
\end{EntryWithPhonetic}

\begin{EntryWithPhonetic}{替}{ti4}{12}{⽈}[HSK 4]
  \definition{prep.}{para; em nome de}
  \definition{s.}{decadência; declínio; enfraquecimento}
  \definition{v.}{substituir; substituir por; tomar o lugar de}
\end{EntryWithPhonetic}

\begin{EntryWithPhonetic}{替代}{ti4 dai4}{12,5}{⽈、⼈}[HSK 4]
  \definition{v.}{substituir; suplantar}
\end{EntryWithPhonetic}

\begin{EntryWithPhonetic}{天}{tian1}{4}{⼤}[HSK 1]
  \definition*{s.}{Sobrenome Tian}
  \definition{adj.}{localizado no topo; suspenso no ar | inato; natural}
  \definition{clas.}{usado para contar dias}
  \definition{s.}{céu; paraíso; espaço onde se encontram o sol, a lua e as estrelas | dia; as 24 horas do dia, às vezes referindo-se especificamente ao período diurno | um período de tempo em um dia; em algum momento do dia | temporada; estação do ano | clima | natureza | Deus; céu; o criador | paraíso; refere-se ao local onde residem os deuses, budas e imortais}
\end{EntryWithPhonetic}

\begin{EntryWithPhonetic}{天才}{tian1cai2}{4,3}{⼤、⼿}[HSK 5]
  \definition{adj.}{talentoso | superdotado | genial}
  \definition[个,位,名]{s.}{dom; genialidade; talento natural; inteligência e sabedoria acima da média}
\end{EntryWithPhonetic}

\begin{EntryWithPhonetic}{天鹅}{tian1'e2}{4,12}{⼤、⿃}
  \definition{s.}{cisne}
\end{EntryWithPhonetic}

\begin{EntryWithPhonetic}{天公}{tian1gong1}{4,4}{⼤、⼋}
  \definition{s.}{céu, paraíso | senhor do céu}
\end{EntryWithPhonetic}

\begin{EntryWithPhonetic}{天花板}{tian1hua1ban3}{4,7,8}{⼤、⾋、⽊}
  \definition{s.}{teto}
\end{EntryWithPhonetic}

\begin{EntryWithPhonetic}{天空}{tian1kong1}{4,8}{⼤、⽳}[HSK 3]
  \definition{s.}{o céu; o firmamento}
\end{EntryWithPhonetic}

\begin{EntryWithPhonetic}{天气}{tian1qi4}{4,4}{⼤、⽓}[HSK 1]
  \definition{s.}{clima, tempo; mudanças meteorológicas que ocorrem na atmosfera em uma determinada área e durante um determinado período de tempo, tais como temperatura, umidade, pressão atmosférica, precipitação, vento, nuvens, etc.}
\end{EntryWithPhonetic}

\begin{EntryWithPhonetic}{天然}{tian1 ran2}{4,12}{⼤、⽕}[HSK 6]
  \definition{adj.}{natural; produzido ou ocorrido narturalmente}
\end{EntryWithPhonetic}

\begin{EntryWithPhonetic}{天然气}{tian1ran2qi4}{4,12,4}{⼤、⽕、⽓}[HSK 5]
  \definition{s.}{gás; gás natural; gás combustível produzido em campos petrolíferos, carboníferos e pântanos}
\end{EntryWithPhonetic}

\begin{EntryWithPhonetic}{天上}{tian1 shang4}{4,3}{⼤、⼀}[HSK 2]
  \definition[期]{s.}{o céu; o paraíso}
\end{EntryWithPhonetic}

\begin{EntryWithPhonetic}{天使}{tian1shi3}{4,8}{⼤、⼈}
  \definition{s.}{anjo}
\end{EntryWithPhonetic}

\begin{EntryWithPhonetic}{天堂}{tian1tang2}{4,11}{⼤、⼟}[HSK 6]
  \definition[间]{s.}{paraíso, céu; em algumas religiões, refere-se ao paraíso para onde as almas das pessoas boas retornam após a morte (diferente do 地狱) | lugar perfeito; ambiente de vida extremamente feliz e bonito; uma metáfora para um ambiente de vida feliz e bonito}
  \seealsoref{地狱}{di4yu4}
\end{EntryWithPhonetic}

\begin{EntryWithPhonetic}{天天}{tian1tian1}{4,4}{⼤、⼤}
  \definition{adv.}{todo dia}
\end{EntryWithPhonetic}

\begin{EntryWithPhonetic}{天文}{tian1wen2}{4,4}{⼤、⽂}[HSK 5]
  \definition[对]{s.}{astronomia; a distribuição e o movimento dos corpos celestes, como o sol, a lua e as estrelas, no universo}
\end{EntryWithPhonetic}

\begin{EntryWithPhonetic}{天下}{tian1 xia4}{4,3}{⼤、⼀}[HSK 6]
  \definition[期]{s.}{China ou o mundo; refere-se à China ou ao mundo | dominação; o poder dominante de um país | situação; um determinado campo; uma metáfora para uma determinada área ou situação}
\end{EntryWithPhonetic}

\begin{EntryWithPhonetic}{天择}{tian1ze2}{4,8}{⼤、⼿}
  \definition{s.}{seleção natural}
\end{EntryWithPhonetic}

\begin{EntryWithPhonetic}{天真}{tian1zhen1}{4,10}{⼤、⼗}[HSK 4]
  \definition{adj.}{ingênuo; inocente; ignorante; simples de coração, direto por natureza, livre de fingimento e hipocrisia}
\end{EntryWithPhonetic}

\begin{EntryWithPhonetic}{天柱}{tian1zhu4}{4,9}{⼤、⽊}
  \definition{s.}{pilar celestial, que sustenta o céu}
\end{EntryWithPhonetic}

\begin{EntryWithPhonetic}{兲}{tian1}{6}{⼋}
  \variantof{天}
\end{EntryWithPhonetic}

\begin{EntryWithPhonetic}{添}{tian1}{11}{⽔}[HSK 6]
  \definition{v.}{adicionar; aumentar | dar à luz}
\end{EntryWithPhonetic}

\begin{EntryWithPhonetic}{田}{tian2}{5}{⽥}[HSK 6][Kangxi 102]
  \definition*{s.}{Sobrenome Tian}
  \definition[亩,块,片]{s.}{campo; terra; terra de cultivo | área aberta rica em algum produto natural; campo}
  \definition{v.}{(arcaico) caçar}
\end{EntryWithPhonetic}

\begin{EntryWithPhonetic}{田径}{tian2jing4}{5,8}{⽥、⼻}[HSK 6]
  \definition{s.}{Esporte: atletismo}[他参加了这次的田径赛。===Ele participou da competição de atletismo.]
\end{EntryWithPhonetic}

\begin{EntryWithPhonetic}{田园}{tian2yuan2}{5,7}{⽥、⼞}
  \definition{adj.}{bucólico}
  \definition{s.}{campo | interior | rural}
\end{EntryWithPhonetic}

\begin{EntryWithPhonetic}{钿}{tian2}{10}{⾦}
  \definition{s.}{(dialeto) moeda | dinheiro; moeda | uma quantia de dinheiro}
  \seeref{钿}{dian4}
\end{EntryWithPhonetic}

\begin{EntryWithPhonetic}{甜}{tian2}{11}{⽢}[HSK 3]
  \definition{adj.}{doce; melado | agradável; confortável; fazer as pessoas se sentirem confortáveis e felizes | (sono) profundo | feliz; descreve o sentimento de felicidade}
\end{EntryWithPhonetic}

\begin{EntryWithPhonetic}{甜酒}{tian2jiu3}{11,10}{⽢、⾣}
  \definition{s.}{licor doce}
\end{EntryWithPhonetic}

\begin{EntryWithPhonetic}{甜菊}{tian2ju2}{11,11}{⽢、⾋}
  \definition{s.}{estévia, arbusto cujas folhas produzem um substituto para o açúcar}
\end{EntryWithPhonetic}

\begin{EntryWithPhonetic}{甜品}{tian2pin3}{11,9}{⽢、⼝}
  \definition{s.}{sobremesa}
\end{EntryWithPhonetic}

\begin{EntryWithPhonetic}{甜食}{tian2shi2}{11,9}{⽢、⾷}
  \definition{s.}{doces | sobremesa}
\end{EntryWithPhonetic}

\begin{EntryWithPhonetic}{甜酸}{tian2suan1}{11,14}{⽢、⾣}
  \definition{adj.}{agridoce}
\end{EntryWithPhonetic}

\begin{EntryWithPhonetic}{甜甜圈}{tian2tian2quan1}{11,11,11}{⽢、⽢、⼞}
  \definition{s.}{rosquinha | \emph{doughnut}}
\end{EntryWithPhonetic}

\begin{EntryWithPhonetic}{甜筒}{tian2tong3}{11,12}{⽢、⽵}
  \definition{s.}{sorvete de casquinha}
\end{EntryWithPhonetic}

\begin{EntryWithPhonetic}{甜头}{tian2tou5}{11,5}{⽢、⼤}
  \definition{s.}{benefício | sabor doce (de poder, sucesso, etc.)}
\end{EntryWithPhonetic}

\begin{EntryWithPhonetic}{甜心}{tian2xin1}{11,4}{⽢、⼼}
  \definition{s.}{querido}
\end{EntryWithPhonetic}

\begin{EntryWithPhonetic}{甜言}{tian2yan2}{11,7}{⽢、⾔}
  \definition{s.}{boa conversa | palavras amáveis}
\end{EntryWithPhonetic}

\begin{EntryWithPhonetic}{甜玉米}{tian2 yu4mi3}{11,5,6}{⽢、⽟、⽶}
  \definition{s.}{milho doce}
\end{EntryWithPhonetic}

\begin{EntryWithPhonetic}{甜稚}{tian2zhi4}{11,13}{⽢、⽲}
  \definition{s.}{doce e inocente}
\end{EntryWithPhonetic}

\begin{EntryWithPhonetic}{填}{tian2}{13}{⼟}
  \definition{v.}{encher; rechear | reabastecer; suplementar; complementar | preencher; escrever dados em uma caixa (em um questionário ou formulário da \emph{Web})}
\end{EntryWithPhonetic}

\begin{EntryWithPhonetic}{填空}{tian2kong4}{13,8}{⼟、⽳}[HSK 4]
  \definition{v.}{preencher o espaço em branco (por exemplo, em um teste)}
\end{EntryWithPhonetic}

\begin{EntryWithPhonetic}{挑}{tiao1}{9}{⼿}[HSK 4]
  \definition{clas.}{usado para coisas que são escolhidas ou selecionadas | usado para coisas que podem ser usadas como palhetas}
  \definition{s.}{vara comprida com algo pendurado nas pontas; haste de transporte}
  \definition{v.}{escolher; selecionar | fazer picuinhas; ser hipercrítico; ser meticuloso; ser excessivamente rigoroso nos detalhes | carregar com uma haste de transporte; carregar no ombro; pendurar coisas nas pontas de varas longas e carregá-las em seus ombros}
  \seeref{挑}{tiao3}
\end{EntryWithPhonetic}

\begin{EntryWithPhonetic}{挑选}{tiao1 xuan3}{9,9}{⼿、⾡}[HSK 4]
  \definition{v.}{escolher; optar; selecionar; escolher a pessoa ou coisa certa para o trabalho}
\end{EntryWithPhonetic}

\begin{EntryWithPhonetic}{条}{tiao2}{7}{⽊}[HSK 2]
  \definition*{s.}{Sobrenome Tiao}
  \definition{clas.}{usado para objetos longos e finos; usado para sintetizar certas coisas longas e retangulares em quantidades fixas | usado para itemização | aplicado ao corpo humano}
  \definition{s.}{galho; galhos finos e longos | tira; faixa | item; artigo | ordem; método | nota; anotação em papel}
\end{EntryWithPhonetic}

\begin{EntryWithPhonetic}{条幅}{tiao2fu2}{7,12}{⽊、⼱}
  \definition{s.}{faixa | banner | pergaminho de parede (para pintura ou caligrafia)}
\end{EntryWithPhonetic}

\begin{EntryWithPhonetic}{条贯}{tiao2guan4}{7,8}{⽊、⾙}
  \definition{s.}{ordem | procedimentos | sequência | sistema}
\end{EntryWithPhonetic}

\begin{EntryWithPhonetic}{条件}{tiao2jian4}{7,6}{⽊、⼈}[HSK 2]
  \definition[个,项,些]{s.}{condição; termo; fator; fatores que restringem a ocorrência, existência ou desenvolvimento das coisas | requisito; pré-requisito; qualificação; requisitos ou padrões estabelecidos para determinadas coisas | situação; estado; condição}
\end{EntryWithPhonetic}

\begin{EntryWithPhonetic}{条例}{tiao2li4}{7,8}{⽊、⼈}
  \definition{s.}{código de conduta | ordenanças | regulamentos | regras | estatutos}
\end{EntryWithPhonetic}

\begin{EntryWithPhonetic}{条目}{tiao2mu4}{7,5}{⽊、⽬}
  \definition{s.}{cláusulas e subcláusulas (em documento formal) | verbete (em um dicionário, enciclopédia, etc.)}
\end{EntryWithPhonetic}

\begin{EntryWithPhonetic}{调}{tiao2}{10}{⾔}[HSK 3]
  \definition{adj.}{harmonioso; boa coordenação}
  \definition{v.}{misturar; ajustar; fazer o ajuste uniforme e apropriado | provocar; importunar; fazer pouco de | incitar; instigar; provocar; semear discórdia | mediar; trazer harmonia}
  \seeref{调}{diao4}
\end{EntryWithPhonetic}

\begin{EntryWithPhonetic}{调节}{tiao2jie2}{10,5}{⾔、⾋}[HSK 5]
  \definition{v.}{regular; ajustar; ajustar e controlar de várias maneiras para atender aos requisitos}
\end{EntryWithPhonetic}

\begin{EntryWithPhonetic}{调解}{tiao2jie3}{10,13}{⾔、⾓}[HSK 5]
  \definition{v.}{mediar; fazer as pazes; resolver conflitos através da persuasão}
\end{EntryWithPhonetic}

\begin{EntryWithPhonetic}{调律}{tiao2lv4}{10,9}{⾔、⼻}
  \definition{v.}{afinar (por exemplo, um piano)}
\end{EntryWithPhonetic}

\begin{EntryWithPhonetic}{调皮}{tiao2pi2}{10,5}{⾔、⽪}[HSK 4]
  \definition{adj.}{travesso; malicioso; malandro | indisciplinado; desordeiro; indomável; astuto | inteligente e desonesto}
\end{EntryWithPhonetic}

\begin{EntryWithPhonetic}{调整}{tiao2zheng3}{10,16}{⾔、⽁}[HSK 3]
  \definition{v.}{ajustar; revisar; regularizar; fazer as alterações apropriadas no estado original para se adaptar à nova situação}
\end{EntryWithPhonetic}

\begin{EntryWithPhonetic}{挑}{tiao3}{9}{⼿}[HSK 4]
  \definition{s.}{um dos traços dos caracteres chineses; inclinado para cima da esquerda para a direita}
  \definition{v.}{levantar; elevar; erguer | levantar ou apoiar com uma extremidade de uma vara ou objeto semelhante; segurar ou apoiar com a ponta de uma vara etc. | causar conflitos deliberadamente; provocar deliberadamente um conflito | (método de bordado) usar uma agulha para levantar os fios de urdidura ou trama, com a agulha e a linha passando por baixo para formar padrões e desenhos}
  \seeref{挑}{tiao1}
\end{EntryWithPhonetic}

\begin{EntryWithPhonetic}{挑衅}{tiao3xin4}{9,11}{⼿、⾎}
  \definition{s.}{provocação}
  \definition{v.}{provocar; causar problemas; tentar causar conflito ou guerra}
\end{EntryWithPhonetic}

\begin{EntryWithPhonetic}{挑战}{tiao3zhan4}{9,9}{⼿、⼽}[HSK 4]
  \definition{v.}{desafiar; deixar um oponente deliberadamente irritado e sair para lutar ou lutar consigo mesmo; estimular um oponente a lutar consigo mesmo}
\end{EntryWithPhonetic}

\begin{EntryWithPhonetic}{跳}{tiao4}{13}{⾜}[HSK 3]
  \definition{v.}{pular; saltar | mover para cima e para baixo | pular (por cima); fazer omissões | quicar; a força elástica faz com que o objeto se mova repentinamente para cima | pulsar; palpitar; contrair-se | pular sobre;  saltar sobre; cruzar}
\end{EntryWithPhonetic}

\begin{EntryWithPhonetic}{跳挡}{tiao4dang3}{13,9}{⾜、⼿}
  \definition{v.}{pular marcha (de um carro) | perder a marcha}
\end{EntryWithPhonetic}

\begin{EntryWithPhonetic}{跳电}{tiao4dian4}{13,5}{⾜、⽥}
  \definition{v.}{desarmar (um disjuntor ou interruptor)}
\end{EntryWithPhonetic}

\begin{EntryWithPhonetic}{跳高}{tiao4 gao1}{13,10}{⾜、⾼}[HSK 3]
  \definition{s.}{salto em altura (atletismo)}
  \definition{v.}{saltar em altura}
\end{EntryWithPhonetic}

\begin{EntryWithPhonetic}{跳频}{tiao4pin2}{13,13}{⾜、⾴}
  \definition{s.}{FHSS, \emph{Frequency-Hopping Spread Spectrum}, método de transmissão de sinais de rádio}
\end{EntryWithPhonetic}

\begin{EntryWithPhonetic}{跳伞}{tiao4san3}{13,6}{⾜、⼈}
  \definition{s.}{paraquedas}
  \definition{v.}{saltar de paraquedas}
\end{EntryWithPhonetic}

\begin{EntryWithPhonetic}{跳绳}{tiao4sheng2}{13,11}{⾜、⽷}
  \definition{v.}{pular corda}
\end{EntryWithPhonetic}

\begin{EntryWithPhonetic}{跳水}{tiao4 shui3}{13,4}{⾜、⽔}[HSK 6]
  \definition{s.}{Esporte: mergulho}
  \definition{v.}{mergulhar | Figurativo: (preços, lucros, etc.) cair drasticamente; cair repentinamente; mergulhar; despencar | cometer suicídio pulando na água | mergulhar (na água)}
\end{EntryWithPhonetic}

\begin{EntryWithPhonetic}{跳跳糖}{tiao4tiao4tang2}{13,13,16}{⾜、⾜、⽶}
  \definition{s.}{\emph{Pop Rocks}, \emph{popping candy}}
\end{EntryWithPhonetic}

\begin{EntryWithPhonetic}{跳舞}{tiao4wu3}{13,14}{⾜、⾇}[HSK 3]
  \definition{v.+compl.}{dançar (como performance); executar dança, especialmente dança de salão}
\end{EntryWithPhonetic}

\begin{EntryWithPhonetic}{跳远}{tiao4 yuan3}{13,7}{⾜、⾡}[HSK 3]
  \definition{s.}{salto em distância (atletismo)}
\end{EntryWithPhonetic}

\begin{EntryWithPhonetic}{跳蚤}{tiao4zao5}{13,9}{⾜、⾍}
  \definition{s.}{pulga}
\end{EntryWithPhonetic}

\begin{EntryWithPhonetic}{贴}{tie1}{9}{⾙}[HSK 4]
  \definition{adj.}{submisso; obediente | apropriado}
  \definition{clas.}{usado em gessos, emplastros}
  \definition{s.}{subsídio; subvenção}
  \definition{v.}{grudar; colar | aninhar-se a; aconchegar-se a; aconchegar-se em | subsidiar; ajudar financeiramente}
\end{EntryWithPhonetic}

\begin{EntryWithPhonetic}{铁}{tie3}{10}{⾦}[HSK 3]
  \definition*{s.}{Sobrenome Tie}
  \definition{adj.}{duro; forte; sólido como ferro; metáfora para natureza dura; vontade forte | violento | inabalável; inalterável; determinado; metáfora para violência ou crueldade}
  \definition{s.}{ferro (Fe) | arma; armamento; refere-se a facas, armas de fogo, etc.}
  \definition{v.}{resolver; determinar}
\end{EntryWithPhonetic}

\begin{EntryWithPhonetic}{铁轨}{tie3gui3}{10,6}{⾦、⾞}
  \definition[根]{s.}{trilho | trilho ferroviário}
\end{EntryWithPhonetic}

\begin{EntryWithPhonetic}{铁路}{tie3 lu4}{10,13}{⾦、⾜}[HSK 3]
  \definition[条,公里]{s.}{ferrovia; estrada de ferro; uma estrada com trilhos de aço dispostos no leito da estrada para a circulação de trens}
\end{EntryWithPhonetic}

\begin{EntryWithPhonetic}{厅}{ting1}{4}{⼚}[HSK 5]
  \definition{s.}{salão; sala grande para reuniões ou receber convidados | escritório; nome de um departamento administrativo de uma grande organização | departamento governamental a nível provincial; nomes de alguns órgãos estaduais}
\end{EntryWithPhonetic}

\begin{EntryWithPhonetic}{听}{ting1}{7}{⼝}[HSK 1]
  \definition{clas.}{latas; usado para bebidas e alimentos para levar consigo}
  \definition{s.}{lata; embalagem metálica; recipiente cilíndrico utilizado para armazenar bebidas, alimentos, etc.}
  \definition{v.}{ouvir; escutar | obedecer; dar ouvidos; aceitar | supervisionar; administrar; tratar (assuntos políticos); julgar (casos) | permitir; deixar ser; deixar fazer}
  \seeref{听}{yin3}
\end{EntryWithPhonetic}

\begin{EntryWithPhonetic}{听到}{ting1 dao4}{7,8}{⼝、⼑}[HSK 1]
  \definition{v.}{ouvir, escutar; ouvir atentamente, escutar atentamente}
\end{EntryWithPhonetic}

\begin{EntryWithPhonetic}{听断}{ting1duan4}{7,11}{⼝、⽄}
  \definition{v.}{ouvir e decidir | julgar (ou seja, ouvir e julgar em um tribunal)}
\end{EntryWithPhonetic}

\begin{EntryWithPhonetic}{听骨}{ting1gu3}{7,9}{⼝、⾻}
  \definition{s.}{ossículos (do ouvido médio)}
  \seealsoref{听小骨}{ting1xiao3gu3}
\end{EntryWithPhonetic}

\begin{EntryWithPhonetic}{听会}{ting1hui4}{7,6}{⼝、⼈}
  \definition{v.}{participar de uma reunião (e ouvir o que é discutido)}
\end{EntryWithPhonetic}

\begin{EntryWithPhonetic}{听见}{ting1 jian4}{7,4}{⼝、⾒}[HSK 1]
  \definition{v.}{ouvir; conseguir ouvir}
\end{EntryWithPhonetic}

\begin{EntryWithPhonetic}{听讲}{ting1 jiang3}{7,6}{⼝、⾔}[HSK 2]
  \definition{v.+compl.}{assistir a uma palestra; ouvir palestras ou discursos}
\end{EntryWithPhonetic}

\begin{EntryWithPhonetic}{听来}{ting1lai2}{7,7}{⼝、⽊}
  \definition{v.}{ouvir de algum lugar | soar (antigo, estrangeiro, excitante, certo, etc.) | soar como se (ou seja, dar uma impressão ao ouvinte)}
\end{EntryWithPhonetic}

\begin{EntryWithPhonetic}{听力}{ting1li4}{7,2}{⼝、⼒}[HSK 3]
  \definition{s.}{audição; capacidade auditiva | compreensão auditiva (na aprendizagem de línguas)}
\end{EntryWithPhonetic}

\begin{EntryWithPhonetic}{听力理解}{ting1li4li3jie3}{7,2,11,13}{⼝、⼒、⽟、⾓}
  \definition{s.}{compreensão auditiva}
\end{EntryWithPhonetic}

\begin{EntryWithPhonetic}{听命}{ting1ming4}{7,8}{⼝、⼝}
  \definition{v.}{obedecer ordens | receber ordens}
\end{EntryWithPhonetic}

\begin{EntryWithPhonetic}{听凭}{ting1ping2}{7,8}{⼝、⼏}
  \definition{v.}{permitir (alguém a fazer o que desejar)}
\end{EntryWithPhonetic}

\begin{EntryWithPhonetic}{听取}{ting1 qu3}{7,8}{⼝、⼜}[HSK 6]
  \definition{v.}{ouvir (opiniões, reflexões, relatórios, etc.)}
\end{EntryWithPhonetic}

\begin{EntryWithPhonetic}{听说}{ting1 shuo1}{7,9}{⼝、⾔}[HSK 2]
  \definition{v.}{ser informado; ouvir falar de; ouvir dizer | ouvir e falar}
\end{EntryWithPhonetic}

\begin{EntryWithPhonetic}{听随}{ting1sui2}{7,11}{⼝、⾩}
  \definition{v.}{permitir | obedecer}
\end{EntryWithPhonetic}

\begin{EntryWithPhonetic}{听戏}{ting1xi4}{7,6}{⼝、⼽}
  \definition{v.}{assistir a uma ópera | ver uma ópera}
\end{EntryWithPhonetic}

\begin{EntryWithPhonetic}{听小骨}{ting1xiao3gu3}{7,3,9}{⼝、⼩、⾻}
  \definition{s.}{ossículos (do ouvido médio)}
  \seealsoref{听骨}{ting1gu3}
\end{EntryWithPhonetic}

\begin{EntryWithPhonetic}{听写}{ting1 xie3}{7,5}{⼝、⼍}[HSK 1]
  \definition{s.}{ditado}
  \definition{v.}{ouvir e escrever}
\end{EntryWithPhonetic}

\begin{EntryWithPhonetic}{听众}{ting1 zhong4}{7,6}{⼝、⼈}[HSK 3]
  \definition[位,名,个]{s.}{audiência; ouvintes; pessoas que ouvem palestras, música ou transmissões}
\end{EntryWithPhonetic}

\begin{EntryWithPhonetic}{聼}{ting1}{19}{⼼}
  \variantof{听}
\end{EntryWithPhonetic}

\begin{EntryWithPhonetic}{亭}{ting2}{9}{⼇}
  \definition{s.}{pavilhão | cabine | quiosque}
\end{EntryWithPhonetic}

\begin{EntryWithPhonetic}{停}{ting2}{11}{⼈}[HSK 2]
  \definition{adj.}{pronto; resolvido; bem organizado}
  \definition{clas.}{usado para partes (de um total); porções}
  \definition{v.}{parar; interromper; cessar; fazer uma pausa | permanecer; ficar; fazer uma parada (para descansar) | estacionar; ancorar; atracar}
\end{EntryWithPhonetic}

\begin{EntryWithPhonetic}{停办}{ting2ban4}{11,4}{⼈、⼒}
  \definition{v.}{cancelar | sair do negócio | desligar | terminar}
\end{EntryWithPhonetic}

\begin{EntryWithPhonetic}{停车}{ting2 che1}{11,4}{⼈、⾞}[HSK 2]
  \definition{v.}{(veículo) parar; frear | estacionar o veículo | parar; deixar de funcionar}
\end{EntryWithPhonetic}

\begin{EntryWithPhonetic}{停车场}{ting2 che1 chang3}{11,4,6}{⼈、⾞、⼟}[HSK 2]
  \definition[个]{s.}{estacionamento; área de estacionamento; local para estacionamento de veículos}
\end{EntryWithPhonetic}

\begin{EntryWithPhonetic}{停当}{ting2dang5}{11,6}{⼈、⼹}
  \definition{adj.}{realizado | preparado | assentado}
\end{EntryWithPhonetic}

\begin{EntryWithPhonetic}{停电}{ting2dian4}{11,5}{⼈、⽥}
  \definition{s.}{corte de energia}
  \definition{v.}{ter uma falha de energia}
\end{EntryWithPhonetic}

\begin{EntryWithPhonetic}{停工}{ting2gong1}{11,3}{⼈、⼯}
  \definition{v.}{parar de trabalhar | parar a produção}
\end{EntryWithPhonetic}

\begin{EntryWithPhonetic}{停火}{ting2huo3}{11,4}{⼈、⽕}
  \definition{s.}{cessar-fogo}
  \definition{v.+compl.}{cessar fogo}
\end{EntryWithPhonetic}

\begin{EntryWithPhonetic}{停课}{ting2ke4}{11,10}{⼈、⾔}
  \definition{v.}{fechar (escola) | parar as aulas}
\end{EntryWithPhonetic}

\begin{EntryWithPhonetic}{停留}{ting2 liu2}{11,10}{⼈、⽥}[HSK 5]
  \definition{v.}{permanecer; ficar por muito tempo; parar temporariamente em algum lugar, sem continuar avançando | permanecer; parar por um longo tempo; parar em um determinado estágio ou nível, sem evoluir}
\end{EntryWithPhonetic}

\begin{EntryWithPhonetic}{停息}{ting2xi1}{11,10}{⼈、⼼}
  \definition{v.}{cessar | parar}
\end{EntryWithPhonetic}

\begin{EntryWithPhonetic}{停下}{ting2 xia4}{11,3}{⼈、⼀}[HSK 4]
  \definition{v.}{encerrar; desligar; parar}
\end{EntryWithPhonetic}

\begin{EntryWithPhonetic}{停歇}{ting2xie1}{11,13}{⼈、⽋}
  \definition{v.}{parar para descansar}
\end{EntryWithPhonetic}

\begin{EntryWithPhonetic}{停业}{ting2ye4}{11,5}{⼈、⼀}
  \definition{v.}{cessar a negociação (temporária ou permanentemente) | fechar}
\end{EntryWithPhonetic}

\begin{EntryWithPhonetic}{停用}{ting2yong4}{11,5}{⼈、⽤}
  \definition{v.}{desabilitar | descontinuar | parar de usar | suspender}
\end{EntryWithPhonetic}

\begin{EntryWithPhonetic}{停止}{ting2 zhi3}{11,4}{⼈、⽌}[HSK 3]
  \definition{v.}{parar; suspender; cessar; cancelar}
\end{EntryWithPhonetic}

\begin{EntryWithPhonetic}{挺}{ting3}{9}{⼿}[HSK 2,4]
  \definition{adj.}{rígido; ereto; vertical; reto | notável; destacado; distinto}
  \definition{adv.}{muito; bastante}
  \definition{clas.}{usado para metralhadoras}
  \definition{v.}{sobressair; endireitar-se; protrudir (protuberância ou saliência) | suportar; aguentar; resistir; perseverar}
\end{EntryWithPhonetic}

\begin{EntryWithPhonetic}{挺拔}{ting3ba2}{9,8}{⼿、⼿}
  \definition{adj.}{alto e reto}
\end{EntryWithPhonetic}

\begin{EntryWithPhonetic}{挺杆}{ting3gan3}{9,7}{⼿、⽊}
  \definition{s.}{tucho (peça de máquina)}
\end{EntryWithPhonetic}

\begin{EntryWithPhonetic}{挺过}{ting3guo4}{9,6}{⼿、⾡}
  \definition{s.}{sobreviver}
\end{EntryWithPhonetic}

\begin{EntryWithPhonetic}{挺好}{ting3 hao3}{9,6}{⼿、⼥}[HSK 2]
  \definition{adj.}{nada mal; surpreendentemente bom}
\end{EntryWithPhonetic}

\begin{EntryWithPhonetic}{挺进}{ting3jin4}{9,7}{⼿、⾡}
  \definition{s.}{progresso | avanço}
  \definition{v.}{progredir | avançar}
\end{EntryWithPhonetic}

\begin{EntryWithPhonetic}{挺立}{ting3li4}{9,5}{⼿、⽴}
  \definition{v.}{ficar ereto | ficar de pé}
\end{EntryWithPhonetic}

\begin{EntryWithPhonetic}{挺身}{ting3shen1}{9,7}{⼿、⾝}
  \definition{v.}{endireitar as costas}
\end{EntryWithPhonetic}

\begin{EntryWithPhonetic}{挺尸}{ting3shi1}{9,3}{⼿、⼫}
  \definition{v.}{(coloquial) dormir | (literalmente) ficar deitado duro como um cadáver}
\end{EntryWithPhonetic}

\begin{EntryWithPhonetic}{挺腰}{ting3yao1}{9,13}{⼿、⾁}
  \definition{v.}{arquear as costas | endireitar as costas}
\end{EntryWithPhonetic}

\begin{EntryWithPhonetic}{挺住}{ting3zhu4}{9,7}{⼿、⼈}
  \definition{v.}{permanecer firme | manter-se firme (diante da adversidade ou da dor)}
\end{EntryWithPhonetic}

\begin{EntryWithPhonetic}{通}{tong1}{10}{⾡}[HSK 2]
  \definition*{s.}{Sobrenome Tong}
  \definition{adj.}{lógico; coerente | geral; comum | tudo; inteiro | aberto; através de | total}
  \definition{clas.}{(antigo) usado para cartas, telegramas, documentos oficiais, etc.}
  \definition{s.}{autoridade; especialista}
  \definition{suf.}{especialista}
  \definition{v.}{abrir; atravessar | abrir ou limpar cutucando ou espetando | levar a; ir a | conectar; comunicar | notificar; informar | compreender; saber | cutucar; dar uma pancada | transmitir; conectar; interagir | dominar; compreender; entender}
  \seeref{通}{tong4}
\end{EntryWithPhonetic}

\begin{EntryWithPhonetic}{通报}{tong1 bao4}{10,7}{⾡、⼿}[HSK 6]
  \definition[份]{s.}{circular | boletim; jornal; publicação | sumário; notificação para informações gerais}
  \definition{v.}{circular um aviso (aviso por escrito) | notificar; dar informações com; compartilhar informações com}
\end{EntryWithPhonetic}

\begin{EntryWithPhonetic}{通常}{tong1chang2}{10,11}{⾡、⼱}[HSK 3]
  \definition{adj.}{usual; normal; geral}
  \definition{adv.}{habitualmente; usualmente; geralmente; ordinariamente}
\end{EntryWithPhonetic}

\begin{EntryWithPhonetic}{通道}{tong1 dao4}{10,12}{⾡、⾡}[HSK 6]
  \definition[条,个]{s.}{acesso; corredor; passagem; caminhos que levam ao exterior de teatros, minas, etc. | passagem; via pública}
\end{EntryWithPhonetic}

\begin{EntryWithPhonetic}{通牒}{tong1die2}{10,13}{⾡、⽚}
  \definition{s.}{nota diplomática}
\end{EntryWithPhonetic}

\begin{EntryWithPhonetic}{通观}{tong1guan1}{10,6}{⾡、⾒}
  \definition{v.}{ter uma visão geral de algo}
\end{EntryWithPhonetic}

\begin{EntryWithPhonetic}{通过}{tong1guo4}{10,6}{⾡、⾡}[HSK 2]
  \definition{prep.}{por; através de; por meio de; por meio de; meios, métodos, etc. para introduzir ações}
  \definition{v.}{atravessar; passar por; transitar | aprovar; adotar | solicitar o consentimento ou aprovação de}
\end{EntryWithPhonetic}

\begin{EntryWithPhonetic}{通红}{tong1 hong2}{10,6}{⾡、⽷}[HSK 6]
  \definition{adj.}{muito vermelho; vermelho por completo}
\end{EntryWithPhonetic}

\begin{EntryWithPhonetic}{通话}{tong1 hua4}{10,8}{⾡、⾔}[HSK 6]
  \definition{v.}{comunicar por telefone | conversar; comunicar; falar em uma língua que ambos possam entender}
\end{EntryWithPhonetic}

\begin{EntryWithPhonetic}{通识}{tong1shi2}{10,7}{⾡、⾔}
  \definition{s.}{conhecimento comum | erudição | conhecimento geral | amplamente conhecido}
\end{EntryWithPhonetic}

\begin{EntryWithPhonetic}{通信}{tong1 xin4}{10,9}{⾡、⼈}[HSK 3]
  \definition{v.+compl.}{corresponder; comunicar por carta; comunicar situações e informações escrevendo cartas | transmitir (ou transportar) mensagem; passar (ou transmitir) informação; usar ondas de rádio e outros sinais para transmitir texto, imagens, etc.}
\end{EntryWithPhonetic}

\begin{EntryWithPhonetic}{通行}{tong1 xing2}{10,6}{⾡、⾏}[HSK 6]
  \definition{adj.}{atual; geral}
  \definition{v.}{passar (ou ir) através; passar por; atravessar | prevalecer; predominar; ser corrente | (pedestres, veículos, etc.) passar na linha de trânsito}
\end{EntryWithPhonetic}

\begin{EntryWithPhonetic}{通讯}{tong1xun4}{10,5}{⾡、⾔}[HSK 6]
  \definition[个,种]{s.}{relatório; comunicação; boletim informativo; correspondência; reportagem; despacho de notícias; artigos que relatam fatos objetivos ou números típicos de forma detalhada e vívida}
  \definition{v.}{usar equipamentos de telecomunicações para transmitir mensagens}
\end{EntryWithPhonetic}

\begin{EntryWithPhonetic}{通用}{tong1yong4}{10,5}{⾡、⽤}[HSK 5]
  \definition[家]{adj.}{de uso comum; universal; (em um determinado âmbito) de uso generalizado | intercambiável; alguns caracteres chineses com grafia diferente, mas pronúncia igual, podem ser usados indistintamente (alguns limitados a um determinado significado)}
\end{EntryWithPhonetic}

\begin{EntryWithPhonetic}{通知}{tong1zhi1}{10,8}{⾡、⽮}[HSK 2]
  \definition[份,个,张]{s.}{aviso; circular; notificação por escrito ou verbal}
  \definition{v.}{aconselhar; notificar; informar; dar aviso prévio}
\end{EntryWithPhonetic}

\begin{EntryWithPhonetic}{通知书}{tong1 zhi1 shu1}{10,8,4}{⾡、⽮、⼄}[HSK 4]
  \definition[份]{s.}{aviso; observação; notificação}
\end{EntryWithPhonetic}

\begin{EntryWithPhonetic}{同}{tong2}{6}{⼝}[HSK 6]
  \definition{adj.}{como; igual; parecido; similar; o mesmo; sem diferença}
  \definition{adv.}{juntos; em comum; indica que diferentes atores realizam uma determinada ação juntos ou estão na mesma situação, o que equivale a 一同 ou 一起}
  \definition{v.}{ser o mesmo que}
  \seeref{同}{tong4}
  \seealsoref{一起}{yi4qi3}
  \seealsoref{一同}{yi4tong2}
\end{EntryWithPhonetic}

\begin{EntryWithPhonetic}{同胞}{tong2bao1}{6,9}{⼝、⾁}[HSK 6]
  \definition{s.}{nascidos dos mesmos pais | compatriota; conterrâneo; pessoas do mesmo país ou etnia}
\end{EntryWithPhonetic}

\begin{EntryWithPhonetic}{同行}{tong2 hang2}{6,6}{⼝、⾏}[HSK 6]
  \definition{s.}{do mesmo ofício ou ocupação; pessoas no mesmo setor}
  \definition{v.}{ser do mesmo ofício ou ocupação; trabalhar no mesmo setor}
\end{EntryWithPhonetic}

\begin{EntryWithPhonetic}{同伙}{tong2huo3}{6,6}{⼝、⼈}
  \definition[个]{s.}{cúmplice | colega}
\end{EntryWithPhonetic}

\begin{EntryWithPhonetic}{同流合污}{tong2liu2he2wu1}{6,10,6,6}{⼝、⽔、⼝、⽔}
  \definition{expr.}{chafurdar na lama com alguém | seguir o mau exemplo dos outros}
\end{EntryWithPhonetic}

\begin{EntryWithPhonetic}{同期}{tong2 qi1}{6,12}{⼝、⽉}[HSK 6]
  \definition{s.}{o período correspondente; o mesmo período; no mesmo tempo}
\end{EntryWithPhonetic}

\begin{EntryWithPhonetic}{同情}{tong2qing2}{6,11}{⼝、⼼}[HSK 4]
  \definition{s.}{simpatia}
  \definition{v.}{simpatizar com; solidarizar-se; compadecer-se; ter empatia emocional pelo que os outros estão passando}
\end{EntryWithPhonetic}

\begin{EntryWithPhonetic}{同时}{tong2shi2}{6,7}{⼝、⽇}[HSK 2]
  \definition{conj.}{além disso; além do mais; ainda mais; indica uma relação de equivalência, geralmente com um significado mais profundo}
  \definition{s.}{enquanto isso; ao mesmo tempo}
\end{EntryWithPhonetic}

\begin{EntryWithPhonetic}{同事}{tong2shi4}{6,8}{⼝、⼅}[HSK 2]
  \definition[个,位,名]{s.}{companheiro; colega; colega de trabalho; pessoas que trabalham juntas}
  \definition{v.}{trabalhar no mesmo lugar; trabalhar juntos; trabalhar na mesma unidade}
\end{EntryWithPhonetic}

\begin{EntryWithPhonetic}{同屋}{tong2wu1}{6,9}{⼝、⼫}
  \definition[个]{s.}{companheiro de quarto | colega de quarto}
\end{EntryWithPhonetic}

\begin{EntryWithPhonetic}{同性恋}{tong2xing4lian4}{6,8,10}{⼝、⼼、⼼}
  \definition{s.}{homossexualidade | pessoa gay | amor gay}
\end{EntryWithPhonetic}

\begin{EntryWithPhonetic}{同学}{tong2xue2}{6,8}{⼝、⼦}[HSK 1]
  \definition[位,个,些]{s.}{colega de escola; colega de turma; colega de estudos; pessoas que estudam na mesma escola}
\end{EntryWithPhonetic}

\begin{EntryWithPhonetic}{同砚}{tong2yan4}{6,9}{⼝、⽯}
  \definition[位,个]{s.}{colega de classe | colega estudante}
\end{EntryWithPhonetic}

\begin{EntryWithPhonetic}{同样}{tong2 yang4}{6,10}{⼝、⽊}[HSK 2]
  \definition{adj.}{igual; semelhante; similar; idêntico; sem diferença}
\end{EntryWithPhonetic}

\begin{EntryWithPhonetic}{同一}{tong2 yi1}{6,1}{⼝、⼀}[HSK 6]
  \definition{adj.}{mesmo; idêntico}
  \definition[讲]{s.}{identidade; unidade}
\end{EntryWithPhonetic}

\begin{EntryWithPhonetic}{同意}{tong2yi4}{6,13}{⼝、⼼}[HSK 3]
  \definition{v.}{concordar; consentir; aprovar; concordar com; dizer sim}
\end{EntryWithPhonetic}

\begin{EntryWithPhonetic}{铜}{tong2}{11}{⾦}
  \definition[块]{s.}{cobre (Cu)}
\end{EntryWithPhonetic}

\begin{EntryWithPhonetic}{铜牌}{tong2 pai2}{11,12}{⾦、⽚}[HSK 6]
  \definition[枚]{s.}{medalha de bronze; o bronze | placa de bronze com nome ou logotipo comercial, etc.}
\end{EntryWithPhonetic}

\begin{EntryWithPhonetic}{童}{tong2}{12}{⽴}
  \definition*{s.}{Sobrenome Tong}
  \definition{adj.}{virgem; solteira | nu; careca | árido; estéril}
  \definition{s.}{criança | jovem servo; antigamente, referia-se a um servo menor de idade.}
\end{EntryWithPhonetic}

\begin{EntryWithPhonetic}{童话}{tong2hua4}{12,8}{⽴、⾔}[HSK 4]
  \definition[个,部]{s.}{conto de fadas; gênero de literatura infantil no qual as histórias adequadas para a diversão das crianças são escritas com muita imaginação, fantasia e exagero}
\end{EntryWithPhonetic}

\begin{EntryWithPhonetic}{童年}{tong2 nian2}{12,6}{⽴、⼲}[HSK 4]
  \definition[对]{s.}{infância; primeiros anos de vida}
\end{EntryWithPhonetic}

\begin{EntryWithPhonetic}{僮}{tong2}{14}{⼈}
  \definition*{s.}{Sobrenome Tong}
  \seeref{僮}{zhuang4}
\end{EntryWithPhonetic}

\begin{EntryWithPhonetic}{獞}{tong2}{15}{⽝}
  \definition{s.}{nome de uma variedade de cão | tribos selvagens no sul da China}
  \seeref{獞}{zhuang4}
\end{EntryWithPhonetic}

\begin{EntryWithPhonetic}{统}{tong3}{9}{⽷}
  \definition{adv.}{todos; juntos; de forma unificada | inteiramente; totalmente}
  \definition{s.}{interligado; inter-relacionado | sistema interconectado | qualquer parte em forma de tubo de uma peça de roupa, etc.; o mesmo que 筒}
  \definition{v.}{reunir em um; unir | unir; liderar; comandar}
  \seealsoref{筒}{tong3}
\end{EntryWithPhonetic}

\begin{EntryWithPhonetic}{统计}{tong3ji4}{9,4}{⽷、⾔}[HSK 4]
  \definition{v.}{compilar estatísticas; refere-se à realização de trabalho estatístico, ou seja, coletar, reunir, analisar e extrapolar dados sobre um fenômeno | somar; adicionar; contar}
\end{EntryWithPhonetic}

\begin{EntryWithPhonetic}{统一}{tong3yi1}{9,1}{⽷、⼀}[HSK 4]
  \definition{adj.}{unificado; unitário; centralizado; consistente}
  \definition{v.}{unificar; unir; integrar; padronizar}
\end{EntryWithPhonetic}

\begin{EntryWithPhonetic}{筒}{tong3}{12}{⽵}
  \definition[个]{s.}{seção de bambu grosso; tubo grosso de bambu | objeto em forma de tubo largo | a parte em forma de tubo das roupas etc.}
\end{EntryWithPhonetic}

\begin{EntryWithPhonetic}{同}{tong4}{6}{⼝}
  \definition[条,处]{s.}{beco; rua estreita}
  \seealsoref{胡同}{hu2tong5}
\end{EntryWithPhonetic}

\begin{EntryWithPhonetic}{通}{tong4}{10}{⾡}
  \definition{clas.}{usado para uma atividade, tomada em sua totalidade (discurso de abuso, período de reprodução de música, bebedeira, etc.)}
  \seeref{通}{tong1}
\end{EntryWithPhonetic}

\begin{EntryWithPhonetic}{痛}{tong4}{12}{⽧}[HSK 3]
  \definition{adv.}{extremamente; profundamente; amargamente}
  \definition{s.}{dor; sofrimento | tristeza; pesar}
\end{EntryWithPhonetic}

\begin{EntryWithPhonetic}{痛苦}{tong4ku3}{12,8}{⽧、⾋}[HSK 3]
  \definition{adj.}{doloroso; angustiado; sentindo-se muito desconfortável física ou mentalmente}
  \definition[降,种]{s.}{dor; agonia; sofrimento; refere-se a um estado ou sentimento de extremo desconforto físico ou mental}
\end{EntryWithPhonetic}

\begin{EntryWithPhonetic}{痛快}{tong4kuai4}{12,7}{⽧、⼼}[HSK 4]
  \definition{adj.}{encantado; alegre; muito feliz; confortável | franco; direto; simples e direto}
\end{EntryWithPhonetic}

\begin{EntryWithPhonetic}{痛骂}{tong4ma4}{12,9}{⽧、⾺}
  \definition{v.}{repreender severamente}
\end{EntryWithPhonetic}

\begin{EntryWithPhonetic}{偷}{tou1}{11}{⼈}[HSK 5]
  \definition{adv.}{furtivamente; secretamente; às escondidas}
  \definition{s.}{ladrão; furtador}
  \definition{v.}{roubar; furtar; levar sem pagar; roubar os bens alheios às escondidas | encontrar (tempo) | deixar-se levar; viver apenas para o presente, sem se preocupar com o futuro}
\end{EntryWithPhonetic}

\begin{EntryWithPhonetic}{偷安}{tou1'an1}{11,6}{⼈、⼧}
  \definition{v.}{buscar facilidade temporária}
\end{EntryWithPhonetic}

\begin{EntryWithPhonetic}{偷渡}{tou1du4}{11,12}{⼈、⽔}
  \definition{s.}{contrabando | imigração ilegal | clandestino (em um navio)}
  \definition{v.}{executar um bloqueio | roubar através da fronteira internacional}
\end{EntryWithPhonetic}

\begin{EntryWithPhonetic}{偷窃}{tou1qie4}{11,9}{⼈、⽳}
  \definition{v.}{furtar | roubar}
\end{EntryWithPhonetic}

\begin{EntryWithPhonetic}{偷情}{tou1qing2}{11,11}{⼈、⼼}
  \definition{v.}{manter um caso de amor clandestino}
\end{EntryWithPhonetic}

\begin{EntryWithPhonetic}{偷税}{tou1shui4}{11,12}{⼈、⽲}
  \definition{s.}{evasão fiscal}
\end{EntryWithPhonetic}

\begin{EntryWithPhonetic}{偷听}{tou1ting1}{11,7}{⼈、⼝}
  \definition{v.}{bisbilhotar; monitorar (secretamente)}
\end{EntryWithPhonetic}

\begin{EntryWithPhonetic}{偷偷}{tou1 tou1}{11,11}{⼈、⼈}[HSK 5]
  \definition{adv.}{secretamente; dissimuladamente; furtivamente; às escondidas; descreve uma ação que não é notada pelos outros; em segredo ou em privado, não revelada}
\end{EntryWithPhonetic}

\begin{EntryWithPhonetic}{偷袭}{tou1xi2}{11,11}{⼈、⾐}
  \definition{s.}{ataque surpresa}
  \definition{v.}{montar um ataque furtivo | invadir}
\end{EntryWithPhonetic}

\begin{EntryWithPhonetic}{偸}{tou1}{11}{⼈}
  \variantof{偷}
\end{EntryWithPhonetic}

\begin{EntryWithPhonetic}{头}{tou2}{5}{⼤}[HSK 2,3]
  \definition{adj.}{(antes de um numeral) primeiro | (antes de 年 ou 天) último; anterior}
  \definition{clas.}{usado para suínos ou gado (animais de criação) | usado para cabeças de alho ou coisas com formato de cabeça}
  \definition{num.}{primeiro}
  \definition{prep.}{antes de; perto de; introduz o tempo de uma ação, equivalente a  在……之前 ou 临近 | (entre dois algarismos, indicando um número aproximado) cerca de}
  \definition[个,颗]{s.}{cabeça; a parte do corpo humano ou animal que possui órgãos como boca, nariz, olhos e ouvidos | cabelo ou penteado | topo; fim; a parte superior ou final de um objeto | começo ou fim; o ponto inicial ou final de algo | fim; remanescente; os restos de algo | cabeça; chefe; líder | lado; aspecto}
  \seeref{头}{tou5}
  \seealsoref{临近}{lin2jin4}
  \seealsoref{年}{nian2}
  \seealsoref{天}{tian1}
  \seealsoref{在}{zai4}
  \seealsoref{之前}{zhi1 qian2}
\end{EntryWithPhonetic}

\begin{EntryWithPhonetic}{头发}{tou2fa5}{5,5}{⼤、⼜}[HSK 2]
  \definition[根,缕,头]{s.}{cabelo}
\end{EntryWithPhonetic}

\begin{EntryWithPhonetic}{头号}{tou2hao4}{5,5}{⼤、⼝}
  \definition{adj.}{primeira classe | número um | \emph{top rank}}
\end{EntryWithPhonetic}

\begin{EntryWithPhonetic}{头脑}{tou2 nao3}{5,10}{⼤、⾁}[HSK 3]
  \definition{s.}{inteligência; mente | pista; tópicos principais | chefe; líder; capitão}
\end{EntryWithPhonetic}

\begin{EntryWithPhonetic}{头脑风暴}{tou2nao3feng1bao4}{5,10,4,15}{⼤、⾁、⾵、⽇}
  \definition{s.}{\emph{brainstorm}}
\end{EntryWithPhonetic}

\begin{EntryWithPhonetic}{头疼}{tou2 teng2}{5,10}{⼤、⽧}[HSK 6]
  \definition{s.}{dor de cabeça}
  \definition{v.}{estar preocupado ou incomodado por alguém ou algo}
\end{EntryWithPhonetic}

\begin{EntryWithPhonetic}{头头}{tou2tou2}{5,5}{⼤、⼤}
  \definition{s.}{chefe | o cabeça}
\end{EntryWithPhonetic}

\begin{EntryWithPhonetic}{头像}{tou2xiang4}{5,13}{⼤、⼈}
  \definition{s.}{retrato | busto | avatar | imagem de perfil (computação)}
\end{EntryWithPhonetic}

\begin{EntryWithPhonetic}{投}{tou2}{7}{⼿}[HSK 4]
  \definition*{s.}{Sobrenome Tou}
  \definition{pron.}{para; indica tempo, equivalente a 到, 临 | para; em direção a; indica orientação, direção, equivalente a 朝 ou 向}
  \definition{s.}{um jogo durante uma festa em que o vencedor era decidido pelo número de flechas lançadas em um pote distante | jogo de dados}
  \definition{v.}{lançar; arremessar; atirar | deixar cair; colocar em; lançar | mergulhar em; lançar-se em; pular dentro | lançar; projetar; sombrear | entregar; postar; enviar | ir até; ir para; buscar; juntar-se | sentir-se atraído por; adaptar-se a; concordar com; atender a}
  \seealsoref{朝}{chao2}
  \seealsoref{到}{dao4}
  \seealsoref{临}{lin2}
  \seealsoref{向}{xiang4}
\end{EntryWithPhonetic}

\begin{EntryWithPhonetic}{投递}{tou2di4}{7,10}{⼿、⾡}
  \definition{v.}{despachar | enviar}
\end{EntryWithPhonetic}

\begin{EntryWithPhonetic}{投票}{tou2piao4}{7,11}{⼿、⽰}[HSK 6]
  \definition{v.+compl.}{votar; dar um voto; um método de eleição no qual os eleitores escrevem o nome da pessoa que querem eleger na cédula, ou marcam a cédula com o nome do candidato impresso e depois a colocam na urna para votar na resolução}
\end{EntryWithPhonetic}

\begin{EntryWithPhonetic}{投入}{tou2ru4}{7,2}{⼿、⼊}[HSK 4]
  \definition{adj.}{sisudo; dedicado; devotado; absorto}
  \definition{s.}{investimento; insumo; refere-se à aplicação de recursos}
  \definition{v.}{lançar em; colocar em; jogar em; por em | entrar em uma situação; participar de | aplicar; investir; colocar fundos em}
\end{EntryWithPhonetic}

\begin{EntryWithPhonetic}{投诉}{tou2su4}{7,7}{⼿、⾔}[HSK 4]
  \definition{v.}{reclamar; queixar-se; reclamar às autoridades ou pessoas envolvidas}
\end{EntryWithPhonetic}

\begin{EntryWithPhonetic}{投资}{tou2zi1}{7,10}{⼿、⾙}[HSK 4]
  \definition[笔]{s.}{investimento}
  \definition{v.}{investir; aplicar dinheiro; investir dinheiro em negócios}
\end{EntryWithPhonetic}

\begin{EntryWithPhonetic}{投资风险}{tou2zi1 feng1xian3}{7,10,4,9}{⼿、⾙、⾵、⾩}
  \definition*{s.}{risco de investimento}
\end{EntryWithPhonetic}

\begin{EntryWithPhonetic}{投资回报率}{tou2zi1 hui2bao4 lv4}{7,10,6,7,11}{⼿、⾙、⼞、⼿、⽞}
  \definition{s.}{retorno sobre o investimento (ROI)}
\end{EntryWithPhonetic}

\begin{EntryWithPhonetic}{投资家}{tou2zi1jia1}{7,10,10}{⼿、⾙、⼧}
  \definition{s.}{investidor}
  \seealsoref{投资人}{tou2zi1ren2}
  \seealsoref{投资者}{tou2zi1zhe3}
\end{EntryWithPhonetic}

\begin{EntryWithPhonetic}{投资人}{tou2zi1ren2}{7,10,2}{⼿、⾙、⼈}
  \definition{s.}{investidor}
  \seealsoref{投资家}{tou2zi1jia1}
  \seealsoref{投资者}{tou2zi1zhe3}
\end{EntryWithPhonetic}

\begin{EntryWithPhonetic}{投资者}{tou2zi1zhe3}{7,10,8}{⼿、⾙、⽼}
  \definition{s.}{investidor}
  \seealsoref{投资家}{tou2zi1jia1}
  \seealsoref{投资人}{tou2zi1ren2}
\end{EntryWithPhonetic}

\begin{EntryWithPhonetic}{透}{tou4}{10}{⾡}[HSK 4]
  \definition{adv.}{totalmente; completamente; minuciosamente | profundamente; extremamente}
  \definition{v.}{penetrar; passar através de; infiltrar-se através de | revelar; deixar transparecer; contar secretamente |mostrar; aparecer}
\end{EntryWithPhonetic}

\begin{EntryWithPhonetic}{透彻}{tou4che4}{10,7}{⾡、⼻}
  \definition{adj.}{minucioso | incisivo | penetrante}
\end{EntryWithPhonetic}

\begin{EntryWithPhonetic}{透澈}{tou4che4}{10,15}{⾡、⽔}
  \variantof{透彻}
\end{EntryWithPhonetic}

\begin{EntryWithPhonetic}{透顶}{tou4ding3}{10,8}{⾡、⾴}
  \definition{adv.}{completamente}
\end{EntryWithPhonetic}

\begin{EntryWithPhonetic}{透过}{tou4guo4}{10,6}{⾡、⾡}
  \definition{v.}{passar através | penetrar}
\end{EntryWithPhonetic}

\begin{EntryWithPhonetic}{透亮}{tou4liang4}{10,9}{⾡、⼇}
  \definition{adj.}{brilhante | claro como cristal}
\end{EntryWithPhonetic}

\begin{EntryWithPhonetic}{透露}{tou4lu4}{10,21}{⾡、⾬}[HSK 6]
  \definition{v.}{vazar; revelar; expor; divulgar; contar deliberadamente um segredo a alguém; revelar um certo significado}
\end{EntryWithPhonetic}

\begin{EntryWithPhonetic}{透明}{tou4ming2}{10,8}{⾡、⽇}[HSK 4]
  \definition{adj.}{transparente; diáfano; capaz de transmitir luz | evidente; transparente; situação ou assunto que seja aberto e não oculto | transparente; diáfano; indica pureza, ausência de impurezas}
\end{EntryWithPhonetic}

\begin{EntryWithPhonetic}{透辟}{tou4pi4}{10,13}{⾡、⾟}
  \definition{adj.}{incisivo | penetrante}
\end{EntryWithPhonetic}

\begin{EntryWithPhonetic}{透气}{tou4qi4}{10,4}{⾡、⽓}
  \definition{v.}{respirar (sobre tecido, etc.) | fluir livremente (sobre ar) | respirar ar fresco | ventilar}
\end{EntryWithPhonetic}

\begin{EntryWithPhonetic}{透水}{tou4shui3}{10,4}{⾡、⽔}
  \definition{adj.}{permeável}
  \definition{s.}{vazamento de água}
\end{EntryWithPhonetic}

\begin{EntryWithPhonetic}{透支}{tou4zhi1}{10,4}{⾡、⽀}
  \definition{v.}{cheque especial (bancário) | saque a descoberto}
\end{EntryWithPhonetic}

\begin{EntryWithPhonetic}{头}{tou5}{5}{⼤}
  \definition{suf.}{adicionado após componentes nominais comuns | adicionado após o componente verbal, forma um substantivo abstrato, geralmente indicando que vale a pena realizar essa ação | adicionado após um componente adjetival, forma um substantivo, geralmente indicando algo abstrato | adicionado após o componente substantivo que indica a direção}
  \seeref{头}{tou2}
\end{EntryWithPhonetic}

\begin{EntryWithPhonetic}{突}{tu1}{9}{⽳}
  \definition{adv.}{de repente; abruptamente; inesperadamente}
  \definition{s.}{chaminé}
  \definition{v.}{avançar rapidamente; atacar | projetar; destacar-se | romper | projetar-se; inchar; fazer bojo}
\end{EntryWithPhonetic}

\begin{EntryWithPhonetic}{突出}{tu1chu1}{9,5}{⽳、⼐}[HSK 3]
  \definition{adj.}{proeminente; excelente; mais que a média}
  \definition{v.}{romper | enfatizar; destacar; dar destaque a | sobressair; projetar-se; destacar-se}
\end{EntryWithPhonetic}

\begin{EntryWithPhonetic}{突破}{tu1po4}{9,10}{⽳、⽯}[HSK 5]
  \definition{v.}{romper; fazer uma descoberta revolucionária; concentrar esforços em um único ponto de ataque, reunir o sucesso | quebrar (limite); superar (dificuldade); superar dificuldades; ultrapassar os números ou limites anteriores, superar recordes anteriores, etc.; romper com as limitações e restrições anteriores}
\end{EntryWithPhonetic}

\begin{EntryWithPhonetic}{突然}{tu1ran2}{9,12}{⽳、⽕}[HSK 3]
  \definition{adj.}{repentino; abrupto; inesperado}
  \definition{adv.}{de repente; abruptamente; inesperadamente; subitamente}
\end{EntryWithPhonetic}

\begin{EntryWithPhonetic}{图}{tu2}{8}{⼞}[HSK 3]
  \definition*{s.}{Sobrenome Tu}
  \definition[张]{s.}{mapa; gráfico; imagem; desenho | plano; esquema; tentativa}
  \definition{v.}{procurar; perseguir; esperar obter| desenhar; retratar; pintar | imaginar; planejar; pensar; maquinar}
\end{EntryWithPhonetic}

\begin{EntryWithPhonetic}{图案}{tu2'an4}{8,10}{⼞、⽊}[HSK 4]
  \definition[种,个]{s.}{padrão; desenho; padrões e gráficos usados para decoração de edifícios, tecidos, artes e artesanato, etc.}
\end{EntryWithPhonetic}

\begin{EntryWithPhonetic}{图画}{tu2 hua4}{8,8}{⼞、⽥}[HSK 3]
  \definition[幅,张,套]{s.}{desenho; imagem; pintura}
\end{EntryWithPhonetic}

\begin{EntryWithPhonetic}{图片}{tu2 pian4}{8,4}{⼞、⽚}[HSK 2]
  \definition[张,幅]{s.}{imagem; fotografia; um termo geral para imagens, fotografias, decalques, etc. usados para ilustrar algo}
\end{EntryWithPhonetic}

\begin{EntryWithPhonetic}{图书}{tu2 shu1}{8,4}{⼞、⼄}[HSK 6]
  \definition{s.}{livros; um termo geral para publicações como livros e álbuns de imagens}[这些图书都可以借阅。===Esses livros estão disponíveis para empréstimo.]
\end{EntryWithPhonetic}

\begin{EntryWithPhonetic}{图书馆}{tu2shu1guan3}{8,4,11}{⼞、⼄、⾷}[HSK 1]
  \definition[个,家]{s.}{biblioteca; instituição que coleta, organiza e armazena livros e materiais para leitura e consulta}
\end{EntryWithPhonetic}

\begin{EntryWithPhonetic}{徒}{tu2}{10}{⼻}
  \definition{adj.}{vazio; nu}
  \definition{adv.}{somente; meramente; apenas | a pé | em vão; sem sucesso; sem sucesso}
  \definition{s.}{aprendiz; aluno | seguidor; crente | (pejorativo) pessoas da mesma facção | (pejorativo) pessoa; companheiro | (prisão) pena; prisão; sentença | estudante}
  \definition{v.}{estar a pé | andar}
\end{EntryWithPhonetic}

\begin{EntryWithPhonetic}{徒弟}{tu2di4}{10,7}{⼻、⼸}[HSK 6]
  \definition[位,名,个]{s.}{discípulo; aprendiz; uma pessoa que aprende com um mestre; geralmente se refere a uma pessoa que aprende com um especialista}[他是我的徒弟。===Ele é meu aprendiz.]
\end{EntryWithPhonetic}

\begin{EntryWithPhonetic}{徒手}{tu2shou3}{10,4}{⼻、⼿}
  \definition{adj.}{com as mãos vazias | desarmado | mão livre (desenho) | lutando mão-a-mão}
\end{EntryWithPhonetic}

\begin{EntryWithPhonetic}{途}{tu2}{10}{⾡}
  \definition[条]{s.}{caminho; estrada; rota | jornada; caminho}
\end{EntryWithPhonetic}

\begin{EntryWithPhonetic}{途径}{tu2jing4}{10,8}{⾡、⼻}[HSK 6]
  \definition[种,条,个]{s.}{caminho; canal; metaforicamente falando, uma maneira ou método de resolver um problema ou fazer algo}
\end{EntryWithPhonetic}

\begin{EntryWithPhonetic}{途中}{tu2 zhong1}{10,4}{⾡、⼁}[HSK 4]
  \definition[家]{adv.}{no caminho; ao longo do caminho}
\end{EntryWithPhonetic}

\begin{EntryWithPhonetic}{土}{tu3}{3}{⼟}[HSK 3,6][Kangxi 32]
  \definition*{s.}{Sobrenome Tu}
  \definition{adj.}{local; nativo; local com características regionais| caseiro; indígena; o que é tradicional no país; popular | não refinado; não esclarecido; não está na moda; não é popular}
  \definition[堆,捧,层]{s.}{solo; terra | terra; território | ópio bruto | cidade natal; terra natal; pátria}
\end{EntryWithPhonetic}

\begin{EntryWithPhonetic}{土地}{tu3di4}{3,6}{⼟、⼟}[HSK 4]
  \definition[片,块,顷]{s.}{terra; solo; chão; superfície terrestre da Terra usada para cultivar, construir edifícios e viver | território; território de um país}
  \seeref{土地}{tu3di5}
\end{EntryWithPhonetic}

\begin{EntryWithPhonetic}{土地}{tu3di5}{3,6}{⼟、⼟}
  \definition[片,块,顷]{s.}{deus da audeia; deus local; \emph{genius loci} deidade protetora de um local; (superstição) refere-se ao deus da terra que governa uma pequena área}
  \seeref{土地}{tu3di4}
\end{EntryWithPhonetic}

\begin{EntryWithPhonetic}{土豆}{tu3dou4}{3,7}{⼟、⾖}[HSK 5]
  \definition[颗,斤,个,棵]{s.}{batata; denominação comum da batata}
\end{EntryWithPhonetic}

\begin{EntryWithPhonetic}{土豆泥}{tu3dou4ni2}{3,7,8}{⼟、⾖、⽔}
  \definition{s.}{purê de batatas}
\end{EntryWithPhonetic}

\begin{EntryWithPhonetic}{土鸡}{tu3ji1}{3,7}{⼟、⿃}
  \definition{s.}{galinha caipira}
\end{EntryWithPhonetic}

\begin{EntryWithPhonetic}{吐}{tu3}{6}{⼝}[HSK 5]
  \definition{v.}{cuspir; sair pela boca | surgir ou aparecer pela boca ou por uma fenda | dizer; contar; falar abertamente}
  \seeref{吐}{tu4}
\end{EntryWithPhonetic}

\begin{EntryWithPhonetic}{吐}{tu4}{6}{⼝}[HSK 5]
  \definition{v.}{vomitar; sair pela boca | vomitar; expelir; metáfora para ser forçado a devolver bens usurpados}
  \seeref{吐}{tu3}
\end{EntryWithPhonetic}

\begin{EntryWithPhonetic}{兔}{tu4}{8}{⼉}[HSK 5]
  \definition[只]{s.}{lebre; coelho}
\end{EntryWithPhonetic}

\begin{EntryWithPhonetic}{兔子}{tu4zi5}{8,3}{⼉、⼦}
  \definition[只]{s.}{coelho | lebre}
\end{EntryWithPhonetic}

\begin{EntryWithPhonetic}{团}{tuan2}{6}{⼞}[HSK 3]
  \definition*{s.}{Liga da Juventude Comunista da China; Liga}
  \definition{adj.}{redondo; circular}
  \definition{clas.}{usado para algo em forma de bola}
  \definition[个]{s.}{bolinho de massa; comida em forma de bola feita de arroz ou farinha | algo em forma de bola | grupo; corpo; sociedade; organização; um grupo envolvido em um determinado trabalho ou atividade | regimento; unidade organizacional militar, geralmente abaixo do nível de divisão e acima do nível de batalhão}
  \definition{v.}{enrolar algo para formar uma bola; rolar | reunir; unir; conglomerar}
\end{EntryWithPhonetic}

\begin{EntryWithPhonetic}{团队}{tuan2 dui4}{6,4}{⼞、⾩}[HSK 6]
  \definition[个,支,种]{s.}{equipe; time; grupo; um grupo de alguma natureza}
\end{EntryWithPhonetic}

\begin{EntryWithPhonetic}{团结}{tuan2jie2}{6,9}{⼞、⽷}[HSK 3]
  \definition{adj.}{unido; amigável; harmonioso; relação harmoniosa e coexistência harmoniosa}
  \definition{v.}{unir; reunir}
\end{EntryWithPhonetic}

\begin{EntryWithPhonetic}{团体}{tuan2ti3}{6,7}{⼞、⼈}[HSK 3]
  \definition[种,个]{s.}{equipe; grupo; organização; um grupo de pessoas com objetivos e interesses comuns}
\end{EntryWithPhonetic}

\begin{EntryWithPhonetic}{团长}{tuan2 zhang3}{6,4}{⼞、⾧}[HSK 5]
  \definition[位,名]{s.}{comandante do regimento | chefe (ou presidente) de uma delegação, trupe, etc. | líder de uma delegação}
\end{EntryWithPhonetic}

\begin{EntryWithPhonetic}{推}{tui1}{11}{⼿}[HSK 2]
  \definition{v.}{empurrar; dar um encontrão | girar um moinho ou uma pedra de amolar; moer | cortar; aparar | impulsionar; promover; avançar | inferir; deduzir | afastar; fugir; deslocar | adiar | eleger; escolher | ter em alta estima; elogiar muito | declinar | selecionar | elogiar muito}
\end{EntryWithPhonetic}

\begin{EntryWithPhonetic}{推迟}{tui1chi2}{11,7}{⼿、⾡}[HSK 4]
  \definition{v.}{adiar; postergar; tardar; deixar para mais tarde}
\end{EntryWithPhonetic}

\begin{EntryWithPhonetic}{推出}{tui1 chu1}{11,5}{⼿、⼐}[HSK 6]
  \definition{v.}{lançar; apresentar; fazer com que apareça diante do público | deduzir; tirar conclusões da análise}
\end{EntryWithPhonetic}

\begin{EntryWithPhonetic}{推动}{tui1 dong4}{11,6}{⼿、⼒}[HSK 3]
  \definition{v.}{promover; atuar; impulsionar; empurrar para a frente; dar ímpeto a; começar ou avançar algo (com alguma força); começar a trabalhar}
\end{EntryWithPhonetic}

\begin{EntryWithPhonetic}{推广}{tui1guang3}{11,3}{⼿、⼴}[HSK 3]
  \definition{v.}{espalhar; estender; promover; popularizar; expandir o escopo de uso ou função de algo}
\end{EntryWithPhonetic}

\begin{EntryWithPhonetic}{推介}{tui1jie4}{11,4}{⼿、⼈}
  \definition{s.}{promoção}
  \definition{v.}{promover | introduzir e recomendar}
\end{EntryWithPhonetic}

\begin{EntryWithPhonetic}{推进}{tui1 jin4}{11,7}{⼿、⾡}[HSK 3]
  \definition{v.}{avançar; empurrar; levar adiante; dar ímpeto a; promover o trabalho e fazê-lo avançar | empurrar; dirigir; avançar; seguir em frente; seguir em frente}
\end{EntryWithPhonetic}

\begin{EntryWithPhonetic}{推开}{tui1 kai1}{11,4}{⼿、⼶}[HSK 3]
  \definition{v.}{declinar; rejeitar | empurrar para longe; aplicar força em uma determinada direção para mover uma pessoa ou objeto para longe de seu lugar original | empurrar para abrir (um portão, etc.); empurrar para fora para abrir algo que está fechado | estender; popularizar; promover para um alcance mais amplo e realizar em uma escala mais ampla}
\end{EntryWithPhonetic}

\begin{EntryWithPhonetic}{推销}{tui1xiao1}{11,12}{⼿、⾦}[HSK 4]
  \definition{v.}{vender; comercializar; promover vendas; promover a comercialização de mercadorias}
\end{EntryWithPhonetic}

\begin{EntryWithPhonetic}{推行}{tui1 xing2}{11,6}{⼿、⾏}[HSK 5]
  \definition{v.}{realizar; prosseguir; praticar | implementar; praticar; implementação generalizada; divulgar (experiências, métodos, etc.)}
\end{EntryWithPhonetic}

\begin{EntryWithPhonetic}{腿}{tui3}{13}{⾁}[HSK 2]
  \definition[条,双]{s.}{perna; as partes dos humanos e dos animais que sustentam o corpo e permitem caminhar | um suporte em forma de perna; a parte inferior de um objeto que atua como uma perna e serve de suporte | presunto}
\end{EntryWithPhonetic}

\begin{EntryWithPhonetic}{腿号}{tui3hao4}{13,5}{⾁、⼝}
  \definition{s.}{anilha numerada (por exemplo, usada para identificar pássaros)}
  \seealsoref{腿号箍}{tui3hao4gu1}
\end{EntryWithPhonetic}

\begin{EntryWithPhonetic}{腿号箍}{tui3hao4gu1}{13,5,14}{⾁、⼝、⽵}
  \definition{s.}{anilha numerada (por exemplo, usada para identificar pássaros)}
  \seealsoref{腿号}{tui3hao4}
\end{EntryWithPhonetic}

\begin{EntryWithPhonetic}{退}{tui4}{9}{⾡}[HSK 3]
  \definition{v.}{recuar; mover-se para trás  (oposto de 進) | remover; retirar; fazer recuar; mover para trás | desistir; retirar-se de | refluir; declinar; retroceder | aposentar-se; deixar o emprego por atingir a idade estipulada ou por problemas de saúde | retornar; reembolsar; devolver | romper; cancelar o que foi decidido}
  \seealsoref{进}{jin4}
\end{EntryWithPhonetic}

\begin{EntryWithPhonetic}{退出}{tui4 chu1}{9,5}{⾡、⼐}[HSK 3]
  \definition{v.}{desistir; retirar-se; separar-se; retirar-se de; abandonar o local ou outro lugar e parar de participar; abandonaar o grupo ou organização}
\end{EntryWithPhonetic}

\begin{EntryWithPhonetic}{退票}{tui4 piao4}{9,11}{⾡、⽰}[HSK 6]
  \definition{s.}{bilhete devolvido (ou não utilizado) | reembolso do bilhete}
  \definition{v.}{devolver um bilhete; obter um reembolso por um bilhete | devolver (um cheque)}
\end{EntryWithPhonetic}

\begin{EntryWithPhonetic}{退休}{tui4xiu1}{9,6}{⾡、⼈}[HSK 3]
  \definition{v.+compl.}{aposentar-se; os trabalhadores que deixarem o emprego por velhice ou invalidez causada pelo trabalho receberão as despesas de subsistência conforme o cronograma}
\end{EntryWithPhonetic}

\begin{EntryWithPhonetic}{吞}{tun1}{7}{⼝}[HSK 6]
  \definition*{s.}{Sobrenome Tun}
  \definition{v.}{engolir; engolir em seco | tomar posse de; anexar | engolir; tragar; devorar; engolir inteiro ou em pedaços | absorver; engolir; engolfar}
\end{EntryWithPhonetic}

\begin{EntryWithPhonetic}{屯}{tun2}{4}{⼬}
  \definition*{s.}{Sobrenome Tun}
  \definition{s.}{vila (geralmente usado em nomes de vilas); vilarejos; aldeias; povoados}
  \definition{v.}{coletar; estocar; armazenar; acumular | estacionar (tropas); aquartelar}
  \seeref{屯}{zhun1}
\end{EntryWithPhonetic}

\begin{EntryWithPhonetic}{托}{tuo1}{6}{⼿}[HSK 6]
  \definition{clas.}{torr, uma unidade de pressão, 1 torr é igual à pressão de 1 mmHg, ou 133,322 Pa}
  \definition{s.}{algo servindo como suporte | fantoche; cúmplice; pessoas que ajudam golpistas a enganar outras pessoas}
  \definition{v.}{segurar na palma; apoiar com a mão ou palma; suportar (um objeto) com um objeto ou com a palma da mão | destacar; servir como contraste | pedir; confiar | implorar; dar como pretexto | dever a; confiar em}
\end{EntryWithPhonetic}

\begin{EntryWithPhonetic}{拖}{tuo1}{8}{⼿}[HSK 6]
  \definition{v.}{puxar; arrastar; transportar; puxar um objeto para movê-lo contra o solo ou outra superfície | esfregar; limpar o chão com uma ferramenta especial para esfregar | atrasar; prolongar; procrastinar; arrastar; coisas que deveriam ser feitas nunca são iniciadas ou concluídas; uma certa nota é prolongada por um longo tempo | atrasar; conter; segurar; restringir}
\end{EntryWithPhonetic}

\begin{EntryWithPhonetic}{拖拉机}{tuo1la1ji1}{8,8,6}{⼿、⼿、⽊}
  \definition[台]{s.}{trator}
\end{EntryWithPhonetic}

\begin{EntryWithPhonetic}{拖鞋}{tuo1 xie2}{8,15}{⼿、⾰}[HSK 6]
  \definition[双,只]{s.}{chinelos; samdálias; babouche; sapatos sem cabedal geralmente são usados ​​em ambientes fechados}
\end{EntryWithPhonetic}

\begin{EntryWithPhonetic}{脱}{tuo1}{11}{⾁}[HSK 4]
  \definition{conj.}{se; no caso}
  \definition{v.}{(cabelo, pele) soltar-se; desprender-se; cair | retirar peça de roupa do corpo | sair de; escapar de | perder (palavras) | livrar-se de algo}
\end{EntryWithPhonetic}

\begin{EntryWithPhonetic}{脱离}{tuo1li2}{11,10}{⾁、⼇}[HSK 5]
  \definition{v.}{separar-se; divorciar-se; afastar-se; sair (de um determinado ambiente ou situação); romper (uma determinada relação)}
\end{EntryWithPhonetic}

\begin{EntryWithPhonetic}{脱毛}{tuo1mao2}{11,4}{⾁、⽑}
  \definition{s.}{depilação}
  \definition{v.}{perder cabelo ou penas | depilar | fazer a barba}
\end{EntryWithPhonetic}

\begin{EntryWithPhonetic}{脱险}{tuo1xian3}{11,9}{⾁、⾩}
  \definition{v.}{sair do perigo}
\end{EntryWithPhonetic}

\begin{EntryWithPhonetic}{鸵}{tuo2}{10}{⿃}
  \definition[只]{s.}{avestruz}
\end{EntryWithPhonetic}

\begin{EntryWithPhonetic}{鸵鸟}{tuo2niao3}{10,5}{⿃、⿃}
  \definition{s.}{avestruz}
\end{EntryWithPhonetic}

\begin{EntryWithPhonetic}{唾}{tuo4}{11}{⼝}
  \definition[口]{s.}{saliva; cuspe}
  \definition{v.}{cuspir (mostrar desprezo) | rejeitar}
\end{EntryWithPhonetic}

\begin{EntryWithPhonetic}{唾骂}{tuo4ma4}{11,9}{⼝、⾺}
  \definition{v.}{insultar | amaldiçoar}
\end{EntryWithPhonetic}

%%%%% EOF %%%%%


%%%%%%%%%%%%%%%%%%%%%%%%%%%%%%% Não existem palavras com pinyin iniciado em "U"
%%%%%%%%%%%%%%%%%%%%%%%%%%%%%%% Não existem palavras com pinyin iniciado em "V"
 %%%
%%% W
%%%
\section*{W}
\addcontentsline{toc}{section}{W}

\begin{verbete}[wai4bian0]{外边}[5;5]
\begin{pronuncia}{wai4bian0}
\significado{p.l.}{ fora; por fora; exterior }
\end{pronuncia}
\end{verbete}

\begin{verbete}[wai4gong1]{外公}[5;4]
\begin{pronuncia}{wai4gong1}
\significado{s.}{ avô materno }
\end{pronuncia}
\end{verbete}

\begin{verbete}[wai4guo2]{外国}[5;8]
\begin{pronuncia}{wai4guo2}
\significado[个]{s.}{ país estrangeiro }
\end{pronuncia}
\end{verbete}

\begin{verbete}[wai4hao4]{外号}[5;5]
\begin{pronuncia}{wai4hao4}
\significado{s.}{ apelido }
\end{pronuncia}
\end{verbete}

\begin{verbete}[wai4mao4]{外贸}[5;9]
\begin{pronuncia}{wai4mao4}
\significado{s.}{ comércio exterior }
\end{pronuncia}
\end{verbete}

\begin{verbete}[wai4mian0]{外面}[5;9]
\begin{pronuncia}{wai4mian0}
\significado{p.l.}{ fora; por fora; exterior }
\end{pronuncia}
\end{verbete}

\begin{verbete}[wai4po2]{外婆}[5;11]
\begin{pronuncia}{wai4po2}
\significado{s.}{ avó materna }
\end{pronuncia}
\end{verbete}

\begin{verbete}[wai4shi4]{外事}[5;8]
\begin{pronuncia}{wai4shi4}
\significado{s.}{ assuntos ou relações exteriores }
\end{pronuncia}
\end{verbete}

\begin{verbete}[wai4sun1]{外孙}[5;6]
\begin{pronuncia}{wai4sun1}
\significado{s.}{ filho da filha }
\end{pronuncia}
\end{verbete}

\begin{verbete}[wai4sun1nv3]{外孙女}[5;6;3]
\begin{pronuncia}{wai4sun1nv3}
\significado{s.}{ filha da filha }
\end{pronuncia}
\end{verbete}

\begin{verbete}[wai4yu3]{外语}[5;9]
\begin{pronuncia}{wai4yu3}
\significado[门]{s.}{ língua estrangeira }
\end{pronuncia}
\end{verbete}

\begin{verbete}[wan1dou4]{豌豆}[15;7]
\begin{pronuncia}{wan1dou4}
\significado{s.}{ ervilha }
\end{pronuncia}
\end{verbete}

\begin{verbete}[wan2]{完}[7]
\begin{pronuncia}{wan2}
\significado{v.}{ acabar; terminar }
\end{pronuncia}
\end{verbete}

\begin{verbete}[wan2]{玩}[8]
\begin{pronuncia}{wan2}
\significado{v.}{ brincar; tocar (intrumento musical) }
\end{pronuncia}
\end{verbete}

\begin{verbete}[wanr2]{玩儿}[8;2]
\begin{pronuncia}{wanr2}
\significado{v.}{ divertir-se }
\end{pronuncia}
\end{verbete}

\begin{verbete}[wan3]{晚}[11]
\begin{pronuncia}{wan3}
\significado{adj.}{ tarde }
\end{pronuncia}
\end{verbete}

\begin{verbete}[wan3fan4]{晚饭}[11;7]
\begin{pronuncia}{wan3fan4}
\significado[份,顿,次,餐]{s.}{ jantar }
\end{pronuncia}
\end{verbete}

\begin{verbete}[wan3shang0]{晚上}[11;3]
\begin{pronuncia}{wan3shang0}
\significado{p.t.}{ noite; à noite }
\end{pronuncia}
\end{verbete}

\begin{verbete}[wan3]{碗}[13]
\begin{pronuncia}{wan3}
\significado[只,个]{n}{ tigela }
\significado{p.c.}{ tigelas }
\end{pronuncia}
\end{verbete}

\begin{verbete}[wan3zi0]{碗子}[13;3]
\begin{pronuncia}{wan3zi0}
\significado{n}{ tigela }
\end{pronuncia}
\end{verbete}

\begin{verbete}[wan4]{万}[3]
\begin{pronuncia}{wan4}
\significado{num.}{ dez mil; 10.000 }
\end{pronuncia}
\end{verbete}

\begin{verbete}[wang3]{往}[8]
\begin{pronuncia}{wang3}
\significado{prep.}{ para; em direção a }
\end{pronuncia}
\end{verbete}

\begin{verbete}[wang3qiu2]{网球}[6;11]
\begin{pronuncia}{wang3qiu2}
\significado[个]{s.}{ tênis (esporte); bola de tênis }
\end{pronuncia}
\end{verbete}

\begin{verbete}[wang4]{忘}[7]
\begin{pronuncia}{wang4}
\significado{v.}{ esquecer }
\end{pronuncia}
\end{verbete}

\begin{verbete}[wen1du4]{温度}[12;9]
\begin{pronuncia}{wen1du4}
\significado[个]{s.}{ temperatura }
\end{pronuncia}
\end{verbete}

\begin{verbete}[wei2]{喂}[12]
\begin{pronuncia}{wei2}
\significado{interj.}{ ei!; chamar atenção (alô, telefone) }
\end{pronuncia}
\begin{pronuncia}{wei4}
\significado*{}{ 喂\p{wei4} }
\end{pronuncia}
\end{verbete}

\begin{verbete}[wei4sheng1jian1]{卫生间}[3;5;7]
\begin{pronuncia}{wei4sheng1jian1}
\significado[间]{s.}{ banheiro; toilette }
\end{pronuncia}
\end{verbete}

\begin{verbete}[wei4]{为}[4]
\begin{pronuncia}{wei4}
\significado{prep.}{ para }
\end{pronuncia}
\end{verbete}

\begin{verbete}[wei4shen2me0]{为什么}[4;4;3]
\begin{pronuncia}{wei4shen2me0}
\significado{interr.}{ por que? }
\end{pronuncia}
\end{verbete}

\begin{verbete}[wei4]{位}[7]
\begin{pronuncia}{wei4}
\significado{p.c.}{ para pessoas (com cortesia) }
\end{pronuncia}
\end{verbete}

\begin{verbete}[wei4dao0]{味道}[8;12]
\begin{pronuncia}{wei4dao0}
\significado{s.}{ sabor }
\end{pronuncia}
\end{verbete}

\begin{verbete}[wei4]{喂}
\begin{pronuncia}{wei4}
\significado{interj.}{ ei!; chamar atenção (alô, telefone) }
\end{pronuncia}
\begin{pronuncia}{wei2}
\significado*{}{ 喂\p{wei2} }
\end{pronuncia}
\end{verbete}

\begin{verbete}[wen2hua4]{文化}[4;4]
\begin{pronuncia}{wen2hua4}
\significado[个,种]{s.}{ cultura; civilização }
\end{pronuncia}
\end{verbete}

\begin{verbete}[Wen2xue2xi4]{文学系}[4;8;7]
\begin{pronuncia}{Wen2xue2xi4}
\significado{s.}{ Faculdade de Letras }
\end{pronuncia}
\end{verbete}

\begin{verbete}[wen4]{问}[6]
\begin{pronuncia}{wen4}
\significado{v.}{ perguntar }
\end{pronuncia}
\end{verbete}

\begin{verbete}[wen4ti2]{问题}[6;15]
\begin{pronuncia}{wen4ti2}
\significado[个]{s.}{ pergunta; questão; problema }
\end{pronuncia}
\end{verbete}

\begin{verbete}[wo3]{我}[7]
\begin{pronuncia}{wo3}
\significado{pron.}{ eu }
\end{pronuncia}
\end{verbete}

\begin{verbete}[wo3de0]{我的}[7;8]
\begin{pronuncia}{wo3de0}
\significado{pron.}{ meu, meus }
\end{pronuncia}
\end{verbete}

\begin{verbete}[wo3men0]{我们}[7;5]
\begin{pronuncia}{wo3men0}
\significado{pron.}{ nós }
\end{pronuncia}
\end{verbete}

\begin{verbete}[wo3men0de0]{我们的}[7;5;8]
\begin{pronuncia}{wo3men0de0}
\significado{pron.}{ nosso, nossos }
\end{pronuncia}
\end{verbete}

\begin{verbete}[wo3shi4]{卧室}[8;9]
\begin{pronuncia}{wo3shi4}
\significado{s.}{ quarto de dormir }
\end{pronuncia}
\end{verbete}

\begin{verbete}[wu3fan4]{午饭}[4;7]
\begin{pronuncia}{wu3fan4}
\significado[份,顿,次,餐]{s.}{ almoço }
\end{pronuncia}
\end{verbete}

\begin{verbete}[wu3]{五}[4]
\begin{pronuncia}{wu3}
\significado{num.}{ cinco; 5 }
\end{pronuncia}
\end{verbete}

\begin{verbete}[wu3]{舞}[14]
\begin{pronuncia}{wu3}
\significado{s.}{ dança }
\end{pronuncia}
\end{verbete}

%%%%% EOF %%%%%

 %%%
%%% X
%%%

\section*{X}\addcontentsline{toc}{section}{X}

\begin{entry}{夕阳}{xi1yang2}{3,6}{⼣、⾩}
  \definition{s.}{pôr do sol}
  \seealsoref{日出}{ri4chu1}
\end{entry}

\begin{entry}{吸}{xi1}{6}{⼝}[HSK 4]
  \definition{v.}{inalar; inspirar; aspirar; itroduzir líquidos, gases, etc. no corpo | absorver; sugar | atrair; atrair para si mesmo; atrair (interesse, investimento etc.)}
\end{entry}

\begin{entry}{吸管}{xi1 guan3}{6,14}{⼝、⽵}[HSK 4]
  \definition[根,个]{s.}{tubo de sucção; sugador; canudo (para beber); refere-se ao tubo fino usado para sugar bebidas | conta-gotas; pipeta; cateter para bombeamento de líquidos usando pressão de ar}
\end{entry}

\begin{entry}{吸收}{xi1shou1}{6,6}{⼝、⽁}[HSK 4]
  \definition{v.}{imbuir; absorver; assimilar; sugar;  chupar; (animais, plantas, etc.) extrair material de fora dos tecidos para o interior dos tecidos | absorver; chupar;  sugar alguma substância de fora para dentro | recrutar; alistar; inscrever-se; matricular-se; admitir; (organizações ou coletivos) aceitar novos membros | absorver; aproveitar e usar a experiência, o conhecimento, o dinheiro e outras coisas valiosas de outras pessoas | absorver; diminuir, atenuar ou eliminar determinados efeitos ou fenômenos}
\end{entry}

\begin{entry}{吸铁石}{xi1tie3shi2}{6,10,5}{⼝、⾦、⽯}
  \definition{s.}{imã | magneto}
  \seealsoref{磁铁}{ci2tie3}
\end{entry}

\begin{entry}{吸烟}{xi1yan1}{6,10}{⼝、⽕}[HSK 4]
  \definition{v.+compl.}{fumar}
\end{entry}

\begin{entry}{吸引}{xi1yin3}{6,4}{⼝、⼸}[HSK 4]
  \definition{v.}{atrair; apelar para; chamar a atenção de outros objetos, forças ou pessoas para si mesmo}
\end{entry}

\begin{entry}{西}{xi1}{6}{⾑}[HSK 1]
  \definition{s.}{oeste}
\end{entry}

\begin{entry}{西安}{xi1'an1}{6,6}{⾑、⼧}
  \definition*{s.}{Xi'an}
\end{entry}

\begin{entry}{西班牙文}{xi1ban1ya2wen2}{6,10,4,4}{⾑、⽟、⽛、⽂}
  \definition{s.}{espanhol, língua espanhola}
  \seealsoref{西文}{xi1wen2}
\end{entry}

\begin{entry}{西班牙语}{xi1ban1ya2yu3}{6,10,4,9}{⾑、⽟、⽛、⾔}
  \definition{s.}{espanhol | língua espanhola}
  \seealsoref{西语}{xi1yu3}
\end{entry}

\begin{entry}{西半球}{xi1ban4qiu2}{6,5,11}{⾑、⼗、⽟}
  \definition{s.}{hemisfério oeste}
\end{entry}

\begin{entry}{西北}{xi1 bei3}{6,5}{⾑、⼔}[HSK 2]
  \definition{s.}{noroeste | noroeste da China}
\end{entry}

\begin{entry}{西边}{xi1bian1}{6,5}{⾑、⾡}[HSK 1]
  \definition{adv.}{ao oeste de | oeste | lado oeste | parte ocidental}
\end{entry}

\begin{entry}{西部}{xi1 bu4}{6,10}{⾑、⾢}[HSK 3]
  \definition{s.}{parte ocidental}
\end{entry}

\begin{entry}{西餐}{xi1 can1}{6,16}{⾑、⾷}[HSK 2]
  \definition[分,顿]{s.}{comida ocidental}
\end{entry}

\begin{entry}{西方}{xi1 fang1}{6,4}{⾑、⽅}[HSK 2]
  \definition{s.}{países ocidentais | o Ocidente | o Oeste}
\end{entry}

\begin{entry}{西瓜}{xi1gua1}{6,5}{⾑、⽠}[HSK 4]
  \definition[个,颗,粒]{s.}{melancia; fruto que é uma baga de formato grande, globular ou oval, com muita polpa aguada e doce}
\end{entry}

\begin{entry}{西兰花}{xi1lan2hua1}{6,5,7}{⾑、⼋、⾋}
  \definition{s.}{brócolis}
\end{entry}

\begin{entry}{西蓝花}{xi1lan2hua1}{6,13,7}{⾑、⾋、⾋}
  \variantof{西兰花}
\end{entry}

\begin{entry}{西面}{xi1mian4}{6,9}{⾑、⾯}
  \definition{s.}{oeste | lado oeste}
\end{entry}

\begin{entry}{西南}{xi1 nan2}{6,9}{⾑、⼗}[HSK 2]
  \definition{s.}{sudoeste | sudoeste da China}
\end{entry}

\begin{entry}{西文}{xi1wen2}{6,4}{⾑、⽂}
  \definition{s.}{espanhol | língua espanhola}
  \seealsoref{西班牙文}{xi1ban1ya2wen2}
\end{entry}

\begin{entry}{西西}{xi1xi1}{6,6}{⾑、⾑}
  \definition{num.}{centímetro cúbico}
\end{entry}

\begin{entry}{西医}{xi1 yi1}{6,7}{⾑、⼖}[HSK 2]
  \definition{s.}{medicina ocidental | um médico treinado em medicina ocidental}
\end{entry}

\begin{entry}{西语}{xi1yu3}{6,9}{⾑、⾔}
  \definition{s.}{espanhol | língua espanhola}
  \seealsoref{西班牙语}{xi1ban1ya2yu3}
\end{entry}

\begin{entry}{希望}{xi1wang4}{7,11}{⼱、⽉}[HSK 3]
  \definition[个]{s.}{esperança; desejo; expectativa | aquilo em que a esperança é depositada}
  \definition{v.}{ter esperança; desejar; esperar}
\end{entry}

\begin{entry}{昔日}{xi1ri4}{8,4}{⽇、⽇}
  \definition{adj.}{passado}
\end{entry}

\begin{entry}{牺牲}{xi1sheng1}{10,9}{⽜、⽜}
  \definition{s.}{abate de um animal como sacrifício}
  \definition{v.}{sacrificar a vida de alguém | sacrificar (algo de valor)}
\end{entry}

\begin{entry}{悉尼}{xi1ni2}{11,5}{⼼、⼫}
  \definition*{s.}{Sidney}
\end{entry}

\begin{entry}{悉数}{xi1shu3}{11,13}{⼼、⽁}
  \definition{adv.}{enumerar em detalhes | explicar claramente}
  \seeref{悉数}{xi1shu4}
\end{entry}

\begin{entry}{悉数}{xi1shu4}{11,13}{⼼、⽁}
  \definition{adv.}{todos | cada um | toda a soma}
  \seeref{悉数}{xi1shu3}
\end{entry}

\begin{entry}{悉心}{xi1xin1}{11,4}{⼼、⼼}
  \definition{adv.}{colocar o coração (e a alma) em algo | com muito cuidado}
\end{entry}

\begin{entry}{蜥易}{xi1yi4}{14,8}{⾍、⽇}
  \variantof{蜥蜴}
\end{entry}

\begin{entry}{蜥蜴}{xi1yi4}{14,14}{⾍、⾍}
  \definition{s.}{lagarto}
\end{entry}

\begin{entry}{习惯}{xi2guan4}{3,11}{⼄、⼼}[HSK 2]
  \definition[个]{s.}{hábito | costume | prática usual}
  \definition{v.}{ser acostumado a | ter o hábito de}
\end{entry}

\begin{entry}{席卷}{xi2juan3}{10,8}{⼱、⼙}
  \definition{v.}{engolfar | varrer | levar tudo para fora}
\end{entry}

\begin{entry}{袭击}{xi2ji1}{11,5}{⾐、⼐}
  \definition{s.}{ataque (especialmente um ataque surpresa) | invasão}
  \definition{v.}{atacar}
\end{entry}

\begin{entry}{洗}{xi3}{9}{⽔}[HSK 1]
  \definition{v.}{lavar | revelar (fotos) | tomar banho}
\end{entry}

\begin{entry}{洗涤}{xi3di2}{9,10}{⽔、⽔}
  \definition{s.}{enxágue | lava}
  \definition{v.}{enxaguar | lavar}
\end{entry}

\begin{entry}{洗涤间}{xi3di2jian1}{9,10,7}{⽔、⽔、⾨}
  \definition{s.}{lavanderia}
\end{entry}

\begin{entry}{洗劫}{xi3jie2}{9,7}{⽔、⼒}
  \definition{v.}{saquear | pilhar | roubar}
\end{entry}

\begin{entry}{洗净}{xi3jing4}{9,8}{⽔、⼎}
  \definition{v.}{lavar (limpeza)}
\end{entry}

\begin{entry}{洗礼}{xi3li3}{9,5}{⽔、⽰}
  \definition{s.}{batismo}
  \definition{v.}{batizar}
\end{entry}

\begin{entry}{洗手}{xi3shou3}{9,4}{⽔、⼿}
  \definition{v.}{ir ao banheiro | lavar as mãos}
\end{entry}

\begin{entry}{洗手不干}{xi3shou3bu2gan4}{9,4,4,3}{⽔、⼿、⼀、⼲}
  \definition{v.}{parar totalmente de fazer algo}
\end{entry}

\begin{entry}{洗手池}{xi3shou3chi2}{9,4,6}{⽔、⼿、⽔}
  \definition{s.}{pia de banheiro | lavatório}
  \seealsoref{洗手盆}{xi3shou3pen2}
\end{entry}

\begin{entry}{洗手间}{xi3shou3jian1}{9,4,7}{⽔、⼿、⾨}[HSK 1]
  \definition{s.}{sanitário | toilette | banheiro}
\end{entry}

\begin{entry}{洗手盆}{xi3shou3pen2}{9,4,9}{⽔、⼿、⽫}
  \definition{s.}{pia de banheiro | lavatório}
  \seealsoref{洗手池}{xi3shou3chi2}
\end{entry}

\begin{entry}{洗手乳}{xi3shou3ru3}{9,4,8}{⽔、⼿、⼄}
  \definition{s.}{sabonete líquido para lavar as mãos}
  \seealsoref{洗手液}{xi3shou3ye4}
\end{entry}

\begin{entry}{洗手液}{xi3shou3ye4}{9,4,11}{⽔、⼿、⽔}
  \definition{s.}{sabonete líquido para lavar as mãos}
  \seealsoref{洗手乳}{xi3shou3ru3}
\end{entry}

\begin{entry}{洗脱}{xi3tuo1}{9,11}{⽔、⾁}
  \definition{v.}{limpar | purgar | lavar}
\end{entry}

\begin{entry}{洗碗}{xi3wan3}{9,13}{⽔、⽯}
  \definition{v.}{lavar pratos}
\end{entry}

\begin{entry}{洗胃}{xi3wei4}{9,9}{⽔、⾁}
  \definition{s.}{(medicina) lavagem gástrica}
  \definition{v.}{ter o estômago lavado}
\end{entry}

\begin{entry}{洗衣机}{xi3 yi1 ji1}{9,6,6}{⽔、⾐、⽊}[HSK 2]
  \definition[台]{s.}{máquina de lavar roupa}
\end{entry}

\begin{entry}{洗澡}{xi3zao3}{9,16}{⽔、⽔}[HSK 2]
  \definition{v.+compl.}{tomar banho | duchar-se | lavar-se}
\end{entry}

\begin{entry}{洗澡间}{xi3zao3jian1}{9,16,7}{⽔、⽔、⾨}
  \definition[间]{s.}{banheiro}
\end{entry}

\begin{entry}{喜爱}{xi3 ai4}{12,10}{⼝、⽖}[HSK 4]
  \definition{v.}{gostar; amar; ter afeição por; estar interessado em; ter uma queda ou sentir interesse por pessoas ou coisas}
\end{entry}

\begin{entry}{喜欢}{xi3huan5}{12,6}{⼝、⽋}[HSK 1]
  \definition{v.}{gostar}
\end{entry}

\begin{entry}{喜剧}{xi3ju4}{12,10}{⼝、⼑}
  \definition[部,出]{s.}{uma comédia}
\end{entry}

\begin{entry}{戏}{xi4}{6}{⼽}
  \definition[出,场,台]{s.}{drama | peça de teatro | \emph{show}}
\end{entry}

\begin{entry}{戏法}{xi4fa3}{6,8}{⼽、⽔}
  \definition{s.}{truque de mágica | prestidigitação}
\end{entry}

\begin{entry}{戏剧}{xi4ju4}{6,10}{⼽、⼑}
  \definition{s.}{drama | suspense | teatro}
\end{entry}

\begin{entry}{戏剧般}{xi4ju4ban1}{6,10,10}{⼽、⼑、⾈}
  \definition{adj.}{melodramático}
\end{entry}

\begin{entry}{戏剧编剧}{xi4ju4bian1ju4}{6,10,12,10}{⼽、⼑、⽷、⼑}
  \definition{s.}{dramaturgo}
\end{entry}

\begin{entry}{戏剧化地}{xi4ju4hua4di4}{6,10,4,6}{⼽、⼑、⼔、⼟}
  \definition{adv.}{dramaticamente | teatralmente}
\end{entry}

\begin{entry}{戏剧家}{xi4ju4jia1}{6,10,10}{⼽、⼑、⼧}
  \definition{s.}{dramaturgo}
\end{entry}

\begin{entry}{戏剧效果}{xi4ju4xiao4guo3}{6,10,10,8}{⼽、⼑、⽁、⽊}
  \definition{s.}{efeito dramático}
\end{entry}

\begin{entry}{戏剧性}{xi4ju4xing4}{6,10,8}{⼽、⼑、⼼}
  \definition{adj.}{dramático}
\end{entry}

\begin{entry}{戏剧演出}{xi4ju4yan3chu1}{6,10,14,5}{⼽、⼑、⽔、⼐}
  \definition{s.}{performance dramática}
\end{entry}

\begin{entry}{戏弄}{xi4nong4}{6,7}{⼽、⼶}
  \definition{v.}{zombar de | pregar peças | provocar}
\end{entry}

\begin{entry}{戏耍}{xi4shua3}{6,9}{⼽、⽽}
  \definition{v.}{divertir-me | brincar com | provocar}
\end{entry}

\begin{entry}{戏谑}{xi4xue4}{6,11}{⼽、⾔}
  \definition{v.}{brincar | fazer piadas | ridicularizar}
\end{entry}

\begin{entry}{戏院}{xi4yuan4}{6,9}{⼽、⾩}
  \definition{s.}{teatro}
\end{entry}

\begin{entry}{系}{xi4}{7}{⽷}[HSK 3,4]
  \definition*{s.}{sobrenome Xi}
  \definition{s.}{faculdade (da universidade) | departamento}
  \definition{v.}{sistema; série | departamento; faculdade}
  \definition{v.}{relacionar-se com; suportar; depender de | sentir-se ansioso; estar preocupado | amarrar; prender | ser}
  \seeref{系}{ji4}
\end{entry}

\begin{entry}{系列}{xi4lie4}{7,6}{⽷、⼑}[HSK 4]
  \definition{s.}{série; conjunto; conjunto de coisas relacionadas (matemática)}
\end{entry}

\begin{entry}{系囚}{xi4qiu2}{7,5}{⽷、⼞}
  \definition{s.}{prisioneiro}
\end{entry}

\begin{entry}{系统}{xi4tong3}{7,9}{⽷、⽷}[HSK 4]
  \definition{adj.}{sistemático; organizado}
  \definition[个]{s.}{sistema; relação de tipos semelhantes (ou seja, grupo de coisas semelhantes)}
\end{entry}

\begin{entry}{细}{xi4}{8}{⽷}[HSK 4]
  \definition{adj.}{fino; delgado; esguio; esbelto; em oposição a ``粗'' | fino; em partículas pequenas; grãos pequenos | fino e macio;  um sussuro | fino; requintado; delicado | cuidadoso; detalhado; meticuloso | ínfimo; minúsculo; insignificante; diminuto | jovem; pequeno}
  \seealsoref{粗}{cu1}
\end{entry}

\begin{entry}{细节}{xi4jie2}{8,5}{⽷、⾋}[HSK 4]
  \definition{s.}{detalhe; particularidade; aspectos secundários ou partes sutis de um enredo ou episódios secundários usados em uma obra literária para expressar o caráter de uma pessoa ou as características essenciais de uma coisa}
\end{entry}

\begin{entry}{细菌战}{xi4jun1zhan4}{8,11,9}{⽷、⾋、⼽}
  \definition{s.}{guerra biológica}
\end{entry}

\begin{entry}{细致}{xi4zhi4}{8,10}{⽷、⾄}[HSK 4]
  \definition{adj.}{meticuloso; cuidadoso; minucioso | intrincado; delicado}
\end{entry}

\begin{entry}{虾}{xia1}{9}{⾍}
  \definition{s.}{camarão}
\end{entry}

\begin{entry}{下}{xia4}{3}{⼀}[HSK 1,2]
  \definition{adv.}{abaixo | em baixo de}
  \definition{clas.}{para número de vezes para ações}
  \definition{v.}{descer | chegar a (uma decisão, conclusão, etc.) | recusar}
\end{entry}

\begin{entry}{下巴}{xia4ba5}{3,4}{⼀、⼰}
  \definition[个]{s.}{queixo}
\end{entry}

\begin{entry}{下班}{xia4 ban1}{3,10}{⼀、⽟}[HSK 1]
  \definition{v.+compl.}{sair do trabalho}
\end{entry}

\begin{entry}{下边}{xia4bian5}{3,5}{⼀、⾡}[HSK 1]
  \definition{adv.}{em baixo | abaixo | parte de baixo}
\end{entry}

\begin{entry}{下车}{xia4 che1}{3,4}{⼀、⾞}[HSK 1]
  \definition{v.}{descer ou sair (de ônibus, carro, etc.)}
\end{entry}

\begin{entry}{下次}{xia4 ci4}{3,6}{⼀、⽋}[HSK 1]
  \definition{s.}{próxima vez}
\end{entry}

\begin{entry}{下蛋}{xia4dan4}{3,11}{⼀、⾍}
  \definition{v.}{botar ovos}
\end{entry}

\begin{entry}{下个月}{xia4 ge4 yue4}{3,3,4}{⼀、⼈、⽉}[HSK 4]
  \definition{s.}{próximo mês; mês que vem; refere-se ao próximo mês do mês atual}
\end{entry}

\begin{entry}{下海}{xia4hai3}{3,10}{⼀、⽔}
  \definition{v.+compl.}{ir para o mar; (barco) deixar o porto e iniciar uma jornada | ir pescar no mar | tornar-se ator profissional}
\end{entry}

\begin{entry}{下降}{xia4 jiang4}{3,8}{⼀、⾩}[HSK 4]
  \definition{v.}{cair; despencar; declinar; descer; diminuir; ir para baixo}
\end{entry}

\begin{entry}{下课}{xia4 ke4}{3,10}{⼀、⾔}[HSK 1]
  \definition{v.+compl.}{acabar a aula | terminar a aula}
\end{entry}

\begin{entry}{下来}{xia4 lai5}{3,7}{⼀、⽊}[HSK 3]
  \definition{part.}{usado depois de um verbo para indicar que uma ação ou comportamento está se movendo em direção ao falante ou que a ação está continuando ou sendo concluída | usado depois de um adjetivo para indicar que um certo estado começou a aparecer e continuará a se desenvolver.}
  \definition{v.}{descer (para a minha localização) | (colheitas/frutas/vegetais, etc.) ser colhido; estar maduro o suficiente para ser colhido | (período de tempo) acabar; passar; chegar ao fim}
\end{entry}

\begin{entry}{下楼}{xia4 lou2}{3,13}{⼀、⽊}[HSK 4]
  \definition{v.}{descer as escadas}
\end{entry}

\begin{entry}{下面}{xia4 mian4}{3,9}{⼀、⾯}[HSK 3]
  \definition{s.}{em baixo; abaixo; parte de baixo | próximo; seguinte | subordinado; o nível inferior; homens nos níveis inferiores}
  \definition{v.}{cozinhar macarrão}
\end{entry}

\begin{entry}{下去}{xia4 qu4}{3,5}{⼀、⼛}[HSK 3]
  \definition{part.}{usado depois de verbos para indicar de alto a baixo | usado depois de um verbo para indicar continuação}
  \definition{v.}{descer (a partir da minha localização)
continuar
obter; crescer; tornar-se}
\end{entry}

\begin{entry}{下水道}{xia4shui3dao4}{3,4,12}{⼀、⽔、⾡}
  \definition{s.}{esgoto}
\end{entry}

\begin{entry}{下午}{xia4wu3}{3,4}{⼀、⼗}[HSK 1]
  \definition{adv.}{tarde | à tarde | período logo após o meio-dia}
\end{entry}

\begin{entry}{下午茶}{xia4wu3cha2}{3,4,9}{⼀、⼗、⾋}
  \definition{s.}{chá da tarde (normalmente chás com doces)}
\end{entry}

\begin{entry}{下线}{xia4xian4}{3,8}{⼀、⽷}
  \definition{v.}{ficar \emph{offline} | (um produto) sair da linha de montagem | pessoa abaixo de si em um esquema de pirâmide}
\end{entry}

\begin{entry}{下雪}{xia4 xue3}{3,11}{⼀、⾬}[HSK 2]
  \definition[场,次]{s.}{neve}
  \definition{v.+compl.}{nevar}
\end{entry}

\begin{entry}{下旬}{xia4xun2}{3,6}{⼀、⽇}
  \definition{adv.}{última dezena do mês}
\end{entry}

\begin{entry}{下雨}{xia4 yu3}{3,8}{⼀、⾬}[HSK 1]
  \definition{v.+compl.}{chover}
\end{entry}

\begin{entry}{下载}{xia4zai3}{3,10}{⼀、⾞}[HSK 4]
  \definition{v.}{\emph{download}; baixar; salvar informações da \emph{Web} em um dispositivo, como um computador}
\end{entry}

\begin{entry}{下崽}{xia4zai3}{3,12}{⼀、⼭}
  \definition{v.}{dar à luz (animais) | parir}
\end{entry}

\begin{entry}{下周}{xia4 zhou1}{3,8}{⼀、⼝}[HSK 2]
  \definition{s.}{próxima semana}
\end{entry}

\begin{entry}{吓人}{xia4ren2}{6,2}{⼝、⼈}
  \definition{adj.}{apavorante | assustador}
  \definition{v.+compl.}{assustar-se | tomar um susto}
\end{entry}

\begin{entry}{夏季}{xia4 ji4}{10,8}{⼢、⼦}[HSK 4]
  \definition{s.}{verão; segundo trimestre do ano, habitualmente chamado na China de período de três meses, do início do verão ao início do outono, também chamado de ``quarto, quinto e sexto'' meses do calendário lunar}
\end{entry}

\begin{entry}{夏日}{xia4ri4}{10,4}{⼢、⽇}
  \definition{s.}{horário de verão}
\end{entry}

\begin{entry}{夏天}{xia4 tian1}{10,4}{⼢、⼤}[HSK 2]
  \definition[个]{s.}{verão}
\end{entry}

\begin{entry}{仙}{xian1}{5}{⼈}
  \definition{s.}{imortal}
\end{entry}

\begin{entry}{先}{xian1}{6}{⼉}[HSK 1]
  \definition{adv.}{em primeiro lugar | primeiramente | antes do tempo | de antemão}
\end{entry}

\begin{entry}{先不先}{xian1bu4xian1}{6,4,6}{⼉、⼀、⼉}
  \definition{adv.}{(dialeto) antes de tudo | em primeiro lugar}
\end{entry}

\begin{entry}{先到先得}{xian1dao4xian1de2}{6,8,6,11}{⼉、⼑、⼉、⼻}
  \definition{expr.}{primeiro a chegar | primeiro a ser servido}
\end{entry}

\begin{entry}{先进}{xian1jin4}{6,7}{⼉、⾡}[HSK 3]
  \definition{adj.}{avançado}
  \definition{s.}{indivíduo avançado; grupo avançado}
\end{entry}

\begin{entry}{先烈}{xian1lie4}{6,10}{⼉、⽕}
  \definition{s.}{mártir}
\end{entry}

\begin{entry}{先期}{xian1qi1}{6,12}{⼉、⽉}
  \definition{adv.}{antecipadamente}
  \definition{s.}{prematuro | \emph{front-end}}
\end{entry}

\begin{entry}{先生}{xian1sheng5}{6,5}{⼉、⽣}[HSK 1]
  \definition[位]{s.}{senhor | marido | professor | (dialeto) doutor}
\end{entry}

\begin{entry}{先天}{xian1tian1}{6,4}{⼉、⼤}
  \definition{adj.}{congênito | inato | natural}
  \definition{s.}{período embrionário}
\end{entry}

\begin{entry}{先验}{xian1yan4}{6,10}{⼉、⾺}
  \definition{adj.}{(filosofia) a priori}
\end{entry}

\begin{entry}{先有}{xian1you3}{6,6}{⼉、⽉}
  \definition{adj.}{preexistente | anterior}
\end{entry}

\begin{entry}{鲜}{xian1}{14}{⿂}[HSK 4]
  \definition*{s.}{sobrenome Xian}
  \definition{adj.}{fresco; novo; fresco (experiência, comida etc.) |brilhante; de cores vivas | saboroso; delicioso | exuberante; luxuriante}
  \definition{s.}{aves e animais recém-abatidos; vegetais recém-colhidos; frutas, etc. | alimentos aquáticos; geralmente, peixes vivos, camarões, etc., para alimentação}
  \seeref{鲜}{xian3}
\end{entry}

\begin{entry}{鲜花}{xian1 hua1}{14,7}{⿂、⾋}[HSK 4]
  \definition[朵,束,支,捧]{s.}{flor; flores frescas; flores bonitas e frescas}
\end{entry}

\begin{entry}{鲜明}{xian1ming2}{14,8}{⿂、⽇}[HSK 4]
  \definition{adj.}{brilhante (cor) | distinto; bem definido; nítido; claro; característico}
\end{entry}

\begin{entry}{咸}{xian2}{9}{⼝}[HSK 4]
  \definition*{s.}{sobrenome Xian}
  \definition{adj.}{salgado; em conserva; sabor salgado}
  \definition{adv.}{todos; indica a totalidade de um intervalo, equivalente a ``全'' e ``都''}
  \seealsoref{都}{dou1}
  \seealsoref{全}{quan2}
\end{entry}

\begin{entry}{咸菜}{xian2cai4}{9,11}{⼝、⾋}
  \definition{s.}{legumes salgados | \emph{pickles}}
\end{entry}

\begin{entry}{咸淡}{xian2dan4}{9,11}{⼝、⽔}
  \definition{s.}{água salobra | grau de salinidade | salgado e sem sal (sabores)}
\end{entry}

\begin{entry}{咸肉}{xian2rou4}{9,6}{⼝、⾁}
  \definition{s.}{\emph{bacon} | carne curada com sal}
\end{entry}

\begin{entry}{咸涩}{xian2se4}{9,10}{⼝、⽔}
  \definition{s.}{ácido | salgado e amargo}
\end{entry}

\begin{entry}{咸水}{xian2shui3}{9,4}{⼝、⽔}
  \definition{s.}{salmora | água salgada}
\end{entry}

\begin{entry}{咸盐}{xian2yan2}{9,10}{⼝、⽫}
  \definition{s.}{(coloquial) sal | sal de mesa}
\end{entry}

\begin{entry}{咸鱼}{xian2yu2}{9,8}{⼝、⿂}
  \definition{s.}{peixe salgado}
\end{entry}

\begin{entry}{显得}{xian3de5}{9,11}{⽇、⼻}[HSK 3]
  \definition{v.}{parecer; aparecer}
\end{entry}

\begin{entry}{显然}{xian3ran2}{9,12}{⽇、⽕}[HSK 3]
  \definition{adj.}{claro; evidente; óbvio}
  \definition{adv.}{claramente; evidentemente; obviamente}
\end{entry}

\begin{entry}{显示}{xian3shi4}{9,5}{⽇、⽰}[HSK 3]
  \definition{v.}{mostrar | exibir}
\end{entry}

\begin{entry}{显著}{xian3zhu4}{9,11}{⽇、⽬}[HSK 4]
  \definition{adj.}{notável; significativo; notável; extraordinário; muito óbvio; muito claramente demonstrado; muito facilmente visto ou sentido}
\end{entry}

\begin{entry}{猃狁}{xian3yun3}{10,7}{⽝、⽝}
  \definition*{s.}{Termo da dinastia Zhou para uma tribo nômade do norte mais tarde chamou o Xiongnu (匈奴) nas dinastias Qin e Han}
  \seealsoref{匈奴}{xiong1nu2}
\end{entry}

\begin{entry}{鲜}{xian3}{14}{⿂}
  \definition{adj.}{raro; pouco; pequeno;}
  \definition{adv.}{raramente}
  \seeref{鲜}{xian1}
\end{entry}

\begin{entry}{见}{xian4}{4}{⾒}
  \definition{v.}{aparecer | também escrito como 现}
  \seeref{见}{jian4}
  \seeref{现}{xian4}
\end{entry}

\begin{entry}{县}{xian4}{7}{⼛}[HSK 4]
  \definition[个]{s.}{condado; unidade de divisão administrativa}
\end{entry}

\begin{entry}{现}{xian4}{8}{⾒}
  \definition{adj.}{presente | atual}
  \definition{v.}{aparecer}
  \seeref{见}{xian4}
\end{entry}

\begin{entry}{现场}{xian4chang3}{8,6}{⾒、⼟}[HSK 3]
  \definition[个,处]{s.}{cena (de um incidente) | local; lugar; sítio}
\end{entry}

\begin{entry}{现代}{xian4dai4}{8,5}{⾒、⼈}[HSK 3]
  \definition*{s.}{Hyundai, empresa sul-coreana}
  \definition{adj.}{moderno; contemporâneo}
  \definition{s.}{tempos modernos; era contemporânea}
\end{entry}

\begin{entry}{现货}{xian4huo4}{8,8}{⾒、⾙}
  \definition{s.}{produtos à vista}
\end{entry}

\begin{entry}{现货的}{xian4huo4 de5}{8,8,8}{⾒、⾙、⽩}
  \definition{s.}{produtos em estoque}
\end{entry}

\begin{entry}{现金}{xian4jin1}{8,8}{⾒、⾦}[HSK 3]
  \definition[笔]{s.}{dinheiro; dinheiro vivo | reserva de dinheiro em um banco}
\end{entry}

\begin{entry}{现实}{xian4shi2}{8,8}{⾒、⼧}[HSK 3]
  \definition{adj.}{real; atual}
  \definition[个]{s.}{realidade; atualidade}
\end{entry}

\begin{entry}{现象}{xian4xiang4}{8,11}{⾒、⾗}[HSK 3]
  \definition[个,种]{s.}{aparência (das coisas); fenômeno}
\end{entry}

\begin{entry}{现有}{xian4you3}{8,6}{⾒、⽉}
  \definition{adj.}{disponível atualmente | atualmente existente}
\end{entry}

\begin{entry}{现在}{xian4zai4}{8,6}{⾒、⼟}[HSK 1]
  \definition{adv.}{agora | neste momento}
\end{entry}

\begin{entry}{现抓}{xian4zhua1}{8,7}{⾒、⼿}
  \definition{v.}{improvisar}
\end{entry}

\begin{entry}{现做}{xian4zuo4}{8,11}{⾒、⼈}
  \definition{adj.}{fresco}
  \definition{v.}{fazer (comida) no local}
\end{entry}

\begin{entry}{线}{xian4}{8}{⽷}[HSK 3]
  \definition{clas.}{para coisas abstratas, o número é limitado a ``一''}
  \definition{s.}{fio; corda; arame | linha | feito de fio de algodão | algo em forma de linha, fio, etc. | rota; linha | linha de demarcação; limite | beira; borda | linha ideológica e política | pista; fio}
\end{entry}

\begin{entry}{线香}{xian4xiang1}{8,9}{⽷、⾹}
  \definition{s.}{bastão ou vareta de incenso}
\end{entry}

\begin{entry}{限制}{xian4zhi4}{8,8}{⾩、⼑}[HSK 4]
  \definition{s.}{limite; restrição; confinamento}
  \definition{v.}{limitar; adstringir; restringir; confinar; fechar em (sobre)}
\end{entry}

\begin{entry}{宪法法院}{xian4fa3fa3yuan4}{9,8,8,9}{⼧、⽔、⽔、⾩}
  \definition{s.}{tribunal constitucional}
\end{entry}

\begin{entry}{宪政}{xian4zheng4}{9,9}{⼧、⽁}
  \definition{s.}{governo constitucional}
\end{entry}

\begin{entry}{宪制}{xian4zhi4}{9,8}{⼧、⼑}
  \definition{adj.}{constitucional}
  \definition{s.}{sistema de governo constitucional}
\end{entry}

\begin{entry}{陷入}{xian4ru4}{10,2}{⾩、⼊}
  \definition{v.}{afundar | ser pego em | pousar (em uma situação)}
\end{entry}

\begin{entry}{羡慕}{xian4mu4}{12,14}{⽺、⼼}
  \definition{v.}{invejar | admirar}
\end{entry}

\begin{entry}{乡巴佬}{xiang1ba1lao3}{3,4,8}{⼄、⼰、⼈}
  \definition{s.}{aldeão | caipira}
\end{entry}

\begin{entry}{乡村}{xiang1cun1}{3,7}{⼄、⽊}
  \definition{adj.}{rural | rústico}
  \definition{s.}{vila | campo}
\end{entry}

\begin{entry}{相比}{xiang1 bi3}{9,4}{⽬、⽐}[HSK 3]
  \definition{v.}{combinar; comparar com |comparar uma coisa com outra, usar uma coisa como padrão para ver as características de outra coisa ou para obter um ponto de vista}
\end{entry}

\begin{entry}{相处}{xiang1chu3}{9,5}{⽬、⼡}[HSK 4]
  \definition{v.}{dar-se bem; viver juntos; dar-se bem (uns com os outros); viver uns com os outros; entrar em contato uns com os outros, tratar uns aos outros}
\end{entry}

\begin{entry}{相当}{xiang1dang1}{9,6}{⽬、⼹}[HSK 3]
  \definition{adj.}{adequado; ajustado; apropriado}
  \definition{adv.}{bastante; razoavelmente; consideravelmente}
  \definition{v.}{combinar; equilibrar; corresponder a; ser aproximadamente igual a; ser compatível com}
\end{entry}

\begin{entry}{相反}{xiang1fan3}{9,4}{⽬、⼜}[HSK 4]
  \definition{adj.}{oposto; contrário; dois aspectos das coisas são contraditórios e mutuamente exclusivos}
  \definition{conj.}{pelo contrário; usado no início ou no meio de uma frase para indicar uma contradição de significado com o que foi dito anteriormente.}
\end{entry}

\begin{entry}{相关}{xiang1guan1}{9,6}{⽬、⼋}[HSK 3]
  \definition{v.}{mutuamente relacionados; inter-relacionados}
\end{entry}

\begin{entry}{相互}{xiang1 hu4}{9,4}{⽬、⼆}[HSK 3]
  \definition{adj.}{mútuo; recíproco}
  \definition{adv.}{mutuamente; um ao outro}
\end{entry}

\begin{entry}{相聚}{xiang1ju4}{9,14}{⽬、⽿}
  \definition{v.}{reunir-se | montar}
\end{entry}

\begin{entry}{相亲}{xiang1qin1}{9,9}{⽬、⼇}
  \definition{s.}{encontro às cegas | entrevista arranjada para avaliar a proposta de um parceiro de casamento | apegar-se profundamente um ao outro}
\end{entry}

\begin{entry}{相思病}{xiang1si1bing4}{9,9,10}{⽬、⼼、⽧}
  \definition{s.}{saudade de amor}
\end{entry}

\begin{entry}{相似}{xiang1si4}{9,6}{⽬、⼈}[HSK 3]
  \definition{v.}{assemelhar-se; ser semelhante; ser igual}
\end{entry}

\begin{entry}{相同}{xiang1tong2}{9,6}{⽬、⼝}[HSK 2]
  \definition{adj.}{igual | idêntico | o mesmo}
\end{entry}

\begin{entry}{相信}{xiang1xin4}{9,9}{⽬、⼈}[HSK 2]
  \definition{v.}{acreditar | estar convencido | aceitar como verdadeiro}
\end{entry}

\begin{entry}{相宜}{xiang1yi2}{9,8}{⽬、⼧}
  \definition{adj.}{adequado | apropriado}
  \definition{v.}{ser adequado ou apropriado}
\end{entry}

\begin{entry}{相遇}{xiang1yu4}{9,12}{⽬、⾡}
  \definition{v.}{encontrar (reunião, encontro, etc.)}
\end{entry}

\begin{entry}{香}{xiang1}{9}{⾹}[HSK 3][Kangxi 186]
  \definition*{s.}{sobrenome Xiang}
  \definition{adj.}{aromático; perfumado; fragrante; cheiroso | saboroso; saboroso; delicioso; apetitoso | com gosto; com bom apetite | (sono) profundo | popular; bem-vindo}
  \definition[束,根,炷]{s.}{especiaria; perfume; fragrância; aromatizante | incenso | relacionado a mulheres ou às próprias mulheres}
\end{entry}

\begin{entry}{香槟酒}{xiang1bin1jiu3}{9,14,10}{⾹、⽊、⾣}
  \definition[杯]{s.}{(empréstimo linguístico) \emph{champagne}}
\end{entry}

\begin{entry}{香波}{xiang1bo1}{9,8}{⾹、⽔}
  \definition{s.}{xampu}
\end{entry}

\begin{entry}{香肠}{xiang1chang2}{9,7}{⾹、⾁}
  \definition[根]{s.}{salsicha}
\end{entry}

\begin{entry}{香港}{xiang1gang3}{9,12}{⾹、⽔}
  \definition*{s.}{Hong Kong}
  \seealsoref{香港岛}{xiang1gang3 dao3}
\end{entry}

\begin{entry}{香港岛}{xiang1gang3 dao3}{9,12,7}{⾹、⽔、⼭}
  \definition*{s.}{Ilha de Hong Kong}
  \seealsoref{香港}{xiang1gang3}
\end{entry}

\begin{entry}{香蕉}{xiang1jiao1}{9,15}{⾹、⾋}[HSK 3]
  \definition[枝,根,个,把,串,束,弓]{s.}{banana}
\end{entry}

\begin{entry}{香炉}{xiang1lu2}{9,8}{⾹、⽕}
  \definition{s.}{incensário (para queimar incenso) | queimador de incenso | insensório, turíbulo}
\end{entry}

\begin{entry}{香气}{xiang1qi4}{9,4}{⾹、⽓}
  \definition{s.}{fragrância | aroma | incenso}
\end{entry}

\begin{entry}{香味}{xiang1wei4}{9,8}{⾹、⼝}
  \definition[股]{s.}{fragrância | cheiro doce}
\end{entry}

\begin{entry}{香蕈}{xiang1xun4}{9,15}{⾹、⾋}
  \definition{s.}{\emph{shiitake}, cogumelo comestível}
\end{entry}

\begin{entry}{香烟}{xiang1yan1}{9,10}{⾹、⽕}
  \definition[支,条]{s.}{cigarro | fumaça de incenso queimado}
\end{entry}

\begin{entry}{香艳}{xiang1yan4}{9,10}{⾹、⾊}
  \definition{adj.}{atraente | erótico | romântico}
\end{entry}

\begin{entry}{香皂}{xiang1zao4}{9,7}{⾹、⽩}
  \definition{s.}{sabonete | sabonete perfumado}
\end{entry}

\begin{entry}{箱}{xiang1}{15}{⾋}[HSK 4]
  \definition{s.}{caixa; estojo; baú | qualquer coisa no formato de caixa}
\end{entry}

\begin{entry}{箱子}{xiang1 zi5}{15,3}{⾋、⼦}[HSK 4]
  \definition[个,只]{s.}{baú; caixa; estojo; maleta; pasta executiva}
\end{entry}

\begin{entry}{享受}{xiang3shou4}{8,8}{⼇、⼜}
  \definition[种]{s.}{prazer}
  \definition{v.}{desfrutar | viver}
\end{entry}

\begin{entry}{响}{xiang3}{9}{⼝}[HSK 2]
  \definition{adj.}{barulhento}
  \definition[声,阵]{s.}{som | barulho | eco}
  \definition{v.}{fazer um som | soar | tocar}
\end{entry}

\begin{entry}{想}{xiang3}{13}{⼼}[HSK 1]
  \definition{v.}{acreditar | sentir falta (sentir-se melancólico com a ausência de alguém ou algo) | supor | pensar | querer | desejar}
\end{entry}

\begin{entry}{想到}{xiang3 dao4}{13,8}{⼼、⼑}[HSK 2]
  \definition{v.}{pensar em | trazer à mente | ter no coração}
\end{entry}

\begin{entry}{想法}{xiang3 fa3}{13,8}{⼼、⽔}[HSK 2]
  \definition[个]{s.}{noção | opinião | jeito de pensar}
  \definition{s.}{maneira de pensar | opinião | noção}
  \definition{v.}{pensar em uma maneira (de fazer algo)}
\end{entry}

\begin{entry}{想念}{xiang3nian4}{13,8}{⼼、⼼}[HSK 4]
  \definition{v.}{sentir falta; pensar em; lembrar com carinho; ficar doente por; desejar ver novamente; lembrar com saudade}
\end{entry}

\begin{entry}{想起}{xiang3 qi3}{13,10}{⼼、⾛}[HSK 2]
  \definition{v.}{recordar | lembrar | pensar em | trazer à mente | cruzar pelos pensamentos de alguém | passar pelo pensamento de alguém}
\end{entry}

\begin{entry}{想想看}{xiang3xiang3kan4}{13,13,9}{⼼、⼼、⽬}
  \definition{v.}{pensar sobre isso}
\end{entry}

\begin{entry}{想象}{xiang3xiang4}{13,11}{⼼、⾗}[HSK 4]
  \definition[个]{s.}{imaginação; refere-se ao processo mental de processamento e transformação de representações armazenadas na mente para formar novas imagens}
  \definition{v.}{imaginar; vislumbrar; visualizar; refere-se a ter uma imagem concreta de algo que não está na frente dos olhos}
\end{entry}

\begin{entry}{向}{xiang4}{6}{⼝}[HSK 2]
  \definition*{s.}{sobrenome Xiang}
  \definition{prep.}{para}
  \definition{v.}{enfrentar | virar para | apoiar}
\end{entry}

\begin{entry}{向汪}{xiang4wang1}{6,7}{⼝、⽔}
  \definition{v.}{esperar que}
\end{entry}

\begin{entry}{向往}{xiang4wang3}{6,8}{⼝、⼻}
  \definition{v.}{ansiar por | esperar ansiosamente por}
\end{entry}

\begin{entry}{相机}{xiang4 ji1}{9,6}{⽬、⽊}[HSK 2]
  \definition[台,个]{s.}{câmera | máquina fotográfica}
  \definition{v.}{ficar atento a uma oportunidade}
\end{entry}

\begin{entry}{相片}{xiang4 pian4}{9,4}{⽬、⽚}[HSK 4]
  \definition[张]{s.}{foto; fotografia; uma imagem de uma pessoa ou objeto feita pela exposição de papel fotográfico a um negativo fotográfico e, em seguida, revelando e fixando a imagem.}
\end{entry}

\begin{entry}{项}{xiang4}{9}{⾴}[HSK 4]
  \definition*{s.}{sobrenome Xiang}
  \definition{clas.}{para itens discriminados; taxonomia}
  \definition{s.}{nuca (do pescoço); a parte de trás do pescoço |
soma (de dinheiro); fundos para fins especiais |
termo; em álgebra, significa uma única fórmula que não é unida por um sinal de mais ou de menos | item}
\end{entry}

\begin{entry}{项目}{xiang4mu4}{9,5}{⾴、⽬}[HSK 4]
  \definition{s.}{evento | item; projeto; trabalhos de engenharia, acadêmicos, etc., de conteúdo específico}
\end{entry}

\begin{entry}{像}{xiang4}{13}{⼈}[HSK 2]
  \definition{s.}{imagem | retrato | aparência}
  \definition{v.}{assemelhar-se | ser como}
\end{entry}

\begin{entry}{消防}{xiao1fang2}{10,6}{⽔、⾩}
  \definition{s.}{combate a incêncios | controle de incêndios}
\end{entry}

\begin{entry}{消防员}{xiao1fang2yuan2}{10,6,7}{⽔、⾩、⼝}
  \definition{s.}{bombeiro}
\end{entry}

\begin{entry}{消费}{xiao1fei4}{10,9}{⽔、⾙}[HSK 3]
  \definition{v.}{gastar; consumir | consumir (recursos naturais)}
\end{entry}

\begin{entry}{消化}{xiao1hua4}{10,4}{⽔、⼔}[HSK 4]
  \definition{v.}{digerir (alimentos) | digerir (conhecimento); pensar e absorver; uma metáfora para a compreensão total de novos conhecimentos ou informações e a capacidade de transformá-los em algo que possa ser usado}
\end{entry}

\begin{entry}{消失}{xiao1shi1}{10,5}{⽔、⼤}[HSK 3]
  \definition{v.}{desaparecer; desvanecer; dissolver; dissipar; evaporar; sumir}
\end{entry}

\begin{entry}{消息}{xiao1xi5}{10,10}{⽔、⼼}[HSK 3]
  \definition[个,条,篇]{s.}{notícias; informação}
\end{entry}

\begin{entry}{销售}{xiao1shou4}{12,11}{⾦、⼝}[HSK 4]
  \definition{v.}{vender; comercializar}
\end{entry}

\begin{entry}{嚣张}{xiao1zhang1}{18,7}{⼝、⼸}
  \definition{adj.}{desenfreado | arrogante | agressivo}
\end{entry}

\begin{entry}{小}{xiao3}{3}{⼩}[HSK 1,2][Kangxi 42]
  \definition{adj.}{pequeno | jovem}
\end{entry}

\begin{entry}{小白菜}{xiao3bai2cai4}{3,5,11}{⼩、⽩、⾋}
  \definition[棵]{s.}{\emph{bok choy} | couve chinesa}
\end{entry}

\begin{entry}{小吃}{xiao3chi1}{3,6}{⼩、⼝}[HSK 4]
  \definition{s.}{lanche; petiscos; comida com especialidades locais, não muito para uma porção | prato frio; prato feito; cortes de frios na culinária ocidental | pratos pequenos e baratos; pratos simples em restaurantes com porções pequenas e preços baixos}
\end{entry}

\begin{entry}{小狗}{xiao3 gou3}{3,8}{⼩、⽝}
  \definition{s.}{filhote de cachorro}
\end{entry}

\begin{entry}{小孩儿}{xiao3hai2r5}{3,9,2}{⼩、⼦、⼉}[HSK 1]
  \definition[个]{s.}{criança | bebê}
\end{entry}

\begin{entry}{小伙子}{xiao3huo3zi5}{3,6,3}{⼩、⼈、⼦}[HSK 4]
  \definition[个]{s.}{rapaz jovem; jovem colega}
\end{entry}

\begin{entry}{小姐}{xiao3jie5}{3,8}{⼩、⼥}[HSK 1]
  \definition[个,位]{s.}{senhorita | jovem senhora | (gíria) prostituta}
\end{entry}

\begin{entry}{小朋友}{xiao3peng2you3}{3,8,4}{⼩、⽉、⼜}[HSK 1]
  \definition{s.}{criança | [termo de tratamento usado por um adulto para uma criança] amiguinho}
\end{entry}

\begin{entry}{小气鬼}{xiao3qi4gui3}{3,4,9}{⼩、⽓、⿁}
  \definition{adj.}{avarento | mão-de-vaca | miserável | pão-duro}
\end{entry}

\begin{entry}{小区}{xiao3qu1}{3,4}{⼩、⼖}
  \definition{s.}{conjunto habitacional, comunidade, bairro | célula (telecomunicações)}
\end{entry}

\begin{entry}{小声}{xiao3 sheng1}{3,7}{⼩、⼠}[HSK 2]
  \definition{v.}{falar em voz baixa | sussurar}
\end{entry}

\begin{entry}{小时}{xiao3shi2}{3,7}{⼩、⽇}[HSK 1]
  \definition{adv.}{hora | para horas}
  \definition[个]{s.}{hora}
\end{entry}

\begin{entry}{小时候}{xiao3 shi2 hou5}{3,7,10}{⼩、⽇、⼈}[HSK 2]
  \definition{s.}{na infância | quando alguém era jovem}
\end{entry}

\begin{entry}{小树}{xiao3shu4}{3,9}{⼩、⽊}
  \definition[棵]{s.}{muda | arbusto | árvore pequena}
\end{entry}

\begin{entry}{小说}{xiao3shuo1}{3,9}{⼩、⾔}[HSK 2]
  \definition[本,部]{s.}{romance | ficção}
\end{entry}

\begin{entry}{小腿}{xiao3tui3}{3,13}{⼩、⾁}
  \definition{s.}{perna (do joelho ao calcanhar) | haste}
\end{entry}

\begin{entry}{小屋}{xiao3wu1}{3,9}{⼩、⼫}
  \definition{s.}{cabana | chalé | cabine}
\end{entry}

\begin{entry}{小小}{xiao3xiao3}{3,3}{⼩、⼩}
  \definition{adj.}{muito pequeno}
\end{entry}

\begin{entry}{小心}{xiao3xin1}{3,4}{⼩、⼼}[HSK 2]
  \definition{adj.}{cuidado}
\end{entry}

\begin{entry}{小型}{xiao3 xing2}{3,9}{⼩、⼟}[HSK 4]
  \definition{adj.}{de tamanho pequeno; em pequena escala; miniatura; tipo pequeno; tamanho de bolso; tipo compacto}
  \definition{s.}{(Mediterrâneo) escunas, pequenos veleiros de pesca ou turismo | pequeno \emph{rover} lunar (duas pessoas)}
\end{entry}

\begin{entry}{小学}{xiao3xue2}{3,8}{⼩、⼦}[HSK 1]
  \definition{s.}{escola ensino fundamental}
\end{entry}

\begin{entry}{小学生}{xiao3xue2sheng1}{3,8,5}{⼩、⼦、⽣}[HSK 1]
  \definition{s.}{aluno, estudante de escola primária}
\end{entry}

\begin{entry}{小洋白菜}{xiao3 yang2bai2cai4}{3,9,5,11}{⼩、⽔、⽩、⾋}
  \definition{s.}{couve de bruxelas}
\end{entry}

\begin{entry}{小众}{xiao3zhong4}{3,6}{⼩、⼈}
  \definition{s.}{minoria da população | nicho (mercado, etc.)}
\end{entry}

\begin{entry}{小组}{xiao3 zu3}{3,8}{⼩、⽷}[HSK 2]
  \definition[个]{s.}{grupo}
\end{entry}

\begin{entry}{哮喘}{xiao4chuan3}{10,12}{⼝、⼝}
  \definition{s.}{asma}
\end{entry}

\begin{entry}{效果}{xiao4guo3}{10,8}{⽁、⽊}[HSK 3]
  \definition[种,个]{s.}{efeito; resultado | efeitos sonoros; vários sons ou fenômenos naturais criados para combinar com o enredo em dramas e filmes, como vento e chuva, tiros, fogo, neve, etc.}
\end{entry}

\begin{entry}{效率}{xiao4lv4}{10,11}{⽁、⽞}[HSK 4]
  \definition[种]{s.}{eficiência; produtividade}
\end{entry}

\begin{entry}{校}{xiao4}{10}{⽊}
  \definition[所]{s.}{oficial militar | escola}
  \seeref{校}{jiao4}
\end{entry}

\begin{entry}{校服}{xiao4fu2}{10,8}{⽊、⽉}
  \definition{s.}{uniforme escolar}
\end{entry}

\begin{entry}{校规}{xiao4gui1}{10,8}{⽊、⾒}
  \definition{s.}{regras e regulamentos escolares}
\end{entry}

\begin{entry}{校监}{xiao4jian1}{10,10}{⽊、⽫}
  \definition{s.}{diretor | supervisor (de escola)}
\end{entry}

\begin{entry}{校园}{xiao4 yuan2}{10,7}{⽊、⼞}[HSK 2]
  \definition{s.}{campus}
\end{entry}

\begin{entry}{校长}{xiao4zhang3}{10,4}{⽊、⾧}[HSK 2]
  \definition[个,位,名]{s.}{diretor de escola | reitor (universidade)}
\end{entry}

\begin{entry}{笑}{xiao4}{10}{⽵}[HSK 1]
  \definition{v.}{sorrir | rir | rir de}
\end{entry}

\begin{entry}{笑话儿}{xiao4 hua4r5}{10,8,2}{⽵、⾔、⼉}[HSK 2]
  \definition{s.}{piada | gracejo}
\end{entry}

\begin{entry}{笑话}{xiao4hua5}{10,8}{⽵、⾔}[HSK 2]
  \definition{adj.}{absurdo | ridículo}
  \definition[个]{s.}{piada | brincadeira}
  \definition{v.}{rir de algo | zombar | ridicularizar}
\end{entry}

\begin{entry}{笑容}{xiao4rong2}{10,10}{⽵、⼧}
  \definition[副]{s.}{sorriso | expressão sorridente}
\end{entry}

\begin{entry}{些}{xie1}{8}{⼆}[HSK 4]
  \definition{adv.}{um pouco; um pouco mais; usado após um adjetivo ou parte de um verbo para indicar uma pequena quantidade, equivalente a ``一点儿''}
  \definition{clas.}{alguns; um pouco; denota uma quantidade indefinida}
  \seealsoref{一点儿}{yi4dian3r5}
\end{entry}

\begin{entry}{些许}{xie1xu3}{8,6}{⼆、⾔}
  \definition{num.}{um pouco}
\end{entry}

\begin{entry}{斜阳}{xie2yang2}{11,6}{⽃、⾩}
  \definition{s.}{sol poente}
\end{entry}

\begin{entry}{谐}{xie2}{11}{⾔}
  \definition{adj.}{harmonioso | humorístico}
\end{entry}

\begin{entry}{鞋}{xie2}{15}{⾰}[HSK 2]
  \definition[双,只]{s.}{sapatos}
\end{entry}

\begin{entry}{写}{xie3}{5}{⼍}[HSK 1]
  \definition{v.}{escrever}
\end{entry}

\begin{entry}{写意}{xie3yi4}{5,13}{⼍、⼼}
  \definition{s.}{estilo de pintura chinesa à mão livre, caracterizado por traços ousados em vez de detalhes precisos}
  \definition{v.}{sugerir (em vez de descrever em detalhes)}
  \seeref{写意}{xie4yi4}
\end{entry}

\begin{entry}{写照}{xie3zhao4}{5,13}{⼍、⽕}
  \definition{s.}{retrato}
\end{entry}

\begin{entry}{写真}{xie3zhen1}{5,10}{⼍、⼗}
  \definition{s.}{retrato}
  \definition{v.}{descrever algo com precisão}
\end{entry}

\begin{entry}{写字}{xie3zi4}{5,6}{⼍、⼦}
  \definition{v.}{escrever (à mão) | praticar caligrafia}
\end{entry}

\begin{entry}{写字匠}{xie3zi4 jiang4}{5,6,6}{⼍、⼦、⼕}
  \definition{s.}{calígrafo}
\end{entry}

\begin{entry}{写作}{xie3zuo4}{5,7}{⼍、⼈}[HSK 3]
  \definition{s.}{escrita; redação; composição}
  \definition{v.}{escrever artigos; escrever livros, etc.; também se refere especificamente à criação de obras literárias}
\end{entry}

\begin{entry}{血}{xie3}{6}{⾎}[Kangxi 143]
  \seeref{血}{xue4}
\end{entry}

\begin{entry}{写意}{xie4yi4}{5,13}{⼍、⼼}
  \definition{adj.}{confortável | agradável | relaxado}
  \seeref{写意}{xie3yi4}
\end{entry}

\begin{entry}{泄气}{xie4qi4}{8,4}{⽔、⽓}
  \definition{adj.}{decepcionante | frustrante | patético}
  \definition{v.+compl.}{perder o coração | sentir-se desencorajado | ficar desanimado}
\end{entry}

\begin{entry}{谢病}{xie4bing4}{12,10}{⾔、⽧}
  \definition{v.}{desculpar-se por causa de doença}
\end{entry}

\begin{entry}{谢恩}{xie4'en1}{12,10}{⾔、⼼}
  \definition{v.}{agradecer a alguém pelo favor (especialmente imperador ou oficial superior)}
\end{entry}

\begin{entry}{谢媒}{xie4mei2}{12,12}{⾔、⼥}
  \definition{v.}{agradecer ao casamenteiro}
\end{entry}

\begin{entry}{谢世}{xie4shi4}{12,5}{⾔、⼀}
  \definition{v.}{morrer | falecer}
\end{entry}

\begin{entry}{谢天谢地}{xie4tian1xie4di4}{12,4,12,6}{⾔、⼤、⾔、⼟}
  \definition{expr.}{agradecer a Deus | agradecer aos céus}
\end{entry}

\begin{entry}{谢谢}{xie4xie5}{12,12}{⾔、⾔}[HSK 1]
  \definition{interj.}{Obrigado!}
  \definition{v.}{agradecer}
\end{entry}

\begin{entry}{谢意}{xie4yi4}{12,13}{⾔、⼼}
  \definition{s.}{gratidão}
\end{entry}

\begin{entry}{心}{xin1}{4}{⼼}[HSK 3][Kangxi 61]
  \definition*{s.}{Xin, uma das mansões lunares}
  \definition[颗]{s.}{o coração | coração; mente; sentimento; intenção | centro; núcleo}
\end{entry}

\begin{entry}{心机}{xin1ji1}{4,6}{⼼、⽊}
  \definition{s.}{pensamento | esquema}
\end{entry}

\begin{entry}{心里}{xin1 li3}{4,7}{⼼、⾥}[HSK 2]
  \definition[把]{s.}{no coração | no coração de alguém | na mente}
\end{entry}

\begin{entry}{心理}{xin1li3}{4,11}{⼼、⽟}[HSK 4]
  \definition{adj.}{psicológico}
  \definition{s.}{mentalidade; refere-se à reflexão da mente humana sobre coisas objetivas, incluindo sensação, percepção, memória, pensamento e emoções | psicologia}
\end{entry}

\begin{entry}{心情}{xin1qing2}{4,11}{⼼、⼼}[HSK 2]
  \definition{s.}{humor | sentimento | estado de espírito}
\end{entry}

\begin{entry}{心声}{xin1sheng1}{4,7}{⼼、⼠}
  \definition{s.}{desejo sincero | voz interior | aspiração}
\end{entry}

\begin{entry}{心疼}{xin1teng2}{4,10}{⼼、⽧}
  \definition{adj.}{angustiado}
  \definition{v.}{sentir pena de alguém | arrepender-se | ressentir-se | ficar angustiado}
\end{entry}

\begin{entry}{心中}{xin1zhong1}{4,4}{⼼、⼁}[HSK 2]
  \definition{adv.}{nos pensamentos | no coração}
  \definition{s.}{ponto central}
\end{entry}

\begin{entry}{芯片}{xin1pian4}{7,4}{⾋、⽚}
  \definition{s.}{chip de computador | microchip}
\end{entry}

\begin{entry}{辛苦}{xin1ku3}{7,8}{⾟、⾋}
  \definition{adj.}{exaustivo | duro | árduo}
  \definition{s.}{dificuldades}
  \definition{v.}{trabalhar duro | ter muitos problemas}
\end{entry}

\begin{entry}{新}{xin1}{13}{⽄}[HSK 1]
  \definition*{s.}{sobrenome Xin | abreviação de Xinjiang (新疆) | abreviação de Singapura (新加坡)}
  \definition{adj.}{novo}
  \definition{adv.}{recentemente}
  \definition{pref.}{(química) meso-}
  \seealsoref{新加坡}{xin1jia1po1}
  \seealsoref{新疆}{xin1jiang1}
\end{entry}

\begin{entry}{新加坡}{xin1jia1po1}{13,5,8}{⽄、⼒、⼟}
  \definition*{s.}{Singapura}
\end{entry}

\begin{entry}{新疆}{xin1jiang1}{13,19}{⽄、⼸}
  \definition*{s.}{Xinjiang}
\end{entry}

\begin{entry}{新疆维吾尔自治区}{xin1jiang1 wei2wu2'er3 zi4zhi4qu1}{13,19,11,7,5,6,8,4}{⽄、⼸、⽷、⼝、⼩、⾃、⽔、⼖}
  \definition*{s.}{Região Autônoma Uigur de Xinjiang}
\end{entry}

\begin{entry}{新郎}{xin1lang2}{13,8}{⽄、⾢}[HSK 4]
  \definition[位,个]{s.}{noivo; homens no momento do casamento}
\end{entry}

\begin{entry}{新年}{xin1nian2}{13,6}{⽄、⼲}[HSK 1]
  \definition*[个]{s.}{Ano Novo}
\end{entry}

\begin{entry}{新娘}{xin1niang2}{13,10}{⽄、⼥}[HSK 4]
  \definition[位,个]{s.}{noiva; a mulher no momento do casamento}
  \seealsoref{新娘子}{xin1niang2zi5}
\end{entry}

\begin{entry}{新娘服装}{xin1niang2 fu2zhuang1}{13,10,8,12}{⽄、⼥、⽉、⾐}
  \definition{s.}{roupas de noiva}
\end{entry}

\begin{entry}{新娘子}{xin1niang2zi5}{13,10,3}{⽄、⼥、⼦}
  \definition{s.}{noiva}
  \seealsoref{新娘}{xin1niang2}
\end{entry}

\begin{entry}{新闻}{xin1wen2}{13,9}{⽄、⾨}[HSK 2]
  \definition[条,个]{s.}{notícia}
\end{entry}

\begin{entry}{新鲜}{xin1xian1}{13,14}{⽄、⿂}
  \definition{adj.}{fresco (experiência, alimento, etc.)}
  \definition{s.}{frescor}
\end{entry}

\begin{entry}{新型}{xin1 xing2}{13,9}{⽄、⼟}[HSK 4]
  \definition[种]{s.}{ultimo modelo; novo tipo; novo padrão; novo estilo}
\end{entry}

\begin{entry}{信}{xin4}{9}{⼈}[HSK 2,3]
  \definition*{s.}{sobrenome Xin}
  \definition{adj.}{verdade}
  \definition{adv.}{à vontade; ao acaso; sem plano}
  \definition[封,个,张]{s.}{carta; correio
mensagem; palavra; informação
sinal; evidência
confiança; fé
fusível
arsênico}
  \definition{v.}{acreditar; fazer um balanço; dar crédito | professar fé em; acreditar em}
\end{entry}

\begin{entry}{信访}{xin4fang3}{9,6}{⼈、⾔}
  \definition{s.}{carta de reclamação | carta de petição}
  \seealsoref{上访}{shang4fang3}
\end{entry}

\begin{entry}{信封}{xin4feng1}{9,9}{⼈、⼨}[HSK 3]
  \definition[个]{s.}{envelope de carta}
\end{entry}

\begin{entry}{信号}{xin4hao4}{9,5}{⼈、⼝}[HSK 2]
  \definition[个]{s.}{sinal | ponte de sinalização}
\end{entry}

\begin{entry}{信经}{xin4jing1}{9,8}{⼈、⽷}
  \definition[个]{s.}{crença | credo (seção da missa católica)}
\end{entry}

\begin{entry}{信任}{xin4ren4}{9,6}{⼈、⼈}[HSK 3]
  \definition[个]{s.}{confiança; certeza; convicção}
  \definition{v.}{confiar; ter confiança em}
\end{entry}

\begin{entry}{信息}{xin4xi1}{9,10}{⼈、⼼}[HSK 2]
  \definition[个,条]{s.}{notícias | informação | mensagem}
\end{entry}

\begin{entry}{信心}{xin4xin1}{9,4}{⼈、⼼}[HSK 2]
  \definition[个]{s.}{confiança | fé (em alguém ou algo)}
\end{entry}

\begin{entry}{信用}{xin4yong4}{9,5}{⼈、⽤}
  \definition{s.}{crédito (comércio)}
\end{entry}

\begin{entry}{信用卡}{xin4yong4ka3}{9,5,5}{⼈、⽤、⼘}[HSK 2]
  \definition[些]{s.}{cartão de crédito}
\end{entry}

\begin{entry}{兴}{xing1}{6}{⼋}
  \definition*{s.}{sobrenome Xing}
  \definition{adv.}{talvez (dialeto)}
  \definition{v.}{subir | florescer | tornar-se popular | começar | encorajar | levantar-se | (frequentemente usado em negativas) permitir (dialeto)}
  \seeref{兴}{xing4}
\end{entry}

\begin{entry}{兴奋}{xing1fen4}{6,8}{⼋、⼤}[HSK 4]
  \definition{adj.}{animado; excitante; empolgante;}
  \definition{s.}{excitação; empolgação}
  \definition{v.}{excitar; intoxicar}
\end{entry}

\begin{entry}{星表}{xing1biao3}{9,8}{⽇、⾐}
  \definition{s.}{catálogo de estrelas}
\end{entry}

\begin{entry}{星辰}{xing1chen2}{9,7}{⽇、⾠}
  \definition{s.}{estrelas}
\end{entry}

\begin{entry}{星火}{xing1huo3}{9,4}{⽇、⽕}
  \definition{s.}{trilha de meteoro (usada principalmente em expressões como 急如星火) | faísca}
\end{entry}

\begin{entry}{星期}{xing1qi1}{9,12}{⽇、⽉}[HSK 1]
  \definition[个]{s.}{semana}
\end{entry}

\begin{entry}{星期二}{xing1qi1'er4}{9,12,2}{⽇、⽉、⼆}[HSK 1]
  \definition{s.}{terça-feira}
\end{entry}

\begin{entry}{星期六}{xing1qi1liu4}{9,12,4}{⽇、⽉、⼋}[HSK 1]
  \definition{s.}{sábado}
\end{entry}

\begin{entry}{星期日}{xing1qi1ri4}{9,12,4}{⽇、⽉、⽇}[HSK 1]
  \definition{s.}{domingo}
  \seealsoref{星期天}{xing1qi1tian1}
\end{entry}

\begin{entry}{星期三}{xing1qi1san1}{9,12,3}{⽇、⽉、⼀}[HSK 1]
  \definition{s.}{quarta-feira}
\end{entry}

\begin{entry}{星期四}{xing1qi1si4}{9,12,5}{⽇、⽉、⼞}[HSK 1]
  \definition{s.}{quinta-feira}
\end{entry}

\begin{entry}{星期天}{xing1qi1tian1}{9,12,4}{⽇、⽉、⼤}[HSK 1]
  \definition{s.}{domingo}
  \seealsoref{星期日}{xing1qi1ri4}
\end{entry}

\begin{entry}{星期五}{xing1qi1wu3}{9,12,4}{⽇、⽉、⼆}[HSK 1]
  \definition{s.}{sexta-feira}
\end{entry}

\begin{entry}{星期一}{xing1qi1yi1}{9,12,1}{⽇、⽉、⼀}[HSK 1]
  \definition{s.}{segunda-feira}
\end{entry}

\begin{entry}{星星}{xing1 xing5}{9,9}{⽇、⽇}[HSK 2]
  \definition{s.}{estrela}
\end{entry}

\begin{entry}{星座}{xing1zuo4}{9,10}{⽇、⼴}
  \definition[张]{s.}{signo astrológico | constelação}
\end{entry}

\begin{entry}{猩猩}{xing1xing5}{12,12}{⽝、⽝}
  \definition{s.}{orangotango}
\end{entry}

\begin{entry}{行}{xing2}{6}{⾏}[HSK 1][Kangxi 144]
  \definition{adj.}{capaz | competente}
  \definition{expr.}{claro que sim | de acordo | está bem}
  \definition{interj.}{OK!}
  \definition{v.}{caminhar | ir | viajar | atuar}
  \seeref{行}{hang2}
\end{entry}

\begin{entry}{行动}{xing2dong4}{6,6}{⾏、⼒}[HSK 2]
  \definition[个]{s.}{ação | operação}
  \definition{v.}{mover}
\end{entry}

\begin{entry}{行进}{xing2jin4}{6,7}{⾏、⾡}
  \definition{s.}{avançar | movimentar-se para frente}
\end{entry}

\begin{entry}{行礼}{xing2li3}{6,5}{⾏、⽰}
  \definition{v.}{saudar | fazer saudação}
\end{entry}

\begin{entry}{行李}{xing2li5}{6,7}{⾏、⽊}[HSK 3]
  \definition[个,件]{s.}{bagagem | pacotes, caixas, cestas, etc. que você carrega quando sai}
\end{entry}

\begin{entry}{行人}{xing2ren2}{6,2}{⾏、⼈}[HSK 2]
  \definition{s.}{transeunte | pedestre | viajante à pé}
\end{entry}

\begin{entry}{行驶}{xing2shi3}{6,8}{⾏、⾺}
  \definition{v.}{viajar ao longo de uma rota (veículos, etc.)}
\end{entry}

\begin{entry}{行为}{xing2wei2}{6,4}{⾏、⼂}[HSK 2]
  \definition[个]{s.}{ação | comportamento | conduta}
\end{entry}

\begin{entry}{行星}{xing2xing1}{6,9}{⾏、⽇}
  \definition[颗]{s.}{planeta}
  \seealsoref{惑星}{huo4xing1}
\end{entry}

\begin{entry}{行凶}{xing2xiong1}{6,4}{⾏、⼐}
  \definition{v.+compl.}{cometer agressão física ou assassinato | fazer algo violento}
\end{entry}

\begin{entry}{形成}{xing2cheng2}{7,6}{⼺、⼽}[HSK 3]
  \definition{v.}{moldar; formar; tomar forma | tornar-se algo ou algo através do desenvolvimento e da mudança}
\end{entry}

\begin{entry}{形而上学}{xing2'er2shang4xue2}{7,6,3,8}{⼺、⽽、⼀、⼦}
  \definition{s.}{metafísica}
\end{entry}

\begin{entry}{形容}{xing2rong2}{7,10}{⼺、⼧}[HSK 4]
  \definition{s.}{aparência; semblante}
  \definition{v.}{descrever}
\end{entry}

\begin{entry}{形式}{xing2shi4}{7,6}{⼺、⼷}[HSK 3]
  \definition[种,个]{s.}{forma; formato; modalidade | aparência, estrutura ou estado de algo}
\end{entry}

\begin{entry}{形势}{xing2shi4}{7,8}{⼺、⼒}[HSK 4]
  \definition[个]{s.}{terreno; características topográficas; situação geográfica, principalmente de uma perspectiva militar | situação; circunstâncias; a situação geral, a tendência de como as coisas estão se desenvolvendo e mudando | geralmente não é usado em situações pessoais}
\end{entry}

\begin{entry}{形象}{xing2xiang4}{7,11}{⼺、⾗}[HSK 3]
  \definition{adj.}{vívido}
  \definition[个]{s.}{imagem; forma; figura | uma forma ou gesto específico que pode despertar os pensamentos ou emoções das pessoas | imagem literária; imagem artística | pessoas ou coisas com características diferentes criadas na literatura, no cinema e em outras artes}
\end{entry}

\begin{entry}{形状}{xing2zhuang4}{7,7}{⼺、⽝}[HSK 3]
  \definition[个]{s.}{forma; aparência | a aparência de um objeto ou figura formada pela combinação de superfícies ou linhas externas}
\end{entry}

\begin{entry}{型}{xing2}{9}{⼟}[HSK 4]
  \definition{s.}{molde; modelo | modelo; tipo; padrão}
\end{entry}

\begin{entry}{型号}{xing2 hao4}{9,5}{⼟、⼝}[HSK 4]
  \definition[个,种]{s.}{modelo; tipo; refere-se ao desempenho, às especificações e ao tamanho de aeronaves, máquinas, implementos agrícolas, etc.}
\end{entry}

\begin{entry}{省}{xing3}{9}{⽬}
  \definition[个]{s.}{governadoria}
  \definition{v.}{examinar minuciosamente | refletir (sobre a conduta de alguém) | realizar | fazer uma visita (aos pais ou idosos)}
  \seeref{省}{sheng3}
\end{entry}

\begin{entry}{省悟}{xing3wu4}{9,10}{⽬、⼼}
  \definition{v.}{voltar a si | constatar | ver a verdade | acordar para a realidade}
\end{entry}

\begin{entry}{醒}{xing3}{16}{⾣}[HSK 4]
  \definition{adj.}{impressionante; notável; admirável; atraente; chamativo}
  \definition{v.}{ficar sóbrio; voltar a si; recuperar a consciência; retornar à normalidade após intoxicação, anestesia ou coma | despertar; estar acordado | ter a mente clara; mover a consciência da confusão para a compreensão | vir a entender; tornar-se ciente de; tomar consciência de}
\end{entry}

\begin{entry}{兴}{xing4}{6}{⼋}
  \definition{s.}{sentimento ou desejo de fazer algo | interesse em algo | excitação}
  \seeref{兴}{xing1}
\end{entry}

\begin{entry}{兴趣}{xing4 qu4}{6,15}{⼋、⾛}[HSK 4]
  \definition[个]{s.}{interesse (desejo de conhecer sobre alguma coisa ou coisa no qual está interessado) | \emph{hobby}}
\end{entry}

\begin{entry}{姓}{xing4}{8}{⼥}[HSK 2]
  \definition[个]{s.}{sobrenome}
  \definition{v.}{ter o sobrenome}
\end{entry}

\begin{entry}{姓名}{xing4ming2}{8,6}{⼥、⼝}[HSK 2]
  \definition{s.}{nome completo}
\end{entry}

\begin{entry}{姓氏}{xing4shi4}{8,4}{⼥、⽒}
  \definition{s.}{sobrenome}
\end{entry}

\begin{entry}{幸福}{xing4fu2}{8,13}{⼲、⽰}[HSK 3]
  \definition{adj.}{feliz | a vida, a família, etc. deixam as pessoas satisfeitas e felizes}
  \definition{s.}{felicidade; bem estar | uma sensação ou experiência satisfatória e feliz}
\end{entry}

\begin{entry}{幸亏}{xing4kui1}{8,3}{⼲、⼆}
  \definition{adv.}{felizmente}
\end{entry}

\begin{entry}{幸运}{xing4yun4}{8,7}{⼲、⾡}[HSK 3]
  \definition{adj.}{sortudo; feliz; afortunado}
  \definition[个]{s.}{boa sorte; boa fortuna}
\end{entry}

\begin{entry}{幸运抽奖}{xing4yun4chou1jiang3}{8,7,8,9}{⼲、⾡、⼿、⼤}
  \definition{s.}{loteria | sorteio}
\end{entry}

\begin{entry}{幸运儿}{xing4yun4'er2}{8,7,2}{⼲、⾡、⼉}
  \definition{s.}{pessoa de sorte}
\end{entry}

\begin{entry}{性}{xing4}{8}{⼼}[HSK 3]
  \definition*{s.}{sobrenome Xing}
  \definition[个]{s.}{natureza; caráter; disposição | propriedade; qualidade | sexo; gênero}
  \definition{suf.}{forma substantivo a partir de adjetivo | indica natureza, escopo ou maneira}
\end{entry}

\begin{entry}{性别}{xing4bie2}{8,7}{⼼、⼑}[HSK 3]
  \definition[种]{s.}{sexo; gênero}
\end{entry}

\begin{entry}{性格}{xing4ge2}{8,10}{⼼、⽊}[HSK 3]
  \definition[种,个]{s.}{caráter; temperamento}
\end{entry}

\begin{entry}{性侵}{xing4qin1}{8,9}{⼼、⼈}
  \definition{s.}{agressão sexual}
  \definition{v.}{agredir sexualmente}
\end{entry}

\begin{entry}{性生活}{xing4sheng1huo2}{8,5,9}{⼼、⽣、⽔}
  \definition{s.}{vida sexual}
\end{entry}

\begin{entry}{性质}{xing4zhi4}{8,8}{⼼、⾙}[HSK 4]
  \definition[个,种,类]{s.}{natureza; qualidade; caráter; propriedade; propriedade fundamental que distingue uma coisa de outra}
\end{entry}

\begin{entry}{兄弟}{xiong1di4}{5,7}{⼉、⼸}[HSK 4]
  \definition{adj.}{fraternal}
  \definition{pron.}{eu, me (termo de uso humilde por homens em discurso público)}
  \definition[个,对]{s.}{irmãos; irmão}
\end{entry}

\begin{entry}{匈奴}{xiong1nu2}{6,5}{⼓、⼥}
  \definition*{s.}{Xiongnu, um povo da estepe oriental que criou um império que floresceu na época das dinastias Qin e Han}
\end{entry}

\begin{entry}{汹涌}{xiong1yong3}{7,10}{⽔、⽔}
  \definition{adj.}{turbulento}
  \definition{v.}{aumentar ou emergir violentamente (oceano, rio, lago, etc.)}
\end{entry}

\begin{entry}{胸}{xiong1}{10}{⾁}
  \definition{s.}{peito | tórax}
\end{entry}

\begin{entry}{胸部}{xiong1 bu4}{10,10}{⾁、⾢}[HSK 4]
  \definition{s.}{peito; tórax; seios}
\end{entry}

\begin{entry}{熊}{xiong2}{14}{⽕}
  \definition*{s.}{sobrenome Xiong}
  \definition{adj.}{incapaz}
  \definition[把]{s.}{urso}
  \definition{v.}{repreender}
\end{entry}

\begin{entry}{熊猫}{xiong2mao1}{14,11}{⽕、⽝}
  \definition[把,只]{s.}{panda gigante}
  \seealsoref{猫熊}{mao1xiong2}
\end{entry}

\begin{entry}{休兵}{xiu1bing1}{6,7}{⼈、⼋}
  \definition{s.}{armistício}
  \definition{v.}{cessar fogo}
\end{entry}

\begin{entry}{休假}{xiu1 jia4}{6,11}{⼈、⼈}[HSK 2]
  \definition{v.+compl.}{ter um feriado | tirar férias | sair de férias}
\end{entry}

\begin{entry}{休憩}{xiu1qi4}{6,16}{⼈、⼼}
  \definition{v.}{relaxar | descansar | dar um tempo}
\end{entry}

\begin{entry}{休息室}{xiu1xi1shi4}{6,10,9}{⼈、⼼、⼧}
  \definition{s.}{saguão | salão}
\end{entry}

\begin{entry}{休息}{xiu1xi5}{6,10}{⼈、⼼}[HSK 1]
  \definition{s.}{descanço}
  \definition{v.}{descansar}
\end{entry}

\begin{entry}{休闲}{xiu1xian2}{6,7}{⼈、⾨}
  \definition{s.}{ócio | lazer}
  \definition{v.}{desfrutar do lazer}
\end{entry}

\begin{entry}{休整}{xiu1zheng3}{6,16}{⼈、⽁}
  \definition{v.}{(militar) descansar e reorganizar}
\end{entry}

\begin{entry}{修}{xiu1}{9}{⼈}[HSK 3]
  \definition*{s.}{sobrenome Xiu}
  \definition{adj.}{comprido; alto e esbelto}
  \definition{s.}{revisionismo}
  \definition{v.}{embelezar; decorar
consertar; reparar; reformar
escrever; compilar
estudar; cultivar
construir; edificar
aparar; podar}
\end{entry}

\begin{entry}{修改}{xiu1gai3}{9,7}{⼈、⽁}[HSK 3]
  \definition{v.}{revisar; alterar}
\end{entry}

\begin{entry}{修规}{xiu1gui1}{9,8}{⼈、⾒}
  \definition{s.}{plano de construção}
\end{entry}

\begin{entry}{修理}{xiu1li3}{9,11}{⼈、⽟}[HSK 4]
  \definition{v.}{consertar; reparar; restaurar algo danificado à sua forma ou função original | aparar; podar; cortar com tesouras e outras ferramentas para deixar árvores, flores, cabelos, etc. arrumados | culpar; punir; criticar ou punir uma pessoa para mostrar que ela está errada}
\end{entry}

\begin{entry}{绣}{xiu4}{10}{⽷}
  \definition{s.}{bordado}
  \definition{v.}{bordar}
\end{entry}

\begin{entry}{臭}{xiu4}{10}{⾃}
  \definition{s.}{odor; cheiro;}
  \definition{v.}{cheirar; farejar; o mesmo que "嗅"}
  \seeref{臭}{chou4}
  \seealsoref{嗅}{xiu4}
\end{entry}

\begin{entry}{袖}{xiu4}{10}{⾐}
  \definition{s.}{manga (de camisa, de camiseta, etc.)}
\end{entry}

\begin{entry}{嗅}{xiu4}{13}{⼝}
  \definition{v.}{cheirar; farejar; identificar odores pelo nariz}
\end{entry}

\begin{entry}{虚伪}{xu1wei3}{11,6}{⾌、⼈}
  \definition{adj.}{falso | hipócrita | artificial}
\end{entry}

\begin{entry}{需求}{xu1qiu2}{14,7}{⾬、⽔}[HSK 3]
  \definition{s.}{necessidades; demanda; requisito; requerimento; exigência | solicitações decorrentes de necessidades}
\end{entry}

\begin{entry}{需要}{xu1yao4}{14,9}{⾬、⾑}[HSK 3]
  \definition{s.}{necessidade | desejo ou solicitação de algo}
  \definition{v.}{precisar; querer; requerer; demandar}
\end{entry}

\begin{entry}{许}{xu3}{6}{⾔}
  \definition*{s.}{sobrenome Xu}
  \definition{adv.}{um pouco | talvez}
  \definition{v.}{permitir | prometer | elogiar}
\end{entry}

\begin{entry}{许多}{xu3duo1}{6,6}{⾔、⼣}[HSK 2]
  \definition{num.}{muitos | muito | numerosos | uma grande quantidade de}
\end{entry}

\begin{entry}{T-恤}{xu4}{9}{⼼}
  \definition{s.}{camiseta | pulôver | suéter}
\end{entry}

\begin{entry}{畜}{xu4}{10}{⽥}
  \definition{v.}{criar (animais)}
  \seeref{畜}{chu4}
\end{entry}

\begin{entry}{宣布}{xuan1bu4}{9,5}{⼧、⼱}[HSK 3]
  \definition{v.}{declarar; proclamar; pronunciar; anunciar | anunciar oficialmente e publicamente as últimas decisões e situações a todos}
\end{entry}

\begin{entry}{宣传}{xuan1chuan2}{9,6}{⼧、⼈}[HSK 3]
  \definition{v.}{propagar; disseminar; conduzir propaganda | explicar às massas para que elas possam acreditar e agir de acordo}
\end{entry}

\begin{entry}{宣扬}{xuan1yang2}{9,6}{⼧、⼿}
  \definition{v.}{divulgar | anunciar | espalhar por toda parte}
\end{entry}

\begin{entry}{玄学}{xuan2xue2}{5,8}{⽞、⼦}
  \definition{s.}{Escola Philosófica Wei e Jin amalgamando os ideais daoísta e confucionistas | tradução da metafísica (形而上学)}
  \seeref{形而上学}{xing2'er2shang4xue2}
\end{entry}

\begin{entry}{悬挂}{xuan2gua4}{11,9}{⼼、⼿}
  \definition{s.}{(veículo) suspensão}
  \definition{v.}{suspender}
\end{entry}

\begin{entry}{悬崖}{xuan2ya2}{11,11}{⼼、⼭}
  \definition{s.}{precipício | penhasco}
\end{entry}

\begin{entry}{旋转}{xuan2zhuan3}{11,8}{⽅、⾞}
  \definition{v.}{girar}
\end{entry}

\begin{entry}{选}{xuan3}{9}{⾡}[HSK 2]
  \definition{s.}{seleções | antologia}
  \definition{v.}{selecionar | escolher | eleger}
\end{entry}

\begin{entry}{选手}{xuan3shou3}{9,4}{⾡、⼿}[HSK 3]
  \definition[位]{s.}{jogador; competidor (selecionado); atleta selecionado para uma competição esportiva}
\end{entry}

\begin{entry}{选择}{xuan3ze2}{9,8}{⾡、⼿}[HSK 4]
  \definition[个,种,次]{s.}{escolha; opção; resultado da escolha; possibilidade de escolha}
  \definition{v.}{selecionar; escolher}
\end{entry}

\begin{entry}{学}{xue2}{8}{⼦}[HSK 1]
  \definition{v.}{aprender | estudar}
\end{entry}

\begin{entry}{学费}{xue2 fei4}{8,9}{⼦、⾙}[HSK 3]
  \definition[个]{s.}{mensalidade (taxa); prêmio; taxas que os alunos devem pagar para estudar na escola, conforme estipulado pela escola | preço pelo que se aprendeu ao custo de cada um; uma metáfora para o preço pago para ganhar uma certa experiência | custo; preço; todas as despesas necessárias para os alunos estudarem}
\end{entry}

\begin{entry}{学分}{xue2fen1}{8,4}{⼦、⼑}[HSK 4]
  \definition{s.}{créditos de um curso; uma unidade de medida do peso e do tempo do curso no ensino superior; cada curso vale um crédito para uma aula por semana durante um semestre; alunos devem concluir o número necessário de créditos para se formar}
\end{entry}

\begin{entry}{学好}{xue2hao3}{8,6}{⼦、⼥}
  \definition{v.}{seguir bons exemplos | aprender bem}
\end{entry}

\begin{entry}{学会}{xue2hui4}{8,6}{⼦、⼈}
  \definition{s.}{instituto | associação (acadêmica) | sociedade científica, douta ou erudita}
  \definition{v.}{aprender | dominar (um assunto)}
\end{entry}

\begin{entry}{学年}{xue2 nian2}{8,6}{⼦、⼲}[HSK 4]
  \definition{s.}{ano letivo; ano acadêmico}
\end{entry}

\begin{entry}{学期}{xue2qi1}{8,12}{⼦、⽉}[HSK 2]
  \definition[个]{s.}{semestre}
\end{entry}

\begin{entry}{学生}{xue2sheng5}{8,5}{⼦、⽣}[HSK 1]
  \definition{s.}{estudante | aluno}
\end{entry}

\begin{entry}{学生证}{xue2sheng5zheng4}{8,5,7}{⼦、⽣、⾔}
  \definition{s.}{cartão de identidade de estudante}
\end{entry}

\begin{entry}{学时}{xue2 shi2}{8,7}{⼦、⽇}[HSK 4]
  \definition{s.}{hora-aula; hora de aula | período}
\end{entry}

\begin{entry}{学术}{xue2shu4}{8,5}{⼦、⽊}[HSK 4]
  \definition[个]{s.}{aprendizagem; aprendizado; ciências; aprendizado sistemático e especializado}
\end{entry}

\begin{entry}{学问}{xue2wen4}{8,6}{⼦、⾨}[HSK 4]
  \definition[个]{s.}{aprendizado, conhecimento, erudição; a compreensão correta do mundo objetivo que alguém tem | conhecimento; aprendizado sistemático; conhecimento sistemático sobre algo ou uma ciência que pode ser aprendido em um livro ou em uma experiência prática}
\end{entry}

\begin{entry}{学习}{xue2xi2}{8,3}{⼦、⼄}[HSK 1]
  \definition{v.}{estudar | aprender}
\end{entry}

\begin{entry}{学校}{xue2xiao4}{8,10}{⼦、⽊}[HSK 1]
  \definition{s.}{escola | instituição de ensino}
\end{entry}

\begin{entry}{学院}{xue2yuan4}{8,9}{⼦、⾩}[HSK 1]
  \definition[所]{s.}{instituto}
\end{entry}

\begin{entry}{雪}{xue3}{11}{⾬}[HSK 2]
  \definition*{s.}{sobrenome Xue}
  \definition[场]{s.}{neve}
\end{entry}

\begin{entry}{雪板}{xue3ban3}{11,8}{⾬、⽊}
  \definition{s.}{prancha de \emph{snowboard}}
  \definition{v.}{praticar \textit{snowboard}}
\end{entry}

\begin{entry}{雪糕}{xue3gao1}{11,16}{⾬、⽶}
  \definition{s.}{picolé}
\end{entry}

\begin{entry}{雪花}{xue3hua1}{11,7}{⾬、⾋}
  \definition{s.}{floco de neve}
\end{entry}

\begin{entry}{雪葩}{xue3pa1}{11,12}{⾬、⾋}
  \definition{s.}{sorvete}
\end{entry}

\begin{entry}{雪人}{xue3ren2}{11,2}{⾬、⼈}
  \definition{s.}{boneco de neve | \emph{Yeti}}
\end{entry}

\begin{entry}{雪山}{xue3shan1}{11,3}{⾬、⼭}
  \definition{s.}{montanha coberta de neve}
\end{entry}

\begin{entry}{雪鞋}{xue3xie2}{11,15}{⾬、⾰}
  \definition[双]{s.}{sapatos de neve}
\end{entry}

\begin{entry}{血}{xue4}{6}{⾎}[HSK 3][Kangxi 143]
  \definition{adj.}{relacionado por sangue; parente consanguíneo}
  \definition[片]{s.}{sangue}
  \seeref{血}{xie3}
\end{entry}

\begin{entry}{血汗}{xue4han4}{6,6}{⾎、⽔}
  \definition{s.}{(fig.) suor e labuta, trabalho duro}
\end{entry}

\begin{entry}{熏香}{xun1xiang1}{14,9}{⽕、⾹}
  \definition{s.}{incenso}
\end{entry}

\begin{entry}{寻找}{xun2zhao3}{6,7}{⼨、⼿}[HSK 4]
  \definition{v.}{buscar; procurar; pesquisar; encontrar, que pode ser usado tanto para coisas concretas quanto para coisas abstratas}
\end{entry}

\begin{entry}{巡逻}{xun2luo2}{6,11}{⾡、⾡}
  \definition{s.}{patrulha}
  \definition{v.}{patrulhar (polícia, exército ou marinha)}
\end{entry}

\begin{entry}{训练}{xun4lian4}{5,8}{⾔、⽷}[HSK 3]
  \definition{v.}{treinar; exercitar; adquirir certas especialidades ou habilidades de forma planejada e passo a passo}
\end{entry}

\begin{entry}{迅速}{xun4su4}{6,10}{⾡、⾡}[HSK 4]
  \definition{adv.}{rapidamente; velozmente; prontamente}
\end{entry}

%%%%% EOF %%%%%


 %%%
%%% Y
%%%
\section*{Y}
\addcontentsline{toc}{section}{Y}
%\begin{multicols*}{2}

\begin{verbete}[ya1sui4qian2]{压岁钱}
\begin{pronuncia}{ya1sui4qian2}
\significado{n.}{
dinheiro da sorte|
dinheiro dado às crianças como presente no Ano Novo Chinês
}
\end{pronuncia}
\end{verbete}

\begin{verbete}[ya1]{鸭}
\begin{pronuncia}{ya1}
\significado[只]{n.}{
pato|
gíria: prostituto
}
\end{pronuncia}
\end{verbete}

\begin{verbete}[ya2]{牙}
\begin{pronuncia}{ya2}
\significado[颗]{n.}{
dente; marfim
}
\end{pronuncia}
\end{verbete}

\begin{verbete}[ya2chi3]{牙齿}
\begin{pronuncia}{ya2chi3}
\significado{adv.}{
dental
}
\significado[颗]{n.}{
dente
}
\end{pronuncia}
\end{verbete}

\begin{verbete}[Ya4zhou1]{亚洲}
\begin{pronuncia}{Ya4zhou1}
\significado{n.}{
Ásia
}
\end{pronuncia}
\end{verbete}

\begin{verbete}[yan2se4]{颜色}
\begin{pronuncia}{yan2se4}
\significado{n.}{
cor; pigmento; tintura
}
\end{pronuncia}
\end{verbete}

\begin{verbete}[yan3jing4]{眼镜}
\begin{pronuncia}{yan3jing4}
\significado[副]{n.}{
óculos
}
\end{pronuncia}
\end{verbete}

\begin{verbete}[yan3jing0]{眼睛}
\begin{pronuncia}{yan3jing0}
\significado[只,双]{n.}{
olho(s)
}
\end{pronuncia}
\end{verbete}

\begin{verbete}[yang3]{养}
\begin{pronuncia}{yang3}
\significado{v.}{
criar (animais ou filhos), plantar (flores), etc
}
\end{pronuncia}
\end{verbete}

\begin{verbete}[yang4zi0]{样子}
\begin{pronuncia}{yang4zi0}
\significado{n.}{
aparência; forma; modelo
}
\end{pronuncia}
\end{verbete}

\begin{verbete}[yao1]{腰}
\begin{pronuncia}{yao1}
\significado{n.}{
cintura
}
\end{pronuncia}
\end{verbete}

\begin{verbete}[yao4]{药}
\begin{pronuncia}{yao4}
\significado[种,服,味]{n.}{
medicamento; remédio; droga
}
\end{pronuncia}
\end{verbete}

\begin{verbete}[yao4]{要}
\begin{pronuncia}{yao4}
\significado{v./v.o.}{
querer; precisar
}
\end{pronuncia}
\end{verbete}

\begin{verbete}[yao4shi0]{要是}
\begin{pronuncia}{yao4shi0}
\significado{conj.}{
se
}
\end{pronuncia}
\end{verbete}

\begin{verbete}[yao4shi0 ...\  de0hua0]{要是······的话}
\begin{pronuncia}[\\]{yao4shi0 ...\  de0hua0}
\significado{conj.}{
se ... no caso de
}
\end{pronuncia}
\end{verbete}

\begin{verbete}[ye2ye0]{爷爷}
\begin{pronuncia}{ye2ye0}
\significado[个]{n.}{
avô (paterno)
}
\end{pronuncia}
\end{verbete}

\begin{verbete}[ye3]{也}
\begin{pronuncia}{ye3}
\significado{adv.}{
também
}
\end{pronuncia}
\end{verbete}

\begin{verbete}[ye4li0]{夜里}
\begin{pronuncia}{ye4li0}
\significado{p.t.}{
noite
}
\end{pronuncia}
\end{verbete}

\begin{verbete}[yi1]{一}
\begin{pronuncia}{yi1}[(quando usado sozinho)]
\significado{num.}{
um, uma; 1
}
\end{pronuncia}
\begin{pronuncia}{yi2}[(antes de quarto tom)]
\significado{num.}{
um, uma; 1|
um, uma (artigo)
}
\end{pronuncia}
\begin{pronuncia}{yi4}
\significado{num.}{
um, uma; 1|
um, uma (artigo)
}
\end{pronuncia}
\end{verbete}

\begin{verbete}[yi2]{一}
\begin{pronuncia}{yi2}[(antes de quarto tom)]
\significado{num.}{
um, uma; 1|
um, uma (artigo)
}
\end{pronuncia}
\begin{pronuncia}{yi1}[(quando usado sozinho)]
\significado{num.}{
um, uma; 1
}
\end{pronuncia}
\begin{pronuncia}{yi4}
\significado{num.}{
um, uma; 1|
um, uma (artigo)
}
\end{pronuncia}
\end{verbete}

\begin{verbete}[yi2ban4]{一半}
\begin{pronuncia}{yi2ban4}
\significado{adj.}{
metade
}
\end{pronuncia}
\end{verbete}

\begin{verbete}[yi2ding4]{一定}
\begin{pronuncia}{yi2ding4}
\significado{adv.}{
certamente; definitivamente
}
\end{pronuncia}
\end{verbete}

\begin{verbete}[yi2gong4]{一共}
\begin{pronuncia}{yi2gong4}
\significado{adv.}{
tudo; no local
}
\end{pronuncia}
\end{verbete}

\begin{verbete}[yi2xia4]{一下}
\begin{pronuncia}{yi2xia4}
\significado{adv.}{
em um curto tempo; rapidamente
}
\end{pronuncia}
\end{verbete}

\begin{verbete}[yi2yang4]{一样}
\begin{pronuncia}{yi2yang4}
\significado{adj.}{
igual; mesmo, mesma
}
\end{pronuncia}
\end{verbete}

\begin{verbete}[yi4]{一}
\begin{pronuncia}{yi4}
\significado{num.}{
um, uma; 1|
um, uma (artigo)
}
\end{pronuncia}
\begin{pronuncia}{yi2}[(antes de quarto tom)]
\significado{num.}{
um, uma; 1|
um, uma (artigo)
}
\end{pronuncia}
\begin{pronuncia}{yi1}[(quando usado sozinho)]
\significado{num.}{
um, uma; 1
}
\end{pronuncia}
\end{verbete}

\begin{verbete}[yi4ban1]{一般}
\begin{pronuncia}{yi4ban1}
\significado{adj.}{
geral; comum; normal
}
\significado{adv.}{
normalmente
}
\end{pronuncia}
\end{verbete}

\begin{verbete}[yi4dianr3]{一点儿}
\begin{pronuncia}{yi4dianr3}
\significado{adv.}{
um pouco (``adj.+一点儿'' ou ``一点儿+n.'')
}
\end{pronuncia}
\end{verbete}

\begin{verbete}[yi4huir4]{一会儿}
\begin{pronuncia}{yi4huir4}
\significado{adv.}{
daqui a pouco tempo; pouco tempo
}
\end{pronuncia}
\end{verbete}

\begin{verbete}[yi4qi3]{一起}
\begin{pronuncia}{yi4qi3}
\significado{adv.}{
juntamente; em conjunto
}
\end{pronuncia}
\end{verbete}

\begin{verbete}[yi4zhi2]{一直}
\begin{pronuncia}{yi4zhi2}
\significado{adv.}{
diretamente; sempre
}
\end{pronuncia}
\end{verbete}

\begin{verbete}[yi4xie1]{一些}
\begin{pronuncia}{yi4xie1}
\significado{pron.}{
uns, umas; alguns, algumas
}
\end{pronuncia}
\end{verbete}

\begin{verbete}[yi1fu0]{衣服}
\begin{pronuncia}{yi1fu0}
\significado[件,套]{n.}{
roupa, vestuário
}
\end{pronuncia}
\end{verbete}

\begin{verbete}[yi1sheng1]{医生}
\begin{pronuncia}{yi1sheng1}
\significado[个,位,名]{n.}{
médico; clínico
}
\end{pronuncia}
\end{verbete}

\begin{verbete}[yi1yuan0]{医院}
\begin{pronuncia}{yi1yuan0}
\significado[所,家,座]{n.}{
hospital
}
\end{pronuncia}
\end{verbete}

\begin{verbete}[yi2he2yuan2]{颐和园}
\begin{pronuncia}{yi2he2yuan2}
\significado{n.}{
Palácio de Verão
}
\end{pronuncia}
\end{verbete}

\begin{verbete}[yi2han4]{遗憾}
\begin{pronuncia}{yi2han4}
\significado{v.}{
ter pena de
}
\end{pronuncia}
\end{verbete}

\begin{verbete}[yi3jing1]{已经}
\begin{pronuncia}{yi3jing1}
\significado{adv.}{
já
}
\end{pronuncia}
\end{verbete}

\begin{verbete}[yi3hou4]{以后}
\begin{pronuncia}{yi3hou4}
\significado{p.t.}{
depois de; depois; após
}
\end{pronuncia}
\end{verbete}

\begin{verbete}[yi3qian2]{以前}
\begin{pronuncia}{yi3qian2}
\significado{p.t.}{
antes de; antes
}
\end{pronuncia}
\end{verbete}

\begin{verbete}[yi4]{亿}
\begin{pronuncia}{yi4}
\significado{num.}{
cem milhões; 100.000.000
}
\end{pronuncia}
\end{verbete}

\begin{verbete}[yi4si0]{意思}
\begin{pronuncia}{yi4si0}
\significado[个]{n.}{
interesse
}
\end{pronuncia}
\end{verbete}

\begin{verbete}[yin1tian1]{阴天}
\begin{pronuncia}{yin1tian1}
\significado{adj.}{
céu muito nublado; céu cinzento
}
\end{pronuncia}
\end{verbete}

\begin{verbete}[yin1wei4]{因为}
\begin{pronuncia}{yin1wei4}
\significado{conj.}{
porque
}
\end{pronuncia}
\end{verbete}

\begin{verbete}[yin1yue4音乐]{音乐}
\begin{pronuncia}{yin1yue4}
\significado[张,曲,段]{n.}{
música
}
\end{pronuncia}
\end{verbete}

\begin{verbete}[yin2hang2]{银行}
\begin{pronuncia}{yin2hang2}
\significado[家,个]{n.}{
banco; agência bancária
}
\end{pronuncia}
\end{verbete}

\begin{verbete}[yin3liao4]{饮料}
\begin{pronuncia}{yin3liao4}
\significado{n.}{
bebida
}
\end{pronuncia}
\end{verbete}

\begin{verbete}[ying1gai1]{应该}
\begin{pronuncia}{ying1gai1}
\significado{v.}{
dever; ter de
}
\end{pronuncia}
\end{verbete}

\begin{verbete}[Ying1guo2]{英国}
\begin{pronuncia}{Ying1guo2}
\significado{n.}{
Reino Unido
}
\end{pronuncia}
\end{verbete}

\begin{verbete}[ying1yu3]{英语}
\begin{pronuncia}{ying1yu3}
\significado{n.}{
inglês, língua inglesa
}
\end{pronuncia}
\end{verbete}

\begin{verbete}[ying1wen2]{英文}
\begin{pronuncia}{ying1wen2}
\significado{n.}{
inglês, língua inglesa
}
\end{pronuncia}
\end{verbete}

\begin{verbete}[you1mei3]{优美}
\begin{pronuncia}{you1mei3}
\significado{adj.}{
gracioso; fino; elegante
}
\end{pronuncia}
\end{verbete}

\begin{verbete}[you2jian4]{邮件}
\begin{pronuncia}{you2jian4}
\significado{n.}{
correspondência; \emph{email}
}
\end{pronuncia}
\end{verbete}

\begin{verbete}[you2ju4]{邮局}
\begin{pronuncia}{you2ju4}
\significado[家,个]{n.}{
correio; agência dos correios
}
\end{pronuncia}
\end{verbete}

\begin{verbete}[you2]{游}
\begin{pronuncia}{you2}
\significado{v.}{
nadar
}
\end{pronuncia}
\end{verbete}

\begin{verbete}[you2yong3]{游泳}
\begin{pronuncia}{you2yong3}
\significado{v.+compl.}{
nadar
}
\end{pronuncia}
\end{verbete}

\begin{verbete}[you2yong3chi2]{游泳池}
\begin{pronuncia}{you2yong3chi2}
\significado[场]{n.}{
piscina
}
\end{pronuncia}
\end{verbete}

\begin{verbete}[you3]{有}
\begin{pronuncia}{you3}
\significado{v.}{
ter; haver
}
\end{pronuncia}
\end{verbete}

\begin{verbete}[you3de0]{有的}
\begin{pronuncia}{you3de0}
\significado{pron.}{
algum, alguma, alguns, algumas
}
\end{pronuncia}
\end{verbete}

\begin{verbete}[you3de0\ shi2hou0]{有的时候}
\begin{pronuncia}{you3de0\ shi2hou0}
\significado{expr.}{
às vezes;
de vez em quando;
de quando em quando
}
\end{pronuncia}
\end{verbete}

\begin{verbete}[you3dianr3]{有点儿}
\begin{pronuncia}{you3dianr3}
\significado{adv.}{
um pouco (``有点儿+n. ou v. mental'')
}
\end{pronuncia}
\end{verbete}

\begin{verbete}[you3ming2]{有名}
\begin{pronuncia}{you3ming2}
\significado{adj.}{
famoso, famosa
}
\end{pronuncia}
\end{verbete}

\begin{verbete}[you3shi2]{有时}
\begin{pronuncia}{you3shi2}
\significado{expr.}{
às vezes;
de vez em quando;
de quando em quando
}
\end{pronuncia}
\end{verbete}

\begin{verbete}[you3shi2hou0]{有时候}
\begin{pronuncia}{you3shi2hou0}
\significado{expr.}{
às vezes;
de vez em quando;
de quando em quando
}
\end{pronuncia}
\end{verbete}

\begin{verbete}[you3yi2si0]{有意思}
\begin{pronuncia}{you3yi2si0}
\significado{adj.}{
interessante
}
\end{pronuncia}
\end{verbete}

\begin{verbete}[you3yong4]{有用}
\begin{pronuncia}{you3yong4}
\significado{adj.}{
útil
}
\end{pronuncia}
\end{verbete}

\begin{verbete}[you4]{右}
\begin{pronuncia}{you4}
\significado{p.l.}{
direita
}
\end{pronuncia}
\end{verbete}

\begin{verbete}[you4bian0]{右边}
\begin{pronuncia}{you4bian0}
\significado{p.l.}{
à direita; ao lado direito
}
\end{pronuncia}
\end{verbete}

\begin{verbete}[you4mian0]{右面}
\begin{pronuncia}{you4mian0}
\significado{p.l.}{
à direita; ao lado direito
}
\end{pronuncia}
\end{verbete}

\begin{verbete}[yong4]{用}
\begin{pronuncia}{yong4}
\significado{v.}{
usar
}
\end{pronuncia}
\end{verbete}

\begin{verbete}[yu2]{鱼}
\begin{pronuncia}{yu2}
\significado[条,尾]{n.}{
peixe
}
\end{pronuncia}
\end{verbete}

\begin{verbete}[yu2pian4]{鱼片}
\begin{pronuncia}{yu2pian4}
\significado{n.}{
fatia de peixe; filé de peixe
}
\end{pronuncia}
\end{verbete}

\begin{verbete}[yu2xiang1rou4si1]{鱼香肉丝}
\begin{pronuncia}{yu2xiang1rou4si1}
\significado{n.}{
tiras de carne de porco salteadas com molho picante
}
\end{pronuncia}
\end{verbete}

\begin{verbete}[yu3]{玉}
\begin{pronuncia}{yu3}
\significado[块]{n.}{
jade
}
\end{pronuncia}
\end{verbete}

\begin{verbete}[yu3mao2qiu2]{羽毛球}
\begin{pronuncia}{yu3mao2qiu2}
\significado{n.}{
badminton
}
\end{pronuncia}
\end{verbete}

\begin{verbete}[yu3]{雨}
\begin{pronuncia}{yu3}
\significado[阵,场]{n.}{
chuva
}
\end{pronuncia}
\end{verbete}

\begin{verbete}[yu3san3]{雨伞}
\begin{pronuncia}{yu3san3}
\significado[把]{n.}{
guarda-chuva
}
\end{pronuncia}
\end{verbete}

\begin{verbete}[yu3yi1]{雨衣}
\begin{pronuncia}{yu3yi1}
\significado[件]{n.}{
impermeável
}
\end{pronuncia}
\end{verbete}

\begin{verbete}[yu3fa3]{语法}
\begin{pronuncia}{yu3fa3}
\significado{n.}{
gramática
}
\end{pronuncia}
\end{verbete}

\begin{verbete}[yu3yan2shi2yan4shi4]{语言实验室}
\begin{pronuncia}[\\]{yu3yan2shi2yan4shi4}
\significado{n.}{
laboratório de línguas
}
\end{pronuncia}
\end{verbete}

\begin{verbete}[yu4bao4]{预报}
\begin{pronuncia}{yu4bao4}
\significado{n.}{
previsão (meteorológica); boletim meteorológico
}
\significado{v.}{
prever (o tempo)
}
\end{pronuncia}
\end{verbete}

\begin{verbete}[yuan2]{元}
\begin{pronuncia}{yuan2}
\significado{p.c.}{
unidade monetária da China
}
\end{pronuncia}
\end{verbete}

\begin{verbete}[yuan3]{远}
\begin{pronuncia}{yuan3}
\significado{adj.}{
longe; longo, longa
}
\end{pronuncia}
\end{verbete}

\begin{verbete}[yuan4zi0]{院子}
\begin{pronuncia}{yuan4zi0}
\significado[个]{n.}{
pátio; jardim; quintal
}
\end{pronuncia}
\end{verbete}

\begin{verbete}[yue1hui4]{约会}
\begin{pronuncia}{yue1hui4}
\significado[次,个]{n.}{
compromisso; encontro marcado
}
\end{pronuncia}
\end{verbete}

\begin{verbete}[yue4]{月}
\begin{pronuncia}{yue4}
\significado[个,轮]{n.}{
mês
}
\end{pronuncia}
\end{verbete}

\begin{verbete}[yue4liang0]{月亮}
\begin{pronuncia}{yue4liang0}
\significado{n.}{
lua
}
\end{pronuncia}
\end{verbete}

\begin{verbete}[yue4du2]{阅读}
\begin{pronuncia}{yue4du2}
\significado{n.}{
leitura
}
\significado{v.}{
ler
}
\end{pronuncia}
\end{verbete}

\begin{verbete}[yue4...\ yue4...]{越······越······}
\begin{pronuncia}{yue4...\ yue4...}
\significado{expr.}{
quanto mais... tanto mais...
}
\end{pronuncia}
\end{verbete}

\begin{verbete}[yue4lai2yue4...]{越来越······}
\begin{pronuncia}{yue4lai2yue4...}
\significado{expr.}{
cada vez mais...
}
\end{pronuncia}
\end{verbete}

\begin{verbete}[yue4lan3shi4]{阅览室}
\begin{pronuncia}{yue4lan3shi4}
\significado[间]{n.}{
sala de leitura
}
\end{pronuncia}
\end{verbete}

\begin{verbete}[Yun2nan2]{云南}
\begin{pronuncia}{Yun2nan2}
\significado{n.}{
Yunnan
}
\end{pronuncia}
\end{verbete}

\begin{verbete}[yun4dong4]{运动}
\begin{pronuncia}{yun4dong4}
\significado[场]{n.}{
esporte; desporto
}
\end{pronuncia}
\end{verbete}

\begin{verbete}[yun4dong4chang3]{运动场}
\begin{pronuncia}{yun4dong4chang3}
\significado{n.}{
campo desportivo; campo de jogos
}
\end{pronuncia}
\end{verbete}

\begin{verbete}[yun4dong4hui4]{运动会}
\begin{pronuncia}{yun4dong4hui4}
\significado[个]{n.}{
jogos desportivos
}
\end{pronuncia}
\end{verbete}

\begin{verbete}[yun4dong4yuan2]{运动员}
\begin{pronuncia}{yun4dong4yuan2}
\significado[名,个]{n.}{
jogador, jogadora; atleta
}
\end{pronuncia}
\end{verbete}

%\end{multicols*}

 %%%
%%% Z
%%%
\section*{Z}
\addcontentsline{toc}{section}{Z}
\begin{multicols*}{2}

\begin{verbete}[zai4]{在}
\begin{pronuncia}{zai4}
\significado{adv.}{
para designar ações que estão passando
}
\significado{prep.}{
em
}
\significado{v.}{
estar; ficar
}
\end{pronuncia}
\end{verbete}

\begin{verbete}[zai4]{再}
\begin{pronuncia}{zai4}
\significado{adv.}{
de novo; outra vez
}
\end{pronuncia}
\end{verbete}

\begin{verbete}[zai4jian4]{再见}
\begin{pronuncia}{zai4jian4}
\significado{v.}{
adeus; até à vista; até à próxima; até logo
}
\end{pronuncia}
\end{verbete}

\begin{verbete}[zan2men0]{咱们}
\begin{pronuncia}{zan2men0}
\significado{pron.}{
nós (eu e você)
}
\end{pronuncia}
\end{verbete}

\begin{verbete}[zang1]{脏}
\begin{pronuncia}{zang1}
\significado{adj.}{
sujo
}
\end{pronuncia}
\end{verbete}

\begin{verbete}[zao3]{早}
\begin{pronuncia}{zao3}
\significado{adj.}{
cedo
}
\end{pronuncia}
\end{verbete}

\begin{verbete}[zao3fan4]{早反}
\begin{pronuncia}{zao3fan4}
\significado{n.}{
café da manhã
}
\end{pronuncia}
\end{verbete}

\begin{verbete}[zao3shang0]{早上}
\begin{pronuncia}{zao3shang0}
\significado{p.t.}{
manhã cedo; manhãzinha
}
\end{pronuncia}
\end{verbete}

\begin{verbete}[zen3me0]{怎么}
\begin{pronuncia}{zen3me0}
\significado{interr.}{
como?
}
\end{pronuncia}
\end{verbete}

\begin{verbete}[zen3me0yang4]{怎么样}
\begin{pronuncia}{zen3me0yang4}
\significado{interr.}{
como?; que tal?
}
\end{pronuncia}
\end{verbete}

\begin{verbete}[zhan4]{站}
\begin{pronuncia}{zhan4}
\significado{n.}{
estação; ponto; paragem
}
\end{pronuncia}
\end{verbete}

\begin{verbete}[zhang1]{张}
\begin{pronuncia}{zhang1}
\significado{p.c.}{
para folha de papéis, mapas, etc
}
\end{pronuncia}
\end{verbete}

\begin{verbete}[zhao2ji2]{着急}
\begin{pronuncia}{zhao2ji2}
\significado{adj.}{
inquieto; ansioso
}
\end{pronuncia}
\end{verbete}

\begin{verbete}[zhao3]{找}
\begin{pronuncia}{zhao3}
\significado{v.}{
andar à procura de; procurar; dar troco
}
\end{pronuncia}
\end{verbete}

\begin{verbete}[zhao4pian4]{照片}
\begin{pronuncia}{zhao4pian4}
\significado[张,套,幅]{n.}{
fotografia; foto
}
\end{pronuncia}
\end{verbete}

\begin{verbete}[zhao4xiang4]{照相}
\begin{pronuncia}{zhao4xiang4}
\significado{v.+compl.}{
tirar fotografia
}
\end{pronuncia}
\end{verbete}

\begin{verbete}[zhao4xiang4ji1]{照相机}
\begin{pronuncia}{zhao4xiang4ji1}
\significado[个,架,部,台,只]{n.}{
câmera/máquina fotográfica
}
\end{pronuncia}
\end{verbete}

\begin{verbete}[zhe4]{这}
\begin{pronuncia}{zhe4}
\significado{pron.}{
este, esta, isto
}
\end{pronuncia}
\end{verbete}

\begin{verbete}[zhe4li0]{这里}
\begin{pronuncia}{zhe4li0}
\significado{pron.}{
aqui
}
\end{pronuncia}
\end{verbete}

\begin{verbete}[zhe4me0]{这么}
\begin{pronuncia}{zhe4me0}
\significado{adv.}{
como este; desta maneira
}
\end{pronuncia}
\end{verbete}

\begin{verbete}[zhe4xie1]{这些}
\begin{pronuncia}{zhe4xie1}
\significado{pron.}{
estes, estas
}
\end{pronuncia}
\end{verbete}

\begin{verbete}[zher4]{这儿}
\begin{pronuncia}{zher4}
\significado{pron.}{
aqui
}
\end{pronuncia}
\end{verbete}

\begin{verbete}[Zhe4jiang1]{浙江}
\begin{pronuncia}{Zhe4jiang1}
\significado{n.}{
Zhejiang
}
\end{pronuncia}
\end{verbete}

\begin{verbete}[zhen1]{真}
\begin{pronuncia}{zhen1}
\significado{adv.}{
que...tão...!|
realmente
}
\end{pronuncia}
\end{verbete}

\begin{verbete}[zheng4qian2]{挣钱}
\begin{pronuncia}{zheng4qian2}
\significado{v.+compl.}{
ganhar dinheiro
}
\end{pronuncia}
\end{verbete}

\begin{verbete}[zheng4zai4]{正在}
\begin{pronuncia}{zheng4zai4}
\significado{adv.}{
estar a + inf.; estar + ger.
}
\end{pronuncia}
\end{verbete}

\begin{verbete}[zhi1]{支}
\begin{pronuncia}{zhi1}
\significado{p.c.}{
para caneta, lápis, etc
}
\end{pronuncia}
\end{verbete}

\begin{verbete}[zhi1]{只}
\begin{pronuncia}{zhi1}
\significado{p.c.}{
para pássaros, gatos, cãezinhos, etc
}
\end{pronuncia}
\begin{pronuncia}{zhi3}
\significado{adv.}{
apenas; só
}
\end{pronuncia}
\end{verbete}

\begin{verbete}[zhi1dao0]{知道}
\begin{pronuncia}{zhi1dao0}
\significado{v.}{
conhecer; saber
}
\end{pronuncia}
\end{verbete}

\begin{verbete}[zhi2yuan2]{职员}
\begin{pronuncia}{zhi2yuan2}
\significado{n.}{
empregado, empregada
}
\end{pronuncia}
\end{verbete}

\begin{verbete}[zhi3]{只}
\begin{pronuncia}{zhi3}
\significado{adv.}{
apenas; só
}
\end{pronuncia}
\begin{pronuncia}{zhi1}
\significado{p.c.}{
para pássaros, gatos, cãezinhos, etc
}
\end{pronuncia}
\end{verbete}

\begin{verbete}[Zhong1guo2]{中国}
\begin{pronuncia}{Zhong1guo2}
\significado{n.}{
China
}
\end{pronuncia}
\end{verbete}

\begin{verbete}[Zhong1guo2tong1]{中国通}
\begin{pronuncia}{Zhong1guo2tong1}
\significado{n.}{
conhecedor da China; especialista em tudo sobre a China
}
\end{pronuncia}
\end{verbete}

\begin{verbete}[zhong1jian1]{中间}
\begin{pronuncia}{zhong1jian1}
\significado{p.l.}{
central; centro; no meio
}
\end{pronuncia}
\end{verbete}

\begin{verbete}[zhong1wen2]{中文}
\begin{pronuncia}{zhong1wen2}
\significado{n.}{
chinês; língua chinesa
}
\end{pronuncia}
\end{verbete}

\begin{verbete}[zhong1xue2]{中学}
\begin{pronuncia}{zhong1xue2}
\significado[个]{n.}{
escola ensino médio
}
\end{pronuncia}
\end{verbete}

\begin{verbete}[zhong1xue2sheng1]{中学生}
\begin{pronuncia}{zhong1xue2sheng1}
\significado{n.}{
estudante da escola ensino médio
}
\end{pronuncia}
\end{verbete}

\begin{verbete}[zhong1xun2]{中询}
\begin{pronuncia}{zhong1xun2}
\significado{p.t.}{
segunda dezena do mês; meio do mês; em meados do mês
}
\end{pronuncia}
\end{verbete}

\begin{verbete}[zhong1]{钟}
\begin{pronuncia}{zhong1}
\significado{p.c.}{
hora
}
\end{pronuncia}
\end{verbete}

\begin{verbete}[zhong3]{种}
\begin{pronuncia}{zhong3}
\significado{p.c.}{
para tipos, espécies e gêneros
}
\end{pronuncia}
\end{verbete}

\begin{verbete}[zhong4]{重}
\begin{pronuncia}{zhong4}
\significado{adj.}{
pesado, pesada
}
\end{pronuncia}
\end{verbete}

\begin{verbete}[zhong4liang4]{重量}
\begin{pronuncia}{zhong4liang4}
\significado[个]{n.}{
peso
}
\end{pronuncia}
\end{verbete}

\begin{verbete}[zhou1mo4]{周末}
\begin{pronuncia}{zhou1mo4}
\significado{n.}{
final-de-semana
}
\end{pronuncia}
\end{verbete}

\begin{verbete}[zhu1]{猪}
\begin{pronuncia}{zhu1}
\significado[口,头]{n.}{
porco, porca
}
\end{pronuncia}
\end{verbete}

\begin{verbete}[Zhu3xi2]{主席}
\begin{pronuncia}{Zhu3xi2}
\significado[个,位]{n.}{
Presidente (da China); Primeiro-Ministro
}
\end{pronuncia}
\end{verbete}

\begin{verbete}[zhu4]{住}
\begin{pronuncia}{zhu4}
\significado{v.}{
morar; viver; alojar-se
}
\end{pronuncia}
\end{verbete}

\begin{verbete}[zhu4zhai2]{住宅}
\begin{pronuncia}{zhu4zhai2}
\significado{n.}{
residência
}
\end{pronuncia}
\end{verbete}

\begin{verbete}[zhu4ce4]{注册}
\begin{pronuncia}{zhu4ce4}
\significado{v.}{
inscrever-se; matricular-se
}
\end{pronuncia}
\end{verbete}

\begin{verbete}[zhu4]{祝}
\begin{pronuncia}{zhu4}
\significado{v.}{
desejar (exprimir um bom desejo);
congratular
}
\end{pronuncia}
\end{verbete}

\begin{verbete}[zhu4fu4]{嘱咐}
\begin{pronuncia}{zhu4fu4}
\significado{v.}{
ordenar; dizer; exortar
}
\end{pronuncia}
\end{verbete}

\begin{verbete}[zhuan1ye4]{专业}
\begin{pronuncia}{zhuan1ye4}
\significado[门,个]{n.}{
área de atuação; especialidade
}
\end{pronuncia}
\end{verbete}

\begin{verbete}[zhuo1zi0]{桌子}
\begin{pronuncia}{zhuo1zi0}
\significado[张,套]{n.}{
mesa
}
\end{pronuncia}
\end{verbete}

\begin{verbete}[zi3se4]{紫色}
\begin{pronuncia}{zi3se4}
\significado{n.}{
cor roxa
}
\end{pronuncia}
\end{verbete}

\begin{verbete}[zi4]{字}
\begin{pronuncia}{zi4}
\significado[个]{n.}{
carácter; letra; símbolo; palavra
}
\end{pronuncia}
\end{verbete}

\begin{verbete}[zi4ji3]{自己}
\begin{pronuncia}{zi4ji3}
\significado{pron.}{
a si próprio; próprio
}
\end{pronuncia}
\end{verbete}

\begin{verbete}[zi4xing2che1]{自行车}
\begin{pronuncia}{zi4xing2che1}
\significado[辆]{n.}{
bicicleta
}
\end{pronuncia}
\end{verbete}

\begin{verbete}[zi4wo3]{自我}
\begin{pronuncia}{zi4wo3}
\significado{pron.}{
a si mesmo; eu próprio|
auto-...
}
\end{pronuncia}
\end{verbete}

\begin{verbete}[Zong3du1]{总督}
\begin{pronuncia}{Zong3du1}
\significado{n.}{
Governador; Governador-Geral; Vice-Rei
}
\end{pronuncia}
\end{verbete}

\begin{verbete}[Zong3li3]{总理}
\begin{pronuncia}{Zong3li3}
\significado[个,位,名]{n.}{
Primeiro-Ministro
}
\end{pronuncia}
\end{verbete}

\begin{verbete}[Zong3tong3]{总统}
\begin{pronuncia}{Zong3tong3}
\significado[个,位,名,届]{n.}{
Presidente (de um país)
}
\end{pronuncia}
\end{verbete}

\begin{verbete}[zou3]{走}
\begin{pronuncia}{zou3}
\significado{v.}{
andar; caminhar
}
\end{pronuncia}
\end{verbete}

\begin{verbete}[zu2qiu2]{足球}
\begin{pronuncia}{zu2qiu2}
\significado[个]{n.}{
futebol; bola de futebol
}
\end{pronuncia}
\end{verbete}

\begin{verbete}[zui3ba0]{嘴巴}
\begin{pronuncia}{zui3ba0}
\significado[张]{n.}{
boca
}
\significado[个]{n.}{
bofetada na cara
}
\end{pronuncia}
\end{verbete}

\begin{verbete}[zui4]{最}
\begin{pronuncia}{zui4}
\significado{adv.}{
o mais, a mais|
grau superlativo relativo de superioridade
}
\end{pronuncia}
\end{verbete}

\begin{verbete}[zui4hou4]{最后}
\begin{pronuncia}{zui4hou4}
\significado{adj.}{
final; último
}
\end{pronuncia}
\end{verbete}

\begin{verbete}[zui4jin4]{最近}
\begin{pronuncia}{zui4jin4}
\significado{adv.}{
ultimamente; recentemente
}
\end{pronuncia}
\end{verbete}

\begin{verbete}[zuo2tian1]{昨天}
\begin{pronuncia}{zuo2tian1}
\significado{p.t.}{
ontem
}
\end{pronuncia}
\end{verbete}

\begin{verbete}[zuo3]{左}
\begin{pronuncia}{zuo3}
\significado{p.l.}{
esquerda
}
\end{pronuncia}
\end{verbete}

\begin{verbete}[zuo3bian0]{左边}
\begin{pronuncia}{zuo3bian0}
\significado{p.l.}{
esquerda; lado esquerdo
}
\end{pronuncia}
\end{verbete}

\begin{verbete}[zuo3mian0]{左面}
\begin{pronuncia}{zuo3mian0}
\significado{p.l.}{
esquerda; lado esquerdo
}
\end{pronuncia}
\end{verbete}

\begin{verbete}[zuo3you4]{左右}
\begin{pronuncia}{zuo3you4}
\significado{part.}{
cerca de; aproximadamente
}
\end{pronuncia}
\end{verbete}

\begin{verbete}[zuo4]{坐}
\begin{pronuncia}{zuo4}
\significado{v.}{
sentar-se|
andar de carro, ônibus, trem, avião, etc
}
\end{pronuncia}
\end{verbete}

\begin{verbete}[zuo4]{做}
\begin{pronuncia}{zuo4}
\significado{v.}{
fazer
}
\end{pronuncia}
\end{verbete}

\end{multicols*}

\end{DictionaryEntries} 

\ifdraftdoc
%%%
\else

\clearpage
\chapter{Termos Gramaticais Chineses}
%%%%%%%%%%%%%%%%%%%%%%%%%%%%%%%%%%%%%%%%%%%%%%%%%%%%%%%%%%%%%%%%%%%%%%%%%%%%%%%
%%%%%%%%%%%%%%%%%%%%%%%%%%%%%%%%%%%%%%%%%%%%%%%%%%%%%%%%%%%%%%%%%%%%%%%%%%%%%%%
%%%%%                                                                     %%%%%
%%%%% termos_gramaticais.tex:                                             %%%%%
%%%%% Tabela dos termos gramaticais utilizados neste Dicionário.          %%%%%
%%%%%                                                                     %%%%%
%%%%%%%%%%%%%%%%%%%%%%%%%%%%%%%%%%%%%%%%%%%%%%%%%%%%%%%%%%%%%%%%%%%%%%%%%%%%%%%
%%%%%%%%%%%%%%%%%%%%%%%%%%%%%%%%%%%%%%%%%%%%%%%%%%%%%%%%%%%%%%%%%%%%%%%%%%%%%%%

%%% Tabela
\begin{center}
\begin{tblr}[m]{lll}
substantivo/nome & \textbf{s.} & 名词 \\
pronome & \textbf{pron.} & 代词 \\
numeral & \textbf{num.} & 数词 \\
classificador & \textbf{clas.} & 量词 \\
verbo & \textbf{v.} & 动词 \\
verbo auxiliar & \textbf{v.aux.} & 助动词 \\
verbo+complemento & \textbf{v.+compl.} & 动宾式\hspace{1em}离合词 \\
adjetivo & \textbf{adj.} & 形容词 \\
advérbio & \textbf{adv.} & 副词 \\
preposição & \textbf{prep.} & 介词 \\
conjunção & \textbf{conj.} & 连词 \\
partícula & \textbf{part.} & 助词 \\
interjeição & \textbf{interj.} & 叹词 \\
prefixo & \textbf{pref.} & 前缀 \\
sufixo & \textbf{suf.} & 后缀 \\
expressão & \textbf{expr.} & 熟语 \\
\end{tblr}
\end{center}

%%%%% EOF %%%%%


\clearpage
\chapter{Classificadores Nominais}
%%%%%%%%%%%%%%%%%%%%%%%%%%%%%%%%%%%%%%%%%%%%%%%%%%%%%%%%%%%%%%%%%%%%%%%%%%%%%%%
%%%%%%%%%%%%%%%%%%%%%%%%%%%%%%%%%%%%%%%%%%%%%%%%%%%%%%%%%%%%%%%%%%%%%%%%%%%%%%%
%%%%%                                                                     %%%%%
%%%%% classificadores_nominais.tex:                                       %%%%%
%%%%% Tabela com as palavras classificadoras para substantivos            %%%%%
%%%%%                                                                     %%%%%
%%%%%%%%%%%%%%%%%%%%%%%%%%%%%%%%%%%%%%%%%%%%%%%%%%%%%%%%%%%%%%%%%%%%%%%%%%%%%%%
%%%%%%%%%%%%%%%%%%%%%%%%%%%%%%%%%%%%%%%%%%%%%%%%%%%%%%%%%%%%%%%%%%%%%%%%%%%%%%%

%%% Ajustes para a tabela
\DefTblrTemplate{caption}{default}{}
\DefTblrTemplate{capcont}{default}{ \UseTblrTemplate{conthead-text}{default} }
\DefTblrTemplate{contfoot-text}{default}{Continua na próxima página.}
\DefTblrTemplate{conthead-text}{default}{(Continuação)}
\DefTblrTemplate{firsthead}{default}{ \UseTblrTemplate{caption}{default} }
\DefTblrTemplate{middlehead,lasthead}{default}{ \UseTblrTemplate{conthead}{default} }
\DefTblrTemplate{firstfoot,middlefoot}{default}{ \UseTblrTemplate{contfoot}{default} }
\DefTblrTemplate{lastfoot}{default}{ \UseTblrTemplate{note}{default} \UseTblrTemplate{remark}{default} }

%%% Tabela
\begin{longtblr}
{
 colspec = {|c|c|X|X|}, hlines,
 width = 1\linewidth,
 rowhead = 1, rowfoot = 0,
 row{1} = {font=\bfseries, fg=white, bg=black},
}
\textbf{Hanzi} & \textbf{Pinyin} & \textbf{Descrição} & \textbf{Exemplos}\\
 把 & \dpy{ba3}     & mão cheia --- objetos que podem ser segurados, objetos relativamente longos e planos & faca, tesoura, espada, chave, guarda-chuva, leque, escova de dentes, colher, garfo, martelo, cadeado, pistola, rifle, carne, punhado de arroz, punhado de areia, ramo de flores, punhado de sementes, ramo de pauzinhos, esqueleto, fogo, bule\\
 班 & \dpy{ban1}    & serviços programados (trens, aviões, etc), grupos de pessoas, uma classe como em alunos & \\
 包 & \dpy{bao1}    & pacote & doces, cigarros, açúcar, biscoitos\\
 杯 & \dpy{bei1}    & copo --- bebidas & chá, vinho, álcool, água, leite, suco de fruta, refrigerante\\
 本 & \dpy{ben3}    & volume --- material impresso encadernado & livro, revista, romance, escritura, dicionário, bloco de notas, livro didático\\
 笔 & \dpy{bi3}     & traços de caracteres; grandes quantidades de dinheiro & \\
 部 & \dpy{bu4}     & máquinas, veículos; produções & celular, telefone, carro, jogo, romance, filme/imagem em movimento, ópera, obra literária\\
 册 & \dpy{ce4}     & volumes de livros & \\
 层 & \dpy{ceng2}   & andar, piso; camada & andares (em um prédio); camada de poeira, bolo, tinta\\
 场 & \dpy{chang3}  & evento de curta duração; precipitação & espetáculo público, jogo, crise, guerra, catástrofe, uma doença, performance, jogo, chuva, queda de neve\\
 串 & \dpy{chuan4}  & conjuntos de números & telefone celular/número de celular\\
 床 & \dpy{chuang4} & cama & cobertores, lençóis\\
 次 & \dpy{ci4}     & tempo, repetições --- oportunidades, acidentes & \\
 出 & \dpy{chu1}    & atuação em uma peça & \\
 打 & \dpy{da2}     & dúzia & lápis, ovos\\
 贷 & \dpy{dai4}    & sacos ou bolsos cheios & açúcar, farinha, arroz\\
 道 & \dpy{dao4}    & projeções lineares (raios de luz, etc.); ordem dada por uma figura de autoridade; pergunta, memorando; curso (de comida); coisas longas e tortas (cume da montanha, relâmpago) & \\
 滴 & \dpy{di1}     & gotículas de água, sangue, outros fluidos semelhantes & \\
 点 & \dpy{dian3}   & ideias, sugestões; um pouco/algum (somente com 一) & \\
 碟 & \dpy{die2}    & pires (molho de soja) & \\
 顶 & \dpy{ding3}   & objetos com topo saliente, algo para colocar sobre a cabeça & chapéu, barraca, mosquiteiro\\
 栋 & \dpy{dong4}   & pilar (edifício menor, casa) & \\
 堵 & \dpy{du3}     & luminárias abrangentes (parede sem teto) & \\
 段 & \dpy{duan4}   & comprimento adjacente --- cabos, estradas, pedaço de giz, parte de uma música & \\
 对 & \dpy{dui4}    & casal, par combinado (para certas coisas), dísticos & casal, brincos, vasos\\
 顿 & \dpy{dun4}    & ações de curta duração & refeição, conflito, espancamento, briga, repreensão\\
 朵 & \dpy{duo3}    & coisas parecidas com flores & flor, nuvem, cogumelo\\
 发 & \dpy{fa1}     & coisas redondas & bala, munição\\
 方 & \dpy{fang4}   & coisas quadradas & pedra de tinta, bacon\\
 份 & \dpy{fen4}    & porções, documentos de várias páginas & porção de comida, jornal, emprego\\
 封 & \dpy{feng1}   & coisas que podem ser seladas & carta, correio, telegrama\\
 峰 & \dpy{feng1}   & animais com corcundas & camelo\\
 幅 & \dpy{fu2}     & objetos semelhantes a imagens & foto, pintura, desenho, banner, obra de arte, quadro, cartaz\\
 服 & \dpy{fu4}     & dose (de remédio)\\
 副 & \dpy{fu4}     & objetos que vêm em pares (luvas, óculos, etc.), baralhos, mahjong & \\
 个 & {\dpy{ge5}\\ \dpy{ge4}} & coisas individuais, pessoas, classificador de uso geral (o uso desse classificador em conjunto com qualquer substantivo é geralmente aceito se a pessoa não souber o classificador adequado) & pessoa, irmão mais velho, estudante, parente, modo de pensar, sugestão, pergunta, nação\\
 根 & \dpy{gen1}    & objetos finos e esguios; fios finos e flexíveis & agulha, pilar, banana, palito de massa frita, coxa de frango frita, picolé, pirulito, pauzinho, vela, incenso, cabelo, linha, barbante\\
 股 & \dpy{gu3}     & fluxos (de ar, cheiro, influência, etc.) & \\
 挂 & \dpy{gua4}    & coisas multi-componentes & cavalo e carroça \\
 罐 & \dpy{guan4}   & lata pequena a média & refrigerante, suco, comida, feijão, óleo, doce\\
 行 & \dpy{hang2}   & objetos que formam linhas (palavras, etc.) & \\
 盒 & \dpy{he2}     & caixa pequena & fita, comida, bolo, doces, chocolate, brinquedos, livros, cigarros, detergente, roupas\\
 户 & \dpy{hu4}     & famílias & \\
 伙 & \dpy{huo3}    & classificador geralmente depreciativo para bandos de pessoas, como gangues ou bandidos & \\
 家 & \dpy{jia1}    & reunião de pessoas, estabelecimentos & famílias, empresas, lojas, restaurantes\\
 架 & \dpy{jia4}    & maquinário, veículo & aeronave, avião, piano, máquinas, computadores\\
 间 & \dpy{jian1}   & quartos, espaços & quarto, dormitório, cozinha, sala de aula, casa, escola, empresa, capela\\
 件 & \dpy{jian4}   & assuntos, roupas (tops), móveis & roupa (top), camiseta, camisa, casaco, lençol, bagagem, presente, questão/matéria/coisa\\
 讲 & \dpy{jiang3}  & longos períodos de aula & \\
 节 & \dpy{jie2}    & seção (de bambu, período curto de aula) & \\
 届 & \dpy{jie4}    & sessões ou reuniões agendadas regularmente, grupos de anos em uma escola (por exemplo, Turma de 2025) & \\
 句 & \dpy{ju4}     & linhas de poemas, frases & \\
 棵 & \dpy{ke1}     & árvores ou outra flora semelhante & pinheiro, rosa\\
 颗 & \dpy{ke1}     & pequenos objetos, objetos que parecem pequenos & corações, pérolas, dentes, diamantes, sementes, estrelas distantes, planetas distantes\\
 课 & \dpy{ke4}     & lições em um texto & \\
 口 & \dpy{kou3}    & população de aldeias (número inferior a 100), familiares, poços, bocados de comida & \\
 块 & \dpy{kuai4}   & pedaço de forma irregular & terra, pedra, tofu, sabonete, pedaço/fatia de bolo, pão (não fatias), melancia, carne, queijo, pizza, chiclete, toalha de mesa, relógio de pulso, bloco de incenso\\
 类 & \dpy{lei4}    & objetos do mesmo tipo ou natureza & \\
 粒 & \dpy{li4}     & grão & um (único) amendoim, uva, arroz cru, semenete, doce, bala, chocolate\\
 辆 & \dpy{liang4}  & veículos com rodas (não trens) & automóvel, bicicleta, carro\\
 列 & \dpy{lie4}    & trens & \\
 轮 & \dpy{lun2}    & lua & \\
 枚 & \dpy{mei2}    & medalhas, pequenas coisas planas como selos, cascas de banana, anéis, distintivos, foguetes, mísseis & \\
 门 & \dpy{men2}    & objetos pertencentes a acadêmicos (cursos, disciplinas, etc.); artilharia (canhão) & \\
 面 & \dpy{mian4}   & objetos planos e lisos & espelho, bandeira, parede com telhado, escudo\\
 名 & \dpy{ming4}   & pessoas de alto escalão (médicos, advogados, políticos, realeza, etc.), membros; em linguagem formal pode ser utilizado para qualquer pessoa não necessariamente de alto escalão & \\
 排 & \dpy{pai2}    & linhas --- objetos agrupados em linhas & cadeiras, assentos, mesas, filas de pessoas\\
 盘 & \dpy{pan2}    & objetos planos ou bobinas & cassete de vídeo ou áudio, pratinho, bobina de incenso\\
 批 & \dpy{pi1}     & pessoas, bens, etc. & \\
 匹 & \dpy{pi3}     & cavalos e outras montarias; rolos/pedaços de panos & cavalo, lobo\\
 篇 & \dpy{pian1}   & escritos & papel, artigo, ensaio, relatório\\
 片 & \dpy{pian4}   & fatia - objetos finos, planos, às vezes irregulares & cartão, lábio, nuvem, praia, chiclete, fatia de pão, fatia de carne, biscoito, queijo, fatia de melancia, folha, pétala de flor, campo, lago, pílula (comprimido de remédio), DVD\\
 瓶 & \dpy{ping2}   & garrafa & álcool, água, óleo, cerveja, bebida, vinho, refrigerante, leite, shampoo\\
 期 & \dpy{qi1}     & revistas & \\
 群 & \dpy{qun2}    & grupo (incl. pessoas), rebanho, multidão, enxame, etc. & pessoas, rebanho de gado, bando de pássaros, bando de cães, enxame de mosquitos, colônia de abelhas/formigas\\
 任 & \dpy{ren4}    & mandato (presidente, senador, deputado, etc.) & \\
 扇 & \dpy{shan4}   & coisas que abrem e fecham com dobradiças & janela, porta\\
 首 & \dpy{shou1}   & coisas com versos & canção, poema \\
 束 & \dpy{shu4}    & cachos & flores, uvas \\
 双 & \dpy{shuang1} & par de objetos que naturalmente vêm em pares & pauzinhos, sapatos, luvas, olhos\\
 艘 & \dpy{sou1}    & navios & \\
 所 & \dpy{suo3}    & pequenos edifícios, instituições & universidade, casa independente\\
 台 & \dpy{tai2}    & objetos pesados, especialmente máquinas; apresentações & TV, computador, piano, aparelho, avião, trem, carro, elevador; apresentação de teatro, jogo\\
 躺 & \dpy{tang2}   & períodos de aulas, suítes de imóveis & aulas, lições, leituras\\
 趟 & \dpy{tang4}   & viagens, serviços de transportes programados & \\
 套 & \dpy{tao4}    & conjuntos & livros, revistas, colecionáveis, roupas, casas/apartamentos com vários cômodos, suítes, selos, móveis, quartos, ternos\\
 题 & \dpy{ti2}     & classificador de perguntas & \\
 条 & \dpy{tiao2}   & objetos longos, estreitos e flexíveis & peixe, cobra, dragão, minhoca, cachorro, cachecol, estrada, fita, rio, raiz, caule, corda, edredom, toalha, fio dental, calças, gravata, saia, sofá/banco, pessoa heróica, barco pequeno\\
 帖 & \dpy{tie4}    & bandagens adesivas & \\
 通 & \dpy{tong1}   & conversa, palestra & \\
 桶 & \dpy{tong3}   & jarro, balde, barril & jarro de leite, barril de óleo\\
 头 & \dpy{tou2}    & cabeça de gado & porco, vaca, bois, iaques, ovelhas, burros, mulas, leopardos, dinossauros\\
 团 & \dpy{tuan2}   & bola --- objetos redondos e enrolados & bola de lã, etc. \\
 碗 & \dpy{wan3}    & tigela & de arroz, de macarrão, de sopa\\
 位 & \dpy{wei4}    & classificador educado e respeitoso para pessoas & professor, cliente, colega\\
 项 & \dpy{xiang4}  & projetos & \\
 些 & \dpy{xie1}    & alguns & somente com 一,这,那,哪\\
 样 & \dpy{yang4}   & itens gerais de diferentes atributos & \\
 页 & \dpy{ye4}     & página & \\
 则 & \dpy{ze2}     & diário, registro do dia & \\
 扎 & \dpy{zha1}    & jarra, caneca --- usado em cantonês no lugar de 束 \dpy{shu4} (por exemplo, um pacote de flores) & bebidas como cerveja, refrigerante, suco, etc. (pint/jar: empréstimo linguístico do inglês, pode ser considerado informal ou gíria)\\
 盏 & \dpy{zhan3}   & luminárias (geralmente lâmpadas), bule de chá, etc. & \\
 站 & \dpy{zhan4}   & paradas (de ônibus ou trens) & \\
 张 & \dpy{zhang1}  & folha --- objetos planos ou de papel & papel, mesa, cama, cartão, sofá, CD/DVD, guardanapo, fotografia, ingresso, pintura, constelação, rosto, boca, arco\\
 阵 & \dpy{zhen4}   & rajada, estouro --- eventos com duração curtas & tempestades com raios, rajadas de vento, ocorrência de chuva\\
 支 & \dpy{zhi1}    & objetos bastante longos, semelhantes a bastões & caneta, lápis, pauzinho, canudo, bambu, rosa, rifle, flecha, lança, projétil de artilharia, míssil, cantigas\\
 只 & \dpy{zhi1}    & um de um par; animais & mão, dedo, olho, pé, cabeça, meia, luva, sapato, brinco, óculos; pássaro, frango, gato, tigre, cachorro, macaco, elefante, ovelha, rato, borboleta, rã, inseto\\
 种 & \dpy{zhong3}  & tipos & de coisas, de livros, de pessoas\\
 株 & \dpy{zhu1}    & árvores/plantas menores & arbusto, planta de arroz, planta de trigo\\
 幢 & \dpy{zhuang2} & edifício de vários andares & \\
 组 & \dpy{zu3}     & conjuntos, linhas, séries, baterias (militares) & \\
 座 & \dpy{zuo4}    & montanha, edifício & montanha, colina, estrutura, grande edifício, cidade, ponte, vila, arranha-céu, torre, templo, bloco de apartamentos\\
\end{longtblr}

%%%%% EOF %%%%%


\clearpage
\chapter{Classificadores Verbais}
\DefTblrTemplate{caption}{default}{}
\DefTblrTemplate{capcont}{default}{ \UseTblrTemplate{conthead-text}{default} }
\DefTblrTemplate{contfoot-text}{default}{Continua na próxima página.}
\DefTblrTemplate{conthead-text}{default}{(Continuação)}
\DefTblrTemplate{firsthead}{default}{ \UseTblrTemplate{caption}{default} }
\DefTblrTemplate{middlehead,lasthead}{default}{ \UseTblrTemplate{conthead}{default} }
\DefTblrTemplate{firstfoot,middlefoot}{default}{ \UseTblrTemplate{contfoot}{default} }
\DefTblrTemplate{lastfoot}{default}{ \UseTblrTemplate{note}{default} \UseTblrTemplate{remark}{default} }

\begin{longtblr}
{
  colspec = {|c|c|X|X|}, hlines,
  width = 1\linewidth,
  rowhead = 1, rowfoot = 0,
  row{1} = {font=\bfseries, fg=white, bg=black},
}
\textbf{Hanzi} & \textbf{Pinyin} & \textbf{Descrição}\\
    遍 & \dpy{bian4}  & o número de vezes que uma ação foi concluída \\
    场 & \dpy{chang3} & a duração de um evento ocorrendo dentro de outro evento\\
    次 & \dpy{ci4}    & vezes (ao contrário de 遍, 次 refere-se ao número de vezes, independente de ter sido concluído ou não)\\
    顿 & \dpy{dun4}   & ações sem repetição\\
    回 & \dpy{hui2}   & ocorrências (usado coloquialmente)\\
    声 & \dpy{sheng1} & gritos, expressões\\
    趟 & \dpy{tang4}  & viagens, visitas\\
    下 & \dpy{xia4}   & ações breves e frequentemente repentinas, muito mais comum em cantonês do que em dialetos do norte\\
\end{longtblr}


\clearpage
\chapter{Verbos Direcionais}
%%%%%%%%%%%%%%%%%%%%%%%%%%%%%%%%%%%%%%%%%%%%%%%%%%%%%%%%%%%%%%%%%%%%%%%%%%%%%%%
%%%%%%%%%%%%%%%%%%%%%%%%%%%%%%%%%%%%%%%%%%%%%%%%%%%%%%%%%%%%%%%%%%%%%%%%%%%%%%%
%%%%%                                                                     %%%%%
%%%%% verbos_direcionais.tex:                                             %%%%%
%%%%% Tabela com os verbos direcionais chineses.                          %%%%%
%%%%%                                                                     %%%%%
%%%%%%%%%%%%%%%%%%%%%%%%%%%%%%%%%%%%%%%%%%%%%%%%%%%%%%%%%%%%%%%%%%%%%%%%%%%%%%%
%%%%%%%%%%%%%%%%%%%%%%%%%%%%%%%%%%%%%%%%%%%%%%%%%%%%%%%%%%%%%%%%%%%%%%%%%%%%%%%

%%% Ajustes para a tabela
\DefTblrTemplate{caption}{default}{}
\DefTblrTemplate{capcont}{default}{ \UseTblrTemplate{conthead-text}{default} }
\DefTblrTemplate{contfoot-text}{default}{Continua na próxima página.}
\DefTblrTemplate{conthead-text}{default}{(Continuação)}
\DefTblrTemplate{firsthead}{default}{ \UseTblrTemplate{caption}{default} }
\DefTblrTemplate{middlehead,lasthead}{default}{ \UseTblrTemplate{conthead}{default} }
\DefTblrTemplate{firstfoot,middlefoot}{default}{ \UseTblrTemplate{contfoot}{default} }
\DefTblrTemplate{lastfoot}{default}{ \UseTblrTemplate{note}{default} \UseTblrTemplate{remark}{default} }

%%% Tabela
\begin{longtblr}
{
  colspec = {cccccccc},
  width = 1\linewidth,
  hlines = {white},
  vlines = {white},
  rowhead = 1, rowfoot = 0,
  row{1} = {font=\bfseries, bg=gray8, fg=black},
  column{1} = {font=\bfseries, bg=gray8, fg=black},
  cell{1}{1} = {bg=white},
  cell{2-Z}{2-Z} = {bg=gray9},
  cell{3}{8} = {bg=white},
}
 & {上\\ \normalsize descer} & {下\\ \normalsize subir} & {进\\ \normalsize entrar} & {出\\ \normalsize sair} & {回\\ \normalsize retornar} & {过\\ \normalsize atravessar} & {起\\ \normalsize levantar} \\
{来\\ \normalsize vir}  &  上来 &  下来 &  进来 &  出来 &  回来 &  过来 &  起来 \\
{去\\ \normalsize ir }  &  上去 &  下去 &  进去 &  出去 &  回去 &  过去 &  \\ 
\end{longtblr}

%%%%% EOF %%%%%


\clearpage
\chapter{Locativos}
%%%%%%%%%%%%%%%%%%%%%%%%%%%%%%%%%%%%%%%%%%%%%%%%%%%%%%%%%%%%%%%%%%%%%%%%%%%%%%%
%%%%%%%%%%%%%%%%%%%%%%%%%%%%%%%%%%%%%%%%%%%%%%%%%%%%%%%%%%%%%%%%%%%%%%%%%%%%%%%
%%%%%                                                                     %%%%%
%%%%% locativos.tex:                                                      %%%%%
%%%%% Tabela com os locativos chineses                                    %%%%%
%%%%%                                                                     %%%%%
%%%%%%%%%%%%%%%%%%%%%%%%%%%%%%%%%%%%%%%%%%%%%%%%%%%%%%%%%%%%%%%%%%%%%%%%%%%%%%%
%%%%%%%%%%%%%%%%%%%%%%%%%%%%%%%%%%%%%%%%%%%%%%%%%%%%%%%%%%%%%%%%%%%%%%%%%%%%%%%

%%% Ajustes para a tabela
\DefTblrTemplate{caption}{default}{}
\DefTblrTemplate{capcont}{default}{ \UseTblrTemplate{conthead-text}{default} }
\DefTblrTemplate{contfoot-text}{default}{Continua na próxima página.}
\DefTblrTemplate{conthead-text}{default}{(Continuação)}
\DefTblrTemplate{firsthead}{default}{ \UseTblrTemplate{caption}{default} }
\DefTblrTemplate{middlehead,lasthead}{default}{ \UseTblrTemplate{conthead}{default} }
\DefTblrTemplate{firstfoot,middlefoot}{default}{ \UseTblrTemplate{contfoot}{default} }
\DefTblrTemplate{lastfoot}{default}{ \UseTblrTemplate{note}{default} \UseTblrTemplate{remark}{default} }

%%% Tabela
\begin{longtblr}
{
 colspec = {cccccc},
 width = 1\linewidth,
 hlines = {white},
 vlines = {white},
 rowhead = 1, rowfoot = 0,
 row{1} = {font=\bfseries, bg=gray8, fg=black},
 column{1} = {font=\bfseries, bg=gray8, fg=black},
 cell{1}{1} = {bg=white},
 cell{2-Z}{2-Z} = {bg=gray9},
 cell{6}{5-6} = {bg=white},
 cell{7}{2-4} = {bg=white},
 cell{9}{2-5} = {bg=white},
 cell{10}{3-6} = {bg=white},
 cell{11}{2-5} = {bg=white},
 cell{12}{4-6} = {bg=white},
 cell{13}{4-6} = {bg=white},
}
                                           & {边\\   \normalsize\dpy{bian1}}        & {面\\   \normalsize\dpy{mian4}}        & {头\\   \normalsize\dpy{tou5}}        & {以\\   \normalsize\dpy{yi3}}        & {之\\   \normalsize\dpy{zhi1}}        \\
{上\\ \normalsize\dpy{shang4}\\ sobre}     & {上边\\ \normalsize\dpy{shang4 bian1}} & {上面\\ \normalsize\dpy{shang4 mian4}} & {上头\\ \normalsize\dpy{shang4 tou5}} & {以上\\ \normalsize\dpy{yi3 shang4}} & {之上\\ \normalsize\dpy{zhi1 shang4}} \\
{下\\ \normalsize\dpy{xia4}\\ sob}         & {下边\\ \normalsize\dpy{xia4 bian1}}   & {下面\\ \normalsize\dpy{xia4 mian4}}   & {下头\\ \normalsize\dpy{xia4 tou5}}   & {以下\\ \normalsize\dpy{yi3 xia4}}   & {之下\\ \normalsize\dpy{zhi1 xia4}}   \\
{前\\ \normalsize\dpy{qian2}\\ na frente}  & {前边\\ \normalsize\dpy{qian2 bian1}}  & {前面\\ \normalsize\dpy{qian2 mian4}}  & {前头\\ \normalsize\dpy{qian2 tou5}}  & {以前\\ \normalsize\dpy{yi3 qian2}}  & {之前\\ \normalsize\dpy{zhi1 qian2}}  \\
{后\\ \normalsize\dpy{hou4}\\ atrás}       & {后边\\ \normalsize\dpy{hou4 bian1}}   & {后面\\ \normalsize\dpy{hou4 mian4}}   & {后头\\ \normalsize\dpy{hou4 tou5}}   & {以后\\ \normalsize\dpy{yi3 hou4}}   & {之后\\ \normalsize\dpy{zhi1 hou4}}   \\
{里\\ \normalsize\dpy{li3}\\ dentro}       & {里边\\ \normalsize\dpy{li3 bian1}}    & {里面\\ \normalsize\dpy{li3 mian4}}    & {里头\\ \normalsize\dpy{li3 tou5}}    &                                      &                                       \\
{内\\ \normalsize\dpy{nei4}\\ no interior} &                                        &                                        &                                       & {以内\\ \normalsize\dpy{yi3 nei4}}   & {之内\\ \normalsize\dpy{zhi1 nei4}}   \\
{外\\ \normalsize\dpy{wai4}\\ no exterior} & {外边\\ \normalsize\dpy{wai4 bian1}}   & {外面\\ \normalsize\dpy{wai4 mian4}}   & {外头\\ \normalsize\dpy{wai4 tou5}}   & {以外\\ \normalsize\dpy{yi3 wai4}}   & {之外\\ \normalsize\dpy{zhi1 wai4}}   \\
{间\\ \normalsize\dpy{jian1}\\ entre}      &                                        &                                        &                                       &                                      & {之间\\ \normalsize\dpy{zhi1 jian1}}  \\
{旁\\ \normalsize\dpy{pang2}\\ ao lado}    & {旁边\\ \normalsize\dpy{pang2 bian1}}  &                                        &                                       &                                      &                                       \\
{中\\ \normalsize\dpy{zhong1}\\ no meio}   &                                        &                                        &                                       &                                      & {之中\\ \normalsize\dpy{zhi1 zhong1}} \\
{左\\ \normalsize\dpy{zuo3}\\ à esquerda}  & {左边\\ \normalsize\dpy{zuo3 bian1}}   & {左面\\ \normalsize\dpy{zuo3 mian4}}   &                                       &                                      &                                       \\
{右\\ \normalsize\dpy{you4}\\ à direita}   & {右边\\ \normalsize\dpy{you4 bian1}}   & {右面\\ \normalsize\dpy{you4 mian4}}   &                                       &                                      &                                       \\
\pagebreak
{东\\ \normalsize\dpy{dong1}\\ no leste}   & {东边\\ \normalsize\dpy{dong1 bian1}}  & {东面\\ \normalsize\dpy{dong1 mian4}}  & {东头\\ \normalsize\dpy{dong1 tou5}}  & {以东\\ \normalsize\dpy{yi3 dong1}}  & {之东\\ \normalsize\dpy{zhi1 dong1}}  \\
{南\\ \normalsize\dpy{nan2}\\ no sul}      & {南边\\ \normalsize\dpy{nan2 bian1}}   & {南面\\ \normalsize\dpy{nan2 mian4}}   & {南头\\ \normalsize\dpy{nan2 tou5}}   & {以南\\ \normalsize\dpy{yi3 nan2}}   & {之南\\ \normalsize\dpy{zhi1 nan2}}   \\
{西\\ \normalsize\dpy{xi1}\\ no oeste}     & {西边\\ \normalsize\dpy{xi1 bian1}}    & {西面\\ \normalsize\dpy{xi1 mian4}}    & {西头\\ \normalsize\dpy{xi1 tou5}}    & {以西\\ \normalsize\dpy{yi3 xi1}}    & {之西\\ \normalsize\dpy{zhi1 xi1}}    \\
{北\\ \normalsize\dpy{bei3}\\ n norte}     & {北边\\ \normalsize\dpy{bei3 bian1}}   & {北面\\ \normalsize\dpy{bei3 mian4}}   & {北头\\ \normalsize\dpy{bei3 tou5}}   & {以北\\ \normalsize\dpy{yi3 bei3}}   & {之北\\ \normalsize\dpy{zhi1 bei3}}   \\
\end{longtblr}

%%%%% EOF %%%%%


\clearpage
\chapter{Radicais Kangxi}
%%%%%%%%%%%%%%%%%%%%%%%%%%%%%%%%%%%%%%%%%%%%%%%%%%%%%%%%%%%%%%%%%%%%%%%%%%%%%%%
%%%%%%%%%%%%%%%%%%%%%%%%%%%%%%%%%%%%%%%%%%%%%%%%%%%%%%%%%%%%%%%%%%%%%%%%%%%%%%%
%%%%%                                                                     %%%%%
%%%%% radicais_kangxi.tex:                                                %%%%%
%%%%% Lista dos 214 radicais Kangxi utilizados nos caracteres chineses.   %%%%%
%%%%%                                                                     %%%%%
%%%%%%%%%%%%%%%%%%%%%%%%%%%%%%%%%%%%%%%%%%%%%%%%%%%%%%%%%%%%%%%%%%%%%%%%%%%%%%%
%%%%%%%%%%%%%%%%%%%%%%%%%%%%%%%%%%%%%%%%%%%%%%%%%%%%%%%%%%%%%%%%%%%%%%%%%%%%%%%

%%% Ajustes para a tabela
\DefTblrTemplate{caption}{default}{}
\DefTblrTemplate{capcont}{default}{ \UseTblrTemplate{conthead-text}{default} }
\DefTblrTemplate{contfoot-text}{default}{Continua na próxima página.}
\DefTblrTemplate{conthead-text}{default}{(Continuação)}
\DefTblrTemplate{firsthead}{default}{ \UseTblrTemplate{caption}{default} }
\DefTblrTemplate{middlehead,lasthead}{default}{ \UseTblrTemplate{conthead}{default} }
\DefTblrTemplate{firstfoot,middlefoot}{default}{ \UseTblrTemplate{contfoot}{default} }
\DefTblrTemplate{lastfoot}{default}{ \UseTblrTemplate{note}{default} \UseTblrTemplate{remark}{default} }

%%% Tabela
\begin{longtblr}
{
  colspec = {|r|XX|X|X|}, hlines,
  width = 1\linewidth,
  rowhead = 1, rowfoot = 0,
  row{1} = {font=\bfseries, fg=white, bg=black},
  row{2-Z} = {font=\normalfont},
}
\textbf{Nº} & \SetCell[c=2]{c}\textbf{Radical e Variantes} & 2-2 & \textbf{Tradução} & \textbf{Pinyin} \\
  1  & 一 &          & um                     & \dictpinyin{yi1}                  \\
  2  & 丨 &          & linha                  & \dictpinyin{shu4}                 \\
  3  & 丶 &          & ponto                  & \dictpinyin{dian3}                \\
  4  & 丿 & 乀,乁    & golpear                & \dictpinyin{pie3}                 \\
  5  & 乙 & 乚,乛    & segundo                & \dictpinyin{yi3}                  \\
  6  & 亅 &          & gancho                 & \dictpinyin{gou1}                 \\
  7  & 二 &          & dois                   & \dictpinyin{er4}                  \\
  8  & 亠 &          & membro                 & \dictpinyin{tou2}                 \\
  9  & 人 & 亻       & homem                  & \dictpinyin{ren2}                 \\
 10  & 儿 &          & pernas                 & \dictpinyin{er2}                  \\
 11  & 入 &          & entra                  & \dictpinyin{ru4}                  \\
 12  & 八 & 丷       & oito                   & \dictpinyin{ba1}                  \\
 13  & 冂 &          & caixa de baixo         & \dictpinyin{jiong3}               \\
 14  & 冖 &          & sobre                  & \dictpinyin{mi4}                  \\
 15  & 冫 &          & gelo                   & \dictpinyin{bing1}                \\
 16  & 几 &          & mesa                   & \dictpinyin{ji1},\dictpinyin{ji3} \\
 17  & 凵 &          & caixa aberta           & \dictpinyin{qu3}                  \\
 18  & 刀 & 刂,⺈    & faca                   & \dictpinyin{dao1}                 \\
 19  & 力 &          & poder                  & \dictpinyin{li4}                  \\
 20  & 勹 &          & embrulho               & \dictpinyin{bao1}                 \\
 21  & 匕 &          & colher                 & \dictpinyin{bi3}                  \\
 22  & 匚 &          & caixa aberta           & \dictpinyin{fang1}                \\
 23  & 匸 &          & esconderijo anexo      & \dictpinyin{xi3}                  \\
 24  & 十 &          & dez                    & \dictpinyin{shi2}                 \\
 25  & 卜 &          & místico                & \dictpinyin{bu3}                  \\
 26  & 卩 & 㔾       & foca                   & \dictpinyin{jie2}                 \\
 27  & 厂 &          & penhasco               & \dictpinyin{han4}                 \\
 28  & 厶 &          & privado                & \dictpinyin{si1}                  \\
 29  & 又 &          & novamente              & \dictpinyin{you4}                 \\
 30  & 口 &          & boca                   & \dictpinyin{kou3}                 \\
 31  & 囗 &          & lugar                  & \dictpinyin{wei2}                 \\
 32  & 土 &          & Terra                  & \dictpinyin{tu3}                  \\
 33  & 士 &          & guerreiro              & \dictpinyin{shi4}                 \\
 34  & 夂 &          & ir                     & \dictpinyin{zhi1}                 \\
 35  & 夊 &          & devagar                & \dictpinyin{sui1}                 \\
 36  & 夕 &          & tarde                  & \dictpinyin{xi1}                  \\
 37  & 大 &          & grande                 & \dictpinyin{da4}                  \\
 38  & 女 &          & mulher                 & \dictpinyin{nv3}                  \\
 39  & 子 &          & criança                & \dictpinyin{zi3}                  \\
 40  & 宀 &          & cobertura              & \dictpinyin{mian2}                \\
 41  & 寸 &          & polegada               & \dictpinyin{cun4}                 \\
 42  & 小 & ⺌,⺍    & pequeno                & \dictpinyin{xiao3}                \\
 43  & 尢 & 尣       & coxo                   & \dictpinyin{you2}                 \\
 44  & 尸 &          & cadáver                & \dictpinyin{shi1}                 \\
 45  & 屮 &          & brotar                 & \dictpinyin{che4}                 \\
 46  & 山 &          & montanha               & \dictpinyin{shan1}                \\
 47  & 川 & 巛,巜    & rio                    & \dictpinyin{chuan1}               \\
 48  & 工 &          & trabalho               & \dictpinyin{gong1}                \\
 49  & 己 &          & a si mesmo             & \dictpinyin{ji3}                  \\
 50  & 巾 &          & turbante               & \dictpinyin{jin1}                 \\
 51  & 干 &          & seco                   & \dictpinyin{gan1}                 \\
 52  & 幺 & 么       & fio curto              & \dictpinyin{yao1}                 \\
 53  & 广 &          & vasto                  & \dictpinyin{guang3}               \\
 54  & 廴 &          & passo longo            & \dictpinyin{yin3}                 \\
 55  & 廾 &          & duas mãos              & \dictpinyin{gong3}                \\
 56  & 弋 &          & atirar flecha          & \dictpinyin{yi4}                  \\
 57  & 弓 &          & arco                   & \dictpinyin{gong1}                \\
 58  & 彐 & 彑       & focinho                & \dictpinyin{ji4}                  \\
 59  & 彡 &          & cerdas                 & \dictpinyin{shan1}                \\
 60  & 彳 &          & dupla                  & \dictpinyin{chi4}                 \\
 61  & 心 & 忄,⺗    & coração                & \dictpinyin{xin1}                 \\
 62  & 戈 &          & lança                  & \dictpinyin{ge1}                  \\
 63  & 户 & 戶,戸    & por                    & \dictpinyin{hu4}                  \\
 64  & 手 & 扌,龵    & mão                    & \dictpinyin{shou3}                \\
 65  & 支 &          & ramo                   & \dictpinyin{zhi1}                 \\
 66  & 攴 & 攵       & batida                 & \dictpinyin{pu1}                  \\
 67  & 文 &          & escrita                & \dictpinyin{wen2}                 \\
 68  & 斗 &          & mergulhador            & \dictpinyin{dou3}                 \\
 69  & 斤 &          & eixo                   & \dictpinyin{jin1}                 \\
 70  & 方 &          & quadrado               & \dictpinyin{fang1}                \\
 71  & 无 & 旡       & não                    & \dictpinyin{wu2}                  \\
 72  & 日 &          & sol                    & \dictpinyin{ri4}                  \\
 73  & 曰 &          & dizer                  & \dictpinyin{yue1}                 \\
 74  & 月 &          & lua                    & \dictpinyin{yue4}                 \\
 75  & 木 &          & árvore                 & \dictpinyin{mu4}                  \\
 76  & 欠 &          & falta                  & \dictpinyin{qian4}                \\
 77  & 止 &          & parar                  & \dictpinyin{zhi3}                 \\
 78  & 歹 & 歺       & morte                  & \dictpinyin{dai3}                 \\
 79  & 殳 &          & arma                   & \dictpinyin{shu1}                 \\
 80  & 母 & 毋       & mãe                    & \dictpinyin{mu3}                  \\
 81  & 比 &          & comparar               & \dictpinyin{bi3}                  \\
 82  & 毛 &          & pelo                   & \dictpinyin{mao2}                 \\
 83  & 氏 &          & clã                    & \dictpinyin{shi4}                 \\
 84  & 气 &          & ar                     & \dictpinyin{qi4}                  \\
 85  & 水 & 氵,氺    & água                   & \dictpinyin{shui3}                \\
 86  & 火 & 灬       & fogo                   & \dictpinyin{huo3}                 \\
 87  & 爪 & 爫       & garra                  & \dictpinyin{zhao3}                \\
 88  & 父 &          & pai                    & \dictpinyin{fu4}                  \\
 89  & 爻 &          & linha                  & \dictpinyin{yao2}                 \\
 90  & 爿 & 丬       & meio tronco            & \dictpinyin{pan2}                 \\
 91  & 片 &          & fatia                  & \dictpinyin{pian4}                \\
 92  & 牙 &          & dente                  & \dictpinyin{ya2}                  \\
 93  & 牛 & 牜,⺧    & vaca                   & \dictpinyin{niu2}                 \\
 94  & 犬 & 犭       & cão                    & \dictpinyin{quan3}                \\
 95  & 玄 &          & profundo               & \dictpinyin{xuan2}                \\
 96  & 玉 & 王,玊    & jade                   & \dictpinyin{yu4}                  \\
 97  & 瓜 &          & melão                  & \dictpinyin{gua1}                 \\
 98  & 瓦 &          & telha                  & \dictpinyin{wa3}                  \\
 99  & 甘 &          & doce                   & \dictpinyin{gan1}                 \\
100  & 生 &          & vida                   & \dictpinyin{sheng1}               \\
101  & 用 &          & usar                   & \dictpinyin{yong4}                \\
102  & 田 &          & campo                  & \dictpinyin{tian2}                \\
103  & 疋 & ⺪       & roupa                  & \dictpinyin{pi3}                  \\
104  & 疒 &          & doença                 & \dictpinyin{ne4}                  \\
105  & 癶 &          & pegadas                & \dictpinyin{bo1}                  \\
106  & 白 &          & branco                 & \dictpinyin{bai2}                 \\
107  & 皮 &          & pele                   & \dictpinyin{pi2}                  \\
108  & 皿 &          & prato                  & \dictpinyin{min3}                 \\
109  & 目 & ⺫       & olho                   & \dictpinyin{mu4}                  \\
110  & 矛 &          & lança                  & \dictpinyin{mao2}                 \\
111  & 矢 &          & seta                   & \dictpinyin{shi3}                 \\
112  & 石 &          & pedra                  & \dictpinyin{shi2}                 \\
113  & 示 & 礻       & espírito               & \dictpinyin{shi4}                 \\
114  & 禸 &          & rastrear               & \dictpinyin{rou2}                 \\
115  & 禾 &          & grão                   & \dictpinyin{he2}                  \\
116  & 穴 &          & caverna                & \dictpinyin{xue2}                 \\
117  & 立 &          & ficar em pé            & \dictpinyin{li4}                  \\
118  & 竹 & ⺮       & bambu                  & \dictpinyin{zhu2}                 \\
119  & 米 &          & arroz                  & \dictpinyin{mi3}                  \\
120  & 糸 & 纟       & seda                   & \dictpinyin{mi4}                  \\
121  & 缶 &          & pote                   & \dictpinyin{fou3}                 \\
122  & 网 & 罒,罓,⺳ & rede                   & \dictpinyin{wang3}                \\
123  & 羊 & ⺶,⺷    & ovelha                 & \dictpinyin{yang2}                \\
124  & 羽 &          & pena                   & \dictpinyin{yu3}                  \\
125  & 老 & 耂       & velho                  & \dictpinyin{lao3}                 \\
126  & 而 &          & e                      & \dictpinyin{er2}                  \\
127  & 耒 &          & arado                  & \dictpinyin{lei3}                 \\
128  & 耳 &          & orelha                 & \dictpinyin{er3}                  \\
129  & 聿 & ⺺,⺻    & escova                 & \dictpinyin{yu4}                  \\
130  & 肉 & ⺼       & carne                  & \dictpinyin{rou4}                 \\
131  & 臣 &          & ministro               & \dictpinyin{chen2}                \\
132  & 自 &          & auto--                 & \dictpinyin{zi4}                  \\
133  & 至 &          & chegar                 & \dictpinyin{zhi4}                 \\
134  & 臼 &          & argamassa              & \dictpinyin{jiu4}                 \\
135  & 舌 &          & língua                 & \dictpinyin{she2}                 \\
136  & 舛 &          & opor                   & \dictpinyin{chuan3}               \\
137  & 舟 &          & barco                  & \dictpinyin{zhou1}                \\
138  & 艮 &          & pausa                  & \dictpinyin{gen3}                 \\
139  & 色 &          & cor                    & \dictpinyin{se4}                  \\
140  & 艸 & 艹       & grama                  & \dictpinyin{cao3}                 \\
141  & 虍 &          & tigre                  & \dictpinyin{hu1}                  \\
142  & 虫 &          & inseto                 & \dictpinyin{chong2}               \\
143  & 血 &          & sangue                 & \dictpinyin{xue4}                 \\
144  & 行 &          & andar                  & \dictpinyin{xing2}                \\
145  & 衣 & 衤       & roupa                  & \dictpinyin{yi1}                  \\
146  & 襾 & 西,覀    & oeste                  & \dictpinyin{ya4}                  \\
147  & 見 & 见       & ver                    & \dictpinyin{jian4}                \\
148  & 角 & ⻇       & chifre                 & \dictpinyin{jiao3}                \\
149  & 言 & 訁       & palavra                & \dictpinyin{yan2}                 \\
150  & 谷 &          & vale                   & \dictpinyin{gu3}                  \\
151  & 豆 &          & grão                   & \dictpinyin{dou4}                 \\
152  & 豕 &          & porco                  & \dictpinyin{shi3}                 \\
153  & 豸 &          & texugo                 & \dictpinyin{zhi4}                 \\
154  & 貝 & 贝       & concha                 & \dictpinyin{bei4}                 \\
155  & 赤 &          & vermelho               & \dictpinyin{chi4}                 \\
156  & 走 &          & andar                  & \dictpinyin{zou3}                 \\
157  & 足 & ⻊       & pé                     & \dictpinyin{zu2}                  \\
158  & 身 &          & corpo                  & \dictpinyin{shen1}                \\
159  & 車 & 车       & carro                  & \dictpinyin{che1}                 \\
160  & 辛 &          & amargo                 & \dictpinyin{xin1}                 \\
161  & 辰 &          & manhã                  & \dictpinyin{chen2}                \\
162  & 辵 & 辶,⻍,⻎ & caminhar               & \dictpinyin{chuo4}                \\
163  & 邑 & 阝       & cidade                 & \dictpinyin{yi4}                  \\
164  & 酉 &          & vinho                  & \dictpinyin{you3}                 \\
165  & 釆 &          & distinto               & \dictpinyin{bian4}                \\
166  & 里 &          & aldeia                 & \dictpinyin{li3}                  \\
167  & 金 & 釒       & ouro                   & \dictpinyin{jin1}                 \\
168  & 長 & 镸       & longo                  & \dictpinyin{zhang3}               \\
169  & 門 & 门       & portão                 & \dictpinyin{men2}                 \\
170  & 阜 & ⻖       & monte                  & \dictpinyin{fu4}                  \\
171  & 隶 &          & escravo                & \dictpinyin{li4}                  \\
172  & 隹 &          & pássaro de cauda curta & \dictpinyin{zhui1}                \\
173  & 雨 &          & chuva                  & \dictpinyin{yu3}                  \\
174  & 青 & 靑       & azul                   & \dictpinyin{qing1}                \\
175  & 非 &          & errado                 & \dictpinyin{fei1}                 \\
176  & 面 & 靣       & face                   & \dictpinyin{mian4}                \\
177  & 革 &          & couro                  & \dictpinyin{ge2}                  \\
178  & 韋 & 韦       & couro tingido          & \dictpinyin{wei2}                 \\
179  & 韭 &          & parecia                & \dictpinyin{jiu3}                 \\
180  & 音 &          & som                    & \dictpinyin{yin1}                 \\
181  & 頁 & 页       & folha                  & \dictpinyin{ye4}                  \\
182  & 風 & 风       & vento                  & \dictpinyin{feng1}                \\
183  & 飛 & 飞       & mosca                  & \dictpinyin{fei1}                 \\
184  & 食 & 饣,飠    & alimento               & \dictpinyin{shi2}                 \\
185  & 首 &          & cabeça                 & \dictpinyin{shou3}                \\
186  & 香 &          & perfume                & \dictpinyin{xiang1}               \\
187  & 馬 & 马       & cavalo                 & \dictpinyin{ma3}                  \\
188  & 骨 &          & osso                   & \dictpinyin{gu3}                  \\
189  & 高 & 髙       & alto                   & \dictpinyin{gao1}                 \\
190  & 髟 &          & cabelo                 & \dictpinyin{biao1}                \\
191  & 鬥 &          & luta                   & \dictpinyin{dou4}                 \\
192  & 鬯 &          & vinho                  & \dictpinyin{chang4}               \\
193  & 鬲 &          & separado               & \dictpinyin{ge2}                  \\
194  & 鬼 &          & fantasma               & \dictpinyin{gui3}                 \\
195  & 魚 & 鱼       & peixe                  & \dictpinyin{yu2}                  \\
196  & 鳥 & 鸟       & pássaro                & \dictpinyin{niao3}                \\
197  & 鹵 &          & sal                    & \dictpinyin{lu3}                  \\
198  & 鹿 &          & veado                  & \dictpinyin{lu4}                  \\
199  & 麥 & 麦       & trigo                  & \dictpinyin{mai4}                 \\
200  & 麻 &          & cânhamo                & \dictpinyin{ma2}                  \\
201  & 黃 &          & amarelo                & \dictpinyin{huang4}               \\
202  & 黍 &          & nação                  & \dictpinyin{shu3}                 \\
203  & 黑 &          & preto                  & \dictpinyin{hei1}                 \\
204  & 黹 &          & costura                & \dictpinyin{zhi3}                 \\
205  & 黽 & 黾       & rã                     & \dictpinyin{mian3}                \\
206  & 鼎 &          & tripé                  & \dictpinyin{ding3}                \\
207  & 鼓 &          & tambor                 & \dictpinyin{gu3}                  \\
208  & 鼠 & 鼡       & rato                   & \dictpinyin{shu3}                 \\
209  & 鼻 &          & nariz                  & \dictpinyin{bi2}                  \\
210  & 齊 & 齐,斉    & até                    & \dictpinyin{qi2}                  \\
211  & 齒 & 齿       & dente                  & \dictpinyin{chi3}                 \\
212  & 龍 & 龙       & dragão                 & \dictpinyin{long2}                \\
213  & 龜 & 龟       & tartaruga              & \dictpinyin{gui1}                 \\
214  & 龠 &          & flauta                 & \dictpinyin{yue4}                 \\
\end{longtblr}

%%%%% EOF %%%%%


\fi

\end{document}

%%%%% EOF %%%%
