
%%%%%%%%%%%%%%%%%%%%%%%%%%%%%%%%%%%%%%%%%
% LuaLaTex
%
% Dicionário Chinês -> Português
% Autor: Luiz Eduardo Roncato Cordeiro
%
% Licença:
% CC BY-NC-SA 3.0 (http://creativecommons.org/licenses/by-nc-sa/3.0/)
%%%%%%%%%%%%%%%%%%%%%%%%%%%%%%%%%%%%%%%%%

\documentclass[a4paper,9pt,twoside,openright,book]{memoir}

\usepackage[brazilian]{babel}
\usepackage{fontspec}
\usepackage[dvipsnames]{xcolor}
\usepackage{imakeidx}
\usepackage[inline]{enumitem}
\usepackage{zhnumber}
\usepackage{tikz}
\usepackage[hyperindex]{hyperref}
\usepackage{pifont}
\usepackage{xstring}
\usepackage{xifthen}
\usepackage{tabularray}
\usepackage[most]{tcolorbox}
\usepackage{luacode}
\usepackage{multicol}

% Meus Comandos
%%%%%%%%%%%%%%%%%%%%%%%%%%%%%%%%%%%%%%%%%%%%%%%%%%%%%%%%%%%%%%%%%%%%%%%%%%%%%%%
%%%%%%%%%%%%%%%%%%%%%%%%%%%%%%%%%%%%%%%%%%%%%%%%%%%%%%%%%%%%%%%%%%%%%%%%%%%%%%%
%%%%%                                                                     %%%%%
%%%%% Funções e Ajustes dos Documentos do Dicionário                      %%%%%
%%%%%                                                                     %%%%%
%%%%%%%%%%%%%%%%%%%%%%%%%%%%%%%%%%%%%%%%%%%%%%%%%%%%%%%%%%%%%%%%%%%%%%%%%%%%%%%
%%%%%%%%%%%%%%%%%%%%%%%%%%%%%%%%%%%%%%%%%%%%%%%%%%%%%%%%%%%%%%%%%%%%%%%%%%%%%%%

%%% Espaçamento das linhas normal
\SingleSpacing

%%% Hyperref em modo 'draft' não gera os hiperlinks
\hypersetup{final}

%%% Ajustes da separação das colunas quando em modo texto de 2 colunas
\setlength{\columnsep}{1.2em}
\setlength{\columnseprule}{0.1mm}

%%% Estilo do capítulo, o melhor que encontrei
\chapterstyle{dash}

%%% Sem identação
\setlength{\parindent}{0cm}
\setlength{\parskip}{0.15\baselineskip}

%%% Ajuste das margens do documento
\setlrmarginsandblock{3cm}{2cm}{*}
\setulmarginsandblock{2cm}{*}{1}
\checkandfixthelayout

%%% Pra evitar viúvas e órfãs
\clubpenalty=10000
\widowpenalty=10000
\raggedbottom

%%% Usando a fonte NoTofu do Google.
\babelfont{rm}[
 Renderer=Harfbuzz,
 Ligatures=TeX,
 BoldFont={NotoSerifCJKsc-Bold},
 BoldSlantedFont={NotoSansCJKsc-Regular},
 AutoFakeSlant=0.25,
 SlantedFeatures={FakeSlant=0.25},
 BoldSlantedFeatures={FakeSlant=0.25}]{Noto Serif CJK SC}
\babelfont{sf}[Renderer=Harfbuzz,Ligatures=TeX]{Noto Sans CJK SC}
\babelfont{tt}[Renderer=Harfbuzz,Ligatures=TeX]{Noto Sans Mono CJK SC}

%%% Ajustes do MultiCol: parar com a indentação do primeiro parágrafo
%\AddToHook{env/multicols/begin}{\AddToHookNext{para/begin}{\OmitIndent}}

%%% Ajustes do Sumário
\setcounter{secnumdepth}{2}
\makeatletter
\renewcommand{\@pnumwidth}{2em} 
\renewcommand{\@tocrmarg}{4em}
\makeatother
\renewcommand\cftbeforechapterskip{5pt plus 1pt}

%%% Cria 'Lista de Hanzis'
\newcommand{\listlohname}{Primeiros Hanzis}
\newlistof{listoffirsthanzis}{loh}{\listlohname}

%%% Ajustes de Cabeçalhos e Rodapés
\setheadfoot{14pt}{28pt}

% Estilo "plain"
\makefootrule{plain}{\textwidth}{\normalrulethickness}{2pt}
\ifdraftdoc
 \makeevenfoot{plain}{\thepage}{汉葡词典}{Draft}
 \makeoddfoot{plain}{Draft}{汉葡词典}{\thepage}
\else
 \makeevenfoot{plain}{\thepage}{汉葡词典}{}
 \makeoddfoot{plain}{}{汉葡词典}{\thepage}
\fi

% Estilo "dictionary"
\makepagestyle{dictionary}
\makeheadrule{dictionary}{\textwidth}{\normalrulethickness}
\makefootrule{dictionary}{\textwidth}{\normalrulethickness}{2pt}
\ifdraftdoc
 \makeevenhead{dictionary}{\rightmark}{Draft}{\leftmark}
 \makeoddhead{dictionary}{\rightmark}{Draft}{\leftmark}
 \makeevenfoot{dictionary}{\thepage}{汉葡词典}{Draft}
 \makeoddfoot{dictionary}{Draft}{汉葡词典}{\thepage}
\else
 \makeevenhead{dictionary}{\rightmark}{}{\leftmark}
 \makeoddhead{dictionary}{\rightmark}{}{\leftmark}
 \makeevenfoot{dictionary}{\thepage}{汉葡词典}{}
 \makeoddfoot{dictionary}{}{汉葡词典}{\thepage}
\fi

\newcommand{\boxedsec}[1]
 {%
  \begin{tcolorbox}%
   [%
    enhanced,%
    nobeforeafter,%
    before={\noindent},%
    colframe=black,%
    colback=black!20!white,%
    boxrule=2pt,%
    leftrule=4mm,%
    left=0mm,%
    right=0mm,%
    top=0mm,%
    bottom=0mm%
   ]
   \hfill\LARGE\bfseries#1
  \end{tcolorbox}
 }
\setsecheadstyle{\boxedsec}
\newcommand{\sectionbreak}{\phantomsection}

\newcommand{\boxedsubsec}[1]
 {%
  \begin{tcolorbox}%
   [%
    enhanced,%
    nobeforeafter,%
    before={\noindent},%
    colframe=black,%
    colback=black!10!white,%
    boxrule=2pt,%
    leftrule=2mm,%
    left=0mm,%
    right=0mm,%
    top=0mm,%
    bottom=0mm%
   ]
   \hfill\Large#1
  \end{tcolorbox}
 }
\setsubsecheadstyle{\boxedsubsec}

%%% Estilo das caixas dos verbetes
\newtcolorbox{lightbox}%
 {%
  enhanced,%
  size=fbox,%
  colframe=black,%
  colback=white,%
  boxrule=1pt,%
  toprule=3pt,%
  left=0mm,%
  right=0mm,%
  top=0mm,%
  bottom=0mm,%
  middle=0mm,%
  nobeforeafter,%
  segmentation empty,%
  before={\noindent}%
 }
\newtcolorbox{darkbox}%
 {%
  enhanced,%
  size=fbox,%
  colframe=black,%
  colback=black!5!white,%
  boxrule=1pt,%
  toprule=3pt,%
  left=0mm,%
  right=0mm,%
  top=0mm,%
  bottom=0mm,%
  middle=0mm,%
  nobeforeafter,%
  segmentation empty,%
  before={\noindent}%
 }


%%% Variáveis tipo "bool" para dizer se tem ou não os campos
%%% "Veja", "Veja também", "Synonym"" e "Antonym"
%%% nas definições dos verbetes
\newbool{hassee}
\newbool{hasseealso}
\newbool{hassynonym}
\newbool{hasantonym}

%%% Converte os pinyins numéricos em pinyins com marcação de tom
\directlua{dofile "include/tex-sx-pinyin-tonemarks.lua"}

%%% Comandos genéricos usados no Dicionário

% Função "\pinyin" faz a conversão
\protected\def\pinyin#1{\directlua{packagedata.pinyintones.convert ([==[#1]==])}}

\ExplSyntaxOn

% Comando "\dictpinyin", coloca o pinyin entre «»
\NewDocumentCommand{\dictpinyin}{m}{\guillemotleft\pinyin{#1}\guillemotright} 

% Comando "\dpy", gera a string do pinyin utilizada no Dicionário
% Este comando realiza uma série de substituições antes
\NewDocumentCommand{\dpy}{m}%
 {%
  \StrSubstitute{#1}{5}{}[\result]%
  \StrSubstitute{\result}{v}{ü}[\result]%
  \StrSubstitute{\result}{V}{Ü}[\result]%
  \edef\py{\dictpinyin{\result}}%
  \mbox{}\py
 }

% Yin, Yang e Os Oito Trigramas
\newfontfamily\dejavusans{DejaVu Sans}
\DeclareRobustCommand{\Yin}{{\dejavusans\symbol{"268A}}}
\DeclareRobustCommand{\Yang}{{\dejavusans\symbol{"268B}}}
\DeclareRobustCommand{\TrigramHeaven}{{\dejavusans\symbol{"2630}}}
\DeclareRobustCommand{\TrigramLake}{{\dejavusans\symbol{"2631}}}
\DeclareRobustCommand{\TrigramFire}{{\dejavusans\symbol{"2632}}}
\DeclareRobustCommand{\TrigramThunder}{{\dejavusans\symbol{"2633}}}
\DeclareRobustCommand{\TrigramWind}{{\dejavusans\symbol{"2634}}}
\DeclareRobustCommand{\TrigramWater}{{\dejavusans\symbol{"2635}}}
\DeclareRobustCommand{\TrigramMountain}{{\dejavusans\symbol{"2636}}}
\DeclareRobustCommand{\TrigramEarth}{{\dejavusans\symbol{"2637}}}

% Comando "\&", insere o caracgter "&"
%\DeclareRobustCommand{\&}%
% {%
%  \ifdim\fontdimen1\font>0pt%
%   \textsl{\symbol{`\&}}%
%  \else%
%   \symbol{`\&}%
%  \fi%
% }

\NewDocumentCommand{\setvar}{mm}
 {
  % clear an existing variable or allocate a new one
  \tl_clear_new:c { g__youthdoo_var_#1_tl }
  % set to the stated value
  \tl_gset:cn { g__youthdoo_var_#1_tl } { #2 }
 }

\NewExpandableDocumentCommand{\usevar}{m}
 {
  % deliver the contents
  \tl_use:c { g__youthdoo_var_#1_tl }
 }

% Ambiente "enumerate" especial utilizado no dicionário, coloca as definições 
% do verbete em uma lista numerada em linha
\NewDocumentCommand{\dictenumerate}{>{\SplitList{|}}m}
 {%
  \begin{enumerate*}[nosep,label={\arabic*},left=0pt,mode=boxed,font=\bfseries]
   \ProcessList{#1}{\insertitem}
  \end{enumerate*}
 }
\NewDocumentCommand{\insertitem}{>{\TrimSpaces}m}{\item #1}

% Ambiente "enumerate" especial utilizado no dicionário, coloca os exemplos
% das definições do verbete em uma lista numerada em linha, utilizando
% algarismos romanos
\makeatletter
\NewDocumentCommand{\dictexamples}{m>{\SplitList{|}}m}
 {%
  \def\@theword{#1}% 
  \begin{sloppypar}
   \begin{enumerate}[nosep,label=\alph*),left=0pt,mode=boxed,font=\bfseries]
    \ProcessList{#2}{\insertexample}
   \end{enumerate}
  \end{sloppypar}
 }
\NewDocumentCommand{\insertexample}{>{\TrimSpaces}m}
 {%
   \IfSubStr{#1}{===}
   {% Com traducao
     \StrCut{#1}{===}\csH\csP%
     \StrSubstitute{\csH}{\@theword}{\underline{\@theword}}[\csHUL]%
     \item\foreignlanguage{chinese-hans}{\csHUL}\\*\textit{\footnotesize``\csP''}
   }
   {% Sem traducao
     \StrSubstitute{#1}{\@theword}{\underline{\@theword}}[\csHUL]%
     \item\foreignlanguage{chinese-hans}{\csHUL}
   }
 }  
\makeatother

%%% Cria listas especializadas (seelist, seealsolist, synonymlist e antonymlist)
\newlist{seelist}{enumerate}{1}
\newlist{seealsolist}{enumerate}{1}
\newlist{synonymlist}{enumerate}{1}
\newlist{antonymlist}{enumerate}{1}

\setlist[seelist]{label={\roman*)},topsep=0pt,nosep,noitemsep,font=\bfseries}
\setlist[seealsolist]{label={\roman*)},topsep=0pt,nosep,noitemsep,font=\bfseries}
\setlist[synonymlist]{label={\roman*)},topsep=0pt,nosep,noitemsep,font=\bfseries}
\setlist[antonymlist]{label={\roman*)},topsep=0pt,nosep,noitemsep,font=\bfseries}

%%% Cria e inicializa a lista "Veja", "Veja também", "Sinônimo(s)" e "Antônimo(s)"
\newcommand\seerefl{}
\newcommand\seealsorefl{}
\newcommand\synonymrefl{}
\newcommand\antonymrefl{}

%%% Comando "\areference", adiciona um item "Veja", "Veja também", "Antônimo(s)" ou "Sinônimo(s)" na lista,
%%% com os pinyins abaixo dos caracteres
\newcommand{\areference}[2]
 {
  \StrLen{#1}[\hlen]%
  \StrLen{#2}[\plen]%
  \ifnumcomp{\hlen+\plen}{>}{24}
   {%
    \foreignlanguage{chinese-hans}{#1}\ (pág.~\pageref{\l_label_tl : #1 : #2})\\*
    \dpy{#2}
   }
   {%
    \foreignlanguage{chinese-hans}{#1}\ \dpy{#2}\ (pág.~\pageref{\l_label_tl : #1 : #2})
   }
 }

%%% Comando "\definition", gera o texto da definição
\NewDocumentCommand{\definition}{sommo}
 {%
  \begin{midsloppypar}
   \IfBooleanTF{#1}%
    {% Substantivo Próprio
     {\small\ding{108}}\ (\textit{S.P.})\IfValueT{#2}{~[clas.:~#2]}{\ \dictenumerate{#4}}\par
    }%
    {%
     {\small\ding{108}}\ (\textit{#3})\IfValueT{#2}{~[clas.:~#2]}{\ \dictenumerate{#4}}\par
    }%
  \end{midsloppypar}
  \IfValueT{#5}%
   {%
    \IfSubStr{#5}{|}{\textbf{Exemplos:}}{\textbf{Exemplo:}}\dictexamples{\l_hanzi_tl}{#5}\par
   }%
 }

%%% Comando "Variante de"
\NewDocumentCommand{\variantof}{m}
 {
  {\small\ding{108}}\ Variante\ de\ \foreignlanguage{chinese-hans}{#1}\ (p.~\pageref{\l_label_tl : #1 : \l_pinyin_tl})\par
 }

%%% Comando "Veja"
\NewDocumentCommand{\seeref}{m}{\booltrue{hassee}\listgadd{\seerefl}{\l_hanzi_tl : #1}}

%%% Comando "Veja também"
\NewDocumentCommand{\seealsoref}{mm}{\booltrue{hasseealso}\listgadd{\seealsorefl}{#1 : #2}}

%%% Comando "Sinônimo(s)"
\NewDocumentCommand{\synonymref}{mm}{\booltrue{hassynonym}\listgadd{\synonymrefl}{#1 : #2}}

%%% Comando "Antônimo(s)"
\NewDocumentCommand{\antonymref}{mm}{\booltrue{hasantonym}\listgadd{\antonymrefl}{#1 : #2}}

%%% Ambiente "DictionaryEntries" para definir o início das entradas dos verbetes
\NewDocumentEnvironment{DictionaryEntries}{m}%
 {%
  \tl_set:Nn \l_label_tl {#1}
  \pagestyle{dictionary}
  \twocolumn
 }%
 {%
  \onecolumn
 }%

%%% Ambiente "EntryWithPhonetic", para os verbetes
\NewDocumentEnvironment{EntryWithPhonetic}{mO{}mO{}mmooo}%
 {%
  \leavevmode
  \markboth{#1{\tiny\dpy{#3}}}{#1{\tiny\dpy{#3}}}
  \tl_set:Nn \l_hanzi_tl {#1}
  \tl_set:Nn \l_pinyin_tl {#3}
  \tl_set:Nn \l_strokes_tl {#5}
  \boolfalse{hassee}\renewcommand\seerefl{}
  \boolfalse{hasseealso}\renewcommand\seealsorefl{}
  \boolfalse{hassynonym}\renewcommand\synonymrefl{}
  \boolfalse{hasantonym}\renewcommand\antonymrefl{}
  \label{\l_label_tl : #1 : #3}
  \StrLen{#1}[\hlen]%
  \StrLen{#3}[\plen]%
  \begin{lightbox}
   \ifnumcomp{\hlen}{>}{10}
    {%
     \mbox{}\hfill\textsuperscript{\tiny(#5画)}\\
     {\Large\foreignlanguage{chinese-hans}{#1}}
    }
    {%
     {\Large\foreignlanguage{chinese-hans}{#1}}\hfill\textsuperscript{\tiny(#5画)}
    }
   \tcblower
   \ifnumcomp{\plen}{>}{25}
    {%
     {\footnotesize#2\ \dpy{#3}\ #4}\\
    }
    {%
      {\footnotesize#2\ \dpy{#3}\ #4}
    }
   \IfValueT{#7}{\mbox{}\hfill{\tiny#7}}{}%
   \IfValueT{#8}{\mbox{}\hfill{\tiny#8}}{}%
   \IfValueT{#9}{\mbox{}\hfill{\tiny#9}}{}%
   \mbox{}\hfill\IfSubStr{#6}{、}{\tiny Radicais\ #6}{\tiny Radical\ #6}
  \end{lightbox}
 }%
 {%
  \ifbool{hassee}%
   {% Processa as referências "Veja"
    \RenewDocumentCommand\do{>{\SplitArgument{1}{:}}m}{\item \areference ##1}
    \textbf{Veja:}%
    \begin{seelist}
     \dolistloop{\seerefl}
    \end{seelist}
   }{}%
  \ifbool{hasseealso}%
   {% Processa as referências "Veja também"
    \RenewDocumentCommand\do{>{\SplitArgument{1}{:}}m}{\item \areference ##1}
    \textbf{Veja\ também:}%
    \begin{seealsolist}
     \dolistloop{\seealsorefl}
    \end{seealsolist}
   }{}%
  \ifbool{hassynonym}%
   {% Processa as referências "Sinônimo"
    \RenewDocumentCommand\do{>{\SplitArgument{1}{:}}m}{\item \areference ##1}
    \textbf{Sinônimo(s):}%
    \begin{synonymlist}
     \dolistloop{\synonymrefl}
    \end{synonymlist}
   }{}%
  \ifbool{hasantonym}%
   {% Processa as referências "Antônimo"
    \RenewDocumentCommand\do{>{\SplitArgument{1}{:}}m}{\item \areference ##1}
    \textbf{Antônimo(s):}%
    \begin{antonymlist}
     \dolistloop{\antonymrefl}
    \end{antonymlist}
   }{}%
 }

%%% Ambiente "Entry", para os verbetes
\NewDocumentEnvironment{Entry}{mmmooo}%
 {%
  \leavevmode
  \markboth{#1{\tiny(#2画)}}{#1{\tiny(#2画)}}
  \tl_set:Nn \l_hanzi_tl {#1}
  \tl_set:Nn \l_strokes_tl {#2}
  \StrLen{#1}[\hlen]%
  \begin{lightbox}
   \ifnumcomp{\hlen}{>}{10}
    {%
     \mbox{}\hfill\textsuperscript{\tiny(#2画)}\\
     {\Large\foreignlanguage{chinese-hans}{#1}}
    }
    {%
     {\LARGE#1}\hfill\textsuperscript{\tiny(#2画)}
    }
   \tcblower
   \IfValueT{#4}{\mbox{}\hfill{\tiny#4}}{}%
   \IfValueT{#5}{\mbox{}\hfill{\tiny#5}}{}%
   \IfValueT{#6}{\mbox{}\hfill{\tiny#6}}{}%
   \mbox{}\hfill\IfSubStr{#3}{,}{\tiny Radicais\ #3}{\tiny Radical\ #3}
  \end{lightbox}\par
 }{}

%%% Ambiente "Phonetics", para as diversas entradas de fonética da palavra
\NewDocumentEnvironment{Phonetics}{mO{}mO{}O{}}%
 {%
  \tl_set:Nn \l_pinyin_tl {#3}
  \boolfalse{hassee}\renewcommand\seerefl{}
  \boolfalse{hasseealso}\renewcommand\seealsorefl{}
  \boolfalse{hassynonym}\renewcommand\synonymrefl{}
  \boolfalse{hasantonym}\renewcommand\antonymrefl{}
  \label{\l_label_tl : #1 : #3}
   \ding{93}\ #2\ \dpy{#3}\ #4\ \ding{93}\hfill \textbf{#5}
 }%
 {%  
  \ifbool{hassee}%
   {% Processa as referências "Veja"
    \RenewDocumentCommand\do{>{\SplitArgument{1}{:}}m}{\item \areference ##1}
    \textbf{Veja:}%
    \begin{seelist}
     \dolistloop{\seerefl}
    \end{seelist}
   }{}%
  \ifbool{hasseealso}%
   {% Processa as referências "Veja também"
    \RenewDocumentCommand\do{>{\SplitArgument{1}{:}}m}{\item \areference ##1}
    \textbf{Veja\ também:}%
    \begin{seealsolist}
     \dolistloop{\seealsorefl}
    \end{seealsolist}
   }{}%
  \ifbool{hassynonym}%
   {% Processa as referências "Sinônimo"
    \RenewDocumentCommand\do{>{\SplitArgument{1}{:}}m}{\item \areference ##1}
    \textbf{Sinônimo(s):}%
    \begin{synonymlist}
     \dolistloop{\synonymrefl}
    \end{synonymlist}
  }{}%
  \ifbool{hasantonym}%
   {% Processa as referências "Antônimo"
    \RenewDocumentCommand\do{>{\SplitArgument{1}{:}}m}{\item \areference ##1}
    \textbf{Antônimo(s):}%
    \begin{antonymlist}
     \dolistloop{\antonymrefl}
    \end{antonymlist}
   }{}%
 }

\ExplSyntaxOff

%%%%% EOF %%%%%

\hyphenation{
post-gresql
or-a-cle
mi-cro-soft
}


% Ajustes do PDF
\hypersetup{
  linktoc=page,
  colorlinks=true,
  urlcolor=blue,
  linkcolor=blue,
  citecolor=blue,
  pdftitle={汉葡词典 - Dicionário Chinês-Português},
  pdfsubject={Dicionário Chinês-Português -- Ordenado por Pinyin},
  pdfauthor={罗学凯, Luiz Eduardo Roncato Cordeiro},
  pdfkeywords={dicionário, chinês, português, instituto confúcio}
}

%%%
%%% Documento começa aqui!
%%%

\begin{document}
\addfontfeatures{CharacterWidth=Proportional}

\input{title.tex}

\pagestyle{plain}

\clearpage
\chapter{Sumário}
\begin{KeepFromToc}
 \renewcommand{\contentsname}{}% Remove \tableofcontents' title/name
 \tableofcontents
\end{KeepFromToc}

\clearpage
\chapter{汉葡词典}

%%%%%%%%%%%%%%%%%%%%%%%%
%
% https://en.wikipedia.org/wiki/Chinese_character_orders
%
%%%%%%%%%%%%%%%%%%%%%%%%

Dicionário Chinês-Português ordenado primeiro pelo pinyin de cada
caracter, depois pelo número de traços e, finalmente, pela ordem do
caracter na tabela UTF-8.

\clearpage
\pagestyle{dictionary}
\begin{DictionaryEntries}{pinyins}
 %%%
%%% A
%%%
\section*{A}
\addcontentsline{toc}{section}{A}
\begin{multicols}{2}

\begin{hanzi}[啊]{a0}
\entry{a0}{part.}{ah!;oh!|no final da sentença para expressar entusiasmo|%
no final da sentença para expressar impaciência ou o que é óbvio|%
no final de uma ordem, aviso, etc|%
no final da sentença para expressar questionamento|%
para indicar uma pausa deliberada|%
para enumerar itens}
\end{hanzi}

\begin{hanzi}[矮]{ai3}
\entry{ai3}{adj.}{baixo (estatura, dimensão, grau ou ranque)}
\end{hanzi}

\begin{hanzi}[爱]{ai4}
\entry{ai4}{n.}{amor; afeição}
\entry{ai4}{v.}{amar; gostar|ter afeição}
\end{hanzi}

\begin{hanzi}[爱好]{ai4hao4}
\entry{ai4hao4}{n.}{passatempo; interesse|\pc{个}}
\entry{ai4hao4}{v.}{ter algo como hobby; ter prazer em fazer algo}
\end{hanzi}

\begin{hanzi}[爱人]{ai4ren0}
\entry{ai4ren0}{n.}{marido ou esposa|querido ou querida|\pc{个}}
\end{hanzi}

\end{multicols}

 %%%
%%% B
%%%

\section*{B}\addcontentsline{toc}{section}{B}

\begin{entry}{八}{ba1}{2}[HSK 1][Kangxi 12][Radical ⼋]
  \definition{num.}{oito; 8}
\end{entry}

\begin{entry}{八八六}{ba1 ba1 liu4}{2,2,4}[Radicais ⼋、⼋、⼋]
  \definition{expr.}{\emph{Bye bye!} (em salas de bate-papo e mensagens de texto)}
\end{entry}

\begin{entry}{巴勒斯坦}{ba1le4si1tan3}{4,11,12,8}[Radicais ⼰、⼒、⽄、⼟]
  \definition*{s.}{Palestina}
\end{entry}

\begin{entry}{巴士}{ba1 shi4}{4,3}[HSK 4][Radicais ⼰、⼠]
  \definition[辆]{s.}{ônibus; transliteração da palavra inglesa ``bus''}
\end{entry}

\begin{entry}{巴西}{ba1xi1}{4,6}[Radicais ⼰、⾑]
  \definition*{s.}{Brasil}
\end{entry}

\begin{entry}{巴西人}{ba1xi1ren2}{4,6,2}[Radicais ⼰、⾑、⼈]
  \definition[个,位]{s.}{brasileiro | pessoa ou povo do Brasil}
  \example{他是巴西人。}[Ele é brasileiro.]
\end{entry}

\begin{entry}{巴西战舞}{ba1xi1zhan4wu3}{4,6,9,14}[Radicais ⼰、⾑、⼽、⾇]
  \definition{s.}{capoeira}
\end{entry}

\begin{entry}{吧}{ba1}{7}[Radical ⼝]
  \definition{s.}{som de estalo, som crepitante}
  \definition{v.}{puxar o cachimbo; fumar | abreviação de ``bar''}
  \seeref{吧}{ba5}
\end{entry}

\begin{entry}{拔尖}{ba2jian1}{8,6}[Radicais ⼿、⼩]
  \definition{adj.}{topo de linha | fora do comum | o melhor}
  \definition{v.+compl.}{empurrar-se para a frente | sentir que é superior aos outros}
\end{entry}

\begin{entry}{把}{ba3}{7}[HSK 3][Radical ⼿]
  \definition{clas.}{para objetos com alça | para objetos pequenos:~punhado}
  \definition{part.}{partícula tornando o substantivo seguinte um objeto direto}
  \definition{v.}{conter | alcançar | segurar}
  \seeref{把}{ba4}
\end{entry}

\begin{entry}{把柄}{ba3bing3}{7,9}[Radicais ⼿、⽊]
  \definition{s.}{(figurativo) informações que podem ser usadas contra alguém}
\end{entry}

\begin{entry}{把持}{ba3chi2}{7,9}[Radicais ⼿、⼿]
  \definition{v.}{controlar | dominar | monopolizar}
\end{entry}

\begin{entry}{把风}{ba3feng1}{7,4}[Radicais ⼿、⾵]
  \definition{v.}{estar atento | vigiar (durante uma atividade clandestina)}
\end{entry}

\begin{entry}{把关}{ba3guan1}{7,6}[Radicais ⼿、⼋]
  \definition{v.}{verificar estritamente | examinar cuidadosamente para ver se algo é feito de acordo com um padrão fixo | fazer a verificação final | guardar uma passagem, fronteira}
\end{entry}

\begin{entry}{把脉}{ba3mai4}{7,9}[Radicais ⼿、⾁]
  \definition{v.}{sentir ou tomar o pulso de alguém}
\end{entry}

\begin{entry}{把式}{ba3shi4}{7,6}[Radicais ⼿、⼷]
  \definition{s.}{pessoa qualificada em um comércio}
\end{entry}

\begin{entry}{把守}{ba3shou3}{7,6}[Radicais ⼿、⼧]
  \definition{v.}{vigiar | guardar}
\end{entry}

\begin{entry}{把玩}{ba3wan2}{7,8}[Radicais ⼿、⽟]
  \definition{v.}{brincar com | mexer com}
\end{entry}

\begin{entry}{把稳}{ba3wen3}{7,14}[Radicais ⼿、⽲]
  \definition{adj.}{confiável}
\end{entry}

\begin{entry}{把握}{ba3wo4}{7,12}[HSK 3][Radicais ⼿、⼿]
  \definition{s.}{seguro | garantia | certeza}
  \definition{v.}{agarrar | segurar | aproveitar}
\end{entry}

\begin{entry}{把戏}{ba3xi4}{7,6}[Radicais ⼿、⼽]
  \definition{s.}{acrobacia | malabarismo | truque barato}
\end{entry}

\begin{entry}{把}{ba4}{7}[Radical ⼿]
  \definition{v.}{lidar}
  \seeref{把}{ba3}
\end{entry}

\begin{entry}{爸}{ba4}{8}[HSK 1][Radical ⽗]
  \definition[个,位]{s.}{(informal) pai}
  \seeref{爸爸}{ba4ba5}
  \seealsoref{爸爸}{ba4ba5}
\end{entry}

\begin{entry}{爸爸}{ba4ba5}{8,8}[HSK 1][Radicais ⽗、⽗]
  \definition[个,位,名,群]{s.}{(informal) pai; papai; papa}
  \seeref{爸}{ba4}
\end{entry}

\begin{entry}{爸妈}{ba4ma1}{8,6}[Radicais ⽗、⼥]
  \definition{s.}{pai e mãe}
\end{entry}

\begin{entry}{罢}{ba4}{10}[Radical ⽹]
  \definition{v.}{parar | cessar | demitir | suspender | desistir | terminar}
  \seeref{罢}{ba5}
\end{entry}

\begin{entry}{霸权}{ba4quan2}{21,6}[Radicais ⾬、⽊]
  \definition{s.}{hegemonia | supremacia}
\end{entry}

\begin{entry}{吧}{ba5}{7}[HSK 1][Radical ⼝]
  \definition{part.}{indica discussão, sugestão, solicitação ou comando no final de uma frase | indica concordância ou aprovação no final de uma frase | indica uma pergunta ou especulação no final de uma frase | indica incerteza no final de uma frase | em uma frase, indica uma pausa, carrega um tom hipotético, frequentemente apresenta um contraste e implica um dilema}
  \seeref{吧}{ba1}
\end{entry}

\begin{entry}{罢}{ba5}{10}[Radical ⽹]
  \definition{part.}{partícula final, a mesma que 吧}
  \seeref{罢}{ba4}
  \seealsoref{吧}{ba5}
\end{entry}

\begin{entry}{白}{bai2}{5}[HSK 1,3][Kangxi 106][Radical ⽩]
  \definition*{s.}{sobrenome Bai}
  \definition{adj.}{branco | claro | puro; claro; simples; sem mistura; em branco | branco (como símbolo de reação) | escrito incorretamente ou pronunciado incorretamente | grátis; sem custos}
  \definition{adv.}{em vão; sem propósito; sem resultados}
  \definition{s.}{parte falada em ópera, etc.; frases de peças de teatro, etc. | dialeto local | funeral}
  \definition{v.}{explicar; apresentar; esclarecer; declarar | branquear | olhar para as pessoas com o branco dos olhos (olhar vazio, de desaprovação)}
\end{entry}

\begin{entry}{白菜}{bai2 cai4}{5,11}[HSK 3][Radicais ⽩、⾋]
  \definition[棵,个]{s.}{acelga | repolho chinês}
\end{entry}

\begin{entry}{白痴}{bai2chi1}{5,13}[Radicais ⽩、⽧]
  \definition{adj./s.}{estúpido | imbecil}
\end{entry}

\begin{entry}{白蛋白}{bai2dan4bai2}{5,11,5}[Radicais ⽩、⾍、⽩]
  \definition{s.}{albumina}
\end{entry}

\begin{entry}{白鹄}{bai2hu2}{5,12}[Radicais ⽩、⿃]
  \definition{s.}{cisne branco}
\end{entry}

\begin{entry}{白拣}{bai2jian3}{5,8}[Radicais ⽩、⼿]
  \definition{s.}{uma escolha barata}
  \definition{v.}{escolher algo que não custa nada}
\end{entry}

\begin{entry}{白萝卜}{bai2luo2bo5}{5,11,2}[Radicais ⽩、⾋、⼘]
  \definition{s.}{rabanete branco}
\end{entry}

\begin{entry}{白色}{bai2 se4}{5,6}[HSK 2][Radicais ⽩、⾊]
  \definition{s.}{cor branca}
\end{entry}

\begin{entry}{白天}{bai2 tian1}{5,4}[HSK 1][Radicais ⽩、⼤]
  \definition{adv.}{dia | de dia}
  \definition[个]{s.}{dia}
\end{entry}

\begin{entry}{白苋}{bai2xian4}{5,7}[Radicais ⽩、⾋]
  \definition{s.}{amaranto branco | brotos e folhas tenras de espinafre chinês usados como alimento}
\end{entry}

\begin{entry}{百}{bai3}{6}[HSK 1][Radical ⽩]
  \definition*{s.}{sobrenome Bai}
  \definition{num.}{cem; 100 | centena | cento}
\end{entry}

\begin{entry}{百般}{bai3ban1}{6,10}[Radicais ⽩、⾈]
  \definition{adv.}{de todas as maneiras possíveis | por todos os meios}
\end{entry}

\begin{entry}{百分}{bai3fen1}{6,4}[Radicais ⽩、⼑]
  \definition{num.}{por cento}
  \definition{s.}{porcentagem}
\end{entry}

\begin{entry}{百货}{bai3 huo4}{6,8}[HSK 4][Radicais ⽩、⾙]
  \definition{s.}{mercadorias em geral; loja de departamentos; um termo geral para bens que incluem principalmente roupas, utensílios e necessidades diárias gerais}
\end{entry}

\begin{entry}{柏树}{bai3shu4}{9,9}[Radicais ⽊、⽊]
  \definition{s.}{cipreste}
\end{entry}

\begin{entry}{摆}{bai3}{13}[HSK 4][Radical ⼿]
  \definition*{s.}{sobrenome Bai | Festival de Ganbai; uma reunião realizada nas áreas Dai durante festivais religiosos, para celebrar uma boa colheita ou para trocar materiais; geralmente se refere a uma reunião em massa}
  \definition{s.}{pêndulo; um dispositivo mecânico que controla a frequência de vibração em relógios e instrumentos | a bainha inferior de um vestido, jaqueta ou saia}
  \definition{v.}{colocar; organizar | vestir; assumir | balançar; acenar; agitar para frente e para trás | expor; declarar claramente; listar | dizer; falar | libertar-se}
\end{entry}

\begin{entry}{摆动}{bai3 dong4}{13,6}[HSK 4][Radicais ⼿、⼒]
  \definition{v.}{balançar; balançar para frente e para trás; oscilar; vibrar}
\end{entry}

\begin{entry}{摆烂}{bai3lan4}{13,9}[Radicais ⼿、⽕]
  \definition{v.}{(neologismo, gíria) parar de lutar (especialmente quando se sabe que não pode ter sucesso) | deixar tudo ir para o inferno}
\end{entry}

\begin{entry}{摆手}{bai3shou3}{13,4}[Radicais ⼿、⼿]
  \definition{v.+compl.}{gesticular com a mão (acenando, acenando adeus, etc.) | balançar os braços | acenar com as mãos}
\end{entry}

\begin{entry}{摆脱}{bai3tuo1}{13,11}[HSK 4][Radicais ⼿、⾁]
  \definition{v.}{sacudir; rejeitar; romper com; libertar-se (ou desembaraçar-se) de; livrar-se de dificuldades, escravidão, controle, etc.}
\end{entry}

\begin{entry}{败}{bai4}{8}[HSK 4][Radical ⾒]
  \definition{adj.}{dilapidado; decadente; murcho; em declínio}
  \definition{v.}{derrota; bater | falhar | quebrar; neutralizar; dissipar | arruinar; estragar; corromper | ser derrotado; perder}
\end{entry}

\begin{entry}{班}{ban1}{10}[HSK 1][Radical ⽟]
  \definition*{s.}{sobrenome Ban}
  \definition{clas.}{para grupos}
  \definition[个]{s.}{equipe| time | esquadrão | turno de trabalho | classificação}
\end{entry}

\begin{entry}{班级}{ban1 ji2}{10,6}[HSK 3][Radicais ⽟、⽷]
  \definition[个]{s.}{classe | série (na escola)}
\end{entry}

\begin{entry}{班长}{ban1 zhang3}{10,4}[HSK 2][Radicais ⽟、⾧]
  \definition[个]{s.}{monitor de classe | líder de equipe | líder de esquadrão}
\end{entry}

\begin{entry}{般}{ban1}{10}[Radical ⾈]
  \definition{s.}{espécie | tipo | classe | caminho | maneira}
  \seeref{般}{bo1}
  \seeref{般}{pan2}
\end{entry}

\begin{entry}{搬}{ban1}{13}[HSK 3][Radical ⼿]
  \definition{v.}{copiar indiscriminadamente | mover-se (ou seja, mudar-se) | mover-se (algo relativamente pesado ou volumoso) | mudar | mudar-se}
\end{entry}

\begin{entry}{搬动}{ban1dong4}{13,6}[Radicais ⼿、⼒]
  \definition{v.}{mover-se (alguma coisa) | se mudar}
\end{entry}

\begin{entry}{搬家}{ban1jia1}{13,10}[HSK 3][Radicais ⼿、⼧]
  \definition{s.}{mudança}
  \definition{v.+compl.}{mudar-se de casa}
\end{entry}

\begin{entry}{搬口}{ban1kou3}{13,3}[Radicais ⼿、⼝]
  \definition{v.}{tagarelar | (idioma) transmitir histórias | semear dissensão | contar histórias}
\end{entry}

\begin{entry}{搬弄}{ban1nong4}{13,7}[Radicais ⼿、⼶]
  \definition{v.}{causar problemas | mexer com alguém | mostrar (o que se pode fazer)}
\end{entry}

\begin{entry}{搬运}{ban1yun4}{13,7}[Radicais ⼿、⾡]
  \definition{s.}{frete | transporte}
  \definition{v.}{carregar | transportar}
\end{entry}

\begin{entry}{搬走}{ban1zou3}{13,7}[Radicais ⼿、⾛]
  \definition{v.}{carregar}
\end{entry}

\begin{entry}{板}{ban3}{8}[HSK 3][Radical ⽊]
  \definition{adj.}{rígido; não natural | duro}
  \definition{clas.}{para cartões, papéis}
  \definition{s.}{tábua; placa; prato | veneziana; persiana; refere-se especificamente aos painéis de portas de lojas | badalos (instrumento musical que marca o ritmo) | uma batida acentuada (ritmo) na música e na ópera tradicional | chefe}
  \definition{v.}{parecer sério | livrar-se de maus hábitos ou falhas}
\end{entry}

\begin{entry}{办}{ban4}{4}[HSK 2][Radical ⼒]
  \definition{v.}{lidar com | lidar | gerenciar | configurar}
\end{entry}

\begin{entry}{办法}{ban4fa3}{4,8}[HSK 2][Radicais ⼒、⽔]
  \definition[条,个]{s.}{meio (de se fazer alguma coisa) | método | medida}
\end{entry}

\begin{entry}{办公}{ban4gong1}{4,4}[Radicais ⼒、⼋]
  \definition{v.+compl.}{lidar com negócios oficiais | trabalhar (especialmente em um escritório)}
\end{entry}

\begin{entry}{办公室}{ban4gong1shi4}{4,4,9}[HSK 2][Radicais ⼒、⼋、⼧]
  \definition[间]{s.}{gabinete | escritório}
\end{entry}

\begin{entry}{办理}{ban4li3}{4,11}[HSK 3][Radicais ⼒、⽟]
  \definition{v.}{conduzir | manusear | transacionar}
\end{entry}

\begin{entry}{办事}{ban4 shi4}{4,8}[HSK 4][Radicais ⼒、⼅]
  \definition{v.}{trabalhar | lidar com assuntos; manipular transações}
\end{entry}

\begin{entry}{半}{ban4}{5}[HSK 1][Radical ⼗]
  \definition{adj.}{incompleto}
  \definition{num.}{(depois de um número) ``e meio''}
  \definition{pref.}{semi}
  \definition{s.}{metade}
\end{entry}

\begin{entry}{半年}{ban4 nian2}{5,6}[HSK 1][Radicais ⼗、⼲]
  \definition{s.}{meio ano}
\end{entry}

\begin{entry}{半球}{ban4qiu2}{5,11}[Radicais ⼗、⽟]
  \definition{s.}{hemisfério}
\end{entry}

\begin{entry}{半天}{ban4 tian1}{5,4}[HSK 1][Radicais ⼗、⼤]
  \definition{s.}{metade do dia | muito tempo | bastante tempo}
\end{entry}

\begin{entry}{半夜}{ban4 ye4}{5,8}[HSK 2][Radicais ⼗、⼣]
  \definition{adv.}{no meio da noite | metade de uma noite}
  \definition{s.}{meia-noite}
\end{entry}

\begin{entry}{半音}{ban4yin1}{5,9}[Radicais ⼗、⾳]
  \definition{s.}{semitom}
\end{entry}

\begin{entry}{伴侣}{ban4lv3}{7,8}[Radicais ⼈、⼈]
  \definition{s.}{companheiro | parceiro}
\end{entry}

\begin{entry}{帮}{bang1}{9}[HSK 1][Radical ⼱]
  \definition{clas.}{para alguém (como uma ajuda)}
  \definition{s.}{gangue | grupo | contratado (como trabalhador) | camada externa | festa | sociedade secreta}
  \definition{v.}{ajudar | apoiar}
\end{entry}

\begin{entry}{帮教}{bang1jiao4}{9,11}[Radicais ⼱、⽁]
  \definition{v.}{orientar}
\end{entry}

\begin{entry}{帮忙}{bang1 mang2}{9,6}[HSK 1][Radicais ⼱、⼼]
  \definition{v.+compl.}{ajudar | dar uma mão | estender a mão | fazer um favor}
\end{entry}

\begin{entry}{帮佣}{bang1yong1}{9,7}[Radicais ⼱、⼈]
  \definition{s.}{ajudante doméstico | servo}
\end{entry}

\begin{entry}{帮助}{bang1zhu4}{9,7}[HSK 2][Radicais ⼱、⼒]
  \definition[种]{s.}{ajuda | assistência}
  \definition{v.}{ajudar | dar assistência}
\end{entry}

\begin{entry}{棒棒糖}{bang4bang4tang2}{12,12,16}[Radicais ⽊、⽊、⽶]
  \definition[根]{s.}{pirulito}
\end{entry}

\begin{entry}{棒冰}{bang4bing1}{12,6}[Radicais ⽊、⼎]
  \definition{s.}{picolé}
\end{entry}

\begin{entry}{包}{bao1}{5}[HSK 1][Radical ⼓]
  \definition*{s.}{sobrenome Bao}
  \definition{clas.}{pacotes, sacos, sacolas, embrulhos}
  \definition[个,只]{s.}{bolsa | pacote | recipiente | embrulho}
  \definition{v.}{contratar | cobrir | segurar ou abraçar | incluir | assumir o comando | embrulhar}
\end{entry}

\begin{entry}{包办}{bao1ban4}{5,4}[Radicais ⼓、⼒]
  \definition{v.}{comandar todo o show | comprometer-se a fazer tudo sozinho}
\end{entry}

\begin{entry}{包干}{bao1gan1}{5,3}[Radicais ⼓、⼲]
  \definition{s.}{tarefa alocada}
  \definition{v.}{ter a responsabilidade total sobre um trabalho}
\end{entry}

\begin{entry}{包裹}{bao1guo3}{5,14}[HSK 4][Radicais ⼓、⾐]
  \definition[个]{s.}{pacote; embrulho}
  \definition{v.}{embrulhar; amarrar; enrolar coisas em pano ou outra coisa}
\end{entry}

\begin{entry}{包含}{bao1han2}{5,7}[HSK 4][Radicais ⼓、⼝]
  \definition{v.}{conter; implicar; incluir; conter dentro, resumir, enfatizar o que está contido dentro, focar em relações internas, muitas vezes coisas abstratas}
\end{entry}

\begin{entry}{包括}{bao1kuo4}{5,9}[HSK 4][Radicais ⼓、⼿]
  \definition{v.}{incluir; compreender; consistir em; conter, conter dentro, resumir junto, enfatizar a listagem de todas as partes, ou a citação de uma parte delas, que podem ser coisas abstratas ou concretas}
\end{entry}

\begin{entry}{包容}{bao1rong2}{5,10}[Radicais ⼓、⼧]
  \definition{adj.}{inclusivo}
  \definition{v.}{perdoar | mostrar tolerância | conter | segurar}
\end{entry}

\begin{entry}{包子}{bao1 zi5}{5,3}[HSK 1][Radicais ⼓、⼦]
  \definition[个]{s.}{pão recheado cozido no vapor}
\end{entry}

\begin{entry}{包租}{bao1zu1}{5,10}[Radicais ⼓、⽲]
  \definition{s.}{aluguel fixo para terras agrícolas}
  \definition{v.}{fretar | alugar | alugar um terreno ou uma casa para subarrendar}
\end{entry}

\begin{entry}{薄}{bao2}{16}[HSK 4][Radical ⾋]
  \definition{adj.}{fino; frágil; pouca espessura |  frio; indiferente; carente de calor; emocionalmente frio; não profundo | leve; fraco | pobre; infértil}
  \seeref{薄}{bo2}
\end{entry}

\begin{entry}{宝}{bao3}{8}[HSK 4][Radical ⼧]
  \definition{adj.}{antigo; precioso; estimado}
  \definition[个,件]{s.}{tesouro; objeto estimado; coisa preciosa | dispositivo de jogo; ferramenta de jogo | dinheiro; moeda; moeda antiga com furo quadrado no centro; moeda de prata}
  \definition{s.}{sobrenome Bao}
\end{entry}

\begin{entry}{宝宝}{bao3 bao5}{8,8}[HSK 4][Radicais ⼧、⼧]
  \definition[个]{s.}{querida; \emph{darling}; \emph{baby}; apelido para crianças}
\end{entry}

\begin{entry}{宝贝}{bao3bei4}{8,4}[HSK 4][Radicais ⼧、⾙]
  \definition{adj.}{excêntrico; estranho; imprestável; um termo depreciativo para uma pessoa incompetente ou ridícula}
  \definition[个,件]{s.}{tesouro; objeto estimado; coisa preciosa | querida; \emph{darling}; \emph{baby}; apelido para crianças}
\end{entry}

\begin{entry}{宝贵}{bao3gui4}{8,9}[HSK 4][Radicais ⼧、⾙]
  \definition{adj.}{precioso; extremamente valioso, muito raro, pode ser usado para descrever coisas específicas, também pode ser usado para descrever coisas abstratas | valioso; como um tesouro}
\end{entry}

\begin{entry}{宝石}{bao3 shi2}{8,5}[HSK 4][Radicais ⼧、⽯]
  \definition[颗,枚,块]{s.}{gema; jóia; pedra preciosa; mineral precioso que tem um brilho lindo e uma dureza de mais de sete graus, não é afetado pela atmosfera ou por produtos químicos e pode ser usado como decoração, suporte de instrumentos ou abrasivos}
\end{entry}

\begin{entry}{饱}{bao3}{8}[HSK 2][Radical ⾷]
  \definition{adj.}{ter comido até ficar satisfeito | estar cheio | cheio}
  \definition{adv.}{completamente | até estar cheio}
  \definition{v.}{satisfazer}
\end{entry}

\begin{entry}{保}{bao3}{9}[HSK 3][Radical ⼈]
  \definition*{s.}{sobrenome Bao}
  \definition{s.}{fiador
oficial responsável
sistema administrativo}
  \definition{v.}{defender | proteger |manter | preservar | conservar em boas condições | garantir | assegurar | ficar como fiador de alguém.}
\end{entry}

\begin{entry}{保安}{bao3 an1}{9,6}[HSK 3][Radicais ⼈、⼧]
  \definition{s.}{guarda de segurança}
  \definition{v.}{manter seguro | garantir a segurança}
\end{entry}

\begin{entry}{保持}{bao3chi2}{9,9}[HSK 3][Radicais ⼈、⼿]
  \definition{v.}{manter | segurar | reter | preservar}
\end{entry}

\begin{entry}{保存}{bao3cun2}{9,6}[HSK 3][Radicais ⼈、⼦]
  \definition{v.}{conservar | preservar | (computação) salvar (um arquivo, etc.)}
\end{entry}

\begin{entry}{保护}{bao3hu4}{9,7}[HSK 3][Radicais ⼈、⼿]
  \definition{s.}{proteção | salvaguarda}
  \definition{v.}{proteger | defender | salvaguardar}
\end{entry}

\begin{entry}{保护国}{bao3hu4guo2}{9,7,8}[Radicais ⼈、⼿、⼞]
  \definition{s.}{protetorado}
\end{entry}

\begin{entry}{保护剂}{bao3hu4ji4}{9,7,8}[Radicais ⼈、⼿、⼑]
  \definition{s.}{agente protetor}
\end{entry}

\begin{entry}{保护区}{bao3hu4qu1}{9,7,4}[Radicais ⼈、⼿、⼖]
  \definition{s.}{área protegida | área de conservação}
\end{entry}

\begin{entry}{保护色}{bao3hu4se4}{9,7,6}[Radicais ⼈、⼿、⾊]
  \definition{s.}{camuflagem}
\end{entry}

\begin{entry}{保护神}{bao3hu4shen2}{9,7,9}[Radicais ⼈、⼿、⽰]
  \definition{s.}{anjo da guarda | santo patrono}
\end{entry}

\begin{entry}{保护物}{bao3hu4 wu4}{9,7,8}[Radicais ⼈、⼿、⽜]
  \definition{s.}{protetor}
\end{entry}

\begin{entry}{保护性}{bao3hu4xing4}{9,7,8}[Radicais ⼈、⼿、⼼]
  \definition{s.}{proteção}
\end{entry}

\begin{entry}{保护者}{bao3hu4zhe3}{9,7,8}[Radicais ⼈、⼿、⽼]
  \definition{s.}{protetor | segurador}
\end{entry}

\begin{entry}{保护主义}{bao3hu4zhu3yi4}{9,7,5,3}[Radicais ⼈、⼿、⼂、⼂]
  \definition{s.}{protecionismo}
\end{entry}

\begin{entry}{保留}{bao3liu2}{9,10}[HSK 3][Radicais ⼈、⽥]
  \definition{v.}{reter | continuar a ter | segurar | reservar}
\end{entry}

\begin{entry}{保密}{bao3mi4}{9,11}[HSK 4][Radicais ⼈、⼧]
  \definition{v.}{manter segredo; manter algo em segredo; manter a confidencialidade}
\end{entry}

\begin{entry}{保守}{bao3shou3}{9,6}[HSK 4][Radicais ⼈、⼧]
  \definition{adj.}{retrógrado; conservador; pensamentos e conceitos que são retrógrados e não conseguem acompanhar o desenvolvimento da situação}
  \definition{v.}{manter; guardar; evitar perder}
\end{entry}

\begin{entry}{保险}{bao3xian3}{9,9}[HSK 3][Radicais ⼈、⾩]
  \definition[个]{adj./s.}{seguro}
  \definition{v.}{ter certeza | estar vinculado a}
\end{entry}

\begin{entry}{保证}{bao3zheng4}{9,7}[HSK 3][Radicais ⼈、⾔]
  \definition[个]{s.}{garantia}
  \definition{v.}{garantir}
\end{entry}

\begin{entry}{报}{bao4}{7}[HSK 3][Radical ⼿]
  \definition[份,张]{s.}{jornal | recompensa | relatório | vingança}
  \definition{v.}{anunciar | informar}
\end{entry}

\begin{entry}{报酬}{bao4chou5}{7,13}[Radicais ⼿、⾣]
  \definition{s.}{recompensa | remuneração}
\end{entry}

\begin{entry}{报到}{bao4dao4}{7,8}[HSK 3][Radicais ⼿、⼑]
  \definition{v.+compl.}{apresentar-se para o serviço | fazer check-in | registrar-se | assinar}
\end{entry}

\begin{entry}{报道}{bao4dao4}{7,12}[HSK 3][Radicais ⼿、⾡]
  \definition[个,篇,分]{s.}{história | reportagem}
  \definition{v.}{cobrir | relatar (notícias)}
\end{entry}

\begin{entry}{报告}{bao4gao4}{7,7}[HSK 3][Radicais ⼿、⼝]
  \definition[份,篇,分,个,通]{s.}{relatório | discurso | palestra | aconselhamento}
  \definition{v.}{relatar | dar a conhecer | informar}
\end{entry}

\begin{entry}{报名}{bao4ming2}{7,6}[HSK 2][Radicais ⼿、⼝]
  \definition{v.+compl.}{matricular-se | alistar-se | inscrever-se | inserir o nome de alguém}
\end{entry}

\begin{entry}{报纸}{bao4zhi3}{7,7}[HSK 2][Radicais ⼿、⽷]
  \definition[张]{s.}{jornal | diário}
\end{entry}

\begin{entry}{抱}{bao4}{8}[HSK 4][Radical ⼿]
  \definition*{s.}{sobrenome Bao}
  \definition{clas.}{braçada; medida dos dois braços}
  \definition{v.}{carregar no peito; segurar com ambos os braços; abraçar | ter o primeiro filho ou neto | adotar um bebê ou criança | ficar juntos, unidos | encaixar ou servir perfeitamente (roupas e sapatos do tamanho certo) | estimar; nutrir; abrigar; ter em mente | continuar; sobrecarregar com | chocar ovos}
\end{entry}

\begin{entry}{抱怨}{bao4yuan4}{8,9}[Radicais ⼿、⼼]
  \definition{v.}{reclamar | resmungar | abrir uma reclamação | sentir-se insatisfeito}
\end{entry}

\begin{entry}{豹子}{bao4zi5}{10,3}[Radicais ⾘、⼦]
  \definition[头]{s.}{leopardo}
\end{entry}

\begin{entry}{暴力}{bao4li4}{15,2}[Radicais ⽇、⼒]
  \definition{adj.}{violento}
  \definition{s.}{violência}
\end{entry}

\begin{entry}{暴乱}{bao4luan4}{15,7}[Radicais ⽇、⼄]
  \definition{s.}{rebelião | revolta | tumulto}
\end{entry}

\begin{entry}{暴行}{bao4xing2}{15,6}[Radicais ⽇、⾏]
  \definition{s.}{ato selvagem | atrocidade | indignação}
\end{entry}

\begin{entry}{暴雨}{bao4yu3}{15,8}[Radicais ⽇、⾬]
  \definition[场,阵]{s.}{tempestade | chuva torrencial}
\end{entry}

\begin{entry}{暴躁}{bao4zao4}{15,20}[Radicais ⽇、⾜]
  \definition{adj.}{irascível | irritável}
\end{entry}

\begin{entry}{爆米花}{bao4mi3hua1}{19,6,7}[Radicais ⽕、⽶、⾋]
  \definition{s.}{pipoca (de milho) | pipoca de arroz}
\end{entry}

\begin{entry}{爆炸}{bao4zha4}{19,9}[Radicais ⽕、⽕]
  \definition{s.}{explosão}
  \definition{v.}{explodir | detonar}
\end{entry}

\begin{entry}{杯}{bei1}{8}[HSK 1][Radical ⽊]
  \definition{clas.}{para certos recipientes de líquidos: copo, xícara, etc.}
  \definition{s.}{copo | caneca | xícara | taça | troféu}
\end{entry}

\begin{entry}{杯具}{bei1ju4}{8,8}[Radicais ⽊、⼋]
  \definition{s.}{parachoque | fiasco | (gíria) tragédia}
\end{entry}

\begin{entry}{杯子}{bei1 zi5}{8,3}[HSK 1][Radicais ⽊、⼦]
  \definition[个,只]{s.}{copo | caneca | xícara | taça}
\end{entry}

\begin{entry}{背}{bei1}{9}[HSK 2][Radical ⾁]
  \definition{v.}{estar sobrecarregado | carregar nas costas ou no ombro}
  \seeref{背}{bei4}
\end{entry}

\begin{entry}{北}{bei3}{5}[HSK 1][Radical ⼔]
  \definition{s.}{norte}
  \definition{v.}{(clássico) ser derrotado}
\end{entry}

\begin{entry}{北边}{bei3 bian1}{5,5}[HSK 1][Radicais ⼔、⾡]
  \definition{adv.}{lado norte | ao norte de}
\end{entry}

\begin{entry}{北部}{bei3 bu4}{5,10}[HSK 3][Radicais ⼔、⾢]
  \definition{s.}{parte norte}
\end{entry}

\begin{entry}{北大西洋公约组织}{bei3 da4xi1 yang2 gong1 yue1 zu3zhi1}{5,3,6,9,4,6,8,8}[Radicais ⼔、⼤、⾑、⽔、⼋、⽷、⽷、⽷]
  \definition*{s.}{Organização do Tratado do Atlântico Norte, OTAN}
\end{entry}

\begin{entry}{北方}{bei3fang1}{5,4}[HSK 2][Radicais ⼔、⽅]
  \definition{s.}{norte | a parte norte de um país}
\end{entry}

\begin{entry}{北极}{bei3ji2}{5,7}[Radicais ⼔、⽊]
  \definition*{s.}{Ártico | Pólo Norte}
  \definition{s.}{pólo norte magnético}
\end{entry}

\begin{entry}{北京}{bei3 jing1}{5,8}[HSK 1][Radicais ⼔、⼇]
  \definition*{s.}{Beijing (Pequim), Capital da República Popular da China | Beijing (Pequim), governo da RPC}
\end{entry}

\begin{entry}{北面}{bei3mian4}{5,9}[Radicais ⼔、⾯]
  \definition{s.}{lado norte}
\end{entry}

\begin{entry}{北约}{bei3yue1}{5,6}[Radicais ⼔、⽷]
  \definition*{s.}{OTAN (Organização do Tratado do Atlântico Norte), abreviação de 北大西洋公约组织}
  \seeref{北大西洋公约组织}{bei3 da4xi1 yang2 gong1 yue1 zu3zhi1}
\end{entry}

\begin{entry}{备份}{bei4fen4}{8,6}[Radicais ⼡、⼈]
  \definition{s.}{cópia de segurança | \emph{backup}}
\end{entry}

\begin{entry}{备胎}{bei4tai1}{8,9}[Radicais ⼡、⾁]
  \definition{s.}{pneu sobressalente | (gíria) substituto}
\end{entry}

\begin{entry}{背}{bei4}{9}[HSK 3][Radical ⾁]
  \definition{adv.}{a parte de trás de um corpo ou objeto}
  \definition{s.}{costas | (gíria) azarado}
  \definition{v.}{esconder algo de | decorar | recitar de memória | virar as costas}
  \seeref{背}{bei1}
\end{entry}

\begin{entry}{背后}{bei4 hou4}{9,6}[HSK 3][Radicais ⾁、⼝]
  \definition{s.}{parte de trás | traseira | nas costas de alguém}
\end{entry}

\begin{entry}{背景}{bei4jing3}{9,12}[HSK 4][Radicais ⾁、⽇]
  \definition[种]{s.}{pano de fundo; fundo; cenário de teatro, filme ou drama de TV | fundo; cenário que permeia a imagem principal na tela | condições sociais; ambientes históricos (significativamente influentes para algo ou alguém) | poder que dá suporte a alguém}
\end{entry}

\begin{entry}{倍}{bei4}{10}[HSK 4][Radical ⼈]
  \definition{adv.}{mais; especialmente}
  \definition{clas.}{vezes; para obter um número igual ao número original, você pode multiplicar o número por esse múltiplo}
  \definition{s.}{dobro; duas vezes mais}
\end{entry}

\begin{entry}{被}{bei4}{10}[HSK 3][Radical ⾐]
  \definition*{s.}{sobrenome Bei}
  \definition{part.}{usada antes de verbos para formar frases verbais passivas}
  \definition{prep.}{usado em uma frase para indicar que o sujeito é o receptor da ação}
  \definition{s.}{colcha}
  \definition{v.}{cobrir; espalhar
sofrer}
\end{entry}

\begin{entry}{被单}{bei4dan1}{10,8}[Radicais ⾐、⼗]
  \definition[床]{s.}{lençol}
\end{entry}

\begin{entry}{被动}{bei4dong4}{10,6}[Radicais ⾐、⼒]
  \definition{adj.}{passivo}
\end{entry}

\begin{entry}{被告}{bei4gao4}{10,7}[Radicais ⾐、⼝]
  \definition{s.}{réu}
\end{entry}

\begin{entry}{被迫}{bei4 po4}{10,8}[HSK 4][Radicais ⾐、⾡]
  \definition{v.}{ser forçado; ser coagido; ser compelido; ser constrangido; ser forçado a fazer algo por força externa}
\end{entry}

\begin{entry}{被套}{bei4tao4}{10,10}[Radicais ⾐、⼤]
  \definition{s.}{capa de \emph{edredon}}
  \definition{v.}{ter dinheiro preso (em ações, imóveis, etc.)}
\end{entry}

\begin{entry}{被窝}{bei4wo1}{10,12}[Radicais ⾐、⽳]
  \definition{s.}{colcha}
\end{entry}

\begin{entry}{被子}{bei4zi5}{10,3}[HSK 3][Radicais ⾐、⼦]
  \definition[床]{s.}{colcha}
\end{entry}

\begin{entry}{本}{ben3}{5}[HSK 1][Radical ⽊]
  \definition{adj.}{o atual | original | inerente}
  \definition{adv.}{originalmente}
  \definition{clas.}{para livros, dicionários, periódicos, arquivos, etc.}
  \definition{s.}{raiz | caule | origem | fonte}
\end{entry}

\begin{entry}{本金}{ben3 jin1}{5,8}[Radicais ⽊、⾦]
  \definition{s.}{capital; capital para a operação do comércio e da indústria; capital para a operação de negócios |
valor principal; dinheiro retirado ao depositar ou tomar emprestado (diferente de ``利息'')}
  \seealsoref{利息}{li4xi1}
\end{entry}

\begin{entry}{本科}{ben3ke1}{5,9}[HSK 4][Radicais ⽊、⽲]
  \definition{s.}{graduação; bacharelado; o curso básico de uma universidade ou faculdade}
\end{entry}

\begin{entry}{本来}{ben3lai2}{5,7}[HSK 3][Radicais ⽊、⽊]
  \definition{adv.}{originalmente | apropriadamente | legalmente}
\end{entry}

\begin{entry}{本领}{ben3 ling3}{5,11}[HSK 3][Radicais ⽊、⾴]
  \definition[项,个]{s.}{capacidade | faculdade | poder | habilidade | talento}
\end{entry}

\begin{entry}{本事}{ben3shi4}{5,8}[Radicais ⽊、⼅]
  \definition{s.}{habilidade | capacidade | \emph{status} | poder | posição | autoridade}
  \seeref{本事}{ben3shi5}
\end{entry}

\begin{entry}{本事}{ben3shi5}{5,8}[HSK 3][Radicais ⽊、⼅]
  \definition{s.}{habilidade | capacidade |\emph{status} | poder | posição | autoridade}
  \seeref{本事}{ben3shi4}
\end{entry}

\begin{entry}{本子}{ben3 zi5}{5,3}[HSK 1][Radicais ⽊、⼦]
  \definition[本]{s.}{caderno}
\end{entry}

\begin{entry}{笨}{ben4}{11}[HSK 4][Radical ⽵]
  \definition{adj.}{estúpido; sem graça; tolo; de pouca habilidade; sem inteligência | desajeitado; tosco; inflexível | incômodo; pesado; desajeitado; difícil de manejar; trabalhoso}
\end{entry}

\begin{entry}{笨蛋}{ben4dan4}{11,11}[Radicais ⽵、⾍]
  \definition{s.}{bobalhão | cabeça-oca | cabeça-dura}
  \definition{v.}{iludir | enganar}
\end{entry}

\begin{entry}{崩}{beng1}{11}[Radical ⼭]
  \definition{s.}{morte de rei ou imperador | desaparecimento}
  \definition{v.}{entrar em colapso | cair em ruínas}
\end{entry}

\begin{entry}{绷带}{beng1dai4}{11,9}[Radicais ⽷、⼱]
  \definition{s.}{curativo | (empréstimo linguístico) \emph{bandage}}
\end{entry}

\begin{entry}{甭}{beng2}{9}[Radical ⽤]
  \definition{v.}{contração de 不用 | não precisar}
  \seeref{不用}{bu2 yong4}
\end{entry}

\begin{entry}{蹦极}{beng4ji2}{18,7}[Radicais ⾜、⽊]
  \definition{s.}{\emph{bungee jumping}}
\end{entry}

\begin{entry}{鼻子}{bi2zi5}{14,3}[Radicais ⿐、⼦]
  \definition[个,只]{s.}{nariz}
\end{entry}

\begin{entry}{比}{bi3}{4}[HSK 1][Kangxi 81][Radical ⽐]
  \definition*{s.}{Bélgica, abreviação de 比利时}
  \definition{part.}{partícula usada para comparação (superioridade)}
  \definition{prep.}{que | do que | (seguido por um substantivo e adjetivo) mais \{adj.\} do que \{s.\}}
  \definition{s.}{razão (taxa)}
  \definition{v.}{comparar | contrastar | gesticular (com as mãos)}
  \seeref{比利时}{bi3li4shi2}
\end{entry}

\begin{entry}{比分}{bi3 fen1}{4,4}[HSK 4][Radicais ⽐、⼑]
  \definition{s.}{pontuação; comparação de pontuações entre as duas equipes em uma partida}
\end{entry}

\begin{entry}{比较}{bi3jiao4}{4,10}[HSK 3][Radicais ⽐、⾞]
  \definition{adv.}{comparativamente | relativamente}
  \definition{s.}{comparação}
  \definition{v.}{comparar}
\end{entry}

\begin{entry}{比利时}{bi3li4shi2}{4,7,7}[Radicais ⽐、⼑、⽇]
  \definition*{s.}{Bélgica}
\end{entry}

\begin{entry}{比例}{bi3li4}{4,8}[HSK 3][Radicais ⽐、⼈]
  \definition{s.}{escala | razão | proporção}
\end{entry}

\begin{entry}{比拼}{bi3pin1}{4,9}[Radicais ⽐、⼿]
  \definition{s.}{concurso}
  \definition{v.}{competir ferozmente}
\end{entry}

\begin{entry}{比如}{bi3ru2}{4,6}[HSK 2][Radicais ⽐、⼥]
  \definition{conj.}{por exemplo | como}
\end{entry}

\begin{entry}{比如说}{bi3 ru2 shuo1}{4,6,9}[HSK 2][Radicais ⽐、⼥、⾔]
  \definition{adv.}{por exemplo}
\end{entry}

\begin{entry}{比萨饼}{bi3sa4bing3}{4,11,9}[Radicais ⽐、⾋、⾷]
  \definition[张]{s.}{pizza}
\end{entry}

\begin{entry}{比赛}{bi3sai4}{4,14}[HSK 3][Radicais ⽐、⾙]
  \definition[场,次]{s.}{competição | concurso}
  \definition{v.}{competir}
\end{entry}

\begin{entry}{比亚迪}{bi3ya4di2}{4,6,8}[Radicais ⽐、⼆、⾡]
  \definition*{s.}{Montadora BYD}
\end{entry}

\begin{entry}{笔}{bi3}{10}[HSK 2][Radical ⽵]
  \definition{clas.}{para somas de dinheiro, negócios}
  \definition[支,枝]{s.}{caneta | lápis}
\end{entry}

\begin{entry}{笔记}{bi3 ji4}{10,5}[HSK 2][Radicais ⽵、⾔]
  \definition[篇,本,个]{s.}{notas | ensaios | esboços}
  \definition{v.}{tomar nota (por escrito)}
\end{entry}

\begin{entry}{笔记本}{bi3ji4ben3}{10,5,5}[HSK 2][Radicais ⽵、⾔、⽊]
  \definition[本]{s.}{caderno}
  \definition{s.}{\emph{laptop}}
\end{entry}

\begin{entry}{必定}{bi4ding4}{5,8}[Radicais ⼼、⼧]
  \definition{adv.}{sem falta | certamente | com certeza | definitivamente | inevitavelmente | com determinação}
  \definition{v.}{estar vinculado a | ter certeza de}
\end{entry}

\begin{entry}{必然}{bi4ran2}{5,12}[HSK 3][Radicais ⼼、⽕]
  \definition{adj.}{certo | inevitável | necessário}
  \definition{adv.}{inevitavelmente}
  \definition{s.}{necessidade}
\end{entry}

\begin{entry}{必须}{bi4xu1}{5,9}[HSK 2][Radicais ⼼、⾴]
  \definition{adv.}{necessariamente | obrigatoriamente}
\end{entry}

\begin{entry}{必要}{bi4yao4}{5,9}[HSK 3][Radicais ⼼、⾑]
  \definition{adj.}{necessário | essencial | indispensável}
  \definition[个,些]{s.}{necessidade}
\end{entry}

\begin{entry}{毕业}{bi4ye4}{6,5}[HSK 4][Radicais ⽐、⼀]
  \definition{s.}{formatura}
  \definition{v.+compl.}{formar-se}
\end{entry}

\begin{entry}{毕业生}{bi4 ye4 sheng1}{6,5,5}[HSK 4][Radicais ⽐、⼀、⽣]
  \definition[个]{s.}{diplomado; graduado; bacharel; pessoa que recebeu um diploma, grau ou certificado}
\end{entry}

\begin{entry}{闭嘴}{bi4zui3}{6,16}[Radicais ⾨、⼝]
  \definition{expr.}{Cale-se!}
\end{entry}

\begin{entry}{壁虎}{bi4hu3}{16,8}[Radicais ⼟、⾌]
  \definition{s.}{lagartixa}
\end{entry}

\begin{entry}{壁纸}{bi4zhi3}{16,7}[Radicais ⼟、⽷]
  \definition{s.}{papel de parede}
\end{entry}

\begin{entry}{避}{bi4}{16}[HSK 4][Radical ⾌]
  \definition{v.}{evitar; evadir; esquivar-se; buscar abrigo; fugir | impedir; manter afastado; repelir; previnir}
\end{entry}

\begin{entry}{避免}{bi4mian3}{16,7}[HSK 4][Radicais ⾌、⼉]
  \definition{v.}{evitar; desviar; abster-se de; tentar não fazer com que algo aconteça; prevenir; tentar impedir (que algo ruim aconteça) com antecedência}
\end{entry}

\begin{entry}{边}{bian1}{5}[HSK 2][Radical ⾡]
  \definition{adv.}{simultaneamente}
  \definition[个]{s.}{fronteira | limite | borda | margem | lado}
  \seeref{边}{bian5}
\end{entry}

\begin{entry}{边防}{bian1fang2}{5,6}[Radicais ⾡、⾩]
  \definition{s.}{defesa da fronteira}
\end{entry}

\begin{entry}{边关}{bian1guan1}{5,6}[Radicais ⾡、⼋]
  \definition{s.}{posto de fronteira | posição defensiva estratégica na fronteira}
\end{entry}

\begin{entry}{编}{bian1}{12}[HSK 4][Radical ⽷]
  \definition*{s.}{sobrenome Bian}
  \definition{s.}{livro; volume; parte de um livro}
  \definition{v.}{tecer; trançar; entrançar | fazer uma lista; organizar em uma lista; organizar; agrupar | editar; compilar | compor; escrever | fabricar; inventar; fazer; preparar}
\end{entry}

\begin{entry}{编程}{bian1cheng2}{12,12}[Radicais ⽷、⽲]
  \definition{s.}{programa de computador}
  \definition{v.}{programar computador}
\end{entry}

\begin{entry}{邉}{bian1}{17}[Radical ⾡]
  \variantof{边}
\end{entry}

\begin{entry}{变}{bian4}{8}[HSK 2][Radical ⼜]
  \definition{v.}{mudar | transformar | variar}
\end{entry}

\begin{entry}{变成}{bian4 cheng2}{8,6}[HSK 2][Radicais ⼜、⼽]
  \definition{v.}{mudar | transformar-se em | tornar-se}
\end{entry}

\begin{entry}{变更}{bian4geng1}{8,7}[Radicais ⼜、⽈]
  \definition{v.}{alterar | mudar | modificar}
\end{entry}

\begin{entry}{变化}{bian4hua4}{8,4}[HSK 3][Radicais ⼜、⼔]
  \definition[个]{s.}{mudança | variação}
  \definition{v.}{(intransitivo) mudar, variar}
\end{entry}

\begin{entry}{变节}{bian4jie2}{8,5}[Radicais ⼜、⾋]
  \definition{s.}{traição | deserção | vira-casaca}
  \definition{v.}{mudar de lado politicamente}
\end{entry}

\begin{entry}{变迁}{bian4qian1}{8,6}[Radicais ⼜、⾡]
  \definition{s.}{mudanças | vicissitudes}
\end{entry}

\begin{entry}{变数}{bian4shu4}{8,13}[Radicais ⼜、⽁]
  \definition{s.}{(matemática) variável}
\end{entry}

\begin{entry}{变为}{bian4 wei2}{8,4}[HSK 3][Radicais ⼜、⼂]
  \definition{v.}{transformar-se em | tornar-se | mudar para}
\end{entry}

\begin{entry}{变心}{bian4xin1}{8,4}[Radicais ⼜、⼼]
  \definition{v.+compl.}{deixar de ser fiel}
\end{entry}

\begin{entry}{变性}{bian4xing4}{8,8}[Radicais ⼜、⼼]
  \definition{s.}{desnaturação | transexual}
  \definition{v.}{desnaturar | mudar de sexo}
\end{entry}

\begin{entry}{变异}{bian4yi4}{8,6}[Radicais ⼜、⼶]
  \definition{s.}{variação | mutação}
\end{entry}

\begin{entry}{变装}{bian4zhuang1}{8,12}[Radicais ⼜、⾐]
  \definition{v.}{trocar de roupa | vestir-se | vestir uma fantasia | disfarçar-se ou fantasiar-se de personagem real ou ficcional, \emph{cosplay} | travestir-se}
\end{entry}

\begin{entry}{遍}{bian4}{12}[HSK 2][Radical ⾡]
  \definition{adv.}{em todos os lugares | por toda parte}
  \definition{clas.}{para a repetição de ações de leitura, fala ou escrita}
\end{entry}

\begin{entry}{辩论}{bian4lun4}{16,6}[HSK 4][Radicais ⾟、⾔]
  \definition[场,次]{s.}{debate; argumento; a atividade comportamental em si de argumentar ou refutar diferentes pontos de vista ou afirmações, ou uma ocasião ou situação em que tal argumentação ou refutação é feita}
  \definition{v.}{debater; obter um entendimento unificado ou correto, ambos os lados usam linguagem, palavras etc. para explicar seus pontos de vista, apontar os erros ou as contradições do outro lado}
\end{entry}

\begin{entry}{辫子}{bian4zi5}{17,3}[Radicais ⾟、⼦]
  \definition[根,条]{s.}{trança | um erro ou falha que pode ser explorado por um oponente | alça}
\end{entry}

\begin{entry}{边}{bian5}{5}[Radical ⾡]
  \definition{suf.}{sufixo de uma palavra de localidade}
  \seeref{边}{bian1}
\end{entry}

\begin{entry}{标题}{biao1ti2}{9,15}[HSK 3][Radicais ⽊、⾴]
  \definition[个,条,篇]{s.}{título | manchete | cabeçalho}
\end{entry}

\begin{entry}{标志}{biao1zhi4}{9,7}[HSK 4][Radicais ⽊、⼼]
  \definition[个,种]{s.}{sinal; marca; logotipo; símbolo; emblema; marcações que caracterizam um objeto para facilitar a identificação}
  \definition{v.}{marcar; indicar; simbolizar; identificar}
\end{entry}

\begin{entry}{标准}{biao1zhun3}{9,10}[HSK 3][Radicais ⽊、⼎]
  \definition{adj.}{criterioso | padronizado | normatizado}
  \definition[个]{s.}{critério | padrão (oficial) | norma}
\end{entry}

\begin{entry}{镖}{biao1}{16}[Radical ⾦]
  \definition{s.}{dardo | arma de arremesso | mercadorias enviadas sob a proteção de uma escolta armada}
\end{entry}

\begin{entry}{表}{biao3}{8}[HSK 2][Radical ⾐]
  \definition*{s.}{sobrenome Biao}
  \definition{s.}{superfície externa | a relação entre os filhos ou netos de um irmão e uma irmã ou de irmãs | exemplo | modelo | memorial a um imperador dos tempos antigos | gráfico | formulário | lista | tabela | medidor | relógio de pulso}
\end{entry}

\begin{entry}{表白}{biao3bai2}{8,5}[Radicais ⾐、⽩]
  \definition{s.}{declaração | confissão}
  \definition{v.}{confessar a si mesmo | expressar | revelar pensamentos ou sentimentos de alguém}
\end{entry}

\begin{entry}{表达}{biao3da2}{8,6}[HSK 3][Radicais ⾐、⾡]
  \definition{v.}{entregar | expressar | mostrar | transmitir | comunicar}
\end{entry}

\begin{entry}{表格}{biao3ge2}{8,10}[HSK 3][Radicais ⾐、⽊]
  \definition[份,张]{s.}{tabela | formulário}
\end{entry}

\begin{entry}{表面}{biao3mian4}{8,9}[HSK 3][Radicais ⾐、⾯]
  \definition{s.}{superfície | lado de fora | aparência | superficialidade}
\end{entry}

\begin{entry}{表明}{biao3ming2}{8,8}[HSK 3][Radicais ⾐、⽇]
  \definition{v.}{deixar claro | tornar conhecido | declarar claramente}
\end{entry}

\begin{entry}{表情}{biao3qing2}{8,11}[HSK 4][Radicais ⾐、⼼]
  \definition[个,种,幅]{s.}{expressão; expressão facial; expressão de pensamentos e sentimentos internos por meio de mudanças faciais ou de gestos}
  \definition{v.}{expressar pensamentos e sentimentos internos por meio de mudanças faciais ou de gestos}
\end{entry}

\begin{entry}{表示}{biao3shi4}{8,5}[HSK 2][Radicais ⾐、⽰]
  \definition{s.}{expressão | indicação}
  \definition{v.}{expressar | mostrar | indicar | significar}
\end{entry}

\begin{entry}{表现}{biao3xian4}{8,8}[HSK 3][Radicais ⾐、⾒]
  \definition[个,种,份]{s.}{desempenho | expressão  manifestação | comportamento}
  \definition{v.}{mostrar | expressar | exibir | manifestar | descrever}
\end{entry}

\begin{entry}{表演}{biao3yan3}{8,14}[HSK 3][Radicais ⾐、⽔]
  \definition[场]{s.}{representação | atuação | exposição}
  \definition{v.}{executar | atuar | jogar | demonstrar | agir | fingir}
\end{entry}

\begin{entry}{表演赛}{biao3yan3sai4}{8,14,14}[Radicais ⾐、⽔、⾙]
  \definition{s.}{partida ou jogo de exibição}
\end{entry}

\begin{entry}{表演特技}{biao3yan3 te4ji4}{8,14,10,7}[Radicais ⾐、⽔、⽜、⼿]
  \definition{s.}{acrobacia | pirueta | façanha}
\end{entry}

\begin{entry}{表演艺术}{biao3yan3 yi4shu4}{8,14,4,5}[Radicais ⾐、⽔、⾋、⽊]
  \definition{s.}{arte performática}
\end{entry}

\begin{entry}{表演游戏}{biao3yan3 you2xi4}{8,14,12,6}[Radicais ⾐、⽔、⽔、⼽]
  \definition{s.}{exibição dramática}
\end{entry}

\begin{entry}{表演者}{biao3yan3 zhe3}{8,14,8}[Radicais ⾐、⽔、⽼]
  \definition{s.}{ator}
\end{entry}

\begin{entry}{表扬}{biao3yang2}{8,6}[HSK 4][Radicais ⾐、⼿]
  \definition[次,种,份]{s.}{elogios públicos por boas ações}
  \definition{v.}{elogiar; louvar}
\end{entry}

\begin{entry}{表扬信}{biao3yang2 xin4}{8,6,9}[Radicais ⾐、⼿、⼈]
  \definition{s.}{carta de elogio | depoimento}
\end{entry}

\begin{entry}{别}{bie2}{7}[HSK 1,4][Radical ⼑]
  \definition*{s.}{sobrenome Bie}
  \definition{adv.}{não; nada de (pedir a alguém para não fazer); é melhor não | talvez, usado em conjunto com a palavra ``是'' para indicar especulação.}
  \definition{pron.}{outro; algum outro}
  \definition{s.}{distinção; diferença | classificação}
  \definition{v.}{deixar; partir; separar | diferenciar; distinguir; encontrar aspectos diferentes | fixar objetos com pinos | girar; virar | aderir; colar; preder}
  \seeref{别}{bie4}
  \seealsoref{是}{shi4}
\end{entry}

\begin{entry}{别的}{bie2 de5}{7,8}[HSK 1][Radicais ⼑、⽩]
  \definition{pron.}{outro}
\end{entry}

\begin{entry}{别人}{bie2ren5}{7,2}[Radicais ⼑、⼈]
  \definition{pron.}{outra pessoa | outro povo | outros}
\end{entry}

\begin{entry}{别说}{bie2shuo1}{7,9}[Radicais ⼑、⾔]
  \definition{v.}{não falar de | não mencionar}
\end{entry}

\begin{entry}{别}{bie4}{7}[Radical ⼑]
  \definition{v.}{fazer com que alguém mude seus hábitos, opiniões, etc.}
  \seeref{别}{bie2}
\end{entry}

\begin{entry}{宾馆}{bin1guan3}{10,11}[Radicais ⼧、⾷]
  \definition[个,家]{s.}{casa de hóspedes | hotel}
\end{entry}

\begin{entry}{冰}{bing1}{6}[HSK 4][Radical ⼎]
  \definition{adj.}{frio (pessoa)| hostil}
  \definition[块]{s.}{gelo; água em estado sólido |  (gíria) metanfetamina}
  \definition{v.}{colocar gelo; colocar gelo ao redor; colocar no gelo; resfriar objetos com gelo ou água fria | sentir frio}
\end{entry}

\begin{entry}{冰糕}{bing1gao1}{6,16}[Radicais ⼎、⽶]
  \definition{s.}{sorvete | picolé}
\end{entry}

\begin{entry}{冰棍}{bing1gun4}{6,12}[Radicais ⼎、⽊]
  \definition[根]{s.}{picolé}
\end{entry}

\begin{entry}{冰激凌}{bing1ji1ling2}{6,16,10}[Radicais ⼎、⽔、⼎]
  \definition{s.}{sorvete}
\end{entry}

\begin{entry}{冰球}{bing1qiu2}{6,11}[Radicais ⼎、⽟]
  \definition{s.}{hóquei no gelo}
\end{entry}

\begin{entry}{冰天雪地}{bing1tian1-xue3di4}{6,4,11,6}[Radicais ⼎、⼤、⾬、⼟]
  \definition{expr.}{um mundo de gelo e neve}
\end{entry}

\begin{entry}{冰箱}{bing1xiang1}{6,15}[HSK 4][Radicais ⼎、⾋]
  \definition[台,个]{s.}{geladeira; freezer; refrigerador; aparelhos para congelar alimentos ou medicamentos com gelo para mantê-los frios}
\end{entry}

\begin{entry}{冰雪}{bing1 xue3}{6,11}[HSK 4][Radicais ⼎、⾬]
  \definition{adj.}{puro como gelo e neve; descreve uma pessoa como pura}
  \definition{s.}{gelo e neve}
\end{entry}

\begin{entry}{兵}{bing1}{7}[HSK 4][Radical ⼋]
  \definition[名]{s.}{armas; armamentos | soldado; pessoal militar | exército; tropas | soldado raso; membro mais jovem do exército | assuntos militares (estratégia) | peão, uma das peças do xadrez chinês}
\end{entry}

\begin{entry}{兵器}{bing1qi4}{7,16}[Radicais ⼋、⼝]
  \definition{s.}{armas | armamento}
\end{entry}

\begin{entry}{饼}{bing3}{9}[Radical ⾷]
  \definition[张]{s.}{panqueca | biscoito | torta}
\end{entry}

\begin{entry}{饼干}{bing3gan1}{9,3}[Radicais ⾷、⼲]
  \definition[片,块]{s.}{bolacha | biscoito}
\end{entry}

\begin{entry}{并}{bing4}{6}[HSK 3,4][Radical ⼲]
  \definition{adv.}{igualmente; simultaneamente; lado a lado; coisas diferentes existem ao mesmo tempo; coisas diferentes estão acontecendo ao mesmo tempo | em absoluto (usado antes de uma negativa para dar ênfase);  usado antes de uma palavra negativa para reforçar o tom e refutá-la ligeiramente}
  \definition{conj.}{além de; e}
  \definition{v.}{combinar; fundir; incorporar; anexar; juntar}
\end{entry}

\begin{entry}{并排}{bing4pai2}{6,11}[Radicais ⼲、⼿]
  \definition{adv.}{lado a lado}
\end{entry}

\begin{entry}{并且}{bing4qie3}{6,5}[HSK 3][Radicais ⼲、⼀]
  \definition{conj.}{além disso | o que é mais | e}
\end{entry}

\begin{entry}{幷}{bing4}{8}[Radical ⼲]
  \variantof{并}
\end{entry}

\begin{entry}{倂}{bing4}{10}[Radical ⼈]
  \variantof{并}
\end{entry}

\begin{entry}{病}{bing4}{10}[HSK 1][Radical ⽧]
  \definition[场]{s.}{doença}
  \definition{v.}{adoecer | estar doente}
\end{entry}

\begin{entry}{病人}{bing4 ren2}{10,2}[HSK 1][Radicais ⽧、⼈]
  \definition{s.}{doente | paciente}
\end{entry}

\begin{entry}{拨转}{bo1zhuan3}{8,8}[Radicais ⼿、⾞]
  \definition{v.}{transferir (fundos, etc.) | virar | dar a volta}
\end{entry}

\begin{entry}{波}{bo1}{8}[Radical ⽔]
  \definition*{s.}{Polônia, abreviação de 波兰}
  \definition{s.}{onda | ondulação | tempestade | surto}
  \seeref{波兰}{bo1lan2}
\end{entry}

\begin{entry}{波兰}{bo1lan2}{8,5}[Radicais ⽔、⼋]
  \definition*{s.}{Polônia}
\end{entry}

\begin{entry}{波音}{bo1yin1}{8,9}[Radicais ⽔、⾳]
  \definition*{s.}{Boeing (empresa aeroespacial)}
  \definition{s.}{mordente (música)}
\end{entry}

\begin{entry}{玻璃}{bo1li5}{9,14}[Radicais ⽟、⽟]
  \definition[张,塊]{s.}{vidro | (gíria) homossexual masculino}
\end{entry}

\begin{entry}{般}{bo1}{10}[Radical ⾈]
  \definition{s.}{utilizado em 般若 \dpy{bo1re3}}
  \seeref{般若}{bo1re3}
\end{entry}

\begin{entry}{般若}{bo1re3}{10,8}[Radicais ⾈、⾋]
  \definition*{s.}{Prajna (sânscrito), \emph{insight} sobre a verdadeira natureza da realidade | (Budismo) sabedoria}
\end{entry}

\begin{entry}{啵}{bo1}{11}[Radical ⼝]
  \definition{s.}{(onomatopéia) borbulhar}
  \seeref{啵}{bo5}
\end{entry}

\begin{entry}{菠菜}{bo1cai4}{11,11}[Radicais ⾋、⾋]
  \definition[棵]{s.}{espinafre}
\end{entry}

\begin{entry}{播出}{bo1 chu1}{15,5}[HSK 3][Radicais ⼿、⼐]
  \definition{v.}{transmitir | estar no ar}
\end{entry}

\begin{entry}{播放}{bo1fang4}{15,8}[HSK 3][Radicais ⼿、⽅]
  \definition{v.}{ir ao ar | transmitir por rádio | mostrar | transmitir (um programa de TV)}
\end{entry}

\begin{entry}{播音}{bo1yin1}{15,9}[Radicais ⼿、⾳]
  \definition{s.}{transmissão}
  \definition{v.+compl.}{transmitir}
\end{entry}

\begin{entry}{脖子}{bo2zi5}{11,3}[Radicais ⾁、⼦]
  \definition[个]{s.}{pescoço}
\end{entry}

\begin{entry}{博文}{bo2wen2}{12,4}[Radicais ⼗、⽂]
  \definition{s.}{artigo em um blog}
  \definition{v.}{escrever um artigo em um blog}
\end{entry}

\begin{entry}{博物馆}{bo2wu4guan3}{12,8,11}[Radicais ⼗、⽜、⾷]
  \definition{s.}{museu}
\end{entry}

\begin{entry}{博主}{bo2zhu3}{12,5}[Radicais ⼗、⼂]
  \definition{s.}{blogueiro}
\end{entry}

\begin{entry}{薄}{bo2}{16}[Radical ⾋]
  \definition{adj.}{ligeiro; escasso; pequeno | mesquinho; pouco generoso; cruel | frívolo; fútil; leviano}
  \seeref{薄}{bao2}
\end{entry}

\begin{entry}{啵}{bo5}{11}[Radical ⼝]
  \definition{part.}{partícula gramaticalmente equivalente a 吧}
  \seeref{啵}{bo1}
  \seealsoref{吧}{ba5}
\end{entry}

\begin{entry}{不}{bu2}[(antes de quarto tom)]{4}[HSK 1][Radical ⼀]
  \definition{adv.}{não}
  \definition{pref.}{prefixo negativo}
  \seeref{不}{bu4}
  \seeref{不}{bu5}
\end{entry}

\begin{entry}{不必}{bu2 bi4}{4,5}[HSK 3][Radicais ⼀、⼼]
  \definition{adv.}{não precisa | não tem que}
\end{entry}

\begin{entry}{不错}{bu2 cuo4}{4,13}[HSK 2][Radicais ⼀、⾦]
  \definition{adj.}{correto | não (é) mau | bastante bom | certo}
\end{entry}

\begin{entry}{不大}{bu2 da4}{4,3}[HSK 1][Radicais ⼀、⼤]
  \definition{adv.}{não muito | não frequentemente | raramente |dificilmente | escassamente}
\end{entry}

\begin{entry}{不大离}{bu2da4li2}{4,3,10}[Radicais ⼀、⼤、⼇]
  \definition{adj.}{bem perto | quase certo | nada mal}
\end{entry}

\begin{entry}{不但}{bu2 dan4}{4,7}[HSK 2][Radicais ⼀、⼈]
  \definition{conj.}{não somente}
\end{entry}

\begin{entry}{不但……而且……}{bu2 dan4 er2qie3}{4,7,6,5}[HSK 2][Radicais ⼀、⼈、⽽、⼀]
  \definition{conj.}{não só\dots mas também\dots}
\end{entry}

\begin{entry}{不到}{bu2dao4}{4,8}[Radicais ⼀、⼑]
  \definition{adj.}{insuficiente}
  \definition{adv.}{menos que}
  \definition{v.}{não chegar}
\end{entry}

\begin{entry}{不断}{bu2duan4}{4,11}[HSK 3][Radicais ⼀、⽄]
  \definition{adv.}{continuamente | sem fim}
\end{entry}

\begin{entry}{不对}{bu2 dui4}{4,5}[HSK 1][Radicais ⼀、⼨]
  \definition{adj.}{incorreto | errado | anormal | estranho | estar em desacordo com | ser difícil de conviver}
\end{entry}

\begin{entry}{不够}{bu2 gou4}{4,11}[HSK 2][Radicais ⼀、⼣]
  \definition{adv.}{insuficiente}
  \definition{v.}{não ser suficiente}
\end{entry}

\begin{entry}{不过}{bu2guo4}{4,6}[HSK 2][Radicais ⼀、⾡]
  \definition{conj.}{mas | contudo | no entanto}
\end{entry}

\begin{entry}{不客气}{bu2 ke4 qi5}{4,9,4}[HSK 1][Radicais ⼀、⼧、⽓]
  \definition{adj.}{indelicado | rude | brusco}
  \definition{expr.}{de nada | não há de que | não mencione isso}
\end{entry}

\begin{entry}{不论}{bu2 lun4}{4,6}[HSK 3][Radicais ⼀、⾔]
  \definition{conj.}{não importa (o que, quem, como, etc.) | se \dots ou \dots}
\end{entry}

\begin{entry}{不论……都……}{bu2lun4 dou1}{4,6,10}[Radicais ⼀、⾔、⾢]
  \definition{conj.}{não apenas\dots, (o que, quem, como, etc.), \dots}
\end{entry}

\begin{entry}{不论……也……}{bu2lun4 ye3}{4,6,3}[Radicais ⼀、⾔、⼄]
  \definition{conj.}{não apenas\dots, (o que, quem, como, etc.), \dots}
\end{entry}

\begin{entry}{不日}{bu2ri4}{4,4}[Radicais ⼀、⽇]
  \definition{adv.}{em alguns dias}
\end{entry}

\begin{entry}{不是话}{bu2shi4hua4}{4,9,8}[Radicais ⼀、⽇、⾔]
  \definition{expr.}{sem razão | demasiado irracionável}
  \seeref{不像话}{bu2xiang4hua4}
  \seeref{不成话}{bu4cheng2hua4}
\end{entry}

\begin{entry}{不太}{bu2 tai4}{4,4}[HSK 2][Radicais ⼀、⼤]
  \definition{adv.}{não bastante | não muito}
\end{entry}

\begin{entry}{不像话}{bu2xiang4hua4}{4,13,8}[Radicais ⼀、⼈、⾔]
  \definition{expr.}{sem razão | demasiado irracionável}
  \seeref{不是话}{bu2shi4hua4}
  \seeref{不成话}{bu4cheng2hua4}
\end{entry}

\begin{entry}{不要}{bu2 yao4}{4,9}[HSK 2][Radicais ⼀、⾑]
  \definition{adv.}{nada de (pedir a alguém para não fazer) | não}
\end{entry}

\begin{entry}{不要紧}{bu2yao4jin3}{4,9,10}[HSK 4][Radicais ⼀、⾑、⽷]
  \definition{adj.}{sem importância; sem seriedade; não problemático | não importa; não é um obstáculo | parece estar tudo bem, mas | à primeira vista, isso não parece atrapalhar}
\end{entry}

\begin{entry}{不用}{bu2 yong4}{4,5}[HSK 1][Radicais ⼀、⽤]
  \definition{v.}{não precisar}
  \seeref{甭}{beng2}
\end{entry}

\begin{entry}{不在乎}{bu2 zai4 hu1}{4,6,5}[HSK 4][Radicais ⼀、⼟、⼃]
  \definition{v.}{não se importar; não dar a mínima; não dar atenção}
\end{entry}

\begin{entry}{不注意}{bu2zhu4yi4}{4,8,13}[Radicais ⼀、⽔、⼼]
  \definition{adj.}{impensado | distraído}
  \definition{s.}{descuido | distração}
\end{entry}

\begin{entry}{补}{bu3}{7}[HSK 3][Radical ⾐]
  \definition*{s.}{sobrenome Bu}
  \definition{s.}{benefício | ajuda | uso}
  \definition{v.}{consertar | remendar | preencher | adicionar suplemento | suprir | compensar |nutrir}
\end{entry}

\begin{entry}{补充}{bu3chong1}{7,6}[HSK 3][Radicais ⾐、⼉]
  \definition{adj.}{adicional | suplementar}
  \definition[个]{s.}{aditivo | suplemento}
  \definition{v.}{reabastecer | suplementar | complementar}
\end{entry}

\begin{entry}{不}{bu4}{4}[HSK 1][Radical ⼀]
  \definition{adv.}{não}
  \definition{pref.}{prefixo negativo}
  \seeref{不}{bu2}
  \seeref{不}{bu5}
\end{entry}

\begin{entry}{不安}{bu4'an1}{4,6}[HSK 3][Radicais ⼀、⼧]
  \definition{adj.}{inquieto | instável | intranquilo | pesaroso}
\end{entry}

\begin{entry}{不成话}{bu4cheng2hua4}{4,6,8}[Radicais ⼀、⼽、⾔]
  \definition{expr.}{sem razão | demasiado irracionável}
  \seeref{不是话}{bu2shi4hua4}
  \seeref{不像话}{bu2xiang4hua4}
\end{entry}

\begin{entry}{不得不}{bu4de2bu4}{4,11,4}[HSK 3][Radicais ⼀、⼻、⼀]
  \definition{adv.}{tem que | não tem escolha a não ser}
\end{entry}

\begin{entry}{不公}{bu4gong1}{4,4}[Radicais ⼀、⼋]
  \definition{adj.}{injusto}
\end{entry}

\begin{entry}{不管}{bu4guan3}{4,14}[HSK 4][Radicais ⼀、⽵]
  \definition{conj.}{não importa (o que, como, etc.); independentemente de; indica que, embora as condições ou circunstâncias tenham mudado, o resultado permanece o mesmo}
  \seeref{不管……都……}{bu4guan3 dou1}
  \seeref{不管……也……}{bu4guan3 ye3}
\end{entry}

\begin{entry}{不管……都……}{bu4guan3 dou1}{4,14,10}[Radicais ⼀、⽵、⾢]
  \definition{conj.}{não apenas\dots, (o que, quem, como, etc.), \dots}
\end{entry}

\begin{entry}{不管……也……}{bu4guan3 ye3}{4,14,3}[Radicais ⼀、⽵、⼄]
  \definition{conj.}{não apenas\dots, (o que, quem, como, etc.), \dots}
\end{entry}

\begin{entry}{不光}{bu4 guang1}{4,6}[HSK 3][Radicais ⼀、⼉]
  \definition{adv.}{não é o único}
  \definition{conj.}{não somente}
\end{entry}

\begin{entry}{不好意思}{bu4 hao3 yi4 si5}{4,6,13,9}[HSK 2][Radicais ⼀、⼥、⼼、⼼]
  \definition{adj.}{pedir desculpas (por incomodar alguém) | sentir-se envergonhado | achar isso embaraçoso}
\end{entry}

\begin{entry}{不仅}{bu4jin3}{4,4}[HSK 3][Radicais ⼀、⼈]
  \definition{adv.}{não apenas (em número, quantidade ou extensão)}
  \definition{conj.}{não somente}
\end{entry}

\begin{entry}{不久}{bu4 jiu3}{4,3}[HSK 2][Radicais ⼀、⼃]
  \definition{adj.}{em breve | futuro próximo | logo depois | não muito depois | não muito tempo (antes ou depois de algo)}
\end{entry}

\begin{entry}{不可避免}{bu4ke3bi4mian3}{4,5,16,7}[Radicais ⼀、⼝、⾌、⼉]
  \definition{adj./adv.}{inevitável}
\end{entry}

\begin{entry}{不满}{bu4 man3}{4,13}[HSK 2][Radicais ⼀、⽔]
  \definition{adj.}{ressentido | insatisfeito | descontente}
  \definition{v.}{estar descontente com |ser menor que}
\end{entry}

\begin{entry}{不然}{bu4ran2}{4,12}[HSK 4][Radicais ⼀、⽕]
  \definition{adj.}{não é assim; não é o caso}
  \definition{conj.}{se não; caso contrário; indica outra consequência ou circunstância que teria ocorrido se não fosse}
\end{entry}

\begin{entry}{不如}{bu4ru2}{4,6}[HSK 2][Radicais ⼀、⼥]
  \definition{conj.}{em vez de | melhor que | seria melhor}
  \definition{v.}{ser inferior a | não ser igual a | não ser tão bom quanto | não poder fazer melhor que}
\end{entry}

\begin{entry}{不少}{bu4 shao3}{4,4}[HSK 2][Radicais ⼀、⼩]
  \definition{adj.}{muitos | bastante | não poucos}
\end{entry}

\begin{entry}{不是……而是}{bu4shi4 er2 shi4}{4,9,6,9}[Radicais ⼀、⽇、⽽、⽇]
  \definition{conj.}{não somente\dots mas também\dots, expressam um relacionamento mais profundo e avançado em significado, mas as orações antes e depois são consistentes em expressar significados negativos e afirmativos, entretanto, a primeira metade da frase expressa negação, e a segunda metade expressa afirmação, e o significado das orações anteriores e seguintes não pode ser de um nível mais alto}
\end{entry}

\begin{entry}{不同}{bu4 tong2}{4,6}[HSK 2][Radicais ⼀、⼝]
  \definition{adj.}{diferente | distinto}
\end{entry}

\begin{entry}{不行}{bu4 xing2}{4,6}[HSK 2][Radicais ⼀、⾏]
  \definition{adj.}{não funciona | não é bom}
  \definition{adv.}{profundamente | terrivelmente | extremamente}
  \definition{v.}{não fazer | não ser permitido | estar fora de questão | estar à beira da morte}
\end{entry}

\begin{entry}{不一定}{bu4 yi2 ding4}{4,1,8}[HSK 2][Radicais ⼀、⼀、⼧]
  \definition{adv.}{talvez | incerto | não tenho certeza | não necessariamente}
\end{entry}

\begin{entry}{不一会儿}{bu4 yi2 hui4r5}{4,1,6,2}[HSK 2][Radicais ⼀、⼀、⼈、⼉]
  \definition{expr.}{em um momento | em pouco tempo |em breve}
\end{entry}

\begin{entry}{不止}{bu4zhi3}{4,4}[Radicais ⼀、⽌]
  \definition{adv.}{incessantemente | sem fim | mais que | não limitado a}
\end{entry}

\begin{entry}{布}{bu4}{5}[HSK 3][Radical ⼱]
  \definition*{s.}{sobrenome Bu}
  \definition[块,幅,匹]{s.}{pano | tecido | uma moeda de cobre nos tempos antigos}
  \definition{v.}{anunciar | declarar | tornar conhecido | proclamar | publicar | espalhar | disseminar |organizar | implantar | dispor}
\end{entry}

\begin{entry}{布谷鸟}{bu4gu3niao3}{5,7,5}[Radicais ⼱、⾕、⿃]
  \definition{s.}{cuco (pássaro)}
  \seealsoref{杜鹃}{du4juan1}
  \seealsoref{杜鹃鸟}{du4juan1niao3}
  \seealsoref{杜宇}{du4yu3}
\end{entry}

\begin{entry}{布署}{bu4shu3}{5,13}[Radicais ⼱、⽹]
  \variantof{部署}
\end{entry}

\begin{entry}{布置}{bu4zhi4}{5,13}[HSK 4][Radicais ⼱、⽹]
  \definition{v.}{arrumar; organizar; decorar; colocar adequadamente objetos ou paisagismo, conforme necessário | designar; tomar providências para; dar instruções sobre; organizar trabalho, atividades, etc.}
\end{entry}

\begin{entry}{步}{bu4}{7}[HSK 3][Radical ⽌]
  \definition*{s.}{sobrenome Bu}
  \definition{clas.}{uma unidade antiga para medida de comprimento, equivalente a cinco chi}
  \definition{s.}{ritmo | passo | estágio | passo | condição | situação | estado}
  \definition{v.}{ir a pé | andar | pisar | contar passos}
\end{entry}

\begin{entry}{步行}{bu4 xing2}{7,6}[HSK 4][Radicais ⽌、⾏]
  \definition{v.}{caminhar; ir a pé; andar a pé (diferente de andar de carro, a cavalo, etc.)}
\end{entry}

\begin{entry}{部}{bu4}{10}[HSK 3][Radical ⾢]
  \definition{clas.}{para obras de literatura, filmes, máquinas etc.}
  \definition[根]{s.}{departamento | divisão | ministério | seção | parte | tropas}
\end{entry}

\begin{entry}{部队}{bu4dui4}{10,4}[Radicais ⾢、⾩]
  \definition[个]{s.}{exército | forças armadas | tropas | unidades}
\end{entry}

\begin{entry}{部分}{bu4fen5}{10,4}[HSK 2][Radicais ⾢、⼑]
  \definition[个]{s.}{parte | parte de | uma parte de | pedaço | secção}
\end{entry}

\begin{entry}{部门}{bu4men2}{10,3}[HSK 3][Radicais ⾢、⾨]
  \definition[个]{s.}{filial | departamento | divisão | seção}
\end{entry}

\begin{entry}{部属}{bu4shu3}{10,12}[Radicais ⾢、⼫]
  \definition{s.}{afiliado a um ministério | subordinado | tropas sob comando de alguém}
\end{entry}

\begin{entry}{部署}{bu4shu3}{10,13}[Radicais ⾢、⽹]
  \definition{s.}{implantação}
  \definition{v.}{implantar}
\end{entry}

\begin{entry}{部下}{bu4xia4}{10,3}[Radicais ⾢、⼀]
  \definition{s.}{subordinado | tropas sob comando de alguém}
\end{entry}

\begin{entry}{部长}{bu4 zhang3}{10,4}[HSK 3][Radicais ⾢、⾧]
  \definition[个,位,名]{s.}{ministro | chefe de departamento | chefe de seção}
\end{entry}

\begin{entry}{部族}{bu4zu2}{10,11}[Radicais ⾢、⽅]
  \definition{adj.}{tribal}
  \definition{s.}{tribo}
\end{entry}

\begin{entry}{不}{bu5}{4}[HSK 1][Radical ⼀]
  \definition{adv.}{não (em expressões ``v.+不+v.'')}
  \seeref{不}{bu2}
  \seeref{不}{bu4}
\end{entry}

%%%%% EOF %%%%%


 %%%
%%% C
%%%
\section*{C}\addcontentsline{toc}{section}{C}\addcontentsline{loh}{figure}{\#\#\#\#\#\#\#\# C}

%%%%%%%%%% 擦 %%%%%%%%%%
\subsection*{擦}\addcontentsline{loh}{figure}{擦 \dpy{ca1}}

\begin{EntryWithPhonetic}{擦}{ca1}{17}{⼿}[HSK 4]
  \definition{v.}{enxugar; esfregar; apagar; limpar; limpar esfregando com um pano, toalha de mão, etc. | espalhar sobre; colocar sobre | passar raspando | ralar (em pedaços); ralar frutas em um ralador para fazer fios finos}
\end{EntryWithPhonetic}

\begin{EntryWithPhonetic}{擦拭}{ca1shi4}{17,9}{⼿,⼿}
  \definition{v.}{limpar com um pano}
\end{EntryWithPhonetic}

%%%%%%%%%% 猜 %%%%%%%%%%
\subsection*{猜}\addcontentsline{loh}{figure}{猜 \dpy{cai1}}

\begin{EntryWithPhonetic}{猜}{cai1}{11}{⽝}[HSK 5]
  \definition{v.}{adivinhar; conjecturar; especular | suspeitar; ser cauteloso com os outros; desconfiar dos outros}
\end{EntryWithPhonetic}

\begin{EntryWithPhonetic}{猜测}{cai1 ce4}{11,9}{⽝,⽔}[HSK 5]
  \definition[个,种]{s.}{advinhação; conjectura; suposição; especulação}
  \definition{v.}{adivinhar; conjecturar; especular; estimar a partir da imaginação}
\end{EntryWithPhonetic}

\begin{EntryWithPhonetic}{猜忌}{cai1 ji4}{11,7}{⽝,⼼}
  \definition{v.}{ser desconfiado e invejoso | ser desconfiado e ciumento de}
\end{EntryWithPhonetic}

\begin{EntryWithPhonetic}{猜谜}{cai1 mi2}{11,11}{⽝,⾔}[HSK 7-9]
  \definition{v.}{adivinhar um enigma}
\end{EntryWithPhonetic}

\begin{EntryWithPhonetic}{猜想}{cai1xiang3}{11,13}{⽝,⼼}[HSK 7-9]
  \definition{s.}{suposição; conjectura; palpite; especulação}
  \definition{v.}{supor; adivinhar; suspeitar}
\end{EntryWithPhonetic}

\begin{EntryWithPhonetic}{猜疑}{cai1 yi2}{11,14}{⽝,⽦}
  \definition{v.}{abrigar suspeitas; ser desconfiado; ter receios; levantar suspeitas do nada}
\end{EntryWithPhonetic}

%%%%%%%%%% 才 %%%%%%%%%%
\subsection*{才}\addcontentsline{loh}{figure}{才 \dpy{cai2}}

\begin{EntryWithPhonetic}{才}{cai2}{3}{⼿}[HSK 2,4]
  \definition*{s.}{Sobrenome: Cai}
  \definition{adv.}{há pouco; agora mesmo | (precedido por uma expressão de tempo) não até | (precedido por uma expressão de razão ou condição) não a menos que; não até que; então e somente então; por nenhuma outra razão | (seguido por uma expressão numérica) apenas; indica um intervalo pequeno ou uma quantidade reduzida, equivalente a 仅仅 ou 只 | (em uma afirmação ou negação, enfatizando o que vem antes de 才, geralmente com 呢 no final da frase) na verdade; realmente | dica que algo acontece tarde ou termina tarde | (precedido por uma expressão de tempo) não até; indicando que não era assim, mas agora surgiu uma nova situação | (precedido por uma expressão de razão ou condição) a menos que; indica que só em determinadas condições e, em seguida, como | (expressa ênfase )}
  \definition{s.}{habilidade; talento; dom | pessoa competente | pessoas de um determinado tipo (frequentemente usado como sufixo) | dotação; talento; habilidade}
  \seealsoref{呢}{ne5}
\end{EntryWithPhonetic}

\begin{EntryWithPhonetic}{才华}{cai2hua2}{3,6}{⼿,⼗}[HSK 7-9]
  \definition[份]{s.}{talento literário; talento artístico; talentos que são exibidos externamente (principalmente nas artes)}
\end{EntryWithPhonetic}

\begin{EntryWithPhonetic}{才略}{cai2lve4}{3,11}{⼿,⽥}
  \definition{s.}{habilidade e sagacidade}
\end{EntryWithPhonetic}

\begin{EntryWithPhonetic}{才能}{cai2 neng2}{3,10}{⼿,⾁}[HSK 3]
  \definition[间]{s.}{talento; habilidade; dom; capacidade; inteligência e habilidade}
\end{EntryWithPhonetic}

%%%%%%%%%% 材 %%%%%%%%%%
\subsection*{材}\addcontentsline{loh}{figure}{材 \dpy{cai2}}

\begin{EntryWithPhonetic}{材}{cai2}{7}{⽊}
  \definition[份]{s.}{madeira | material; geralmente se refere a coisas que podem ser transformadas diretamente em produtos acabados | material; materiais para escrita ou referência | pessoa capaz; pessoas talentosas | habilidade; talento; aptidão | caixão}
\end{EntryWithPhonetic}

\begin{EntryWithPhonetic}{材料}{cai2liao4}{7,10}{⽊,⽃}[HSK 4]
  \definition[份,个,种]{s.}{material; algo para fazer um produto acabado | material (figura de linguagem) | dados; material para estudo, pesquisa, etc.; conteúdo de uma obra}
\end{EntryWithPhonetic}

%%%%%%%%%% 财 %%%%%%%%%%
\subsection*{财}\addcontentsline{loh}{figure}{财 \dpy{cai2}}

\begin{EntryWithPhonetic}{财}{cai2}{7}{⾙}
  \definition[笔]{s.}{riqueza; dinheiro; fortuna | propriedade; objetos de valor; um termo geral para dinheiro e materiais}
\end{EntryWithPhonetic}

\begin{EntryWithPhonetic}{财产}{cai2chan3}{7,6}{⾙,⼇}[HSK 4]
  \definition[笔]{s.}{ativos; propriedade; pertences; refere"-se à posse de riqueza material, como dinheiro, bens, casas, terras, etc.}
\end{EntryWithPhonetic}

\begin{EntryWithPhonetic}{财富}{cai2fu4}{7,12}{⾙,⼧}[HSK 4]
  \definition{s.}{riqueza; fortuna; algo de valor}
\end{EntryWithPhonetic}

\begin{EntryWithPhonetic}{财经}{cai2jing1}{7,8}{⾙,⽷}[HSK 7-9]
  \definition{s.}{finanças e economia}
\end{EntryWithPhonetic}

\begin{EntryWithPhonetic}{财力}{cai2li4}{7,2}{⾙,⼒}[HSK 7-9]
  \definition{s.}{capacidade financeira; recursos financeiros; poder econômico (refere"-se principalmente ao capital)}
\end{EntryWithPhonetic}

\begin{EntryWithPhonetic}{财务}{cai2wu4}{7,5}{⾙,⼒}[HSK 7-9]
  \definition{s.}{finanças; assuntos financeiros; assuntos relacionados à administração de bens ou retirada de dinheiro, guarda e cálculo}
\end{EntryWithPhonetic}

\begin{EntryWithPhonetic}{财物}{cai2wu4}{7,8}{⾙,⽜}[HSK 7-9]
  \definition{s.}{dinheiro e bens; propriedade (espólio não incluído) | propriedade; pertences; dinheiro e suprimentos}
\end{EntryWithPhonetic}

\begin{EntryWithPhonetic}{财政}{cai2zheng4}{7,9}{⾙,⽁}[HSK 7-9]
  \definition{s.}{finanças; a gestão estatal das receitas e despesas financeiras}
\end{EntryWithPhonetic}

%%%%%%%%%% 裁 %%%%%%%%%%
\subsection*{裁}\addcontentsline{loh}{figure}{裁 \dpy{cai2}}

\begin{EntryWithPhonetic}{裁}{cai2}{12}{⾐}[HSK 7-9]
  \definition{clas.}{divisão de papel de impressão de tamanho padrão}
  \definition{s.}{planejamento | tipo de escrita | planejamento mental; arranjo e seleção, usados principalmente na literatura e na arte | sanção; restrição | estilo; forma | (impressão) tamanho do corte de papel}
  \definition{v.}{cortar (papel, tecido, etc.) em partes | reduzir; cortar; dispensar | julgar; decidir | verificar; sancionar | cortar; eliminar; remover coisas desnecessárias ou redundantes | discernir; medir; julgar}
\end{EntryWithPhonetic}

\begin{EntryWithPhonetic}{裁定}{cai2ding4}{12,8}{⾐,⼧}[HSK 7-9]
  \definition{v.}{decidir (ou declarar) judicialmente; governar}
\end{EntryWithPhonetic}

\begin{EntryWithPhonetic}{裁决}{cai2jue2}{12,6}{⾐,⼎}[HSK 7-9]
  \definition[项]{s.}{sentença arbitral; decisão; adjudicação}
  \definition{v.}{fazer uma decisão; julgar; decidir; adjudicar veredicto}
\end{EntryWithPhonetic}

\begin{EntryWithPhonetic}{裁判}{cai2pan4}{12,7}{⾐,⼑}[HSK 5]
  \definition[个,位,名]{s.}{árbitro; juiz; pessoa que desempenha funções de arbitragem em esportes e outras competições}
  \definition{v.}{arbitrar; atuar como árbitro; em esportes e outras atividades competitivas, julgar o desempenho dos atletas, vitórias e derrotas, classificações e problemas que ocorrem durante a competição de acordo com as regras da competição | julgar; refere"-se a um terceiro que faz um julgamento quando surge uma disputa entre duas partes}
\end{EntryWithPhonetic}

%%%%%%%%%% 采 %%%%%%%%%%
\subsection*{采}\addcontentsline{loh}{figure}{采 \dpy{cai3}}

\begin{EntryWithPhonetic}{采}{cai3}{8}{⾤}[HSK 7-9]
  \definition*{s.}{Sobrenome: Cai}
  \definition{s.}{espírito; tez; cor e expressão facial | cores}
  \definition{v.}{escolher; arrancar; reunir; colher (flores, folhas, frutas) | minerar; extrair | reunir; coletar | adotar; pegar; selecionar}
  \seeref{cai4}
\end{EntryWithPhonetic}

\begin{EntryWithPhonetic}{采访}{cai3fang3}{8,6}{⾤,⾔}[HSK 4]
  \definition{s.}{cobertura; entrevista; coleta de notícias; entrevistas, pesquisas, gravações de áudio e vídeo, etc., com o objetivo de coletar os materiais necessários}
  \definition{v.}{cobrir; entrevistar; reunir novas informações}
\end{EntryWithPhonetic}

\begin{EntryWithPhonetic}{采购}{cai3gou4}{8,8}{⾤,⾙}[HSK 5]
  \definition[名]{s.}{comprador; responsável pelas compras}
  \definition{v.}{adquirir; comprar; fazer compras para uma organização; fazer compras para uma empresa}
\end{EntryWithPhonetic}

\begin{EntryWithPhonetic}{采集}{cai3ji2}{8,12}{⾤,⾫}[HSK 7-9]
  \definition{v.}{reunir; coletar}
\end{EntryWithPhonetic}

\begin{EntryWithPhonetic}{采矿}{cai3/kuang4}{8,8}{⾤,⽯}[HSK 7-9]
  \definition{s.}{mina}
  \definition{v.+compl.}{minerar; extrair minerais}
\end{EntryWithPhonetic}

\begin{EntryWithPhonetic}{采纳}{cai3na4}{8,7}{⾤,⽷}[HSK 6]
  \definition{v.}{aceitar; adotar; tomar (opiniões, sugestões, solicitações, etc.)}
\end{EntryWithPhonetic}

\begin{EntryWithPhonetic}{采取}{cai3qu3}{8,8}{⾤,⼜}[HSK 3]
  \definition{v.}{adotar; escolha da implementação (diretrizes, políticas, métodos, ações, etc.) | reunir; coletar; tomar; assumir}
\end{EntryWithPhonetic}

\begin{EntryWithPhonetic}{采用}{cai3 yong4}{8,5}{⾤,⽤}[HSK 3]
  \definition{v.}{selecionar e usar; adotar; considerar adequado e utilizar}
\end{EntryWithPhonetic}

%%%%%%%%%% 彩 %%%%%%%%%%
\subsection*{彩}\addcontentsline{loh}{figure}{彩 \dpy{cai3}}

\begin{EntryWithPhonetic}{彩}{cai3}{11}{⼺}
  \definition{s.}{cor | aplausos; vivas | variedade; brilho; esplendor | prêmio; loteria | sangue de uma ferida | habilidades especiais empregadas em mágica ou ópera para alcançar um efeito desejado | seda colorida | cores variadas | graça na arte; graciosidade | prêmio de loteria; ganhos | efeitos especiais no teatro chinês (simbolizando sangue, fogo, etc.)}
\end{EntryWithPhonetic}

\begin{EntryWithPhonetic}{彩电}{cai3dian4}{11,5}{⼺,⽥}[HSK 7-9]
  \definition[台,个]{s.}{TV à cores}
\end{EntryWithPhonetic}

\begin{EntryWithPhonetic}{彩虹}{cai3hong2}{11,9}{⼺,⾍}[HSK 7-9]
  \definition[道,条]{s.}{arco-íris}
\end{EntryWithPhonetic}

\begin{EntryWithPhonetic}{彩票}{cai3piao4}{11,11}{⼺,⽰}[HSK 5]
  \definition[张,注]{s.}{bilhete de loteria; um título com números, vendido pelo valor de face; após o sorteio, o portador do bilhete premiado pode reivindicar o prêmio de acordo com o regulamento}
\end{EntryWithPhonetic}

\begin{EntryWithPhonetic}{彩色}{cai3 se4}{11,6}{⼺,⾊}[HSK 3]
  \definition[个,种]{s.}{multicolorido; cor; várias cores}
\end{EntryWithPhonetic}

\begin{EntryWithPhonetic}{彩霞}{cai3xia2}{11,17}{⼺,⾬}[HSK 7-9]
  \definition[片]{s.}{nuvens rosadas (ou cor-de-rosa) | nuvens tingidas com tons de pôr do sol; nuvens coloridas}
\end{EntryWithPhonetic}

%%%%%%%%%% 踩 %%%%%%%%%%
\subsection*{踩}\addcontentsline{loh}{figure}{踩 \dpy{cai3}}

\begin{EntryWithPhonetic}{踩}{cai3}{15}{⾜}[HSK 6]
  \definition{v.}{pisar; pisotear | pisar; metáfora: depreciar ou estragar | rastrear; antigamente significava rastrear (bandidos) ou investigar (casos)}
\end{EntryWithPhonetic}

%%%%%%%%%% 采 %%%%%%%%%%
\subsection*{采}\addcontentsline{loh}{figure}{采 \dpy{cai4}}

\begin{EntryWithPhonetic}{采}{cai4}{8}{⾤}
  \definition{s.}{atribuição a um nobre feudal; a terra (incluindo os escravos que cultivavam a terra) concedida pelos antigos príncipes aos nobres; também chamada de feudo}
  \seeref{cai3}
\end{EntryWithPhonetic}

%%%%%%%%%% 菜 %%%%%%%%%%
\subsection*{菜}\addcontentsline{loh}{figure}{菜 \dpy{cai4}}

\begin{EntryWithPhonetic}{菜}{cai4}{11}{⾋}[HSK 1]
  \definition*{s.}{Sobrenome: Cai}
  \definition{adj.}{pouca habilidade; baixo nível; baixa capacidade}
  \definition[棵,个,道]{s.}{legumes; verduras; plantas que podem ser usadas como alimentos complementares | óleo de canola | prato; item ou prato do cardápio (seja de carne ou de vegetais)}
\end{EntryWithPhonetic}

\begin{EntryWithPhonetic}{菜单}{cai4dan1}{11,8}{⾋,⼗}[HSK 2]
  \definition[个,分,张]{s.}{menu; lista de pratos | menu (para computadores); lista utilizada para selecionar várias operações diferentes}
\end{EntryWithPhonetic}

\begin{EntryWithPhonetic}{菜刀}{cai4dao1}{11,2}{⾋,⼑}
  \definition[把]{s.}{faca de vegetais | faca de cozinha | cutelo}
\end{EntryWithPhonetic}

\begin{EntryWithPhonetic}{菜市场}{cai4shi4chang3}{11,5,6}{⾋,⼱,⼟}[HSK 7-9]
  \definition[个,家]{s.}{mercado de alimentos; mercearia verde; mercado de produtos agrícolas; mercado de vegetais; um mercado em uma cidade ou município que vende vegetais, carne, ovos e outros alimentos não básicos}
\end{EntryWithPhonetic}

%%%%%%%%%% 参 %%%%%%%%%%
\subsection*{参}\addcontentsline{loh}{figure}{参 \dpy{can1}}

\begin{EntryWithPhonetic}{参}{can1}{8}{⼛}
  \definition{v.}{juntar"-se; entrar; tomar parte em; participar | referir; consultar; comparar com outros materiais | ligar para prestar homenagem a; fazer uma visita |  (significado antigo) acusar um funcionário perante o imperador; relatar ou expor ao imperador | explorar e compreender (verdade, significado, etc.)}
\end{EntryWithPhonetic}

\begin{EntryWithPhonetic}{参观}{can1guan1}{8,6}{⼛,⾒}[HSK 2]
  \definition{v.}{visitar; dar uma olhada; observação no local (resultados do trabalho, carreira, instalações, locais históricos e pontos turísticos, etc.)}
\end{EntryWithPhonetic}

\begin{EntryWithPhonetic}{参加}{can1jia1}{8,5}{⼛,⼒}[HSK 2]
  \definition{v.}{aderir (a organizações); participar; participar (de atividades); participar de alguma organização ou atividade | dar (conselho, sugestão, etc.)}
\end{EntryWithPhonetic}

\begin{EntryWithPhonetic}{参见}{can1jian4}{8,4}{⼛,⾒}[HSK 7-9]
  \definition{v.}{(em referências) ver; ver também | consultar; visualizar | ler algo para referência | prestar homenagem a (um superior, etc.)}
\end{EntryWithPhonetic}

\begin{EntryWithPhonetic}{参军}{can1/jun1}{8,6}{⼛,⼍}[HSK 7-9]
  \definition{s.}{oficial do estado"-maior militar; título oficial antigo}
  \definition{v.+compl.}{juntar"-se ao exército; alistar"-se}
\end{EntryWithPhonetic}

\begin{EntryWithPhonetic}{参考}{can1kao3}{8,6}{⼛,⽼}[HSK 4]
  \definition{v.}{consultar; referir"-se a; acessar informações relevantes para estudo ou pesquisa | consultar; referir"-se a; lidar com coisas, observar, ler, aprender e usar materiais relevantes}
\end{EntryWithPhonetic}

\begin{EntryWithPhonetic}{参谋}{can1mou2}{8,11}{⼛,⾔}[HSK 7-9]
  \definition{s.}{oficial de estado-maior; pessoal militar envolvido em planejamento militar e outros assuntos | conselheiro; consultor}
  \definition{v.}{aconselhar; dar conselhos}
\end{EntryWithPhonetic}

\begin{EntryWithPhonetic}{参赛}{can1 sai4}{8,14}{⼛,⾙}[HSK 6]
  \definition{v.}{participar de uma partida (ou competição); competir}
\end{EntryWithPhonetic}

\begin{EntryWithPhonetic}{参与}{can1yu4}{8,3}{⼛,⼀}[HSK 4]
  \definition{v.}{participar de; tomar parte em; ter uma mão em; envolver"-se em; participar (no planejamento, discussão e condução dos assuntos)}
\end{EntryWithPhonetic}

\begin{EntryWithPhonetic}{参展}{can1 zhan3}{8,10}{⼛,⼫}[HSK 6]
  \definition{v.}{expor ou participar de uma feira comercial, etc.}
\end{EntryWithPhonetic}

\begin{EntryWithPhonetic}{参照}{can1zhao4}{8,13}{⼛,⽕}[HSK 7-9]
  \definition{v.}{consultar; referir"-se a; referir-se e imitar (métodos, experiências, etc.)}
\end{EntryWithPhonetic}

%%%%%%%%%% 餐 %%%%%%%%%%
\subsection*{餐}\addcontentsline{loh}{figure}{餐 \dpy{can1}}

\begin{EntryWithPhonetic}{餐}{can1}{16}{⾷}[HSK 6]
  \definition{clas.}{comer; fazer uma refeição}
  \definition{clas.}{usado para refeições}
  \definition{s.}{comida; refeição}
\end{EntryWithPhonetic}

\begin{EntryWithPhonetic}{餐馆}{can1 guan3}{16,11}{⾷,⾷}[HSK 5]
  \definition[家,个]{s.}{restaurante}
\end{EntryWithPhonetic}

\begin{EntryWithPhonetic}{餐厅}{can1ting1}{16,4}{⾷,⼚}[HSK 5]
  \definition[间]{s.}{sala de jantar}
  \definition[间,家,个]{s.}{restaurante; refeitório em um hotel | cantina, refeitório; também é chamado de 食堂}
  \seealsoref{食堂}{shi2 tang2}
\end{EntryWithPhonetic}

\begin{EntryWithPhonetic}{餐饮}{can1 yin3}{16,7}{⾷,⾷}[HSK 5]
  \definition[个]{s.}{comidas e bebidas; refere"-se a atividades de bufê em restaurantes e hotéis}
\end{EntryWithPhonetic}

\begin{EntryWithPhonetic}{餐桌}{can1zhuo1}{16,10}{⾷,⽊}[HSK 7-9]
  \definition[张]{s.}{mesa de jantar}
\end{EntryWithPhonetic}

%%%%%%%%%% 残 %%%%%%%%%%
\subsection*{残}\addcontentsline{loh}{figure}{残 \dpy{can2}}

\begin{EntryWithPhonetic}{残}{can2}{9}{⽍}[HSK 7-9]
  \definition{adj.}{incompleto; fragmentário; deficiente | remanescente; restante | cruel; feroz | opressivo; selvagem; bárbaro}
  \definition{v.}{ferir; danificar | estragar; prejudicar; destruir}
\end{EntryWithPhonetic}

\begin{EntryWithPhonetic}{残疾}{can2ji2}{9,10}{⽍,⽧}[HSK 6]
  \definition{s.}{deformidade; deficiência; deficiência física; defeitos de membros, órgãos ou funções fisiológicas}
\end{EntryWithPhonetic}

\begin{EntryWithPhonetic}{残疾人}{can2 ji2 ren2}{9,10,2}{⽍,⽧,⼈}[HSK 6]
  \definition[位,名]{s.}{pessoa com deficiência (ou incapacitada); o incapacitado (ou deficiente); pessoas com deficiência visual, auditiva, de linguagem, intelectual, física ou mental são os principais alvos da medicina de reabilitação}
\end{EntryWithPhonetic}

\begin{EntryWithPhonetic}{残酷}{can2ku4}{9,14}{⽍,⾣}[HSK 6]
  \definition{adj.}{cruel; brutal; implacável}
\end{EntryWithPhonetic}

\begin{EntryWithPhonetic}{残留}{can2liu2}{9,10}{⽍,⽥}[HSK 7-9]
  \definition{adj.}{residual; restante}
  \definition{s.}{vestígio; resto}
  \definition{v.}{sobrar}
\end{EntryWithPhonetic}

\begin{EntryWithPhonetic}{残缺}{can2que1}{9,10}{⽍,⽸}[HSK 7-9]
  \definition{adj.}{incompleto; fragmentário; com partes faltando}
\end{EntryWithPhonetic}

\begin{EntryWithPhonetic}{残忍}{can2ren3}{9,7}{⽍,⼼}[HSK 7-9]
  \definition{adj.}{cruel; implacável; impiedoso}
\end{EntryWithPhonetic}

%%%%%%%%%% 蚕 %%%%%%%%%%
\subsection*{蚕}\addcontentsline{loh}{figure}{蚕 \dpy{can2}}

\begin{EntryWithPhonetic}{蚕}{can2}{10}{⾍}
  \definition[只,条]{s.}{bicho"-da"-seda; um inseto que pode fiar seda e fazer casulos}
\end{EntryWithPhonetic}

\begin{EntryWithPhonetic}{蚕纸}{can2zhi3}{10,7}{⾍,⽷}
  \definition{s.}{papel onde o bicho"-da"-seda põe seus ovos}
\end{EntryWithPhonetic}

%%%%%%%%%% 惭 %%%%%%%%%%
\subsection*{惭}\addcontentsline{loh}{figure}{惭 \dpy{can2}}

\begin{EntryWithPhonetic}{惭}{can2}{11}{⼼}
  \definition{adj.}{envergonhado}
  \definition{s.}{vergonha}
  \definition{v.}{sentir vergonha}
\end{EntryWithPhonetic}

\begin{EntryWithPhonetic}{惭愧}{can2kui4}{11,12}{⼼,⼼}[HSK 7-9]
  \definition{adj.}{envergonhado; sentir"-se inseguro por ter deficiências, fazer algo errado ou não cumprir responsabilidades}
\end{EntryWithPhonetic}

%%%%%%%%%% 惨 %%%%%%%%%%
\subsection*{惨}\addcontentsline{loh}{figure}{惨 \dpy{can3}}

\begin{EntryWithPhonetic}{惨}{can3}{11}{⽕}[HSK 6]
  \definition{adj.}{miserável; trágico | cruel; brutal; implacável | desastroso; terrível; esmagador | lamentável; desaventurado | em um grau sério; grau grave; dano grave | selvagem; desumano; vicioso; cruel}
\end{EntryWithPhonetic}

\begin{EntryWithPhonetic}{惨白}{can3bai2}{11,5}{⽕,⽩}[HSK 7-9]
  \definition{adj.}{(cenário) escuro; fraco; sombrio; sombrio | (rosto) mortalmente pálido; medonho | terrivelmente (fantasmagórico) pálido; pálido; escuro}
\end{EntryWithPhonetic}

\begin{EntryWithPhonetic}{惨痛}{can3tong4}{11,12}{⽕,⽧}[HSK 7-9]
  \definition{adj.}{grave; terrivelmente doloroso; amargo; agonizante | profundamente triste; doloroso}
\end{EntryWithPhonetic}

\begin{EntryWithPhonetic}{惨重}{can3zhong4}{11,9}{⽕,⾥}[HSK 7-9]
  \definition{adj.}{pesado; doloroso; desastroso | calamitoso (perdas extremamente graves)}
\end{EntryWithPhonetic}

%%%%%%%%%% 灿 %%%%%%%%%%
\subsection*{灿}\addcontentsline{loh}{figure}{灿 \dpy{can4}}

\begin{EntryWithPhonetic}{灿}{can4}{7}{⽕}
  \definition{adj.}{brilhante; luminoso; resplandecente; flamejante; deslumbrante}
\end{EntryWithPhonetic}

\begin{EntryWithPhonetic}{灿烂}{can4lan4}{7,9}{⽕,⽕}[HSK 7-9]
  \definition{adj.}{brilhante; glorioso; esplêndido; resplandecente; magnífico; deslumbrante}
\end{EntryWithPhonetic}

%%%%%%%%%% 掺 %%%%%%%%%%
\subsection*{掺}\addcontentsline{loh}{figure}{掺 \dpy{can4}}

\begin{EntryWithPhonetic}{掺}{can4}{11}{⼿}
  \definition{s.}{um estilo antigo de tocar bateria; uma antiga canção de tambor}
  \seeref{chan1}
  \seeref{shan3}
\end{EntryWithPhonetic}

%%%%%%%%%% 仓 %%%%%%%%%%
\subsection*{仓}\addcontentsline{loh}{figure}{仓 \dpy{cang1}}

\begin{EntryWithPhonetic}{仓}{cang1}{4}{⼈}
  \definition*{s.}{Sobrenome: Cang}
  \definition{s.}{armazém; depósito}
\end{EntryWithPhonetic}

\begin{EntryWithPhonetic}{仓库}{cang1ku4}{4,7}{⼈,⼴}[HSK 6]
  \definition[间,座,个]{s.}{armazém; depósito; um edifício usado para armazenar grandes quantidades de alimentos ou outros suprimentos}
\end{EntryWithPhonetic}

%%%%%%%%%% 沧 %%%%%%%%%%
\subsection*{沧}\addcontentsline{loh}{figure}{沧 \dpy{cang1}}

\begin{EntryWithPhonetic}{沧}{cang1}{7}{⽔}
  \definition{adj.}{(mar) azul profundo | azul"-esverdeado ou azul-celeste (água) | frio | vasto (água)}
\end{EntryWithPhonetic}

\begin{EntryWithPhonetic}{沧海桑田}{cang1 hai3 sang1 tian2}{7,10,10,5}{⽔,⽔,⽊,⽥}
  \definition{expr.}{mudança dos mares para campos de amoreiras e dos campos de amoreiras para mares, o tempo traz grandes mudanças; vicissitudes | Figurativo: as transformações do mundo | Literário: o mar azul se transformou em campos de amoreiras}
\end{EntryWithPhonetic}

\begin{EntryWithPhonetic}{沧桑}{cang1sang1}{7,10}{⽔,⽊}[HSK 7-9]
  \definition{adj.}{vicissitudes; grandes mudanças; altos e baixos; abreviação de 沧海桑田}
  \seealsoref{沧海桑田}{cang1 hai3 sang1 tian2}
\end{EntryWithPhonetic}

%%%%%%%%%% 苍 %%%%%%%%%%
\subsection*{苍}\addcontentsline{loh}{figure}{苍 \dpy{cang1}}

\begin{EntryWithPhonetic}{苍}{cang1}{7}{⾋}
  \definition*{s.}{Sobrenome: Cang}
  \definition{adj.}{verde escuro (ou azul); ciano (inclui azul e verde) | cinza; acinzentado}
  \definition{s.}{o céu azul; o céu acima}
\end{EntryWithPhonetic}

\begin{EntryWithPhonetic}{苍蝇}{cang1ying5}{7,14}{⾋,⾍}[HSK 7-9]
  \definition[只,个,群]{s.}{mosca;mosca doméstica}
\end{EntryWithPhonetic}

%%%%%%%%%% 舱 %%%%%%%%%%
\subsection*{舱}\addcontentsline{loh}{figure}{舱 \dpy{cang1}}

\begin{EntryWithPhonetic}{舱}{cang1}{10}{⾈}[HSK 7-9]
  \definition{s.}{cabine (de um avião ou navio) | módulo (de uma nave espacial) | espaço em um navio ou aeronave para transportar pessoas, carga ou máquinas}
\end{EntryWithPhonetic}

%%%%%%%%%% 藏 %%%%%%%%%%
\subsection*{藏}\addcontentsline{loh}{figure}{藏 \dpy{cang2}}

\begin{EntryWithPhonetic}{藏}{cang2}{17}{⾋}[HSK 6]
  \definition*{s.}{Sobrenome: Cang}
  \definition{v.}{esconder; ocultar; esconder da vista | armazenar; coletar; colocar de lado}
  \seeref{zang4}
\end{EntryWithPhonetic}

\begin{EntryWithPhonetic}{藏匿}{cang2ni4}{17,10}{⾋,⼖}[HSK 7-9]
  \definition{v.}{esconder; esconder"-se | abrigar}
\end{EntryWithPhonetic}

\begin{EntryWithPhonetic}{藏品}{cang2pin3}{17,9}{⾋,⼝}[HSK 7-9]
  \definition[件]{s.}{artigos coletados; coleção | item de colecionador | peça de museu | objeto precioso}
\end{EntryWithPhonetic}

\begin{EntryWithPhonetic}{藏身}{cang2shen1}{17,7}{⾋,⾝}[HSK 7-9]
  \definition{v.}{esconder"-se; esconder | abrigar"-se; estabelecer"-se}
\end{EntryWithPhonetic}

%%%%%%%%%% 操 %%%%%%%%%%
\subsection*{操}\addcontentsline{loh}{figure}{操 \dpy{cao1}}

\begin{EntryWithPhonetic}{操}{cao1}{16}{⼿}
  \definition*{s.}{Sobrenome: Cao}
  \definition[节,套]{s.}{exercício; ginástica | conduta; comportamento; moralidade, a moral e o código de conduta que as pessoas seguem}
  \definition{v.}{segurar; agarrar; segurar na mão | fazer algo; envolver"-se em | falar (uma língua ou dialeto) | treinar (tropas); exercitar (corpo); praticar ou treinar de acordo com uma determinada forma ou postura | dirigir; manusear}
\end{EntryWithPhonetic}

\begin{EntryWithPhonetic}{操场}{cao1chang3}{16,6}{⼿,⼟}[HSK 4]
  \definition[个,片,座,处]{s.}{\emph{playground}; campo esportivo; locais para exercícios físicos ou exercícios militares}
\end{EntryWithPhonetic}

\begin{EntryWithPhonetic}{操控}{cao1kong4}{16,11}{⼿,⼿}[HSK 7-9]
  \definition{v.}{controlar; manipular}
\end{EntryWithPhonetic}

\begin{EntryWithPhonetic}{操劳}{cao1lao2}{16,7}{⼿,⼒}[HSK 7-9]
  \definition{v.}{trabalhar duro | cuidar; cuidar de}
\end{EntryWithPhonetic}

\begin{EntryWithPhonetic}{操心}{cao1/xin1}{16,4}{⼿,⼼}[HSK 7-9]
  \definition{v.+compl.}{esforçar"-se; preocupar"-se com; incomodar"-se com}
\end{EntryWithPhonetic}

\begin{EntryWithPhonetic}{操纵}{cao1zong4}{16,7}{⼿,⽷}[HSK 6]
  \definition{v.}{operar; controlar (uma máquina, instrumento, etc.) | manipular; controlar secretamente; assumir o controle de (uma pessoa, organização, situação, etc.)}
\end{EntryWithPhonetic}

\begin{EntryWithPhonetic}{操作}{cao1zuo4}{16,7}{⼿,⼈}[HSK 4]
  \definition{v.}{operar; seguir os requisitos e procedimentos prescritos | implementar; realizar; executar; refere"-se à implementação concreta (planos, medidas, etc.)}
\end{EntryWithPhonetic}

%%%%%%%%%% 槽 %%%%%%%%%%
\subsection*{槽}\addcontentsline{loh}{figure}{槽 \dpy{cao2}}

\begin{EntryWithPhonetic}{槽}{cao2}{15}{⽊}[HSK 7-9]
  \definition{clas.}{usado para portas | usado para porcos}
  \definition[个,道]{s.}{cocho | sulco; entalhe | canal | manjedoura (para água, ração animal, vinho, cuba); um recipiente para alimentar o gado, geralmente é retangular, alto em todos os lados e côncavo no meio, como uma caixa sem tampa | tanque de fermentação; cuba de vinho; geralmente se refere a certos utensílios com lados altos e côncavos no meio | leito do rio; fossa; refere"-se a certos cursos d'água ou valas com lados altos e um meio côncavo | ranhura; fenda; uma depressão semelhante a um sulco em um objeto}
\end{EntryWithPhonetic}

%%%%%%%%%% 草 %%%%%%%%%%
\subsection*{草}\addcontentsline{loh}{figure}{草 \dpy{cao3}}

\begin{EntryWithPhonetic}{草}{cao3}{9}{⾋}[HSK 2]
  \definition*{s.}{Sobrenome: Cao}
  \definition{adj.}{descuidado; rude | rascunho; inicial | femea; na linguagem coloquial, refere"-se a animais domésticos e aves fêmeas | precipitado; pouco cuidadoso | rascunho; não definitivo; preliminar; informal}
  \definition[种,棵,撮,株,根]{s.}{grama; gramado | palha | campo; zona rural; área selvagem | letra cursiva | letra cursiva (ou caligráfica) de um alfabeto fonético | rascunho | caligrafia cursiva; um tipo de escrita chinesa}
  \definition{v.}{esboçar; redigir}
\end{EntryWithPhonetic}

\begin{EntryWithPhonetic}{草案}{cao3'an4}{9,10}{⾋,⽊}[HSK 7-9]
  \definition[个,项,份,部]{s.}{rascunho (de um plano, lei, etc.); leis e regulamentos que foram escritos, mas ainda não foram finalizados pelos departamentos relevantes ou ainda estão sendo testados}
\end{EntryWithPhonetic}

\begin{EntryWithPhonetic}{草地}{cao3 di4}{9,6}{⾋,⼟}[HSK 2]
  \definition[片,块]{s.}{prado; gramado; campo; pastagem ou grande área de terra plantada com pastagem | gramado; relvado; local com grama alta ou gramado}
\end{EntryWithPhonetic}

\begin{EntryWithPhonetic}{草莓}{cao3mei2}{9,10}{⾋,⾋}
  \definition[颗]{s.}{morango}
\end{EntryWithPhonetic}

\begin{EntryWithPhonetic}{草坪}{cao3ping2}{9,8}{⾋,⼟}[HSK 7-9]
  \definition{s.}{gramado; prado plano; agora se refere principalmente a um pasto plano inteiro cultivado artificialmente}
\end{EntryWithPhonetic}

\begin{EntryWithPhonetic}{草原}{cao3 yuan2}{9,10}{⾋,⼚}[HSK 5]
  \definition[片,个]{s.}{estepe; pradaria; grandes áreas de terra coberta de vegetação em áreas semiáridas, intercaladas com árvores tolerantes à seca}
\end{EntryWithPhonetic}

\begin{EntryWithPhonetic}{草纸}{cao3zhi3}{9,7}{⾋,⽷}
  \definition{s.}{papel pardo | pergaminho | papel de palha áspero | papel higiênico}
\end{EntryWithPhonetic}

%%%%%%%%%% 肏 %%%%%%%%%%
\subsection*{肏}\addcontentsline{loh}{figure}{肏 \dpy{cao4}}

\begin{EntryWithPhonetic}{肏}{cao4}{8}{⼊}
  \definition{v.}{(vulgar) foder; palavras sujas usadas para insultar pessoas; refere"-se à relação sexual masculina}
\end{EntryWithPhonetic}

%%%%%%%%%% 册 %%%%%%%%%%
\subsection*{册}\addcontentsline{loh}{figure}{册 \dpy{ce4}}

\begin{EntryWithPhonetic}{册}{ce4}{5}{⼌}[HSK 5]
  \definition{clas.}{usado para cópias de livros}
  \definition{s.}{volume; livro | cópia; volume | ordem imperial para conferir um título}
  \definition{v.}{conferir um título}
\end{EntryWithPhonetic}

%%%%%%%%%% 侧 %%%%%%%%%%
\subsection*{侧}\addcontentsline{loh}{figure}{侧 \dpy{ce4}}

\begin{EntryWithPhonetic}{侧}{ce4}{8}{⼈}[HSK 6]
  \definition*{s.}{Sobrenome: Ce}
  \definition{s.}{lado | inclinação}
  \definition{v.}{inclinar; inclinar"-se}
  \seeref{zhai1}
\end{EntryWithPhonetic}

\begin{EntryWithPhonetic}{侧面}{ce4mian4}{8,9}{⼈,⾯}[HSK 7-9]
  \definition[个]{s.}{lado; flanco | vias indiretas; canais informais; algum aspecto; outro aspecto}
\end{EntryWithPhonetic}

\begin{EntryWithPhonetic}{侧重}{ce4zhong4}{8,9}{⼈,⾥}[HSK 7-9]
  \definition{v.}{enfatizar; focar em (um certo aspecto)}
\end{EntryWithPhonetic}

%%%%%%%%%% 厕 %%%%%%%%%%
\subsection*{厕}\addcontentsline{loh}{figure}{厕 \dpy{ce4}}

\begin{EntryWithPhonetic}{厕}{ce4}{8}{⼚}
  \definition[个,间]{s.}{latrina; fossa sanitária; (componente formador de palavras)}
  \seeref{si5}
  \seealsoref{茅厕}{mao2ce4}
\end{EntryWithPhonetic}

\begin{EntryWithPhonetic}{厕所}{ce4suo3}{8,8}{⼚,⼾}[HSK 6]
  \definition[个,间]{s.}{banheiro; lavatório; sanitário; latrina; um lugar para as pessoas urinarem e defecarem}
\end{EntryWithPhonetic}

\begin{EntryWithPhonetic}{厕纸}{ce4zhi3}{8,7}{⼚,⽷}
  \definition{s.}{papel higiênico}
\end{EntryWithPhonetic}

%%%%%%%%%% 测 %%%%%%%%%%
\subsection*{测}\addcontentsline{loh}{figure}{测 \dpy{ce4}}

\begin{EntryWithPhonetic}{测}{ce4}{9}{⽔}[HSK 4]
  \definition{v.}{pesquisar; sondar; medir | conjecturar; advinhar}
\end{EntryWithPhonetic}

\begin{EntryWithPhonetic}{测定}{ce4 ding4}{9,8}{⽔,⼧}[HSK 6]
  \definition{v.}{verificar por medição (ou levantamento); determinar; medir; avaliar}
\end{EntryWithPhonetic}

\begin{EntryWithPhonetic}{测量}{ce4liang2}{9,12}{⽔,⾥}[HSK 4]
  \definition{v.}{aferir; pesquisar; medir; determinar valores relevantes para espaço, tempo, temperatura, velocidade, função, etc.}
\end{EntryWithPhonetic}

\begin{EntryWithPhonetic}{测试}{ce4 shi4}{9,8}{⽔,⾔}[HSK 4]
  \definition[次,场]{s.}{exame; teste; medição do conhecimento humano, das habilidades ou do funcionamento de máquinas, ferramentas ou instrumentos}
  \definition{v.}{examinar | testar, medição do desempenho e da precisão de máquinas, instrumentos, aparelhos, etc.}
\end{EntryWithPhonetic}

\begin{EntryWithPhonetic}{测算}{ce4suan4}{9,14}{⽔,⽵}[HSK 7-9]
  \definition{v.}{medir e calcular | adivinhar e estimar; especular}
\end{EntryWithPhonetic}

\begin{EntryWithPhonetic}{测验}{ce4yan4}{9,10}{⽔,⾺}[HSK 7-9]
  \definition[次,个]{s.}{teste; execução de teste; testes com instrumentos ou outros métodos}
  \definition{v.}{testar; verificar o desempenho acadêmico, etc.}
\end{EntryWithPhonetic}

%%%%%%%%%% 策 %%%%%%%%%%
\subsection*{策}\addcontentsline{loh}{figure}{策 \dpy{ce4}}

\begin{EntryWithPhonetic}{策}{ce4}{12}{⽵}
  \definition*{s.}{Sobrenome: Ce}
  \definition[个,项,根]{s.}{plano; esquema | tiras de bambu ou madeira usadas para escrever na China antiga | questões sobre atualidades definidas para os exames imperiais | chicote de montaria antigo | um tipo de ensaio na China antiga; um estilo de escrita para exames antigos | estratégia; método}
  \definition{v.}{chicotear (um cavalo) com um chicote de montaria | incitar com um chicote de cavalo, espora}
\end{EntryWithPhonetic}

\begin{EntryWithPhonetic}{策划}{ce4hua4}{12,6}{⽵,⼑}[HSK 6]
  \definition{v.}{planejar; traçar; esquematizar; pensar repetidamente para elaborar um plano}
\end{EntryWithPhonetic}

\begin{EntryWithPhonetic}{策略}{ce4lve4}{12,11}{⽵,⽥}[HSK 6]
  \definition{adj.}{diplomático; (métodos) flexíveis sem sacrificar princípios}
  \definition[种,个,条,套]{s.}{tática; estratégia; política; para atingir determinadas tarefas estratégicas, o curso de ação e os métodos de luta são formulados de acordo com o desenvolvimento da situação}
\end{EntryWithPhonetic}

%%%%%%%%%% 层 %%%%%%%%%%
\subsection*{层}\addcontentsline{loh}{figure}{层 \dpy{ceng2}}

\begin{EntryWithPhonetic}{层}{ceng2}{7}{⼫}[HSK 2]
  \definition{clas.}{usado para coisas que se sobrepõem e se acumulam, como andares, camadas e estratos | usado para coisas que podem ser divididas em itens e etapas | usado para coisas que podem ser removidas ou apagadas da superfície de um objeto}
  \definition{s.}{camada; nível; coisas que se sobrepõem | nível; classificação; camada}
  \definition{v.}{sobrepor; empilhar camada sobre camada}
\end{EntryWithPhonetic}

\begin{EntryWithPhonetic}{层层}{ceng2ceng2}{7,7}{⼫,⼫}
  \definition{s.}{camada sobre camada}
\end{EntryWithPhonetic}

\begin{EntryWithPhonetic}{层出不穷}{ceng2chu1-bu4qiong2}{7,5,4,7}{⼫,⼐,⼀,⽳}[HSK 7-9]
  \definition{expr.}{surgem um após o outro; surgem em um fluxo sem fim; aparecem em sucessão; surgem continuamente; sem fim}
\end{EntryWithPhonetic}

\begin{EntryWithPhonetic}{层次}{ceng2ci4}{7,6}{⼫,⽋}[HSK 5]
  \definition[个]{s.}{disposição ordenada do conteúdo (de um discurso ou texto) | nível ou estrutura administrativa; distinções entre a mesma coisa devido a diferenças de tamanho, altura, etc. | nível; níveis de afiliação}
\end{EntryWithPhonetic}

\begin{EntryWithPhonetic}{层面}{ceng2 mian4}{7,9}{⼫,⾯}[HSK 6]
  \definition[个]{s.}{escopo; alcance | aspecto; campo}
\end{EntryWithPhonetic}

%%%%%%%%%% 曾 %%%%%%%%%%
\subsection*{曾}\addcontentsline{loh}{figure}{曾 \dpy{ceng2}}

\begin{EntryWithPhonetic}{曾}{ceng2}{12}{⽈}[HSK 4]
  \definition{adv.}{indica que uma ação já aconteceu ou um estado já existiu}
  \seeref{zeng1}
\end{EntryWithPhonetic}

\begin{EntryWithPhonetic}{曾经}{ceng2jing1}{12,8}{⽈,⽷}[HSK 3]
  \definition{adv.}{uma vez; indica que houve algum comportamento ou situação}
\end{EntryWithPhonetic}

%%%%%%%%%% 蹭 %%%%%%%%%%
\subsection*{蹭}\addcontentsline{loh}{figure}{蹭 \dpy{ceng4}}

\begin{EntryWithPhonetic}{蹭}{ceng4}{19}{⾜}[HSK 7-9]
  \definition{v.}{esfregar; raspar; arranhar | esfregar em algo e ficar manchado; ser manchado com; manchar por fricção | mover"-se lentamente; demorar"-se; arrastar"-se | Dialeto: roubar}
\end{EntryWithPhonetic}

\begin{EntryWithPhonetic}{蹭蹬}{ceng4deng4}{19,19}{⾜,⾜}
  \definition{interj.}{``Droga!''}
  \definition{v.}{enfrentar contratempos; estar sem sorte; ter má sorte}
\end{EntryWithPhonetic}

%%%%%%%%%% 叉 %%%%%%%%%%
\subsection*{叉}\addcontentsline{loh}{figure}{叉 \dpy{cha1}}

\begin{EntryWithPhonetic}{叉}{cha1}{3}{⼜}[HSK 5]
  \definition{s.}{garfo; forquilha | símbolo de cruz, ``×''}
  \definition{v.}{trabalhar com um garfo; garfar; pegar coisas com um garfo}
  \seeref{cha2}
  \seeref{cha3}
\end{EntryWithPhonetic}

\begin{EntryWithPhonetic}{叉子}{cha1zi5}{3,3}{⼜,⼦}[HSK 5]
  \definition[把,个]{s.}{garfo; ferramenta com mais de duas pontas em uma extremidade | tridente; forquilha; ferramentas de agricultura antigas}
\end{EntryWithPhonetic}

%%%%%%%%%% 差 %%%%%%%%%%
\subsection*{差}\addcontentsline{loh}{figure}{差 \dpy{cha1}}

\begin{EntryWithPhonetic}{差}{cha1}{9}{⼯}
  \definition{adj.}{diferente; diferente ou inconsistente com um determinado padrão}
  \definition{adv.}{ligeiramente; comparativamente; um pouco}
  \definition{s.}{diferença; resto após a subtração de dois números | erro; engano}
  \seeref{cha4}
  \seeref{chai1}
\end{EntryWithPhonetic}

\begin{EntryWithPhonetic}{差别}{cha1bie2}{9,7}{⼯,⼑}[HSK 5]
  \definition{s.}{diferença; disparidade; dissimilaridade; distinção; não semelhança; diferenças na forma ou no conteúdo}
\end{EntryWithPhonetic}

\begin{EntryWithPhonetic}{差错}{cha1cuo4}{9,13}{⼯,⾦}[HSK 7-9]
  \definition[出]{s.}{deslize; erro; engano | acidente; contratempo; mudanças inesperadas}
\end{EntryWithPhonetic}

\begin{EntryWithPhonetic}{差额}{cha1'e2}{9,15}{⼯,⾴}[HSK 7-9]
  \definition{s.}{equilíbrio; diferencial; margem; a diferença entre um valor usado como padrão ou comparação}
\end{EntryWithPhonetic}

\begin{EntryWithPhonetic}{差距}{cha1ju4}{9,11}{⼯,⾜}[HSK 5]
  \definition[个,些,段]{s.}{lacuna; disparidade; discrepância; diferença; grau de diferença entre as coisas, especialmente em termos de distância de algum padrão.}
\end{EntryWithPhonetic}

\begin{EntryWithPhonetic}{差(一)点儿}{cha1yi4dian3r5}{9,1,9,2}{⼯,⼀,⽕,⼉}
  \definition{adj.}{não é bom o suficiente; ligeiramente inferior a; não está à altura da marca;  (qualidade, tecnologia, desempenho, etc.) ligeiramente inferior}
  \definition{adv.}{quase; à beira de; praticamente; aproximadamente; significa que algo está perto de ser alcançado, mas não foi alcançado, ou algo foi alcançado, mas mal foi alcançado}
\end{EntryWithPhonetic}

\begin{EntryWithPhonetic}{差异}{cha1 yi4}{9,6}{⼯,⼶}[HSK 6]
  \definition{s.}{diferença; divergência; discrepância}
\end{EntryWithPhonetic}

%%%%%%%%%% 插 %%%%%%%%%%
\subsection*{插}\addcontentsline{loh}{figure}{插 \dpy{cha1}}

\begin{EntryWithPhonetic}{插}{cha1}{12}{⼿}[HSK 5]
  \definition{v.}{perfurar; inserir | interpor; inserir; colocar no meio}
\end{EntryWithPhonetic}

\begin{EntryWithPhonetic}{插话}{cha1/hua4}{12,8}{⼿,⾔}
  \definition{s.}{interrupção | digressão}
  \definition{v.+compl.}{interromper (a fala de alguém)}
\end{EntryWithPhonetic}

\begin{EntryWithPhonetic}{插手}{cha1/shou3}{12,4}{⼿,⼿}[HSK 7-9]
  \definition{v.+compl.}{participar; dar uma mão | meter a mão em; meter o nariz em; intrometer"-se | ter (tomar) uma mão em}
\end{EntryWithPhonetic}

\begin{EntryWithPhonetic}{插图}{cha1tu2}{12,8}{⼿,⼞}[HSK 7-9]
  \definition[张,幅]{s.}{ilustração (artística ou científica) | ilustração; figura; mapa; demonstração; inserção}
\end{EntryWithPhonetic}

\begin{EntryWithPhonetic}{插嘴}{cha1/zui3}{12,16}{⼿,⼝}[HSK 7-9]
  \definition{v.+compl.}{interromper; intrometer"-se; participar da conversa (geralmente de forma inadequada)}
\end{EntryWithPhonetic}

%%%%%%%%%% 叉 %%%%%%%%%%
\subsection*{叉}\addcontentsline{loh}{figure}{叉 \dpy{cha2}}

\begin{EntryWithPhonetic}{叉}{cha2}{3}{⼜}
  \definition{v.}{bloquear; emperrar; congestionar}
  \seeref{cha1}
  \seeref{cha3}
\end{EntryWithPhonetic}

%%%%%%%%%% 查 %%%%%%%%%%
\subsection*{查}\addcontentsline{loh}{figure}{查 \dpy{cha2}}

\begin{EntryWithPhonetic}{查}{cha2}{9}{⽊}[HSK 2]
  \definition{v.}{examinar; verificar cuidadosamente | examinar; investigar; entender bem a situação | procurar; consultar; revisar (documentos bibliográficos)}
  \seeref{zha1}
\end{EntryWithPhonetic}

\begin{EntryWithPhonetic}{查出}{cha2 chu1}{9,5}{⽊,⼐}[HSK 6]
  \definition{v.}{rastrear; desentocar}
\end{EntryWithPhonetic}

\begin{EntryWithPhonetic}{查处}{cha2chu3}{9,5}{⽊,⼡}[HSK 7-9]
  \definition{v.}{investigar e lidar (com um caso criminal)}
\end{EntryWithPhonetic}

\begin{EntryWithPhonetic}{查看}{cha2 kan4}{9,9}{⽊,⽬}[HSK 6]
  \definition{v.}{verificar; examinar; checar; investigar; verificar e observar a existência das coisas}
\end{EntryWithPhonetic}

\begin{EntryWithPhonetic}{查明}{cha2ming2}{9,8}{⽊,⽇}[HSK 7-9]
  \definition{v.}{provar por meio de investigação; descobrir; apurar}
\end{EntryWithPhonetic}

\begin{EntryWithPhonetic}{查询}{cha2 xun2}{9,8}{⽊,⾔}[HSK 5]
  \definition{v.}{indagar; inquirir; perguntar sobre}
\end{EntryWithPhonetic}

\begin{EntryWithPhonetic}{查找}{cha2zhao3}{9,7}{⽊,⼿}[HSK 7-9]
  \definition{v.}{procurar; pesquisar; tentar encontrar as informações que você precisa}
\end{EntryWithPhonetic}

%%%%%%%%%% 茶 %%%%%%%%%%
\subsection*{茶}\addcontentsline{loh}{figure}{茶 \dpy{cha2}}

\begin{EntryWithPhonetic}{茶}{cha2}{9}{⾋}[HSK 1]
  \definition{adj.}{moreno; fulvo; amarelo-acastanhado}
  \definition[杯,壶]{s.}{chá (a bebida); bebida feita com folhas de chá | chá (a planta) | certos tipos de bebidas ou alimentos líquidos | árvore de chá-de-óleo | camélia}
\end{EntryWithPhonetic}

\begin{EntryWithPhonetic}{茶道}{cha2dao4}{9,12}{⾋,⾡}[HSK 7-9]
  \definition{s.}{cerimônia do chá ; cerimônia japonesa do chá; Sadō}
\end{EntryWithPhonetic}

\begin{EntryWithPhonetic}{茶馆儿}{cha2guan3r5}{9,11,2}{⾋,⾷,⼉}[HSK 7-9]
  \definition[家,个,间]{s.}{casa de chá}
\end{EntryWithPhonetic}

\begin{EntryWithPhonetic}{茶叶}{cha2 ye4}{9,5}{⾋,⼝}[HSK 4]
  \definition[包,袋,盒,斤,把,种]{s.}{chá; folhas de chá; as folhas jovens da planta do chá que são processadas para produzir bebidas}
\end{EntryWithPhonetic}

%%%%%%%%%% 察 %%%%%%%%%%
\subsection*{察}\addcontentsline{loh}{figure}{察 \dpy{cha2}}

\begin{EntryWithPhonetic}{察}{cha2}{14}{⼧}
  \definition*{s.}{Sobrenome: Cha}
  \definition{v.}{examinar; investigar; escrutinar | observar; olhar atentamente; investigar}
\end{EntryWithPhonetic}

\begin{EntryWithPhonetic}{察觉}{cha2jue2}{14,9}{⼧,⾒}[HSK 7-9]
  \definition{v.}{detectar; perceber; estar ciente de; estar consciente de; descobrir; ver}
\end{EntryWithPhonetic}

\begin{EntryWithPhonetic}{察看}{cha2kan4}{14,9}{⼧,⽬}[HSK 7-9]
  \definition{v.}{observar; olhar atentamente; inspecionar}
\end{EntryWithPhonetic}

%%%%%%%%%% 叉 %%%%%%%%%%
\subsection*{叉}\addcontentsline{loh}{figure}{叉 \dpy{cha3}}

\begin{EntryWithPhonetic}{叉}{cha3}{3}{⼜}
  \definition{v.}{separar de modo a formar uma bifurcação; bifurcar}
  \seeref{cha1}
  \seeref{cha2}
\end{EntryWithPhonetic}

%%%%%%%%%% 刹 %%%%%%%%%%
\subsection*{刹}\addcontentsline{loh}{figure}{刹 \dpy{cha4}}

\begin{EntryWithPhonetic}{刹}{cha4}{8}{⼑}
  \definition*{s.}{abreviação de 刹多罗 (Kshatara), sânscrito ``ksetra''}
  \definition{s.}{mosteiro, templo ou santuário budista}
  \seeref{sha1}
  \seealsoref{刹多罗}{sha1duo1luo2}
\end{EntryWithPhonetic}

%%%%%%%%%% 诧 %%%%%%%%%%
\subsection*{诧}\addcontentsline{loh}{figure}{诧 \dpy{cha4}}

\begin{EntryWithPhonetic}{诧}{cha4}{8}{⾔}
  \definition{v.}{ficar surpreso}
\end{EntryWithPhonetic}

\begin{EntryWithPhonetic}{诧异}{cha4yi4}{8,6}{⾔,⼶}[HSK 7-9]
  \definition{v.}{ficar surpreso; ficar espantado}
\end{EntryWithPhonetic}

%%%%%%%%%% 差 %%%%%%%%%%
\subsection*{差}\addcontentsline{loh}{figure}{差 \dpy{cha4}}

\begin{EntryWithPhonetic}{差}{cha4}{9}{⼯}[HSK 1]
  \definition{adj.}{não está de acordo com o padrão; pobre; ruim; inferior | errado; incorreto | mesmo significado de 差 \dpy{cha1}}
  \definition{v.}{faltar}
  \seeref{cha1}
  \seeref{chai1}
\end{EntryWithPhonetic}

\begin{EntryWithPhonetic}{差不多}{cha4bu5duo1}{9,4,6}{⼯,⼀,⼣}[HSK 2]
  \definition{adj.}{semelhante; aproximadamente igual | não muito longe; quase certo (suficiente); basicamente, próximo dos padrões e requisitos; normal | prestes a (terminar; acabar); descreve que (algo) está quase acabando; (uma tarefa) está quase concluída}
  \definition{adv.}{quase; perto; indica proximidade}
\end{EntryWithPhonetic}

\begin{EntryWithPhonetic}{差点儿}{cha4dian3r5}{9,9,2}{⼯,⽕,⼉}[HSK 5]
  \definition{adv.}{por pouco | por um triz | quase}
\end{EntryWithPhonetic}

%%%%%%%%%% 拆 %%%%%%%%%%
\subsection*{拆}\addcontentsline{loh}{figure}{拆 \dpy{chai1}}

\begin{EntryWithPhonetic}{拆}{chai1}{8}{⼿}[HSK 5]
  \definition{v.}{rasgar; desmontar; separar o que está unido | derrubar; desmantelar; demolir; refere"-se especificamente à demolição de edifícios}
\end{EntryWithPhonetic}

\begin{EntryWithPhonetic}{拆除}{chai1 chu2}{8,9}{⼿,⾩}[HSK 5]
  \definition{v.}{desmantelar; demolir; derrubar; remover (um edifício, etc.)}
\end{EntryWithPhonetic}

\begin{EntryWithPhonetic}{拆迁}{chai1 qian1}{8,6}{⼿,⾡}[HSK 6]
  \definition{v.}{demolir uma casa velha e realocar seus ocupantes em outro lugar; devido às necessidades de construção, unidades ou casas residenciais são demolidas e realocadas em outros lugares}
\end{EntryWithPhonetic}

%%%%%%%%%% 差 %%%%%%%%%%
\subsection*{差}\addcontentsline{loh}{figure}{差 \dpy{chai1}}

\begin{EntryWithPhonetic}{差}{chai1}{9}{⼯}
  \definition{s.}{tarefa; trabalho; ser enviado para fazer algo; deveres oficiais; posição | corvéia; mensageiro ou oficial de justiça em um yamen feudal; Antigo: refere"-se a pessoas que são enviadas para fazer coisas}
  \definition{v.}{enviar uma mensagem; despachar; fnviar (para fazer algo)}
  \seeref{cha1}
  \seeref{cha4}
\end{EntryWithPhonetic}

%%%%%%%%%% 掺 %%%%%%%%%%
\subsection*{掺}\addcontentsline{loh}{figure}{掺 \dpy{chan1}}

\begin{EntryWithPhonetic}{掺}{chan1}{11}{⼿}[HSK 7-9]
  \definition{v.}{misturar; mesclar; adicionar}
  \seeref{can4}
  \seeref{shan3}
\end{EntryWithPhonetic}

%%%%%%%%%% 搀 %%%%%%%%%%
\subsection*{搀}\addcontentsline{loh}{figure}{搀 \dpy{chan1}}

\begin{EntryWithPhonetic}{搀}{chan1}{12}{⼿}[HSK 7-9]
  \definition{v.}{apoiar alguém pelo braço; apoiar alguém com a mão; apoiar | misturar}
\end{EntryWithPhonetic}

%%%%%%%%%% 单 %%%%%%%%%%
\subsection*{单}\addcontentsline{loh}{figure}{单 \dpy{chan2}}

\begin{EntryWithPhonetic}{单}{chan2}{8}{⼗}
  \definition{s.}{usado em 单于 \dpy{chan2yu2}}
  \seeref{dan1}
  \seeref{shan4}
  \seealsoref{单于}{chan2yu2}
\end{EntryWithPhonetic}

\begin{EntryWithPhonetic}{单于}{chan2yu2}{8,3}{⼗,⼆}
  \definition{s.}{rei de Xiongnu (匈奴)}
  \seealsoref{匈奴}{xiong1nu2}
\end{EntryWithPhonetic}

%%%%%%%%%% 禅 %%%%%%%%%%
\subsection*{禅}\addcontentsline{loh}{figure}{禅 \dpy{chan2}}

\begin{EntryWithPhonetic}{禅}{chan2}{12}{⽰}
  \definition{s.}{Budismo: contemplação prolongada e intensa; meditação profunda | budista; refere"-se geralmente a coisas relacionadas ao budismo}
  \seeref{shan4}
\end{EntryWithPhonetic}

\begin{EntryWithPhonetic}{禅杖}{chan2zhang4}{12,7}{⽰,⽊}[HSK 7-9]
  \definition[根,支]{s.}{cajado (bastão) do monge budista | uma bengala com cabeça acolchoada para bater na cabeça de quem adormece}
\end{EntryWithPhonetic}

%%%%%%%%%% 馋 %%%%%%%%%%
\subsection*{馋}\addcontentsline{loh}{figure}{馋 \dpy{chan2}}

\begin{EntryWithPhonetic}{馋}{chan2}{12}{⾷}[HSK 7-9]
  \definition{adj.}{cobiçoso; invejoso | guloso; comilão; glutão}
  \definition{v.}{desejar comida; querer comer (alguma coisa)}
\end{EntryWithPhonetic}

%%%%%%%%%% 缠 %%%%%%%%%%
\subsection*{缠}\addcontentsline{loh}{figure}{缠 \dpy{chan2}}

\begin{EntryWithPhonetic}{缠}{chan2}{13}{⽷}[HSK 7-9]
  \definition*{s.}{Sobrenome: Chan}
  \definition{v.}{enrolar; entrelaçar; bobinar | emaranhar | amarrar; importunar; perturbar | Dialeto: lidar com}
\end{EntryWithPhonetic}

%%%%%%%%%% 蝉 %%%%%%%%%%
\subsection*{蝉}\addcontentsline{loh}{figure}{蝉 \dpy{chan2}}

\begin{EntryWithPhonetic}{蝉}{chan2}{14}{⾍}
  \definition[只,个]{s.}{cigarra}
  \seealsoref{知了}{zhi1liao3}
\end{EntryWithPhonetic}

%%%%%%%%%% 产 %%%%%%%%%%
\subsection*{产}\addcontentsline{loh}{figure}{产 \dpy{chan3}}

\begin{EntryWithPhonetic}{产}{chan3}{6}{⼇}[HSK 7-9]
  \definition*{s.}{Sobrenome: Chan}
  \definition{s.}{produto | propriedade; espólio | (abreviação) indústria}
  \definition{v.}{dar à luz; ser entregue a | produzir; render | separar um ser humano ou animal de sua mãe}
\end{EntryWithPhonetic}

\begin{EntryWithPhonetic}{产地}{chan3di4}{6,6}{⼇,⼟}[HSK 7-9]
  \definition{s.}{local de produção (ou origem); área de produção; o local onde o item é produzido}
\end{EntryWithPhonetic}

\begin{EntryWithPhonetic}{产后}{chan3hou4}{6,6}{⼇,⼝}
  \definition{s.}{pós-parto}
\end{EntryWithPhonetic}

\begin{EntryWithPhonetic}{产量}{chan3 liang4}{6,12}{⼇,⾥}[HSK 6]
  \definition{v.}{rendimento; produção; a quantidade de produção; a quantidade total de produtos produzidos em um determinado período de tempo}
\end{EntryWithPhonetic}

\begin{EntryWithPhonetic}{产品}{chan3pin3}{6,9}{⼇,⼝}[HSK 4]
  \definition[个,件,种,批,项,类]{s.}{produto; item produzido}
\end{EntryWithPhonetic}

\begin{EntryWithPhonetic}{产生}{chan3sheng1}{6,5}{⼇,⽣}[HSK 3]
  \definition{v.}{produzir; evoluir; emergir; provocar; vir a ser; dar origem a; criar coisas novas e novos fenômenos a partir do que já existe}
\end{EntryWithPhonetic}

\begin{EntryWithPhonetic}{产物}{chan3wu4}{6,8}{⼇,⽜}[HSK 7-9]
  \definition{s.}{resultado; produto; coisas que ocorrem sob certas condições}
\end{EntryWithPhonetic}

\begin{EntryWithPhonetic}{产业}{chan3ye4}{6,5}{⼇,⼀}[HSK 5]
  \definition{s.}{patrimônio; propriedade; bens pessoais, como terrenos, casas, fábricas, etc. | indústria; refere"-se especificamente à produção industrial moderna | setor; indústria; indústrias e setores da economia nacional}
\end{EntryWithPhonetic}

\begin{EntryWithPhonetic}{产值}{chan3zhi2}{6,10}{⼇,⼈}[HSK 7-9]
  \definition{s.}{valor de saída; o valor monetário de todos os produtos ou de um produto específico em um período de tempo}
\end{EntryWithPhonetic}

%%%%%%%%%% 铲 %%%%%%%%%%
\subsection*{铲}\addcontentsline{loh}{figure}{铲 \dpy{chan3}}

\begin{EntryWithPhonetic}{铲}{chan3}{11}{⾦}[HSK 7-9]
  \definition[个,把]{s.}{pá}
  \definition{v.}{trabalhar com uma pá (ou enxada) | levantar (mover) com uma pá}
\end{EntryWithPhonetic}

\begin{EntryWithPhonetic}{铲车}{chan3che1}{11,4}{⾦,⾞}
  \definition[台]{s.}{empilhadeira}
\end{EntryWithPhonetic}

\begin{EntryWithPhonetic}{铲子}{chan3zi5}{11,3}{⾦,⼦}[HSK 7-9]
  \definition[把]{s.}{pá; uma ferramenta com uma chapa de ferro grossa, quase quadrada, em uma extremidade e um cabo longo na outra}
\end{EntryWithPhonetic}

%%%%%%%%%% 阐 %%%%%%%%%%
\subsection*{阐}\addcontentsline{loh}{figure}{阐 \dpy{chan3}}

\begin{EntryWithPhonetic}{阐}{chan3}{11}{⾨}
  \definition{v.}{explicar; expor; expressar; divulgar; esclarecer; elucidar}
\end{EntryWithPhonetic}

\begin{EntryWithPhonetic}{阐述}{chan3shu4}{11,8}{⾨,⾡}[HSK 7-9]
  \definition{v.}{explicar; expor; elaborar; discutir}
\end{EntryWithPhonetic}

%%%%%%%%%% 颤 %%%%%%%%%%
\subsection*{颤}\addcontentsline{loh}{figure}{颤 \dpy{chan4}}

\begin{EntryWithPhonetic}{颤}{chan4}{19}{⾴}
  \definition{v.}{tremer; estremecer | vibrar; tremer; sacudir}
\end{EntryWithPhonetic}

\begin{EntryWithPhonetic}{颤抖}{chan4dou3}{19,7}{⾴,⼿}[HSK 7-9]
  \definition{v.}{tremer; estremecer; tremular; tiritar}
\end{EntryWithPhonetic}

%%%%%%%%%% 昌 %%%%%%%%%%
\subsection*{昌}\addcontentsline{loh}{figure}{昌 \dpy{chang1}}

\begin{EntryWithPhonetic}{昌}{chang1}{8}{⽇}
  \definition*{s.}{Sobrenome: Chang}
  \definition{adj.}{próspero; florescente | adequado; bom}
\end{EntryWithPhonetic}

\begin{EntryWithPhonetic}{昌盛}{chang1 sheng4}{8,11}{⽇,⽫}[HSK 6]
  \definition{adj.}{(país, nação, etc.) próspero; florescente}
\end{EntryWithPhonetic}

%%%%%%%%%% 猖 %%%%%%%%%%
\subsection*{猖}\addcontentsline{loh}{figure}{猖 \dpy{chang1}}

\begin{EntryWithPhonetic}{猖}{chang1}{11}{⽝}
  \definition{adj.}{louco; indisciplinado; dissoluto; licencioso; precipitado; imprudente | Literário: feroz}
\end{EntryWithPhonetic}

\begin{EntryWithPhonetic}{猖狂}{chang1kuang2}{11,7}{⽝,⽝}[HSK 7-9]
  \definition{adj.}{selvagem; desenfreado; furioso; imprudente; arrogante e presunçoso}
\end{EntryWithPhonetic}

%%%%%%%%%% 长 %%%%%%%%%%
\subsection*{长}\addcontentsline{loh}{figure}{长 \dpy{chang2}}

\begin{EntryWithPhonetic}{长}{chang2}{4}{⾧}[HSK 2][Kangxi 168]
  \definition*{s.}{Sobrenome: Chang}
  \definition{adj.}{longo; comprido; a distância entre uma extremidade e outra é grande | para sempre; duradouro; a distância entre o ponto inicial e o ponto final de um determinado período é grande | excesso; o que sobra; o que é desnecessário}
  \definition{s.}{comprimento; na direção horizontal, a distância percorrida por um objeto de uma extremidade à outra | ponto forte; especialidade; especialização em determinada área; vantagem}
  \definition{v.}{ser bom em; ser forte em; ser muito bom em algo; ser especialista em algo}
  \seeref{zhang3}
  \antonymref{短}{duan3}
\end{EntryWithPhonetic}

\begin{EntryWithPhonetic}{长城}{chang2cheng2}{4,9}{⾧,⼟}[HSK 3]
  \definition*[段,座,条]{s.}{A Grande Muralha da China; também é usado como metáfora para uma força poderosa e inabalável, um obstáculo intransponível, etc.}
\end{EntryWithPhonetic}

\begin{EntryWithPhonetic}{长处}{chang2 chu4}{4,5}{⾧,⼡}[HSK 3]
  \definition[个]{s.}{forte; boas qualidades; pontos fortes; especialização em determinada área}
\end{EntryWithPhonetic}

\begin{EntryWithPhonetic}{长达}{chang2 da2}{4,6}{⾧,⾡}[HSK 7-9]
  \definition{v.}{estender"-se tanto quanto | alongar"-se para}
\end{EntryWithPhonetic}

\begin{EntryWithPhonetic}{长度}{chang2 du4}{4,9}{⾧,⼴}[HSK 5]
  \definition{s.}{comprimento; extensão; distância entre dois pontos}
\end{EntryWithPhonetic}

\begin{EntryWithPhonetic}{长短}{chang2 duan3}{4,12}{⾧,⽮}[HSK 6]
  \definition{s.}{comprimento | acidente; infortúnio | certo e errado; pontos fortes e fracos}
\end{EntryWithPhonetic}

\begin{EntryWithPhonetic}{长假}{chang2 jia4}{4,11}{⾧,⼈}[HSK 6]
  \definition{s.}{longa licença de ausência | feriado prolongado | Obsoleto: renúncia | férias longas | refere"-se a um feriado nacional de uma semana na China, começando em 1º de maio e 1º de outubro}
\end{EntryWithPhonetic}

\begin{EntryWithPhonetic}{长颈鹿}{chang2jing3lu4}{4,11,11}{⾧,⾴,⿅}
  \definition[只]{s.}{girafa}
\end{EntryWithPhonetic}

\begin{EntryWithPhonetic}{长久}{chang2 jiu3}{4,3}{⾧,⼃}[HSK 6]
  \definition{adj.}{longo; permanente; duradouro; por muito tempo; já faz muito tempo}
\end{EntryWithPhonetic}

\begin{EntryWithPhonetic}{长跑}{chang2 pao3}{4,12}{⾧,⾜}[HSK 6]
  \definition{s.}{corrida de longa distância}
  \antonymref{短跑}{duan3 pao3}
\end{EntryWithPhonetic}

\begin{EntryWithPhonetic}{长期}{chang2 qi1}{4,12}{⾧,⽉}[HSK 3]
  \definition{adj.}{secular; longo prazo; longo alcance; durante um longo período de tempo}
  \definition{s.}{longo prazo; por muito tempo}
\end{EntryWithPhonetic}

\begin{EntryWithPhonetic}{长期以来}{chang2qi1 yi3lai2}{4,12,4,7}{⾧,⽉,⼈,⽊}[HSK 7-9]
  \definition{adv.}{por muito tempo passado | desde muito tempo atrás}
\end{EntryWithPhonetic}

\begin{EntryWithPhonetic}{长寿}{chang2 shou4}{4,7}{⾧,⼨}[HSK 5]
  \definition{adj.}{vida longa; longevidade}
\end{EntryWithPhonetic}

\begin{EntryWithPhonetic}{长途}{chang2tu2}{4,10}{⾧,⾡}[HSK 4]
  \definition{adj.}{de longa distância; longe}
  \definition[段,次,程]{s.}{longa"-distância; referindo"-se especificamente a chamadas telefônicas de longa distância ou ônibus de longa distância}
\end{EntryWithPhonetic}

\begin{EntryWithPhonetic}{长效}{chang2xiao4}{4,10}{⾧,⽁}[HSK 7-9]
  \definition{adj.}{duradouro}
  \definition{s.}{efeito duradouro}
  \definition{v.}{ser eficaz por um período prolongado}
\end{EntryWithPhonetic}

\begin{EntryWithPhonetic}{长远}{chang2 yuan3}{4,7}{⾧,⾡}[HSK 6]
  \definition{adj.}{longo prazo; longo alcance; duradouro; há muito tempo (referindo"-se ao futuro); crônico}
\end{EntryWithPhonetic}

\begin{EntryWithPhonetic}{长征}{chang2zheng1}{4,8}{⾧,⼻}[HSK 7-9]
  \definition*{s.}{A Longa Marcha (1934-1935), um importante movimento estratégico do Exército Vermelho dos Trabalhadores e Camponeses da China que conseguiu alcançar a base revolucionária no norte de Shaanxi depois de atravessar onze províncias e cobrir 25.000 li ou 12.500 quilômetros}
  \definition{s.}{expedição; longa marcha; viagem de longa distância; expedição de longa distância}
\end{EntryWithPhonetic}

\begin{EntryWithPhonetic}{长足}{chang2zu2}{4,7}{⾧,⾜}[HSK 7-9]
  \definition{adj.}{Literário: aos trancos e barrancos; substancial; progredindo rapidamente}
\end{EntryWithPhonetic}

%%%%%%%%%% 场 %%%%%%%%%%
\subsection*{场}\addcontentsline{loh}{figure}{场 \dpy{chang2}}

\begin{EntryWithPhonetic}{场}{chang2}{6}{⼟}
  \definition{clas.}{usado para descrever o desenrolar dos acontecimentos}
  \definition{s.}{eira; espaço aberto e plano; um terreno plano, geralmente usado para secar grãos e moer cereais | mercado; feira rural}
  \seeref{chang3}
\end{EntryWithPhonetic}

%%%%%%%%%% 肠 %%%%%%%%%%
\subsection*{肠}\addcontentsline{loh}{figure}{肠 \dpy{chang2}}

\begin{EntryWithPhonetic}{肠}{chang2}{7}{⾁}[HSK 5]
  \definition[根,段,片]{s.}{intestinos; parte do sistema digestivo | salsicha; linguiça; alimentos de tripas recheadas com carne, peixe, etc. | sentimentos; emoções; humor}
\end{EntryWithPhonetic}

%%%%%%%%%% 尝 %%%%%%%%%%
\subsection*{尝}\addcontentsline{loh}{figure}{尝 \dpy{chang2}}

\begin{EntryWithPhonetic}{尝}{chang2}{9}{⼩}[HSK 5]
  \definition{adv.}{alguma vez; uma vez}
  \definition{v.}{provar; experimentar o sabor de | provar; experimentar; conhecer | tentar; testar}
\end{EntryWithPhonetic}

\begin{EntryWithPhonetic}{尝试}{chang2shi4}{9,8}{⼩,⾔}[HSK 5]
  \definition{v.}{tentar; provar; experimentar}
\end{EntryWithPhonetic}

%%%%%%%%%% 倘 %%%%%%%%%%
\subsection*{倘}\addcontentsline{loh}{figure}{倘 \dpy{chang2}}

\begin{EntryWithPhonetic}{倘}{chang2}{10}{⼈}
  \seeref{tang3}
\end{EntryWithPhonetic}

%%%%%%%%%% 偿 %%%%%%%%%%
\subsection*{偿}\addcontentsline{loh}{figure}{偿 \dpy{chang2}}

\begin{EntryWithPhonetic}{偿}{chang2}{11}{⼈}
  \definition{v.}{reembolsar; compensar | realizar; cumprir | cumprir; satisfazer}
\end{EntryWithPhonetic}

\begin{EntryWithPhonetic}{偿还}{chang2huan2}{11,7}{⼈,⾡}[HSK 7-9]
  \definition{v.}{reembolsar; pagar de volta; pagar (uma dívida)}
\end{EntryWithPhonetic}

%%%%%%%%%% 常 %%%%%%%%%%
\subsection*{常}\addcontentsline{loh}{figure}{常 \dpy{chang2}}

\begin{EntryWithPhonetic}{常}{chang2}{11}{⼱}[HSK 1]
  \definition*{s.}{Sobrenome: Chang}
  \definition{adj.}{normal; comum; ordinário; indica frequência, normalidade, universalidade | constante; invariável; imutável; permanente}
  \definition{adv.}{frequentemente; geralmente; com frequência;}
  \definition{s.}{normas; disciplina, ordem social e lei e ordem do Estado}
\end{EntryWithPhonetic}

\begin{EntryWithPhonetic}{常常}{chang2 chang2}{11,11}{⼱,⼱}[HSK 1]
  \definition{adv.}{frequentemente; muitas vezes; geralmente; indica que a ação ocorreu várias vezes}
\end{EntryWithPhonetic}

\begin{EntryWithPhonetic}{常规}{chang2 gui1}{11,8}{⼱,⾒}[HSK 6]
  \definition[个,种]{s.}{convenção; prática comum; rotina | (medicina) rotina | regra; sulco}
\end{EntryWithPhonetic}

\begin{EntryWithPhonetic}{常见}{chang2 jian4}{11,4}{⼱,⾒}[HSK 2]
  \definition{adj.}{comum; frequentemente visto}
\end{EntryWithPhonetic}

\begin{EntryWithPhonetic}{常理}{chang2li3}{11,11}{⼱,⽟}[HSK 7-9]
  \definition{s.}{regra geral; o que é normal | senso comum; pensamento lógico | raciocínio convencional e moral}
\end{EntryWithPhonetic}

\begin{EntryWithPhonetic}{常年}{chang2 nian2}{11,6}{⼱,⼲}[HSK 6]
  \definition{adj.}{perene; anual}
  \definition{adv.}{ano após ano; ao longo do ano; durante todo o ano; longo prazo}
\end{EntryWithPhonetic}

\begin{EntryWithPhonetic}{常人}{chang2ren2}{11,2}{⼱,⼈}[HSK 7-9]
  \definition{s.}{pessoa comum; homem da rua}
\end{EntryWithPhonetic}

\begin{EntryWithPhonetic}{常识}{chang2shi2}{11,7}{⼱,⾔}[HSK 4]
  \definition[门]{s.}{senso comum; conhecimento geral; conhecimento que uma pessoa comum deve ter}
\end{EntryWithPhonetic}

\begin{EntryWithPhonetic}{常态}{chang2tai4}{11,8}{⼱,⼼}[HSK 7-9]
  \definition{s.}{normalidade; \emph{habitus}; comportamento normal; condições normais; estado normal ou usual}
\end{EntryWithPhonetic}

\begin{EntryWithPhonetic}{常温}{chang2wen1}{11,12}{⼱,⽔}[HSK 7-9]
  \definition{s.}{temperatura atmosférica normal; temperatura ordinária | homeotermia}
\end{EntryWithPhonetic}

\begin{EntryWithPhonetic}{常问问题}{chang2wen4wen4ti2}{11,6,6,15}{⼱,⾨,⾨,⾴}
  \definition{s.}{FAQ; perguntas frequentes}
\end{EntryWithPhonetic}

\begin{EntryWithPhonetic}{常用}{chang2 yong4}{11,5}{⼱,⽤}[HSK 2]
  \definition{adj.}{em uso comum; frequentemente utilizado}
\end{EntryWithPhonetic}

%%%%%%%%%% 嫦 %%%%%%%%%%
\subsection*{嫦}\addcontentsline{loh}{figure}{嫦 \dpy{chang2}}

\begin{EntryWithPhonetic}{嫦}{chang2}{14}{⼥}
  \definition{s.}{uma beleza lendária que voou para a lua | a dama da lua}
\end{EntryWithPhonetic}

\begin{EntryWithPhonetic}{嫦娥}{chang2'e2}{14,10}{⼥,⼥}[HSK 7-9]
  \definition*{s.}{Chang'e, a dama da lua (mitologia chinesa); uma fada que voou do mundo humano para o Palácio da Lua na mitologia}
\end{EntryWithPhonetic}

%%%%%%%%%% 厂 %%%%%%%%%%
\subsection*{厂}\addcontentsline{loh}{figure}{厂 \dpy{chang3}}

\begin{EntryWithPhonetic}{厂}{chang3}{2}{⼚}[HSK 3][Kangxi 27]
  \definition[家,间]{s.}{fábrica; moinho; planta; obra | pátio; depósito; refere"-se a um estabelecimento comercial com um amplo espaço para armazenamento de mercadorias e processamento}
  \seeref{an1}
  \seeref{han3}
\end{EntryWithPhonetic}

\begin{EntryWithPhonetic}{厂家}{chang3jia1}{2,10}{⼚,⼧}[HSK 7-9]
  \definition[个]{s.}{fábrica; refere"-se aos aspectos de fábrica ou fábrica}
\end{EntryWithPhonetic}

\begin{EntryWithPhonetic}{厂商}{chang3 shang1}{2,11}{⼚,⼝}[HSK 6]
  \definition[家,个]{s.}{empresa; fornecedor; fábrica; negócio; fabricante; uma unidade que produz e vende produtos; uma pessoa que administra uma fábrica}
\end{EntryWithPhonetic}

\begin{EntryWithPhonetic}{厂长}{chang3 zhang3}{2,4}{⼚,⾧}[HSK 5]
  \definition[位,个,名]{s.}{diretor de fábrica; gerente de fábrica; líder responsável pela produção, pela vida e por todos os outros assuntos de toda a fábrica}
\end{EntryWithPhonetic}

%%%%%%%%%% 场 %%%%%%%%%%
\subsection*{场}\addcontentsline{loh}{figure}{场 \dpy{chang3}}

\begin{EntryWithPhonetic}{场}{chang3}{6}{⼟}[HSK 2]
  \definition*{s.}{Sobrenome: Chang}
  \definition{clas.}{usado para atividades culturais, recreativas e esportivas | usado para pequenos trechos de uma peça}
  \definition{s.}{um local amplo utilizado para um fim específico | palco; campo | cena | Física: campo (por exemplo, campo magnético) | local para atividades recreativas, esportivas ou outras | um lugar onde as pessoas se reúnem | fazenda; quinta | abertura; encerramento; refere"-se ao processo completo de uma apresentação ou competição | local; ponto; o local onde ocorreu o incidente}
  \seeref{chang2}
\end{EntryWithPhonetic}

\begin{EntryWithPhonetic}{场地}{chang3 di4}{6,6}{⼟,⼟}[HSK 6]
  \definition[片,块,个]{s.}{área; pátio; espaço; lugar; quadra; campo; um lugar onde construções ou atividades são realizadas}
\end{EntryWithPhonetic}

\begin{EntryWithPhonetic}{场馆}{chang3 guan3}{6,11}{⼟,⾷}[HSK 6]
  \definition{s.}{ginásios e estádios | arena | local esportivo}
\end{EntryWithPhonetic}

\begin{EntryWithPhonetic}{场合}{chang3he2}{6,6}{⼟,⼝}[HSK 3]
  \definition[个,些,种,类]{s.}{ocasião; situação; um certo tempo, lugar ou situação}
\end{EntryWithPhonetic}

\begin{EntryWithPhonetic}{场景}{chang3 jing3}{6,12}{⼟,⽇}[HSK 6]
  \definition[个,幕,种]{s.}{espetáculo; cena (em drama, ficção, etc.); refere"-se a cenas de drama, cinema, televisão e obras literárias | cena; visão; circunstâncias; cenas e situações}
\end{EntryWithPhonetic}

\begin{EntryWithPhonetic}{场面}{chang3mian4}{6,9}{⼟,⾯}[HSK 5]
  \definition[个,种,番]{s.}{espetáculo; cena (em teatro, ficção, etc.); uma cena em uma produção teatral, cinematográfica ou televisiva que consiste em um cenário, música e personagens | cena; ocasião; literatura narrativa que consiste em situações da vida em que os personagens se relacionam entre si em determinadas ocasiões | orquestra ou instrumentos de acompanhamento (em ópera); refere"-se às pessoas e aos instrumentos musicais que acompanham a apresentação de uma ópera, divididos em dois tipos: música de sopro e cordas é uma cena cultural, e gongos e tambores são uma cena marcial | situação; referência geral a uma situação em um determinado contexto | frente; fachada; aparência; espetáculo superficial}
\end{EntryWithPhonetic}

\begin{EntryWithPhonetic}{场所}{chang3suo3}{6,8}{⼟,⼾}[HSK 3]
  \definition{s.}{lugar; sítio; arena; local da atividade}
\end{EntryWithPhonetic}

%%%%%%%%%% 敞 %%%%%%%%%%
\subsection*{敞}\addcontentsline{loh}{figure}{敞 \dpy{chang3}}

\begin{EntryWithPhonetic}{敞}{chang3}{12}{⽁}
  \definition{adj.}{espaçoso; aberto; desobstruído | Dialeto: (casa, pátio, etc.) espaçoso}
  \definition{v.}{abrir; descobrir}
\end{EntryWithPhonetic}

\begin{EntryWithPhonetic}{敞开}{chang3kai1}{12,4}{⽁,⼶}[HSK 7-9]
  \definition{adj.}{aberto; irrestrito}
  \definition{adv.}{livremente; sem reservas; ilimitadamente; irrestritamente; totalmente aberto; infinitamente; sem limite}
  \definition{v.}{abrir o máximo possível}
\end{EntryWithPhonetic}

%%%%%%%%%% 畅 %%%%%%%%%%
\subsection*{畅}\addcontentsline{loh}{figure}{畅 \dpy{chang4}}

\begin{EntryWithPhonetic}{畅}{chang4}{8}{⽥}
  \definition*{s.}{Sobrenome: Chang}
  \definition{adj.}{suave; desimpedido; sem obstáculos; desobstruído | livre; desinibido}
\end{EntryWithPhonetic}

\begin{EntryWithPhonetic}{畅谈}{chang4tan2}{8,10}{⽥,⾔}[HSK 7-9]
  \definition{v.}{falar livremente e com entusiasmo; falar livremente e com satisfação; falar com entusiasmo sobre}
\end{EntryWithPhonetic}

\begin{EntryWithPhonetic}{畅通}{chang4tong1}{8,10}{⽥,⾡}[HSK 6]
  \definition{adj.}{Suave
1. desimpedido; sem bloqueios
(Canal, rota) Desobstruído; não bloqueado.}
  \synonymref{贯通}{guan4tong1}
  \synonymref{流畅}{liu2chang4}
  \synonymref{流利}{liu2li4}
  \synonymref{流通}{liu2tong1}
  \antonymref{堵车}{du3/che1}
  \antonymref{堵塞}{du3se4}
  \antonymref{瓶颈}{ping2jing3}
  \antonymref{阻碍}{zu3'ai4}
\end{EntryWithPhonetic}

\begin{EntryWithPhonetic}{畅销}{chang4xiao1}{8,12}{⽥,⾦}[HSK 7-9]
  \definition{adj.}{mais vendido; \emph{best-seller}}
  \definition{v.}{vender bem; ter grande procura; ter um mercado pronto; ser um \emph{best-seller}}
\end{EntryWithPhonetic}

%%%%%%%%%% 倡 %%%%%%%%%%
\subsection*{倡}\addcontentsline{loh}{figure}{倡 \dpy{chang4}}

\begin{EntryWithPhonetic}{倡}{chang4}{10}{⼈}
  \definition{v.}{iniciar; propor; defender | promover; assumir a liderança}
\end{EntryWithPhonetic}

\begin{EntryWithPhonetic}{倡导}{chang4dao3}{10,6}{⼈,⼨}[HSK 5]
  \definition{v.}{iniciar; propor; promover; defender; advogar}
\end{EntryWithPhonetic}

\begin{EntryWithPhonetic}{倡议}{chang4yi4}{10,5}{⼈,⾔}[HSK 7-9]
  \definition[项,条,次]{s.}{proposta; iniciativa; primeiras sugestões}
  \definition{v.}{propor; iniciar; defender}
\end{EntryWithPhonetic}

%%%%%%%%%% 鬯 %%%%%%%%%%
\subsection*{鬯}\addcontentsline{loh}{figure}{鬯 \dpy{chang4}}

\begin{EntryWithPhonetic}{鬯}{chang4}{10}{⾿}[Kangxi 192]
  \definition{adj.}{suave; desimpedido | livre; desinibido}
  \definition{s.}{um tipo de vinho usado em sacrifícios antigos | (antigo) estojo ou bolsa para arco | o mesmo que 畅}
  \seealsoref{畅}{chang4}
\end{EntryWithPhonetic}

%%%%%%%%%% 唱 %%%%%%%%%%
\subsection*{唱}\addcontentsline{loh}{figure}{唱 \dpy{chang4}}

\begin{EntryWithPhonetic}{唱}{chang4}{11}{⼝}[HSK 1]
  \definition*{s.}{Sobrenome: Chang}
  \definition{s.}{uma música ou uma parte cantada de uma ópera chinesa; canções; letras de óperas tradicionais}
  \definition{v.}{cantar; seguir o ritmo da música | chorar; chamar; gritar, falar ou recitar em voz alta}
\end{EntryWithPhonetic}

\begin{EntryWithPhonetic}{唱歌}{chang4/ge1}{11,14}{⼝,⽋}[HSK 1]
  \definition{v.+compl.}{cantar (uma música); emitir sons com entonação ritmada e melodiosa; emitir sons (musicais) com a boca; emitir sons de acordo com a melodia}
\end{EntryWithPhonetic}

\begin{EntryWithPhonetic}{唱片}{chang4 pian4}{11,4}{⼝,⽚}[HSK 4]
  \definition[枚,张]{s.}{disco; disco feito de goma-laca, plástico, etc. com ranhuras em espiral na superfície para registrar alterações no som que podem reproduzir o som gravado em um fonógrafo}
\end{EntryWithPhonetic}

%%%%%%%%%% 抄 %%%%%%%%%%
\subsection*{抄}\addcontentsline{loh}{figure}{抄 \dpy{chao1}}

\begin{EntryWithPhonetic}{抄}{chao1}{7}{⼿}[HSK 4]
  \definition*{s.}{Sobrenome: Chao}
  \definition{v.}{copiar; transcrever | plagiar | registrar as leituras de um medidor | revistar e confiscar; fazer uma incursão em  | pegar um atalho | dobrar (os braços) | agarrar; pegar | ir (andar) embora com}
\end{EntryWithPhonetic}

\begin{EntryWithPhonetic}{抄表}{chao1 biao3}{7,8}{⼿,⾐}
  \definition{s.}{leitura do medidor}
\end{EntryWithPhonetic}

\begin{EntryWithPhonetic}{抄袭}{chao1xi2}{7,11}{⼿,⾐}[HSK 7-9]
  \definition{v.}{plagiar | emular indiscriminadamente; copiar | Militar: lançar um ataque surpresa ao inimigo fazendo um desvio; atacar a retaguarda ou os flancos do inimigo | Militar: tomar emprestado indiscriminadamente da experiência de outras pessoas; copiar a experiência e os métodos de outras pessoas}
\end{EntryWithPhonetic}

\begin{EntryWithPhonetic}{抄写}{chao1 xie3}{7,5}{⼿,⼍}[HSK 4]
  \definition{v.}{copiar; transcrever}
\end{EntryWithPhonetic}

%%%%%%%%%% 钞 %%%%%%%%%%
\subsection*{钞}\addcontentsline{loh}{figure}{钞 \dpy{chao1}}

\begin{EntryWithPhonetic}{钞}{chao1}{9}{⾦}
  \definition*{s.}{Sobrenome: Chao}
  \definition[张,把,叠,摞]{s.}{cédula; nota de banco; papel-moeda; nota bancária | escritos coletados; texto transcrito}
  \definition{v.}{copiar; transcrever; o mesmo que 抄}
  \seealsoref{抄}{chao1}
\end{EntryWithPhonetic}

\begin{EntryWithPhonetic}{钞票}{chao1piao4}{9,11}{⾦,⽰}[HSK 7-9]
  \definition[张,扎]{s.}{cédula; nota de banco; papel-moeda}
\end{EntryWithPhonetic}

%%%%%%%%%% 超 %%%%%%%%%%
\subsection*{超}\addcontentsline{loh}{figure}{超 \dpy{chao1}}

\begin{EntryWithPhonetic}{超}{chao1}{12}{⾛}[HSK 6]
  \definition{adj.}{super; extremamente; maior (ou menor) que o padrão geral}
  \definition{v.}{exceder; ultrapassar; vir para a frente por trás; prevalecer | transcender; ir além; não ser sujeito a certas restrições; ir além de um certo intervalo | exceder; superar; exceder o limite prescrito}
\end{EntryWithPhonetic}

\begin{EntryWithPhonetic}{超标}{chao1/biao1}{12,9}{⾛,⽊}[HSK 7-9]
  \definition{v.+compl.}{exceder uma cota (ou padrão); exceder o limite prescrito}
\end{EntryWithPhonetic}

\begin{EntryWithPhonetic}{超车}{chao1/che1}{12,4}{⾛,⾞}[HSK 7-9]
  \definition{v.+compl.}{ultrapassar (um veículo); passar por um veículo que trafega na mesma direção}
\end{EntryWithPhonetic}

\begin{EntryWithPhonetic}{超出}{chao1 chu1}{12,5}{⾛,⼐}[HSK 6]
  \definition{v.}{exceder; ultrapassar; ir além (de uma certa quantidade ou intervalo)}
\end{EntryWithPhonetic}

\begin{EntryWithPhonetic}{超过}{chao1guo4}{12,6}{⾛,⾡}[HSK 2]
  \definition{v.}{ultrapassar; superar (algo ou alguém); passar de trás para a frente de alguém ou algo | exceder; ser mais do que; ultrapassar (um padrão)}
\end{EntryWithPhonetic}

\begin{EntryWithPhonetic}{超级}{chao1ji2}{12,6}{⾛,⽷}[HSK 3]
  \definition{adj.}{super; além do nível geral}
  \definition{pref.}{super-; ultra-; hiper-}
\end{EntryWithPhonetic}

\begin{EntryWithPhonetic}{超级市场}{chao1 ji2 shi4 chang3}{12,6,5,6}{⾛,⽷,⼱,⼟}
  \definition[个,间,所,家]{s.}{supermercado; hipermercado}
\end{EntryWithPhonetic}

\begin{EntryWithPhonetic}{超前}{chao1qian2}{12,9}{⾛,⼑}[HSK 7-9]
  \definition{v.}{transcender os tempos; estar à frente dos tempos; estar à frente do seu tempo | superar os antecessores; liderar; assumir a liderança}
\end{EntryWithPhonetic}

\begin{EntryWithPhonetic}{超声}{chao1sheng1}{12,7}{⾛,⼠}
  \definition{adj.}{ultrasônico}
  \definition{s.}{ultrasom}
\end{EntryWithPhonetic}

\begin{EntryWithPhonetic}{超市}{chao1shi4}{12,5}{⾛,⼱}[HSK 2]
  \definition[家]{s.}{supermercado; abreviação de 超级市场}
  \seealsoref{超级市场}{chao1 ji2 shi4 chang3}
\end{EntryWithPhonetic}

\begin{EntryWithPhonetic}{超速}{chao1su4}{12,10}{⾛,⾡}[HSK 7-9]
  \definition{s.}{Física: hipervelocidade | excesso de velocidade}
  \definition{v.}{exceder o limite de velocidade}
\end{EntryWithPhonetic}

\begin{EntryWithPhonetic}{超越}{chao1yue4}{12,12}{⾛,⾛}[HSK 5]
  \definition{v.}{ultrapassar; superar; passar por cima; transcender}
\end{EntryWithPhonetic}

%%%%%%%%%% 巢 %%%%%%%%%%
\subsection*{巢}\addcontentsline{loh}{figure}{巢 \dpy{chao2}}

\begin{EntryWithPhonetic}{巢}{chao2}{11}{⼮}
  \definition*{s.}{Sobrenome: Chao}
  \definition[个]{s.}{ninho (de aves, insetos, etc.)}
\end{EntryWithPhonetic}

%%%%%%%%%% 朝 %%%%%%%%%%
\subsection*{朝}\addcontentsline{loh}{figure}{朝 \dpy{chao2}}

\begin{EntryWithPhonetic}{朝}{chao2}{12}{⽉}[HSK 3]
  \definition*{s.}{Sobrenome: Chao}
  \definition{prep.}{para; em direção a; a direção ou o objeto da ação introduzida, equivalente a 向 ou 对}
  \definition[个]{s.}{corte real; governo; assembleia realizada por um soberano; também se refere à posição no poder | dinastia, todo o período de governo transmitido de geração em geração por um determinado sobrenome imperial | reinado (de um soberano); o período de reinado de um determinado monarca}
  \definition{v.}{fazer uma peregrinação para; ter uma audiência com (um rei, um imperador, etc.) | estar voltado para; estar em frente a}
  \seeref{zhao1}
  \seealsoref{对}{dui4}
  \seealsoref{向}{xiang4}
  \antonymref{野}{ye3}
\end{EntryWithPhonetic}

\begin{EntryWithPhonetic}{朝代}{chao2dai4}{12,5}{⽉,⼈}[HSK 7-9]
  \definition[个]{s.}{dinastia; refere"-se a um período ou era histórica}
\end{EntryWithPhonetic}

\begin{EntryWithPhonetic}{朝廷}{chao2ting2}{12,6}{⽉,⼵}
  \definition{s.}{corte imperial | dinastia}
\end{EntryWithPhonetic}

\begin{EntryWithPhonetic}{朝鲜}{chao2xian3}{12,14}{⽉,⿂}
  \definition*{s.}{Coréia do Norte}
\end{EntryWithPhonetic}

\begin{EntryWithPhonetic}{朝着}{chao2zhe5}{12,11}{⽉,⽬}[HSK 7-9]
  \definition{prep.}{voltado para; em direção a; pessoas ou coisas voltadas para uma direção}
\end{EntryWithPhonetic}

%%%%%%%%%% 嘲 %%%%%%%%%%
\subsection*{嘲}\addcontentsline{loh}{figure}{嘲 \dpy{chao2}}

\begin{EntryWithPhonetic}{嘲}{chao2}{15}{⼝}
  \definition{v.}{ridicularizar; zombar; fazer piada de}
  \seeref{zhao1}
\end{EntryWithPhonetic}

\begin{EntryWithPhonetic}{嘲弄}{chao2nong4}{15,7}{⼝,⼶}[HSK 7-9]
  \definition{v.}{zombar; zombar de}
\end{EntryWithPhonetic}

\begin{EntryWithPhonetic}{嘲笑}{chao2xiao4}{15,10}{⼝,⽵}[HSK 7-9]
  \definition{v.}{ridicularizar; zombar; rir de; zombar de; fazer graça de; usar palavras para zombar de alguém}
\end{EntryWithPhonetic}

%%%%%%%%%% 潮 %%%%%%%%%%
\subsection*{潮}\addcontentsline{loh}{figure}{潮 \dpy{chao2}}

\begin{EntryWithPhonetic}{潮}{chao2}{15}{⽔}[HSK 4]
  \definition{adj.}{úmido; molhado | inferior; de qualidade ruim | inferior; não muito habilidoso}
  \definition{s.}{maré; água da maré | surto; corrente; maré; uma metáfora para mudanças sociais em grande escala ou para os altos e baixos de um movimento (social)}
  \definition{s.}{Chaozhou, uma cidade na província de Guangdong}
\end{EntryWithPhonetic}

\begin{EntryWithPhonetic}{潮流}{chao2liu2}{15,10}{⽔,⽔}[HSK 4]
  \definition[种,股,个]{s.}{maré; corrente de maré; movimento da água devido às marés | tendência; analogia com mudanças sociais ou tendências de desenvolvimento}
\end{EntryWithPhonetic}

\begin{EntryWithPhonetic}{潮湿}{chao2shi1}{15,12}{⽔,⽔}[HSK 4]
  \definition{adj.}{molhado; úmido; umedecido; que contém mais água do que o normal}
\end{EntryWithPhonetic}

\begin{EntryWithPhonetic}{潮绣}{chao2xiu4}{15,10}{⽔,⽷}
  \definition*{s.}{Bordado Chaozhou}
\end{EntryWithPhonetic}

%%%%%%%%%% 鼂 %%%%%%%%%%
\subsection*{鼂}\addcontentsline{loh}{figure}{鼂 \dpy{chao2}}

\begin{EntryWithPhonetic}{鼂}{chao2}{18}{⿌}
  \definition*{s.}{Sobrenome: Chao}
  \definition{s.}{tartaruga marinha}
\end{EntryWithPhonetic}

%%%%%%%%%% 吵 %%%%%%%%%%
\subsection*{吵}\addcontentsline{loh}{figure}{吵 \dpy{chao3}}

\begin{EntryWithPhonetic}{吵}{chao3}{7}{⼝}[HSK 3]
  \definition{adj.}{barulhento; ruidoso e perturbador}
  \definition{v.}{brigar; discutir; disputar}
\end{EntryWithPhonetic}

\begin{EntryWithPhonetic}{吵架}{chao3/jia4}{7,9}{⼝,⽊}[HSK 3]
  \definition{v.+compl.}{brigar; discutir; ter uma discussão acalorada}
\end{EntryWithPhonetic}

\begin{EntryWithPhonetic}{吵嘴}{chao3/zui3}{7,16}{⼝,⼝}[HSK 7-9]
  \definition{v.}{brigar; discutir}[我们之间从不吵嘴。===Nós nunca brigamos.]
\end{EntryWithPhonetic}

%%%%%%%%%% 炒 %%%%%%%%%%
\subsection*{炒}\addcontentsline{loh}{figure}{炒 \dpy{chao3}}

\begin{EntryWithPhonetic}{炒}{chao3}{8}{⽕}[HSK 6]
  \definition{v.}{saltear; refogar; aquecer os alimentos em uma panela e mexer repetidamente para cozinhá-los ou secá-los | especular (na bolsa de valores, etc.) | exagerar; dar publicidade exagerada; a fim de ampliar a influência, por meio de publicidade repetida e exagerada na mídia | demitir; despedir}
\end{EntryWithPhonetic}

\begin{EntryWithPhonetic}{炒股}{chao3/gu3}{8,8}{⽕,⾁}[HSK 6]
  \definition{v.+compl.}{especular em ações; comprar e vender ações; jogar no mercado}
\end{EntryWithPhonetic}

\begin{EntryWithPhonetic}{炒作}{chao3 zuo4}{8,7}{⽕,⼈}[HSK 6]
  \definition{v.}{promover (na mídia); exagerar artificialmente e promover ou desvalorizar de forma inadequada | especular; comprar e vender frequentemente no mercado de negociação para obter lucros}
\end{EntryWithPhonetic}

%%%%%%%%%% 车 %%%%%%%%%%
\subsection*{车}\addcontentsline{loh}{figure}{车 \dpy{che1}}

\begin{EntryWithPhonetic}{车}{che1}{4}{⾞}[HSK 1][Kangxi 159]
  \definition*{s.}{Sobrenome: Che}
  \definition[辆]{s.}{veículo; meios de transporte terrestres sobre rodas | máquina ou instrumento com rodas; ferramentas com eixo giratório | máquina}
  \definition{v.}{tornear; usinar com torno mecânico | elevar água por meio de uma roda d'água; usar caminhão-pipa para coletar água | girar, geralmente se refere ao corpo}
  \seeref{ju1}
\end{EntryWithPhonetic}

\begin{EntryWithPhonetic}{车次}{che1ci4}{4,6}{⾞,⽋}
  \definition{s.}{número do trem}
\end{EntryWithPhonetic}

\begin{EntryWithPhonetic}{车道}{che1dao4}{4,12}{⾞,⾡}[HSK 7-9]
  \definition{s.}{(tráfego) pista; faixa; estrada}[三车道公路。===Rodovia de três pistas.]
\end{EntryWithPhonetic}

\begin{EntryWithPhonetic}{车号}{che1 hao4}{4,5}{⾞,⼝}[HSK 6]
  \definition[个]{s.}{número da licença (de um veículo) | número do carro}
\end{EntryWithPhonetic}

\begin{EntryWithPhonetic}{车祸}{che1huo4}{4,11}{⾞,⽰}[HSK 7-9]
  \definition[场,起]{s.}{acidente de trânsito; acidente automobilístico; acidentes envolvendo vítimas durante a condução (principalmente carros)}
\end{EntryWithPhonetic}

\begin{EntryWithPhonetic}{车间}{che1jian1}{4,7}{⾞,⾨}[HSK 7-9]
  \definition[个,栋]{s.}{fábrica; oficina; uma unidade dentro de uma empresa que conclui certos processos no processo de produção ou produz certos produtos de forma independente}
\end{EntryWithPhonetic}

\begin{EntryWithPhonetic}{车库}{che1ku4}{4,7}{⾞,⼴}
  \definition{s.}{garagem}
\end{EntryWithPhonetic}

\begin{EntryWithPhonetic}{车辆}{che1 liang4}{4,11}{⾞,⾞}[HSK 2]
  \definition{s.}{veículo; carro; termo genérico para todos os tipos de veículos}
\end{EntryWithPhonetic}

\begin{EntryWithPhonetic}{车轮}{che1lun2}{4,8}{⾞,⾞}[HSK 7-9]
  \definition[个]{s.}{roda (de um veículo); a parte redonda e giratória de um carro; também chamada de 车轮子}
  \seealsoref{车轮子}{che1lun2zi5}
\end{EntryWithPhonetic}

\begin{EntryWithPhonetic}{车轮子}{che1lun2zi5}{4,8,3}{⾞,⾞,⼦}
  \definition{s.}{roda}
\end{EntryWithPhonetic}

\begin{EntryWithPhonetic}{车牌}{che1 pai2}{4,12}{⾞,⽚}[HSK 6]
  \definition[个,块]{s.}{placa de licença; placa instalada no veículo}
\end{EntryWithPhonetic}

\begin{EntryWithPhonetic}{车票}{che1 piao4}{4,11}{⾞,⽰}[HSK 1]
  \definition{s.}{passagem de trem ou ônibus; bilhete; bilhete de transporte público}
\end{EntryWithPhonetic}

\begin{EntryWithPhonetic}{车上}{che1 shang4}{4,3}{⾞,⼀}[HSK 1]
  \definition{adv.}{no carro; no interior do veículo}
\end{EntryWithPhonetic}

\begin{EntryWithPhonetic}{车水马龙}{che1shui3-ma3long2}{4,4,3,5}{⾞,⽔,⾺,⿓}
  \definition{expr.}{``Carruagens fluindo como água, cavalos nadando como dragões.'', descreve um grande número de carruagens e cavalos indo e vindo em um fluxo interminável; fluxo interminável de tráfego intenso; um fluxo de tráfego; estar lotado de pessoas e veículos; tráfego engarrafado; engarrafamento}
\end{EntryWithPhonetic}

\begin{EntryWithPhonetic}{车速}{che1su4}{4,10}{⾞,⾡}[HSK 7-9]
  \definition{s.}{velocidade de um veículo | velocidade de um torno}
\end{EntryWithPhonetic}

\begin{EntryWithPhonetic}{车位}{che1wei4}{4,7}{⾞,⼈}[HSK 7-9]
  \definition[个]{s.}{vaga de estacionamento | vaga de garagem | ponto de táxi | ponto de descarga}
\end{EntryWithPhonetic}

\begin{EntryWithPhonetic}{车厢}{che1xiang1}{4,11}{⾞,⼚}[HSK 7-9]
  \definition[节,个,号,辆]{s.}{compartimento; vagão ferroviário; trem de vagões; a parte de um carro ou outro veículo usado para transportar pessoas ou coisas}
\end{EntryWithPhonetic}

\begin{EntryWithPhonetic}{车型}{che1xing2}{4,9}{⾞,⼟}[HSK 7-9]
  \definition{s.}{marca e modelo do carro; tipo de motocicleta | modelo (de um carro)}
\end{EntryWithPhonetic}

\begin{EntryWithPhonetic}{车展}{che1 zhan3}{4,10}{⾞,⼫}[HSK 6]
  \definition{s.}{salão do automóvel; exposição de carros}
\end{EntryWithPhonetic}

\begin{EntryWithPhonetic}{车站}{che1 zhan4}{4,10}{⾞,⽴}[HSK 1]
  \definition[个,处]{s.}{estação; estação ferroviária; parada; pontos de parada estabelecidos nas linhas de transporte rodoviário são locais para embarque e desembarque de passageiros ou carga e descarga de mercadorias}
\end{EntryWithPhonetic}

\begin{EntryWithPhonetic}{车轴}{che1zhou2}{4,9}{⾞,⾞}[HSK 7-9]
  \definition[根,条,个]{s.}{eixo (de um veículo)}[这辆货车一共有六个轮子,所以有三条车轴。===Este caminhão tem seis rodas, então ele tem três eixos.]
\end{EntryWithPhonetic}

\begin{EntryWithPhonetic}{车主}{che1 zhu3}{4,5}{⾞,⼂}[HSK 5]
  \definition[位,个]{s.}{proprietário do carro; uma pessoa física ou família que possui um veículo motorizado}
\end{EntryWithPhonetic}

\begin{EntryWithPhonetic}{车子}{che1zi5}{4,3}{⾞,⼦}
  \definition{s.}{qualquer veículo (carro, bicicleta, caminhão, etc)}
\end{EntryWithPhonetic}

%%%%%%%%%% 尺 %%%%%%%%%%
\subsection*{尺}\addcontentsline{loh}{figure}{尺 \dpy{che3}}

\begin{EntryWithPhonetic}{尺}{che3}{4}{⼫}
  \definition{s.}{(tom) uma nota da escala em Gongchepu (工尺谱), correspondente a 2 na notação musical numerada}
  \seeref{chi3}
  \seealsoref{工尺谱}{gong1 che3 pu3}
\end{EntryWithPhonetic}

%%%%%%%%%% 扯 %%%%%%%%%%
\subsection*{扯}\addcontentsline{loh}{figure}{扯 \dpy{che3}}

\begin{EntryWithPhonetic}{扯}{che3}{7}{⼿}[HSK 7-9]
  \definition{v.}{puxar | rasgar; arrancar | comprar (tecido, linha, etc.) | conversar; fofocar; bater papo}
\end{EntryWithPhonetic}

%%%%%%%%%% 彻 %%%%%%%%%%
\subsection*{彻}\addcontentsline{loh}{figure}{彻 \dpy{che4}}

\begin{EntryWithPhonetic}{彻}{che4}{7}{⼻}
  \definition{adj.}{minucioso; completo; penetrante}
  \definition{adv.}{minuciosamente; profundamente}
\end{EntryWithPhonetic}

\begin{EntryWithPhonetic}{彻底}{che4di3}{7,8}{⼻,⼴}[HSK 4]
  \definition{adj.}{minucioso; completo; exaustivo; profundo e completo; nada é deixado de fora}
\end{EntryWithPhonetic}

\begin{EntryWithPhonetic}{彻夜}{che4ye4}{7,8}{⼻,⼣}[HSK 7-9]
  \definition{adv.}{a noite toda; durante toda a noite; do anoitecer ao amanhecer}
\end{EntryWithPhonetic}

%%%%%%%%%% 撤 %%%%%%%%%%
\subsection*{撤}\addcontentsline{loh}{figure}{撤 \dpy{che4}}

\begin{EntryWithPhonetic}{撤}{che4}{15}{⼿}[HSK 7-9]
  \definition{v.}{remover, tirar | demitir; liberar | retirar"-se; evacuar}
\end{EntryWithPhonetic}

\begin{EntryWithPhonetic}{撤换}{che4huan4}{15,10}{⼿,⼿}[HSK 7-9]
  \definition{v.}{demitir e substituir (alguém); revogar; substituir (alguém ou alguma coisa)}
\end{EntryWithPhonetic}

\begin{EntryWithPhonetic}{撤离}{che4 li2}{15,10}{⼿,⼇}[HSK 6]
  \definition{v.}{retirar"-se de; deixar; evacuar}
\end{EntryWithPhonetic}

\begin{EntryWithPhonetic}{撤销}{che4xiao1}{15,12}{⼿,⾦}[HSK 6]
  \definition{v.}{cancelar; rescindir; revogar; remover}
\end{EntryWithPhonetic}

%%%%%%%%%% 沉 %%%%%%%%%%
\subsection*{沉}\addcontentsline{loh}{figure}{沉 \dpy{chen2}}

\begin{EntryWithPhonetic}{沉}{chen2}{7}{⽔}[HSK 4]
  \definition{adj.}{profundo | pesado | pesado (sentir"-se pesado)}
  \definition{v.}{afundar; submergir; imergir | manter baixo; abaixar | descansar; parar}
\end{EntryWithPhonetic}

\begin{EntryWithPhonetic}{沉甸甸}{chen2dian4dian4}{7,7,7}{⽔,⽥,⽥}[HSK 7-9]
  \definition{adj.}{pesado; pesado e difícil de manejar}
\end{EntryWithPhonetic}

\begin{EntryWithPhonetic}{沉淀}{chen2dian4}{7,11}{⽔,⽔}[HSK 7-9]
  \definition{s.}{acúmulo; precipitado; sedimento; matéria sólida que afunda no fundo de líquidos como água ou óleo}
  \definition{v.}{assentar; baixar | acumular; reunir}
\end{EntryWithPhonetic}

\begin{EntryWithPhonetic}{沉浸}{chen2jin4}{7,10}{⽔,⽔}[HSK 7-9]
  \definition{v.}{estar imerso em; estar submerso em; estar permeado com; invasão de água, frequentemente usada como uma metáfora para estar em um determinado estado ou atividade de pensamento}
\end{EntryWithPhonetic}

\begin{EntryWithPhonetic}{沉闷}{chen2men4}{7,7}{⽔,⾨}[HSK 7-9]
  \definition{adj.}{triste; opressivo; deprimente; o clima e a atmosfera fazem as pessoas se sentirem pesadas e deprimidas | deprimido; desanimado; (humor) baixo; (caráter) nada alegre | (som e voz) baixo}
\end{EntryWithPhonetic}

\begin{EntryWithPhonetic}{沉迷}{chen2mi2}{7,9}{⽔,⾡}[HSK 7-9]
  \definition{v.}{entregar"-se; chafurdar; remoer}[不要沉迷在回忆中。===Não fique remoendo memórias.]
\end{EntryWithPhonetic}

\begin{EntryWithPhonetic}{沉默}{chen2mo4}{7,16}{⽔,⿊}[HSK 4]
  \definition{adj.}{silencioso; reticente; taciturno; não comunicativo}
  \definition{v.}{silenciar; não falar por causa de alguma coisa}
\end{EntryWithPhonetic}

\begin{EntryWithPhonetic}{沉思}{chen2si1}{7,9}{⽔,⼼}[HSK 7-9]
  \definition{v.}{ponderar; meditar; contemplar; perder"-se em pensamentos}
\end{EntryWithPhonetic}

\begin{EntryWithPhonetic}{沉稳}{chen2wen3}{7,14}{⽔,⽲}[HSK 7-9]
  \definition{adj.}{estável; sóbrio; calmo; sereno | calmo; imperturbável}
\end{EntryWithPhonetic}

\begin{EntryWithPhonetic}{沉重}{chen2zhong4}{7,9}{⽔,⾥}[HSK 4]
  \definition{adj.}{(pressão, fardo, etc.) muito pesado; profundo | sério; pesado; humor pouco animador; fardo pesado de pensamentos}
\end{EntryWithPhonetic}

\begin{EntryWithPhonetic}{沉着}{chen2zhuo2}{7,11}{⽔,⽬}[HSK 7-9]
  \definition{adj.}{calmo; estável; composto; cabeça fria | (diante de problemas) calmo; sem pressa}
\end{EntryWithPhonetic}

%%%%%%%%%% 陈 %%%%%%%%%%
\subsection*{陈}\addcontentsline{loh}{figure}{陈 \dpy{chen2}}

\begin{EntryWithPhonetic}{陈}{chen2}{7}{⾩}
  \definition*{s.}{Chen, um estado vassalo da Dinastia Zhou (1046-256 a.C.) | Dinastia Chen (557-589), uma das dinastias do sul | Sobrenome: Chen}
  \definition{adj.}{velho; obsoleto; desatualizado}
  \definition{v.}{expor; colocar em exposição; organizar; colocar para fora | declarar; explicar; expressar pensamentos, opiniões, etc. de maneira organizada}
\end{EntryWithPhonetic}

\begin{EntryWithPhonetic}{陈旧}{chen2jiu4}{7,5}{⾩,⽇}[HSK 7-9]
  \definition{adj.}{ultrapassado; fora de moda; antiquado; velho; desatualizado}
\end{EntryWithPhonetic}

\begin{EntryWithPhonetic}{陈列}{chen2lie4}{7,6}{⾩,⼑}[HSK 7-9]
  \definition{v.}{exibir; expor; organizar itens de uma determinada maneira}[博物馆陈列了古代的陶瓷器。===O museu exibe cerâmicas antigas.]
\end{EntryWithPhonetic}

\begin{EntryWithPhonetic}{陈述}{chen2shu4}{7,8}{⾩,⾡}[HSK 7-9]
  \definition{v.}{declarar; alegar; afirmar; falar de forma ordenada}
\end{EntryWithPhonetic}

%%%%%%%%%% 衬 %%%%%%%%%%
\subsection*{衬}\addcontentsline{loh}{figure}{衬 \dpy{chen4}}

\begin{EntryWithPhonetic}{衬}{chen4}{8}{⾐}
  \definition[件,个]{s.}{forro}
  \definition{v.}{forrar; colocar algo embaixo | fornecer um pano de fundo para; destacar; servir como contraste para}
\end{EntryWithPhonetic}

\begin{EntryWithPhonetic}{衬衫}{chen4shan1}{8,8}{⾐,⾐}[HSK 3]
  \definition[件,个]{s.}{camisa; blusa; camisa ocidental usada por baixo}
\end{EntryWithPhonetic}

\begin{EntryWithPhonetic}{衬托}{chen4tuo1}{8,6}{⾐,⼿}[HSK 7-9]
  \definition{v.}{destacar; acentuar}[红色衬托了她的笑容。===O vermelho acentuava seu sorriso.]
\end{EntryWithPhonetic}

\begin{EntryWithPhonetic}{衬衣}{chen4 yi1}{8,6}{⾐,⾐}[HSK 3]
  \definition[件,个]{s.}{camisa; também se refere a uma peça de roupa usada por baixo do casaco}
\end{EntryWithPhonetic}

%%%%%%%%%% 称 %%%%%%%%%%
\subsection*{称}\addcontentsline{loh}{figure}{称 \dpy{chen4}}

\begin{EntryWithPhonetic}{称}{chen4}{10}{⽲}
  \definition{adj.}{ajustado; encaixado; adequado}
  \definition{v.}{ajustar; adequar; combinar; estar em conformidade com; ser adequado para | ter; possuir}
  \seeref{cheng1}
\end{EntryWithPhonetic}

%%%%%%%%%% 趁 %%%%%%%%%%
\subsection*{趁}\addcontentsline{loh}{figure}{趁 \dpy{chen4}}

\begin{EntryWithPhonetic}{趁}{chen4}{12}{⾛}[HSK 7-9]
  \definition{prep.}{aproveitar"-se de; tirar vantagem de (tempo, oportunidade, etc.); indica o tempo e as condições de uso}
  \definition{v.}{ser rico em; possuir}
\end{EntryWithPhonetic}

\begin{EntryWithPhonetic}{趁机}{chen4ji1}{12,6}{⾛,⽊}[HSK 7-9]
  \definition{adv./adv.}{aproveitar a ocasião; aproveitar a oportunidade}
\end{EntryWithPhonetic}

\begin{EntryWithPhonetic}{趁早}{chen4zao3}{12,6}{⾛,⽇}[HSK 7-9]
  \definition{adv.}{o mais cedo possível; antes que seja tarde demais; na primeira oportunidade}
\end{EntryWithPhonetic}

\begin{EntryWithPhonetic}{趁着}{chen4zhe5}{12,11}{⾛,⽬}[HSK 7-9]
  \definition{v.}{aproveitar"-se de; tirar vantagem de}
\end{EntryWithPhonetic}

%%%%%%%%%% 称 %%%%%%%%%%
\subsection*{称}\addcontentsline{loh}{figure}{称 \dpy{cheng1}}

\begin{EntryWithPhonetic}{称}{cheng1}{10}{⽲}[HSK 2,5]
  \definition*{s.}{Sobrenome: Cheng}
  \definition{s.}{nome}
  \definition{v.}{chamar; ser chamado | dizer; declarar | elogiar; louvar; expressar afirmação ou elogio a pessoas ou coisas por meio de palavras | pesar; medir o peso | elevar; levantar; erguer | aplaudir; concordar; expressar suas opiniões ou sentimentos por meio de palavras ou ações | declarar"-se como; declarar que é; reivindicar ser alguém em virtude do próprio poder}
  \seeref{chen4}
\end{EntryWithPhonetic}

\begin{EntryWithPhonetic}{称号}{cheng1hao4}{10,5}{⽲,⼝}[HSK 5]
  \definition{s.}{título; nome; designação; nome dado a alguém, a uma organização ou a alguma coisa (geralmente usado de forma honrosa)}
\end{EntryWithPhonetic}

\begin{EntryWithPhonetic}{称呼}{cheng1hu5}{10,8}{⽲,⼝}[HSK 7-9]
  \definition{v.}{chamar; dirigir"-se (a alguém)}[我称呼他为老师。===Eu o chamo de professor.]
\end{EntryWithPhonetic}

\begin{EntryWithPhonetic}{称为}{cheng1 wei2}{10,4}{⽲,⼂}[HSK 3]
  \definition{v.}{ser chamado de; ser conhecido como; denominar}
\end{EntryWithPhonetic}

\begin{EntryWithPhonetic}{称赞}{cheng1zan4}{10,16}{⽲,⾙}[HSK 4]
  \definition[句,声,番,次]{s.}{elogio; aclamação; louvor; avaliação positiva de um desempenho ou conquista}
  \definition{v.}{elogiar; aclamar; louvar; usar palavras para expressar um carinho pelas virtudes de uma pessoa ou coisa}
\end{EntryWithPhonetic}

\begin{EntryWithPhonetic}{称作}{cheng1zuo4}{10,7}{⽲,⼈}[HSK 7-9]
  \definition{v.}{ser chamado | ser conhecido como}
\end{EntryWithPhonetic}

%%%%%%%%%% 撑 %%%%%%%%%%
\subsection*{撑}\addcontentsline{loh}{figure}{撑 \dpy{cheng1}}

\begin{EntryWithPhonetic}{撑}{cheng1}{15}{⼿}[HSK 6]
  \definition{s.}{suporte; escora;  apoio; esteio}
  \definition{v.}{sustentar; apoiar; resistir a | empurrar (ou mover) com uma vara; usar um mastro para empurrar a margem ou o leito do rio para fazer o barco avançar | manter; manter"-se atualizado | abrir; desdobrar; expandir (um objeto contraído) | encher até estourar (inchaço devido a excesso de comida ou alimentação excessiva)}
\end{EntryWithPhonetic}

%%%%%%%%%% 瞠 %%%%%%%%%%
\subsection*{瞠}\addcontentsline{loh}{figure}{瞠 \dpy{cheng1}}

\begin{EntryWithPhonetic}{瞠}{cheng1}{16}{⽬}
  \definition{v.}{Literário: olhar fixamente para algo além do alcance}
\end{EntryWithPhonetic}

%%%%%%%%%% 成 %%%%%%%%%%
\subsection*{成}\addcontentsline{loh}{figure}{成 \dpy{cheng2}}

\begin{EntryWithPhonetic}{成}{cheng2}{6}{⼽}[HSK 2,6]
  \definition*{s.}{Sobrenome: Cheng}
  \definition{adj.}{capaz; competente | totalmente crescido; totalmente desenvolvido; maduro | estabelecido; Já definido; pronto para uso | em números ou quantidades consideráveis; inteiro; suficiente: enfatiza a quantidade ou a duração}
  \definition{clas.}{um décimo}
  \definition{interj.}{``O.K.!''; ``Tudo bem!''}
  \definition{s.}{resultado; conquista}
  \definition{v.}{ter sucesso; conseguir; ser bem"-sucedido | tornar"-se; transformar"-se | ajudar a completar; realizar}
\end{EntryWithPhonetic}

\begin{EntryWithPhonetic}{成本}{cheng2ben3}{6,5}{⼽,⽊}[HSK 5]
  \definition{s.}{custo principal; custo; custo capitalizado; custo final; primeiro custo; custo próprio; custo de produção de um produto; inclui o custo dos materiais de produção consumidos durante o processo produtivo e a remuneração paga aos trabalhadores}
\end{EntryWithPhonetic}

\begin{EntryWithPhonetic}{成才}{cheng2cai2}{6,3}{⼽,⼿}[HSK 7-9]
  \definition{v.}{tornar"-se uma pessoa útil | tornar"-se uma pessoa digna de respeito | fazer algo de si mesmo}
\end{EntryWithPhonetic}

\begin{EntryWithPhonetic}{成都}{cheng2du1}{6,10}{⼽,⾢}
  \definition*{s.}{Chengdu}
\end{EntryWithPhonetic}

\begin{EntryWithPhonetic}{成分}{cheng2fen4}{6,4}{⼽,⼑}[HSK 6]
  \definition[个,些,种]{s.}{composição; ingrediente; elemento; parte componente; as várias substâncias ou fatores que compõem as coisas | a condição de classe de alguém; a profissão ou a condição econômica de alguém; refere"-se à classe à qual uma família pertence; à principal experiência ou ocupação anterior de uma pessoa}
\end{EntryWithPhonetic}

\begin{EntryWithPhonetic}{成功}{cheng2gong1}{6,5}{⼽,⼒}[HSK 3]
  \definition{adj.}{bem-sucedido; frutífero}
  \definition[个,次]{s.}{sucesso}
  \definition{v.}{ter sucesso; obter os resultados esperados}
\end{EntryWithPhonetic}

\begin{EntryWithPhonetic}{成功率}{cheng2gong1lv4}{6,5,11}{⼽,⼒,⽞}
  \definition{s.}{taxa de sucesso}
\end{EntryWithPhonetic}

\begin{EntryWithPhonetic}{成果}{cheng2guo3}{6,8}{⼽,⽊}[HSK 3]
  \definition[个]{s.}{realização; resultado; conquista; recompensas no trabalho ou na carreira}
\end{EntryWithPhonetic}

\begin{EntryWithPhonetic}{成婚}{cheng2hun1}{6,11}{⼽,⼥}
  \definition{v.}{casar"-se}
\end{EntryWithPhonetic}

\begin{EntryWithPhonetic}{成活}{cheng2huo2}{6,9}{⼽,⽔}
  \definition{v.}{sobreviver}
\end{EntryWithPhonetic}

\begin{EntryWithPhonetic}{成吉思汗}{cheng2ji2si1han2}{6,6,9,6}{⼽,⼝,⼼,⽔}
  \definition*{s.}{Genghis Khan (1162-1227), fundador e governante do Império Mongol}
\end{EntryWithPhonetic}

\begin{EntryWithPhonetic}{成绩}{cheng2ji4}{6,11}{⼽,⽷}[HSK 2]
  \definition[项,个]{s.}{realização; sucesso; resultado (de trabalho ou estudo); refere"-se à pontuação obtida em exames e competições; classificação, também se refere aos resultados alcançados no trabalho}
\end{EntryWithPhonetic}

\begin{EntryWithPhonetic}{成家}{cheng2/jia1}{6,10}{⼽,⼧}[HSK 7-9]
  \definition{v.+compl.}{(um homem) casar; (um homem) estabelecer"-se e casar"-se | tornar"-se um especialista (ou \emph{expert}); tornar"-se um especialista reconhecido}
\end{EntryWithPhonetic}

\begin{EntryWithPhonetic}{成交}{cheng2/jiao1}{6,6}{⼽,⼇}[HSK 5]
  \definition{v.+compl.}{fechar um acordo; fazer uma barganha; concluir uma transação}
\end{EntryWithPhonetic}

\begin{EntryWithPhonetic}{成就}{cheng2jiu4}{6,12}{⼽,⼪}[HSK 3]
  \definition[个,项]{s.}{realização; conquista; sucesso; realizações profissionais}
  \definition{v.}{realizar; alcançar; completar; concluir (carreira)}
\end{EntryWithPhonetic}

\begin{EntryWithPhonetic}{成立}{cheng2li4}{6,5}{⼽,⽴}[HSK 3]
  \definition{v.}{fundar; estabelecer; criar; (organizações, instituições, etc.) começar a existir e a funcionar | ser válido; ser sustentável; fazer sentido; (teorias, pontos de vista, razões, etc.) fundamentados e válidos}
\end{EntryWithPhonetic}

\begin{EntryWithPhonetic}{成年}{cheng2nian2}{6,6}{⼽,⼲}[HSK 7-9]
  \definition{adv.}{o ano todo; durante todo o ano}
  \definition{v.}{atingir a maioridade (ser humano, animal, madeira); refere"-se à idade em que uma pessoa atinge a maturidade, ou ao período em que animais superiores ou árvores atingem a maturidade}
\end{EntryWithPhonetic}

\begin{EntryWithPhonetic}{成批}{cheng2pi1}{6,7}{⼽,⼿}
  \definition{s.}{em lotes | a granel}
\end{EntryWithPhonetic}

\begin{EntryWithPhonetic}{成品}{cheng2 pin3}{6,9}{⼽,⼝}[HSK 6]
  \definition[批]{s.}{produto final; produto acabado; produto processado e pronto para ser fornecido}
\end{EntryWithPhonetic}

\begin{EntryWithPhonetic}{成器}{cheng2qi4}{6,16}{⼽,⼝}
  \definition{v.}{tornar"-se uma pessoa digna de respeito | fazer algo de si mesmo}
\end{EntryWithPhonetic}

\begin{EntryWithPhonetic}{成千成万}{cheng2qian1-cheng2wan4}{6,3,6,3}{⼽,⼗,⼽,⼀}
  \definition{expr.}{milhares e dezenas de milhares; milhares e milhares | inumeráveis | Literário: aos milhares e dezenas de milhares; números incontáveis}
  \seealsoref{成千累万}{cheng2qian1-lei3wan4}
  \seealsoref{成千上万}{cheng2qian1-shang4wan4}
\end{EntryWithPhonetic}

\begin{EntryWithPhonetic}{成千累万}{cheng2qian1-lei3wan4}{6,3,11,3}{⼽,⼗,⽷,⼀}
  \definition{expr.}{milhares e dezenas de milhares | milhares e milhares; inumeráveis | Literário: aos milhares e dezenas de milhares; números incontáveis}
  \seealsoref{成千成万}{cheng2qian1-cheng2wan4}
  \seealsoref{成千上万}{cheng2qian1-shang4wan4}
\end{EntryWithPhonetic}

\begin{EntryWithPhonetic}{成千上万}{cheng2qian1-shang4wan4}{6,3,3,3}{⼽,⼗,⼀,⼀}[HSK 7-9]
  \definition{expr.}{aos milhares e dezenas de milhares; números incontáveis; descreve um grande número, também escrito como 成千成万 ou 成千累万 | milhares e dezenas de milhares; milhares e milhares}
  \seealsoref{成千成万}{cheng2qian1-cheng2wan4}
  \seealsoref{成千累万}{cheng2qian1-lei3wan4}
\end{EntryWithPhonetic}

\begin{EntryWithPhonetic}{成群结队}{cheng2qun2-jie2dui4}{6,13,9,4}{⼽,⽺,⽷,⾩}[HSK 7-9]
  \definition{expr.}{em multidões; como uma grande multidão; horda; formar um grupo, constituir uma trupe; em grande número}
\end{EntryWithPhonetic}

\begin{EntryWithPhonetic}{成群结对}{cheng2qun2-jie2dui4}{6,13,9,5}{⼽,⽺,⽷,⼨}
  \definition{expr.}{em grupos}
\end{EntryWithPhonetic}

\begin{EntryWithPhonetic}{成人}{cheng2ren2}{6,2}{⼽,⼈}[HSK 4]
  \definition[个,名,位]{s.}{adulto; crescido; pessoa adulta}
  \definition{v.}{crescer; tornar"-se adulto}
\end{EntryWithPhonetic}

\begin{EntryWithPhonetic}{成色}{cheng2se4}{6,6}{⼽,⾊}
  \definition{v.}{sair"-se bem | ser bem sucedido}
\end{EntryWithPhonetic}

\begin{EntryWithPhonetic}{成熟}{cheng2shu2}{6,15}{⼽,⽕}[HSK 3]
  \definition{adj.}{maduro; amadurecido; totalmente desenvolvido; descreve que as oportunidades, condições, etc. estão perfeitas e que não haverá nenhum problema}
  \definition{v.}{amadurecer; atingir a maturidade; estar totalmente desenvolvido; frutas e outros frutos totalmente maduros, referindo"-se ao desenvolvimento completo de organismos vivos}
\end{EntryWithPhonetic}

\begin{EntryWithPhonetic}{成天}{cheng2tian1}{6,4}{⼽,⼤}[HSK 7-9]
  \definition{adv.}{o dia todo; o tempo todo}
\end{EntryWithPhonetic}

\begin{EntryWithPhonetic}{成为}{cheng2wei2}{6,4}{⼽,⼂}[HSK 2]
  \definition{v.}{tornar"-se; transformar"-se; revelar"-se; passar de uma situação, identidade ou estado para outro}
\end{EntryWithPhonetic}

\begin{EntryWithPhonetic}{成问题}{cheng2wen4ti2}{6,6,15}{⼽,⾨,⾴}[HSK 7-9]
  \definition{v.}{Coloquial: ser um problema; estar aberto a questionamentos (ou dúvidas, objeções)}
\end{EntryWithPhonetic}

\begin{EntryWithPhonetic}{成效}{cheng2xiao4}{6,10}{⼽,⽁}[HSK 5]
  \definition{s.}{efeito; resultado}
\end{EntryWithPhonetic}

\begin{EntryWithPhonetic}{成型}{cheng2xing2}{6,9}{⼽,⼟}[HSK 7-9]
  \definition{v.}{(peças ou produtos) estar em forma acabada; assumir a forma necessária}
\end{EntryWithPhonetic}

\begin{EntryWithPhonetic}{成语}{cheng2yu3}{6,9}{⼽,⾔}[HSK 5]
  \definition[条,则,句,个]{s.}{expressão idiomática; frase de conjunto (frases de quatro caracteres em chinês, geralmente com alusões literárias)}
\end{EntryWithPhonetic}

\begin{EntryWithPhonetic}{成员}{cheng2yuan2}{6,7}{⼽,⼝}[HSK 3]
  \definition[个,些,名,位]{s.}{membro; membros de um grupo ou família}
\end{EntryWithPhonetic}

\begin{EntryWithPhonetic}{成长}{cheng2zhang3}{6,4}{⼽,⾧}[HSK 3]
  \definition{v.}{crescer; amadurecer; tornar"-se adulto; o desenvolvimento de seres humanos, animais ou plantas desde a infância até a maturidade}
\end{EntryWithPhonetic}

%%%%%%%%%% 呈 %%%%%%%%%%
\subsection*{呈}\addcontentsline{loh}{figure}{呈 \dpy{cheng2}}

\begin{EntryWithPhonetic}{呈}{cheng2}{7}{⼝}
  \definition*{s.}{Sobrenome: Cheng}
  \definition{s.}{documento submetido a um superior; petição; memorial}
  \definition{v.}{apresentar; assumir (forma, cor, etc.) | submeter; apresentar; enviar respeitosamente}
\end{EntryWithPhonetic}

\begin{EntryWithPhonetic}{呈现}{cheng2xian4}{7,8}{⼝,⾒}[HSK 7-9]
  \definition{v.}{apresentar (uma certa aparência); aparecer. mostrar uma determinada forma, cor ou tendência, etc., para que as pessoas possam ver}[大海呈现出碧蓝的颜色。===O mar apresenta uma cor azul vibrante.]
\end{EntryWithPhonetic}

%%%%%%%%%% 承 %%%%%%%%%%
\subsection*{承}\addcontentsline{loh}{figure}{承 \dpy{cheng2}}

\begin{EntryWithPhonetic}{承}{cheng2}{8}{⼿}
  \definition*{s.}{Sobrenome: Cheng}
  \definition{v.}{suportar; segurar; carregar; sustentar | empreender; contratar (para fazer um trabalho) | estar em dívida (com alguém por uma gentileza); receber um favor | continuar; prosseguir | receber de cima (instruções, mandato)}
\end{EntryWithPhonetic}

\begin{EntryWithPhonetic}{承办}{cheng2ban4}{8,4}{⼿,⼒}[HSK 5]
  \definition{v.}{ocupar"-se de; encarregar"-se de; (pessoas, organizações, instituições) aceitar (atividades, reuniões, negócios, etc.)}
\end{EntryWithPhonetic}

\begin{EntryWithPhonetic}{承包}{cheng2bao1}{8,5}{⼿,⼓}[HSK 7-9]
  \definition{v.}{contratar (com; para); aceitar projetos ou pedidos em massa, etc. e ser responsável por concluí-los}[他承包了这个工程。===Ele foi contratado para esse projeto.]
\end{EntryWithPhonetic}

\begin{EntryWithPhonetic}{承担}{cheng2dan1}{8,8}{⼿,⼿}[HSK 4]
  \definition{v.}{suportar; empreender; assumir; tomar conta de algo}
\end{EntryWithPhonetic}

\begin{EntryWithPhonetic}{承诺}{cheng2nuo4}{8,10}{⼿,⾔}[HSK 6]
  \definition[个,句,份]{s.}{juramento; promessa; compromisso}
  \definition{v.}{prometer fazer algo; prometer empreender; comprometer"-se a fazer algo}
\end{EntryWithPhonetic}

\begin{EntryWithPhonetic}{承认}{cheng2ren4}{8,4}{⼿,⾔}[HSK 4]
  \definition{s.}{reconhecimento (diplomático, artístico, etc.)}
  \definition{v.}{admitir; reconhecer | dar reconhecimento diplomático; reconhecer}
\end{EntryWithPhonetic}

\begin{EntryWithPhonetic}{承受}{cheng2shou4}{8,8}{⼿,⼜}[HSK 4]
  \definition{v.}{suportar; resistir; realizar (tarefas, dificuldades, pressões, etc.); submeter"-se a (testes, etc.) | herdar}
\end{EntryWithPhonetic}

\begin{EntryWithPhonetic}{承载}{cheng2zai4}{8,10}{⼿,⾞}[HSK 7-9]
  \definition{v.}{suportar o peso; segurar o objeto e suportar seu peso}[桥梁承载着巨大的重量。===A ponte suporta uma carga pesada.]
\end{EntryWithPhonetic}

%%%%%%%%%% 诚 %%%%%%%%%%
\subsection*{诚}\addcontentsline{loh}{figure}{诚 \dpy{cheng2}}

\begin{EntryWithPhonetic}{诚}{cheng2}{8}{⾔}
  \definition{adj.}{sincero; honesto; verdadeiro}
  \definition{adv.}{na verdade; realmente; de fato}
  \definition{s.}{sinceridade; genuinidade; seriedade}
\end{EntryWithPhonetic}

\begin{EntryWithPhonetic}{诚恳}{cheng2ken3}{8,10}{⾔,⼼}[HSK 7-9]
  \definition{adj.}{sincero; sério; a atitude é muito real e pé no chão}
\end{EntryWithPhonetic}

\begin{EntryWithPhonetic}{诚实}{cheng2shi2}{8,8}{⾔,⼧}[HSK 4]
  \definition{adj.}{honesto; sincero e honesto, não hipócrita}
\end{EntryWithPhonetic}

\begin{EntryWithPhonetic}{诚实地}{cheng2shi2 di4}{8,8,6}{⾔,⼧,⼟}
  \definition{adv.}{honestamente}
\end{EntryWithPhonetic}

\begin{EntryWithPhonetic}{诚心诚意}{cheng2xin1-cheng2yi4}{8,4,8,13}{⾔,⼼,⾔,⼼}[HSK 7-9]
  \definition{expr.}{sincero e sério; com toda a sinceridade; é uma expressão idiomática chinesa que vem da Biografia de Ma Yuan no Livro da Dinastia Han Posterior | genuíno; sincero}
\end{EntryWithPhonetic}

\begin{EntryWithPhonetic}{诚信}{cheng2 xin4}{8,9}{⾔,⼈}[HSK 4]
  \definition{adj.}{honesto e confiável}
  \definition[种]{s.}{fé; honestidade; padrão e princípio de comportamento: não contar mentiras, prometer aos outros o que eles podem fazer e ter a confiança dos outros}
\end{EntryWithPhonetic}

\begin{EntryWithPhonetic}{诚意}{cheng2yi4}{8,13}{⾔,⼼}[HSK 7-9]
  \definition{s.}{boa fé; sinceridade; intenções sinceras}
\end{EntryWithPhonetic}

\begin{EntryWithPhonetic}{诚挚}{cheng2zhi4}{8,10}{⾔,⼿}[HSK 7-9]
  \definition{adj.}{sincero; cordial; honesto}
\end{EntryWithPhonetic}

%%%%%%%%%% 城 %%%%%%%%%%
\subsection*{城}\addcontentsline{loh}{figure}{城 \dpy{cheng2}}

\begin{EntryWithPhonetic}{城}{cheng2}{9}{⼟}[HSK 3]
  \definition*{s.}{Sobrenome: Cheng}
  \definition[座,道,个]{s.}{muralha da cidade; muralha | cidade | centro de um determinado tipo (por exemplo, negócios, entretenimento, etc.)}
\end{EntryWithPhonetic}

\begin{EntryWithPhonetic}{城堡}{cheng2bao3}{9,12}{⼟,⼟}
  \definition[座,个]{s.}{forte; castelo; cidadela; uma pequena cidade com muralhas que facilitam a defesa}
\end{EntryWithPhonetic}

\begin{EntryWithPhonetic}{城度}{cheng2du4}{9,9}{⼟,⼴}
  \definition{s.}{na cidade; dentro da cidade; originalmente se referia à área dentro das muralhas da cidade, agora se refere principalmente à área urbana}
\end{EntryWithPhonetic}

\begin{EntryWithPhonetic}{城里}{cheng2 li3}{9,7}{⼟,⾥}[HSK 5]
  \definition{s.}{na cidade; dentro da cidade; originalmente referia"-se à área dentro das muralhas da cidade, agora refere"-se principalmente à área urbana}
\end{EntryWithPhonetic}

\begin{EntryWithPhonetic}{城墙}{cheng2qiang2}{9,14}{⼟,⼟}[HSK 7-9]
  \definition{s.}{muralha (da cidade); as altas e grossas muralhas de proteção que cercam a cidade antiga}
\end{EntryWithPhonetic}

\begin{EntryWithPhonetic}{城区}{cheng2 qu1}{9,4}{⼟,⼖}[HSK 6]
  \definition{s.}{cidade propriamente dita | área metropolitana; área urbana; áreas urbanas e suburbanas}
  \antonymref{郊区}{jiao1 qu1}
\end{EntryWithPhonetic}

\begin{EntryWithPhonetic}{城市}{cheng2shi4}{9,5}{⼟,⼱}[HSK 3]
  \definition[个,座]{s.}{cidade; regiões com alta densidade populacional, comércio e indústria desenvolvidos e cuja população é predominantemente não agrícola são geralmente centros políticos, econômicos e culturais das regiões vizinhas}
\end{EntryWithPhonetic}

\begin{EntryWithPhonetic}{城乡}{cheng2 xiang1}{9,3}{⼟,⼄}[HSK 6]
  \definition{s.}{cidade e campo; áreas urbanas e rurais; a cidade e o campo | cidade e campo; urbano e rural}
\end{EntryWithPhonetic}

\begin{EntryWithPhonetic}{城镇}{cheng2 zhen4}{9,15}{⼟,⾦}[HSK 6]
  \definition[个]{s.}{cidade; cidades e vilas}
\end{EntryWithPhonetic}

%%%%%%%%%% 乘 %%%%%%%%%%
\subsection*{乘}\addcontentsline{loh}{figure}{乘 \dpy{cheng2}}

\begin{EntryWithPhonetic}{乘}{cheng2}{10}{⽲}[HSK 5]
  \definition*{s.}{Sobrenome: Cheng}
  \definition{s.}{uma divisão principal das escolas budistas; uma seita ou doutrina do budismo}
  \definition{v.}{cavalgar; andar a cavalo; utilizar um veículo ou animal em vez de caminhar | aproveitar"-se de; valer"-se de; tirar vantagem de; tirar proveito de | multiplicar; realizar multiplicação | perseguir; caçar}
  \seeref{sheng4}
\end{EntryWithPhonetic}

\begin{EntryWithPhonetic}{乘车}{cheng2 che1}{10,4}{⽲,⾞}[HSK 5]
  \definition{v.}{montar; dirigir; conduzir; andar a cavalo, de moto, de bicicleta, etc.}
\end{EntryWithPhonetic}

\begin{EntryWithPhonetic}{乘积}{cheng2ji1}{10,10}{⽲,⽲}
  \definition{s.}{Matemática: produto (resultado da multiplicação)}
\end{EntryWithPhonetic}

\begin{EntryWithPhonetic}{乘客}{cheng2 ke4}{10,9}{⽲,⼧}[HSK 5]
  \definition[个,位,名]{s.}{passageiro; pessoas viajando de carro, navio ou avião}
\end{EntryWithPhonetic}

\begin{EntryWithPhonetic}{乘客数}{cheng2ke4 shu4}{10,9,13}{⽲,⼧,⽁}
  \definition{s.}{número de passageiros}
\end{EntryWithPhonetic}

\begin{EntryWithPhonetic}{乘人之危}{cheng2ren2zhi1wei1}{10,2,3,6}{⽲,⼈,⼂,⼙}[HSK 7-9]
  \definition{expr.}{tirar vantagem das dificuldades dos outros (posição precária; problema); capitalizar as dificuldades de alguém (desastres); fazer uso (utilizar) da situação precária em que alguém se encontra; tentar usar o dilema de alguém para\dots; aproveitar"-se da angústia dos outros para prejudicá"-los; atacar em um momento de crise}
\end{EntryWithPhonetic}

\begin{EntryWithPhonetic}{乘坐}{cheng2zuo4}{10,7}{⽲,⼟}[HSK 5]
  \definition{v.}{pegar (um trem, ônibus, etc.); andar de (bicicleta, moto, etc.)}
\end{EntryWithPhonetic}

%%%%%%%%%% 盛 %%%%%%%%%%
\subsection*{盛}\addcontentsline{loh}{figure}{盛 \dpy{cheng2}}

\begin{EntryWithPhonetic}{盛}{cheng2}{11}{⽫}[HSK 7-9]
  \definition{v.}{encher; encher com uma concha; colocar as coisas em recipientes; especialmente colocar alimentos em tigelas, pratos e outros recipientes | segurar; conter; acomodar}
  \seeref{sheng4}
\end{EntryWithPhonetic}

%%%%%%%%%% 惩 %%%%%%%%%%
\subsection*{惩}\addcontentsline{loh}{figure}{惩 \dpy{cheng2}}

\begin{EntryWithPhonetic}{惩}{cheng2}{12}{⼼}
  \definition{v.}{receber ou dar aviso | punir; penalizar}
\end{EntryWithPhonetic}

\begin{EntryWithPhonetic}{惩处}{cheng2chu3}{12,5}{⼼,⼡}[HSK 7-9]
  \definition{v.}{penalizar; punir | punir; administrar justiça}
\end{EntryWithPhonetic}

\begin{EntryWithPhonetic}{惩罚}{cheng2fa2}{12,9}{⼼,⽹}[HSK 7-9]
  \definition[次,种]{s.}{punição; o ato ou método de punição}
  \definition{v.}{punir (severamente); penalizar}
\end{EntryWithPhonetic}

%%%%%%%%%% 程 %%%%%%%%%%
\subsection*{程}\addcontentsline{loh}{figure}{程 \dpy{cheng2}}

\begin{EntryWithPhonetic}{程}{cheng2}{12}{⽲}
  \definition{s.}{regra; regulamento; lei | ordem; procedimento | jornada; etapa de uma jornada; estrada; um trecho de estrada | distância percorrida ou movida por um objeto | programação | medição; termo geral para pesos e medidas}
\end{EntryWithPhonetic}

\begin{EntryWithPhonetic}{程度}{cheng2du4}{12,9}{⽲,⼴}[HSK 3]
  \definition[种]{s.}{nível; grau (de cultura, educação, aprendizagem, etc.) | extensão; grau; a situação, o nível ou o estágio em que as coisas mudam}
\end{EntryWithPhonetic}

\begin{EntryWithPhonetic}{程控}{cheng2kong4}{12,11}{⽲,⼿}
  \definition{s.}{programado | sob controle automático}
\end{EntryWithPhonetic}

\begin{EntryWithPhonetic}{程序}{cheng2xu4}{12,7}{⽲,⼴}[HSK 4]
  \definition[个,套,种]{s.}{ordem; curso; sequência; procedimento; ordem em que algo é feito; também, um determinado número de etapas em um trabalho | programa; conjunto de instruções de computador projetado em sequência para permitir que um computador execute uma ou mais operações}
\end{EntryWithPhonetic}

\begin{EntryWithPhonetic}{程序库}{cheng2xu4ku4}{12,7,7}{⽲,⼴,⼴}
  \definition{s.}{biblioteca de funções e procedimentos para programas de computador}
\end{EntryWithPhonetic}

\begin{EntryWithPhonetic}{程序设计}{cheng2xu4she4ji4}{12,7,6,4}{⽲,⼴,⾔,⾔}
  \definition{s.}{programação de computadores}
\end{EntryWithPhonetic}

%%%%%%%%%% 澄 %%%%%%%%%%
\subsection*{澄}\addcontentsline{loh}{figure}{澄 \dpy{cheng2}}

\begin{EntryWithPhonetic}{澄}{cheng2}{15}{⽔}
  \definition*{s.}{Sobrenome: Cheng}
  \definition{adj.}{claro; transparente}
  \definition{v.}{esclarecer; purificar}
  \seeref{deng4}
\end{EntryWithPhonetic}

\begin{EntryWithPhonetic}{澄清}{cheng2qing1}{15,11}{⽔,⽔}[HSK 7-9]
  \definition{adj.}{claro; transparente}
  \definition{v.}{esclarecer; deixar claro; entender | purificar; limpar; esclarecer a turbidez, uma metáfora para esclarecer uma situação caótica}
\end{EntryWithPhonetic}

%%%%%%%%%% 橙 %%%%%%%%%%
\subsection*{橙}\addcontentsline{loh}{figure}{橙 \dpy{cheng2}}

\begin{EntryWithPhonetic}{橙}{cheng2}{16}{⽊}
  \definition{s.}{laranja; fruta da laranjeira | laranjeira; pé de laranja | cor laranja}
\end{EntryWithPhonetic}

\begin{EntryWithPhonetic}{橙色}{cheng2 se4}{16,6}{⽊,⾊}
  \definition{s.}{cor de laranja}
\end{EntryWithPhonetic}

\begin{EntryWithPhonetic}{橙汁}{cheng2zhi1}{16,5}{⽊,⽔}[HSK 7-9]
  \definition[瓶,杯,罐,盒]{s.}{laranjada; suco de laranja}
  \seealsoref{橘子汁}{ju2zi5zhi1}
  \seealsoref{柳橙汁}{liu3cheng2zhi1}
\end{EntryWithPhonetic}

%%%%%%%%%% 逞 %%%%%%%%%%
\subsection*{逞}\addcontentsline{loh}{figure}{逞 \dpy{cheng3}}

\begin{EntryWithPhonetic}{逞}{cheng3}{10}{⾡}
  \definition*{s.}{Sobrenome: Cheng}
  \definition{v.}{exibir"-se; ostentar; gabar"-se | executar (um plano maligno); ter sucesso (em um esquema) | saciar; satisfazer; dar rédea solta a; deliciar"-se}
\end{EntryWithPhonetic}

\begin{EntryWithPhonetic}{逞能}{cheng3/neng2}{10,10}{⾡,⾁}[HSK 7-9]
  \definition{v.+compl.}{exibir a própria habilidade (ou capacidade); exibir a própria capacidade | mostrar sua habilidade ou capacidade}
\end{EntryWithPhonetic}

\begin{EntryWithPhonetic}{逞强}{cheng3/qiang2}{10,12}{⾡,⼸}[HSK 7-9]
  \definition{v.+compl.}{exibir"-se; ser orgulhoso; ser teimoso; ostentar a própria superioridade}
\end{EntryWithPhonetic}

%%%%%%%%%% 秤 %%%%%%%%%%
\subsection*{秤}\addcontentsline{loh}{figure}{秤 \dpy{cheng4}}

\begin{EntryWithPhonetic}{秤}{cheng4}{10}{⽲}[HSK 7-9]
  \definition[把,杆,台]{s.}{balança; balança romana; um instrumento para medir o peso de um objeto}
\end{EntryWithPhonetic}

%%%%%%%%%% 吃 %%%%%%%%%%
\subsection*{吃}\addcontentsline{loh}{figure}{吃 \dpy{chi1}}

\begin{EntryWithPhonetic}{吃}{chi1}{6}{⼝}[HSK 1]
  \definition{s.}{alimentos; necessidades básicas}
  \definition{v.}{comer; pegar; fazer; colocar alimentos na boca, mastigar e engolir (incluindo sugar e beber) | viver; depender de algo para viver | aniquilar; eliminar (usado principalmente em jogos de guerra e jogos de tabuleiro) | esgotar; exaurir; ser um fardo; ser um esforço | absorver | sofrer; incorrer | entender; compreender | entrar um objeto em outro | expressar aceitação psicológica | fazer suas refeições; comer}
\end{EntryWithPhonetic}

\begin{EntryWithPhonetic}{吃不上}{chi1bu5shang4}{6,4,3}{⼝,⼀,⼀}[HSK 7-9]
  \definition{v.}{incapaz de comer alguma coisa | pular uma refeição; perder a chance de comer | não conseguir comer alguma coisa; não ter comida para comer}
\end{EntryWithPhonetic}

\begin{EntryWithPhonetic}{吃饭}{chi1/fan4}{6,7}{⼝,⾷}[HSK 1]
  \definition{v.+compl.}{comer; ter (comer) uma refeição | manter"-se vivo;  ganhar a vida; refere"-se à vida ou à sobrevivência em geral}
\end{EntryWithPhonetic}

\begin{EntryWithPhonetic}{吃喝玩乐}{chi1-he1-wan2-le4}{6,12,8,5}{⼝,⼝,⽟,⼃}[HSK 7-9]
  \definition{expr.}{comer, beber e se divertir, passar o tempo com prazer | abandonar"-se a uma vida de prazer}
\end{EntryWithPhonetic}

\begin{EntryWithPhonetic}{吃惊}{chi1/jing1}{6,11}{⼝,⼼}[HSK 4]
  \definition{v.+compl.}{ficar assustado; ficar chocado; ficar espantado; pegar de surpresa; ficar assustado inesperadamente}
\end{EntryWithPhonetic}

\begin{EntryWithPhonetic}{吃苦}{chi1/ku3}{6,8}{⼝,⾋}[HSK 7-9]
  \definition{v.+compl.}{suportar dificuldades; sofrer}[他在工作中吃了很多苦。===Ele sofreu muito em seu trabalho.]
\end{EntryWithPhonetic}

\begin{EntryWithPhonetic}{吃亏}{chi1/kui1}{6,3}{⼝,⼆}[HSK 7-9]
  \definition{adv.}{em desvantagem; em situação desfavorável}
  \definition{v.+compl.}{sofrer perdas; sofrer aflição; levar a pior; levar uma surra}
\end{EntryWithPhonetic}

\begin{EntryWithPhonetic}{吃力}{chi1li4}{6,2}{⼝,⼒}[HSK 5]
  \definition{adj.}{suado; extenuante; trabalhoso; laborioso | cansado; fatigado}
\end{EntryWithPhonetic}

\begin{EntryWithPhonetic}{吃屎}{chi1 shi3}{6,9}{⼝,⼫}
  \definition{expr.}{Coma merda!}
\end{EntryWithPhonetic}

%%%%%%%%%% 痴 %%%%%%%%%%
\subsection*{痴}\addcontentsline{loh}{figure}{痴 \dpy{chi1}}

\begin{EntryWithPhonetic}{痴}{chi1}{13}{⽧}
  \definition{adj.}{bobo; idiota; estúpido | louco por alguém (ou algo); extremamente obcecado por alguém ou alguma coisa | Dialeto: louco; insano; mentalmente perturbado}
\end{EntryWithPhonetic}

\begin{EntryWithPhonetic}{痴呆}{chi1dai1}{13,7}{⽧,⼝}[HSK 7-9]
  \definition{adj.}{estúpido; tolo; expressão ou comportamento enfadonho}
  \definition{s.}{demência}
\end{EntryWithPhonetic}

\begin{EntryWithPhonetic}{痴迷}{chi1mi2}{13,9}{⽧,⾡}[HSK 7-9]
  \definition{v.}{ser ou estar apaixonado; ser ou estar obcecado; ser ou estar louco por}
\end{EntryWithPhonetic}

\begin{EntryWithPhonetic}{痴心}{chi1xin1}{13,4}{⽧,⼼}[HSK 7-9]
  \definition{adj.}{apaixonado; obcecado por alguém ou alguma coisa}
  \definition{s.}{desejo tolo; amor cego; paixão cega}
\end{EntryWithPhonetic}

%%%%%%%%%% 池 %%%%%%%%%%
\subsection*{池}\addcontentsline{loh}{figure}{池 \dpy{chi2}}

\begin{EntryWithPhonetic}{池}{chi2}{6}{⽔}
  \definition*{s.}{Sobrenome: Chi}
  \definition[个,片]{s.}{piscina; lagoa | qualquer espaço fechado com laterais elevadas | baias (em um teatro); a parte frontal do salão principal do teatro | fosso}
\end{EntryWithPhonetic}

\begin{EntryWithPhonetic}{池塘}{chi2tang2}{6,13}{⽔,⼟}[HSK 7-9]
  \definition[个]{s.}{lagoa; açude; grande poço de armazenamento de água | piscina comum (em um balneário)}
\end{EntryWithPhonetic}

\begin{EntryWithPhonetic}{池子}{chi2 zi5}{6,3}{⽔,⼦}[HSK 5]
  \definition{s.}{lago; lagoa; viveiro | piscina; piscina do balneário | (antigo) arquibancada (primeiras fileiras em um teatro) | pista de dança de um salão de baile}
\end{EntryWithPhonetic}

%%%%%%%%%% 驰 %%%%%%%%%%
\subsection*{驰}\addcontentsline{loh}{figure}{驰 \dpy{chi2}}

\begin{EntryWithPhonetic}{驰}{chi2}{6}{⾺}
  \definition*{s.}{Sobrenome: Chi}
  \definition{v.}{(veículos, carruagens, cavalos, etc.) acelerar; galopar; fazer correr muito rápido | espalhar | (pensamentos) voltar"-se ansiosamente para; vire"-se ansiosamente para}
\end{EntryWithPhonetic}

\begin{EntryWithPhonetic}{驰名}{chi2ming2}{6,6}{⾺,⼝}[HSK 7-9]
  \definition{s.}{famoso; bem conhecido; renomado}
  \definition{v.}{tornar"-se famoso (celebrado): conhecido em toda parte}
\end{EntryWithPhonetic}

%%%%%%%%%% 迟 %%%%%%%%%%
\subsection*{迟}\addcontentsline{loh}{figure}{迟 \dpy{chi2}}

\begin{EntryWithPhonetic}{迟}{chi2}{7}{⾡}[HSK 5]
  \definition*{s.}{Sobrenome: Chi}
  \definition{adj.}{lento; tardio; demorado | atrasado | lento; obtuso}
\end{EntryWithPhonetic}

\begin{EntryWithPhonetic}{迟迟}{chi2chi2}{7,7}{⾡,⾡}[HSK 7-9]
  \definition{adv.}{lentamente; tardiamente}
\end{EntryWithPhonetic}

\begin{EntryWithPhonetic}{迟到}{chi2dao4}{7,8}{⾡,⼑}[HSK 4]
  \definition{v.}{chegar atrasado; atrasar"-se; chegar depois do horário estipulado, geralmente usado para aulas, reuniões ou encontros em horário combinado, etc.}
\end{EntryWithPhonetic}

\begin{EntryWithPhonetic}{迟疑}{chi2yi2}{7,14}{⾡,⽦}[HSK 7-9]
  \definition{v.}{hesitar; titubear}
\end{EntryWithPhonetic}

\begin{EntryWithPhonetic}{迟早}{chi2zao3}{7,6}{⾡,⽇}[HSK 7-9]
  \definition{adv.}{mais cedo ou mais tarde; cedo ou tarde}[我们迟早会成功的。===Teremos sucesso mais cedo ou mais tarde.]
\end{EntryWithPhonetic}

%%%%%%%%%% 持 %%%%%%%%%%
\subsection*{持}\addcontentsline{loh}{figure}{持 \dpy{chi2}}

\begin{EntryWithPhonetic}{持}{chi2}{9}{⼿}[HSK 7-9]
  \definition{v.}{segurar; agarrar | opor; confrontar | apoiar; manter | gerenciar; supervisionar | sequestrar; agarrar (controlar; forçar)}
\end{EntryWithPhonetic}

\begin{EntryWithPhonetic}{持久}{chi2jiu3}{9,3}{⼿,⼃}[HSK 7-9]
  \definition{adj.}{duradouro; prolongado; persistente; permanente}
\end{EntryWithPhonetic}

\begin{EntryWithPhonetic}{持续}{chi2xu4}{9,11}{⼿,⽷}[HSK 3]
  \definition{v.}{durar; continuar; sustentar; manter a situação ou as condições como estão, sem alterações}
\end{EntryWithPhonetic}

\begin{EntryWithPhonetic}{持有}{chi2 you3}{9,6}{⼿,⽉}[HSK 6]
  \definition{v.}{segurar; possuir | segurar; ter; abrigar; ter em mente (ideias, opiniões, etc.)}
\end{EntryWithPhonetic}

\begin{EntryWithPhonetic}{持之以恒}{chi2zhi1-yi3heng2}{9,3,4,9}{⼿,⼂,⼈,⼼}[HSK 7-9]
  \definition{expr.}{perseguir incessantemente | de forma persistente | perseverar em (fazer algo)}
\end{EntryWithPhonetic}

%%%%%%%%%% 尺 %%%%%%%%%%
\subsection*{尺}\addcontentsline{loh}{figure}{尺 \dpy{chi3}}

\begin{EntryWithPhonetic}{尺}{chi3}{4}{⼫}[HSK 4]
  \definition{clas.}{chi, uma unidade de comprimento (=13 metros)}
  \definition[支,把]{s.}{régua; instrumentos de medição | um instrumento no formato de uma régua}
  \seeref{che3}
\end{EntryWithPhonetic}

\begin{EntryWithPhonetic}{尺寸}{chi3 cun4}{4,3}{⼫,⼨}[HSK 4]
  \definition{s.}{tamanho; medida; dimensão}
\end{EntryWithPhonetic}

\begin{EntryWithPhonetic}{尺度}{chi3du4}{4,9}{⼫,⼴}[HSK 7-9]
  \definition{s.}{padrão; critério; medida}
\end{EntryWithPhonetic}

\begin{EntryWithPhonetic}{尺子}{chi3zi5}{4,3}{⼫,⼦}[HSK 4]
  \definition[把,根]{s.}{régua de madeira ou metal para orientar a caneta ou o lápis para desenhar linhas ou fazer medições}
\end{EntryWithPhonetic}

%%%%%%%%%% 齿 %%%%%%%%%%
\subsection*{齿}\addcontentsline{loh}{figure}{齿 \dpy{chi3}}

\begin{EntryWithPhonetic}{齿}{chi3}{8}{⿒}[Kangxi 211]
  \definition[颗]{s.}{dente | uma parte de qualquer coisa semelhante a um dente; parte dentada de um objeto | idade (de uma pessoa); faixa etária}
  \definition{v.}{mencionar; falar de}
\end{EntryWithPhonetic}

\begin{EntryWithPhonetic}{齿儿}{chi3r5}{8,2}{⿒,⼉}
  \definition{s.}{dentes}
\end{EntryWithPhonetic}

%%%%%%%%%% 耻 %%%%%%%%%%
\subsection*{耻}\addcontentsline{loh}{figure}{耻 \dpy{chi3}}

\begin{EntryWithPhonetic}{耻}{chi3}{10}{⽿}
  \definition{s.}{vergonha; desgraça; humilhação}
  \definition{v.}{estar envergonhado de; considerar vergonhoso}
\end{EntryWithPhonetic}

\begin{EntryWithPhonetic}{耻辱}{chi3ru3}{10,10}{⽿,⾠}[HSK 7-9]
  \definition{s.}{vergonha; desgraça; humilhação; danos à reputação; incidente vergonhoso}
\end{EntryWithPhonetic}

\begin{EntryWithPhonetic}{耻笑}{chi3xiao4}{10,10}{⽿,⽵}[HSK 7-9]
  \definition{v.}{ridicularizar alguém; zombar; zombar de; rir de}
\end{EntryWithPhonetic}

%%%%%%%%%% 斥 %%%%%%%%%%
\subsection*{斥}\addcontentsline{loh}{figure}{斥 \dpy{chi4}}

\begin{EntryWithPhonetic}{斥}{chi4}{5}{⽄}
  \definition*{s.}{Sobrenome: Chi}
  \definition{adj.}{(solo) salino; alcalino}
  \definition{s.}{terra impregnada de sal, portanto estéril}
  \definition{v.}{repreender; censurar; denunciar; reprimir | repelir; excluir; expulsar | fornecer; prover | Literário: abrir; expandir | culpar; reprovar | estender; ampliar | Obsoleto: reconhecer; detectar}
\end{EntryWithPhonetic}

\begin{EntryWithPhonetic}{斥骂}{chi4ma4}{5,9}{⽄,⾺}
  \definition{v.}{repreender}
\end{EntryWithPhonetic}

%%%%%%%%%% 赤 %%%%%%%%%%
\subsection*{赤}\addcontentsline{loh}{figure}{赤 \dpy{chi4}}

\begin{EntryWithPhonetic}{赤}{chi4}{7}{⾚}[Kangxi 155]
  \definition*{s.}{Sobrenome: Chi}
  \definition{adj.}{vermelho | um tipo de vermelho um pouco mais claro que o vermelhão | (história)  revolucionário; comunista | leal; sincero | nú; exposto}
  \definition{s.}{ouro puro}
\end{EntryWithPhonetic}

\begin{EntryWithPhonetic}{赤字}{chi4zi4}{7,6}{⾚,⼦}[HSK 7-9]
  \definition{s.}{déficit; refere"-se à diferença entre despesas e receitas nas atividades econômicas}
\end{EntryWithPhonetic}

%%%%%%%%%% 翅 %%%%%%%%%%
\subsection*{翅}\addcontentsline{loh}{figure}{翅 \dpy{chi4}}

\begin{EntryWithPhonetic}{翅}{chi4}{10}{⽻}
  \definition[只]{s.}{asa | barbatana de tubarão | coisa parecida com uma asa}
\end{EntryWithPhonetic}

\begin{EntryWithPhonetic}{翅膀}{chi4bang3}{10,14}{⽻,⾁}[HSK 7-9]
  \definition[只,个,对]{s.}{asa; os órgãos de voo de animais como pássaros e insetos geralmente aparecem em pares | barbatana; aba; lâmina; a parte de algo que tem o formato ou age como uma asa}
\end{EntryWithPhonetic}

%%%%%%%%%% 充 %%%%%%%%%%
\subsection*{充}\addcontentsline{loh}{figure}{充 \dpy{chong1}}

\begin{EntryWithPhonetic}{充}{chong1}{6}{⼉}[HSK 7-9]
  \definition*{s.}{Sobrenome: Chong}
  \definition{adj.}{suficiente; completo; amplo; cheio}
  \definition{v.}{encher; carregar; atulhar | servir como; agir como | fingir ser; posar como; passar algo como}
\end{EntryWithPhonetic}

\begin{EntryWithPhonetic}{充当}{chong1dang1}{6,6}{⼉,⼹}[HSK 7-9]
  \definition{v.}{agir como; servir como; desempenhar o papel de; assumir o comando de}
\end{EntryWithPhonetic}

\begin{EntryWithPhonetic}{充电}{chong1 dian4}{6,5}{⼉,⽥}[HSK 4]
  \definition{v.}{carregar (uma bateria); conectar uma fonte de alimentação CC aos terminais da bateria para recarregar a bateria | relaxar; passar o tempo livre; ``recarregar as baterias''; estudar para adquirir mais conhecimento; reabastecer (ou ampliar) o conhecimento; metaforicamente falando, para reabastecer a força física e a energia por meio do descanso e da recreação; também metaforicamente falando, para reabastecer novos conhecimentos e desenvolver novas habilidades por meio do reaprendizado}
\end{EntryWithPhonetic}

\begin{EntryWithPhonetic}{充电器}{chong1dian4qi4}{6,5,16}{⼉,⽥,⼝}[HSK 4]
  \definition[个,台]{s.}{carregador de bateria; dispositivo para alimentar uma bateria com energia, forçando uma corrente através dela}
\end{EntryWithPhonetic}

\begin{EntryWithPhonetic}{充分}{chong1fen4}{6,4}{⼉,⼑}[HSK 4]
  \definition{adj.}{cheio; amplo; abundante; suficiente; adequado}
  \definition{adv.}{totalmente; até o fim}
\end{EntryWithPhonetic}

\begin{EntryWithPhonetic}{充满}{chong1man3}{6,13}{⼉,⽔}[HSK 3]
  \definition{v.}{preencher; encher; cobrir completamente | estar cheio de; estar repleto de; estar transbordando de; estar impregnado de}
\end{EntryWithPhonetic}

\begin{EntryWithPhonetic}{充沛}{chong1pei4}{6,7}{⼉,⽔}[HSK 7-9]
  \definition{adj.}{abundante; cheio de}
\end{EntryWithPhonetic}

\begin{EntryWithPhonetic}{充实}{chong1shi2}{6,8}{⼉,⼧}[HSK 7-9]
  \definition{adj.}{rico; cheio; substancial; gratificante}
  \definition{v.}{enriquecer; aumentar; substanciar (um argumento)}
\end{EntryWithPhonetic}

\begin{EntryWithPhonetic}{充足}{chong1zu2}{6,7}{⼉,⾜}[HSK 5]
  \definition{adj.}{bastante; adequado; suficiente; mais do que suficiente para atender às necessidades (usado principalmente para coisas mais específicas)}
\end{EntryWithPhonetic}

%%%%%%%%%% 冲 %%%%%%%%%%
\subsection*{冲}\addcontentsline{loh}{figure}{冲 \dpy{chong1}}

\begin{EntryWithPhonetic}{冲}{chong1}{6}{⼎}[HSK 4,6]
  \definition{s.}{via pública; local importante; via de passagem; via local importante | um trecho de planície em uma área montanhosa | (astronomia) oposição; os planetas externos orbitam até ficarem alinhados com a Terra e o Sol, e a Terra está no meio}
  \definition{v.}{atacar; apressar; correr; passar rapidamente; passar por um obstáculo | colidir; chocar; bater | despejar água fervente sobre | enxaguar; dar descarga; lavar | revelar (filme) | neutralizar a má sorte}
  \seeref{chong4}
\end{EntryWithPhonetic}

\begin{EntryWithPhonetic}{冲刺}{chong1ci4}{6,8}{⼎,⼑}[HSK 7-9]
  \definition{v.}{arrancar; correr; disparar | metáfora para fazer o maior esforço ao se aproximar de uma meta ou estar prestes a ter sucesso}
\end{EntryWithPhonetic}

\begin{EntryWithPhonetic}{冲动}{chong1dong4}{6,6}{⼎,⼒}[HSK 5]
  \definition{adj.}{impulsivo; impetuoso}
  \definition{s.}{impulso; impetuosidade; impulso de movimento; fenômeno psicológico no qual as emoções são particularmente fortes e o controle racional é fraco}
  \definition{v.}{ficar animado; ser impetuoso; agir por impulso}
\end{EntryWithPhonetic}

\begin{EntryWithPhonetic}{冲锋}{chong1feng1}{6,12}{⼎,⾦}
  \definition{v.}{cobrar | tomar de assalto}
\end{EntryWithPhonetic}

\begin{EntryWithPhonetic}{冲击}{chong1ji1}{6,5}{⼎,⼐}[HSK 6]
  \definition{v.}{chicotear; bater | correr; voar; atacar; assaltar; atacar bravamente em direção a um alvo predeterminado | chocar; metáfora para interferência ou golpe sério}
\end{EntryWithPhonetic}

\begin{EntryWithPhonetic}{冲浪}{chong1lang4}{6,10}{⼎,⽔}[HSK 7-9]
  \definition{v.}{surfar; um esporte aquático em que os atletas surfam em pranchas especialmente construídas e deslizam ao longo das ondas | metáfora para navegar na \emph{Internet}}
\end{EntryWithPhonetic}

\begin{EntryWithPhonetic}{冲突}{chong1tu1}{6,9}{⼎,⽳}[HSK 5]
  \definition{v.}{chocar"-se; entrar em conflito; conflitar | contradizer; duas coisas opostas que interferem uma na outra}
\end{EntryWithPhonetic}

\begin{EntryWithPhonetic}{冲洗}{chong1xi3}{6,9}{⼎,⽔}[HSK 7-9]
  \definition{v.}{lavar; enxaguar | revelar filme; revelar e fixar o material fotossensível exposto}
\end{EntryWithPhonetic}

\begin{EntryWithPhonetic}{冲撞}{chong1zhuang4}{6,15}{⼎,⼿}[HSK 7-9]
  \definition{v.}{colidir; bater; sofrer impacto; voar; errar contra | ofender; ofender}
\end{EntryWithPhonetic}

%%%%%%%%%% 憧 %%%%%%%%%%
\subsection*{憧}\addcontentsline{loh}{figure}{憧 \dpy{chong1}}

\begin{EntryWithPhonetic}{憧}{chong1}{15}{⼼}
  \definition{adj.}{irresoluto; indeciso | estúpido; imbecil; confuso}
\end{EntryWithPhonetic}

\begin{EntryWithPhonetic}{憧憬}{chong1jing3}{15,15}{⼼,⼼}
  \definition{v.}{ansiar por | esperar por}
\end{EntryWithPhonetic}

%%%%%%%%%% 虫 %%%%%%%%%%
\subsection*{虫}\addcontentsline{loh}{figure}{虫 \dpy{chong2}}

\begin{EntryWithPhonetic}{虫}{chong2}{6}{⾍}[Kangxi 142]
  \definition[只,条]{s.}{inseto; verme | (pejorativo) pessoas que se comportam de forma desprezível | fã; viciado | forma inferior de vida animal, incluindo insetos, larvas de insetos, vermes e criaturas semelhantes | pessoa com uma característica indesejável específica}
\end{EntryWithPhonetic}

\begin{EntryWithPhonetic}{虫子}{chong2 zi5}{6,3}{⾍,⼦}[HSK 4]
  \definition[条,只,种]{s.}{percevejo; besouro; inseto; verme; criaturas semelhantes a insetos}
\end{EntryWithPhonetic}

%%%%%%%%%% 重 %%%%%%%%%%
\subsection*{重}\addcontentsline{loh}{figure}{重 \dpy{chong2}}

\begin{EntryWithPhonetic}{重}{chong2}{9}{⾥}
  \definition*{s.}{Sobrenome: Chong}
  \definition{adv.}{novamente; mais uma vez}
  \definition{clas.}{usado para camadas}
  \definition{v.}{repetir; duplicar}
  \seeref{zhong4}
\end{EntryWithPhonetic}

\begin{EntryWithPhonetic}{重播}{chong2bo1}{9,15}{⾥,⼿}[HSK 7-9]
  \definition{v.}{retransmitir (um programa de rádio ou TV) | Agricultura: ressemear (o mesmo campo)}
\end{EntryWithPhonetic}

\begin{EntryWithPhonetic}{重重}{chong2chong2}{9,9}{⾥,⾥}
  \definition{adv.}{camada após camada | um após o outro}
  \seeref{zhong4zhong4}
\end{EntryWithPhonetic}

\begin{EntryWithPhonetic}{重点}{chong2dian3}{9,9}{⾥,⽕}
  \definition[个]{adj./adv./s.}{nota principal; ponto-chave; ponto focal; ênfase}
  \seeref{zhong4dian3}
\end{EntryWithPhonetic}

\begin{EntryWithPhonetic}{重叠}{chong2die2}{9,13}{⾥,⼜}[HSK 7-9]
  \definition{s.}{reduplicação; gramaticalmente, refere"-se ao uso da mesma palavra ou frase duas vezes}
  \definition{v.}{sobrepor; por um em cima do outro; colocar as mesmas coisas juntas camada por camada}
\end{EntryWithPhonetic}

\begin{EntryWithPhonetic}{重返}{chong2fan3}{9,7}{⾥,⾡}[HSK 7-9]
  \definition{v.}{retornar; voltar; de volta ao lugar original}
\end{EntryWithPhonetic}

\begin{EntryWithPhonetic}{重逢}{chong2feng2}{9,10}{⾥,⾡}
  \definition{v.}{reunir; encontrar"-se novamente; reencontrar"-se após uma longa separação; rever"-se}
\end{EntryWithPhonetic}

\begin{EntryWithPhonetic}{重复}{chong2fu4}{9,9}{⾥,⼢}[HSK 2]
  \definition{v.}{repetir; iterar; duplicar; reduplicar | fazer algo novamente; repetir as mesmas palavras, fazer as mesmas coisas}
\end{EntryWithPhonetic}

\begin{EntryWithPhonetic}{重合}{chong2he2}{9,6}{⾥,⼝}[HSK 7-9]
  \definition{s.}{coincidir; sobrepor"-se}[两个假期的时间重合了。===Os dois feriados se sobrepõem.]
\end{EntryWithPhonetic}

\begin{EntryWithPhonetic}{重建}{chong2 jian4}{9,8}{⾥,⼵}[HSK 6]
  \definition{s.}{restabelecimento; reconstrução}
  \definition{v.}{reconstruir; reconstruir; restabelecer; reabilitar}
\end{EntryWithPhonetic}

\begin{EntryWithPhonetic}{重申}{chong2shen1}{9,5}{⾥,⽥}[HSK 7-9]
  \definition{v.}{reafirmar; reiterar}[他重申了自己对古巴的承诺。===Ele reiterou seu compromisso com Cuba.]
\end{EntryWithPhonetic}

\begin{EntryWithPhonetic}{重现}{chong2xian4}{9,8}{⾥,⾒}[HSK 7-9]
  \definition{v.}{reaparecer | reproduzir}
\end{EntryWithPhonetic}

\begin{EntryWithPhonetic}{重新}{chong2xin1}{9,13}{⾥,⽄}[HSK 2]
  \definition{adv.}{novamente; de novo; significa repetir uma ação ou comportamento já realizado | indica que se deve começar do início (mudança de método ou conteúdo)}
\end{EntryWithPhonetic}

\begin{EntryWithPhonetic}{重阳节}{chong2yang2jie2}{9,6,5}{⾥,⾩,⾋}
  \definition*{s.}{Festa do Duplo Nove, Festival Yang, dia de subir aos lugares mais altos para evitar calamidades e dia do culto aos antepassados (9º dia do nono mês lunar)}
\end{EntryWithPhonetic}

\begin{EntryWithPhonetic}{重组}{chong2 zu3}{9,8}{⾥,⽷}[HSK 6]
  \definition{v.}{reestruturar; reorganizar; remanejar}
\end{EntryWithPhonetic}

%%%%%%%%%% 崇 %%%%%%%%%%
\subsection*{崇}\addcontentsline{loh}{figure}{崇 \dpy{chong2}}

\begin{EntryWithPhonetic}{崇}{chong2}{11}{⼭}
  \definition*{s.}{Sobrenome: Chong}
  \definition{adj.}{alto; elevado; sublime}
  \definition{v.}{adorar; reverenciar; venerar; estimar | respeitar}
\end{EntryWithPhonetic}

\begin{EntryWithPhonetic}{崇拜}{chong2bai4}{11,9}{⼭,⼿}[HSK 6]
  \definition{v.}{adorar; idolatrar; venerar}
\end{EntryWithPhonetic}

\begin{EntryWithPhonetic}{崇高}{chong2gao1}{11,10}{⼭,⾼}[HSK 7-9]
  \definition{adj.}{alto; elevado; sublime; nobre}
\end{EntryWithPhonetic}

\begin{EntryWithPhonetic}{崇尚}{chong2shang4}{11,8}{⼭,⼩}[HSK 7-9]
  \definition{v.}{sustentar; defender; valorizar}[我们崇尚公平与正义。===Nós defendemos a justiça e a equidade.]
\end{EntryWithPhonetic}

%%%%%%%%%% 宠 %%%%%%%%%%
\subsection*{宠}\addcontentsline{loh}{figure}{宠 \dpy{chong3}}

\begin{EntryWithPhonetic}{宠}{chong3}{8}{⼧}[HSK 7-9]
  \definition*{s.}{Sobrenome: Chong}
  \definition{v.}{mimar; estragar; conceder favor a | regalar; encontrar favor com alguém; estar nas boas graças de alguém}
\end{EntryWithPhonetic}

\begin{EntryWithPhonetic}{宠爱}{chong3'ai4}{8,10}{⼧,⽖}[HSK 7-9]
  \definition{v.}{mimar; fazer de alguém um animal de estimação}[爷爷总是宠爱他的孙子。===O avô sempre mima o neto.]
\end{EntryWithPhonetic}

\begin{EntryWithPhonetic}{宠物}{chong3wu4}{8,8}{⼧,⽜}[HSK 6]
  \definition[只]{s.}{animal de estimação; refere"-se a pequenos animais criados na família}
\end{EntryWithPhonetic}

%%%%%%%%%% 冲 %%%%%%%%%%
\subsection*{冲}\addcontentsline{loh}{figure}{冲 \dpy{chong4}}

\begin{EntryWithPhonetic}{冲}{chong4}{6}{⼎}
  \definition{adj.}{poderoso; com vigor; com muita força; vigoroso | forte; odor forte e pungente (olfato)}
  \definition{prep.}{de frente; em direção a | na força de; com base em; em virtude de}
  \definition{v.}{estampar (máquina de estamparia)}
  \seeref{chong1}
\end{EntryWithPhonetic}

%%%%%%%%%% 抽 %%%%%%%%%%
\subsection*{抽}\addcontentsline{loh}{figure}{抽 \dpy{chou1}}

\begin{EntryWithPhonetic}{抽}{chou1}{8}{⼿}[HSK 4]
  \definition{v.}{retirar; tirar (do meio); retirar, puxar ou arrancar algo que está preso ou emaranhado em outra coisa | tirar, retirar (uma parte de um todo) | (certas plantas) começar a crescer, produzir | bombear | encolher; contrair | chicotear; açoitar; surrar | dirigir; conduzir | encontrar tempo; libertar"-se; sair de alguma coisa}
\end{EntryWithPhonetic}

\begin{EntryWithPhonetic}{抽奖}{chou1 jiang3}{8,9}{⼿,⼤}[HSK 4]
  \definition{s.}{loteria; sorteio de loteria}
\end{EntryWithPhonetic}

\begin{EntryWithPhonetic}{抽签}{chou1/qian1}{8,13}{⼿,⽵}[HSK 7-9]
  \definition{v.+compl.}{tirar/lançar sorte; realizar/fazer um sorteio}[他们抽签决定胜者。===Eles fizeram um sorteio para decidir o vencedor.]
\end{EntryWithPhonetic}

\begin{EntryWithPhonetic}{抽屉}{chou1ti4}{8,8}{⼿,⼫}[HSK 7-9]
  \definition[个,层,组]{s.}{gaveta}
\end{EntryWithPhonetic}

\begin{EntryWithPhonetic}{抽象}{chou1xiang4}{8,11}{⼿,⾗}[HSK 7-9]
  \definition{adj.}{abstrato}[抽象的艺术需要想象力。===A arte abstrata requer imaginação.]
  \definition{v.}{abstrair}[这个理论很难抽象。===Essa teoria é difícil de abstrair.]
\end{EntryWithPhonetic}

\begin{EntryWithPhonetic}{抽烟}{chou1/yan1}{8,10}{⼿,⽕}[HSK 4]
  \definition{v.+compl.}{fumar (um cigarro ou um cachimbo)}
\end{EntryWithPhonetic}

%%%%%%%%%% 仇 %%%%%%%%%%
\subsection*{仇}\addcontentsline{loh}{figure}{仇 \dpy{chou2}}

\begin{EntryWithPhonetic}{仇}{chou2}{4}{⼈}[HSK 7-9]
  \definition{s.}{inimigo; adversário | ódio; inimizade}
  \definition{v.}{odiar; detestar}
  \seeref{qiu2}
\end{EntryWithPhonetic}

\begin{EntryWithPhonetic}{仇恨}{chou2hen4}{4,9}{⼈,⼼}[HSK 7-9]
  \definition{s.}{ódio; inimizade; hostilidade; velho rancor}[他对她充满仇恨。===Ele estava cheio de ódio por ela.]
  \definition{v.}{odiar}
\end{EntryWithPhonetic}

\begin{EntryWithPhonetic}{仇人}{chou2ren2}{4,2}{⼈,⼈}[HSK 7-9]
  \definition[个]{s.}{inimigo; inimigo pessoal; pessoas que são hostis por causa do ódio}
\end{EntryWithPhonetic}

\begin{EntryWithPhonetic}{仇外心理}{chou2wai4xin1li3}{4,5,4,11}{⼈,⼣,⼼,⽟}
  \definition{s.}{xenofobia}[他的仇外心理很严重。===Ele tem uma mentalidade xenófoba séria.]
\end{EntryWithPhonetic}

%%%%%%%%%% 愁 %%%%%%%%%%
\subsection*{愁}\addcontentsline{loh}{figure}{愁 \dpy{chou2}}

\begin{EntryWithPhonetic}{愁}{chou2}{13}{⼼}[HSK 5]
  \definition{adj.}{triste; pesaroso; angustiado; desconsolado; preocupado; deprimido}
  \definition{s.}{pesar; sofrimento; dor; tristeza}
  \definition{v.}{preocupar"-se; estar preocupado; ficar ansioso; sentir ansiedade}
\end{EntryWithPhonetic}

\begin{EntryWithPhonetic}{愁眉苦脸}{chou2mei2-ku3lian3}{13,9,8,11}{⼼,⽬,⾋,⾁}[HSK 7-9]
  \definition{expr.}{com uma expressão preocupada e angustiada; parecendo perturbado; usar uma expressão preocupada; fazer uma cara feia}
\end{EntryWithPhonetic}

%%%%%%%%%% 稠 %%%%%%%%%%
\subsection*{稠}\addcontentsline{loh}{figure}{稠 \dpy{chou2}}

\begin{EntryWithPhonetic}{稠}{chou2}{13}{⽲}[HSK 7-9]
  \definition*{s.}{Sobrenome: Chou}
  \definition{adj.}{(sopa, etc.) grosso | denso}[人烟稠密。===Densamente povoado.]
\end{EntryWithPhonetic}

\begin{EntryWithPhonetic}{稠密}{chou2mi4}{13,11}{⽲,⼧}[HSK 7-9]
  \definition{adj.}{denso; muitos e densos}[这个地区的森林非常稠密。===As florestas nesta área são muito densas.]
\end{EntryWithPhonetic}

%%%%%%%%%% 筹 %%%%%%%%%%
\subsection*{筹}\addcontentsline{loh}{figure}{筹 \dpy{chou2}}

\begin{EntryWithPhonetic}{筹}{chou2}{13}{⽵}[HSK 7-9]
  \definition{clas.}{usado para pessoas, visto principalmente no vernáculo antigo}
  \definition{s.}{estratégia; estratagema; meios; método | ficha; contador; um pequeno pedaço de bambu ou madeira; frequentemente usado para contagem ou como um voucher}
  \definition{v.}{preparar; planejar; levantar}
\end{EntryWithPhonetic}

\begin{EntryWithPhonetic}{筹办}{chou2ban4}{13,4}{⽵,⼒}[HSK 7-9]
  \definition{v.}{fazer preparativos; fazer arranjos}
\end{EntryWithPhonetic}

\begin{EntryWithPhonetic}{筹备}{chou2bei4}{13,8}{⽵,⼡}[HSK 7-9]
  \definition{v.}{preparar; organizar; planejar}
\end{EntryWithPhonetic}

\begin{EntryWithPhonetic}{筹措}{chou2cuo4}{13,11}{⽵,⼿}[HSK 7-9]
  \definition{v.}{arrecadar (dinheiro) | coletar (fundos)}
\end{EntryWithPhonetic}

\begin{EntryWithPhonetic}{筹划}{chou2hua4}{13,6}{⽵,⼑}[HSK 7-9]
  \definition{v.}{planejar e preparar; encontrar um caminho; fazer um plano}
\end{EntryWithPhonetic}

\begin{EntryWithPhonetic}{筹集}{chou2ji2}{13,12}{⽵,⾫}[HSK 7-9]
  \definition{v.}{arrecadar (dinheiro)}
\end{EntryWithPhonetic}

\begin{EntryWithPhonetic}{筹码}{chou2ma3}{13,8}{⽵,⽯}[HSK 7-9]
  \definition{s.}{ficha; contador (usado para cálculos, frequentemente em jogos de azar como substituto de moeda)}
\end{EntryWithPhonetic}

%%%%%%%%%% 酬 %%%%%%%%%%
\subsection*{酬}\addcontentsline{loh}{figure}{酬 \dpy{chou2}}

\begin{EntryWithPhonetic}{酬}{chou2}{13}{⾣}
  \definition[份]{s.}{recompensa; pagamento}
  \definition{v.}{trocar amigavelmente | cumprir; perceber | (literário) propor um brinde; brindar | pagar; reembolsar | completar; concluir}
\end{EntryWithPhonetic}

\begin{EntryWithPhonetic}{酬劳}{chou2lao2}{13,7}{⾣,⼒}
  \definition{s.}{recompensa}
\end{EntryWithPhonetic}

%%%%%%%%%% 丑 %%%%%%%%%%
\subsection*{丑}\addcontentsline{loh}{figure}{丑 \dpy{chou3}}

\begin{EntryWithPhonetic}{丑}{chou3}{4}{⼀}[HSK 5]
  \definition*{s.}{Sobrenome: Chou}
  \definition{adj.}{feio, sem atrativos | vergonhoso; desavergonhado; escandaloso; censurável; questionável}
  \definition{s.}{palhaço na ópera de Pequim, etc. | o segundo dos Doze Ramos Terrestres}
  \antonymref{美}{mei3}
\end{EntryWithPhonetic}

\begin{EntryWithPhonetic}{丑恶}{chou3'e4}{4,10}{⼀,⼼}[HSK 7-9]
  \definition{adj./s.}{feio; repulsivo; hediondo}
\end{EntryWithPhonetic}

\begin{EntryWithPhonetic}{丑陋}{chou3lou4}{4,8}{⼀,⾩}[HSK 7-9]
  \definition{adj.}{feio; de aparência muito feia | desgraçado; inglório; refere"-se a pensamentos e comportamentos desprezíveis}
\end{EntryWithPhonetic}

\begin{EntryWithPhonetic}{丑闻}{chou3wen2}{4,9}{⼀,⾨}[HSK 7-9]
  \definition[个,件,些]{s.}{escândalo; rumores ou notícias sobre escândalos}
\end{EntryWithPhonetic}

%%%%%%%%%% 瞅 %%%%%%%%%%
\subsection*{瞅}\addcontentsline{loh}{figure}{瞅 \dpy{chou3}}

\begin{EntryWithPhonetic}{瞅}{chou3}{14}{⽬}[HSK 7-9]
  \definition{v.}{Dialeto: olhar para}[让我瞅瞅。===Deixe-me dar uma olhada.]
\end{EntryWithPhonetic}

%%%%%%%%%% 臭 %%%%%%%%%%
\subsection*{臭}\addcontentsline{loh}{figure}{臭 \dpy{chou4}}

\begin{EntryWithPhonetic}{臭}{chou4}{10}{⾃}[HSK 5]
  \definition{adj.}{sujo; malcheiroso; fedorento; contrário de 香 | repugnante; nojento; repulsivo | ruim; pobre; péssimo}
  \definition{adv.}{severamente; firmemente}
  \definition{v.}{falhar em detonar (bala)}
  \seeref{xiu4}
  \seealsoref{香}{xiang1}
\end{EntryWithPhonetic}

\begin{EntryWithPhonetic}{臭气}{chou4qi4}{10,4}{⾃,⽓}
  \definition{s.}{fedor}
\end{EntryWithPhonetic}

%%%%%%%%%% 出 %%%%%%%%%%
\subsection*{出}\addcontentsline{loh}{figure}{出 \dpy{chu1}}

\begin{EntryWithPhonetic}{出}{chu1}{5}{⼐}[HSK 1]
  \definition{clas.}{usado para dramas, peças, óperas, etc.}
  \definition{v.}{deixar; sair (ir); de dentro para fora | vir; chegar | exceder; ir além | emitir; levar para fora | produzir; despejar | surgir; acontecer; ter lugar | publicar; divulgar | ventilar; emitir; descarregar | aparecer; revelar | gastar; pagar}
\end{EntryWithPhonetic}

\begin{EntryWithPhonetic}{出版}{chu1ban3}{5,8}{⼐,⽚}[HSK 5]
  \definition{v.}{aparecer; publicar; sair; sair da imprensa}
\end{EntryWithPhonetic}

\begin{EntryWithPhonetic}{出版社}{chu1ban3she4}{5,8,7}{⼐,⽚,⽰}[HSK 7-9]
  \definition{s.}{editora; instituições que se dedicam à edição e publicação de livros, periódicos, pinturas, produtos audiovisuais, etc.}
\end{EntryWithPhonetic}

\begin{EntryWithPhonetic}{出差}{chu1/chai1}{5,9}{⼐,⼯}[HSK 5]
  \definition{v.+compl.}{fazer uma viagem de negócios | assumir tarefas de curto prazo em transporte, construção, etc.}
\end{EntryWithPhonetic}

\begin{EntryWithPhonetic}{出厂}{chu1/chang3}{5,2}{⼐,⼚}[HSK 7-9]
  \definition{v.+compl.}{(produtos) ser despachado da fábrica | sair da fábrica (produtos acabados)}
\end{EntryWithPhonetic}

\begin{EntryWithPhonetic}{出场}{chu1 chang3}{5,6}{⼐,⼟}[HSK 6]
  \definition{v.}{entrar no palco; aparecer em cena; entrar atores no palco (performance) | entrar na arena ou campo de esportes; entrar atletas no estádio (para participar de uma apresentação ou competição)}
\end{EntryWithPhonetic}

\begin{EntryWithPhonetic}{出丑}{chu1/chou3}{5,4}{⼐,⼀}[HSK 7-9]
  \definition{v.+compl.}{fazer alguém de bobo | fazer papel de bobo; tornar"-se motivo de chacota; expor"-se ao ridículo por falta de jeito (estranheza, incompetência, etc.) | envergonhar"-se; expor os próprios pontos fracos}
\end{EntryWithPhonetic}

\begin{EntryWithPhonetic}{出道}{chu1/dao4}{5,12}{⼐,⾡}[HSK 7-9]
  \definition{v.+compl.}{(anteriormente um aprendiz) começar a trabalhar depois de cumprir seu aprendizado; refere"-se à aprendizagem de uma habilidade e ao início do trabalho independente; agora geralmente se refere aos jovens que entram na sociedade pela primeira vez | tornar"-se conhecido, estrear"-se na sociedade; começar uma carreira (no show business, etc.); refere"-se a um artista que entra formalmente na indústria do entretenimento e ganha reconhecimento na carreira}
\end{EntryWithPhonetic}

\begin{EntryWithPhonetic}{出动}{chu1 dong4}{5,6}{⼐,⼒}[HSK 6]
  \definition{v.}{partir; começar; passear em equipe | chamar; enviar; despachar; enviar forças militares | entrar em ação; manifestar"-se; tomar uma atitude}
\end{EntryWithPhonetic}

\begin{EntryWithPhonetic}{出发}{chu1fa1}{5,5}{⼐,⼜}[HSK 2]
  \definition{v.}{sair; partir; ir embora; deixar; sair do lugar onde se está e ir para outro lugar | começar a partir de; partir de; considerar ou tratar uma questão a partir de um determinado ponto de vista}
\end{EntryWithPhonetic}

\begin{EntryWithPhonetic}{出发点}{chu1fa1dian3}{5,5,9}{⼐,⼜,⽕}[HSK 7-9]
  \definition{s.}{ponto de partida de uma jornada | ponto de vista (em uma discussão, argumento, etc.); ponto de partida | origem; ponto de partida}
\end{EntryWithPhonetic}

\begin{EntryWithPhonetic}{出访}{chu1 fang3}{5,6}{⼐,⾔}[HSK 6]
  \definition{v.}{ir ao exterior em visita oficial | ir visitar em caráter oficial ou para investigação}
\end{EntryWithPhonetic}

\begin{EntryWithPhonetic}{出风头}{chu1 feng1tou5}{5,4,5}{⼐,⾵,⼤}[HSK 7-9]
  \definition{v.}{procurar (estar em; gostar de) holofotes; fazer uma figura elegante; impulsionar"-se; exibir"-se | Literário: buscar os holofotes}
\end{EntryWithPhonetic}

\begin{EntryWithPhonetic}{出国}{chu1/guo2}{5,8}{⼐,⼞}[HSK 2]
  \definition{v.+compl.}{ir para o exterior; deixar a terra natal; viajar para o exterior}
\end{EntryWithPhonetic}

\begin{EntryWithPhonetic}{出汗}{chu1 han4}{5,6}{⼐,⽔}[HSK 5]
  \definition{v.}{suar; transpirar}
\end{EntryWithPhonetic}

\begin{EntryWithPhonetic}{出击}{chu1ji1}{5,5}{⼐,⼐}
  \definition{v.}{atacar}
\end{EntryWithPhonetic}

\begin{EntryWithPhonetic}{出家}{chu1 jia1}{5,10}{⼐,⼧}
  \definition{v.}{renunciar à família (para se tornar monge ou monja) | tornar"-se monge, monja ou sacerdote taoísta}
  \antonymref{在家}{zai4 jia1}
\end{EntryWithPhonetic}

\begin{EntryWithPhonetic}{出境}{chu1/jing4}{5,14}{⼐,⼟}[HSK 7-9]
  \definition{v.+compl.}{sair do país | sair de uma determinada região | partir; sair; emigrar}
\end{EntryWithPhonetic}

\begin{EntryWithPhonetic}{出局}{chu1/ju2}{5,7}{⼐,⼫}[HSK 7-9]
  \definition{v.+compl.}{estar fora (no beisebol, softball, etc.) | estar fora (de uma competição, etc.); abandonar; ser eliminado}
\end{EntryWithPhonetic}

\begin{EntryWithPhonetic}{出具}{chu1ju4}{5,8}{⼐,⼋}[HSK 7-9]
  \definition{v.}{emitir; escrever; produzir ou escrever (algum tipo de prova)}
\end{EntryWithPhonetic}

\begin{EntryWithPhonetic}{出口}{chu1/kou3}{5,3}{⼐,⼝}[HSK 2,4]
  \definition[个]{s.}{saída; porta ou passagem que dá acesso ao exterior}
  \definition{v.+compl.}{falar; proferir; manifestar"-se | exportar mercadorias do país ou da região para venda no exterior ou em outro lugar | deixar o porto (de um navio)}
\end{EntryWithPhonetic}

\begin{EntryWithPhonetic}{出口成章}{chu1kou3-cheng2zhang1}{5,3,6,11}{⼐,⼝,⼽,⾳}[HSK 7-9]
  \definition{expr.}{``As palavras fluem da boca como da pena de um mestre.''; ``Seja um excelente orador.''  | (discurso de alguém) eloquente | articular}
\end{EntryWithPhonetic}

\begin{EntryWithPhonetic}{出来}{chu1 lai2}{5,7}{⼐,⽊}[HSK 1]
  \definition{v.}{emergir; sair; (para a minha direção) |  surgir; aparecer; emergir | concluir ou algo acontecer}
\end{EntryWithPhonetic}

\begin{EntryWithPhonetic}{出路}{chu1lu4}{5,13}{⼐,⾜}[HSK 6]
  \definition[条,个]{s.}{saída; futuro; uma maneira de manter a sobrevivência ou progredir intermitentemente; também pode se referir ao futuro | saída; formas de vender produtos | saída; um caminho que leva para fora ou para frente}
\end{EntryWithPhonetic}

\begin{EntryWithPhonetic}{出卖}{chu1mai4}{5,8}{⼐,⼗}[HSK 7-9]
  \definition{v.}{vender; oferecer para venda; trocar trabalho por uma certa quantia de compensação | vender; trair; negociar; fazer coisas que beneficiam o inimigo para ganho pessoal, causando danos ao país, à nação, a parentes e amigos, etc.}
\end{EntryWithPhonetic}

\begin{EntryWithPhonetic}{出毛病}{chu1 mao2 bing4}{5,4,10}{⼐,⽑,⽧}[HSK 7-9]
  \definition{v.}{estar (sair) fora de ordem; dar errado; quebrar; ter problemas com; falhar; sofrer um acidente | Coloquial:
estar (ou ficar) fora de serviço}
\end{EntryWithPhonetic}

\begin{EntryWithPhonetic}{出门}{chu1/men2}{5,3}{⼐,⾨}[HSK 2]
  \definition{v.+compl.}{sair | sair de casa; estar longe de casa; viajar para longe de casa | casar"-se}
\end{EntryWithPhonetic}

\begin{EntryWithPhonetic}{出面}{chu1 mian4}{5,9}{⼐,⾯}[HSK 6]
  \definition{v.}{comparecer pessoalmente; agir em sua própria capacidade (ou em nome de alguém) | agir em sua própria capacidade (em nome de uma organização); apresentar"-se; fazer algo individualmente ou coletivamente}
\end{EntryWithPhonetic}

\begin{EntryWithPhonetic}{出名}{chu1 ming2}{5,6}{⼐,⼝}[HSK 6]
  \definition{adj.}{famoso; bem conhecido; renomado}
  \definition{v.}{tornar"-se famoso (ou conhecido) | emprestar o próprio nome (para uma ocasião ou empresa); usar o nome de}
\end{EntryWithPhonetic}

\begin{EntryWithPhonetic}{出难题}{chu1 nan2ti2}{5,10,15}{⼐,⾫,⾴}[HSK 7-9]
  \definition{v.}{levantar uma questão difícil; dificultar as coisas para alguém | fazer perguntas difíceis; propor um problema difícil; definir uma tarefa muito difícil}
\end{EntryWithPhonetic}

\begin{EntryWithPhonetic}{出去}{chu1 qu4}{5,5}{⼐,⼛}[HSK 1]
  \definition{v.}{sair; ir para fora;  (a partir da minha localização)}
\end{EntryWithPhonetic}

\begin{EntryWithPhonetic}{出人意料}{chu1ren2yi4liao4}{5,2,13,10}{⼐,⼈,⼼,⽃}[HSK 7-9]
  \definition{expr.}{excedendo todas as expectativas; além de todas as expectativas | surpreendente | inesperado}
\end{EntryWithPhonetic}

\begin{EntryWithPhonetic}{出任}{chu1ren4}{5,6}{⼐,⼈}[HSK 7-9]
  \definition{v.}{assumir o cargo de; assumir (uma determinada posição oficial)}
\end{EntryWithPhonetic}

\begin{EntryWithPhonetic}{出入}{chu1 ru4}{5,2}{⼐,⼊}[HSK 6]
  \definition{s.}{discrepância; divergência; inconsistência; diferença}
  \definition{v.}{entrar e sair; sair e entrar}
\end{EntryWithPhonetic}

\begin{EntryWithPhonetic}{出色}{chu1se4}{5,6}{⼐,⾊}[HSK 4]
  \definition{adj.}{esplêndido; extraordinário; notável; excepcionalmente bom; acima da média}
\end{EntryWithPhonetic}

\begin{EntryWithPhonetic}{出山}{chu1/shan1}{5,3}{⼐,⼭}[HSK 7-9]
  \definition{v.+compl.}{deixar a área da montanha | sair da aposentadoria e assumir um cargo | deixar a aposentadoria e assumir um cargo governamental; tornar"-se um funcionário público}
\end{EntryWithPhonetic}

\begin{EntryWithPhonetic}{出身}{chu1shen1}{5,7}{⼐,⾝}[HSK 7-9]
  \definition{s.}{origem familiar; origem (de classe) | experiência anterior (ou ocupação)}
  \definition{v.}{ser descendente de; vir de}
\end{EntryWithPhonetic}

\begin{EntryWithPhonetic}{出生}{chu1sheng1}{5,5}{⼐,⽣}[HSK 2]
  \definition{v.}{nascer}
\end{EntryWithPhonetic}

\begin{EntryWithPhonetic}{出示}{chu1shi4}{5,5}{⼐,⽰}[HSK 7-9]
  \definition{v.}{mostrar; produzir}
\end{EntryWithPhonetic}

\begin{EntryWithPhonetic}{出事}{chu1 shi4}{5,8}{⼐,⼅}[HSK 6]
  \definition{v.}{sofrer um acidente; ocorrer um acidente ou incidente}
\end{EntryWithPhonetic}

\begin{EntryWithPhonetic}{出手}{chu1/shou3}{5,4}{⼐,⼿}[HSK 7-9]
  \definition{s.}{a habilidade que você demonstra quando começa a fazer algo | comprimento da manga}
  \definition{v.+compl.}{vender; dispor de; tirar (bens acumulados, etc.) das mãos de alguém; vender mercadorias (usado principalmente para negociação, liquidação, etc.) | atacar; começar a agir; refere"-se à luta física | oferecer; produzir; tirar (dinheiro ou algo); gastar dinheiro}
\end{EntryWithPhonetic}

\begin{EntryWithPhonetic}{出售}{chu1 shou4}{5,11}{⼐,⼝}[HSK 4]
  \definition{v.}{vender; oferecer para venda}
\end{EntryWithPhonetic}

\begin{EntryWithPhonetic}{出台}{chu1 tai2}{5,5}{⼐,⼝}[HSK 6]
  \definition{v.}{aparecer no palco; entrar atores no palco | aparecer publicamente; metáfora (política, plano, programa, etc.) lançar oficialmente}
\end{EntryWithPhonetic}

\begin{EntryWithPhonetic}{出头}{chu1/tou2}{5,5}{⼐,⼤}[HSK 7-9]
  \definition{s.}{fração ímpar restante após uma divisão | (após um número redondo) um pouco superior; um pouco mais que}
  \definition{v.+compl.}{erguer a cabeça; libertar-se (da miséria, da perseguição, etc.); livrar"-se de circunstâncias miseráveis; sair dos problemas | apresentar"-se; aparecer em público; agir em nome de; assumir a liderança; avançar | (algo) mostrar sua ponta; apenas/se destacar (objeto) mostrando o topo}
\end{EntryWithPhonetic}

\begin{EntryWithPhonetic}{出土}{chu1/tu3}{5,3}{⼐,⼟}[HSK 7-9]
  \definition{v.+compl.}{ser desenterrado; ser escavado | sair do chão | subir; emergir}
\end{EntryWithPhonetic}

\begin{EntryWithPhonetic}{出席}{chu1xi2}{5,10}{⼐,⼱}[HSK 4]
  \definition{v.}{comparecer; estar presente; participar de reuniões com o direito de falar e votar; juntar"-se a uma organização ou atividade}
\end{EntryWithPhonetic}

\begin{EntryWithPhonetic}{出息}{chu1xi5}{5,10}{⼐,⼼}[HSK 7-9]
  \definition{s.}{perspectivas; futuro brilhante; refere"-se ao desenvolvimento ou ambição futura}
  \definition{v.+compl.}{progredir; progredir bem; há progresso e a capacidade se torna mais forte}
\end{EntryWithPhonetic}

\begin{EntryWithPhonetic}{出现}{chu1xian4}{5,8}{⼐,⾒}[HSK 2]
  \definition{v.}{aparecer; surgir; emergir; crescer; revelar}
\end{EntryWithPhonetic}

\begin{EntryWithPhonetic}{出血}{chu1/xie3}{5,6}{⼐,⾎}[HSK 7-9]
  \definition{s.}{Medicina: hemorragia; sangramento}
  \definition{v.+compl.}{perder sangue; sangrar | desembolsar (dinheiro, etc.)}
\end{EntryWithPhonetic}

\begin{EntryWithPhonetic}{出行}{chu1 xing2}{5,6}{⼐,⾏}[HSK 6]
  \definition{v.}{viajar; sair}
\end{EntryWithPhonetic}

\begin{EntryWithPhonetic}{出演}{chu1yan3}{5,14}{⼐,⽔}[HSK 7-9]
  \definition{s.}{uma aparição (no palco, etc.)}
  \definition{v.}{desempenhar o papel de; atuar | aparecer (em um show, etc.)}
\end{EntryWithPhonetic}

\begin{EntryWithPhonetic}{出洋相}{chu1 yang2xiang4}{5,9,9}{⼐,⽔,⽬}[HSK 7-9]
  \definition{v.}{fazer papel de bobo}
\end{EntryWithPhonetic}

\begin{EntryWithPhonetic}{出游}{chu1you2}{5,12}{⼐,⽔}[HSK 7-9]
  \definition{v.}{fazer um passeio turístico (ou viagem)}
\end{EntryWithPhonetic}

\begin{EntryWithPhonetic}{出于}{chu1 yu2}{5,3}{⼐,⼆}[HSK 5]
  \definition{prep.}{de; fora de; por causa de; em função de; de um certo ponto de vista}
  \definition{v.}{iniciar a partir de; originar"-se de; prosseguir a partir de}
\end{EntryWithPhonetic}

\begin{EntryWithPhonetic}{出院}{chu1 yuan4}{5,9}{⼐,⾩}[HSK 2]
  \definition{v.}{sair do hospital; estar fora do hospital; receber alta do hospital}
\end{EntryWithPhonetic}

\begin{EntryWithPhonetic}{出站}{chu1 zhan4}{5,10}{⼐,⽴}
  \definition{s.}{saída da estação}
\end{EntryWithPhonetic}

\begin{EntryWithPhonetic}{出众}{chu1zhong4}{5,6}{⼐,⼈}[HSK 7-9]
  \definition{adj.}{fora do comum; excepcional}
\end{EntryWithPhonetic}

\begin{EntryWithPhonetic}{出主意}{chu1 zhu3yi5}{5,5,13}{⼐,⼂,⼼}[HSK 7-9]
  \definition{v.}{oferecer conselhos; fornecer ideias; fazer sugestões; apresentar ideias}
\end{EntryWithPhonetic}

\begin{EntryWithPhonetic}{出资}{chu1zi1}{5,10}{⼐,⾙}[HSK 7-9]
  \definition{v.}{financiar; investir capital; fornecer fundos (ou capital); fornecer financiamento}
\end{EntryWithPhonetic}

\begin{EntryWithPhonetic}{出自}{chu1zi4}{5,6}{⼐,⾃}[HSK 7-9]
  \definition{v.}{vir de; provir de; originar"-se de (um certo tempo ou lugar)}
\end{EntryWithPhonetic}

\begin{EntryWithPhonetic}{出走}{chu1zou3}{5,7}{⼐,⾛}[HSK 7-9]
  \definition{v.}{deixar a própria casa (ou país) sob compulsão; fugir; ir embora; sair de casa}
\end{EntryWithPhonetic}

\begin{EntryWithPhonetic}{出租}{chu1 zu1}{5,10}{⼐,⽲}[HSK 2]
  \definition{s.}{taxi; abreviação de 出租车}
  \definition{v.}{alugar; arrendar; receber dinheiro de outras pessoas para permitir que elas utilizem algo (como uma casa, um carro, livros, etc.) por um determinado período de tempo}
  \seealsoref{出租车}{chu1zu1che1}
\end{EntryWithPhonetic}

\begin{EntryWithPhonetic}{出租车}{chu1zu1che1}{5,10,4}{⼐,⽲,⾞}[HSK 2]
  \definition[辆]{s.}{táxi; carro de aluguel; veículos de transporte urbano disponíveis para aluguel, com cobrança por quilometragem ou tempo}
  \seealsoref{出租汽车}{chu1zu1qi4che1}
\end{EntryWithPhonetic}

\begin{EntryWithPhonetic}{出租汽车}{chu1zu1qi4che1}{5,10,7,4}{⼐,⽲,⽔,⾞}
  \definition[辆]{s.}{táxi}
  \seealsoref{出租车}{chu1zu1che1}
\end{EntryWithPhonetic}

\begin{EntryWithPhonetic}{出租司机}{chu1zu1si1ji1}{5,10,5,6}{⼐,⽲,⼝,⽊}
  \definition{s.}{motorista de táxi}
\end{EntryWithPhonetic}

%%%%%%%%%% 初 %%%%%%%%%%
\subsection*{初}\addcontentsline{loh}{figure}{初 \dpy{chu1}}

\begin{EntryWithPhonetic}{初}{chu1}{7}{⾐}[HSK 3]
  \definition*{s.}{Sobrenome: Chu}
  \definition{adj.}{primeiro (em ordem) | elementar; rudimentar | original}
  \definition{adv.}{pela primeira vez; apenas começando; indica que a ação está ocorrendo pela primeira vez ou acabou de começar}
  \definition{pref.}{anexado a alguns substantivos ou números cardinais, indicando o primeiro}
  \definition{s.}{no início de; na primeira parte de | o estágio júnior (pleno; sênior)}
\end{EntryWithPhonetic}

\begin{EntryWithPhonetic}{初步}{chu1bu4}{7,7}{⾐,⽌}[HSK 3]
  \definition{adj.}{inicial; preliminar; imaturo, incompleto}
\end{EntryWithPhonetic}

\begin{EntryWithPhonetic}{初次}{chu1ci4}{7,6}{⾐,⽋}[HSK 7-9]
  \definition{adv.}{a primeira vez}
\end{EntryWithPhonetic}

\begin{EntryWithPhonetic}{初等}{chu1 deng3}{7,12}{⾐,⽵}[HSK 6]
  \definition{adj.}{elementar; primário; rudimentar | elementar (ou seja, fácil)}
\end{EntryWithPhonetic}

\begin{EntryWithPhonetic}{初级}{chu1ji2}{7,6}{⾐,⽷}[HSK 3]
  \definition{adj.}{elementar; primário; júnior; inicial; o nível mais baixo; de baixa qualidade}
\end{EntryWithPhonetic}

\begin{EntryWithPhonetic}{初期}{chu1 qi1}{7,12}{⾐,⽉}[HSK 5]
  \definition{s.}{primórdio; estágio inicial; primeiros dias; estágio preliminar; período inicial}
\end{EntryWithPhonetic}

\begin{EntryWithPhonetic}{初心}{chu1xin1}{7,4}{⾐,⼼}
  \definition{s.}{intenção original de alguém, aspiração, etc. | Budismo: ``mente do iniciante'' (ter a mente aberta quando estudando um assunto como um iniciante no assunto teria)}
\end{EntryWithPhonetic}

\begin{EntryWithPhonetic}{初中}{chu1 zhong1}{7,4}{⾐,⼁}[HSK 3]
  \definition[所,个]{s.}{ensino médio; ensino fundamental}
\end{EntryWithPhonetic}

\begin{EntryWithPhonetic}{初衷}{chu1zhong1}{7,10}{⾐,⾐}[HSK 7-9]
  \definition{s.}{intenção original}
\end{EntryWithPhonetic}

%%%%%%%%%% 除 %%%%%%%%%%
\subsection*{除}\addcontentsline{loh}{figure}{除 \dpy{chu2}}

\begin{EntryWithPhonetic}{除}{chu2}{9}{⾩}[HSK 6]
  \definition*{s.}{Sobrenome: Chu}
  \definition{prep.}{exceto; não incluído | além do mais}
  \definition{s.}{degraus de uma casa; degraus de uma porta; escadaria}
  \definition{v.}{remover; livrar"-se de; eliminar; limpar | dividir; executar operação de divisão | nomear para o cargo}
\end{EntryWithPhonetic}

\begin{EntryWithPhonetic}{除此之外}{chu2ci3zhi1wai4}{9,6,3,5}{⾩,⽌,⼂,⼣}[HSK 7-9]
  \definition{expr.}{além disso; além destes; além do mais}
\end{EntryWithPhonetic}

\begin{EntryWithPhonetic}{除非}{chu2fei1}{9,8}{⾩,⾮}[HSK 5]
  \definition{conj.}{a menos que; somente se; indica a única condição, equivalente a 只有, frequentemente combinada com 才, 否则, 不然, etc.}
  \seealsoref{不然}{bu4ran2}
  \seealsoref{才}{cai2}
  \seealsoref{否则}{fou3ze2}
  \seealsoref{只有}{zhi3 you3}
\end{EntryWithPhonetic}

\begin{EntryWithPhonetic}{除了}{chu2le5}{9,2}{⾩,⼅}[HSK 3]
  \definition{prep.}{exceto; à parte; indica que o que foi dito não é levado em consideração | além disso; além de; usado em conjunto com 还, 也 e 只, indica que, além de algo, há ainda outra coisa | ou\dots ou\dots; usado em conjunto com 就是, significa ``ou assim ou assado''}
  \seealsoref{还}{hai2}
  \seealsoref{就是}{jiu4 shi4}
  \seealsoref{也}{ye3}
  \seealsoref{只}{zhi3}
\end{EntryWithPhonetic}

\begin{EntryWithPhonetic}{除去}{chu2qu4}{9,5}{⾩,⼛}[HSK 7-9]
  \definition{prep.}{além disso; além de; exceto por}
  \definition{v.}{livrar"-se de; eliminar; remover}
\end{EntryWithPhonetic}

\begin{EntryWithPhonetic}{除外}{chu2wai4}{9,5}{⾩,⼣}[HSK 7-9]
  \definition{prep.}{exceto; não contando; não incluindo | mas; exclusivo de; com exceção de; impedindo; além de; exceto; exceto por; a menos que; aquém de | além de; diferente de; de outra forma que não; em (para) o negócio; em cima de}
\end{EntryWithPhonetic}

\begin{EntryWithPhonetic}{除夕}{chu2xi1}{9,3}{⾩,⼣}[HSK 5]
  \definition*{s.}{Véspera de Ano Novo Lunar; a noite do último dia do ano, também se refere ao último dia do ano}
\end{EntryWithPhonetic}

%%%%%%%%%% 厨 %%%%%%%%%%
\subsection*{厨}\addcontentsline{loh}{figure}{厨 \dpy{chu2}}

\begin{EntryWithPhonetic}{厨}{chu2}{12}{⼚}
  \definition[个]{s.}{cozinha}
\end{EntryWithPhonetic}

\begin{EntryWithPhonetic}{厨房}{chu2fang2}{12,8}{⼚,⼾}[HSK 5]
  \definition[间,个]{s.}{cozinha}
\end{EntryWithPhonetic}

\begin{EntryWithPhonetic}{厨师}{chu2 shi1}{12,6}{⼚,⼱}[HSK 6]
  \definition[名,位,个]{s.}{chefe de cozinha; cozinheiro; alguém que é bom em cozinhar e faz disso uma profissão}
\end{EntryWithPhonetic}

%%%%%%%%%% 处 %%%%%%%%%%
\subsection*{处}\addcontentsline{loh}{figure}{处 \dpy{chu3}}

\begin{EntryWithPhonetic}{处}{chu3}{5}{⼡}[HSK 4]
  \definition*{s.}{Sobrenome: Chu}
  \definition{v.}{morar; habitar; viver em um lugar | dar"-se bem (com alguém); relacionar-se; interagir | estar situado em; estar em uma determinada condição; estar em (um lugar, período ou ocasião) | gerenciar; manejar; lidar com | punir; sentenciar; tomar medidas disciplinares contra (alguém)}
  \seeref{chu4}
\end{EntryWithPhonetic}

\begin{EntryWithPhonetic}{处罚}{chu3 fa2}{5,9}{⼡,⽹}[HSK 5]
  \definition[个,次,种]{s.}{punição; castigo; penalidade}
  \definition{v.}{punir; disciplinar; castigar; advertir o transgressor ou infrator sobre perdas políticas ou financeiras}
\end{EntryWithPhonetic}

\begin{EntryWithPhonetic}{处方}{chu3fang1}{5,4}{⼡,⽅}[HSK 7-9]
  \definition[张,份]{s.}{prescrição; receita; uma receita médica ou uma receita pronta para preparar medicamentos}
  \definition{v.}{escrever uma receita; prescrever}
\end{EntryWithPhonetic}

\begin{EntryWithPhonetic}{处分}{chu3fen4}{5,4}{⼡,⼑}[HSK 5]
  \definition{s.}{punição; castigo; refere"-se a uma decisão de impor uma penalidade ou uma disposição}
  \definition{v.}{punir; tomar medidas disciplinares contra; fornecer algum tratamento ou disposição para aqueles que cometeram erros ou falhas}
\end{EntryWithPhonetic}

\begin{EntryWithPhonetic}{处境}{chu3jing4}{5,14}{⼡,⼟}[HSK 7-9]
  \definition[种,个]{s.}{situação difícil; situação desfavorável}
\end{EntryWithPhonetic}

\begin{EntryWithPhonetic}{处理}{chu3li3}{5,11}{⼡,⽟}[HSK 3]
  \definition{s.}{manuseio; descarte}
  \definition{v.}{lidar com; dispor de; organizar; resolver | resolver; punir; lidar | vender a preços reduzidos; liquidar | lidar com; processar; processar algo de uma maneira ou método específico; processar uma peça de trabalho ou produto de uma maneira específica para que a peça de trabalho ou produto obtenha o desempenho necessário}
\end{EntryWithPhonetic}

\begin{EntryWithPhonetic}{处于}{chu3 yu2}{5,3}{⼡,⼆}[HSK 4]
  \definition{v.}{estar em (uma condição, estado)}
\end{EntryWithPhonetic}

\begin{EntryWithPhonetic}{处在}{chu3 zai4}{5,6}{⼡,⼟}[HSK 5]
  \definition{v.}{estar situado em; encontrar"-se em; estar em (algum estado, posição ou condição)}
\end{EntryWithPhonetic}

\begin{EntryWithPhonetic}{处置}{chu3zhi4}{5,13}{⼡,⽹}[HSK 7-9]
  \definition{v.}{lidar com; gerenciar; descartar | punir}
\end{EntryWithPhonetic}

%%%%%%%%%% 储 %%%%%%%%%%
\subsection*{储}\addcontentsline{loh}{figure}{储 \dpy{chu3}}

\begin{EntryWithPhonetic}{储}{chu3}{12}{⼈}
  \definition*{s.}{Sobrenome: Chu}
  \definition{s.}{herdeiro de um trono | herdeiro}
  \definition{v.}{armazenar | guardar; manter (ter) em reserva}
\end{EntryWithPhonetic}

\begin{EntryWithPhonetic}{储备}{chu3bei4}{12,8}{⼈,⼡}[HSK 7-9]
  \definition{v.}{guardar; armazenar para uso futuro}
\end{EntryWithPhonetic}

\begin{EntryWithPhonetic}{储存}{chu3cun2}{12,6}{⼈,⼦}[HSK 6]
  \definition{v.}{armazenar; depositar; colocar em; economizar dinheiro ou coisas que você não precisará em um futuro próximo}
\end{EntryWithPhonetic}

\begin{EntryWithPhonetic}{储蓄}{chu3xu4}{12,13}{⼈,⾋}[HSK 7-9]
  \definition[笔,份]{s.}{depósitos; poupanças; refere"-se a dinheiro ou coisas acumuladas}
  \definition{v.}{poupar; depositar; guardar dinheiro ou coisas que são guardadas ou não são usadas temporariamente, geralmente significa depositar dinheiro em um banco}
\end{EntryWithPhonetic}

%%%%%%%%%% 褚 %%%%%%%%%%
\subsection*{褚}\addcontentsline{loh}{figure}{褚 \dpy{chu3}}

\begin{EntryWithPhonetic}{褚}{chu3}{13}{⾐}
  \definition*{s.}{Sobrenome: Chu}
  \seeref{zhu3}
\end{EntryWithPhonetic}

\begin{EntryWithPhonetic}{褚遂良}{chu3 sui4liang2}{13,12,7}{⾐,⾡,⾉}
  \definition*{s.}{Chu Suiliang (596-659), um dos quatro grandes calígrafos do início da dinastia Tang, 唐初四大家}
  \seealsoref{唐初四大家}{tang2 chu1 si4 da4jia1}
\end{EntryWithPhonetic}

%%%%%%%%%% 处 %%%%%%%%%%
\subsection*{处}\addcontentsline{loh}{figure}{处 \dpy{chu4}}

\begin{EntryWithPhonetic}{处}{chu4}{5}{⼡}
  \definition{clas.}{usado para lugares ou para ocorrências ou atividades em lugares diferentes}
  \definition{s.}{lugar; local; instalação; dependência | parte; ponto; aspecto ou parte de um objeto | escritório; departamento; nomes de determinados órgãos, organizações ou unidades em órgãos por empresa}
  \seeref{chu3}
\end{EntryWithPhonetic}

\begin{EntryWithPhonetic}{处处}{chu4 chu4}{5,5}{⼡,⼡}[HSK 6]
  \definition{adv.}{em todos os lugares; em todos os aspectos}
\end{EntryWithPhonetic}

\begin{EntryWithPhonetic}{处长}{chu4 zhang3}{5,4}{⼡,⾧}[HSK 6]
  \definition{s.}{chefe de um departamento (ou escritório); chefe de seção}
\end{EntryWithPhonetic}

%%%%%%%%%% 畜 %%%%%%%%%%
\subsection*{畜}\addcontentsline{loh}{figure}{畜 \dpy{chu4}}

\begin{EntryWithPhonetic}{畜}{chu4}{10}{⽥}
  \definition*{s.}{Sobrenome: Chu}
  \definition{s.}{animal doméstico; gado; bestas, principalmente referindo"-se ao gado}
  \seeref{xu4}
\end{EntryWithPhonetic}

%%%%%%%%%% 触 %%%%%%%%%%
\subsection*{触}\addcontentsline{loh}{figure}{触 \dpy{chu4}}

\begin{EntryWithPhonetic}{触}{chu4}{13}{⾓}
  \definition{v.}{tocar; contatar | atacar; dar um toque | tocar/mover alguém emocionalmente; agitar os sentimentos de alguém | levar um choque elétrico; ser eletrocutado}
\end{EntryWithPhonetic}

\begin{EntryWithPhonetic}{触动}{chu4dong4}{13,6}{⾓,⼒}[HSK 7-9]
  \definition{v.}{tocar em algo; tocar (um interruptor, uma tela, etc.) | comover alguém; despertar os sentimentos de alguém; ter uma mudança emocional causada por algum estímulo; ser movido | afetar; um comportamento que afeta, perturba ou prejudica outras pessoas}
\end{EntryWithPhonetic}

\begin{EntryWithPhonetic}{触犯}{chu4fan4}{13,5}{⾓,⽝}[HSK 7-9]
  \definition{v.}{ofender; violar; ir contra; infringir}
\end{EntryWithPhonetic}

\begin{EntryWithPhonetic}{触觉}{chu4jue2}{13,9}{⾓,⾒}[HSK 7-9]
  \definition{s.}{sensação tátil; sentido do tato | tato; recepção do toque; tigmestesia; pselaphesia; pselaphesis; tactilidade}
\end{EntryWithPhonetic}

\begin{EntryWithPhonetic}{触摸}{chu4mo1}{13,13}{⾓,⼿}[HSK 7-9]
  \definition{s.}{tocar; acariciar brevemente uma parte do corpo}
\end{EntryWithPhonetic}

\begin{EntryWithPhonetic}{触目惊心}{chu4mu4-jing1xin1}{13,5,11,4}{⾓,⽬,⼼,⼼}[HSK 7-9]
  \definition{expr.}{ver a cena que é terrível para a mente; atingir os olhos e despertar a mente; uma visão medonha; horripilante; chocante (para as pessoas); assustador | (uma cena) surpreendente; chocante}
\end{EntryWithPhonetic}

%%%%%%%%%% 揣 %%%%%%%%%%
\subsection*{揣}\addcontentsline{loh}{figure}{揣 \dpy{chuai1}}

\begin{EntryWithPhonetic}{揣}{chuai1}{12}{⼿}[HSK 7-9]
  \definition{v.}{esconder (ou carregar) nas roupas | Dialeto: encher"-se de comida; comer demais; encher alguém de comida; alimentar em excesso}
  \seeref{chuai3}
\end{EntryWithPhonetic}

\begin{EntryWithPhonetic}{揣}{chuai3}{12}{⼿}
  \definition*{s.}{Sobrenome: Chuai}
  \definition{v.}{contar; calcular; medir | estimar; palpitar; conjecturar}
  \seeref{chuai1}
\end{EntryWithPhonetic}

\begin{EntryWithPhonetic}{揣测}{chuai3ce4}{12,9}{⼿,⽔}[HSK 7-9]
  \definition{v.}{adivinhar; conjecturar | supor; calcular; especular}
\end{EntryWithPhonetic}

\begin{EntryWithPhonetic}{揣摩}{chuai3mo2}{12,15}{⼿,⼿}[HSK 7-9]
  \definition{v.}{tentar compreender; tentar descobrir; obter algo por meio de estudo cuidadoso; pesar e considerar}
\end{EntryWithPhonetic}

%%%%%%%%%% 踹 %%%%%%%%%%
\subsection*{踹}\addcontentsline{loh}{figure}{踹 \dpy{chuai4}}

\begin{EntryWithPhonetic}{踹}{chuai4}{16}{⾜}[HSK 7-9]
  \definition{v.}{chutar (com a sola do pé) | pisar; pisotear; pisar em}
\end{EntryWithPhonetic}

%%%%%%%%%% 川 %%%%%%%%%%
\subsection*{川}\addcontentsline{loh}{figure}{川 \dpy{chuan1}}

\begin{EntryWithPhonetic}{川}{chuan1}{3}{⼮}[Kangxi 47]
  \definition*{s.}{Província de Sichuan, abreviação de 四川}
  \definition{s.}{rio; córrego | planície; terra plana}
  \seealsoref{四川}{si4chuan1}
\end{EntryWithPhonetic}

\begin{EntryWithPhonetic}{川流不息}{chuan1liu2-bu4xi1}{3,10,4,10}{⼮,⽔,⼀,⼼}[HSK 7-9]
  \definition{expr.}{``O fluxo nunca para de fluir.''; um fluxo contínuo (de); vem e vai em um fluxo sem fim; vem e vai o tempo todo; flui continuamente; continua continuamente; flui em um fluxo sem fim; flui sem cessar; flui perpetuamente; em um fluxo sem fim; sem fim; despeja em um fluxo sem fim; ``O fluxo flui incessantemente.''; descreve os pedestres, veículos, etc. indo e vindo tão continuamente quanto um fluxo de água}
\end{EntryWithPhonetic}

%%%%%%%%%% 穿 %%%%%%%%%%
\subsection*{穿}\addcontentsline{loh}{figure}{穿 \dpy{chuan1}}

\begin{EntryWithPhonetic}{穿}{chuan1}{9}{⽳}[HSK 1]
  \definition{adj.}{direto; através; usado após certos verbos, indica um estado de revelação completa}
  \definition{s.}{vestuário; roupas; refere"-se a roupas, sapatos, meias, etc.}
  \definition{v.}{usar; vestir; estar vestido; ter\dots vestido;  vestir roupas, sapatos, meias, etc. | perfurar através de; penetrar; formar orifícios por meio de cinzéis, brocas ou pontas afiadas | enfiar; amarrar; usar cordas e fios para ligar coisas | passar por; atravessar; passar por; através de (buracos, fendas, espaços vazios, etc.)}
\end{EntryWithPhonetic}

\begin{EntryWithPhonetic}{穿过}{chuan1guo4}{9,6}{⽳,⾡}[HSK 7-9]
  \definition{v.}{atravessar; penetrar; passar; passar por cima}
\end{EntryWithPhonetic}

\begin{EntryWithPhonetic}{穿上}{chuan1 shang4}{9,3}{⽳,⼀}[HSK 4]
  \definition{v.}{vestir (roupas, etc.); colocar roupas}
\end{EntryWithPhonetic}

\begin{EntryWithPhonetic}{穿小鞋}{chuan1 xiao3xie2}{9,3,15}{⽳,⼩,⾰}[HSK 7-9]
  \definition{v.}{dar a alguém sapatos apertados para usar; dificultar as coisas para alguém abusando do seu poder; metáfora para dificultar secretamente as coisas para alguém ou impor restrições ou obstáculos a essa pessoa}
\end{EntryWithPhonetic}

\begin{EntryWithPhonetic}{穿越}{chuan1yue4}{9,12}{⽳,⾛}[HSK 7-9]
  \definition{v.}{passar através de; cortar através de; atravessar; passar por cima}
\end{EntryWithPhonetic}

\begin{EntryWithPhonetic}{穿着}{chuan1zhuo2}{9,11}{⽳,⽬}[HSK 7-9]
  \definition[次]{s.}{vestido; vestuário; o que alguém veste; o efeito geral das roupas e decorações que as pessoas usam}
\end{EntryWithPhonetic}

%%%%%%%%%% 传 %%%%%%%%%%
\subsection*{传}\addcontentsline{loh}{figure}{传 \dpy{chuan2}}

\begin{EntryWithPhonetic}{传}{chuan2}{6}{⼈}[HSK 3]
  \definition{v.}{passar; passar adiante | passar adiante; legar; passar de\dots para\dots; passar da geração anterior para a seguinte | transmitir (conhecimento, habilidade, etc.); comunicar; ensinar | espalhar; propagar | transmitir; conduzir; transferir | transmitir; expressar | convocar; dar ordem para chamar alguém | infectar; ser contagioso | enviar documentos por e-mail ou fax}
  \seeref{zhuan4}
\end{EntryWithPhonetic}

\begin{EntryWithPhonetic}{传播}{chuan2bo1}{6,15}{⼈,⼿}[HSK 3]
  \definition{v.}{espalhar; difundir; propagar; disseminar}
\end{EntryWithPhonetic}

\begin{EntryWithPhonetic}{传承}{chuan2cheng2}{6,8}{⼈,⼿}[HSK 7-9]
  \definition{v.}{herdar; transmitir (para as gerações futuras); passar adiante (desde os tempos antigos)}
\end{EntryWithPhonetic}

\begin{EntryWithPhonetic}{传出}{chuan2 chu1}{6,5}{⼈,⼐}[HSK 6]
  \definition{adj.}{eferente (nervo)}
  \definition{v.}{disseminar | transmitir para fora}
\end{EntryWithPhonetic}

\begin{EntryWithPhonetic}{传达}{chuan2da2}{6,6}{⼈,⾡}[HSK 5]
  \definition{s.}{recepção e registro de chamadas em um estabelecimento público | zelador | recepcionista}
  \definition{v.}{passar adiante (informações, etc.); transmitir; retransmitir; comunicar; expressar}
\end{EntryWithPhonetic}

\begin{EntryWithPhonetic}{传递}{chuan2 di4}{6,10}{⼈,⾡}[HSK 5]
  \definition{v.}{transmitir; entregar; transferir; passar adiante}
\end{EntryWithPhonetic}

\begin{EntryWithPhonetic}{传给}{chuan2gei3}{6,9}{⼈,⽷}
  \definition{v.}{passar para | transferir para | entregar a}
\end{EntryWithPhonetic}

\begin{EntryWithPhonetic}{传来}{chuan2 lai2}{6,7}{⼈,⽊}[HSK 3]
  \definition{v.}{(um som) passar; transmitir de algum lugar para o local onde o locutor se encontra | (notícias) chegar; transmitir mensagens ou informações}
\end{EntryWithPhonetic}

\begin{EntryWithPhonetic}{传媒}{chuan2 mei2}{6,12}{⼈,⼥}[HSK 6]
  \definition{s.}{meios de comunicação de massa; mídia; jornais, rádio, televisão, \emph{Internet} e outras ferramentas de notícias | meio; veículo; vetor; o meio ou via de transmissão da doença}
\end{EntryWithPhonetic}

\begin{EntryWithPhonetic}{传奇}{chuan2qi2}{6,8}{⼈,⼤}[HSK 7-9]
  \definition[个,项,段]{s.}{contos das dinastias Tang e Song (618-1279); contos de maravilhas | dramas poéticos das dinastias Ming e Qing (1368-1911); dramas em verso | lenda; romance; histórias lendárias}
\end{EntryWithPhonetic}

\begin{EntryWithPhonetic}{传染}{chuan2ran3}{6,9}{⼈,⽊}[HSK 7-9]
  \definition{v.}{infectar; contagiar; comunicar; ser contagioso; patógenos que entram em outros organismos a partir de organismos doentes}
\end{EntryWithPhonetic}

\begin{EntryWithPhonetic}{传染病}{chuan2ran3bing4}{6,9,10}{⼈,⽊,⽧}[HSK 7-9]
  \definition{s.}{doença contagiosa; doença infecciosa; pestilência}
\end{EntryWithPhonetic}

\begin{EntryWithPhonetic}{传人}{chuan2ren2}{6,2}{⼈,⼈}[HSK 7-9]
  \definition{s.}{Literátio: alguém que pode herdar uma determinada disciplina acadêmica e fazê-la se espalhar}
  \definition{v.}{transmitir uma habilidade especial, etc.; ensinar}
\end{EntryWithPhonetic}

\begin{EntryWithPhonetic}{传授}{chuan2shou4}{6,11}{⼈,⼿}[HSK 7-9]
  \definition{v.}{transmitir; ensinar; passar adiante (conhecimento, habilidade, etc.); ensinar conhecimento e habilidades aos outros}
\end{EntryWithPhonetic}

\begin{EntryWithPhonetic}{传输}{chuan2 shu1}{6,13}{⼈,⾞}[HSK 6]
  \definition{v.}{transmitir, transportar (energia, informação, etc.)}
\end{EntryWithPhonetic}

\begin{EntryWithPhonetic}{传说}{chuan2shuo1}{6,9}{⼈,⾔}[HSK 3]
  \definition[个,种,段]{s.}{lenda; conto popular; folclore; coisas lendárias; especificamente, lendas populares}
  \definition{v.}{dizer que; ser dito; passar de boca em boca; transmitir oralmente, segundo a tradição}
\end{EntryWithPhonetic}

\begin{EntryWithPhonetic}{传统}{chuan2tong3}{6,9}{⼈,⽷}[HSK 4]
  \definition{adj.}{tradicional; histórico; transmitido de geração em geração | antiquado, conservador e fora de sintonia com os tempos}
  \definition[个,种,项]{s.}{tradição; costume; fatores sociais, como costumes, moral, ideias, estilos, artes, instituições etc., que são transmitidos de uma geração para outra e que são característicos da sociedade}
\end{EntryWithPhonetic}

\begin{EntryWithPhonetic}{传闻}{chuan2wen2}{6,9}{⼈,⾨}[HSK 7-9]
  \definition{s.}{boato; rumor; conversa}
  \definition{v.}{diz"-se; eles disseram; notícias não verificadas espalhadas por pessoas}
\end{EntryWithPhonetic}

\begin{EntryWithPhonetic}{传言}{chuan2 yan2}{6,7}{⼈,⾔}[HSK 6]
  \definition[个,种,些]{s.}{boato; rumor}
  \definition{v.}{passar uma mensagem | Obsoleto: fazer um discurso | falar; fazer uma declaração}
\end{EntryWithPhonetic}

\begin{EntryWithPhonetic}{传真}{chuan2zhen1}{6,10}{⼈,⼗}[HSK 5]
  \definition[部,台,份,个]{s.}{\emph{FAX}, facsímile; texto, diagramas, fotografias, etc., transmitidos por aparelho de fax}
  \definition{v.}{enviar um fax | retratar; reproduzir}[她的画传真了古代建筑。===Suas pinturas são reproduções fiéis da arquitetura antiga.]
\end{EntryWithPhonetic}

%%%%%%%%%% 船 %%%%%%%%%%
\subsection*{船}\addcontentsline{loh}{figure}{船 \dpy{chuan2}}

\begin{EntryWithPhonetic}{船}{chuan2}{11}{⾈}[HSK 2]
  \definition*{s.}{Sobrenome: Chuan}
  \definition[条,艘,叶,只]{s.}{barco; navio | embarcação; meio de transporte aquático, nome genérico para embarcações}
\end{EntryWithPhonetic}

\begin{EntryWithPhonetic}{船舶}{chuan2bo2}{11,11}{⾈,⾈}[HSK 7-9]
  \definition[艘,条,只]{s.}{transporte marítimo; barcos e navios; refere"-se a vários navios}
\end{EntryWithPhonetic}

\begin{EntryWithPhonetic}{船桨}{chuan2jiang3}{11,10}{⾈,⽊}[HSK 7-9]
  \definition{s.}{remo}
\end{EntryWithPhonetic}

\begin{EntryWithPhonetic}{船员}{chuan2 yuan2}{11,7}{⾈,⼝}[HSK 6]
  \definition[名,位,个]{s.}{tripulação (do navio) | membro da tripulação (do navio); marinheiro; marujo; barqueiro; velejador}
\end{EntryWithPhonetic}

\begin{EntryWithPhonetic}{船长}{chuan2 zhang3}{11,4}{⾈,⾧}[HSK 6]
  \definition{s.}{capitão do navio; mestre; marinheiro; comandante; o oficial chefe a bordo}
\end{EntryWithPhonetic}

\begin{EntryWithPhonetic}{船只}{chuan2 zhi1}{11,5}{⾈,⼝}[HSK 6]
  \definition[艘,条]{s.}{transporte marítimo; embarcação | navio; veleiro}
\end{EntryWithPhonetic}

%%%%%%%%%% 喘 %%%%%%%%%%
\subsection*{喘}\addcontentsline{loh}{figure}{喘 \dpy{chuan3}}

\begin{EntryWithPhonetic}{喘}{chuan3}{12}{⼝}[HSK 7-9]
  \definition{s.}{Medicina: asma}
  \definition{v.}{respirar pesadamente; ofegar por ar; ofegar | sopro; respiração rápida}
\end{EntryWithPhonetic}

\begin{EntryWithPhonetic}{喘息}{chuan3xi1}{12,10}{⼝,⼼}[HSK 7-9]
  \definition{s.}{sopro; respiração rápida |respirador; pausas curtas durante atividades intensas | síndrome caracterizada por dispneia | (ponto de acupuntura) chuanxi}
  \definition{v.}{ofegar; ofegar por ar | fazer uma pausa para respirar; fazer uma pausa}
\end{EntryWithPhonetic}

%%%%%%%%%% 串 %%%%%%%%%%
\subsection*{串}\addcontentsline{loh}{figure}{串 \dpy{chuan4}}

\begin{EntryWithPhonetic}{串}{chuan4}{7}{⼁}[HSK 6]
  \definition{clas.}{corda; grupo; aglomerado; usado para amarrar coisas}
  \definition{s.}{espeto}
  \definition{v.}{atuar; desempenhar um papel (em uma peça) | misturar as coisas (de maneira caótica) | vagar; correr; ir de um lugar para outro | conspirar (conluio, depreciativo) | encadear juntos; conectar as coisas uma a uma para formar um todo | misturar; refere"-se à mistura de coisas diferentes e à alteração de suas características originais}
\end{EntryWithPhonetic}

\begin{EntryWithPhonetic}{串门}{chuan4/men2}{7,3}{⼁,⾨}[HSK 7-9]
  \definition{v.+compl.}{aparecer; ligar para a casa de alguém; visitar a casa de um vizinho, amigo ou parente para conversar e cumprimentá-lo}
\end{EntryWithPhonetic}

%%%%%%%%%% 创 %%%%%%%%%%
\subsection*{创}\addcontentsline{loh}{figure}{创 \dpy{chuang1}}

\begin{EntryWithPhonetic}{创}{chuang1}{6}{⼑}
  \definition{s.}{ferimento; trauma}
  \seeref{chuang4}
\end{EntryWithPhonetic}

\begin{EntryWithPhonetic}{创伤}{chuang1shang1}{6,6}{⼑,⼈}[HSK 7-9]
  \definition{s.}{ferida; parte do corpo lesionada, geralmente se refere a trauma | trauma; metáfora para dano emocional ou dano material}
\end{EntryWithPhonetic}

%%%%%%%%%% 窗 %%%%%%%%%%
\subsection*{窗}\addcontentsline{loh}{figure}{窗 \dpy{chuang1}}

\begin{EntryWithPhonetic}{窗}{chuang1}{12}{⽳}
  \definition[扇,个]{s.}{janela}
\end{EntryWithPhonetic}

\begin{EntryWithPhonetic}{窗户}{chuang1hu5}{12,4}{⽳,⼾}[HSK 4]
  \definition[个,扇,面,排]{s.}{janela; dispositivo de ventilação e transmissão de luz nas paredes}
\end{EntryWithPhonetic}

\begin{EntryWithPhonetic}{窗口}{chuang1 kou3}{12,3}{⽳,⼝}[HSK 6]
  \definition[个,号]{s.}{janela; em frente à janela; perto da janela | janela; postigo; refere"-se a uma abertura especial em forma de janela | janela; meio; intermediário; peça de exibição; campo de testes; uma metáfora para um lugar com muitas interações com o mundo exterior e através do qual o entendimento mútuo é alcançado |  janela; uma metáfora para um lugar que pode refletir ou exibir a totalidade ou parte de algo |  caixa de diálogo; uma caixa de operação quadrada para aplicativos ou arquivos que aparece na tela do computador}
\end{EntryWithPhonetic}

\begin{EntryWithPhonetic}{窗帘}{chuang1lian2}{12,8}{⽳,⼱}[HSK 5]
  \definition[个,套,片,对]{s.}{cortinas para janelas}
\end{EntryWithPhonetic}

\begin{EntryWithPhonetic}{窗台}{chuang1 tai2}{12,5}{⽳,⼝}[HSK 4]
  \definition{s.}{parapeito da janela; peitoril; parte plana de uma janela que segura a moldura}
\end{EntryWithPhonetic}

\begin{EntryWithPhonetic}{窗子}{chuang1 zi5}{12,3}{⽳,⼦}[HSK 4]
  \definition[扇,个]{s.}{janela}
\end{EntryWithPhonetic}

%%%%%%%%%% 床 %%%%%%%%%%
\subsection*{床}\addcontentsline{loh}{figure}{床 \dpy{chuang2}}

\begin{EntryWithPhonetic}{床}{chuang2}{7}{⼴}[HSK 1]
  \definition{clas.}{usado para colchas, roupas de cama, etc.}
  \definition[张]{s.}{cama; sofá; móveis para dormir | algo com o formato de uma cama}
\end{EntryWithPhonetic}

\begin{EntryWithPhonetic}{床位}{chuang2wei4}{7,7}{⼴,⼈}[HSK 7-9]
  \definition{s.}{beliche; cama; camas para pacientes, viajantes e hóspedes em hospitais, navios e dormitórios}
\end{EntryWithPhonetic}

%%%%%%%%%% 闯 %%%%%%%%%%
\subsection*{闯}\addcontentsline{loh}{figure}{闯 \dpy{chuang3}}

\begin{EntryWithPhonetic}{闯}{chuang3}{6}{⾨}[HSK 5]
  \definition*{s.}{Sobrenome: Chuang}
  \definition{v.}{apressar"-se; correr | moderar a si mesmo (lutando contra dificuldades e perigos); aventurar"-se no mundo | incorrer; causar (um desastre, etc.)}
\end{EntryWithPhonetic}

%%%%%%%%%% 创 %%%%%%%%%%
\subsection*{创}\addcontentsline{loh}{figure}{创 \dpy{chuang4}}

\begin{EntryWithPhonetic}{创}{chuang4}{6}{⼑}[HSK 7-9]
  \definition{v.}{começar (fazer algo); alcançar (algo pela primeira vez); estabelecer; fazer pela primeira vez | estabelecer; fundar; criar; perceber algo novo, como um começo | ferir; machucar}
  \seeref{chuang1}
\end{EntryWithPhonetic}

\begin{EntryWithPhonetic}{创办}{chuang4 ban4}{6,4}{⼑,⼒}[HSK 6]
  \definition{v.}{estabelecer; montar; fundar}
\end{EntryWithPhonetic}

\begin{EntryWithPhonetic}{创建}{chuang4 jian4}{6,8}{⼑,⼵}[HSK 6]
  \definition{v.}{fundar; estabelecer; montar}
\end{EntryWithPhonetic}

\begin{EntryWithPhonetic}{创立}{chuang4li4}{6,5}{⼑,⽴}[HSK 5]
  \definition{v.}{fundar; originar; estabelecer}
\end{EntryWithPhonetic}

\begin{EntryWithPhonetic}{创始人}{chuang4shi3ren2}{6,8,2}{⼑,⼥,⼈}[HSK 7-9]
  \definition{s.}{fundador; originador; iniciador | criador}
\end{EntryWithPhonetic}

\begin{EntryWithPhonetic}{创新}{chuang4xin1}{6,13}{⼑,⽄}[HSK 3]
  \definition[个,种,次]{s.}{inovação; algo novo ou diferente, uma ideia}
  \definition{v.}{trazer novas ideias; inovar; abrir novos caminhos; criar ou fazer algo novo, diferente do que era antes}
\end{EntryWithPhonetic}

\begin{EntryWithPhonetic}{创业}{chuang4ye4}{6,5}{⼑,⼀}[HSK 3]
  \definition{s.}{empreendedorismo}
  \definition{v.}{começar um empreendimento; iniciar/fundar um negócio, uma empresa;}
\end{EntryWithPhonetic}

\begin{EntryWithPhonetic}{创意}{chuang4 yi4}{6,13}{⼑,⼼}[HSK 6]
  \definition[个]{s.}{criatividade; originalidade; novidade; uma ideia original, conceito, etc.}
  \definition{v.}{inovar; criar um novo conceito, ideia, etc. | propor designs criativos, ideias, etc.}
\end{EntryWithPhonetic}

\begin{EntryWithPhonetic}{创造}{chuang4zao4}{6,10}{⼑,⾡}[HSK 3]
  \definition{s.}{criação; inovação; primeiro a concluir ou a alcançar resultados}
  \definition{v.}{criar; produzir; trazer à tona; fazer ou estabelecer pela primeira vez; referir"-se de maneira geral a fazer ou estabelecer}
\end{EntryWithPhonetic}

\begin{EntryWithPhonetic}{创作}{chuang4zuo4}{6,7}{⼑,⼈}[HSK 3]
  \definition[个]{s.}{criação; trabalho criativo; obras literárias e artísticas}
  \definition{v.}{escrever; criar; produzir; compor; criar obras artísticas}
\end{EntryWithPhonetic}

%%%%%%%%%% 吹 %%%%%%%%%%
\subsection*{吹}\addcontentsline{loh}{figure}{吹 \dpy{chui1}}

\begin{EntryWithPhonetic}{吹}{chui1}{7}{⼝}[HSK 2]
  \definition{v.}{soprar; baforar | tocar (instrumentos de sopro) | (do vento) soprar | gabar"-se; vangloriar"-se | elogiar; louvar aos céus; adular | (relacionamento) romper; separar"-se; (assunto) fracassar}
\end{EntryWithPhonetic}

\begin{EntryWithPhonetic}{吹了}{chui1 le5}{7,2}{⼝,⼅}[HSK 7-9]
  \definition{v.}{quebrar; falhar | ter esfriado (um relacionamento) | ter morrido | não ter tido sucesso | ter se separado}
\end{EntryWithPhonetic}

\begin{EntryWithPhonetic}{吹牛}{chui1/niu2}{7,4}{⼝,⽜}[HSK 7-9]
  \definition{v.+compl.}{gabar"-se; vangloriar"-se; falar alto; falar com arrogância; falar sem parar}
\end{EntryWithPhonetic}

\begin{EntryWithPhonetic}{吹捧}{chui1peng3}{7,11}{⼝,⼿}[HSK 7-9]
  \definition{v.}{bajular; elogiar até os céus; elogiar abundantemente}
\end{EntryWithPhonetic}

%%%%%%%%%% 垂 %%%%%%%%%%
\subsection*{垂}\addcontentsline{loh}{figure}{垂 \dpy{chui2}}

\begin{EntryWithPhonetic}{垂}{chui2}{8}{⼠}[HSK 7-9]
  \definition{adv.}{perto de; uma palavra respeitosa usada para se referir às ações de outros (geralmente anciãos ou superiores) em relação a si mesmo.}
  \definition{v.}{cair; deixar cair; pendurar (objeto de cabeça para baixo)  | Literário: (mais velhos ou superiores) condescender; um termo respeitoso usado para se referir às ações de outros (geralmente anciãos ou superiores) em relação a si mesmo | transmitir; legar à posteridade; passar para as gerações posteriores}
\end{EntryWithPhonetic}

\begin{EntryWithPhonetic}{垂头丧气}{chui2tou2-sang4qi4}{8,5,8,4}{⼠,⼤,⼗,⽓}[HSK 7-9]
  \definition{v.}{estar desanimado; estar abatido; abaixar a cabeça com um olhar abatido}
\end{EntryWithPhonetic}

%%%%%%%%%% 捶 %%%%%%%%%%
\subsection*{捶}\addcontentsline{loh}{figure}{捶 \dpy{chui2}}

\begin{EntryWithPhonetic}{捶}{chui2}{11}{⼿}[HSK 7-9]
  \definition{v.}{bater (com um pedaço de pau, martelo ou punho)}
\end{EntryWithPhonetic}

\begin{EntryWithPhonetic}{捶子}{chui2zi5}{11,3}{⼿,⼦}
  \definition[把]{s.}{martelo}
\end{EntryWithPhonetic}

%%%%%%%%%% 锤 %%%%%%%%%%
\subsection*{锤}\addcontentsline{loh}{figure}{锤 \dpy{chui2}}

\begin{EntryWithPhonetic}{锤}{chui2}{13}{⾦}
  \definition[把,个]{s.}{uma bola de metal com uma alça ou corrente, usada como arma; maça | algo como um martelo | martelo}
  \definition{v.}{martelar para dar forma; bater com um martelo}
\end{EntryWithPhonetic}

\begin{EntryWithPhonetic}{锤子}{chui2zi5}{13,3}{⾦,⼦}[HSK 7-9]
  \definition[把,个]{s.}{martelo; a ferramenta para golpear coisas tem uma cabeça de ferro e um cabo perpendicular à cabeça}
\end{EntryWithPhonetic}

%%%%%%%%%% 春 %%%%%%%%%%
\subsection*{春}\addcontentsline{loh}{figure}{春 \dpy{chun1}}

\begin{EntryWithPhonetic}{春}{chun1}{9}{⽇}
  \definition*{s.}{Sobrenome: Chun}
  \definition{s.}{primavera | amor; luxúria | vida; vitalidade}
\end{EntryWithPhonetic}

\begin{EntryWithPhonetic}{春季}{chun1 ji4}{9,8}{⽇,⼦}[HSK 4]
  \definition{s.}{primavera; primeiro trimestre do ano, que na China se refere ao período de três meses entre o início da primavera e o início do verão, e também se refere aos três meses do calendário lunar, a saber, o primeiro, o segundo e o terceiro meses}
\end{EntryWithPhonetic}

\begin{EntryWithPhonetic}{春节}{chun1 jie2}{9,5}{⽇,⾋}[HSK 2]
  \definition*[个]{s.}{Festival da Primavera (Ano Novo Chinês); o primeiro dia do primeiro mês do calendário lunar, também se refere aos dias seguintes ao primeiro dia do primeiro mês}
\end{EntryWithPhonetic}

\begin{EntryWithPhonetic}{春天}{chun1 tian1}{9,4}{⽇,⼤}
  \definition[个,段,季,番]{s.}{primavera; época da primavera | primavera; renascimento; uma atmosfera cheia de energia e esperança}
\end{EntryWithPhonetic}

%%%%%%%%%% 纯 %%%%%%%%%%
\subsection*{纯}\addcontentsline{loh}{figure}{纯 \dpy{chun2}}

\begin{EntryWithPhonetic}{纯}{chun2}{7}{⽷}[HSK 4]
  \definition{adj.}{puro; não misturado; livre de impurezas | simples; puro e simples | habilidoso; proficiente; bem versado}
  \definition{adv.}{puramente; completamente; totalmente | genuinamente}
\end{EntryWithPhonetic}

\begin{EntryWithPhonetic}{纯粹}{chun2cui4}{7,14}{⽷,⽶}[HSK 7-9]
  \definition{adj.}{puro; sem mistura; não adulterado; sem outros ingredientes}
  \definition{adv.}{unicamente; puramente; somente; simplesmente, apenas}
\end{EntryWithPhonetic}

\begin{EntryWithPhonetic}{纯洁}{chun2jie2}{7,9}{⽷,⽔}[HSK 7-9]
  \definition{adj.}{puro; limpo e honesto; sem impurezas e manchas; metáfora para pensamentos puros, sem pensamentos egoístas}
  \definition{v.}{purificar}
\end{EntryWithPhonetic}

\begin{EntryWithPhonetic}{纯净水}{chun2 jing4 shui3}{7,8,4}{⽷,⼎,⽔}[HSK 4]
  \definition{s.}{água pura | água purificada}
\end{EntryWithPhonetic}

\begin{EntryWithPhonetic}{纯朴}{chun2pu3}{7,6}{⽷,⽊}[HSK 7-9]
  \definition{adj.}{honesto; simples; pouco sofisticado}
\end{EntryWithPhonetic}

\begin{EntryWithPhonetic}{纯真}{chun2zhen1}{7,10}{⽷,⼗}
  \definition{adj.}{inocente e não afetado | puro e não adulterado}
  \definition{s.}{inocência}
\end{EntryWithPhonetic}

%%%%%%%%%% 唇 %%%%%%%%%%
\subsection*{唇}\addcontentsline{loh}{figure}{唇 \dpy{chun2}}

\begin{EntryWithPhonetic}{唇}{chun2}{10}{⼝}
  \definition[片]{s.}{lábios}
\end{EntryWithPhonetic}

%%%%%%%%%% 醇 %%%%%%%%%%
\subsection*{醇}\addcontentsline{loh}{figure}{醇 \dpy{chun2}}

\begin{EntryWithPhonetic}{醇}{chun2}{15}{⾣}
  \definition{adj.}{Literário: puro; puro e suave; não misturado}
  \definition{s.}{Literário: vinho suave; bom vinho ; Química: álcool}
\end{EntryWithPhonetic}

\begin{EntryWithPhonetic}{醇厚}{chun2hou4}{15,9}{⾣,⼚}[HSK 7-9]
  \definition{adj.}{suave; rico; cheiro e sabor puros e ricos | puro e honesto; simples e gentil}
\end{EntryWithPhonetic}

%%%%%%%%%% 蠢 %%%%%%%%%%
\subsection*{蠢}\addcontentsline{loh}{figure}{蠢 \dpy{chun3}}

\begin{EntryWithPhonetic}{蠢}{chun3}{21}{⾍}[HSK 7-9]
  \definition{adj.}{estúpido; tolo; sem graça; desajeitado}
  \definition{v.}{Literário: contorcer"-se}
\end{EntryWithPhonetic}

%%%%%%%%%% 戳 %%%%%%%%%%
\subsection*{戳}\addcontentsline{loh}{figure}{戳 \dpy{chuo1}}

\begin{EntryWithPhonetic}{戳}{chuo1}{18}{⼽}[HSK 7-9]
  \definition{s.}{selo; carimbo, abreviação de 戳记}
  \definition{v.}{cutucar; esfaquear | Dialeto: torcer; embotar | Dialeto: ficar em pé}
  \seealsoref{戳记}{chuo1ji4}
\end{EntryWithPhonetic}

\begin{EntryWithPhonetic}{戳记}{chuo1ji4}{18,5}{⼽,⾔}
  \definition{s.}{carimbo; selo}
\end{EntryWithPhonetic}

%%%%%%%%%% 绰 %%%%%%%%%%
\subsection*{绰}\addcontentsline{loh}{figure}{绰 \dpy{chuo4}}

\begin{EntryWithPhonetic}{绰}{chuo4}{11}{⽷}
  \definition{adj.}{amplo; espaçoso | (do porte de uma menina) graciosa; flexível}
\end{EntryWithPhonetic}

\begin{EntryWithPhonetic}{绰号}{chuo4hao4}{11,5}{⽷,⼝}[HSK 7-9]
  \definition[个]{s.}{apelido; um nome informal dado a alguém com base em suas características, muitas vezes expressando afeição, antipatia ou brincadeira; também chamado de 外号}
  \seealsoref{外号}{wai4hao4}
\end{EntryWithPhonetic}

%%%%%%%%%% 刺 %%%%%%%%%%
\subsection*{刺}\addcontentsline{loh}{figure}{刺 \dpy{ci1}}

\begin{EntryWithPhonetic}{刺}{ci1}{8}{⼑}
  \definition{s.}{(onomatopéia) som de rasgo, fricção, etc.}
  \seeref{ci4}
\end{EntryWithPhonetic}

%%%%%%%%%% 词 %%%%%%%%%%
\subsection*{词}\addcontentsline{loh}{figure}{词 \dpy{ci2}}

\begin{EntryWithPhonetic}{词}{ci2}{7}{⾔}[HSK 2]
  \definition[个,组,句,段,首]{s.}{palavra; termo; antigamente, referia"-se a palavras vazias; atualmente, refere"-se a palavras com forma fonética fixa e significado específico na língua; a menor unidade que pode ser usada de forma independente | discurso; declaração; linguagem; texto | ci (um tipo de poesia clássica chinesa, originária da dinastia Tang e plenamente desenvolvida na dinastia Song); gênero poético escrito de acordo com uma estrutura fixa, com versos de comprimentos variados | palavras; redação; refere"-se genericamente ao teatro; a parte da letra cantada em harmonia com a melodia em canções e certas artes vocais}
\end{EntryWithPhonetic}

\begin{EntryWithPhonetic}{词典}{ci2 dian3}{7,8}{⾔,⼋}[HSK 2]
  \definition[本,部]{s.}{dicionário, livro de referência que reúne palavras e explicações para consulta}
  \seealsoref{字典}{zi4 dian3}
\end{EntryWithPhonetic}

\begin{EntryWithPhonetic}{词汇}{ci2hui4}{7,5}{⾔,⽔}[HSK 4]
  \definition[个,组,批,串,堆]{s.}{vocabulário; termo geral para palavras usadas em um idioma}
\end{EntryWithPhonetic}

\begin{EntryWithPhonetic}{词语}{ci2yu3}{7,9}{⾔,⾔}[HSK 2]
  \definition[个,租]{s.}{termo; palavra; expressão; conjunto de palavras e frases}
\end{EntryWithPhonetic}

%%%%%%%%%% 瓷 %%%%%%%%%%
\subsection*{瓷}\addcontentsline{loh}{figure}{瓷 \dpy{ci2}}

\begin{EntryWithPhonetic}{瓷}{ci2}{10}{⽡}[HSK 7-9]
  \definition{adj.}{Dialeto: (relação) próxima; íntima}
  \definition{s.}{artigos de porcelana}
\end{EntryWithPhonetic}

\begin{EntryWithPhonetic}{瓷器}{ci2qi4}{10,16}{⽡,⼝}[HSK 7-9]
  \definition[件,种]{s.}{porcelana; louça; utensílios feitos de argila de porcelana, feldspato, quartzo, etc.}
\end{EntryWithPhonetic}

%%%%%%%%%% 慈 %%%%%%%%%%
\subsection*{慈}\addcontentsline{loh}{figure}{慈 \dpy{ci2}}

\begin{EntryWithPhonetic}{慈}{ci2}{13}{⼼}
  \definition*{s.}{Sobrenome: Ci}
  \definition{adj.}{gentil; amoroso}
  \definition{s.}{mãe; refere"-se à mãe}
  \definition{v.}{Literário: amar (amor de cima para baixo)}
\end{EntryWithPhonetic}

\begin{EntryWithPhonetic}{慈善}{ci2shan4}{13,12}{⼼,⼝}[HSK 7-9]
  \definition{adj.}{caridoso; benevolente; filantrópico; gentil e caridoso}
\end{EntryWithPhonetic}

\begin{EntryWithPhonetic}{慈祥}{ci2xiang2}{13,10}{⼼,⽰}[HSK 7-9]
  \definition{adj.}{gentil; benigno; amável; descreve a aparência ou atitude de uma pessoa idosa como sendo gentil, amável e acessível}
\end{EntryWithPhonetic}

%%%%%%%%%% 辞 %%%%%%%%%%
\subsection*{辞}\addcontentsline{loh}{figure}{辞 \dpy{ci2}}

\begin{EntryWithPhonetic}{辞}{ci2}{13}{⾟}[HSK 7-9]
  \definition[首]{s.}{dicção; fraseologia | um tipo de literatura clássica chinesa; um gênero da literatura clássica | uma forma de poesia clássica}
  \definition{v.}{despedir"-se | declinar | renunciar | dispensar; demitir | fugir; evitar}
\end{EntryWithPhonetic}

\begin{EntryWithPhonetic}{辞呈}{ci2cheng2}{13,7}{⾟,⼝}[HSK 7-9]
  \definition{s.}{renúncia (por escrito)}
\end{EntryWithPhonetic}

\begin{EntryWithPhonetic}{辞典}{ci2 dian3}{13,8}{⾟,⼋}[HSK 5]
  \definition[本,部]{s.}{dicionário; coleção de termos especializados ou enciclopédicos, organizados em uma determinada ordem e explicados, para fins de referência}
  \variantof{词典}
\end{EntryWithPhonetic}

\begin{EntryWithPhonetic}{辞去}{ci2qu4}{13,5}{⾟,⼛}[HSK 7-9]
  \definition{v.}{desistir | renunciar}
\end{EntryWithPhonetic}

\begin{EntryWithPhonetic}{辞退}{ci2tui4}{13,9}{⾟,⾡}[HSK 7-9]
  \definition{v.}{dispensar; demitir; a unidade ou instituição toma a iniciativa de encerrar a relação de trabalho com o funcionário | recusar; retornar educadamente}
\end{EntryWithPhonetic}

\begin{EntryWithPhonetic}{辞职}{ci2/zhi2}{13,11}{⾟,⽿}[HSK 5]
  \definition{v.+compl.}{renunciar; deixar o cargo; entregar a renúncia; pedir para ser dispensado de suas funções}
\end{EntryWithPhonetic}

%%%%%%%%%% 磁 %%%%%%%%%%
\subsection*{磁}\addcontentsline{loh}{figure}{磁 \dpy{ci2}}

\begin{EntryWithPhonetic}{磁}{ci2}{14}{⽯}
  \definition[块]{s.}{porcelana | (física) magnetismo; propriedade de atrair ferro, níquel, etc. | (dialeto)  (de relação) próximo; íntimo}
\end{EntryWithPhonetic}

\begin{EntryWithPhonetic}{磁带}{ci2dai4}{14,9}{⽯,⼱}[HSK 7-9]
  \definition[盘,盒,卷]{s.}{fita; fita magnética; cassete; uma fita plástica tratada com material magnético que pode gravar som ou imagens}
\end{EntryWithPhonetic}

\begin{EntryWithPhonetic}{磁卡}{ci2ka3}{14,5}{⽯,⼘}[HSK 7-9]
  \definition[张]{s.}{cartão magnético}
\end{EntryWithPhonetic}

\begin{EntryWithPhonetic}{磁盘}{ci2pan2}{14,11}{⽯,⽫}[HSK 7-9]
  \definition{s.}{Computação: disco; disquete; um disco é um dispositivo de armazenamento que usa tecnologia de gravação magnética para armazenar dados}
\end{EntryWithPhonetic}

\begin{EntryWithPhonetic}{磁铁}{ci2tie3}{14,10}{⽯,⾦}
  \definition{s.}{imã | magneto}
  \seealsoref{吸铁石}{xi1tie3shi2}
\end{EntryWithPhonetic}

%%%%%%%%%% 此 %%%%%%%%%%
\subsection*{此}\addcontentsline{loh}{figure}{此 \dpy{ci3}}

\begin{EntryWithPhonetic}{此}{ci3}{6}{⽌}[HSK 4]
  \definition*{s.}{Sobrenome: Ci}
  \definition{pron.}{esse; essa; isso; este; esta; isto; indica ou se refere a uma pessoa ou coisa que está mais próxima, equivalente a 这 ou 这个 | aqui e agora; refere"-se a um tempo ou lugar recente, equivalente a 这会儿 ou 这里}
  \seealsoref{这}{zhe4}
  \seealsoref{这会儿}{zhe4 hui4r5}
  \seealsoref{这里}{zhe4 li3}
  \seealsoref{这个}{zhe4ge5}
  \antonymref{彼}{bi3}
\end{EntryWithPhonetic}

\begin{EntryWithPhonetic}{此处}{ci3 chu4}{6,5}{⽌,⼡}[HSK 6]
  \definition{pron.}{este lugar; aqui (literário)}
\end{EntryWithPhonetic}

\begin{EntryWithPhonetic}{此次}{ci3 ci4}{6,6}{⽌,⽋}[HSK 6]
  \definition{adv.}{desta vez; refere"-se a um ponto específico no tempo ou período de tempo}
\end{EntryWithPhonetic}

\begin{EntryWithPhonetic}{此后}{ci3 hou4}{6,6}{⽌,⼝}[HSK 5]
  \definition{s.}{daqui em diante; doravante; depois disso; após isso; de agora em diante}
\end{EntryWithPhonetic}

\begin{EntryWithPhonetic}{此刻}{ci3 ke4}{6,8}{⽌,⼑}[HSK 5]
  \definition{s.}{agora; no momento; exatamente agora; neste momento}
\end{EntryWithPhonetic}

\begin{EntryWithPhonetic}{此起彼伏}{ci3qi3-bi3fu2}{6,10,8,6}{⽌,⾛,⼻,⼈}[HSK 7-9]
  \definition{expr.}{``Quando um cai, outro se levanta.''  ou se levanta um após o outro | ``Assim que um desaparece, o próximo surge.'' | ocorrendo repetidamente (de aplausos, incêndios, acenos, protestos, conflitos, revoltas etc.) | repetindo continuamente | aqui em cima, lá embaixo; subir e descer em sucessão}
\end{EntryWithPhonetic}

\begin{EntryWithPhonetic}{此前}{ci3 qian2}{6,9}{⽌,⼑}[HSK 6]
  \definition{adv.}{literário: antes; anteriormente | literário: antes disso}
\end{EntryWithPhonetic}

\begin{EntryWithPhonetic}{此时}{ci3 shi2}{6,7}{⽌,⽇}[HSK 5]
  \definition{s.}{agora; no presente; agora mesmo; neste momento; por enquanto}
\end{EntryWithPhonetic}

\begin{EntryWithPhonetic}{此事}{ci3 shi4}{6,8}{⽌,⼅}[HSK 6]
  \definition{s.}{matéria; assunto}
\end{EntryWithPhonetic}

\begin{EntryWithPhonetic}{此外}{ci3wai4}{6,5}{⽌,⼣}[HSK 4]
  \definition{conj.}{além disso; em adição; além das coisas ou situações mencionadas acima}
\end{EntryWithPhonetic}

\begin{EntryWithPhonetic}{此致}{ci3 zhi4}{6,10}{⽌,⾄}[HSK 6]
  \definition{expr.}{Atenciosamente; Sinceramente; Com os melhores votos; usada no final de uma carta ou correspondência oficial}
\end{EntryWithPhonetic}

%%%%%%%%%% 次 %%%%%%%%%%
\subsection*{次}\addcontentsline{loh}{figure}{次 \dpy{ci4}}

\begin{EntryWithPhonetic}{次}{ci4}{6}{⽋}[HSK 1,4]
  \definition*{s.}{Sobrenome: Ci}
  \definition{adj.}{de segunda categoria; de qualidade inferior}
  \definition{clas.}{usado para coisas ou ações que podem ser repetidas}
  \definition{num.}{segundo; próximo}
  \definition{pref.}{(química) hipo-, radical ácido ou composto contendo dois átomos de oxigênio a menos}
  \definition{s.}{ordem; sequência; classificação | local de parada em uma viagem; escala}
\end{EntryWithPhonetic}

\begin{EntryWithPhonetic}{次日}{ci4ri4}{6,4}{⽋,⽇}[HSK 7-9]
  \definition{s.}{dia seguinte; amanhã}
\end{EntryWithPhonetic}

\begin{EntryWithPhonetic}{次数}{ci4 shu4}{6,13}{⽋,⽁}[HSK 6]
  \definition{s.}{frequência; número de vezes; o número de vezes que uma ação ou evento é repetido}
\end{EntryWithPhonetic}

%%%%%%%%%% 伺 %%%%%%%%%%
\subsection*{伺}\addcontentsline{loh}{figure}{伺 \dpy{ci4}}

\begin{EntryWithPhonetic}{伺}{ci4}{7}{⼈}
  \definition{v.}{servir}
  \seeref{si4}
  \seealsoref{伺候}{ci4hou5}
\end{EntryWithPhonetic}

\begin{EntryWithPhonetic}{伺候}{ci4hou5}{7,10}{⼈,⼈}[HSK 7-9]
  \definition{v.}{servir; esperar; estar à disposição de alguém; cuidar de}
\end{EntryWithPhonetic}

%%%%%%%%%% 刺 %%%%%%%%%%
\subsection*{刺}\addcontentsline{loh}{figure}{刺 \dpy{ci4}}

\begin{EntryWithPhonetic}{刺}{ci4}{8}{⼑}[HSK 4]
  \definition*{s.}{Sobrenome: Ci}
  \definition{s.}{espinho; farpa; algo afiado como uma agulha | cartão de visita | saliências; projeções pequenas e pontiagudas na superfície de um objeto ou na pele de uma pessoa}
  \definition{v.}{esfaquear; perfurar | irritar; estimular | assassinar | espionar; detectar | criticar}
  \seeref{ci1}
\end{EntryWithPhonetic}

\begin{EntryWithPhonetic}{刺耳}{ci4'er3}{8,6}{⼑,⽿}[HSK 7-9]
  \definition{adj.}{irritante (desagradável) para o ouvido; estridente; penetrante; áspero}
\end{EntryWithPhonetic}

\begin{EntryWithPhonetic}{刺骨}{ci4gu3}{8,9}{⼑,⾻}[HSK 7-9]
  \definition{adj.}{perfurante (até os ossos); cortante}
\end{EntryWithPhonetic}

\begin{EntryWithPhonetic}{刺激}{ci4ji1}{8,16}{⼑,⽔}[HSK 4]
  \definition{adj.}{animado; entusiasmado; sensação de empolgação e nervosismo}
  \definition[个]{s.}{estímulo; estimulação; fortes efeitos físicos ou psicológicos}
  \definition{v.}{irritar; provocar; estimular | incentivar; estimular; incitar; (por algum meio) para mudar as coisas para melhor, para fazer coisas positivas}
\end{EntryWithPhonetic}

\begin{EntryWithPhonetic}{刺猬}{ci4wei5}{8,12}{⼑,⽝}
  \definition{s.}{porco-espinho | ouriço}
\end{EntryWithPhonetic}

\begin{EntryWithPhonetic}{刺绣}{ci4xiu4}{8,10}{⼑,⽷}[HSK 7-9]
  \definition{s.}{bordado; um artesanato popular tradicional que usa fios de seda coloridos para bordar padrões ou imagens em tecidos; produtos bordados}
  \definition{v.}{bordar; bordar padrões ou imagens em tecido usando fios de seda coloridos}
\end{EntryWithPhonetic}

%%%%%%%%%% 赐 %%%%%%%%%%
\subsection*{赐}\addcontentsline{loh}{figure}{赐 \dpy{ci4}}

\begin{EntryWithPhonetic}{赐}{ci4}{12}{⾙}[HSK 7-9]
  \definition{s.}{favor; bênção; presente; uma recompensa, um benefício concedido}
  \definition{v.}{conceder; conferir; favorecer; presentear; dar, antigamente, se referia ao superior dando ao subordinado ou ao mais velho dando ao mais novo | responder; dar conselho; instruir; uma palavra usada para demonstrar respeito; quando alguém lhe dá instruções, responde ou lhe entrega algo}
\end{EntryWithPhonetic}

\begin{EntryWithPhonetic}{赐教}{ci4jiao4}{12,11}{⾙,⽁}[HSK 7-9]
  \definition{v.}{condescender em ensinar; conceder instrução | importa"-se em me esclarecer com suas instruções}
\end{EntryWithPhonetic}

%%%%%%%%%% 匆 %%%%%%%%%%
\subsection*{匆}\addcontentsline{loh}{figure}{匆 \dpy{cong1}}

\begin{EntryWithPhonetic}{匆}{cong1}{5}{⼓}
  \definition{adj.}{apressado}
  \definition{adv.}{apressadamente}
\end{EntryWithPhonetic}

\begin{EntryWithPhonetic}{匆匆}{cong1cong1}{5,5}{⼓,⼓}[HSK 7-9]
  \definition{adj.}{apressado; com pressa}
\end{EntryWithPhonetic}

\begin{EntryWithPhonetic}{匆忙}{cong1mang2}{5,6}{⼓,⼼}[HSK 7-9]
  \definition{adj.}{apressado; com pressa; ocupado}
\end{EntryWithPhonetic}

%%%%%%%%%% 葱 %%%%%%%%%%
\subsection*{葱}\addcontentsline{loh}{figure}{葱 \dpy{cong1}}

\begin{EntryWithPhonetic}{葱}{cong1}{12}{⾋}[HSK 7-9]
  \definition{adj.}{verde; turquesa}
  \definition[根,把,捆]{s.}{cebola; cebolinha}
\end{EntryWithPhonetic}

%%%%%%%%%% 聪 %%%%%%%%%%
\subsection*{聪}\addcontentsline{loh}{figure}{聪 \dpy{cong1}}

\begin{EntryWithPhonetic}{聪}{cong1}{15}{⽿}
  \definition{adj.}{audição aguçada | brilhante; inteligente; esperto | perspicaz}
  \definition{s.}{(literário) faculdades auditivas}
\end{EntryWithPhonetic}

\begin{EntryWithPhonetic}{聪慧}{cong1hui4}{15,15}{⽿,⼼}
  \definition{adj.}{inteligente | brilhante}
\end{EntryWithPhonetic}

\begin{EntryWithPhonetic}{聪明}{cong1ming5}{15,8}{⽿,⽇}[HSK 5]
  \definition{adj.}{brilhante; esperto; inteligente; intelecto bem desenvolvido com boa memória e capacidade de compreensão}
\end{EntryWithPhonetic}

%%%%%%%%%% 从 %%%%%%%%%%
\subsection*{从}\addcontentsline{loh}{figure}{从 \dpy{cong2}}

\begin{EntryWithPhonetic}{从}{cong2}{4}{⼈}[HSK 1]
  \definition*{s.}{Sobrenome: Cong}
  \definition{adj.}{secundário; acessório | relacionamento entre primos, etc., do mesmo avô paterno, bisavô ou de um ancestral comum ainda mais antigo; do mesmo clã}
  \definition{adv.}{(seguido de uma negativa) jamais | jamais; usado antes de palavras negativas, indica que algo nunca aconteceu desde o passado, equivalente a 从来}
  \definition{prep.}{de (um tempo, um lugar ou um ponto de vista) | via, através ou após (um local) | de; via; através de; passado (um lugar); introdução das ações, trajetórias e locais| (de comportamento) de; introdução de ações e comportamentos com base em referências e fundamentos}
  \definition{s.}{seguidor; acompanhante}
  \definition{v.}{seguir; cumprir; obedecer | participar; estar envolvido em | seguir o princípio de; empregar o método de | estar envolvido em | seguir; adotar (um determinado princípio ou atitude)}
  \seealsoref{从来}{cong2lai2}
\end{EntryWithPhonetic}

\begin{EntryWithPhonetic}{从不}{cong2bu4}{4,4}{⼈,⼀}[HSK 6]
  \definition{adv.}{nunca; impossível (para expressar surpresa ou choque)}
\end{EntryWithPhonetic}

\begin{EntryWithPhonetic}{从此}{cong2ci3}{4,6}{⼈,⽌}[HSK 4]
  \definition{conj.}{doravante; portanto; a partir deste momento; de agora em diante; a partir de então}
\end{EntryWithPhonetic}

\begin{EntryWithPhonetic}{从而}{cong2'er2}{4,6}{⼈,⽽}[HSK 5]
  \definition{conj.}{assim; por isso; portanto; desse modo; por esse motivo; conjunção usada no início do texto seguinte para expressar o resultado, propósito ou ação posterior, o que é equivalente a 因此就}
  \seealsoref{因此就}{yin1ci3 jiu4}
\end{EntryWithPhonetic}

\begin{EntryWithPhonetic}{从今以后}{cong2 jin1 yi3hou4}{4,4,4,6}{⼈,⼈,⼈,⼝}[HSK 7-9]
  \definition{conj.}{de agora em diante; doravante}
\end{EntryWithPhonetic}

\begin{EntryWithPhonetic}{从来}{cong2lai2}{4,7}{⼈,⽊}[HSK 3]
  \definition{adv.}{sempre; o tempo todo; em todos os momentos; do passado até o presente}
\end{EntryWithPhonetic}

\begin{EntryWithPhonetic}{从来不}{cong2lai2 bu4}{4,7,4}{⼈,⽊,⼀}[HSK 7-9]
  \definition{adv.}{nunca; indica que algo não foi feito no passado ou no presente}
\end{EntryWithPhonetic}

\begin{EntryWithPhonetic}{从没}{cong2 mei2}{4,7}{⼈,⽔}[HSK 6]
  \definition{adv.}{nunca (no passado)}
\end{EntryWithPhonetic}

\begin{EntryWithPhonetic}{从前}{cong2qian2}{4,9}{⼈,⼑}[HSK 3]
  \definition{s.}{antes; antigamente; no passado | era uma vez; há muito tempo atrás}
\end{EntryWithPhonetic}

\begin{EntryWithPhonetic}{从容}{cong2rong2}{4,10}{⼈,⼧}[HSK 7-9]
  \definition{adj.}{calmo; sem pressa; ao se deparar com situações especiais, sua expressão e atitude são calmas, sem nervosismo ou agitação | abundante (tempo, dinheiro); tempo ou dinheiro suficiente}
\end{EntryWithPhonetic}

\begin{EntryWithPhonetic}{从容不迫}{cong2rong2-bu2po4}{4,10,4,8}{⼈,⼧,⼀,⾡}[HSK 7-9]
  \definition{expr.}{ir com calma e sem pressa; em etapas fáceis; calmo e sem pressa; levar o seu tempo; calmamente; vagarosamente; confiantemente e sem pressa; ir com calma; autoconfiante; de maneira vagarosa}
\end{EntryWithPhonetic}

\begin{EntryWithPhonetic}{从事}{cong2shi4}{4,8}{⼈,⼅}[HSK 3]
  \definition{v.}{trabalhar; empreender; empenhar"-se em; envolver-se em; dedicar"-se ou comprometer"-se (a uma causa); participar (de algo) | lidar com; deve ter um advérbio antes e não pode ter um objeto depois}
\end{EntryWithPhonetic}

\begin{EntryWithPhonetic}{从头}{cong2tou2}{4,5}{⼈,⼤}[HSK 7-9]
  \definition{adv.}{desde o começo | de novo; mais uma vez}
\end{EntryWithPhonetic}

\begin{EntryWithPhonetic}{从未}{cong2wei4}{4,5}{⼈,⽊}[HSK 7-9]
  \definition{adv.}{nunca}
\end{EntryWithPhonetic}

\begin{EntryWithPhonetic}{从小}{cong2 xiao3}{4,3}{⼈,⼩}[HSK 2]
  \definition{adv.}{desde a infância; desde muito jovem; quando criança}
\end{EntryWithPhonetic}

\begin{EntryWithPhonetic}{从业}{cong2ye4}{4,5}{⼈,⼀}[HSK 7-9]
  \definition{v.}{obter emprego | praticar (um ofício)}
\end{EntryWithPhonetic}

\begin{EntryWithPhonetic}{从早到晚}{cong2zao3-dao4wan3}{4,6,8,11}{⼈,⽇,⼑,⽇}[HSK 7-9]
  \definition{adv.}{do amanhecer ao anoitecer; da manhã à noite}
\end{EntryWithPhonetic}

\begin{EntryWithPhonetic}{从中}{cong2 zhong1}{4,4}{⼈,⼁}[HSK 5]
  \definition{adv.}{de; dentre; daí}
\end{EntryWithPhonetic}

%%%%%%%%%% 丛 %%%%%%%%%%
\subsection*{丛}\addcontentsline{loh}{figure}{丛 \dpy{cong2}}

\begin{EntryWithPhonetic}{丛}{cong2}{5}{⼀}
  \definition*{s.}{Sobrenome: Cong}
  \definition{s.}{aglomerado; matagal; bosque | coleção; multidão}
\end{EntryWithPhonetic}

\begin{EntryWithPhonetic}{丛林}{cong2lin2}{5,8}{⼀,⽊}[HSK 7-9]
  \definition[片]{s.}{selva; floresta | mosteiro budista}
\end{EntryWithPhonetic}

%%%%%%%%%% 凑 %%%%%%%%%%
\subsection*{凑}\addcontentsline{loh}{figure}{凑 \dpy{cou4}}

\begin{EntryWithPhonetic}{凑}{cou4}{11}{⼎}[HSK 7-9]
  \definition{v.}{reunir; coletar; ajuntar | acontecer por acaso; aproveitar; esbarrar em; alcançar; tirar vantagem de | aproximar; mover"-se para perto de}
\end{EntryWithPhonetic}

\begin{EntryWithPhonetic}{凑合}{cou4he5}{11,6}{⼎,⼝}[HSK 7-9]
  \definition{v.}{contentar"-se com algo; ser razoável; ser razoavelmente bom, mas não excelente; aceitar relutantemente coisas ou condições de um nível ou grau inferior | improvisar | reunir}
\end{EntryWithPhonetic}

\begin{EntryWithPhonetic}{凑巧}{cou4qiao3}{11,5}{⼎,⼯}[HSK 7-9]
  \definition{adj.}{afortunado; sortudo; coincidente; significa que é o momento certo ou que algo que você quer ou não quer está acontecendo}
\end{EntryWithPhonetic}

%%%%%%%%%% 粗 %%%%%%%%%%
\subsection*{粗}\addcontentsline{loh}{figure}{粗 \dpy{cu1}}

\begin{EntryWithPhonetic}{粗}{cu1}{11}{⽶}[HSK 4]
  \definition{adj.}{largo (em diâmetro); grosso | grosseiro; rude; áspero | áspero; rouco | descuidado; negligente | rude; sem refinamento; vulgar}
  \definition{adv.}{grosseiramente; vagamente}
\end{EntryWithPhonetic}

\begin{EntryWithPhonetic}{粗暴}{cu1bao4}{11,15}{⽶,⽇}[HSK 7-9]
  \definition{adj.}{rude; áspero; bruto; brutal; violento}
\end{EntryWithPhonetic}

\begin{EntryWithPhonetic}{粗糙}{cu1cao1}{11,16}{⽶,⽶}[HSK 7-9]
  \definition{adj.}{áspero; grosseiro; não é liso; não é redondo; não é fino | desleixado; descuidado; não meticuloso}
\end{EntryWithPhonetic}

\begin{EntryWithPhonetic}{粗鲁}{cu1lu3}{11,12}{⽶,⿂}[HSK 7-9]
  \definition{adj.}{rude; grosseiro; incivilizado}
\end{EntryWithPhonetic}

\begin{EntryWithPhonetic}{粗略}{cu1lve4}{11,11}{⽶,⽥}[HSK 7-9]
  \definition{adj.}{grosseiro; rudimentar; superficial}
\end{EntryWithPhonetic}

\begin{EntryWithPhonetic}{粗心}{cu1xin1}{11,4}{⽶,⼼}[HSK 4]
  \definition{adj.}{descuidado; irrefletido; (fazer as coisas) de forma desleixada, sem cuidado}
\end{EntryWithPhonetic}

\begin{EntryWithPhonetic}{粗心大意}{cu1xin1-da4yi4}{11,4,3,13}{⽶,⼼,⼤,⼼}[HSK 7-9]
  \definition{expr.}{ser negligente; descuidado; inadvertido; desmiolado; descuidado e negligente; negligente; remisso; refere"-se a fazer as coisas de forma descuidada}
\end{EntryWithPhonetic}

\begin{EntryWithPhonetic}{粗心地做}{cu1xin1 di4 zuo4}{11,4,6,11}{⽶,⼼,⼟,⼈}
  \definition{adj.}{feito descuidadamente}
\end{EntryWithPhonetic}

%%%%%%%%%% 卒 %%%%%%%%%%
\subsection*{卒}\addcontentsline{loh}{figure}{卒 \dpy{cu4}}

\begin{EntryWithPhonetic}{卒}{cu4}{8}{⼗}
  \variantof{猝}
  \seeref{zu2}
\end{EntryWithPhonetic}

%%%%%%%%%% 促 %%%%%%%%%%
\subsection*{促}\addcontentsline{loh}{figure}{促 \dpy{cu4}}

\begin{EntryWithPhonetic}{促}{cu4}{9}{⼈}
  \definition{adj.}{curto; apressado; urgente}
  \definition{v.}{urgir; promover | estar perto de; estar perto}
\end{EntryWithPhonetic}

\begin{EntryWithPhonetic}{促成}{cu4cheng2}{9,6}{⼈,⼽}[HSK 7-9]
  \definition{v.}{facilitar; ajudar a concretizar}
\end{EntryWithPhonetic}

\begin{EntryWithPhonetic}{促进}{cu4jin4}{9,7}{⼈,⾡}[HSK 4]
  \definition{v.}{impulsionar; promover; avançar; incentivar o desenvolvimento}
\end{EntryWithPhonetic}

\begin{EntryWithPhonetic}{促使}{cu4shi3}{9,8}{⼈,⼈}[HSK 4]
  \definition{v.}{incitar; estimular; impelir; causar; provocar uma mudança em alguém ou em algo}
\end{EntryWithPhonetic}

\begin{EntryWithPhonetic}{促销}{cu4 xiao1}{9,12}{⼈,⾦}[HSK 4]
  \definition{v.}{promover vendas}
\end{EntryWithPhonetic}

%%%%%%%%%% 猝 %%%%%%%%%%
\subsection*{猝}\addcontentsline{loh}{figure}{猝 \dpy{cu4}}

\begin{EntryWithPhonetic}{猝}{cu4}{11}{⽝}
  \definition{adj.}{Literário: repentino; abrupto; inesperado}
  \definition{adv.}{Literário: de repente; inesperadamente}
\end{EntryWithPhonetic}

%%%%%%%%%% 酢 %%%%%%%%%%
\subsection*{酢}\addcontentsline{loh}{figure}{酢 \dpy{cu4}}

\begin{EntryWithPhonetic}{酢}{cu4}{12}{⾣}
  \definition{s.}{vinagre | (figurativo) ciúme (como em um caso de amor)}
  \variantof{醋}
\end{EntryWithPhonetic}

%%%%%%%%%% 醋 %%%%%%%%%%
\subsection*{醋}\addcontentsline{loh}{figure}{醋 \dpy{cu4}}

\begin{EntryWithPhonetic}{醋}{cu4}{15}{⾣}[HSK 6]
  \definition[瓶,坛,碟,碗]{s.}{(condimento) vinagre | ciúme (como em caso de amor); uma metáfora para o ciúme, referindo"-se principalmente aos relacionamentos entre pessoas}
\end{EntryWithPhonetic}

%%%%%%%%%% 簇 %%%%%%%%%%
\subsection*{簇}\addcontentsline{loh}{figure}{簇 \dpy{cu4}}

\begin{EntryWithPhonetic}{簇}{cu4}{17}{⽵}
  \definition{clas.}{aglomerado; grupo; usado para pessoas ou coisas que se reúnem em grupos ou pilhas}
  \definition{s.}{pilha; aglomerado; buquê}
  \definition{v.}{aglomerar"-se; formar um aglomerado; empilhar}
\end{EntryWithPhonetic}

\begin{EntryWithPhonetic}{簇拥}{cu4yong1}{17,8}{⽵,⼿}[HSK 7-9]
  \definition{v.}{aglomerar"-se em volta de  | escoltar}
\end{EntryWithPhonetic}

%%%%%%%%%% 窜 %%%%%%%%%%
\subsection*{窜}\addcontentsline{loh}{figure}{窜 \dpy{cuan4}}

\begin{EntryWithPhonetic}{窜}{cuan4}{12}{⽳}[HSK 7-9]
  \definition{v.}{fugir; correr (usado para bandidos, tropas inimigas e animais) | Literário: exilar; expulsar | Obsoleto: mudar (a redação de um texto, manuscrito, etc.); alterar}
\end{EntryWithPhonetic}

%%%%%%%%%% 窾 %%%%%%%%%%
\subsection*{窾}\addcontentsline{loh}{figure}{窾 \dpy{cuan4}}

\begin{EntryWithPhonetic}{窾}{cuan4}{17}{⽳}
  \definition{adj.}{vazio | seco | destituído; pobre}
  \definition{s.}{buraco | lei}
  \definition{v.}{esconder}
  \seeref{kuan3}
\end{EntryWithPhonetic}

%%%%%%%%%% 催 %%%%%%%%%%
\subsection*{催}\addcontentsline{loh}{figure}{催 \dpy{cui1}}

\begin{EntryWithPhonetic}{催}{cui1}{13}{⼈}[HSK 7-9]
  \definition*{s.}{Sobrenome: Cui}
  \definition{v.}{instar; apressar; pressionar | apressar; agilizar; acelerar}
\end{EntryWithPhonetic}

\begin{EntryWithPhonetic}{催促}{cui1cu4}{13,9}{⼈,⼈}[HSK 7-9]
  \definition{v.}{instar; apressar; pressionar; urgir}
\end{EntryWithPhonetic}

\begin{EntryWithPhonetic}{催眠}{cui1mian2}{13,10}{⼈,⽬}[HSK 7-9]
  \definition{adj.}{hipnótico}
  \definition{v.}{hipnotizar; mesmerizar; embalar (para dormir); usar drogas ou sons, movimentos, etc. para induzir o sono}
\end{EntryWithPhonetic}

%%%%%%%%%% 摧 %%%%%%%%%%
\subsection*{摧}\addcontentsline{loh}{figure}{摧 \dpy{cui1}}

\begin{EntryWithPhonetic}{摧}{cui1}{14}{⼿}
  \definition{v.}{quebrar; destruir}
\end{EntryWithPhonetic}

\begin{EntryWithPhonetic}{摧毁}{cui1hui3}{14,13}{⼿,⽎}[HSK 7-9]
  \definition{v.}{destruir; esmagar; nocautear; destruir com grande força}
\end{EntryWithPhonetic}

%%%%%%%%%% 脆 %%%%%%%%%%
\subsection*{脆}\addcontentsline{loh}{figure}{脆 \dpy{cui4}}

\begin{EntryWithPhonetic}{脆}{cui4}{10}{⾁}[HSK 5]
  \definition{adj.}{frágil; quebradiço | crocante | (voz) clara; nítida | puro}
  \antonymref{韧}{ren4}
\end{EntryWithPhonetic}

\begin{EntryWithPhonetic}{脆弱}{cui4ruo4}{10,10}{⾁,⼸}[HSK 7-9]
  \definition{adj.}{frágil; débil; fraco; incapaz de suportar contratempos}
\end{EntryWithPhonetic}

%%%%%%%%%% 翠 %%%%%%%%%%
\subsection*{翠}\addcontentsline{loh}{figure}{翠 \dpy{cui4}}

\begin{EntryWithPhonetic}{翠}{cui4}{14}{⽻}
  \definition{adj.}{verde; verde esmeralda}
  \definition{s.}{martim-pescador | jadeíte; jade}
\end{EntryWithPhonetic}

\begin{EntryWithPhonetic}{翠绿}{cui4lv4}{14,11}{⽻,⽷}[HSK 7-9]
  \definition{adj.}{verde esmeralda; verde jade}
\end{EntryWithPhonetic}

%%%%%%%%%% 村 %%%%%%%%%%
\subsection*{村}\addcontentsline{loh}{figure}{村 \dpy{cun1}}

\begin{EntryWithPhonetic}{村}{cun1}{7}{⽊}[HSK 3]
  \definition{adj.}{rústico; grosseiro}
  \definition[个,座]{s.}{aldeia; vila | área povoada de certo tipo}
\end{EntryWithPhonetic}

\begin{EntryWithPhonetic}{村儿}{cun1r5}{7,2}{⽊,⼉}
  \definition{s.}{vila; aldeia}
\end{EntryWithPhonetic}

\begin{EntryWithPhonetic}{村庄}{cun1 zhuang1}{7,6}{⽊,⼴}[HSK 6]
  \definition[个,座,片]{s.}{aldeia; vila; onde vivem os agricultores}
\end{EntryWithPhonetic}

%%%%%%%%%% 存 %%%%%%%%%%
\subsection*{存}\addcontentsline{loh}{figure}{存 \dpy{cun2}}

\begin{EntryWithPhonetic}{存}{cun2}{6}{⼦}[HSK 3]
  \definition{v.}{existir; viver; sobreviver | armazenar; manter | acumular; coletar | depositar | sair com; verificar | reservar; reter | permanecer em equilíbrio; estar em estoque | estimar; abrigar}
\end{EntryWithPhonetic}

\begin{EntryWithPhonetic}{存放}{cun2fang4}{6,8}{⼦,⽅}[HSK 7-9]
  \definition{v.}{armazenar; guardar; deixar com}
\end{EntryWithPhonetic}

\begin{EntryWithPhonetic}{存款}{cun2 kuan3}{6,12}{⼦,⽋}[HSK 5]
  \definition[些,笔]{s.}{depósito; poupança bancária}
  \definition{v.}{depositar dinheiro; colocar dinheiro no banco}
\end{EntryWithPhonetic}

\begin{EntryWithPhonetic}{存心}{cun2xin1}{6,4}{⼦,⼼}[HSK 7-9]
  \definition{adv.}{intencionalmente; deliberadamente; de propósito}
\end{EntryWithPhonetic}

\begin{EntryWithPhonetic}{存在}{cun2zai4}{6,6}{⼦,⼟}[HSK 3]
  \definition{s.}{existência; ser; ente; o mundo objetivo, que não depende da consciência humana para mudar, ou seja, a matéria}
  \definition{v.}{existir; ser; as coisas ocupam continuamente o tempo e o espaço; na verdade, ainda não desapareceram}
\end{EntryWithPhonetic}

\begin{EntryWithPhonetic}{存折}{cun2zhe2}{6,7}{⼦,⼿}[HSK 7-9]
  \definition{s.}{caderneta bancária; caderneta de poupança; livro de depósitos; livro de poupança bancária; um \emph{voucher} emitido por uma instituição financeira a um depositante como um certificado}
\end{EntryWithPhonetic}

%%%%%%%%%% 寸 %%%%%%%%%%
\subsection*{寸}\addcontentsline{loh}{figure}{寸 \dpy{cun4}}

\begin{EntryWithPhonetic}{寸}{cun4}{3}{⼨}[HSK 5][Kangxi 41]
  \definition*{s.}{Sobrenome: Cun}
  \definition{adj.}{muito pouco; muito curto; pequeno | Dialeto: coincidência}
  \definition{clas.}{cun, uma unidade tradicional de comprimento, igual a 0,1 市尺 e equivalente a 3,333 centímetros ou 1,312 polegadas | cun, uma unidade de comprimento (=13 decímetros)}
  \seealsoref{市尺}{shi4 chi3}
\end{EntryWithPhonetic}

%%%%%%%%%% 搓 %%%%%%%%%%
\subsection*{搓}\addcontentsline{loh}{figure}{搓 \dpy{cuo1}}

\begin{EntryWithPhonetic}{搓}{cuo1}{12}{⼿}[HSK 7-9]
  \definition{s.}{torção}
  \definition{v.}{esfregar ou rolar entre as mãos ou dedos |  (no tênis, tênis de mesa, críquete, etc.) cortar | (roupa, etc.) torcer}
\end{EntryWithPhonetic}

%%%%%%%%%% 磋 %%%%%%%%%%
\subsection*{磋}\addcontentsline{loh}{figure}{磋 \dpy{cuo1}}

\begin{EntryWithPhonetic}{磋}{cuo1}{14}{⽯}
  \definition{v.}{moer e polir marfim (significado original) | Figurativo: consultar; trocar opiniões | moer; polir}
\end{EntryWithPhonetic}

\begin{EntryWithPhonetic}{磋商}{cuo1shang1}{14,11}{⽯,⼝}[HSK 7-9]
  \definition{v.}{consultar; negociar; trocar pontos de vista; discutir repetidamente; discutir cuidadosamente}
\end{EntryWithPhonetic}

%%%%%%%%%% 鹾 %%%%%%%%%%
\subsection*{鹾}\addcontentsline{loh}{figure}{鹾 \dpy{cuo2}}

\begin{EntryWithPhonetic}{鹾}{cuo2}{16}{⿄}
  \definition{adj.}{salgado}
  \definition{s.}{sal}
\end{EntryWithPhonetic}

%%%%%%%%%% 挫 %%%%%%%%%%
\subsection*{挫}\addcontentsline{loh}{figure}{挫 \dpy{cuo4}}

\begin{EntryWithPhonetic}{挫}{cuo4}{10}{⼿}
  \definition{v.}{frustrar | diminuir; embotar; desinflar | pressionar para baixo; abaixar}
\end{EntryWithPhonetic}

\begin{EntryWithPhonetic}{挫折}{cuo4zhe2}{10,7}{⼿,⼿}[HSK 7-9]
  \definition[个,次]{s.}{retrocesso; reversão; frustração | derrota, fracasso, insucesso}
  \definition{v.}{falhar; derrotar; fracassar}
\end{EntryWithPhonetic}

%%%%%%%%%% 措 %%%%%%%%%%
\subsection*{措}\addcontentsline{loh}{figure}{措 \dpy{cuo4}}

\begin{EntryWithPhonetic}{措}{cuo4}{11}{⼿}
  \definition{s.}{iniciativa; solução; medida}
  \definition{v.}{organizar; gerenciar; lidar | fazer planos; administrar; organizar}
\end{EntryWithPhonetic}

\begin{EntryWithPhonetic}{措施}{cuo4shi1}{11,9}{⼿,⽅}[HSK 4]
  \definition[项,个]{s.}{medida; etapa; passo; abordagem adotada para lidar com as coisas}
\end{EntryWithPhonetic}

\begin{EntryWithPhonetic}{措手不及}{cuo4shou3-bu4ji2}{11,4,4,3}{⼿,⼿,⼀,⼃}[HSK 7-9]
  \definition{expr.}{ser pego de surpresa; ser pego de surpresa (despreparado); ser tarde demais para fazer algo a respeito; ficar surpreso demais para se defender; não conseguir fazer uma defesa adequada; não conseguir pensar a tempo em uma maneira de se defender; não ter tempo para colocar em prática; surpreender alguém; pegar alguém desprevenido | ser pego desprevenido; ser pego de surpresa}
\end{EntryWithPhonetic}

%%%%%%%%%% 错 %%%%%%%%%%
\subsection*{错}\addcontentsline{loh}{figure}{错 \dpy{cuo4}}

\begin{EntryWithPhonetic}{错}{cuo4}{13}{⾦}[HSK 1]
  \definition{adj.}{errado; equivocado; errôneo | (na negativa) nada ruim; muito bom | entrelaçado e recortado; intrincado; complexo | ruim; pobre; péssimo (usado apenas em negativas)}
  \definition{s.}{falha; demérito | erro; engano | (arcaico) pedra de amolar para polir jade}
  \definition{v.}{estar entrelaçado e serrilhado; ser intrincado | moer; esfregar | abrir caminho; sair do caminho | alternar; escalonar | estar fora de alinhamento | deslocar | evitar; fazer com que não se encontre ou não entre em conflito | polir; polir pedras preciosas | (literário) incrustar ou revestir com ouro, prata, etc. | interseccionar; cruzar; entrecruzar}
\end{EntryWithPhonetic}

\begin{EntryWithPhonetic}{错别字}{cuo4bie2zi4}{13,7,6}{⾦,⼑,⼦}[HSK 7-9]
  \definition{s.}{caracteres escritos incorretamente ou mal pronunciados; erros de digitação e ortografia}[文章里有一些错别字。===Há alguns erros de digitação no artigo.]
\end{EntryWithPhonetic}

\begin{EntryWithPhonetic}{错过}{cuo4 guo4}{13,6}{⾦,⾡}[HSK 6]
  \definition{v.}{perder (oportunidade); deixar escapar}
\end{EntryWithPhonetic}

\begin{EntryWithPhonetic}{错觉}{cuo4jue2}{13,9}{⾦,⾒}[HSK 7-9]
  \definition{s.}{ilusão; concepção errônea; impressão errada; percepção incorreta de coisas objetivas devido a alguns motivos}
\end{EntryWithPhonetic}

\begin{EntryWithPhonetic}{错位}{cuo4/wei4}{13,7}{⾦,⼈}[HSK 7-9]
  \definition{s.}{deslocamento (por exemplo, de ossos quebrados) | julgamento errôneo | contato defeituoso | inversão (médica, por exemplo, parto prematuro) | fora de alinhamento}
  \definition{v.+compl.}{Medicina: deslocar | deslocar; inverter | extraviar}
\end{EntryWithPhonetic}

\begin{EntryWithPhonetic}{错误}{cuo4wu4}{13,9}{⾦,⾔}[HSK 3]
  \definition{adj.}{equivocado; errado; errôneo; incorreto; não condizente com a realidade objetiva}
  \definition[个,次]{s.}{engano; erro; erro grosseiro; falha; coisas, comportamentos, etc. incorretos}
\end{EntryWithPhonetic}

\begin{EntryWithPhonetic}{错综复杂}{cuo4zong1-fu4za2}{13,11,9,6}{⾦,⽷,⼢,⽊}[HSK 7-9]
  \definition{expr.}{desconcertante; complicado e confuso; complexo e misturado; muito complicado; intrincado; complexo}
\end{EntryWithPhonetic}

%%%%% EOF %%%%%


 %%%
%%% D
%%%
\section*{D}\addcontentsline{toc}{section}{D}\addcontentsline{loh}{figure}{\#\#\#\#\#\#\#\# D}

%%%%%%%%%% 搭 %%%%%%%%%%
\subsection*{搭}\addcontentsline{loh}{figure}{搭 \dpy{da1}}

\begin{EntryWithPhonetic}{搭}{da1}{12}{⼿}[HSK 6]
  \definition{v.}{colocar em prática; construir | ficar pendurado; colocar para cima | entrar em contato; juntar"-se | adicionar (mais pessoas, dinheiro, etc.) | levantar algo junto |
pegar (um navio, avião, etc.); viajar (ou ir) por}
  \variantof{褡}
\end{EntryWithPhonetic}

\begin{EntryWithPhonetic}{搭乘}{da1cheng2}{12,10}{⼿,⽲}[HSK 7-9]
  \definition{v.}{viajar de (carro, barco, avião etc.)}
\end{EntryWithPhonetic}

\begin{EntryWithPhonetic}{搭档}{da1dang4}{12,10}{⼿,⽊}[HSK 6]
  \definition[个,名,位]{s.}{parceiro; colega de trabalho}
  \definition{v.}{cooperar; trabalhar em conjunto; formar pares; colaborar; formar uma parceria}
\end{EntryWithPhonetic}

\begin{EntryWithPhonetic}{搭建}{da1jian4}{12,8}{⼿,⼵}[HSK 7-9]
  \definition{v.}{montar (um galpão, abrigo temporário, etc.) | criar (uma organização) | construir (especialmente com materiais simples) | juntar (um galpão temporário) | armar}
\end{EntryWithPhonetic}

\begin{EntryWithPhonetic}{搭配}{da1pei4}{12,10}{⼿,⾣}[HSK 6]
  \definition{v.}{emparelhar; organizar em pares ou grupos; organizar a distribuição de acordo com certos requisitos | encaixar; combinar}
\end{EntryWithPhonetic}

\begin{EntryWithPhonetic}{搭讪}{da1shan4}{12,5}{⼿,⾔}
  \definition{v.}{bater em alguém | incitar uma conversa | começar a conversar para acabar com um silêncio constrangedor ou uma situação embaraçosa}
\end{EntryWithPhonetic}

%%%%%%%%%% 答 %%%%%%%%%%
\subsection*{答}\addcontentsline{loh}{figure}{答 \dpy{da1}}

\begin{EntryWithPhonetic}{答}{da1}{12}{⽵}
  \definition{v.}{concordar; responder | responder; prestar atenção}
  \seeref{da2}
\end{EntryWithPhonetic}

\begin{EntryWithPhonetic}{答应}{da1ying5}{12,7}{⽵,⼴}[HSK 2]
  \definition{v.}{responder; retribuir; reagir; retrucar | concordar; prometer; cumprir}
\end{EntryWithPhonetic}

%%%%%%%%%% 褡 %%%%%%%%%%
\subsection*{褡}\addcontentsline{loh}{figure}{褡 \dpy{da1}}

\begin{EntryWithPhonetic}{褡}{da1}{14}{⾐}
  \definition{s.}{bolsa; malote; algibeira | jaqueta sem mangas}
\end{EntryWithPhonetic}

%%%%%%%%%% 打 %%%%%%%%%%
\subsection*{打}\addcontentsline{loh}{figure}{打 \dpy{da2}}

\begin{EntryWithPhonetic}{打}{da2}{5}{⼿}[HSK 4]
  \definition{clas./s.}{(empréstimo linguístico) dúzia}
  \seeref{da3}
\end{EntryWithPhonetic}

%%%%%%%%%% 达 %%%%%%%%%%
\subsection*{达}\addcontentsline{loh}{figure}{达 \dpy{da2}}

\begin{EntryWithPhonetic}{达}{da2}{6}{⾡}
  \definition*{s.}{Sobrenome: Da}
  \definition{adj.}{eminente; distinto; refere"-se a um funcionário distinto; \emph{status} elevado | otimista; de mente aberta}
  \definition{v.}{prolongar | alcançar; atingir; equivaler a | entender completamente; compreender (assuntos) | expressar; comunicar}
\end{EntryWithPhonetic}

\begin{EntryWithPhonetic}{达标}{da2biao1}{6,9}{⾡,⽊}[HSK 7-9]
  \definition{v.}{atingir um padrão definido (até o padrão) | atingir um padrão definido (qualifica)}
\end{EntryWithPhonetic}

\begin{EntryWithPhonetic}{达成}{da2/cheng2}{6,6}{⾡,⼽}[HSK 5]
  \definition{v.+compl.}{concluir; chegar (a um acordo); conseguir; obter (principalmente como resultado de uma negociação)}
\end{EntryWithPhonetic}

\begin{EntryWithPhonetic}{达到}{da2/dao4}{6,8}{⾡,⼑}[HSK 3]
  \definition{v.}{alcançar; atender o padrão; atingir (refere"-se principalmente a coisas abstratas ou graus); chegar a um determinado ponto ou grau}
\end{EntryWithPhonetic}

%%%%%%%%%% 怛 %%%%%%%%%%
\subsection*{怛}\addcontentsline{loh}{figure}{怛 \dpy{da2}}

\begin{EntryWithPhonetic}{怛}{da2}{8}{⼼}
  \definition{adj.}{Arcaico: aflito; angustiado | Arcaico: aterrorizado | Arcaico: alarmado; entristecido | Arcaico: chocado}
\end{EntryWithPhonetic}

%%%%%%%%%% 答 %%%%%%%%%%
\subsection*{答}\addcontentsline{loh}{figure}{答 \dpy{da2}}

\begin{EntryWithPhonetic}{答}{da2}{12}{⽵}[HSK 5]
  \definition{v.}{responder; dar resposta a; responder a | retribuir; devolver (uma visita, etc.); retribuir um favor feito a alguém por outro; fazer o bem}
  \seeref{da1}
\end{EntryWithPhonetic}

\begin{EntryWithPhonetic}{答案}{da2'an4}{12,10}{⽵,⽊}[HSK 4]
  \definition[个,条,种,些]{s.}{chave; resposta; solução}
\end{EntryWithPhonetic}

\begin{EntryWithPhonetic}{答辩}{da2bian4}{12,16}{⽵,⾟}[HSK 7-9]
  \definition{v.}{responder a perguntas, acusações, etc. de outras pessoas; defender as próprias opiniões ou ações}
\end{EntryWithPhonetic}

\begin{EntryWithPhonetic}{答复}{da2fu5}{12,9}{⽵,⼢}[HSK 5]
  \definition[个]{s.}{resposta; respostas a perguntas ou solicitações}
  \definition{v.}{responder; dar uma resposta}
\end{EntryWithPhonetic}

%%%%%%%%%% 打 %%%%%%%%%%
\subsection*{打}\addcontentsline{loh}{figure}{打 \dpy{da3}}

\begin{EntryWithPhonetic}{打}{da3}{5}{⼿}[HSK 1,5]
  \definition{prep.}{de; desde; ponto de partida que indica lugar, tempo ou extensão; indica rotas e locais percorridos | devido a; origem da introdução de coisas novas}
  \definition{v.}{golpear; acertar; bater | quebrar; esmagar | lutar; atacar; espancar | entrar com uma ação judicial; negociar; fazer representações | construir; edificar | fabricar (em uma ferraria); forjar | misturar; mexer; bater | amarrar; embalar | tricotar; tecer | desenhar; pintar; deixar uma marca; imprimir | abrir; perfurar; cavar | içar; levantar | enviar; despachar; projetar | emitir ou receber (um certificado, etc.) | remover; livrar-se de | colher; tirar; retirar | comprar | capturar; caçar | reunir; coletar; colher; recolher através de ações como cortar e podar | estimar; calcular; contar; determinar | fazer; envolver-se em | jogar algum tipo de jogo | expressar certos movimentos corporais | adotar; usar; adotar uma determinada abordagem | pegar (um táxi) | indicar a melhora de seu estado mental; melhorar o estado mental}
  \seeref{da2}
\end{EntryWithPhonetic}

\begin{EntryWithPhonetic}{打败}{da3 bai4}{5,8}{⼿,⾒}[HSK 4]
  \definition{v.}{derrotar; vencer; piorar | sofrer uma derrota; ser derrotado}
\end{EntryWithPhonetic}

\begin{EntryWithPhonetic}{打扮}{da3ban5}{5,7}{⼿,⼿}[HSK 5]
  \definition{s.}{estilo de se vestir; o modo de se vestir; as roupas que se usa}
  \definition{v.}{vestir-se bem; maquiar-se; dar uma boa aparência e vestir-se bem; adornar}
\end{EntryWithPhonetic}

\begin{EntryWithPhonetic}{打包}{da3/bao1}{5,5}{⼿,⼓}[HSK 5]
  \definition{v.+compl.}{levar a comida embora; levar para viagem; refere"-se especificamente a comer em um restaurante e levar as sobras em uma caixa, sacola ou outro recipiente | embalar; empacotar | desembalar; desempacotar}
\end{EntryWithPhonetic}

\begin{EntryWithPhonetic}{打岔}{da3/cha4}{5,7}{⼿,⼭}[HSK 7-9]
  \definition{s.}{interrupção}
  \definition{v.+compl.}{interromper; cortar | mudar de assunto | interromper (especialmente a fala)}
\end{EntryWithPhonetic}

\begin{EntryWithPhonetic}{打车}{da3che1}{5,4}{⼿,⾞}[HSK 1]
  \definition{v.}{pegar um táxi; chamar um táxi; dar sinal para um táxi}
\end{EntryWithPhonetic}

\begin{EntryWithPhonetic}{打倒}{da3/dao3}{5,10}{⼿,⼈}[HSK 7-9]
  \definition{v.+compl.}{atacar e derrubar no chão; cair | derrubar; tombar}[打倒法西斯政权。===Derrubar o regime fascista.]
\end{EntryWithPhonetic}

\begin{EntryWithPhonetic}{打的}{da3/di1}{5,8}{⼿,⽩}
  \definition{v.+compl.}{(coloquial) pegar um táxi | ir de táxi}
\end{EntryWithPhonetic}

\begin{EntryWithPhonetic}{打电话}{da3 dian4hua4}{5,5,8}{⼿,⽥,⾔}[HSK 1]
  \definition{v.}{telefonar; fazer uma chamada telefônica; dar um telefonema}
  \seealsoref{给……打电话}{gei3 da3 dian4hua4}
\end{EntryWithPhonetic}

\begin{EntryWithPhonetic}{打动}{da3dong4}{5,6}{⼿,⼒}[HSK 6]
  \definition{v.}{mover; tocar}[这番话打动了她的心。===Essas palavras tocaram seu coração.]
\end{EntryWithPhonetic}

\begin{EntryWithPhonetic}{打断}{da3 duan4}{5,11}{⼿,⽄}[HSK 6]
  \definition{v.}{interromper uma atividade (fala; pensamento ou ação) | fraturar (osso do corpo)  com força | arrombar; bater com força para quebrar}
\end{EntryWithPhonetic}

\begin{EntryWithPhonetic}{打盹儿}{da3/dun3r5}{5,9,2}{⼿,⽬,⼉}[HSK 7-9]
  \definition{v.+compl.}{cochilar; tirar uma soneca}
\end{EntryWithPhonetic}

\begin{EntryWithPhonetic}{打发}{da3fa5}{5,5}{⼿,⼜}[HSK 6]
  \definition{v.}{enviar; despachar | dispensar; mandar embora | passar o tempo; matar o tempo}
\end{EntryWithPhonetic}

\begin{EntryWithPhonetic}{打工}{da3/gong1}{5,3}{⼿,⼯}[HSK 2]
  \definition{v.+compl.}{contratar para trabalhar; trabalhar em tempo parcial; realizar trabalho manual (para alguém, geralmente temporariamente)}
\end{EntryWithPhonetic}

\begin{EntryWithPhonetic}{打工人}{da3gong1ren2}{5,3,2}{⼿,⼯,⼈}
  \definition{s.}{trabalhador}
\end{EntryWithPhonetic}

\begin{EntryWithPhonetic}{打官司}{da3/guan1si5}{5,8,5}{⼿,⼧,⼝}[HSK 6]
  \definition{v.+compl.}{ir ao tribunal (ou à lei); envolver-se em um processo judicial}
\end{EntryWithPhonetic}

\begin{EntryWithPhonetic}{打击}{da3ji1}{5,5}{⼿,⼐}[HSK 5]
  \definition{v.}{golpear; atacar; reprimir; atacar para frustrar; machucar | bater; bater (em um tambor, etc.); golpear ou bater em algo}
\end{EntryWithPhonetic}

\begin{EntryWithPhonetic}{打架}{da3/jia4}{5,9}{⼿,⽊}[HSK 5]
  \definition{v.+compl.}{brigar; discutir; entrar em conflito | contradizer; conflitar; ser inconsistente}
\end{EntryWithPhonetic}

\begin{EntryWithPhonetic}{打交道}{da3 jiao1dao5}{5,6,12}{⼿,⼇,⾡}[HSK 7-9]
  \definition{v.}{mediar; formar equipe; entrar em contato com; fazer contato com; ter relações com}
\end{EntryWithPhonetic}

\begin{EntryWithPhonetic}{打搅}{da3jiao3}{5,12}{⼿,⼿}[HSK 7-9]
  \definition{v.}{perturbar; incomodar | interromper}
\end{EntryWithPhonetic}

\begin{EntryWithPhonetic}{打结}{da3jie2}{5,9}{⼿,⽷}
  \definition{v.}{dar um nó | amarrar}
\end{EntryWithPhonetic}

\begin{EntryWithPhonetic}{打开}{da3 kai1}{5,4}{⼿,⼶}[HSK 1]
  \definition{v.}{abrir; desdobrar; desenrolar | descobrir; revelar; desvendar | ativar; ligar; ligar o circuito | romper | abrir-se; espalhar-se; expandir; ampliar | abrir; iniciar o funcionamento do software, etc.}
\end{EntryWithPhonetic}

\begin{EntryWithPhonetic}{打瞌睡}{da3ke1shui4}{5,15,13}{⼿,⽬,⽬}
  \definition{v.}{cochilar}
\end{EntryWithPhonetic}

\begin{EntryWithPhonetic}{打捞}{da3lao1}{5,10}{⼿,⼿}[HSK 7-9]
  \definition{s.}{salvamento; resgate}
  \definition{v.}{sair da água; resgatar | encontrar e recuperar objetos que afundaram na água}
\end{EntryWithPhonetic}

\begin{EntryWithPhonetic}{打雷}{da3 lei2}{5,13}{⼿,⾬}[HSK 4]
  \definition{v.}{trovejar; produzir ruídos altos quando as nuvens descarregam eletricidade}
\end{EntryWithPhonetic}

\begin{EntryWithPhonetic}{打量}{da3liang5}{5,12}{⼿,⾥}[HSK 7-9]
  \definition{v.}{aumentar o tamanho; medir com o olho; olhar de cima a baixo; examinar; observar | pensar; calcular; supor; estimar}
\end{EntryWithPhonetic}

\begin{EntryWithPhonetic}{打猎}{da3/lie4}{5,11}{⼿,⽝}[HSK 7-9]
  \definition{v.+compl.}{ir caçar}
\end{EntryWithPhonetic}

\begin{EntryWithPhonetic}{打骂}{da3ma4}{5,9}{⼿,⾺}
  \definition{v.}{bater e repreender}
\end{EntryWithPhonetic}

\begin{EntryWithPhonetic}{打磨}{da3mo2}{5,16}{⼿,⽯}[HSK 7-9]
  \definition{v.}{polir; dar brilho; fazer brilhar; esfregar a superfície de um objeto para torná-lo liso e delicado}
\end{EntryWithPhonetic}

\begin{EntryWithPhonetic}{打牌}{da3 pai2}{5,12}{⼿,⽚}[HSK 6]
  \definition{v.}{jogar cartas, usar cartas para entretenimento ou jogos de azar}
\end{EntryWithPhonetic}

\begin{EntryWithPhonetic}{打屁股}{da3pi4gu5}{5,7,8}{⼿,⼫,⾁}
  \definition{v.}{dar um tapa no bumbum de alguém}
\end{EntryWithPhonetic}

\begin{EntryWithPhonetic}{打破}{da3po5}{5,10}{⼿,⽯}[HSK 3]
  \definition{v.}{quebrar; esmagar; quebrar recordes, regras ou restrições existentes, etc.}
\end{EntryWithPhonetic}

\begin{EntryWithPhonetic}{打球}{da3qiu2}{5,11}{⼿,⽟}[HSK 1]
  \definition{v.}{jogar bola (com as mãos) | jogar (basquetebol, handbol, etc.) | jogar um jogo de bola}
\end{EntryWithPhonetic}

\begin{EntryWithPhonetic}{打扰}{da3rao3}{5,7}{⼿,⼿}[HSK 5]
  \definition{v.}{perturbar; incomodar; interferir no trabalho normal, na vida ou no que as outras pessoas estão fazendo, etc. | usado para expressar um pedido de desculpas por ajuda; gratidão por ajuda; hospitalidade recebida}
\end{EntryWithPhonetic}

\begin{EntryWithPhonetic}{打扫}{da3sao3}{5,6}{⼿,⼿}[HSK 4]
  \definition{v.}{varrer; limpar; varrer para limpar}
\end{EntryWithPhonetic}

\begin{EntryWithPhonetic}{打算}{da3suan5}{5,14}{⼿,⽵}[HSK 2]
  \definition[个,项]{s.}{plano; intenção; consideração; cálculo; ideias sobre a direção e os métodos da ação; pensamentos}
  \definition{v.}{pretender; planejar; calcular; considerar com antecedência}
\end{EntryWithPhonetic}

\begin{EntryWithPhonetic}{打听}{da3ting5}{5,7}{⼿,⼝}[HSK 3]
  \definition{v.}{perguntar sobre; indagar sobre; obter uma linha sobre}
\end{EntryWithPhonetic}

\begin{EntryWithPhonetic}{打通}{da3/tong1}{5,10}{⼿,⾡}[HSK 7-9]
  \definition{v.+compl.}{passar; abrir-se | estabelecer contato | abrir o acesso | passar (uma conexão telefônica) | remover um bloco}
\end{EntryWithPhonetic}

\begin{EntryWithPhonetic}{打压}{da3ya1}{5,6}{⼿,⼚}
  \definition{v.}{reprimir | derrotar}
\end{EntryWithPhonetic}

\begin{EntryWithPhonetic}{打印}{da3yin4}{5,5}{⼿,⼙}[HSK 2]
  \definition{v.}{imprimir; imprimir em papel ou outro suporte de gravação, como uma impressora}
\end{EntryWithPhonetic}

\begin{EntryWithPhonetic}{打印机}{da3yin4ji1}{5,5,6}{⼿,⼙,⽊}[HSK 6]
  \definition[个,部,台]{s.}{impressora; uma máquina de escrever controlada por um microcomputador, sem teclado, que converte códigos de caracteres em caracteres e os imprime}
\end{EntryWithPhonetic}

\begin{EntryWithPhonetic}{打造}{da3zao4}{5,10}{⼿,⾡}[HSK 6]
  \definition{v.}{forjar (trabalhar em metal); fabricar (principalmente objetos de metal) | fazer; criar; construir; desenvolver}
\end{EntryWithPhonetic}

\begin{EntryWithPhonetic}{打仗}{da3/zhang4}{5,5}{⼿,⼈}[HSK 7-9]
  \definition{v.+compl.}{lutar; ir à guerra; fazer guerra}
\end{EntryWithPhonetic}

\begin{EntryWithPhonetic}{打招呼}{da3 zhao1hu5}{5,8,8}{⼿,⼿,⼝}[HSK 7-9]
  \definition{v.}{cumprimentar alguém; dizer olá; tirar o chapéu; saudações por meio de palavras ou gestos | avisar; lembrar; informar; notificar com antecedência; dar aviso prévio}
\end{EntryWithPhonetic}

\begin{EntryWithPhonetic}{打折}{da3/zhe2}{5,7}{⼿,⼿}[HSK 4]
  \definition{v.+compl.}{dar desconto; dar um desconto; vender produtos a um preço reduzido em uma determinada porcentagem do preço original; metáfora para não cumprir 100\% do que foi originalmente padronizado ou prometido}
\end{EntryWithPhonetic}

\begin{EntryWithPhonetic}{打针}{da3/zhen1}{5,7}{⼿,⾦}[HSK 4]
  \definition{v.+compl.}{dar ou receber uma injeção; injetar um medicamento líquido em um organismo com uma seringa}
\end{EntryWithPhonetic}

%%%%%%%%%% 大 %%%%%%%%%%
\subsection*{大}\addcontentsline{loh}{figure}{大 \dpy{da4}}

\begin{EntryWithPhonetic}{大}{da4}{3}{⼤}[HSK 1][Kangxi 37]
  \definition*{s.}{Sobrenome: Da}
  \definition{adj.}{grande; amplo; grande em volume, área, etc. | mais velho; em primeiro lugar no ranking | tamanho; descreve o grau de grandeza | usado em certas épocas do ano, condições climáticas, feriados ou antes de um determinado momento, para enfatizar | o tempo mais distante; há muito tempo}
  \definition{adv.}{grandemente; totalmente; expressa um grau muito profundo | não muito; não frequentemente; usado após 不, indica um grau baixo ou poucas vezes}
  \definition{s.}{adulto; crescido; pessoas idosas | pai | irmão do pai de alguém; tio}
  \seeref{dai4}
  \seealsoref{不}{bu4}
\end{EntryWithPhonetic}

\begin{EntryWithPhonetic}{大巴}{da4ba1}{3,4}{⼤,⼰}[HSK 4]
  \definition{s.}{ônibus}
\end{EntryWithPhonetic}

\begin{EntryWithPhonetic}{大伯子}{da4 bai3zi5}{3,7,3}{⼤,⼈,⼦}
  \definition{s.}{Coloquial: irmão mais velho do marido; cunhado}
\end{EntryWithPhonetic}

\begin{EntryWithPhonetic}{大包大揽}{da4bao1-da4lan3}{3,5,3,12}{⼤,⼓,⼤,⼿}[HSK 7-9]
  \definition{v.}{assumir o controle total}
\end{EntryWithPhonetic}

\begin{EntryWithPhonetic}{大笔}{da4bi3}{3,10}{⼤,⽵}[HSK 7-9]
  \definition{adj.}{grande quantidade (de dinheiro) | grande soma (de capital, dinheiro, etc.)}
  \definition{s.}{caneta | Cortês: sua escrita; sua caligrafia}
\end{EntryWithPhonetic}

\begin{EntryWithPhonetic}{大部分}{da4bu4fen5}{3,10,4}{⼤,⾢,⼑}[HSK 2]
  \definition[把]{s.}{a maioria; a maior parte; em grande parte; refere"-se a uma quantidade superior a metade do total}
\end{EntryWithPhonetic}

\begin{EntryWithPhonetic}{大臣}{da4chen2}{3,6}{⼤,⾂}[HSK 7-9]
  \definition[位,个]{s.}{secretário; ministro (de uma monarquia); altos funcionários da monarquia}
\end{EntryWithPhonetic}

\begin{EntryWithPhonetic}{大城}{da4cheng2}{3,9}{⼤,⼟}
  \definition*[个,座]{s.}{Condado de Dacheng em Langfang 廊坊, Hebei | Município de Tacheng no condado de Changhua | Condado de Changhua, Taiwan}
  \seealsoref{廊坊}{lang2fang2}
\end{EntryWithPhonetic}

\begin{EntryWithPhonetic}{大吃一惊}{da4chi1-yi1jing1}{3,6,1,11}{⼤,⼝,⼀,⼼}[HSK 7-9]
  \definition{expr.}{ficar assustado com; ficar espantado com; ficar completamente surpreso; ficar muito surpreso; ficar completamente chocado; ser pego de surpresa; levar um choque; ter (levar) um susto; levar um susto com; ficar pasmo}
\end{EntryWithPhonetic}

\begin{EntryWithPhonetic}{大大}{da4da5}{3,3}{⼤,⼤}[HSK 2]
  \definition{adv.}{grandemente; enormemente; enfatizar grande quantidade ou grau profundo}
\end{EntryWithPhonetic}

\begin{EntryWithPhonetic}{大大咧咧}{da4da5lie1lie1}{3,3,9,9}{⼤,⼤,⼝,⼝}[HSK 7-9]
  \definition{expr.}{despreocupado; descuidado; casual}
\end{EntryWithPhonetic}

\begin{EntryWithPhonetic}{大胆}{da4dan3}{3,9}{⼤,⾁}[HSK 5]
  \definition{adj.}{ousado; atrevido; audacioso; corajoso; destemido}
\end{EntryWithPhonetic}

\begin{EntryWithPhonetic}{大道}{da4dao4}{3,12}{⼤,⾡}[HSK 6]
  \definition*{s.}{O Grande Tao; O Grande Caminho}
  \definition[条]{s.}{estrada principal | o caminho da justiça | avenida | rua principal}
\end{EntryWithPhonetic}

\begin{EntryWithPhonetic}{大抵}{da4di3}{3,8}{⼤,⼿}
  \definition{adv.}{no geral; de um modo geral; provavelmente; principalmente}
\end{EntryWithPhonetic}

\begin{EntryWithPhonetic}{大地}{da4di4}{3,6}{⼤,⼟}[HSK 7-9]
  \definition*{s.}{A Terra}[大地正在遭受污染。===A Terra está sendo poluída.]
  \definition[片,块,方]{s.}{terreno espaçoso; terra}
\end{EntryWithPhonetic}

\begin{EntryWithPhonetic}{大都}{da4dou1}{3,10}{⼤,⾢}
  \seeref{da4du1}
\end{EntryWithPhonetic}

\begin{EntryWithPhonetic}{大豆}{da4dou4}{3,7}{⼤,⾖}
  \definition{s.}{soja}
\end{EntryWithPhonetic}

\begin{EntryWithPhonetic}{大都}{da4du1}{3,10}{⼤,⾢}[HSK 5]
  \definition*{s.}{Dadu, capital da China durante a Dinastia Yuan (1280-1368), atual Pequim}
  \definition{adv.}{em sua maior parte; na maior parte; indica que a maioria das pessoas ou coisas em um determinado intervalo tem a mesma natureza e características; também pronunciado como \dpy{da4dou1} na língua falada}
  \seeref{da4dou1}
\end{EntryWithPhonetic}

\begin{EntryWithPhonetic}{大队}{da4dui4}{3,4}{⼤,⾩}[HSK 7-9]
  \definition{adj.}{grande (contingente de tropas, corpo de manifestantes, número de pessoas, etc.)}
  \definition{s.}{unidade militar correspondente ao batalhão ou regimento; grupo | História: brigada de produção (de uma comuna popular rural); brigada}
\end{EntryWithPhonetic}

\begin{EntryWithPhonetic}{大多}{da4duo1}{3,6}{⼤,⼣}[HSK 4]
  \definition{adv.}{majoritariamente; em sua maior parte; em sua maioria; em grande parte}
\end{EntryWithPhonetic}

\begin{EntryWithPhonetic}{大多数}{da4duo1shu4}{3,6,13}{⼤,⼣,⽁}[HSK 2]
  \definition{s.}{grande maioria; vasta maioria; a maior parte; mais da metade, um número significativo}
\end{EntryWithPhonetic}

\begin{EntryWithPhonetic}{大方}{da4fang1}{3,4}{⼤,⽅}
  \definition{s.}{generosidades; liberalidades | estudioso; pessoas com conhecimento especializado | um tipo de chá verde, produzido principalmente no Condado de Shexian, Província de Anhui, Condado de Chun'an, Província de Zhejiang, etc.}
  \seeref{da4fang5}
\end{EntryWithPhonetic}

\begin{EntryWithPhonetic}{大方}{da4fang5}{3,4}{⼤,⽅}[HSK 4]
  \definition{adj.}{generoso | não afetado; natural e equilibrado |  de bom gosto}
  \seeref{da4fang1}
\end{EntryWithPhonetic}

\begin{EntryWithPhonetic}{大夫}{da4fu1}{3,4}{⼤,⼤}
  \definition[个,位,名]{s.}{oficial sênior (na China Imperial)}
  \seeref{dai4fu5}
\end{EntryWithPhonetic}

\begin{EntryWithPhonetic}{大幅度}{da4 fu2du4}{3,12,9}{⼤,⼱,⼴}[HSK 7-9]
  \definition{adv.}{drasticamente; substancial; por uma ampla margem; a extensão da mudança ou alteração é significativa}
\end{EntryWithPhonetic}

\begin{EntryWithPhonetic}{大概}{da4gai4}{3,13}{⼤,⽊}[HSK 3]
  \definition{adj.}{geral; grosseiro; aproximado; não é muito preciso ou muito detalhado}
  \definition{adv.}{sobre; provavelmente; estimativas ou suposições imprecisas sobre eventos, quantidades, tempo, localização, etc. | geralmente; brevemente; não muito seriamente, casualmente; não muito cuidadosamente}
  \definition{s.}{ideia geral; esboço geral; conteúdo geral ou situação}
\end{EntryWithPhonetic}

\begin{EntryWithPhonetic}{大纲}{da4gang1}{3,7}{⼤,⽷}[HSK 5]
  \definition{s.}{esboço; compêndio; programa de estudos; resumo; fundamentos da organização sistemática de conteúdos (livros, discursos, programas, etc.)}
\end{EntryWithPhonetic}

\begin{EntryWithPhonetic}{大哥}{da4ge1}{3,10}{⼤,⼝}[HSK 4]
  \definition{s.}{irmão mais velho | \emph{big brother}; tratamento educado para um homem da mesma idade que você | líder de gangue; pessoa mais poderosa em uma organização que realiza atividades ilegais na sociedade}
\end{EntryWithPhonetic}

\begin{EntryWithPhonetic}{大公无私}{da4gong1-wu2si1}{3,4,4,7}{⼤,⼋,⽆,⽲}[HSK 7-9]
  \definition{expr.}{altruísta; generoso; desinteressado | perfeitamente imparcial | nenhuma consideração pessoal (egoísta); não pensar em si mesmo; justo e altruísta; grande imparcialidade exclui consideração de si mesmo; desinteresse e altruísmo}
\end{EntryWithPhonetic}

\begin{EntryWithPhonetic}{大规模}{da4gui1mo2}{3,8,14}{⼤,⾒,⽊}[HSK 4]
  \definition{adj.}{em larga escala; extensivo; maciço; massivo}
  \definition{adv.}{em larga escala; extensivo; maciço; massa}
\end{EntryWithPhonetic}

\begin{EntryWithPhonetic}{大海}{da4hai3}{3,10}{⼤,⽔}[HSK 2]
  \definition{s.}{o mar; o oceano; o mar aberto, ou seja, a parte do oceano que não está fechada entre cabos nem incluída em estreitos}
\end{EntryWithPhonetic}

\begin{EntryWithPhonetic}{大后天}{da4 hou4 tian1}{3,6,4}{⼤,⼝,⼤}
  \definition{s.}{daqui a três dias}
\end{EntryWithPhonetic}

\begin{EntryWithPhonetic}{大黄}{da4huang2}{3,11}{⼤,⿈}
  \definition{s.}{ruibarbo chinês}
\end{EntryWithPhonetic}

\begin{EntryWithPhonetic}{大会}{da4hui4}{3,6}{⼤,⼈}[HSK 4]
  \definition[场,次,个,届]{s.}{sessão plenária; reunião geral de membros; reuniões convocadas por partidos políticos socialistas | reunião de massa; comício de massa}
\end{EntryWithPhonetic}

\begin{EntryWithPhonetic}{大伙儿}{da4huo3r5}{3,6,2}{⼤,⼈,⼉}[HSK 5]
  \definition{pron.}{todos nós; todos vocês; todo mundo; todos; equivalente a 大家}
  \seealsoref{大家}{da4jia1}
\end{EntryWithPhonetic}

\begin{EntryWithPhonetic}{大家}{da4jia1}{3,10}{⼤,⼧}[HSK 2]
  \definition{pron.}{todos; toda a gente; refere"-se a todas as pessoas dentro de um determinado âmbito}
  \definition{s.}{grande mestre; autoridade; especialista renomado | família nobre; família rica e influente; família tradicional}
\end{EntryWithPhonetic}

\begin{EntryWithPhonetic}{大家庭}{da4jia1ting2}{3,10,9}{⼤,⼧,⼴}[HSK 7-9]
  \definition[个,些]{s.}{grande família; comunidade | uma grande família é frequentemente usada para descrever um grupo com muitos membros e harmonia interna}
\end{EntryWithPhonetic}

\begin{EntryWithPhonetic}{大奖赛}{da4jiang3sai4}{3,9,14}{⼤,⼤,⾙}[HSK 5]
  \definition{s.}{grande competição; grande prêmio; \emph{grand prix}}
\end{EntryWithPhonetic}

\begin{EntryWithPhonetic}{大街}{da4jie1}{3,12}{⼤,⾏}[HSK 6]
  \definition[条,个]{s.}{avenida; rua; rua principal}
\end{EntryWithPhonetic}

\begin{EntryWithPhonetic}{大街小巷}{da4jie1-xiao3xiang4}{3,12,3,9}{⼤,⾏,⼩,⼰}[HSK 7-9]
  \definition{expr.}{grandes ruas e pequenos becos; em todos os lugares da cidade}
\end{EntryWithPhonetic}

\begin{EntryWithPhonetic}{大姐}{da4jie3}{3,8}{⼤,⼥}[HSK 4]
  \definition[个,位]{s.}{irmã mais velha (também um termo educado para se dirigir a uma garota ou mulher um pouco mais velha do que a pessoa que fala)}
\end{EntryWithPhonetic}

\begin{EntryWithPhonetic}{大惊小怪}{da4jing1-xiao3guai4}{3,11,3,8}{⼤,⼼,⼩,⼼}[HSK 7-9]
  \definition{expr.}{ficar animado com uma coisa pequena; uma tempestade em um copo d'água; ficar surpreso com algo perfeitamente normal; ficar alarmado com algo perfeitamente normal; sentir-se surpreso; ficar alarmado com; ficar todo agitado por nada; grande alarme com um pequeno fantasma; fazer um alarido; fazer um alarido por nada; fazer um grande alarido sobre (algo); fazer um raro alarido sobre (algo); fazer muito alarido; fazer barulho por nada}
\end{EntryWithPhonetic}

\begin{EntryWithPhonetic}{大局}{da4ju2}{3,7}{⼤,⼫}[HSK 7-9]
  \definition{s.}{situação geral (ou inteira); toda a situação}
\end{EntryWithPhonetic}

\begin{EntryWithPhonetic}{大口}{da4kou3}{3,3}{⼤,⼝}
  \definition{s.}{grande bocado (de comida, bebida, fumo, etc.)}
\end{EntryWithPhonetic}

\begin{EntryWithPhonetic}{大款}{da4kuan3}{3,12}{⼤,⽋}[HSK 7-9]
  \definition[个]{s.}{o rico; pessoa muito rica}
\end{EntryWithPhonetic}

\begin{EntryWithPhonetic}{大力}{da4li4}{3,2}{⼤,⼒}[HSK 6]
  \definition{adv.}{energicamente; vigorosamente; indica uso de grande força}
  \definition{s.}{grande força, poder}
\end{EntryWithPhonetic}

\begin{EntryWithPhonetic}{大量}{da4liang4}{3,12}{⼤,⾥}[HSK 2]
  \definition{adj.}{numeroso; em grande quantidade; grande em número ou quantidade | generoso; magnânimo; descreve uma pessoa que não fica zangada quando os outros cometem erros e que costuma perdoar os outros}
\end{EntryWithPhonetic}

\begin{EntryWithPhonetic}{大楼}{da4lou2}{3,13}{⼤,⽊}[HSK 4]
  \definition[座,幢]{s.}{edifício; mansão; edifício de vários andares disponível para uso residencial e comercial}
\end{EntryWithPhonetic}

\begin{EntryWithPhonetic}{大陆}{da4lu4}{3,7}{⼤,⾩}[HSK 4]
  \definition*{s.}{China continental; refere"-se especificamente à vasta porção terrestre do território da China}
  \definition[个,块]{s.}{terra firme; continente; vasta extensão de terra}
\end{EntryWithPhonetic}

\begin{EntryWithPhonetic}{大妈}{da4ma1}{3,6}{⼤,⼥}[HSK 4]
  \definition[个,位]{s.}{tia; esposa do irmão mais velho do pai | tratamento respeitoso às mulheres idosas}
\end{EntryWithPhonetic}

\begin{EntryWithPhonetic}{大马}{da4ma3}{3,3}{⼤,⾺}
  \definition*{s.}{Malásia}
\end{EntryWithPhonetic}

\begin{EntryWithPhonetic}{大门}{da4men2}{3,3}{⼤,⾨}[HSK 2]
  \definition{s.}{portão; entrada; portão grande, referindo"-se especificamente ao portão principal de um edifício (como uma casa, pátio ou parque) que dá para a rua (em contraste com o segundo portão e as portas das várias divisões)}
\end{EntryWithPhonetic}

\begin{EntryWithPhonetic}{大米}{da4mi3}{3,6}{⼤,⽶}[HSK 6]
  \definition[颗,粒,斤,包,袋]{s.}{arroz; arroz descascado; arroz bom}
\end{EntryWithPhonetic}

\begin{EntryWithPhonetic}{大面积}{da4 mian4ji1}{3,9,10}{⼤,⾯,⽲}[HSK 7-9]
  \definition{s.}{grande área}
\end{EntryWithPhonetic}

\begin{EntryWithPhonetic}{大名鼎鼎}{da4ming2-ding3ding3}{3,6,12,12}{⼤,⼝,⿍,⿍}[HSK 7-9]
  \definition{expr.}{``O nome de alguém é conhecido em toda parte.''; ser muito famoso; ser amplamente conhecido; celebrado; desfrutar de um grande nome; bem conhecido}
\end{EntryWithPhonetic}

\begin{EntryWithPhonetic}{大模大样}{da4mu2-da4yang4}{3,14,3,10}{⼤,⽊,⼤,⽊}[HSK 7-9]
  \definition{expr.}{pomposamente; corajosamente; ostensivamente; equilibrado; autoconfiante}
\end{EntryWithPhonetic}

\begin{EntryWithPhonetic}{大脑}{da4nao3}{3,10}{⼤,⾁}[HSK 5]
  \definition{s.}{cérebro; encéfalo}
\end{EntryWithPhonetic}

\begin{EntryWithPhonetic}{大娘}{da4niang2}{3,10}{⼤,⼥}
  \definition[个,位]{s.}{a esposa do irmão mais velho do pai; tia | tia; termo respeitoso para se referir a uma mulher mais velha}
  \synonymref{阿姨}{a1yi2}
  \synonymref{大妈}{da4ma1}
\end{EntryWithPhonetic}

\begin{EntryWithPhonetic}{大棚}{da4peng2}{3,12}{⼤,⽊}[HSK 7-9]
  \definition{s.}{grande abrigo em arco coberto com lona plástica usada na agricultura; grande estufa | estufa}
\end{EntryWithPhonetic}

\begin{EntryWithPhonetic}{大批}{da4pi1}{3,7}{⼤,⼿}[HSK 6]
  \definition{num.}{grandes quantidades de; exércitos; inundações}[大批书籍被印刷出来。===Grandes quantidades de livros foram impressas.]
\end{EntryWithPhonetic}

\begin{EntryWithPhonetic}{大片}{da4pian4}{3,4}{⼤,⽚}[HSK 7-9]
  \definition{adj.}{grande; enorme; imenso}
  \definition{s.}{um filme de grande orçamento; um sucesso de bilheteria; \emph{block-buster}; refere"-se a um filme caro e bem feito | grande área; vasta extensão; ampla extensão}
\end{EntryWithPhonetic}

\begin{EntryWithPhonetic}{大气}{da4qi4}{3,4}{⼤,⽓}[HSK 7-9]
  \definition{adj.}{generoso; magnânimo; refere"-se a um porte extraordinário, não convencional}
  \definition{s.}{ar; atmosfera | respiração pesada}
\end{EntryWithPhonetic}

\begin{EntryWithPhonetic}{大前天}{da4qian2tian1}{3,9,4}{⼤,⼑,⼤}
  \definition{adv.}{três dias atrás}
\end{EntryWithPhonetic}

\begin{EntryWithPhonetic}{大全}{da4quan2}{3,6}{⼤,⼊}
  \definition{s.}{coleção abrangente}
\end{EntryWithPhonetic}

\begin{EntryWithPhonetic}{大人}{da4ren2}{3,2}{⼤,⼈}[HSK 2]
  \definition[个,位]{s.}{senhor; ilustre; sua excelência; antigo título honorífico para funcionários públicos | adulto; crescido; maduro;}
\end{EntryWithPhonetic}

\begin{EntryWithPhonetic}{大赛}{da4sai4}{3,14}{⼤,⾙}[HSK 6]
  \definition{s.}{grande torneio; competição importante; um evento de grande porte e alto nível; um grande evento}
\end{EntryWithPhonetic}

\begin{EntryWithPhonetic}{大厦}{da4sha4}{3,12}{⼤,⼚}[HSK 7-9]
  \definition[栋,座,撞]{s.}{prédio; edifício grande (frequentemente usado em nomes de grandes edifícios)}
\end{EntryWithPhonetic}

\begin{EntryWithPhonetic}{大神}{da4shen2}{3,9}{⼤,⽰}
  \definition{s.}{deidade | (gíria da Internet) guru | \emph{expert} | gênio}
\end{EntryWithPhonetic}

\begin{EntryWithPhonetic}{大声}{da4sheng1}{3,7}{⼤,⼠}[HSK 2]
  \definition{adj.}{alto; volume alto; em voz alta}
\end{EntryWithPhonetic}

\begin{EntryWithPhonetic}{大师}{da4shi1}{3,6}{⼤,⼱}[HSK 6]
  \definition*{s.}{Grande Mestre, título de cortesia usado para se dirigir a um monge budista}
  \definition{s.}{grande mestre; mestre; maestro; uma pessoa com realizações profundas}
\end{EntryWithPhonetic}

\begin{EntryWithPhonetic}{大使}{da4shi3}{3,8}{⼤,⼈}[HSK 6]
  \definition[位,任]{s.}{embaixador; o representante diplomático de mais alto nível enviado por um país a outro país}
  \seealsoref{全称特命全权大使}{quan2cheng1 te4ming4 quan2quan2 da4shi3}
\end{EntryWithPhonetic}

\begin{EntryWithPhonetic}{大使馆}{da4shi3guan3}{3,8,11}{⼤,⼈,⾷}[HSK 3]
  \definition[座,个]{s.}{embaixada; uma representação diplomática de um país em outro país, chefiada por um embaixador}
\end{EntryWithPhonetic}

\begin{EntryWithPhonetic}{大事}{da4shi4}{3,8}{⼤,⼅}[HSK 5]
  \definition{adv.}{em grande escala; em grande estilo; em grande parte}
  \definition[件,桩]{s.}{grande evento; grande acontecimento; assunto importante; grande questão; algo importante | situação geral}
\end{EntryWithPhonetic}

\begin{EntryWithPhonetic}{大数据}{da4shu4ju4}{3,13,11}{⼤,⽁,⼿}[HSK 7-9]
  \definition*{s.}{\emph{Big Data}}
\end{EntryWithPhonetic}

\begin{EntryWithPhonetic}{大肆}{da4si4}{3,13}{⼤,⾀}[HSK 7-9]
  \definition{adv.}{desenfreadamente; violentamente; imprudentemente; sem restrições; sem escrúpulos (geralmente referindo"-se a fazer coisas ruins)}
\end{EntryWithPhonetic}

\begin{EntryWithPhonetic}{大蒜}{da4suan4}{3,13}{⼤,⾋}
  \definition[瓣,头]{s.}{alho}
\end{EntryWithPhonetic}

\begin{EntryWithPhonetic}{大体}{da4ti3}{3,7}{⼤,⼈}[HSK 7-9]
  \definition{adv.}{aproximadamente; em geral; mais ou menos}
  \definition{s.}{princípio primordial; interesse geral; um princípio basilar baseado na situação geral; um princípio relativo à situação geral}
\end{EntryWithPhonetic}

\begin{EntryWithPhonetic}{大体上}{da4ti3 shang4}{3,7,3}{⼤,⼈,⼀}[HSK 7-9]
  \definition{adv.}{em geral; grosso modo; de um modo geral}
\end{EntryWithPhonetic}

\begin{EntryWithPhonetic}{大厅}{da4ting1}{3,4}{⼤,⼚}[HSK 5]
  \definition{s.}{\emph{hall}; saguão, uma sala grande para reuniões ou atividades em um edifício de grande porte}
\end{EntryWithPhonetic}

\begin{EntryWithPhonetic}{大同小异}{da4tong2-xiao3yi4}{3,6,3,6}{⼤,⼝,⼩,⼶}[HSK 7-9]
  \definition{expr.}{muito parecidos, mas com pequenas diferenças; semelhante, exceto por pequenas diferenças; ser o mesmo em aspectos essenciais, embora diferindo em pontos menores; uma semelhança geral com pequenas (menores) diferenças; ser semelhante em todos os aspectos essenciais importantes, sendo as diferenças de natureza menor; ser em grande parte idêntico com apenas pequenas diferenças; diferir apenas em (em) pequenos pontos; essencialmente o mesmo, embora diferindo em pontos menores; na maior parte, eles são os mesmos; idênticos em questões importantes, embora com pequenas diferenças; principalmente semelhantes, exceto (por) pequenas diferenças; virtualmente os mesmos}
\end{EntryWithPhonetic}

\begin{EntryWithPhonetic}{大腿}{da4tui3}{3,13}{⼤,⾁}
  \definition{s.}{coxa}
\end{EntryWithPhonetic}

\begin{EntryWithPhonetic}{大腕儿}{da4wan4r5}{3,12,2}{⼤,⾁,⼉}[HSK 7-9]
  \definition[位]{s.}{grande nome, figurão; uma pessoa que tem grande habilidade, fama e influência em um determinado setor ou aspecto}
\end{EntryWithPhonetic}

\begin{EntryWithPhonetic}{大王}{da4wang2}{3,4}{⼤,⽟}
  \definition{s.}{rei; magnata | pessoa da mais alta classe ou habilidade em algo; ás | barões | pessoa com habilidade especializada em algo}
  \seeref{dai4wang5}
\end{EntryWithPhonetic}

\begin{EntryWithPhonetic}{大戏}{da4xi4}{3,6}{⼤,⼽}
  \definition*{s.}{Drama, Ópera Chinesa}
\end{EntryWithPhonetic}

\begin{EntryWithPhonetic}{大象}{da4xiang4}{3,11}{⼤,⾗}[HSK 5]
  \definition[只,头,群,个]{s.}{elefante}
\end{EntryWithPhonetic}

\begin{EntryWithPhonetic}{大小}{da4xiao3}{3,3}{⼤,⼩}[HSK 2]
  \definition{adv.}{no mínimo; grande ou pequeno (geralmente pequeno), significa que ainda pode ser considerado}
  \definition[家]{s.}{tamanho; o grau de tamanho | ordem de senioridade; hierarquia | adultos e crianças | grande ou pequeno}
\end{EntryWithPhonetic}

\begin{EntryWithPhonetic}{大猩猩}{da4xing1xing5}{3,12,12}{⼤,⽝,⽝}
  \definition{s.}{gorila}
\end{EntryWithPhonetic}

\begin{EntryWithPhonetic}{大型}{da4xing2}{3,9}{⼤,⼟}[HSK 4]
  \definition{adj.}{grande; em larga escala; tamanho e volume grandes | larga escala (importante e influente)}
\end{EntryWithPhonetic}

\begin{EntryWithPhonetic}{大熊猫}{da4xiong2mao1}{3,14,11}{⼤,⽕,⽝}[HSK 5]
  \definition{s.}{panda gigante}
\end{EntryWithPhonetic}

\begin{EntryWithPhonetic}{大选}{da4xuan3}{3,9}{⼤,⾡}[HSK 7-9]
  \definition[次]{s.}{eleição geral; refere"-se à eleição de membros do parlamento ou presidente em alguns países}
\end{EntryWithPhonetic}

\begin{EntryWithPhonetic}{大学}{da4xue2}{3,8}{⼤,⼦}[HSK 1]
  \definition[所,座]{s.}{universidade; faculdade; tipo de instituição de ensino superior que, na China, geralmente se refere a uma universidade abrangente}
\end{EntryWithPhonetic}

\begin{EntryWithPhonetic}{大学生}{da4xue2sheng1}{3,8,5}{⼤,⼦,⽣}[HSK 1]
  \definition[名,个]{s.}{estudante universitário; estudante de faculdade; estudantes de graduação ou cursos técnicos em instituições de ensino superior}
\end{EntryWithPhonetic}

\begin{EntryWithPhonetic}{大雁}{da4yan4}{3,12}{⼤,⾫}[HSK 7-9]
  \definition[只,群,行]{s.}{ganso selvagem}
\end{EntryWithPhonetic}

\begin{EntryWithPhonetic}{大洋洲}{da4yang2zhou1}{3,9,9}{⼤,⽔,⽔}
  \definition*{s.}{Oceania}
\end{EntryWithPhonetic}

\begin{EntryWithPhonetic}{大爷}{da4ye2}{3,6}{⼤,⽗}
  \definition[个,位]{s.}{Coloquial: irmão mais velho do pai; tio | tratamento respeitoso para um homem mais velho}
  \seeref{da4ye5}
\end{EntryWithPhonetic}

\begin{EntryWithPhonetic}{大爷}{da4ye5}{3,6}{⼤,⽗}[HSK 4]
  \definition[个,位]{s.}{irmão mais velho do pai; tio | tio (homenagem aos homens mais velhos)}
  \seeref{da4ye2}
\end{EntryWithPhonetic}

\begin{EntryWithPhonetic}{大衣}{da4yi1}{3,6}{⼤,⾐}[HSK 2]
  \definition[件,个]{s.}{sobretudo; casaco; casaco ocidental mais comprido}
\end{EntryWithPhonetic}

\begin{EntryWithPhonetic}{大姨}{da4yi2}{3,9}{⼤,⼥}
  \definition{s.}{irmã mais velha da mãe; tia | Coloquial: irmã mais velha da esposa | cunhada}
  \seealsoref{大姨儿}{da4yi2r5}
  \synonymref{阿姨}{a1yi2}
\end{EntryWithPhonetic}

\begin{EntryWithPhonetic}{大姨儿}{da4yi2r5}{3,9,2}{⼤,⼥,⼉}
  \definition{s.}{tia}
\end{EntryWithPhonetic}

\begin{EntryWithPhonetic}{大意}{da4yi4}{3,13}{⼤,⼼}[HSK 7-9]
  \definition{s.}{essência; teor; ideia geral; pontos principais; o significado principal}
  \seeref{da4yi5}
\end{EntryWithPhonetic}

\begin{EntryWithPhonetic}{大意}{da4yi5}{3,13}{⼤,⼼}[HSK 7-9]
  \definition{adj.}{descuidado; negligente; desatento}
  \seeref{da4yi4}
\end{EntryWithPhonetic}

\begin{EntryWithPhonetic}{大有可为}{da4you3-ke3wei2}{3,6,5,4}{⼤,⽉,⼝,⼂}[HSK 7-9]
  \definition{expr.}{ter um futuro brilhante; valer a pena; poder realizar grandes coisas; ter perspectivas brilhantes; pode ser bem administrado; há amplo escopo para nossas habilidades em\dots; valer a pena fazer; ter perspectivas brilhantes}
\end{EntryWithPhonetic}

\begin{EntryWithPhonetic}{大于}{da4yu2}{3,3}{⼤,⼆}[HSK 5]
  \definition{v.}{ser maior, mais numeroso, mais importante, etc. do que}
\end{EntryWithPhonetic}

\begin{EntryWithPhonetic}{大雨}{da4yu3}{3,8}{⼤,⾬}
  \definition[场]{s.}{chuva pesada, forte}
\end{EntryWithPhonetic}

\begin{EntryWithPhonetic}{大约}{da4yue1}{3,6}{⼤,⽷}[HSK 3]
  \definition{adv.}{aproximadamente; sobre; estimativa não muito precisa | provavelmente; expressar suposições sobre a situação}
\end{EntryWithPhonetic}

\begin{EntryWithPhonetic}{大战}{da4zhan4}{3,9}{⼤,⼽}
  \definition{s.}{guerra}
  \definition{v.}{guerrear | lutar em uma guerra}
\end{EntryWithPhonetic}

\begin{EntryWithPhonetic}{大致}{da4zhi4}{3,10}{⼤,⾄}[HSK 5]
  \definition{adj.}{geral; no todo}
  \definition{adv.}{grosso modo; aproximadamente; mais ou menos; indica uma estimativa aproximada da situação}
\end{EntryWithPhonetic}

\begin{EntryWithPhonetic}{大众}{da4zhong4}{3,6}{⼤,⼈}[HSK 4]
  \definition{s.}{massas; população; pessoas comuns; público em geral}
\end{EntryWithPhonetic}

\begin{EntryWithPhonetic}{大自然}{da4zi4ran2}{3,6,12}{⼤,⾃,⽕}[HSK 2]
  \definition{s.}{natureza}
\end{EntryWithPhonetic}

\begin{EntryWithPhonetic}{大宗}{da4zong1}{3,8}{⼤,⼧}[HSK 7-9]
  \definition{adj.}{grande; volumoso}
  \definition{s.}{itens principais; produtos ou bens básicos}
\end{EntryWithPhonetic}

%%%%%%%%%% 呆 %%%%%%%%%%
\subsection*{呆}\addcontentsline{loh}{figure}{呆 \dpy{dai1}}

\begin{EntryWithPhonetic}{呆}{dai1}{7}{⼝}[HSK 5]
  \definition*{s.}{Sobrenome: Dai}
  \definition{adj.}{maçante; de raciocínio lento | em branco; de madeira; rígido; inflexível}
  \definition{v.}{ficar; permanecer}
\end{EntryWithPhonetic}

%%%%%%%%%% 待 %%%%%%%%%%
\subsection*{待}\addcontentsline{loh}{figure}{待 \dpy{dai1}}

\begin{EntryWithPhonetic}{待}{dai1}{9}{⼻}[HSK 5]
  \definition{v.}{ficar; permanecer | ir além (de um período de tempo)}
  \seeref{dai4}
\end{EntryWithPhonetic}

\begin{EntryWithPhonetic}{待会儿}{dai1hui4r5}{9,6,2}{⼻,⼈,⼉}[HSK 6]
  \definition{adv.}{em um momento; depois de um tempo | mais tarde; depois}
\end{EntryWithPhonetic}

%%%%%%%%%% 歹 %%%%%%%%%%
\subsection*{歹}\addcontentsline{loh}{figure}{歹 \dpy{dai3}}

\begin{EntryWithPhonetic}{歹}{dai3}{4}{⽍}[Kangxi 78]
  \definition{adj.}{maligno; cruel; ruim, refere"-se principalmente a pessoas e coisas}
\end{EntryWithPhonetic}

\begin{EntryWithPhonetic}{歹徒}{dai3tu2}{4,10}{⽍,⼻}[HSK 7-9]
  \definition[个,伙,群]{s.}{rufião; malfeitor; arruaceiro; canalha}
\end{EntryWithPhonetic}

%%%%%%%%%% 逮 %%%%%%%%%%
\subsection*{逮}\addcontentsline{loh}{figure}{逮 \dpy{dai3}}

\begin{EntryWithPhonetic}{逮}{dai3}{11}{⾡}[HSK 7-9]
  \definition{v.}{capturar; pegar}[猫逮老鼠。===Gatos pegam ratos.]
  \seeref{dai4}
\end{EntryWithPhonetic}

%%%%%%%%%% 大 %%%%%%%%%%
\subsection*{大}\addcontentsline{loh}{figure}{大 \dpy{dai4}}

\begin{EntryWithPhonetic}{大}{dai4}{3}{⼤}[Kangxi 37]
  \definition{s.}{usado em 大夫: médico, doutor | usado em 大王: grande rei}
  \seeref{da4}
  \seealsoref{大夫}{dai4fu5}
  \seealsoref{大王}{dai4wang5}
\end{EntryWithPhonetic}

\begin{EntryWithPhonetic}{大夫}{dai4fu5}{3,4}{⼤,⼤}[HSK 3]
  \definition[个,位,名]{s.}{médico, doutor}
  \seeref{da4fu1}
\end{EntryWithPhonetic}

\begin{EntryWithPhonetic}{大王}{dai4wang5}{3,4}{⼤,⽟}
  \definition{s.}{magnata; barões | barão ladrão (em ópera, histórias antigas)}
  \seeref{da4wang2}
\end{EntryWithPhonetic}

%%%%%%%%%% 代 %%%%%%%%%%
\subsection*{代}\addcontentsline{loh}{figure}{代 \dpy{dai4}}

\begin{EntryWithPhonetic}{代}{dai4}{5}{⼈}[HSK 3]
  \definition*{s.}{Sobrenome: Dai}
  \definition{s.}{dinastia | geração; hierarquia familiar | era; o segundo nível da divisão geológica é o período, acima do qual está a era e abaixo do qual está o período, por exemplo, o Paleozóico, o Mesozóico e o Cenozóico pertencem à era Phanerozoico | período histórico; época}
  \definition{v.}{tomar o lugar de; estar no lugar de | agir em nome de; exercer}
\end{EntryWithPhonetic}

\begin{EntryWithPhonetic}{代表}{dai4biao3}{5,8}{⼈,⾐}[HSK 3]
  \definition[位,名,个,些]{s.}{deputado; delegado; representante; pessoas eleitas para representar eleitores ou expressar opiniões, ou pessoas encarregadas ou designadas para representar indivíduos, grupos ou governos ou expressar opiniões | representante oficial; pessoas ou coisas que refletem as características comuns de um grupo específico}
  \definition{v.}{representar; defender | usar pessoas ou coisas para expressar um significado ou conceito específico}
\end{EntryWithPhonetic}

\begin{EntryWithPhonetic}{代表团}{dai4biao3tuan2}{5,8,6}{⼈,⾐,⼞}[HSK 3]
  \definition[个]{s.}{delegação; contingente; um grupo temporário de grande dimensão formado para participar de uma determinada atividade em nome de um país, governo ou outra organização social}
\end{EntryWithPhonetic}

\begin{EntryWithPhonetic}{代称}{dai4cheng1}{5,10}{⼈,⽲}
  \definition{s.}{nome alternativo | antonomásia}
  \definition{v.}{referir-se a algo ou alguém por outro nome}
\end{EntryWithPhonetic}

\begin{EntryWithPhonetic}{代号}{dai4hao4}{5,5}{⼈,⼝}[HSK 7-9]
  \definition[个]{s.}{codinome | designação; marca; código}
\end{EntryWithPhonetic}

\begin{EntryWithPhonetic}{代价}{dai4jia4}{5,6}{⼈,⼈}[HSK 5]
  \definition[种,个]{s.}{preço; material, energia gasta ou sacrifícios feitos para atingir um objetivo | custo; preço; dinheiro pago para obter algo}
\end{EntryWithPhonetic}

\begin{EntryWithPhonetic}{代理}{dai4li3}{5,11}{⼈,⽟}[HSK 5]
  \definition{v.}{agir em nome de alguém em uma posição de responsabilidade; substituir alguém | agir como procurador; agir como agente; ser encarregado pelas partes de realizar atividades e conduzir assuntos em seu nome dentro do escopo de sua autorização}
\end{EntryWithPhonetic}

\begin{EntryWithPhonetic}{代理人}{dai4li3ren2}{5,11,2}{⼈,⽟,⼈}[HSK 7-9]
  \definition{s.}{agente; procurador; representante; uma pessoa encarregada de agir em nome de uma parte | advogado; procurador; refere"-se a profissionais que possuem qualificações em prática jurídica e são encarregados de fornecer serviços jurídicos aos clientes}
\end{EntryWithPhonetic}

\begin{EntryWithPhonetic}{代替}{dai4ti4}{5,12}{⼈,⽈}[HSK 4]
  \definition{v.}{substituir; substituir por; tomar o lugar de}
\end{EntryWithPhonetic}

\begin{EntryWithPhonetic}{代言}{dai4yan2}{5,7}{⼈,⾔}
  \definition{v.}{ser um porta-voz | ser um embaixador (para uma marca) | endossar}
\end{EntryWithPhonetic}

\begin{EntryWithPhonetic}{代言人}{dai4yan2ren2}{5,7,2}{⼈,⾔,⼈}[HSK 7-9]
  \definition{s.}{porta-voz; uma pessoa que fala em nome de um determinado partido (classe, grupo, etc.)}
\end{EntryWithPhonetic}

%%%%%%%%%% 带 %%%%%%%%%%
\subsection*{带}\addcontentsline{loh}{figure}{带 \dpy{dai4}}

\begin{EntryWithPhonetic}{带}{dai4}{9}{⼱}[HSK 2]
  \definition*{s.}{Sobrenome: Dai}
  \definition[根]{s.}{cinto; faixa; banda; fita; fita adesiva; algo parecido com uma fita | pneu | zona; área; faixa; cinturão; região; uma determinada área geográfica com determinadas características | leucorreia; corrimento branco; corrimento vaginal}
  \definition{v.}{levar; trazer; transportar | liderar; dirigir; conduzir; assumir | cuidar de crianças; criar filhos; educar | fazer uma coisa e, ao mesmo tempo, fazer outra coisa |suportar; conter | ter algo anexado, simultâneo | trazer consigo | carregar consigo | demonstrar; parecer | incluir; acrescentar}
\end{EntryWithPhonetic}

\begin{EntryWithPhonetic}{带动}{dai4 dong4}{9,6}{⼱,⼒}[HSK 3]
  \definition{v.}{dirigir; ativar; fazer algo funcionar; acionar | liderar; trazer; estimular; motivar; atrair; liderar o avanço; dar o exemplo e fazer com que os outros sigam o exemplo}
\end{EntryWithPhonetic}

\begin{EntryWithPhonetic}{带队}{dai4dui4}{9,4}{⼱,⾩}[HSK 7-9]
  \definition{v.}{liderar um grupo (para fazer algo)}
\end{EntryWithPhonetic}

\begin{EntryWithPhonetic}{带来}{dai4 lai2}{9,7}{⼱,⽊}[HSK 2]
  \definition{v.}{provocar; produzir; causar}
\end{EntryWithPhonetic}

\begin{EntryWithPhonetic}{带领}{dai4ling3}{9,11}{⼱,⾴}[HSK 3]
  \definition{v.}{guiar, na frente, liderando | liderar e comandar}
\end{EntryWithPhonetic}

\begin{EntryWithPhonetic}{带路}{dai4/lu4}{9,13}{⼱,⾜}[HSK 7-9]
  \definition{v.+compl.}{mostrar o caminho; agir como um guia; guiar}
\end{EntryWithPhonetic}

\begin{EntryWithPhonetic}{带头}{dai4/tou2}{9,5}{⼱,⼤}[HSK 7-9]
  \definition{v.+compl.}{assumir a liderança; ser o primeiro; tomar a iniciativa; dar o exemplo}
\end{EntryWithPhonetic}

\begin{EntryWithPhonetic}{带头人}{dai4tou2 ren2}{9,5,2}{⼱,⼤,⼈}[HSK 7-9]
  \definition{s.}{líder; pioneiro}
\end{EntryWithPhonetic}

\begin{EntryWithPhonetic}{带有}{dai4you3}{9,6}{⼱,⽉}[HSK 5]
  \definition{v.}{ter; envolver; carregar; implicar}
\end{EntryWithPhonetic}

%%%%%%%%%% 待 %%%%%%%%%%
\subsection*{待}\addcontentsline{loh}{figure}{待 \dpy{dai4}}

\begin{EntryWithPhonetic}{待}{dai4}{9}{⼻}[HSK 7-9]
  \definition*{s.}{Sobrenome: Dai}
  \definition{v.}{tratar; lidar com | entreter; receber (convidados) | aguardar; esperar por | precisar; necessitar | desejar; pretender; querer}
  \seeref{dai1}
\end{EntryWithPhonetic}

\begin{EntryWithPhonetic}{待遇}{dai4yu4}{9,12}{⼻,⾡}[HSK 4]
  \definition[种,项,份]{s.}{tratamento; refere"-se a direitos, status social, etc. | salário; ordenado; remuneração}
\end{EntryWithPhonetic}

%%%%%%%%%% 怠 %%%%%%%%%%
\subsection*{怠}\addcontentsline{loh}{figure}{怠 \dpy{dai4}}

\begin{EntryWithPhonetic}{怠}{dai4}{9}{⼼}
  \definition{adj.}{ocioso; relaxado; negligente | preguiçoso; indolente}
  \definition{v.}{ficar ocioso; negligenciar; folgar}
\end{EntryWithPhonetic}

\begin{EntryWithPhonetic}{怠工}{dai4/gong1}{9,3}{⼼,⼯}[HSK 7-9]
  \definition{v.+compl.}{ir devagar (como uma forma de greve); ir devagar | afrouxar no trabalho}
\end{EntryWithPhonetic}

\begin{EntryWithPhonetic}{怠慢}{dai4man4}{9,14}{⼼,⼼}[HSK 7-9]
  \definition{v.}{ignorar; tratar com indiferença | deixar de dar a devida atenção para alguém}
\end{EntryWithPhonetic}

%%%%%%%%%% 贷 %%%%%%%%%%
\subsection*{贷}\addcontentsline{loh}{figure}{贷 \dpy{dai4}}

\begin{EntryWithPhonetic}{贷}{dai4}{9}{⾙}
  \definition[笔]{s.}{empréstimo; valor do empréstimo}
  \definition{v.}{pedir dinheiro emprestado ou emprestar dinheiro | fugir da responsabilidade | perdoar}
\end{EntryWithPhonetic}

\begin{EntryWithPhonetic}{贷款}{dai4kuan3}{9,12}{⾙,⽋}[HSK 5]
  \definition[个,笔]{s.}{empréstimo; crédito}
  \definition{v.}{fornecer um empréstimo; conceder um empréstimo; conceder crédito a; emprestar dinheiro para quem precisa}
\end{EntryWithPhonetic}

%%%%%%%%%% 袋 %%%%%%%%%%
\subsection*{袋}\addcontentsline{loh}{figure}{袋 \dpy{dai4}}

\begin{EntryWithPhonetic}{袋}{dai4}{11}{⾐}[HSK 4]
  \definition{clas.}{usado para coisas que podem ser colocadas nos bolsos | usado para cigarros, narguilé ou tabaco seco}
  \definition[口]{s.}{saco; sacola; bolso; bolsa}
\end{EntryWithPhonetic}

%%%%%%%%%% 逮 %%%%%%%%%%
\subsection*{逮}\addcontentsline{loh}{figure}{逮 \dpy{dai4}}

\begin{EntryWithPhonetic}{逮}{dai4}{11}{⾡}
  \definition*{s.}{Sobrenome: Dai}
  \definition{v.}{alcançar | prender, usado em 逮捕}
  \seeref{dai3}
  \seealsoref{逮捕}{dai4bu3}
\end{EntryWithPhonetic}

\begin{EntryWithPhonetic}{逮捕}{dai4bu3}{11,10}{⾡,⼿}[HSK 7-9]
  \definition{v.}{prender; apreender; levar sob custódia}
\end{EntryWithPhonetic}

%%%%%%%%%% 戴 %%%%%%%%%%
\subsection*{戴}\addcontentsline{loh}{figure}{戴 \dpy{dai4}}

\begin{EntryWithPhonetic}{戴}{dai4}{17}{⼽}[HSK 4]
  \definition*{s.}{Sobrenome: Dai}
  \definition{v.}{usar/vestir (óculos, gravata, relógio de pulso, luvas); colocar objetos em sua cabeça, rosto, pescoço, peito, braços etc. | honrar; respeitar;}
\end{EntryWithPhonetic}

%%%%%%%%%% 单 %%%%%%%%%%
\subsection*{单}\addcontentsline{loh}{figure}{单 \dpy{dan1}}

\begin{EntryWithPhonetic}{单}{dan1}{8}{⼗}[HSK 4]
  \definition*{s.}{Sobrenome: Dan}
  \definition{adj.}{sozinho; único | ímpar; número ímpar | simples; poucos projetos e tipos; estrutura e ideias simples | fino; fraco; frágil}
  \definition{adv.}{isoladamente; sozinho; indica que uma ação ou coisa está dentro de um escopo limitado e não é combinada com outras; equivale a 只 ou 仅}
  \definition[个]{s.}{lençol; um único pedaço grande de pano usado para cobrir | conta; lista; pedaços de papel para anotações detalhadas (geralmente folhas soltas)}
  \seeref{chan2}
  \seeref{shan4}
  \seealsoref{仅}{jin3}
  \seealsoref{只}{zhi3}
  \antonymref{双}{shuang1}
\end{EntryWithPhonetic}

\begin{EntryWithPhonetic}{单边}{dan1bian1}{8,5}{⼗,⾡}[HSK 7-9]
  \definition{adj.}{unilateral}
\end{EntryWithPhonetic}

\begin{EntryWithPhonetic}{单薄}{dan1bo2}{8,16}{⼗,⾋}[HSK 7-9]
  \definition{adj.}{fino; pouco | frágil; magro e fraco | fino; frágil; insubstancial}
\end{EntryWithPhonetic}

\begin{EntryWithPhonetic}{单纯}{dan1chun2}{8,7}{⼗,⽷}[HSK 4]
  \definition{adj.}{puro; simples; descomplicado}
  \definition{adv.}{sozinho; puramente; meramente}
\end{EntryWithPhonetic}

\begin{EntryWithPhonetic}{单打}{dan1da3}{8,5}{⼗,⼿}[HSK 6]
  \definition[场,局,次]{s.}{Esporte: simples; competição um contra um}
\end{EntryWithPhonetic}

\begin{EntryWithPhonetic}{单单}{dan1dan1}{8,8}{⼗,⼗}
  \definition{adv.}{somente; sozinho; exceto; indica a identificação de um indivíduo dentro de um grupo geral de pessoas ou coisas}[别人都去了,单单她没去。===Todos os outros foram, somente ela não foi.]
\end{EntryWithPhonetic}

\begin{EntryWithPhonetic}{单调}{dan1diao4}{8,10}{⼗,⾔}[HSK 4]
  \definition{adj.}{maçante; monótono}
\end{EntryWithPhonetic}

\begin{EntryWithPhonetic}{单独}{dan1du2}{8,9}{⼗,⽝}[HSK 4]
  \definition{adv.}{solo; sozinho; por si mesmo; por conta própria}
\end{EntryWithPhonetic}

\begin{EntryWithPhonetic}{单方面}{dan1fang1mian4}{8,4,9}{⼗,⽅,⾯}[HSK 7-9]
  \definition{adj.}{unilateral}
  \definition{adv.}{unilateralmente}
\end{EntryWithPhonetic}

\begin{EntryWithPhonetic}{单脚滑行车}{dan1jiao3hua2xing2che1}{8,11,12,6,4}{⼗,⾁,⽔,⾏,⾞}
  \definition{s.}{\emph{scooter}}
\end{EntryWithPhonetic}

\begin{EntryWithPhonetic}{单身}{dan1shen1}{8,7}{⼗,⾝}[HSK 7-9]
  \definition{s.}{solteiro}
\end{EntryWithPhonetic}

\begin{EntryWithPhonetic}{单位}{dan1wei4}{8,7}{⼗,⼈}[HSK 2]
  \definition[个,家]{s.}{unidade (como padrão de medida) | unidade (como uma organização, departamento, divisão, seção, etc.) | unidade (grupo de pessoas como um todo) | unidade de trabalho (local de trabalho, especialmente na República Popular da China antes da reforma econômica)}
\end{EntryWithPhonetic}

\begin{EntryWithPhonetic}{单向}{dan1xiang4}{8,6}{⼗,⼝}
  \definition{adj.}{1. de mão única; unidirecional (oposto de 双向)}
  \seealsoref{双向}{shuang1xiang4}
\end{EntryWithPhonetic}

\begin{EntryWithPhonetic}{单一}{dan1yi1}{8,1}{⼗,⼀}[HSK 5]
  \definition{adj.}{único; unitário; exclusivo}
\end{EntryWithPhonetic}

\begin{EntryWithPhonetic}{单元}{dan1yuan2}{8,4}{⼗,⼉}[HSK 3]
  \definition[个,组,套]{s.}{unidade (de algo); um conjunto completo, com parágrafos e sistemas próprios, que forma uma unidade independente}
\end{EntryWithPhonetic}

\begin{EntryWithPhonetic}{单质}{dan1zhi4}{8,8}{⼗,⾙}
  \definition{s.}{substância simples (consistindo puramente de um elemento, como diamante, grafite, etc.)}
\end{EntryWithPhonetic}

%%%%%%%%%% 担 %%%%%%%%%%
\subsection*{担}\addcontentsline{loh}{figure}{担 \dpy{dan1}}

\begin{EntryWithPhonetic}{担}{dan1}{8}{⼿}[HSK 7-9]
  \definition{v.}{carregar em uma vara de ombro e baldes; carregar nos ombros | assumir; empreender; não ter medo de correr riscos}
  \seeref{dan4}
\end{EntryWithPhonetic}

\begin{EntryWithPhonetic}{担保}{dan1bao3}{8,9}{⼿,⼈}[HSK 4]
  \definition{v.}{garantir; atestar; expressar responsabilidade e garantir que não haverá problemas ou que eles serão resolvidos}
\end{EntryWithPhonetic}

\begin{EntryWithPhonetic}{担当}{dan1dang1}{8,6}{⼿,⼹}[HSK 7-9]
  \definition{v.}{aceitar e assumir responsabilidade; empreender (responsabilidade, trabalho, despesas)}
\end{EntryWithPhonetic}

\begin{EntryWithPhonetic}{担负}{dan1fu4}{8,6}{⼿,⾙}[HSK 7-9]
  \definition{v.}{suportar; carregar; assumir; ser encarregado de}
\end{EntryWithPhonetic}

\begin{EntryWithPhonetic}{担任}{dan1ren4}{8,6}{⼿,⼈}[HSK 4]
  \definition{v.}{servir como; assumir o cargo de; ocupar o posto de; ocupar um determinado cargo ou emprego}
\end{EntryWithPhonetic}

\begin{EntryWithPhonetic}{担心}{dan1/xin1}{8,4}{⼿,⼼}[HSK 4]
  \definition{v.+compl.}{preocupar-se; ficar ansioso; sentir-se desconfortável com algo}
\end{EntryWithPhonetic}

\begin{EntryWithPhonetic}{担忧}{dan1you1}{8,7}{⼿,⼼}[HSK 6]
  \definition[项,条,套,种]{v.}{preocupar-se; estar ansioso}
\end{EntryWithPhonetic}

%%%%%%%%%% 耽 %%%%%%%%%%
\subsection*{耽}\addcontentsline{loh}{figure}{耽 \dpy{dan1}}

\begin{EntryWithPhonetic}{耽}{dan1}{10}{⽿}
  \definition*{s.}{Sobrenome: Dan}
  \definition{v.}{atrasar | Literário: abandonar"-se a; entregar"-se a}
\end{EntryWithPhonetic}

\begin{EntryWithPhonetic}{耽搁}{dan1ge5}{10,12}{⽿,⼿}[HSK 7-9]
  \definition{v.}{ficar; fazer uma parada | atrasar | perder (uma oportunidade, um prazo)}
\end{EntryWithPhonetic}

\begin{EntryWithPhonetic}{耽误}{dan1wu5}{10,9}{⽿,⾔}[HSK 7-9]
  \definition{v.}{atrasar; segurar; perder algo devido a atraso ou oportunidade perdida; perder (oportunidade)}
\end{EntryWithPhonetic}

\begin{EntryWithPhonetic}{耽心}{dan1/xin1}{10,4}{⽿,⼼}
  \variantof{担心}
\end{EntryWithPhonetic}

%%%%%%%%%% 躭 %%%%%%%%%%
\subsection*{躭}\addcontentsline{loh}{figure}{躭 \dpy{dan1}}

\begin{EntryWithPhonetic}{躭}{dan1}{11}{⾝}
  \definition{v.}{entregar-se a; adiar}
\end{EntryWithPhonetic}

%%%%%%%%%% 胆 %%%%%%%%%%
\subsection*{胆}\addcontentsline{loh}{figure}{胆 \dpy{dan3}}

\begin{EntryWithPhonetic}{胆}{dan3}{9}{⾁}[HSK 5]
  \definition[个,颗]{s.}{vesícula biliar | coragem; bravura | um recipiente interno semelhante a uma bexiga; algo que se encaixa dentro de um objeto e pode conter água, ar, etc.}
\end{EntryWithPhonetic}

\begin{EntryWithPhonetic}{胆怯}{dan3qie4}{9,8}{⾁,⼼}[HSK 7-9]
  \definition{v.+compl.}{tímido; covarde; descreve a aparência ou sentimento de alguém que tem muito medo de fazer algo}
\end{EntryWithPhonetic}

\begin{EntryWithPhonetic}{胆小}{dan3xiao3}{9,3}{⾁,⼩}[HSK 5]
  \definition{adj.}{tímido; covarde}
\end{EntryWithPhonetic}

\begin{EntryWithPhonetic}{胆小鬼}{dan3xiao3gui3}{9,3,9}{⾁,⼩,⿁}
  \definition{adj.}{covarde | medroso}
\end{EntryWithPhonetic}

\begin{EntryWithPhonetic}{胆子}{dan3zi5}{9,3}{⾁,⼦}[HSK 7-9]
  \definition{s.}{coragem}
\end{EntryWithPhonetic}

%%%%%%%%%% 石 %%%%%%%%%%
\subsection*{石}\addcontentsline{loh}{figure}{石 \dpy{dan4}}

\begin{EntryWithPhonetic}{石}{dan4}{5}{⽯}[Kangxi 112]
  \definition{clas.}{dan, uma unidade de medida seca para grãos; unidade de capacidade, 10 斗 é igual a 1 石}
  \seeref{shi2}
  \seealsoref{斗}{dou4}
\end{EntryWithPhonetic}

%%%%%%%%%% 但 %%%%%%%%%%
\subsection*{但}\addcontentsline{loh}{figure}{但 \dpy{dan4}}

\begin{EntryWithPhonetic}{但}{dan4}{7}{⼈}[HSK 2]
  \definition*{s.}{Sobrenome: Dan}
  \definition{adv.}{apenas; meramente; indica uma restrição ao âmbito da ação, equivalente a 只 ou 仅}
  \definition{conj.}{mas; ainda assim; mesmo assim; no entanto; contudo; usado na última oração, conecta duas orações, expressando uma relação de transição, equivalente a 可是 ou 不过}
  \seealsoref{不过}{bu2guo4}
  \seealsoref{仅}{jin3}
  \seealsoref{可是}{ke3shi4}
  \seealsoref{只}{zhi3}
\end{EntryWithPhonetic}

\begin{EntryWithPhonetic}{但是}{dan4shi4}{7,9}{⼈,⽇}[HSK 2]
  \definition{conj.}{mas; contudo; no entanto; mesmo assim; usado na segunda parte da frase para indicar uma mudança, geralmente acompanhada de expressões como 虽然 ou 尽管}
  \seealsoref{尽管}{jin3guan3}
  \seealsoref{虽然}{sui1ran2}
\end{EntryWithPhonetic}

\begin{EntryWithPhonetic}{但愿}{dan4yuan4}{7,14}{⼈,⽕}[HSK 7-9]
  \definition{v.}{se ao menos (algo fosse possível); desejaria (que); espero (que)}
\end{EntryWithPhonetic}

%%%%%%%%%% 担 %%%%%%%%%%
\subsection*{担}\addcontentsline{loh}{figure}{担 \dpy{dan4}}

\begin{EntryWithPhonetic}{担}{dan4}{8}{⼿}[HSK 7-9]
  \definition{clas.}{dan, uma unidade de peso (=50 quilogramas) ; 100 jin = 1 dan | usado em coisas usadas para transportar cargas}
  \definition{s.}{carga; fardo; cargas de mercadorias transportadas em uma vara de ombro por um mascate itinerante}
  \seeref{dan1}
\end{EntryWithPhonetic}

\begin{EntryWithPhonetic}{担子}{dan4zi5}{8,3}{⼿,⼦}[HSK 7-9]
  \definition[副,个]{s.}{vara de transporte (ou de ombro) e as cargas sob ela; canga; carga; fardo | tarefa}
\end{EntryWithPhonetic}

%%%%%%%%%% 诞 %%%%%%%%%%
\subsection*{诞}\addcontentsline{loh}{figure}{诞 \dpy{dan4}}

\begin{EntryWithPhonetic}{诞}{dan4}{8}{⾔}
  \definition{adj.}{absurdo; fantástico; irreal; irracional}
  \definition{adv.}{absurdamente; fantasticamente}
  \definition{s.}{aniversário de nascimento | nascimento}
  \definition{v.}{nascer | dar à luz}
\end{EntryWithPhonetic}

\begin{EntryWithPhonetic}{诞辰}{dan4chen2}{8,7}{⾔,⾠}[HSK 7-9]
  \definition[周年]{s.}{aniversário (usado principalmente para pessoas respeitadas)}[9月28日是孔子诞辰日。===28 de setembro é o aniversário de Confúcio.]
\end{EntryWithPhonetic}

\begin{EntryWithPhonetic}{诞生}{dan4sheng1}{8,5}{⾔,⽣}[HSK 6]
  \definition{v.}{nascer; vir a existir; uma pessoa nasce; também significa que algo novo surgiu e tem um impacto positivo na sociedade}
\end{EntryWithPhonetic}

%%%%%%%%%% 弹 %%%%%%%%%%
\subsection*{弹}\addcontentsline{loh}{figure}{弹 \dpy{dan4}}

\begin{EntryWithPhonetic}{弹}{dan4}{11}{⼸}
  \definition{s.}{bola; pelota; pequenas bolas disparadas com um estilingue | bomba; bala; explosivos que podem ser lançados ou arremessados, com poder destrutivo e letal}
  \seeref{tan2}
\end{EntryWithPhonetic}

%%%%%%%%%% 淡 %%%%%%%%%%
\subsection*{淡}\addcontentsline{loh}{figure}{淡 \dpy{dan4}}

\begin{EntryWithPhonetic}{淡}{dan4}{11}{⽔}[HSK 4]
  \definition*{s.}{Sobrenome: Dan}
  \definition{adj.}{sem gosto; fraco; não tem sabor forte; não é salgado | leve; fraco; pálido | indiferente; frio; sem entusiasmo | frouxo; sem brilho | sem sentido; trivial | fino; leve}
\end{EntryWithPhonetic}

\begin{EntryWithPhonetic}{淡化}{dan4hua4}{11,4}{⽔,⼔}[HSK 7-9]
  \definition{v.}{dessalinizar; transformar água com alto teor de sal em água com baixo teor de sal | desaparecer; enfraquecer; tornar ou tornar-se menos importante}
\end{EntryWithPhonetic}

\begin{EntryWithPhonetic}{淡季}{dan4ji4}{11,8}{⽔,⼦}[HSK 7-9]
  \definition{s.}{baixa temporada; temporada fraca (ou monótona, fora de temporada) | uma estação em que a produção de um determinado produto é baixa; uma estação em que os negócios estão lentos (diferente da 旺季)}
  \seealsoref{旺季}{wang4ji4}
\end{EntryWithPhonetic}

%%%%%%%%%% 蛋 %%%%%%%%%%
\subsection*{蛋}\addcontentsline{loh}{figure}{蛋 \dpy{dan4}}

\begin{EntryWithPhonetic}{蛋}{dan4}{11}{⾍}[HSK 2]
  \definition[个,只]{s.}{ovo; ovos produzidos por aves, tartarugas, cobras, etc. | algo em forma de ovo | tolo; idiota; metáfora para pessoas com determinadas características (com conotação pejorativa) | se perder; colocado após certos verbos, forma um verbo com conotação pejorativa | testículos; em algumas regiões, refere"-se aos testículos de certos animais ou pessoas}
\end{EntryWithPhonetic}

\begin{EntryWithPhonetic}{蛋白质}{dan4bai2zhi4}{11,5,8}{⾍,⽩,⾙}[HSK 7-9]
  \definition{s.}{proteína}
\end{EntryWithPhonetic}

\begin{EntryWithPhonetic}{蛋糕}{dan4gao1}{11,16}{⾍,⽶}[HSK 5]
  \definition[个,块,盒]{s.}{bolo; bolo fofo feito de ovos e farinha com açúcar e óleo}
\end{EntryWithPhonetic}

%%%%%%%%%% 当 %%%%%%%%%%
\subsection*{当}\addcontentsline{loh}{figure}{当 \dpy{dang1}}

\begin{EntryWithPhonetic}{当}{dang1}{6}{⼹}[HSK 2]
  \definition*{s.}{Sobrenome: Dang}
  \definition{adj.}{igual; adequado; compatível}
  \definition{prep.}{na presença de alguém; na cara de alguém | exatamente em (um momento ou lugar); em algum momento, em algum lugar | na frente de alguém}
  \definition{s.}{Onomatopéia: barulho metálico, som de um gongo ou sino}
  \definition{s.}{topo; cume | uma lacuna no espaço ou no tempo; refere"-se a um espaço ou intervalo de tempo}
  \definition{v.}{dever; ter que; dever ser | trabalhar como; servir como; ser; assumir; desempenhar a função de | suportar; aceitar; merecer | dirigir; gerenciar; estar no comando; ser responsável por; presidir | conter; bloquear; segurar; reter; resistir}
  \seeref{dang4}
\end{EntryWithPhonetic}

\begin{EntryWithPhonetic}{当场}{dang1chang3}{6,6}{⼹,⼟}[HSK 5]
  \definition{adv.}{na hora; de imediato; na mesma hora}
\end{EntryWithPhonetic}

\begin{EntryWithPhonetic}{当初}{dang1chu1}{6,7}{⼹,⾐}[HSK 3]
  \definition{s.}{no começo; originalmente; no início; em primeiro lugar; refere"-se a algo que aconteceu no passado, seja em geral ou especificamente}
\end{EntryWithPhonetic}

\begin{EntryWithPhonetic}{当代}{dang1dai4}{6,5}{⼹,⼈}[HSK 5]
  \definition{s.}{a era atual; a era contemporânea}
\end{EntryWithPhonetic}

\begin{EntryWithPhonetic}{当地}{dang1di4}{6,6}{⼹,⼟}[HSK 3]
  \definition{s.}{local; o lugar onde as pessoas e as coisas estão ou onde as coisas acontecem}
\end{EntryWithPhonetic}

\begin{EntryWithPhonetic}{当即}{dang1ji2}{6,7}{⼹,⼙}[HSK 7-9]
  \definition{adv.}{imediatamente}
\end{EntryWithPhonetic}

\begin{EntryWithPhonetic}{当今}{dang1jin1}{6,4}{⼹,⼈}[HSK 7-9]
  \definition{s.}{o presente; hoje | Arcaico: imperador no trono; imperador reinante | agora; no presente; hoje em dia}
\end{EntryWithPhonetic}

\begin{EntryWithPhonetic}{当面}{dang1mian4}{6,9}{⼹,⾯}[HSK 7-9]
  \definition{adv.}{na cara de alguém; na presença de alguém; cara a cara}
\end{EntryWithPhonetic}

\begin{EntryWithPhonetic}{当年}{dang1nian2}{6,6}{⼹,⼲}[HSK 5]
  \definition{s.}{aqueles anos (ou dias) | naqueles anos (ou dias) | durante esse tempo}
  \definition{v.}{estar no auge da vida}
  \seeref{dang4nian2}
\end{EntryWithPhonetic}

\begin{EntryWithPhonetic}{当前}{dang1qian2}{6,9}{⼹,⼑}[HSK 5]
  \definition{s.}{presente; atual}
  \definition{v.}{estar diante de alguém; estar frente a frente com alguém; na frente de, geralmente refere"-se a uma situação perigosa}
\end{EntryWithPhonetic}

\begin{EntryWithPhonetic}{当然}{dang1ran2}{6,12}{⼹,⽕}[HSK 3]
  \definition{adj.}{natural; verdadeiro; espontâneo}
  \definition{adv.}{sem dúvida; certamente; claro}
\end{EntryWithPhonetic}

\begin{EntryWithPhonetic}{当日}{dang1ri4}{6,4}{⼹,⽇}[HSK 7-9]
  \definition[点]{s.}{nessa ocasião; naquela época; no mesmo dia; naquele mesmo dia}
  \seeref{dang4ri4}
\end{EntryWithPhonetic}

\begin{EntryWithPhonetic}{当时}{dang1shi2}{6,7}{⼹,⽇}[HSK 2]
  \definition{s.}{naquela época; aquela ocasião; aquela vez; refere"-se a algo que aconteceu no passado}
  \definition{v.}{ser o momento adequado; acontecer no momento certo}
  \seeref{dang4shi2}
\end{EntryWithPhonetic}

\begin{EntryWithPhonetic}{当事人}{dang1shi4ren2}{6,8,2}{⼹,⼅,⼈}[HSK 7-9]
  \definition{s.}{litigante; parte (em um processo judicial); refere"-se especificamente a pessoas que têm uma relação direta com os fatos do caso, como a vítima, o promotor particular, o réu, etc. em um processo criminal | partes interessadas; pessoa (ou parte) envolvida; alguém que tem uma relação direta com algo}
\end{EntryWithPhonetic}

\begin{EntryWithPhonetic}{当天}{dang1tian1}{6,4}{⼹,⼤}[HSK 6]
  \definition{s.}{no mesmo dia; naquele mesmo dia; refere"-se ao dia em que algo aconteceu no passado}
\end{EntryWithPhonetic}

\begin{EntryWithPhonetic}{当晚}{dang1wan3}{6,11}{⼹,⽇}
  \definition{s.}{naquela noite; esta noite; a mesma noite}
  \seeref{dang4wan3}
\end{EntryWithPhonetic}

\begin{EntryWithPhonetic}{当务之急}{dang1wu4zhi1ji2}{6,5,3,9}{⼹,⼒,⼂,⼼}[HSK 7-9]
  \definition{expr.}{assunto mais urgente do momento; uma tarefa de alta prioridade; assunto urgente | assunto de vital importância; preocupações | trabalho de alta prioridade}
\end{EntryWithPhonetic}

\begin{EntryWithPhonetic}{当下}{dang1xia4}{6,3}{⼹,⼀}[HSK 7-9]
  \definition{adv.}{instantaneamente; imediatamente; de uma vez}
  \definition{s.}{o tempo presente}
\end{EntryWithPhonetic}

\begin{EntryWithPhonetic}{当心}{dang1xin1}{6,4}{⼹,⼼}[HSK 7-9]
  \definition{s.}{centro; o centro do peito}
  \definition{v.}{ter cuidado com; ter cuidado}
\end{EntryWithPhonetic}

\begin{EntryWithPhonetic}{当选}{dang1xuan3}{6,9}{⼹,⾡}[HSK 5]
  \definition{v.}{ser eleito}
\end{EntryWithPhonetic}

\begin{EntryWithPhonetic}{当着}{dang1zhe5}{6,11}{⼹,⽬}[HSK 7-9]
  \definition{prep.}{na frente de | na presença de}
\end{EntryWithPhonetic}

\begin{EntryWithPhonetic}{当之无愧}{dang1zhi1wu2kui4}{6,3,4,12}{⼹,⼂,⽆,⼼}[HSK 7-9]
  \definition{expr.}{merecer plenamente (um título, uma honra, etc.); merecer a recompensa; ser merecedor | ser digno de; ser digno do nome}
\end{EntryWithPhonetic}

\begin{EntryWithPhonetic}{当中}{dang1zhong1}{6,4}{⼹,⼁}[HSK 3]
  \definition{prep.}{no meio; no centro | entre; dentro}
\end{EntryWithPhonetic}

\begin{EntryWithPhonetic}{当众}{dang1zhong4}{6,6}{⼹,⼈}[HSK 7-9]
  \definition{adv.}{abertamente; publicamente; em público; diante do público; na presença de todos; na frente de todos; de frente para a multidão}
\end{EntryWithPhonetic}

%%%%%%%%%% 挡 %%%%%%%%%%
\subsection*{挡}\addcontentsline{loh}{figure}{挡 \dpy{dang3}}

\begin{EntryWithPhonetic}{挡}{dang3}{9}{⼿}[HSK 5]
  \definition{s.}{persiana; veneziana; paralama; coisas para cobrir ou bloquear | caixa de câmbio (automóvel)}
  \definition{v.}{bloquear; resistir; manter afastado; afastar | cobrir; bloquear; atrapalhar}
  \seeref{dang4}
\end{EntryWithPhonetic}

\begin{EntryWithPhonetic}{挡风玻璃}{dang3feng1bo1li5}{9,4,9,14}{⼿,⾵,⽟,⽟}
  \definition{s.}{parabrisa}
\end{EntryWithPhonetic}

%%%%%%%%%% 党 %%%%%%%%%%
\subsection*{党}\addcontentsline{loh}{figure}{党 \dpy{dang3}}

\begin{EntryWithPhonetic}{党}{dang3}{10}{⼉}[HSK 6]
  \definition*{s.}{O Partido (Partido Comunista da China) | Sobrenome: Dang}
  \definition{s.}{partido político; partido | camarilha; facção; gangue | Obsoleto: parentes}
  \definition{v.}{ser parcial; tomar partido de}
\end{EntryWithPhonetic}

%%%%%%%%%% 当 %%%%%%%%%%
\subsection*{当}\addcontentsline{loh}{figure}{当 \dpy{dang4}}

\begin{EntryWithPhonetic}{当}{dang4}{6}{⼹}[HSK 6]
  \definition{adj.}{adequado; correto; apropriado | igual; o mesmo}
  \definition{pron.}{naquele mesmo (dia, etc.); refere"-se ao momento em que algo aconteceu}
  \definition{s.}{algo penhorado; penhor; garantia; objetos físicos penhorados em casas de penhores}
  \definition{v.}{corresponder; ser igual a; combinar | tratar como; considerar como; tomar como | pensar que; achar que | penhorar; empréstimo com garantia real em uma loja de penhores}
  \seeref{dang1}
\end{EntryWithPhonetic}

\begin{EntryWithPhonetic}{当成}{dang4cheng2}{6,6}{⼹,⼽}[HSK 6]
  \definition{v.}{considerar como; tratar como; tomar por}
\end{EntryWithPhonetic}

\begin{EntryWithPhonetic}{当年}{dang4nian2}{6,6}{⼹,⼲}
  \definition{s.}{no mesmo ano; naquele mesmo ano}
  \seeref{dang1nian2}
\end{EntryWithPhonetic}

\begin{EntryWithPhonetic}{当日}{dang4ri4}{6,4}{⼹,⽇}
  \definition[点]{s.}{mesmo dia; naquele mesmo dia; refere"-se ao mesmo dia em que algo aconteceu; (neste) dia}
  \seeref{dang1ri4}
\end{EntryWithPhonetic}

\begin{EntryWithPhonetic}{当时}{dang4shi2}{6,7}{⼹,⽇}
  \definition{adv.}{(depois de fazer algo ou algo acontecer) imediatamente; de imediato; agora mesmo}
  \seeref{dang1shi2}
\end{EntryWithPhonetic}

\begin{EntryWithPhonetic}{当晚}{dang4wan3}{6,11}{⼹,⽇}[HSK 7-9]
  \definition{s.}{na mesma noite; esta noite}
  \seeref{dang1wan3}
\end{EntryWithPhonetic}

\begin{EntryWithPhonetic}{当真}{dang4zhen1}{6,10}{⼹,⼗}[HSK 7-9]
  \definition{adj.}{verdadeiro; real; confiável}
  \definition{adv.}{realmente; verdadeiramente}
  \definition{v.}{levar a sério; acreditar}
\end{EntryWithPhonetic}

\begin{EntryWithPhonetic}{当作}{dang4zuo4}{6,7}{⼹,⼈}[HSK 6]
  \definition{v.}{tratar como; considerar como}
\end{EntryWithPhonetic}

%%%%%%%%%% 挡 %%%%%%%%%%
\subsection*{挡}\addcontentsline{loh}{figure}{挡 \dpy{dang4}}

\begin{EntryWithPhonetic}{挡}{dang4}{9}{⼿}
  \definition{v.}{organizar}
  \seeref{dang3}
\end{EntryWithPhonetic}

%%%%%%%%%% 荡 %%%%%%%%%%
\subsection*{荡}\addcontentsline{loh}{figure}{荡 \dpy{dang4}}

\begin{EntryWithPhonetic}{荡}{dang4}{9}{⾋}
  \definition*{s.}{Sobrenome: Dang}
  \definition{adj.}{indiferente às restrições morais; devasso, libertino, depravado, desregrado | vasto, amplo e nivelado}
  \definition{s.}{lago raso; pântano}
  \definition{v.}{oscilar; gingar; ondular | vadiar; vagabundear | enxaguar | limpar; varrer | vagar; vaguear; andar por aí; passear por aí}
\end{EntryWithPhonetic}

\begin{EntryWithPhonetic}{荡漾}{dang4yang4}{9,14}{⾋,⽔}[HSK 7-9]
  \definition{v.}{ondular; agitar; ser agitado}
\end{EntryWithPhonetic}

%%%%%%%%%% 档 %%%%%%%%%%
\subsection*{档}\addcontentsline{loh}{figure}{档 \dpy{dang4}}

\begin{EntryWithPhonetic}{档}{dang4}{10}{⽊}[HSK 6]
  \definition{clas.}{festa; usado para eventos, shows}
  \definition{s.}{prateleiras (para arquivos); compartimentos para documentos | arquivos; arquivos | travessa (de uma mesa, etc.) | qualidade; nota}
\end{EntryWithPhonetic}

\begin{EntryWithPhonetic}{档案}{dang4'an4}{10,10}{⽊,⽊}[HSK 6]
  \definition[份,个]{s.}{arquivos; registro; dossiê; arquivos e materiais armazenados de forma classificada para referência futura}
\end{EntryWithPhonetic}

\begin{EntryWithPhonetic}{档次}{dang4ci4}{10,6}{⽊,⽋}[HSK 7-9]
  \definition{s.}{classe; grau; qualidade; nível; diferentes níveis divididos de acordo com certos padrões}
\end{EntryWithPhonetic}

%%%%%%%%%% 刀 %%%%%%%%%%
\subsection*{刀}\addcontentsline{loh}{figure}{刀 \dpy{dao1}}

\begin{EntryWithPhonetic}{刀}{dao1}{2}{⼑}[HSK 3][Kangxi 18]
  \definition*{s.}{Sobrenome: Dao}
  \definition{clas.}{unidade de medida para papel, geralmente cem folhas por pacote}
  \definition[把,口]{s.}{faca; espada; armas antigas, referindo"-se a ferramentas para cortar, retalhar, raspar, golpear e fatiar, geralmente feitas de ferro e aço | ferramenta; ferramenta de corte; lâminas para tornos; fresas (ferramentas; ferramentas de ferro para máquinas) | algo com a forma de uma faca}
\end{EntryWithPhonetic}

%%%%%%%%%% 导 %%%%%%%%%%
\subsection*{导}\addcontentsline{loh}{figure}{导 \dpy{dao3}}

\begin{EntryWithPhonetic}{导}{dao3}{6}{⼨}
  \definition[个,位,名,些]{s.}{guia turístico | diretor}
  \definition{v.}{liderar; guiar | conduzir; transmitir | ensinar; instruir; dar orientação a}
\end{EntryWithPhonetic}

\begin{EntryWithPhonetic}{导弹}{dao3dan4}{6,11}{⼨,⼸}[HSK 7-9]
  \definition[枚,颗,个]{s.}{míssil (guiado)}
  \seealsoref{飞弹}{fei1dan4}
\end{EntryWithPhonetic}

\begin{EntryWithPhonetic}{导航}{dao3hang2}{6,10}{⼨,⾈}[HSK 7-9]
  \definition{s.}{navegação; tecnologia que guia aviões, navios ou carros por rotas seguras}
  \definition{v.}{navegar; guiar um avião, navio ou carro por uma rota segura}
\end{EntryWithPhonetic}

\begin{EntryWithPhonetic}{导火索}{dao3huo3suo3}{6,4,10}{⼨,⽕,⽷}[HSK 7-9]
  \definition{s.}{fusível (para explosivo)}
\end{EntryWithPhonetic}

\begin{EntryWithPhonetic}{导师}{dao3shi1}{6,6}{⼨,⼱}[HSK 7-9]
  \definition[位,个]{s.}{tutor; professor; orientador; supervisor; uma pessoa que orienta outras pessoas em seus estudos, educação continuada ou redação de trabalhos em faculdades, universidades ou instituições de pesquisa | mentor; guia de uma grande causa; uma pessoa que fornece orientação em grandes empreendimentos e movimentos}
\end{EntryWithPhonetic}

\begin{EntryWithPhonetic}{导向}{dao3xiang4}{6,6}{⼨,⼝}[HSK 7-9]
  \definition{s.}{orientação; direção}
  \definition{v.}{guiar; orientar; dirigir}
\end{EntryWithPhonetic}

\begin{EntryWithPhonetic}{导演}{dao3yan3}{6,14}{⼨,⽔}[HSK 3]
  \definition[位,名,个]{s.}{diretor; pessoa que exerce a função de diretor}
  \definition{v.}{dirigir (um filme, peça, etc.); ensaio de peças teatrais ou filmagem de filmes e séries de TV; organização e orientação do trabalho de produção}
\end{EntryWithPhonetic}

\begin{EntryWithPhonetic}{导游}{dao3you2}{6,12}{⼨,⽔}[HSK 4]
  \definition[个,位,名]{s.}{guia turístico; pessoas que trabalham como guias turísticos}
  \definition{v.}{guiar; conduzir um passeio turístico}
\end{EntryWithPhonetic}

\begin{EntryWithPhonetic}{导致}{dao3zhi4}{6,10}{⼨,⾄}[HSK 4]
  \definition{v.}{causar; levar a; dar origem a (um resultado ruim)}
\end{EntryWithPhonetic}

%%%%%%%%%% 岛 %%%%%%%%%%
\subsection*{岛}\addcontentsline{loh}{figure}{岛 \dpy{dao3}}

\begin{EntryWithPhonetic}{岛}{dao3}{7}{⼭}[HSK 6]
  \definition[个,座]{s.}{ilha; uma massa de terra menor que um continente cercada por água}
\end{EntryWithPhonetic}

\begin{EntryWithPhonetic}{岛屿}{dao3yu3}{7,6}{⼭,⼭}[HSK 7-9]
  \definition[座,些,群]{s.}{ilha; ilhota}
\end{EntryWithPhonetic}

%%%%%%%%%% 倒 %%%%%%%%%%
\subsection*{倒}\addcontentsline{loh}{figure}{倒 \dpy{dao3}}

\begin{EntryWithPhonetic}{倒}{dao3}{10}{⼈}
  \definition{v.}{cair; tombar | falhar; entrar em colapso | ficar rouco | mudar; trocar; transferir; converter | movimentar-se; manobrar | oferecer (casa, loja) para venda; vender mercadorias ou lojas a terceiros a um preço fixo | derrubar; derrubar com}
  \seeref{dao4}
\end{EntryWithPhonetic}

\begin{EntryWithPhonetic}{倒闭}{dao3bi4}{10,6}{⼈,⾨}[HSK 4]
  \definition{v.}{fechar; ir à falência; entrar em liquidação; sair do negócio; (empresa, loja ou banco) deixar de operar devido ao baixo desempenho}
\end{EntryWithPhonetic}

\begin{EntryWithPhonetic}{倒车}{dao3/che1}{10,4}{⼈,⾞}[HSK 4]
  \definition{v.+compl.}{trocar de trem ou ônibus (no meio do caminho)}
  \seeref{dao4/che1}
\end{EntryWithPhonetic}

\begin{EntryWithPhonetic}{倒地}{dao3di4}{10,6}{⼈,⼟}
  \definition{v.}{cair no chão}
\end{EntryWithPhonetic}

\begin{EntryWithPhonetic}{倒卖}{dao3mai4}{10,8}{⼈,⼗}[HSK 7-9]
  \definition{v.}{revender com lucro}
\end{EntryWithPhonetic}

\begin{EntryWithPhonetic}{倒楣}{dao3mei2}{10,13}{⼈,⽊}
  \definition{adj.}{azarado; infeliz; tendo má sorte}
\end{EntryWithPhonetic}

\begin{EntryWithPhonetic}{倒霉}{dao3/mei2}{10,15}{⼈,⾬}[HSK 7-9]
  \definition{adj.}{azarado}
  \definition{s.}{azar; má sorte}
  \definition{v.+compl.}{cair em dias maus; cair em tempos difíceis; encontrar coisas desfavoráveis; ter má sorte}
  \seealsoref{倒血霉}{dao3xue4mei2}
\end{EntryWithPhonetic}

\begin{EntryWithPhonetic}{倒塌}{dao3ta1}{10,13}{⼈,⼟}[HSK 7-9]
  \definition{v.}{colapsar; desabar}
\end{EntryWithPhonetic}

\begin{EntryWithPhonetic}{倒下}{dao3xia4}{10,3}{⼈,⼀}[HSK 7-9]
  \definition{v.}{entrar em colapso | tombar}
\end{EntryWithPhonetic}

\begin{EntryWithPhonetic}{倒血霉}{dao3xue4mei2}{10,6,15}{⼈,⾎,⾬}
  \definition{v.}{ter muito azar (versão mais forte de 倒霉)}
  \seealsoref{倒霉}{dao3/mei2}
\end{EntryWithPhonetic}

%%%%%%%%%% 捣 %%%%%%%%%%
\subsection*{捣}\addcontentsline{loh}{figure}{捣 \dpy{dao3}}

\begin{EntryWithPhonetic}{捣}{dao3}{10}{⼿}
  \definition{v.}{bater com um pilão, etc.; bater; esmagar | assediar; perturbar | bater com um pedaço de pau}
\end{EntryWithPhonetic}

\begin{EntryWithPhonetic}{捣乱}{dao3/luan4}{10,7}{⼿,⼄}[HSK 7-9]
  \definition{v.+compl.}{causar problemas; criar uma perturbação; causar intencionalmente problemas para os outros; interromper | perturbar; interferir com; causar problemas intencionalmente}
\end{EntryWithPhonetic}

%%%%%%%%%% 到 %%%%%%%%%%
\subsection*{到}\addcontentsline{loh}{figure}{到 \dpy{dao4}}

\begin{EntryWithPhonetic}{到}{dao4}{8}{⼑}[HSK 1]
  \definition*{s.}{Sobrenome: Dao}
  \definition{adj.}{atencioso}
  \definition{prep.}{a; até; para; indica o tempo em que a ação ou comportamento foi alcançado}
  \definition{v.}{ir para; partir para | chegar; alcançar; chegar a | como complemento de um verbo para mostrar o resultado de uma ação}
\end{EntryWithPhonetic}

\begin{EntryWithPhonetic}{到处}{dao4chu4}{8,5}{⼑,⼡}[HSK 2]
  \definition{adv.}{em todos os lugares; em todos os locais; por toda parte}
\end{EntryWithPhonetic}

\begin{EntryWithPhonetic}{到达}{dao4da2}{8,6}{⼑,⾡}[HSK 3]
  \definition{v.}{chegar (a um determinado local, a uma determinada fase); alcançar}
\end{EntryWithPhonetic}

\begin{EntryWithPhonetic}{到底}{dao4di3}{8,8}{⼑,⼴}[HSK 3]
  \definition{adv.}{na terra (usado em frases interrogativas para expressar a determinação de alguém em encontrar uma resposta definitiva) | afinal | finalmente; por fim; no fim; indica uma situação que finalmente se concretizou após várias mudanças ou reviravoltas}
\end{EntryWithPhonetic}

\begin{EntryWithPhonetic}{到来}{dao4lai2}{8,7}{⼑,⽊}[HSK 5]
  \definition{v.}{chegar; chegar aqui de outro lugar}
\end{EntryWithPhonetic}

\begin{EntryWithPhonetic}{到期}{dao4/qi1}{8,12}{⼑,⽉}[HSK 6]
  \definition{v.+compl.}{expirar; amadurecer; tornar-se devido; tornar-se devido}
\end{EntryWithPhonetic}

\begin{EntryWithPhonetic}{到头来}{dao4tou2lai2}{8,5,7}{⼑,⼤,⽊}[HSK 7-9]
  \definition{adv.}{finalmente; no fim; no final; resultado (usado principalmente em aspectos ruins)}
\end{EntryWithPhonetic}

\begin{EntryWithPhonetic}{到位}{dao4/wei4}{8,7}{⼑,⼈}[HSK 7-9]
  \definition{adj.}{bom; muito preciso; até um nível adequado/padrão/satisfatório}
  \definition{v.+compl.}{estar no lugar/posição; chegar ao local designado; alcançar o local especificado; atender aos requisitos especificados}
\end{EntryWithPhonetic}

%%%%%%%%%% 倒 %%%%%%%%%%
\subsection*{倒}\addcontentsline{loh}{figure}{倒 \dpy{dao4}}

\begin{EntryWithPhonetic}{倒}{dao4}{10}{⼈}[HSK 2]
  \definition{adj.}{inverso; invertido; de cabeça para baixo}
  \definition{adv.}{mas; pelo contrário; expressa o contrário do esperado, equivalente a 反倒 | indicando que algo não é o que se pensa; indica que as coisas não são assim | usado para indicar uma transição ou concessão | transmitindo a sensação de ``urgência''; expressa pressa ou insistência, com um tom impaciente}
  \definition{v.}{ser inverso; estar invertido; estar de cabeça para baixo; inverter a posição original para cima e para baixo ou para a frente e para trás | recuar; virar de cabeça para baixo; fazer mover na direção oposta ou inverter | inclinar ou virar o recipiente para retirar o conteúdo; inclinar; derramar}
  \seeref{dao3}
  \seealsoref{反倒}{fan3dao4}
\end{EntryWithPhonetic}

\begin{EntryWithPhonetic}{倒车}{dao4/che1}{10,4}{⼈,⾞}[HSK 4]
  \definition{v.+compl.}{dar marcha à ré (em um veículo)}
  \seeref{dao3/che1}
\end{EntryWithPhonetic}

\begin{EntryWithPhonetic}{倒计时}{dao4ji4shi2}{10,4,7}{⼈,⾔,⽇}[HSK 7-9]
  \definition{s.}{contagem regressiva; contagem de tempo a partir de um determinado ponto no futuro até o presente, de mais para menos, até que o tempo chegue a zero; é frequentemente usado para expressar que um certo momento está se aproximando}
\end{EntryWithPhonetic}

\begin{EntryWithPhonetic}{倒是}{dao4shi4}{10,9}{⼈,⽇}[HSK 5]
  \definition{adv.}{usado para indicar o oposto do que geralmente é verdade; ao contrário do senso comum; pelo contrário | usado para indicar o que é contrário aos fatos, com um toque de crítica; indica que as coisas não são assim (com um sentimento de culpa) | usado de algo inesperado; expressando surpresa | usado para indicar concessão | usado para indicar uma mudança de significado; indica um ponto de virada | usado para modificar ou suavizar uma declaração anterior; para suavizar o tom | usado para pressionar ou questionar alguém; para instar ou perguntar}
\end{EntryWithPhonetic}

\begin{EntryWithPhonetic}{倒数}{dao4shu3}{10,13}{⼈,⽁}[HSK 7-9]
  \definition{v.}{contar de trás para frente (contagem regressiva)}
  \seeref{dao4shu4}
\end{EntryWithPhonetic}

\begin{EntryWithPhonetic}{倒数}{dao4shu4}{10,13}{⼈,⽁}
  \definition{s.}{número inverso; Matemática: recíproco}
  \seeref{dao4shu3}
\end{EntryWithPhonetic}

%%%%%%%%%% 悼 %%%%%%%%%%
\subsection*{悼}\addcontentsline{loh}{figure}{悼 \dpy{dao4}}

\begin{EntryWithPhonetic}{悼}{dao4}{11}{⼼}
  \definition*{s.}{Sobrenome: Dao}
  \definition{v.}{lamentar; expressar pesar}
\end{EntryWithPhonetic}

\begin{EntryWithPhonetic}{悼念}{dao4nian4}{11,8}{⼼,⼼}[HSK 7-9]
  \definition{v.}{lamentar; lamentar-se por | lamentar por; expressar pesar}
\end{EntryWithPhonetic}

%%%%%%%%%% 盗 %%%%%%%%%%
\subsection*{盗}\addcontentsline{loh}{figure}{盗 \dpy{dao4}}

\begin{EntryWithPhonetic}{盗}{dao4}{11}{⽫}[HSK 7-9]
  \definition[个,伙,帮,窝]{s.}{ladrão; assaltante}
  \definition{v.}{roubar; saquear | usurpar; buscar ganho pessoal ou ganho por meios impróprios}
\end{EntryWithPhonetic}

\begin{EntryWithPhonetic}{盗版}{dao4ban3}{11,8}{⽫,⽚}[HSK 6]
  \definition{s.}{cópia ilegal; cópia pirata; refere"-se a livros, periódicos e produtos audiovisuais pirateados (diferentes dos 正版)}
  \definition{v.}{piratear; copiar ou vender ilegalmente; para obter lucros enormes, reimprimir ou copiar livros, periódicos ou produtos audiovisuais em grandes quantidades sem o consentimento do detentor dos direitos autorais}
  \seealsoref{正版}{zheng4ban3}
\end{EntryWithPhonetic}

\begin{EntryWithPhonetic}{盗窃}{dao4qie4}{11,9}{⽫,⽳}[HSK 7-9]
  \definition{v.}{roubar; furtar; obter ilegalmente por meios secretos}
\end{EntryWithPhonetic}

%%%%%%%%%% 道 %%%%%%%%%%
\subsection*{道}\addcontentsline{loh}{figure}{道 \dpy{dao4}}

\begin{EntryWithPhonetic}{道}{dao4}{12}{⾡}[HSK 2]
  \definition*{s.}{Taoismo;  Taoista | Sobrenome: Dao}
  \definition{clas.}{usado para pratos em refeições, etapas em um procedimento, etc. | usado para certos objetos longos e estreitos; tira | usado para portas, paredes, etc.; pesado | usado para comandos, títulos, etc.}
  \definition[条]{s.}{estrada; caminho; trilha | curso; canal; o caminho percorrido pelo fluxo da água | maneira; método; princípio; raciocínio | moral; moralidade | habilidade; técnica | doutrina; princípio; sistema de pensamento acadêmico ou religioso; origem de todas as coisas no universo | taoísta; taoísmo; pertencente ao taoísmo | seita supersticiosa; certas organizações reacionárias e supersticiosas | linha; traços finos e alongados | trato; os canais dentro do corpo}
  \definition{v.}{dizer; falar; expressar-se | pensar; supor; considerar; acreditar que}
\end{EntryWithPhonetic}

\begin{EntryWithPhonetic}{道德}{dao4de2}{12,15}{⾡,⼻}[HSK 5]
  \definition{adj.}{moral; descreve uma pessoa ou comportamento que atende aos requisitos morais; mais usado em situações negativas}
  \definition[种]{s.}{moral; ética; moralidade; regras e normas para que as pessoas vivam juntas e se comportem em comum}
\end{EntryWithPhonetic}

\begin{EntryWithPhonetic}{道行}{dao4 heng2}{12,6}{⾡,⾏}
  \definition{s.}{realizações de um monge budista ou sacerdote taoísta | habilidades; capacidades; aptidões | (figurativo) habilidade | habilidades adquiridas através da prática religiosa}
\end{EntryWithPhonetic}

\begin{EntryWithPhonetic}{道教}{dao4jiao4}{12,11}{⾡,⽁}[HSK 6]
  \definition*{s.}{Taoísmo (sistema de crenças chinês)}
  \definition{s.}{a religião taoísta; taoísmo}
\end{EntryWithPhonetic}

\begin{EntryWithPhonetic}{道具}{dao4ju4}{12,8}{⾡,⼋}[HSK 7-9]
  \definition{s.}{adereços; objetos de cena; artigos de palco; objetos usados em apresentações, como mesas e cadeiras, são chamados de grandes adereços, enquanto cigarros e xícaras de chá são chamados de pequenos adereços}
\end{EntryWithPhonetic}

\begin{EntryWithPhonetic}{道理}{dao4li5}{12,11}{⾡,⽟}[HSK 2]
  \definition[个,种]{s.}{verdade; princípio; a lei das coisas | sentido; razão}
\end{EntryWithPhonetic}

\begin{EntryWithPhonetic}{道路}{dao4lu4}{12,13}{⾡,⾜}[HSK 2]
  \definition[条,段]{s.}{estrada; caminho; os canais de comunicação entre os dois lugares, incluindo terrestres e aquáticos | caminho; processo; refere"-se à vida, à existência (significado abstrato)}
\end{EntryWithPhonetic}

\begin{EntryWithPhonetic}{道歉}{dao4/qian4}{12,14}{⾡,⽋}[HSK 6]
  \definition{v.+compl.}{pedir desculpas; fazer um pedido de desculpas; dizer aos outros que você estava errado e pedir perdão}
\end{EntryWithPhonetic}

%%%%%%%%%% 稻 %%%%%%%%%%
\subsection*{稻}\addcontentsline{loh}{figure}{稻 \dpy{dao4}}

\begin{EntryWithPhonetic}{稻}{dao4}{15}{⽲}
  \definition{s.}{arroz; arroz com casca}
\end{EntryWithPhonetic}

\begin{EntryWithPhonetic}{稻草}{dao4cao3}{15,9}{⽲,⾋}[HSK 7-9]
  \definition[捆,根,抱,束]{s.}{palha de arroz (pode ser usada para fazer cordas ou esteiras de palha, para fazer papel, ou para ser usada como ração, combustível, etc.)}
\end{EntryWithPhonetic}

%%%%%%%%%% 得 %%%%%%%%%%
\subsection*{得}\addcontentsline{loh}{figure}{得 \dpy{de2}}

\begin{EntryWithPhonetic}{得}{de2}{11}{⼻}[HSK 2]
  \definition{adj.}{adequado; apropriado | satisfeito; complacente; orgulhoso de si mesmo}
  \definition{interj.}{usado para encerrar uma conversa para indicar concordância ou proibição | usado quando a situação não é a esperada, para expressar impotência}
  \definition{v.}{obter; conseguir; ganhar |  (de um cálculo) igual; resultar em | estar pronto; estar acabado | pegar; apanhar; contrair uma doença}
  \definition{v.aux.}{usado antes de outros verbos para expressar permissão | usado antes de outros verbos para indicar que é possível (usado principalmente na forma negativa) | usado em conversas para indicar que não há necessidade de dizer mais nada}
  \seeref{de5}
  \seeref{dei3}
  \antonymref{失}{shi1}
\end{EntryWithPhonetic}

\begin{EntryWithPhonetic}{得不偿失}{de2bu4chang2shi1}{11,4,11,5}{⼻,⼀,⼈,⼤}[HSK 7-9]
  \definition{expr.}{``A perda supera o ganho.''; ``Os ganhos não compensam as perdas.''; perder mais do que ganhar; ``O jogo não vale a pena.''; ``O que é ganho não compensa o que é perdido.''}
\end{EntryWithPhonetic}

\begin{EntryWithPhonetic}{得出}{de2 chu1}{11,5}{⼻,⼐}[HSK 2]
  \definition{v.}{chegar (a uma conclusão); obter (a um resultado); deduzir ou calcular (conclusão ou resultado)}
\end{EntryWithPhonetic}

\begin{EntryWithPhonetic}{得当}{de2dang4}{11,6}{⼻,⼹}[HSK 7-9]
  \definition{adj.}{apropriado; próprio; adequado | apto}
\end{EntryWithPhonetic}

\begin{EntryWithPhonetic}{得到}{de2 dao4}{11,8}{⼻,⼑}[HSK 1]
  \definition{v.}{obter; conseguir; ganhar; receber; possuir algo; adquirir}
\end{EntryWithPhonetic}

\begin{EntryWithPhonetic}{得分}{de2 fen1}{11,4}{⼻,⼑}[HSK 3]
  \definition{s.}{pontuação; classificação; nota; pontuação obtida em jogos ou competições}
  \definition{v.}{fazer pontos; pontuar}
\end{EntryWithPhonetic}

\begin{EntryWithPhonetic}{得了}{de2le5}{11,2}{⼻,⼅}[HSK 5]
  \definition{expr.}{Tudo bem!; É o bastante!}
  \seeref{de2liao3}
\end{EntryWithPhonetic}

\begin{EntryWithPhonetic}{得力}{de2li4}{11,2}{⼻,⼒}[HSK 7-9]
  \definition{adj.}{capaz; competente; capaz de fazer coisas | eficiente; poderoso}
  \definition{v.}{beneficiar-se de; obter ajuda de; beneficiar}
\end{EntryWithPhonetic}

\begin{EntryWithPhonetic}{得了}{de2liao3}{11,2}{⼻,⼅}
  \definition{adj.}{(enfaticamente, em perguntas retóricas) possível; indica que a situação é séria (usado principalmente em perguntas retóricas ou formas negativas)}
  \seeref{de2le5}
\end{EntryWithPhonetic}

\begin{EntryWithPhonetic}{得失}{de2shi1}{11,5}{⼻,⼤}[HSK 7-9]
  \definition{s.}{ganho e perda; sucesso e fracasso | méritos e deméritos; vantagens e desvantagens; prós e contras}
\end{EntryWithPhonetic}

\begin{EntryWithPhonetic}{得手}{de2shou3}{11,4}{⼻,⼿}[HSK 7-9]
  \definition{adj.}{Coloquial: prático; conveniente e fácil de usar}
  \definition{v.}{fazer algo suavemente; ter sucesso; atingir seu objetivo | ir suavemente; sair; fazer bem; fazer as coisas sem problemas}
\end{EntryWithPhonetic}

\begin{EntryWithPhonetic}{得体}{de2ti3}{11,7}{⼻,⼈}[HSK 7-9]
  \definition{adj.}{(fala, comportamento, etc.) apropriado; moderado}
\end{EntryWithPhonetic}

\begin{EntryWithPhonetic}{得天独厚}{de2tian1du2hou4}{11,4,9,9}{⼻,⼤,⽝,⼚}[HSK 7-9]
  \definition{expr.}{ser ricamente dotado pela natureza; abundar em dádivas da natureza; desfrutar de vantagens excepcionais | abençoado pelo céu | desfrutar de vantagens excepcionais | favorecido pela natureza}
\end{EntryWithPhonetic}

\begin{EntryWithPhonetic}{得以}{de2yi3}{11,4}{⼻,⼈}[HSK 5]
  \definition{v.}{ser capaz de; para que\dots possa (ou possa)\dots}
\end{EntryWithPhonetic}

\begin{EntryWithPhonetic}{得益于}{de2yi4 yu2}{11,10,3}{⼻,⽫,⼆}[HSK 7-9]
  \definition{s.}{correlação positiva; benefício}
\end{EntryWithPhonetic}

\begin{EntryWithPhonetic}{得意}{de2yi4}{11,13}{⼻,⼼}[HSK 4]
  \definition{adj.}{complacente; orgulhoso de si mesmo; satisfeito consigo mesmo}
\end{EntryWithPhonetic}

\begin{EntryWithPhonetic}{得意扬扬}{de2yi4-yang2yang2}{11,13,6,6}{⼻,⼼,⼿,⼿}[HSK 7-9]
  \definition{expr.}{orgulhoso e complacente | estar imensamente orgulhoso; parecer triunfante}
\end{EntryWithPhonetic}

\begin{EntryWithPhonetic}{得知}{de2zhi1}{11,8}{⼻,⽮}[HSK 7-9]
  \definition{v.}{saber; ser informado de; aprender}
\end{EntryWithPhonetic}

\begin{EntryWithPhonetic}{得罪}{de2zui4}{11,13}{⼻,⽹}[HSK 7-9]
  \definition{v.}{ofender; desagradar; causar desprazer ou ressentimento}
\end{EntryWithPhonetic}

%%%%%%%%%% 德 %%%%%%%%%%
\subsection*{德}\addcontentsline{loh}{figure}{德 \dpy{de2}}

\begin{EntryWithPhonetic}{德}{de2}{15}{⼻}[HSK 7-9]
  \definition*{s.}{Alemanha, abreviação de 德国 | Sobrenome: De}
  \definition{s.}{virtude; moral; caráter moral; moralidade; conduta; qualidades políticas | coração; mente; pensamentos | bondade; favor; graça}
  \seealsoref{德国}{de2guo2}
\end{EntryWithPhonetic}

\begin{EntryWithPhonetic}{德国}{de2guo2}{15,8}{⼻,⼞}
  \definition*{s.}{Alemanha}
\end{EntryWithPhonetic}

\begin{EntryWithPhonetic}{德国人}{de2guo2ren2}{15,8,2}{⼻,⼞,⼈}
  \definition{s.}{alemão | pessoa ou povo da Alemanha}
\end{EntryWithPhonetic}

%%%%%%%%%% 地 %%%%%%%%%%
\subsection*{地}\addcontentsline{loh}{figure}{地 \dpy{de5}}

\begin{EntryWithPhonetic}{地}{de5}{6}{⼟}[HSK 1]
  \definition{part.}{(estrutural) utilizada antes de um verbo ou adjetivo, ligando-o ao adjunto adverbial modificador precedente}
  \seeref{di4}
\end{EntryWithPhonetic}

%%%%%%%%%% 底 %%%%%%%%%%
\subsection*{底}\addcontentsline{loh}{figure}{底 \dpy{de5}}

\begin{EntryWithPhonetic}{底}{de5}{8}{⼴}
  \definition{part.}{usada após uma palavra ou frase que é usada como determinante para indicar subordinação à palavra central}
  \seeref{di3}
\end{EntryWithPhonetic}

%%%%%%%%%% 的 %%%%%%%%%%
\subsection*{的}\addcontentsline{loh}{figure}{的 \dpy{de5}}

\begin{EntryWithPhonetic}{的}{de5}{8}{⽩}[HSK 1]
  \definition{part.}{usado para indicar posse | formar uma frase nominal ou expressão nominal | substituir a pessoa ou coisa mencionada anteriormente | no final de uma frase declarativa, para dar ênfase; usado após o verbo predicativo, enfatiza o agente da ação, o tempo, o local, etc. | usado no final de uma frase declarativa, expressa afirmação, ênfase, certeza, etc. | indica que alguém obteve uma determinada posição ou status | usado com 是 para indicar predicado ou ênfase; indica que alguém é o objeto da ação | e assim por diante; e assim por diante; e similares; usado após palavras paralelas, significa 等等, 之类 | indica uma ação (o pronome é o objeto da ação); combinado com o verbo anterior, expressa uma ação, e o pronome é o objeto dessa ação}
  \seeref{di1}
  \seeref{di2}
  \seeref{di4}
  \seealsoref{等等}{deng3 deng3}
  \seealsoref{是}{shi4}
  \seealsoref{之类}{zhi1lei4}
\end{EntryWithPhonetic}

\begin{EntryWithPhonetic}{的话}{de5hua4}{8,8}{⽩,⾔}[HSK 2]
  \definition{part.}{se; caso; suponha que; partícula usada após uma frase hipotética para introduzir o texto seguinte}
\end{EntryWithPhonetic}

\begin{EntryWithPhonetic}{的时候}{de5 shi2hou4}{8,7,10}{⽩,⽇,⼈}
  \definition{part.}{naquele momento; quando; em; descreve o momento específico em que um evento ocorreu}
\end{EntryWithPhonetic}

%%%%%%%%%% 得 %%%%%%%%%%
\subsection*{得}\addcontentsline{loh}{figure}{得 \dpy{de5}}

\begin{EntryWithPhonetic}{得}{de5}{11}{⼻}[HSK 2]
  \definition{part.}{depois de um verbo ou adjetivo para expressar possibilidade ou capacidade | entre um verbo e seu complemento para expressar possibilidade | ligando um verbo ou um adjetivo a um complemento que descreve a maneira ou o grau}
  \seeref{de2}
  \seeref{dei3}
\end{EntryWithPhonetic}

\begin{EntryWithPhonetic}{得}{dei3}{11}{⼻}[HSK 4]
  \definition{v.}{precisar; expressa uma necessidade lógica, factual ou subjetiva; deve; é necessário | ter de; ser obrigado a; indica uma necessidade de vontade ou de fato | certamente irá; expressa a inevitabilidade da especulação}
  \seeref{de2}
  \seeref{de5}
\end{EntryWithPhonetic}

%%%%%%%%%% 灯 %%%%%%%%%%
\subsection*{灯}\addcontentsline{loh}{figure}{灯 \dpy{deng1}}

\begin{EntryWithPhonetic}{灯}{deng1}{6}{⽕}[HSK 2]
  \definition*{s.}{Sobrenome: Deng}
  \definition[盏,个]{s.}{lâmpada; luz; lanterna; dispositivo luminoso, usado principalmente para iluminação | queimador; um aparelho que brilha e aquece como uma lâmpada e pode ser usado para aquecer | tubo; válvula; o nome popular dado aos tubos eletrônicos com formato semelhante a lâmpadas encontrados em aparelhos antigos, como rádios}
\end{EntryWithPhonetic}

\begin{EntryWithPhonetic}{灯标}{deng1biao1}{6,9}{⽕,⽊}
  \definition{s.}{farol | luz de farol}
\end{EntryWithPhonetic}

\begin{EntryWithPhonetic}{灯光}{deng1guang1}{6,6}{⽕,⼉}[HSK 4]
  \definition[束,盏,点,打]{s.}{iluminação; luminosidade da lâmpada | luminação (palco); equipamento de iluminação para palco ou estúdio}
\end{EntryWithPhonetic}

\begin{EntryWithPhonetic}{灯号}{deng1hao4}{6,5}{⽕,⼝}
  \definition{s.}{sinal luminoso | luz indicadora}
\end{EntryWithPhonetic}

\begin{EntryWithPhonetic}{灯笼}{deng1long5}{6,11}{⽕,⽵}[HSK 7-9]
  \definition[个,盏,只]{s.}{lanterna; luminárias suspensas ou portáteis, geralmente feitas de finas tiras de bambu ou arame de ferro como estrutura, cobertas com areia ou papel, com velas dentro; atualmente, lâmpadas elétricas são usadas principalmente como fontes de luz e como decoração}
\end{EntryWithPhonetic}

\begin{EntryWithPhonetic}{灯泡}{deng1pao4}{6,8}{⽕,⽔}[HSK 7-9]
  \definition[只,个]{s.}{lâmpada (bulbo) | (gíria) terceiro indesejado estragando encontro de casal; é frequentemente usado para descrever a si mesmo ou a outros se sentindo estranhos ou indesejados em situações sociais}
  \seealsoref{电灯泡}{dian4deng1pao4}
\end{EntryWithPhonetic}

\begin{EntryWithPhonetic}{灯丝}{deng1si1}{6,5}{⽕,⼀}
  \definition{s.}{filamento (de uma lâmpada)}
\end{EntryWithPhonetic}

%%%%%%%%%% 登 %%%%%%%%%%
\subsection*{登}\addcontentsline{loh}{figure}{登 \dpy{deng1}}

\begin{EntryWithPhonetic}{登}{deng1}{12}{⽨}[HSK 4]
  \definition{v.}{subir; montar; escalar (uma altura) | publicar; registrar; inserir | recolher e levar para a eira | pisar em; pisar | calçar (calçados ou calças) | partir; começar uma jornada; embarcar em uma jornada}
\end{EntryWithPhonetic}

\begin{EntryWithPhonetic}{登机}{deng1ji1}{12,6}{⽨,⽊}[HSK 7-9]
  \definition{v.}{embarcar; embarcar em um avião}
\end{EntryWithPhonetic}

\begin{EntryWithPhonetic}{登记}{deng1/ji4}{12,5}{⽨,⾔}[HSK 4]
  \definition{v.+compl.}{registrar-se; fazer o \emph{check-in} | registrar; reportar; informar; relatar por escrito a um superior ou autoridade relevante (usado principalmente para documentos legais)}
\end{EntryWithPhonetic}

\begin{EntryWithPhonetic}{登陆}{deng1/lu4}{12,7}{⽨,⾩}[HSK 7-9]
  \definition{v.+compl.}{desembarcar; chegar à costa | entrar em um mercado; Metáfora: mercadorias entram em um determinado mercado e começam a ser vendidas; comerciantes entram em um determinado mercado e começam a fazer negócios}
\end{EntryWithPhonetic}

\begin{EntryWithPhonetic}{登录}{deng1lu4}{12,8}{⽨,⼹}[HSK 4]
  \definition{v.}{fazer \emph{logon}; fazer \emph{login} | gravar; registrar; computadores eletrônicos e sua terminologia de rede, referindo"-se ao acesso ao sistema operacional ou ao site a ser visitado}
\end{EntryWithPhonetic}

\begin{EntryWithPhonetic}{登山}{deng1 shan1}{12,3}{⽨,⼭}[HSK 4]
  \definition{s.}{escalar; fazer alpinismo; subir uma montanha}
\end{EntryWithPhonetic}

%%%%%%%%%% 蹬 %%%%%%%%%%
\subsection*{蹬}\addcontentsline{loh}{figure}{蹬 \dpy{deng1}}

\begin{EntryWithPhonetic}{蹬}{deng1}{19}{⾜}[HSK 7-9]
  \definition{v.}{pressionar com o pé; pisar; pisar em | Dialeto: calçar (sapatos ou calças); usar (sapatos) | Gíria: despejar (algo)}
  \seeref{deng4}
\end{EntryWithPhonetic}

%%%%%%%%%% 等 %%%%%%%%%%
\subsection*{等}\addcontentsline{loh}{figure}{等 \dpy{deng3}}

\begin{EntryWithPhonetic}{等}{deng3}{12}{⽵}[HSK 1,2]
  \definition*{s.}{Sobrenome: Deng}
  \definition{adj.}{igual; na mesma medida ou quantidade}
  \definition{clas.}{usado para classe, grau, classificação | usado para tipo}
  \definition{part.}{e assim por diante; etc.; indica que a enumeração não está completa (pode ser usada repetidamente) | indica o fim de uma enumeração; após a enumeração, é usado para encerrar; geralmente é seguido pelo total dos itens anteriores}
  \definition{pron.}{usado após pronomes pessoais ou substantivos que se referem a pessoas; indica plural}
  \definition{s.}{classe; série; posição | equilíbrio; balança para pesar pequenas quantidades de objetos valiosos e ervas medicinais; atualmente, geralmente escrita como 戥}
  \definition{v.}{esperar; aguardar | esperar até}
\end{EntryWithPhonetic}

\begin{EntryWithPhonetic}{等待}{deng3dai4}{12,9}{⽵,⼻}[HSK 3]
  \definition{v.}{esperar; aguardar; não agir até que a pessoa, coisa ou situação desejada apareça}
\end{EntryWithPhonetic}

\begin{EntryWithPhonetic}{等到}{deng3dao4}{12,8}{⽵,⼑}[HSK 2]
  \definition{prep.}{na hora; quando; expressão de condições temporais | esperar até; aguardar até}
\end{EntryWithPhonetic}

\begin{EntryWithPhonetic}{等等}{deng3 deng3}{12,12}{⽵,⽵}
  \definition{part.}{etc.; e assim por diante; usada depois de duas ou mais palavras paralelas para indicar que a lista não está completa}
\end{EntryWithPhonetic}

\begin{EntryWithPhonetic}{等候}{deng3hou4}{12,10}{⽵,⼈}[HSK 5]
  \definition{v.}{esperar; aguardar; expectar; usado principalmente para objetos específicos}
\end{EntryWithPhonetic}

\begin{EntryWithPhonetic}{等级}{deng3ji2}{12,6}{⽵,⽷}[HSK 5]
  \definition[个]{s.}{grau; classificação; posição; distinções por qualidade, grau, status, etc. | estado social; estrato social; ordem e grau; grupos sociais desiguais em termos de status social e legal}
\end{EntryWithPhonetic}

\begin{EntryWithPhonetic}{等于}{deng3yu2}{12,3}{⽵,⼆}[HSK 2]
  \definition{adv.}{igual a | equivalente a}
  \definition{v.}{equivaler a; ser equivalente a; ser quase igual a; não ter diferença}
\end{EntryWithPhonetic}

%%%%%%%%%% 凳 %%%%%%%%%%
\subsection*{凳}\addcontentsline{loh}{figure}{凳 \dpy{deng4}}

\begin{EntryWithPhonetic}{凳}{deng4}{14}{⼏}
  \definition[条]{s.}{banco; banqueta}
\end{EntryWithPhonetic}

\begin{EntryWithPhonetic}{凳子}{deng4zi5}{14,3}{⼏,⼦}[HSK 7-9]
  \definition[把,条,个]{s.}{banco; banqueta; um móvel que tem pernas para sentar, mas não tem encosto}
\end{EntryWithPhonetic}

%%%%%%%%%% 澄 %%%%%%%%%%
\subsection*{澄}\addcontentsline{loh}{figure}{澄 \dpy{deng4}}

\begin{EntryWithPhonetic}{澄}{deng4}{15}{⽔}
  \definition{adj.}{(água, ar, etc.) claro; transparente; límpido}
  \definition{v.}{esclarecer; aclarar | sedimentar; fazer com que impurezas em um líquido afundem}
  \seeref{cheng2}
\end{EntryWithPhonetic}

%%%%%%%%%% 瞪 %%%%%%%%%%
\subsection*{瞪}\addcontentsline{loh}{figure}{瞪 \dpy{deng4}}

\begin{EntryWithPhonetic}{瞪}{deng4}{17}{⽬}[HSK 7-9]
  \definition{v.}{abrir bem os olhos; encarar; olhar fixamente com os olhos bem abertos; expressar insatisfação}
\end{EntryWithPhonetic}

%%%%%%%%%% 蹬 %%%%%%%%%%
\subsection*{蹬}\addcontentsline{loh}{figure}{蹬 \dpy{deng4}}

\begin{EntryWithPhonetic}{蹬}{deng4}{19}{⾜}
  \definition{s.}{lutar; ter dificuldade}
  \seeref{deng1}
  \seealsoref{蹭蹬}{ceng4deng4}
\end{EntryWithPhonetic}

%%%%%%%%%% 低 %%%%%%%%%%
\subsection*{低}\addcontentsline{loh}{figure}{低 \dpy{di1}}

\begin{EntryWithPhonetic}{低}{di1}{7}{⼈}[HSK 2]
  \definition*{s.}{Sobrenome: Di}
  \definition{adj.}{baixo; distância pequena de baixo para cima; próximo ao solo | abaixo da média; abaixo do padrão geral | inferior (em grau); de nível inferior}
  \definition{v.}{deixar cair; pendurar; abaixar (a cabeça)}
\end{EntryWithPhonetic}

\begin{EntryWithPhonetic}{低潮}{di1chao2}{7,15}{⼈,⽔}
  \definition{s.}{maré baixa/vazante; o nível mais baixo da maré durante um ciclo de maré (distinto da 高潮) | vazante baixa; o ponto mais baixo; uma metáfora para o baixo estágio de desenvolvimento das coisas}
  \seealsoref{高潮}{gao1chao2}
\end{EntryWithPhonetic}

\begin{EntryWithPhonetic}{低等}{di1 deng3}{7,12}{⼈,⽵}
  \definition{adj.}{inferior; classe baixa | inferior}
  \antonymref{高等}{gao1deng3}
\end{EntryWithPhonetic}

\begin{EntryWithPhonetic}{低调}{di1diao4}{7,10}{⼈,⾔}[HSK 7-9]
  \definition{adj.}{moderado; discreto; metaforicamente, abordagem discreta ou gentil}
  \definition{s.}{um perfil/tom/chave baixo; uma metáfora para pensamentos ou observações pessimistas e negativas}
\end{EntryWithPhonetic}

\begin{EntryWithPhonetic}{低估}{di1gu1}{7,7}{⼈,⼈}[HSK 7-9]
  \definition{v.}{subestimar; menosprezar}
\end{EntryWithPhonetic}

\begin{EntryWithPhonetic}{低谷}{di1gu3}{7,7}{⼈,⾕}[HSK 7-9]
  \definition{s.}{vale | maré baixa; uma metáfora para um estado de impróprio; o ponto mais baixo}[雾气笼罩着整个低谷。===A neblina cobria todo o vale.]
\end{EntryWithPhonetic}

\begin{EntryWithPhonetic}{低价}{di1jia4}{7,6}{⼈,⼈}[HSK 7-9]
  \definition{s.}{preço baixo}
\end{EntryWithPhonetic}

\begin{EntryWithPhonetic}{低空}{di1kong1}{7,8}{⼈,⽳}
  \definition{s.}{baixa altitude; baixo nível}
  \antonymref{高空}{gao1kong1}
\end{EntryWithPhonetic}

\begin{EntryWithPhonetic}{低落}{di1luo4}{7,12}{⼈,⾋}
  \definition{adj.}{desanimado; deprimido; abatido; muitas vezes se refere a mau humor}
  \definition{v.}{cair; declinar; diminuir; (preços, sentimento, etc.) cair rapidamente}
  \antonymref{高涨}{gao1zhang3}
\end{EntryWithPhonetic}

\begin{EntryWithPhonetic}{低迷}{di1mi2}{7,9}{⼈,⾡}[HSK 7-9]
  \definition{adj.}{baixo; estagnado; deprimente; abatido | escuro; nebuloso; borrado}
\end{EntryWithPhonetic}

\begin{EntryWithPhonetic}{低碳}{di1tan4}{7,14}{⼈,⽯}[HSK 7-9]
  \definition{adj.}{baixo carbono; nos últimos anos, termos como 低碳 e 碳足迹 tornaram-se termos populares tanto no país quanto no exterior; o ``carbono'' aqui se refere principalmente ao gás dióxido de carbono}
  \seealsoref{碳足迹}{tan4 zu2ji4}
\end{EntryWithPhonetic}

\begin{EntryWithPhonetic}{低头}{di1 tou2}{7,5}{⼈,⼤}[HSK 6]
  \definition{v.}{abaixar a cabeça; curvar a cabeça; pendurar a cabeça | ceder; submeter"-se; refere"-se à rendição e à admissão da derrota}
\end{EntryWithPhonetic}

\begin{EntryWithPhonetic}{低温}{di1wen1}{7,12}{⼈,⽔}[HSK 6]
  \definition{s.}{baixa temperatura | Meteorologia: microtermia  | Medicina: hipotermia}
\end{EntryWithPhonetic}

\begin{EntryWithPhonetic}{低下}{di1xia4}{7,3}{⼈,⼀}[HSK 7-9]
  \definition{adj.}{(\emph{status} ou padrões de vida) baixo; modesto; humilde | (gosto, etc.) vulgar; baixo}
\end{EntryWithPhonetic}

\begin{EntryWithPhonetic}{低压}{di1ya1}{7,6}{⼈,⼚}
  \definition{s.}{Física: baixa pressão | Eletricidade: baixa tensão; baixa voltagem | Meteorologia: baixa pressão; depressão | Medicina: pressão diastólica, pressão mínima}
\end{EntryWithPhonetic}

\begin{EntryWithPhonetic}{低于}{di1yu2}{7,3}{⼈,⼆}[HSK 5]
  \definition{v.}{ser inferior a; algo ou fenômeno é, de alguma forma, inferior ou pior do que outra coisa}
\end{EntryWithPhonetic}

%%%%%%%%%% 的 %%%%%%%%%%
\subsection*{的}\addcontentsline{loh}{figure}{的 \dpy{di1}}

\begin{EntryWithPhonetic}{的}{di1}{8}{⽩}
  \definition{s.}{abreviação de 的士: um táxi}
  \seeref{de5}
  \seeref{di2}
  \seeref{di4}
  \seealsoref{的士}{di1shi4}
\end{EntryWithPhonetic}

\begin{EntryWithPhonetic}{的士}{di1shi4}{8,3}{⽩,⼠}
  \definition{s.}{(empréstimo linguístico) táxi}
\end{EntryWithPhonetic}

%%%%%%%%%% 堤 %%%%%%%%%%
\subsection*{堤}\addcontentsline{loh}{figure}{堤 \dpy{di1}}

\begin{EntryWithPhonetic}{堤}{di1}{12}{⼟}[HSK 7-9]
  \definition[道,条]{s.}{dique; aterro}
\end{EntryWithPhonetic}

\begin{EntryWithPhonetic}{堤坝}{di1ba4}{12,7}{⼟,⼟}[HSK 7-9]
  \definition[座,道,个]{s.}{diques; barragem; represa}
\end{EntryWithPhonetic}

%%%%%%%%%% 提 %%%%%%%%%%
\subsection*{提}\addcontentsline{loh}{figure}{提 \dpy{di1}}

\begin{EntryWithPhonetic}{提防}{di1fang5}{12,6}{⼿,⾩}[HSK 7-9]
  \definition{v.}{proteger-se contra; ter cuidado com; tomar precauções contra; tomar cuidado}
\end{EntryWithPhonetic}

%%%%%%%%%% 滴 %%%%%%%%%%
\subsection*{滴}\addcontentsline{loh}{figure}{滴 \dpy{di1}}

\begin{EntryWithPhonetic}{滴}{di1}{14}{⽔}[HSK 6]
  \definition{clas.}{gota; quantificador para ``gotejamento''}
  \definition{s.}{uma gota}
  \definition{v.}{pingar}
\end{EntryWithPhonetic}

%%%%%%%%%% 的 %%%%%%%%%%
\subsection*{的}\addcontentsline{loh}{figure}{的 \dpy{di2}}

\begin{EntryWithPhonetic}{的}{di2}{8}{⽩}
  \definition{adv.}{verdadeiramente; exatamente; realmente}
  \seeref{de5}
  \seeref{di1}
  \seeref{di4}
\end{EntryWithPhonetic}

\begin{EntryWithPhonetic}{的确}{di2que4}{8,12}{⽩,⽯}[HSK 4]
  \definition{adv.}{realmente; de fato, ao expressar certeza sobre a situação}
\end{EntryWithPhonetic}

%%%%%%%%%% 敌 %%%%%%%%%%
\subsection*{敌}\addcontentsline{loh}{figure}{敌 \dpy{di2}}

\begin{EntryWithPhonetic}{敌}{di2}{10}{⾆}
  \definition[个,名,位,种]{s.}{inimigo; adversário}
  \definition{v.}{opor-se; lutar; resistir; suportar | combinar; igualar}
\end{EntryWithPhonetic}

\begin{EntryWithPhonetic}{敌人}{di2ren2}{10,2}{⾆,⼈}[HSK 4]
  \definition[群,伙,帮,个,队]{s.}{inimigo; pessoa hostil; parte hostil}
\end{EntryWithPhonetic}

%%%%%%%%%% 笛 %%%%%%%%%%
\subsection*{笛}\addcontentsline{loh}{figure}{笛 \dpy{di2}}

\begin{EntryWithPhonetic}{笛}{di2}{11}{⽵}
  \definition[只]{s.}{flauta de bambu | sirene; apito; buzina}
\end{EntryWithPhonetic}

\begin{EntryWithPhonetic}{笛子}{di2zi5}{11,3}{⽵,⼦}[HSK 7-9]
  \definition[管]{s.}{flauta; flauta de bambu; um instrumento de sopro transversal feito de bambu ou metal com seis furos de tom dispostos em uma fileira de acordo com o tom}
\end{EntryWithPhonetic}

%%%%%%%%%% 底 %%%%%%%%%%
\subsection*{底}\addcontentsline{loh}{figure}{底 \dpy{di3}}

\begin{EntryWithPhonetic}{底}{di3}{8}{⼴}[HSK 4]
  \definition*{s.}{Sobrenome: Di}
  \definition{pron.}{o que? |  isto; isso; aqui | assim; tal}
  \definition{s.}{base; fundo; parte inferior de um objeto | detalhes; o cerne da questão; base, fonte ou contexto de uma coisa | rascunho; cópia mantida como registro; rascunho que pode ser usado como base | final de um ano ou mês | chão; fundo; fundação | a última parte de algo}
  \seeref{de5}
\end{EntryWithPhonetic}

\begin{EntryWithPhonetic}{底层}{di3ceng2}{8,7}{⼴,⼫}[HSK 7-9]
  \definition[个]{s.}{andar térreo | fundo; o degrau mais baixo; classe social mais baixa | porão | subcamada; camada de base; subcapa; substrato}
\end{EntryWithPhonetic}

\begin{EntryWithPhonetic}{底气}{di3qi4}{8,4}{⼴,⽓}
  \definition{s.}{capacidade pulmonar | ousadia | confiança | autoconfiança | vigor}
\end{EntryWithPhonetic}

\begin{EntryWithPhonetic}{底下}{di3xia5}{8,3}{⼴,⼀}[HSK 3]
  \definition{adv.}{em baixo; abaixo; sob | próximo; mais tarde; depois; daqui para a frente}
\end{EntryWithPhonetic}

\begin{EntryWithPhonetic}{底线}{di3xian4}{8,8}{⼴,⽷}[HSK 7-9]
  \definition{s.}{linha de base (em esportes); limites em ambas as extremidades de campos esportivos como futebol, basquete, vôlei e badminton | um mínimo; o limite mais baixo; um limite mínimo; a menor quantidade possível; refere"-se às condições mínimas | um fantoche; um informante; um agente infiltrado; uma pessoa que se esconde dentro do inimigo para reunir informações ou conduzir outras atividades; um \emph{insider}}
\end{EntryWithPhonetic}

\begin{EntryWithPhonetic}{底蕴}{di3yun4}{8,15}{⼴,⾋}[HSK 7-9]
  \definition{s.}{detalhes; informações privilegiadas; história interna}
\end{EntryWithPhonetic}

\begin{EntryWithPhonetic}{底子}{di3zi5}{8,3}{⼴,⼦}[HSK 7-9]
  \definition{s.}{fundo; base; a parte mais baixa de um objeto | solo; base; fundo; fundação | rascunho ou esboço; um rascunho para servir de base | cópia mantida como registro; cópia de arquivo | remanescente | detalhes; prós e contras | Literário: configuração (o padrão base)}
\end{EntryWithPhonetic}

%%%%%%%%%% 抵 %%%%%%%%%%
\subsection*{抵}\addcontentsline{loh}{figure}{抵 \dpy{di3}}

\begin{EntryWithPhonetic}{抵}{di3}{8}{⼿}
  \definition{v.}{apoiar; sustentar | resistir; suportar | compensar; fazer o bem | hipotecar; dar como garantia; garantir | equilibrar; cancelar; compensar | ser igual a; corresponder | alcançar; chegar a | colidir; dar cabeçada (por animais com chifres)}
\end{EntryWithPhonetic}

\begin{EntryWithPhonetic}{抵触}{di3chu4}{8,13}{⼿,⾓}[HSK 7-9]
  \definition{adj.}{conflitante; contraditório}
  \definition{s.}{conflito}
  \definition{v.}{entrar em conflito; contradizer}
\end{EntryWithPhonetic}

\begin{EntryWithPhonetic}{抵达}{di3da2}{8,6}{⼿,⾡}[HSK 6]
  \definition{v.}{chegar; alcançar}
\end{EntryWithPhonetic}

\begin{EntryWithPhonetic}{抵挡}{di3dang3}{8,9}{⼿,⼿}[HSK 7-9]
  \definition{v.}{resistir; suportar; bloquear}
\end{EntryWithPhonetic}

\begin{EntryWithPhonetic}{抵抗}{di3kang4}{8,7}{⼿,⼿}[HSK 6]
  \definition{s.}{resistência}
  \definition{v.}{resistir; usar ação para resistir ou parar o ataque da outra parte}
\end{EntryWithPhonetic}

\begin{EntryWithPhonetic}{抵消}{di3xiao1}{8,10}{⼿,⽔}[HSK 7-9]
  \definition{v.}{compensar; neutralizar; anular; cancelar}
  \antonymref{加强}{jia1qiang2}
\end{EntryWithPhonetic}

\begin{EntryWithPhonetic}{抵销}{di3xiao1}{8,12}{⼿,⾦}
  \definition{v.}{compensar}[三笔债务可以抵销。===As três dívidas podem ser compensadas.]
\end{EntryWithPhonetic}

\begin{EntryWithPhonetic}{抵押}{di3ya1}{8,8}{⼿,⼿}[HSK 7-9]
  \definition{s.}{hipoteca; segurança; garantia}
  \definition{v.}{hipotecar; manter em penhor; penhorar}
\end{EntryWithPhonetic}

\begin{EntryWithPhonetic}{抵御}{di3yu4}{8,12}{⼿,⼻}[HSK 7-9]
  \definition{v.}{resistir; suportar; afastar}[我们要抵御外敌的侵略。===Devemos resistir à invasão estrangeira.]
\end{EntryWithPhonetic}

\begin{EntryWithPhonetic}{抵制}{di3zhi4}{8,8}{⼿,⼑}[HSK 7-9]
  \definition{v.}{resistir; boicotar; bloquear, prevenir e impedir que forças externas invadam ou causem danos}
\end{EntryWithPhonetic}

%%%%%%%%%% 地 %%%%%%%%%%
\subsection*{地}\addcontentsline{loh}{figure}{地 \dpy{di4}}

\begin{EntryWithPhonetic}{地}{di4}{6}{⼟}
  \definition*{s.}{A Terra | Sobrenome: Di}
  \definition[块,片]{s.}{terra; solo | campos | chão; piso | posição; situação | contexto; base | distância percorrida (medida em 里 ou paradas 站) | indicando estado de espírito | território | lugar; local | parte do espaço | distância}
  \seeref{de5}
\end{EntryWithPhonetic}

\begin{EntryWithPhonetic}{地板}{di4ban3}{6,8}{⼟,⽊}[HSK 6]
  \definition[块]{s.}{piso de madeira; tábuas de madeira especiais para pavimentação do piso | piso; piso interno pavimentado com tábuas de madeira; geralmente se refere ao piso de um edifício}
\end{EntryWithPhonetic}

\begin{EntryWithPhonetic}{地步}{di4bu4}{6,7}{⼟,⽌}[HSK 7-9]
  \definition[个,种]{s.}{condição; situação; estado; geralmente ruim | grau; extensão; o grau de realização | margem de manobra; espaço}
\end{EntryWithPhonetic}

\begin{EntryWithPhonetic}{地带}{di4dai4}{6,9}{⼟,⼱}[HSK 5]
  \definition[个,条]{s.}{distrito; região; zona; área de uma determinada natureza ou extensão}
\end{EntryWithPhonetic}

\begin{EntryWithPhonetic}{地道}{di4dao4}{6,12}{⼟,⾡}[HSK 7-9]
  \definition[条,个]{s.}{metrô; túnel; passagem subterrânea; galeria}
  \seeref{di4dao5}
\end{EntryWithPhonetic}

\begin{EntryWithPhonetic}{地道}{di4dao5}{6,12}{⼟,⾡}[HSK 7-9]
  \definition{adj.}{autêntico; puro; típico; alta qualidade, atendendo a certos padrões; como o produto real | excelente; honesto; de acordo com os padrões; de alta qualidade | (da moral e das qualidades de uma pessoa) muito bom; nobre, frequentemente usado em frases negativas}
  \seeref{di4dao4}
\end{EntryWithPhonetic}

\begin{EntryWithPhonetic}{地点}{di4dian3}{6,9}{⼟,⽕}[HSK 1]
  \definition[个]{s.}{lugar; local; região; localização}
\end{EntryWithPhonetic}

\begin{EntryWithPhonetic}{地段}{di4duan4}{6,9}{⼟,⽎}[HSK 7-9]
  \definition[个]{s.}{setor (ou seção) de uma cidade, etc.; área | um setor de uma área; uma seção de uma área; alcance; extensão; lote}
\end{EntryWithPhonetic}

\begin{EntryWithPhonetic}{地方}{di4fang1}{6,4}{⼟,⽅}[HSK 4]
  \definition[个]{s.}{distrito; localidade; o número total de unidades administrativas em todos os níveis abaixo do centro | governo local e população; refere"-se a outros setores que não o militar}
  \seeref{di4fang5}
  \antonymref{中央}{zhong1yang1}
\end{EntryWithPhonetic}

\begin{EntryWithPhonetic}{地方}{di4fang5}{6,4}{⼟,⽅}[HSK 1]
  \definition[个,处,块]{s.}{lugar; cômodo; área; refere"-se a um espaço específico | parte}
  \seeref{di4fang1}
\end{EntryWithPhonetic}

\begin{EntryWithPhonetic}{地核}{di4he2}{6,10}{⼟,⽊}
  \definition{s.}{(geologia) núcleo da Terra}
\end{EntryWithPhonetic}

\begin{EntryWithPhonetic}{地理}{di4li3}{6,11}{⼟,⽟}[HSK 7-9]
  \definition{s.}{características geográficas de um lugar; a situação geral do mundo ou das montanhas e rios de uma região; ambiente natural, como clima e produtos; transporte; assentamentos e outros fatores socioeconômicos | geografia}
\end{EntryWithPhonetic}

\begin{EntryWithPhonetic}{地面}{di4mian4}{6,9}{⼟,⾯}[HSK 4]
  \definition{s.}{a superfície da Terra | térreo; piso; camada de material colocada no chão dentro e ao redor dos edifícios | localidade; chão | região; território; principalmente áreas administrativas}
\end{EntryWithPhonetic}

\begin{EntryWithPhonetic}{地名}{di4ming2}{6,6}{⼟,⼝}[HSK 6]
  \definition{s.}{nome de um lugar | nome de lugar | topônimo}
\end{EntryWithPhonetic}

\begin{EntryWithPhonetic}{地球}{di4qiu2}{6,11}{⼟,⽟}[HSK 2]
  \definition[个]{s.}{o planeta Terra}
\end{EntryWithPhonetic}

\begin{EntryWithPhonetic}{地区}{di4qu1}{6,4}{⼟,⼖}[HSK 3]
  \definition[个,片]{s.}{área; distrito; região; um lugar maior | prefeitura; unidade administrativa | latitudes; localidade; lado | em determinadas circunstâncias, algumas regiões administrativas locais da China, como Hong Kong e Macau, participam individualmente em algumas atividades internacionais}
  \definition{suf.}{como sufixo do nome da cidade, significa prefeitura ou condado}
\end{EntryWithPhonetic}

\begin{EntryWithPhonetic}{地上}{di4shang5}{6,3}{⼟,⼀}[HSK 1]
  \definition{adv.}{no chão; no solo; em terra}
\end{EntryWithPhonetic}

\begin{EntryWithPhonetic}{地毯}{di4tan3}{6,12}{⼟,⽑}[HSK 7-9]
  \definition[块,张]{s.}{tapete; carpete}
\end{EntryWithPhonetic}

\begin{EntryWithPhonetic}{地铁}{di4tie3}{6,10}{⼟,⾦}[HSK 2]
  \definition[条,班,列,趟]{s.}{metrô; trem subterrâneo; também se refere ao vagão do metrô}
\end{EntryWithPhonetic}

\begin{EntryWithPhonetic}{地铁站}{di4tie3zhan4}{6,10,10}{⼟,⾦,⽴}[HSK 2]
  \definition[个,座]{s.}{estação de metrô}
\end{EntryWithPhonetic}

\begin{EntryWithPhonetic}{地图}{di4tu2}{6,8}{⼟,⼞}[HSK 1]
  \definition[张,本]{s.}{mapa; mapa que mostra a distribuição de coisas e fenômenos na superfície da Terra, com símbolos e textos, e às vezes também com cores}
\end{EntryWithPhonetic}

\begin{EntryWithPhonetic}{地位}{di4wei4}{6,7}{⼟,⼈}[HSK 4]
  \definition[个]{s.}{lugar; status; posição; posição da pessoa ou do grupo nas relações sociais | lugar; posição (ocupada por uma pessoa ou coisa); espaço ocupado por uma pessoa ou coisa}
\end{EntryWithPhonetic}

\begin{EntryWithPhonetic}{地下室}{di4xia4shi4}{6,3,9}{⼟,⼀,⼧}[HSK 6]
  \definition{s.}{subterrâneo; porão; adega | abóbadas; cripta}
\end{EntryWithPhonetic}

\begin{EntryWithPhonetic}{地下水}{di4xia4shui3}{6,3,4}{⼟,⼀,⽔}[HSK 7-9]
  \definition{s.}{água subterrânea, principalmente água da chuva e outras águas superficiais que se infiltram no solo e se acumulam nas fendas do solo ou nas formações rochosas}
\end{EntryWithPhonetic}

\begin{EntryWithPhonetic}{地下}{di4xia5}{6,3}{⼟,⼀}[HSK 4]
  \definition{s.}{subterrâneo | secreta (atividade) | recursos ocultos}
\end{EntryWithPhonetic}

\begin{EntryWithPhonetic}{地形}{di4xing2}{6,7}{⼟,⼺}[HSK 5]
  \definition{s.}{topografia; forma do terreno; relevo; disposição do terreno; característica do relevo; característica da superfície; terreno}
\end{EntryWithPhonetic}

\begin{EntryWithPhonetic}{地狱}{di4yu4}{6,9}{⼟,⽝}[HSK 7-9]
  \definition[个,层,重,处]{s.}{inferno; submundo; algumas religiões se referem ao lugar onde a alma sofre após a morte | inferno na terra; lugar de tormento como o inferno; uma metáfora para um ambiente de vida sombrio e miserável}
\end{EntryWithPhonetic}

\begin{EntryWithPhonetic}{地域}{di4yu4}{6,11}{⼟,⼟}[HSK 7-9]
  \definition{s.}{região; distrito}
\end{EntryWithPhonetic}

\begin{EntryWithPhonetic}{地震}{di4zhen4}{6,15}{⼟,⾬}[HSK 5]
  \definition[场,次,级]{s.}{sismo; terremoto; tremor de terra; vibrações na crosta terrestre}
  \definition{v.}{sacudir com vibrações sísmicas}
\end{EntryWithPhonetic}

\begin{EntryWithPhonetic}{地址}{di4zhi3}{6,7}{⼟,⼟}[HSK 4]
  \definition[个,条]{s.}{endereço; local de residência ou correspondência}
\end{EntryWithPhonetic}

\begin{EntryWithPhonetic}{地质}{di4zhi4}{6,8}{⼟,⾙}[HSK 7-9]
  \definition[种]{s.}{geologia; composição e estrutura da crosta terrestre}
\end{EntryWithPhonetic}

\begin{EntryWithPhonetic}{地砖}{di4zhuan1}{6,9}{⼟,⽯}
  \definition{s.}{ladrilho de piso}
\end{EntryWithPhonetic}

%%%%%%%%%% 弟 %%%%%%%%%%
\subsection*{弟}\addcontentsline{loh}{figure}{弟 \dpy{di4}}

\begin{EntryWithPhonetic}{弟}{di4}{7}{⼸}
  \definition*{s.}{Sobrenome: Di}
  \definition[个]{s.}{irmão mais novo | (entre amigos homens) eu | geralmente se refere a colegas do sexo masculino mais jovens na família ou entre parentes | forma humilde que os amigos usam para se referir uns aos outros, usada principalmente em correspondência}
\end{EntryWithPhonetic}

\begin{EntryWithPhonetic}{弟弟}{di4di5}{7,7}{⼸,⼸}[HSK 1]
  \definition[个,位]{s.}{irmão mais novo | primo}
\end{EntryWithPhonetic}

\begin{EntryWithPhonetic}{弟妹}{di4mei4}{7,8}{⼸,⼥}
  \definition{s.}{esposa do irmão mais novo}
\end{EntryWithPhonetic}

\begin{EntryWithPhonetic}{弟子}{di4zi3}{7,3}{⼸,⼦}[HSK 7-9]
  \definition{s.}{discípulo; pupilo; seguidor; nome antigo para estudante, aprendiz}
\end{EntryWithPhonetic}

%%%%%%%%%% 的 %%%%%%%%%%
\subsection*{的}\addcontentsline{loh}{figure}{的 \dpy{di4}}

\begin{EntryWithPhonetic}{的}{di4}{8}{⽩}
  \definition{adj.}{alvo; centro do alvo}
  \seeref{de5}
  \seeref{di1}
  \seeref{di2}
\end{EntryWithPhonetic}

%%%%%%%%%% 帝 %%%%%%%%%%
\subsection*{帝}\addcontentsline{loh}{figure}{帝 \dpy{di4}}

\begin{EntryWithPhonetic}{帝}{di4}{9}{⼱}
  \definition*{s.}{Ser Supremo; Deus}
  \definition[位,名,个]{s.}{imperador | (abreviação) imperialismo}
\end{EntryWithPhonetic}

\begin{EntryWithPhonetic}{帝国}{di4guo2}{9,8}{⼱,⼞}[HSK 7-9]
  \definition[个]{s.}{império}[罗马帝国。===Império romano.]
\end{EntryWithPhonetic}

\begin{EntryWithPhonetic}{帝国主义}{di4guo2 zhu3yi4}{9,8,5,3}{⼱,⼞,⼂,⼂}[HSK 7-9]
  \definition{s.}{imperialismo}[帝国主义是垄断的、寄生的、垂死的资本主义。===Imperialismo é capitalismo monopolista, parasitário e moribundo.]
\end{EntryWithPhonetic}

%%%%%%%%%% 递 %%%%%%%%%%
\subsection*{递}\addcontentsline{loh}{figure}{递 \dpy{di4}}

\begin{EntryWithPhonetic}{递}{di4}{10}{⾡}[HSK 5]
  \definition{adv.}{na ordem correta; sucessivamente}
  \definition{v.}{entregar; passar; dar; transmitir}
\end{EntryWithPhonetic}

\begin{EntryWithPhonetic}{递给}{di4gei3}{10,9}{⾡,⽷}[HSK 5]
  \definition{v.}{entregar algo a alguém; passar itens ou coisas para outras pessoas}
\end{EntryWithPhonetic}

\begin{EntryWithPhonetic}{递交}{di4jiao1}{10,6}{⾡,⼇}[HSK 7-9]
  \definition{v.}{apresentar; submeter; entregar; entregar pessoalmente}
\end{EntryWithPhonetic}

%%%%%%%%%% 第 %%%%%%%%%%
\subsection*{第}\addcontentsline{loh}{figure}{第 \dpy{di4}}

\begin{EntryWithPhonetic}{第}{di4}{11}{⽵}[HSK 1]
  \definition*{s.}{Sobrenome: Di}
  \definition{adv.}{mas, apenas, somente; Indica que a ação não está sujeita a restrições ou condições; equivalente a 只管}
  \definition{conj.}{mas; contudo; orações de conexão; indicando uma relação de transição; equivalente a 但是}
  \definition{pref.}{palavra auxiliar para números ordinais; usado antes de números inteiros, indica ordem}
  \definition{s.}{diferentes notas dos candidatos aprovados nos exames imperiais | a residência de um alto funcionário; grandes residências dos burocratas da era feudal}
  \seealsoref{但是}{dan4shi4}
  \seealsoref{只管}{zhi3guan3}
\end{EntryWithPhonetic}

\begin{EntryWithPhonetic}{第一手}{di4yi1shou3}{11,1,4}{⽵,⼀,⼿}[HSK 7-9]
  \definition{s.}{em primeira mão; obtido por meio de prática e investigação pessoal; obtido diretamente}
\end{EntryWithPhonetic}

\begin{EntryWithPhonetic}{第一线}{di4yi1xian4}{11,1,8}{⽵,⼀,⽷}[HSK 7-9]
  \definition{s.}{vanguarda; linha de frente; primeira linha | frente (linha), primeira linha; a linha de frente do campo de batalha também se refere ao local onde um determinado trabalho é realizado diretamente}
\end{EntryWithPhonetic}

%%%%%%%%%% 墬 %%%%%%%%%%
\subsection*{墬}\addcontentsline{loh}{figure}{墬 \dpy{di4}}

\begin{EntryWithPhonetic}{墬}{di4}{14}{⼟}
  \variantof{地}
\end{EntryWithPhonetic}

%%%%%%%%%% 颠 %%%%%%%%%%
\subsection*{颠}\addcontentsline{loh}{figure}{颠 \dpy{dian1}}

\begin{EntryWithPhonetic}{颠}{dian1}{16}{⾴}
  \definition*{s.}{Sobrenome: Dian}
  \definition{adj.}{mentalmente perturbado; insano; o mesmo que 癫}
  \definition{s.}{coroa (da cabeça) | topo; cume}
  \definition{v.}{sacudir; bater | cair; virar; tombar | Dialeto: correr; ir embora}
  \seealsoref{癫}{dian1}
\end{EntryWithPhonetic}

\begin{EntryWithPhonetic}{颠倒}{dian1dao3}{16,10}{⾴,⼈}[HSK 7-9]
  \definition{adj.}{confuso; desordenado}
  \definition{v.}{inverter; reverter; virar de cabeça para baixo}
\end{EntryWithPhonetic}

\begin{EntryWithPhonetic}{颠覆}{dian1fu4}{16,18}{⾴,⾑}[HSK 7-9]
  \definition{v.}{derrubar; subverter; virar; tombar | tombar; derrubar (um regime legítimo) por conspiração}
\end{EntryWithPhonetic}

%%%%%%%%%% 巅 %%%%%%%%%%
\subsection*{巅}\addcontentsline{loh}{figure}{巅 \dpy{dian1}}

\begin{EntryWithPhonetic}{巅}{dian1}{19}{⼭}
  \definition[个]{s.}{pico da montanha; cume; topo da montanha}
\end{EntryWithPhonetic}

\begin{EntryWithPhonetic}{巅峰}{dian1feng1}{19,10}{⼭,⼭}[HSK 7-9]
  \definition{s.}{um cume; um pico de montanha}
\end{EntryWithPhonetic}

%%%%%%%%%% 癫 %%%%%%%%%%
\subsection*{癫}\addcontentsline{loh}{figure}{癫 \dpy{dian1}}

\begin{EntryWithPhonetic}{癫}{dian1}{21}{⽧}
  \definition{adj.}{mentalmente perturbado; insano; louco}
\end{EntryWithPhonetic}

%%%%%%%%%% 典 %%%%%%%%%%
\subsection*{典}\addcontentsline{loh}{figure}{典 \dpy{dian3}}

\begin{EntryWithPhonetic}{典}{dian3}{8}{⼋}
  \definition{s.}{lei; cânone; padrão; sistema; regulamentos | trabalho padrão de bolsa de estudos; livros que podem servir como padrões ou especificações | alusão; citação literária | cerimônia; uma grande cerimônia (nos tempos antigos, a etiqueta era um dos sistemas importantes do estado) | modelo; normas; regras}
  \definition{v.}{estar no comando de | hipotecar; usar imóveis ou casas como garantia ao pedir dinheiro emprestado}
\end{EntryWithPhonetic}

\begin{EntryWithPhonetic}{典范}{dian3fan4}{8,9}{⼋,⾋}[HSK 7-9]
  \definition{s.}{modelo; exemplo; paradigma; uma pessoa ou coisa que pode ser usada como padrão para aprendizagem ou emulação}
\end{EntryWithPhonetic}

\begin{EntryWithPhonetic}{典礼}{dian3li3}{8,5}{⼋,⽰}[HSK 5]
  \definition[个,次,场]{s.}{cerimônia; celebração; comemoração}
\end{EntryWithPhonetic}

\begin{EntryWithPhonetic}{典型}{dian3xing2}{8,9}{⼋,⼟}[HSK 4]
  \definition{adj.}{típico; representativo}
  \definition[个,种]{s.}{modelo; caso típico; indivíduo ou evento representativo | personagens típicos; personalidades modelo (em obras literárias); personagens na literatura e na arte que refletem a natureza de uma determinada sociedade e têm uma personalidade distinta}
\end{EntryWithPhonetic}

%%%%%%%%%% 点 %%%%%%%%%%
\subsection*{点}\addcontentsline{loh}{figure}{点 \dpy{dian3}}

\begin{EntryWithPhonetic}{点}{dian3}{9}{⽕}[HSK 1]
  \definition{clas.}{hora cheia | ponto, uma unidade de medida para tipos; antigamente, a contagem do tempo durante a noite era feita por turnos, sendo cada turno dividido em cinco pontos | quantidade ínfima; um pouco; um pouquinho; alguma coisa; indica uma pequena quantidade | usado para itens}
  \definition{s.}{gota (de líquido); (ponto) pequena gota de líquido | mancha; ponto; salpico; (um pouco) Um pequeno vestígio | (ponto) traço de um caractere chinês, cuja forma é ``、''  | Matemática: ponto; refere"-se a uma figura geométrica que não tem comprimento, largura ou altura, mas apenas uma posição | gongo, instrumento musical de metal | ponto decimal; refere"-se ao ponto decimal, símbolo matemático que representa os números decimais | lugar específico | lanche leve; petisco | lugar; grau; sinalização de um determinado local ou grau | hora marcada; hora regulamentar | aspecto; característica; partes ou aspectos específicos de algo | ritmo; batida}
  \definition{v.}{andar na ponta dos pés | dar uma dica, sugestão | tocar levemente com o dedo, pincel ou vara; tocar muito brevemente; passar rapidamente | acenar; baixar ligeiramente a cabeça e levantar rapidamente | gotejar; fazer cair líquido | semear em buracos; plantar com um plantador | verificar um por um | colocar um ponto; usar caneta e outras ferramentas para adicionar ideias | sugerir; indicar; dar uma dica | decorar; realçar | selecionar; escolher; especificar o que é exigido | acender; queimar; inflamar | (pedido) comer uma pequena quantidade de comida para saciar a fome}
\end{EntryWithPhonetic}

\begin{EntryWithPhonetic}{点火}{dian3/huo3}{9,4}{⽕,⽕}[HSK 7-9]
  \definition{s.}{ignição}
  \definition{v.+compl.}{acender; acender o fogo; acender uma luz; disparar; iniciar a combustão; acender as chamas | inflamar | causar problemas}
\end{EntryWithPhonetic}

\begin{EntryWithPhonetic}{点击率}{dian3ji1lv4}{9,5,11}{⽕,⼐,⽞}[HSK 7-9]
  \definition{s.}{Internet: taxa de cliques (CTR, \emph{click-through rate})}
\end{EntryWithPhonetic}

\begin{EntryWithPhonetic}{点名}{dian3 ming2}{9,6}{⽕,⼝}[HSK 4]
  \definition{v.}{fazer a lista de chamada; manter o controle da presença de alguém; chamar nomes para controle de presença | mencionar alguém pelo nome}
\end{EntryWithPhonetic}

\begin{EntryWithPhonetic}{点评}{dian3ping2}{9,7}{⽕,⾔}[HSK 7-9]
  \definition{s.}{comentário; crítica | um comentário ponto por ponto}
  \definition{v.}{comentar; fazer comentários}
\end{EntryWithPhonetic}

\begin{EntryWithPhonetic}{点燃}{dian3ran2}{9,16}{⽕,⽕}[HSK 5]
  \definition{v.}{acender; inflamar; acender uma fogueira, para iluminar}
\end{EntryWithPhonetic}

\begin{EntryWithPhonetic}{点头}{dian3tou2}{9,5}{⽕,⼤}[HSK 2]
  \definition{v.}{acenar com a cabeça; balançar a cabeça; mover ligeiramente a cabeça para baixo; indicar permissão, aprovação, compreensão ou saudação}
\end{EntryWithPhonetic}

\begin{EntryWithPhonetic}{点心}{dian3xin1}{9,4}{⽕,⼼}
  \definition[块,份,袋,包,盒]{s.}{lanche; refeição leve}
  \seeref{dian3xin5}
\end{EntryWithPhonetic}

\begin{EntryWithPhonetic}{点心}{dian3xin5}{9,4}{⽕,⼼}[HSK 7-9]
  \definition[块,份,袋,包,盒]{s.}{doces; docinhos; sobremesa; lanches leves; pequenos alimentos além de arroz e vegetais, como bolos, biscoitos, etc.}
  \seeref{dian3xin1}
\end{EntryWithPhonetic}

\begin{EntryWithPhonetic}{点缀}{dian3zhui4}{9,11}{⽕,⽷}[HSK 7-9]
  \definition{v.}{adornar; embelezar; ornamentar; realçar, decorar, fazer com que pareça melhor | usar algo apenas para mostrar; decorar a fachada; adaptar-se à ocasião}
\end{EntryWithPhonetic}

\begin{EntryWithPhonetic}{点子}{dian3zi5}{9,3}{⽕,⼦}[HSK 7-9]
  \definition[滴,个]{s.}{gota (de líquido); pequenas gotas | mancha; ponto; pinta | batidas de instrumentos de percussão | ponto-chave; aspecto vital | ideia; indicador; método}
\end{EntryWithPhonetic}

%%%%%%%%%% 电 %%%%%%%%%%
\subsection*{电}\addcontentsline{loh}{figure}{电 \dpy{dian4}}

\begin{EntryWithPhonetic}{电}{dian4}{5}{⽥}[HSK 1]
  \definition*{s.}{Sobrenome: Dian}
  \definition{s.}{eletricidade; energia elétrica | telegrama | relâmpago}
  \definition{v.}{dar ou receber um choque elétrico | enviar telegrama, telefonar ou enviar fax}
\end{EntryWithPhonetic}

\begin{EntryWithPhonetic}{电报}{dian4bao4}{5,7}{⽥,⼿}[HSK 7-9]
  \definition[封,份,个]{s.}{telegrama; cabo; telégrafo; mensagem}
\end{EntryWithPhonetic}

\begin{EntryWithPhonetic}{电冰箱}{dian4bing1xiang1}{5,6,15}{⽥,⼎,⾋}
  \definition[台]{s.}{frigorífico | refrigerador}
\end{EntryWithPhonetic}

\begin{EntryWithPhonetic}{电车}{dian4che1}{5,4}{⽥,⾞}[HSK 6]
  \definition[辆,班,趟,路]{s.}{bonde; veículos de transporte público urbano movidos por linhas aéreas e acionados por motores de tração}
\end{EntryWithPhonetic}

\begin{EntryWithPhonetic}{电车司机}{dian4che1 si1ji1}{5,4,5,6}{⽥,⾞,⼝,⽊}
  \definition{s.}{motorista de bonde}
\end{EntryWithPhonetic}

\begin{EntryWithPhonetic}{电池}{dian4chi2}{5,6}{⽥,⽔}[HSK 5]
  \definition[节,块,组,个]{s.}{célula; bateria}
\end{EntryWithPhonetic}

\begin{EntryWithPhonetic}{电灯}{dian4deng1}{5,6}{⽥,⽕}[HSK 4]
  \definition[盏,个]{s.}{luz elétrica; lâmpada elétrica; lâmpadas que usam eletricidade como fonte de energia}
\end{EntryWithPhonetic}

\begin{EntryWithPhonetic}{电灯泡}{dian4deng1pao4}{5,6,8}{⽥,⽕,⽔}
  \definition{s.}{lâmpada elétrica | (gíria) terceiro convidado indesejado}
\end{EntryWithPhonetic}

\begin{EntryWithPhonetic}{电动}{dian4dong4}{5,6}{⽥,⼒}[HSK 6]
  \definition{adj.}{motorizado; acionado por energia elétrica; operado por energia elétrica elétrico}
\end{EntryWithPhonetic}

\begin{EntryWithPhonetic}{电动车}{dian4dong4che1}{5,6,4}{⽥,⼒,⾞}[HSK 4]
  \definition{s.}{veículo elétrico (\emph{scooter}, bicicleta, carro, etc.)}
\end{EntryWithPhonetic}

\begin{EntryWithPhonetic}{电饭锅}{dian4fan4guo1}{5,7,12}{⽥,⾷,⾦}[HSK 5]
  \definition[台,个]{s.}{panela elétrica de arroz}
\end{EntryWithPhonetic}

\begin{EntryWithPhonetic}{电话}{dian4hua4}{5,8}{⽥,⾔}[HSK 1]
  \definition[部]{s.}{telefone; aparelho telefônico; telefonia}
  \definition[通]{s.}{chamada telefônica; telefonema}
\end{EntryWithPhonetic}

\begin{EntryWithPhonetic}{电力}{dian4li4}{5,2}{⽥,⼒}[HSK 6]
  \definition{s.}{energia elétrica; fornecimento de energia elétrica | energia elétrica | eletricidade}
\end{EntryWithPhonetic}

\begin{EntryWithPhonetic}{电铃}{dian4ling2}{5,10}{⽥,⾦}[HSK 7-9]
  \definition[个]{s.}{campainha elétrica}
\end{EntryWithPhonetic}

\begin{EntryWithPhonetic}{电门铃}{dian4 men2ling2}{5,3,10}{⽥,⾨,⾦}
  \definition{s.}{campainha elétrica}
\end{EntryWithPhonetic}

\begin{EntryWithPhonetic}{电脑}{dian4nao3}{5,10}{⽥,⾁}[HSK 1]
  \definition[个,台]{s.}{computador eletrônico}
\end{EntryWithPhonetic}

\begin{EntryWithPhonetic}{电脑语言}{dian4nao3yu3yan2}{5,10,9,7}{⽥,⾁,⾔,⾔}
  \definition{s.}{linguagem de programação | linguagem de computador}
\end{EntryWithPhonetic}

\begin{EntryWithPhonetic}{电器}{dian4qi4}{5,16}{⽥,⼝}[HSK 6]
  \definition[件,种]{s.}{dispositivo elétrico; cargas em circuitos e dispositivos usados para controlar, regular ou proteger circuitos, motores, etc.; como alto"-falantes; interruptores; resistores; fusíveis, etc. | eletrodomésticos ou aparelhos elétricos domésticos; refere"-se a eletrodomésticos, como televisores, gravadores, geladeiras, máquinas de lavar, etc.}
\end{EntryWithPhonetic}

\begin{EntryWithPhonetic}{电视}{dian4shi4}{5,8}{⽥,⾒}[HSK 1]
  \definition[部,台,个]{s.}{televisão; TV; televisor}
\end{EntryWithPhonetic}

\begin{EntryWithPhonetic}{电视机}{dian4shi4ji1}{5,8,6}{⽥,⾒,⽊}[HSK 1]
  \definition[个,台]{s.}{aparelho de TV; receptor de televisão; receptor de imagem; televisor; aparelho de televisão}
\end{EntryWithPhonetic}

\begin{EntryWithPhonetic}{电视剧}{dian4shi4ju4}{5,8,10}{⽥,⾒,⼑}[HSK 3]
  \definition[部,集,个]{s.}{série de TV; drama de TV; novela; drama escrito e gravado para transmissão pela televisão}
\end{EntryWithPhonetic}

\begin{EntryWithPhonetic}{电视台}{dian4shi4tai2}{5,8,5}{⽥,⾒,⼝}[HSK 3]
  \definition[家,座,个]{s.}{canal de TV; estação de televisão; locais e instituições que transmitem programas de televisão}
\end{EntryWithPhonetic}

\begin{EntryWithPhonetic}{电台}{dian4tai2}{5,5}{⽥,⼝}[HSK 3]
  \definition[个,家]{s.}{transceptor; transmissor-receptor | aparelho de rádio; estação de rádio; estação de transmissão}
\end{EntryWithPhonetic}

\begin{EntryWithPhonetic}{电梯}{dian4ti1}{5,11}{⽥,⽊}[HSK 4]
  \definition[部,台,架]{s.}{elevador}
\end{EntryWithPhonetic}

\begin{EntryWithPhonetic}{电梯司机}{dian4ti1 si1ji1}{5,11,5,6}{⽥,⽊,⼝,⽊}
  \definition{s.}{ascensorista}
\end{EntryWithPhonetic}

\begin{EntryWithPhonetic}{电网}{dian4wang3}{5,6}{⽥,⽹}[HSK 7-9]
  \definition[张,个]{s.}{rede de arame eletrificada; emaranhamento de fios energizados | rede elétrica (ou grade)}
\end{EntryWithPhonetic}

\begin{EntryWithPhonetic}{电线}{dian4xian4}{5,8}{⽥,⽷}[HSK 7-9]
  \definition[根,个]{s.}{fio; fio elétrico; condutor de corrente; condutor que transmite energia elétrica}
\end{EntryWithPhonetic}

\begin{EntryWithPhonetic}{电信}{dian4xin4}{5,9}{⽥,⼈}[HSK 7-9]
  \definition{s.}{telecomunicações; métodos de transmissão de informações usando tecnologias de comunicação com fio, rádio e óptica, incluindo telégrafo, telefone, etc.}
\end{EntryWithPhonetic}

\begin{EntryWithPhonetic}{电讯}{dian4xun4}{5,5}{⽥,⾔}[HSK 7-9]
  \definition{s.}{despacho (telegráfico) | telecomunicações; uma mensagem enviada por telefone, telégrafo ou rádio (telegráfico) | sinais de comunicação de rádio; sinais de rádio}
\end{EntryWithPhonetic}

\begin{EntryWithPhonetic}{电影}{dian4ying3}{5,15}{⽥,⼺}[HSK 1]
  \definition[部,片,幕,场]{s.}{filme; longa-metragem; cinema}
\end{EntryWithPhonetic}

\begin{EntryWithPhonetic}{电影奖}{dian4ying3jiang3}{5,15,9}{⽥,⼺,⼤}
  \definition{s.}{premiações de cinema}
\end{EntryWithPhonetic}

\begin{EntryWithPhonetic}{电影节}{dian4ying3jie2}{5,15,5}{⽥,⼺,⾋}
  \definition{s.}{festival de cinema}
\end{EntryWithPhonetic}

\begin{EntryWithPhonetic}{电影界}{dian4ying3jie4}{5,15,9}{⽥,⼺,⽥}
  \definition{s.}{indústria cinematográfica}
\end{EntryWithPhonetic}

\begin{EntryWithPhonetic}{电影票}{dian4ying3piao4}{5,15,11}{⽥,⼺,⽰}
  \definition{s.}{ingresso de filme}
\end{EntryWithPhonetic}

\begin{EntryWithPhonetic}{电影术}{dian4ying3 shu4}{5,15,5}{⽥,⼺,⽊}
  \definition{s.}{cinematografia}
\end{EntryWithPhonetic}

\begin{EntryWithPhonetic}{电影艺术}{dian4ying3 yi4shu4}{5,15,4,5}{⽥,⼺,⾋,⽊}
  \definition{s.}{arte cinematográfica}
\end{EntryWithPhonetic}

\begin{EntryWithPhonetic}{电影音乐}{dian4ying3 yin1yue4}{5,15,9,5}{⽥,⼺,⾳,⼃}
  \definition{s.}{música cinematográfica}
\end{EntryWithPhonetic}

\begin{EntryWithPhonetic}{电影院}{dian4ying3 yuan4}{5,15,9}{⽥,⼺,⾩}[HSK 1]
  \definition[家,座,个]{s.}{cinema; sala de cinema; teatro; salão de cinema; local comercial dedicado à exibição de filmes}
\end{EntryWithPhonetic}

\begin{EntryWithPhonetic}{电邮}{dian4you2}{5,7}{⽥,⾢}
  \definition{s.}{correio eletrônico, \emph{e-mail} | abreviação de~电子邮件}
  \seealsoref{电子邮件}{dian4zi3you2jian4}
\end{EntryWithPhonetic}

\begin{EntryWithPhonetic}{电源}{dian4yuan2}{5,13}{⽥,⽔}[HSK 4]
  \definition[台,个,套]{s.}{fonte de alimentação; fonte de energia; fonte de energia elétrica; dispositivo que fornece energia elétrica a um aparelho, como uma bateria, um gerador, etc.}
\end{EntryWithPhonetic}

\begin{EntryWithPhonetic}{电子}{dian4zi3}{5,3}{⽥,⼦}
  \definition{s.}{eletrônico | elétron}
\end{EntryWithPhonetic}

\begin{EntryWithPhonetic}{电子版}{dian4zi3ban3}{5,3,8}{⽥,⼦,⽚}[HSK 5]
  \definition[个]{s.}{edição eletrônica}
\end{EntryWithPhonetic}

\begin{EntryWithPhonetic}{电子名片}{dian4zi3 ming2pian4}{5,3,6,4}{⽥,⼦,⼝,⽚}
  \definition{s.}{cartão de visita eletrônico}
\end{EntryWithPhonetic}

\begin{EntryWithPhonetic}{电子邮件}{dian4zi3you2jian4}{5,3,7,6}{⽥,⼦,⾢,⼈}[HSK 3]
  \definition[封,份,个,条]{s.}{correio eletrônico; \emph{e-mail}}
  \seealsoref{电邮}{dian4you2}
\end{EntryWithPhonetic}

%%%%%%%%%% 店 %%%%%%%%%%
\subsection*{店}\addcontentsline{loh}{figure}{店 \dpy{dian4}}

\begin{EntryWithPhonetic}{店}{dian4}{8}{⼴}[HSK 2]
  \definition[家,间,个]{s.}{loja; armazém; loja de venda de mercadorias | pousada; pequena pousada com instalações simples | usado para nomes de lugares}
\end{EntryWithPhonetic}

\begin{EntryWithPhonetic}{店员}{dian4yuan2}{8,7}{⼴,⼝}
  \definition{s.}{assistente de loja | balconista | vendedor}
\end{EntryWithPhonetic}

\begin{EntryWithPhonetic}{店主}{dian4zhu3}{8,5}{⼴,⼂}
  \definition{s.}{lojista | dono de loja}
\end{EntryWithPhonetic}

%%%%%%%%%% 垫 %%%%%%%%%%
\subsection*{垫}\addcontentsline{loh}{figure}{垫 \dpy{dian4}}

\begin{EntryWithPhonetic}{垫}{dian4}{9}{⼟}[HSK 7-9]
  \definition[个]{s.}{almofada}
  \definition{v.}{colocar algo sob; elevar ou nivelar; encher; preencher | pagar por alguém e esperar ser reembolsado mais tarde | colocar algo sob algo para elevá-lo ou nivelá-lo; usar algo para apoiar, espalhar ou forrar algo para torná-lo mais alto, mais grosso ou mais plano | preencher uma vaga; preencher uma lacuna}
\end{EntryWithPhonetic}

\begin{EntryWithPhonetic}{垫底}{dian4/di3}{9,8}{⼟,⼴}[HSK 7-9]
  \definition{v.+compl.}{rebasear; forrar; cobrir}
\end{EntryWithPhonetic}

\begin{EntryWithPhonetic}{垫子}{dian4zi5}{9,3}{⼟,⼦}[HSK 7-9]
  \definition{s.}{colchão; esteira; almofada; algo para colocar em uma cama, cadeira, banquinho ou em outro lugar}
\end{EntryWithPhonetic}

%%%%%%%%%% 钿 %%%%%%%%%%
\subsection*{钿}\addcontentsline{loh}{figure}{钿 \dpy{dian4}}

\begin{EntryWithPhonetic}{钿}{dian4}{10}{⾦}
  \definition{s.}{ornamento incrustado antigo em forma de flor | enfeite de cabelo feminino com flores douradas | incrustação de madrepérola; um padrão incrustado com conchas de caracóis em madeira e laca}
  \definition{v.}{incrustar com ouro, prata, etc.}
  \seeref{tian2}
\end{EntryWithPhonetic}

%%%%%%%%%% 惦 %%%%%%%%%%
\subsection*{惦}\addcontentsline{loh}{figure}{惦 \dpy{dian4}}

\begin{EntryWithPhonetic}{惦}{dian4}{11}{⼼}
  \definition{v.}{lembrar com preocupação; estar preocupado com; continuar pensando sobre; ficar pensando em alguém ou em alguma coisa e se preocupar com eles; sentir falta deles}
\end{EntryWithPhonetic}

\begin{EntryWithPhonetic}{惦记}{dian4ji4}{11,5}{⼼,⾔}[HSK 7-9]
  \definition{s.}{lembrar com preocupação; estar preocupado com; continuar pensando sobre; (sobre uma pessoa ou coisa) continuar pensando nisso e não deixar passar}
\end{EntryWithPhonetic}

%%%%%%%%%% 淀 %%%%%%%%%%
\subsection*{淀}\addcontentsline{loh}{figure}{淀 \dpy{dian4}}

\begin{EntryWithPhonetic}{淀}{dian4}{11}{⽔}
  \definition{s.}{lago raso, frequentemente usado em nomes de lugares}
  \definition{v.}{formar sedimentos | sedimentar; precipitar}
\end{EntryWithPhonetic}

\begin{EntryWithPhonetic}{淀粉}{dian4fen3}{11,10}{⽔,⽶}[HSK 7-9]
  \definition[包,勺,克,的]{s.}{amido; amilo; os carboidratos são os principais componentes das sementes, raízes e tubérculos}
\end{EntryWithPhonetic}

%%%%%%%%%% 奠 %%%%%%%%%%
\subsection*{奠}\addcontentsline{loh}{figure}{奠 \dpy{dian4}}

\begin{EntryWithPhonetic}{奠}{dian4}{12}{⼤}
  \definition{v.}{estabelecer; construir; fundar; lançar a pedra fundamental | fazer oferendas aos espíritos dos mortos | consertar}
\end{EntryWithPhonetic}

\begin{EntryWithPhonetic}{奠定}{dian4ding4}{12,8}{⼤,⼧}[HSK 7-9]
  \definition{v.}{estabelecer; estabelecer de forma estável; tornar estável; estabelecer a base}
\end{EntryWithPhonetic}

%%%%%%%%%% 殿 %%%%%%%%%%
\subsection*{殿}\addcontentsline{loh}{figure}{殿 \dpy{dian4}}

\begin{EntryWithPhonetic}{殿}{dian4}{13}{⽎}
  \definition*{s.}{Sobrenome: Dian}
  \definition[座]{s.}{salão; palácio; templo}
  \definition{v.}{fechar a retaguarda (em uma marcha)}
\end{EntryWithPhonetic}

\begin{EntryWithPhonetic}{殿堂}{dian4tang2}{13,11}{⽎,⼟}[HSK 7-9]
  \definition{s.}{palácio; templo; santuário | salão do palácio; salão do templo}
\end{EntryWithPhonetic}

%%%%%%%%%% 刁 %%%%%%%%%%
\subsection*{刁}\addcontentsline{loh}{figure}{刁 \dpy{diao1}}

\begin{EntryWithPhonetic}{刁}{diao1}{2}{⼑}
  \definition*{s.}{Sobrenome: Diao}
  \definition{adj.}{traiçoeiro; astuto | exigente; exigente com a comida; difícil}
  \definition{v.}{dificultar as coisas}
\end{EntryWithPhonetic}

\begin{EntryWithPhonetic}{刁难}{diao1nan4}{2,10}{⼑,⾫}[HSK 7-9]
  \definition{v.}{criar dificuldades; tornar as coisas difíceis; dificultar deliberadamente as coisas para os outros}
\end{EntryWithPhonetic}

%%%%%%%%%% 叼 %%%%%%%%%%
\subsection*{叼}\addcontentsline{loh}{figure}{叼 \dpy{diao1}}

\begin{EntryWithPhonetic}{叼}{diao1}{5}{⼝}[HSK 7-9]
  \definition{v.}{segurar na boca; segurar com a boca}
\end{EntryWithPhonetic}

%%%%%%%%%% 貂 %%%%%%%%%%
\subsection*{貂}\addcontentsline{loh}{figure}{貂 \dpy{diao1}}

\begin{EntryWithPhonetic}{貂}{diao1}{12}{⾘}
  \definition*{s.}{Sobrenome: Diao}
  \definition[只]{s.}{marta; fuinha; arminho}
\end{EntryWithPhonetic}

%%%%%%%%%% 雕 %%%%%%%%%%
\subsection*{雕}\addcontentsline{loh}{figure}{雕 \dpy{diao1}}

\begin{EntryWithPhonetic}{雕}{diao1}{16}{⾫}[HSK 7-9]
  \definition*{s.}{Sobrenome: Diao}
  \definition{s.}{abutre; águia | escultura ou obras esculpidas}
  \definition{v.}{esculpir; gravar}
\end{EntryWithPhonetic}

\begin{EntryWithPhonetic}{雕刻}{diao1ke4}{16,8}{⾫,⼑}[HSK 7-9]
  \definition[件,尊,个]{s.}{escultura; entalhe; obras de arte esculpidas}
  \definition{v.}{esculpir; gravar; esculpir uma imagem em metal, marfim, osso ou outros materiais}
\end{EntryWithPhonetic}

\begin{EntryWithPhonetic}{雕塑}{diao1su4}{16,13}{⾫,⼟}[HSK 7-9]
  \definition[座,件,尊,个]{s.}{escultura; obras de arte tridimensionais feitas de madeira, pedra ou metal por meio de entalhe, empilhamento, batida, etc.}
  \definition{v.}{esculpir; entalhar e moldar; usar madeira, pedra ou metal para criar formas artísticas tridimensionais esculpindo, empilhando, batendo, etc.}
\end{EntryWithPhonetic}

%%%%%%%%%% 鸟 %%%%%%%%%%
\subsection*{鸟}\addcontentsline{loh}{figure}{鸟 \dpy{diao3}}

\begin{EntryWithPhonetic}{鸟}{diao3}{5}{⿃}[Kangxi 196]
  \definition{s.}{(em romances tradicionais, como termo pejorativo) maldito; condenado; fudido; o mesmo que 屌}
  \seeref{niao3}
  \seealsoref{屌}{diao3}
\end{EntryWithPhonetic}

%%%%%%%%%% 屌 %%%%%%%%%%
\subsection*{屌}\addcontentsline{loh}{figure}{屌 \dpy{diao3}}

\begin{EntryWithPhonetic}{屌}{diao3}{9}{⼫}
  \definition{adj.}{(gíria) legal ou extraordinário}
  \definition{s.}{órgão genital masculino; pênis}
  \definition{v.}{(cantonês) foder}
\end{EntryWithPhonetic}

\begin{EntryWithPhonetic}{屌丝}{diao3si1}{9,5}{⼫,⼀}
  \definition{adj.}{panaca | zé-ninguém | (gíria de \emph{Internet}) \emph{looser}}
\end{EntryWithPhonetic}

%%%%%%%%%% 吊 %%%%%%%%%%
\subsection*{吊}\addcontentsline{loh}{figure}{吊 \dpy{diao4}}

\begin{EntryWithPhonetic}{吊}{diao4}{6}{⼝}[HSK 6]
  \definition{clas.}{uma sequência de 1.000 em dinheiro; antigamente, uma unidade monetária geralmente era composta por mil pequenas moedas de cobre}
  \definition{s.}{guindaste}
  \definition{v.}{pendurar; suspender | levantar ou abaixar com uma corda, etc. | colocar um forro de pele; adicionar revestimentos ou forros aos barris de couro para fazer roupas | revogar; retirar; recuperar documentos emitidos | lamentar; prestar homenagem os mortos ou oferecer condolências às famílias ou grupos que sofreram uma perda}
\end{EntryWithPhonetic}

\begin{EntryWithPhonetic}{吊销}{diao4xiao1}{6,12}{⼝,⾦}[HSK 7-9]
  \definition{v.}{revogar; retirar; desativar; cancelar; reclamar e cancelar (certificados emitidos)}
\end{EntryWithPhonetic}

%%%%%%%%%% 钓 %%%%%%%%%%
\subsection*{钓}\addcontentsline{loh}{figure}{钓 \dpy{diao4}}

\begin{EntryWithPhonetic}{钓}{diao4}{8}{⾦}
  \definition{v.}{pescar com anzol e linha | buscar (fama e ganho pessoal) | fisgar; defraudar por meios desleais}
\end{EntryWithPhonetic}

\begin{EntryWithPhonetic}{钓鱼}{diao4yu2}{8,8}{⾦,⿂}[HSK 7-9]
  \definition{v.}{pescar; ir pescar; atividade de captura de peixes com equipamentos de pesca na beira da água, que é uma forma de lazer e entretenimento | Figurativo: aprisionar; Internet: 钓鱼 significa que alguém publica deliberadamente algo que pode causar controvérsia, raiva ou outras emoções fortes, a fim de atrair pessoas para responder e discutir}
\end{EntryWithPhonetic}

%%%%%%%%%% 调 %%%%%%%%%%
\subsection*{调}\addcontentsline{loh}{figure}{调 \dpy{diao4}}

\begin{EntryWithPhonetic}{调}{diao4}{10}{⾔}[HSK 3]
  \definition{s.}{sotaque; pronúncia | nota (musical) | melodia; música | tom; refere"-se ao tom da fala, ou seja, a elevação e descida do tom das palavras | estilo; ambiente; estilo metafórico, talento, etc. | argumento; discurso}
  \definition{v.}{deslocar; mover; transferir; mover (pessoas, objetos, etc.) de um lugar para outro | examinar; investigar}
  \seeref{tiao2}
\end{EntryWithPhonetic}

\begin{EntryWithPhonetic}{调查}{diao4cha2}{10,9}{⾔,⽊}[HSK 3]
  \definition[项,个,份]{s.}{pesquisa; investigação; informações obtidas após perguntar a outras pessoas ou investigar}
  \definition{v.}{investigar; indagar; inquerir; examinar; realizar uma investigação (geralmente no local) para entender a situação}
\end{EntryWithPhonetic}

\begin{EntryWithPhonetic}{调动}{diao4dong4}{10,6}{⾔,⼒}[HSK 5]
  \definition{v.}{mudar; transferir; pessoal, trabalho | mobilizar; despertar; pôr em jogo; melhorar (motivação, entusiasmo, etc.) por meio de alguns meios | reunir; manobrar; mover (tropas); mobilizar forças militares}
\end{EntryWithPhonetic}

\begin{EntryWithPhonetic}{调度}{diao4du4}{10,9}{⾔,⼴}[HSK 7-9]
  \definition[位,个]{s.}{despachante; pessoal responsável pelo despacho do trabalho}
  \definition{v.}{organizar e despachar; arranjar e despachar}
\end{EntryWithPhonetic}

\begin{EntryWithPhonetic}{调研}{diao4yan2}{10,9}{⾔,⽯}[HSK 6]
  \definition{v.}{pesquisar e estudar; investigar e pesquisar; pesquisar}
\end{EntryWithPhonetic}

%%%%%%%%%% 掉 %%%%%%%%%%
\subsection*{掉}\addcontentsline{loh}{figure}{掉 \dpy{diao4}}

\begin{EntryWithPhonetic}{掉}{diao4}{11}{⼿}[HSK 2]
  \definition{v.}{cair; soltar-se; desprender-se | ficar para trás | perder; desaparecer; omitir | diminuir; reduzir | balançar; abanar; oscilar | virar; voltar; retornar | alterar; trocar; intercambiar}
  \definition{v.aux.}{usado após certos verbos para indicar a conclusão de uma ação}
\end{EntryWithPhonetic}

\begin{EntryWithPhonetic}{掉包}{diao4bao1}{11,5}{⼿,⼓}
  \definition{v.}{vender uma falsificação pelo artigo genuíno | roubar o item valioso de alguém e substituí-lo por um item de aparência semelhante, mas sem valor}
\end{EntryWithPhonetic}

\begin{EntryWithPhonetic}{掉膘}{diao4biao1}{11,15}{⼿,⾁}
  \definition{v.}{perder peso (gado)}
\end{EntryWithPhonetic}

\begin{EntryWithPhonetic}{掉队}{diao4/dui4}{11,4}{⼿,⾩}[HSK 7-9]
  \definition{v.+compl.}{abandonar (ou sair); ficar para trás; cair fora}
\end{EntryWithPhonetic}

\begin{EntryWithPhonetic}{掉落}{diao4luo4}{11,12}{⼿,⾋}
  \definition{v.}{derrubar}
\end{EntryWithPhonetic}

\begin{EntryWithPhonetic}{掉头}{diao4/tou2}{11,5}{⼿,⼤}[HSK 7-9]
  \definition{v.+compl.}{virar-se; afastar-se (das pessoas) | dar meia-volta (carro, barco, etc.); (carro, barco, etc.) virar na direção oposta}
\end{EntryWithPhonetic}

\begin{EntryWithPhonetic}{掉线}{diao4xian4}{11,8}{⼿,⽷}
  \definition{v.}{desconectar-se (da \emph{Internet})}
\end{EntryWithPhonetic}

\begin{EntryWithPhonetic}{掉转}{diao4zhuan3}{11,8}{⼿,⾞}
  \definition{v.}{dar a volta}
\end{EntryWithPhonetic}

%%%%%%%%%% 爹 %%%%%%%%%%
\subsection*{爹}\addcontentsline{loh}{figure}{爹 \dpy{die1}}

\begin{EntryWithPhonetic}{爹}{die1}{10}{⽗}[HSK 7-9]
  \definition[个]{s.}{Coloquial: pai; papai | velho pai; um título respeitoso para homens idosos em algumas áreas}
\end{EntryWithPhonetic}

%%%%%%%%%% 跌 %%%%%%%%%%
\subsection*{跌}\addcontentsline{loh}{figure}{跌 \dpy{die1}}

\begin{EntryWithPhonetic}{跌}{die1}{12}{⾜}[HSK 6]
  \definition{s.}{(de um objeto, etc.) queda; tombo | (de preços, etc.) queda}
  \definition{v.}{cair; tombar; perder o equilíbrio e cair | cair (objetos caindo); descer | cair (queda de preços)}
\end{EntryWithPhonetic}

%%%%%%%%%% 迭 %%%%%%%%%%
\subsection*{迭}\addcontentsline{loh}{figure}{迭 \dpy{die2}}

\begin{EntryWithPhonetic}{迭}{die2}{8}{⾡}
  \definition*{s.}{Sobrenome: Die}
  \definition{adv.}{repetidamente; de novo e de novo | a tempo para}
  \definition{v.}{alternar; mudar; revezar-se; substituir}
\end{EntryWithPhonetic}

\begin{EntryWithPhonetic}{迭起}{die2qi3}{8,10}{⾡,⾛}[HSK 7-9]
  \definition{v.}{ocorrer repetidamente; acontecer com frequência | surgir repetidamente}
\end{EntryWithPhonetic}

%%%%%%%%%% 叠 %%%%%%%%%%
\subsection*{叠}\addcontentsline{loh}{figure}{叠 \dpy{die2}}

\begin{EntryWithPhonetic}{叠}{die2}{13}{⼜}[HSK 7-9]
  \definition{clas.}{maço; pacote; pilha}
  \definition{v.}{acumular; empilhar | dobrar}
\end{EntryWithPhonetic}

%%%%%%%%%% 丁 %%%%%%%%%%
\subsection*{丁}\addcontentsline{loh}{figure}{丁 \dpy{ding1}}

\begin{EntryWithPhonetic}{丁}{ding1}{2}{⼀}[HSK 7-9]
  \definition*{s.}{O quarto dos Dez Troncos Celestiais | Sobrenome: Ding}
  \definition{s.}{homem; homem adulto | população; membro de uma família | uma pessoa envolvida em uma determinada ocupação; pessoas em certas profissões | cubos; pequenos cubos de carne ou vegetais}
  \definition{v.}{encontrar; encontrar-se com; esbarrar em}
  \seeref{zheng1}
\end{EntryWithPhonetic}

%%%%%%%%%% 叮 %%%%%%%%%%
\subsection*{叮}\addcontentsline{loh}{figure}{叮 \dpy{ding1}}

\begin{EntryWithPhonetic}{叮}{ding1}{5}{⼝}
  \definition{v.}{picar; ferroar | dizer ou perguntar novamente para ter certeza; verificar; insistir; certificar-se | sondar; perseguir}
\end{EntryWithPhonetic}

\begin{EntryWithPhonetic}{叮嘱}{ding1zhu3}{5,15}{⼝,⼝}[HSK 7-9]
  \definition{v.}{advertir; exortar; insistir repetidamente; instruir repetidamente; dizer à outra pessoa para lembrar o que deve e o que não deve ser feito}
\end{EntryWithPhonetic}

%%%%%%%%%% 盯 %%%%%%%%%%
\subsection*{盯}\addcontentsline{loh}{figure}{盯 \dpy{ding1}}

\begin{EntryWithPhonetic}{盯}{ding1}{7}{⽬}[HSK 7-9]
  \definition{v.}{olhar; fixar os olhos em; olhar fixamente para; encarar; concentrar seu olhar em um ponto}
\end{EntryWithPhonetic}

%%%%%%%%%% 钉 %%%%%%%%%%
\subsection*{钉}\addcontentsline{loh}{figure}{钉 \dpy{ding1}}

\begin{EntryWithPhonetic}{钉}{ding1}{7}{⾦}
  \definition[根,颗,枚]{s.}{prego; tacha}
  \definition{v.}{seguir de perto | instar; pressionar | olhar fixamente; fixar os olhos em; o mesmo que 盯}
  \seeref{ding4}
  \seealsoref{盯}{ding1}
\end{EntryWithPhonetic}

\begin{EntryWithPhonetic}{钉子}{ding1zi5}{7,3}{⾦,⼦}[HSK 7-9]
  \definition[颗,根,枚]{s.}{prego; prego de ferro | obstáculo; dificuldade; inconveniente | sabotador}
  \definition{v.}{bater em um obstáculo; encontrar um obstáculo; encontrar uma rejeição (de); correr contra um obstáculo}
\end{EntryWithPhonetic}

%%%%%%%%%% 顶 %%%%%%%%%%
\subsection*{顶}\addcontentsline{loh}{figure}{顶 \dpy{ding3}}

\begin{EntryWithPhonetic}{顶}{ding3}{8}{⾴}[HSK 4]
  \definition{adv.}{muito (linguagem falada); a maioria; extremamente; expressa o grau mais alto, equivalente a 最 e 极}
  \definition{clas.}{usado para coisas que têm um topo}
  \definition{prep.}{até}
  \definition{s.}{coroa da cabeça; parte mais alta do corpo ou objeto | topo; limite superior; ponto mais alto}
  \definition{v.}{carregar na cabeça; carregar em sua cabeça | empurrar (ou apoiar) para cima; empurrar por baixo (ou por trás) | dar cabeçadas; dar uma coronhada | sustentar; apoiar; suportar | resistir; ir contra; enfrentar | rebater; retorquir; responder de volta | cooperar; enfrentar; apoiar; dar suporte | igualar; ser equivalente a | substituir; tomar o lugar de | assumir o controle; transferir ou adquirir o direito de administrar um negócio ou alugar uma casa ou terreno}
  \seealsoref{极}{ji2}
  \seealsoref{最}{zui4}
\end{EntryWithPhonetic}

\begin{EntryWithPhonetic}{顶多}{ding3duo1}{8,6}{⾴,⼣}[HSK 7-9]
  \definition{adv.}{na melhor das hipóteses; no máximo, na opinião do orador, o número real não será maior que o maior número estimado}
\end{EntryWithPhonetic}

\begin{EntryWithPhonetic}{顶级}{ding3ji2}{8,6}{⾴,⽷}[HSK 7-9]
  \definition{adj.}{de primeira classe; de alta qualidade; de ponta}
\end{EntryWithPhonetic}

\begin{EntryWithPhonetic}{顶尖}{ding3jian1}{8,6}{⾴,⼩}[HSK 7-9]
  \definition{adj.}{melhor; de primeira classe; de mais alto nível}
  \definition{s.}{centro; ápice | topo pontiagudo; ponta; pico; a parte mais alta e pontiaguda}
\end{EntryWithPhonetic}

%%%%%%%%%% 鼎 %%%%%%%%%%
\subsection*{鼎}\addcontentsline{loh}{figure}{鼎 \dpy{ding3}}

\begin{EntryWithPhonetic}{鼎}{ding3}{12}{⿍}[Kangxi 206]
  \definition{adj.}{grande; generoso | importante; grandioso}
  \definition{adv.}{exatamente quando; exatamente o momento para}
  \definition[尊]{s.}{um antigo recipiente de cozinha com duas alças e três ou quatro pernas | pote; caldeirão | poder do estado; o trono | como símbolo de dinastia; nos tempos antigos, era considerada uma ferramenta importante para estabelecer um país}
\end{EntryWithPhonetic}

%%%%%%%%%% 订 %%%%%%%%%%
\subsection*{订}\addcontentsline{loh}{figure}{订 \dpy{ding4}}

\begin{EntryWithPhonetic}{订}{ding4}{4}{⾔}[HSK 3]
  \definition{v.}{concluir; elaborar; concordar com |assinar (um jornal, etc.); reservar (assentos, ingressos, etc.); encomendar (mercadorias, etc.) | fazer correções; revisar | grampear junto; unir; usar linha ou arame para encadernar páginas soltas ou folhas de papel | julgar; determinar}
\end{EntryWithPhonetic}

\begin{EntryWithPhonetic}{订单}{ding4dan1}{4,8}{⾔,⼗}[HSK 7-9]
  \definition{s.}{pedido de mercadorias; formulário de pedido; contratos e documentos para encomenda de mercadorias}
\end{EntryWithPhonetic}

\begin{EntryWithPhonetic}{订购}{ding4gou4}{4,8}{⾔,⾙}[HSK 7-9]
  \definition{v.}{encomendar (mercadorias); enviar para; fazer um pedido de algo | concordar em comprar (bens, etc.)}
\end{EntryWithPhonetic}

\begin{EntryWithPhonetic}{订婚}{ding4/hun1}{4,11}{⾔,⼥}[HSK 7-9]
  \definition{v.+compl.}{estar noivo(a); estar prometido(a) a}
\end{EntryWithPhonetic}

\begin{EntryWithPhonetic}{订金}{ding4jin1}{4,8}{⾔,⾦}
  \definition{s.}{depósito; sinal; entrada; o pagamento antecipado de parte do pagamento das mercadorias tem um certo significado de compromisso, mas não garante legalmente a execução do contrato}
\end{EntryWithPhonetic}

\begin{EntryWithPhonetic}{订立}{ding4li4}{4,5}{⾔,⽴}[HSK 7-9]
  \definition{v.}{concluir (um tratado, acordo, etc.); fazer (um contrato, etc.)}
\end{EntryWithPhonetic}

%%%%%%%%%% 钉 %%%%%%%%%%
\subsection*{钉}\addcontentsline{loh}{figure}{钉 \dpy{ding4}}

\begin{EntryWithPhonetic}{钉}{ding4}{7}{⾦}[HSK 7-9]
  \definition{v.}{pregar; prender com pregos | costurar; costurar com agulha e linha}
  \seeref{ding1}
\end{EntryWithPhonetic}

%%%%%%%%%% 定 %%%%%%%%%%
\subsection*{定}\addcontentsline{loh}{figure}{定 \dpy{ding4}}

\begin{EntryWithPhonetic}{定}{ding4}{8}{⼧}[HSK 4]
  \definition{adj.}{calmo; estável}
  \definition{adv.}{certamente; com certeza; definitivamente; espressa certeza ou necessidade}
  \definition{v.}{decidir; fixar; definir; determinar; ter certeza | acalmar; estabilizar; tornar estável | assinar (um jornal, etc.); reservar (assentos, ingressos, etc.); encomendar (mercadorias, etc.)}
\end{EntryWithPhonetic}

\begin{EntryWithPhonetic}{定价}{ding4jia4}{8,6}{⼧,⼈}[HSK 6]
  \definition{s.}{fixação de preços; preço especificado}
  \definition{v.}{fixar um preço | fazer um preço; definir um preço}
\end{EntryWithPhonetic}

\begin{EntryWithPhonetic}{定金}{ding4jin1}{8,8}{⼧,⾦}[HSK 7-9]
  \definition{s.}{sinal; depósito; o mesmo que 订金}
  \seealsoref{订金}{ding4jin1}
\end{EntryWithPhonetic}

\begin{EntryWithPhonetic}{定居}{ding4/ju1}{8,8}{⼧,⼫}[HSK 7-9]
  \definition{v.+compl.}{estabelecer-se; fixar residência; viver permanentemente em um determinado lugar}
\end{EntryWithPhonetic}

\begin{EntryWithPhonetic}{定论}{ding4lun4}{8,6}{⼧,⾔}[HSK 7-9]
  \definition{s.}{conclusão final | veredito final}
  \definition{v.}{concluir; chegar a uma conclusão (ou julgamento)}
\end{EntryWithPhonetic}

\begin{EntryWithPhonetic}{定期}{ding4qi1}{8,12}{⼧,⽉}[HSK 3]
  \definition{adj.}{regular; periódico; em intervalos regulares; com prazo determinado; por tempo limitado}
  \definition{v.}{fixar (definir) uma data; determinar a data; confirmar a data}
\end{EntryWithPhonetic}

\begin{EntryWithPhonetic}{定时}{ding4shi2}{8,7}{⼧,⽇}[HSK 6]
  \definition{s.}{em um horário fixo; em intervalos regulares}
  \definition{v.}{cronometrar; fixar um tempo (para fazer algo)}
\end{EntryWithPhonetic}

\begin{EntryWithPhonetic}{定为}{ding4wei2}{8,4}{⼧,⼂}[HSK 7-9]
  \definition{v.}{prescrever como; estar marcado para}
\end{EntryWithPhonetic}

\begin{EntryWithPhonetic}{定位}{ding4wei4}{8,7}{⼧,⼈}[HSK 6]
  \definition{s.}{posição; localização; posição medida ou definida}
  \definition{v.}{localizar; posicionar; orientar; avaliar algo; usar instrumentos para determinar a localização de objetos; definir o \emph{status} das coisas}
\end{EntryWithPhonetic}

\begin{EntryWithPhonetic}{定向}{ding4xiang4}{8,6}{⼧,⼝}[HSK 7-9]
  \definition{adj.}{direcional; orientado}
  \definition{s.}{orientação; sentido de orientação; orientação predeterminada}
  \definition{v.}{orientar}
\end{EntryWithPhonetic}

\begin{EntryWithPhonetic}{定心丸}{ding4xin1wan2}{8,4,3}{⼧,⼼,⼂}[HSK 7-9]
  \definition{s.}{algo que tranquiliza a mente; alívio | algo capaz de tranquilizar a mente de alguém; algo que acalma os nervos; tranquiliza a mente (paz); palavras ou ações que podem acalmar pensamentos e emoções}
\end{EntryWithPhonetic}

\begin{EntryWithPhonetic}{定义}{ding4yi4}{8,3}{⼧,⼂}[HSK 7-9]
  \definition[个,种]{s.}{definição; delimitação; uma descrição precisa e concisa das características essenciais de uma coisa ou da conotação e extensão de um conceito}
  \definition{v.}{definir}
\end{EntryWithPhonetic}

\begin{EntryWithPhonetic}{定做}{ding4zuo4}{8,11}{⼧,⼈}[HSK 7-9]
  \definition{v.}{ter algo feito sob encomenda (medida); feito sob medida; personalizar}
\end{EntryWithPhonetic}

%%%%%%%%%% 丢 %%%%%%%%%%
\subsection*{丢}\addcontentsline{loh}{figure}{丢 \dpy{diu1}}

\begin{EntryWithPhonetic}{丢}{diu1}{6}{⼛}[HSK 5]
  \definition{v.}{perder; extraviar; estar ausente | lançar; atirar | colocar (deixar) de lado | deixar (para trás)}
\end{EntryWithPhonetic}

\begin{EntryWithPhonetic}{丢掉}{diu1diao4}{6,11}{⼛,⼿}[HSK 7-9]
  \definition{v.}{perder | descartar; abandonar; jogar fora; lançar fora}
\end{EntryWithPhonetic}

\begin{EntryWithPhonetic}{丢官}{diu1guan1}{6,8}{⼛,⼧}
  \definition{v.}{(um funcionário) perder o emprego; ser demitido}
\end{EntryWithPhonetic}

\begin{EntryWithPhonetic}{丢开}{diu1kai1}{6,4}{⼛,⼶}
  \definition{v.}{jogar fora ou deixar de lado | esquecer por um tempo}
\end{EntryWithPhonetic}

\begin{EntryWithPhonetic}{丢脸}{diu1/lian3}{6,11}{⼛,⾁}[HSK 7-9]
  \definition{v.+compl.}{perder a dignidade; ser desonroso; estar envergonhado}
\end{EntryWithPhonetic}

\begin{EntryWithPhonetic}{丢弃}{diu1qi4}{6,7}{⼛,⼶}[HSK 7-9]
  \definition{v.}{abandonar; descartar; livrar-se de; jogar fora}
\end{EntryWithPhonetic}

\begin{EntryWithPhonetic}{丢人}{diu1/ren2}{6,2}{⼛,⼈}[HSK 7-9]
  \definition{v.+compl.}{ser desonrado; ser vergonhoso}
\end{EntryWithPhonetic}

\begin{EntryWithPhonetic}{丢失}{diu1shi1}{6,5}{⼛,⼤}[HSK 7-9]
  \definition{v.}{perder}
\end{EntryWithPhonetic}

\begin{EntryWithPhonetic}{丢下}{diu1xia4}{6,3}{⼛,⼀}
  \definition{v.}{abandonar}
\end{EntryWithPhonetic}

%%%%%%%%%% 东 %%%%%%%%%%
\subsection*{东}\addcontentsline{loh}{figure}{东 \dpy{dong1}}

\begin{EntryWithPhonetic}{东}{dong1}{5}{⼀}[HSK 1]
  \definition*{s.}{Sobrenome: Dong}
  \definition{s.}{leste; uma das quatro direções básicas, o lado onde o sol nasce | proprietário; dono | anfitrião (antigamente, o anfitrião ficava a leste e os convidados a oeste)}
\end{EntryWithPhonetic}

\begin{EntryWithPhonetic}{东半球}{dong1ban4qiu2}{5,5,11}{⼀,⼗,⽟}
  \definition*{s.}{Hemisfério Oriental}
\end{EntryWithPhonetic}

\begin{EntryWithPhonetic}{东北}{dong1bei3}{5,5}{⼀,⼔}[HSK 2]
  \definition*{s.}{Nordeste da China; O Nordeste | Manchúria}
  \definition{s.}{nordeste}
\end{EntryWithPhonetic}

\begin{EntryWithPhonetic}{东奔西走}{dong1ben1-xi1zou3}{5,8,6,7}{⼀,⼤,⾑,⾛}[HSK 7-9]
  \definition{expr.}{correr de um lado para o outro; correr atarefadamente; ir em todas as direções em busca de algo; correr ou se apressar aqui e ali (procurando emprego, ganhando a vida, tentando a sorte em algo ou outro, etc.)}
\end{EntryWithPhonetic}

\begin{EntryWithPhonetic}{东边}{dong1bian1}{5,5}{⼀,⾡}[HSK 1]
  \definition{s.}{leste; o lado leste; refere"-se à fronteira oriental}
\end{EntryWithPhonetic}

\begin{EntryWithPhonetic}{东部}{dong1bu4}{5,10}{⼀,⾢}[HSK 3]
  \definition{s.}{o leste; parte oriental; a parte oriental de uma determinada região}
\end{EntryWithPhonetic}

\begin{EntryWithPhonetic}{东道主}{dong1dao4zhu3}{5,12,5}{⼀,⾡,⼂}[HSK 7-9]
  \definition[个,些,位]{s.}{anfitrião; aquele que paga por uma refeição; o anfitrião do banquete}
\end{EntryWithPhonetic}

\begin{EntryWithPhonetic}{东方}{dong1fang1}{5,4}{⼀,⽅}[HSK 2]
  \definition*{s.}{Sobrenome: Dongfang}
  \definition{s.}{leste | oriente; o leste; o Oriente}
\end{EntryWithPhonetic}

\begin{EntryWithPhonetic}{东方学院}{dong1fang1 xue2yuan4}{5,4,8,9}{⼀,⽅,⼦,⾩}
  \definition*{s.}{Instituto Oriental}
\end{EntryWithPhonetic}

\begin{EntryWithPhonetic}{东面}{dong1mian4}{5,9}{⼀,⾯}
  \definition{s.}{lado leste (de algo)}
\end{EntryWithPhonetic}

\begin{EntryWithPhonetic}{东南}{dong1nan2}{5,9}{⼀,⼗}[HSK 2]
  \definition*{s.}{Sudeste da China; O Sudeste; refere"-se à região costeira sudeste da China, incluindo as províncias e cidades de Xangai, Jiangsu, Zhejiang, Fujian, Taiwan, etc.}
  \definition{s.}{sudeste}
\end{EntryWithPhonetic}

\begin{EntryWithPhonetic}{东西}{dong1xi1}{5,6}{⼀,⾑}
  \definition{s.}{leste e oeste | de leste a oeste; a distância de um lugar de leste a oeste}
  \seeref{dong1xi5}
\end{EntryWithPhonetic}

\begin{EntryWithPhonetic}{东西}{dong1xi5}{5,6}{⼀,⾑}[HSK 1]
  \definition[个,件]{s.}{coisa; refere"-se a todos os tipos de coisas concretas ou abstratas | coisa; criatura; refere"-se especificamente a pessoas ou coisas que causam repulsa ou simpatia}
  \seeref{dong1xi1}
\end{EntryWithPhonetic}

\begin{EntryWithPhonetic}{东张西望}{dong1zhang1-xi1wang4}{5,7,6,11}{⼀,⼸,⾑,⽉}[HSK 7-9]
  \definition{expr.}{olhar (ou espreitar) ao redor; olhar para um lado e para o outro; olhar em todas as direções | olhar (olhar) para um lado e para o outro; olhar (espreitar) ao redor; olhar para a direita e para a esquerda; olhar para todos os lados; olhar para leste e oeste; olhar em todas as direções | olhar (espiar) ao redor}
\end{EntryWithPhonetic}

%%%%%%%%%% 冬 %%%%%%%%%%
\subsection*{冬}\addcontentsline{loh}{figure}{冬 \dpy{dong1}}

\begin{EntryWithPhonetic}{冬}{dong1}{5}{⼎}
  \definition*{s.}{Sobrenome: Dong}
  \definition{s.}{inverno}
  \definition{s.}{onomatopéia: som de um tambor batendo, batendo na porta, etc.}
\end{EntryWithPhonetic}

\begin{EntryWithPhonetic}{冬瓜}{dong1gua1}{5,5}{⼎,⽠}
  \definition{s.}{melão de inverno}
\end{EntryWithPhonetic}

\begin{EntryWithPhonetic}{冬季}{dong1ji4}{5,8}{⼎,⼦}[HSK 4]
  \definition[个,次,种]{s.}{inverno; o quarto trimestre do ano, habitualmente referido na China como o período de três meses entre o início do inverno e o início da primavera, e também referido como ``décimo, décimo primeiro e décimo segundo'' meses do calendário lunar}
\end{EntryWithPhonetic}

\begin{EntryWithPhonetic}{冬天}{dong1tian1}{5,4}{⼎,⼤}[HSK 2]
  \definition[个]{s.}{inverno; a quarta estação do ano, na China, geralmente se refere aos três meses entre outubro e dezembro do calendário lunar}
\end{EntryWithPhonetic}

%%%%%%%%%% 董 %%%%%%%%%%
\subsection*{董}\addcontentsline{loh}{figure}{董 \dpy{dong3}}

\begin{EntryWithPhonetic}{董}{dong3}{12}{⾋}
  \definition*{s.}{Sobrenome: Dong}
  \definition{s.}{diretor; administrador}
  \definition{v.}{Literário: dirigir; supervisionar; supervisionar}
\end{EntryWithPhonetic}

\begin{EntryWithPhonetic}{董事}{dong3shi4}{12,8}{⾋,⼅}[HSK 7-9]
  \definition[个,位,名]{s.}{administrador; diretor}
\end{EntryWithPhonetic}

\begin{EntryWithPhonetic}{董事会}{dong3shi4hui4}{12,8,6}{⾋,⼅,⼈}[HSK 7-9]
  \definition[届]{s.}{conselho de administração (numa empresa); conselho de curadores (numa instituição de ensino); órgão decisório de uma sociedade anônima, escola ou grupo}
\end{EntryWithPhonetic}

\begin{EntryWithPhonetic}{董事长}{dong3shi4zhang3}{12,8,4}{⾋,⼅,⾧}[HSK 7-9]
  \definition[个,位,名]{s.}{presidente; presidente do conselho; \emph{chairman}; o principal responsável pelo conselho de administração}
\end{EntryWithPhonetic}

%%%%%%%%%% 懂 %%%%%%%%%%
\subsection*{懂}\addcontentsline{loh}{figure}{懂 \dpy{dong3}}

\begin{EntryWithPhonetic}{懂}{dong3}{15}{⼼}[HSK 2]
  \definition*{s.}{Sobrenome: Dong}
  \definition{v.}{compreender; entender}
\end{EntryWithPhonetic}

\begin{EntryWithPhonetic}{懂得}{dong3de5}{15,11}{⼼,⼻}[HSK 2]
  \definition{v.}{saber (significado, prática, etc.); compreender; entender}
\end{EntryWithPhonetic}

\begin{EntryWithPhonetic}{懂事}{dong3shi4}{15,8}{⼼,⼅}[HSK 7-9]
  \definition{adj.}{sensato; inteligente; muito compreensivo da natureza e da razão humana}
\end{EntryWithPhonetic}

%%%%%%%%%% 动 %%%%%%%%%%
\subsection*{动}\addcontentsline{loh}{figure}{动 \dpy{dong4}}

\begin{EntryWithPhonetic}{动}{dong4}{6}{⼒}[HSK 1]
  \definition{adj.}{não estacionário; móvel; variável; mutável}
  \definition{adv.}{facilmente; frequentemente}
  \definition{s.}{ação; movimento}
  \definition{v.}{mover; mexer; (pessoas ou coisas) mudar a posição ou o estado original | agir; começar a agir; entrar em ação | alterar; mudar; alterar a posição ou o estado original | usar; utilizar; tornar ativo | despertar; tocar (o coração de alguém); provocar mudanças emocionais, reações | [geralmente na forma negativa] comer ou beber | emocionar; deixar emocionado}
\end{EntryWithPhonetic}

\begin{EntryWithPhonetic}{动不动}{dong4bu5dong4}{6,4,6}{⼒,⼀,⼒}[HSK 7-9]
  \definition{adv.}{facilmente; frequentemente; a cada passo; indica que uma determinada ação ou situação (geralmente algo que você não quer que aconteça) provavelmente ocorrerá, geralmente usado com 就}
  \seealsoref{就}{jiu4}
\end{EntryWithPhonetic}

\begin{EntryWithPhonetic}{动荡}{dong4dang4}{6,9}{⼒,⾋}[HSK 7-9]
  \definition{adj.}{(situação política, vida, etc.) caótico; instável; turbulento; metaforicamente falando, uma situação ou condição instável}
  \definition{v.}{ser turbulento; ser instável}
\end{EntryWithPhonetic}

\begin{EntryWithPhonetic}{动感}{dong4gan3}{6,13}{⼒,⼼}[HSK 7-9]
  \definition{adj.}{realista | vívido}
  \definition{s.}{senso de movimento (geralmente em uma obra de arte estática)}
\end{EntryWithPhonetic}

\begin{EntryWithPhonetic}{动工}{dong4/gong1}{6,3}{⼒,⼯}[HSK 7-9]
  \definition{v.+compl.}{começar a construção; começar a construir | construir; estar em construção | iniciar (um projeto de construção)}
\end{EntryWithPhonetic}

\begin{EntryWithPhonetic}{动画}{dong4hua4}{6,8}{⼒,⽥}[HSK 6]
  \definition[部]{s.}{desenho animado; animação; a imagem em movimento formada pela fotografia contínua das imagens desenhadas}
\end{EntryWithPhonetic}

\begin{EntryWithPhonetic}{动画片}{dong4hua4pian4}{6,8,4}{⼒,⽥,⽚}[HSK 4]
  \definition[部,集,个]{s.}{desenho animado; animações; filme de animação}
\end{EntryWithPhonetic}

\begin{EntryWithPhonetic}{动机}{dong4ji1}{6,6}{⼒,⽊}[HSK 5]
  \definition[个]{s.}{motivo; razão; intenção; ideias que motivam as pessoas a se envolverem em determinados comportamentos}
\end{EntryWithPhonetic}

\begin{EntryWithPhonetic}{动静}{dong4jing5}{6,14}{⼒,⾭}[HSK 7-9]
  \definition{s.}{o som de algo se mexendo; ações ou sons da fala | atividade; movimento; (indagar ou explorar) a situação}
\end{EntryWithPhonetic}

\begin{EntryWithPhonetic}{动力}{dong4li4}{6,2}{⼒,⼒}[HSK 3]
  \definition[种,个]{s.}{poder; a força que faz com que as máquinas funcionem, por exemplo, energia elétrica, eólica, hidráulica, etc. | ímpeto; força motriz (ou propulsora); refere"-se, de maneira geral, à força que impulsiona o desenvolvimento das coisas}
\end{EntryWithPhonetic}

\begin{EntryWithPhonetic}{动脉}{dong4mai4}{6,9}{⼒,⾁}[HSK 7-9]
  \definition[条]{s.}{Anatomia: artéria | Figurativo: estrada principal, linha ferroviária, rio, etc.}
\end{EntryWithPhonetic}

\begin{EntryWithPhonetic}{动漫}{dong4man4}{6,14}{⼒,⽔}
  \definition{s.}{desenhos animados; quadrinhos; anime; mangás; refere"-se a filmes de animação e histórias em quadrinhos}
  \seealsoref{漫画}{man4hua4}
\end{EntryWithPhonetic}

\begin{EntryWithPhonetic}{动人}{dong4ren2}{6,2}{⼒,⼈}[HSK 3]
  \definition{adj.}{comovente; emocionante; tocante}
\end{EntryWithPhonetic}

\begin{EntryWithPhonetic}{动身}{dong4/shen1}{6,7}{⼒,⾝}[HSK 7-9]
  \definition{v.+compl.}{partir em uma jornada; partir (para um lugar distante); fazer uma viagem; começar uma viagem; partir; partir para outro lugar; começar uma jornada}
\end{EntryWithPhonetic}

\begin{EntryWithPhonetic}{动手}{dong4/shou3}{6,4}{⼒,⼿}[HSK 5]
  \definition{v.+compl.}{iniciar o trabalho; começar a trabalhar | tocar; manusear; manipular | bater; levantar a mão (para bater); espancar}
\end{EntryWithPhonetic}

\begin{EntryWithPhonetic}{动态}{dong4tai4}{6,8}{⼒,⼼}[HSK 5]
  \definition{s.}{tendências; desenvolvimentos; tendência geral dos assuntos; causa provável de ação; curso dos acontecimentos | expressão; comportamento ativo | estado dinâmico; condição dinâmica; de ou em relação a um estado de movimento}
\end{EntryWithPhonetic}

\begin{EntryWithPhonetic}{动弹}{dong4tan5}{6,11}{⼒,⼸}[HSK 7-9]
  \definition{v.}{mexer; mover (pessoas, animais ou coisas que se movem)}
\end{EntryWithPhonetic}

\begin{EntryWithPhonetic}{动听}{dong4ting1}{6,7}{⼒,⼝}[HSK 7-9]
  \definition{adj.}{interessante de ouvir; agradável de ouvir}
\end{EntryWithPhonetic}

\begin{EntryWithPhonetic}{动物}{dong4wu4}{6,8}{⼒,⽜}[HSK 2]
  \definition[个,只,群,种]{s.}{animal; uma grande classe de seres vivos, que se alimentam principalmente de matéria orgânica, possuem sistema nervoso, são sensíveis e capazes de se mover; refere"-se a todos os tipos de coisas concretas ou abstratas}
\end{EntryWithPhonetic}

\begin{EntryWithPhonetic}{动物园}{dong4wu4yuan2}{6,8,7}{⼒,⽜,⼞}[HSK 2]
  \definition[个,座,家]{s.}{jardim zoológico; zoo; parque que cria muitos tipos de animais (especialmente animais com valor científico ou raros na região) para exibição ao público}
\end{EntryWithPhonetic}

\begin{EntryWithPhonetic}{动向}{dong4xiang4}{6,6}{⼒,⼝}[HSK 7-9]
  \definition{s.}{tendência; direção de movimento; atividades ou direções de desenvolvimento}
\end{EntryWithPhonetic}

\begin{EntryWithPhonetic}{动摇}{dong4yao2}{6,13}{⼒,⼿}[HSK 4]
  \definition{adj.}{instável}
  \definition{v.}{ondular; pairar; agitar; balançar; sacudir | hesitar; vacilar; esmorecer; abalar}
\end{EntryWithPhonetic}

\begin{EntryWithPhonetic}{动用}{dong4yong4}{6,5}{⼒,⽤}[HSK 7-9]
  \definition{v.}{empregar; recorrer a; colocar em uso; usar (pessoal, dinheiro, etc. que se destinam a uso específico ou que não se destinam a ser usados casualmente)}
\end{EntryWithPhonetic}

\begin{EntryWithPhonetic}{动员}{dong4yuan2}{6,7}{⼒,⼝}[HSK 5]
  \definition{v.}{despertar; mobilizar; iniciar (para fazer algo ou participar de uma atividade) | mobilizar toda a nação; transferir dos setores militar, político e econômico para uma situação de guerra}
\end{EntryWithPhonetic}

\begin{EntryWithPhonetic}{动作}{dong4zuo4}{6,7}{⼒,⼈}[HSK 1]
  \definition[个]{s.}{movimento; ação; atividade de todo o corpo ou parte do corpo}
  \definition{v.}{agir; começar a se mover; entrar em ação}
\end{EntryWithPhonetic}

%%%%%%%%%% 冻 %%%%%%%%%%
\subsection*{冻}\addcontentsline{loh}{figure}{冻 \dpy{dong4}}

\begin{EntryWithPhonetic}{冻}{dong4}{7}{⼎}[HSK 5]
  \definition*{s.}{Sobrenome: Dong}
  \definition{s.}{geleia; gelatina}
  \definition{v.}{congelar; ser congelado | ficar com frio ou sentir frio}
\end{EntryWithPhonetic}

\begin{EntryWithPhonetic}{冻结}{dong4jie2}{7,9}{⼎,⽷}[HSK 7-9]
  \definition{adj.}{congelado}
  \definition{s.}{(salários, preços, etc.) congelamento; bloqueio de fluxo ou mudança}
  \definition{v.}{congelar}
\end{EntryWithPhonetic}

%%%%%%%%%% 栋 %%%%%%%%%%
\subsection*{栋}\addcontentsline{loh}{figure}{栋 \dpy{dong4}}

\begin{EntryWithPhonetic}{栋}{dong4}{9}{⽊}[HSK 7-9]
  \definition*{s.}{Sobrenome: Dong}
  \definition{clas.}{edifício; prédio}
  \definition{s.}{Literário: cumeeira; viga principal}
\end{EntryWithPhonetic}

\begin{EntryWithPhonetic}{栋梁}{dong4liang2}{9,11}{⽊,⽊}[HSK 7-9]
  \definition{s.}{cumeeira e vigas; esteio (de organização); viga de cumeeira; cumeeira; placa de cumeeira; tábua de cumeeira | pessoa capaz de suportar grandes responsabilidades | pilar (do estado)}
\end{EntryWithPhonetic}

%%%%%%%%%% 洞 %%%%%%%%%%
\subsection*{洞}\addcontentsline{loh}{figure}{洞 \dpy{dong4}}

\begin{EntryWithPhonetic}{洞}{dong4}{9}{⽔}[HSK 5]
  \definition{adj.}{profundo; minucioso; claro; completo; abrangente}
  \definition{s.}{buraco; cavidade; orifício; furo; parte penetrante ou profundamente recuada de um objeto; uma caverna}
\end{EntryWithPhonetic}

\begin{EntryWithPhonetic}{洞穴}{dong4xue2}{9,5}{⽔,⽳}
  \definition{s.}{caverna}
\end{EntryWithPhonetic}

%%%%%%%%%% 都 %%%%%%%%%%
\subsection*{都}\addcontentsline{loh}{figure}{都 \dpy{dou1}}

\begin{EntryWithPhonetic}{都}{dou1}{10}{⾢}[HSK 1]
  \definition{adv.}{todos; representa a soma total | apenas por causa de; usado em conjunto com a palavra 是, explica o motivo | mesmo; até; indicativo de ênfase | já; significa 已经}
  \seeref{du1}
  \seealsoref{是}{shi4}
  \seealsoref{已经}{yi3jing1}
\end{EntryWithPhonetic}

%%%%%%%%%% 兜 %%%%%%%%%%
\subsection*{兜}\addcontentsline{loh}{figure}{兜 \dpy{dou1}}

\begin{EntryWithPhonetic}{兜}{dou1}{11}{⼉}[HSK 7-9]
  \definition{s.}{bolsa; bolso; coisas tipo bolso}
  \definition{v.}{embrulhar em um pedaço de pano, etc.; fazer um formato de bolso para guardar coisas | mover-se; dar uma volta; fazer um desvio | solicitar; sondar; recrutamentar | assumir a responsabilidade por algo; assumir o controle de}
\end{EntryWithPhonetic}

\begin{EntryWithPhonetic}{兜儿}{dou1r5}{11,2}{⼉,⼉}[HSK 7-9]
  \definition{s.}{bolso}
\end{EntryWithPhonetic}

\begin{EntryWithPhonetic}{兜售}{dou1shou4}{11,11}{⼉,⼝}[HSK 7-9]
  \definition{v.}{vender; apregoar | vender; fazer uma venda de}
\end{EntryWithPhonetic}

%%%%%%%%%% 斗 %%%%%%%%%%
\subsection*{斗}\addcontentsline{loh}{figure}{斗 \dpy{dou3}}

\begin{EntryWithPhonetic}{斗}{dou3}{4}{⽃}[Kangxi 68]
  \definition*{s.}{Ursa Maior; Abreviação de Estrela Polar | Dou, uma das mansões lunares; uma das 28 constelações | Sobrenome: Dou}
  \definition{clas.}{dou, unidade de medida seca para grãos (=1 decalitro)}
  \definition{s.}{dou, medida para grãos; instrumento para medir grãos | um objeto com a forma de um copo ou concha; algo que se parece com um balde | espiral (de uma impressão digital)}
  \seeref{dou4}
\end{EntryWithPhonetic}

%%%%%%%%%% 抖 %%%%%%%%%%
\subsection*{抖}\addcontentsline{loh}{figure}{抖 \dpy{dou3}}

\begin{EntryWithPhonetic}{抖}{dou3}{7}{⼿}[HSK 7-9]
  \definition{v.}{tremer; estremecer; arrepiar | sacudir | despertar; agitar | Coloquial geralmente sarcástico: (usualmente com 起来) progredir no mundo; ganhar fama (ou fortuna)}
  \seealsoref{起来}{qi3lai5}
\end{EntryWithPhonetic}

%%%%%%%%%% 陡 %%%%%%%%%%
\subsection*{陡}\addcontentsline{loh}{figure}{陡 \dpy{dou3}}

\begin{EntryWithPhonetic}{陡}{dou3}{9}{⾩}[HSK 7-9]
  \definition{adj.}{íngreme; precipitado}
  \definition{adv.}{Literário: abruptamente; de repente; inesperadamente; significa que a ação ou situação acontece de forma rápida e inesperada, o que equivale a 突然}
  \seealsoref{突然}{tu1ran2}
\end{EntryWithPhonetic}

%%%%%%%%%% 斗 %%%%%%%%%%
\subsection*{斗}\addcontentsline{loh}{figure}{斗 \dpy{dou4}}

\begin{EntryWithPhonetic}{斗}{dou4}{4}{⽃}[HSK 7-9][Kangxi 68]
  \definition{adj.}{Dialeto: engraçado}
  \definition{v.}{brigar; lutar | lutar contra; denunciar | competir com; disputar com | fazer animais ou insetos lutarem (como um jogo) | encaixar-se | brincar com; provocar | provocar (risos, etc.); divertir | ser remendado; juntar; unir; combinar}
  \seeref{dou3}
\end{EntryWithPhonetic}

\begin{EntryWithPhonetic}{斗争}{dou4zheng1}{4,6}{⽃,⼑}[HSK 6]
  \definition[场,番]{s.}{luta; conflito; batalha; os dois lados entram em conflito entre si}
  \definition{v.}{lutar; combater; batalhar; os dois lados estão em conflito, lutando para derrotar um ao outro | esforçar-se por; lutar por; trabalhar duro por; expor e criticar; atacar | lutar; batalhar}
\end{EntryWithPhonetic}

\begin{EntryWithPhonetic}{斗志}{dou4zhi4}{4,7}{⽃,⼼}[HSK 7-9]
  \definition{s.}{vontade de lutar; espírito combativo}
\end{EntryWithPhonetic}

%%%%%%%%%% 豆 %%%%%%%%%%
\subsection*{豆}\addcontentsline{loh}{figure}{豆 \dpy{dou4}}

\begin{EntryWithPhonetic}{豆}{dou4}{7}{⾖}[Kangxi 151]
  \definition*{s.}{Sobrenome: Dou}
  \definition{s.}{planta que produz vagens ou suas sementes | coisa em forma de feijão | leguminosas ou sementes de leguminosas; feijões; ervilhas | uma xícara ou tigela antiga com haste}
\end{EntryWithPhonetic}

\begin{EntryWithPhonetic}{豆腐}{dou4fu5}{7,14}{⾖,⾁}[HSK 4]
  \definition[块,盒,斤,盘]{s.}{\emph{tofu}}
\end{EntryWithPhonetic}

\begin{EntryWithPhonetic}{豆荚}{dou4jia2}{7,9}{⾖,⾋}
  \definition{s.}{vagem (de legumes)}
\end{EntryWithPhonetic}

\begin{EntryWithPhonetic}{豆浆}{dou4jiang1}{7,10}{⾖,⽔}[HSK 7-9]
  \definition[杯,碗]{s.}{leite de soja}
\end{EntryWithPhonetic}

\begin{EntryWithPhonetic}{豆角}{dou4jiao3}{7,7}{⾖,⾓}
  \definition{s.}{feijão verde}
\end{EntryWithPhonetic}

\begin{EntryWithPhonetic}{豆制品}{dou4zhi4pin3}{7,8,9}{⾖,⼑,⼝}[HSK 5]
  \definition{s.}{produtos de soja}
\end{EntryWithPhonetic}

\begin{EntryWithPhonetic}{豆子}{dou4zi5}{7,3}{⾖,⼦}[HSK 7-9]
  \definition[颗,粒,把,袋]{s.}{planta que produz vagens ou suas sementes | algo em forma de feijão | leguminosas; feijões}
\end{EntryWithPhonetic}

%%%%%%%%%% 读 %%%%%%%%%%
\subsection*{读}\addcontentsline{loh}{figure}{读 \dpy{dou4}}

\begin{EntryWithPhonetic}{读}{dou4}{10}{⾔}
  \definition{s.}{vírgula; uma breve pausa na leitura}
  \seeref{du2}
\end{EntryWithPhonetic}

%%%%%%%%%% 逗 %%%%%%%%%%
\subsection*{逗}\addcontentsline{loh}{figure}{逗 \dpy{dou4}}

\begin{EntryWithPhonetic}{逗}{dou4}{10}{⾡}[HSK 7-9]
  \definition{adj.}{engraçado; divertido}
  \definition{s.}{ligeira pausa na leitura; antigamente, referia"-se ao lugar em um artigo onde o significado de uma frase não era completado e uma pausa era necessária durante a leitura}
  \definition{v.}{provocar; brincar com | divertir; provocar (risos, etc.) | ficar; parar}
\end{EntryWithPhonetic}

%%%%%%%%%% 都 %%%%%%%%%%
\subsection*{都}\addcontentsline{loh}{figure}{都 \dpy{du1}}

\begin{EntryWithPhonetic}{都}{du1}{10}{⾢}
  \definition*{s.}{Sobrenome: Du}
  \definition[座]{s.}{capital | cidade grande; metrópole}
  \seeref{dou1}
\end{EntryWithPhonetic}

\begin{EntryWithPhonetic}{都会}{du1hui4}{10,6}{⾢,⼈}[HSK 7-9]
  \definition{s.}{cidade; metrópole}
\end{EntryWithPhonetic}

\begin{EntryWithPhonetic}{都市}{du1shi4}{10,5}{⾢,⼱}[HSK 6]
  \definition[个]{s.}{cidade grande; grandes cidades}
\end{EntryWithPhonetic}

%%%%%%%%%% 嘟 %%%%%%%%%%
\subsection*{嘟}\addcontentsline{loh}{figure}{嘟 \dpy{du1}}

\begin{EntryWithPhonetic}{嘟}{du1}{13}{⼝}
  \definition{part.}{(onomatopéia) buzina}
  \definition{v.}{fazer beicinho}
\end{EntryWithPhonetic}

%%%%%%%%%% 督 %%%%%%%%%%
\subsection*{督}\addcontentsline{loh}{figure}{督 \dpy{du1}}

\begin{EntryWithPhonetic}{督}{du1}{13}{⽬}
  \definition*{s.}{Sobrenome: Du}
  \definition{v.}{supervisionar e comandar}
\end{EntryWithPhonetic}

\begin{EntryWithPhonetic}{督促}{du1cu4}{13,9}{⽬,⼈}[HSK 7-9]
  \definition{v.}{supervisionar e incentivar}
\end{EntryWithPhonetic}

%%%%%%%%%% 毒 %%%%%%%%%%
\subsection*{毒}\addcontentsline{loh}{figure}{毒 \dpy{du2}}

\begin{EntryWithPhonetic}{毒}{du2}{9}{⽏}[HSK 5]
  \definition*{s.}{Sobrenome: Du}
  \definition{adj.}{veneno; toxina; propriedade ou substância prejudicial aos organismos vivos | droga; narcóticos | vírus; vírus de computador | influência venenosa}
  \definition{adj.}{venenoso; tóxico; envenenado | malicioso; cruel; feroz}
  \definition{v.}{matar com veneno; envenenar | envenenar (a mente de alguém)}
\end{EntryWithPhonetic}

\begin{EntryWithPhonetic}{毒害}{du2hai4}{9,10}{⽏,⼧}
  \definition{s.}{envenenamento}
  \definition{v.}{envenenar (prejudicar com uma substância tóxica) | envenenar (as mentes das pessoas)}
\end{EntryWithPhonetic}

\begin{EntryWithPhonetic}{毒品}{du2pin3}{9,9}{⽏,⼝}[HSK 6]
  \definition[种,点]{s.}{drogas; veneno; narcóticos; refere"-se ao ópio, morfina, heroína, etc. usados como vício}
\end{EntryWithPhonetic}

\begin{EntryWithPhonetic}{毒杀}{du2sha1}{9,6}{⽏,⽊}
  \definition{v.}{matar por envenenamento}
\end{EntryWithPhonetic}

\begin{EntryWithPhonetic}{毒蛇}{du2she2}{9,11}{⽏,⾍}
  \definition{s.}{víbora | cobra venenosa}
\end{EntryWithPhonetic}

\begin{EntryWithPhonetic}{毒物}{du2wu4}{9,8}{⽏,⽜}
  \definition{s.}{substância venenosa | toxina}
\end{EntryWithPhonetic}

%%%%%%%%%% 独 %%%%%%%%%%
\subsection*{独}\addcontentsline{loh}{figure}{独 \dpy{du2}}

\begin{EntryWithPhonetic}{独}{du2}{9}{⽝}[HSK 7-9]
  \definition*{s.}{Sobrenome: Du}
  \definition{adj.}{só; solteiro | (coloquial) distante | único; só}
  \definition{adv.}{unicamente; somente | sozinho; por si mesmo; em solidão}
  \definition{s.}{idosos sem descendência; os sem filhos}
\end{EntryWithPhonetic}

\begin{EntryWithPhonetic}{独唱}{du2chang4}{9,11}{⽝,⼝}[HSK 7-9]
  \definition{s.}{(em canto) solo}
\end{EntryWithPhonetic}

\begin{EntryWithPhonetic}{独家}{du2jia1}{9,10}{⽝,⼧}[HSK 7-9]
  \definition{adj.}{único; exclusivo}
\end{EntryWithPhonetic}

\begin{EntryWithPhonetic}{独立}{du2li4}{9,5}{⽝,⽴}[HSK 4]
  \definition{adj.}{independente; por conta própria | separado; respectivo; descreve algo que é separado e não está em contato com outra coisa}
  \definition{v.}{ficar sozinho | alcançar a independência; tornar-se um país independente; liberdade de um Estado, regime ou organização contra interferência, controle e dominação por forças externas}
\end{EntryWithPhonetic}

\begin{EntryWithPhonetic}{独立自主}{du2li4-zi4zhu3}{9,5,6,5}{⽝,⽴,⾃,⼂}[HSK 7-9]
  \definition{expr.}{manter a independência e manter a iniciativa; agir de forma independente e manter a iniciativa; manter-se independente; ser independente}
\end{EntryWithPhonetic}

\begin{EntryWithPhonetic}{独身}{du2shen1}{9,7}{⽝,⾝}[HSK 7-9]
  \definition{adj.}{separado da família | solteiro}
  \definition{s.}{celibato; solteirice}
\end{EntryWithPhonetic}

\begin{EntryWithPhonetic}{独特}{du2te4}{9,10}{⽝,⽜}[HSK 4]
  \definition{adj.}{único; distinto; original; especial}
\end{EntryWithPhonetic}

\begin{EntryWithPhonetic}{独一无二}{du2yi1-wu2'er4}{9,1,4,2}{⽝,⼀,⽆,⼆}[HSK 7-9]
  \definition{expr.}{único; incomparável; inigualável | o único; em uma classe própria (si mesmo); incomparável; o único de seu tipo; o original sem cópias; sem igual; não há semelhança; não há comparação}
\end{EntryWithPhonetic}

\begin{EntryWithPhonetic}{独资}{du2zi1}{9,10}{⽝,⾙}
  \definition[家]{s.}{investimento exclusivo | empresa unipessoal; propriedade integral (geralmente de uma empresa estrangeira)}
\end{EntryWithPhonetic}

\begin{EntryWithPhonetic}{独自}{du2zi4}{9,6}{⽝,⾃}[HSK 4]
  \definition{adv.}{sozinho; por si mesmo; por conta própria}
\end{EntryWithPhonetic}

%%%%%%%%%% 读 %%%%%%%%%%
\subsection*{读}\addcontentsline{loh}{figure}{读 \dpy{du2}}

\begin{EntryWithPhonetic}{读}{du2}{10}{⾔}[HSK 1]
  \definition*{s.}{Sobrenome: Du}
  \definition{v.}{ler em voz alta | ler; ler o texto e compreendera seu significado | frequentar a escola; refere"-se a ir à escola ou estudar | Computação: ler dados}
  \seeref{dou4}
\end{EntryWithPhonetic}

\begin{EntryWithPhonetic}{读书}{du2/shu1}{10,4}{⾔,⼄}[HSK 1]
  \definition{v.+compl.}{ler; estudar | frequentar a escola}
\end{EntryWithPhonetic}

\begin{EntryWithPhonetic}{读音}{du2yin1}{10,9}{⾔,⾳}[HSK 2]
  \definition[种]{s.}{pronúncia}
\end{EntryWithPhonetic}

\begin{EntryWithPhonetic}{读者}{du2zhe3}{10,8}{⾔,⽼}[HSK 3]
  \definition[个,位,名,些,群]{s.}{leitor; (para obras, autores, revistas, etc.) Pessoas que compram ou leem livros, revistas, artigos, jornais, etc.}
\end{EntryWithPhonetic}

%%%%%%%%%% 肚 %%%%%%%%%%
\subsection*{肚}\addcontentsline{loh}{figure}{肚 \dpy{du3}}

\begin{EntryWithPhonetic}{肚}{du3}{7}{⾁}
  \definition{s.}{tripas; entranhas}
  \seeref{du4}
\end{EntryWithPhonetic}

%%%%%%%%%% 堵 %%%%%%%%%%
\subsection*{堵}\addcontentsline{loh}{figure}{堵 \dpy{du3}}

\begin{EntryWithPhonetic}{堵}{du3}{11}{⼟}[HSK 4]
  \definition*{s.}{Sobrenome: Du}
  \definition{adj.}{asfixiado; abafado; sufocado; oprimido}
  \definition{clas.}{usado para paredes}
  \definition{s.}{parede}
  \definition{v.}{impedir; bloquear}
\end{EntryWithPhonetic}

\begin{EntryWithPhonetic}{堵车}{du3/che1}{11,4}{⼟,⾞}[HSK 4]
  \definition{s.}{congestionamento; tráfego intenso; ficar congestionado (no tráfego); bloqueio de vias devido ao excesso de tráfego, etc.}
  \definition{v.+compl.}{congestionar (trânsito)}
\end{EntryWithPhonetic}

\begin{EntryWithPhonetic}{堵塞}{du3se4}{11,13}{⼟,⼟}[HSK 7-9]
  \definition{v.}{parar; bloquear; tornar obstruído}
\end{EntryWithPhonetic}

%%%%%%%%%% 赌 %%%%%%%%%%
\subsection*{赌}\addcontentsline{loh}{figure}{赌 \dpy{du3}}

\begin{EntryWithPhonetic}{赌}{du3}{12}{⾙}[HSK 6]
  \definition{v.}{jogar | apostar; geralmente se refere à luta pela vitória ou derrota}
\end{EntryWithPhonetic}

\begin{EntryWithPhonetic}{赌博}{du3bo2}{12,12}{⾙,⼗}[HSK 6]
  \definition{v.}{apostar; jogar; usar jogos de cartas, rolagem de dados, etc., para apostar dinheiro}
\end{EntryWithPhonetic}

%%%%%%%%%% 妒 %%%%%%%%%%
\subsection*{妒}\addcontentsline{loh}{figure}{妒 \dpy{du4}}

\begin{EntryWithPhonetic}{妒}{du4}{7}{⼥}
  \definition{v.}{ter ciúmes (inveja) de}
\end{EntryWithPhonetic}

\begin{EntryWithPhonetic}{妒忌}{du4ji4}{7,7}{⼥,⼼}[HSK 7-9]
  \definition{v.}{invejar; ter ciúmes (ou inveja) de}
\end{EntryWithPhonetic}

%%%%%%%%%% 杜 %%%%%%%%%%
\subsection*{杜}\addcontentsline{loh}{figure}{杜 \dpy{du4}}

\begin{EntryWithPhonetic}{杜}{du4}{7}{⽊}
  \definition*{s.}{Sobrenome: Du}
  \definition{s.}{pêra de folha de bétula}
  \definition{v.}{excluir; parar; impedir; bloquear}
\end{EntryWithPhonetic}

\begin{EntryWithPhonetic}{杜鹃}{du4juan1}{7,12}{⽊,⿃}
  \definition{s.}{cuco (pássaro)}
  \seealsoref{布谷鸟}{bu4gu3niao3}
  \seealsoref{杜鹃鸟}{du4juan1niao3}
  \seealsoref{杜宇}{du4yu3}
\end{EntryWithPhonetic}

\begin{EntryWithPhonetic}{杜鹃鸟}{du4juan1niao3}{7,12,5}{⽊,⿃,⿃}
  \definition{s.}{cuco (pássaro)}
  \seealsoref{布谷鸟}{bu4gu3niao3}
  \seealsoref{杜鹃}{du4juan1}
  \seealsoref{杜宇}{du4yu3}
\end{EntryWithPhonetic}

\begin{EntryWithPhonetic}{杜绝}{du4jue2}{7,9}{⽊,⽷}[HSK 7-9]
  \definition{v.}{parar; pôr fim a; bloquear a fonte e parar (coisas ruins)}
\end{EntryWithPhonetic}

\begin{EntryWithPhonetic}{杜宇}{du4yu3}{7,6}{⽊,⼧}
  \definition{s.}{cuco (pássaro)}
  \seealsoref{布谷鸟}{bu4gu3niao3}
  \seealsoref{杜鹃}{du4juan1}
  \seealsoref{杜鹃鸟}{du4juan1niao3}
\end{EntryWithPhonetic}

%%%%%%%%%% 肚 %%%%%%%%%%
\subsection*{肚}\addcontentsline{loh}{figure}{肚 \dpy{du4}}

\begin{EntryWithPhonetic}{肚}{du4}{7}{⾁}
  \definition{s.}{barriga; abdômen; estômago | tolerância}
  \seeref{du3}
\end{EntryWithPhonetic}

\begin{EntryWithPhonetic}{肚子}{du4zi5}{7,3}{⾁,⼦}[HSK 4]
  \definition[个,只]{s.}{abdômen; barriguinha; ventre; barriga}
\end{EntryWithPhonetic}

%%%%%%%%%% 度 %%%%%%%%%%
\subsection*{度}\addcontentsline{loh}{figure}{度 \dpy{du4}}

\begin{EntryWithPhonetic}{度}{du4}{9}{⼴}[HSK 2]
  \definition*{s.}{Sobrenome: Du}
  \definition{clas.}{grau; unidade de medida para ângulos, temperatura, etc. | quilowatt"-hora (kWh) | usado para indicar a quantidade de álcool presente no vinho | usado para arcos e ângulos | usado para indicar o grau de curvatura da lente dos óculos ou o grau de miopia | tempo; número de vezes | usado para longitude e latitude, localização geográfica}
  \definition{s.}{medida linear; padrões e instrumentos para medir comprimentos | grau de intensidade; refere"-se especificamente ao grau alcançado por uma determinada propriedade de uma coisa | limite; extensão; grau; quota | regras; código de conduta; diretrizes | tolerância; magnanimidade; refere"-se especificamente ao grau de tolerância | maneira; temperamento; disposição; a personalidade ou aparência de uma pessoa | indicador de grau, nível alcançado por algo | tempo ou espaço limitado; um determinado período de tempo ou espaço}
  \definition{v.}{passar; atravessar; passar por cima | (em termos de tempo) passar; passar por | (de monges ou monjas budistas, ou sacerdotes taoístas) pregar; converter; proselitar}
  \seeref{duo2}
\end{EntryWithPhonetic}

\begin{EntryWithPhonetic}{度过}{du4guo4}{9,6}{⼴,⾡}[HSK 4]
  \definition{s.}{passar o tempo; fazer o tempo desaparecer no trabalho, na vida, no lazer e no descanso}
\end{EntryWithPhonetic}

\begin{EntryWithPhonetic}{度假}{du4jia4}{9,11}{⼴,⼈}[HSK 7-9]
  \definition{v.}{sair de férias; passar as férias}
\end{EntryWithPhonetic}

\begin{EntryWithPhonetic}{度知名度}{du4 zhi1ming2du4}{9,8,6,9}{⼴,⽮,⼝,⼴}
  \definition{s.}{popularidade}
\end{EntryWithPhonetic}

%%%%%%%%%% 渡 %%%%%%%%%%
\subsection*{渡}\addcontentsline{loh}{figure}{渡 \dpy{du4}}

\begin{EntryWithPhonetic}{渡}{du4}{12}{⽔}[HSK 6]
  \definition{s.}{(usualmente em nomes de lugares) travessia de balsa}
  \definition{v.}{atravessar (um rio, o mar, etc.) | superar; sobreviver | transportar (pessoas, mercadorias, etc.) através}
\end{EntryWithPhonetic}

\begin{EntryWithPhonetic}{渡过}{du4guo4}{12,6}{⽔,⾡}[HSK 7-9]
  \definition{v.}{viajar; atravessar; cruzar}[渡过最困难的时期。===Atravessar os momentos mais difíceis.]
\end{EntryWithPhonetic}

%%%%%%%%%% 镀 %%%%%%%%%%
\subsection*{镀}\addcontentsline{loh}{figure}{镀 \dpy{du4}}

\begin{EntryWithPhonetic}{镀}{du4}{14}{⾦}
  \definition{v.}{cobrir ou revestir (com um metal)}
\end{EntryWithPhonetic}

\begin{EntryWithPhonetic}{镀金}{du4jin1}{14,8}{⾦,⾦}
  \definition{v.}{banhar a ouro | dourar | (figurativo) fazer algo muito comum parecer especial}
\end{EntryWithPhonetic}

%%%%%%%%%% 端 %%%%%%%%%%
\subsection*{端}\addcontentsline{loh}{figure}{端 \dpy{duan1}}

\begin{EntryWithPhonetic}{端}{duan1}{14}{⽴}[HSK 6]
  \definition*{s.}{Sobrenome: Duan}
  \definition{adj.}{adequado; próprio | reto; correto}
  \definition{s.}{fim; extremidade | começo | item; ponto; pista, projeto ou aspecto | causa; razão | problema; incidente; coisas (geralmente se refere a coisas ruins, como acidentes, disputas, etc.)}
  \definition{v.}{carregar; segurar algo nivelado com ambas as mãos; segurar algo horizontalmente | erradicar; eliminar; acabar com; remover completamente; varrer | dar ares de superioridade | revelar}
\end{EntryWithPhonetic}

\begin{EntryWithPhonetic}{端午节}{duan1wu3jie2}{14,4,5}{⽴,⼗,⾋}[HSK 6]
  \definition*[个]{s.}{Festa do Duplo Cinco, Festival dos Barcos-Dragão (5º~dia do quinto mês lunar)}
\end{EntryWithPhonetic}

\begin{EntryWithPhonetic}{端正}{duan1zheng4}{14,5}{⽴,⽌}[HSK 7-9]
  \definition{adj.}{apropriado; correto; não torto ou inclinado | ereto; integridade; decência}
  \definition{v.}{corrigir; fazer o certo}
\end{EntryWithPhonetic}

%%%%%%%%%% 短 %%%%%%%%%%
\subsection*{短}\addcontentsline{loh}{figure}{短 \dpy{duan3}}

\begin{EntryWithPhonetic}{短}{duan3}{12}{⽮}[HSK 2]
  \definition{adj.}{curto; comprimento pequeno de uma extremidade à outra | curto; breve; a distância entre o ponto inicial e o ponto final de um determinado período é pequena | raso; superficial}
  \definition{s.}{falha; defeito; ponto fraco; desvantagens | tonelada curta (EUA)}
  \definition{v.}{dever; carecer}
  \antonymref{长}{zhang3}
\end{EntryWithPhonetic}

\begin{EntryWithPhonetic}{短处}{duan3chu5}{12,5}{⽮,⼡}[HSK 3]
  \definition[个]{s.}{deficiência; ponto fraco; defeito; fraqueza}
\end{EntryWithPhonetic}

\begin{EntryWithPhonetic}{短促}{duan3cu4}{12,9}{⽮,⼈}
  \definition{adj.}{curto (tom de voz) | fugaz | ofegante (respiração) | curto no tempo}
\end{EntryWithPhonetic}

\begin{EntryWithPhonetic}{短裤}{duan3ku4}{12,12}{⽮,⾐}[HSK 3]
  \definition[条]{s.}{calças curtas; calção; \emph{shorts}; calças com bainha acima do joelho}
\end{EntryWithPhonetic}

\begin{EntryWithPhonetic}{短跑}{duan3 pao3}{12,12}{⽮,⾜}
  \definition{s.}{corrida de curta distância; corrida rápida}
  \antonymref{长跑}{chang2pao3}
\end{EntryWithPhonetic}

\begin{EntryWithPhonetic}{短片}{duan3pian4}{12,4}{⽮,⽚}[HSK 6]
  \definition{s.}{curta-metragem; curtas-metragens documentais ou educativos exibidos individualmente ou em série}
\end{EntryWithPhonetic}

\begin{EntryWithPhonetic}{短期}{duan3qi1}{12,12}{⽮,⽉}[HSK 3]
  \definition{adj.}{de curta duração; de prazo curto}
  \definition[个]{s.}{curto prazo}
\end{EntryWithPhonetic}

\begin{EntryWithPhonetic}{短缺}{duan3que1}{12,10}{⽮,⽸}[HSK 7-9]
  \definition{s.}{falta; déficit; escassez; insuficiência}
\end{EntryWithPhonetic}

\begin{EntryWithPhonetic}{短少}{duan3shao3}{12,4}{⽮,⼩}
  \definition{v.}{estar aquém do valor total}
\end{EntryWithPhonetic}

\begin{EntryWithPhonetic}{短视}{duan3shi4}{12,8}{⽮,⾒}
  \definition{adj.}{míope}
\end{EntryWithPhonetic}

\begin{EntryWithPhonetic}{短信}{duan3xin4}{12,9}{⽮,⼈}[HSK 2]
  \definition[条,个,封]{s.}{mensagem de texto; refere"-se especificamente a mensagens de texto curtas, imagens, etc., enviadas ou recebidas por celular}
\end{EntryWithPhonetic}

\begin{EntryWithPhonetic}{短暂}{duan3zan4}{12,12}{⽮,⽇}[HSK 7-9]
  \definition{adj.}{breve; transitório; momentâneo; de curta duração}
\end{EntryWithPhonetic}

%%%%%%%%%% 段 %%%%%%%%%%
\subsection*{段}\addcontentsline{loh}{figure}{段 \dpy{duan4}}

\begin{EntryWithPhonetic}{段}{duan4}{9}{⽎}[HSK 2]
  \definition*{s.}{Sobrenome: Duan}
  \definition{clas.}{parte; seção; segmento; usado para dividir objetos em várias partes | passagem; parágrafo; parte de algo que tem características de continuidade | seção; período; usado para uma certa distância no tempo ou no espaço}
  \definition{s.}{nível; dan (no judô, weiqi, etc.) | seção (como nível administrativo em uma mina ou fábrica) | parte; etapa; estágio}
  \definition{v.}{cortar; separar}
\end{EntryWithPhonetic}

\begin{EntryWithPhonetic}{段落}{duan4luo4}{9,12}{⽎,⾋}[HSK 7-9]
  \definition[个]{s.}{seção; parágrafo; (artigo, evento) dividido em seções de acordo com o conteúdo | fase; estágio; o estágio relativamente independente de trabalho, coisas, etc.}
\end{EntryWithPhonetic}

%%%%%%%%%% 断 %%%%%%%%%%
\subsection*{断}\addcontentsline{loh}{figure}{断 \dpy{duan4}}

\begin{EntryWithPhonetic}{断}{duan4}{11}{⽄}[HSK 3]
  \definition*{s.}{Sobrenome: Duan}
  \definition{adv.}{(geralmente na forma negativa) absolutamente; decididamente}
  \definition{v.}{quebrar; partir; (objetos longos) dividir em segmentos não conectados | parar; interromper; romper; isolar; fazer com que não se sucedam mais | desistir; abster-se de; parar de fumar, beber, etc. | julgar; decidir | interceptar}
\end{EntryWithPhonetic}

\begin{EntryWithPhonetic}{断定}{duan4ding4}{11,8}{⽄,⼧}[HSK 7-9]
  \definition{v.}{decidir; determinar; concluir; formar um julgamento; fazer um julgamento definitivo}
\end{EntryWithPhonetic}

\begin{EntryWithPhonetic}{断断续续}{duan4duan4xu4xu4}{11,11,11,11}{⽄,⽄,⽷,⽷}[HSK 7-9]
  \definition{adj.}{intermitente; esporádico; ocasional; aos trancos e barrancos}
\end{EntryWithPhonetic}

\begin{EntryWithPhonetic}{断交}{duan4/jiao1}{11,6}{⽄,⼇}
  \definition{v.+compl.}{terminar uma amizade | romper relações diplomáticas}
\end{EntryWithPhonetic}

\begin{EntryWithPhonetic}{断裂}{duan4lie4}{11,12}{⽄,⾐}[HSK 7-9]
  \definition{s.}{fraturar; quebrar; surtar | fender; rachar; romper}
\end{EntryWithPhonetic}

%%%%%%%%%% 锻 %%%%%%%%%%
\subsection*{锻}\addcontentsline{loh}{figure}{锻 \dpy{duan4}}

\begin{EntryWithPhonetic}{锻}{duan4}{14}{⾦}
  \definition{v.}{forjar; moldar}
\end{EntryWithPhonetic}

\begin{EntryWithPhonetic}{锻炼}{duan4lian4}{14,9}{⾦,⽕}[HSK 4]
  \definition{v.}{exercitar"-se; fazer (ou fazer) exercícios; submeter"-se a treinamento físico; fortalecer o corpo por meio do esporte | fortalecer; endurecer; aprimorar as habilidades de trabalho e de vida por meio de trabalho e outras atividades | forjar ou moldar metal para torná"-lo mais refinado; refere"-se à transformação de materiais metálicos em objetos de determinada forma e tamanho por meio de aquecimento, batimento, prensagem etc.}
\end{EntryWithPhonetic}

%%%%%%%%%% 堆 %%%%%%%%%%
\subsection*{堆}\addcontentsline{loh}{figure}{堆 \dpy{dui1}}

\begin{EntryWithPhonetic}{堆}{dui1}{11}{⼟}[HSK 5]
  \definition{clas.}{amontoado; pilha; multidão; usado para pilhas de coisas}
  \definition{s.}{amontoado; pilha; empilhamento | (em nomes de lugares)  colina; monte| multidão de pessoas ou coisas}
  \definition{v.}{empilhar; amontoar; acumular; juntar; reunir}
\end{EntryWithPhonetic}

\begin{EntryWithPhonetic}{堆砌}{dui1qi4}{11,9}{⼟,⽯}[HSK 7-9]
  \definition{v.}{Figurativo: preencher (escrever com frases elaboradas) | Literário: empilhar (tijolos) | embalar}
\end{EntryWithPhonetic}

%%%%%%%%%% 队 %%%%%%%%%%
\subsection*{队}\addcontentsline{loh}{figure}{队 \dpy{dui4}}

\begin{EntryWithPhonetic}{队}{dui4}{4}{⾩}[HSK 2]
  \definition*{s.}{Jovens Pioneiros, refere"-se especificamente à Patrulha de Jovens Pioneiros}
  \definition[条,个,支]{s.}{fila de pessoas | equipe; grupo}
\end{EntryWithPhonetic}

\begin{EntryWithPhonetic}{队伍}{dui4wu5}{4,6}{⾩,⼈}[HSK 6]
  \definition[条,行,列,支,个]{s.}{tropas; exército | fileiras; contingente | desfile}
\end{EntryWithPhonetic}

\begin{EntryWithPhonetic}{队形}{dui4xing2}{4,7}{⾩,⼺}[HSK 7-9]
  \definition{s.}{formação; disposição}
\end{EntryWithPhonetic}

\begin{EntryWithPhonetic}{队友}{dui4you3}{4,4}{⾩,⼜}
  \definition{s.}{companheiro de equipe}
\end{EntryWithPhonetic}

\begin{EntryWithPhonetic}{队员}{dui4yuan2}{4,7}{⾩,⼝}[HSK 3]
  \definition[名,位,个]{s.}{membro da equipe; a composição de uma equipe}
\end{EntryWithPhonetic}

\begin{EntryWithPhonetic}{队长}{dui4zhang3}{4,4}{⾩,⾧}[HSK 2]
  \definition[个,位,名]{s.}{capitão (de equipe); capitães | líder de equipe}
\end{EntryWithPhonetic}

%%%%%%%%%% 对 %%%%%%%%%%
\subsection*{对}\addcontentsline{loh}{figure}{对 \dpy{dui4}}

\begin{EntryWithPhonetic}{对}{dui4}{5}{⼨}[HSK 1,2]
  \definition{adj.}{certo; correto; em conformidade com determinados padrões | oposto; contrário}
  \definition{adv.}{mutuamente; cara a cara}
  \definition{clas.}{usado para pessoas ou coisas que formam pares; casais}
  \definition{prep.}{o que diz respeito a; relativo a; com relação a; introduz o objeto da ação}
  \definition[幅]{s.}{dístico; refere"-se a um par de versos | par; parceiro; pessoas ou coisas que se complementam}
  \definition{v.}{responder; dar uma resposta | tratar; lidar com; combater | ser treinado para; ser direcionado para; enfrentar | colocar (duas coisas) em contato; encaixar uma na outra; combinar ou cooperar entre si | comparar; verificar; identificar; comparar e verificar se estão de acordo | definir; ajustar; ajustar para atender a determinados requisitos | misturar (refere"-se principalmente a líquidos); adicionar | dividir ao meio; dividir em duas partes iguais | combinar; concordar; dar"-se bem; harmonizar"-se}
\end{EntryWithPhonetic}

\begin{EntryWithPhonetic}{对白}{dui4bai2}{5,5}{⼨,⽩}[HSK 7-9]
  \definition{s.}{diálogo | diálogo entre personagens em peças e filmes}
\end{EntryWithPhonetic}

\begin{EntryWithPhonetic}{对比}{dui4bi3}{5,4}{⼨,⽐}[HSK 4]
  \definition{s.}{razão; proporção | contraste; comparação; diferenças ou lacunas encontradas após comparação}
  \definition{v.}{contrastar; comparar}
\end{EntryWithPhonetic}

\begin{EntryWithPhonetic}{对不起}{dui4bu5qi3}{5,4,10}{⼨,⼀,⾛}[HSK 1]
  \definition{interj.}{``Desculpe!''; ``Desculpe"-me!''; ``Perdoe"-me!''; ``Com licença?''}
  \definition{v.}{desculpar; pedir desculpas; perdoar}
\end{EntryWithPhonetic}

\begin{EntryWithPhonetic}{对策}{dui4ce4}{5,12}{⼨,⽵}[HSK 7-9]
  \definition{s.}{contra-ataque; contramedida; maneira de lidar com uma situação; estratégias ou soluções para problemas a serem resolvidos}
  \definition{v.}{abordar; elaborar estratégias; traçar estratégias}
\end{EntryWithPhonetic}

\begin{EntryWithPhonetic}{对称}{dui4chen4}{5,10}{⼨,⽲}[HSK 7-9]
  \definition{adj.}{simétrico; refere"-se a uma figura ou objeto que tem uma correspondência um"-para"-um em tamanho, forma e disposição em relação a um ponto, linha ou plano, como os lados esquerdo e direito de um corpo humano, um navio ou um avião}
\end{EntryWithPhonetic}

\begin{EntryWithPhonetic}{对待}{dui4dai4}{5,9}{⼨,⼻}[HSK 3]
  \definition{v.}{tratar; abordar; manusear; estar em uma posição relacionada ou comparada a outra; expressar uma certa atitude ou agir de determinada maneira em relação a pessoas ou coisas}
\end{EntryWithPhonetic}

\begin{EntryWithPhonetic}{对得起}{dui4de5qi3}{5,11,10}{⼨,⼻,⾛}[HSK 7-9]
  \definition{v.}{tratar alguém de forma justa; ser digno de; não decepcionar alguém}
\end{EntryWithPhonetic}

\begin{EntryWithPhonetic}{对方}{dui4fang1}{5,4}{⼨,⽅}[HSK 3]
  \definition{s.}{outro lado; lado oposto; outra parte; a parte contrária ao sujeito da ação ou outras pessoas envolvidas em um determinado evento ou situação}
\end{EntryWithPhonetic}

\begin{EntryWithPhonetic}{对付}{dui4fu5}{5,5}{⼨,⼈}[HSK 4]
  \definition{adj.}{em bons termos; estar em termos agradáveis (frequentemente usado em negativas); dialeto usado para descrever duas pessoas que têm um bom relacionamento e se dão bem, frequentemente usado para negar}
  \definition{v.}{enfrentar; tratar; lidar com | fazer acontecer; (informal) fazer algo que você não quer fazer; aceitar algo que você não gosta}
\end{EntryWithPhonetic}

\begin{EntryWithPhonetic}{对……感兴趣}{dui4 gan3xing4qu4}{5,13,6,15}{⼨,⼼,⼋,⾛}
  \definition{expr.}{estar interessado em\dots; ter interesse em\dots; interessar-se por\dots}
  \seealsoref{对……有兴趣}{dui4 you3xing4qu4}
\end{EntryWithPhonetic}

\begin{EntryWithPhonetic}{对话}{dui4hua4}{5,8}{⼨,⾔}[HSK 2]
  \definition[段,番,个]{s.}{diálogo; conversa; refere"-se especificamente a diálogos entre personagens em obras literárias, como peças de teatro e romances}
  \definition{v.}{conversar com; comunicar"-se com | manter um diálogo; conversar uns com os outros}
\end{EntryWithPhonetic}

\begin{EntryWithPhonetic}{对抗}{dui4kang4}{5,7}{⼨,⼿}[HSK 6]
  \definition{v.}{antagonizar; confrontar | resistir; opor-se; contra-atacar}
\end{EntryWithPhonetic}

\begin{EntryWithPhonetic}{对立}{dui4li4}{5,5}{⼨,⽴}[HSK 5]
  \definition{v.}{opor"-se; contrastar; filosoficamente, refere"-se a duas coisas ou dois aspectos da mesma coisa que se contradizem, se excluem ou entram em conflito entre si | opor"-se; ser antagônico a}
\end{EntryWithPhonetic}

\begin{EntryWithPhonetic}{对联}{dui4lian2}{5,12}{⼨,⽿}[HSK 7-9]
  \definition[副,幅]{s.}{dístico (escrito em pergaminhos, etc.); dísticos: frases paralelas escritas em papel, tecido ou esculpidas em bambu, madeira ou pilares}
\end{EntryWithPhonetic}

\begin{EntryWithPhonetic}{对面}{dui4mian4}{5,9}{⼨,⾯}[HSK 2]
  \definition{adv.}{cara a cara}
  \definition[面]{s.}{lado oposto; o outro lado; os nomes dados às duas margens opostas de ruas, rios, etc. | bem na frente; diretamente à frente}
\end{EntryWithPhonetic}

\begin{EntryWithPhonetic}{对手}{dui4shou3}{5,4}{⼨,⼿}[HSK 3]
  \definition[个,名,位,对]{s.}{oponente; adversário na competição | igual; correspondente; refere"-se especificamente ao adversário em uma competição em que as habilidades e o nível são praticamente iguais}
\end{EntryWithPhonetic}

\begin{EntryWithPhonetic}{对……熟悉}{dui4 shu2xi5}{5,15,11}{⼨,⽕,⼼}
  \definition{expr.}{estar familiarizado com\dots}
\end{EntryWithPhonetic}

\begin{EntryWithPhonetic}{对……说}{dui4 shuo5}{5,9}{⼨,⾔}
  \definition{v.}{dizer a alguém}
\end{EntryWithPhonetic}

\begin{EntryWithPhonetic}{对外}{dui4wai4}{5,5}{⼨,⼣}[HSK 6]
  \definition{adj.}{externo; para fora | estrangeiro; no exterior}
\end{EntryWithPhonetic}

\begin{EntryWithPhonetic}{对象}{dui4xiang4}{5,11}{⼨,⾗}[HSK 3]
  \definition[个,位]{s.}{alvo; objeto; a pessoa ou coisa que serve como objetivo ao agir ou pensar | parceiro; namorado; namorada; refere"-se especificamente à pessoa amada}
\end{EntryWithPhonetic}

\begin{EntryWithPhonetic}{对弈}{dui4yi4}{5,9}{⼨,⼶}[HSK 7-9]
  \definition{v.}{Literário: jogar go, xadrez etc.}
\end{EntryWithPhonetic}

\begin{EntryWithPhonetic}{对应}{dui4ying4}{5,7}{⼨,⼴}[HSK 5]
  \definition{adj.}{homólogo; correspondente}
  \definition{v.}{corresponder; ser equivalente a}
\end{EntryWithPhonetic}

\begin{EntryWithPhonetic}{对……有兴趣}{dui4 you3xing4qu4}{5,6,6,15}{⼨,⽉,⼋,⾛}
  \definition{expr.}{estar interessado em\dots; ter interesse em\dots; interessar-se por\dots}
  \seealsoref{对……感兴趣}{dui4 gan3xing4qu4}
\end{EntryWithPhonetic}

\begin{EntryWithPhonetic}{对于}{dui4yu2}{5,3}{⼨,⼆}[HSK 4]
  \definition{prep.}{para; relativo a; no que diz respeito a; a respeito de}
\end{EntryWithPhonetic}

\begin{EntryWithPhonetic}{对照}{dui4zhao4}{5,13}{⼨,⽕}[HSK 7-9]
  \definition{s.}{contraste; comparação}
  \definition{v.aux.}{contrastar; comparar ; fazer referência cruzada; fazer comparação}
\end{EntryWithPhonetic}

\begin{EntryWithPhonetic}{对峙}{dui4zhi4}{5,9}{⼨,⼭}[HSK 7-9]
  \definition{v.}{ficar de frente um para o outro; confrontar um ao outro}
\end{EntryWithPhonetic}

\begin{EntryWithPhonetic}{对准}{dui4zhun3}{5,10}{⼨,⼎}[HSK 7-9]
  \definition{s.}{alinhamento; mira; registro}
  \definition{v.}{treinar em; ser direcionado para; mirar (apontar; direcionar) para}
\end{EntryWithPhonetic}

%%%%%%%%%% 兑 %%%%%%%%%%
\subsection*{兑}\addcontentsline{loh}{figure}{兑 \dpy{dui4}}

\begin{EntryWithPhonetic}{兑}{dui4}{7}{⼉}
  \definition*{s.}{Sobrenome: Dui}
  \definition{s.}{dui, um dos Oito Trigramas que representa pântano}
  \definition{v.}{trocar; converter | adicionar (água, etc.) | sacar; pagar ou receber dinheiro por fatura}
\end{EntryWithPhonetic}

\begin{EntryWithPhonetic}{兑换}{dui4huan4}{7,10}{⼉,⼿}[HSK 7-9]
  \definition{v.}{converter; trocar; trocar uma moeda por outra; trocar um cheque, etc., por dinheiro}
\end{EntryWithPhonetic}

\begin{EntryWithPhonetic}{兑现}{dui4xian4}{7,8}{⼉,⾒}[HSK 7-9]
  \definition{v.}{sacar (dinheiro, cheques, etc.) | cumprir; fazer o bem; honrar (um compromisso, etc.); metáfora para cumprir uma promessa}
\end{EntryWithPhonetic}

%%%%%%%%%% 敦 %%%%%%%%%%
\subsection*{敦}\addcontentsline{loh}{figure}{敦 \dpy{dui4}}

\begin{EntryWithPhonetic}{敦}{dui4}{12}{⽁}
  \definition{s.}{um recipiente tradicional para armazenar arroz ou grãos; utensílios antigos para guardar painço}
  \seeref{dun1}
\end{EntryWithPhonetic}

%%%%%%%%%% 吨 %%%%%%%%%%
\subsection*{吨}\addcontentsline{loh}{figure}{吨 \dpy{dun1}}

\begin{EntryWithPhonetic}{吨}{dun1}{7}{⼝}[HSK 5]
  \definition{clas.}{tonelada}
\end{EntryWithPhonetic}

%%%%%%%%%% 敦 %%%%%%%%%%
\subsection*{敦}\addcontentsline{loh}{figure}{敦 \dpy{dun1}}

\begin{EntryWithPhonetic}{敦}{dun1}{12}{⽁}
  \definition*{s.}{Sobrenome: Dun}
  \definition{adj.}{honesto; sincero}
  \seeref{dui4}
\end{EntryWithPhonetic}

\begin{EntryWithPhonetic}{敦促}{dun1cu4}{12,9}{⽁,⼈}[HSK 7-9]
  \definition{v.}{instar; pressionar; urgir}
\end{EntryWithPhonetic}

\begin{EntryWithPhonetic}{敦厚}{dun1hou4}{12,9}{⽁,⼚}[HSK 7-9]
  \definition{adj.}{genuíno | honesto e sincero}
\end{EntryWithPhonetic}

%%%%%%%%%% 蹲 %%%%%%%%%%
\subsection*{蹲}\addcontentsline{loh}{figure}{蹲 \dpy{dun1}}

\begin{EntryWithPhonetic}{蹲}{dun1}{19}{⾜}[HSK 6]
  \definition{v.}{agachamento sobre os calcanhares; dobrar as pernas o máximo possível, como se estivesse sentado, mas não deixar as nádegas tocarem o chão | ficar; metáfora para ficar ocioso em casa}
\end{EntryWithPhonetic}

\begin{EntryWithPhonetic}{蹲下}{dun1xia4}{19,3}{⾜,⼀}
  \definition{v.}{agachar | agachar-se}
\end{EntryWithPhonetic}

%%%%%%%%%% 炖 %%%%%%%%%%
\subsection*{炖}\addcontentsline{loh}{figure}{炖 \dpy{dun4}}

\begin{EntryWithPhonetic}{炖}{dun4}{8}{⽕}[HSK 7-9]
  \definition{v.}{cozinhar | aquecer algo colocando o recipiente em água quente}
\end{EntryWithPhonetic}

%%%%%%%%%% 钝 %%%%%%%%%%
\subsection*{钝}\addcontentsline{loh}{figure}{钝 \dpy{dun4}}

\begin{EntryWithPhonetic}{钝}{dun4}{9}{⾦}
  \definition{adj.}{sem corte; opaco | estúpido; sem noção | maçante}
  \antonymref{快}{kuai4}
  \antonymref{利}{li4}
  \antonymref{锐}{rui4}
\end{EntryWithPhonetic}

%%%%%%%%%% 顿 %%%%%%%%%%
\subsection*{顿}\addcontentsline{loh}{figure}{顿 \dpy{dun4}}

\begin{EntryWithPhonetic}{顿}{dun4}{10}{⾴}[HSK 3]
  \definition*{s.}{Sobrenome: Dun}
  \definition{adj.}{cansado; fatigado}
  \definition{adv.}{de repente; imediatamente; indica que o tempo é curto, equivalente a 立刻}
  \definition{clas.}{usado para refeições | usado para surras, repreensões, castigos físicos, etc.}
  \definition{s.}{um lugar para ficar; acomodação e alimentação}
  \definition{v.}{pausar; parar; fazer uma pausa | pausar na escrita para reforçar o início ou o fim de um traço; ao escrever com pincel, pressione o pincel com força e pare um pouco sobre o papel | tocar o chão (com a cabeça) | bater o pé); chutar o chão ou bater no chão com um objeto | resolver; arranjar | montar acampamento; ficar temporariamente; parar para se hospedar; acampar}
  \seealsoref{立刻}{li4ke4}
\end{EntryWithPhonetic}

\begin{EntryWithPhonetic}{顿时}{dun4shi2}{10,7}{⾴,⽇}[HSK 7-9]
  \definition{adv.}{de repente; imediatamente; repentinamente; indica que uma ação ou comportamento ocorre sob certas circunstâncias ou imediatamente após algo; usado principalmente na escrita; usado apenas para descrever eventos passados}
\end{EntryWithPhonetic}

%%%%%%%%%% 多 %%%%%%%%%%
\subsection*{多}\addcontentsline{loh}{figure}{多 \dpy{duo1}}

\begin{EntryWithPhonetic}{多}{duo1}{6}{⼣}[HSK 1,2]
  \definition*{s.}{Sobrenome: Duo}
  \definition{adj.}{grande quantidade | excessivo; desnecessário | excessivo; em demasia; indica um grande grau de diferença | mais do que o número correto ou necessário; em excesso}
  \definition{adv.}{acima de um valor especificado; e mais | em que medida; usado em frases interrogativas para indagar sobre grau ou quantidade, equivalente a 多么 | uma extensão não especificada; usado em frases exclamativas para expressar um alto grau, equivalente a 多么 | quase; significa que a maior parte do intervalo é assim | mais;  sobre; ímpar; usado depois de um quantificador para indicar uma fração}
  \definition{num.}{(após um número) ímpar}
  \definition{pref.}{multi- | poli-}
  \definition{v.}{ter (uma quantidade específica) a mais ou a mais | ter algo em abundância  | (em perguntas) até que ponto | (em exclamações) até que ponto | ter mais}
  \seealsoref{多么}{duo1me5}
  \antonymref{寡}{gua3}
  \antonymref{少}{shao3}
\end{EntryWithPhonetic}

\begin{EntryWithPhonetic}{多半}{duo1ban4}{6,5}{⼣,⼗}[HSK 6]
  \definition{adv.}{geralmente; mais frequentemente do que não}
  \definition{num.}{a maioria; a maior parte; mais da metade}
\end{EntryWithPhonetic}

\begin{EntryWithPhonetic}{多边}{duo1bian1}{6,5}{⼣,⾡}[HSK 7-9]
  \definition{adj.}{multilateral (três ou mais partes)}
\end{EntryWithPhonetic}

\begin{EntryWithPhonetic}{多重}{duo1chong2}{6,9}{⼣,⾥}
  \definition{pref.}{multi (facetado, cultural, étnico, etc.)}
\end{EntryWithPhonetic}

\begin{EntryWithPhonetic}{多次}{duo1ci4}{6,6}{⼣,⽋}[HSK 4]
  \definition{adv.}{muitas vezes; de vez em quando; repetidamente; em muitas ocasiões}
\end{EntryWithPhonetic}

\begin{EntryWithPhonetic}{多大}{duo1da4}{6,3}{⼣,⼤}
  \definition{adj.}{quantos anos? | que idade? | quão grande?}
\end{EntryWithPhonetic}

\begin{EntryWithPhonetic}{多方面}{duo1fang1mian4}{6,4,9}{⼣,⽅,⾯}[HSK 6]
  \definition{adj.}{de muitas maneiras; todos os aspectos}
  \definition{s.}{multifacetado; multiaspecto}
\end{EntryWithPhonetic}

\begin{EntryWithPhonetic}{多功能}{duo1gong1neng2}{6,5,10}{⼣,⼒,⾁}[HSK 7-9]
  \definition{adj.}{multifuncional; multiuso (ou para todos os fins)}
\end{EntryWithPhonetic}

\begin{EntryWithPhonetic}{多久}{duo1jiu3}{6,3}{⼣,⼃}[HSK 2]
  \definition{pron.}{quanto tempo?; quanto tempo; perguntar quanto tempo leva}
\end{EntryWithPhonetic}

\begin{EntryWithPhonetic}{多亏}{duo1kui1}{6,3}{⼣,⼆}[HSK 7-9]
  \definition{adv.}{felizmente; graças a; devido a; significa que alguém evitou algo desagradável ou ganhou algo bom devido à ajuda de outros ou alguns fatores favoráveis, e implica gratidão ou alívio}
\end{EntryWithPhonetic}

\begin{EntryWithPhonetic}{多劳多得}{duo1lao2-duo1de2}{6,7,6,11}{⼣,⼒,⼣,⼻}[HSK 7-9]
  \definition{expr.}{trabalhe mais e ganhe mais; o princípio socialista de distribuição é que quanto mais você trabalha, mais você se beneficia, e se você não trabalha, você não ganha comida}
\end{EntryWithPhonetic}

\begin{EntryWithPhonetic}{多么}{duo1me5}{6,3}{⼣,⼃}[HSK 2]
  \definition{adv.}{(em exclamações) como; o quê; em que medida; usado em frases exclamativas, indica um grau muito alto | em grau indeterminado; usado em frases declarativas, indica um grau mais profundo | como (usado em uma frase interrogativa para perguntar sobre grau ou número)}
\end{EntryWithPhonetic}

\begin{EntryWithPhonetic}{多媒体}{duo1mei2ti3}{6,12,7}{⼣,⼥,⼈}[HSK 6]
  \definition{s.}{multimídia; uma combinação de múltiplas mídias}
\end{EntryWithPhonetic}

\begin{EntryWithPhonetic}{多年}{duo1nian2}{6,6}{⼣,⼲}[HSK 4]
  \definition{adv.}{por muitos anos; durante muitos anos}
\end{EntryWithPhonetic}

\begin{EntryWithPhonetic}{多年来}{duo1nian2lai2}{6,6,7}{⼣,⼲,⽊}[HSK 7-9]
  \definition{s.}{nos últimos anos}
\end{EntryWithPhonetic}

\begin{EntryWithPhonetic}{多少}{duo1shao3}{6,4}{⼣,⼩}
  \definition{adv.}{um pouco; mais ou menos; até certo ponto}
  \definition{s.}{número; quantidade; volume}
  \seeref{duo1shao5}
\end{EntryWithPhonetic}

\begin{EntryWithPhonetic}{多少}{duo1shao5}{6,4}{⼣,⼩}[HSK 1]
  \definition{adv.}{quantos?; quanto?; usado em perguntas para perguntar sobre quantidade | expressar uma quantidade ou número não especificado; quantidade indefinida}
  \seeref{duo1shao3}
\end{EntryWithPhonetic}

\begin{EntryWithPhonetic}{多数}{duo1shu4}{6,13}{⼣,⽁}[HSK 2]
  \definition{adj.}{maioria; a maioria; plural}
  \definition{pref.}{pluri-}
\end{EntryWithPhonetic}

\begin{EntryWithPhonetic}{多心}{duo1xin1}{6,4}{⼣,⼼}[HSK 7-9]
  \definition{adj.}{hipersensível; paranóico; sensível | suspeito}
\end{EntryWithPhonetic}

\begin{EntryWithPhonetic}{多样}{duo1yang4}{6,10}{⼣,⽊}[HSK 4]
  \definition{adj.}{diversos; variados; diversificado}
  \definition{s.}{diversidade}
\end{EntryWithPhonetic}

\begin{EntryWithPhonetic}{多余}{duo1yu2}{6,7}{⼣,⼈}[HSK 7-9]
  \definition{adj.}{extra; excedente; excessivo; supérfluo; desnecessário}
  \definition{v.}{exceder; superar; transbordar}
\end{EntryWithPhonetic}

\begin{EntryWithPhonetic}{多元}{duo1yuan2}{6,4}{⼣,⼉}[HSK 7-9]
  \definition{adj.}{diverso; pluralista; multivariado}
\end{EntryWithPhonetic}

\begin{EntryWithPhonetic}{多云}{duo1yun2}{6,4}{⼣,⼆}[HSK 2]
  \definition{adj.}{céu nublado; em meteorologia, refere"-se a condições atmosféricas em que a cobertura de nuvens médias e baixas ocupa entre 40\% e 70\% da área do céu, ou a cobertura de nuvens altas ocupa entre 60\% e 100\% da área do céu}
\end{EntryWithPhonetic}

\begin{EntryWithPhonetic}{多咱}{duo1 zan5}{6,9}{⼣,⼝}
  \definition{adv.}{que horas?; quando?}
\end{EntryWithPhonetic}

\begin{EntryWithPhonetic}{多种}{duo1zhong3}{6,9}{⼣,⽲}[HSK 4]
  \definition{adj.}{diverso; vários tipos de; múltiplo; diversificado}
  \definition{pref.}{multi-}
\end{EntryWithPhonetic}

%%%%%%%%%% 哆 %%%%%%%%%%
\subsection*{哆}\addcontentsline{loh}{figure}{哆 \dpy{duo1}}

\begin{EntryWithPhonetic}{哆}{duo1}{9}{⼝}
  \definition{part.}{usado em 哆嗦}
  \seealsoref{哆嗦}{duo1suo5}
\end{EntryWithPhonetic}

\begin{EntryWithPhonetic}{哆嗦}{duo1suo5}{9,13}{⼝,⼝}[HSK 7-9]
  \definition{v.}{tremer; estremecer (tremores corporais involuntários devido a estímulos externos)}
\end{EntryWithPhonetic}

%%%%%%%%%% 夺 %%%%%%%%%%
\subsection*{夺}\addcontentsline{loh}{figure}{夺 \dpy{duo2}}

\begin{EntryWithPhonetic}{夺}{duo2}{6}{⼤}[HSK 6]
  \definition{v.}{tomar à força; apreender; arrancar; roubar | forçar a passagem; empurrar para abrir | lutar por; competir por; esforçar-se por; obter primeiro | privar; perder | perder; tirar | decidir; tomar uma decisão | omitir (palavra em um texto)}
\end{EntryWithPhonetic}

\begin{EntryWithPhonetic}{夺冠}{duo2/guan4}{6,9}{⼤,⼍}[HSK 7-9]
  \definition{v.+compl.}{chegar em primeiro lugar; ficar em primeiro lugar; ganhar um campeonato}
\end{EntryWithPhonetic}

\begin{EntryWithPhonetic}{夺魁}{duo2/kui2}{6,13}{⼤,⿁}[HSK 7-9]
  \definition{v.+compl.}{Literário: ganhar o primeiro prêmio; ganhar o campeonato; conquistar; vencer}
\end{EntryWithPhonetic}

\begin{EntryWithPhonetic}{夺取}{duo2qu3}{6,8}{⼤,⼜}[HSK 6]
  \definition{v.}{capturar; apreender; arrancar; tomar à força | esforçar-se para; alcançar}
\end{EntryWithPhonetic}

%%%%%%%%%% 度 %%%%%%%%%%
\subsection*{度}\addcontentsline{loh}{figure}{度 \dpy{duo2}}

\begin{EntryWithPhonetic}{度}{duo2}{9}{⼴}
  \definition{v.}{supor; estimar; especular}
  \seeref{du4}
\end{EntryWithPhonetic}

%%%%%%%%%% 朵 %%%%%%%%%%
\subsection*{朵}\addcontentsline{loh}{figure}{朵 \dpy{duo3}}

\begin{EntryWithPhonetic}{朵}{duo3}{6}{⽊}[HSK 5]
  \definition*{s.}{Sobrenome: Duo}
  \definition{clas.}{usado para flores, nuvens ou coisas que se assemelham a flores e nuvens}
\end{EntryWithPhonetic}

%%%%%%%%%% 躲 %%%%%%%%%%
\subsection*{躲}\addcontentsline{loh}{figure}{躲 \dpy{duo3}}

\begin{EntryWithPhonetic}{躲}{duo3}{13}{⾝}[HSK 5]
  \definition{v.}{esconder (a si mesmo); ocultar (a si mesmo); esconder-se | evitar; esquivar-se}
\end{EntryWithPhonetic}

\begin{EntryWithPhonetic}{躲避}{duo3bi4}{13,16}{⾝,⾌}[HSK 7-9]
  \definition{v.}{esquivar; evitar; fugir; sair ou se esconder deliberadamente para que as pessoas não possam vê-lo | evitar; esquivar-se; fugir; deixar para trás as coisas que não são boas para você}
\end{EntryWithPhonetic}

\begin{EntryWithPhonetic}{躲藏}{duo3cang2}{13,17}{⾝,⾋}[HSK 7-9]
  \definition{v.}{esconder-se; esconder seu corpo da vista}
\end{EntryWithPhonetic}

\begin{EntryWithPhonetic}{躲闪}{duo3shan3}{13,5}{⾝,⾨}
  \definition{v.}{desviar | evadir | esquivar (para fora do caminho)}
\end{EntryWithPhonetic}

%%%%%%%%%% 堕 %%%%%%%%%%
\subsection*{堕}\addcontentsline{loh}{figure}{堕 \dpy{duo4}}

\begin{EntryWithPhonetic}{堕}{duo4}{11}{⼟}
  \definition{v.}{cair; afundar}
\end{EntryWithPhonetic}

\begin{EntryWithPhonetic}{堕落}{duo4luo4}{11,12}{⼟,⾋}[HSK 7-9]
  \definition{adj.}{corrupto; decadente}
  \definition{v.}{cair; afundar; degenerar; deteriorar}
\end{EntryWithPhonetic}

%%%%%%%%%% 舵 %%%%%%%%%%
\subsection*{舵}\addcontentsline{loh}{figure}{舵 \dpy{duo4}}

\begin{EntryWithPhonetic}{舵}{duo4}{11}{⾈}
  \definition{s.}{leme; dispositivos para controlar a direção de navios, aeronaves, etc.}
  \seealsoref{柁}{tuo2}
\end{EntryWithPhonetic}

\begin{EntryWithPhonetic}{舵手}{duo4shou3}{11,4}{⾈,⼿}[HSK 7-9]
  \definition{s.}{timoneiro}
\end{EntryWithPhonetic}

%%%%% EOF %%%%%


 %%%
%%% E
%%%
\section*{E}\addcontentsline{toc}{section}{E}\addcontentsline{loh}{figure}{\#\#\#\#\#\#\#\# E}

%%%%%%%%%% 阿 %%%%%%%%%%
\subsection*{阿}\addcontentsline{loh}{figure}{阿 \dpy{e1}}

\begin{EntryWithPhonetic}{阿}{e1}{7}{⾩}
  \definition*{s.}{Dong'e, um condado na província de Shandong | Sobrenome: E}
  \definition{s.}{grande monte (ou colina) | um lugar sinuoso (montanha, água, etc.)}
  \definition{v.}{bajular; satisfazer}
  \seeref{a1}
\end{EntryWithPhonetic}

%%%%%%%%%% 讹 %%%%%%%%%%
\subsection*{讹}\addcontentsline{loh}{figure}{讹 \dpy{e2}}

\begin{EntryWithPhonetic}{讹}{e2}{6}{⾔}
  \definition{adj.}{errôneo; equivocado}
  \definition{v.}{extorquir sob falsos pretextos; chantagear; enganar}
\end{EntryWithPhonetic}

\begin{EntryWithPhonetic}{讹诈}{e2zha4}{6,7}{⾔,⾔}[HSK 7-9]
  \definition{v.}{extorquir sob falsos pretextos; chantagear; intimidar}
\end{EntryWithPhonetic}

%%%%%%%%%% 俄 %%%%%%%%%%
\subsection*{俄}\addcontentsline{loh}{figure}{俄 \dpy{e2}}

\begin{EntryWithPhonetic}{俄}{e2}{9}{⼈}
  \definition*{s.}{Rússia, abreviação de 俄罗斯}
  \definition{adv.}{muito em breve; em breve; de repente}
  \seealsoref{俄罗斯}{e2luo2si1}
\end{EntryWithPhonetic}

\begin{EntryWithPhonetic}{俄罗斯}{e2luo2si1}{9,8,12}{⼈,⽹,⽄}
  \definition*{s.}{Rússia}
\end{EntryWithPhonetic}

\begin{EntryWithPhonetic}{俄罗斯人}{e2luo2si1ren2}{9,8,12,2}{⼈,⽹,⽄,⼈}
  \definition{s.}{russo | pessoa ou povo da Rússia}
\end{EntryWithPhonetic}

\begin{EntryWithPhonetic}{俄语}{e2yu3}{9,9}{⼈,⾔}[HSK 7-9]
  \definition{s.}{russo; língua russa}
\end{EntryWithPhonetic}

%%%%%%%%%% 哦 %%%%%%%%%%
\subsection*{哦}\addcontentsline{loh}{figure}{哦 \dpy{e2}}

\begin{EntryWithPhonetic}{哦}{e2}{10}{⼝}
  \definition{v.}{cantar suavemente (um poema)}
  \seeref{o2}
  \seeref{o4}
  \seeref{o5}
\end{EntryWithPhonetic}

%%%%%%%%%% 鹅 %%%%%%%%%%
\subsection*{鹅}\addcontentsline{loh}{figure}{鹅 \dpy{e2}}

\begin{EntryWithPhonetic}{鹅}{e2}{12}{⿃}[HSK 7-9]
  \definition[只,群]{s.}{ganso}
\end{EntryWithPhonetic}

%%%%%%%%%% 额 %%%%%%%%%%
\subsection*{额}\addcontentsline{loh}{figure}{额 \dpy{e2}}

\begin{EntryWithPhonetic}{额}{e2}{15}{⾴}
  \definition*{s.}{Sobrenome: E}
  \definition[块]{s.}{testa; a área abaixo do cabelo e acima das sobrancelhas em humanos; a área aproximadamente equivalente na cabeça de alguns animais | uma tábua horizontal; placa horizontal inscrita; uma placa pendurada no lintel de uma porta ou na parede | um número específico (ou quantidade); limite superior de número; número limitado | a parte superior de algo}
\end{EntryWithPhonetic}

\begin{EntryWithPhonetic}{额外}{e2wai4}{15,5}{⾴,⼣}[HSK 7-9]
  \definition{adj.}{extra; adicional; excede a quantidade ou intervalo prescrito}
\end{EntryWithPhonetic}

%%%%%%%%%% 恶 %%%%%%%%%%
\subsection*{恶}\addcontentsline{loh}{figure}{恶 \dpy{e3}}

\begin{EntryWithPhonetic}{恶}{e3}{10}{⼼}
  \definition{part.}{elementos formadores de palavras}
  \seeref{e4}
  \seeref{wu1}
  \seeref{wu4}
\end{EntryWithPhonetic}

\begin{EntryWithPhonetic}{恶心}{e3xin5}{10,4}{⼼,⼼}[HSK 4]
  \definition{adj.}{nauseante; repugnante}
  \definition{s.}{náusea; repugnância}
  \definition{v.}{repugnar; ser nauseante; sentir-se mal | envergonhar (deliberadamente)}
  \seeref{e4xin1}
\end{EntryWithPhonetic}

%%%%%%%%%% 厄 %%%%%%%%%%
\subsection*{厄}\addcontentsline{loh}{figure}{厄 \dpy{e4}}

\begin{EntryWithPhonetic}{厄}{e4}{4}{⼚}
  \definition[个]{s.}{ponto estratégico; lugares perigosos | adversidade; desastre; dificuldade}
  \definition{v.}{estar em perigo; estar abandonado; estar; encurralado}
\end{EntryWithPhonetic}

\begin{EntryWithPhonetic}{厄运}{e4yun4}{4,7}{⼚,⾡}[HSK 7-9]
  \definition{s.}{adversidade; infortúnio; experiência infeliz}
\end{EntryWithPhonetic}

%%%%%%%%%% 恶 %%%%%%%%%%
\subsection*{恶}\addcontentsline{loh}{figure}{恶 \dpy{e4}}

\begin{EntryWithPhonetic}{恶}{e4}{10}{⼼}[HSK 7-9]
  \definition{adj.}{feroz | ruim; maligno; perverso | vicioso | feio | grosseiro}
  \definition{s.}{mal; vício; crime | maldade; comportamento muito ruim; coisas criminosas}
  \seeref{e3}
  \seeref{wu1}
  \seeref{wu4}
  \antonymref{善}{shan4}
\end{EntryWithPhonetic}

\begin{EntryWithPhonetic}{恶霸}{e4ba4}{10,21}{⼼,⾬}
  \definition{s.}{tirano local; déspota local; valentão; pessoas más que se apoiam em forças reacionárias para dominar uma região e oprimir o povo}
  \synonymref{霸王}{ba4wang2}
  \antonymref{绅士}{shen1shi4}
\end{EntryWithPhonetic}

\begin{EntryWithPhonetic}{恶化}{e4hua4}{10,4}{⼼,⼔}[HSK 7-9]
  \definition{v.}{piorar; deteriorar; exacerbar | piorar a situação}
\end{EntryWithPhonetic}

\begin{EntryWithPhonetic}{恶劣}{e4lie4}{10,6}{⼼,⼒}[HSK 7-9]
  \definition{adj.}{mau; odioso; abominável; repugnante; desprezível; muito mau; muito ruim}
\end{EntryWithPhonetic}

\begin{EntryWithPhonetic}{恶心}{e4xin1}{10,4}{⼼,⼼}
  \definition{s.}{mau hábito; hábito vicioso; vício}
  \seeref{e3xin5}
\end{EntryWithPhonetic}

\begin{EntryWithPhonetic}{恶性}{e4xing4}{10,8}{⼼,⼼}[HSK 7-9]
  \definition{adj.}{maligno; pernicioso; vicioso | produzindo o mal | rápido (declínio) | descontrolada (inflação) | vicioso (círculo) | perverso}
  \antonymref{良性}{liang2xing4}
\end{EntryWithPhonetic}

\begin{EntryWithPhonetic}{恶意}{e4yi4}{10,13}{⼼,⼼}[HSK 7-9]
  \definition[丝]{s.}{malícia; má vontade; más intenções}
\end{EntryWithPhonetic}

%%%%%%%%%% 饿 %%%%%%%%%%
\subsection*{饿}\addcontentsline{loh}{figure}{饿 \dpy{e4}}

\begin{EntryWithPhonetic}{饿}{e4}{10}{⾷}[HSK 1]
  \definition{adj.}{faminto}
  \definition{v.}{passar fome; causar fome}
\end{EntryWithPhonetic}

%%%%%%%%%% 遏 %%%%%%%%%%
\subsection*{遏}\addcontentsline{loh}{figure}{遏 \dpy{e4}}

\begin{EntryWithPhonetic}{遏}{e4}{12}{⾡}
  \definition{v.}{reprimir; restringir; reter; impedir; proibir}
\end{EntryWithPhonetic}

\begin{EntryWithPhonetic}{遏制}{e4zhi4}{12,8}{⾡,⼑}[HSK 7-9]
  \definition{v.}{conter; restringir; controlar e prevenir ativamente o desenvolvimento de coisas que possam trazer perigo; usado principalmente para discutir tópicos formais}
\end{EntryWithPhonetic}

%%%%%%%%%% 鳄 %%%%%%%%%%
\subsection*{鳄}\addcontentsline{loh}{figure}{鳄 \dpy{e4}}

\begin{EntryWithPhonetic}{鳄}{e4}{17}{⿂}
  \definition{s.}{crocodilo;  jacaré}
\end{EntryWithPhonetic}

\begin{EntryWithPhonetic}{鳄鱼}{e4yu2}{17,8}{⿂,⿂}[HSK 7-9]
  \definition[只,群,些]{s.}{crocodilo; jacaré}
\end{EntryWithPhonetic}

%%%%%%%%%% 恩 %%%%%%%%%%
\subsection*{恩}\addcontentsline{loh}{figure}{恩 \dpy{en1}}

\begin{EntryWithPhonetic}{恩}{en1}{10}{⼼}
  \definition*{s.}{Sobrenome: En}
  \definition{s.}{bondade; favor; graça; gentileza}
\end{EntryWithPhonetic}

\begin{EntryWithPhonetic}{恩赐}{en1ci4}{10,12}{⼼,⾙}[HSK 7-9]
  \definition{s.}{favor; caridade; esmola}
  \definition{v.}{conceder (favores, caridade, etc.); recompensar}
\end{EntryWithPhonetic}

\begin{EntryWithPhonetic}{恩惠}{en1hui4}{10,12}{⼼,⼼}[HSK 7-9]
  \definition[份]{s.}{favor; generosidade | bondade; graça; benefícios concedidos ou recebidos}
\end{EntryWithPhonetic}

\begin{EntryWithPhonetic}{恩情}{en1qing2}{10,11}{⼼,⼼}[HSK 7-9]
  \definition{s.}{amor; bondade; afeição profunda}
\end{EntryWithPhonetic}

\begin{EntryWithPhonetic}{恩人}{en1 ren2}{10,2}{⼼,⼈}[HSK 6]
  \definition{s.}{benfeitor; uma pessoa que ajudou significativamente alguém}
\end{EntryWithPhonetic}

\begin{EntryWithPhonetic}{恩怨}{en1yuan4}{10,9}{⼼,⼼}[HSK 7-9]
  \definition{s.}{sentimento de gratidão ou ressentimento (inimizade) | ressentimento; queixa; velhas contas}
\end{EntryWithPhonetic}

%%%%%%%%%% 儿 %%%%%%%%%%
\subsection*{儿}\addcontentsline{loh}{figure}{儿 \dpy{er2}}

\begin{EntryWithPhonetic}{儿}{er2}{2}{⼉}[Kangxi 10]
  \definition{adj.}{macho}
  \definition{s.}{criança | jovem; juventude | filho}
  \definition{suf.}{adicionado a substantivos para expressar pequenez  | adicionado a verbos, adjetivos e classificadores para formar substantivos | adicionado a substantivos para formar substantivos com significados diferentes | sufixos de alguns verbos | anexado após adjetivos duplicados}
  \seeref{r5}
\end{EntryWithPhonetic}

\begin{EntryWithPhonetic}{儿科}{er2 ke1}{2,9}{⼉,⽲}[HSK 6]
  \definition{s.}{(departamento de) pediatria | pediatria; o ramo da medicina que trata do desenvolvimento, cuidado e doença das crianças}
\end{EntryWithPhonetic}

\begin{EntryWithPhonetic}{儿女}{er2 nv3}{2,3}{⼉,⼥}[HSK 5]
  \definition{s.}{crianças; filhos e filhas | homem e mulher jovens (apaixonados)}
\end{EntryWithPhonetic}

\begin{EntryWithPhonetic}{儿童}{er2tong2}{2,12}{⼉,⽴}[HSK 4]
  \definition[个,群]{s.}{criança; menor de idade (mais jovem do que 少年)}
  \seealsoref{少年}{shao4 nian2}
\end{EntryWithPhonetic}

\begin{EntryWithPhonetic}{儿媳}{er2xi2}{2,13}{⼉,⼥}
  \definition{s.}{esposa do filho}
\end{EntryWithPhonetic}

\begin{EntryWithPhonetic}{儿子}{er2zi5}{2,3}{⼉,⼦}[HSK 1]
  \definition[个]{s.}{filho}
  \seealsoref{女儿}{nv3'er2}
\end{EntryWithPhonetic}

%%%%%%%%%% 而 %%%%%%%%%%
\subsection*{而}\addcontentsline{loh}{figure}{而 \dpy{er2}}

\begin{EntryWithPhonetic}{而}{er2}{6}{⽽}[HSK 4][Kangxi 126]
  \definition{conj.}{e (coordenação) | e ainda (restrição) | conexão de componentes com continuidade semântica | conexão de componentes afirmativos e negativos que se complementam | conexão de componentes com significados opostos para indicar um contraste | conexão de componentes de causa e efeito no raciocínio | significa ``chegar'' ou ``alcançar'' | conexão de componentes que indicam tempo ou modo ao verbo | inserido entre o sujeito e o predicado, significa 如果}
  \seealsoref{如果}{ru2guo3}
\end{EntryWithPhonetic}

\begin{EntryWithPhonetic}{而况}{er2kuang4}{6,7}{⽽,⼎}
  \definition{conj.}{além disso | além do mais}
\end{EntryWithPhonetic}

\begin{EntryWithPhonetic}{而且}{er2 qie3}{6,5}{⽽,⼀}[HSK 2]
  \definition{conj.}{e também; indica igualdade | e isso; não só\dots mas (também); indica um passo adiante}
\end{EntryWithPhonetic}

\begin{EntryWithPhonetic}{而是}{er2 shi4}{6,9}{⽽,⽇}[HSK 4]
  \definition{conj.}{mas; em vez disso; geralmente usada em conjunto com 不是 para formar o correlativo 不是……而是, indicando uma relação paralela}
  \seealsoref{不是…而是}{bu4shi4 er2 shi4}
\end{EntryWithPhonetic}

\begin{EntryWithPhonetic}{而已}{er2yi3}{6,3}{⽽,⼰}[HSK 7-9]
  \definition{part.}{isso é tudo; nada mais; usado no final de uma frase declarativa, geralmente é precedido por 不过 ou 只 para expressar que é exatamente assim (罢了)}
  \seealsoref{罢了}{ba4 le5}
  \seealsoref{不过}{bu2guo4}
  \seealsoref{只}{zhi3}
\end{EntryWithPhonetic}

%%%%%%%%%% 耳 %%%%%%%%%%
\subsection*{耳}\addcontentsline{loh}{figure}{耳 \dpy{er3}}

\begin{EntryWithPhonetic}{耳}{er3}{6}{⽿}[Kangxi 128]
  \definition*{s.}{Sobrenome: Er}
  \definition{part.}{(clássico) somente; apenas}
  \definition{s.}{orelha | coisa parecida com uma orelha | em ambos os lados; lado | orelha de um utensílio}
\end{EntryWithPhonetic}

\begin{EntryWithPhonetic}{耳朵}{er3duo5}{6,6}{⽿,⽊}[HSK 5]
  \definition[双,只,个,对]{s.}{orelha; ouvido; órgão da audição e do equilíbrio}
\end{EntryWithPhonetic}

\begin{EntryWithPhonetic}{耳光}{er3guang1}{6,6}{⽿,⼉}[HSK 7-9]
  \definition[个,记]{s.}{uma bofetada na orelha; um tapa na cara; (bater) no rosto em frente à orelha; a ação de bater no rosto}
\end{EntryWithPhonetic}

\begin{EntryWithPhonetic}{耳机}{er3 ji1}{6,6}{⽿,⽊}[HSK 4]
  \definition[副,个,对]{s.}{fone de ouvido; receptor (de telefone); dispositivos que permitem que uma pessoa ouça sons sozinha, como ouvir música, histórias, chamadas telefônicas etc., usados na cabeça ou inseridos nos ouvidos}
\end{EntryWithPhonetic}

\begin{EntryWithPhonetic}{耳目一新}{er3mu4-yi4xin1}{6,5,1,13}{⽿,⽬,⼀,⽄}[HSK 7-9]
  \definition{expr.}{encontrar tudo fresco e novo; encontrar-se em um mundo inteiramente novo; apresentar uma nova aparência (de um lugar); uma mudança agradável de atmosfera; ``Tudo o que ouço e vejo mudou e parece novo.''}
\end{EntryWithPhonetic}

\begin{EntryWithPhonetic}{耳熟能详}{er3shu2-neng2xiang2}{6,15,10,8}{⽿,⽕,⾁,⾔}[HSK 7-9]
  \definition{expr.}{o que é ouvido com frequência pode ser repetido em detalhes; já ouvi isso muitas vezes e estou familiarizado o suficiente para falar sobre isso em detalhes}
\end{EntryWithPhonetic}

\begin{EntryWithPhonetic}{耳闻目睹}{er3wen2-mu4du3}{6,9,5,13}{⽿,⾨,⽬,⽬}[HSK 7-9]
  \definition{expr.}{testemunhar pessoalmente; ver e ouvir pessoalmente; o que se vê e se ouve}
\end{EntryWithPhonetic}

%%%%%%%%%% 二 %%%%%%%%%%
\subsection*{二}\addcontentsline{loh}{figure}{二 \dpy{er4}}

\begin{EntryWithPhonetic}{二}{er4}{2}{⼆}[HSK 1][Kangxi 7]
  \definition{adj.}{diferente; refere"-se a duas coisas ou coisas diferentes | bobo; pateta; tolo; sem inteligência | desleal; infiel; indiferente; sem determinação}
  \definition{num.}{dois; 2}
  \seealsoref{两}{liang3}
\end{EntryWithPhonetic}

\begin{EntryWithPhonetic}{二胡}{er4hu2}{2,9}{⼆,⾁}
  \definition{s.}{erhu; um instrumento de arco de duas cordas com um registro mais baixo que o 京胡; um tipo de 胡琴, a caixa de som é feita de bambu, madeira, etc., coberta com pele de cobra, etc., tem duas cordas e o tom é baixo e suave}
  \seealsoref{胡琴}{hu2qin2}
  \seealsoref{京胡}{jing1hu2}
\end{EntryWithPhonetic}

\begin{EntryWithPhonetic}{二手}{er4 shou3}{2,4}{⼆,⼿}[HSK 4]
  \definition{adj.}{usado; de segunda mão; refere"-se especificamente a usados e revendidos}
\end{EntryWithPhonetic}

\begin{EntryWithPhonetic}{二手车}{er4shou3che1}{2,4,4}{⼆,⼿,⾞}[HSK 7-9]
  \definition{s.}{carro usado; carro de segunda mão}
\end{EntryWithPhonetic}

\begin{EntryWithPhonetic}{二维码}{er4 wei2 ma3}{2,11,8}{⼆,⽷,⽯}[HSK 5]
  \definition[个]{s.}{\emph{QR code}; um identificador gráfico que distribui formas geométricas específicas em um plano ou direção bidimensional de acordo com certas regras para expressar um conjunto de informações}
\end{EntryWithPhonetic}

\begin{EntryWithPhonetic}{二氧化碳}{er4yang3hua4tan4}{2,10,4,14}{⼆,⽓,⼔,⽯}[HSK 7-9]
  \definition{s.}{CO$_2$; dióxido de carbono; gás carbônico}
\end{EntryWithPhonetic}

\begin{EntryWithPhonetic}{二战}{er4zhan4}{2,9}{⼆,⼽}
  \definition*{s.}{Segunda Guerra Mundial (1039-1945)}
\end{EntryWithPhonetic}

%%%%% EOF %%%%%


 %%%
%%% F
%%%
\section*{F}\addcontentsline{toc}{section}{F}

%%%%%%%%%% 发 %%%%%%%%%%
\subsection*{发}

\begin{EntryWithPhonetic}{发}{fa1}{5}{⼜}[HSK 2]
  \definition*{s.}{Sobrenome: Fa}
  \definition{clas.}{bala, usada para munições e cartuchos}
  \definition{v.}{distribuir; enviar; entregar | emitir; disparar; lançar; descarregar | produzir; gerar; criar (dar origem a) | proferir; emitir; expressar | expandir; desenvolver | prosperar; prosperidade graças à aquisição de bens materiais | crescer ou expandir quando fermentado ou embebido | difundir; dispersar; espalhar | expor; descobrir; revelar | transformar-se; tornar-se; entrar em um determinado estado | demonstrar seus sentimentos; expressar (sentimentos) | sentir; ter um sentimento | começar; estabelecer | fazer com que se faça; iniciar um empreendimento; começar a agir; provocar uma ação}
  \seeref{fa4}
\end{EntryWithPhonetic}

\begin{EntryWithPhonetic}{发表}{fa1biao3}{5,8}{⼜、⾐}[HSK 3]
  \definition{v.}{publicar; entregar; emitir; expressar; anunciar; expressar (opiniões) ou divulgar (assuntos) ao público, verbalmente ou por escrito | publicar em jornais (artigos, etc.)}
\end{EntryWithPhonetic}

\begin{EntryWithPhonetic}{发病}{fa1 bing4}{5,10}{⼜、⽧}[HSK 6]
  \definition{v.}{(de uma doença) avanço | patogênese; morbidade | surto (de uma doença)}
\end{EntryWithPhonetic}

\begin{EntryWithPhonetic}{发布}{fa1bu4}{5,5}{⼜、⼱}[HSK 5]
  \definition{v.}{emitir; publicar; liberar; anunciar; fazer ordens públicas, anúncios, notícias, etc.}
\end{EntryWithPhonetic}

\begin{EntryWithPhonetic}{发布会}{fa1bu4hui4}{5,5,6}{⼜、⼱、⼈}[HSK 7-9]
  \definition[次]{s.}{coletiva de imprensa; um formato de conferência usado para divulgar notícias ou responder perguntas da mídia e do público | \emph{briefing}; atividades de exposição para promover novos produtos, etc.}
\end{EntryWithPhonetic}

\begin{EntryWithPhonetic}{发财}{fa1/cai2}{5,7}{⼜、⾙}[HSK 7-9]
  \definition{v.+compl.}{ficar rico; fazer fortuna; acumular fortuna; ganhar muito dinheiro ou propriedades | trabalhar; conseguir um emprego; uma maneira educada de perguntar a alguém onde ele trabalha é dizer onde ele fez fortuna}
\end{EntryWithPhonetic}

\begin{EntryWithPhonetic}{发愁}{fa1/chou2}{5,13}{⼜、⼼}[HSK 7-9]
  \definition{v.+compl.}{preocupar-se; ficar ansioso; ficar triste; sentir-se deprimido por não ter ideias ou soluções}
\end{EntryWithPhonetic}

\begin{EntryWithPhonetic}{发出}{fa1 chu1}{5,5}{⼜、⼐}[HSK 3]
  \definition{v.}{fazer; produzir; deixar sair; ocorrer (som, dúvida, etc.) | emitir; anunciar; publicar; divulgar (ordens, instruções) | enviar (mercadorias, cartas, etc.); partir (veículos, etc.) | emitir; exalar (cheiro, calor, etc.)}
\end{EntryWithPhonetic}

\begin{EntryWithPhonetic}{发达}{fa1da2}{5,6}{⼜、⾡}[HSK 3]
  \definition{adj.}{desenvolvido; florescente; (coisas) Já estão bem desenvolvidas; (negócios) prosperam}
  \definition{v.}{desenvolver; promover; florescer; a pessoa tem um bom desempenho profissional e é muito bem-sucedida}
\end{EntryWithPhonetic}

\begin{EntryWithPhonetic}{发电}{fa1 dian4}{5,5}{⼜、⽥}[HSK 6]
  \definition{s.}{geração de energia elétrica; produção de eletricidade; fornecimento de energia}
  \definition{v.}{gerar eletricidade (ou energia elétrica) | enviar um telegrama}
\end{EntryWithPhonetic}

\begin{EntryWithPhonetic}{发电机}{fa1dian4ji1}{5,5,6}{⼜、⽥、⽊}[HSK 7-9]
  \definition{s.}{gerador; dínamo | alternador; gerador elétrico}
\end{EntryWithPhonetic}

\begin{EntryWithPhonetic}{发动}{fa1dong4}{5,6}{⼜、⼒}[HSK 3]
  \definition{v.}{iniciar; começar; lançar | chamar à ação; mobilizar; despertar | ligar o motor; dar a partida; dar o pontapé inicial (motor de combustão interna) | estimular; colocar em ação}
\end{EntryWithPhonetic}

\begin{EntryWithPhonetic}{发动机}{fa1dong4ji1}{5,6,6}{⼜、⼒、⽊}
  \definition[台]{s.}{motor}
\end{EntryWithPhonetic}

\begin{EntryWithPhonetic}{发抖}{fa1dou3}{5,7}{⼜、⼿}[HSK 7-9]
  \definition{v.}{tremer; sacudir; estremecer; tremer devido ao medo, raiva ou frio}
\end{EntryWithPhonetic}

\begin{EntryWithPhonetic}{发放}{fa1 fang4}{5,8}{⼜、⽅}[HSK 6]
  \definition{v.}{conceder; estender; fornecer; (governo, organização) distribuir dinheiro ou suprimentos para os necessitados | emitir; enviar}
\end{EntryWithPhonetic}

\begin{EntryWithPhonetic}{发奋图强}{fa1fen4-tu2qiang2}{5,8,8,12}{⼜、⼤、⼞、⼸}[HSK 7-9]
  \definition{expr.}{fazer um esforço para se tornar forte (expressão idiomática); determinado a fazer melhor | arregaçar as mangas}
\end{EntryWithPhonetic}

\begin{EntryWithPhonetic}{发光}{fa1/guang1}{5,6}{⼜、⼉}[HSK 7-9]
  \definition{s.}{luminescência}
  \definition{v.+compl.}{emitir luz; brilhar; ser luminoso; cintilar}
\end{EntryWithPhonetic}

\begin{EntryWithPhonetic}{发挥}{fa1hui1}{5,9}{⼜、⼿}[HSK 4]
  \definition{v.}{colocar em jogo; dar jogo a; dar espaço a; dar rédea solta a; revelar a natureza ou a capacidade interior | expressar; desenvolver (uma ideia, um tema, etc.); elaborar; fazer valer o ponto ou o motivo}
\end{EntryWithPhonetic}

\begin{EntryWithPhonetic}{发火}{fa1/huo3}{5,4}{⼜、⽕}[HSK 7-9]
  \definition{v.+compl.}{ficar com raiva; explodir; ficar furioso; perder a paciência | detonar; explodir | inflamar; pegar fogo; acender; começar a queimar}
\end{EntryWithPhonetic}

\begin{EntryWithPhonetic}{发酵}{fa1/jiao4}{5,14}{⼜、⾣}[HSK 7-9]
  \definition{s.}{fermentação; zimólise; compostos orgânicos complexos são decompostos em substâncias mais simples sob a ação de microrganismos}
  \definition{v.+compl.}{fermentar}
\end{EntryWithPhonetic}

\begin{EntryWithPhonetic}{发觉}{fa1jue2}{5,9}{⼜、⾒}[HSK 5]
  \definition{v.}{vir a saber; estar ciente (de); perceber; tornar-se consciente | encontrar; detectar; perceber; descobrir}
\end{EntryWithPhonetic}

\begin{EntryWithPhonetic}{发掘}{fa1jue2}{5,11}{⼜、⼿}[HSK 7-9]
  \definition{v.}{explorar; escavar; desenterrar}
\end{EntryWithPhonetic}

\begin{EntryWithPhonetic}{发愣}{fa1/leng4}{5,12}{⼜、⼼}[HSK 7-9]
  \definition{v.+compl.}{olhar fixamente; estar em transe (ou atordoado)}
\end{EntryWithPhonetic}

\begin{EntryWithPhonetic}{发明}{fa1ming2}{5,8}{⼜、⽇}[HSK 3]
  \definition[个,项,种]{s.}{invenção; novos produtos ou métodos inventados}
  \definition{v.}{inventar; pesquisa que cria (novos produtos ou novos métodos) | expor; explicar; explicação criativa}
\end{EntryWithPhonetic}

\begin{EntryWithPhonetic}{发明者}{fa1ming2zhe3}{5,8,8}{⼜、⽇、⽼}
  \definition{s.}{inventor}
\end{EntryWithPhonetic}

\begin{EntryWithPhonetic}{发怒}{fa1 nu4}{5,9}{⼜、⼼}[HSK 6]
  \definition{v.}{ficar com raiva; explodir; perder a paciência | entrar em fúria | entrar em fúria (paixão)}
\end{EntryWithPhonetic}

\begin{EntryWithPhonetic}{发脾气}{fa1 pi2qi5}{5,12,4}{⼜、⾁、⽓}[HSK 7-9]
  \definition{v.}{ficar com raiva; perder a paciência; ficar furioso; fazer barulho ou xingar porque as coisas não saem do seu jeito}
\end{EntryWithPhonetic}

\begin{EntryWithPhonetic}{发票}{fa1piao4}{5,11}{⼜、⽰}[HSK 4]
  \definition[张,种]{s.}{conta; recibo; fatura; recibos emitidos por lojas ou outros escritórios de cobrança}
\end{EntryWithPhonetic}

\begin{EntryWithPhonetic}{发起}{fa1 qi3}{5,10}{⼜、⾛}[HSK 6]
  \definition{s.}{iniciador; patrocinador}
  \definition{v.}{iniciar; patrocinar; começar; lançar}
\end{EntryWithPhonetic}

\begin{EntryWithPhonetic}{发起人}{fa1qi3ren2}{5,10,2}{⼜、⾛、⼈}[HSK 7-9]
  \definition{s.}{iniciador; patrocinador | membro fundador; originadores; autores | propositor}
\end{EntryWithPhonetic}

\begin{EntryWithPhonetic}{发热}{fa1/re4}{5,10}{⼜、⽕}[HSK 7-9]
  \definition{pref.}{piro-}
  \definition{s.}{ebulição; febre; calor; pirexia}
  \definition{v.+compl.}{emitir calor; gerar calor; aquecer; esquentar | ter febre | ser cabeça quente}
\end{EntryWithPhonetic}

\begin{EntryWithPhonetic}{发烧}{fa1shao1}{5,10}{⼜、⽕}[HSK 4]
  \definition{v.}{ter febre; a temperatura corporal normal de uma pessoa é de cerca de 37ºC; se exceder 37,5ºC, é febre}
\end{EntryWithPhonetic}

\begin{EntryWithPhonetic}{发射}{fa1she4}{5,10}{⼜、⼨}[HSK 5]
  \definition{v.}{subir; disparar; lançar; irradiar; projetar; descarregar; enviar algo (como uma bala, um projétil, um satélite, etc.) de um dispositivo em uma velocidade muito alta}
\end{EntryWithPhonetic}

\begin{EntryWithPhonetic}{发生}{fa1sheng1}{5,5}{⼜、⽣}[HSK 3]
  \definition{v.}{ocorrer; acontecer; tomar lugar; surgir algo que não existia antes}
\end{EntryWithPhonetic}

\begin{EntryWithPhonetic}{发誓}{fa1/shi4}{5,14}{⼜、⾔}[HSK 7-9]
  \definition{v.+compl.}{jurar; prometer; fazer um juramento; expressar solenemente a resolução e a promessa de fazer o que foi acordado ou dito}
\end{EntryWithPhonetic}

\begin{EntryWithPhonetic}{发送}{fa1 song4}{5,9}{⼜、⾡}[HSK 3]
  \definition{v.}{enviar; despachar | transmitir (rádio)}
\end{EntryWithPhonetic}

\begin{EntryWithPhonetic}{发现}{fa1xian4}{5,8}{⼜、⾒}[HSK 2]
  \definition[个,项]{s.}{descoberta; achado}
  \definition{v.}{encontrar; descobrir; detectar; identificar; através de pesquisa, exploração, etc., ver ou encontrar coisas ou leis que os antepassados não viram | descobrir; perceber; perceber; notar; estar ciente de}
\end{EntryWithPhonetic}

\begin{EntryWithPhonetic}{发现者}{fa1xian4 zhe3}{5,8,8}{⼜、⾒、⽼}
  \definition{s.}{descobridor}
\end{EntryWithPhonetic}

\begin{EntryWithPhonetic}{发泄}{fa1xie4}{5,8}{⼜、⽔}[HSK 7-9]
  \definition{v.}{soltar; abreviar; dar vazão a; desabafar emoções ou desejos}
\end{EntryWithPhonetic}

\begin{EntryWithPhonetic}{发行}{fa1xing2}{5,6}{⼜、⾏}[HSK 5]
  \definition{v.}{emitir; liberar; publicar; emitir ou vender de publicações recém-impressas, moeda, selos, etc.}
\end{EntryWithPhonetic}

\begin{EntryWithPhonetic}{发言}{fa1/yan2}{5,7}{⼜、⾔}[HSK 3]
  \definition[个]{s.}{discurso; declaração; palestra; opiniões publicadas}
  \definition{v.+compl.}{falar; fazer uma declaração (discurso); expressar opinião (geralmente em reuniões)}
\end{EntryWithPhonetic}

\begin{EntryWithPhonetic}{发言人}{fa1 yan2 ren2}{5,7,2}{⼜、⾔、⼈}[HSK 6]
  \definition{s.}{porta-voz}
\end{EntryWithPhonetic}

\begin{EntryWithPhonetic}{发炎}{fa1yan2}{5,8}{⼜、⽕}[HSK 6]
  \definition{s.}{inflamação}
  \definition{v.}{irritar; inflamar; reação complexa de organismos a fatores patogênicos, como microrganismos, substâncias químicas e estímulos físicos; os sintomas sistêmicos incluem aumento da temperatura corporal, alterações na composição do sangue, vermelhidão local, inchaço, febre, dor, etc.}
\end{EntryWithPhonetic}

\begin{EntryWithPhonetic}{发扬}{fa1yang2}{5,6}{⼜、⼿}[HSK 7-9]
  \definition{v.}{desenvolver; continuar; levar adiante; desenvolver e promover (boas práticas, tradições, etc.) | aproveitar ao máximo; fazer uso total de; exercer ou mostrar (algum poder, habilidade, etc.) tanto quanto possível}
\end{EntryWithPhonetic}

\begin{EntryWithPhonetic}{发扬光大}{fa1yang2-guang1da4}{5,6,6,3}{⼜、⼿、⼉、⼤}[HSK 7-9]
  \definition{expr.}{levar adiante; desenvolver; aprimorar; fomentar e aprimorar; dar pleno uso a; dar maior escopo a; levar a um maior nível de desenvolvimento; desenvolver para um estágio mais alto; espalhar e florescer; ``O desenvolvimento e a promoção tornam-no cada vez mais grandioso.''}
\end{EntryWithPhonetic}

\begin{EntryWithPhonetic}{发音}{fa1yin1}{5,9}{⼜、⾳}
  \definition{s.}{pronúncia}
  \definition{v.}{pronunciar}
\end{EntryWithPhonetic}

\begin{EntryWithPhonetic}{发育}{fa1yu4}{5,8}{⼜、⾁}[HSK 7-9]
  \definition{s.}{crescimento}
  \definition{v.}{crescer; desenvolver; a estrutura e a função dos organismos evoluem do simples para o complexo ou do imaturo para o maduro}
\end{EntryWithPhonetic}

\begin{EntryWithPhonetic}{发源地}{fa1yuan2di4}{5,13,6}{⼜、⽔、⼟}[HSK 7-9]
  \definition{s.}{fonte; berço; lar; terra natal; lugar de origem; terra de origem | local de nascimento}[青藏高原是藏族文化的发源地。===O Planalto Qinghai-Tibete é o berço da cultura tibetana.]
\end{EntryWithPhonetic}

\begin{EntryWithPhonetic}{发展}{fa1zhan3}{5,10}{⼜、⼫}[HSK 3]
  \definition{v.}{crescer; expandir; avançar; desenvolver; a mudança das coisas de pequeno para grande, de simples para complexo, de inferior para superior | recrutar; admitir expandir (organização, escala, etc.)}
\end{EntryWithPhonetic}

\begin{EntryWithPhonetic}{发作}{fa1zuo4}{5,7}{⼜、⼈}[HSK 7-9]
  \definition{v.}{sair; mostrar efeito; a doença no corpo se manifesta repentinamente ou o álcool ou as drogas fazem efeito | explodir; ter um ataque de raiva; perder a paciência porque está muito zangado ou insatisfeito}
\end{EntryWithPhonetic}

%%%%%%%%%% 罚 %%%%%%%%%%
\subsection*{罚}

\begin{EntryWithPhonetic}{罚}{fa2}{9}{⽹}[HSK 5]
  \definition{s.}{punição; penalidade}
  \definition{v.}{punir; penalizar; multar; confiscar}
\end{EntryWithPhonetic}

\begin{EntryWithPhonetic}{罚款}{fa2/kuan3}{9,12}{⽹、⽋}[HSK 5]
  \definition[笔,次,宗]{s.}{multa; penalidade; refere-se ao dinheiro pago por uma pessoa ou entidade de acordo com as disposições de um delito ou violação de contrato ou contrato}
  \definition{v.+compl.}{multar; penalizar; exigir, de acordo com os regulamentos, uma determinada quantia de dinheiro de uma pessoa ou entidade que tenha violado a lei ou descumprido um regulamento ou contrato}
\end{EntryWithPhonetic}

%%%%%%%%%% 阀 %%%%%%%%%%
\subsection*{阀}

\begin{EntryWithPhonetic}{阀}{fa2}{9}{⾨}
  \definition[个]{s.}{casa estabelecida ou grupo de poder; uma pessoa ou família poderosa; refere-se a uma pessoa ou família que tem uma influência dominante em uma determinada área | válvula (mecânica)}
\end{EntryWithPhonetic}

\begin{EntryWithPhonetic}{阀门}{fa2men2}{9,3}{⾨、⾨}[HSK 7-9]
  \definition{s.}{válvula (mecânica); dispositivos para controlar o fluxo de água e ar em máquinas e tubulações}
\end{EntryWithPhonetic}

%%%%%%%%%% 筏 %%%%%%%%%%
\subsection*{筏}

\begin{EntryWithPhonetic}{筏}{fa2}{12}{⽵}
  \definition[条]{s.}{jangada (de troncos, bambus, etc.)}
\end{EntryWithPhonetic}

%%%%%%%%%% 法 %%%%%%%%%%
\subsection*{法}

\begin{EntryWithPhonetic}{法}{fa3}{8}{⽔}[HSK 4]
  \definition*{s.}{Doutrina budista; o dharma | França, abreviação de 法国 | Sobrenome: Fa}
  \definition{adj.}{(usado após advérbios negativos) legal; cumpridor da lei}
  \definition{clas.}{F; Farad, medida de capacitância}
  \definition{s.}{lei; termo geral para regras de comportamento estabelecidas ou endossadas pelo Estado | maneira; método; modo; meios | padrão; modelo | artes mágicas; feitiço}
  \definition{v.}{seguir; imitar; aprender (os pontos fortes dos outros) |}
  \seealsoref{法国}{fa3guo2}
\end{EntryWithPhonetic}

\begin{EntryWithPhonetic}{法官}{fa3 guan1}{8,8}{⽔、⼧}[HSK 4]
  \definition[位,名,个,些]{s.}{juiz; justiça; termo genérico para um membro do judiciário em um tribunal de justiça}
\end{EntryWithPhonetic}

\begin{EntryWithPhonetic}{法规}{fa3 gui1}{8,8}{⽔、⾒}[HSK 5]
  \definition[部,项,条,套,个]{s.}{lei e regulamento; estatuto; termo geral para leis, decretos, regulamentos, regras, estatutos, etc.}
\end{EntryWithPhonetic}

\begin{EntryWithPhonetic}{法国}{fa3guo2}{8,8}{⽔、⼞}
  \definition*{s.}{França}
\end{EntryWithPhonetic}

\begin{EntryWithPhonetic}{法国人}{fa3guo2ren2}{8,8,2}{⽔、⼞、⼈}
  \definition{s.}{francês | pessoa ou povo da França}
\end{EntryWithPhonetic}

\begin{EntryWithPhonetic}{法律}{fa3lv4}{8,9}{⽔、⼻}[HSK 4]
  \definition[项,条,套,个]{s.}{lei; estatuto; regras de conduta formuladas pelo legislativo e cuja aplicação é garantida pelo poder estatal}
\end{EntryWithPhonetic}

\begin{EntryWithPhonetic}{法庭}{fa3 ting2}{8,9}{⽔、⼴}[HSK 6]
  \definition{s.}{corte; tribunal | tribunal; um órgão estatal que exerce o poder judicial de forma independente}
\end{EntryWithPhonetic}

\begin{EntryWithPhonetic}{法网}{fa3wang3}{8,6}{⽔、⽹}
  \definition*{s.}{Torneio de Roland Garros (French Open), torneio de tênis}
\end{EntryWithPhonetic}

\begin{EntryWithPhonetic}{法文}{fa3wen2}{8,4}{⽔、⽂}
  \definition[份]{s.}{françês, língua francesa}
\end{EntryWithPhonetic}

\begin{EntryWithPhonetic}{法语}{fa3 yu3}{8,9}{⽔、⾔}[HSK 6]
  \definition[种,门,句,段]{s.}{françês, língua francesa}
\end{EntryWithPhonetic}

\begin{EntryWithPhonetic}{法院}{fa3yuan4}{8,9}{⽔、⾩}[HSK 4]
  \definition[所,座]{s.}{tribunal; corte; órgãos estatais que exercem poder judicial independente}
\end{EntryWithPhonetic}

\begin{EntryWithPhonetic}{法制}{fa3 zhi4}{8,8}{⽔、⼑}[HSK 5]
  \definition{s.}{legalidade; instituições jurídicas; sistema jurídico}
\end{EntryWithPhonetic}

%%%%%%%%%% 发 %%%%%%%%%%
\subsection*{发}

\begin{EntryWithPhonetic}{发}{fa4}{5}{⼜}
  \definition*{s.}{Sobrenome: Fa}
  \definition[件]{s.}{cabelo}
  \seeref{fa1}
\end{EntryWithPhonetic}

\begin{EntryWithPhonetic}{发型}{fa4xing2}{5,9}{⼜、⼟}[HSK 7-9]
  \definition[个,种]{s.}{penteado}
\end{EntryWithPhonetic}

\begin{EntryWithPhonetic}{发簪}{fa4zan1}{5,18}{⼜、⽵}
  \definition{s.}{grampo de cabelo}
\end{EntryWithPhonetic}

%%%%%%%%%% 帆 %%%%%%%%%%
\subsection*{帆}

\begin{EntryWithPhonetic}{帆}{fan1}{6}{⼱}[HSK 7-9]
  \definition{s.}{vela (de barco) | Literário: barco à vela; veleiro}
\end{EntryWithPhonetic}

\begin{EntryWithPhonetic}{帆船}{fan1chuan2}{6,11}{⼱、⾈}[HSK 7-9]
  \definition[艘,条]{s.}{veleiro; barco a vela; um navio que usa velas para se impulsionar com a ajuda do vento}
\end{EntryWithPhonetic}

%%%%%%%%%% 番 %%%%%%%%%%
\subsection*{番}

\begin{EntryWithPhonetic}{番}{fan1}{12}{⽥}[HSK 6]
  \definition{adj.}{estrangeiro; de tribos estrangeiras; estrangeiro ou alienígena}
  \definition{clas.}{usado para o número de vezes que uma ação é executada, equivalente a 回 ou 次 | usado para o tipo de coisas, equivalente a 种}
  \definition{s.}{estrangeiro; de tribos estrangeiras; (velho) refere-se a países estrangeiros ou raças estrangeiras | tomate; batata-doce | aborígenes; nativos; povos indígenas}
  \definition{v.}{revezar; rotacionar; substituir}
  \seealsoref{次}{ci4}
  \seealsoref{回}{hui2}
  \seealsoref{种}{zhong3}
\end{EntryWithPhonetic}

\begin{EntryWithPhonetic}{番茄}{fan1 qie2}{12,8}{⽥、⾋}[HSK 6]
  \definition[个,斤,磅,公斤]{s.}{tomate | tomateiro}
\end{EntryWithPhonetic}

%%%%%%%%%% 蕃 %%%%%%%%%%
\subsection*{蕃}

\begin{EntryWithPhonetic}{蕃}{fan1}{15}{⾋}
  \definition[种]{s.}{estrangeiros; aborígenes}
  \seeref{bo1}
  \seeref{fan2}
\end{EntryWithPhonetic}

\begin{EntryWithPhonetic}{蕃茄}{fan1 qie2}{15,8}{⾋、⾋}
  \variantof{番茄}
\end{EntryWithPhonetic}

%%%%%%%%%% 翻 %%%%%%%%%%
\subsection*{翻}

\begin{EntryWithPhonetic}{翻}{fan1}{18}{⽻}[HSK 4]
  \definition{v.}{virar; dar a volta; inverter; mudar de posição; torcer; reverter | vasculhar; procurar; pesquisar; mover objetos para localizar algo | reverter; retrair; retirar | passar por cima; ultrapassar; cruzar | multiplicar | traduzir; decodificar | romper-se; cair; desentender-se com alguém}
\end{EntryWithPhonetic}

\begin{EntryWithPhonetic}{翻番}{fan1/fan1}{18,12}{⽻、⽥}[HSK 7-9]
  \definition{v.+compl.}{aumentar em um número especificado de vezes; dobrar}
\end{EntryWithPhonetic}

\begin{EntryWithPhonetic}{翻过}{fan1guo4}{18,6}{⽻、⾡}
  \definition{v.}{virar |  transformar}
\end{EntryWithPhonetic}

\begin{EntryWithPhonetic}{翻来覆去}{fan1lai2-fu4qu4}{18,7,18,5}{⽻、⽊、⾑、⼛}[HSK 7-9]
  \definition{expr.}{jogar de um lado para o outro; tocar a mesma corda; repetir várias vezes; virar e se virar; dizer repetidamente; jogar inquieto de um lado para o outro; virar de um lado para o outro}
\end{EntryWithPhonetic}

\begin{EntryWithPhonetic}{翻脸}{fan1/lian3}{18,11}{⽻、⾁}
  \definition{v.+compl.}{brigar com alguém | tornar-se hostil}
\end{EntryWithPhonetic}

\begin{EntryWithPhonetic}{翻天覆地}{fan1tian1-fu4di4}{18,4,18,6}{⽻、⼤、⾑、⼟}[HSK 7-9]
  \definition{expr.}{virar o mundo de cabeça para baixo; uma mudança tremenda; abalar a terra; marcar época; virar o céu e a terra; sacudir o próprio chão (mundo); virar o mundo de cabeça para baixo; mudanças titânicas; ``Céu e terra virados de cabeça para baixo.''}
\end{EntryWithPhonetic}

\begin{EntryWithPhonetic}{翻译}{fan1yi4}{18,7}{⽻、⾔}[HSK 4]
  \definition[个,位,名]{s.}{tradutor; intérprete; pessoas que fazem trabalhos de tradução}
  \definition{v.}{traduzir; interpretar; colocar o significado de palavras de um idioma em palavras de outro idioma (expressão idiomática); expressar um significado em outro idioma}
\end{EntryWithPhonetic}

%%%%%%%%%% 凡 %%%%%%%%%%
\subsection*{凡}

\begin{EntryWithPhonetic}{凡}{fan2}{3}{⼏}[HSK 7-9]
  \definition*{s.}{Sobrenome: Fan}
  \definition{adj.}{comum; ordinário}
  \definition{adv.}{Literário: qualquer; todos; todo | Literário: em tudo; completamente}
  \definition{s.}{este mundo mortal; a terra | o mundo secular; refere-se ao mundo humano | uma nota da escala em Gongchepu (工尺谱), correspondente a 4 na notação musical numerada | Literário: ideia geral; esboço}
  \seealsoref{工尺谱}{gong1 che3 pu3}
\end{EntryWithPhonetic}

\begin{EntryWithPhonetic}{凡是}{fan2shi4}{3,9}{⼏、⽇}[HSK 6]
  \definition{adv.}{todos; qualquer; cada; resumir tudo dentro de um determinado âmbito}
\end{EntryWithPhonetic}

%%%%%%%%%% 烦 %%%%%%%%%%
\subsection*{烦}

\begin{EntryWithPhonetic}{烦}{fan2}{10}{⽕}[HSK 4]
  \definition{adj.}{redundante e confuso | supérfluo e confuso; muito bagunçado}
  \definition{v.}{aborrecer | irritar; incomodar; estar cansado de; ficar irritado | incomodar; solicitar}
\end{EntryWithPhonetic}

\begin{EntryWithPhonetic}{烦闷}{fan2men4}{10,7}{⽕、⾨}[HSK 7-9]
  \definition{adj.}{infeliz; deprimido; mal-humorado | desconfortável}
\end{EntryWithPhonetic}

\begin{EntryWithPhonetic}{烦恼}{fan2nao3}{10,9}{⽕、⼼}[HSK 7-9]
  \definition{adj.}{irritado; preocupado; incomodado}
  \definition[个,种,些]{s.}{aborrecimento; coisas que te incomodam}
\end{EntryWithPhonetic}

\begin{EntryWithPhonetic}{烦躁}{fan2zao4}{10,20}{⽕、⾜}[HSK 7-9]
  \definition{adj.}{inquieto; agitado; irritável}
\end{EntryWithPhonetic}

%%%%%%%%%% 蕃 %%%%%%%%%%
\subsection*{蕃}

\begin{EntryWithPhonetic}{蕃}{fan2}{15}{⾋}
  \definition{adj.}{exuberante; próspero}
  \definition{v.}{multiplicar; proliferar}
  \seeref{bo1}
  \seeref{fan1}
\end{EntryWithPhonetic}

%%%%%%%%%% 繁 %%%%%%%%%%
\subsection*{繁}

\begin{EntryWithPhonetic}{繁}{fan2}{17}{⽷}
  \definition{adj.}{em grande número; numerosos; múltiplos (oposto a 简) | em grande número; numerosos; complexos; complicado}
  \definition{v.}{propagar; multiplicar}
  \seealsoref{简}{jian3}
\end{EntryWithPhonetic}

\begin{EntryWithPhonetic}{繁花}{fan2hua1}{17,7}{⽷、⾋}
  \definition{s.}{Literário: flores desabrochadas; flores de cores diferentes; flores exuberantes; uma massa de flores}
\end{EntryWithPhonetic}

\begin{EntryWithPhonetic}{繁华}{fan2hua2}{17,6}{⽷、⼗}[HSK 7-9]
  \definition{adj.}{ocupado; agitado; próspero; florescente; (cidade, mercado de rua) movimentado e próspero}
\end{EntryWithPhonetic}

\begin{EntryWithPhonetic}{繁忙}{fan2mang2}{17,6}{⽷、⼼}[HSK 7-9]
  \definition{adj.}{ocupado; agitado; muita coisa para fazer, pouco tempo livre}
\end{EntryWithPhonetic}

\begin{EntryWithPhonetic}{繁荣}{fan2rong2}{17,9}{⽷、⾋}[HSK 5]
  \definition{adj.}{florescente; próspero}
  \definition{v.}{promover; prosperar}
\end{EntryWithPhonetic}

\begin{EntryWithPhonetic}{繁体字}{fan2ti3zi4}{17,7,6}{⽷、⼈、⼦}[HSK 7-9]
  \definition{s.}{forma complexa tradicional de um caractere chinês simplificado; caracteres chineses com mais traços que foram substituídos por caracteres simplificados; oposto a 简体字}
  \seealsoref{简体字}{jian3ti3zi4}
\end{EntryWithPhonetic}

\begin{EntryWithPhonetic}{繁殖}{fan2zhi2}{17,12}{⽷、⽍}[HSK 6]
  \definition{v.}{criar; reproduzir; propagar; multiplicar; os organismos produzem novos indivíduos}
\end{EntryWithPhonetic}

\begin{EntryWithPhonetic}{繁重}{fan2zhong4}{17,9}{⽷、⾥}[HSK 7-9]
  \definition{adj.}{pesado; oneroso; árduo; penoso; (trabalho, tarefas) muitas e pesadas}[我每天都有繁重的工作。===Tenho uma carga de trabalho pesada todos os dias.]
\end{EntryWithPhonetic}

%%%%%%%%%% 反 %%%%%%%%%%
\subsection*{反}

\begin{EntryWithPhonetic}{反}{fan3}{4}{⼜}[HSK 4]
  \definition{adj.}{oposto; contrário; invertido}
  \definition{adv.}{pelo contrário; inversamente}
  \definition{v.}{inverter o lado; de cabeça para baixo; na direção oposta | virar; converter | retornar | opor-se; combater; voltar-se contra | rebelar-se; revoltar-se | inferir; deduzir; raciocinar por analogia}
\end{EntryWithPhonetic}

\begin{EntryWithPhonetic}{反驳}{fan3bo2}{4,7}{⼜、⾺}[HSK 7-9]
  \definition{v.}{refutar; replicar; apresentar suas próprias razões para refutar teorias ou opiniões de outras pessoas que diferem das suas}
\end{EntryWithPhonetic}

\begin{EntryWithPhonetic}{反差}{fan3cha1}{4,9}{⼜、⼯}[HSK 7-9]
  \definition{s.}{contraste; o contraste de cores da foto ou do cenário é muito diferente, como o contraste entre o preto e o branco | discrepância; o contraste entre o bem e o mal, o alto e o baixo, etc. de pessoas ou coisas}
\end{EntryWithPhonetic}

\begin{EntryWithPhonetic}{反常}{fan3chang2}{4,11}{⼜、⼱}[HSK 7-9]
  \definition{adj.}{incomum; anormal; diferente da situação normal}
\end{EntryWithPhonetic}

\begin{EntryWithPhonetic}{反倒}{fan3dao4}{4,10}{⼜、⼈}[HSK 7-9]
  \definition{adv.}{em vez disso; pelo contrário}
  \definition{conj.}{em vez disso; pelo contrário; frequentemente acompanhadas por várias palavras que expressam negação}
\end{EntryWithPhonetic}

\begin{EntryWithPhonetic}{反对}{fan3dui4}{4,5}{⼜、⼨}[HSK 3]
  \definition{v.}{lutar; opor-se; objetar a; ser contra; discordar}
\end{EntryWithPhonetic}

\begin{EntryWithPhonetic}{反对党}{fan3dui4dang3}{4,5,10}{⼜、⼨、⼉}
  \definition{s.}{partido de oposição}
\end{EntryWithPhonetic}

\begin{EntryWithPhonetic}{反对派}{fan3dui4pai4}{4,5,9}{⼜、⼨、⽔}
  \definition{s.}{facção de oposição}
\end{EntryWithPhonetic}

\begin{EntryWithPhonetic}{反对票}{fan3dui4piao4}{4,5,11}{⼜、⼨、⽰}
  \definition{s.}{voto dissidente}
\end{EntryWithPhonetic}

\begin{EntryWithPhonetic}{反而}{fan3'er2}{4,6}{⼜、⽽}[HSK 4]
  \definition{adv.}{em vez disso; ao contrário de; contrário ao significado da frase anterior ou inesperado, desempenha o papel de uma reviravolta em uma frase}
\end{EntryWithPhonetic}

\begin{EntryWithPhonetic}{反复}{fan3fu4}{4,9}{⼜、⼢}[HSK 3]
  \definition{adv.}{repetidamente; de ​​novo e de novo; várias vezes}
  \definition{s.}{reversão; recaída; a situação anterior se repetiu}
  \definition{v.}{recuar; cortar e mudar; virar de cabeça para baixo; arrepender-se; aparecer várias vezes (usado principalmente em situações ruins)}
\end{EntryWithPhonetic}

\begin{EntryWithPhonetic}{反感}{fan3gan3}{4,13}{⼜、⼼}[HSK 7-9]
  \definition{adj.}{avesso; enojado; desgostoso; insatisfeito}
  \definition{s.}{antipatia; aversão; oposição ou insatisfação}
\end{EntryWithPhonetic}

\begin{EntryWithPhonetic}{反过来}{fan3 guo4lai2}{4,6,7}{⼜、⾡、⽊}[HSK 7-9]
  \definition{adv.}{inversamente; na ordem inversa; em direção oposta; indica uma reversão ou mudança de uma ação ou estado}
  \definition{conj.}{vice-versa; inversamente; na ordem inversa; ao contrário; usado para orientar relacionamentos de transição, como condições e causas}
  \definition{v.}{virar; virar para frente e para trás}
\end{EntryWithPhonetic}

\begin{EntryWithPhonetic}{反击}{fan3ji1}{4,5}{⼜、⼐}[HSK 7-9]
  \definition{v.}{revidar; responder fogo; contra-atacar}
\end{EntryWithPhonetic}

\begin{EntryWithPhonetic}{反抗}{fan3kang4}{4,7}{⼜、⼿}[HSK 6]
  \definition{s.}{resistência}
  \definition{v.}{revoltar-se; resistir; opor-se com ação}
\end{EntryWithPhonetic}

\begin{EntryWithPhonetic}{反馈}{fan3kui4}{4,12}{⼜、⾶}[HSK 7-9]
  \definition{s.}{\emph{feedback}; resposta; uma resposta ou reação a algo, informação, etc.}
  \definition{v.}{dar \emph{feedback}; enviar informações de volta; (informações, \emph{feedback}, etc.) retornar ao local de onde foi enviado}
\end{EntryWithPhonetic}

\begin{EntryWithPhonetic}{反面}{fan3mian4}{4,9}{⼜、⾯}[HSK 7-9]
  \definition{adj.}{oposto; negativo; ruim}
  \definition{s.}{costas; lado reverso; lado avesso; o lado de um objeto oposto à frente | o reverso de um estado de coisas, um problema, etc.; o outro lado de uma questão, problema, etc.}
\end{EntryWithPhonetic}

\begin{EntryWithPhonetic}{反思}{fan3si1}{4,9}{⼜、⼼}[HSK 7-9]
  \definition{v.}{refletir; introspectar; refletir sobre o passado e tirar lições dele}
\end{EntryWithPhonetic}

\begin{EntryWithPhonetic}{反弹}{fan3tan2}{4,11}{⼜、⼸}[HSK 7-9]
  \definition{s.}{rebote}
  \definition{v.}{recuperar; um objeto elástico retorna à sua forma original após ser deformado por uma força externa | rebater; saltar de volta; ressurgir; metáfora para recuperação de preço ou mercado | rebotar; quicar; ricochetear; um objeto em movimento salta na direção oposta quando encontra um obstáculo}
\end{EntryWithPhonetic}

\begin{EntryWithPhonetic}{反问}{fan3wen4}{4,6}{⼜、⾨}[HSK 6]
  \definition{v.}{fazer uma pergunta em resposta; responder a uma pergunta com outra pergunta | fazer uma pergunta retórica (uma pergunta com significado negativo)}
\end{EntryWithPhonetic}

\begin{EntryWithPhonetic}{反响}{fan3 xiang3}{4,9}{⼜、⼝}[HSK 6]
  \definition{s.}{eco; reverberação; repercusão}
\end{EntryWithPhonetic}

\begin{EntryWithPhonetic}{反省}{fan3xing3}{4,9}{⼜、⽬}[HSK 7-9]
  \definition{v.}{refletir sobre si mesmo; envolver-se em introspecção e autoexame; refletir sobre seus pensamentos e ações e examinar quaisquer erros; examinar a consciência; questionar-se; sondar a alma}
\end{EntryWithPhonetic}

\begin{EntryWithPhonetic}{反应}{fan3ying4}{4,7}{⼜、⼴}[HSK 3]
  \definition[个]{s.}{reação; resposta; opiniões, atitudes ou ações causadas pelo acontecimento}
  \definition{v.}{reagir; responder; atividade correspondente causada pela estimulação do organismo}
\end{EntryWithPhonetic}

\begin{EntryWithPhonetic}{反映}{fan3ying4}{4,9}{⼜、⽇}[HSK 4]
  \definition{s.}{reflexão; opiniões sobre pessoas ou situações}
  \definition{v.}{refletir; espelhar; figurativamente, trazer à tona a essência de uma questão objetiva (expressão idiomática); expressar a essência de algo objetivamente | relatar; tornar conhecido; informar às autoridades superiores | refletir; espelhar; a imagem de um objeto aparece invertida em outro objeto}
\end{EntryWithPhonetic}

\begin{EntryWithPhonetic}{反正}{fan3zheng4}{4,5}{⼜、⽌}[HSK 3]
  \definition{adv.}{de qualquer forma; de qualquer maneira; embora as circunstâncias sejam diferentes, o resultado é o mesmo | tudo igual; em qualquer caso; tom de voz que expressa afirmação categórica}
\end{EntryWithPhonetic}

%%%%%%%%%% 返 %%%%%%%%%%
\subsection*{返}

\begin{EntryWithPhonetic}{返}{fan3}{7}{⾡}
  \definition{v.}{retornar; vir ou voltar}
\end{EntryWithPhonetic}

\begin{EntryWithPhonetic}{返还}{fan3huan2}{7,7}{⾡、⾡}[HSK 7-9]
  \definition{s.}{remessa | restituição | devolução de algo ao seu dono original}
\end{EntryWithPhonetic}

\begin{EntryWithPhonetic}{返回}{fan3 hui2}{7,6}{⾡、⼞}[HSK 5]
  \definition{v.}{retornar; ir (voltar); reverter; recorrer; retroceder; voltar para (o lugar original)}
\end{EntryWithPhonetic}

%%%%%%%%%% 犯 %%%%%%%%%%
\subsection*{犯}

\begin{EntryWithPhonetic}{犯}{fan4}{5}{⽝}[HSK 6]
  \definition{s.}{criminoso}
  \definition{v.}{ofender; violar; ir contra | atacar; violar; trabalhar contra | fazer; ocorrer | voltar a; ter uma recorrência de; recair; retornar a (velhos hábitos)}
\end{EntryWithPhonetic}

\begin{EntryWithPhonetic}{犯愁}{fan4/chou2}{5,13}{⽝、⼼}[HSK 7-9]
  \definition{v.+compl.}{preocupar-se; estar ansioso}
\end{EntryWithPhonetic}

\begin{EntryWithPhonetic}{犯法}{fan4fa3}{5,8}{⽝、⽔}
  \definition{v.}{violar (quebrar) a lei}
\end{EntryWithPhonetic}

\begin{EntryWithPhonetic}{犯规}{fan4 gui1}{5,8}{⽝、⾒}[HSK 6]
  \definition{v.}{quebrar as regras; violar regras | Esporte: cometer uma falta contra}
\end{EntryWithPhonetic}

\begin{EntryWithPhonetic}{犯罪}{fan4/zui4}{5,13}{⽝、⽹}[HSK 6]
  \definition{v.+compl.}{cometer  um crime}
\end{EntryWithPhonetic}

%%%%%%%%%% 泛 %%%%%%%%%%
\subsection*{泛}

\begin{EntryWithPhonetic}{泛}{fan4}{7}{⽔}
  \definition{adj.}{extenso; amplo; geral; inespecífico | superficial; raso | amarelo (ficar amarelo)}
  \definition{v.}{Literário: flutuar; derivar | afastar-se; espalhar-se; transbordar | transbordar; inundar | emergir; sair}
\end{EntryWithPhonetic}

\begin{EntryWithPhonetic}{泛滥}{fan4lan4}{7,13}{⽔、⽔}[HSK 7-9]
  \definition{v.}{fluir; transbordar; inundar | espalhar sem controle; metáfora para coisas ruins se tornarem populares sem restrições}
\end{EntryWithPhonetic}

%%%%%%%%%% 饭 %%%%%%%%%%
\subsection*{饭}

\begin{EntryWithPhonetic}{饭}{fan4}{7}{⾷}[HSK 1]
  \definition{s.}{(empréstimo linguístico) fã, devoto}
  \definition[顿,份,碗,口,锅]{s.}{cereais cozidos; grãos cozidos | refeição; alimentos consumidos diariamente em horários regulares | trabalho; meio de subsistência; meio de vida}
\end{EntryWithPhonetic}

\begin{EntryWithPhonetic}{饭店}{fan4dian4}{7,8}{⾷、⼴}[HSK 1]
  \definition[家,个]{s.}{restaurante | hotel; hotel grande e bem equipado}
\end{EntryWithPhonetic}

\begin{EntryWithPhonetic}{饭馆}{fan4 guan3}{7,11}{⾷、⾷}[HSK 2]
  \definition[家,个]{s.}{restaurante; lanchonete}
\end{EntryWithPhonetic}

\begin{EntryWithPhonetic}{饭碗}{fan4wan3}{7,13}{⾷、⽯}[HSK 7-9]
  \definition[个,只]{s.}{tigela de arroz | Coloquial: emprego; meio de subsistência | Figurativo: meio de subsistência; maneira de ganhar a vida}
\end{EntryWithPhonetic}

%%%%%%%%%% 贩 %%%%%%%%%%
\subsection*{贩}

\begin{EntryWithPhonetic}{贩}{fan4}{8}{⾙}
  \definition[个]{s.}{comerciante; mascate; negociante; vendedor ambulante}
  \definition{v.}{(comerciantes) comprar para revender}
\end{EntryWithPhonetic}

\begin{EntryWithPhonetic}{贩卖}{fan4mai4}{8,8}{⾙、⼗}[HSK 7-9]
  \definition{v.}{vender; traficar}
\end{EntryWithPhonetic}

%%%%%%%%%% 范 %%%%%%%%%%
\subsection*{范}

\begin{EntryWithPhonetic}{范}{fan4}{9}{⾋}
  \definition*{s.}{Sobrenome: Fan}
  \definition{s.}{padrão; molde; matriz | modelo; exemplo; modelo a seguir | limites; escopo | restrição; limite}
\end{EntryWithPhonetic}

\begin{EntryWithPhonetic}{范成大}{fan4 cheng2da4}{9,6,3}{⾋、⼽、⼤}
  \definition*{s.}{Fan Chengda (1126–1193), de nome de cortesia Zhineng 致能 e também Youyuan 幼元, autodenominou-se Cishan Jushi 此山 居士em seus primeiros anos e Shihu Jushi 石湖 居士em seus últimos anos; natural do Condado de Wu, Suzhou (atual Cidade de Suzhou, Província de Jiangsu), foi um oficial, poeta e escritor durante a Dinastia Song do Sul; seu nome póstumo foi Wenmu 文穆}
\end{EntryWithPhonetic}

\begin{EntryWithPhonetic}{范畴}{fan4chou2}{9,12}{⾋、⽥}[HSK 7-9]
  \definition{s.}{tipo; domínio; escopo; alcance; categoria}
\end{EntryWithPhonetic}

\begin{EntryWithPhonetic}{范围}{fan4wei2}{9,7}{⾋、⼞}[HSK 3]
  \definition[个]{s.}{escopo; limite; alcance}
  \definition{v.}{estabelecer limites para; limitar o escopo de}
\end{EntryWithPhonetic}

%%%%%%%%%% 方 %%%%%%%%%%
\subsection*{方}

\begin{EntryWithPhonetic}{方}{fang1}{4}{⽅}[HSK 4][Kangxi 70]
  \definition*{s.}{Alquimia, 方术 | Sobrenome: Fang}
  \definition{adj.}{reto; honesto; imparcial}
  \definition{adv.}{exatamente quando; no momento em que}
  \definition{clas.}{usado para coisas quadradas | quadrado ou cúbico (geralmente metro quadrado ou cúbico)}
  \definition[个,张]{s.}{quadrado; um quadrado ou sólido com seis faces quadradas | matemática: potência; o número de vezes que uma quantidade deve ser multiplicada por si mesma | direção | lado; festa | lugar; região; localidade | maneira; método; solução | prescrição | lei; regra}
  \seealsoref{方术}{fang1 shu4}
\end{EntryWithPhonetic}

\begin{EntryWithPhonetic}{方案}{fang1'an4}{4,10}{⽅、⽊}[HSK 4]
  \definition[个,些,种]{s.}{plano; esquema; programa; planos específicos para tratar de um determinado problema | o esquema criado pelo governo; medidas ou regulamentações formuladas e implementadas pelo governo ou autoridades relevantes}
\end{EntryWithPhonetic}

\begin{EntryWithPhonetic}{方便}{fang1bian4}{4,9}{⽅、⼈}[HSK 2]
  \definition{adj.}{conveniente; sem complicações; sem dificuldades; muito fácil| adequado; condições ou circunstâncias adequadas}
  \definition{s.}{conveniência}
  \definition{v.}{ir ao banheiro; uma maneira delicada de dizer ``ir ao banheiro'' | facilitar; tornar algo conveniente para alguém; facilitar a realização de tarefas ou o alcance de objetivos | ter dinheiro sobrando}
\end{EntryWithPhonetic}

\begin{EntryWithPhonetic}{方便面}{fang1 bian4 mian4}{4,9,9}{⽅、⼈、⾯}[HSK 2]
  \definition[袋,包,碗,桶]{s.}{macarrão instantâneo}
\end{EntryWithPhonetic}

\begin{EntryWithPhonetic}{方法}{fang1fa3}{4,8}{⽅、⽔}[HSK 2]
  \definition[种,个,套,类]{s.}{método; meio; maneira; sobre os meios e procedimentos para resolver questões relacionadas com o pensamento, a fala e as ações, etc.}
\end{EntryWithPhonetic}

\begin{EntryWithPhonetic}{方方面面}{fang1fang1mian4mian4}{4,4,9,9}{⽅、⽅、⾯、⾯}[HSK 7-9]
  \definition{expr.}{todos os aspectos; todos os lados | multifacetado}
\end{EntryWithPhonetic}

\begin{EntryWithPhonetic}{方面}{fang1mian4}{4,9}{⽅、⾯}[HSK 2]
  \definition[个,种]{s.}{lado; campo; aspecto; respeito}
\end{EntryWithPhonetic}

\begin{EntryWithPhonetic}{方片}{fang1 pian4}{4,4}{⽅、⽚}
  \definition{s.}{ouros ♦ (em jogos de cartas)}
  \seealsoref{黑桃}{hei1 tao2}
  \seealsoref{红心}{hong2 xin1}
  \seealsoref{梅花}{mei2 hua1}
\end{EntryWithPhonetic}

\begin{EntryWithPhonetic}{方式}{fang1shi4}{4,6}{⽅、⼷}[HSK 3]
  \definition[种,个]{s.}{maneira; método}
\end{EntryWithPhonetic}

\begin{EntryWithPhonetic}{方术}{fang1 shu4}{4,5}{⽅、⽊}
  \definition{s.}{artes de cura, adivinhação, horóscopo etc. | Arcaico: artes sobrenaturais}
\end{EntryWithPhonetic}

\begin{EntryWithPhonetic}{方向}{fang1xiang4}{4,6}{⽅、⼝}[HSK 2]
  \definition[个,种]{s.}{direção; orientação; referindo-se a leste, sul, oeste, norte, sudeste, sudoeste, nordeste, noroeste, etc. | objetivo; meta; finalidade}
\end{EntryWithPhonetic}

\begin{EntryWithPhonetic}{方向盘}{fang1xiang4pan2}{4,6,11}{⽅、⼝、⽫}[HSK 7-9]
  \definition[个]{s.}{volante; um dispositivo em forma de roda para controlar a direção de viagem de navios, carros, etc.}
\end{EntryWithPhonetic}

\begin{EntryWithPhonetic}{方言}{fang1yan2}{4,7}{⽅、⾔}[HSK 7-9]
  \definition*{s.}{O primeiro Dicionário de Dialeto Chinês, editado por Yang Xiong, 扬雄, no século I, contendo mais de 9.000 caracteres}
  \definition[口]{s.}{dialeto; um ramo regional de uma língua, formado durante sua evolução, que difere da língua padrão e é usado apenas em uma determinada área}
  \seealsoref{扬雄}{yang2xiong2}
\end{EntryWithPhonetic}

\begin{EntryWithPhonetic}{方针}{fang1zhen1}{4,7}{⽅、⾦}[HSK 4]
  \definition[个,项]{s.}{política; diretriz; princípio orientador; orientação da direção e das metas de um empreendimento}
\end{EntryWithPhonetic}

%%%%%%%%%% 防 %%%%%%%%%%
\subsection*{防}

\begin{EntryWithPhonetic}{防}{fang2}{6}{⾩}[HSK 3]
  \definition*{s.}{Sobrenome: Fang}
  \definition{s.}{defesa | dique; aterro | barragem; represa; estrutura para conter a água}
  \definition{v.}{proteger contra; prevenir contra; tomar precauções contra | defender-se contra}
\end{EntryWithPhonetic}

\begin{EntryWithPhonetic}{防盗}{fang2dao4}{6,11}{⾩、⽫}[HSK 7-9]
  \definition{v.}{proteger-se contra roubos; tomar precauções contra ladrões; impedir que bandidos roubem}
\end{EntryWithPhonetic}

\begin{EntryWithPhonetic}{防盗门}{fang2dao4men2}{6,11,3}{⾩、⽫、⾨}[HSK 7-9]
  \definition{s.}{porta de segurança; equipado com trava antirroubo, ela resiste à abertura anormal sob certas condições e por um período determinado | porta à prova de roubo}
\end{EntryWithPhonetic}

\begin{EntryWithPhonetic}{防范}{fang2 fan4}{6,9}{⾩、⾋}[HSK 6]
  \definition{v.}{vigiar; estar em guarda; ficar de olho}
\end{EntryWithPhonetic}

\begin{EntryWithPhonetic}{防护}{fang2hu4}{6,7}{⾩、⼿}[HSK 7-9]
  \definition{s.}{abrigo; proteção; meios ou medidas para proteger pessoas, coisas, o meio ambiente, etc. de adoecer de receber danos ou destruição}
  \definition{v.}{abrigar; proteger; proteger pessoas, coisas e o meio ambiente de doenças, danos ou destruição}
\end{EntryWithPhonetic}

\begin{EntryWithPhonetic}{防火墙}{fang2huo3qiang2}{6,4,14}{⾩、⽕、⼟}[HSK 7-9]
  \definition[个,堵]{s.}{\emph{firewall} (Internet)}
\end{EntryWithPhonetic}

\begin{EntryWithPhonetic}{防晒}{fang2shai4}{6,10}{⾩、⽇}
  \definition{s.}{protetor solar}
\end{EntryWithPhonetic}

\begin{EntryWithPhonetic}{防守}{fang2shou3}{6,6}{⾩、⼧}[HSK 6]
  \definition{v.}{defender; guardar}
\end{EntryWithPhonetic}

\begin{EntryWithPhonetic}{防卫}{fang2wei4}{6,3}{⾩、⼙}[HSK 7-9]
  \definition{v.}{defender}
\end{EntryWithPhonetic}

\begin{EntryWithPhonetic}{防汛}{fang2xun4}{6,6}{⾩、⽔}[HSK 7-9]
  \definition{s.}{prevenção ou controle de inundações | controle de enchentes}
\end{EntryWithPhonetic}

\begin{EntryWithPhonetic}{防疫}{fang2yi4}{6,9}{⾩、⽧}[HSK 7-9]
  \definition{s.}{prevenção de epidemias | prevenção de doenças | proteção contra epidemias}
\end{EntryWithPhonetic}

\begin{EntryWithPhonetic}{防御}{fang2yu4}{6,12}{⾩、⼻}[HSK 7-9]
  \definition{v.}{guardar; defender; resistir ao ataque do inimigo}
\end{EntryWithPhonetic}

\begin{EntryWithPhonetic}{防止}{fang2zhi3}{6,4}{⾩、⽌}[HSK 3]
  \definition{v.}{evitar; prevenir; prevenir; proteger contra; preparar-se com antecedência para evitar que coisas ruins aconteçam}
\end{EntryWithPhonetic}

\begin{EntryWithPhonetic}{防治}{fang2zhi4}{6,8}{⾩、⽔}[HSK 5]
  \definition{s.}{tratamento preventivo; prevenção e cura; profilaxia e tratamento}
\end{EntryWithPhonetic}

%%%%%%%%%% 妨 %%%%%%%%%%
\subsection*{妨}

\begin{EntryWithPhonetic}{妨}{fang2}{7}{⼥}
  \definition{v.}{dificultar; entravar; impedir; obstruir | (no negativo ou interrogativo) prejudicar | interferir com}
\end{EntryWithPhonetic}

\begin{EntryWithPhonetic}{妨碍}{fang2'ai4}{7,13}{⼥、⽯}[HSK 7-9]
  \definition{v.}{dificultar; entravar; impedir; obstruir; impedir que as coisas corram bem}
\end{EntryWithPhonetic}

\begin{EntryWithPhonetic}{妨害}{fang2hai4}{7,10}{⼥、⼧}[HSK 7-9]
  \definition{v.}{prejudicar; pôr em risco; ser prejudicial a; causar dano; ferir}
\end{EntryWithPhonetic}

%%%%%%%%%% 房 %%%%%%%%%%
\subsection*{房}

\begin{EntryWithPhonetic}{房}{fang2}{8}{⼾}
  \definition*{s.}{Fang, a quarta das vinte e oito constelações nas quais a esfera celeste foi dividida, consistindo de quatro estrelas quase em linha reta em Escorpião | Sobrenome: Fang}
  \definition[幢,个,间]{s.}{casa; edifício | sala; quarto; câmara | estrutura semelhante a uma casa | um ramo de uma família extensa | loja; estoque | local de trabalho do artesão; oficina; moinho}
\end{EntryWithPhonetic}

\begin{EntryWithPhonetic}{房地产}{fang2di4chan3}{8,6,6}{⼾、⼟、⼇}[HSK 7-9]
  \definition{s.}{imóveis; um termo geral para imóveis e terrenos}
\end{EntryWithPhonetic}

\begin{EntryWithPhonetic}{房东}{fang2dong1}{8,5}{⼾、⼀}[HSK 3]
  \definition[个,位,名]{s.}{dono;  proprietário; senhorio; pessoas que alugam ou emprestam imóveis (para os 房客 )}
  \seealsoref{房客}{fang2ke4}
\end{EntryWithPhonetic}

\begin{EntryWithPhonetic}{房价}{fang2 jia4}{8,6}{⼾、⼈}[HSK 6]
  \definition{s.}{custo de moradia; tarifa de quarto | preço da casa}
\end{EntryWithPhonetic}

\begin{EntryWithPhonetic}{房间}{fang2jian1}{8,7}{⼾、⾨}[HSK 1]
  \definition[个,间,套]{s.}{sala; câmara; escritório; apartamento; divisões internas da casa}
\end{EntryWithPhonetic}

\begin{EntryWithPhonetic}{房客}{fang2ke4}{8,9}{⼾、⼧}[HSK 3]
  \definition{s.}{inquilino (de um quarto ou casa); hóspede (oposto a 房东) | inquilino; hóspede; pessoas que alugam ou emprestam imóveis para moradia (para o 房东)}
  \seealsoref{房东}{fang2dong1}
\end{EntryWithPhonetic}

\begin{EntryWithPhonetic}{房屋}{fang2 wu1}{8,9}{⼾、⼫}[HSK 3]
  \definition[间,所,套]{s.}{casas; habitação; edifícios}
\end{EntryWithPhonetic}

\begin{EntryWithPhonetic}{房主}{fang2zhu3}{8,5}{⼾、⼂}
  \definition{s.}{proprietário | dono de um imóvel}
\end{EntryWithPhonetic}

\begin{EntryWithPhonetic}{房子}{fang2 zi5}{8,3}{⼾、⼦}[HSK 1]
  \definition[栋,幢,座,套,间]{s.}{casa; edifício; prédio}
\end{EntryWithPhonetic}

\begin{EntryWithPhonetic}{房租}{fang2 zu1}{8,10}{⼾、⽲}[HSK 3]
  \definition[笔]{s.}{aluguel}
\end{EntryWithPhonetic}

%%%%%%%%%% 仿 %%%%%%%%%%
\subsection*{仿}

\begin{EntryWithPhonetic}{仿}{fang3}{6}{⼈}[HSK 7-9]
  \definition{adv.}{semelhante; como}
  \definition{s.}{caracteres escritos segundo um modelo de caligrafia | cartas modeladas a partir de uma cópia; palavras escritas de acordo com o modelo}
  \definition{v.}{imitar; copiar | assemelhar-se; ser como}
\end{EntryWithPhonetic}

\begin{EntryWithPhonetic}{仿佛}{fang3fu2}{6,7}{⼈、⼈}[HSK 6]
  \definition{adv.}{parece que; como se}
  \definition{v.}{ser como; parecer}
\end{EntryWithPhonetic}

\begin{EntryWithPhonetic}{仿制}{fang3zhi4}{6,8}{⼈、⼑}[HSK 7-9]
  \definition{v.}{copiar; imitar; ser modelado em}
\end{EntryWithPhonetic}

%%%%%%%%%% 访 %%%%%%%%%%
\subsection*{访}

\begin{EntryWithPhonetic}{访}{fang3}{6}{⾔}
  \definition{v.}{visitar; fazer uma visita; ligar para | procurar por meio de investigação ou busca; tentar obter; obter uma entrevista | entrevistar | investigar; procurar por meio de investigação (pesquisar)}
\end{EntryWithPhonetic}

\begin{EntryWithPhonetic}{访谈}{fang3tan2}{6,10}{⾔、⾔}[HSK 7-9]
  \definition{v.}{entrevistar; conversar; visitar e conversar}
\end{EntryWithPhonetic}

\begin{EntryWithPhonetic}{访问}{fang3wen4}{6,6}{⾔、⾨}[HSK 3]
  \definition{v.}{visitar; ligar; entrevistar; visitar e conversar com um objetivo específico | visitar um \emph{site}}
\end{EntryWithPhonetic}

%%%%%%%%%% 纺 %%%%%%%%%%
\subsection*{纺}

\begin{EntryWithPhonetic}{纺}{fang3}{7}{⽷}
  \definition{s.}{tecido de seda fina | um pano de seda fino}
  \definition{v.}{girar | enrolar}
\end{EntryWithPhonetic}

\begin{EntryWithPhonetic}{纺织}{fang3zhi1}{7,8}{⽷、⽷}[HSK 7-9]
  \definition{v.}{tecer; fiar; fiar algodão, linho, seda, lã e outras fibras em fios ou linhas e tecê-los em tecidos, cetim, lã, etc.}
\end{EntryWithPhonetic}

%%%%%%%%%% 放 %%%%%%%%%%
\subsection*{放}

\begin{EntryWithPhonetic}{放}{fang4}{8}{⽅}[HSK 1]
  \definition{v.}{deixar ir; libertar; soltar | ceder; deixar-se levar | levar para se alimentar; pastar | soltar; liberar (ou expelir) | exibir (um filme, etc.); reproduzir (um disco, etc.) | acender; inflamar | emprestar (dinheiro) com juros | tornar maior ou mais longo; soltar; abaixar | moderar (a atitude ou o comportamento de alguém) | (de flores) florescer; abrir | colocar; posicionar; deitar | fazer com que algo (ou alguém) caia no chão | deixar de lado; guardar (para uso futuro); conservar | (seguido por 着\dots 不\dots) permitir que algo permaneça (por fazer, por pegar, por usar, etc.) | adicionar; colocar | colocar em pastagem; soltar para caçar | deixar de lado; suspender; interromper | remover; aliviar; livrar-se; proteger; libertar | deixar-se levar; sem restrições; libertino | mandar embora; tirar o prisioneiro da prisão e deportá-lo para uma região remota | distribuir; emitir; lançar | atear fogo | expandir; ampliar; prolongar | reajustar-se até certo ponto; controlar suas ações, adotar uma determinada atitude, atingir um certo equilíbrio | derrubar}
\end{EntryWithPhonetic}

\begin{EntryWithPhonetic}{放鞭炮}{fang4bian1pao4}{8,18,9}{⽅、⾰、⽕}
  \definition{s.}{um conjunto de bombinhas ou traques}
\end{EntryWithPhonetic}

\begin{EntryWithPhonetic}{放出}{fang4chu1}{8,5}{⽅、⼐}
  \definition{v.}{liberar | libertar}
\end{EntryWithPhonetic}

\begin{EntryWithPhonetic}{放大}{fang4da4}{8,3}{⽅、⼤}[HSK 5]
  \definition{v.}{amplificar; magnificar; aumentar; ampliar; aumentar o tamanho de imagens, textos, sons, etc.}
\end{EntryWithPhonetic}

\begin{EntryWithPhonetic}{放到}{fang4 dao4}{8,8}{⽅、⼑}[HSK 3]
  \definition{v.}{colocar em; meter}
\end{EntryWithPhonetic}

\begin{EntryWithPhonetic}{放电}{fang4dian4}{8,5}{⽅、⽥}
  \definition{s.}{descarga elétrica}
\end{EntryWithPhonetic}

\begin{EntryWithPhonetic}{放飞}{fang4fei1}{8,3}{⽅、⾶}
  \definition{s.}{deixar voar}
\end{EntryWithPhonetic}

\begin{EntryWithPhonetic}{放过}{fang4guo4}{8,6}{⽅、⾡}[HSK 7-9]
  \definition{v.}{deixar escapar; perder}
\end{EntryWithPhonetic}

\begin{EntryWithPhonetic}{放假}{fang4/jia4}{8,11}{⽅、⼈}[HSK 1]
  \definition{v.}{tirar férias (ou feriado); ter um dia de folga}
  \definition{v.+compl.}{tirar férias (ou feriado); começar as férias; ter um dia de folga; estar de férias (feriado)}
\end{EntryWithPhonetic}

\begin{EntryWithPhonetic}{放弃}{fang4qi4}{8,7}{⽅、⼶}[HSK 5]
  \definition{v.}{desistir, abandonar; descartar (direitos originais, reivindicações, opiniões, etc.)}
\end{EntryWithPhonetic}

\begin{EntryWithPhonetic}{放弃权利}{fang4qi4 quan2li4}{8,7,6,7}{⽅、⼶、⽊、⼑}
  \definition{s.}{renúncia}
\end{EntryWithPhonetic}

\begin{EntryWithPhonetic}{放弃者}{fang4qi4zhe3}{8,7,8}{⽅、⼶、⽼}
  \definition{s.}{desistente}
\end{EntryWithPhonetic}

\begin{EntryWithPhonetic}{放任}{fang4ren4}{8,6}{⽅、⼈}
  \definition{v.}{ignorar | saciar-se | deixar sozinho}
\end{EntryWithPhonetic}

\begin{EntryWithPhonetic}{放水}{fang4/shui3}{8,4}{⽅、⽔}[HSK 7-9]
  \definition{v.+compl.}{ligar a água; deixar a água fluir, geralmente significa abrir a fonte de água ou fornecer uma determinada vazão de água | (reservatório, etc.) retirar água; drenar água de reservatórios, lagoas, etc. para irrigação ou outros fins | (em uma competição, etc.) facilitar as coisas para alguém; perder um jogo intencionalmente; deixar deliberadamente o adversário vencer facilmente durante uma partida}
\end{EntryWithPhonetic}

\begin{EntryWithPhonetic}{放肆}{fang4si4}{8,13}{⽅、⾀}[HSK 7-9]
  \definition{adj.}{desenfreado; devasso; atrevido; descontrolado; descreve agir de forma imprudente e sem escrúpulos}
\end{EntryWithPhonetic}

\begin{EntryWithPhonetic}{放松}{fang4song1}{8,8}{⽅、⽊}[HSK 4]
  \definition{v.}{relaxar; afrouxar; soltar; desprender}
\end{EntryWithPhonetic}

\begin{EntryWithPhonetic}{放下}{fang4 xia4}{8,3}{⽅、⼀}[HSK 2]
  \definition{v.}{deitar-se; colocar no chão| deixar ir; soltar; desistir; largar | colocar; acomodar; depositar}
\end{EntryWithPhonetic}

\begin{EntryWithPhonetic}{放心}{fang4xin1}{8,4}{⽅、⼼}[HSK 2]
  \definition{adj.}{despreocupado}
  \definition{v.}{confiar; ter confiança em alguém; sentir-se aliviado; ficar tranquilo; ficar com a consciência tranquila}
\end{EntryWithPhonetic}

\begin{EntryWithPhonetic}{放学}{fang4/xue2}{8,8}{⽅、⼦}[HSK 1]
  \definition{v.+compl.}{encerrar; sair da escola; as aulas terminaram; a escola acabou (por hoje); voltar para casa depois de um dia ou meio dia de aula}
\end{EntryWithPhonetic}

\begin{EntryWithPhonetic}{放养}{fang4yang3}{8,9}{⽅、⼋}
  \definition{v.}{criar (gado, peixes, culturas, etc.) | crescer | criar}
\end{EntryWithPhonetic}

\begin{EntryWithPhonetic}{放映}{fang4ying4}{8,9}{⽅、⽇}[HSK 7-9]
  \definition{v.}{mostrar (um filme); exibir; projetar; usar um dispositivo de luz forte para iluminar a imagem de uma foto ou filme em uma tela ou parede}
\end{EntryWithPhonetic}

\begin{EntryWithPhonetic}{放置}{fang4zhi4}{8,13}{⽅、⽹}[HSK 7-9]
  \definition{v.}{colocar; deitar; deixar de lado}
\end{EntryWithPhonetic}

\begin{EntryWithPhonetic}{放纵}{fang4zong4}{8,7}{⽅、⽷}[HSK 7-9]
  \definition{adj.}{grosseiro; inculto; autoindulgente; indisciplinado}
  \definition{v.}{satisfazer; ser conivente com; bajular; deixar alguém fazer o que quer}
\end{EntryWithPhonetic}

\begin{EntryWithPhonetic}{放走}{fang4zou3}{8,7}{⽅、⾛}
  \definition{v.}{permitir (uma pessoa ou um animal) ir | liberar | libertar}
\end{EntryWithPhonetic}

%%%%%%%%%% 飞 %%%%%%%%%%
\subsection*{飞}

\begin{EntryWithPhonetic}{飞}{fei1}{3}{⾶}[HSK 1][Kangxi 183]
  \definition{adj.}{inesperado; acidental; surgido do nada}
  \definition{adv.}{rapidamente; velozmente}
  \definition{s.}{roda livre de uma bicicleta}
  \definition{v.}{voar; esvoaçar; (pássaros, insetos, etc.) voar pelo ar batendo as asas | voar; utilizar máquinas motorizadas para se deslocar no ar | voar; (objetos naturais) flutuar ou esvoaçar no ar | volatilizar; evaporar; um gás se dissipar no ar | ir muito rapidamente; movimentar-se rapidamente, como se estivesse voando}
\end{EntryWithPhonetic}

\begin{EntryWithPhonetic}{飞船}{fei1 chuan2}{3,11}{⾶、⾈}[HSK 6]
  \definition{s.}{nave espacial; espaçonave | dirigível; aerobarco}
\end{EntryWithPhonetic}

\begin{EntryWithPhonetic}{飞弹}{fei1dan4}{3,11}{⾶、⼸}
  \definition{s.}{míssil (guiado) | bala perdida; bala disparada aleatoriamente | avião de mísseis; bomba voadora; bombas com equipamento de voo automático | dardo}
\end{EntryWithPhonetic}

\begin{EntryWithPhonetic}{飞碟}{fei1die2}{3,14}{⾶、⽯}
  \definition{s.}{disco-voador, OVNI, \emph{UFO} | \emph{frisbee}}
\end{EntryWithPhonetic}

\begin{EntryWithPhonetic}{飞机}{fei1ji1}{3,6}{⾶、⽊}[HSK 1]
  \definition[架,个]{s.}{avião; aeronave; aroplano}
\end{EntryWithPhonetic}

\begin{EntryWithPhonetic}{飞机票}{fei1ji1 piao4}{3,6,11}{⾶、⽊、⽰}
  \definition[张]{s.}{bilhete de avião; documento emitido mediante pagamento de passagem aérea, que autoriza o titular a viajar}
  \seealsoref{机票}{ji1 piao4}
\end{EntryWithPhonetic}

\begin{EntryWithPhonetic}{飞速}{fei1su4}{3,10}{⾶、⾡}[HSK 7-9]
  \definition{adj.}{em velocidade máxima}
\end{EntryWithPhonetic}

\begin{EntryWithPhonetic}{飞往}{fei1wang3}{3,8}{⾶、⼻}[HSK 7-9]
  \definition{v.}{voar para}[飞机将径直飞往圣保罗。===O avião voará diretamente para São Paulo.]
\end{EntryWithPhonetic}

\begin{EntryWithPhonetic}{飞翔}{fei1xiang2}{3,12}{⾶、⽻}[HSK 7-9]
  \definition{v.}{pairar; voar; circular no ar; voar em círculos}
\end{EntryWithPhonetic}

\begin{EntryWithPhonetic}{飞行}{fei1 xing2}{3,6}{⾶、⾏}[HSK 3]
  \definition{s.}{voo; aviação}
  \definition{v.}{voar; fazer um voo; (aviões, foguetes, etc.) voar no ar}
\end{EntryWithPhonetic}

\begin{EntryWithPhonetic}{飞行员}{fei1 xing2 yuan2}{3,6,7}{⾶、⾏、⼝}[HSK 6]
  \definition[名,班]{s.}{piloto; aviador; pilotos de aeronaves}
\end{EntryWithPhonetic}

\begin{EntryWithPhonetic}{飞跃}{fei1yue4}{3,11}{⾶、⾜}[HSK 7-9]
  \definition{v.}{saltar; crescer rapidamente}
\end{EntryWithPhonetic}

%%%%%%%%%% 非 %%%%%%%%%%
\subsection*{非}

\begin{EntryWithPhonetic}{非}{fei1}{8}{⾮}[HSK 4][Kangxi 175]
  \definition*{s.}{África, abreviação de 非洲 | Sobrenome: Fei}
  \definition{adv.}{Em resposta a 不, indica necessidade (deve)}
  \definition{pref.}{indicando negatividade ou exclusão}
  \definition{s.}{engano; erro}
  \definition{v.}{opor-se a; culpar; censurar | não estar em conformidade com; ser contrário a | não ser | ter que; simplesmente precisar (fazer algo)}
  \seealsoref{不}{bu4}
  \seealsoref{非洲}{fei1zhou1}
\end{EntryWithPhonetic}

\begin{EntryWithPhonetic}{非常}{fei1chang2}{8,11}{⾮、⼱}[HSK 1]
  \definition{adj.}{extraordinário; incomum; especial}
  \definition{adv.}{muito; extremamente; altamente}
\end{EntryWithPhonetic}

\begin{EntryWithPhonetic}{非得}{fei1dei3}{8,11}{⾮、⼻}[HSK 7-9]
  \definition{adv.}{(geralmente usado comcomitantemente com 不 ou 才) tem que; deve}[你非得服从命令不可。===Você deve obedecer às ordens.]
  \seealsoref{不}{bu4}
  \seealsoref{才}{cai2}
\end{EntryWithPhonetic}

\begin{EntryWithPhonetic}{非法}{fei1fa3}{8,8}{⾮、⽔}[HSK 7-9]
  \definition{adj.}{ilegal; ilícito; fora da lei}
\end{EntryWithPhonetic}

\begin{EntryWithPhonetic}{非凡}{fei1fan2}{8,3}{⾮、⼏}[HSK 7-9]
  \definition{adj.}{excepcional; extraordinário; incomum; mais do que o normal}
\end{EntryWithPhonetic}

\begin{EntryWithPhonetic}{非金属}{fei1jin4shu3}{8,8,12}{⾮、⾦、⼫}[HSK 7-9]
  \definition{s.}{Química: não metal; metalóide; com exceção do bromo, os elementos que geralmente não têm brilho metálico nem ductilidade e não conduzem facilmente eletricidade ou calor são gases ou sólidos à temperatura ambiente, como oxigênio, enxofre, nitrogênio, fósforo, etc.}
\end{EntryWithPhonetic}

\begin{EntryWithPhonetic}{非洲}{fei1zhou1}{8,9}{⾮、⽔}
  \definition*{s.}{África}
\end{EntryWithPhonetic}

\begin{EntryWithPhonetic}{非洲人}{fei1zhou1ren2}{8,9,2}{⾮、⽔、⼈}
  \definition{s.}{africano | pessoa ou povo da África}
\end{EntryWithPhonetic}

%%%%%%%%%% 绯 %%%%%%%%%%
\subsection*{绯}

\begin{EntryWithPhonetic}{绯}{fei1}{11}{⽷}
  \definition{adj.}{escarlate; vermelho; vermelho escuro; vermelho profundo}
\end{EntryWithPhonetic}

\begin{EntryWithPhonetic}{绯闻}{fei1wen2}{11,9}{⽷、⾨}[HSK 7-9]
  \definition{s.}{boato/fofoca sobre escândalos sexuais | escândalo sexual; rumores sobre relacionamentos entre homens e mulheres}
\end{EntryWithPhonetic}

%%%%%%%%%% 肥 %%%%%%%%%%
\subsection*{肥}

\begin{EntryWithPhonetic}{肥}{fei2}{8}{⾁}[HSK 4]
  \definition{adj.}{gordo; gorduroso; contém muita gordura (o oposto de 瘦, geralmente não usado para descrever pessoas) | fértil; rico | solto; largo; folgado; (roupas, etc.) largas (em oposição a 瘦) | lucrativo; rendendo bons lucros}
  \definition{s.}{fertilizante; esterco}
  \definition{v.}{fertilizar; tornar fértil ou obeso | enriquecer com renda ilegal, ilícita}
  \seealsoref{瘦}{shou4}
\end{EntryWithPhonetic}

\begin{EntryWithPhonetic}{肥料}{fei2liao4}{8,10}{⾁、⽃}[HSK 7-9]
  \definition[种,袋,把]{s.}{esterco; fertilizante}
\end{EntryWithPhonetic}

\begin{EntryWithPhonetic}{肥胖}{fei2pang4}{8,9}{⾁、⾁}[HSK 7-9]
  \definition{adj.}{gordo; obeso; corpulento; excesso de gordura corporal}
\end{EntryWithPhonetic}

\begin{EntryWithPhonetic}{肥沃}{fei2wo4}{8,7}{⾁、⽔}[HSK 7-9]
  \definition{adj.}{fértil; rico (de solo); (terra) contém mais nutrientes e água adequados para o crescimento das plantas}
\end{EntryWithPhonetic}

\begin{EntryWithPhonetic}{肥皂}{fei2zao4}{8,7}{⾁、⽩}[HSK 7-9]
  \definition[块,条]{s.}{sabão; produtos químicos usados ​​para limpeza}
\end{EntryWithPhonetic}

%%%%%%%%%% 诽 %%%%%%%%%%
\subsection*{诽}

\begin{EntryWithPhonetic}{诽}{fei3}{10}{⾔}
  \definition{v.}{calúnia}
\end{EntryWithPhonetic}

\begin{EntryWithPhonetic}{诽谤}{fei3bang4}{10,12}{⾔、⾔}[HSK 7-9]
  \definition{v.}{difamar; caluniar; falar mal; espalhar boatos e caluniar os outros; falar mal dos outros e prejudicar sua reputação}
\end{EntryWithPhonetic}

%%%%%%%%%% 废 %%%%%%%%%%
\subsection*{废}

\begin{EntryWithPhonetic}{废}{fei4}{8}{⼴}[HSK 7-9]
  \definition{adj.}{desperdíçado; inútil; fora de uso; inválido; tendo perdido sua função original | Literário: incapacitado; mutilado; aleijado; desabilitado}
  \definition{v.}{desistir; abandonar; abolir; revogar  | Coloquial: punir; bater em alguém | descartar; abandonar}
\end{EntryWithPhonetic}

\begin{EntryWithPhonetic}{废除}{fei4chu2}{8,9}{⼴、⾩}[HSK 7-9]
  \definition{v.}{revogar; anular; cancelar; abolir (uma lei, sistema, tratado, etc.)}
\end{EntryWithPhonetic}

\begin{EntryWithPhonetic}{废话}{fei4hua4}{8,8}{⼴、⾔}[HSK 7-9]
  \definition{s.}{lixo; absurdo; palavras supérfluas; palavras redundantes e inúteis}
  \definition{v.}{falar bobagens; conversa fiada}
\end{EntryWithPhonetic}

\begin{EntryWithPhonetic}{废品}{fei4pin3}{8,9}{⼴、⼝}[HSK 7-9]
  \definition[件,吨,批,堆]{s.}{produto residual; rejeito; descarte; produtos não qualificados; produto descartado; sucata; refugo; material rejeitado}
\end{EntryWithPhonetic}

\begin{EntryWithPhonetic}{废寝忘食}{fei4qin3-wang4shi2}{8,13,7,9}{⼴、⼧、⼼、⾷}[HSK 7-9]
  \definition{expr.}{esquecer de comer e dormir; estar totalmente absorvido em}
\end{EntryWithPhonetic}

\begin{EntryWithPhonetic}{废物}{fei4wu4}{8,8}{⼴、⽜}[HSK 7-9]
  \definition{s.}{lixo; material residual; coisas que perderam seu valor de uso original}
  \seeref{fei4wu5}
\end{EntryWithPhonetic}

\begin{EntryWithPhonetic}{废物}{fei4wu5}{8,8}{⼴、⽜}
  \definition{s.}{pessoa inútil; imprestável (insulto); uma metáfora para uma pessoa inútil (palavrão)}
  \seeref{fei4wu4}
\end{EntryWithPhonetic}

\begin{EntryWithPhonetic}{废墟}{fei4xu1}{8,14}{⼴、⼟}[HSK 7-9]
  \definition[片,堆,个]{s.}{ruínas; terreno baldio; um lugar como uma cidade ou vila que ficou deserta e desolada após ser destruída ou sofrer um desastre natural}
\end{EntryWithPhonetic}

%%%%%%%%%% 沸 %%%%%%%%%%
\subsection*{沸}

\begin{EntryWithPhonetic}{沸}{fei4}{8}{⽔}
  \definition{adj.}{fervente; borbulhante; em ebulição}
  \definition{v.}{ferver | borbulhar}
\end{EntryWithPhonetic}

\begin{EntryWithPhonetic}{沸沸扬扬}{fei4fei4yang2yang2}{8,8,6,6}{⽔、⽔、⼿、⼿}[HSK 7-9]
  \definition{expr.}{levantar uma confusão de críticas sobre; borbulhando de barulho; discutir animadamente; dar origem a muita discussão; em um rebuliço; ``Tão barulhento quanto água fervente.'', frequentemente usado para descrever muita discussão}
\end{EntryWithPhonetic}

\begin{EntryWithPhonetic}{沸腾}{fei4teng2}{8,13}{⽔、⾁}[HSK 7-9]
  \definition{v.}{ferver; vaporizar | fervilhar de excitação; uma metáfora para alto astral ou vozes barulhentas}
\end{EntryWithPhonetic}

%%%%%%%%%% 狒 %%%%%%%%%%
\subsection*{狒}

\begin{EntryWithPhonetic}{狒}{fei4}{8}{⽝}
  \definition{s.}{babuíno (uma espécie de macaco)}
\end{EntryWithPhonetic}

\begin{EntryWithPhonetic}{狒狒}{fei4fei4}{8,8}{⽝、⽝}
  \definition{s.}{babuíno}
\end{EntryWithPhonetic}

%%%%%%%%%% 肺 %%%%%%%%%%
\subsection*{肺}

\begin{EntryWithPhonetic}{肺}{fei4}{8}{⾁}[HSK 6]
  \definition[叶]{s.}{pulmão | pulmões; órgãos respiratórios de humanos e animais superiores}
\end{EntryWithPhonetic}

%%%%%%%%%% 费 %%%%%%%%%%
\subsection*{费}

\begin{EntryWithPhonetic}{费}{fei4}{9}{⾙}[HSK 3]
  \definition*{s.}{Sobrenome: Fei}
  \definition{s.}{taxa; despesa; encargo}
  \definition{v.}{custar; gastar; despender | ser desperdiçador; consumir em excesso; gastar algo muito rapidamente; consumo excessivo (oposto a 省)}
  \seealsoref{省}{sheng3}
\end{EntryWithPhonetic}

\begin{EntryWithPhonetic}{费劲}{fei4/jin4}{9,7}{⾙、⼒}[HSK 7-9]
  \definition{adj.}{extenuante}
  \definition{v.+compl.}{ser extenuante; precisar ou usar grande esforço; gastar energia; fazer coisas ou falar com cuidado; trabalhar duro}
\end{EntryWithPhonetic}

\begin{EntryWithPhonetic}{费用}{fei4 yong4}{9,5}{⾙、⽤}[HSK 3]
  \definition[笔,个]{s.}{custo; despesa; desembolso}
\end{EntryWithPhonetic}

%%%%%%%%%% 分 %%%%%%%%%%
\subsection*{分}

\begin{EntryWithPhonetic}{分}{fen1}{4}{⼑}[HSK 1,2]
  \definition{adj.}{filial (de uma organização)}
  \definition{clas.}{parte ou subdivisão | fração | um décimo (de certas unidades) | unidade de comprimento equivalente a 0,33cm | unidade de área (=66,666 metros quadrados) | unidade de peso (=1/2 grama) | minuto (unidade de tempo) | minuto (unidade de medida angular) | 0,01 yuan (unidade de dinheiro) | taxa de juros | marca; ponto; unidade de contagem para avaliação de notas, etc.}
  \definition{s.}{fração}
  \definition{v.}{separar; dividir; partir; dividir algo inteiro em várias partes ou separar coisas que estão ligadas entre si | atribuir; designar; distribuir | distinguir; diferenciar; diferenciar um do outro}
  \seeref{fen4}
\end{EntryWithPhonetic}

\begin{EntryWithPhonetic}{分辨}{fen1bian4}{4,16}{⼑、⾟}[HSK 7-9]
  \definition{s.}{resolução; descreve o grau em que imagens, exibições, sensores, etc. podem distinguir claramente os detalhes}
  \definition{v.}{distinguir; diferenciar}
\end{EntryWithPhonetic}

\begin{EntryWithPhonetic}{分别}{fen1bie2}{4,7}{⼑、⼑}[HSK 3]
  \definition{adv.}{diferentemente; de maneiras diferentes; expressar de maneiras diferentes | separadamente; individualmente; respectivamente}
  \definition{s.}{diferença; pontos diferentes}
  \definition{v.}{partir; deixar um ao outro; não estar mais junto | distinguir; diferenciar}
\end{EntryWithPhonetic}

\begin{EntryWithPhonetic}{分布}{fen1bu4}{4,5}{⼑、⼱}[HSK 4]
  \definition{v.}{espalhar; distribuir; dispersar (em uma determinada área)}
\end{EntryWithPhonetic}

\begin{EntryWithPhonetic}{分成}{fen1 cheng2}{4,6}{⼑、⼽}[HSK 5]
  \definition{v.}{dividir em; separar em; dividir dinheiro, bens, etc. de acordo com a porcentagem}
\end{EntryWithPhonetic}

\begin{EntryWithPhonetic}{分寸}{fen1cun5}{4,3}{⼑、⼨}[HSK 7-9]
  \definition{s.}{limites adequados para a fala ou ação; senso de propriedade (ou proporção)}
\end{EntryWithPhonetic}

\begin{EntryWithPhonetic}{分担}{fen1dan1}{4,8}{⼑、⼿}[HSK 7-9]
  \definition{v.}{contribuir; compartilhar a responsabilidade por; participar}
\end{EntryWithPhonetic}

\begin{EntryWithPhonetic}{分割}{fen1ge1}{4,12}{⼑、⼑}[HSK 7-9]
  \definition{v.}{cortar; separar; quebrar; separar uma coisa inteira ou conectada}
\end{EntryWithPhonetic}

\begin{EntryWithPhonetic}{分工}{fen1 gong1}{4,3}{⼑、⼯}[HSK 6]
  \definition[种]{s.}{divisão do trabalho}
  \definition{v.}{dividir o trabalho; envolver-se em várias tarefas diferentes, mas complementares}
\end{EntryWithPhonetic}

\begin{EntryWithPhonetic}{分公司}{fen1gong1si1}{4,4,5}{⼑、⼋、⼝}
  \definition{s.}{sucursal | filial de companhia}
\end{EntryWithPhonetic}

\begin{EntryWithPhonetic}{分红}{fen1/hong2}{4,6}{⼑、⽷}[HSK 7-9]
  \definition{v.+compl.}{receber dividendos; distribuir bônus; obter dividendos extras (lucros)}
\end{EntryWithPhonetic}

\begin{EntryWithPhonetic}{分化}{fen1hua4}{4,4}{⼑、⼔}[HSK 7-9]
  \definition{s.}{diferenciação}
  \definition{v.}{dividir-se; fragmentar; romper; separar-se; separar | Biologia: (células ou tecidos) diferenciar}
\end{EntryWithPhonetic}

\begin{EntryWithPhonetic}{分解}{fen1jie3}{4,13}{⼑、⾓}[HSK 5]
  \definition{v.}{quebrar; separar em partes; dividir um todo em seus componentes | resolver; decompor | Química: transformar uma substância em duas ou mais substâncias por meio de uma reação química | desintegrar-se; dividir-se; desunir uma organização | mediar; fazer a paz; resolver conflitos e disputas | explicar; defender-se}
\end{EntryWithPhonetic}

\begin{EntryWithPhonetic}{分开}{fen1/kai1}{4,4}{⼑、⼶}[HSK 2]
  \definition{v.+compl.}{separar; dividir; desacoplar; desembalar; romper; desfolhar; decolar; romper; distribuir; separar de (em); dividir\dots de\dots | separar; fazer com que uma pessoa ou algo deixe de estar junto com outra pessoa ou coisa}
\end{EntryWithPhonetic}

\begin{EntryWithPhonetic}{分类}{fen1/lei4}{4,9}{⼑、⽶}[HSK 5]
  \definition{v.+compl.}{ordenar; classificar; categorizar; classificar as coisas de acordo com sua natureza e características}
\end{EntryWithPhonetic}

\begin{EntryWithPhonetic}{分离}{fen1 li2}{4,10}{⼑、⼇}[HSK 5]
  \definition{v.}{cortar; separar (de coisas) | separar; sair; separar (de pessoas); partir (em uma longa viagem)}
\end{EntryWithPhonetic}

\begin{EntryWithPhonetic}{分量}{fen1liang4}{4,12}{⼑、⾥}
  \definition{s.}{componente vetorial}
  \seeref{fen4liang4}
  \seeref{fen4liang5}
\end{EntryWithPhonetic}

\begin{EntryWithPhonetic}{分裂}{fen1lie4}{4,12}{⼑、⾐}[HSK 6]
  \definition{s.}{fissão; divisão}
  \definition{v.}{dividir; separar; romper}
\end{EntryWithPhonetic}

\begin{EntryWithPhonetic}{分泌}{fen1mi4}{4,8}{⼑、⽔}[HSK 7-9]
  \definition{v.}{secretar; excretar}
\end{EntryWithPhonetic}

\begin{EntryWithPhonetic}{分明}{fen1ming2}{4,8}{⼑、⽇}[HSK 7-9]
  \definition{adj.}{claro; distinto; óbvio}
  \definition{adv.}{claramente; evidentemente; obviamente; os fatos são claros, óbvios e inquestionáveis}
\end{EntryWithPhonetic}

\begin{EntryWithPhonetic}{分配}{fen1pei4}{4,10}{⼑、⾣}[HSK 3]
  \definition{v.}{atribuir; dispor; organizar o trabalho, as tarefas, os recursos, o tempo, etc. | atribuir; compartilhar; distribuir dinheiro ou bens às pessoas envolvidas de acordo com um determinado plano, padrão ou regulamento}
\end{EntryWithPhonetic}

\begin{EntryWithPhonetic}{分歧}{fen1qi2}{4,8}{⼑、⽌}[HSK 7-9]
  \definition[点,个]{s.}{diferença; divergência; diferenças de pontos de vista, opiniões, etc.}
\end{EntryWithPhonetic}

\begin{EntryWithPhonetic}{分散}{fen1san4}{4,12}{⼑、⽁}[HSK 4]
  \definition{adj.}{espalhado; disperso; desviado; fragmentado; sem foco}
  \definition{v.}{dispersar; espalhar; descentralizar | separar-se; desunir-se}
\end{EntryWithPhonetic}

\begin{EntryWithPhonetic}{分手}{fen1/shou3}{4,4}{⼑、⼿}[HSK 4]
  \definition{v.+compl.}{separar; romper; terminar um relacionamento ou um casal | separar-se (de uma empresa); dizer adeus; despedir-se da família, dos amigos, etc.}
\end{EntryWithPhonetic}

\begin{EntryWithPhonetic}{分数}{fen1 shu4}{4,13}{⼑、⽁}[HSK 2]
  \definition[个]{s.}{fração; número fracionário | nota; classificação; ponto; pontuação registrada ao avaliar o resultado ou a vitória/derrota}
\end{EntryWithPhonetic}

\begin{EntryWithPhonetic}{分为}{fen1 wei2}{4,4}{⼑、⼂}[HSK 4]
  \definition{v.}{subdividir; dividir algo em}
\end{EntryWithPhonetic}

\begin{EntryWithPhonetic}{分析}{fen1xi1}{4,8}{⼑、⽊}[HSK 5]
  \definition{v.}{analisar; dividir uma coisa, um fenômeno, um conceito em componentes mais simples e descobrir as propriedades essenciais desses componentes e a relação entre eles (em oposição à 综合)}
  \seealsoref{综合}{zong1he2}
\end{EntryWithPhonetic}

\begin{EntryWithPhonetic}{分享}{fen1 xiang3}{4,8}{⼑、⼇}[HSK 5]
  \definition{v.}{compartilhar; partilhar}
\end{EntryWithPhonetic}

\begin{EntryWithPhonetic}{分赃}{fen1/zang1}{4,10}{⼑、⾙}[HSK 7-9]
  \definition{v.+compl.}{dividir ganhos ilícitos; compartilhar o saque}
\end{EntryWithPhonetic}

\begin{EntryWithPhonetic}{分之}{fen1 zhi1}{4,3}{⼑、⼂}[HSK 4]
  \definition{expr.}{indicando uma fração; formatação e leitura de frações, ou seja, partes de um total}[顾客减少了三分之一。===O número de clientes caiu em um terço.]
\end{EntryWithPhonetic}

\begin{EntryWithPhonetic}{分支}{fen1zhi1}{4,4}{⼑、⽀}[HSK 7-9]
  \definition{s.}{subdivisão; filial; afiliada; uma parte separada de um sistema ou corpo}
\end{EntryWithPhonetic}

\begin{EntryWithPhonetic}{分钟}{fen1zhong1}{4,9}{⼑、⾦}[HSK 2]
  \definition{clas.}{minuto (usado em intervalos de tempo); 60 segundos}
\end{EntryWithPhonetic}

\begin{EntryWithPhonetic}{分子}{fen1zi3}{4,3}{⼑、⼦}
  \definition{s.}{molécula | (matemática) numerador de uma fração}
  \seeref{fen4zi3}
\end{EntryWithPhonetic}

\begin{EntryWithPhonetic}{分组}{fen1 zu3}{4,8}{⼑、⽷}[HSK 3]
  \definition{v.}{dividir em grupos}
\end{EntryWithPhonetic}

%%%%%%%%%% 吩 %%%%%%%%%%
\subsection*{吩}

\begin{EntryWithPhonetic}{吩}{fen1}{7}{⼝}
  \definition{v.}{deixar instruções; instruir | ordenar; mandar}
\end{EntryWithPhonetic}

\begin{EntryWithPhonetic}{吩咐}{fen1fu4}{7,8}{⼝、⼝}[HSK 7-9]
  \definition{v.}{dizer; instruir; comandar; dizer a alguém para lembrar o que deve ou não ser feito}
\end{EntryWithPhonetic}

%%%%%%%%%% 纷 %%%%%%%%%%
\subsection*{纷}

\begin{EntryWithPhonetic}{纷}{fen1}{7}{⽷}
  \definition[场]{adj.}{confuso; emaranhado; desordenado | muitos e variados; profusos; numerosos}
\end{EntryWithPhonetic}

\begin{EntryWithPhonetic}{纷纷}{fen1fen1}{7,7}{⽷、⽷}[HSK 4]
  \definition{adj.}{numeroso e confuso; muitos e desordenados}
  \definition{adv.}{um após o outro; em sucessão; em rápida sucessão}
\end{EntryWithPhonetic}

%%%%%%%%%% 芬 %%%%%%%%%%
\subsection*{芬}

\begin{EntryWithPhonetic}{芬}{fen1}{7}{⾋}
  \definition*{s.}{Sobrenome: Fen}
  \definition{adj.}{perfumado}
  \definition[阵,股]{s.}{cheiro doce; fragrância; aroma}
\end{EntryWithPhonetic}

\begin{EntryWithPhonetic}{芬芳}{fen1fang1}{7,7}{⾋、⾋}[HSK 7-9]
  \definition{adj.}{perfumado; com cheiro doce}
  \definition{s.}{fragrância; aroma; cheiro doce}
\end{EntryWithPhonetic}

%%%%%%%%%% 氛 %%%%%%%%%%
\subsection*{氛}

\begin{EntryWithPhonetic}{氛}{fen1}{8}{⽓}
  \definition{s.}{atmosfera; gás}
\end{EntryWithPhonetic}

\begin{EntryWithPhonetic}{氛围}{fen1wei2}{8,7}{⽓、⼞}[HSK 7-9]
  \definition[种,片,股]{s.}{atmosfera; a atmosfera e o humor ao redor}
\end{EntryWithPhonetic}

%%%%%%%%%% 坟 %%%%%%%%%%
\subsection*{坟}

\begin{EntryWithPhonetic}{坟}{fen2}{7}{⼟}[HSK 7-9]
  \definition[座,片,个]{s.}{sepultura; túmulo; cova}
\end{EntryWithPhonetic}

\begin{EntryWithPhonetic}{坟墓}{fen2mu4}{7,13}{⼟、⼟}[HSK 7-9]
  \definition[座,片,个]{s.}{sepultura; túmulo; a cova funerária e a sepultura acima dela}
\end{EntryWithPhonetic}

%%%%%%%%%% 焚 %%%%%%%%%%
\subsection*{焚}

\begin{EntryWithPhonetic}{焚}{fen2}{12}{⽕}
  \definition{v.}{queimar}
\end{EntryWithPhonetic}

\begin{EntryWithPhonetic}{焚烧}{fen2shao1}{12,10}{⽕、⽕}[HSK 7-9]
  \definition{v.}{queimar; incendiar; colocar fogo}
\end{EntryWithPhonetic}

\begin{EntryWithPhonetic}{焚香}{fen2xiang1}{12,9}{⽕、⾹}
  \definition{v.}{queimar incenso}
\end{EntryWithPhonetic}

%%%%%%%%%% 粉 %%%%%%%%%%
\subsection*{粉}

\begin{EntryWithPhonetic}{粉}{fen3}{10}{⽶}[HSK 7-9]
  \definition{adj.}{branco | rosa}
  \definition{s.}{pó | cosméticos em pó | farinha de trigo | macarrão ou outro alimento feito de feijão, arroz, batata, amido de batata-doce, etc. | macarrão de arroz}
  \definition{v.}{virar pó | Dialeto: caiar}
\end{EntryWithPhonetic}

\begin{EntryWithPhonetic}{粉色}{fen3 se4}{10,6}{⽶、⾊}
  \definition{s.}{cor-de-rosa}
\end{EntryWithPhonetic}

\begin{EntryWithPhonetic}{粉丝}{fen3si1}{10,5}{⽶、⼀}[HSK 7-9]
  \definition{s.}{(empréstimo linguístico) fã | entusiasta de alguém ou alguma coisa}
  \definition[个,群,位,名,些,批]{s.}{aletria de amido de feijão ou batata; aletria chinesa; macarrão de celofane ou macarrão de vidro (transparente) | Empréstimo linguístico: fã; refere-se a uma pessoa que é obcecada ou adora uma celebridade}
\end{EntryWithPhonetic}

\begin{EntryWithPhonetic}{粉碎}{fen3sui4}{10,13}{⽶、⽯}[HSK 7-9]
  \definition{adj.}{pulverizado; quebrado em pedaços; descreve algo que está muito quebrado, quebrado em partículas muito pequenas}
  \definition{v.}{esmagar; transformar as coisas em partículas muito pequenas | esmagar; quebrar; estilhaçar; fazer com que a outra parte falhe ou seja completamente destruída}
\end{EntryWithPhonetic}

%%%%%%%%%% 分 %%%%%%%%%%
\subsection*{分}

\begin{EntryWithPhonetic}{分}{fen4}{4}{⼑}[HSK 2]
  \definition{s.}{componente | o que está dentro dos deveres ou direitos de alguém; limites das responsabilidades e direitos | afeto; sentimento de amizade}
  \definition{v.}{pensar; esperar; estimar}
  \seeref{fen1}
\end{EntryWithPhonetic}

\begin{EntryWithPhonetic}{分量}{fen4liang4}{4,12}{⼑、⾥}[HSK 7-9]
  \definition{s.}{tamanho da porção (comida)}[这道菜的分量太少了。===A porção deste prato era muito pequena.]
  \seeref{fen1liang4}
  \seeref{fen4liang5}
\end{EntryWithPhonetic}

\begin{EntryWithPhonetic}{分量}{fen4liang5}{4,12}{⼑、⾥}
  \definition{s.}{quantidade; peso; medida | Figurativo: peso (importância, prestígio, autoridade etc.); (de material escrito) densidade}
  \seeref{fen1liang4}
  \seeref{fen4liang4}
\end{EntryWithPhonetic}

\begin{EntryWithPhonetic}{分外}{fen4wai4}{4,5}{⼑、⼣}[HSK 7-9]
  \definition{adj.}{não é tarefa (dever) de alguém; além do dever de alguém; fora do escopo do dever de alguém}[他对分外的工作总是抢着干。===Ele está sempre disposto e ansioso para fazer trabalho extra.]
  \definition{adv.}{particularmente; especialmente; excepcionalmente}[雨后初晴的天空分外明朗。===O céu depois da chuva está excepcionalmente claro.]
\end{EntryWithPhonetic}

\begin{EntryWithPhonetic}{分子}{fen4zi3}{4,3}{⼑、⼦}
  \definition{s.}{membros de uma classe ou grupo | elementos políticos (como intelectuais ou extremistas)}
  \seeref{fen1zi3}
\end{EntryWithPhonetic}

%%%%%%%%%% 份 %%%%%%%%%%
\subsection*{份}

\begin{EntryWithPhonetic}{份}{fen4}{6}{⼈}
  \definition{clas.}{usado para emparelhar itens em grupos | usado para jornais, documentos, etc. | usado para partes de um todo | usado para aparência, estado, etc.}
  \definition{s.}{porção; parte | a unidade de divisão; usado após 省, 县, 年, 月,  indica a unidade de divisão | grau; extensão de algo}
  \seealsoref{年}{nian2}
  \seealsoref{省}{sheng3}
  \seealsoref{县}{xian4}
  \seealsoref{月}{yue4}
\end{EntryWithPhonetic}

\begin{EntryWithPhonetic}{份额}{fen4'e2}{6,15}{⼈、⾴}[HSK 7-9]
  \definition{s.}{quota; quinhão; porção; a proporção ou percentagem do todo}
\end{EntryWithPhonetic}

\begin{EntryWithPhonetic}{份量}{fen4liang5}{6,12}{⼈、⾥}
  \variantof{分量}
\end{EntryWithPhonetic}

%%%%%%%%%% 奋 %%%%%%%%%%
\subsection*{奋}

\begin{EntryWithPhonetic}{奋}{fen4}{8}{⼤}
  \definition{adv.}{energicamente; com força e espírito}
  \definition{v.}{esforçar-se; agir vigorosamente; preparar-se | levantar | aplicar energia; resolver; animar-se | acenar; sacudir; levantar}
\end{EntryWithPhonetic}

\begin{EntryWithPhonetic}{奋斗}{fen4dou4}{8,4}{⼤、⽃}[HSK 4]
  \definition{v.}{lutar; esforçar-se; batalhar; trabalhar duro para atingir um determinado objetivo}
\end{EntryWithPhonetic}

\begin{EntryWithPhonetic}{奋力}{fen4li4}{8,2}{⼤、⼒}[HSK 7-9]
  \definition{v.}{fazer tudo o que puder; não poupar esforços}
\end{EntryWithPhonetic}

\begin{EntryWithPhonetic}{奋勇}{fen4yong3}{8,9}{⼤、⼒}[HSK 7-9]
  \definition{v.}{reunir toda a coragem e energia; criar coragem}
\end{EntryWithPhonetic}

\begin{EntryWithPhonetic}{奋战}{fen4zhan4}{8,9}{⼤、⼽}
  \definition{v.}{lutar bravamente | trabalhar duro}
\end{EntryWithPhonetic}

%%%%%%%%%% 愤 %%%%%%%%%%
\subsection*{愤}

\begin{EntryWithPhonetic}{愤}{fen4}{12}{⼼}
  \definition{s.}{raiva; indignação; ressentimento; exasperação}
  \definition{v.}{ressentir-se; ficar indignado; ficar com raiva}
\end{EntryWithPhonetic}

\begin{EntryWithPhonetic}{愤怒}{fen4nu4}{12,9}{⼼、⼼}[HSK 6]
  \definition{adj.}{zangado; enraivecido; iracundo; furioso; emocionalmente agitado por extrema insatisfação}
\end{EntryWithPhonetic}

\begin{EntryWithPhonetic}{愤世嫉俗}{fen4shi4ji2su2}{12,5,13,9}{⼼、⼀、⼥、⼈}
  \definition{v.}{ser cínico | ser amargurado}
\end{EntryWithPhonetic}

%%%%%%%%%% 粪 %%%%%%%%%%
\subsection*{粪}

\begin{EntryWithPhonetic}{粪}{fen4}{12}{⽶}[HSK 7-9]
  \definition[缸,桶]{s.}{excremento; fezes; esterco}
  \definition{v.}{Literário: aplicar esterco; fertilizar | Literário: limpar; remover; eliminar; acabar com}
\end{EntryWithPhonetic}

\begin{EntryWithPhonetic}{粪便}{fen4bian4}{12,9}{⽶、⼈}[HSK 7-9]
  \definition{s.}{excremento; fezes; esterco; excrementos e urina}
\end{EntryWithPhonetic}

%%%%%%%%%% 丰 %%%%%%%%%%
\subsection*{丰}

\begin{EntryWithPhonetic}{丰}{feng1}{4}{⼁}
  \definition*{s.}{Sobrenome: Feng}
  \definition[阵,丝]{adj.}{cheio; rico; abundante | ótimo | bonito; de boa aparência; cheio e redondo}
\end{EntryWithPhonetic}

\begin{EntryWithPhonetic}{丰富}{feng1fu4}{4,12}{⼁、⼧}[HSK 3]
  \definition{adj.}{rico; abundante; pleno; (riqueza material, conhecimento, experiência, etc.) variedade ou quantidade}
  \definition{v.}{enriquecer}
\end{EntryWithPhonetic}

\begin{EntryWithPhonetic}{丰富多彩}{feng1fu4-duo1cai3}{4,12,6,11}{⼁、⼧、⼣、⼺}[HSK 7-9]
  \definition{expr.}{rica e colorida; variada e colorida; abundante e diversa; descreve uma grande variedade de tipos e cores}
\end{EntryWithPhonetic}

\begin{EntryWithPhonetic}{丰厚}{feng1hou4}{4,9}{⼁、⼚}[HSK 7-9]
  \definition{adj.}{rico e generoso; (renda, presentes, etc.) grande em quantidade e alto em valor | grosso; abundante e espesso}
\end{EntryWithPhonetic}

\begin{EntryWithPhonetic}{丰满}{feng1man3}{4,13}{⼁、⽔}[HSK 7-9]
  \definition{adj.}{(corpo ou partes do corpo) roliço; adulto; cheio e redondo; bem desenvolvido; roliço e bem proporcionado | abundante; adequado | completo; cheio}
\end{EntryWithPhonetic}

\begin{EntryWithPhonetic}{丰盛}{feng1sheng4}{4,11}{⼁、⽫}[HSK 7-9]
  \definition{adj.}{rico; abundante; substancial; suntuoso; descreve comida que é abundante e boa}[他吃了一顿丰盛的早餐。===Ele tomou um café da manhã substancial.]
\end{EntryWithPhonetic}

\begin{EntryWithPhonetic}{丰收}{feng1shou1}{4,6}{⼁、⽁}[HSK 5]
  \definition{v.}{ter uma boa colheita; obter uma colheita boa e abundante; obter bons resultados}
\end{EntryWithPhonetic}

\begin{EntryWithPhonetic}{丰硕}{feng1shuo4}{4,11}{⼁、⽯}[HSK 7-9]
  \definition{adj.}{rico; frutífero; abundante e substancial; usado principalmente para coisas abstratas}[她的研究成果非常丰硕。===Os resultados de sua pesquisa são muito frutíferos.]
\end{EntryWithPhonetic}

%%%%%%%%%% 风 %%%%%%%%%%
\subsection*{风}

\begin{EntryWithPhonetic}{风}{feng1}{4}{⾵}[HSK 1][Kangxi 182]
  \definition*{s.}{Sobrenome: Feng}
  \definition{adj.}{lendário; sem fundamento concreto | rápido; veloz | promíscuo; libertino; sedutor}
  \definition[阵,丝]{s.}{vento; fluxo de ar | prática; ambiente; costume | cena; vista | notícias; fofocas; rumores | comportamento; maneira; estilo | canção folclórica | certas doenças}
  \definition{v.}{colocar para secar ou arejar; secar ao vento}
\end{EntryWithPhonetic}

\begin{EntryWithPhonetic}{风暴}{feng1bao4}{4,15}{⾵、⽇}[HSK 6]
  \definition{s.}{tempestade; vendaval; um termo geral para perturbações violentas na atmosfera e mudanças drásticas no clima, como tempestades de areia, tornados, ciclones tropicais, etc. | tempestade; comoção violenta; uma metáfora para um evento tão poderoso que abala toda a sociedade}
\end{EntryWithPhonetic}

\begin{EntryWithPhonetic}{风波}{feng1bo1}{4,8}{⾵、⽔}[HSK 7-9]
  \definition[场]{s.}{Meteorologia: vento e ondas --- perturbação | Literário: tempestade | perturbação; onda de vento; uma tempestade em copo d'água | tempestade; crise; perturbação}
\end{EntryWithPhonetic}

\begin{EntryWithPhonetic}{风采}{feng1cai3}{4,8}{⾵、⾤}[HSK 7-9]
  \definition{s.}{graça; elegância; comportamento; carisma | estilo; graça; talento literário}
\end{EntryWithPhonetic}

\begin{EntryWithPhonetic}{风餐露宿}{feng1can1-lu4su4}{4,16,21,11}{⾵、⾷、⾬、⼧}[HSK 7-9]
  \definition{expr.}{comer ao vento e dormir no orvalho --- suportar as dificuldades de uma jornada árdua | o sol brilha no chão; para descrever as dificuldades de viajar ou viver ao ar livre, também é dito ``dormir ao ar livre''}
\end{EntryWithPhonetic}

\begin{EntryWithPhonetic}{风度}{feng1du4}{4,9}{⾵、⼴}[HSK 5]
  \definition{s.}{postura; comportamento; porte; conduta; atitude}
\end{EntryWithPhonetic}

\begin{EntryWithPhonetic}{风范}{feng1fan4}{4,9}{⾵、⾋}[HSK 7-9]
  \definition{s.}{comportamento; porte; postura | estilo; maneira; ar | modelo; protótipo}
\end{EntryWithPhonetic}

\begin{EntryWithPhonetic}{风风雨雨}{feng1feng1yu3yu3}{4,4,8,8}{⾵、⾵、⾬、⾬}[HSK 7-9]
  \definition{expr.}{altos e baixos | dificuldades e sofrimentos | fofocas infundadas}
\end{EntryWithPhonetic}

\begin{EntryWithPhonetic}{风格}{feng1ge2}{4,10}{⾵、⽊}[HSK 4]
  \definition{s.}{modo; estilo; maneira; caráter | características das criações literárias de diferentes épocas, povos, escolas ou indivíduos em termos de conteúdo ideológico e técnicas artísticas}
\end{EntryWithPhonetic}

\begin{EntryWithPhonetic}{风光}{feng1guang1}{4,6}{⾵、⼉}[HSK 5]
  \definition{s.}{cena; vista; paisagens naturais e humanas}
\end{EntryWithPhonetic}

\begin{EntryWithPhonetic}{风和日丽}{feng1he2-ri4li4}{4,8,4,7}{⾵、⼝、⽇、⼀}[HSK 7-9]
  \definition{expr.}{vento moderado, sol bonito; tempo bom e ensolarado; sol brilhante e uma brisa suave; clima quente e ensolarado; descreve o clima ensolarado e ameno (usado principalmente na primavera)}
\end{EntryWithPhonetic}

\begin{EntryWithPhonetic}{风景}{feng1jing3}{4,12}{⾵、⽇}[HSK 4]
  \definition[种,处,道]{s.}{cenário; paisagem; cenários e vistas que podem ser apreciados, inclui paisagens, flores, árvores, edifícios e determinados fenômenos naturais}
\end{EntryWithPhonetic}

\begin{EntryWithPhonetic}{风浪}{feng1lang4}{4,10}{⾵、⽔}[HSK 7-9]
  \definition{s.}{tempestade; ondas tempestuosas; vento e ondas na água | dificuldades; sofrimento; uma metáfora para uma experiência difícil ou perigosa}
\end{EntryWithPhonetic}

\begin{EntryWithPhonetic}{风力}{feng1li4}{4,2}{⾵、⼒}[HSK 7-9]
  \definition{s.}{força do vento; o poder do vento | energia eólica; alimentado pelo vento}
\end{EntryWithPhonetic}

\begin{EntryWithPhonetic}{风流}{feng1liu2}{4,10}{⾵、⽔}[HSK 7-9]
  \definition{adj.}{refinado e saboroso | descontrolado em espírito e comportamento | romântico; amoroso; licencioso | distinto e admirável | meritório e talentoso | talentoso e romântico; talentoso em letras e não convencional no estilo de vida | dissoluto; solto}
\end{EntryWithPhonetic}

\begin{EntryWithPhonetic}{风貌}{feng1mao4}{4,14}{⾵、⾘}[HSK 7-9]
  \definition{s.}{estilo e características; o estilo e a aparência das coisas | vista; cena; cenário | aparência e porte elegantes; o comportamento e a aparência de uma pessoa}
\end{EntryWithPhonetic}

\begin{EntryWithPhonetic}{风气}{feng1qi4}{4,4}{⾵、⽓}[HSK 7-9]
  \definition[种]{s.}{ethos; atmosfera; humor geral; prática comum; um \emph{hobby} ou hábito popular na sociedade ou em um grupo}
  \seealsoref{风尚}{feng1shang4}
\end{EntryWithPhonetic}

\begin{EntryWithPhonetic}{风情}{feng1qing2}{4,11}{⾵、⼼}[HSK 7-9]
  \definition{s.}{charme; bom temperamento; graça; expressão | charme; panorama; gosto elegante | charme; sentimentos amorosos; pornografia | costumes e práticas locais | a condição do vento; informações sobre direção e velocidade do vento | sensação; a sensação do ambiente circundante em um determinado ambiente}
\end{EntryWithPhonetic}

\begin{EntryWithPhonetic}{风趣}{feng1qu4}{4,15}{⾵、⾛}[HSK 7-9]
  \definition{adj.}{espirituoso; humorístico; (fala, texto, etc.) humorístico e interessante}
  \definition{s.}{sagacidade; humor; refere-se ao humor e ao gosto humorísticos e interessantes}
\end{EntryWithPhonetic}

\begin{EntryWithPhonetic}{风沙}{feng1sha1}{4,7}{⾵、⽔}[HSK 7-9]
  \definition{s.}{areia soprada pelo vento; areia levada pelo vento}
\end{EntryWithPhonetic}

\begin{EntryWithPhonetic}{风扇}{feng1shan4}{4,10}{⾵、⼾}
  \definition{s.}{ventilador elétrico}
\end{EntryWithPhonetic}

\begin{EntryWithPhonetic}{风尚}{feng1shang4}{4,8}{⾵、⼩}[HSK 7-9]
  \definition{s.}{costume (ou prática, hábito) predominante; as tendências e costumes sociais predominantes durante um determinado período}
  \seealsoref{风气}{feng1qi4}
\end{EntryWithPhonetic}

\begin{EntryWithPhonetic}{风水}{feng1shui3}{4,4}{⾵、⽔}[HSK 7-9]
  \definition*{s.}{Feng Shui}
  \definition{s.}{geomancia; refere-se à localização geográfica de locais residenciais, cemitérios, etc., como a direção dos veios de terra, montanhas e rios}
\end{EntryWithPhonetic}

\begin{EntryWithPhonetic}{风俗}{feng1su2}{4,9}{⾵、⼈}[HSK 4]
  \definition[种,个,些]{s.}{costumes; a soma de costumes sociais, maneiras, hábitos, etc., desenvolvidos ao longo do tempo}
\end{EntryWithPhonetic}

\begin{EntryWithPhonetic}{风味}{feng1wei4}{4,8}{⾵、⼝}[HSK 7-9]
  \definition{s.}{sabor especial; cor (ou sabor) local; características das coisas (referindo-se principalmente às características locais)}
\end{EntryWithPhonetic}

\begin{EntryWithPhonetic}{风险}{feng1xian3}{4,9}{⾵、⾩}[HSK 3]
  \definition[个,种]{s.}{risco; perigo; ameaça; riscos possíveis}
\end{EntryWithPhonetic}

\begin{EntryWithPhonetic}{风雨}{feng1yu3}{4,8}{⾵、⾬}[HSK 7-9]
  \definition{s.}{o clima; vento e chuva | tribulações; estresse e tempestade; provações e dificuldades; uma metáfora para uma situação difícil e dura}
\end{EntryWithPhonetic}

\begin{EntryWithPhonetic}{风云}{feng1yun2}{4,4}{⾵、⼆}[HSK 7-9]
  \definition{s.}{vento e nuvem; uma situação tempestuosa ou instável}
\end{EntryWithPhonetic}

\begin{EntryWithPhonetic}{风筝}{feng1zheng5}{4,12}{⾵、⽵}[HSK 7-9]
  \definition[个,只]{s.}{pipa; papagaio; pandorga; é feito de tiras de bambu amarradas em um esqueleto em forma de pássaros, insetos, peixes, dragões, etc., coberto com papel ou seda, e flutua no ar com a ajuda do vento, as pessoas puxam a longa linha amarrada a ele para controlá-lo}
\end{EntryWithPhonetic}

%%%%%%%%%% 枫 %%%%%%%%%%
\subsection*{枫}

\begin{EntryWithPhonetic}{枫}{feng1}{8}{⽊}
  \definition[棵]{s.}{goma doce chinesa | bordo; \emph{maple}}
\end{EntryWithPhonetic}

\begin{EntryWithPhonetic}{枫叶}{feng1ye4}{8,5}{⽊、⼝}
  \definition{s.}{folha de bordo (maple, tipo de árvore)}
\end{EntryWithPhonetic}

%%%%%%%%%% 封 %%%%%%%%%%
\subsection*{封}

\begin{EntryWithPhonetic}{封}{feng1}{9}{⼨}[HSK 2,5]
  \definition*{s.}{Sobrenome: Feng}
  \definition{clas.}{usado para objetos selados, especialmente cartas}
  \definition{s.}{feudalismo | embalagem; envelope | pacote}
  \definition{v.}{conferir (um título, território, etc.) a | selar | acender uma fogueira | fechar}
\end{EntryWithPhonetic}

\begin{EntryWithPhonetic}{封闭}{feng1bi4}{9,6}{⼨、⾨}[HSK 4]
  \definition{adj.}{fechado; aqueles que não têm contato com o mundo exterior; aqueles que são muito conservadores (em seu pensamento) e não se comunicam com os outros}
  \definition{v.}{selar; fechar; lacrar; vedar; de modo a impedir a passagem, o uso ou a abertura}
\end{EntryWithPhonetic}

\begin{EntryWithPhonetic}{封底}{feng1di3}{9,8}{⼨、⼴}
  \definition{s.}{contracapa de um livro}
\end{EntryWithPhonetic}

\begin{EntryWithPhonetic}{封顶}{feng1ding3}{9,8}{⼨、⾴}[HSK 7-9]
  \definition{v.}{parar de crescer (de broto ou galho de planta) | Figurativo: impor um teto (sobre preços, salários, bônus, etc.) | cobrir (um edifício, etc.); cobrir o telhado (finalizar um projeto de construção); colocar um telhado (em um edifício); completar; encerrar | Figurativo: atingir o ponto mais alto (de crescimento, lucro, taxas de juros)}
\end{EntryWithPhonetic}

\begin{EntryWithPhonetic}{封冻}{feng1dong4}{9,7}{⼨、⼎}
  \definition{v.}{congelar (água ou terra)}
\end{EntryWithPhonetic}

\begin{EntryWithPhonetic}{封盖}{feng1gai4}{9,11}{⼨、⽫}
  \definition{s.}{arremesso bloqueado (basquete) | boné | capa | selo}
  \definition{v.}{(no basquete) bloquear (um arremesso) | cobrir}
\end{EntryWithPhonetic}

\begin{EntryWithPhonetic}{封建}{feng1jian4}{9,8}{⼨、⼵}[HSK 7-9]
  \definition{adj.}{feudal}
  \definition{s.}{feudalismo; sistema de feudo; o sistema político de feudos e estabelecimento de estados vassalos; esse sistema foi implementado pela primeira vez durante a Dinastia Zhou Ocidental | ideologia feudal; pensamento feudal}
\end{EntryWithPhonetic}

\begin{EntryWithPhonetic}{封口}{feng1kou3}{9,3}{⼨、⼝}
  \definition{v.}{selar | fechar | curar (uma ferida) | manter os lábios selados}
\end{EntryWithPhonetic}

\begin{EntryWithPhonetic}{封面}{feng1mian4}{9,9}{⼨、⾯}[HSK 7-9]
  \definition{s.}{capa (de uma publicação) ; capa frontal; sobrecapa}
\end{EntryWithPhonetic}

\begin{EntryWithPhonetic}{封锁}{feng1suo3}{9,12}{⼨、⾦}[HSK 7-9]
  \definition{v.}{bloquear; selar; tomar medidas militares, etc.) impedir a passagem}
\end{EntryWithPhonetic}

\begin{EntryWithPhonetic}{封印}{feng1yin4}{9,5}{⼨、⼙}
  \definition{s.}{selo (em envelopes)}
\end{EntryWithPhonetic}

\begin{EntryWithPhonetic}{封斋}{feng1zhai1}{9,10}{⼨、⽂}
  \definition*{s.}{Ramadã (Islã)}
\end{EntryWithPhonetic}

%%%%%%%%%% 疯 %%%%%%%%%%
\subsection*{疯}

\begin{EntryWithPhonetic}{疯}{feng1}{9}{⽧}[HSK 5]
  \definition{adj.}{louco; insano | tolo; leviano | (de uma planta, safra de grãos, etc.) esguia; refere-se ao crescimento vigoroso das plantações, mas sem frutos | com todas as forças; fazer o máximo possível}
  \definition{v.}{jogar sem restrições}
\end{EntryWithPhonetic}

\begin{EntryWithPhonetic}{疯狂}{feng1kuang2}{9,7}{⽧、⽝}[HSK 5]
  \definition{adj.}{louco; insano; frenético; desenfreado}
\end{EntryWithPhonetic}

\begin{EntryWithPhonetic}{疯子}{feng1zi5}{9,3}{⽧、⼦}[HSK 7-9]
  \definition[个,些,种]{s.}{maníaco; lunático; louco; pessoas com doenças mentais graves}[别把我当成疯子!===Não me trate como um louco!]
\end{EntryWithPhonetic}

%%%%%%%%%% 峰 %%%%%%%%%%
\subsection*{峰}

\begin{EntryWithPhonetic}{峰}{feng1}{10}{⼭}
  \definition{clas.}{usado para camelos}
  \definition{s.}{pico; cume; o pico proeminente de uma montanha | coisa parecida com um pico; coisas em forma de montanhas}
\end{EntryWithPhonetic}

\begin{EntryWithPhonetic}{峰回路转}{feng1hui2-lu4zhuan3}{10,6,13,8}{⼭、⼞、⾜、⾞}[HSK 7-9]
  \definition{expr.}{cume em meio a elevações circundantes e estradas sinuosas;  (estrada de montanha) torcendo e virando; a estrada da montanha serpenteia em torno de cada novo pico | boa (ou nova) reviravolta nos acontecimentos; uma oportunidade surgiu inesperadamente; as coisas tomaram um novo rumo}
\end{EntryWithPhonetic}

\begin{EntryWithPhonetic}{峰会}{feng1 hui4}{10,6}{⼭、⼈}[HSK 6]
  \definition{s.}{cúpula; reunião de cúpula}
\end{EntryWithPhonetic}

%%%%%%%%%% 蜂 %%%%%%%%%%
\subsection*{蜂}

\begin{EntryWithPhonetic}{蜂}{feng1}{13}{⾍}
  \definition{adv.}{em enxames}
  \definition[只,群,窝]{s.}{vespa | abelha}
\end{EntryWithPhonetic}

\begin{EntryWithPhonetic}{蜂蜜}{feng1mi4}{13,14}{⾍、⾍}[HSK 7-9]
  \definition[杯,瓶,罐,斤,碗,勺]{s.}{mel (de abelhas)}
\end{EntryWithPhonetic}

%%%%%%%%%% 逢 %%%%%%%%%%
\subsection*{逢}

\begin{EntryWithPhonetic}{逢}{feng2}{10}{⾡}[HSK 7-9]
  \definition*{s.}{Sobrenome: Feng}
  \definition{v.}{encontrar; vir até; encontrar-se por acaso}
\end{EntryWithPhonetic}

%%%%%%%%%% 缝 %%%%%%%%%%
\subsection*{缝}

\begin{EntryWithPhonetic}{缝}{feng2}{13}{⽷}[HSK 7-9]
  \definition{v.}{costurar; dar um ponto}
  \seeref{feng4}
\end{EntryWithPhonetic}

\begin{EntryWithPhonetic}{缝合}{feng2he2}{13,6}{⽷、⼝}[HSK 7-9]
  \definition{v.}{suturar; costurar uma ferida com agulhas e linhas especiais}
\end{EntryWithPhonetic}

\begin{EntryWithPhonetic}{缝纫}{feng2ren4}{13,6}{⽷、⽷}
  \definition{v.}{costurar}
\end{EntryWithPhonetic}

\begin{EntryWithPhonetic}{缝纫机}{feng2ren4ji1}{13,6,6}{⽷、⽷、⽊}
  \definition[架]{s.}{máquina de costura}
\end{EntryWithPhonetic}

%%%%%%%%%% 讽 %%%%%%%%%%
\subsection*{讽}

\begin{EntryWithPhonetic}{讽}{feng3}{6}{⾔}
  \definition{v.}{satirizar; zombar | Literário: cantar; entoar}
\end{EntryWithPhonetic}

\begin{EntryWithPhonetic}{讽刺}{feng3ci4}{6,8}{⾔、⼑}[HSK 7-9]
  \definition{adj.}{irônico; satírico; sarcástico}
  \definition{v.}{satirizar; ridicularizar; usar metáforas, exageros, ironia e outras expressões para expor, criticar ou ridicularizar}
\end{EntryWithPhonetic}

%%%%%%%%%% 凤 %%%%%%%%%%
\subsection*{凤}

\begin{EntryWithPhonetic}{凤}{feng4}{4}{⼏}
  \definition*{s.}{Sobrenome: Feng}
  \definition[只]{s.}{Mitologia: fênix}
\end{EntryWithPhonetic}

\begin{EntryWithPhonetic}{凤凰}{feng4huang2}{4,11}{⼏、⼏}[HSK 7-9]
  \definition[只,个,对]{s.}{fênix; o rei dos pássaros nas lendas antigas, com belas penas, o macho é chamado de 凤 e a fêmea é chamada de 凰, frequentemente usado para simbolizar auspiciosidade}
  \seealsoref{凤}{feng4}
  \seealsoref{凰}{huang2}
\end{EntryWithPhonetic}

%%%%%%%%%% 奉 %%%%%%%%%%
\subsection*{奉}

\begin{EntryWithPhonetic}{奉}{feng4}{8}{⼤}
  \definition*{s.}{Sobrenome: Feng}
  \definition{v.}{Literário: dedicar ou presentear com respeito | receber (pedidos, instruções, etc.) | Literário: estimar; reverenciar | Litrário: acreditar em  | esperar; atender; servir}
\end{EntryWithPhonetic}

\begin{EntryWithPhonetic}{奉献}{feng4xian4}{8,13}{⼤、⽝}[HSK 6]
  \definition{v.}{dedicar; oferecer como tributo; apresentar com todo respeito; entregar respeitosamente}
\end{EntryWithPhonetic}

%%%%%%%%%% 缝 %%%%%%%%%%
\subsection*{缝}

\begin{EntryWithPhonetic}{缝}{feng4}{13}{⽷}[HSK 7-9]
  \definition[道]{s.}{costura | fenda; rachadura; fissura; brecha; fresta}
  \seeref{feng2}
\end{EntryWithPhonetic}

%%%%%%%%%% 佛 %%%%%%%%%%
\subsection*{佛}

\begin{EntryWithPhonetic}{佛}{fo2}{7}{⼈}[HSK 6]
  \definition*{s.}{Buda, abreviação de 佛陀 | Budismo}
  \definition{s.}{imagem de Buda | budista | nome de Buda; escritura budista | uma pessoa que alcançou a perfeição na prática espiritual; budista real | estátua do Buda}
  \seeref{fu2}
  \seealsoref{佛陀}{fo2tuo2}
\end{EntryWithPhonetic}

\begin{EntryWithPhonetic}{佛教}{fo2 jiao4}{7,11}{⼈、⽁}[HSK 6]
  \definition*{s.}{Budismo; uma das principais religiões do mundo, diz-se que foi fundada por Sakyamuni, um príncipe do antigo reino indiano de Kapilavastu (no atual Nepal), no século VI ou V a.C.; foi amplamente difundida em muitos países asiáticos e introduzida na China no final da Dinastia Han Ocidental}
\end{EntryWithPhonetic}

\begin{EntryWithPhonetic}{佛陀}{fo2tuo2}{7,7}{⼈、⾩}
  \definition{s.}{Buda, um título para Sakyamuni ou uma pessoa que atingiu a iluminação | Buda, uma pessoa que atingiu a Budeidade, ou especificamente Siddhartha Gautama}
\end{EntryWithPhonetic}

%%%%%%%%%% 否 %%%%%%%%%%
\subsection*{否}

\begin{EntryWithPhonetic}{否}{fou3}{7}{⼝}
  \definition{adv.}{não; expressa discordância, equivalente à palavra falada 不 | usado no final de uma pergunta para indicar investigação | 是否, 能否 e 可否 que significa respectivamente 是不是, 能不能 e 可不可}
  \definition{v.}{negar}
  \seeref{pi3}
  \seealsoref{不}{bu4}
  \seealsoref{可}{ke3}
  \seealsoref{能}{neng2}
  \seealsoref{是}{shi4}
\end{EntryWithPhonetic}

\begin{EntryWithPhonetic}{否定}{fou3ding4}{7,8}{⼝、⼧}[HSK 3]
  \definition{adj.}{negativo; contrário}
  \definition{v.}{rejeitar; negar a existência ou a autenticidade de algo}
\end{EntryWithPhonetic}

\begin{EntryWithPhonetic}{否决}{fou3jue2}{7,6}{⼝、⼎}[HSK 7-9]
  \definition{v.}{rejeitar; votar contra; vetar; anular}
\end{EntryWithPhonetic}

\begin{EntryWithPhonetic}{否认}{fou3ren4}{7,4}{⼝、⾔}[HSK 3]
  \definition{v.}{negar; repudiar; não reconhecer}
\end{EntryWithPhonetic}

\begin{EntryWithPhonetic}{否则}{fou3ze2}{7,6}{⼝、⼑}[HSK 4]
  \definition{conj.}{senão; se não; ou então; se não for isso}
\end{EntryWithPhonetic}

%%%%%%%%%% 夫 %%%%%%%%%%
\subsection*{夫}

\begin{EntryWithPhonetic}{夫}{fu1}{4}{⼤}
  \definition{s.}{marido | homem | (velho) alguém que faz algum tipo de trabalho manual | (velho) uma pessoa que serviu em trabalho forçado}
  \seeref{fu2}
\end{EntryWithPhonetic}

\begin{EntryWithPhonetic}{夫妇}{fu1fu4}{4,6}{⼤、⼥}[HSK 4]
  \definition[对]{s.}{casal; marido e mulher}
\end{EntryWithPhonetic}

\begin{EntryWithPhonetic}{夫妻}{fu1qi1}{4,8}{⼤、⼥}[HSK 4]
  \definition[对]{s.}{casal; marido e mulher}
\end{EntryWithPhonetic}

\begin{EntryWithPhonetic}{夫人}{fu1ren2}{4,2}{⼤、⼈}[HSK 4]
  \definition[位,名,个]{s.}{senhora; \emph{lady}; madame; na antiguidade, as esposas dos senhores feudais eram chamadas de ``madame'' e, nas dinastias Ming e Qing, as esposas dos oficiais de primeiro e segundo escalão eram chamadas de ``madame'', que mais tarde foi usada para homenagear as esposas das pessoas em geral e agora é usada principalmente em ocasiões diplomáticas}
\end{EntryWithPhonetic}

%%%%%%%%%% 孵 %%%%%%%%%%
\subsection*{孵}

\begin{EntryWithPhonetic}{孵}{fu1}{14}{⼦}
  \definition{v.}{chocar; incubar; (pássaros) sentar em ovos}
\end{EntryWithPhonetic}

\begin{EntryWithPhonetic}{孵化}{fu1hua4}{14,4}{⼦、⼔}[HSK 7-9]
  \definition{v.}{chocar; incubar | incubar; metaforicamente, cultivar e desenvolver coisas novas (agora se refere principalmente ao suporte a empresas de alta tecnologia recém-criadas)}
\end{EntryWithPhonetic}

%%%%%%%%%% 敷 %%%%%%%%%%
\subsection*{敷}

\begin{EntryWithPhonetic}{敷}{fu1}{15}{⽁}[HSK 7-9]
  \definition*{s.}{Sobrenome: Fu}
  \definition{v.}{aplicar (pó, pomada, etc.) | espalhar; dispor | ser suficiente para | espalhar-se}
\end{EntryWithPhonetic}

%%%%%%%%%% 夫 %%%%%%%%%%
\subsection*{夫}

\begin{EntryWithPhonetic}{夫}{fu2}{4}{⼤}
  \definition{part.}{usado no início de uma frase | usado no final de uma frase ou em uma pausa no meio de uma frase para expressar uma exclamação}
  \definition{pron.}{isto; isso; aqueles; estes | ele}
  \seeref{fu1}
\end{EntryWithPhonetic}

%%%%%%%%%% 佛 %%%%%%%%%%
\subsection*{佛}

\begin{EntryWithPhonetic}{佛}{fu2}{7}{⼈}
  \definition{adv.}{aparentemente}
  \definition{s.}{ornamento da cabeça (feminino)}
  \seeref{fo2}
\end{EntryWithPhonetic}

%%%%%%%%%% 扶 %%%%%%%%%%
\subsection*{扶}

\begin{EntryWithPhonetic}{扶}{fu2}{7}{⼿}[HSK 5]
  \definition*{s.}{Sobrenome: Fu}
  \definition{v.}{segurar; apoiar com a mão; segurar algo com o apoio das mãos para que ninguém, objeto ou pessoa caia | dar apoio a; ajudar uma pessoa deitada ou caída a se levantar com as mãos; endireitar um objeto caído com as mãos | ajudar; tirar de baixo}
\end{EntryWithPhonetic}

\begin{EntryWithPhonetic}{扶持}{fu2chi2}{7,9}{⼿、⼿}[HSK 7-9]
  \definition{v.}{apoiar com a mão; colocar uma mão em alguém para apoio; apoiar | apoiar; dar ajuda a; ajudar a sustentar}
\end{EntryWithPhonetic}

\begin{EntryWithPhonetic}{扶梯}{fu2ti1}{7,11}{⼿、⽊}
  \definition{s.}{escada rolante}
\end{EntryWithPhonetic}

%%%%%%%%%% 服 %%%%%%%%%%
\subsection*{服}

\begin{EntryWithPhonetic}{服}{fu2}{8}{⽉}[HSK 6]
  \definition*{s.}{Sobrenome: Fu}
  \definition{s.}{roupas | vestuário de luto; refere-se a roupas de luto}
  \definition{v.}{vestir (roupas) | tomar (remédio) | envolver-se em; servir | obedecer; ser convencido | convencer; persuadir | adaptar-se; acostumar-se a}
  \seeref{fu4}
\end{EntryWithPhonetic}

\begin{EntryWithPhonetic}{服从}{fu2cong2}{8,4}{⽉、⼈}[HSK 5]
  \definition{v.}{obedecer; submeter-se a; estar subordinado a}
\end{EntryWithPhonetic}

\begin{EntryWithPhonetic}{服饰}{fu2shi4}{8,8}{⽉、⾷}[HSK 7-9]
  \definition[套]{s.}{roupa; vestido; traje; vestimenta e adorno pessoal}
\end{EntryWithPhonetic}

\begin{EntryWithPhonetic}{服务}{fu2 wu4}{8,5}{⽉、⼒}[HSK 2]
  \definition{v.}{prestar serviço a; estar a serviço de; servir; trabalhar para o benefício coletivo (ou de outras pessoas) ou para uma causa específica | trabalhar; servir}
\end{EntryWithPhonetic}

\begin{EntryWithPhonetic}{服务器}{fu2wu4qi4}{8,5,16}{⽉、⼒、⼝}[HSK 7-9]
  \definition[个,台]{s.}{Computção: servidor; um dispositivo dedicado que fornece serviços aos usuários em uma rede eletrônica de computadores}
\end{EntryWithPhonetic}

\begin{EntryWithPhonetic}{服务员}{fu2wu4yuan2}{8,5,7}{⽉、⼒、⼝}
  \definition{s.}{atendente | garçom | garçonete | pessoal de atendimento ao cliente}
\end{EntryWithPhonetic}

\begin{EntryWithPhonetic}{服用}{fu2yong4}{8,5}{⽉、⽤}[HSK 7-9]
  \definition{v.}{tomar (remédio)}[他已开始服用这种药。===Ele começou a tomar o remédio.]
\end{EntryWithPhonetic}

\begin{EntryWithPhonetic}{服装}{fu2zhuang1}{8,12}{⽉、⾐}[HSK 3]
  \definition[套,件,身]{s.}{roupas; vestuário; trajes; termo genérico para roupas, sapatos e chapéus, geralmente referido especificamente a roupas}
\end{EntryWithPhonetic}

%%%%%%%%%% 俘 %%%%%%%%%%
\subsection*{俘}

\begin{EntryWithPhonetic}{俘}{fu2}{9}{⼈}
  \definition{s.}{prisioneiro de guerra; cativo}
  \definition{v.}{capturar; fazer prisioneiro | fazer prisioneiro de guerra}
\end{EntryWithPhonetic}

\begin{EntryWithPhonetic}{俘获}{fu2huo4}{9,10}{⼈、⾋}[HSK 7-9]
  \definition{s.}{Física: captura; aprisionamento}[中子俘获===Captura de nêutrons]
  \definition{v.}{capturar; apreender}[军队成功俘获敌方指挥官。===O exército conseguiu capturar o comandante inimigo.]
\end{EntryWithPhonetic}

\begin{EntryWithPhonetic}{俘虏}{fu2lu3}{9,8}{⼈、⾌}[HSK 7-9]
  \definition[个,名,批,群]{s.}{cativo; prisioneiro de guerra; inimigos capturados durante a batalha}
  \definition{v.}{capturar (um inimigo) durante o combate; fazer prisioneiro; capturar}
\end{EntryWithPhonetic}

%%%%%%%%%% 浮 %%%%%%%%%%
\subsection*{浮}

\begin{EntryWithPhonetic}{浮}{fu2}{10}{⽔}[HSK 6]
  \definition*{s.}{Sobrenome: Fu}
  \definition{adj.}{superficial; na superfície | móvel; removível | temporário; provisório | superficial e frívolo; volátil; impetuoso | oco; vazio; inflado | excessivo; excedente}
  \definition{v.}{flutuar (oposto a 沉) | (dialeto) nadar | flutuar; derivar; flutuar na superfície do líquido}
  \seealsoref{沉}{chen2}
\end{EntryWithPhonetic}

\begin{EntryWithPhonetic}{浮力}{fu2li4}{10,2}{⽔、⼒}[HSK 7-9]
  \definition{s.}{flutuabilidade; a força de empuxo, a força ascendente exercida sobre um objeto em um fluido, é igual ao peso do fluido deslocado pelo objeto}
\end{EntryWithPhonetic}

\begin{EntryWithPhonetic}{浮图}{fu2tu2}{10,8}{⽔、⼞}
  \definition*{s.}{Termo alternativo para 佛陀}
  \variantof{浮屠}
  \seealsoref{佛陀}{fo2tuo2}
\end{EntryWithPhonetic}

\begin{EntryWithPhonetic}{浮屠}{fu2tu2}{10,11}{⽔、⼫}
  \definition*{s.}{Buda | Templo (Stupa) Budista (transliteração de Pali Thuo)}
\end{EntryWithPhonetic}

\begin{EntryWithPhonetic}{浮现}{fu2xian4}{10,8}{⽔、⾒}[HSK 7-9]
  \definition{v.}{(experiência passada) ressurgir; vir à mente | aparecer; apresentar; revelar}
\end{EntryWithPhonetic}

\begin{EntryWithPhonetic}{浮躁}{fu2zao4}{10,20}{⽔、⾜}[HSK 7-9]
  \definition{adj.}{inquieto; impetuoso; impaciente; frívolo e impaciente}
\end{EntryWithPhonetic}

%%%%%%%%%% 符 %%%%%%%%%%
\subsection*{符}

\begin{EntryWithPhonetic}{符}{fu2}{11}{⽵}
  \definition*{s.}{Sobrenome: Fu}
  \definition[个]{s.}{registro emitido por um governante para generais, enviados, etc., como credenciais na China antiga | símbolo; emblema | figuras mágicas desenhadas por sacerdotes taoístas para invocar ou expulsar espíritos e trazer boa ou má sorte | marca; sinal}
  \definition{v.}{(usado com 相 xiāng ou 不) coincidir com; concordar com | encaixar bem; combinar com; em conformidade com}
  \seealsoref{不}{bu4}
  \seealsoref{相}{xiang1}
\end{EntryWithPhonetic}

\begin{EntryWithPhonetic}{符号}{fu2hao4}{11,5}{⽵、⼝}[HSK 4]
  \definition[个]{s.}{marca; símbolo; sinais que marcam as coisas | insígnia; emblema; um símbolo usado no corpo para indicar posição, \emph{status}, etc.}
\end{EntryWithPhonetic}

\begin{EntryWithPhonetic}{符合}{fu2he2}{11,6}{⽵、⼝}[HSK 4]
  \definition{v.}{conformar-se com, estar de acordo com, estar em conformidade com}
\end{EntryWithPhonetic}

%%%%%%%%%% 幅 %%%%%%%%%%
\subsection*{幅}

\begin{EntryWithPhonetic}{幅}{fu2}{12}{⼱}[HSK 5]
  \definition{clas.}{usado para tecidos, telas de lã, pinturas, etc.}
  \definition{s.}{largura do tecido, seda, tweed, etc. | tamanho; largura; geralmente se refere à largura}
\end{EntryWithPhonetic}

\begin{EntryWithPhonetic}{幅度}{fu2du4}{12,9}{⼱、⼴}[HSK 5]
  \definition{s.}{alcance; escopo; extensão; largura; largura da propagação de um objeto que vibra ou balança, uma metáfora para a magnitude de uma mudança em algo}
\end{EntryWithPhonetic}

%%%%%%%%%% 福 %%%%%%%%%%
\subsection*{福}

\begin{EntryWithPhonetic}{福}{fu2}{13}{⽰}[HSK 3]
  \definition*{s.}{Província de Fujian | Sobrenome: Fu}
  \definition{s.}{benção; felicidade; boa sorte; boa fortuna; sorte (oposto de 祸)}
  \definition{v.}{(de uma mulher) fazer uma reverência; antigamente, as mulheres faziam a reverência 万福 (colocando as duas mãos na cintura do mesmo lado e dobrando ligeiramente os joelhos)}
  \seealsoref{祸}{huo4}
  \seealsoref{万福}{wan4fu2}
\end{EntryWithPhonetic}

\begin{EntryWithPhonetic}{福克斯}{fu2ke4si1}{13,7,12}{⽰、⼗、⽄}
  \definition*{s.}{Fox (empresa de mídia) | Focus (automóvel fabricado pela Ford)}
\end{EntryWithPhonetic}

\begin{EntryWithPhonetic}{福利}{fu2li4}{13,7}{⽰、⼑}[HSK 5]
  \definition[项,种]{s.}{bem-estar; benefícios materiais}
  \definition{v.}{melhorar suas condições de vida; facilitar a vida}
\end{EntryWithPhonetic}

\begin{EntryWithPhonetic}{福气}{fu2qi5}{13,4}{⽰、⽓}[HSK 7-9]
  \definition{s.}{bênção; boa sorte; refere-se ao destino de desfrutar de uma vida feliz}
\end{EntryWithPhonetic}

\begin{EntryWithPhonetic}{福泽}{fu2ze2}{13,8}{⽰、⽔}
  \definition{s.}{boa sorte}
\end{EntryWithPhonetic}

%%%%%%%%%% 辐 %%%%%%%%%%
\subsection*{辐}

\begin{EntryWithPhonetic}{辐}{fu2}{13}{⾞}
  \definition{s.}{raio (de uma roda); a conexão entre o cubo e o aro de uma roda}
\end{EntryWithPhonetic}

\begin{EntryWithPhonetic}{辐射}{fu2she4}{13,10}{⾞、⼨}[HSK 7-9]
  \definition{s.}{radiação;  uma forma de propagação de calor que se irradia da fonte de calor em linha reta para a área circundante; a propagação de ondas eletromagnéticas, como luz e ondas de rádio}
  \definition{v.}{irradiar}
\end{EntryWithPhonetic}

%%%%%%%%%% 抚 %%%%%%%%%%
\subsection*{抚}

\begin{EntryWithPhonetic}{抚}{fu3}{7}{⼿}
  \definition{v.}{confortar; consolar | nutrir; fomentar | Literário: acariciar | proteger; promover; criar | o mesmo que 拊}
  \seealsoref{拊}{fu3}
\end{EntryWithPhonetic}

\begin{EntryWithPhonetic}{抚摸}{fu3mo1}{7,13}{⼿、⼿}[HSK 7-9]
  \definition{v.}{acariciar; afagar; amimar}
\end{EntryWithPhonetic}

\begin{EntryWithPhonetic}{抚恤}{fu3xu4}{7,9}{⼿、⼼}[HSK 7-9]
  \definition{v.}{(estado ou organização) fornecer conforto e assistência material às famílias de pessoal ferido ou incapacitado no cumprimento do dever, ou daqueles que morreram de doença ou morreram no cumprimento do dever}
\end{EntryWithPhonetic}

\begin{EntryWithPhonetic}{抚养}{fu3yang3}{7,9}{⼿、⼋}[HSK 7-9]
  \definition{v.}{criar; cuidar; proporcionar às crianças as condições de vida necessárias para que possam crescer com saúde}
\end{EntryWithPhonetic}

\begin{EntryWithPhonetic}{抚养费}{fu3yang3fei4}{7,9,9}{⼿、⼋、⾙}[HSK 7-9]
  \definition{s.}{pensão alimentícia (após o divórcio) | pagamento pela educação dos filhos (como após o divórcio)}
\end{EntryWithPhonetic}

%%%%%%%%%% 拊 %%%%%%%%%%
\subsection*{拊}

\begin{EntryWithPhonetic}{拊}{fu3}{8}{⼿}
  \definition{v.}{Literário: bater palmas; esbofetear; golpear}
\end{EntryWithPhonetic}

%%%%%%%%%% 斧 %%%%%%%%%%
\subsection*{斧}

\begin{EntryWithPhonetic}{斧}{fu3}{8}{⽄}
  \definition[把,只]{s.}{machado; machadinha | machado de batalha (um tipo de arma usada na China antiga)}
\end{EntryWithPhonetic}

\begin{EntryWithPhonetic}{斧子}{fu3zi5}{8,3}{⽄、⼦}[HSK 7-9]
  \definition[把,个]{s.}{machado; machadinha}
\end{EntryWithPhonetic}

%%%%%%%%%% 俯 %%%%%%%%%%
\subsection*{俯}

\begin{EntryWithPhonetic}{俯}{fu3}{10}{⼈}
  \definition{v.}{curvar (a cabeça), oposto a 仰 | inclinar-se | Obsoleto: (em documentos ou cartas oficiais) condescender com | curvar-se; fazer uma reverência}
  \seealsoref{仰}{yang3}
\end{EntryWithPhonetic}

\begin{EntryWithPhonetic}{俯首}{fu3shou3}{10,9}{⼈、⾸}[HSK 7-9]
  \definition{v.}{abaixar a cabeça; curvar-se; inclinar-se}
\end{EntryWithPhonetic}

%%%%%%%%%% 辅 %%%%%%%%%%
\subsection*{辅}

\begin{EntryWithPhonetic}{辅}{fu3}{11}{⾞}
  \definition*{s.}{Sobrenome: Fu}
  \definition{adj.}{subsidiário}
  \definition{s.}{barras laterais do carrinho atuando como proteção da roda; duas barras retas de madeira são adicionadas na parte externa da roda para prender o cubo | maçã do rosto | assistente oficial; títulos oficiais antigos | (literário) território que circunda a capital}
  \definition{v.}{auxiliar; complementar; suplementar | ajudar}
\end{EntryWithPhonetic}

\begin{EntryWithPhonetic}{辅导}{fu3dao3}{11,6}{⾞、⼨}[HSK 7-9]
  \definition{v.}{orientar no estudo ou treinamento; treinar; guiar; dar aulas particulares}
\end{EntryWithPhonetic}

\begin{EntryWithPhonetic}{辅助}{fu3zhu4}{11,7}{⾞、⼒}[HSK 5]
  \definition{adj.}{auxiliar; suplementar; complementar}
  \definition{v.}{auxiliar; ajudar; colocar os outros em primeiro lugar e dar-lhes alguma ajuda externa}
\end{EntryWithPhonetic}

%%%%%%%%%% 腐 %%%%%%%%%%
\subsection*{腐}

\begin{EntryWithPhonetic}{腐}{fu3}{14}{⾁}
  \definition{adj.}{podre; obsoleto; corrupto | corroído; pútrido}
  \definition{s.}{tofu}
  \definition{v.}{apodrecer; corroer; estragar; decair}
\end{EntryWithPhonetic}

\begin{EntryWithPhonetic}{腐败}{fu3bai4}{14,8}{⾁、⾒}[HSK 7-9]
  \definition{adj.}{(ideias) corrupto; decadente; (pensamento) obsoleto; (comportamento) degenerado | (sistema, organização, instituição, medida, etc.) corrupto}
  \definition{s.}{deterioração; podridão}
  \definition{v.}{apodrecer; decair}
\end{EntryWithPhonetic}

\begin{EntryWithPhonetic}{腐化}{fu3hua4}{14,4}{⾁、⼔}[HSK 7-9]
  \definition{adj.}{degenerado; corrupto, dissoluto ou depravado; desmoralizado; decadente}
  \definition{v.}{decompor; apodrecer; tornar-se pútrido | quebrar; corroer}
\end{EntryWithPhonetic}

\begin{EntryWithPhonetic}{腐烂}{fu3lan4}{14,9}{⾁、⽕}[HSK 7-9]
  \definition{adj.}{corrupto; depravado | (pensamentos) obsoletos; (comportamento) degenerado}
  \definition{v.}{apodrecer; decompor; tornar-se pútrido}
\end{EntryWithPhonetic}

\begin{EntryWithPhonetic}{腐蚀}{fu3shi2}{14,9}{⾁、⾷}[HSK 7-9]
  \definition{v.}{corroer; destruir gradualmente um objeto por meio de reações químicas | corroer; corromper (pensamentos e comportamentos)}
\end{EntryWithPhonetic}

\begin{EntryWithPhonetic}{腐朽}{fu3xiu3}{14,6}{⾁、⽊}[HSK 7-9]
  \definition{adj.}{decaído; decadente; degenerado; uma metáfora para as ideias ultrapassadas das pessoas ou para a moral social corrupta}
  \definition{v.}{apodrecer; decair; apodrecimento e deterioração da madeira e outros materiais fibrosos}
\end{EntryWithPhonetic}

%%%%%%%%%% 父 %%%%%%%%%%
\subsection*{父}

\begin{EntryWithPhonetic}{父}{fu4}{4}{⽗}[Kangxi 88]
  \definition{s.}{pai | um homem mais velho na família ou parentes | fundador; uma pessoa que inventa ou inicia algo}
\end{EntryWithPhonetic}

\begin{EntryWithPhonetic}{父母}{fu4 mu3}{4,5}{⽗、⽏}[HSK 3]
  \definition{s.}{pai e mãe; pais}
\end{EntryWithPhonetic}

\begin{EntryWithPhonetic}{父母亲}{fu4mu3qin1}{4,5,9}{⽗、⽏、⼇}
  \definition{s.}{os pais; pai e mãe}
\end{EntryWithPhonetic}

\begin{EntryWithPhonetic}{父女}{fu4 nv3}{4,3}{⽗、⼥}[HSK 6]
  \definition{s.}{pai e filha}
\end{EntryWithPhonetic}

\begin{EntryWithPhonetic}{父亲}{fu4qin1}{4,9}{⽗、⼇}[HSK 3]
  \definition[个,位,名]{s.}{pai; homem com filhos; pai dos filhos}
\end{EntryWithPhonetic}

\begin{EntryWithPhonetic}{父子}{fu4 zi3}{4,3}{⽗、⼦}[HSK 6]
  \definition{s.}{pai e filho}
\end{EntryWithPhonetic}

%%%%%%%%%% 讣 %%%%%%%%%%
\subsection*{讣}

\begin{EntryWithPhonetic}{讣}{fu4}{4}{⾔}
  \definition[个,则]{s.}{obituário}
  \definition{v.}{anunciar a morte de alguém | relatar um luto}
\end{EntryWithPhonetic}

%%%%%%%%%% 付 %%%%%%%%%%
\subsection*{付}

\begin{EntryWithPhonetic}{付}{fu4}{5}{⼈}[HSK 3]
  \definition*{s.}{Sobrenome: Fu}
  \definition{clas.}{usado para pares ou conjuntos de coisas | usado para expressões faciais}
  \definition{v.}{comprometer-se com; entregar (ou transferir) para | pagar; refere-se especificamente a dar dinheiro}
\end{EntryWithPhonetic}

\begin{EntryWithPhonetic}{付出}{fu4 chu1}{5,5}{⼈、⼐}[HSK 4]
  \definition{v.}{pagar; gastar; entregar (dinheiro, consideração, etc.)}
\end{EntryWithPhonetic}

\begin{EntryWithPhonetic}{付费}{fu4fei4}{5,9}{⼈、⾙}[HSK 7-9]
  \definition{v.}{pagar}
\end{EntryWithPhonetic}

\begin{EntryWithPhonetic}{付款}{fu4kuan3}{5,12}{⼈、⽋}[HSK 7-9]
  \definition[次]{s.}{pagamento}
  \definition{v.}{pagar uma quantia em dinheiro}
\end{EntryWithPhonetic}

%%%%%%%%%% 妇 %%%%%%%%%%
\subsection*{妇}

\begin{EntryWithPhonetic}{妇}{fu4}{6}{⼥}
  \definition{s.}{mulher | mulher casada | esposa}
\end{EntryWithPhonetic}

\begin{EntryWithPhonetic}{妇女}{fu4nv3}{6,3}{⼥、⼥}[HSK 6]
  \definition[个,位,群,名,帮]{s.}{mulher; mulheres; um termo geral para mulheres adultas}
\end{EntryWithPhonetic}

%%%%%%%%%% 负 %%%%%%%%%%
\subsection*{负}

\begin{EntryWithPhonetic}{负}{fu4}{6}{⾙}[HSK 6]
  \definition{adj.}{negativo; menor que zero | negativo; referindo-se ao que recebe elétrons (oposto a 正)}
  \definition{v.}{carregar; transportar nas costas ou nos ombros | suportar; assumir; encarar | confiar em; contar com; depender | sofrer | aproveitar; desfrutar | ter dívidas | trair; violar | perder; ser derrotado}
  \seealsoref{正}{zheng4}
\end{EntryWithPhonetic}

\begin{EntryWithPhonetic}{负担}{fu4dan1}{6,8}{⾙、⼿}[HSK 4]
  \definition{s.}{carga; fardo; frete; ônus; pressão ou responsabilidade, despesas, etc.}
  \definition{v.}{carregar; carregar (um fardo); assumir (responsabilidade, trabalho, despesas, etc.)}
\end{EntryWithPhonetic}

\begin{EntryWithPhonetic}{负面}{fu4mian4}{6,9}{⾙、⾯}[HSK 7-9]
  \definition{adj.}{ruim; negativo; prejudicial; desvantajoso}
  \definition{s.}{lado reverso; o negativo; refere-se aos aspectos ou partes ruins, negativas, prejudiciais ou desfavoráveis}
\end{EntryWithPhonetic}

\begin{EntryWithPhonetic}{负有}{fu4you3}{6,6}{⾙、⽉}[HSK 7-9]
  \definition{v.}{ser responsável por}
\end{EntryWithPhonetic}

\begin{EntryWithPhonetic}{负责}{fu4ze2}{6,8}{⾙、⾙}[HSK 3]
  \definition{adj.}{consciencioso; ser sério e responsável}
  \definition{v.}{ser responsável por; estar encarregado de; assumir responsabilidades}
\end{EntryWithPhonetic}

\begin{EntryWithPhonetic}{负责人}{fu4 ze2 ren2}{6,8,2}{⾙、⾙、⼈}[HSK 5]
  \definition[位]{s.}{pessoa responsável; pessoa encarregada; pessoas com responsabilidades de liderança}
\end{EntryWithPhonetic}

%%%%%%%%%% 附 %%%%%%%%%%
\subsection*{附}

\begin{EntryWithPhonetic}{附}{fu4}{7}{⾩}[HSK 7-9]
  \definition*{s.}{Sobrenome: Fu}
  \definition{v.}{adicionar; anexar; incluir | chegar perto de; estar perto de | depender de; confiar em; cumprir com | concordar com; anexar a; aderir a; cumprir com; depender de}
\end{EntryWithPhonetic}

\begin{EntryWithPhonetic}{附带}{fu4dai4}{7,9}{⾩、⼱}[HSK 7-9]
  \definition{adj.}{subsidiário; suplementar; incidental}
  \definition{adv.}{de passagem; a propósito; incidentalmente}
  \definition{v.}{anexar}
\end{EntryWithPhonetic}

\begin{EntryWithPhonetic}{附和}{fu4he4}{7,8}{⾩、⼝}[HSK 7-9]
  \definition{v.}{ecoar; entrar na conversa com; repetir o que os outros dizem (alguém disse)}
\end{EntryWithPhonetic}

\begin{EntryWithPhonetic}{附加}{fu4jia1}{7,5}{⾩、⼒}[HSK 7-9]
  \definition{adj.}{adicional; anexado; extra}
  \definition{v.}{adicionar; anexar; exceder a quantidade ou intervalo prescrito}
\end{EntryWithPhonetic}

\begin{EntryWithPhonetic}{附件}{fu4jian4}{7,6}{⾩、⼈}[HSK 5]
  \definition*{s.}{\emph{Adnexa Uteri}, refere-se à genitália interna feminina que não seja o útero, as trompas de falópio e os ovários}
  \definition{s.}{apêndice; documentos que acompanham o documento principal | acessório; anexo; peças ou sobressalentes que não sejam peças principais de máquinas e equipamentos | anexo; documentos ou itens relevantes emitidos com o documento}
\end{EntryWithPhonetic}

\begin{EntryWithPhonetic}{附近}{fu4jin4}{7,7}{⾩、⾡}[HSK 4]
  \definition{adj.}{perto; vizinho}
  \definition{s.}{vizinhança; bairro}
\end{EntryWithPhonetic}

\begin{EntryWithPhonetic}{附属}{fu4shu3}{7,12}{⾩、⼫}[HSK 7-9]
  \definition{adj.}{anexado; afiliado; dependente ou pertencente a uma instituição}
  \definition{v.}{afiliar-se; filiar-se; inscrever-se; agregar-se}
\end{EntryWithPhonetic}

%%%%%%%%%% 服 %%%%%%%%%%
\subsection*{服}

\begin{EntryWithPhonetic}{服}{fu4}{8}{⽉}
  \definition{clas.}{usado para remédio: dose; usado na medicina tradicional chinesa}
  \seeref{fu2}
\end{EntryWithPhonetic}

%%%%%%%%%% 复 %%%%%%%%%%
\subsection*{复}

\begin{EntryWithPhonetic}{复}{fu4}{9}{⼢}
  \definition*{s.}{Sobrenome: Fu}
  \definition{adj.}{composto; complexo; nem um único; dois ou mais}
  \definition{adv.}{de novo; novamente; indica o reaparecimento de uma situação, equivalente a 再}
  \definition{s.}{jaqueta; roupas forradas}
  \definition{v.}{virar; virar-se | responder; retornar | recuperar; retornar a; restaurar | vingar | duplicar; repetir}
  \seealsoref{再}{zai4}
\end{EntryWithPhonetic}

\begin{EntryWithPhonetic}{复查}{fu4cha2}{9,9}{⼢、⽊}[HSK 7-9]
  \definition{v.}{verificar novamente; reexaminar; revisar}
\end{EntryWithPhonetic}

\begin{EntryWithPhonetic}{复发}{fu4fa1}{9,5}{⼢、⼜}[HSK 7-9]
  \definition{v.}{ter uma recaída; recorrer | reaparecer; recrudescer | recorrer (de uma doença) | recair (em um antigo estado ruim)}
\end{EntryWithPhonetic}

\begin{EntryWithPhonetic}{复合}{fu4he2}{9,6}{⼢、⼝}[HSK 7-9]
  \definition{v.}{compor; tornar complexo; combinar; juntar | recombinar; juntar; metáfora para estarem juntos novamente depois de estarem separados}
\end{EntryWithPhonetic}

\begin{EntryWithPhonetic}{复活}{fu4huo2}{9,9}{⼢、⽔}[HSK 7-9]
  \definition{s.}{ressurreição (cristianismo)}
  \definition{v.}{reviver; voltar à vida; morrer e voltar à vida, frequentemente usado como metáfora}
\end{EntryWithPhonetic}

\begin{EntryWithPhonetic}{复活节}{fu4huo2jie2}{9,9,5}{⼢、⽔、⾋}
  \definition*{s.}{Páscoa; festival cristão que comemora a ressurreição de Jesus ocorre no primeiro domingo após a primeira lua cheia após o equinócio da primavera}
\end{EntryWithPhonetic}

\begin{EntryWithPhonetic}{复刻}{fu4ke4}{9,8}{⼢、⼑}
  \definition{v.}{reimprimir (um trabalho que esteve fora do catálogo) | reeditar (um disco de vinil, um CD, etc.) | replicar | recriar | (empréstimo linguístico) (computação) \emph{fork}}
\end{EntryWithPhonetic}

\begin{EntryWithPhonetic}{复苏}{fu4 su1}{9,7}{⼢、⾋}[HSK 6]
  \definition{s.}{recuperação}
  \definition{v.}{reviver; recuperar; ressuscitar; voltar à vida}
\end{EntryWithPhonetic}

\begin{EntryWithPhonetic}{复习}{fu4xi2}{9,3}{⼢、⼄}[HSK 2]
  \definition{s.}{revisão}
  \definition{v.}{revisar; corrigir (lições, etc.); repetir o que já aprendeu para consolidar o conhecimento}
\end{EntryWithPhonetic}

\begin{EntryWithPhonetic}{复兴}{fu4xing1}{9,6}{⼢、⼋}[HSK 7-9]
  \definition{v.}{reviver; rejuvenescer | reviver; desenvolver-se e tornar-se mais forte}
\end{EntryWithPhonetic}

\begin{EntryWithPhonetic}{复印}{fu4yin4}{9,5}{⼢、⼙}[HSK 3]
  \definition{v.}{fotografar; fotocopiar; duplicar; sem passar pelo processo de impressão, obter uma cópia diretamente do original (geralmente referindo-se à cópia feita com uma copiadora)}
\end{EntryWithPhonetic}

\begin{EntryWithPhonetic}{复元}{fu4/yuan2}{9,4}{⼢、⼉}
  \variantof{复原}
\end{EntryWithPhonetic}

\begin{EntryWithPhonetic}{复原}{fu4/yuan2}{9,10}{⼢、⼚}[HSK 7-9]
  \definition{s.}{reconversão; recuperação; redefinição; reabilitação; restauração; recura; analepsia; analepse}
  \definition{v.+compl.}{recuperar-se de uma doença; ter a saúde restaurada | restaurar; reabilitar}
\end{EntryWithPhonetic}

\begin{EntryWithPhonetic}{复杂}{fu4za2}{9,6}{⼢、⽊}[HSK 3]
  \definition{adj.}{complexo; complicado; em oposição a 单纯 e 简单}
  \seealsoref{单纯}{dan1chun2}
  \seealsoref{简单}{jian3dan1}
\end{EntryWithPhonetic}

\begin{EntryWithPhonetic}{复制}{fu4zhi4}{9,8}{⼢、⼑}[HSK 4]
  \definition{v.}{copiar; duplicar; reproduzir; fazer uma cópia de; fazer uma cópia do original ou reproduzi-lo, reimprimi-lo ou copiá-lo em sua forma original (geralmente referindo-se a relíquias culturais ou obras de arte)}
\end{EntryWithPhonetic}

%%%%%%%%%% 赴 %%%%%%%%%%
\subsection*{赴}

\begin{EntryWithPhonetic}{赴}{fu4}{9}{⾛}[HSK 7-9]
  \definition{v.}{ir para | comparecer}
\end{EntryWithPhonetic}

%%%%%%%%%% 副 %%%%%%%%%%
\subsection*{副}

\begin{EntryWithPhonetic}{副}{fu4}{11}{⼑}[HSK 6]
  \definition{adj.}{segundo em exercício; deputado; auxiliar | subsidiário; incidental; secundário}
  \definition{clas.}{usado para conjuntos completos de itens; usado para \emph{kits} | usado para expressões faciais | usado para som ou voz}
  \definition{pref.}{vice-}
  \definition{s.}{assistente; ajudante; auxiliar; posição auxiliar; pessoa que ocupa uma posição auxiliar}
  \definition{v.}{ajustar; corresponder a; conformar-se a}
\end{EntryWithPhonetic}

\begin{EntryWithPhonetic}{副研}{fu4yan2}{11,9}{⼑、⽯}
  \definition{s.}{pesquisador adjunto}
\end{EntryWithPhonetic}

\begin{EntryWithPhonetic}{副作用}{fu4zuo4yong4}{11,7,5}{⼑、⼈、⽤}[HSK 7-9]
  \definition{s.}{efeito colateral; efeitos adversos além dos efeitos principais}
\end{EntryWithPhonetic}

%%%%%%%%%% 富 %%%%%%%%%%
\subsection*{富}

\begin{EntryWithPhonetic}{富}{fu4}{12}{⼧}[HSK 3]
  \definition*{s.}{Sobrenome: Fu}
  \definition{adj.}{rico; abastado; abundante; refere-se a ter muito dinheiro (oposto de 贫) | rico; abundante}
  \definition{v.}{tornar-se rico; enriquecer}
  \seealsoref{贫}{pin2}
\end{EntryWithPhonetic}

\begin{EntryWithPhonetic}{富含}{fu4han2}{12,7}{⼧、⼝}[HSK 7-9]
  \definition{v.}{ser rico em}[橘子富含维生素。===As laranjas são ricas em vitaminas.]
\end{EntryWithPhonetic}

\begin{EntryWithPhonetic}{富豪}{fu4hao2}{12,14}{⼧、⾗}[HSK 7-9]
  \definition{s.}{rico; pessoa com muito dinheiro e grande poder}
\end{EntryWithPhonetic}

\begin{EntryWithPhonetic}{富强}{fu4qiang2}{12,12}{⼧、⼸}[HSK 7-9]
  \definition{adj.}{próspero e forte; próspero e poderoso; rico e poderoso; (país) rico e poderoso}
\end{EntryWithPhonetic}

\begin{EntryWithPhonetic}{富人}{fu4 ren2}{12,2}{⼧、⼈}[HSK 6]
  \definition{s.}{os ricos; os abastados}
\end{EntryWithPhonetic}

\begin{EntryWithPhonetic}{富翁}{fu4weng1}{12,10}{⼧、⽺}[HSK 7-9]
  \definition[个,名,位]{s.}{homem rico; homem de riqueza; pessoas que possuem muitas propriedades}
\end{EntryWithPhonetic}

\begin{EntryWithPhonetic}{富有}{fu4 you3}{12,6}{⼧、⽉}[HSK 6]
  \definition{adj.}{rico; abastado; possuir uma grande quantidade de propriedades | rico em espírito; metáfora para uma vida espiritual rica}
  \definition{v.}{ser rico ou abundante em; principalmente referindo-se a coisas abstratas com significados positivos que são suficientes}
\end{EntryWithPhonetic}

\begin{EntryWithPhonetic}{富裕}{fu4yu4}{12,12}{⼧、⾐}[HSK 7-9]
  \definition{adj.}{rico; próspero; abastado; em boas condições econômicas e com dinheiro suficiente}
\end{EntryWithPhonetic}

\begin{EntryWithPhonetic}{富足}{fu4zu2}{12,7}{⼧、⾜}[HSK 7-9]
  \definition{adj.}{rico; pleno; abundante}
\end{EntryWithPhonetic}

%%%%%%%%%% 赋 %%%%%%%%%%
\subsection*{赋}

\begin{EntryWithPhonetic}{赋}{fu4}{12}{⾙}
  \definition{s.}{dotação (natural) | Obsoleto: imposto territorial | fu, estilo antigo, uma forma literária complexa que combina elementos de poesia e prosa, muito cultivada desde a época Han até o período das Seis Dinastias}
  \definition{v.}{compor (versos); escrever poemas e letras | Literário: conceder; entregar; dotar com}
\end{EntryWithPhonetic}

\begin{EntryWithPhonetic}{赋予}{fu4yu3}{12,4}{⾙、⼅}[HSK 7-9]
  \definition{v.}{conceder; dotar (uma tarefa importante, missão, etc.); dar a alguém uma tarefa, responsabilidade, direito, autoridade, etc. | investir com (significado, característica, etc.); dar a algo uma cor, significado, importância, etc.}
\end{EntryWithPhonetic}

%%%%%%%%%% 腹 %%%%%%%%%%
\subsection*{腹}

\begin{EntryWithPhonetic}{腹}{fu4}{13}{⾁}
  \definition*{s.}{Sobrenome: Fu}
  \definition[个]{s.}{barriga (do corpo); abdômen; estômago | barriga (de uma garrafa, etc.) | coração; mente | parte vazia e saliente no meio de um recipiente ou vaso}
\end{EntryWithPhonetic}

\begin{EntryWithPhonetic}{腹部}{fu4bu4}{13,10}{⾁、⾢}[HSK 7-9]
  \definition{s.}{abdômen; estômago; barriga}
\end{EntryWithPhonetic}

\begin{EntryWithPhonetic}{腹泻}{fu4xie4}{13,8}{⾁、⽔}[HSK 7-9]
  \definition{s.}{diarreia; refere-se ao aumento da frequência de fezes aquosas, com pus ou com sangue, acompanhadas de dor abdominal, causadas por infecção intestinal ou disfunção digestiva}
\end{EntryWithPhonetic}

%%%%%%%%%% 覆 %%%%%%%%%%
\subsection*{覆}

\begin{EntryWithPhonetic}{覆}{fu4}{18}{⾑}
  \definition{v.}{cobrir; encapar | derrubar; perturbar; virar de cabeça para baixo}
\end{EntryWithPhonetic}

\begin{EntryWithPhonetic}{覆盖}{fu4gai4}{18,11}{⾑、⽫}[HSK 7-9]
  \definition{s.}{vegetação; cobertura vegetal; refere-se às plantas que cobrem o solo}
  \definition{v.}{cobrir}
\end{EntryWithPhonetic}

\begin{EntryWithPhonetic}{覆盆子}{fu4pen2zi5}{18,9,3}{⾑、⽫、⼦}
  \definition{s.}{framboesa}
\end{EntryWithPhonetic}

%%%%% EOF %%%%%


 %%%
%%% G
%%%
\section*{G}\addcontentsline{toc}{section}{G}\addcontentsline{loh}{figure}{\#\#\#\#\#\#\#\# G}

%%%%%%%%%% 夹 %%%%%%%%%%
\subsection*{夹}\addcontentsline{loh}{figure}{夹 \dpy{ga1}}

\begin{EntryWithPhonetic}{夹}{ga1}{6}{⼤}
  \definition{s.}{axila; sovaco; atualmente, costuma-se escrever 胳肢窝}
  \seeref{jia1}
  \seeref{jia2}
  \seealsoref{胳肢窝}{ga1 zhi1 wo1}
\end{EntryWithPhonetic}

%%%%%%%%%% 胳 %%%%%%%%%%
\subsection*{胳}\addcontentsline{loh}{figure}{胳 \dpy{ga1}}

\begin{EntryWithPhonetic}{胳}{ga1}{10}{⾁}
  \definition{s.}{usado em 胳肢窝}
  \seeref{ge1}
  \seeref{ge2}
  \seealsoref{胳肢窝}{ga1 zhi1 wo1}
\end{EntryWithPhonetic}

\begin{EntryWithPhonetic}{胳肢窝}{ga1 zhi1 wo1}{10,8,12}{⾁,⾁,⽳}
  \definition{s.}{axila; sovaco; também escrito 夹肢窝}
  \seealsoref{夹肢窝}{jia1 zhi1 wo1}
\end{EntryWithPhonetic}

%%%%%%%%%% 该 %%%%%%%%%%
\subsection*{该}\addcontentsline{loh}{figure}{该 \dpy{gai1}}

\begin{EntryWithPhonetic}{该}{gai1}{8}{⾔}[HSK 2,7-9]
  \definition{adj.}{completo; integral; abrangente; inclusivo; o mesmo que 赅}
  \definition{pron.}{isto; aquilo; o referido; o acima mencionado; indica a pessoa ou coisa mencionada acima, equivalente a 此, 这个, etc.}
  \definition{v.}{deveria ser; deveria ser assim | caber a alguém; ser a vez (ou dever) de alguém fazer algo | merecer; servir a alguém de direito; indica que algo deve ser feito | dever | deve; provavelmente irá; muito provavelmente; pode ser razoavelmente ou naturalmente esperado que; expressa uma conclusão lógica ou provável com base na razão ou na experiência}
  \definition{v.aux.}{usado em frases exclamativas, tem a função de reforçar o tom}
  \seealsoref{此}{ci3}
  \seealsoref{赅}{gai1}
  \seealsoref{这个}{zhe4ge5}
\end{EntryWithPhonetic}

%%%%%%%%%% 赅 %%%%%%%%%%
\subsection*{赅}\addcontentsline{loh}{figure}{赅 \dpy{gai1}}

\begin{EntryWithPhonetic}{赅}{gai1}{10}{⾙}
  \definition*{s.}{Sobrenome: Gai}
  \definition{adj.}{completo; integral; abrangente; inclusivo}
\end{EntryWithPhonetic}

%%%%%%%%%% 改 %%%%%%%%%%
\subsection*{改}\addcontentsline{loh}{figure}{改 \dpy{gai3}}

\begin{EntryWithPhonetic}{改}{gai3}{7}{⽁}[HSK 2]
  \definition{v.}{mudar; converter; transformar; alterar; substituir | alterar; revisar; aperfeiçoar; modificar | corrigir; retificar; remediar; consertar}
\end{EntryWithPhonetic}

\begin{EntryWithPhonetic}{改版}{gai3/ban3}{7,8}{⽁,⽚}[HSK 7-9]
  \definition{s.}{(programas de rádio ou TV) reformulação; ajuste | edição revisada}
  \definition{v.+compl.}{alterar o layout de uma folha impressa | alterar ou corrigir uma página definida | revisar a edição atual}
\end{EntryWithPhonetic}

\begin{EntryWithPhonetic}{改编}{gai3bian1}{7,12}{⽁,⽷}[HSK 7-9]
  \definition{v.}{adaptar; revisar; converter; reorganizar; transcrever; reescrever com base no trabalho original (geralmente em um gênero diferente) | reorganizar; redesignar; alterar a organização original (referindo"-se principalmente ao exército)}
\end{EntryWithPhonetic}

\begin{EntryWithPhonetic}{改变}{gai3bian4}{7,8}{⽁,⼜}[HSK 2]
  \definition{v.}{mudar; alterar; transformar; converter; moldar; modificar | causar mudanças; alterar}
\end{EntryWithPhonetic}

\begin{EntryWithPhonetic}{改动}{gai3dong4}{7,6}{⽁,⼒}[HSK 7-9]
  \definition{v.}{mudar; alterar; modificar; polir; melhorar | alterar (texto, itens, ordem, etc.)}
\end{EntryWithPhonetic}

\begin{EntryWithPhonetic}{改革}{gai3ge2}{7,9}{⽁,⾰}[HSK 5]
  \definition[项,次,种]{s.}{reforma; reformação; iniciativas para aprimorar a inovação}
  \definition{v.}{reformar; transformar as antigas partes irracionais das coisas em novas que possam ser adaptadas à situação objetiva}
\end{EntryWithPhonetic}

\begin{EntryWithPhonetic}{改革开放}{gai3ge2 kai1fang4}{7,9,4,8}{⽁,⾰,⼶,⽅}[HSK 7-9]
  \definition{v.}{reformar e abrir"-se ao mundo exterior (refere"-se às políticas de Deng Xiaoping por volta de 1980)}
\end{EntryWithPhonetic}

\begin{EntryWithPhonetic}{改进}{gai3jin4}{7,7}{⽁,⾡}[HSK 3]
  \definition[个,些]{s.}{melhoria}
  \definition{v.}{aprimorar; aperfeiçoar; melhorar; tornar melhor; mudar a situação antiga para melhorar | modificar (mudança mecânica)}
\end{EntryWithPhonetic}

\begin{EntryWithPhonetic}{改良}{gai3liang2}{7,7}{⽁,⾉}[HSK 7-9]
  \definition{v.}{melhorar; amenizar; remover as deficiências individuais das coisas para torná-las mais adequadas às necessidades | reformar; melhorar | Metalurgia: modificar}
\end{EntryWithPhonetic}

\begin{EntryWithPhonetic}{改名}{gai3ming2}{7,6}{⽁,⼝}[HSK 7-9]
  \definition{v.}{mudar o próprio nome; alterar nome}
\end{EntryWithPhonetic}

\begin{EntryWithPhonetic}{改日}{gai3ri4}{7,4}{⽁,⽇}[HSK 7-9]
  \definition{adv.}{algum outro dia; outro dia}
\end{EntryWithPhonetic}

\begin{EntryWithPhonetic}{改善}{gai3shan4}{7,12}{⽁,⼝}[HSK 4]
  \definition{v.}{melhorar; amenizar; mudar a situação original para torná-la melhor}
\end{EntryWithPhonetic}

\begin{EntryWithPhonetic}{改善关系}{gai3shan4guan1xi5}{7,12,6,7}{⽁,⼝,⼋,⽷}
  \definition{v.}{melhorar a relação}
\end{EntryWithPhonetic}

\begin{EntryWithPhonetic}{改善通讯}{gai3shan4tong1xun4}{7,12,10,5}{⽁,⼝,⾡,⾔}
  \definition{v.}{melhorar a comunicação}
\end{EntryWithPhonetic}

\begin{EntryWithPhonetic}{改为}{gai3wei2}{7,4}{⽁,⼂}[HSK 7-9]
  \definition{v.}{mudar para}[原计划改为明天开始。===O plano original foi mudado para começar amanhã.]
\end{EntryWithPhonetic}

\begin{EntryWithPhonetic}{改邪归正}{gai3xie2-gui1zheng4}{7,6,5,5}{⽁,⾢,⼹,⽌}[HSK 7-9]
  \definition{expr.}{abandonar os maus caminhos e retornar ao caminho certo; abandonar o vício e voltar-se para a virtude; virar uma nova página; retornar a um modo de vida cumpridor da lei}
\end{EntryWithPhonetic}

\begin{EntryWithPhonetic}{改造}{gai3zao4}{7,10}{⽁,⾡}[HSK 3]
  \definition{v.}{transformar; renovar; modificar o original para melhor se adequar às necessidades; usado principalmente para coisas específicas | remodelar; mudar radicalmente o que é velho e ruim; criar algo novo e bom, para se adaptar às novas circunstâncias e necessidades; usado principalmente para coisas abstratas}
\end{EntryWithPhonetic}

\begin{EntryWithPhonetic}{改正}{gai3zheng4}{7,5}{⽁,⽌}[HSK 4]
  \definition{v.}{corrigir; emendar; mudar o errado para o correto}
\end{EntryWithPhonetic}

\begin{EntryWithPhonetic}{改装}{gai3zhuang1}{7,12}{⽁,⾐}[HSK 6]
  \definition{v.}{mudar de traje ou vestido | reembalar | reequipar; reaparelhar | modificar; alterar o dispositivo original}
\end{EntryWithPhonetic}

%%%%%%%%%% 芥 %%%%%%%%%%
\subsection*{芥}\addcontentsline{loh}{figure}{芥 \dpy{gai4}}

\begin{EntryWithPhonetic}{芥}{gai4}{7}{⾋}
  \definition{s.}{mostarda}
  \seeref{jie4}
  \seealsoref{芥蓝}{gai4lan2}
\end{EntryWithPhonetic}

\begin{EntryWithPhonetic}{芥兰}{gai4lan2}{7,5}{⾋,⼋}
  \variantof{芥蓝}
\end{EntryWithPhonetic}

\begin{EntryWithPhonetic}{芥蓝}{gai4lan2}{7,13}{⾋,⾋}
  \definition{s.}{brócolis chinês; couve chinesa; mostarda}
  \seealsoref{格兰菜}{ge2lan2cai4}
\end{EntryWithPhonetic}

%%%%%%%%%% 钙 %%%%%%%%%%
\subsection*{钙}\addcontentsline{loh}{figure}{钙 \dpy{gai4}}

\begin{EntryWithPhonetic}{钙}{gai4}{9}{⾦}[HSK 7-9]
  \definition[克,毫克]{s.}{Ca, cálcio}
\end{EntryWithPhonetic}

%%%%%%%%%% 盖 %%%%%%%%%%
\subsection*{盖}\addcontentsline{loh}{figure}{盖 \dpy{gai4}}

\begin{EntryWithPhonetic}{盖}{gai4}{11}{⽫}[HSK 4]
  \definition*{s.}{Sobrenome: Gai}
  \definition{adj.}{excelente; soberbo; fantástico}
  \definition{adv.}{cerca de; ao redor; aproximadamente; expressa um julgamento especulativo sobre algo, ou uma explicação da causa, o que é equivalente a 大概 ou 原来}
  \definition{conj.}{para; porque; dando continuidade à frase anterior, afirmando a razão ou causa, com tom incerto}
  \definition{s.}{tampa; capa; cobertura; algo que cobre ou sela a parte superior de um objeto | carapaça; concha (de tartaruga, caranguejo, etc.); ossos em formato de crânio em certas partes do corpo humano; as conchas nas costas de certos animais | dossel; capota; toldo | nivelador (uma ferramenta agrícola usada para nivelar terras)}
  \definition{v.}{cobrir; proteger; colocar uma capa em; colocar uma tampa em um objeto | selar; afixar um selo em | superar; sobressair; sobrepujar; ultrapassar | construir; colocar para cima | esconder; ocultar; encobrir | nivelar o terreno com um nivelador (ferramenta agrícola)}
  \seeref{ge3}
  \seealsoref{大概}{da4gai4}
  \seealsoref{原来}{yuan2lai2}
\end{EntryWithPhonetic}

\begin{EntryWithPhonetic}{盖子}{gai4zi5}{11,3}{⽫,⼦}[HSK 7-9]
  \definition[个]{s.}{tampa; cobertura; capa; topo; algo que tem um efeito de proteção na parte superior de um objeto | casco (de tartaruga, etc.); conchas nas costas dos animais}
\end{EntryWithPhonetic}

%%%%%%%%%% 概 %%%%%%%%%%
\subsection*{概}\addcontentsline{loh}{figure}{概 \dpy{gai4}}

\begin{EntryWithPhonetic}{概}{gai4}{13}{⽊}
  \definition{adj.}{geral; aproximado}
  \definition{adv.}{sem exceção; categoricamente}
  \definition{s.}{ideia principal; esboço geral | maneira de se portar e conduzir; comportamento}
  \definition{v.}{generalizar; exemplificar; tipificar}
\end{EntryWithPhonetic}

\begin{EntryWithPhonetic}{概况}{gai4kuang4}{13,7}{⽊,⼎}[HSK 7-9]
  \definition{s.}{situação geral; levantamento; breve relato (de algo); fatos básicos}[个人概况===Perfil Pessoal]
\end{EntryWithPhonetic}

\begin{EntryWithPhonetic}{概括}{gai4kuo4}{13,9}{⽊,⼿}[HSK 4]
  \definition{adj.}{genérico; simples e claro, captando o conteúdo principal}
  \definition{s.}{generalização}
  \definition{v.}{generalizar; resumir}
\end{EntryWithPhonetic}

\begin{EntryWithPhonetic}{概论}{gai4lun4}{13,6}{⽊,⾔}[HSK 7-9]
  \definition{s.}{esboço; introdução; enquete; frequentemente usado em títulos de livros: um resumo da discussão}[«艺术历史概论»===«Introdução à História da Arte»]
\end{EntryWithPhonetic}

\begin{EntryWithPhonetic}{概率}{gai4lv4}{13,11}{⽊,⽞}[HSK 7-9]
  \definition{s.}{acaso; probabilidade; a probabilidade de um certo tipo de evento ocorrer nas mesmas condições}[成功的概率只有8\%。===A probabilidade de sucesso é de apenas 8\%.]
\end{EntryWithPhonetic}

\begin{EntryWithPhonetic}{概念}{gai4nian4}{13,8}{⽊,⼼}[HSK 3]
  \definition[个,种,项]{s.}{ideia; noção; conceito; concepção; uma forma de pensamento que resume as características comuns de algo em uma palavra}
\end{EntryWithPhonetic}

%%%%%%%%%% 干 %%%%%%%%%%
\subsection*{干}\addcontentsline{loh}{figure}{干 \dpy{gan1}}

\begin{EntryWithPhonetic}{干}{gan1}{3}{⼲}[Kangxi 51]
  \definition*{s.}{Sobrenome: Gan}
  \definition{adj.}{seco | vazio; oco; seco | sem substância; vazio | de parentesco nominal; (parentes) não ligados por laços sanguíneos | sem água; (água) esgotada; completamente vazia | assumido como parente nominal; relação familiar reconhecida por adoção | rude; grosseiro; mal-educado; descreve alguém que fala de forma muito direta e rude (sem delicadeza).}
  \definition{adv.}{em vão; fútil; sem propósito; para nada; sem resultado | apenas; sem nada mais | inutilmente; sem uso, sem aproveitamento | superficialmente; significa que não há conteúdo, apenas forma}
  \definition{s.}{Arcaico: escudo | margem; ribeira; margem das águas | alimentos desidratados | abreviação para os dez troncos celestiais}
  \definition{v.}{ofender; afrontar | ter a ver com; estar relacionado com; estar implicado em; interferir com | (antiquado) buscar (cargo público, remuneração, etc.) | (dialeto) deixar alguém de fora; tratar alguém com indiferença; desprezar | assediar; perturbar; criar confusão; causar estragos; bagunçar | solicitar; procurar; buscar (cargo, salário, etc.) | beber até o fim | tratar com indiferença; ignorar}
  \seeref{gan4}
  \seealsoref{干儿}{gan1 er2}
  \seealsoref{干儿}{gan1r5}
  \antonymref{湿}{shi1}
\end{EntryWithPhonetic}

\begin{EntryWithPhonetic}{干杯}{gan1/bei1}{3,8}{⼲,⽊}[HSK 2]
  \definition{interj.}{`Saúde!''}
  \definition{v.+compl.}{fazer um brinde;  brindar até a última gota}
\end{EntryWithPhonetic}

\begin{EntryWithPhonetic}{干脆}{gan1cui4}{3,10}{⼲,⾁}[HSK 5]
  \definition{adj.}{claro; direto; (falar, fazer coisas) sem hesitação; atitude clara}
  \definition{adv.}{justamente; diretamente; sem maiores considerações}
\end{EntryWithPhonetic}

\begin{EntryWithPhonetic}{干儿}{gan1 er2}{3,2}{⼲,⼉}
  \definition{s.}{filho adotivo (adoção tradicional, ou seja, sem implicações legais)}
  \seeref{gan1r5}
\end{EntryWithPhonetic}

\begin{EntryWithPhonetic}{干戈}{gan1ge1}{3,4}{⼲,⼽}[HSK 7-9]
  \definition{s.}{armas de guerra; armas; guerra}[化干戈为玉帛。===Transforme guerra em paz.]
  \antonymref{玉帛}{yu4bo2}
\end{EntryWithPhonetic}

\begin{EntryWithPhonetic}{干旱}{gan1han4}{3,7}{⼲,⽇}[HSK 7-9]
  \definition{adj.}{árido; seco}
\end{EntryWithPhonetic}

\begin{EntryWithPhonetic}{干净}{gan1jing4}{3,8}{⼲,⼎}[HSK 1]
  \definition{adj.}{limpo; limpo e arrumado; sem poeira, impurezas, etc.}
  \definition{adv.}{completamente; totalmente; sem deixar nada para trás}
\end{EntryWithPhonetic}

\begin{EntryWithPhonetic}{干你屁事}{gan1 ni3 pi4shi4}{3,7,7,8}{⼲,⼈,⼫,⼅}
  \definition{interj.}{`Foda-se!''}
\end{EntryWithPhonetic}

\begin{EntryWithPhonetic}{干儿}{gan1r5}{3,2}{⼲,⼉}
  \definition{s.}{alimentos secos, desidratados}
  \seeref{gan1 er2}
\end{EntryWithPhonetic}

\begin{EntryWithPhonetic}{干扰}{gan1rao3}{3,7}{⼲,⼿}[HSK 5]
  \definition{v.}{perturbar; incomodar | interferir; interromper o funcionamento adequado de equipamentos eletrônicos com sinais eletrônicos dispersos}
\end{EntryWithPhonetic}

\begin{EntryWithPhonetic}{干涉}{gan1she4}{3,10}{⼲,⽔}[HSK 6]
  \definition{s.}{interferência; refere"-se ao ato ou comportamento de interferir nos assuntos dos outros}
  \definition{v.}{interferir; intervir; intrometer"-se; pedir ou impedir algo geralmente significa interferir quando não se deve}
\end{EntryWithPhonetic}

\begin{EntryWithPhonetic}{干与}{gan1yu4}{3,3}{⼲,⼀}
  \variantof{干预}
\end{EntryWithPhonetic}

\begin{EntryWithPhonetic}{干预}{gan1yu4}{3,10}{⼲,⾴}[HSK 5]
  \definition{s.}{intromissão; intervenção}
  \definition{v.}{intrometer-se; intervir; interpor-se;}
\end{EntryWithPhonetic}

\begin{EntryWithPhonetic}{干燥}{gan1zao4}{3,17}{⼲,⽕}[HSK 7-9]
  \definition{adj.}{seco; árido; sem umidade ou muito pouca umidade | enfadonho; desinteressante; chato, sem graça}
\end{EntryWithPhonetic}

%%%%%%%%%% 甘 %%%%%%%%%%
\subsection*{甘}\addcontentsline{loh}{figure}{甘 \dpy{gan1}}

\begin{EntryWithPhonetic}{甘}{gan1}{5}{⽢}[Kangxi 99]
  \definition*{s.}{Província de Gansu, abreviação de 甘肃 | Sobrenome: Gan}
  \definition{adj.}{doce; agradável; satisfatório}
  \definition{v.}{estar disposto a; estar contente ou satisfeito com}
  \seealsoref{甘肃}{gan1su4}
\end{EntryWithPhonetic}

\begin{EntryWithPhonetic}{甘薯}{gan1shu3}{5,16}{⽢,⾋}
  \definition{s.}{batata doce}
\end{EntryWithPhonetic}

\begin{EntryWithPhonetic}{甘肃}{gan1su4}{5,8}{⽢,⾀}
  \definition*{s.}{Província de Gansu}
\end{EntryWithPhonetic}

\begin{EntryWithPhonetic}{甘心}{gan1xin1}{5,4}{⽢,⼼}[HSK 7-9]
  \definition{v.}{estar contente com; estar disposto a | reconciliar"-se com; resignar"-se com; contentar"-se com}
\end{EntryWithPhonetic}

%%%%%%%%%% 杆 %%%%%%%%%%
\subsection*{杆}\addcontentsline{loh}{figure}{杆 \dpy{gan1}}

\begin{EntryWithPhonetic}{杆}{gan1}{7}{⽊}
  \definition{s.}{poste; pólo; mastro}
  \seeref{gan3}
\end{EntryWithPhonetic}

%%%%%%%%%% 肝 %%%%%%%%%%
\subsection*{肝}\addcontentsline{loh}{figure}{肝 \dpy{gan1}}

\begin{EntryWithPhonetic}{肝}{gan1}{7}{⾁}[HSK 6]
  \definition[个]{s.}{fígado; um dos órgãos digestivos dos humanos e dos animais superiores}
\end{EntryWithPhonetic}

\begin{EntryWithPhonetic}{肝脏}{gan1zang4}{7,10}{⾁,⾁}[HSK 7-9]
  \definition{s.}{fígado}
\end{EntryWithPhonetic}

%%%%%%%%%% 尴 %%%%%%%%%%
\subsection*{尴}\addcontentsline{loh}{figure}{尴 \dpy{gan1}}

\begin{EntryWithPhonetic}{尴}{gan1}{13}{⼪}
  \definition{adj.}{envergonhado; uma situação ou assunto difícil de lidar | pouco à vontade; expressão não natural; envergonhado}
\end{EntryWithPhonetic}

\begin{EntryWithPhonetic}{尴尬}{gan1ga4}{13,7}{⼪,⼪}[HSK 7-9]
  \definition{adj.}{estranho; envergonhado; quando você se depara com algo difícil de lidar ou algo que o deixa envergonhado}
\end{EntryWithPhonetic}

%%%%%%%%%% 杆 %%%%%%%%%%
\subsection*{杆}\addcontentsline{loh}{figure}{杆 \dpy{gan3}}

\begin{EntryWithPhonetic}{杆}{gan3}{7}{⽊}[HSK 6]
  \definition{clas.}{usado para objetos semelhantes a hastes}
  \definition{s.}{eixo; braço | haste; barra; poste; a parte longa e fina de um objeto, semelhante a um bastão}
  \seeref{gan1}
\end{EntryWithPhonetic}

%%%%%%%%%% 赶 %%%%%%%%%%
\subsection*{赶}\addcontentsline{loh}{figure}{赶 \dpy{gan3}}

\begin{EntryWithPhonetic}{赶}{gan3}{10}{⾛}[HSK 3]
  \definition*{s.}{Sobrenome: Gan}
  \definition{prep.}{por; até; até que; até quando; introduzir o momento em que algo aconteceu, indicando que se espera até um determinado momento}
  \definition{v.}{ultrapassar; alcançar | perseguir; correr atrás; tentar alcançar; dar uma corrida; acelerar ou intensificar  | dirigir; conduzir | expulsar; afugentar; afastar | encontrar; deparar-se com; esbarrar em; acontecer; encontrar-se em (uma situação); aproveitar-se de (uma oportunidade) | ir para; participar (atividades com horário marcado)}
\end{EntryWithPhonetic}

\begin{EntryWithPhonetic}{赶不上}{gan3bu5shang4}{10,4,3}{⾛,⼀,⼀}[HSK 6]
  \definition{v.}{ficar para trás; ser incapaz de alcançar; não conseguir alcançar; não conseguir acompanhar | ser tarde demais (para fazer algo); (não) existir tempo suficiente (para fazer algo) |  deixar de ter; ser incapaz de encontrar ou ter a chance de encontrar; não encontrar; não encontrar (boa oportunidade) | não poder ser comparado a}
\end{EntryWithPhonetic}

\begin{EntryWithPhonetic}{赶到}{gan3dao4}{10,8}{⾛,⼑}[HSK 3]
  \definition{v.}{correr (para algum lugar); apressar-se}
\end{EntryWithPhonetic}

\begin{EntryWithPhonetic}{赶赴}{gan3fu4}{10,9}{⾛,⾛}[HSK 7-9]
  \definition{v.}{apressar-se para; correr para | apressar-se}
\end{EntryWithPhonetic}

\begin{EntryWithPhonetic}{赶集}{gan3ji2}{10,12}{⾛,⾫}
  \definition{v.}{ir a uma feira | ir ao mercado}
\end{EntryWithPhonetic}

\begin{EntryWithPhonetic}{赶脚}{gan3jiao3}{10,11}{⾛,⾁}
  \definition{v.}{transportar mercadorias para ganhar a vida (especialmente de burro) | trabalhar como carroceiro ou porteiro}
\end{EntryWithPhonetic}

\begin{EntryWithPhonetic}{赶紧}{gan3jin3}{10,10}{⾛,⽷}[HSK 3]
  \definition{adv.}{apressadamente; precipitadamente; às pressas; significa agir imediatamente, sem demora}
\end{EntryWithPhonetic}

\begin{EntryWithPhonetic}{赶快}{gan3kuai4}{10,7}{⾛,⼼}[HSK 3]
  \definition{adv.}{rapidamente; imediatamente; aproveite o momento e acelere o ritmo}
\end{EntryWithPhonetic}

\begin{EntryWithPhonetic}{赶路}{gan3lu4}{10,13}{⾛,⾜}
  \definition{v.}{apressar a jornada | apressar-se}
\end{EntryWithPhonetic}

\begin{EntryWithPhonetic}{赶忙}{gan3mang2}{10,6}{⾛,⼼}[HSK 6]
  \definition{adv.}{imediatamente; com pressa; às pressas; rapidamente}
\end{EntryWithPhonetic}

\begin{EntryWithPhonetic}{赶跑}{gan3pao3}{10,12}{⾛,⾜}
  \definition{v.}{afastar | forçar a saída | repelir}
\end{EntryWithPhonetic}

\begin{EntryWithPhonetic}{赶上}{gan3 shang4}{10,3}{⾛,⼀}[HSK 6]
  \definition{v.}{alcançar; manter o ritmo com; acompanhar alguém ou o padrão do planejador | chegar a tempo para; ter tempo suficiente; não ser tarde demais | encontrar; topar com; cruzar com; encontrar-se com; acontecer de encontrar; encontrar algo, em um determinado momento ou oportunidade}
\end{EntryWithPhonetic}

\begin{EntryWithPhonetic}{赶往}{gan3wang3}{10,8}{⾛,⼻}[HSK 7-9]
  \definition{v.}{apressar-se para (algum lugar)}
\end{EntryWithPhonetic}

\begin{EntryWithPhonetic}{赶早}{gan3zao3}{10,6}{⾛,⽇}
  \definition{adv.}{o mais breve possível | na primeira oportunidade | antes que seja tarde | quanto antes melhor}
\end{EntryWithPhonetic}

\begin{EntryWithPhonetic}{赶走}{gan3zou3}{10,7}{⾛,⾛}
  \definition{v.}{expulsar | voltar atrás}
\end{EntryWithPhonetic}

%%%%%%%%%% 敢 %%%%%%%%%%
\subsection*{敢}\addcontentsline{loh}{figure}{敢 \dpy{gan3}}

\begin{EntryWithPhonetic}{敢}{gan3}{11}{⽁}[HSK 3]
  \definition{adj.}{ousado; corajoso; audacioso; valente}
  \definition{adv.}{talvez; provavelmente}
  \definition{v.}{ser ousado o suficiente; atrever-se | ter confiança em; ter certeza; estar certo | aventurar-se; ter coragem de fazer algo | ser ousado; arriscar-se}
\end{EntryWithPhonetic}

\begin{EntryWithPhonetic}{敢情}{gan3qing5}{11,11}{⽁,⼼}[HSK 7-9]
  \definition{adv.}{por que; então; eu digo; indica a descoberta de algo que não foi descoberto anteriormente | claro; de fato; realmente; isso significa que a razão é óbvia e não há necessidade de duvidar dela}
\end{EntryWithPhonetic}

\begin{EntryWithPhonetic}{敢于}{gan3yu2}{11,3}{⽁,⼆}[HSK 6]
  \definition{v.}{ousar; ser ousado em; ter determinação; ter coragem (para fazer ou se esforçar para fazer)}
\end{EntryWithPhonetic}

%%%%%%%%%% 感 %%%%%%%%%%
\subsection*{感}\addcontentsline{loh}{figure}{感 \dpy{gan3}}

\begin{EntryWithPhonetic}{感}{gan3}{13}{⼼}[HSK 7-9]
  \definition{s.}{sentido; sensação; sentimento; impressão | emoção; sentimento}
  \definition{v.}{sentir; perceber; estar ciente | mover; tocar; afetar | ser grato; ser agradecido | ser afetado (pelo frio); pegar um resfriado | (fotografia) sensibilizar | ser grato; apreciar | ser afetado}
\end{EntryWithPhonetic}

\begin{EntryWithPhonetic}{感触}{gan3chu4}{13,13}{⼼,⾓}[HSK 7-9]
  \definition{s.}{sentimento; pensamentos e sentimentos; pensamentos e emoções causados por fatores externos}
\end{EntryWithPhonetic}

\begin{EntryWithPhonetic}{感到}{gan3dao4}{13,8}{⼼,⼑}[HSK 2]
  \definition{v.}{sentir; achar; perceber}
\end{EntryWithPhonetic}

\begin{EntryWithPhonetic}{感动}{gan3dong4}{13,6}{⼼,⼒}[HSK 2]
  \definition{v.}{mover (alguém) | tocar (alguém emocionalmente)}
\end{EntryWithPhonetic}

\begin{EntryWithPhonetic}{感恩}{gan3/en1}{13,10}{⼼,⼼}[HSK 7-9]
  \definition{v.+compl.}{sentir-se grato; ser grato; expressar gratidão pela ajuda dada por outros}
\end{EntryWithPhonetic}

\begin{EntryWithPhonetic}{感激}{gan3ji1}{13,16}{⼼,⽔}[HSK 7-9]
  \definition{v.}{apreciar; ser grato; sentir-se grato; sentir-se em dívida; desenvolver uma impressão favorável de alguém por causa de sua gentileza ou ajuda}
\end{EntryWithPhonetic}

\begin{EntryWithPhonetic}{感觉}{gan3jue2}{13,9}{⼼,⾒}[HSK 2]
  \definition[个]{s.}{sentimento; sensação; percepção sensorial;}
  \definition{v.}{sentir; perceber; tomar consciência; sentir no coração, acreditar}
\end{EntryWithPhonetic}

\begin{EntryWithPhonetic}{感慨}{gan3kai3}{13,12}{⼼,⼼}[HSK 7-9]
  \definition{v.}{suspirar de emoção; estar cheio de emoções; ficar profundamente comovido;  geralmente expresso em palavras}
\end{EntryWithPhonetic}

\begin{EntryWithPhonetic}{感冒}{gan3mao4}{13,9}{⼼,⽇}[HSK 3]
  \definition{adj.}{interessado em}
  \definition[场,次]{s.}{resfriado; gripe comum; \emph{influenza}; doença infecciosa causada por um vírus, que tende a causar sintomas como garganta seca, congestão nasal, tosse, espirros, dor de cabeça e febre quando o corpo está excessivamente cansado, resfriado ou com a imunidade enfraquecida}
  \definition{v.}{pegar (ter) um resfriado}
\end{EntryWithPhonetic}

\begin{EntryWithPhonetic}{感情}{gan3qing2}{13,11}{⼼,⼼}[HSK 3]
  \definition[份,个,种]{s.}{emoção; sentimento; reações psicológicas como amor, ódio, alegria, raiva, tristeza e felicidade, geradas por estímulos externos | amor; afeto; apego; preocupação e afeição por pessoas ou coisas}
\end{EntryWithPhonetic}

\begin{EntryWithPhonetic}{感染}{gan3ran3}{13,9}{⼼,⽊}[HSK 7-9]
  \definition{v.}{infectar; ser infectado com | infectar; afetar; influenciar; evocar os mesmos pensamentos e sentimentos}
\end{EntryWithPhonetic}

\begin{EntryWithPhonetic}{感染力}{gan3ran3li4}{13,9,2}{⼼,⽊,⼒}[HSK 7-9]
  \definition{s.}{poder de mover os sentimentos; apelo | infeccioso (entusiasmo) | inspiração}
\end{EntryWithPhonetic}

\begin{EntryWithPhonetic}{感人}{gan3ren2}{13,2}{⼼,⼈}[HSK 6]
  \definition{adj.}{comovente; tocante}
\end{EntryWithPhonetic}

\begin{EntryWithPhonetic}{感受}{gan3shou4}{13,8}{⼼,⼜}[HSK 3]
  \definition{s.}{percepção ; compreenção; sentimento; experiência; influência do contato com o mundo exterior}
  \definition{v.}{sentir; sentir (através dos sentidos); experimentar; ser afetado}
\end{EntryWithPhonetic}

\begin{EntryWithPhonetic}{感叹}{gan3tan4}{13,5}{⼼,⼝}[HSK 7-9]
  \definition{v.}{suspirar com sentimento; suspirar por causa de um sentimento}
\end{EntryWithPhonetic}

\begin{EntryWithPhonetic}{感想}{gan3xiang3}{13,13}{⼼,⼼}[HSK 5]
  \definition[个,条]{s.}{pensamentos; impressões; reflexões; resposta do pensamento decorrente da exposição ao mundo exterior}
\end{EntryWithPhonetic}

\begin{EntryWithPhonetic}{感谢}{gan3xie4}{13,12}{⼼,⾔}[HSK 2]
  \definition{v.}{agradecer; ser grato; expressar gratidão com palavras ou ações}
\end{EntryWithPhonetic}

\begin{EntryWithPhonetic}{感兴趣}{gan3/xing4qu4}{13,6,15}{⼼,⼋,⾛}[HSK 4]
  \definition{v.+compl.}{estar interessado}
  \seealsoref{对……感兴趣}{dui4 gan3xing4qu4}
\end{EntryWithPhonetic}

\begin{EntryWithPhonetic}{感性}{gan3xing4}{13,8}{⼼,⼼}[HSK 7-9]
  \definition{adj.}{perceptivo; sentimental; emocional; pertencente a formas intuitivas como sensação, percepção e representação}
  \definition{s.}{percepção; sensibilidade; refere"-se a pessoas que são emocionalmente ricas, sentimentais, capazes de ter empatia pelos outros, que têm grande sensibilidade e conseguem entender as mudanças emocionais de qualquer coisa}
\end{EntryWithPhonetic}

%%%%%%%%%% 橄 %%%%%%%%%%
\subsection*{橄}\addcontentsline{loh}{figure}{橄 \dpy{gan3}}

\begin{EntryWithPhonetic}{橄}{gan3}{15}{⽊}
  \definition*{s.}{Sobrenome: Gan}
\end{EntryWithPhonetic}

\begin{EntryWithPhonetic}{橄榄球}{gan3lan3qiu2}{15,13,11}{⽊,⽊,⽟}
  \definition{s.}{futebol jogado com bola oval (rúgbi, futebol americano, regras australianas, etc.)}
\end{EntryWithPhonetic}

%%%%%%%%%% 干 %%%%%%%%%%
\subsection*{干}\addcontentsline{loh}{figure}{干 \dpy{gan4}}

\begin{EntryWithPhonetic}{干}{gan4}{3}{⼲}[HSK 1][Kangxi 51]
  \definition{adj.}{capaz; competente; habilidoso}
  \definition{s.}{tronco; parte principal; corpo principal ou parte importante de algo | habilidade; capacidade; competência}
  \definition{v.}{fazer; trabalhar; cuidar; fazer coisas | ocupar o cargo de; estar envolvido em; assumir, exercer | lutar; golpear; esforçar-se}
  \seeref{gan1}
\end{EntryWithPhonetic}

\begin{EntryWithPhonetic}{干部}{gan4bu4}{3,10}{⼲,⾢}[HSK 7-9]
  \definition[名,个,位]{s.}{quadro; funcionário público; funcionário do governo; servidores públicos (excluindo soldados e pessoal geral) em órgãos estatais, militares e organizações populares; refere"-se àqueles que desempenham determinado trabalho de liderança ou gestão | líder; pessoal que ocupa determinados cargos de liderança}
\end{EntryWithPhonetic}

\begin{EntryWithPhonetic}{干活}{gan4/huo2}{3,9}{⼲,⽔}
  \definition{v.+compl.}{trabalhar | trabalhar em um emprego}
\end{EntryWithPhonetic}

\begin{EntryWithPhonetic}{干活儿}{gan4huo2r5}{3,9,2}{⼲,⽔,⼉}[HSK 2]
  \definition{v.}{trabalhar; gastar energia física ou mental para fazer algo, especialmente trabalho árduo ou esforçado.}
\end{EntryWithPhonetic}

\begin{EntryWithPhonetic}{干吗}{gan4ma2}{3,6}{⼲,⼝}[HSK 3]
  \definition{pron.}{por que?}
  \definition{v.}{o que fazer?}
\end{EntryWithPhonetic}

\begin{EntryWithPhonetic}{干什么}{gan4shen2me5}{3,4,3}{⼲,⼈,⼃}[HSK 1]
  \definition{adv.}{o que fazer; o que ele está fazendo?; o que você está fazendo?; perguntar a razão ou o objetivo}
\end{EntryWithPhonetic}

\begin{EntryWithPhonetic}{干事}{gan4shi5}{3,8}{⼲,⼅}[HSK 7-9]
  \definition{s.}{secretário (ou funcionário administrativo) encarregado de algo | administrador | secretária executiva}
\end{EntryWithPhonetic}

%%%%%%%%%% 刚 %%%%%%%%%%
\subsection*{刚}\addcontentsline{loh}{figure}{刚 \dpy{gang1}}

\begin{EntryWithPhonetic}{刚}{gang1}{6}{⼑}[HSK 2]
  \definition*{s.}{Sobrenome: Gang}
  \definition{adj.}{duro; firme; rígido; forte; (personalidade, atitude) forte; (vontade) determinada}
  \definition{adv.}{apenas; exatamente; justamente | apenas; apenas por pouco; significa atingir um certo nível com dificuldade | apenas; há pouco tempo; indica que a ação ou situação ocorreu há pouco tempo | assim que; somente neste momento; aconteceu que; use a palavra 就 para indicar que duas coisas estão intimamente relacionadas}
  \seealsoref{就}{jiu4}
\end{EntryWithPhonetic}

\begin{EntryWithPhonetic}{刚才}{gang1cai2}{6,3}{⼑,⼿}[HSK 2]
  \definition{s.}{agora mesmo; há pouco; há pouco tempo; referindo"-se ao período recente que acabou de passar}
\end{EntryWithPhonetic}

\begin{EntryWithPhonetic}{刚刚}{gang1gang1}{6,6}{⼑,⼑}[HSK 2]
  \definition{adv.}{apenas; somente; exatamente; refere"-se a algo que é adequado em termos de grau, quantidade, tempo, etc., nem mais nem menos, nem cedo nem tarde, atingindo um estado satisfatório ou que atende exatamente às necessidades | agora mesmo; há pouco; há um momento atrás; referindo"-se a um período de tempo muito curto no passado}
\end{EntryWithPhonetic}

\begin{EntryWithPhonetic}{刚好}{gang1hao3}{6,6}{⼑,⼥}[HSK 6]
  \definition{adj.}{apropriado; na medida certa}
  \definition{adv.}{apenas; acontece que; por acaso}
\end{EntryWithPhonetic}

\begin{EntryWithPhonetic}{刚毅}{gang1yi4}{6,15}{⼑,⽎}[HSK 7-9]
  \definition{adj.}{resoluto e firme | resoluto | robusto | firme}
\end{EntryWithPhonetic}

%%%%%%%%%% 扛 %%%%%%%%%%
\subsection*{扛}\addcontentsline{loh}{figure}{扛 \dpy{gang1}}

\begin{EntryWithPhonetic}{扛}{gang1}{6}{⼿}
  \definition{v.}{levantar com as duas mãos | carregar alguma coisa juntos (duas ou mais pessoas)}
  \seeref{kang2}
\end{EntryWithPhonetic}

%%%%%%%%%% 杠 %%%%%%%%%%
\subsection*{杠}\addcontentsline{loh}{figure}{杠 \dpy{gang1}}

\begin{EntryWithPhonetic}{杠}{gang1}{7}{⽊}
  \definition{s.}{pequena ponte | mastro de bandeira}
  \seeref{gang4}
\end{EntryWithPhonetic}

%%%%%%%%%% 纲 %%%%%%%%%%
\subsection*{纲}\addcontentsline{loh}{figure}{纲 \dpy{gang1}}

\begin{EntryWithPhonetic}{纲}{gang1}{7}{⽷}
  \definition[条,个]{s.}{a corda principal para içar uma rede, frequentemente usada como metáfora | elo-chave; princípio orientador; esboço; programa | Biologia: classe | Arcaico: transporte de mercadorias em comboio (na China feudal)}
  \definition{v.}{amarrar; comprender | corrigir; esclarecer o contorno do problema}
\end{EntryWithPhonetic}

\begin{EntryWithPhonetic}{纲领}{gang1ling3}{7,11}{⽷,⾴}[HSK 7-9]
  \definition{s.}{programa | princípios de orientação; normas diretivas; diretrizes | credo | esboço}
\end{EntryWithPhonetic}

\begin{EntryWithPhonetic}{纲要}{gang1yao4}{7,9}{⽷,⾑}[HSK 7-9]
  \definition{s.}{esboço; contorno | (usualmente em títulos de livros) essenciais; compêndio | fundamentos; princípios básicos}
\end{EntryWithPhonetic}

%%%%%%%%%% 缸 %%%%%%%%%%
\subsection*{缸}\addcontentsline{loh}{figure}{缸 \dpy{gang1}}

\begin{EntryWithPhonetic}{缸}{gang1}{9}{⽸}[HSK 7-9]
  \definition[口,个]{s.}{jarra; pote de barro; recipientes feitos de barro, porcelana, vidro, etc. geralmente têm uma boca grande e um fundo pequeno | composto de areia, argila, etc. para fazer louça de barro | vaso em forma de jarro; objetos em forma de potes}
\end{EntryWithPhonetic}

%%%%%%%%%% 钢 %%%%%%%%%%
\subsection*{钢}\addcontentsline{loh}{figure}{钢 \dpy{gang1}}

\begin{EntryWithPhonetic}{钢}{gang1}{9}{⾦}[HSK 7-9]
  \definition[吨,块,根]{s.}{aço; liga de ferro e carbono}
\end{EntryWithPhonetic}

\begin{EntryWithPhonetic}{钢笔}{gang1bi3}{9,10}{⾦,⽵}[HSK 5]
  \definition[支,杆]{s.}{caneta-tinteiro; canetas com ponta metálica}
\end{EntryWithPhonetic}

\begin{EntryWithPhonetic}{钢琴}{gang1qin2}{9,12}{⾦,⽟}[HSK 5]
  \definition[架,台]{s.}{piano}
\end{EntryWithPhonetic}

\begin{EntryWithPhonetic}{钢丝}{gang1si1}{9,5}{⾦,⼀}
  \definition{s.}{cabo de aço | corda bamba}
\end{EntryWithPhonetic}

%%%%%%%%%% 岗 %%%%%%%%%%
\subsection*{岗}\addcontentsline{loh}{figure}{岗 \dpy{gang3}}

\begin{EntryWithPhonetic}{岗}{gang3}{7}{⼭}
  \definition{s.}{outeiro; monte | crista; vergão (no rosto, pele, etc.) | sentinela; posto | trabalho | batida policial}
\end{EntryWithPhonetic}

\begin{EntryWithPhonetic}{岗位}{gang3wei4}{7,7}{⼭,⼈}[HSK 6]
  \definition[个,类]{s.}{posto; estação; originalmente se refere ao local guardado pelos militares e pela polícia, agora se refere a uma posição geral}
\end{EntryWithPhonetic}

%%%%%%%%%% 港 %%%%%%%%%%
\subsection*{港}\addcontentsline{loh}{figure}{港 \dpy{gang3}}

\begin{EntryWithPhonetic}{港}{gang3}{12}{⽔}[HSK 7-9]
  \definition*{s.}{Hong Kong, abreviação de 香港 | Sobrenome: Gang}
  \definition{s.}{porto; ancoradouro}
  \seealsoref{香港}{xiang1gang3}
\end{EntryWithPhonetic}

\begin{EntryWithPhonetic}{港口}{gang3kou3}{12,3}{⽔,⼝}[HSK 6]
  \definition[个,座]{s.}{porto; locais com certas condições naturais e instalações portuárias para atracação de navios, embarque e desembarque de passageiros e coleta e distribuição de cargas}
\end{EntryWithPhonetic}

%%%%%%%%%% 杠 %%%%%%%%%%
\subsection*{杠}\addcontentsline{loh}{figure}{杠 \dpy{gang4}}

\begin{EntryWithPhonetic}{杠}{gang4}{7}{⽊}
  \definition{s.}{vara grossa | (esportes) barra | peça sobressalente em forma de haste; peça sobressalente em forma de haste usada para máquinas-ferramentas | varas robustas usadas para carregar um caixão | (em um texto) linha grossa desenhada ao lado ou abaixo das palavras como uma marca | (coloquial) padrão; critério}
  \definition{v.}{marcar com uma linha grossa | afiar (faca, navalha, etc.)}
  \seeref{gang1}
\end{EntryWithPhonetic}

\begin{EntryWithPhonetic}{杠铃}{gang4ling2}{7,10}{⽊,⾦}[HSK 7-9]
  \definition{s.}{barra; levantamento de peso; equipamento de levantamento de peso, com placas metálicas em forma de disco instaladas em ambas as extremidades da barra horizontal}
\end{EntryWithPhonetic}

%%%%%%%%%% 高 %%%%%%%%%%
\subsection*{高}\addcontentsline{loh}{figure}{高 \dpy{gao1}}

\begin{EntryWithPhonetic}{高}{gao1}{10}{⾼}[HSK 1][Kangxi 189]
  \definition*{s.}{Sobrenome: Gao}
  \definition{adj.}{alto; elevado; grande distância de baixo para cima; longe do chão | barulhento | sofisticado; caro; de preço elevado; acima do valor real ou do preço de mercado | acima da média; de alto nível ou grau; acima do padrão geral ou da média; de nível superior}
  \definition{s.}{altura; altitude}
\end{EntryWithPhonetic}

\begin{EntryWithPhonetic}{高昂}{gao1'ang2}{10,8}{⾼,⽇}[HSK 7-9]
  \definition{adj.}{alto; eufórico; exaltado | caro; exorbitante}
  \definition{v.}{manter erguida (a cabeça, etc.)}
\end{EntryWithPhonetic}

\begin{EntryWithPhonetic}{高傲}{gao1'ao4}{10,12}{⾼,⼈}[HSK 7-9]
  \definition{adj.}{arrogante; altivo | orgulhoso; respeitoso; nobre}
\end{EntryWithPhonetic}

\begin{EntryWithPhonetic}{高层}{gao1ceng2}{10,7}{⾼,⼫}[HSK 6]
  \definition{adj.}{(de um edifício) arranha-céu | (de posição oficial) alto nível}
  \definition{s.}{nível superior; piso, camada, etc. | arranha-céus; um prédio de apartamentos alto}
\end{EntryWithPhonetic}

\begin{EntryWithPhonetic}{高超}{gao1chao1}{10,12}{⾼,⾛}[HSK 7-9]
  \definition{adj.}{soberbo; excelente; descreve um nível muito alto, excedendo a maioria dos níveis}
\end{EntryWithPhonetic}

\begin{EntryWithPhonetic}{高潮}{gao1chao2}{10,15}{⾼,⽔}[HSK 4]
  \definition[个,场]{s.}{maré alta; o nível mais alto da maré em um ciclo de maré | pico; aumento; maré alta; uma metáfora para o estágio mais próspero de desenvolvimento das coisas (diferente de 低潮) | (ficção, drama e filmes) clímax}
  \seealsoref{低潮}{di1chao2}
\end{EntryWithPhonetic}

\begin{EntryWithPhonetic}{高大}{gao1da4}{10,3}{⾼,⼤}[HSK 5]
  \definition{adj.}{alto e grande; alto | elevado; sublime; nobre}
\end{EntryWithPhonetic}

\begin{EntryWithPhonetic}{高档}{gao1dang4}{10,10}{⾼,⽊}[HSK 6]
  \definition{adj.}{grau superior; alta qualidade; alo grau; qualidade superior; boa qualidade, preço alto (produto)}
\end{EntryWithPhonetic}

\begin{EntryWithPhonetic}{高等}{gao1deng3}{10,12}{⾼,⽵}[HSK 6]
  \definition{adj.}{superior; avançado | alto nível}
  \antonymref{低等}{di1 deng3}
\end{EntryWithPhonetic}

\begin{EntryWithPhonetic}{高低}{gao1di1}{10,7}{⾼,⼈}[HSK 7-9]
  \definition{adv.}{apenas; simplesmente; em qualquer caso; de qualquer forma; em qualquer conta; indica que não importa o que | finalmente; no final, depois de tudo}
  \definition{s.}{inclinação; nível; altura | diferença de grau; superioridade ou inferioridade relativa | discrição; senso de propriedade | (falar ou fazer coisas) medida; profundidade, leveza e peso}
\end{EntryWithPhonetic}

\begin{EntryWithPhonetic}{高调}{gao1diao4}{10,10}{⾼,⾔}[HSK 7-9]
  \definition{adj.}{alto perfil; retaliação; significa agir de forma muito ostensiva e chamativa, deixando claro para todos; também pode significar opor-se deliberadamente aos outros, chegando até a provocar uma briga}
  \definition{s.}{tom elevado; palavras de alto som; sons mais agudos do que o normal ao cantar ou falar}
\end{EntryWithPhonetic}

\begin{EntryWithPhonetic}{高度}{gao1du4}{10,9}{⾼,⼴}[HSK 5]
  \definition{adj.}{alto; elevado; avançado; alto grau | alta concentração; intenso}
  \definition[个]{s.}{altura; altitude; elevação; distância de baixo para cima; o grau e o nível em que as coisas se desenvolveram}
\end{EntryWithPhonetic}

\begin{EntryWithPhonetic}{高额}{gao1'e2}{10,15}{⾼,⾴}[HSK 7-9]
  \definition{s.}{quantidade enorme; cota grande | grande quantidade}
\end{EntryWithPhonetic}

\begin{EntryWithPhonetic}{高尔夫}{gao1'er3fu1}{10,5,4}{⾼,⼩,⼤}
  \definition{s.}{(empréstimo linguístico) \emph{golf}}
\end{EntryWithPhonetic}

\begin{EntryWithPhonetic}{高尔夫球}{gao1'er3fu1qiu2}{10,5,4,11}{⾼,⼩,⼤,⽟}[HSK 7-9]
  \definition[个,只,场,些]{s.}{golfe | bola de golfe}
\end{EntryWithPhonetic}

\begin{EntryWithPhonetic}{高峰}{gao1feng1}{10,10}{⾼,⼭}[HSK 6]
  \definition[个,座]{s.}{cume; pináculo; pico da montanha | pico (de atividade, qualidade ou realização); uma metáfora para o ponto mais alto no desenvolvimento das coisas | cúpula; principais líderes; uma metáfora para o mais alto nível de liderança}
\end{EntryWithPhonetic}

\begin{EntryWithPhonetic}{高峰期}{gao1feng1qi1}{10,10,12}{⾼,⼭,⽉}[HSK 7-9]
  \definition[个]{s.}{período de pico; (de tráfego) horas de pico; o período em que ocorre com mais frequência ou se desenvolve mais próspero}
\end{EntryWithPhonetic}

\begin{EntryWithPhonetic}{高跟鞋}{gao1gen1xie2}{10,13,15}{⾼,⾜,⾰}[HSK 5]
  \definition[双]{s.}{salto alto; sapatos de salto alto; sapato feminino com salto mais alto e mais distante do chão}
\end{EntryWithPhonetic}

\begin{EntryWithPhonetic}{高贵}{gao1gui4}{10,9}{⾼,⾙}[HSK 7-9]
  \definition{adj.}{(caráter pessoal) nobre; honrado; moralmente elevado; magnânimo | grandeza; extremamente valioso | elitista; altamente privilegiado; refere"-se àqueles com \emph{status} elevado e vida superior}
\end{EntryWithPhonetic}

\begin{EntryWithPhonetic}{高级}{gao1ji2}{10,6}{⾼,⽷}[HSK 2]
  \definition{adj.}{sênior; de alto escalão; de alto nível; elevado; excelente; superior; estágio avançado | e alta qualidade; de primeira qualidade; avançado}
\end{EntryWithPhonetic}

\begin{EntryWithPhonetic}{高技术}{gao1ji4shu4}{10,7,5}{⾼,⼿,⽊}
  \definition{s.}{alta tecnologia; \emph{hight tech}}
  \seealsoref{高科技}{gao1ke1ji4}
\end{EntryWithPhonetic}

\begin{EntryWithPhonetic}{高价}{gao1jia4}{10,6}{⾼,⼈}[HSK 4]
  \definition{s.}{preço alto; bilhete caro; custo elevado; dispendioso}
\end{EntryWithPhonetic}

\begin{EntryWithPhonetic}{高考}{gao1kao3}{10,6}{⾼,⽼}[HSK 6]
  \definition[次,回,场]{s.}{vestibular; exame de admissão em instituições de ensino superior}
\end{EntryWithPhonetic}

\begin{EntryWithPhonetic}{高科技}{gao1ke1ji4}{10,9,7}{⾼,⽲,⼿}[HSK 6]
  \definition[种,类]{s.}{alta tecnologia; \emph{high tech}}
  \seealsoref{高技术}{gao1ji4shu4}
\end{EntryWithPhonetic}

\begin{EntryWithPhonetic}{高空}{gao1kong1}{10,8}{⾼,⽳}[HSK 7-9]
  \definition{s.}{alta altitude; ar superior}
  \antonymref{低空}{di1kong1}
\end{EntryWithPhonetic}

\begin{EntryWithPhonetic}{高龄}{gao1ling2}{10,13}{⾼,⿒}[HSK 7-9]
  \definition{adj.}{mais velho que o normal | avançado em anos}
  \definition{s.}{idade avançada; idade venerável}
\end{EntryWithPhonetic}

\begin{EntryWithPhonetic}{高楼}{gao1lou2}{10,13}{⾼,⽊}
  \definition[座]{s.}{edifício alto | edifício de muitos andares | arranha-céu}
\end{EntryWithPhonetic}

\begin{EntryWithPhonetic}{高明}{gao1ming2}{10,8}{⾼,⽇}[HSK 7-9]
  \definition{adj.}{sábio; brilhante; (percepção, habilidades) excelente}
  \definition{s.}{pessoa sábia; pessoa habilidosa}
\end{EntryWithPhonetic}

\begin{EntryWithPhonetic}{高山}{gao1shan1}{10,3}{⾼,⼭}[HSK 7-9]
  \definition[座]{s.}{alta montanha; alpes}
\end{EntryWithPhonetic}

\begin{EntryWithPhonetic}{高尚}{gao1shang4}{10,8}{⾼,⼩}[HSK 4]
  \definition{adj.}{nobre; elevado; descreve um alto padrão moral e uma boa qualidade de pensamento | significativo e não de mau gosto}
\end{EntryWithPhonetic}

\begin{EntryWithPhonetic}{高手}{gao1shou3}{10,4}{⾼,⼿}[HSK 6]
  \definition[位,个,名,些,群]{s.}{ás; mestre; especialista; \emph{expert}; uma pessoa com habilidades excepcionais}
\end{EntryWithPhonetic}

\begin{EntryWithPhonetic}{高速}{gao1su4}{10,10}{⾼,⾡}[HSK 3]
  \definition{adj.}{alta velocidade; veloz; rápido}
  \definition[条]{s.}{autoestrada; via expressa; rodovia}
\end{EntryWithPhonetic}

\begin{EntryWithPhonetic}{高速公路}{gao1su4gong1lu4}{10,10,4,13}{⾼,⾡,⼋,⾜}[HSK 3]
  \definition[条]{s.}{via expressa; rodovia; autoestrada; as rodovias destinadas exclusivamente ao tráfego de veículos em alta velocidade são retas e, quando cruzam outras vias, utilizam cruzamentos em nível}
\end{EntryWithPhonetic}

\begin{EntryWithPhonetic}{高铁}{gao1tie3}{10,10}{⾼,⾦}[HSK 4]
  \definition{s.}{trem de alta velocidade; trem bala}
\end{EntryWithPhonetic}

\begin{EntryWithPhonetic}{高温}{gao1wen1}{10,12}{⾼,⽔}[HSK 5]
  \definition{s.}{alta temperatura; temperatura elevada; hipertermia; megatemperatura; inferno}
  \antonymref{低温}{di1wen1}
\end{EntryWithPhonetic}

\begin{EntryWithPhonetic}{高效}{gao1xiao4}{10,10}{⾼,⽁}[HSK 7-9]
  \definition{adj.}{altamente eficiente | eficiente | altamente eficaz}
\end{EntryWithPhonetic}

\begin{EntryWithPhonetic}{高新技术}{gao1xin1-ji4shu4}{10,13,7,5}{⾼,⽄,⼿,⽊}[HSK 7-9]
  \definition[项,门,套]{s.}{nova e alta tecnologia; \emph{high‐tech}}
\end{EntryWithPhonetic}

\begin{EntryWithPhonetic}{高兴}{gao1xing4}{10,6}{⾼,⼋}[HSK 1]
  \definition{adj.}{contente; feliz; exultante; alegre; satisfeito; animado}
  \definition{v.}{estar contente; estar feliz; estar animado; estar de bom humor; fazer algo com alegria; gostar}
\end{EntryWithPhonetic}

\begin{EntryWithPhonetic}{高血压}{gao1xue4ya1}{10,6,6}{⾼,⾎,⼚}[HSK 7-9]
  \definition{adj.}{hipertenso}
  \definition[点儿]{s.}{hipertenção; pressão alta}
\end{EntryWithPhonetic}

\begin{EntryWithPhonetic}{高压}{gao1ya1}{10,6}{⾼,⼚}[HSK 7-9]
  \definition{s.}{Física, Meteorologia: alta pressão| Eletricidade: alta tensão; alta voltagem | Política: mão de ferro; arrogância | Medicina: pressão sistólica; pressão máxima | perseguição cruel; opressão extrema}
  \antonymref{低压}{di1ya1}
\end{EntryWithPhonetic}

\begin{EntryWithPhonetic}{高雅}{gao1ya3}{10,12}{⾼,⾫}[HSK 7-9]
  \definition{adj.}{delicado | elegante}
  \definition{s.}{delicadeza; decoro | elegância}
\end{EntryWithPhonetic}

\begin{EntryWithPhonetic}{高于}{gao1yu2}{10,3}{⾼,⼆}[HSK 5]
  \definition{v.}{ser mais alto do que; sobrepujar}
\end{EntryWithPhonetic}

\begin{EntryWithPhonetic}{高原}{gao1yuan2}{10,10}{⾼,⼚}[HSK 5]
  \definition[片]{s.}{platô; terras altas; planalto | planalto continental}
\end{EntryWithPhonetic}

\begin{EntryWithPhonetic}{高涨}{gao1zhang3}{10,10}{⾼,⽔}[HSK 7-9]
  \definition{v.}{ascender; subir alto; avançar; (preços, sentimento, etc.) subir rapidamente}
  \antonymref{低落}{di1luo4}
\end{EntryWithPhonetic}

\begin{EntryWithPhonetic}{高中}{gao1zhong1}{10,4}{⾼,⼁}[HSK 2]
  \definition[所,个]{s.}{ensino médio; escola secundária de ensino médio}
\end{EntryWithPhonetic}

%%%%%%%%%% 糕 %%%%%%%%%%
\subsection*{糕}\addcontentsline{loh}{figure}{糕 \dpy{gao1}}

\begin{EntryWithPhonetic}{糕}{gao1}{16}{⽶}
  \definition{s.}{bolo; alimentos feitos de farinha de arroz, farinha de trigo, etc.}
\end{EntryWithPhonetic}

\begin{EntryWithPhonetic}{糕点}{gao1dian3}{16,9}{⽶,⽕}
  \definition{s.}{bolos | pastéis}
\end{EntryWithPhonetic}

\begin{EntryWithPhonetic}{糕点店}{gao1dian3 dian4}{16,9,8}{⽶,⽕,⼴}
  \definition{s.}{confeitaria}
\end{EntryWithPhonetic}

\begin{EntryWithPhonetic}{糕点师}{gao1dian3 shi1}{16,9,6}{⽶,⽕,⼱}
  \definition{s.}{confeiteiro}
\end{EntryWithPhonetic}

%%%%%%%%%% 搞 %%%%%%%%%%
\subsection*{搞}\addcontentsline{loh}{figure}{搞 \dpy{gao3}}

\begin{EntryWithPhonetic}{搞}{gao3}{13}{⼿}[HSK 5]
  \definition{v.}{fazer; realizar; estar envolvido em; engajar-se em um estudo, fazer algo em relação a, etc. | fazer; produzir; gerar; trabalhar | iniciar; estabelecer; organizar; configurar | consertar (mudar) alguém; fazer alguém sofrer | obter; assegurar; agarrar |  (seguido de um complemento) fazer com que se torne; produzir um determinado efeito ou resultado}
\end{EntryWithPhonetic}

\begin{EntryWithPhonetic}{搞错}{gao3cuo4}{13,13}{⼿,⾦}
  \definition{v.}{cometer um erro}
\end{EntryWithPhonetic}

\begin{EntryWithPhonetic}{搞定}{gao3ding4}{13,8}{⼿,⼧}
  \definition{v.}{consertar | resolver}
\end{EntryWithPhonetic}

\begin{EntryWithPhonetic}{搞鬼}{gao3/gui3}{13,9}{⼿,⿁}[HSK 7-9]
  \definition{v.+compl.}{Coloquial: pregar peças; planejar em segredo; fazer alguma travessura}
\end{EntryWithPhonetic}

\begin{EntryWithPhonetic}{搞好}{gao3 hao3}{13,6}{⼿,⼥}[HSK 5]
  \definition{v.}{fazer um bom trabalho; fazer bem; suar; tornar submisso, tornar útil, por meio de solicitações e presentes amigáveis; amolecer}
\end{EntryWithPhonetic}

\begin{EntryWithPhonetic}{搞混}{gao3hun4}{13,11}{⼿,⽔}
  \definition{v.}{confundir; embaralhar}
\end{EntryWithPhonetic}

\begin{EntryWithPhonetic}{搞乱}{gao3luan4}{13,7}{⼿,⼄}
  \definition{v.}{estragar | confundir | bagunçar}
\end{EntryWithPhonetic}

\begin{EntryWithPhonetic}{搞钱}{gao3qian2}{13,10}{⼿,⾦}
  \definition{v.}{fazer dinheiro | acumular dinheiro}
\end{EntryWithPhonetic}

\begin{EntryWithPhonetic}{搞通}{gao3tong1}{13,10}{⼿,⾡}
  \definition{v.}{entender algo}
\end{EntryWithPhonetic}

\begin{EntryWithPhonetic}{搞笑}{gao3xiao4}{13,10}{⼿,⽵}[HSK 7-9]
  \definition{adj.}{engraçado; divertido; descreve um estado ou qualidade que é interessante, engraçado ou faz as pessoas rirem}
  \definition{v.}{fazer palhaçadas para provocar risos; fazer as pessoas rirem deliberadamente; criar piadas e fazer as pessoas rirem}
\end{EntryWithPhonetic}

%%%%%%%%%% 稿 %%%%%%%%%%
\subsection*{稿}\addcontentsline{loh}{figure}{稿 \dpy{gao3}}

\begin{EntryWithPhonetic}{稿}{gao3}{15}{⽲}
  \definition[篇]{s.}{(significado original) talo de grão; palha | rascunho; esboço; manuscrito | texto original}
\end{EntryWithPhonetic}

\begin{EntryWithPhonetic}{稿纸}{gao3zhi3}{15,7}{⽲,⽷}
  \definition{s.}{rascunho | manuscrito}
\end{EntryWithPhonetic}

\begin{EntryWithPhonetic}{稿子}{gao3zi5}{15,3}{⽲,⼦}[HSK 6]
  \definition[篇,份,堆,叠]{s.}{rascunho; esboço; rascunhos de poemas, ensaios, desenhos, etc. | rascunho; manuscrito; poemas escritos | ideia; plano; plano preliminar ou conceito de trabalho}
\end{EntryWithPhonetic}

%%%%%%%%%% 告 %%%%%%%%%%
\subsection*{告}\addcontentsline{loh}{figure}{告 \dpy{gao4}}

\begin{EntryWithPhonetic}{告}{gao4}{7}{⼝}[HSK 7-9]
  \definition*{s.}{Sobrenome: Gao}
  \definition{s.}{anúncio; declaração; notificação}
  \definition{v.}{informar; contar; notificar; explicar aos outros | acusar; processar; relatar | pedir; requisitar; solicitar | dar a conhecer; mostrar | anunciar; declarar; proclamar}
\end{EntryWithPhonetic}

\begin{EntryWithPhonetic}{告别}{gao4/bie2}{7,7}{⼝,⼑}[HSK 3]
  \definition{v.+compl.}{dizer adeus a; expressar a outros, por meio de palavras, que está prestes a partir | deixar; sair; partir de | prestar as últimas homenagens ao falecido}
\end{EntryWithPhonetic}

\begin{EntryWithPhonetic}{告辞}{gao4ci2}{7,13}{⼝,⾟}[HSK 7-9]
  \definition{v.}{despedir"-se (do anfitrião) | despedir"-se}
\end{EntryWithPhonetic}

\begin{EntryWithPhonetic}{告急}{gao4ji2}{7,9}{⼝,⼼}
  \definition{v.}{estar em estado de emergência; ser crítico | relatar uma emergência; pedir ajuda de emergência | estar em uma emergência; fazer uma solicitação urgente de ajuda em uma emergência}
\end{EntryWithPhonetic}

\begin{EntryWithPhonetic}{告诫}{gao4jie4}{7,9}{⼝,⾔}[HSK 7-9]
  \definition{v.}{avisar; advertir; exortar; admoestar}
\end{EntryWithPhonetic}

\begin{EntryWithPhonetic}{告示}{gao4shi5}{7,5}{⼝,⽰}[HSK 7-9]
  \definition[张,条,篇]{s.}{nota oficial; boletim; cartaz | Obsoleto: slogan; cartaz}
  \definition{v.}{notificar; anunciar}
\end{EntryWithPhonetic}

\begin{EntryWithPhonetic}{告诉}{gao4su4}{7,7}{⼝,⾔}
  \definition{v.}{dizer; informar (dar a conhecer); dizer aos outros, para que todos saibam}
  \seeref{gao4su5}
\end{EntryWithPhonetic}

\begin{EntryWithPhonetic}{告诉}{gao4su5}{7,7}{⼝,⾔}[HSK 1]
  \definition{v.}{dizer; informar (dar a conhecer)}
  \seeref{gao4su4}
\end{EntryWithPhonetic}

\begin{EntryWithPhonetic}{告知}{gao4zhi1}{7,8}{⼝,⽮}[HSK 7-9]
  \definition{v.}{contar; informar; transmitir; familiarizar}
\end{EntryWithPhonetic}

\begin{EntryWithPhonetic}{告状}{gao4/zhuang4}{7,7}{⼝,⽝}[HSK 7-9]
  \definition{v.+compl.}{ir à justiça contra alguém; apresentar uma queixa ao tribunal e solicitar que o caso seja aberto para julgamento | apresentar uma acusação ou queixa; informar aos pais ou superiores que você ou outras pessoas foram vítimas de \emph{bullying} ou injustiças}
\end{EntryWithPhonetic}

%%%%%%%%%% 戈 %%%%%%%%%%
\subsection*{戈}\addcontentsline{loh}{figure}{戈 \dpy{ge1}}

\begin{EntryWithPhonetic}{戈}{ge1}{4}{⼽}[Kangxi 62]
  \definition*{s.}{Sobrenome: Ge}
  \definition{s.}{Arcaico: machado"-adaga (com cabo longo e lâmina horizontal) | punhal"-machado; arma antiga, lâmina cruzada, feita de bronze ou ferro, com cabo longo}
\end{EntryWithPhonetic}

\begin{EntryWithPhonetic}{戈壁}{ge1bi4}{4,16}{⼽,⼟}[HSK 7-9]
  \definition*{s.}{Deserto de Gobi}
  \definition{s.}{deserto; refere"-se a uma área desértica onde o solo é quase coberto por areia grossa e cascalho e há poucas plantas}
\end{EntryWithPhonetic}

%%%%%%%%%% 哥 %%%%%%%%%%
\subsection*{哥}\addcontentsline{loh}{figure}{哥 \dpy{ge1}}

\begin{EntryWithPhonetic}{哥}{ge1}{10}{⼝}
  \definition[个,位,名,些]{s.}{irmão mais velho | forma de se dirigir a um parente masculino mais velho de sua geração | irmão; termo amigável para se dirigir a conhecidos mais velhos do sexo masculino}
  \seealsoref{哥哥}{ge1ge5}
\end{EntryWithPhonetic}

\begin{EntryWithPhonetic}{哥哥}{ge1ge5}{10,10}{⼝,⼝}[HSK 1]
  \definition[个,位]{s.}{irmão mais velho | primo}
\end{EntryWithPhonetic}

\begin{EntryWithPhonetic}{哥们}{ge1men5}{10,5}{⼝,⼈}
  \definition{expr.}{\emph{Brothers!}}
  \definition{s.}{(coloquial) cara | irmão (forma diminuta de tratamento entre homens)}
\end{EntryWithPhonetic}

\begin{EntryWithPhonetic}{哥斯拉}{ge1si1la1}{10,12,8}{⼝,⽄,⼿}
  \definition*{s.}{Godzilla}
  \seealsoref{酷斯拉}{ku4si1la1}
\end{EntryWithPhonetic}

%%%%%%%%%% 格 %%%%%%%%%%
\subsection*{格}\addcontentsline{loh}{figure}{格 \dpy{ge1}}

\begin{EntryWithPhonetic}{格}{ge1}{10}{⽊}
  \definition{s.}{Onomatopéia: estalo (som); riso zombeteiro}
  \seeref{ge2}
\end{EntryWithPhonetic}

%%%%%%%%%% 胳 %%%%%%%%%%
\subsection*{胳}\addcontentsline{loh}{figure}{胳 \dpy{ge1}}

\begin{EntryWithPhonetic}{胳}{ge1}{10}{⾁}
  \definition{s.}{axila; sovaco}
  \seeref{ga1}
  \seeref{ge2}
\end{EntryWithPhonetic}

\begin{EntryWithPhonetic}{胳膊}{ge1bo5}{10,14}{⾁,⾁}[HSK 7-9]
  \definition[条,双,只]{s.}{braço; a área abaixo do ombro e acima do pulso}
\end{EntryWithPhonetic}

%%%%%%%%%% 鸽 %%%%%%%%%%
\subsection*{鸽}\addcontentsline{loh}{figure}{鸽 \dpy{ge1}}

\begin{EntryWithPhonetic}{鸽}{ge1}{11}{⿃}
  \definition[只]{s.}{pombo}[和平鸽。===Pomba da Paz.]
\end{EntryWithPhonetic}

\begin{EntryWithPhonetic}{鸽子}{ge1zi5}{11,3}{⿃,⼦}[HSK 7-9]
  \definition[只,对,群]{s.}{pombo}
\end{EntryWithPhonetic}

%%%%%%%%%% 割 %%%%%%%%%%
\subsection*{割}\addcontentsline{loh}{figure}{割 \dpy{ge1}}

\begin{EntryWithPhonetic}{割}{ge1}{12}{⼑}[HSK 7-9]
  \definition{v.}{cortar; ceifar | dividir; cortar}
\end{EntryWithPhonetic}

%%%%%%%%%% 搁 %%%%%%%%%%
\subsection*{搁}\addcontentsline{loh}{figure}{搁 \dpy{ge1}}

\begin{EntryWithPhonetic}{搁}{ge1}{12}{⼿}[HSK 7-9]
  \definition{v.}{pôr; colocar | colocar à parte; deixar para trás; deixar para mais tarde| deixar de lado}
  \seeref{ge2}
\end{EntryWithPhonetic}

\begin{EntryWithPhonetic}{搁浅}{ge1/qian3}{12,8}{⼿,⽔}[HSK 7-9]
  \definition{v.+compl.}{ficar encalhado (navio); encalhar | ser retido; chegar a um impasse; metaforicamente, algo está bloqueado e não pode prosseguir}
\end{EntryWithPhonetic}

\begin{EntryWithPhonetic}{搁置}{ge1zhi4}{12,13}{⼿,⽹}[HSK 7-9]
  \definition{v.}{arquivar; deixar de lado; suspender; classificar; deitar; adiar; colocar na prateleira}
\end{EntryWithPhonetic}

%%%%%%%%%% 歌 %%%%%%%%%%
\subsection*{歌}\addcontentsline{loh}{figure}{歌 \dpy{ge1}}

\begin{EntryWithPhonetic}{歌}{ge1}{14}{⽋}[HSK 1]
  \definition[首,支,段]{s.}{canção; poesia cantável}
  \definition{v.}{cantar; entoar | louvar; exaltar; cantar louvores a}
\end{EntryWithPhonetic}

\begin{EntryWithPhonetic}{歌唱}{ge1chang4}{14,11}{⽋,⼝}[HSK 6]
  \definition{v.}{cantar | cantar em louvor de; louvor através de cânticos, recitações, etc.}
\end{EntryWithPhonetic}

\begin{EntryWithPhonetic}{歌词}{ge1ci2}{14,7}{⽋,⾔}[HSK 6]
  \definition{s.}{letra da música; libreto}
\end{EntryWithPhonetic}

\begin{EntryWithPhonetic}{歌剧}{ge1ju4}{14,10}{⽋,⼑}[HSK 7-9]
  \definition[场,出]{s.}{ópera | ópera ocidental; um drama que integra poesia, música, dança e outras artes, tendo o canto como principal característica}
\end{EntryWithPhonetic}

\begin{EntryWithPhonetic}{歌迷}{ge1mi2}{14,9}{⽋,⾡}[HSK 3]
  \definition{s.}{fã de um cantor; pessoas que gostam de ouvir música ou cantar e ficam fascinadas por isso}
\end{EntryWithPhonetic}

\begin{EntryWithPhonetic}{歌曲}{ge1qu3}{14,6}{⽋,⽈}[HSK 5]
  \definition[首,支]{s.}{música; obra para as pessoas cantarem, uma combinação de poesia e música}
\end{EntryWithPhonetic}

\begin{EntryWithPhonetic}{歌声}{ge1sheng1}{14,7}{⽋,⼠}[HSK 3]
  \definition{s.}{canto; voz cantada; som do canto}
\end{EntryWithPhonetic}

\begin{EntryWithPhonetic}{歌手}{ge1shou3}{14,4}{⽋,⼿}[HSK 3]
  \definition[个,位,名]{s.}{cantor; vocalista; pessoa com talento para cantar}
\end{EntryWithPhonetic}

\begin{EntryWithPhonetic}{歌颂}{ge1song4}{14,10}{⽋,⾴}[HSK 7-9]
  \definition{v.}{cantar louvores de; exaltar; elogiar; elogio com poesia, geralmente se refere a elogiar com palavras, etc.}
\end{EntryWithPhonetic}

\begin{EntryWithPhonetic}{歌舞}{ge1wu3}{14,14}{⽋,⾇}[HSK 7-9]
  \definition{s.}{canto e dança}
\end{EntryWithPhonetic}

\begin{EntryWithPhonetic}{歌星}{ge1xing1}{14,9}{⽋,⽇}[HSK 6]
  \definition[位,名]{s.}{cantor famoso; estrela da música}
\end{EntryWithPhonetic}

\begin{EntryWithPhonetic}{歌咏}{ge1yong3}{14,8}{⽋,⼝}[HSK 7-9]
  \definition{v.}{cantar; cantar canções}
\end{EntryWithPhonetic}

%%%%%%%%%% 阁 %%%%%%%%%%
\subsection*{阁}\addcontentsline{loh}{figure}{阁 \dpy{ge2}}

\begin{EntryWithPhonetic}{阁}{ge2}{9}{⾨}
  \definition{s.}{pavilhão (geralmente de dois andares) | gabinete (de um governo) | Obsoleto: quarto da mulher; \emph{boudoir} | prateleira}
\end{EntryWithPhonetic}

\begin{EntryWithPhonetic}{阁下}{ge2xia4}{9,3}{⾨,⼀}
  \definition{pron.}{Sua Excelência | Sua Majestade | \emph{Sire}}
\end{EntryWithPhonetic}

%%%%%%%%%% 革 %%%%%%%%%%
\subsection*{革}\addcontentsline{loh}{figure}{革 \dpy{ge2}}

\begin{EntryWithPhonetic}{革}{ge2}{9}{⾰}[Kangxi 177]
  \definition*{s.}{Sobrenome: Ge}
  \definition{s.}{couro; pele; peles de animais depiladas e processadas}
  \definition{v.}{mudar; transformar; reformar | demitir; remover do cargo; expulsar}
\end{EntryWithPhonetic}

\begin{EntryWithPhonetic}{革命}{ge2ming4}{9,8}{⾰,⼝}[HSK 7-9]
  \definition{adj.}{revolucionário}
  \definition[次,场]{s.}{revolução; a classe oprimida toma o poder pela violência, destrói o antigo sistema social decadente e estabelece um novo sistema social progressista; a revolução destrói as antigas relações de produção, estabelece novas relações de produção, libera as forças produtivas e promove o desenvolvimento social}
  \definition{v.}{participar da revolução; originalmente se referia à reforma do Mandato do Céu, ou seja, à mudança de dinastias; agora, refere"-se à classe oprimida usando a violência para tomar o poder, destruir o antigo sistema social, estabelecer um novo sistema social e promover o desenvolvimento social}
\end{EntryWithPhonetic}

\begin{EntryWithPhonetic}{革新}{ge2xin1}{9,13}{⾰,⽄}[HSK 6]
  \definition{v.}{inovar; renovar; livrar"-se do velho e criar o novo}
\end{EntryWithPhonetic}

%%%%%%%%%% 格 %%%%%%%%%%
\subsection*{格}\addcontentsline{loh}{figure}{格 \dpy{ge2}}

\begin{EntryWithPhonetic}{格}{ge2}{10}{⽊}[HSK 7-9]
  \definition*{s.}{Sobrenome: Ge}
  \definition{s.}{quadrados formados por linhas cruzadas; quadriculado; grade | divisão (horizontal ou não); treliça | padrão; forma; formato; estilo | caso; as categorias morfológicas de substantivos, pronomes e adjetivos em algumas línguas}
  \definition{v.}{resistir; dificultar; obstruir; impedir | estudar cuidadosamente; investigar | lutar; bater}
  \seeref{ge1}
\end{EntryWithPhonetic}

\begin{EntryWithPhonetic}{格格不入}{ge2ge2-bu2ru4}{10,10,4,2}{⽊,⽊,⼀,⼊}[HSK 7-9]
  \definition{expr.}{incompatível com; fora de sintonia com; estranho; fora do seu elemento; como uma estaca quadrada em um buraco redondo; desarmônico}
\end{EntryWithPhonetic}

\begin{EntryWithPhonetic}{格局}{ge2ju2}{10,7}{⽊,⼫}[HSK 7-9]
  \definition{s.}{padrão; configuração; estrutura; estilo; maneira; arranjo | a visão ou percepção de uma situação geral; a visão de uma pessoa, a altura e a profundidade da consideração do problema}
\end{EntryWithPhonetic}

\begin{EntryWithPhonetic}{格兰菜}{ge2lan2cai4}{10,5,11}{⽊,⼋,⾋}
  \definition{s.}{brócolis chinês | couve chinesa | mostarda}
  \seealsoref{芥蓝}{gai4lan2}
\end{EntryWithPhonetic}

\begin{EntryWithPhonetic}{格式}{ge2shi5}{10,6}{⽊,⼷}[HSK 7-9]
  \definition[种]{s.}{forma; estilo; \emph{layout}; padrão; formato; modo}
\end{EntryWithPhonetic}

\begin{EntryWithPhonetic}{格外}{ge2wai4}{10,5}{⽊,⼣}[HSK 4]
  \definition{adv.}{especialmente; particularmente; ainda mais; indica mais do que a média | adicionalmente; indica adicional ou extra}
\end{EntryWithPhonetic}

%%%%%%%%%% 胳 %%%%%%%%%%
\subsection*{胳}\addcontentsline{loh}{figure}{胳 \dpy{ge2}}

\begin{EntryWithPhonetic}{胳}{ge2}{10}{⾁}
  \definition{v.}{usado em 胳肢}
  \seeref{ga1}
  \seeref{ge1}
  \seealsoref{胳肢}{ge2zhi5}
\end{EntryWithPhonetic}

\begin{EntryWithPhonetic}{胳肢}{ge2zhi5}{10,8}{⾁,⾁}
  \definition{v.}{(dialeto) fazer cócegas}
\end{EntryWithPhonetic}

%%%%%%%%%% 鬲 %%%%%%%%%%
\subsection*{鬲}\addcontentsline{loh}{figure}{鬲 \dpy{ge2}}

\begin{EntryWithPhonetic}{鬲}{ge2}{10}{⿀}[Kangxi 193]
  \definition{s.}{um antigo utensílio de cozinha semelhante a um caldeirão; uma grande panela de barro | utilizado em nomes geográficos ou pessoais}
  \seeref{li4}
\end{EntryWithPhonetic}

%%%%%%%%%% 搁 %%%%%%%%%%
\subsection*{搁}\addcontentsline{loh}{figure}{搁 \dpy{ge2}}

\begin{EntryWithPhonetic}{搁}{ge2}{12}{⼿}
  \definition{v.}{suportar; resistir}
  \seeref{ge1}
\end{EntryWithPhonetic}

%%%%%%%%%% 隔 %%%%%%%%%%
\subsection*{隔}\addcontentsline{loh}{figure}{隔 \dpy{ge2}}

\begin{EntryWithPhonetic}{隔}{ge2}{12}{⾩}[HSK 4]
  \definition{adj.}{seguinte; vizinho}
  \definition{v.}{separar; cortar; dividir; particionar | estar a uma distância de, após ou em um intervalo de | ficar de pé ou deitar entre}
\end{EntryWithPhonetic}

\begin{EntryWithPhonetic}{隔壁}{ge2bi4}{12,16}{⾩,⼟}[HSK 5]
  \definition{s.}{vizinho; casas ou pessoas vizinhas | septo; distante (socialmente distante) | anteparo; partição}
\end{EntryWithPhonetic}

\begin{EntryWithPhonetic}{隔阂}{ge2he2}{12,9}{⾩,⾨}[HSK 7-9]
  \definition[层,种,点]{s.}{estranhamento; mal-entendido; há uma falta de conexão emocional e uma distância de pensamento entre eles}
\end{EntryWithPhonetic}

\begin{EntryWithPhonetic}{隔开}{ge2kai1}{12,4}{⾩,⼶}[HSK 4]
  \definition{v.}{separar; manter separado; barricar; separar completamente duas pessoas (ou coisas) ou duas partes de uma coisa que estão intimamente unidas}
\end{EntryWithPhonetic}

\begin{EntryWithPhonetic}{隔离}{ge2li2}{12,10}{⾩,⼇}[HSK 7-9]
  \definition{v.}{segregar; não permitir que as pessoas se reúnam, cortar o contato | isolar; colocar em quarentena; separar pessoas e animais com doenças infecciosas de pessoas e animais saudáveis para evitar o contato}
\end{EntryWithPhonetic}

%%%%%%%%%% 个 %%%%%%%%%%
\subsection*{个}\addcontentsline{loh}{figure}{个 \dpy{ge3}}

\begin{EntryWithPhonetic}{个}{ge3}{3}{⼈}
  \definition{pron.}{usado em 自个儿}
  \seeref{ge4}
  \seealsoref{自个儿}{zi4ge3r5}
\end{EntryWithPhonetic}

%%%%%%%%%% 盖 %%%%%%%%%%
\subsection*{盖}\addcontentsline{loh}{figure}{盖 \dpy{ge3}}

\begin{EntryWithPhonetic}{盖}{ge3}{11}{⽫}
  \definition*{s.}{Sobrenome: Ge}
  \seeref{gai4}
\end{EntryWithPhonetic}

%%%%%%%%%% 个 %%%%%%%%%%
\subsection*{个}\addcontentsline{loh}{figure}{个 \dpy{ge4}}

\begin{EntryWithPhonetic}{个}{ge4}{3}{⼈}[HSK 1]
  \definition{adj.}{individual}
  \definition{clas.}{usado antes de substantivos que não têm palavras de medida específicas | usado na frente do divisor; usado na frente do número aproximado | usado após verbos com objeto direto |  usado entre verbos e complementos}
  \definition{part.}{usado após pronomes demonstrativos | adicionado após certas palavras de tempo}
  \seeref{ge3}
\end{EntryWithPhonetic}

\begin{EntryWithPhonetic}{个案}{ge4'an4}{3,10}{⼈,⽊}[HSK 7-9]
  \definition[个,些]{s.}{caso individual (ou especial); caso; caso a caso}
\end{EntryWithPhonetic}

\begin{EntryWithPhonetic}{个别}{ge4bie2}{3,7}{⼈,⼑}[HSK 4]
  \definition{adj.}{muito poucos; excepcionais}
  \definition{adv.}{separadamente; individualmente; isoladamente}
\end{EntryWithPhonetic}

\begin{EntryWithPhonetic}{个儿}{ge4r5}{3,2}{⼈,⼉}[HSK 5]
  \definition{s.}{tamanho; altura; estatura; tamanho do corpo ou do objeto | pessoas ou coisas consideradas isoladamente; referir-se a uma pessoa ou coisa individualmente}
\end{EntryWithPhonetic}

\begin{EntryWithPhonetic}{个人}{ge4ren2}{3,2}{⼈,⼈}[HSK 3]
  \definition{pron.}{pessoal; si mesmo}
  \definition[个]{s.}{indivíduo; pessoa}
\end{EntryWithPhonetic}

\begin{EntryWithPhonetic}{个体}{ge4ti3}{3,7}{⼈,⼈}[HSK 4]
  \definition[个,位]{s.}{uma única pessoa ou organismo}
\end{EntryWithPhonetic}

\begin{EntryWithPhonetic}{个头儿}{ge4tou2er5}{3,5,2}{⼈,⼤,⼉}[HSK 7-9]
  \definition{s.}{tamanho; altura}
\end{EntryWithPhonetic}

\begin{EntryWithPhonetic}{个性}{ge4xing4}{3,8}{⼈,⼼}[HSK 3]
  \definition[种,点儿]{s.}{individualidade; personalidade; caráter individual; as características relativamente fixas de uma pessoa, formadas sob determinadas condições sociais e influências educacionais | propriedade específica; caráter específico; a propriedade ou característica especial que distingue uma coisa de outras coisas}
\end{EntryWithPhonetic}

\begin{EntryWithPhonetic}{个子}{ge4zi5}{3,3}{⼈,⼦}[HSK 2]
  \definition[个,种,些]{s.}{altura; estatura; refere"-se ao tamanho do corpo humano e também ao tamanho do corpo dos animais}
\end{EntryWithPhonetic}

%%%%%%%%%% 各 %%%%%%%%%%
\subsection*{各}\addcontentsline{loh}{figure}{各 \dpy{ge4}}

\begin{EntryWithPhonetic}{各}{ge4}{6}{⼝}[HSK 3]
  \definition{adv.}{de várias maneiras; de diversas formas; respectivamente; indica que algo é feito separadamente ou que possui uma determinada característica separadamente}
  \definition{pron.}{todo; todos; cada; refere"-se a todos os indivíduos dentro de um determinado intervalo, equivalente a 每个}
  \seealsoref{每个}{mei3ge4}
\end{EntryWithPhonetic}

\begin{EntryWithPhonetic}{各奔前程}{ge4ben4qian2cheng2}{6,8,9,12}{⼝,⼤,⼑,⽲}[HSK 7-9]
  \definition{expr.}{``Cada um segue seu próprio caminho.''; cada pessoa tem sua própria vida para viver; cada um deles desenvolve sua própria carreira ambiciosa; cada um segue seu próprio curso}
\end{EntryWithPhonetic}

\begin{EntryWithPhonetic}{各地}{ge4di4}{6,6}{⼝,⼟}[HSK 3]
  \definition{s.}{em todos os lugares; em vários locais}
\end{EntryWithPhonetic}

\begin{EntryWithPhonetic}{各个}{ge4ge4}{6,3}{⼝,⼈}[HSK 4]
  \definition{adv./pron.}{cada | um a um; um após o outro}
\end{EntryWithPhonetic}

\begin{EntryWithPhonetic}{各式各样}{ge4shi4-ge4yang4}{6,6,6,10}{⼝,⼷,⼝,⽊}[HSK 7-9]
  \definition{expr.}{todo tipo de\dots; todos os tipos de\dots; todos os tipos de; de várias maneiras; de todas as descrições; uma variedade de; uma variedade de variedades com cores diferentes}
\end{EntryWithPhonetic}

\begin{EntryWithPhonetic}{各位}{ge4wei4}{6,7}{⼝,⼈}[HSK 3]
  \definition{pron.}{todos; toda a gente; todo mundo | cada um}
\end{EntryWithPhonetic}

\begin{EntryWithPhonetic}{各种}{ge4zhong3}{6,9}{⼝,⽲}[HSK 3]
  \definition{adv.}{todos os tipos; vários tipos}
\end{EntryWithPhonetic}

\begin{EntryWithPhonetic}{各自}{ge4zi4}{6,6}{⼝,⾃}[HSK 3]
  \definition{pron.}{por si mesmo; por conta própria; cada um por si | cada um; indica cada uma das partes envolvidas}
\end{EntryWithPhonetic}

%%%%%%%%%% 给 %%%%%%%%%%
\subsection*{给}\addcontentsline{loh}{figure}{给 \dpy{gei3}}

\begin{EntryWithPhonetic}{给}{gei3}{9}{⽷}[HSK 1]
  \definition{prep.}{por; expressa significado passivo; tem o mesmo significado que 被, 叫; pode ser seguido pelo agente da ação; o agente da ação também pode não aparecer na frase | para; a; seguido por quem se beneficia da ação; igual a 为 | em direção a; seguido pelo destinatário da ação; o mesmo que 向 | indica transmissão}
  \definition{v.}{dar; conceder; fazer com que a outra parte obtenha algo | passar; pagar; indicar que a outra pessoa faça algo | deixar; permitir que alguém faça algo; autorizar alguém a fazer algo}
  \definition{v.aux.}{usado antes de verbos predicativos que expressam passividade, disposição, etc., para reforçar o tom}
  \seeref{ji3}
  \seealsoref{被}{bei4}
  \seealsoref{叫}{jiao4}
  \seealsoref{为}{wei4}
  \seealsoref{向}{xiang4}
\end{EntryWithPhonetic}

\begin{EntryWithPhonetic}{给……打电话}{gei3 da3 dian4hua4}{9,5,5,8}{⽷,⼿,⽥,⾔}
  \definition{expr.}{dar um telefonema para alguém}
  \seealsoref{打电话}{da3 dian4hua4}
\end{EntryWithPhonetic}

\begin{EntryWithPhonetic}{给……定向}{gei3 ding4xiang4}{9,8,6}{⽷,⼧,⼝}
  \definition{v.}{dar orientação para algo; orientar algo}
\end{EntryWithPhonetic}

%%%%%%%%%% 根 %%%%%%%%%%
\subsection*{根}\addcontentsline{loh}{figure}{根 \dpy{gen1}}

\begin{EntryWithPhonetic}{根}{gen1}{10}{⽊}[HSK 4]
  \definition*{s.}{Sobrenome: Gen}
  \definition{adv.}{completamente; minuciosamente; radicalmente}
  \definition{clas.}{usado para objetos finos, alongados}
  \definition{s.}{raiz (de uma planta) | descendentes; posteridade; analogia com as gerações futuras | raiz (abreviação de raiz quadrada) | radical (química, refere"-se a radicais carregados) | base; pé; raiz; parte inferior, base ou parte de um objeto que está presa a outra coisa | a parte de baixo das coisas; fonte; a origem  das coisas | base; fundamento}
\end{EntryWithPhonetic}

\begin{EntryWithPhonetic}{根本}{gen1ben3}{10,5}{⽊,⽊}[HSK 3]
  \definition{adj.}{básico; essencial; fundamental; importante; decisivo}
  \definition{adv.}{nunca; simplesmente; de forma alguma | radicalmente; completamente; nunca (mais usado em negativas)}
  \definition[个]{s.}{base; raiz; fundação; a origem, a base ou a parte mais importante das coisas}
\end{EntryWithPhonetic}

\begin{EntryWithPhonetic}{根基}{gen1ji1}{10,11}{⽊,⼟}[HSK 7-9]
  \definition{s.}{base; fundação; alicerce; parte subterrânea de um edifício | recursos; propriedade acumulada ao longo de um longo período}
\end{EntryWithPhonetic}

\begin{EntryWithPhonetic}{根据}{gen1ju4}{10,11}{⽊,⼿}[HSK 4]
  \definition{prep.}{com base em; de acordo com; à luz de}
  \definition[个]{s.}{base; fundamentos; razão; fundo; alicerce}
  \definition{v.}{basear; usar algo como premissa para uma conclusão ou como base para uma ação verbal}
\end{EntryWithPhonetic}

\begin{EntryWithPhonetic}{根深蒂固}{gen1shen1-di4gu4}{10,11,12,8}{⽊,⽔,⾋,⼞}[HSK 7-9]
  \definition{expr.}{arraigado; inveterado; tornar-se profundamente enraizado em; profundamente enraizado; profundamente enraizado; profundamente enraizado e firmemente plantado -- bem fundado; ter uma base firme; ter raízes profundas e uma base firme; bem estabelecido; significa que a fundação é sólida e não se abala facilmente}
\end{EntryWithPhonetic}

\begin{EntryWithPhonetic}{根源}{gen1yuan2}{10,13}{⽊,⽔}[HSK 7-9]
  \definition{s.}{fonte; origem; raiz | raízes da grama; fonte; nascente; raiz; fundo}
  \definition{v.}{originar-se; provir de}
\end{EntryWithPhonetic}

\begin{EntryWithPhonetic}{根治}{gen1zhi4}{10,8}{⽊,⽔}[HSK 7-9]
  \definition{v.}{efetuar uma cura radical; curar de uma vez por todas; colocar sob controle permanente; curar completamente (referindo"-se à erradicação de pragas ou doenças)}
\end{EntryWithPhonetic}

%%%%%%%%%% 跟 %%%%%%%%%%
\subsection*{跟}\addcontentsline{loh}{figure}{跟 \dpy{gen1}}

\begin{EntryWithPhonetic}{跟}{gen1}{13}{⾜}[HSK 1]
  \definition{conj.}{e; expressa uma relação de união; 和}
  \definition{prep.}{com; Introduzir objetos relacionados à mesma ação, equivalente a 同 | para; em direção a | de; introduzir o objeto de comparação; equivalente a 从, 由 | como; objetos que causam comparações e semelhanças}
  \definition[个]{s.}{calcanhar; parte posterior do pé ou parte posterior do sapato ou meia | base (de um objeto)}
  \definition{v.}{seguir; acompanhar; seguir imediatamente na mesma direção | (uma mulher) estar casada com; casar"-se com alguém}
  \seealsoref{从}{cong2}
  \seealsoref{和}{he2}
  \seealsoref{同}{tong2}
  \seealsoref{由}{you2}
\end{EntryWithPhonetic}

\begin{EntryWithPhonetic}{跟不上}{gen1 bu5 shang4}{13,4,3}{⾜,⼀,⼀}[HSK 7-9]
  \definition{v.}{não é capaz de acompanhar; não conseguir alcançar}
\end{EntryWithPhonetic}

\begin{EntryWithPhonetic}{跟前}{gen1qian2}{13,9}{⾜,⼑}[HSK 5]
  \definition{s.}{próximo; perto de; na frente de; (na ou para) a presença de alguém | o tempo imediatamente anterior a algum evento; tempo que se aproxima}
  \seeref{gen1qian5}
\end{EntryWithPhonetic}

\begin{EntryWithPhonetic}{跟前}{gen1qian5}{13,9}{⾜,⼑}
  \definition{v.}{(filhos de alguém) viver com alguém (exclusivamente com relação à presença ou ausência de crianças)}
  \seeref{gen1qian2}
\end{EntryWithPhonetic}

\begin{EntryWithPhonetic}{跟上}{gen1shang5}{13,3}{⾜,⼀}[HSK 7-9]
  \definition{v.}{acompanhar; alcançar; manter-se a par de}
\end{EntryWithPhonetic}

\begin{EntryWithPhonetic}{跟随}{gen1sui2}{13,11}{⾜,⾩}[HSK 5]
  \definition{s.}{seguidor; usado para se referir a alguém que seguiu}
  \definition{v.}{seguir; ir atrás; acompanhar}
\end{EntryWithPhonetic}

\begin{EntryWithPhonetic}{跟踪}{gen1zong1}{13,15}{⾜,⾜}[HSK 7-9]
  \definition{v.}{rastrear; alcançar; seguir; seguir atrás; seguir alguém; seguir os rastros de; seguir de perto}
\end{EntryWithPhonetic}

%%%%%%%%%% 更 %%%%%%%%%%
\subsection*{更}\addcontentsline{loh}{figure}{更 \dpy{geng1}}

\begin{EntryWithPhonetic}{更}{geng1}{7}{⽈}
  \definition*{s.}{Sobrenome: Geng}
  \definition{clas.}{um dos cinco períodos de duas horas em que a noite era anteriormente dividida; vigília; antigamente, a noite era dividida em cinco turnos, cada um com aproximadamente duas horas de duração}
  \definition{v.}{alterar; substituir | experimentar}
  \seeref{geng4}
\end{EntryWithPhonetic}

\begin{EntryWithPhonetic}{更改}{geng1gai3}{7,7}{⽈,⽁}[HSK 7-9]
  \definition{v.}{alterar; mudar}
\end{EntryWithPhonetic}

\begin{EntryWithPhonetic}{更换}{geng1huan4}{7,10}{⽈,⼿}[HSK 5]
  \definition{v.}{alterar; mudar; substituir; comutar}
\end{EntryWithPhonetic}

\begin{EntryWithPhonetic}{更新}{geng1xin1}{7,13}{⽈,⽄}[HSK 5]
  \definition{v.}{renovar; atualizar; substituir; remover o antigo e substituir pelo novo}
\end{EntryWithPhonetic}

\begin{EntryWithPhonetic}{更衣室}{geng1yi1shi4}{7,6,9}{⽈,⾐,⼧}[HSK 7-9]
  \definition{s.}{vestiário | camarim | \emph{toilet}; toucador}
\end{EntryWithPhonetic}

%%%%%%%%%% 耕 %%%%%%%%%%
\subsection*{耕}\addcontentsline{loh}{figure}{耕 \dpy{geng1}}

\begin{EntryWithPhonetic}{耕}{geng1}{10}{⽾}
  \definition{v.}{arar; cultivar | trabalhar; fazer | ganhar a vida}
\end{EntryWithPhonetic}

\begin{EntryWithPhonetic}{耕地}{geng1/di4}{10,6}{⽾,⼟}[HSK 7-9]
  \definition[块,公顷]{s.}{terra cultivada; terra para cultivo}
  \definition{v.+compl.}{lavrar; arar}
\end{EntryWithPhonetic}

%%%%%%%%%% 耿 %%%%%%%%%%
\subsection*{耿}\addcontentsline{loh}{figure}{耿 \dpy{geng3}}

\begin{EntryWithPhonetic}{耿}{geng3}{10}{⽿}
  \definition*{s.}{Sobrenome: Geng}
  \definition{adj.}{Literário: brilhante | honesto e justo; correto; íntegro | dedicado; leal}
\end{EntryWithPhonetic}

\begin{EntryWithPhonetic}{耿直}{geng3zhi2}{10,8}{⽿,⽬}[HSK 7-9]
  \definition{adj.}{íntregro; franco; correto; honesto e franco}
\end{EntryWithPhonetic}

%%%%%%%%%% 颈 %%%%%%%%%%
\subsection*{颈}\addcontentsline{loh}{figure}{颈 \dpy{geng3}}

\begin{EntryWithPhonetic}{颈}{geng3}{11}{⾴}
  \definition{s.}{nuca}
  \seeref{jing3}
\end{EntryWithPhonetic}

%%%%%%%%%% 更 %%%%%%%%%%
\subsection*{更}\addcontentsline{loh}{figure}{更 \dpy{geng4}}

\begin{EntryWithPhonetic}{更}{geng4}{7}{⽈}[HSK 2]
  \definition{adv.}{mais; ainda mais | além disso; além do mais; ainda mais}
  \seeref{geng1}
\end{EntryWithPhonetic}

\begin{EntryWithPhonetic}{更加}{geng4jia1}{7,5}{⽈,⼒}[HSK 3]
  \definition{adv.}{mais; ainda mais; em maior grau; indica um nível mais profundo ou um aumento ou diminuição quantitativa adicional}
\end{EntryWithPhonetic}

\begin{EntryWithPhonetic}{更是}{geng4shi4}{7,9}{⽈,⽇}[HSK 6]
  \definition{adv.}{ainda mais (assim)}
\end{EntryWithPhonetic}

%%%%%%%%%% 工 %%%%%%%%%%
\subsection*{工}\addcontentsline{loh}{figure}{工 \dpy{gong1}}

\begin{EntryWithPhonetic}{工}{gong1}{3}{⼯}[Kangxi 48]
  \definition*{s.}{Sobrenome: Gong}
  \definition{adj.}{fino; requintado; delicado}
  \definition{s.}{trabalhador; operário; artesão | trabalho; labor; trabalho produtivo | projeto; construção; refere"-se à engenharia | indústria; refere"-se à indústria | homem"-dia; a quantidade de trabalho que um trabalhador faz em um dia | uma nota da escala em Gongchepu (工尺谱), correspondente a 3 na notação musical numerada | engenheiro; refere"-se a engenheiros}
  \definition{v.}{ser versado em; ser bom em | trabalhar em; agora geralmente escrito como 功}
  \seealsoref{功}{gong1}
  \seealsoref{工尺谱}{gong1 che3 pu3}
\end{EntryWithPhonetic}

\begin{EntryWithPhonetic}{工厂}{gong1chang3}{3,2}{⼯,⼚}[HSK 3]
  \definition[个,家,座,间]{s.}{fábrica; moinho; planta; unidades que realizam atividades de produção industrial diretamente, geralmente incluindo diferentes oficinas}
\end{EntryWithPhonetic}

\begin{EntryWithPhonetic}{工尺谱}{gong1 che3 pu3}{3,4,14}{⼯,⼫,⾔}
  \definition*{s.}{Gongchepu, notação musical tradicional chinesa}
  \definition{s.}{notação musical tradicional chinesa que usa caracteres chineses para representar notas musicais}
\end{EntryWithPhonetic}

\begin{EntryWithPhonetic}{工程}{gong1cheng2}{3,12}{⼯,⽲}[HSK 4]
  \definition[个,项]{s.}{projeto; programa; trabalhos que utilizam equipamentos grandes e complexos, como projetos de reconstrução urbana e projetos de cestas de alimentos, etc. | engenharia; departamentos de produção e manufatura usam equipamentos grandes e complexos para realizar seu trabalho}
\end{EntryWithPhonetic}

\begin{EntryWithPhonetic}{工程师}{gong1cheng2shi1}{3,12,6}{⼯,⽲,⼱}[HSK 3]
  \definition[个,位,名,些]{s.}{engenheiro; um dos cargos técnicos é o de especialista capaz de realizar de forma independente o projeto e a execução de uma tarefa técnica específica}
\end{EntryWithPhonetic}

\begin{EntryWithPhonetic}{工地}{gong1di4}{3,6}{⼯,⼟}[HSK 7-9]
  \definition{s.}{canteiro de obras; locais onde são realizadas construções, desenvolvimentos, produções, etc.}
\end{EntryWithPhonetic}

\begin{EntryWithPhonetic}{工夫}{gong1fu1}{3,4}{⼯,⼤}
  \definition[个]{s.}{tempo | tempo livre; lazer}
  \seeref{gong1fu5}
\end{EntryWithPhonetic}

\begin{EntryWithPhonetic}{工夫}{gong1fu5}{3,4}{⼯,⼤}[HSK 3]
  \definition[个]{s.}{(um período de) tempo; o tempo ou energia gastos para realizar uma tarefa | tempo livre}
  \seeref{gong1fu1}
\end{EntryWithPhonetic}

\begin{EntryWithPhonetic}{工会}{gong1hui4}{3,6}{⼯,⼈}[HSK 7-9]
  \definition[个]{s.}{sindicato; sindicato trabalhista; organizações de massa criadas pelos trabalhadores para proteger os seus próprios interesses}
\end{EntryWithPhonetic}

\begin{EntryWithPhonetic}{工具}{gong1ju4}{3,8}{⼯,⼋}[HSK 3]
  \definition[个,件,套]{s.}{ferramenta; ferramentas utilizadas na produção | ferramenta; meio; instrumento; (metáfora) algo ou meio utilizado para atingir um determinado objetivo}
\end{EntryWithPhonetic}

\begin{EntryWithPhonetic}{工科}{gong1ke1}{3,9}{⼯,⽲}[HSK 7-9]
  \definition{s.}{curso de engenharia | engenharia como disciplina acadêmica; um termo geral para disciplinas de engenharia no ensino}
\end{EntryWithPhonetic}

\begin{EntryWithPhonetic}{工龄}{gong1ling2}{3,13}{⼯,⿒}
  \definition{s.}{tempo de serviço | senioridade}
\end{EntryWithPhonetic}

\begin{EntryWithPhonetic}{工人}{gong1ren5}{3,2}{⼯,⼈}[HSK 1]
  \definition[个,名]{s.}{trabalhador; operário; mão de obra; trabalhadores braçais que vivem do salário}
\end{EntryWithPhonetic}

\begin{EntryWithPhonetic}{工商}{gong1shang1}{3,11}{⼯,⼝}[HSK 6]
  \definition{s.}{indústria e comércio; um termo combinado para indústria e comércio}
\end{EntryWithPhonetic}

\begin{EntryWithPhonetic}{工商界}{gong1shang1jie4}{3,11,9}{⼯,⼝,⽥}[HSK 7-9]
  \definition{s.}{círculos industriais e comerciais; círculos de negócios | indústria | o mundo dos negócios}
\end{EntryWithPhonetic}

\begin{EntryWithPhonetic}{工序}{gong1xu4}{3,7}{⼯,⼴}[HSK 7-9]
  \definition[道]{s.}{processo; procedimento de trabalho; sequência do processo de produção}
\end{EntryWithPhonetic}

\begin{EntryWithPhonetic}{工业}{gong1ye4}{3,5}{⼯,⼀}[HSK 3]
  \definition{s.}{indústria; utilização de recursos naturais; fabricação de meios de produção; meios de subsistência; ou processamento de produtos agrícolas, produtos semiacabados, etc.}
\end{EntryWithPhonetic}

\begin{EntryWithPhonetic}{工艺}{gong1yi4}{3,4}{⼯,⾋}[HSK 5]
  \definition{s.}{técnica; tecnologia; arte industrial; técnicas ou métodos de fabricação e processamento de produtos | artesanato; arte artesanal}
\end{EntryWithPhonetic}

\begin{EntryWithPhonetic}{工艺流程}{gong1yi4 liu2cheng2}{3,4,10,12}{⼯,⾋,⽔,⽲}
  \definition{s.}{fluxograma do processo; fluxo do processo}
\end{EntryWithPhonetic}

\begin{EntryWithPhonetic}{工艺品}{gong1 yi4 pin3}{3,4,9}{⼯,⾋,⼝}
  \definition[个,件]{s.}{trabalho manual; artesanato; habilidade manual; artigo artesanal; itens delicados produzidos com técnicas artesanais. Por exemplo, esculturas em jade, esmaltes Jingtailan, bordados, etc.}
\end{EntryWithPhonetic}

\begin{EntryWithPhonetic}{工整}{gong1zheng3}{3,16}{⼯,⽁}[HSK 7-9]
  \definition{adj.}{limpo; organizado; meticuloso e organizado; não desleixado | fino; requintado}
\end{EntryWithPhonetic}

\begin{EntryWithPhonetic}{工资}{gong1zi1}{3,10}{⼯,⾙}[HSK 3]
  \definition[份,笔,月,天]{s.}{pagamento; salário; remuneração; vencimentos; o pagamento em dinheiro ou em espécie feito ao trabalhador como remuneração pelo trabalho realizado}
\end{EntryWithPhonetic}

\begin{EntryWithPhonetic}{工作}{gong1zuo4}{3,7}{⼯,⼈}[HSK 1]
  \definition[份,个,分,项]{s.}{trabalho; emprego | dever; tarefa; negócio}
  \definition{v.}{trabalhar; operar (uma máquina); envolver-se em trabalho físico ou intelectual, também se refere de maneira geral a máquinas e ferramentas operadas por pessoas para realizar funções produtivas}
\end{EntryWithPhonetic}

\begin{EntryWithPhonetic}{工作量}{gong1zuo4liang4}{3,7,12}{⼯,⼈,⾥}[HSK 7-9]
  \definition{s.}{quantidade de trabalho; volume de trabalho; carga de trabalho}
\end{EntryWithPhonetic}

\begin{EntryWithPhonetic}{工作日}{gong1zuo4ri4}{3,7,4}{⼯,⼈,⽇}[HSK 5]
  \definition{s.}{dia de trabalho; dia útil; dias em que você deveria estar trabalhando de acordo com as regras | horas de trabalho por dia; horas do dia para fazer o trabalho necessário}
\end{EntryWithPhonetic}

%%%%%%%%%% 弓 %%%%%%%%%%
\subsection*{弓}\addcontentsline{loh}{figure}{弓 \dpy{gong1}}

\begin{EntryWithPhonetic}{弓}{gong1}{3}{⼸}[HSK 7-9][Kangxi 57]
  \definition*{s.}{Sobrenome: Gong}
  \definition{clas.}{uma antiga unidade de comprimento para medir a terra, igual a cinco 尺}
  \definition[张]{s.}{arco | qualquer coisa em forma de arco | Obsoleto: ferramenta de madeira de medição de terreno; divisores de madeira para medição de terrenos; arco de medição; régua escalonada (1,5m)}
  \definition{v.}{dobrar; arquear; curvar; entortar}
  \seealsoref{尺}{chi3}
\end{EntryWithPhonetic}

%%%%%%%%%% 公 %%%%%%%%%%
\subsection*{公}\addcontentsline{loh}{figure}{公 \dpy{gong1}}

\begin{EntryWithPhonetic}{公}{gong1}{4}{⼋}[HSK 6]
  \definition*{s.}{Sobrenome: Gong}
  \definition{adj.}{público; estatal; coletivo | comum; geral | do mundo; internacional; universal; métrico | imparcial; justo; equitativo |  (de um animal) masculino}
  \definition{s.}{assuntos públicos; negócios oficiais (ou deveres) | autoridade; coletivo | duque | títulos respeitosos para homens idosos; uma saudação respeitosa | marido}
  \definition{v.}{tornar público; divulgar; abrir a todos; exibir}
  \antonymref{母}{mu3}
  \antonymref{私}{si1}
\end{EntryWithPhonetic}

\begin{EntryWithPhonetic}{公安}{gong1'an1}{4,6}{⼋,⼧}[HSK 6]
  \definition[名,位]{s.}{segurança pública; a segurança e estabilidade dos direitos dos cidadãos, da propriedade da segurança pública e da ordem social | agente de segurança pública; pessoal que mantém a segurança pública}
\end{EntryWithPhonetic}

\begin{EntryWithPhonetic}{公安局}{gong1'an1ju2}{4,6,7}{⼋,⼧,⼫}[HSK 7-9]
  \definition{s.}{departamento de polícia; departamento de segurança pública; o departamento responsável pelo trabalho de segurança pública do Governo Popular}
\end{EntryWithPhonetic}

\begin{EntryWithPhonetic}{公办}{gong1ban4}{4,4}{⼋,⼒}
  \definition{adj.}{público; estatal; administrado pelo governo}
\end{EntryWithPhonetic}

\begin{EntryWithPhonetic}{公布}{gong1bu4}{4,5}{⼋,⼱}[HSK 3]
  \definition{v.}{(leis, decretos, comunicados e avisos de órgãos governamentais) promulgar; anunciar; publicar; tornar público; divulgar publicamente}
\end{EntryWithPhonetic}

\begin{EntryWithPhonetic}{公车}{gong1che1}{4,4}{⼋,⾞}[HSK 7-9]
  \definition[辆]{s.}{ônibus, abreviação de 公共汽车 | carro pertencente a uma organização e usado por seus membros (carro do governo, carro de polícia, carro da empresa etc.), abreviação de 公务用车}
  \seealsoref{公共}{gong1gong4}
  \seealsoref{公共汽车}{gong1gong4 qi4che1}
  \seealsoref{公务用车}{gong1wu4yong4che1}
\end{EntryWithPhonetic}

\begin{EntryWithPhonetic}{公道}{gong1dao4}{4,12}{⼋,⾡}
  \definition{s.}{justiça; o princípio da justiça}
  \seeref{gong1dao5}
\end{EntryWithPhonetic}

\begin{EntryWithPhonetic}{公道}{gong1dao5}{4,12}{⼋,⾡}[HSK 7-9]
  \definition{adj.}{equitativo | justo}
  \seeref{gong1dao4}
\end{EntryWithPhonetic}

\begin{EntryWithPhonetic}{公费}{gong1fei4}{4,9}{⼋,⾙}[HSK 7-9]
  \definition{s.}{despesa pública; despesas fornecidas pelo estado ou grupo}
\end{EntryWithPhonetic}

\begin{EntryWithPhonetic}{公告}{gong1gao4}{4,7}{⼋,⼝}[HSK 5]
  \definition[张,份,项]{s.}{anúncio; notificação de assuntos importantes ao público em geral pelo governo ou por um órgão importante}
  \definition{v.}{anunciar; o governo ou órgão governamental informa publicamente às pessoas algo importante}
\end{EntryWithPhonetic}

\begin{EntryWithPhonetic}{公共}{gong1gong4}{4,6}{⼋,⼋}[HSK 3]
  \definition{adj.}{público; comum; comunal; comunitário; pertencente à sociedade}
  \definition[辆]{s.}{ônibus}
  \seealsoref{公车}{gong1che1}
  \seealsoref{公共汽车}{gong1gong4 qi4che1}
\end{EntryWithPhonetic}

\begin{EntryWithPhonetic}{公共场所}{gong1gong4 chang3suo3}{4,6,6,8}{⼋,⼋,⼟,⼾}[HSK 7-9]
  \definition{s.}{lugares públicos; locais onde o público pode ir}
\end{EntryWithPhonetic}

\begin{EntryWithPhonetic}{公共关系}{gong1gong4 guan1xi4}{4,6,6,7}{⼋,⼋,⼋,⽷}
  \definition{s.}{relações públicas; refere"-se à relação entre grupos, empresas ou indivíduos em atividades sociais, denominada relações públicas}
\end{EntryWithPhonetic}

\begin{EntryWithPhonetic}{公共汽车}{gong1gong4 qi4che1}{4,6,7,4}{⼋,⼋,⽔,⾞}[HSK 2]
  \definition[辆,个]{s.}{ônibus}
  \seealsoref{公车}{gong1che1}
  \seealsoref{公共}{gong1gong4}
\end{EntryWithPhonetic}

\begin{EntryWithPhonetic}{公关}{gong1guan1}{4,6}{⼋,⼋}[HSK 7-9]
  \definition{s.}{relações públicas, abreviação de 公共关系 | pessoa que trabalha em relações públicas}
  \seealsoref{公共关系}{gong1gong4 guan1xi4}
\end{EntryWithPhonetic}

\begin{EntryWithPhonetic}{公函}{gong1han2}{4,8}{⼋,⼐}[HSK 7-9]
  \definition{s.}{carta oficial; correspondência oficial entre departamentos paralelos e não relacionados}
  \antonymref{便函}{bian4han2}
  \antonymref{私函}{si1han2}
\end{EntryWithPhonetic}

\begin{EntryWithPhonetic}{公鸡}{gong1ji1}{4,7}{⼋,⿃}[HSK 6]
  \definition{s.}{galo; frango macho}
\end{EntryWithPhonetic}

\begin{EntryWithPhonetic}{公积金}{gong1ji1jin1}{4,10,8}{⼋,⽲,⾦}[HSK 7-9]
  \definition{s.}{fundo de acumulação (comum); fundo de reserva pública; fundo de reserva comum | fundo acumulado | reservas oficiais}
\end{EntryWithPhonetic}

\begin{EntryWithPhonetic}{公交车}{gong1jiao1che1}{4,6,4}{⼋,⼇,⾞}[HSK 2]
  \definition[辆]{s.}{ônibus urbano; veículo de transporte público}
\end{EntryWithPhonetic}

\begin{EntryWithPhonetic}{公斤}{gong1jin1}{4,4}{⼋,⽄}[HSK 2]
  \definition{clas.}{quilograma (kg)}
\end{EntryWithPhonetic}

\begin{EntryWithPhonetic}{公开}{gong1kai1}{4,4}{⼋,⼶}[HSK 3]
  \definition{adj.}{aberto; público; não oculto; exposto ao público}
  \definition{v.}{tornar público}
\end{EntryWithPhonetic}

\begin{EntryWithPhonetic}{公开信}{gong1kai1xin4}{4,4,9}{⼋,⼶,⼈}[HSK 7-9]
  \definition{s.}{carta aberta; cartas endereçadas a indivíduos ou grupos que o autor acredita serem necessárias para divulgação pública}
\end{EntryWithPhonetic}

\begin{EntryWithPhonetic}{公克}{gong1ke4}{4,7}{⼋,⼗}
  \definition{s.}{grama (medida de peso)}
\end{EntryWithPhonetic}

\begin{EntryWithPhonetic}{公款}{gong1kuan3}{4,12}{⼋,⽋}[HSK 7-9]
  \definition{s.}{dinheiro público (ou fundo); despesa do governo | fundos públicos}
\end{EntryWithPhonetic}

\begin{EntryWithPhonetic}{公里}{gong1li3}{4,7}{⼋,⾥}[HSK 2]
  \definition{s.}{quilômetro (km)}
\end{EntryWithPhonetic}

\begin{EntryWithPhonetic}{公立}{gong1li4}{4,5}{⼋,⽴}[HSK 7-9]
  \definition{adj.}{estabelecido e mantido pelo governo; público | financiado publicamente e administrado pelo governo; estabelecido pelo governo e operado com fundos governamentais para fornecer serviços ao público}
  \antonymref{私立}{si1li4}
\end{EntryWithPhonetic}

\begin{EntryWithPhonetic}{公路}{gong1lu4}{4,13}{⼋,⾜}[HSK 2]
  \definition[条,段]{s.}{rodovia; via de acesso; via de tráfego; estrada; estrada principal;}
\end{EntryWithPhonetic}

\begin{EntryWithPhonetic}{公民}{gong1min2}{4,5}{⼋,⽒}[HSK 3]
  \definition[个,位]{s.}{cidadão; civil; pessoa que possui a nacionalidade de um país, goza dos direitos e cumpre as obrigações previstos na Constituição e nas demais leis desse país}
\end{EntryWithPhonetic}

\begin{EntryWithPhonetic}{公募}{gong1mu4}{4,12}{⼋,⼒}
  \definition{s.}{financiamento público; arrecadação pública de fundos (investimento)}
\end{EntryWithPhonetic}

\begin{EntryWithPhonetic}{公墓}{gong1mu4}{4,13}{⼋,⼟}[HSK 7-9]
  \definition[顿]{s.}{cemitério; cemitério público | Arcaico: túmulos ou cemitérios reais ou aristocráticos | parque memorial}
\end{EntryWithPhonetic}

\begin{EntryWithPhonetic}{公平}{gong1ping2}{4,5}{⼋,⼲}[HSK 2]
  \definition{adj.}{justo; imparcial; equitativo; equidade}
\end{EntryWithPhonetic}

\begin{EntryWithPhonetic}{公仆}{gong1pu2}{4,4}{⼋,⼈}[HSK 7-9]
  \definition[个,位,名,些]{s.}{servidor público; oficial | funcionário público; pessoas que servem o público}
\end{EntryWithPhonetic}

\begin{EntryWithPhonetic}{公顷}{gong1qing3}{4,8}{⼋,⾴}[HSK 7-9]
  \definition{clas.}{hectare; é uma unidade de área terrestre no sistema métrico;  equivale a 10.000 metros quadrados, ou 15 市亩}
  \seealsoref{市亩}{shi4mu3}
\end{EntryWithPhonetic}

\begin{EntryWithPhonetic}{公然}{gong1ran2}{4,12}{⼋,⽕}[HSK 7-9]
  \definition{adv.}{Pejorativo: abertamente; publicamente; sem disfarces; descaradamente; flagrantemente}
  \antonymref{暗自}{an4zi4}
\end{EntryWithPhonetic}

\begin{EntryWithPhonetic}{公认}{gong1ren4}{4,4}{⼋,⾔}[HSK 5]
  \definition{v.}{(geralmente) reconhecer; (universalmente) aceitar}
\end{EntryWithPhonetic}

\begin{EntryWithPhonetic}{公示}{gong1shi4}{4,5}{⼋,⽰}[HSK 7-9]
  \definition{v.}{dar a conhecer ao público e pedir opiniões}
\end{EntryWithPhonetic}

\begin{EntryWithPhonetic}{公式}{gong1shi4}{4,6}{⼋,⼷}[HSK 5]
  \definition[个,些,种]{s.}{fórmula; expressão}
\end{EntryWithPhonetic}

\begin{EntryWithPhonetic}{公事}{gong1shi4}{4,8}{⼋,⼅}[HSK 7-9]
  \definition{s.}{assuntos públicos; negócios oficiais (ou deveres) | Coloquial: documento oficial}
  \antonymref{私事}{si1shi4}
\end{EntryWithPhonetic}

\begin{EntryWithPhonetic}{公司}{gong1si1}{4,5}{⼋,⼝}[HSK 2]
  \definition[个,家]{s.}{empresa; companhia; corporação; uma organização industrial e comercial que opera na produção de produtos, circulação de mercadorias ou certos empreendimentos de construção, etc.}
\end{EntryWithPhonetic}

\begin{EntryWithPhonetic}{公司治理}{gong1si1zhi4li3}{4,5,8,11}{⼋,⼝,⽔,⽟}
  \definition{s.}{governança corporativa}
\end{EntryWithPhonetic}

\begin{EntryWithPhonetic}{公务}{gong1wu4}{4,5}{⼋,⼒}[HSK 7-9]
  \definition{s.}{assuntos públicos; negócios oficiais; em relação a assuntos nacionais ou coletivos}
\end{EntryWithPhonetic}

\begin{EntryWithPhonetic}{公务用车}{gong1wu4yong4che1}{4,5,5,4}{⼋,⼒,⽤,⾞}
  \definition{s.}{veículos oficiais}
\end{EntryWithPhonetic}

\begin{EntryWithPhonetic}{公务员}{gong1wu4yuan2}{4,5,7}{⼋,⼒,⼝}[HSK 3]
  \definition[个,位,名,些]{s.}{funcionário público; funcionário de órgãos governamentais}
\end{EntryWithPhonetic}

\begin{EntryWithPhonetic}{公益}{gong1yi4}{4,10}{⼋,⽫}[HSK 7-9]
  \definition{s.}{bem público; comunitário; bem"-estar; interesse público (referindo"-se principalmente ao bem"-estar público, como saúde e assistência)}
\end{EntryWithPhonetic}

\begin{EntryWithPhonetic}{公益性}{gong1yi4xing4}{4,10,8}{⼋,⽫,⼼}[HSK 7-9]
  \definition{s.}{bem-estar público}
\end{EntryWithPhonetic}

\begin{EntryWithPhonetic}{公用}{gong1yong4}{4,5}{⼋,⽤}[HSK 7-9]
  \definition{adj.}{público; comunitário; para uso público, uso comum}
\end{EntryWithPhonetic}

\begin{EntryWithPhonetic}{公用电话}{gong1yong4dian4hua4}{4,5,5,8}{⼋,⽤,⽥,⾔}
  \definition[部]{s.}{telefone público}
\end{EntryWithPhonetic}

\begin{EntryWithPhonetic}{公寓}{gong1yu4}{4,12}{⼋,⼧}[HSK 7-9]
  \definition[套,座,栋,间]{s.}{apartamentos; moradias; colônias de férias; prédio de apartamentos; construções que podem acomodar várias famílias}
\end{EntryWithPhonetic}

\begin{EntryWithPhonetic}{公元}{gong1yuan2}{4,4}{⼋,⼉}[HSK 4]
  \definition{s.}{D.C. (Depois de~Cristo); a era cristã; um método internacionalmente aceito de registro de datas, o ano lendário do nascimento de Jesus é 1 d.C., também conhecido como o primeiro ano da Era Comum, e é denotado por D.C.}
  \seealsoref{前}{qian2}
\end{EntryWithPhonetic}

\begin{EntryWithPhonetic}{公园}{gong1yuan2}{4,7}{⼋,⼞}[HSK 2]
  \definition[个,座]{s.}{parque; jardim público; os jardins abertos ao público para passeios e descanso geralmente ficam nas cidades, têm muitas flores, árvores e, em alguns casos, lagos}
\end{EntryWithPhonetic}

\begin{EntryWithPhonetic}{公约}{gong1yue1}{4,6}{⼋,⽷}[HSK 7-9]
  \definition[项,条]{s.}{convenção; pacto; aliança; tratado | compromisso conjunto; regulamentos acordados coletivamente dentro de uma organização | regulamentos acordados coletivamente dentro de uma unidade de trabalho}
\end{EntryWithPhonetic}

\begin{EntryWithPhonetic}{公正}{gong1zheng4}{4,5}{⼋,⽌}[HSK 5]
  \definition{adj.}{justo; equitativo; imparcial; de mente justa; equidade e integridade sem favoritismo}
\end{EntryWithPhonetic}

\begin{EntryWithPhonetic}{公证}{gong1zheng4}{4,7}{⼋,⾔}[HSK 7-9]
  \definition{adj.}{autenticado em cartório}
  \definition{s.}{reconhecimento de firma}
  \definition{v.}{autenticar}
\end{EntryWithPhonetic}

\begin{EntryWithPhonetic}{公职}{gong1zhi2}{4,11}{⼋,⽿}[HSK 7-9]
  \definition{s.}{cargo público; emprego público; cargos ou patentes oficiais}
\end{EntryWithPhonetic}

\begin{EntryWithPhonetic}{公众}{gong1zhong4}{4,6}{⼋,⼈}[HSK 6]
  \definition[对]{s.}{o público; as massas; refere"-se à maioria das pessoas na sociedade}
\end{EntryWithPhonetic}

\begin{EntryWithPhonetic}{公主}{gong1zhu3}{4,5}{⼋,⼂}[HSK 6]
  \definition[个,位,名,些]{s.}{princesa; a filha do monarca}
\end{EntryWithPhonetic}

%%%%%%%%%% 功 %%%%%%%%%%
\subsection*{功}\addcontentsline{loh}{figure}{功 \dpy{gong1}}

\begin{EntryWithPhonetic}{功}{gong1}{5}{⼒}[HSK 7-9]
  \definition*{s.}{Sobrenome: Gong}
  \definition[次,大]{s.}{mérito; façanha; serviço meritório (ação) | resultado; eficácia; realização | habilidade; habilidade técnica; tecnologia e qualificação técnica | trabalho; uma força faz com que um objeto se desloque uma certa distância na direção da força}
\end{EntryWithPhonetic}

\begin{EntryWithPhonetic}{功臣}{gong1chen2}{5,6}{⼒,⾂}[HSK 7-9]
  \definition[个]{s.}{pessoa que prestou serviço excepcional | ministro que prestou serviço excepcional; um funcionário meritório se refere a alguém que fez contribuições significativas para uma determinada causa.}
  \antonymref{罪人}{zui4ren2}
\end{EntryWithPhonetic}

\begin{EntryWithPhonetic}{功底}{gong1di3}{5,8}{⼒,⼴}[HSK 7-9]
  \definition{s.}{fundação; base sólida; formação profunda; conhecimento dos fundamentos; boas habilidades básicas}
\end{EntryWithPhonetic}

\begin{EntryWithPhonetic}{功夫}{gong1fu5}{5,4}{⼒,⼤}[HSK 3]
  \definition*{s.}{Gongfu (Kung Fu), arte marcial}
  \definition[番]{s.}{habilidade; destreza; conhecimento | luta acrobática; habilidade em artes marciais | esforço; tempo e energia}
\end{EntryWithPhonetic}

\begin{EntryWithPhonetic}{功绩}{gong1ji4}{5,11}{⼒,⽷}
  \definition{s.}{mérito e realização; feito; contribuição}
  \antonymref{过失}{guo4shi1}
\end{EntryWithPhonetic}

\begin{EntryWithPhonetic}{功课}{gong1ke4}{5,10}{⼒,⾔}[HSK 3]
  \definition[份,门]{s.}{trabalho escolar; dever de casa; refere"-se aos trabalhos de casa atribuídos pelos professores aos alunos| tarefa; lições; lição escolar | preparações; preparação necessária antes de fazer algo}
\end{EntryWithPhonetic}

\begin{EntryWithPhonetic}{功劳}{gong1lao2}{5,7}{⼒,⼒}[HSK 7-9]
  \definition{s.}{contribuição; crédito; contribuição para a causa}
\end{EntryWithPhonetic}

\begin{EntryWithPhonetic}{功力}{gong1li4}{5,2}{⼒,⼒}[HSK 7-9]
  \definition{s.}{efeito; eficácia | habilidade}
\end{EntryWithPhonetic}

\begin{EntryWithPhonetic}{功率}{gong1lv4}{5,11}{⼒,⽞}[HSK 7-9]
  \definition[瓦,千瓦,兆瓦]{s.}{potência (W); uma grandeza física que indica a velocidade com que o trabalho é realizado; ou seja, o trabalho realizado ou consumido por unidade de tempo; a unidade é Watt}
\end{EntryWithPhonetic}

\begin{EntryWithPhonetic}{功能}{gong1neng2}{5,10}{⼒,⾁}[HSK 3]
  \definition[种,项]{s.}{função; os efeitos positivos produzidos por coisas ou métodos}
\end{EntryWithPhonetic}

\begin{EntryWithPhonetic}{功效}{gong1xiao4}{5,10}{⼒,⽁}[HSK 7-9]
  \definition{s.}{efeito; eficiência; comportamento; eficácia; o efeito, função ou eficiência de um medicamento, método ou outra coisa}
\end{EntryWithPhonetic}

%%%%%%%%%% 攻 %%%%%%%%%%
\subsection*{攻}\addcontentsline{loh}{figure}{攻 \dpy{gong1}}

\begin{EntryWithPhonetic}{攻}{gong1}{7}{⽁}[HSK 7-9]
  \definition*{s.}{Sobrenome: Gong}
  \definition{v.}{atacar; assaltar; tomar a ofensiva | acusar; cobrar | estudar; trabalhar em; especializar-se em}
\end{EntryWithPhonetic}

\begin{EntryWithPhonetic}{攻读}{gong1du2}{7,10}{⽁,⾔}[HSK 7-9]
  \definition{v.}{especializar-se em; trabalhar arduamente em uma matéria para obter um diploma ou certificado nessa matéria; estudar bastante; estudar ou aprofundar-se em um assunto}
\end{EntryWithPhonetic}

\begin{EntryWithPhonetic}{攻关}{gong1guan1}{7,6}{⽁,⼋}[HSK 7-9]
  \definition{v.}{ter que encarar; superar esseobstáculo; começar essa jornada; resolver esse problema; abordar os principais problemas}
\end{EntryWithPhonetic}

\begin{EntryWithPhonetic}{攻击}{gong1ji1}{7,5}{⽁,⼐}[HSK 6]
  \definition{v.}{atacar; assaltar; lançar uma ofensiva | difamar; caluniar; acusar; atacar (verbalmente)}
\end{EntryWithPhonetic}

%%%%%%%%%% 供 %%%%%%%%%%
\subsection*{供}\addcontentsline{loh}{figure}{供 \dpy{gong1}}

\begin{EntryWithPhonetic}{供}{gong1}{8}{⼈}[HSK 7-9]
  \definition*{s.}{Sobrenome: Gong}
  \definition{v.}{fornecer; alimentar |  fornecer algo (para uso ou conveniência de); fornecer algumas condições de exploração à outra parte}
  \seeref{gong4}
\end{EntryWithPhonetic}

\begin{EntryWithPhonetic}{供不应求}{gong1bu2ying4qiu2}{8,4,7,7}{⼈,⼀,⼴,⽔}[HSK 7-9]
  \definition{expr.}{``A oferta fica aquém da demanda.'' ou ``A demanda excede a oferta.''}
\end{EntryWithPhonetic}

\begin{EntryWithPhonetic}{供给}{gong1ji3}{8,9}{⼈,⽷}[HSK 6]
  \definition{s.}{fornecer; prover; fornecer produção e necessidades de vida, dinheiro, etc. para aqueles que precisam}
\end{EntryWithPhonetic}

\begin{EntryWithPhonetic}{供暖}{gong1nuan3}{8,13}{⼈,⽇}[HSK 7-9]
  \definition{s.}{fornecimento de aquecimento}
  \definition{v.}{fornecer aquecimento}
\end{EntryWithPhonetic}

\begin{EntryWithPhonetic}{供求}{gong1qiu2}{8,7}{⼈,⽔}[HSK 7-9]
  \definition{s.}{Economia: oferta e procura (principalmente de commodities)}
\end{EntryWithPhonetic}

\begin{EntryWithPhonetic}{供应}{gong1ying4}{8,7}{⼈,⼴}[HSK 4]
  \definition{v.}{fornecer; prover de}
\end{EntryWithPhonetic}

%%%%%%%%%% 宫 %%%%%%%%%%
\subsection*{宫}\addcontentsline{loh}{figure}{宫 \dpy{gong1}}

\begin{EntryWithPhonetic}{宫}{gong1}{9}{⼧}[HSK 6]
  \definition*{s.}{Sobrenome: Gong}
  \definition[座]{s.}{palácio imperial; palácio; casas onde o imperador, a imperatriz, o príncipe, etc. vivem | morada de seres sobrenaturais; palácio; paraíso; casas onde vivem os deuses na mitologia | templo (usado em um nome de templo) | local para atividades culturais e recreativas; um edifício para atividades culturais e recreativas; casas para fins culturais e de entretenimento | útero | uma nota da antiga escala chinesa de cinco tons, correspondente a 1 na notação musical numerada}
\end{EntryWithPhonetic}

\begin{EntryWithPhonetic}{宫殿}{gong1dian4}{9,13}{⼧,⽎}[HSK 7-9]
  \definition[座]{s.}{palácio; geralmente se refere às casas magníficas onde os imperadores vivem}
\end{EntryWithPhonetic}

\begin{EntryWithPhonetic}{宫廷}{gong1ting2}{9,6}{⼧,⼵}[HSK 7-9]
  \definition{s.}{palácio imperial; residência do imperador; palácio | corte real ou imperial; o monarca e seus funcionários; corte}
\end{EntryWithPhonetic}

%%%%%%%%%% 恭 %%%%%%%%%%
\subsection*{恭}\addcontentsline{loh}{figure}{恭 \dpy{gong1}}

\begin{EntryWithPhonetic}{恭}{gong1}{10}{⼼}
  \definition{adj.}{respeitoso; reverente | educado}
\end{EntryWithPhonetic}

\begin{EntryWithPhonetic}{恭维}{gong1wei2}{10,11}{⼼,⽷}[HSK 7-9]
  \definition{v.}{bajular; elogiar}
\end{EntryWithPhonetic}

\begin{EntryWithPhonetic}{恭喜}{gong1xi3}{10,12}{⼼,⼝}[HSK 7-9]
  \definition{v.}{parabenizar; uma maneira educada de parabenizar alguém por seu feliz evento}
\end{EntryWithPhonetic}

%%%%%%%%%% 巩 %%%%%%%%%%
\subsection*{巩}\addcontentsline{loh}{figure}{巩 \dpy{gong3}}

\begin{EntryWithPhonetic}{巩}{gong3}{6}{⼯}
  \definition*{s.}{Sobrenome: Gong}
  \definition{s.}{seguro | sólido}
  \definition{v.}{consolidar}
\end{EntryWithPhonetic}

\begin{EntryWithPhonetic}{巩固}{gong3gu4}{6,8}{⼯,⼞}[HSK 6]
  \definition{adj.}{sólido; estável; consolidado; não facilmente abalado (usado principalmente para coisas abstratas)}
  \definition{v.}{consolidar}
\end{EntryWithPhonetic}

%%%%%%%%%% 拱 %%%%%%%%%%
\subsection*{拱}\addcontentsline{loh}{figure}{拱 \dpy{gong3}}

\begin{EntryWithPhonetic}{拱}{gong3}{9}{⼿}[HSK 7-9]
  \definition*{s.}{Sobrenome: Gong}
  \definition{s.}{Arquitetura: arco}[游客们在拱门前留影。===Turistas tiram fotos em frente ao arco.]
  \definition{v.}{colocar uma mão na outra em frente ao peito (em saudação) | cercar | arquear-se | empurrar sem usar as mãos; bater em um objeto com seu corpo | (porcos, etc.) cavar a terra com o focinho; (minhocas, etc.) contorcer-se na terra | brotar através da terra}
\end{EntryWithPhonetic}

%%%%%%%%%% 共 %%%%%%%%%%
\subsection*{共}\addcontentsline{loh}{figure}{共 \dpy{gong4}}

\begin{EntryWithPhonetic}{共}{gong4}{6}{⼋}[HSK 4]
  \definition*{s.}{Partido Comunista, abreviação de 共产党 | Sobrenome: Gong}
  \definition{adj.}{conjunto; mútuo; geral; comum; o mesmo para todos}
  \definition{adv.}{juntos; juntamente; conjuntamente | em sua totalidade; em todos}
  \definition{v.}{compartilhar com; empreender ou realizar em conjunto}
  \seealsoref{共产党}{gong4chan3dang3}
\end{EntryWithPhonetic}

\begin{EntryWithPhonetic}{共产}{gong4chan3}{6,6}{⼋,⼇}
  \definition{adj.}{comunista}
  \definition{s.}{comunismo}
\end{EntryWithPhonetic}

\begin{EntryWithPhonetic}{共产党}{gong4chan3dang3}{6,6,10}{⼋,⼇,⼉}
  \definition*{s.}{Partido Comunista}
\end{EntryWithPhonetic}

\begin{EntryWithPhonetic}{共产主义}{gong4chan3 zhu3yi4}{6,6,5,3}{⼋,⼇,⼂,⼂}
  \definition*{s.}{Comunismo}
\end{EntryWithPhonetic}

\begin{EntryWithPhonetic}{共计}{gong4ji4}{6,4}{⼋,⾔}[HSK 5]
  \definition{s.}{total; total geral; agregado; montante}
  \definition{v.}{contar até; somar até; totalizar}
\end{EntryWithPhonetic}

\begin{EntryWithPhonetic}{共鸣}{gong4ming2}{6,8}{⼋,⿃}[HSK 7-9]
  \definition{s.}{ressonância; fenômeno que ocorre quando um objeto ressoa, por exemplo, quando dois diapasões com a mesma frequência são colocados próximos um do outro, quando um vibra e emite um som, o outro também emite um som | resposta simpática; uma metáfora para ter as mesmas emoções que outra pessoa}
\end{EntryWithPhonetic}

\begin{EntryWithPhonetic}{共时}{gong4shi2}{6,7}{⼋,⽇}
  \definition{adj.}{sincrônico; simultâneo}
  \antonymref{历时}{li4shi2}
\end{EntryWithPhonetic}

\begin{EntryWithPhonetic}{共识}{gong4shi2}{6,7}{⼋,⾔}[HSK 7-9]
  \definition{s.}{consenso; entendimento comum}
\end{EntryWithPhonetic}

\begin{EntryWithPhonetic}{共同}{gong4tong2}{6,6}{⼋,⼝}[HSK 3]
  \definition{adj.}{comum; compartilhado; colaborativo; todos têm}
  \definition{adv.}{juntos; conjuntamente; todos juntos (fazemos)}
\end{EntryWithPhonetic}

\begin{EntryWithPhonetic}{共同体}{gong4tong2ti3}{6,6,7}{⼋,⼝,⼈}[HSK 7-9]
  \definition[个]{s.}{comunidade}[欧洲经济共同体===Comunidade Econômica Europeia]
\end{EntryWithPhonetic}

\begin{EntryWithPhonetic}{共享}{gong4xiang3}{6,8}{⼋,⼇}[HSK 5]
  \definition{v.}{compartilhar; desfrutar juntos; aproveitar as coisas boas juntos}
\end{EntryWithPhonetic}

\begin{EntryWithPhonetic}{共性}{gong4xing4}{6,8}{⼋,⼼}[HSK 7-9]
  \definition{s.}{caráter geral (comum); natureza comum; generalidade; semelhança; universalidade}
\end{EntryWithPhonetic}

\begin{EntryWithPhonetic}{共有}{gong4you3}{6,6}{⼋,⽉}[HSK 3]
  \definition{v.}{compartilhar; possuir (por todos); possuir ou desfrutar em conjunto}
\end{EntryWithPhonetic}

%%%%%%%%%% 贡 %%%%%%%%%%
\subsection*{贡}\addcontentsline{loh}{figure}{贡 \dpy{gong4}}

\begin{EntryWithPhonetic}{贡}{gong4}{7}{⾙}
  \definition*{s.}{Sobrenome: Gong}
  \definition[个,批]{s.}{tributo}
  \definition{v.}{recomendar uma pessoa adequada à corte imperial; recomendar talentos à corte na era feudal | prestar homenagem; pagar tributos (à corte imperial)}
\end{EntryWithPhonetic}

\begin{EntryWithPhonetic}{贡献}{gong4xian4}{7,13}{⾙,⽝}[HSK 6]
  \definition[份]{s.}{contribuição; boas ações feitas para o país ou para o público}
  \definition{v.}{dedicar; contribuir; contribuir com materiais, força, experiência, etc. para o país ou para o público}
\end{EntryWithPhonetic}

%%%%%%%%%% 供 %%%%%%%%%%
\subsection*{供}\addcontentsline{loh}{figure}{供 \dpy{gong4}}

\begin{EntryWithPhonetic}{供}{gong4}{8}{⼈}
  \definition{s.}{oferendas | confissão}
  \definition{v.}{depositar (oferendas) | confessar}
  \seeref{gong1}
\end{EntryWithPhonetic}

\begin{EntryWithPhonetic}{供奉}{gong4feng4}{8,8}{⼈,⼤}[HSK 7-9]
  \definition{s.}{artista servindo ao imperador; uma pessoa que serve ao imperador com alguma habilidade; especialmente um ator que é convocado ao palácio para atuar}
  \definition{v.}{consagrar; consagrar e adorar; colocar incenso e velas em frente aos retratos ou tábuas de deuses, Budas ou ancestrais; colocar oferendas; mostrar respeito | prestar homenagem à corte imperial}
\end{EntryWithPhonetic}

%%%%%%%%%% 勾 %%%%%%%%%%
\subsection*{勾}\addcontentsline{loh}{figure}{勾 \dpy{gou1}}

\begin{EntryWithPhonetic}{勾}{gou1}{4}{⼓}[HSK 7-9]
  \definition*{s.}{Sobrenome: Gou}
  \definition{s.}{nos tempos antigos, referia"-se ao lado mais curto de um triângulo retângulo isósceles}
  \definition{v.}{cancelar; riscar; marcar | delinear; desenhar | preencher as juntas da alvenaria com argamassa ou cimento; apontar | adicionar algo para engrossar; engrossar | induzir; evocar; trazer à mente | conspirar com; unir"-se a | excluir; apagar | seduzir; atrair}
  \variantof{钩}
  \seeref{gou4}
\end{EntryWithPhonetic}

\begin{EntryWithPhonetic}{勾画}{gou1hua4}{4,8}{⼓,⽥}[HSK 7-9]
  \definition{v.}{esboçar; delinear; desenhar o contorno de; descrever em palavras curtas}
\end{EntryWithPhonetic}

\begin{EntryWithPhonetic}{勾结}{gou1jie2}{4,9}{⼓,⽷}[HSK 7-9]
  \definition{v.}{conspirar com; estar em aliança com; ser de mão e luva com; colaborar com; unir-se a; conspirar secretamente e combinar-se entre si para realizar atividades impróprias}
\end{EntryWithPhonetic}

%%%%%%%%%% 沟 %%%%%%%%%%
\subsection*{沟}\addcontentsline{loh}{figure}{沟 \dpy{gou1}}

\begin{EntryWithPhonetic}{沟}{gou1}{7}{⽔}[HSK 5]
  \definition[条,道,段]{s.}{canal; vala; sarjeta; trincheira; cursos d'água ou fortificações escavados | ranhura; sulco raso; uma depressão que se assemelha a uma vala | ravina; barranco; cursos d'água}
\end{EntryWithPhonetic}

\begin{EntryWithPhonetic}{沟通}{gou1tong1}{7,10}{⽔,⾡}[HSK 5]
  \definition{v.}{comunicar; comunicar-se para entender as ideias, opiniões, etc. | conectar; ligar; estabelecer um paralelo entre os dois}
\end{EntryWithPhonetic}

%%%%%%%%%% 钩 %%%%%%%%%%
\subsection*{钩}\addcontentsline{loh}{figure}{钩 \dpy{gou1}}

\begin{EntryWithPhonetic}{钩}{gou1}{9}{⾦}[HSK 7-9]
  \definition*{s.}{Sobrenome: Gou}
  \definition[只,个]{s.}{gancho | traço de gancho em caracteres chineses | marca de verificação; visto; \emph{tick}; \emph{check mark} | marca em forma de gancho | uma espada em forma de gancho | forma falada do numeral 9 em certas ocasiões}
  \definition{v.}{prender com um gancho; enganchar | fazer crochê | costurar com pontos grandes | costurar com pontos longos}
\end{EntryWithPhonetic}

\begin{EntryWithPhonetic}{钩子}{gou1zi5}{9,3}{⾦,⼦}[HSK 7-9]
  \definition[个]{s.}{gancho | coisa parecida com um gancho}
\end{EntryWithPhonetic}

%%%%%%%%%% 狗 %%%%%%%%%%
\subsection*{狗}\addcontentsline{loh}{figure}{狗 \dpy{gou3}}

\begin{EntryWithPhonetic}{狗}{gou3}{8}{⽝}[HSK 2]
  \definition[条,只,群]{s.}{cão; cachorro | palavrão usado para se referir a pessoas más ou seus capangas}
\end{EntryWithPhonetic}

%%%%%%%%%% 勾 %%%%%%%%%%
\subsection*{勾}\addcontentsline{loh}{figure}{勾 \dpy{gou4}}

\begin{EntryWithPhonetic}{勾}{gou4}{4}{⼓}
  \definition*{s.}{Sobrenome: Gou}
  \definition{s.}{usado em 勾当}
  \seeref{gou1}
  \seealsoref{勾当}{gou4dang4}
\end{EntryWithPhonetic}

\begin{EntryWithPhonetic}{勾当}{gou4dang4}{4,6}{⼓,⼹}
  \definition{s.}{(depreciativo, obscuro) negócio; acordo; esquema; atividade}
\end{EntryWithPhonetic}

%%%%%%%%%% 句 %%%%%%%%%%
\subsection*{句}\addcontentsline{loh}{figure}{句 \dpy{gou4}}

\begin{EntryWithPhonetic}{句}{gou4}{5}{⼝}
  \variantof{勾}
  \seeref{ju4}
\end{EntryWithPhonetic}

%%%%%%%%%% 构 %%%%%%%%%%
\subsection*{构}\addcontentsline{loh}{figure}{构 \dpy{gou4}}

\begin{EntryWithPhonetic}{构}{gou4}{8}{⽊}
  \definition{s.}{composição literária}
  \definition{v.}{construir; formar; compor | fabricar; inventar | construir; erguer uma casa}
  \variantof{够}
\end{EntryWithPhonetic}

\begin{EntryWithPhonetic}{构成}{gou4/cheng2}{8,6}{⽊,⼽}[HSK 4]
  \definition{s.}{parte; componente; composição; estrutura}
  \definition{v.+compl.}{formar; compor; constituir; compor; encaixar muitas partes para formar um todo | consistir; causar; formar (principalmente em termos jurídicos)}
\end{EntryWithPhonetic}

\begin{EntryWithPhonetic}{构建}{gou4jian4}{8,8}{⽊,⼵}[HSK 6]
  \definition{v.}{estabelecer (usado principalmente para coisas abstratas); montar; instalar}
\end{EntryWithPhonetic}

\begin{EntryWithPhonetic}{构思}{gou4si1}{8,9}{⽊,⼼}[HSK 7-9]
  \definition{s.}{concepção (ideia); o resultado da concepção}
  \definition{v.}{elaborar o enredo de uma obra literária ou a composição de uma pintura; pensar bem antes de escrever artigos ou criar obras literárias}
\end{EntryWithPhonetic}

\begin{EntryWithPhonetic}{构想}{gou4xiang3}{8,13}{⽊,⼼}[HSK 7-9]
  \definition{s.}{ideia; concepção; ideias formadas}
  \definition[种,个]{v.}{pensar (em um plano, projeto, etc.); conceber; usar a mente ao escrever ou criar arte}
\end{EntryWithPhonetic}

\begin{EntryWithPhonetic}{构造}{gou4zao4}{8,10}{⽊,⾡}[HSK 4]
  \definition[种]{s.}{estrutura; construção; disposição, organização e inter-relação dos componentes}
  \definition{v.}{formar; construir}
\end{EntryWithPhonetic}

%%%%%%%%%% 诟 %%%%%%%%%%
\subsection*{诟}\addcontentsline{loh}{figure}{诟 \dpy{gou4}}

\begin{EntryWithPhonetic}{诟}{gou4}{8}{⾔}
  \definition*{s.}{Sobrenome: Gou}
  \definition{s.}{vergonha; humilhação}
  \definition{v.}{insultar; xingar; falar de forma abusiva}
\end{EntryWithPhonetic}

\begin{EntryWithPhonetic}{诟骂}{gou4ma4}{8,9}{⾔,⾺}
  \definition{v.}{abusar verbalmente | insultar | criticar}
\end{EntryWithPhonetic}

%%%%%%%%%% 购 %%%%%%%%%%
\subsection*{购}\addcontentsline{loh}{figure}{购 \dpy{gou4}}

\begin{EntryWithPhonetic}{购}{gou4}{8}{⾙}[HSK 7-9]
  \definition{v.}{comprar}
\end{EntryWithPhonetic}

\begin{EntryWithPhonetic}{购买}{gou4mai3}{8,6}{⾙,⼄}[HSK 4]
  \definition{v.}{comprar; adquirir; usar dinheiro para obter itens}
\end{EntryWithPhonetic}

\begin{EntryWithPhonetic}{购物}{gou4wu4}{8,8}{⾙,⽜}[HSK 4]
  \definition{s.}{compras; itens comprados; \emph{shopping}}
  \definition{v.}{ir às compras; fazer compras}
\end{EntryWithPhonetic}

%%%%%%%%%% 够 %%%%%%%%%%
\subsection*{够}\addcontentsline{loh}{figure}{够 \dpy{gou4}}

\begin{EntryWithPhonetic}{够}{gou4}{11}{⼣}[HSK 2]
  \definition{adj.}{suficiente; adequado; apropriado; atingir e ultrapassar um determinado limite, difícil de suportar}
  \definition{adv.}{suficientemente; o suficiente (para atingir um determinado nível); indica que atingiu um determinado padrão ou nível elevado}
  \definition{v.}{alcançar (algo, esticando-se); (usando membros, etc.) esticar-se para alcançar ou tocar em locais de difícil acesso | atingir (um padrão ou nível); satisfazer ou atingir a quantidade, os padrões, etc. necessários}
\end{EntryWithPhonetic}

\begin{EntryWithPhonetic}{够本}{gou4ben3}{11,5}{⼣,⽊}
  \definition{v.}{empatar | fazer valer o dinheiro}
\end{EntryWithPhonetic}

\begin{EntryWithPhonetic}{够不着}{gou4bu5zhao2}{11,4,11}{⼣,⼀,⽬}
  \definition{v.}{ser incapaz de alcançar}
\end{EntryWithPhonetic}

\begin{EntryWithPhonetic}{够得着}{gou4de5zhao2}{11,11,11}{⼣,⼻,⽬}
  \definition{v.}{estar à altura | alcançar}
\end{EntryWithPhonetic}

\begin{EntryWithPhonetic}{够格}{gou4ge2}{11,10}{⼣,⽊}
  \definition{adj.}{apto | qualificado | apresentável}
\end{EntryWithPhonetic}

\begin{EntryWithPhonetic}{够朋友}{gou4peng2you5}{11,8,4}{⼣,⽉,⼜}
  \definition{v.}{ser um amigo verdadeiro}
\end{EntryWithPhonetic}

\begin{EntryWithPhonetic}{够呛}{gou4qiang4}{11,7}{⼣,⼝}[HSK 7-9]
  \definition{adj.}{terrível; insuportável; descreve uma situação extremamente grave e insuportável | improvável; bastante improvável; quase impossível; descreve como difícil de alcançar}
\end{EntryWithPhonetic}

\begin{EntryWithPhonetic}{够戗}{gou4qiang4}{11,8}{⼣,⼽}
  \variantof{够呛}
\end{EntryWithPhonetic}

\begin{EntryWithPhonetic}{够味}{gou4wei4}{11,8}{⼣,⼝}
  \definition{adj.}{excelente | na medida}
\end{EntryWithPhonetic}

%%%%%%%%%% 彀 %%%%%%%%%%
\subsection*{彀}\addcontentsline{loh}{figure}{彀 \dpy{gou4}}

\begin{EntryWithPhonetic}{彀}{gou4}{13}{⼸}
  \definition{adj.}{suficiente; adequado}
  \definition{v.}{puxar um arco ao máximo}
\end{EntryWithPhonetic}

%%%%%%%%%% 估 %%%%%%%%%%
\subsection*{估}\addcontentsline{loh}{figure}{估 \dpy{gu1}}

\begin{EntryWithPhonetic}{估}{gu1}{7}{⼈}
  \definition{v.}{estimar; avaliar; aferir}
  \seeref{gu4}
\end{EntryWithPhonetic}

\begin{EntryWithPhonetic}{估计}{gu1ji4}{7,4}{⼈,⾔}[HSK 5]
  \definition{v.}{fazer contas; estimar; calcular; julgar a natureza, quantidade, mudança, etc. de uma coisa em uma determinada situação | parecer; parecer como se; aparentar; fazer inferências aproximadas sobre a natureza, a quantidade e a mudança das coisas com base em determinadas circunstâncias}
\end{EntryWithPhonetic}

\begin{EntryWithPhonetic}{估算}{gu1suan4}{7,14}{⼈,⽵}[HSK 7-9]
  \definition{v.}{calcular; estimar}
\end{EntryWithPhonetic}

%%%%%%%%%% 姑 %%%%%%%%%%
\subsection*{姑}\addcontentsline{loh}{figure}{姑 \dpy{gu1}}

\begin{EntryWithPhonetic}{姑}{gu1}{8}{⼥}
  \definition{adv.}{provisoriamente; por enquanto}
  \definition[个,位,名,些]{s.}{irmã do pai; tia | irmã do marido; cunhada | mãe do marido; sogra | freira; mulher que exerce uma ocupação religiosa | a irmã do pai de alguém | mulheres jovens (no campo)}
\end{EntryWithPhonetic}

\begin{EntryWithPhonetic}{姑姑}{gu1gu5}{8,8}{⼥,⼥}[HSK 6]
  \definition[个,位,名]{s.}{tia; tia paterna}
\end{EntryWithPhonetic}

\begin{EntryWithPhonetic}{姑娘}{gu1niang5}{8,10}{⼥,⼥}[HSK 3]
  \definition[位,名,个,些]{s.}{menina; jovem senhora; mulher solteira | filha}
\end{EntryWithPhonetic}

\begin{EntryWithPhonetic}{姑且}{gu1qie3}{8,5}{⼥,⼀}
  \definition{adv.}{provisoriamente; por enquanto; temporariamente; indica temporário}
\end{EntryWithPhonetic}

%%%%%%%%%% 孤 %%%%%%%%%%
\subsection*{孤}\addcontentsline{loh}{figure}{孤 \dpy{gu1}}

\begin{EntryWithPhonetic}{孤}{gu1}{8}{⼦}
  \definition*{s.}{Sobrenome: Gu}
  \definition{adj.}{sozinho; solitário; isolado}
  \definition{pron.}{eu; meu humilde eu (usado por príncipes feudais); título autoproclamado dos príncipes feudais}
  \definition[个,名,位]{s.}{órfão}
\end{EntryWithPhonetic}

\begin{EntryWithPhonetic}{孤单}{gu1dan1}{8,8}{⼦,⼗}[HSK 7-9]
  \definition{adj.}{sozinho; solitário | fraco; inadequado; descreve um pequeno número de pessoas e poder fraco}
\end{EntryWithPhonetic}

\begin{EntryWithPhonetic}{孤独}{gu1du2}{8,9}{⼦,⽝}[HSK 6]
  \definition{adj.}{sozinho; solitário}
\end{EntryWithPhonetic}

\begin{EntryWithPhonetic}{孤儿}{gu1'er2}{8,2}{⼦,⼉}[HSK 6]
  \definition[个,名,位]{s.}{órfão; criança sem pais; crianças que perderam os pais}
\end{EntryWithPhonetic}

\begin{EntryWithPhonetic}{孤立}{gu1li4}{8,5}{⼦,⽴}[HSK 7-9]
  \definition{adj.}{isolado; condenado ao ostracismo; descreve a falta de ajuda e simpatia}
  \definition{v.}{isolar; ostracizar; privar uma pessoa de ajuda, apoio e confiança}
\end{EntryWithPhonetic}

\begin{EntryWithPhonetic}{孤零零}{gu1ling2ling2}{8,13,13}{⼦,⾬,⾬}[HSK 7-9]
  \definition{adj.}{solitário; sozinho; completamente sozinho; sem apoio ou companhia}
\end{EntryWithPhonetic}

\begin{EntryWithPhonetic}{孤陋寡闻}{gu1lou4-gua3wen2}{8,8,14,9}{⼦,⾩,⼧,⾨}[HSK 7-9]
  \definition{expr.}{ignorante e mal informado | ignorante e inexperiente | mal informado e tacanho}
\end{EntryWithPhonetic}

%%%%%%%%%% 沽 %%%%%%%%%%
\subsection*{沽}\addcontentsline{loh}{figure}{沽 \dpy{gu1}}

\begin{EntryWithPhonetic}{沽}{gu1}{8}{⽔}
  \definition*{s.}{Município de Tianjin; outro nome para Tianjin}
  \definition{v.}{comprar | vender}
\end{EntryWithPhonetic}

\begin{EntryWithPhonetic}{沽名钓誉}{gu1ming2-diao4yu4}{8,6,8,13}{⽔,⼝,⾦,⾔}[HSK 7-9]
  \definition{expr.}{``Buscando fama e reputação.''; pescar fama e elogios; tentar alcançar a fama}
\end{EntryWithPhonetic}

%%%%%%%%%% 辜 %%%%%%%%%%
\subsection*{辜}\addcontentsline{loh}{figure}{辜 \dpy{gu1}}

\begin{EntryWithPhonetic}{辜}{gu1}{12}{⾟}
  \definition*{s.}{Sobrenome: Gu}
  \definition{s.}{culpa; crime}
\end{EntryWithPhonetic}

\begin{EntryWithPhonetic}{辜负}{gu1fu4}{12,6}{⾟,⾙}[HSK 7-9]
  \definition{v.}{desapontar; decepcionar; ser indigno de; não corresponder a}
\end{EntryWithPhonetic}

%%%%%%%%%% 古 %%%%%%%%%%
\subsection*{古}\addcontentsline{loh}{figure}{古 \dpy{gu3}}

\begin{EntryWithPhonetic}{古}{gu3}{5}{⼝}[HSK 3]
  \definition*{s.}{Cuba, abreviação de 古巴 | Sobrenome: Gu}
  \definition{adj.}{antigo; milenar; ancestral; secular | simples e sincero | velho | arcaico}
  \definition{pref.}{(distante no tempo; antigo; primitivo) paleo-; arqueo-}
  \definition{s.}{tempos antigos | antiguidade; ancestralidade | livros ou ortodoxias dos sábios antigos, a tradição do Tao | uma forma de poesia pré-Tang}
  \seealsoref{古巴}{gu3ba1}
  \antonymref{今}{jin1}
\end{EntryWithPhonetic}

\begin{EntryWithPhonetic}{古巴}{gu3ba1}{5,4}{⼝,⼰}
  \definition*{s.}{Cuba}
\end{EntryWithPhonetic}

\begin{EntryWithPhonetic}{古板}{gu3ban3}{5,8}{⼝,⽊}
  \definition{adj.}{antiquado e inflexível | reto; intolerante; conservador; lento | (pensamentos e estilo de trabalho) teimoso e conservador; rígido e avesso a mudanças}
  \synonymref{痴呆}{chi1dai1}
  \synonymref{传统}{chuan2tong3}
  \synonymref{固执}{gu4zhi5}
  \synonymref{严肃}{yan2su4}
  \antonymref{风尚}{feng1shang4}
  \antonymref{活泼}{huo2po5}
  \antonymref{机灵}{ji1ling5}
  \antonymref{开通}{kai1tong5}
\end{EntryWithPhonetic}

\begin{EntryWithPhonetic}{古城}{gu3cheng2}{5,9}{⼝,⼟}
  \definition{s.}{cidade antiga}
\end{EntryWithPhonetic}

\begin{EntryWithPhonetic}{古代}{gu3dai4}{5,5}{⼝,⼈}[HSK 3]
  \definition{s.}{tempos antigos; o passado é um período muito distante do presente (diferentemente de 近代 e 现代); na periodização histórica chinesa, geralmente se refere ao período anterior a meados do século XIX | sociedade antiga; sociedade primitiva; refere"-se especificamente à era da sociedade escravista (às vezes também inclui a era comunal primitiva) |antigamente; tempos antigos; no passado}
  \seealsoref{近代}{jin4dai4}
  \seealsoref{现代}{xian4dai4}
\end{EntryWithPhonetic}

\begin{EntryWithPhonetic}{古典}{gu3dian3}{5,8}{⼝,⼋}[HSK 6]
  \definition{adj.}{clássico; descreve uma obra ou coisa como tendo características tradicionais ou exemplares}
  \definition{s.}{os clássicos}
\end{EntryWithPhonetic}

\begin{EntryWithPhonetic}{古董}{gu3dong3}{5,12}{⼝,⾋}[HSK 7-9]
  \definition{adj.}{antiquado; uma metáfora para coisas ultrapassadas ou pessoas teimosas}
  \definition[件,个]{s.}{objeto de arte; raridade; antiguidade; artefatos transmitidos desde os tempos antigos podem ser usados como referência para a compreensão da cultura antiga}
\end{EntryWithPhonetic}

\begin{EntryWithPhonetic}{古怪}{gu3guai4}{5,8}{⼝,⼼}[HSK 7-9]
  \definition{adj.}{pitoresco; excêntrico; esquisito; estranho; muito diferente do habitual, surpreendente; desconhecido e raro; raro e inovador}
\end{EntryWithPhonetic}

\begin{EntryWithPhonetic}{古迹}{gu3ji4}{5,9}{⼝,⾡}[HSK 7-9]
  \definition[处]{s.}{sítio histórico; local de interesse histórico; construções antigas ou outras relíquias de grande importância}
\end{EntryWithPhonetic}

\begin{EntryWithPhonetic}{古今中外}{gu3jin1-zhong1wai4}{5,4,4,5}{⼝,⼈,⼁,⼣}[HSK 7-9]
  \definition{expr.}{``Antigo e moderno, chinês e estrangeiro.''; em todos os tempos e em todas as terras}
\end{EntryWithPhonetic}

\begin{EntryWithPhonetic}{古柯碱}{gu3ke1jian3}{5,9,14}{⼝,⽊,⽯}
  \definition{s.}{cocaína}
\end{EntryWithPhonetic}

\begin{EntryWithPhonetic}{古老}{gu3lao3}{5,6}{⼝,⽼}[HSK 5]
  \definition{adj.}{antigo; antiquado; histórico}
\end{EntryWithPhonetic}

\begin{EntryWithPhonetic}{古朴}{gu3pu3}{5,6}{⼝,⽊}[HSK 7-9]
  \definition{adj.}{simples e pouco sofisticado (arte, arquitetura, etc.); descreve a aparência sem muita decoração ou modificação, dando às pessoas uma sensação antiga, e também descreve o comportamento das pessoas como simples e sincero}
\end{EntryWithPhonetic}

\begin{EntryWithPhonetic}{古人}{gu3ren2}{5,2}{⼝,⼈}[HSK 7-9]
  \definition{s.}{os antigos; antepassados | pessoas dos tempos antigos | espécies humanas extintas, como \emph{Homo erectus} ou \emph{Homo neanderthalensis} | Lliterário: pessoa falecida}
  \antonymref{今人}{jin1ren2}
\end{EntryWithPhonetic}

\begin{EntryWithPhonetic}{古铜色}{gu3tong2 se4}{5,11,6}{⼝,⾦,⾊}
  \definition{s.}{cor bronze}
\end{EntryWithPhonetic}

\begin{EntryWithPhonetic}{古装}{gu3 zhuang1}{5,12}{⼝,⾐}
  \definition[套]{s.}{traje antigo; roupas tradicionais; roupas de estilo antigo}
\end{EntryWithPhonetic}

%%%%%%%%%% 谷 %%%%%%%%%%
\subsection*{谷}\addcontentsline{loh}{figure}{谷 \dpy{gu3}}

\begin{EntryWithPhonetic}{谷}{gu3}{7}{⾕}[Kangxi 150]
  \definition*{s.}{Sobrenome: Gu}
  \definition{adj.}{bom; gentil}
  \definition{s.}{vale; ravina; desfiladeiro; garganta; faixa estreita de terra com uma saída no meio de duas colinas ou dois platôs | arroz não descascado | salário de funcionário (na época feudal) | calha; cocho; canal | fossa sob o cerebelo (anatomia); valécula | dificuldade; dilema}
  \definition{v.}{criar (filhos) | crescer}
\end{EntryWithPhonetic}

%%%%%%%%%% 股 %%%%%%%%%%
\subsection*{股}\addcontentsline{loh}{figure}{股 \dpy{gu3}}

\begin{EntryWithPhonetic}{股}{gu3}{8}{⾁}[HSK 6]
  \definition*{s.}{Sobrenome: Gu}
  \definition{clas.}{usado para coisas em tiras, longas e estreitas | usado para gás, odor, força, etc. | Pejorativo: usado para um grupo de pessoas}
  \definition{s.}{coxa; ancas | seção (de um escritório, empresa, etc.); unidades organizacionais em agências governamentais, empresas e grupos | fio; camada | uma das várias partes iguais de propriedade | ação; \emph{stock}; ação do capital social; uma parte igual de fundos ou propriedade | a perna mais longa de um triângulo retângulo}
\end{EntryWithPhonetic}

\begin{EntryWithPhonetic}{股东}{gu3dong1}{8,5}{⾁,⼀}[HSK 6]
  \definition[个,位,名,家]{s.}{acionista de uma sociedade anônima com direito a participar e votar nas assembleias gerais; refere"-se também a investidores em outras empresas industriais e comerciais administradas por sociedades}
\end{EntryWithPhonetic}

\begin{EntryWithPhonetic}{股份}{gu3fen4}{8,6}{⾁,⼈}[HSK 7-9]
  \definition{s.}{ação; unidade de distribuição de capital de uma sociedade anônima ou de uma empresa cooperativa, com uma parcela igual do capital total}
\end{EntryWithPhonetic}

\begin{EntryWithPhonetic}{股民}{gu3min2}{8,5}{⾁,⽒}[HSK 7-9]
  \definition{s.}{pessoa que compra e vende ações; acionista | corretor de ações | investidor em ações}
\end{EntryWithPhonetic}

\begin{EntryWithPhonetic}{股票}{gu3piao4}{8,11}{⾁,⽰}[HSK 6]
  \definition[只,股]{s.}{ação; quotas; certificado de ações; título de capital; capital social; títulos utilizados para representar ações}
\end{EntryWithPhonetic}

\begin{EntryWithPhonetic}{股市}{gu3shi4}{8,5}{⾁,⼱}[HSK 7-9]
  \definition{s.}{mercado de ações; mercado de compra e venda de ações | cotações na bolsa de valores}
\end{EntryWithPhonetic}

%%%%%%%%%% 骨 %%%%%%%%%%
\subsection*{骨}\addcontentsline{loh}{figure}{骨 \dpy{gu3}}

\begin{EntryWithPhonetic}{骨}{gu3}{9}{⾻}[Kangxi 188]
  \definition*{s.}{Sobrenome: Gu}
  \definition[根,块]{s.}{osso | esqueleto; estrutura | caráter; espírito | cadáver; corpo}
\end{EntryWithPhonetic}

\begin{EntryWithPhonetic}{骨干}{gu3gan4}{9,3}{⾻,⼲}[HSK 7-9]
  \definition[名,个,位]{s.}{diáfise; a parte central de um osso longo, conectada à epífise em ambas as extremidades, contém uma cavidade | espinha dorsal; esteio; metaforicamente falando, uma pessoa ou coisa que desempenha um papel importante}
\end{EntryWithPhonetic}

\begin{EntryWithPhonetic}{骨气}{gu3qi4}{9,4}{⾻,⽓}[HSK 7-9]
  \definition[些,种]{s.}{espinha dorsal; integridade moral; força de caráter | vigor dos traços caligráficos; refere"-se ao impulso forte e vertical expresso pela caligrafia}
\end{EntryWithPhonetic}

\begin{EntryWithPhonetic}{骨头}{gu3tou5}{9,5}{⾻,⼤}[HSK 4]
  \definition[根,块]{s.}{osso; tecidos mais duros no corpo de uma pessoa ou de alguns animais que sustentam o corpo ou protegem os órgãos do corpo | caráter de uma pessoa; refere"-se à qualidade do caráter de uma pessoa}
\end{EntryWithPhonetic}

\begin{EntryWithPhonetic}{骨折}{gu3zhe2}{9,7}{⾻,⼿}[HSK 7-9]
  \definition{v.}{sofrer uma fratura; quebrar (um osso)}
\end{EntryWithPhonetic}

%%%%%%%%%% 鼓 %%%%%%%%%%
\subsection*{鼓}\addcontentsline{loh}{figure}{鼓 \dpy{gu3}}

\begin{EntryWithPhonetic}{鼓}{gu3}{13}{⿎}[HSK 5][Kangxi 207]
  \definition*{s.}{Sobrenome: Gu}
  \definition{adj.}{abaulado; inchado; saliente; protuberante}
  \definition{clas.}{unidades antigas de cronometragem noturna; vigílias da noite}
  \definition[个,架,面,张]{s.}{tambor; instrumento de percussão | coisas semelhantes a tambores; formato, som e função semelhantes aos de um tambor}
  \definition{v.}{soar; bater; golpear; fazer um objeto soar | ventilar; soprar com fole | agitar; despertar; ativar; incitar; revigorar | bater asas | aumentar; fazer beicinho}
\end{EntryWithPhonetic}

\begin{EntryWithPhonetic}{鼓吹}{gu3chui1}{13,7}{⿎,⼝}
  \definition{v.}{defender (um réu) | Pejorativo: pregar; anunciar; exagerar gabar-se}
\end{EntryWithPhonetic}

\begin{EntryWithPhonetic}{鼓动}{gu3dong4}{13,6}{⿎,⼒}[HSK 7-9]
  \definition{v.}{promover; ativar; agitar; despertar; inspirar as pessoas a agir | instigar; incitar}
\end{EntryWithPhonetic}

\begin{EntryWithPhonetic}{鼓励}{gu3li4}{13,7}{⿎,⼒}[HSK 5]
  \definition{v.}{incitar; encorajar; provocar e incentivar}
\end{EntryWithPhonetic}

\begin{EntryWithPhonetic}{鼓舞}{gu3wu3}{13,14}{⿎,⾇}[HSK 7-9]
  \definition{adj.}{animado; elevado; inspirado; encorajado; motivado}
  \definition{v.}{animar; elevar; inspirar; encorajar; motivar}
\end{EntryWithPhonetic}

\begin{EntryWithPhonetic}{鼓掌}{gu3/zhang3}{13,12}{⿎,⼿}[HSK 5]
  \definition{v.+compl.}{aplaudir; bater palmas, principalmente para expressar felicidade, aprovação ou boas-vindas}
\end{EntryWithPhonetic}

%%%%%%%%%% 估 %%%%%%%%%%
\subsection*{估}\addcontentsline{loh}{figure}{估 \dpy{gu4}}

\begin{EntryWithPhonetic}{估}{gu4}{7}{⼈}
  \definition{adj.}{velho | roupas de segunda mão}
  \seeref{gu1}
\end{EntryWithPhonetic}

%%%%%%%%%% 固 %%%%%%%%%%
\subsection*{固}\addcontentsline{loh}{figure}{固 \dpy{gu4}}

\begin{EntryWithPhonetic}{固}{gu4}{8}{⼞}
  \definition*{s.}{Sobrenome: Gu}
  \definition{adj.}{sólido; firme; forte | duro; sólido | mal informado; superficial; ignorante}
  \definition{adv.}{firmemente; resolutamente | originalmente; em primeiro lugar | certamente; reconhecidamente; seguramente}
  \definition{conj.}{usado da mesma forma que 固然}
  \definition{v.}{solidificar; consolidar; fortalecer | defender; proteger}
  \seealsoref{固然}{gu4ran2}
\end{EntryWithPhonetic}

\begin{EntryWithPhonetic}{固定}{gu4ding4}{8,8}{⼞,⼧}[HSK 4]
  \definition{adj.}{fixo; regular; inalterado ou imóvel}
  \definition{v.}{consertar; tornar fixo, não mover novamente; colocar as coisas em ordem, não mudá-las novamente}
\end{EntryWithPhonetic}

\begin{EntryWithPhonetic}{固然}{gu4ran2}{8,12}{⼞,⽕}[HSK 7-9]
  \definition{conj.}{usado para introduzir uma cláusula adversativa admitindo primeiro um certo fato; quando usado na primeira metade de uma frase, a segunda metade geralmente tem 可是 ou 但是 para ecoá-lo, indicando que o fato A é reconhecido, mas o fato B não se torna inválido por causa do fato A | admitir um fato sem negar outro; indica o reconhecimento de um fato, levando a uma transição no texto seguinte; indica o reconhecimento do fato A e não nega o fato B}
  \seealsoref{但是}{dan4shi4}
  \seealsoref{可是}{ke3shi4}
\end{EntryWithPhonetic}

\begin{EntryWithPhonetic}{固执}{gu4zhi5}{8,6}{⼞,⼿}[HSK 7-9]
  \definition{adj.}{obstinado; teimoso; mantém suas próprias opiniões e não quer mudá-las, mesmo que estejam erradas}
\end{EntryWithPhonetic}

%%%%%%%%%% 故 %%%%%%%%%%
\subsection*{故}\addcontentsline{loh}{figure}{故 \dpy{gu4}}

\begin{EntryWithPhonetic}{故}{gu4}{9}{⽁}[HSK 7-9]
  \definition*{s.}{Sobrenome: Gu}
  \definition{adj.}{velho; antigo; original}
  \definition{adv.}{propositalmente; intencionalmente; deliberadamente}
  \definition{conj.}{assim; portanto; consequentemente; pelo contrário}
  \definition{s.}{evento; incidente; acontecimento; acidente | causa; razão | amigo; conhecido | o velho; refere"-se a coisas antigas e passadas}
  \definition{v.}{morrer}
\end{EntryWithPhonetic}

\begin{EntryWithPhonetic}{故宫}{gu4gong1}{9,9}{⽁,⼧}
  \definition*{s.}{O Palácio Imperial; O Museu do Palácio (em Pequim); A Cidade Proibida}
\end{EntryWithPhonetic}

\begin{EntryWithPhonetic}{故事}{gu4shi5}{9,8}{⽁,⼅}[HSK 2]
  \definition[个,段,篇,则]{s.}{história; conto; coisas reais ou fictícias usadas como objeto de narrativa, com coerência, atraentes e capazes de emocionar as pessoas | enredo; trama; enredo que consegue mostrar a personalidade dos personagens e refletir a ideia central da obra literária}
\end{EntryWithPhonetic}

\begin{EntryWithPhonetic}{故乡}{gu4xiang1}{9,3}{⽁,⼄}[HSK 3]
  \definition[个]{s.}{cidade natal; terra natal; local de nascimento ou onde viveu por muito tempo}
\end{EntryWithPhonetic}

\begin{EntryWithPhonetic}{故意}{gu4yi4}{9,13}{⽁,⼼}[HSK 2]
  \definition{adv.}{deliberadamente; intencionalmente; não é por descuido, mas sim conscientemente (geralmente coisas que não se devem fazer ou que não são necessárias)}
  \definition{s.}{intenção; um tipo de mentalidade, uma pessoa sabe claramente que seus atos podem causar danos a outras pessoas ou trazer consequências negativas para a sociedade, mas mesmo assim não faz nada para impedir isso}
\end{EntryWithPhonetic}

\begin{EntryWithPhonetic}{故障}{gu4zhang4}{9,13}{⽁,⾩}[HSK 6]
  \definition[出]{s.}{problema; falha; parada; mau funcionamento; avaria; situações em que máquinas, instrumentos, etc. não podem funcionar normalmente devido a problemas}
\end{EntryWithPhonetic}

%%%%%%%%%% 顾 %%%%%%%%%%
\subsection*{顾}\addcontentsline{loh}{figure}{顾 \dpy{gu4}}

\begin{EntryWithPhonetic}{顾}{gu4}{10}{⾴}[HSK 6]
  \definition*{s.}{Sobrenome: Gu}
  \definition{adv.}{em vez disso; pelo contrário; indica o oposto, equivalente a 却 ou 反而}
  \definition{conj.}{mas; no entanto}
  \definition{v.}{olhar para trás; olhar para; virar-se e olhar para | cuidar de; atender a; levar em conta ou consideração | visitar; chamar | sentir pena de}
  \seealsoref{反而}{fan3'er2}
  \seealsoref{却}{que4}
\end{EntryWithPhonetic}

\begin{EntryWithPhonetic}{顾不得}{gu4bu5de5}{10,4,11}{⾴,⼀,⼻}[HSK 7-9]
  \definition{v.}{incapaz de mudar algo | incapaz de lidar com}
\end{EntryWithPhonetic}

\begin{EntryWithPhonetic}{顾不上}{gu4bu5shang4}{10,4,3}{⾴,⼀,⼀}[HSK 7-9]
  \definition{v.}{não conseguir; não conseguir atender; incapaz de cuidar de (fazer algo)}
\end{EntryWithPhonetic}

\begin{EntryWithPhonetic}{顾及}{gu4ji2}{10,3}{⾴,⼃}[HSK 7-9]
  \definition{v.}{atender a; levar em conta; dar consideração a; cuidar de; notar}
\end{EntryWithPhonetic}

\begin{EntryWithPhonetic}{顾客}{gu4ke4}{10,9}{⾴,⼧}[HSK 2]
  \definition[个,位,名,些]{s.}{cliente; comprador; consumidor; paciente}
\end{EntryWithPhonetic}

\begin{EntryWithPhonetic}{顾虑}{gu4lv4}{10,10}{⾴,⾌}[HSK 7-9]
  \definition[丝,点]{s.}{preocupação; escrúpulo; receio; apreensão}
  \definition{v.}{estar apreensivo (sobre as consequências da própria ação)}
\end{EntryWithPhonetic}

\begin{EntryWithPhonetic}{顾全大局}{gu4quan2-da4ju2}{10,6,3,7}{⾴,⼊,⼤,⼫}[HSK 7-9]
  \definition{expr.}{``Considere a situação geral.''; levar em conta os interesses do todo; considerar a situação como um todo; levar em consideração o panorama geral; trabalhar para o benefício de todos}
\end{EntryWithPhonetic}

\begin{EntryWithPhonetic}{顾问}{gu4wen4}{10,6}{⾴,⾨}[HSK 5]
  \definition[个,位,名]{s.}{conselheiro; consultor; assessor; pessoas com conhecimento especializado ou experiência contratadas para prestar consultoria a organizações ou indivíduos}
\end{EntryWithPhonetic}

%%%%%%%%%% 雇 %%%%%%%%%%
\subsection*{雇}\addcontentsline{loh}{figure}{雇 \dpy{gu4}}

\begin{EntryWithPhonetic}{雇}{gu4}{12}{⾫}[HSK 7-9]
  \definition{v.}{contratar; empregar; pagar pessoas para fazerem coisas por você | contratar (transporte de aluguel)}
\end{EntryWithPhonetic}

\begin{EntryWithPhonetic}{雇佣}{gu4yong1}{12,7}{⾫,⼈}[HSK 7-9]
  \definition{v.}{contratar; empregar; comprar mão de obra com dinheiro}
\end{EntryWithPhonetic}

\begin{EntryWithPhonetic}{雇员}{gu4yuan2}{12,7}{⾫,⼝}[HSK 7-9]
  \definition[名,位,个]{s.}{empregado; servo; pessoal contratado ou temporário fora do estabelecimento}
\end{EntryWithPhonetic}

\begin{EntryWithPhonetic}{雇主}{gu4zhu3}{12,5}{⾫,⼂}[HSK 7-9]
  \definition[名]{s.}{empregador; uma pessoa que contrata trabalhadores, veículos ou barcos}
\end{EntryWithPhonetic}

%%%%%%%%%% 瓜 %%%%%%%%%%
\subsection*{瓜}\addcontentsline{loh}{figure}{瓜 \dpy{gua1}}

\begin{EntryWithPhonetic}{瓜}{gua1}{5}{⽠}[HSK 4][Kangxi 97]
  \definition*{s.}{Sobrenome: Gua}
  \definition[个]{s.}{qualquer tipo de melão ou cabaça | companheiro (termo depreciativo para uma pessoa)}
  \definition{v.}{fofocar}
\end{EntryWithPhonetic}

\begin{EntryWithPhonetic}{瓜分}{gua1fen1}{5,4}{⽠,⼑}[HSK 7-9]
  \definition{v.}{cortar um melão -- cortar; desmembrar; dividir; particionar}
\end{EntryWithPhonetic}

\begin{EntryWithPhonetic}{瓜子}{gua1zi3}{5,3}{⽠,⼦}[HSK 7-9]
  \definition[个,把,颗,粒,些]{s.}{sementes de melão; sementes de girassol; sementes de abóbora}
\end{EntryWithPhonetic}

%%%%%%%%%% 刮 %%%%%%%%%%
\subsection*{刮}\addcontentsline{loh}{figure}{刮 \dpy{gua1}}

\begin{EntryWithPhonetic}{刮}{gua1}{8}{⼑}[HSK 6]
  \definition{v.}{barbear; raspar; depilar | untar com (pasta, etc.)  | extorquir; pilhar; adquirir avidamente (propriedade) por vários meios | (do vento) soprar}
\end{EntryWithPhonetic}

\begin{EntryWithPhonetic}{刮风}{gua1/feng1}{8,4}{⼑,⾵}[HSK 7-9]
  \definition{v.+compl.}{ventar; fazer vento; soprar (vento)}
\end{EntryWithPhonetic}

%%%%%%%%%% 寡 %%%%%%%%%%
\subsection*{寡}\addcontentsline{loh}{figure}{寡 \dpy{gua3}}

\begin{EntryWithPhonetic}{寡}{gua3}{14}{⼧}
  \definition{adj.}{poucos; escassos | insípido; sem sabor | pouco; escasso | insípido; sem graça}
  \definition{pron.}{eu; título autoproclamado de um antigo monarca}
  \definition{s.}{viúva | viuvez; a natureza ou estado de uma mulher viúva que vive sozinha}
  \antonymref{多}{duo1}
  \antonymref{众}{zhong4}
\end{EntryWithPhonetic}

\begin{EntryWithPhonetic}{寡妇}{gua3fu5}{14,6}{⼧,⼥}[HSK 7-9]
  \definition[个]{s.}{viúva; uma mulher cujo marido morreu}
\end{EntryWithPhonetic}

%%%%%%%%%% 挂 %%%%%%%%%%
\subsection*{挂}\addcontentsline{loh}{figure}{挂 \dpy{gua4}}

\begin{EntryWithPhonetic}{挂}{gua4}{9}{⼿}[HSK 3]
  \definition{clas.}{usado principalmente para coisas que vêm em conjuntos ou séries}
  \definition{v.}{pendurar; colocar; suspender; usando cordas, ganchos, pregos e outros itens para prender objetos em um ou mais pontos específicos | interromper chamada (telefônica) | colocar alguém em contato com; ligar; telefonar; refere"-se a ligar o telefone, bem como a fazer uma chamada | falhar; fracassar | colocar em registro; registrar | pegar carona; ser pego | preocupar"-se com | ser revestido com; ser coberto com | estar pendente; deixar algo sem solução}
\end{EntryWithPhonetic}

\begin{EntryWithPhonetic}{挂钩}{gua4gou1}{9,9}{⼿,⾦}[HSK 7-9]
  \definition[个,种]{s.}{(vagões ferroviários) acoplamento; manilha; engate | gancho}
  \definition{v.}{acoplar (dois vagões ferroviários); articular | conectar-se com; estabelecer contato com; entrar em contato com; vincular-se a}
\end{EntryWithPhonetic}

\begin{EntryWithPhonetic}{挂号}{gua4/hao4}{9,5}{⼿,⼝}[HSK 7-9]
  \definition{v.+compl.}{registrar-se (em um hospital, etc.) | enviar através de carta registrada}
\end{EntryWithPhonetic}

\begin{EntryWithPhonetic}{挂号信}{gua4hao4xin4}{9,5,9}{⼿,⼝,⼈}
  \definition{s.}{carta registrada}
\end{EntryWithPhonetic}

\begin{EntryWithPhonetic}{挂念}{gua4nian4}{9,8}{⼿,⼼}[HSK 7-9]
  \definition{v.}{sentir falta; preocupar-se com alguém que está ausente}
\end{EntryWithPhonetic}

\begin{EntryWithPhonetic}{挂失}{gua4/shi1}{9,5}{⼿,⼤}[HSK 7-9]
  \definition{v.+compl.}{relatar a perda de algo; se perder uma nota ou certificado, você deve registrá-lo junto à autoridade emissora ou declará-lo inválido}
\end{EntryWithPhonetic}

%%%%%%%%%% 乖 %%%%%%%%%%
\subsection*{乖}\addcontentsline{loh}{figure}{乖 \dpy{guai1}}

\begin{EntryWithPhonetic}{乖}{guai1}{8}{⼃}[HSK 7-9]
  \definition{adj.}{(uma criança) bem comportado; bom; obediente | inteligente; astuto; esperto | (caráter, comportamento, etc.) estranho; anormal; irracional}
  \definition{v.}{perverter; ser contrário à razão; ir contra | (caráter, comportamento, etc.) ser anormal; ser estranho}
\end{EntryWithPhonetic}

\begin{EntryWithPhonetic}{乖乖}{guai1guai1}{8,8}{⼃,⼃}
  \definition{adj.}{bem-comportado; obediente}
  \definition{s.}{bebezinho; pequenino; querido; docinho (usado apenas para crianças)}
  \seeref{guai1guai5}
\end{EntryWithPhonetic}

\begin{EntryWithPhonetic}{乖乖}{guai1guai5}{8,8}{⼃,⼃}
  \definition{expr.}{Uau!; Nossa!; Meu Deus!; Oh meu Deus!}
  \seeref{guai1guai1}
\end{EntryWithPhonetic}

\begin{EntryWithPhonetic}{乖巧}{guai1qiao3}{8,5}{⼃,⼯}[HSK 7-9]
  \definition{adj.}{fofo; adorável; agradável; descreve crianças, pequenos animais, etc. como sendo obedientes, fofos e simpáticos | inteligente; engenhoso; descreve uma pessoa que sempre fala ou faz coisas de acordo com os desejos de outras pessoas e é querida por elas}
\end{EntryWithPhonetic}

\begin{EntryWithPhonetic}{乖张}{guai1zhang1}{8,7}{⼃,⼸}
  \definition{adj.}{excêntrico e irracional; perverso; recalcitrante | não suave; sem sucesso | irritadiço | irrazoável}
  \synonymref{荒诞}{huang1dan4}
  \synonymref{荒谬}{huang1miu4}
  \synonymref{荒唐}{huang1tang2}
  \antonymref{随和}{sui2he5}
  \antonymref{温和}{wen1he2}
  \antonymref{温柔}{wen1rou2}
  \antonymref{温顺}{wen1shun4}
\end{EntryWithPhonetic}

%%%%%%%%%% 拐 %%%%%%%%%%
\subsection*{拐}\addcontentsline{loh}{figure}{拐 \dpy{guai3}}

\begin{EntryWithPhonetic}{拐}{guai3}{8}{⼿}[HSK 6]
  \definition[支,根,副]{s.}{muleta; bengala; uma bengala com uma barra horizontal na parte superior, usada por pessoas com doenças ou deficiências nos membros inferiores para ajudá-las a caminhar |  sete; forma falada do numeral 七 | esquina; curva; canto}
  \definition{v.}{virar; girar; mudar de direção enquanto se move | enganar | mudar; transformar | mancar}
  \seealsoref{七}{qi1}
\end{EntryWithPhonetic}

\begin{EntryWithPhonetic}{拐弯}{guai3/wan1}{8,9}{⼿,⼸}[HSK 7-9]
  \definition[个]{s.}{esquina; curva; canto}
  \definition{v.}{virar; virar uma esquina; indica mudança de direção da viagem | dar meia-volta; seguir um novo curso; indica mudança de ideias, linguagem, etc.}
\end{EntryWithPhonetic}

\begin{EntryWithPhonetic}{拐杖}{guai3zhang4}{8,7}{⼿,⽊}[HSK 7-9]
  \definition[个,根,支,副]{s.}{muleta; bengala}
\end{EntryWithPhonetic}

%%%%%%%%%% 怪 %%%%%%%%%%
\subsection*{怪}\addcontentsline{loh}{figure}{怪 \dpy{guai4}}

\begin{EntryWithPhonetic}{怪}{guai4}{8}{⼼}[HSK 4,5]
  \definition*{s.}{Sobrenome: Guai}
  \definition{adj.}{estranho; esquisito; desconcertante | peculiar; excêntrico; pitoresco; monstruoso; anormal; incomum}
  \definition{adv.}{bastante; muito}
  \definition{s.}{monstro; demônio | diabo; ser maligno}
  \definition{v.}{culpar | achar algo estranho; maravilhar-se com; ficar surpreso | repreender; culpar; reclamar}
\end{EntryWithPhonetic}

\begin{EntryWithPhonetic}{怪不得}{guai4bu5de5}{8,4,11}{⼼,⼀,⼻}[HSK 7-9]
  \definition{adv.}{não é de admirar; então é por isso; isso explica por que; isso significa que você entende o motivo e não acha mais uma situação estranha}
  \definition{v.}{não culpar; não acusar; não poder culpar, não se ofender}[你做错了,怪不得别人。===Você cometeu um erro, então não culpe os outros.]
\end{EntryWithPhonetic}

\begin{EntryWithPhonetic}{怪癖}{guai4pi3}{8,18}{⼼,⽧}
  \definition{adj.}{peculiar}
  \definition{s.}{excentricidade | peculiaridade | hobby estranho}
\end{EntryWithPhonetic}

\begin{EntryWithPhonetic}{怪兽}{guai4shou4}{8,11}{⼼,⼋}
  \definition{s.}{animal raro | animal mítico | monstro}
\end{EntryWithPhonetic}

\begin{EntryWithPhonetic}{怪物}{guai4wu5}{8,8}{⼼,⽜}[HSK 7-9]
  \definition{s.}{monstro; aberração; coisas imaginárias que parecem estranhas, mas têm habilidades especiais | pessoa excêntrica; pássaro estranho; uma pessoa com temperamento excêntrico}
\end{EntryWithPhonetic}

\begin{EntryWithPhonetic}{怪异}{guai4yi4}{8,6}{⼼,⼶}[HSK 7-9]
  \definition{adj.}{monstruoso; estranho; incomum}
  \definition{s.}{fenômeno estranho; presságio; prodígio | monstruosidade}
\end{EntryWithPhonetic}

%%%%%%%%%% 关 %%%%%%%%%%
\subsection*{关}\addcontentsline{loh}{figure}{关 \dpy{guan1}}

\begin{EntryWithPhonetic}{关}{guan1}{6}{⼋}[HSK 1,4]
  \definition*{s.}{Sobrenome: Guan}
  \definition{s.}{passagem; ponto de controle | alfândega; escritórios de cobrança de impostos para exportação e importação de mercadorias | ponto de inflexão ou barreira; ponto de virada ou dificuldade | momento crítico; mecanismo}
  \definition{v.}{fechar; encerrar; amarrar algo | fechar; trancar | encerrar; sair do mercado; falir | conceder ou sacar o pagamento de um salário | desligar | envolver; preocupar-se; conectar-se}
\end{EntryWithPhonetic}

\begin{EntryWithPhonetic}{关爱}{guan1'ai4}{6,10}{⼋,⽖}[HSK 6]
  \definition{v.}{cuidar; cuidar e amar}
\end{EntryWithPhonetic}

\begin{EntryWithPhonetic}{关闭}{guan1bi4}{6,6}{⼋,⾨}[HSK 4]
  \definition{v.}{fechar | (empresa) falir}
\end{EntryWithPhonetic}

\begin{EntryWithPhonetic}{关掉}{guan1diao4}{6,11}{⼋,⼿}[HSK 7-9]
  \definition{v.}{desligar}
\end{EntryWithPhonetic}

\begin{EntryWithPhonetic}{关怀}{guan1huai2}{6,7}{⼋,⼼}[HSK 5]
  \definition{v.}{mostrar cuidado amoroso por; mostrar solicitude por; cuidar, amar, apoiar ou ajudar os fracos ou grupos em dificuldade | geralmente usado para superiores para subordinados, anciãos para juniores ou organizações para indivíduos}
\end{EntryWithPhonetic}

\begin{EntryWithPhonetic}{关机}{guan1 ji1}{6,6}{⼋,⽊}[HSK 2]
  \definition{v.}{encerrar; terminar; refere"-se especificamente à conclusão das filmagens de um filme ou série de TV | desligar; desligar a fonte de alimentação; parar o funcionamento da máquina}
\end{EntryWithPhonetic}

\begin{EntryWithPhonetic}{关键}{guan1jian4}{6,13}{⼋,⾦}[HSK 5]
  \definition{adj.}{crucial; decisivo; importante; que pode determinar o curso e o resultado dos eventos}
  \definition[个,点,些]{s.}{chave; ponto crucial; aspectos ou condições mais importantes que determinam o desenvolvimento e o resultado de algo}
\end{EntryWithPhonetic}

\begin{EntryWithPhonetic}{关节}{guan1jie2}{6,5}{⼋,⾋}[HSK 7-9]
  \definition{s.}{articulação; as partes onde os ossos se conectam e que possibilitam o movimento | suborno; relacionamentos que podem ajudar as pessoas a obter benefícios por meios impróprios | elo (ou ponto) chave (ou crucial)}
\end{EntryWithPhonetic}

\begin{EntryWithPhonetic}{关联}{guan1lian2}{6,12}{⼋,⽿}[HSK 6]
  \definition{s.}{conexão; inter-relação; a conexão entre as coisas}
  \definition{v.}{estar relacionado; estar conectado; as coisas estão envolvidas e influenciam umas às outras}
\end{EntryWithPhonetic}

\begin{EntryWithPhonetic}{关上}{guan1shang4}{6,3}{⼋,⼀}[HSK 1]
  \definition{v.}{fechar (uma porta); fechar um objeto | desligar (luz, equipamento elétrico etc.); parar ou encerrar (uma atividade, situação, etc.)}
\end{EntryWithPhonetic}

\begin{EntryWithPhonetic}{关税}{guan1shui4}{6,12}{⼋,⽲}[HSK 7-9]
  \definition{s.}{tarifa; taxa aduaneira; impostos cobrados pelo estado sobre mercadorias importadas e exportadas}
\end{EntryWithPhonetic}

\begin{EntryWithPhonetic}{关头}{guan1tou2}{6,5}{⼋,⼤}[HSK 7-9]
  \definition{s.}{conjuntura; momento; um momento decisivo ou ponto de virada}
\end{EntryWithPhonetic}

\begin{EntryWithPhonetic}{关系}{guan1xi5}{6,7}{⼋,⽷}[HSK 3]
  \definition[个,种]{s.}{relações; conexões; relacionamento; a interligação entre pessoas ou coisas | consequência; impacto; significado a influência ou importância de algo; algo digno de nota (geralmente usado com 没有, 有). | causa; razão (geralmente usado com 由于 ou 因为); refere"-se genericamente a causas, condições, etc. | credenciais que mostram filiação a uma organização; documento que comprova a existência de algum tipo de relação organizacional}
  \definition{v.}{preocupar; afetar; ter influência sobre; ter a ver com}
  \seealsoref{没有}{mei2you5}
  \seealsoref{因为}{yin1wei5}
  \seealsoref{由于}{you2yu2}
  \seealsoref{有}{you3}
\end{EntryWithPhonetic}

\begin{EntryWithPhonetic}{关心}{guan1xin1}{6,4}{⼋,⼼}[HSK 2]
  \definition{v.}{cuidar; preocupar-se com; manifestar interesse por; demonstrar solicitude por; (colocar uma pessoa ou coisa) sempre no coração; valorizar e cuidar}
\end{EntryWithPhonetic}

\begin{EntryWithPhonetic}{关于}{guan1yu2}{6,3}{⼋,⼆}[HSK 4]
  \definition{prep.}{sobre; relativo a; pertencente a; uma questão de; com relação a}
\end{EntryWithPhonetic}

\begin{EntryWithPhonetic}{关张}{guan1zhang1}{6,7}{⼋,⼸}
  \definition{v.}{Dialeto: (uma loja) fechar as portas; falir}
\end{EntryWithPhonetic}

\begin{EntryWithPhonetic}{关照}{guan1zhao4}{6,13}{⼋,⽕}[HSK 7-9]
  \definition{v.}{cuidar de; ficar de olho em; preocupar-se e cuidar de alguém e tomar a iniciativa de ajudar quando perceber que essa pessoa está com problemas | contar; notificar de boca em boca; notificação verbal para que as pessoas saibam ou se lembrem de algo}
\end{EntryWithPhonetic}

\begin{EntryWithPhonetic}{关注}{guan1zhu4}{6,8}{⼋,⽔}[HSK 3]
  \definition{v.}{prestar atenção em; seguir algo de perto; seguir (nas redes sociais)}
\end{EntryWithPhonetic}

%%%%%%%%%% 观 %%%%%%%%%%
\subsection*{观}\addcontentsline{loh}{figure}{观 \dpy{guan1}}

\begin{EntryWithPhonetic}{观}{guan1}{6}{⾒}
  \definition*{s.}{Templo taoísta; ``Koon''}
  \definition{s.}{visão; vista | perspectiva; visão; conceito | aparência; perspectiva | alcance de visão | noção; ideia; conhecimento ou visão das coisas | ponto de vista; postura; uma visão de uma coisa}
  \definition{v.}{olhar para; assistir; observar | contemplar}
  \seeref{guan4}
\end{EntryWithPhonetic}

\begin{EntryWithPhonetic}{观测}{guan1ce4}{6,9}{⾒,⽔}[HSK 7-9]
  \definition{v.}{pesquisar; observar e medir; observar e medir (astronomia, geografia, clima, direção, etc.) | observar; assistir e analisar; observar e medir (situação)}
\end{EntryWithPhonetic}

\begin{EntryWithPhonetic}{观察}{guan1cha2}{6,14}{⾒,⼧}[HSK 3]
  \definition{v.}{assistir; pesquisar; observar; examinar cuidadosamente coisas ou fenômenos}
\end{EntryWithPhonetic}

\begin{EntryWithPhonetic}{观点}{guan1dian3}{6,9}{⾒,⽕}[HSK 2]
  \definition[个,种]{s.}{ponto de vista; perspectiva; a visão ou atitude que se tem sobre algo a partir de uma determinada posição ou perspectiva | ponto de vista; perspectiva; a posição ou perspectiva adotada ao analisar uma questão}
\end{EntryWithPhonetic}

\begin{EntryWithPhonetic}{观感}{guan1gan3}{6,13}{⾒,⼼}[HSK 7-9]
  \definition{s.}{impressões; observações | impressões de alguém}
\end{EntryWithPhonetic}

\begin{EntryWithPhonetic}{观光}{guan1guang1}{6,6}{⾒,⼉}[HSK 6]
  \definition{v.}{visitar; passear; fazer turismo; fazer um passeio em um país ou lugar estrangeiro}
\end{EntryWithPhonetic}

\begin{EntryWithPhonetic}{观看}{guan1kan4}{6,9}{⾒,⽬}[HSK 3]
  \definition{v.}{assistir; ver propositadamente; observar}
\end{EntryWithPhonetic}

\begin{EntryWithPhonetic}{观摩}{guan1mo2}{6,15}{⾒,⼿}[HSK 7-9]
  \definition{v.}{inspecionar e aprender com o trabalho uns dos outros; visualizar e emular; observar, refere"-se principalmente a observar as conquistas uns dos outros, trocar experiências e aprender uns com os outros}
\end{EntryWithPhonetic}

\begin{EntryWithPhonetic}{观念}{guan1nian4}{6,8}{⾒,⼼}[HSK 3]
  \definition[种,个]{s.}{ideia; conceito; consciência ideológica}
\end{EntryWithPhonetic}

\begin{EntryWithPhonetic}{观赏}{guan1shang3}{6,12}{⾒,⾙}[HSK 7-9]
  \definition{v.}{ver e admirar; apreciar a vista de; assistir e aproveitar}
\end{EntryWithPhonetic}

\begin{EntryWithPhonetic}{观望}{guan1wang4}{6,11}{⾒,⽉}[HSK 7-9]
  \definition{v.}{esperar para ver; observar (de lado) | olhar ao redor}
\end{EntryWithPhonetic}

\begin{EntryWithPhonetic}{观众}{guan1zhong4}{6,6}{⾒,⼈}[HSK 3]
  \definition[位,名,批,个]{s.}{espectador; público; audiência; pessoas que assistem a espetáculos ou competições}
\end{EntryWithPhonetic}

%%%%%%%%%% 官 %%%%%%%%%%
\subsection*{官}\addcontentsline{loh}{figure}{官 \dpy{guan1}}

\begin{EntryWithPhonetic}{官}{guan1}{8}{⼧}[HSK 4]
  \definition*{s.}{Sobrenome: Guan}
  \definition{adj.}{propriedade do governo; pertencente ao governo ou ao público | público}
  \definition[个,位,名,些]{s.}{funcionário do governo; oficial; servidor público; titular de cargo; funcionário público nomeado acima de um determinado nível | órgão (parte do tecido do corpo)}
\end{EntryWithPhonetic}

\begin{EntryWithPhonetic}{官兵}{guan1bing1}{8,7}{⼧,⼋}[HSK 7-9]
  \definition{s.}{oficiais e soldados | Obsoleto: tropas governamentais}
\end{EntryWithPhonetic}

\begin{EntryWithPhonetic}{官方}{guan1fang1}{8,4}{⼧,⽅}[HSK 4]
  \definition{s.}{autoridade; (do ou pelo) governo | oficial (de uma organização ou instituição)}
\end{EntryWithPhonetic}

\begin{EntryWithPhonetic}{官桂}{guan1gui4}{8,10}{⼧,⽊}
  \definition{s.}{canela; também escrito como 肉桂}
  \seealsoref{肉桂}{rou4gui4}
\end{EntryWithPhonetic}

\begin{EntryWithPhonetic}{官吏}{guan1li4}{8,6}{⼧,⼝}[HSK 7-9]
  \definition{s.}{funcionários do governo | burocrata | oficial}
\end{EntryWithPhonetic}

\begin{EntryWithPhonetic}{官僚}{guan1liao2}{8,14}{⼧,⼈}[HSK 7-9]
  \definition{s.}{burocrata | burocracia | oficial}
\end{EntryWithPhonetic}

\begin{EntryWithPhonetic}{官僚主义}{guan1liao2 zhu3yi4}{8,14,5,3}{⼧,⼈,⼂,⼂}[HSK 7-9]
  \definition{s.}{burocracia; burocratismo}
\end{EntryWithPhonetic}

\begin{EntryWithPhonetic}{官司}{guan1si5}{8,5}{⼧,⼝}[HSK 6]
  \definition[场,个]{s.}{ação judicial}
\end{EntryWithPhonetic}

\begin{EntryWithPhonetic}{官员}{guan1yuan2}{8,7}{⼧,⼝}[HSK 7-9]
  \definition[名,位]{s.}{oficial; funcionários do governo de um determinado nível}
\end{EntryWithPhonetic}

%%%%%%%%%% 冠 %%%%%%%%%%
\subsection*{冠}\addcontentsline{loh}{figure}{冠 \dpy{guan1}}

\begin{EntryWithPhonetic}{冠}{guan1}{9}{⼍}
  \definition{s.}{chapéu | corona; coroa; copa | crista}
  \seeref{guan4}
\end{EntryWithPhonetic}

%%%%%%%%%% 棺 %%%%%%%%%%
\subsection*{棺}\addcontentsline{loh}{figure}{棺 \dpy{guan1}}

\begin{EntryWithPhonetic}{棺}{guan1}{12}{⽊}
  \definition[副]{s.}{caixão; esquife; ataúde}
\end{EntryWithPhonetic}

\begin{EntryWithPhonetic}{棺材}{guan1cai5}{12,7}{⽊,⽊}[HSK 7-9]
  \definition[具,口]{s.}{caixão; esquife; ataúde; féretro; urna funerária usada para enterrar os mortos, geralmente feito de madeira}
\end{EntryWithPhonetic}

%%%%%%%%%% 管 %%%%%%%%%%
\subsection*{管}\addcontentsline{loh}{figure}{管 \dpy{guan3}}

\begin{EntryWithPhonetic}{管}{guan3}{14}{⽵}[HSK 3]
  \definition*{s.}{Guan, um estado da dinastia Zhou | Sobrenome: Guan}
  \definition{adj.}{estreito; restrito; limitado; pequeno}
  \definition{clas.}{usado para objetos cilíndricos longos e finos}
  \definition{conj.}{não importa (quem, o quê, como, etc.)}
  \definition{prep.}{função semelhante a 把, usada especificamente em conjunto com 叫}
  \definition[根,条,排]{s.}{cano; tubo | instrumento musical de sopro | válvula; tubo | duto; canal; vasos}
  \definition{v.}{administrar; dirigir; controlar; cuidar; ser responsável por | ter jurisdição sobre; administrar | disciplinar (crianças ou alunos) | preocupar-se com; importar-se com; incomodar-se com; intervir | fornecer; garantir | supervisionar | governar | submeter alguém a disciplina | assumir; arcar com | incomodar; interferir | assegurar; garantir}
  \seealsoref{把}{ba3}
  \seealsoref{叫}{jiao4}
\end{EntryWithPhonetic}

\begin{EntryWithPhonetic}{管道}{guan3dao4}{14,12}{⽵,⾡}[HSK 6]
  \definition[根,千米,公里]{s.}{oleoduto; canal; túnel; tubulação; um tubo feito de metal ou outro material usado para transportar ou descarregar fluidos (como vapor, gás, óleo, água, etc.) | caminho; canal; abordagem}
\end{EntryWithPhonetic}

\begin{EntryWithPhonetic}{管家}{guan3jia1}{14,10}{⽵,⼧}[HSK 7-9]
  \definition[个]{s.}{mordomo; antigamente, referia"-se a alguém que administrava os negócios de uma família rica | governanta; alguém que gerencia as tarefas domésticas | gerente; governanta; uma pessoa que administra bens ou negócios familiares ou coletivos}
  \definition{v.}{administrar uma casa}
\end{EntryWithPhonetic}

\begin{EntryWithPhonetic}{管……叫……}{guan3 jiao4}{14,5}{⽵,⼝}
  \definition{expr.}{chamar alguém (ou algo) de alguém (ou algo)}
\end{EntryWithPhonetic}

\begin{EntryWithPhonetic}{管教}{guan3jiao4}{14,11}{⽵,⽁}[HSK 7-9]
  \definition{adv.}{Dialeto: certamente; seguramente}
  \definition{v.}{corrigir; disciplinar alguém júnior | responsabilizar"-se por | ensinar}
\end{EntryWithPhonetic}

\begin{EntryWithPhonetic}{管理}{guan3li3}{14,11}{⽵,⽟}[HSK 3]
  \definition{v.}{gerenciar; executar; administrar; governar; estar encarregado de; responsável por garantir o bom andamento de uma determinada tarefa | controlar; gerenciar; fazer com que pessoas e animais obedeçam ou se comportem de maneira ordeira | cuidar; zelar por; proteger; cuidar, organizar coisas}
\end{EntryWithPhonetic}

\begin{EntryWithPhonetic}{管理费}{guan3li3fei4}{14,11,9}{⽵,⽟,⾙}[HSK 7-9]
  \definition{s.}{despesas de gestão; custos de administração | taxa de administração}
\end{EntryWithPhonetic}

\begin{EntryWithPhonetic}{管辖}{guan3xia2}{14,14}{⽵,⾞}[HSK 7-9]
  \definition{v.}{gerenciar; governar (pessoal, assuntos, áreas, casos, etc.)}
\end{EntryWithPhonetic}

\begin{EntryWithPhonetic}{管用}{guan3yong4}{14,5}{⽵,⽤}[HSK 7-9]
  \definition{adj.}{eficaz; funcional}
\end{EntryWithPhonetic}

\begin{EntryWithPhonetic}{管仲}{guan3 zhong4}{14,6}{⽵,⼈}
  \definition*{s.}{uma visão restrita através de um tubo de bambu | conhecido como tubo de Guangzi 管子}
  \definition*{s.}{Guan Zhong (-645 aC), famoso político do Qi (齐国) do período da Primavera e Outono}
  \seealsoref{管子}{guan3zi5}
  \seealsoref{齐国}{qi2 guo2}
\end{EntryWithPhonetic}

\begin{EntryWithPhonetic}{管子}{guan3zi5}{14,3}{⽵,⼦}[HSK 7-9]
  \definition*{s.}{Guanzi ou Guan Zhong 管仲 (-645 a.C.), famoso político de Qi (齐国) do período da Primavera e do Outono | Guanzi, livro clássico contendo escritos de Guan Zhong e sua escola}
  \seealsoref{管仲}{guan3 zhong4}
  \seealsoref{齐国}{qi2 guo2}
\end{EntryWithPhonetic}

%%%%%%%%%% 观 %%%%%%%%%%
\subsection*{观}\addcontentsline{loh}{figure}{观 \dpy{guan4}}

\begin{EntryWithPhonetic}{观}{guan4}{6}{⾒}
  \definition*{s.}{Sobrenome: Guan}
  \definition{s.}{mosteiro taoísta | torre de vigia do portão do palácio | plataforma}
  \seeref{guan1}
\end{EntryWithPhonetic}

%%%%%%%%%% 贯 %%%%%%%%%%
\subsection*{贯}\addcontentsline{loh}{figure}{贯 \dpy{guan4}}

\begin{EntryWithPhonetic}{贯}{guan4}{8}{⾙}
  \definition*{s.}{Sobrenome: Guan}
  \definition{clas.}{uma sequência de 1.000 em dinheiro; antigamente, o dinheiro era amarrado com cordas, e cada mil moedas era uma corda.}
  \definition{s.}{lugar nativo; local de nascimento; lugar do lar ancestral; lugar onde gerações viveram | Literário: exemplo; instância; caso; precedente | Arcaico: guan (uma corda de 1.000 moedas de cobre); corda para amarrar dinheiro nos tempos antigos}
  \definition{v.}{passar através de; perfurar; enfiar; penetrar | estar ligados entre si; seguir em linha contínua; estar conectado | Literário: comparecer}
\end{EntryWithPhonetic}

\begin{EntryWithPhonetic}{贯彻}{guan4che4}{8,7}{⾙,⼻}[HSK 7-9]
  \definition{v.}{executar; implementar; pôr em prática; realizar ou incorporar completamente (diretrizes, políticas, espírito, etc.)}
\end{EntryWithPhonetic}

\begin{EntryWithPhonetic}{贯穿}{guan4chuan1}{8,9}{⾙,⽳}[HSK 7-9]
  \definition{v.}{cruzar; conectar; penetrar; correr através; passar através | permear; estar cheio de}
\end{EntryWithPhonetic}

\begin{EntryWithPhonetic}{贯通}{guan4tong1}{8,10}{⾙,⾡}[HSK 7-9]
  \definition{v.}{ter um conhecimento profundo de; ser bem versado (em); (acadêmico, ideológico, etc.) ter compreensão completa | ligar; encadear}
\end{EntryWithPhonetic}

%%%%%%%%%% 冠 %%%%%%%%%%
\subsection*{冠}\addcontentsline{loh}{figure}{冠 \dpy{guan4}}

\begin{EntryWithPhonetic}{冠}{guan4}{9}{⼍}
  \definition*{s.}{Sobrenome: Guan}
  \definition{s.}{primeiro lugar; o melhor; classificado em primeiro lugar}
  \definition{v.}{colocar um chapéu (boné) | preceder com (por); coroar com; adicionar um nome ou texto na frente}
  \seeref{guan1}
\end{EntryWithPhonetic}

\begin{EntryWithPhonetic}{冠军}{guan4jun1}{9,6}{⼍,⼍}[HSK 5]
  \definition[位,名,项,个]{s.}{campeão; medalhista de ouro; primeiro lugar em esportes e outras competições}
\end{EntryWithPhonetic}

%%%%%%%%%% 惯 %%%%%%%%%%
\subsection*{惯}\addcontentsline{loh}{figure}{惯 \dpy{guan4}}

\begin{EntryWithPhonetic}{惯}{guan4}{11}{⼼}[HSK 7-9]
  \definition{adj.}{habitual; costumeiro; usual | incorrigível; endurecido}
  \definition{v.}{estar acostumado a; ter o hábito de | mimar; estragar}
\end{EntryWithPhonetic}

\begin{EntryWithPhonetic}{惯例}{guan4li4}{11,8}{⼼,⼈}[HSK 7-9]
  \definition[个]{s.}{rotina; convenção; prática usual; prática habitual | precedente; embora não haja nenhuma disposição explícita na lei, há práticas que foram implementadas no passado e podem ser imitadas}
\end{EntryWithPhonetic}

\begin{EntryWithPhonetic}{惯性}{guan4xing4}{11,8}{⼼,⼼}[HSK 7-9]
  \definition{s.}{Física: inércia; a força da inércia}
\end{EntryWithPhonetic}

%%%%%%%%%% 灌 %%%%%%%%%%
\subsection*{灌}\addcontentsline{loh}{figure}{灌 \dpy{guan4}}

\begin{EntryWithPhonetic}{灌}{guan4}{20}{⽔}[HSK 7-9]
  \definition*{s.}{Sobrenome: Guan}
  \definition{s.}{arbusto; aglomerados de árvores baixas | irrigação}
  \definition{v.}{irrigar (rega e irrigação do solo) | encher; despejar; injetar | gravar; refere"-se à gravação (música)}
\end{EntryWithPhonetic}

\begin{EntryWithPhonetic}{灌溉}{guan4gai4}{20,12}{⽔,⽔}[HSK 7-9]
  \definition{v.}{regar; irrigar}
\end{EntryWithPhonetic}

\begin{EntryWithPhonetic}{灌输}{guan4shu1}{20,13}{⽔,⾞}[HSK 7-9]
  \definition{v.}{implantar; incutir em; inculcar; imbuir com (ideias, conhecimento); transmitir (ideias, conhecimento, etc.) | canalizar água; despejar água em; direcionar a água para onde ela é necessária}
\end{EntryWithPhonetic}

%%%%%%%%%% 罐 %%%%%%%%%%
\subsection*{罐}\addcontentsline{loh}{figure}{罐 \dpy{guan4}}

\begin{EntryWithPhonetic}{罐}{guan4}{23}{⽸}[HSK 7-9]
  \definition{clas.}{lata; jarra; gavetas e recipientes de água feitos de cerâmica ou metal}[我买了一罐可乐。===Comprei uma lata de Coca-Cola.]
  \definition{s.}{lata; jarra; jarro; pote; tanque | cuba de carvão; vagão de caçamba para carregamento de carvão em minas de carvão}
\end{EntryWithPhonetic}

\begin{EntryWithPhonetic}{罐头食品}{guan4tou2 shi2pin3}{23,5,9,9}{⽸,⼤,⾷,⼝}
  \definition{s.}{alimentos enlatados; produtos enlatados}
\end{EntryWithPhonetic}

\begin{EntryWithPhonetic}{罐头}{guan4tou5}{23,5}{⽸,⼤}[HSK 7-9]
  \definition[个,盒,瓶]{s.}{lata; jarra | enlatado; comida enlatada é a abreviação de 罐头食品, que é processada e embalada em latas de ferro seladas ou garrafas de vidro, e pode ser armazenada por um longo tempo}
  \seealsoref{罐头食品}{guan4tou2 shi2pin3}
\end{EntryWithPhonetic}

%%%%%%%%%% 光 %%%%%%%%%%
\subsection*{光}\addcontentsline{loh}{figure}{光 \dpy{guang1}}

\begin{EntryWithPhonetic}{光}{guang1}{6}{⼉}[HSK 3]
  \definition*{s.}{Sobrenome: Guang}
  \definition{adj.}{suave; liso; brilhante | esgotado; sem nada sobrando | brilhante}
  \definition{adv.}{somente; sozinho; meramente}
  \definition{s.}{luz; raio | cenário; paisagem | honra; glória; brilho | claridade | favor; graça | momento | corpo celeste; referindo"-se especificamente a corpos celestes, como o sol, a lua e as estrelas}
  \definition{v.}{glorificar; recuperar; reconquistar | estar nu; expor}
\end{EntryWithPhonetic}

\begin{EntryWithPhonetic}{光彩}{guang1cai3}{6,11}{⼉,⼺}[HSK 7-9]
  \definition{adj.}{glorioso; honroso; decente}
  \definition{s.}{brilho; esplendor; radiância}
\end{EntryWithPhonetic}

\begin{EntryWithPhonetic}{光碟}{guang1die2}{6,14}{⼉,⽯}[HSK 7-9]
  \definition[个,片,张]{s.}{disco compacto (CD); videodisco; CD; CD-ROM; disco ótico}
\end{EntryWithPhonetic}

\begin{EntryWithPhonetic}{光顾}{guang1gu4}{6,10}{⼉,⾴}[HSK 7-9]
  \definition{v.}{patrocinar; honrar com; uma palavra que demonstra respeito a alguém, referindo"-se à chegada de um convidado; restaurantes e lojas costumam usá"-la para dar as boas"-vindas aos clientes; também é usada de forma metafórica e irônica}
\end{EntryWithPhonetic}

\begin{EntryWithPhonetic}{光滑}{guang1hua2}{6,12}{⼉,⽔}[HSK 7-9]
  \definition{adj.}{liso; suave; brilhante}
\end{EntryWithPhonetic}

\begin{EntryWithPhonetic}{光环}{guang1huan2}{6,8}{⼉,⽟}[HSK 7-9]
  \definition[道]{s.}{um anel de luz; matéria brilhante ao redor de alguns planetas | halo; auréola; o halo anular na cabeça de uma divindade | um halo colorido que às vezes aparece ao redor do sol ou da lua | glória; distinção; esplendor; metáfora para fama e honra}
\end{EntryWithPhonetic}

\begin{EntryWithPhonetic}{光辉}{guang1hui1}{6,12}{⼉,⾞}[HSK 6]
  \definition{adj.}{brilhante; magnífico; glorioso}
  \definition{s.}{esplendor; brilho; glória | chama; brilho; halo; labareda; fulguração; lustre}
\end{EntryWithPhonetic}

\begin{EntryWithPhonetic}{光缆}{guang1lan3}{6,12}{⼉,⽷}[HSK 7-9]
  \definition[根,条]{s.}{cabo óptico; cabo de fibra óptica}
\end{EntryWithPhonetic}

\begin{EntryWithPhonetic}{光临}{guang1lin2}{6,9}{⼉,⼁}[HSK 4]
  \definition{v.}{honrar com sua presença, uma palavra de honra, usada para dizer que um convidado chegou}
\end{EntryWithPhonetic}

\begin{EntryWithPhonetic}{光芒}{guang1mang2}{6,6}{⼉,⾋}[HSK 7-9]
  \definition[道]{s.}{brilho; radiância; raios brilhantes; raios de luz; luz forte irradiando em todas as direções}
\end{EntryWithPhonetic}

\begin{EntryWithPhonetic}{光明}{guang1ming2}{6,8}{⼉,⽇}[HSK 3]
  \definition{adj.}{brilhante; luminoso | sincero; ingênuo; metáfora da justiça e da esperança | justo; honesto; franco}
  \definition{s.}{luz}
\end{EntryWithPhonetic}

\begin{EntryWithPhonetic}{光明磊落}{guang1ming2-lei3luo4}{6,8,15,12}{⼉,⽇,⽯,⾋}[HSK 7-9]
  \definition{expr.}{aberto e sincero; direto e honesto; descreve ser altruísta e de mente aberta; aberto e transparente}
\end{EntryWithPhonetic}

\begin{EntryWithPhonetic}{光盘}{guang1pan2}{6,11}{⼉,⽫}[HSK 4]
  \definition[张,套,片]{s.}{CD; disco compacto; um disco circular feito de plástico rígido composto que usa um laser para registrar e ler informações}
\end{EntryWithPhonetic}

\begin{EntryWithPhonetic}{光槃}{guang1pan2}{6,14}{⼉,⽊}
  \variantof{光盘}
\end{EntryWithPhonetic}

\begin{EntryWithPhonetic}{光荣}{guang1rong2}{6,9}{⼉,⾋}[HSK 5]
  \definition{adj.}{honroso; honrado; glorioso; por fazer algo que é benéfico para o país ou para a coletividade e que é considerado por todos como digno de respeito ou elogio}
  \definition{s.}{honra; glória; crédito; sentimento de honra decorrente do fato de ser respeitado ou elogiado por fazer algo importante ou grandioso}
\end{EntryWithPhonetic}

\begin{EntryWithPhonetic}{光污染}{guang1 wu1ran3}{6,6,9}{⼉,⽔,⽊}
  \definition{s.}{poluição luminosa}
\end{EntryWithPhonetic}

\begin{EntryWithPhonetic}{光线}{guang1xian4}{6,8}{⼉,⽷}[HSK 5]
  \definition[条,道]{s.}{luz; feixe luminoso; raio de luz}
\end{EntryWithPhonetic}

\begin{EntryWithPhonetic}{光泽}{guang1ze2}{6,8}{⼉,⽔}[HSK 7-9]
  \definition{s.}{brilho; lustro; fulgor; luz brilhante refletida de uma superfície; cor e brilho}
\end{EntryWithPhonetic}

%%%%%%%%%% 广 %%%%%%%%%%
\subsection*{广}\addcontentsline{loh}{figure}{广 \dpy{guang3}}

\begin{EntryWithPhonetic}{广}{guang3}{3}{⼴}[HSK 5][Kangxi 53]
  \definition*{s.}{Sobrenome: Guang}
  \definition{adj.}{largo; vasto; amplo; extenso | numeroso | comum; universal}
  \definition{s.}{Guangdong, 广东, e Guangxi, 广州}
  \definition{v.}{expandir; espalhar; ampliar}
  \seeref{an1}
  \seeref{yan3}
  \seealsoref{广东}{guang3dong1}
  \seealsoref{广州}{guang3zhou1}
  \antonymref{狭}{xia2}
\end{EntryWithPhonetic}

\begin{EntryWithPhonetic}{广播}{guang3bo1}{3,15}{⼴,⼿}[HSK 3]
  \definition[个,次,段,则,条]{s.}{programa de rádio; transmissão (de rádio); refere"-se a programas transmitidos por estações de rádio ou televisão a cabo}
  \definition{v.}{transmitir; estar no ar | espalhar"-se amplamente; ser conhecido em toda parte; divulgar amplamente}
\end{EntryWithPhonetic}

\begin{EntryWithPhonetic}{广场}{guang3chang3}{3,6}{⼴,⼟}[HSK 2]
  \definition{s.}{praça; praça pública; esplanada; área ampla, especificamente uma área ampla na cidade}
\end{EntryWithPhonetic}

\begin{EntryWithPhonetic}{广场舞}{guang3chang3wu3}{3,6,14}{⼴,⼟,⾇}
  \definition{s.}{quadrilha, uma rotina de exercícios tocada com música em quadrados públicos, parques e praças, popular especialmente entre mulheres de meia-idade e aposentados na China}
\end{EntryWithPhonetic}

\begin{EntryWithPhonetic}{广大}{guang3da4}{3,3}{⼴,⼤}[HSK 3]
  \definition{adj.}{muito difundido; enorme (alcance, escala) | (uma área ou espaço) vasto; extenso; em grande escala; amplo (área, espaço) | numeroso; muitos (número de pessoas)}
\end{EntryWithPhonetic}

\begin{EntryWithPhonetic}{广东}{guang3dong1}{3,5}{⼴,⼀}
  \definition*{s.}{Província de Guangdong}
  \seealsoref{粤}{yue4}
\end{EntryWithPhonetic}

\begin{EntryWithPhonetic}{广泛}{guang3fan4}{3,7}{⼴,⽔}[HSK 5]
  \definition{adj.}{amplo; extenso; de grande alcance; disseminado; escopo e cobertura amplos}
\end{EntryWithPhonetic}

\begin{EntryWithPhonetic}{广告}{guang3gao4}{3,7}{⼴,⼝}[HSK 2]
  \definition[则,条,段,项,个]{s.}{anúncio; propaganda; uma forma de divulgação ao público de produtos, serviços ou programas culturais e esportivos, geralmente realizada por meio de jornais, televisão, rádio, cartazes, etc.}
  \definition{v.}{anunciar; a ação ou ato de promover ou divulgar algo}
\end{EntryWithPhonetic}

\begin{EntryWithPhonetic}{广阔}{guang3kuo4}{3,12}{⼴,⾨}[HSK 6]
  \definition{adj.}{vasto; largo; amplo}
\end{EntryWithPhonetic}

\begin{EntryWithPhonetic}{广西}{guang3xi1}{3,6}{⼴,⾑}
  \definition*{s.}{Guangxi (Região Autônoma de Zhuang)}
  \seealsoref{壮}{zhuang4}
\end{EntryWithPhonetic}

\begin{EntryWithPhonetic}{广义}{guang3yi4}{3,3}{⼴,⼂}[HSK 7-9]
  \definition*{s.}{Província de Quang Ngai; nome de lugar vietnamita, uma das províncias do Vietnã Central}
  \definition{s.}{sentido amplo; sentido geral; definição mais ampla}
\end{EntryWithPhonetic}

\begin{EntryWithPhonetic}{广州}{guang3zhou1}{3,6}{⼴,⼮}
  \definition*{s.}{Guangzhou, antigamente Cantão; Capital da Província de Guangdong}
\end{EntryWithPhonetic}

%%%%%%%%%% 逛 %%%%%%%%%%
\subsection*{逛}\addcontentsline{loh}{figure}{逛 \dpy{guang4}}

\begin{EntryWithPhonetic}{逛}{guang4}{10}{⾡}[HSK 4]
  \definition{v.}{perambular; passear; vaguear}
\end{EntryWithPhonetic}

%%%%%%%%%% 归 %%%%%%%%%%
\subsection*{归}\addcontentsline{loh}{figure}{归 \dpy{gui1}}

\begin{EntryWithPhonetic}{归}{gui1}{5}{⼹}[HSK 4]
  \definition*{s.}{Sobrenome: Gui}
  \definition{s.}{divisão no ábaco com divisor de um dígito}
  \definition{v.}{retornar; voltar para; voltar (ou ir) | devolver algo a; dar de volta a | convergir; juntar-se | encarregar alguém de algo | atribuir a; pertencer a}
  \definition{v.aux.}{usado entre dois verbos idênticos, indicando que a ação não levou ao resultado correspondente}
\end{EntryWithPhonetic}

\begin{EntryWithPhonetic}{归根到底}{gui1gen1-dao4di3}{5,10,8,8}{⼹,⽊,⼑,⼴}[HSK 7-9]
  \definition{expr.}{``Em última análise.'', significa que, no final, as coisas acabarão de uma certa maneira; ela vem de 《何典》, de 张南庄, da Dinastia Qing (清); na análise final (última); no longo prazo; afinal; na análise final; em essência; fundamentalmente}
  \seealsoref{何典}{he2 dian3}
  \seealsoref{清}{qing1}
  \seealsoref{张南庄}{zhang1 nan2zhuang1}
\end{EntryWithPhonetic}

\begin{EntryWithPhonetic}{归还}{gui1huan2}{5,7}{⼹,⾡}[HSK 7-9]
  \definition{v.}{retornar; reverter; devolver dinheiro ou itens emprestados ao proprietário original}
  \antonymref{借用}{jie4yong4}
\end{EntryWithPhonetic}

\begin{EntryWithPhonetic}{归结}{gui1jie2}{5,9}{⼹,⽷}[HSK 7-9]
  \definition{s.}{fim; final (de uma história, etc.) | resolução}
  \definition{v.}{chegar a uma conclusão; resumir; colocar em poucas palavras}
\end{EntryWithPhonetic}

\begin{EntryWithPhonetic}{归来}{gui1lai2}{5,7}{⼹,⽊}[HSK 7-9]
  \definition{v.}{retornar; voltar ou estar de volta; retornar ao local de onde você começou ou partiu de outro lugar}
\end{EntryWithPhonetic}

\begin{EntryWithPhonetic}{归纳}{gui1na4}{5,7}{⼹,⽷}[HSK 7-9]
  \definition{s.}{indução; método indutivo}
  \definition{v.}{induzir; concluir; mesclar e classificar; resumir (usado principalmente para coisas abstratas)}
\end{EntryWithPhonetic}

\begin{EntryWithPhonetic}{归属}{gui1shu3}{5,12}{⼹,⼫}[HSK 7-9]
  \definition{v.}{pertencer a; estar sob a jurisdição de; definir afiliação}
\end{EntryWithPhonetic}

\begin{EntryWithPhonetic}{归宿}{gui1su4}{5,11}{⼹,⼧}[HSK 7-9]
  \definition{s.}{um lar para retornar; destino final; fim}
\end{EntryWithPhonetic}

%%%%%%%%%% 龟 %%%%%%%%%%
\subsection*{龟}\addcontentsline{loh}{figure}{龟 \dpy{gui1}}

\begin{EntryWithPhonetic}{龟}{gui1}{7}{⿔}[HSK 7-9][Kangxi 213]
  \definition[只]{s.}{tartaruga; cágado}
\end{EntryWithPhonetic}

\begin{EntryWithPhonetic}{龟速}{gui1su4}{7,10}{⿔,⾡}
  \definition{adv.}{tão lento quanto uma tartaruga}
\end{EntryWithPhonetic}

%%%%%%%%%% 规 %%%%%%%%%%
\subsection*{规}\addcontentsline{loh}{figure}{规 \dpy{gui1}}

\begin{EntryWithPhonetic}{规}{gui1}{8}{⾒}
  \definition*{s.}{Sobrenome: Gui}
  \definition[个,种]{s.}{bússola | regulamentação; regra | (mecânica) medidor | compasso; ferramenta para desenhar círculos}
  \definition{v.}{admoestar; aconselhar; advertir | planejar; fazer planos}
\end{EntryWithPhonetic}

\begin{EntryWithPhonetic}{规定}{gui1ding4}{8,8}{⾒,⼧}[HSK 3]
  \definition[个,条,项,款]{s.}{regra; regulamento; estipulação; tomar decisões sobre a forma, o método, a quantidade ou a qualidade de algo}
  \definition{v.}{estipular; prover; prescrever; estabelecer requisitos ou restrições em termos de métodos, qualidade, quantidade, tempo, etc.}
\end{EntryWithPhonetic}

\begin{EntryWithPhonetic}{规范}{gui1fan4}{8,9}{⾒,⾋}[HSK 3]
  \definition{adj.}{regular; normal; padrão; que atende às especificações; em conformidade com as normas}
  \definition{s.}{norma; padrão; diretriz}
  \definition{v.}{regular; padronizar; tornar conforme as normas}
\end{EntryWithPhonetic}

\begin{EntryWithPhonetic}{规格}{gui1ge2}{8,10}{⾒,⽊}[HSK 7-9]
  \definition[种]{s.}{normas; padrões; especificações; padrões de qualidade do produto, como determinados tamanho, peso, precisão, desempenho, etc. | formato; padrão; requisito; geralmente se refere a requisitos ou condições especificados}
\end{EntryWithPhonetic}

\begin{EntryWithPhonetic}{规划}{gui1hua4}{8,6}{⾒,⼑}[HSK 5]
  \definition[个,项]{s.}{plano; projeto; planejamento; programa; programação; esquematização; plano de desenvolvimento de longo prazo mais abrangente}
  \definition{v.}{planejar; programar}
\end{EntryWithPhonetic}

\begin{EntryWithPhonetic}{规矩}{gui1ju5}{8,9}{⾒,⽮}[HSK 7-9]
  \definition{adj.}{adequado; bem comportado; bem disciplinado; honesto e correto; de acordo com os padrões ou o senso comum}
  \definition[条,个,项]{s.}{regra; costume; prática estabelecida; certos padrões, regras ou costumes}
\end{EntryWithPhonetic}

\begin{EntryWithPhonetic}{规律}{gui1lv4}{8,9}{⾒,⼻}[HSK 4]
  \definition{adj.}{estável; regular; coisas, comportamentos, fenômenos, etc. que ocorrem em um determinado momento}
  \definition{s.}{lei; padrão regular; conexão essencial e recorrente entre as coisas}
\end{EntryWithPhonetic}

\begin{EntryWithPhonetic}{规模}{gui1mo2}{8,14}{⾒,⽊}[HSK 4]
  \definition[个,种]{s.}{escala; escopo; dimensões; padrão, forma ou escopo (de um empreendimento, instituição, projeto, movimento, etc.)}
\end{EntryWithPhonetic}

\begin{EntryWithPhonetic}{规则}{gui1ze2}{8,6}{⾒,⼑}[HSK 4]
  \definition{adj.}{ordenado; regular; descreve a forma, estrutura, arranjo, etc., que se conformam a uma determinada maneira organizada}
  \definition{s.}{regra; regulamento; sistema ou código de conduta prescrito para observância comum | lei; norma}
\end{EntryWithPhonetic}

%%%%%%%%%% 闺 %%%%%%%%%%
\subsection*{闺}\addcontentsline{loh}{figure}{闺 \dpy{gui1}}

\begin{EntryWithPhonetic}{闺}{gui1}{9}{⾨}
  \definition{s.}{Arcaico: (em uma casa) porta pequena; porta com arco | quarto da senhora; \emph{boudoir} | Literário: um pequeno portão; parte superior redonda e porta pequena na parte inferior}
\end{EntryWithPhonetic}

\begin{EntryWithPhonetic}{闺女}{gui1nv5}{9,3}{⾨,⼥}[HSK 7-9]
  \definition[个]{s.}{menina; donzela; mulher solteira | filha}
\end{EntryWithPhonetic}

%%%%%%%%%% 瑰 %%%%%%%%%%
\subsection*{瑰}\addcontentsline{loh}{figure}{瑰 \dpy{gui1}}

\begin{EntryWithPhonetic}{瑰}{gui1}{13}{⽟}
  \definition{adj.}{Literário: raro; maravilhoso; fabuloso}
  \definition[朵]{s.}{jaspe fino | Arcaico: uma espécie de pedra semelhante ao jade}
\end{EntryWithPhonetic}

\begin{EntryWithPhonetic}{瑰宝}{gui1bao3}{13,8}{⽟,⼧}[HSK 7-9]
  \definition{s.}{raridade; tesouro; joia; coisas muito preciosas}
\end{EntryWithPhonetic}

%%%%%%%%%% 轨 %%%%%%%%%%
\subsection*{轨}\addcontentsline{loh}{figure}{轨 \dpy{gui3}}

\begin{EntryWithPhonetic}{轨}{gui3}{6}{⾞}
  \definition{s.}{trilho; pista | curso; caminho | ordem; regulamento; regra | rotina; metaforicamente falando, métodos, regras, ordem, etc.}
  \definition{v.}{seguir | Literário: cumprir; aderir a}
\end{EntryWithPhonetic}

\begin{EntryWithPhonetic}{轨道}{gui3dao4}{6,12}{⾞,⾡}[HSK 6]
  \definition[条]{s.}{trilha; uma rota pavimentada com trilhos de aço para trens, bondes, etc. | órbita; trajetória; corpos celestes e objetos têm trajetórias de movimento regulares | caminho; curso; maneira adequada de fazer as coisas; curso adequado; uma metáfora para o desenvolvimento normal das coisas ou as normas e procedimentos que as pessoas devem seguir}
\end{EntryWithPhonetic}

\begin{EntryWithPhonetic}{轨迹}{gui3ji4}{6,9}{⾞,⾡}[HSK 7-9]
  \definition{s.}{trilha; caminho; trajetória | órbita; caminho | trilha; pegada; uma metáfora para experiências de vida ou o caminho de desenvolvimento das coisas}
\end{EntryWithPhonetic}

%%%%%%%%%% 鬼 %%%%%%%%%%
\subsection*{鬼}\addcontentsline{loh}{figure}{鬼 \dpy{gui3}}

\begin{EntryWithPhonetic}{鬼}{gui3}{9}{⿁}[HSK 5][Kangxi 194]
  \definition*{s.}{Gui, uma das mansões lunares | Gui, a vigésima terceira das vinte e oito constelações em que a esfera celeste foi dividida, consistindo de quatro estrelas em Câncer | Sobrenome: Gui}
  \definition{adj.}{evasivo; furtivo; sub"-reptício; ardiloso; enganoso, malicioso; obscuro | terrível; ruim; severo; vil | esperto; astuto; inteligente}
  \definition{s.}{espírito; fantasma; aparição; refere"-se à alma de uma pessoa após a morte | usado para formar um termo de abuso para caráter ignóbil; refere"-se a pessoas que têm maus hábitos ou cujo comportamento é repugnante | companheiro; pessoa que é considerada divertida}
\end{EntryWithPhonetic}

\begin{EntryWithPhonetic}{鬼怪}{gui3guai4}{9,8}{⿁,⼼}
  \definition{s.}{\emph{hobgoblin} | bicho-papão | fantasma}
\end{EntryWithPhonetic}

\begin{EntryWithPhonetic}{鬼火}{gui3huo3}{9,4}{⿁,⽕}
  \definition{s.}{fogo-fátuo | boitatá | fogo corredor | fogo de santelmo}
\end{EntryWithPhonetic}

%%%%%%%%%% 柜 %%%%%%%%%%
\subsection*{柜}\addcontentsline{loh}{figure}{柜 \dpy{gui4}}

\begin{EntryWithPhonetic}{柜}{gui4}{8}{⽊}
  \definition{s.}{baú; armário; gabinete | loja; balcão}
  \seeref{ju3}
\end{EntryWithPhonetic}

\begin{EntryWithPhonetic}{柜台}{gui4tai2}{8,5}{⽊,⼝}[HSK 7-9]
  \definition[个,排,组]{s.}{bar; balcão; uma longa área semelhante a uma mesa em uma loja ou banco usada para vender mercadorias ou conduzir negócios}
\end{EntryWithPhonetic}

\begin{EntryWithPhonetic}{柜子}{gui4zi5}{8,3}{⽊,⼦}[HSK 5]
  \definition[个]{s.}{gabinete; armário; dispositivo para guardar roupas, documentos, livros, etc.}
\end{EntryWithPhonetic}

%%%%%%%%%% 贵 %%%%%%%%%%
\subsection*{贵}\addcontentsline{loh}{figure}{贵 \dpy{gui4}}

\begin{EntryWithPhonetic}{贵}{gui4}{9}{⾙}[HSK 1]
  \definition*{s.}{Província de Guizhou, abreviação de 贵州 | Sobrenome: Gui}
  \definition{adj.}{caro; dispendioso | altamente valorizado; valioso | de alta patente; nobre | caro; preço ou valor elevado | digno de ser valorizado ou apreciado | nobre; honrado; posição social elevada}
  \definition{pron.}{Honrado: Seu}
  \seealsoref{贵州}{gui4zhou1}
  \antonymref{贱}{jian4}
\end{EntryWithPhonetic}

\begin{EntryWithPhonetic}{贵宾}{gui4bin1}{9,10}{⾙,⼧}[HSK 7-9]
  \definition[位]{s.}{convidado de honra; convidado distinto; um convidado de alto escalão, importante e respeitado}
\end{EntryWithPhonetic}

\begin{EntryWithPhonetic}{贵姓}{gui4xing4}{9,8}{⾙,⼥}
  \definition{expr.}{qual seu sobrenome?}
\end{EntryWithPhonetic}

\begin{EntryWithPhonetic}{贵重}{gui4zhong4}{9,9}{⾙,⾥}[HSK 7-9]
  \definition{adj.}{valioso; precioso; alto valor; digno de atenção}
\end{EntryWithPhonetic}

\begin{EntryWithPhonetic}{贵州}{gui4zhou1}{9,6}{⾙,⼮}
  \definition*{s.}{Província de Guizhou}
\end{EntryWithPhonetic}

\begin{EntryWithPhonetic}{贵族}{gui4zu2}{9,11}{⾙,⽅}[HSK 7-9]
  \definition{s.}{nobre; nobreza; aristocracia; a classe alta da classe dominante na sociedade escravista ou feudal e na monarquia moderna goza de privilégios}
\end{EntryWithPhonetic}

%%%%%%%%%% 桂 %%%%%%%%%%
\subsection*{桂}\addcontentsline{loh}{figure}{桂 \dpy{gui4}}

\begin{EntryWithPhonetic}{桂}{gui4}{10}{⽊}
  \definition*{s.}{outro nome para o rio Guijiang 桂江 (em Guangxi 广西) | outro nome para Guangxi 广西 (Região Autônoma de Zhuang) | Sobrenome: Gui}
  \definition[棵]{s.}{louro; loureiro | osmanthus de aroma doce | árvore de casca de cássia | canela; osmanthus}
  \seealsoref{广西}{guang3xi1}
  \seealsoref{桂江}{gui4jiang1}
\end{EntryWithPhonetic}

\begin{EntryWithPhonetic}{桂花}{gui4hua1}{10,7}{⽊,⾋}[HSK 7-9]
  \definition{s.}{jasmim do imperador; um arbusto perene ou pequena árvore, cujas flores também são chamadas de osmanthus, são muito perfumadas e podem ser usadas para extrair óleos aromáticos ou fazer especiarias. Variedades comuns incluem Jingui 金桂 (flores amarelo-alaranjadas), Dangui 丹桂 (flores vermelho-alaranjadas), Yingui 银桂 (flores branco-amareladas) e Sijigui 四季桂 (flores branco-amareladas).}
\end{EntryWithPhonetic}

\begin{EntryWithPhonetic}{桂江}{gui4jiang1}{10,6}{⽊,⽔}
  \definition*{s.}{Rio Guijiang}
\end{EntryWithPhonetic}

%%%%%%%%%% 跪 %%%%%%%%%%
\subsection*{跪}\addcontentsline{loh}{figure}{跪 \dpy{gui4}}

\begin{EntryWithPhonetic}{跪}{gui4}{13}{⾜}[HSK 6]
  \definition{v.}{ajoelhar-se; dobrar os joelhos de modo que um ou ambos os joelhos toquem o chão}
\end{EntryWithPhonetic}

\begin{EntryWithPhonetic}{跪拜}{gui4bai4}{13,9}{⾜,⼿}
  \definition{v.}{prostrar-se | ajoelhar-se e adorar}
\end{EntryWithPhonetic}

%%%%%%%%%% 滚 %%%%%%%%%%
\subsection*{滚}\addcontentsline{loh}{figure}{滚 \dpy{gun3}}

\begin{EntryWithPhonetic}{滚}{gun3}{13}{⽔}[HSK 5]
  \definition*{s.}{Sobrenome: Gun}
  \definition{adj.}{rolante | fervente | precipitado; torrencial}
  \definition{adv.}{muito; em um grau elevado}
  \definition{v.}{rolar; girar; virar | escapar; fugir; ir embora | ferver | amarrar; aparar; fazer bainha}
\end{EntryWithPhonetic}

\begin{EntryWithPhonetic}{滚动}{gun3dong4}{13,6}{⽔,⼒}[HSK 7-9]
  \definition{adv.}{(fazer algo) em intervalos regulares | (fazer algo) em um loop; onduladamente | expandir progressivamente (economia)}
  \definition{v.}{rolar; girar; fazer rodízio | (um trovão) fazer barulho; ribombar; estrondear | Computação: rolar}
\end{EntryWithPhonetic}

\begin{EntryWithPhonetic}{滚滚}{gun3 gun3}{13,13}{⽔,⽔}
  \definition{adj.}{ondulante | rolando continuamente}
  \definition{v.}{rolar; ondular; fluir}
\end{EntryWithPhonetic}

\begin{EntryWithPhonetic}{滚轮}{gun3lun2}{13,8}{⽔,⾞}
  \definition{s.}{pneu | dial rotativo | roda de rolagem (\emph{scroll})  (mouse de computador)}
\end{EntryWithPhonetic}

%%%%%%%%%% 棍 %%%%%%%%%%
\subsection*{棍}\addcontentsline{loh}{figure}{棍 \dpy{gun4}}

\begin{EntryWithPhonetic}{棍}{gun4}{12}{⽊}[HSK 7-9]
  \definition[根]{s.}{vara; bastão; porrete | canalha; patife; ladino; bandido}
\end{EntryWithPhonetic}

\begin{EntryWithPhonetic}{棍子}{gun4zi5}{12,3}{⽊,⼦}[HSK 7-9]
  \definition[根]{s.}{vara; bastão; um objeto longo e redondo feito de madeira, bambu ou metal}
\end{EntryWithPhonetic}

%%%%%%%%%% 过 %%%%%%%%%%
\subsection*{过}\addcontentsline{loh}{figure}{过 \dpy{guo1}}

\begin{EntryWithPhonetic}{过}{guo1}{6}{⾡}
  \definition*{s.}{Sobrenome: Guo}
  \seeref{guo4}
  \seeref{guo5}
\end{EntryWithPhonetic}

%%%%%%%%%% 锅 %%%%%%%%%%
\subsection*{锅}\addcontentsline{loh}{figure}{锅 \dpy{guo1}}

\begin{EntryWithPhonetic}{锅}{guo1}{12}{⾦}[HSK 5]
  \definition[口,个,只]{s.}{panela; frigideira; utensílios de cozinha, redondos e côncavos, feitos principalmente de ferro, alumínio, etc. | parte que se parece com um pote em alguns objetos}
\end{EntryWithPhonetic}

%%%%%%%%%% 囯 %%%%%%%%%%
\subsection*{囯}\addcontentsline{loh}{figure}{囯 \dpy{guo2}}

\begin{EntryWithPhonetic}{囯}{guo2}{7}{⼞}
  \definition*{s.}{Sobrenome: Guo}
  \definition{adj.}{do estado; nacional | do nosso país; Chinês | do país}
  \definition{s.}{país; nação; estado | o melhor da nação | o melhor; o mais bonito do país}
  \variantof{国}
\end{EntryWithPhonetic}

%%%%%%%%%% 国 %%%%%%%%%%
\subsection*{国}\addcontentsline{loh}{figure}{国 \dpy{guo2}}

\begin{EntryWithPhonetic}{国}{guo2}{8}{⼞}[HSK 1]
  \definition*{s.}{Sobrenome: Guo}
  \definition{adj.}{nacional; do estado; representante do país | o melhor de um país}
  \definition[个]{s.}{estado; nação; país}
\end{EntryWithPhonetic}

\begin{EntryWithPhonetic}{国宝}{guo2bao3}{8,8}{⼞,⼧}[HSK 7-9]
  \definition[件]{s.}{tesouro nacional}
\end{EntryWithPhonetic}

\begin{EntryWithPhonetic}{国宾馆}{guo2bin1guan3}{8,10,11}{⼞,⼧,⾷}
  \definition{s.}{pousada estadual}
\end{EntryWithPhonetic}

\begin{EntryWithPhonetic}{国产}{guo2chan3}{8,6}{⼞,⼇}[HSK 6]
  \definition{adj.}{doméstico; feito na China; produzido internamente, especificamente na China}
\end{EntryWithPhonetic}

\begin{EntryWithPhonetic}{国防}{guo2fang2}{8,6}{⼞,⾩}[HSK 7-9]
  \definition{s.}{defesa nacional; as instalações humanas, materiais e militares que um país possui para defender sua soberania territorial e impedir invasões estrangeiras}
\end{EntryWithPhonetic}

\begin{EntryWithPhonetic}{国歌}{guo2ge1}{8,14}{⼞,⽋}[HSK 6]
  \definition[首,支]{s.}{hino nacional; o hino nacional da China, oficialmente designado pelo estado como a música que representa o país, é ``Marcha dos Voluntários''}
\end{EntryWithPhonetic}

\begin{EntryWithPhonetic}{国画}{guo2hua4}{8,8}{⼞,⽥}[HSK 7-9]
  \definition[幅,张,卷]{s.}{pintura tradicional chinesa | arte chinesa | pintura nacional}
\end{EntryWithPhonetic}

\begin{EntryWithPhonetic}{国徽}{guo2hui1}{8,17}{⼞,⼻}[HSK 7-9]
  \definition{s.}{emblema nacional; o emblema nacional da China, oficialmente designado pelo estado para representar o país, apresenta a Praça da Paz Celestial sob o céu brilhante de cinco estrelas, cercada por espigas de grãos e engrenagens}
\end{EntryWithPhonetic}

\begin{EntryWithPhonetic}{国会}{guo2hui4}{8,6}{⼞,⼈}[HSK 6]
  \definition{s.}{parlamento; congresso}
\end{EntryWithPhonetic}

\begin{EntryWithPhonetic}{国籍}{guo2ji2}{8,20}{⼞,⽵}[HSK 5]
  \definition[个]{s.}{nacionalidade; cidadania; refere"-se à identidade de um indivíduo como pertencente a um Estado | identidade nacional (de um avião, navio, etc.)}
\end{EntryWithPhonetic}

\begin{EntryWithPhonetic}{国际}{guo2ji4}{8,7}{⼞,⾩}[HSK 2]
  \definition{adj.}{internacional; entre países; entre nações}
  \definition{s.}{internacional; o mundo; entre nações; entre países de todo o mundo}
\end{EntryWithPhonetic}

\begin{EntryWithPhonetic}{国际儿童节}{guo2ji4 er2tong2jie2}{8,7,2,12,5}{⼞,⾩,⼉,⽴,⾋}
  \definition*{s.}{Dia Internacional das Crianças (1 de junho)}
\end{EntryWithPhonetic}

\begin{EntryWithPhonetic}{国际妇女节}{guo2ji4 fu4nv3jie2}{8,7,6,3,5}{⼞,⾩,⼥,⼥,⾋}
  \definition*{s.}{Dia Internacional das Mulheres (8 de março)}
\end{EntryWithPhonetic}

\begin{EntryWithPhonetic}{国际劳动节}{guo2ji4 lao2dong4 jie2}{8,7,7,6,5}{⼞,⾩,⼒,⼒,⾋}
  \definition*{s.}{Dia Internacional dos Trabalhadores (1 de maio)}
\end{EntryWithPhonetic}

\begin{EntryWithPhonetic}{国家}{guo2jia1}{8,10}{⼞,⼧}[HSK 1]
  \definition[个]{s.}{país; estado; nação; um lugar reconhecido internacionalmente e com soberania independente, incluindo as pessoas e as instituições administrativas desse lugar}
\end{EntryWithPhonetic}

\begin{EntryWithPhonetic}{国民}{guo2min2}{8,5}{⼞,⽒}[HSK 5]
  \definition{adj.}{nacional}
  \definition[个]{s.}{membro de uma nação; povo de uma nação}
\end{EntryWithPhonetic}

\begin{EntryWithPhonetic}{国内}{guo2nei4}{8,4}{⼞,⼌}[HSK 3]
  \definition{s.}{interno (a um país); doméstico; lar; dentro de um determinado país}
\end{EntryWithPhonetic}

\begin{EntryWithPhonetic}{国旗}{guo2qi2}{8,14}{⼞,⽅}[HSK 6]
  \definition[面]{s.}{bandeira (de um país)}
\end{EntryWithPhonetic}

\begin{EntryWithPhonetic}{国情}{guo2qing2}{8,11}{⼞,⼼}[HSK 7-9]
  \definition{s.}{condição (ou estado) do país; condições nacionais; as condições e características básicas da natureza social, política, economia, cultura etc. de um país também se referem especificamente às condições e características básicas de um país em um determinado período de tempo}
\end{EntryWithPhonetic}

\begin{EntryWithPhonetic}{国庆}{guo2qing4}{8,6}{⼞,⼴}[HSK 3]
  \definition*{s.}{Dia Nacional, o dia em que um país comemora sua independência ou fundação}
\end{EntryWithPhonetic}

\begin{EntryWithPhonetic}{国庆节}{guo2qing4jie2}{8,6,5}{⼞,⼴,⾋}
  \definition*{s.}{Dia Nacional (1~de~outubro)}
\end{EntryWithPhonetic}

\begin{EntryWithPhonetic}{国人}{guo2ren2}{8,2}{⼞,⼈}
  \definition{s.}{compatriota}
\end{EntryWithPhonetic}

\begin{EntryWithPhonetic}{国土}{guo2tu3}{8,3}{⼞,⼟}[HSK 7-9]
  \definition{s.}{terra; território; território nacional}
\end{EntryWithPhonetic}

\begin{EntryWithPhonetic}{国外}{guo2wai4}{8,5}{⼞,⼣}[HSK 1]
  \definition{adj.}{externo; no exterior; fora do país; outros lugares fora do país; geralmente chamados de exterior;  exterior não é o mesmo que estrangeiro}
\end{EntryWithPhonetic}

\begin{EntryWithPhonetic}{国王}{guo2wang2}{8,4}{⼞,⽟}[HSK 6]
  \definition[位,名,个,些]{s.}{rei; soberanos; o governante supremo de algumas monarquias antigas; nos tempos modernos, refere"-se ao chefe de estado de algumas monarquias}
\end{EntryWithPhonetic}

\begin{EntryWithPhonetic}{国学}{guo2xue2}{8,8}{⼞,⼦}[HSK 7-9]
  \definition*{s.}{Arcaico: O Colégio Imperial}
  \definition{s.}{estudos da cultura clássica chinesa (história, filosofia, literatura, língua, etc.) | cultura nacional chinesa | estudos da antiga civilização chinesa}
\end{EntryWithPhonetic}

\begin{EntryWithPhonetic}{国有}{guo2you3}{8,6}{⼞,⽉}[HSK 7-9]
  \definition{v.}{pertencer ao estado; ser nacionalizado}
\end{EntryWithPhonetic}

\begin{EntryWithPhonetic}{国语}{guo2yu3}{8,9}{⼞,⾔}
  \definition*{s.}{Língua Chinesa (Mandarim), enfatizando sua natureza nacional}
\end{EntryWithPhonetic}

%%%%%%%%%% 果 %%%%%%%%%%
\subsection*{果}\addcontentsline{loh}{figure}{果 \dpy{guo3}}

\begin{EntryWithPhonetic}{果}{guo3}{8}{⽊}
  \definition*{s.}{Sobrenome: Guo}
  \definition{adj.}{resoluto; determinado; sem exitação}
  \definition{adv.}{realmente; como esperado; com certeza; isso significa que as coisas são consistentes com as expectativas, equivalente a 果然}
  \definition{conj.}{se realmente; se de fato}
  \definition[个,些,种]{s.}{fruta; fruto da planta | resultado; consequência; o resultado final de um assunto}
  \seealsoref{果然}{guo3ran2}
  \antonymref{因}{yin1}
\end{EntryWithPhonetic}

\begin{EntryWithPhonetic}{果断}{guo3duan4}{8,11}{⽊,⽄}[HSK 7-9]
  \definition{adj.}{resoluto; decisivo; agir decisivamente sem hesitação}
\end{EntryWithPhonetic}

\begin{EntryWithPhonetic}{果酱}{guo3jiang4}{8,13}{⽊,⾣}[HSK 6]
  \definition{s.}{geléia | compota ou doce (de frutas); fruta em conserva}
\end{EntryWithPhonetic}

\begin{EntryWithPhonetic}{果然}{guo3ran2}{8,12}{⽊,⽕}[HSK 3]
  \definition{adv.}{realmente; como esperado; com certeza; indica que os fatos correspondem ao que foi dito ou esperado}
  \definition{conj.}{se realmente; se de fato; suponha que os fatos correspondam ao que foi dito ou esperado}
\end{EntryWithPhonetic}

\begin{EntryWithPhonetic}{果实}{guo3shi2}{8,8}{⽊,⼧}[HSK 4]
  \definition[种]{s.}{fruta; o órgão que se desenvolve a partir do ovário ou com outras partes da flor após a fertilização da flor | ganhos; frutos;  uma metáfora para conquista ou recompensa por trabalho árduo}
\end{EntryWithPhonetic}

\begin{EntryWithPhonetic}{果树}{guo3shu4}{8,9}{⽊,⽊}[HSK 6]
  \definition[棵,个,片]{s.}{árvore frutífera; árvores cujos frutos são principalmente comestíveis, como pessegueiros e macieiras}
\end{EntryWithPhonetic}

\begin{EntryWithPhonetic}{果园}{guo3yuan2}{8,7}{⽊,⼞}[HSK 7-9]
  \definition[个,座]{s.}{pomar; um jardim onde são plantadas árvores frutíferas}
\end{EntryWithPhonetic}

\begin{EntryWithPhonetic}{果真}{guo3zhen1}{8,10}{⽊,⼗}[HSK 7-9]
  \definition{adv.}{realmente; como esperado; com certeza}
  \definition{conj.}{se de fato; se realmente; se for o caso}[果真如此, 我就放心了。===Se for esse o caso, então ficarei aliviado.]
\end{EntryWithPhonetic}

\begin{EntryWithPhonetic}{果汁}{guo3zhi1}{8,5}{⽊,⽔}[HSK 3]
  \definition[杯,瓶,种]{s.}{suco; suco de frutas frescas; também se refere a bebidas feitas com suco de frutas frescas}
\end{EntryWithPhonetic}

\begin{EntryWithPhonetic}{果子}{guo3zi5}{8,3}{⽊,⼦}
  \definition{s.}{fruta}
\end{EntryWithPhonetic}

%%%%%%%%%% 裹 %%%%%%%%%%
\subsection*{裹}\addcontentsline{loh}{figure}{裹 \dpy{guo3}}

\begin{EntryWithPhonetic}{裹}{guo3}{14}{⾐}[HSK 7-9]
  \definition{s.}{pacote; encomenda}
  \definition{v.}{amarrar; embrulhar; envolver | levar embora; varrer com violência | Dialeto: sugar (leite) | pressionar a servir; fugir com (algo)}
\end{EntryWithPhonetic}

%%%%%%%%%% 过 %%%%%%%%%%
\subsection*{过}\addcontentsline{loh}{figure}{过 \dpy{guo4}}

\begin{EntryWithPhonetic}{过}{guo4}{6}{⾡}[HSK 1,2]
  \definition{adv.}{excessivamente; em excesso}
  \definition{clas.}{tempo; número de vezes usado para a ação}
  \definition{s.}{falha; erro; demérito; equívoco; negligência}
  \definition{v.}{cruzar; passar; mudar-se de um lugar para outro; passar por | exceder; ir além; ultrapassar; usado após um adjetivo, significa ``mais do que'' | gastar (tempo); passar (tempo); exceder (um determinado limite ou limite) | celebrar; comemorar | mudar; transferir; transferir de um lado para o outro | passar por um processo; passar por; submeter a (algum tipo de tratamento) | visitar; fazer uma visita | falecer; morrer | infectar; ser contagioso; espalhar | exceder; ir além; usado após o verbo com o sufixo 得, significa ``superar'' ou ``passar'' | viver | revisar; examinar; usar os olhos para ver ou a mente para lembrar}
  \seeref{guo1}
  \seeref{guo5}
  \seealsoref{得}{de5}
  \antonymref{功}{gong1}
\end{EntryWithPhonetic}

\begin{EntryWithPhonetic}{过半}{guo4ban4}{6,5}{⾡,⼗}[HSK 7-9]
  \definition{s.}{maioria; mais da metade; mais de cinquenta por cento}
\end{EntryWithPhonetic}

\begin{EntryWithPhonetic}{过不惯}{guo4 bu5 guan4}{6,4,11}{⾡,⼀,⼼}
  \definition{v.}{não se acostumar; não se habituar}
  \seealsoref{过惯}{guo4guan4}
\end{EntryWithPhonetic}

\begin{EntryWithPhonetic}{过不去}{guo4bu5qu4}{6,4,5}{⾡,⼀,⼛}[HSK 7-9]
  \definition{v.}{não poder passar; ser incapaz de passar; ser ou estar bloqueado; ser intransitável | Coloquial: ser duro com; dificultar; envergonhar; colocar para fora | sentir pena; sentir-se mal | encontrar falhas em}
\end{EntryWithPhonetic}

\begin{EntryWithPhonetic}{过程}{guo4cheng2}{6,12}{⾡,⽲}[HSK 3]
  \definition[个,段]{s.}{curso dos eventos; processo; o processo pelo qual as coisas acontecem ou se desenvolvem.}
\end{EntryWithPhonetic}

\begin{EntryWithPhonetic}{过错}{guo4cuo4}{6,13}{⾡,⾦}[HSK 7-9]
  \definition{s.}{falha; erro; engano | ações ilícitas; no direito civil, refere"-se a atos ilegais que prejudicam outras pessoas intencionalmente ou negligentemente}
\end{EntryWithPhonetic}

\begin{EntryWithPhonetic}{过道}{guo4dao4}{6,12}{⾡,⾡}[HSK 7-9]
  \definition{s.}{corredor; caminho; passarela; passagem; o corredor da porta para cada cômodo da nova casa | passagem; uma passarela que conecta os pátios de uma casa antiga, especialmente o cômodo ou metade do cômodo onde o portão está localizado}
\end{EntryWithPhonetic}

\begin{EntryWithPhonetic}{过度}{guo4du4}{6,9}{⾡,⼴}[HSK 5]
  \definition{adj.}{excessivo; acima do limite; além do limite; além do que é apropriado}
\end{EntryWithPhonetic}

\begin{EntryWithPhonetic}{过渡}{guo4du4}{6,12}{⾡,⽔}[HSK 6]
  \definition{v.}{fazer a transição; estar em transição; estar em fase de transição; mudar de um estágio para outro | atravessar; cruzar}
\end{EntryWithPhonetic}

\begin{EntryWithPhonetic}{过分}{guo4fen4}{6,4}{⾡,⼑}[HSK 4]
  \definition{adj.}{excessivo; muito longe; demais; falar ou agir além dos limites ou graus adequados}
  \definition{adv.}{excessivamente; indevidamente; muito mesmo}
\end{EntryWithPhonetic}

\begin{EntryWithPhonetic}{过关}{guo4/guan1}{6,6}{⾡,⼋}[HSK 7-9]
  \definition{v.+compl.}{passar (um teste); alcançar (um padrão); cruzar uma barreira; superar (uma provação) ; passar por um posto de controle, frequentemente usado como metáfora}
\end{EntryWithPhonetic}

\begin{EntryWithPhonetic}{过惯}{guo4guan4}{6,11}{⾡,⼼}
  \definition{v.}{estar acostumado (a um certo estilo de vida, etc.)}
  \seealsoref{过不惯}{guo4 bu5 guan4}
\end{EntryWithPhonetic}

\begin{EntryWithPhonetic}{过后}{guo4hou4}{6,6}{⾡,⼝}[HSK 6]
  \definition[期]{s.}{depois; mais tarde}
\end{EntryWithPhonetic}

\begin{EntryWithPhonetic}{过奖}{guo4jiang3}{6,9}{⾡,⼤}[HSK 7-9]
  \definition{v.}{elogiar demais; bajular; dar elogios imerecidos}
\end{EntryWithPhonetic}

\begin{EntryWithPhonetic}{过节}{guo4/jie2}{6,5}{⾡,⾋}[HSK 7-9]
  \definition{v.+compl.}{celebrar um festival; passar as férias; comemorar durante as férias}[今年我们一起过节吧!===Vamos comemorar as festas juntos este ano!]
\end{EntryWithPhonetic}

\begin{EntryWithPhonetic}{过境}{guo4/jing4}{6,14}{⾡,⼟}[HSK 7-9]
  \definition{v.+compl.}{estar em trânsito; passar pelo território de um país}
\end{EntryWithPhonetic}

\begin{EntryWithPhonetic}{过来}{guo4/lai2}{6,7}{⾡,⽊}[HSK 2]
  \definition{v.+compl.}{vir até aqui | ser capaz de cuidar de | lidar com | administrar}
\end{EntryWithPhonetic}

\begin{EntryWithPhonetic}{过路人}{guo4lu4 ren2}{6,13,2}{⾡,⾜,⼈}
  \definition{s.}{transeunte}
\end{EntryWithPhonetic}

\begin{EntryWithPhonetic}{过滤}{guo4lv4}{6,13}{⾡,⽔}[HSK 7-9]
  \definition{v.}{filtrar; separar sólidos ou componentes nocivos de gases ou líquidos por meio de materiais porosos, como papel de filtro e pano de filtro}[所有饮用水必须经过过滤。===Toda água potável deve ser filtrada.]
\end{EntryWithPhonetic}

\begin{EntryWithPhonetic}{过敏}{guo4min3}{6,11}{⾡,⽁}[HSK 5]
  \definition{adj.}{sensível; excessivamente sensível; resposta acima do normal; ceticismo excessivo}
  \definition{v.}{ser alérgico a}
\end{EntryWithPhonetic}

\begin{EntryWithPhonetic}{过年}{guo4/nian2}{6,6}{⾡,⼲}[HSK 2]
  \definition{v.+compl.}{comemorar o Ano Novo; comemorar o Festival da Primavera; passar o Ano Novo; passar o Festival da Primavera; realizar atividades comemorativas durante o Ano Novo ou o Festival da Primavera}
\end{EntryWithPhonetic}

\begin{EntryWithPhonetic}{过期}{guo4/qi1}{6,12}{⾡,⽉}[HSK 7-9]
  \definition{v.+compl.}{expirar; estar vencido; exceder o limite de tempo; exceder o período prescrito ou acordado}
\end{EntryWithPhonetic}

\begin{EntryWithPhonetic}{过去}{guo4 qu5}{6,5}{⾡,⼛}[HSK 2,3]
  \definition{adv.}{(no) passado}
  \definition{s.}{o passado; refere"-se a um período anterior; também se refere a coisas anteriores}
  \definition{v.}{atravessar; passar; sair do local onde o interlocutor se encontra e deslocar"-se para outro local | acabar; passar; ficar para trás; indica que já passou por uma determinada fase | passar; indica que um determinado período ou situação já não existe mais | falecer | ir lá | passar por}
\end{EntryWithPhonetic}

\begin{EntryWithPhonetic}{过日子}{guo4 ri4zi5}{6,4,3}{⾡,⽇,⼦}[HSK 7-9]
  \definition{v.}{viver; conviver; passar/viver a própria vida}
\end{EntryWithPhonetic}

\begin{EntryWithPhonetic}{过剩}{guo4sheng4}{6,12}{⾡,⼑}[HSK 7-9]
  \definition{v.}{exceder; a quantidade excede em muito o limite necessário | saturar; oferecer em excesso; a oferta excede a demanda do mercado ou o poder de compra}
\end{EntryWithPhonetic}

\begin{EntryWithPhonetic}{过失}{guo4shi1}{6,5}{⾡,⼤}[HSK 7-9]
  \definition{s.}{falha; deslize; erro; erros cometidos por negligência | negligência; crime por negligência}
\end{EntryWithPhonetic}

\begin{EntryWithPhonetic}{过时}{guo4 shi2}{6,7}{⾡,⽇}[HSK 6]
  \definition{adj.}{fora de moda; obsoleto; antiquado; desatualizado; o que era popular no passado não é mais popular}
  \definition{v.}{passar do tempo marcado; estar após o tempo estipulado}
\end{EntryWithPhonetic}

\begin{EntryWithPhonetic}{过头}{guo4/tou2}{6,5}{⾡,⼤}[HSK 7-9]
  \definition{adv.}{excessivamente; acima da cabeça; por cima; ao alto}
  \definition{v.+compl.}{exagerar; ir além do limite; exceder o limite; ser excessivo}
\end{EntryWithPhonetic}

\begin{EntryWithPhonetic}{过屠门而大嚼}{guo4 tu2men2 er2 da4 jiao2}{6,11,3,6,3,20}{⾡,⼫,⾨,⽽,⼤,⼝}
  \definition{expr.}{``Passar pelo portão do açougueiro e comer com apetite.''; essa metáfora descreve alguém que admira algo, mas não pode tê-lo, e usa métodos irreais para se consolar; alimente-se de ilusões; tem significado negativo}
\end{EntryWithPhonetic}

\begin{EntryWithPhonetic}{过往}{guo4wang3}{6,8}{⾡,⼻}[HSK 7-9]
  \definition{v.}{ir e vir | ter relações amigáveis com; associar"-se a; lidar com}
\end{EntryWithPhonetic}

\begin{EntryWithPhonetic}{过意不去}{guo4yi4bu2qu4}{6,13,4,5}{⾡,⼼,⼀,⼛}[HSK 7-9]
  \definition{expr.}{sentir-se arrependido ou culpado; sentir-se mal ou envergonhado; sentir-se envergonhado ou arrependido; metáfora para aceitar um favor de alguém, mas não retribuí-lo, ou sentir pena de algo e não ser culpado, o que faz com que alguém se sinta arrependido e desconfortável}
\end{EntryWithPhonetic}

\begin{EntryWithPhonetic}{过瘾}{guo4/yin3}{6,16}{⾡,⽧}[HSK 7-9]
  \definition{adj.}{gratificante; imensamente agradável; satisfatório; realizador}
  \definition{v.+compl.}{satisfazer um desejo; divertir-se ao máximo; fazer algo à vontade}
\end{EntryWithPhonetic}

\begin{EntryWithPhonetic}{过硬}{guo4/ying4}{6,12}{⾡,⽯}[HSK 7-9]
  \definition{adj.}{perfeito; soberbo; à altura; verdadeiramente proficiente}
  \definition{v.+compl.}{ter domínio perfeito de algo; estar à altura; resistir a testes ou exames rigorosos}
\end{EntryWithPhonetic}

\begin{EntryWithPhonetic}{过于}{guo4yu2}{6,3}{⾡,⼆}[HSK 5]
  \definition{adv.}{demais; indevidamente; excessivamente; advérbios de grau ou quantidade excessiva}
\end{EntryWithPhonetic}

\begin{EntryWithPhonetic}{过早}{guo4 zao3}{6,6}{⾡,⽇}[HSK 7-9]
  \definition{adj.}{prematuro; inoportuno | Dialeto: café da manha}
  \definition{adv.}{muito cedo; prematuramente | Dialeto: tomar o café da manhã}
\end{EntryWithPhonetic}

\begin{EntryWithPhonetic}{过}{guo5}{6}{⾡}
  \definition{part.}{usado depois de um verbo para indicar conclusão | usado depois de um verbo para indicar que uma ação ou mudança ocorreu | usado depois de um adjetivo para indicar que algo já teve uma certa qualidade ou estado e para compará-lo com o presente}
  \seeref{guo1}
  \seeref{guo4}
\end{EntryWithPhonetic}

%%%%% EOF %%%%%


 %%%
%%% H
%%%
\section*{H}\addcontentsline{toc}{section}{H}\addcontentsline{loh}{figure}{\#\#\#\#\#\#\#\# H}

%%%%%%%%%% 哈 %%%%%%%%%%
\subsection*{哈}\addcontentsline{loh}{figure}{哈 \dpy{ha1}}

\begin{EntryWithPhonetic}{哈}{ha1}{9}{⼝}
  \definition{interj.}{Onomatopéia: ``Ha''; descreve o riso, usado principalmente em duplicata | indica orgulho ou satisfação, frequentemente usado de forma duplicada}
  \definition{v.}{soprar; expirar (com a boca aberta) | dobrar}
  \seeref{ha3}
  \seealsoref{哈哈}{ha1ha5}
\end{EntryWithPhonetic}

\begin{EntryWithPhonetic}{哈哈}{ha1ha5}{9,9}{⼝,⼝}[HSK 3]
  \definition{expr.}{Onomatopéia: ``Ha Ha''; o som de uma gargalhada}
\end{EntryWithPhonetic}

\begin{EntryWithPhonetic}{哈马斯}{ha1ma3si1}{9,3,12}{⼝,⾺,⽄}
  \definition*{s.}{Hamas (Grupo Palestino)}
\end{EntryWithPhonetic}

\begin{EntryWithPhonetic}{哈}{ha3}{9}{⼝}
  \definition*{s.}{Sobrenome: Ha}
  \definition{v.}{repreender}
  \seeref{ha1}
\end{EntryWithPhonetic}

%%%%%%%%%% 咳 %%%%%%%%%%
\subsection*{咳}\addcontentsline{loh}{figure}{咳 \dpy{hai1}}

\begin{EntryWithPhonetic}{咳}{hai1}{9}{⼝}
  \definition{interj.}{expressa tristeza, arrependimento ou espanto}
  \seeref{ke2}
\end{EntryWithPhonetic}

%%%%%%%%%% 还 %%%%%%%%%%
\subsection*{还}\addcontentsline{loh}{figure}{还 \dpy{hai2}}

\begin{EntryWithPhonetic}{还}{hai2}{7}{⾡}[HSK 1]
  \definition{adv.}{ainda; indica que a ação ou estado permanece inalterado, equivalente a 仍然 | também; além disso; em adição; indica que há um aumento ou suplemento além do escopo já indicado | ainda mais; usado com 比 para indicar que as características e o grau das coisas comparadas aumentaram, o que é equivalente a 更加 razoavelmente; medianamente; usado antes de um adjetivo, indica que algo atinge apenas o nível mínimo exigido | mesmo; usado na primeira parte da frase como complemento, e na segunda parte como conclusão, equivalente a 尚且 | que expressa realização ou descoberta; expressa surpresa por algo que não se esperava, mas que acabou acontecendo | tão cedo quanto; por um curto período de tempo; indica que já era assim há muito tempo | para dar ênfase; para reforçar o tom}
  \seeref{huan2}
  \seealsoref{比}{bi3}
  \seealsoref{更加}{geng4jia1}
  \seealsoref{仍然}{reng2ran2}
  \seealsoref{尚且}{shang4 qie3}
\end{EntryWithPhonetic}

\begin{EntryWithPhonetic}{还是}{hai2shi5}{7,9}{⾡,⽇}[HSK 1]
  \definition{adv.}{ainda; ainda assim; não é a continuação de um determinado estado, fenômeno ou ação; o resultado é o mesmo de antes, sem mudanças | que expressa uma preferência por uma alternativa; expressa comparação ou escolha feita após consideração cuidadosa, frequentemente usado para fazer sugestões | que expressa realização ou descoberta; indica que o resultado final foi inesperado}
  \definition{conj.}{ou (somente para frases interrogativas); indica várias opções, geralmente usado em perguntas | tudo; se; não importa; independentemente de; significa que, independentemente das mudanças que ocorram, o resultado permanecerá o mesmo}
\end{EntryWithPhonetic}

\begin{EntryWithPhonetic}{还有}{hai2you3}{7,6}{⾡,⽉}[HSK 1]
  \definition{adv.}{também; ainda; além disso; então novamente; enfatizar as partes complementares, excedentes ou não mencionadas além do que já é conhecido}
\end{EntryWithPhonetic}

%%%%%%%%%% 孩 %%%%%%%%%%
\subsection*{孩}\addcontentsline{loh}{figure}{孩 \dpy{hai2}}

\begin{EntryWithPhonetic}{孩}{hai2}{9}{⼦}
  \definition[个]{s.}{criança}
\end{EntryWithPhonetic}

\begin{EntryWithPhonetic}{孩子}{hai2zi5}{9,3}{⼦,⼦}[HSK 1]
  \definition[个]{s.}{criança; crianças; pessoas com idade entre alguns anos ou na adolescência, geralmente com menos de 14 anos | crianças; filho ou filha}
\end{EntryWithPhonetic}

%%%%%%%%%% 海 %%%%%%%%%%
\subsection*{海}\addcontentsline{loh}{figure}{海 \dpy{hai3}}

\begin{EntryWithPhonetic}{海}{hai3}{10}{⽔}[HSK 2]
  \definition*{s.}{Sobrenome: Hai}
  \definition{adj.}{extragrande; de grande capacidade; descreve capacidade, tom de voz, etc.}
  \definition{adv.}{aleatoriamente; sem rumo; sem limites; sem restrições}
  \definition[片]{s.}{mar; grande lago; a parte do oceano próxima à costa, alguns grandes lagos também são chamados de mar | grande número de pessoas ou coisas reunidas; metáfora para muitas coisas semelhantes que formam um grande conjunto}
\end{EntryWithPhonetic}

\begin{EntryWithPhonetic}{海岸}{hai3'an4}{10,8}{⽔,⼭}[HSK 7-9]
  \definition[段]{s.}{litoral; costa; praia}
\end{EntryWithPhonetic}

\begin{EntryWithPhonetic}{海拔}{hai3ba2}{10,8}{⽔,⼿}[HSK 7-9]
  \definition{s.}{altitude; altura em relação ao nível médio do mar}
\end{EntryWithPhonetic}

\begin{EntryWithPhonetic}{海报}{hai3bao4}{10,7}{⽔,⼿}[HSK 6]
  \definition[张,份,幅]{s.}{pôster; cartaz; cartazes anunciando apresentações culturais, exibições de filmes ou competições esportivas, etc.}
\end{EntryWithPhonetic}

\begin{EntryWithPhonetic}{海边}{hai3bian1}{10,5}{⽔,⾡}[HSK 2]
  \definition{s.}{praia; costa; litoral; orla marítima; a parte marginal do oceano e as grandes áreas de água salgada cercadas por terra firme, onde a terra e a água se encontram, formam a costa}
\end{EntryWithPhonetic}

\begin{EntryWithPhonetic}{海滨}{hai3bin1}{10,13}{⽔,⽔}[HSK 7-9]
  \definition{s.}{praia; beira-mar; litoral; um lugar perto do mar}
\end{EntryWithPhonetic}

\begin{EntryWithPhonetic}{海盗}{hai3dao4}{10,11}{⽔,⽫}[HSK 7-9]
  \definition{s.}{pirata; viajante do mar | bucaneiro; viking; pirata}
\end{EntryWithPhonetic}

\begin{EntryWithPhonetic}{海底}{hai3di3}{10,8}{⽔,⼴}[HSK 6]
  \definition{s.}{fundo do mar; fundo do oceano; solo oceânico}
\end{EntryWithPhonetic}

\begin{EntryWithPhonetic}{海风}{hai3feng1}{10,4}{⽔,⾵}
  \definition{s.}{brisa do mar | vento que vem do mar}
\end{EntryWithPhonetic}

\begin{EntryWithPhonetic}{海关}{hai3guan1}{10,6}{⽔,⼋}[HSK 3]
  \definition[个]{s.}{alfândega; órgão administrativo nacional, sua principal função é supervisionar e inspecionar os bens e meios de transporte que entram e saem do país, cobrar impostos alfandegários e reprimir o contrabando}
\end{EntryWithPhonetic}

\begin{EntryWithPhonetic}{海军}{hai3jun1}{10,6}{⽔,⼍}[HSK 6]
  \definition[支,名,位,个]{s.}{marinha; o exército que luta no mar geralmente é composto por navios de superfície, submarinos, aviação naval, fuzileiros navais e outros ramos, além de diversas forças profissionais}
\end{EntryWithPhonetic}

\begin{EntryWithPhonetic}{海浪}{hai3lang4}{10,10}{⽔,⽔}[HSK 6]
  \definition{s.}{ondas do mar}
\end{EntryWithPhonetic}

\begin{EntryWithPhonetic}{海里}{hai3li3}{10,7}{⽔,⾥}
  \definition{s.}{milha náutica}
\end{EntryWithPhonetic}

\begin{EntryWithPhonetic}{海量}{hai3liang4}{10,12}{⽔,⾥}[HSK 7-9]
  \definition{s.}{magnanimidade | alta tolerância ao álcool | cargas de; uma grande quantidade de}
\end{EntryWithPhonetic}

\begin{EntryWithPhonetic}{海绵}{hai3mian2}{10,11}{⽔,⽷}[HSK 7-9]
  \definition{s.}{espuma de borracha; espuma de plástico; esponja; um material poroso feito de borracha ou plástico que é elástico, como uma esponja | esponja marinha seca; poríferos; osso esponjoso; refere"-se especificamente ao esqueleto queratinoso das esponjas}
\end{EntryWithPhonetic}

\begin{EntryWithPhonetic}{海面}{hai3mian4}{10,9}{⽔,⾯}[HSK 7-9]
  \definition{s.}{nível do mar | superfície do mar}
\end{EntryWithPhonetic}

\begin{EntryWithPhonetic}{海内外}{hai3 nei4wai4}{10,4,5}{⽔,⼌,⼣}[HSK 7-9]
  \definition{s.}{em casa e no exterior | nacional e internacional}
\end{EntryWithPhonetic}

\begin{EntryWithPhonetic}{海鸥}{hai3'ou1}{10,9}{⽔,⿃}
  \definition{s.}{gaivota}
\end{EntryWithPhonetic}

\begin{EntryWithPhonetic}{海水}{hai3shui3}{10,4}{⽔,⽔}[HSK 4]
  \definition[把]{s.}{água do mar; salmoura}
\end{EntryWithPhonetic}

\begin{EntryWithPhonetic}{海滩}{hai3tan1}{10,13}{⽔,⽔}[HSK 7-9]
  \definition[个,片]{s.}{praia; praia com declive suave em direção ao mar}
\end{EntryWithPhonetic}

\begin{EntryWithPhonetic}{海棠}{hai3tang2}{10,12}{⽔,⽊}
  \definition{s.}{begônia}
\end{EntryWithPhonetic}

\begin{EntryWithPhonetic}{海外}{hai3wai4}{10,5}{⽔,⼣}[HSK 6]
  \definition[次]{s.}{fora das fronteiras nacionais; no exterior}
\end{EntryWithPhonetic}

\begin{EntryWithPhonetic}{海湾}{hai3wan1}{10,12}{⽔,⽔}[HSK 6]
  \definition{s.}{baía; golfo | lago}
\end{EntryWithPhonetic}

\begin{EntryWithPhonetic}{海峡}{hai3xia2}{10,9}{⽔,⼭}[HSK 7-9]
  \definition*{s.}{Estreito de Taiwan}
  \definition[个]{s.}{estreito; canal; um canal estreito que conecta dois oceanos entre duas massas de terra}
\end{EntryWithPhonetic}

\begin{EntryWithPhonetic}{海鲜}{hai3xian1}{10,14}{⽔,⿂}[HSK 4]
  \definition[种,份,桌,批,些]{s.}{frutos do mar; mariscos; peixes marinhos frescos, camarões, etc., para consumo}
\end{EntryWithPhonetic}

\begin{EntryWithPhonetic}{海啸}{hai3xiao4}{10,11}{⽔,⼝}[HSK 7-9]
  \definition{s.}{\emph{tsunami}; maremoto}
\end{EntryWithPhonetic}

\begin{EntryWithPhonetic}{海洋}{hai3yang2}{10,9}{⽔,⽔}[HSK 6]
  \definition[片,个]{s.}{mar; oceano; um termo geral para os mares e oceanos que formam uma entidade contínua na superfície da Terra; também pode ser usado para descrever um grande número de coisas semelhantes}
\end{EntryWithPhonetic}

\begin{EntryWithPhonetic}{海域}{hai3yu4}{10,11}{⽔,⼟}[HSK 7-9]
  \definition{s.}{área marítima; espaço marítimo; refere"-se a uma determinada área do oceano (tanto acima quanto abaixo da água)}
\end{EntryWithPhonetic}

\begin{EntryWithPhonetic}{海运}{hai3yun4}{10,7}{⽔,⾡}[HSK 7-9]
  \definition{s.}{transporte marítimo; transporte oceânico}
  \definition{v.}{transportar pelo mar}
\end{EntryWithPhonetic}

\begin{EntryWithPhonetic}{海藻}{hai3zao3}{10,19}{⽔,⾋}[HSK 7-9]
  \definition{s.}{alga marinha; planta marítima}
\end{EntryWithPhonetic}

%%%%%%%%%% 骇 %%%%%%%%%%
\subsection*{骇}\addcontentsline{loh}{figure}{骇 \dpy{hai4}}

\begin{EntryWithPhonetic}{骇}{hai4}{9}{⾺}
  \definition{adj.}{assustado; chocado}
  \definition{v.}{ficar surpreso; ficar chocado}
\end{EntryWithPhonetic}

\begin{EntryWithPhonetic}{骇人听闻}{hai4ren2ting1wen2}{9,2,7,9}{⾺,⼈,⼝,⾨}[HSK 7-9]
  \definition{expr.}{chocante; terrível; assustador para os ouvidos; espantoso; fabuloso; chocante (notícia); aterrorizante; horripilante}
\end{EntryWithPhonetic}

%%%%%%%%%% 害 %%%%%%%%%%
\subsection*{害}\addcontentsline{loh}{figure}{害 \dpy{hai4}}

\begin{EntryWithPhonetic}{害}{hai4}{10}{⼧}[HSK 5]
  \definition{adj.}{prejudicial; destrutivo; injurioso; nocivo}
  \definition{s.}{mal; maldade; dano; calamidade}
  \definition{v.}{prejudicar; fazer mal a; causar problemas a | matar; assassinar | sofrer de; contrair (uma doença) | sentir"-se (envergonhado, com medo, etc.); despertar (um sentimento ou uma emoção)}
\end{EntryWithPhonetic}

\begin{EntryWithPhonetic}{害虫}{hai4chong2}{10,6}{⼧,⾍}[HSK 7-9]
  \definition[种,只,个]{s.}{verme; bicho; inseto nocivo (ou destrutivo); praga}
  \antonymref{益虫}{yi4chong2}
\end{EntryWithPhonetic}

\begin{EntryWithPhonetic}{害怕}{hai4pa4}{10,8}{⼧,⼼}[HSK 3]
  \definition{v.}{estar assustado; ter medo; encontrar dificuldades, perigos, etc., e sentir"-se inquieto ou nervoso}
\end{EntryWithPhonetic}

\begin{EntryWithPhonetic}{害臊}{hai4/sao4}{10,17}{⼧,⾁}[HSK 7-9]
  \definition{v.+compl.}{sentir vergonha; ser tímido}
\end{EntryWithPhonetic}

\begin{EntryWithPhonetic}{害羞}{hai4/xiu1}{10,10}{⼧,⽺}[HSK 7-9]
  \definition{v.+compl.}{ser tímido; parecer tímido; tornar-se tímido}
\end{EntryWithPhonetic}

%%%%%%%%%% 酣 %%%%%%%%%%
\subsection*{酣}\addcontentsline{loh}{figure}{酣 \dpy{han1}}

\begin{EntryWithPhonetic}{酣}{han1}{12}{⾣}
  \definition{adj.}{intoxicado}
\end{EntryWithPhonetic}

\begin{EntryWithPhonetic}{酣畅}{han1chang4}{12,8}{⾣,⽥}[HSK 7-9]
  \definition{adj.}{alegre e animado (com bebida) | profundo (sono profundo)}
  \definition{adv.}{com facilidade e entusiasmo; totalmente; refere"-se a obras literárias e artísticas}
\end{EntryWithPhonetic}

\begin{EntryWithPhonetic}{酣睡}{han1shui4}{12,13}{⾣,⽬}[HSK 7-9]
  \definition{v.}{dormir profundamente; estar em sono profundo | estar profundamente adormecido; cair em sono profundo}
\end{EntryWithPhonetic}

%%%%%%%%%% 汗 %%%%%%%%%%
\subsection*{汗}\addcontentsline{loh}{figure}{汗 \dpy{han2}}

\begin{EntryWithPhonetic}{汗}{han2}{6}{⽔}
  \definition*{s.}{Abreviação de Khan}[他是成吉思汗。===Ele é Genghis Khan.]
  \seeref{han4}
\end{EntryWithPhonetic}

%%%%%%%%%% 含 %%%%%%%%%%
\subsection*{含}\addcontentsline{loh}{figure}{含 \dpy{han2}}

\begin{EntryWithPhonetic}{含}{han2}{7}{⼝}[HSK 4]
  \definition{v.}{manter na boca (sem engolir ou cuspir) | conter; incluir | cuidar; acalentar; abrigar}
\end{EntryWithPhonetic}

\begin{EntryWithPhonetic}{含糊}{han2hu5}{7,15}{⼝,⽶}[HSK 7-9]
  \definition{adj.}{(atitude, palavras, etc.) vago; ambíguo; pouco claro | (falar, fazer coisas, etc.) descuidado; desleixado (usado principalmente em termos negativos) | covarde; demonstrando fraqueza (usado principalmente em sentido negativo)}
\end{EntryWithPhonetic}

\begin{EntryWithPhonetic}{含金量}{han2jin1liang4}{7,8,12}{⼝,⾦,⾥}
  \definition{adj.}{conteúdo de ouro | (fig.) valioso}
\end{EntryWithPhonetic}

\begin{EntryWithPhonetic}{含量}{han2liang4}{7,12}{⼝,⾥}[HSK 4]
  \definition{s.}{conteúdo; a quantidade de um componente contido em uma substância}
\end{EntryWithPhonetic}

\begin{EntryWithPhonetic}{含蓄}{han2xu4}{7,13}{⼝,⾋}[HSK 7-9]
  \definition{v.}{conter; incorporar}
\end{EntryWithPhonetic}

\begin{EntryWithPhonetic}{含义}{han2yi4}{7,3}{⼝,⼂}[HSK 4]
  \definition[个,种,层]{s.}{sentido; mensagem; significado; implicação; o significado contido em (palavras, frases, sentenças e discursos)}
\end{EntryWithPhonetic}

\begin{EntryWithPhonetic}{含有}{han2you3}{7,6}{⼝,⽉}[HSK 4]
  \definition{v.}{conter; ter; incluir}
\end{EntryWithPhonetic}

%%%%%%%%%% 函 %%%%%%%%%%
\subsection*{函}\addcontentsline{loh}{figure}{函 \dpy{han2}}

\begin{EntryWithPhonetic}{函}{han2}{8}{⼐}
  \definition*{s.}{Sobrenome: Han}
  \definition[封]{s.}{caixa; envelope; capa | carta}
\end{EntryWithPhonetic}

\begin{EntryWithPhonetic}{函授}{han2shou4}{8,11}{⼐,⼿}[HSK 7-9]
  \definition{v.}{ensinar por correspondência; utilizar principalmente tutoria por correspondência para ministrar cursos}
\end{EntryWithPhonetic}

\begin{EntryWithPhonetic}{函数}{han2shu4}{8,13}{⼐,⽁}
  \definition[个]{s.}{Matemática: função; em um determinado processo, duas variáveis $x$ e $y$ têm um certo valor de $y$ correspondente a cada valor de $x$ dentro de um determinado intervalo, $y$ é uma função de $x$; essa relação geralmente é expressa como $y = f(x)$}
\end{EntryWithPhonetic}

%%%%%%%%%% 涵 %%%%%%%%%%
\subsection*{涵}\addcontentsline{loh}{figure}{涵 \dpy{han2}}

\begin{EntryWithPhonetic}{涵}{han2}{11}{⽔}
  \definition{s.}{bueiro; galeria}
  \definition{v.}{conter; incorporar}
\end{EntryWithPhonetic}

\begin{EntryWithPhonetic}{涵盖}{han2gai4}{11,11}{⽔,⽫}[HSK 7-9]
  \definition{v.}{cobrir; incluir; conter; conter completamente}
\end{EntryWithPhonetic}

\begin{EntryWithPhonetic}{涵义}{han2yi4}{11,3}{⽔,⼂}[HSK 7-9]
  \definition[层,种]{s.}{significado; implicação | conotação | conteúdo}
\end{EntryWithPhonetic}

%%%%%%%%%% 寒 %%%%%%%%%%
\subsection*{寒}\addcontentsline{loh}{figure}{寒 \dpy{han2}}

\begin{EntryWithPhonetic}{寒}{han2}{12}{⼧}
  \definition*{s.}{Sobrenome: Han}
  \definition{adj.}{frio | pobre; necessitado | (autodepreciativo) meu/minha humilde\dots | assustado; medroso | com medo; tremendo (de medo) | humilde}
  \definition{s.}{estação fria; inverno | (medicina chinesa) sintomas causados por fatores frios}
  \antonymref{暑}{shu3}
\end{EntryWithPhonetic}

\begin{EntryWithPhonetic}{寒假}{han2jia4}{12,11}{⼧,⼈}[HSK 4]
  \definition[个,段]{s.}{férias de inverno (feriados); férias escolares no meio do inverno, em janeiro e fevereiro (na China)}
\end{EntryWithPhonetic}

\begin{EntryWithPhonetic}{寒冷}{han2leng3}{12,7}{⼧,⼎}[HSK 4]
  \definition[度,阵,股]{adj.}{frio; frígido; gélido; gelado}
\end{EntryWithPhonetic}

%%%%%%%%%% 韩 %%%%%%%%%%
\subsection*{韩}\addcontentsline{loh}{figure}{韩 \dpy{han2}}

\begin{EntryWithPhonetic}{韩}{han2}{12}{⾱}
  \definition*{s.}{Um estado durante o Período dos Estados Combatentes nas atuais províncias centrais de Henan e sudeste de Shanxi | O nome de um estado feudal durante a dinastia Zhou, localizado no que hoje é o nordeste de Hejin, província de Shanxi | Coreia do Sul, abreviação de 韩国; República da Coreia (RC) | Sobrenome: Han}
  \seealsoref{韩国}{han2guo2}
\end{EntryWithPhonetic}

\begin{EntryWithPhonetic}{韩国}{han2guo2}{12,8}{⾱,⼞}
  \definition*{s.}{Coréia do Sul; República da Coreia}
\end{EntryWithPhonetic}

\begin{EntryWithPhonetic}{韩国人}{han2guo2ren2}{12,8,2}{⾱,⼞,⼈}
  \definition{s.}{coreano | pessoa ou povo da Coréia}
\end{EntryWithPhonetic}

%%%%%%%%%% 厂 %%%%%%%%%%
\subsection*{厂}\addcontentsline{loh}{figure}{厂 \dpy{han3}}

\begin{EntryWithPhonetic}{厂}{han3}{2}{⼚}[Kangxi 27]
  \definition[家,间]{s.}{radical ``penhasco'' em caracteres chineses}
  \seeref{an1}
  \seeref{chang3}
\end{EntryWithPhonetic}

%%%%%%%%%% 罕 %%%%%%%%%%
\subsection*{罕}\addcontentsline{loh}{figure}{罕 \dpy{han3}}

\begin{EntryWithPhonetic}{罕}{han3}{7}{⽹}
  \definition*{s.}{Sobrenome: Han}
  \definition{adj.}{raro; escasso | raro; incomum}
\end{EntryWithPhonetic}

\begin{EntryWithPhonetic}{罕见}{han3jian4}{7,4}{⽹,⾒}[HSK 7-9]
  \definition{adj.}{raro; raramente visto}
\end{EntryWithPhonetic}

%%%%%%%%%% 喊 %%%%%%%%%%
\subsection*{喊}\addcontentsline{loh}{figure}{喊 \dpy{han3}}

\begin{EntryWithPhonetic}{喊}{han3}{12}{⼝}[HSK 2]
  \definition{v.}{gritar; clamar; berrar | chamar (uma pessoa) | chamar; dirigir-se a}
\end{EntryWithPhonetic}

%%%%%%%%%% 汉 %%%%%%%%%%
\subsection*{汉}\addcontentsline{loh}{figure}{汉 \dpy{han4}}

\begin{EntryWithPhonetic}{汉}{han4}{5}{⽔}
  \definition*{s.}{Dinastia Han (206 a.C.-220 d.C.)  | Astronomia: A Via Láctea | Sobrenome: Han}
  \definition{s.}{grupo étnico Han | chinês (língua) | homem}
\end{EntryWithPhonetic}

\begin{EntryWithPhonetic}{汉堡包}{han4bao3bao1}{5,12,5}{⽔,⼟,⼓}
  \definition[个]{s.}{hambúrguer}
\end{EntryWithPhonetic}

\begin{EntryWithPhonetic}{汉堡王}{han4bao3wang2}{5,12,4}{⽔,⼟,⽟}
  \definition*{s.}{Burguer King, restaurante de \emph{fast-food}}
\end{EntryWithPhonetic}

\begin{EntryWithPhonetic}{汉服}{han4fu2}{5,8}{⽔,⽉}
  \definition{s.}{vestido chinês tradicional Han}
\end{EntryWithPhonetic}

\begin{EntryWithPhonetic}{汉葡词典}{han4-pu2 ci2dian3}{5,12,7,8}{⽔,⾋,⾔,⼋}
  \definition[部,本]{s.}{dicionário chinês-português}
  \seealsoref{葡汉词典}{pu2-han4 ci2dian3}
\end{EntryWithPhonetic}

\begin{EntryWithPhonetic}{汉语}{han4yu3}{5,9}{⽔,⾔}[HSK 1]
  \definition[门]{s.}{língua chinesa, mandarim}
\end{EntryWithPhonetic}

\begin{EntryWithPhonetic}{汉字}{han4zi4}{5,6}{⽔,⼦}[HSK 1]
  \definition[个]{s.}{caractere chinês; ideograma chinês; sinograma; com pouquíssimas exceções, os caracteres chineses representam uma sílaba cada um}
\end{EntryWithPhonetic}

%%%%%%%%%% 汗 %%%%%%%%%%
\subsection*{汗}\addcontentsline{loh}{figure}{汗 \dpy{han4}}

\begin{EntryWithPhonetic}{汗}{han4}{6}{⽔}[HSK 5]
  \definition{s.}{suor; transpiração; perspiração}
  \seeref{han2}
\end{EntryWithPhonetic}

\begin{EntryWithPhonetic}{汗水}{han4shui3}{6,4}{⽔,⽔}[HSK 7-9]
  \definition{s.}{transpiração; suor (em grandes quantidades)}
\end{EntryWithPhonetic}

\begin{EntryWithPhonetic}{汗腺}{han4xian4}{6,13}{⽔,⾁}
  \definition{s.}{glândula sudorípara}
\end{EntryWithPhonetic}

\begin{EntryWithPhonetic}{汗液}{han4ye4}{6,11}{⽔,⽔}
  \definition{s.}{suor}
\end{EntryWithPhonetic}

%%%%%%%%%% 旱 %%%%%%%%%%
\subsection*{旱}\addcontentsline{loh}{figure}{旱 \dpy{han4}}

\begin{EntryWithPhonetic}{旱}{han4}{7}{⽇}[HSK 7-9]
  \definition{adj.}{atingido pela seca | seco; árido | em terra; terrestre}
  \definition{s.}{período de seca; nenhuma precipitação ou precipitação muito baixa; seca | rota terrestre (ou comunicação) | terra firme}
  \antonymref{涝}{lao4}
\end{EntryWithPhonetic}

\begin{EntryWithPhonetic}{旱灾}{han4zai1}{7,7}{⽇,⽕}[HSK 7-9]
  \definition[场]{s.}{seca; desastres causados por secas prolongadas e escassez de água que resultam na morte de colheitas ou redução significativa da produção}
  \antonymref{水灾}{shui3zai1}
\end{EntryWithPhonetic}

%%%%%%%%%% 捍 %%%%%%%%%%
\subsection*{捍}\addcontentsline{loh}{figure}{捍 \dpy{han4}}

\begin{EntryWithPhonetic}{捍}{han4}{10}{⼿}
  \definition{v.}{defender; guardar | defender-se | afastar (um golpe) | resistir}
\end{EntryWithPhonetic}

\begin{EntryWithPhonetic}{捍卫}{han4wei4}{10,3}{⼿,⼙}[HSK 7-9]
  \definition{v.}{defender; guardar; proteger; defender-se pela força ou outros meios de ser violado ou prejudicado}
\end{EntryWithPhonetic}

%%%%%%%%%% 焊 %%%%%%%%%%
\subsection*{焊}\addcontentsline{loh}{figure}{焊 \dpy{han4}}

\begin{EntryWithPhonetic}{焊}{han4}{11}{⽕}[HSK 7-9]
  \definition{v.}{soldar; usar metal fundido para reparar objetos de metal ou conectar peças de metal}
\end{EntryWithPhonetic}

%%%%%%%%%% 撼 %%%%%%%%%%
\subsection*{撼}\addcontentsline{loh}{figure}{撼 \dpy{han4}}

\begin{EntryWithPhonetic}{撼}{han4}{16}{⼿}
  \definition{v.}{agitar; sacudir}
\end{EntryWithPhonetic}

%%%%%%%%%% 行 %%%%%%%%%%
\subsection*{行}\addcontentsline{loh}{figure}{行 \dpy{hang2}}

\begin{EntryWithPhonetic}{行}{hang2}{6}{⾏}[HSK 3][Kangxi 144]
  \definition{adj.}{temporário; improvisado | capaz; competente}
  \definition{adv.}{logo; em breve}
  \definition{clas.}{linha; fileira; coisas usadas para formar filas, linhas}
  \definition{s.}{comportamento; conduta | linha; fileira | empresa comercial; certas instituições comerciais | comércio; profissão; ramo de atividade | especialista; conhecedor; refere"-se ao conhecimento e experiência em um determinado setor}
  \definition{v.}{ir; caminhar; viajar | estar atualizado; circular | fazer; executar; realizar | (antes de um verbo dissílabo, indicando a realização de alguma ação) | ficar bem; vai dar certo | (remédio) fazer efeito | classificar (entre irmãos e irmãs por ordem de idade)}
  \seeref{heng2}
  \seeref{xing2}
\end{EntryWithPhonetic}

\begin{EntryWithPhonetic}{行家}{hang2jia5}{6,10}{⾏,⼧}[HSK 7-9]
  \definition[位,名,个,些]{s.}{especialista; \emph{expert}; conhecedor; \emph{connoisseur}}
\end{EntryWithPhonetic}

\begin{EntryWithPhonetic}{行列}{hang2lie4}{6,6}{⾏,⼑}[HSK 7-9]
  \definition{s.}{fileiras}
\end{EntryWithPhonetic}

\begin{EntryWithPhonetic}{行情}{hang2qing2}{6,11}{⾏,⼼}[HSK 7-9]
  \definition{s.}{preço; cotações de mercado; o preço geral dos bens no mercado também se refere à situação geral das taxas de juros, taxas de câmbio, preços de títulos, etc. no mercado financeiro}
\end{EntryWithPhonetic}

\begin{EntryWithPhonetic}{行业}{hang2ye4}{6,5}{⾏,⼀}[HSK 4]
  \definition[种,个]{s.}{comércio; indústria; setor; profissão; categorias em negócios e indústria referem-se a ocupações em geral}
\end{EntryWithPhonetic}

%%%%%%%%%% 航 %%%%%%%%%%
\subsection*{航}\addcontentsline{loh}{figure}{航 \dpy{hang2}}

\begin{EntryWithPhonetic}{航}{hang2}{10}{⾈}
  \definition*{s.}{Sobrenome: Hang}
  \definition[趟]{s.}{barco; navio}
  \definition{v.}{navegar (por água ou ar) | velejar}
\end{EntryWithPhonetic}

\begin{EntryWithPhonetic}{航班}{hang2ban1}{10,10}{⾈,⽟}[HSK 4]
  \definition[个,次]{s.}{número do voo; voo programado; o horário de um navio ou avião de passageiros}
\end{EntryWithPhonetic}

\begin{EntryWithPhonetic}{航海}{hang2hai3}{10,10}{⾈,⽔}[HSK 7-9]
  \definition{v.}{velejar; navegar}
\end{EntryWithPhonetic}

\begin{EntryWithPhonetic}{航空}{hang2kong1}{10,8}{⾈,⽳}[HSK 4]
  \definition{s.}{viagem; aviação; refere"-se ao voo de uma aeronave no ar}
\end{EntryWithPhonetic}

\begin{EntryWithPhonetic}{航天}{hang2tian1}{10,4}{⾈,⼤}[HSK 7-9]
  \definition{s.}{voo espacial; astronáutica}
  \definition{v.}{voar ou viajar no espaço}
\end{EntryWithPhonetic}

\begin{EntryWithPhonetic}{航天员}{hang2tian1yuan2}{10,4,7}{⾈,⼤,⼝}[HSK 7-9]
  \definition[名,位,个]{s.}{astronauta}
\end{EntryWithPhonetic}

\begin{EntryWithPhonetic}{航行}{hang2xing2}{10,6}{⾈,⾏}[HSK 7-9]
  \definition{v.}{velejar; voar; navegar pela água, pelo ar}
\end{EntryWithPhonetic}

\begin{EntryWithPhonetic}{航运}{hang2yun4}{10,7}{⾈,⾡}[HSK 7-9]
  \definition{s.}{transporte hidroviário; transporte marítimo}
\end{EntryWithPhonetic}

%%%%%%%%%% 号 %%%%%%%%%%
\subsection*{号}\addcontentsline{loh}{figure}{号 \dpy{hao2}}

\begin{EntryWithPhonetic}{号}{hao2}{5}{⼝}
  \definition{v.}{uivar; gritar; gritar em voz alta e prolongada | lamentar; chorar alto | uivar; (vento) assobiar, assoviar}
  \seeref{hao4}
\end{EntryWithPhonetic}

%%%%%%%%%% 蚝 %%%%%%%%%%
\subsection*{蚝}\addcontentsline{loh}{figure}{蚝 \dpy{hao2}}

\begin{EntryWithPhonetic}{蚝}{hao2}{10}{⾍}
  \definition[只]{s.}{ostra}
\end{EntryWithPhonetic}

%%%%%%%%%% 毫 %%%%%%%%%%
\subsection*{毫}\addcontentsline{loh}{figure}{毫 \dpy{hao2}}

\begin{EntryWithPhonetic}{毫}{hao2}{11}{⽊}
  \definition{adv.}{nem um pouco; absolutamente nenhum; completamente sem}
  \definition{clas.}{hao, uma unidade de comprimento igual a um milésimo de polegada ou 1/30 de milímetro | hao, uma unidade de peso igual a um milésimo de um centavo ou 0,005 grama |
uma fração minúscula; uma parte muito pequena}
  \definition{pref.}{mili-, usado com a unidade de uma quantidade física para representar um milésimo dessa quantidade}
  \definition{s.}{cabelo longo e fino | pincel de escrita | uma das duas ou três alças de uma balança para pendurar na mão do usuário | cerda; uma corda de mão em uma balança ou equilíbrio | fio de cabelo}
\end{EntryWithPhonetic}

\begin{EntryWithPhonetic}{毫不费力}{hao2bu2fei4li4}{11,4,9,2}{⽊,⼀,⾙,⼒}
  \definition{expr.}{sem esforço; não despender o menor esforço}
\end{EntryWithPhonetic}

\begin{EntryWithPhonetic}{毫不}{hao2 bu4}{11,4}{⽊,⼀}[HSK 7-9]
  \definition{adv.}{dificilmente; de jeito nenhum; nem um pouco}
\end{EntryWithPhonetic}

\begin{EntryWithPhonetic}{毫不犹豫}{hao2 bu4 you2yu4}{11,4,7,15}{⽊,⼀,⽝,⾗}[HSK 7-9]
  \definition{expr.}{sem hesitação; sem a menor hesitação}
\end{EntryWithPhonetic}

\begin{EntryWithPhonetic}{毫米}{hao2mi3}{11,6}{⽊,⽶}[HSK 4]
  \definition{clas.}{milímetro; unidade legal de medida de comprimento, 1 mm equivale a 0,1 cm}
\end{EntryWithPhonetic}

\begin{EntryWithPhonetic}{毫升}{hao2sheng1}{11,4}{⽊,⼗}[HSK 4]
  \definition{clas.}{mililitro; unidade de volume, milésimo de um litro (ml)}
\end{EntryWithPhonetic}

\begin{EntryWithPhonetic}{毫无}{hao2wu2}{11,4}{⽊,⽆}[HSK 7-9]
  \definition{adv.}{não; nada; de jeito nenhum}
\end{EntryWithPhonetic}

%%%%%%%%%% 豪 %%%%%%%%%%
\subsection*{豪}\addcontentsline{loh}{figure}{豪 \dpy{hao2}}

\begin{EntryWithPhonetic}{豪}{hao2}{14}{⾗}
  \definition*{s.}{Sobrenome: Hao}
  \definition{adj.}{direto; irrestrito; ousado | despótico; intimidador | rico e poderoso}
  \definition{s.}{pessoa com poderes ou dons extraordinários}
\end{EntryWithPhonetic}

\begin{EntryWithPhonetic}{豪华}{hao2hua2}{14,6}{⾗,⼗}[HSK 7-9]
  \definition{adj.}{luxo; luxuoso; (edifício, equipamento ou decoração) magnífico; muito lindo}
\end{EntryWithPhonetic}

%%%%%%%%%% 好 %%%%%%%%%%
\subsection*{好}\addcontentsline{loh}{figure}{好 \dpy{hao3}}

\begin{EntryWithPhonetic}{好}{hao3}{6}{⼥}[HSK 1,2,4]
  \definition{adj.}{bom; ótimo; agradável; vantajoso; satisfatório | amigável; gentil; amistoso; amável | saudável; bem | pronto; concluído; usado após um verbo para indicar conclusão ou perfeição | fácil (de fazer); conveniente; responsável (por)}
  \definition{adv.}{muito; bastante; tão; usado na frente de uma palavra de quantidade ou uma palavra de tempo para indicar muito ou por muito tempo | em que medida; como; usado antes de adjetivos e verbos para indicar profundidade e com exclamação}
  \definition{interj.}{``O.K.!''; ``Tudo bem!''; aprovação, acordo ou encerramento | (no início de uma frase ou oração) expressa concordância (ou desaprovação, surpresa, etc.)}
  \definition{prep.}{de modo a; para que}
  \definition{s.}{referindo"-se a palavras de elogio ou aplauso | saudações; cumprimentos}
  \definition{suf.}{sufixo que indica conclusão ou prontidão | depois de um pronome significa ``olá''}
  \definition{v.}{deve; precisa; tem que; deveria | apaixonar"-se}
  \seeref{hao4}
\end{EntryWithPhonetic}

\begin{EntryWithPhonetic}{好比}{hao3bi3}{6,4}{⼥,⽐}[HSK 7-9]
  \definition{v.}{pode ser comparado a; ser exatamente como}
\end{EntryWithPhonetic}

\begin{EntryWithPhonetic}{好(不)容易}{hao3 bu4 rong2 yi4}{6,4,10,8}{⼥,⼀,⼧,⽇}
  \definition{adv.}{com grande dificuldade; muito difícil}
  \definition{v.}{ter dificuldade (em fazer algo)}
\end{EntryWithPhonetic}

\begin{EntryWithPhonetic}{好吃}{hao3chi1}{6,6}{⼥,⼝}[HSK 1]
  \definition{adj.}{bom; saboroso; delicioso; descreve o sabor agradável de algo, que as pessoas gostam de comer}
  \seeref{hao4chi1}
\end{EntryWithPhonetic}

\begin{EntryWithPhonetic}{好处}{hao3chu5}{6,5}{⼥,⼡}[HSK 2]
  \definition[个]{s.}{bom; benefício; vantagem; fatores favoráveis a pessoas ou coisas | ganho; lucro; algo que não se deveria receber, dado por outra pessoa ou obtido através de uma oportunidade; geralmente tem conotação pejorativa}
\end{EntryWithPhonetic}

\begin{EntryWithPhonetic}{好歹}{hao3dai3}{6,4}{⼥,⽍}[HSK 7-9]
  \definition{adv.}{de qualquer forma; em qualquer caso | de alguma forma; não importa de que maneira; não importa o que}
  \definition{s.}{bom e mau; o que é bom e o que é mau | acidente; desastre; refere"-se a situações de risco de vida}
\end{EntryWithPhonetic}

\begin{EntryWithPhonetic}{好多}{hao3duo1}{6,6}{⼥,⼣}[HSK 2]
  \definition{adj.}{muitos; uma boa quantidade; uma grande quantidade; uma quantidade enorme}
  \definition{pron.}{quantos?; quanto?; frequentemente usado para perguntar sobre quantidade}
\end{EntryWithPhonetic}

\begin{EntryWithPhonetic}{好感}{hao3gan3}{6,13}{⼥,⼼}[HSK 7-9]
  \definition{s.}{boa opinião; impressão favorável; sentimentos de satisfação ou simpatia por pessoas ou coisas}
\end{EntryWithPhonetic}

\begin{EntryWithPhonetic}{好汉}{hao3han4}{6,5}{⼥,⽔}
  \definition[条]{s.}{herói | pessoa forte e corajosa}
\end{EntryWithPhonetic}

\begin{EntryWithPhonetic}{好好}{hao3hao3}{6,6}{⼥,⼥}[HSK 3]
  \definition{adj.}{realmente bom/bem; em perfeitas condições; quando tudo está bem}
  \definition{adv.}{diretamente; seriamente; cuidadosamente; com todo o empenho; ao máximo}
\end{EntryWithPhonetic}

\begin{EntryWithPhonetic}{好坏}{hao3huai4}{6,7}{⼥,⼟}[HSK 7-9]
  \definition{s.}{bom e mau; o que é bom e o que é mau | bom ou ruim | qualidade |padrão}
\end{EntryWithPhonetic}

\begin{EntryWithPhonetic}{好家伙}{hao3jia1huo5}{6,10,6}{⼥,⼧,⼈}[HSK 7-9]
  \definition[个]{interj.}{``Bom Deus!''; ``Céus!''; ``Bom Senhor!''; expressa surpresa ou admiração}
\end{EntryWithPhonetic}

\begin{EntryWithPhonetic}{好久}{hao3jiu3}{6,3}{⼥,⼃}[HSK 2]
  \definition{adv.}{por muito tempo | por eras (no passado)}
\end{EntryWithPhonetic}

\begin{EntryWithPhonetic}{好看}{hao3kan4}{6,9}{⼥,⽬}[HSK 1]
  \definition{adj.}{de boa aparência; agradável; bonito | interessante; descreve o enredo ou conteúdo de filmes, romances, performances, etc., como sendo cativante, agradável ou apreciável}
\end{EntryWithPhonetic}

\begin{EntryWithPhonetic}{好评}{hao3ping2}{6,7}{⼥,⾔}[HSK 7-9]
  \definition{s.}{comentário favorável; opinião elevada; boas críticas; altas críticas}
\end{EntryWithPhonetic}

\begin{EntryWithPhonetic}{好人}{hao3ren2}{6,2}{⼥,⼈}[HSK 2]
  \definition[个,位,名]{s.}{pessoa boa (ou excelente) | pessoa saudável | pessoa gentil que tenta se dar bem com todos (muitas vezes em detrimento dos princípios)}
  \antonymref{坏人}{huai4ren2}
\end{EntryWithPhonetic}

\begin{EntryWithPhonetic}{好容易}{hao3rong2yi4}{6,10,8}{⼥,⼧,⽇}[HSK 6]
  \definition{adv.}{com grande dificuldade; ter muita dificuldade (em fazer algo)}
\end{EntryWithPhonetic}

\begin{EntryWithPhonetic}{好生}{hao3sheng1}{6,5}{⼥,⽣}
  \definition{adv.}{bastante; extremamente | cuidadosamente; apropriadamente}
\end{EntryWithPhonetic}

\begin{EntryWithPhonetic}{好事}{hao3shi4}{6,8}{⼥,⼅}[HSK 2]
  \definition[个,件]{s.}{boa ação; gentileza | (antigo) obra de caridade | acontecimento feliz; evento festivo}
  \seeref{hao4shi4}
\end{EntryWithPhonetic}

\begin{EntryWithPhonetic}{好说}{hao3shuo1}{6,9}{⼥,⾔}[HSK 7-9]
  \definition{adj.}{palavras elogiosas; usadas quando alguém agradece ou elogia você; usadas para expressar que você não é digno do elogio | sem problemas; expressa concordância ou vontade de negociar}
\end{EntryWithPhonetic}

\begin{EntryWithPhonetic}{好似}{hao3si4}{6,6}{⼥,⼈}[HSK 6]
  \definition{v.}{parecer; ser como}
\end{EntryWithPhonetic}

\begin{EntryWithPhonetic}{好听}{hao3ting1}{6,7}{⼥,⼝}[HSK 1]
  \definition{adj.}{agradável de ouvir (de som ou voz) | bom; palatável; satisfatório (de palavras)  | decente; honrado (de ações, etc.); descreve uma coisa que parece prestigiosa | interessante; descreve palavras, histórias e outras coisas interessantes}
\end{EntryWithPhonetic}

\begin{EntryWithPhonetic}{好玩儿}{hao3wan2r5}{6,8,2}{⼥,⽟,⼉}[HSK 1]
  \definition{adj.}{divertido; interessante; capaz de despertar interesse}
\end{EntryWithPhonetic}

\begin{EntryWithPhonetic}{好象}{hao3xiang4}{6,11}{⼥,⾗}
  \variantof{好像}
\end{EntryWithPhonetic}

\begin{EntryWithPhonetic}{好像}{hao3xiang4}{6,13}{⼥,⼈}[HSK 2]
  \definition{adv.}{como se; um pouco parecido; como se fosse}
  \definition{v.}{parecer; ser como; parecer-se com}
\end{EntryWithPhonetic}

\begin{EntryWithPhonetic}{好笑}{hao3xiao4}{6,10}{⼥,⽵}[HSK 7-9]
  \definition{adj.}{engraçado; divertido; ridículo}
\end{EntryWithPhonetic}

\begin{EntryWithPhonetic}{好心}{hao3xin1}{6,4}{⼥,⼼}[HSK 7-9]
  \definition{adj./s.}{bondade; boas intenções}
\end{EntryWithPhonetic}

\begin{EntryWithPhonetic}{好心人}{hao3xin1ren2}{6,4,2}{⼥,⼼,⼈}[HSK 7-9]
  \definition{s.}{boa alma; pessoa de bom coração | pessoa gentil}
\end{EntryWithPhonetic}

\begin{EntryWithPhonetic}{好学}{hao3xue2}{6,8}{⼥,⼦}
  \definition{adj.}{fácil de aprender}
  \seeref{hao4xue2}
\end{EntryWithPhonetic}

\begin{EntryWithPhonetic}{好意}{hao3yi4}{6,13}{⼥,⼼}[HSK 7-9]
  \definition{s.}{boas intenções; gentileza}
\end{EntryWithPhonetic}

\begin{EntryWithPhonetic}{好用}{hao3yong4}{6,5}{⼥,⽤}
  \definition{adj.}{fácil de usar | adequado ao uso}
\end{EntryWithPhonetic}

\begin{EntryWithPhonetic}{好友}{hao3you3}{6,4}{⼥,⼜}[HSK 4]
  \definition[位,名,个,些]{s.}{bom amigo; amigo próximo}
\end{EntryWithPhonetic}

\begin{EntryWithPhonetic}{好运}{hao3yun4}{6,7}{⼥,⾡}[HSK 5]
  \definition{s.}{boa sorte, fortuna ou oportunidade}
\end{EntryWithPhonetic}

\begin{EntryWithPhonetic}{好在}{hao3zai4}{6,6}{⼥,⼟}[HSK 7-9]
  \definition{adv.}{felizmente; afortunadamente; indica que existem fatores favoráveis em condições difíceis ou desfavoráveis}
\end{EntryWithPhonetic}

\begin{EntryWithPhonetic}{好转}{hao3zhuan3}{6,8}{⼥,⾞}[HSK 6]
  \definition{v.}{melhorar; dar uma guinada para melhor; tomar um rumo favorável}
\end{EntryWithPhonetic}

%%%%%%%%%% 号 %%%%%%%%%%
\subsection*{号}\addcontentsline{loh}{figure}{号 \dpy{hao4}}

\begin{EntryWithPhonetic}{号}{hao4}{5}{⼝}[HSK 1]
  \definition{clas.}{usado para o número de pessoas |  tipo; espécie; classificação | usado para pessoas ou negócios; número de vezes utilizado para transações}
  \definition[把]{s.}{nome | nome presumido; nome alternativo; pseudônimo; apelido | casa de negócios; loja | marca; sinal; sinalização | número | data | ordem; no exército, as ordens são transmitidas verbalmente ou por meio de clarins | qualquer instrumento de sopro e latão; trombeta usada no exército ou em bandas | qualquer coisa usada como buzina | chamada de corneta; qualquer chamada feita em uma corneta; usar um apito para emitir um som com um significado específico | pessoa em uma condição especial; pessoas que se encontram em uma situação especial}
  \definition{suf.}{sufixo de navio}
  \definition{v.}{marcar; fazer uma marca | sentir; colocar a mão no pulso do paciente e avaliar a situação através do fluxo sanguíneo}
  \seeref{hao2}
\end{EntryWithPhonetic}

\begin{EntryWithPhonetic}{号称}{hao4cheng1}{5,10}{⼝,⽲}[HSK 7-9]
  \definition{v.}{ser conhecido como; ser conhecido por um certo nome | afirmar ser; alegar}
\end{EntryWithPhonetic}

\begin{EntryWithPhonetic}{号角}{hao4jiao3}{5,7}{⼝,⾓}
  \definition{s.}{corneta | trombeta}
\end{EntryWithPhonetic}

\begin{EntryWithPhonetic}{号码}{hao4ma3}{5,8}{⼝,⽯}[HSK 4]
  \definition[个,组,串]{s.}{número}
\end{EntryWithPhonetic}

\begin{EntryWithPhonetic}{号召}{hao4zhao4}{5,5}{⼝,⼝}[HSK 5]
  \definition{s.}{chamado; apelo; desejo ou pedido solene (de um governo, partido político, organização etc.) para que as massas façam algo}
  \definition{v.}{chamar;  (governo, partido político, organização, etc.) fazer um pedido solene às massas para que façam algo, na esperança de que todos se esforcem para alcançá-lo}
\end{EntryWithPhonetic}

%%%%%%%%%% 好 %%%%%%%%%%
\subsection*{好}\addcontentsline{loh}{figure}{好 \dpy{hao4}}

\begin{EntryWithPhonetic}{好}{hao4}{6}{⼥}[HSK 4]
  \definition*{s.}{Sobrenome: Hao}
  \definition{adv.}{algo que acontece com frequência, que é fácil de acontecer}
  \definition{v.}{gostar; amar; ter afeição por}
  \seeref{hao3}
\end{EntryWithPhonetic}

\begin{EntryWithPhonetic}{好吃}{hao4chi1}{6,6}{⼥,⼝}
  \definition{v.}{ser guloso; gostar de comer boa comida}
  \seeref{hao3chi1}
\end{EntryWithPhonetic}

\begin{EntryWithPhonetic}{好客}{hao4ke4}{6,9}{⼥,⼧}[HSK 7-9]
  \definition{adj.}{hospitaleiro; refere"-se a estar disposto a receber convidados e ser afetuoso com eles}
\end{EntryWithPhonetic}

\begin{EntryWithPhonetic}{好奇}{hao4qi2}{6,8}{⼥,⼤}[HSK 3]
  \definition{adj.}{curioso; curiosidade e interesse por coisas não conhecidas}
  \definition{s.}{curiosidade}
  \definition{v.}{ser ou estar curioso}
\end{EntryWithPhonetic}

\begin{EntryWithPhonetic}{好奇心}{hao4qi2xin1}{6,8,4}{⼥,⼤,⼼}[HSK 7-9]
  \definition{s.}{curiosidade; uma emoção que expressa atenção especial a algo}
\end{EntryWithPhonetic}

\begin{EntryWithPhonetic}{好事}{hao4shi4}{6,8}{⼥,⼅}
  \definition[个,件]{s.}{intrometido; gostar de se meter na vida dos outros}
  \seeref{hao3shi4}
\end{EntryWithPhonetic}

\begin{EntryWithPhonetic}{好学}{hao4xue2}{6,8}{⼥,⼦}[HSK 6]
  \definition[个]{s.}{apaixonado para aprender; estudioso; erudito}
  \seeref{hao3xue2}
\end{EntryWithPhonetic}

%%%%%%%%%% 浩 %%%%%%%%%%
\subsection*{浩}\addcontentsline{loh}{figure}{浩 \dpy{hao4}}

\begin{EntryWithPhonetic}{浩}{hao4}{10}{⽔}
  \definition*{s.}{Sobrenome: Hao}
  \definition{adj.}{grande; vasto; grandioso; sem limites | um grande número; infinito}
\end{EntryWithPhonetic}

\begin{EntryWithPhonetic}{浩劫}{hao4jie2}{10,7}{⽔,⼒}[HSK 7-9]
  \definition[场,次]{s.}{grande calamidade; catástrofe | devastação; holocausto; flagelo}
\end{EntryWithPhonetic}

%%%%%%%%%% 耗 %%%%%%%%%%
\subsection*{耗}\addcontentsline{loh}{figure}{耗 \dpy{hao4}}

\begin{EntryWithPhonetic}{耗}{hao4}{10}{⽾}[HSK 7-9]
  \definition{s.}{más notícias}[听到噩耗,他碎心裂胆。===Ele ficou arrasado ao ouvir as más notícias.]
  \definition{v.}{consumir; custar | perder tempo; procrastinar}
\end{EntryWithPhonetic}

\begin{EntryWithPhonetic}{耗费}{hao4fei4}{10,9}{⽾,⾙}[HSK 7-9]
  \definition{v.}{gastar; consumir; esgotar}
\end{EntryWithPhonetic}

\begin{EntryWithPhonetic}{耗时}{hao4shi2}{10,7}{⽾,⽇}[HSK 7-9]
  \definition{adj.}{demorado; levar um período de ($x$ quantidade de tempo)}
\end{EntryWithPhonetic}

%%%%%%%%%% 呵 %%%%%%%%%%
\subsection*{呵}\addcontentsline{loh}{figure}{呵 \dpy{he1}}

\begin{EntryWithPhonetic}{呵}{he1}{8}{⼝}
  \definition{interj.}{``Meu Deus!'' | ``Ah!''; ``Oh!''}
  \definition{v.}{expirar (com a boca aberta) | repreender}
  \seeref{a1}
\end{EntryWithPhonetic}

\begin{EntryWithPhonetic}{呵护}{he1hu4}{8,7}{⼝,⼿}[HSK 7-9]
  \definition{v.}{proteger; cuidar bem de}
\end{EntryWithPhonetic}

%%%%%%%%%% 欱 %%%%%%%%%%
\subsection*{欱}\addcontentsline{loh}{figure}{欱 \dpy{he1}}

\begin{EntryWithPhonetic}{欱}{he1}{10}{⽋}
  \definition{v.}{beber | beber bebida alcoólica}
  \variantof{喝}
\end{EntryWithPhonetic}

%%%%%%%%%% 喝 %%%%%%%%%%
\subsection*{喝}\addcontentsline{loh}{figure}{喝 \dpy{he1}}

\begin{EntryWithPhonetic}{喝}{he1}{12}{⼝}[HSK 1]
  \definition{interj.}{``Meu Deus!''; ``Oh!''; ``Ah!''; ``Uau!''}
  \definition{s.}{bebida; especificamente, vinho}
  \definition{v.}{beber; engolir líquidos ou alimentos líquidos | beber bebida alcoólica; referência específica ao consumo de álcool}
  \seeref{he4}
\end{EntryWithPhonetic}

\begin{EntryWithPhonetic}{喝醉}{he1zui4}{12,15}{⼝,⾣}
  \definition{v.}{ficar bêbado}
\end{EntryWithPhonetic}

%%%%%%%%%% 禾 %%%%%%%%%%
\subsection*{禾}\addcontentsline{loh}{figure}{禾 \dpy{he2}}

\begin{EntryWithPhonetic}{禾}{he2}{5}{⽲}[Kangxi 115]
  \definition[棵]{s.}{mudas (especialmente de arroz) | painço}
\end{EntryWithPhonetic}

\begin{EntryWithPhonetic}{禾苗}{he2miao2}{5,8}{⽲,⾋}[HSK 7-9]
  \definition[棵,片]{s.}{mudas de cereais | muda (de arroz ou outro grão)}
\end{EntryWithPhonetic}

%%%%%%%%%% 合 %%%%%%%%%%
\subsection*{合}\addcontentsline{loh}{figure}{合 \dpy{he2}}

\begin{EntryWithPhonetic}{合}{he2}{6}{⼝}[HSK 3]
  \definition{adj.}{todo; completo; inteiro}
  \definition{clas.}{usado para rodadas | 100ml | medida para grãos secos igual a um décimo de 升, ou um centésimo de 斗}
  \definition{s.}{casamento; união matrimonial | (astronomia) conjunção | nota da escala em Gongchepu (工尺谱), correspondente ao 5 na notação musical numerada}
  \definition{v.}{fechar | juntar; combinar | adequar-se; concordar; conformar-se a | ser igual a; somar | ser adequado}
  \seealsoref{斗}{dou4}
  \seealsoref{工尺谱}{gong1 che3 pu3}
  \seealsoref{升}{sheng1}
  \antonymref{分}{fen1}
\end{EntryWithPhonetic}

\begin{EntryWithPhonetic}{合并}{he2bing4}{6,6}{⼝,⼲}[HSK 5]
  \definition{v.}{fundir; amalgamar; combinar várias coisas em uma coisa só | (doença) ser complicada por outra doença; uma doença levar a outra, ataques simultâneos (de várias doenças)}
\end{EntryWithPhonetic}

\begin{EntryWithPhonetic}{合唱}{he2chang4}{6,11}{⼝,⼝}[HSK 7-9]
  \definition{v.}{cantar em coro; cantar ou apresentar-se junto}[他们一起合唱一台戏。===Eles cantam uma ópera juntos.]
\end{EntryWithPhonetic}

\begin{EntryWithPhonetic}{合成}{he2cheng2}{6,6}{⼝,⼽}[HSK 5]
  \definition{s.}{compor; integrar; combinar; misturar | Química: sintetizar, reação química para transformar uma substância com uma composição simples em uma substância com uma composição complexa}
\end{EntryWithPhonetic}

\begin{EntryWithPhonetic}{合法}{he2fa3}{6,8}{⼝,⽔}[HSK 3]
  \definition{adj.}{legal; legítimo; lícito;  justo; válido; em conformidade com as disposições legais}
\end{EntryWithPhonetic}

\begin{EntryWithPhonetic}{合格}{he2ge2}{6,10}{⼝,⽊}[HSK 3]
  \definition{adj.}{qualificado; dentro dos padrões; em conformidade com os requisitos ou normas}
\end{EntryWithPhonetic}

\begin{EntryWithPhonetic}{合乎}{he2hu1}{6,5}{⼝,⼃}[HSK 7-9]
  \definition{v.}{conformar-se com (ou a); corresponder a; concordar com; coincidir com; ser consistente com}
\end{EntryWithPhonetic}

\begin{EntryWithPhonetic}{合伙}{he2huo3}{6,6}{⼝,⼈}[HSK 7-9]
  \definition{v.}{formar uma parceria; formar uma parceria relativamente fixa (para se envolver em atividades comerciais ou fazer coisas ruins)}
\end{EntryWithPhonetic}

\begin{EntryWithPhonetic}{合计}{he2ji4}{6,4}{⼝,⾔}[HSK 7-9]
  \definition{v.}{pensar sobre; descobrir | consultar | somar; totalizar}
  \seeref{he2ji5}
\end{EntryWithPhonetic}

\begin{EntryWithPhonetic}{合计}{he2ji5}{6,4}{⼝,⾔}
  \definition{v.}{totalizar; somar; estimar | discutir; negociar; deliberar}
  \seeref{he2ji4}
\end{EntryWithPhonetic}

\begin{EntryWithPhonetic}{合理}{he2li3}{6,11}{⼝,⽟}[HSK 3]
  \definition{adj.}{racional; razoável; equitativo; razoável ou lógico}
\end{EntryWithPhonetic}

\begin{EntryWithPhonetic}{合情合理}{he2qing2-he2li3}{6,11,6,11}{⼝,⼼,⼝,⽟}[HSK 7-9]
  \definition{expr.}{razoável; razoável e lógico; justo e racional; justificável e sensato; justo e razoável; justo e sensato}
\end{EntryWithPhonetic}

\begin{EntryWithPhonetic}{合适}{he2shi4}{6,9}{⼝,⾡}[HSK 2]
  \definition{adj.}{correto; adequado; apropriado; conveniente; em conformidade com a realidade ou com os requisitos objetivos}
\end{EntryWithPhonetic}

\begin{EntryWithPhonetic}{合同}{he2tong5}{6,6}{⼝,⼝}[HSK 4]
  \definition[个,份]{s.}{contrato; acordo; uma disposição para observância mútua por duas ou mais partes na condução de um assunto com o objetivo de determinar seus respectivos direitos e obrigações.}
\end{EntryWithPhonetic}

\begin{EntryWithPhonetic}{合宪性}{he2xian4xing4}{6,9,8}{⼝,⼧,⼼}
  \definition{s.}{constitucionalismo}
\end{EntryWithPhonetic}

\begin{EntryWithPhonetic}{合影}{he2/ying3}{6,15}{⼝,⼺}[HSK 7-9]
  \definition[张,个]{s.}{foto de grupo; imagem de grupo}
  \definition{v.+compl.}{tirar uma foto em grupo; tirar uma foto}
\end{EntryWithPhonetic}

\begin{EntryWithPhonetic}{合约}{he2yue1}{6,6}{⼝,⽷}[HSK 6]
  \definition[份]{s.}{contrato; geralmente se refere a contratos com cláusulas mais simples}
\end{EntryWithPhonetic}

\begin{EntryWithPhonetic}{合资}{he2zi1}{6,10}{⼝,⾙}[HSK 7-9]
  \definition{s.}{consórcio; \emph{joint-venture} com capitais mistos; investimento conjunto por duas ou mais partes (diferente de 独资)}
  \definition{v.}{investir conjuntamente em}
  \seealsoref{独资}{du2zi1}
\end{EntryWithPhonetic}

\begin{EntryWithPhonetic}{合作}{he2zuo4}{6,7}{⼝,⼈}[HSK 3]
  \definition{v.}{cooperar; colaborar; trabalhar em conjunto; trabalhar em conjunto para realizar algo ou concluir uma tarefa}
\end{EntryWithPhonetic}

\begin{EntryWithPhonetic}{合作社}{he2zuo4she4}{6,7,7}{⼝,⼈,⽰}[HSK 7-9]
  \definition{s.}{cooperativa | cooperativa de trabalhadores ou produtores agrícolas, etc.}
\end{EntryWithPhonetic}

%%%%%%%%%% 何 %%%%%%%%%%
\subsection*{何}\addcontentsline{loh}{figure}{何 \dpy{he2}}

\begin{EntryWithPhonetic}{何}{he2}{7}{⼈}
  \definition*{s.}{Sobrenome: He}
  \definition{adv.}{enfatiza um alto grau de intensidade, equivalente a 多么}
  \definition{pron.}{O que?; Qual?; em nome de pessoas ou coisas, equivalente a 什么 | Onde?; em nome do lugar, equivalente a 哪里 | Por que?; Como?; a razão, é equivalente a 为什么 e 怎么}
  \seealsoref{多么}{duo1me5}
  \seealsoref{哪里}{na3li5}
  \seealsoref{岂}{qi3}
  \seealsoref{什么}{shen2me5}
  \seealsoref{为什么}{wei4shen2me5}
  \seealsoref{怎}{zen3}
  \seealsoref{怎么}{zen3me5}
\end{EntryWithPhonetic}

\begin{EntryWithPhonetic}{何必}{he2bi4}{7,5}{⼈,⼼}[HSK 7-9]
  \definition{adv.}{por que?; não há necessidade; use um tom interrogativo para expressar que não é necessário}[你何必如此匆忙?===Por que você está com tanta pressa?]
\end{EntryWithPhonetic}

\begin{EntryWithPhonetic}{何不}{he2bu4}{7,4}{⼈,⼀}
  \definition{adv.}{``Por que não?''; use o tom interrogativo para expressar ``deveria'' ou ``pode''}
\end{EntryWithPhonetic}

\begin{EntryWithPhonetic}{何处}{he2chu4}{7,5}{⼈,⼡}[HSK 7-9]
  \definition{pron.}{onde; que lugar}[你要去何处?===Onde você está indo?]
\end{EntryWithPhonetic}

\begin{EntryWithPhonetic}{何典}{he2 dian3}{7,8}{⼈,⼋}
  \definition*{s.}{He Dian; este é um romance clássico chinês único, com uma arte inesquecível e um estilo humorístico único; o romance satiriza o submundo}
\end{EntryWithPhonetic}

\begin{EntryWithPhonetic}{何故}{he2gu4}{7,9}{⼈,⽁}
  \definition{adv.}{qual razão?; por que? | para quê? | qual é o motivo?}
\end{EntryWithPhonetic}

\begin{EntryWithPhonetic}{何苦}{he2ku3}{7,8}{⼈,⾋}[HSK 7-9]
  \definition{adv.}{por que se preocupar?; vale a pena o esforço?; use uma pergunta retórica para expressar que não vale a pena e por que se preocupar}[为了这点事,何苦生气呢?===Por que ficar bravo com uma coisa dessas?]
\end{EntryWithPhonetic}

\begin{EntryWithPhonetic}{何况}{he2kuang4}{7,7}{⼈,⼎}[HSK 7-9]
  \definition{conj.}{muito menos; use um tom interrogativo para expressar que é mais óbvio e razoável em comparação | além disso; além do mais; indique outras razões ou razões adicionais}
\end{EntryWithPhonetic}

\begin{EntryWithPhonetic}{何时}{he2shi2}{7,7}{⼈,⽇}[HSK 7-9]
  \definition{adv.}{quando?; que horas?}[我何时能再见到你?===Quando poderei vê-lo novamente?]
\end{EntryWithPhonetic}

%%%%%%%%%% 和 %%%%%%%%%%
\subsection*{和}\addcontentsline{loh}{figure}{和 \dpy{he2}}

\begin{EntryWithPhonetic}{和}{he2}{8}{⼝}[HSK 1]
  \definition*{s.}{Sobrenome: He}
  \definition{adj.}{gentil; suave; amável | harmonioso; em boas condições}
  \definition{conj.}{e (somente para palavras); unidos com}
  \definition{prep.}{relacionado com | para; com; indica correlação; comparação, etc.}
  \definition{s.}{soma; soma total | japonês; refere"-se ao Japão}
  \definition{v.}{disputar; reconciliar; acabar com a guerra ou a disputa | empatar; (próxima edição ou torneio) sem vencedor}
  \seeref{he4}
  \seeref{hu2}
  \seeref{huo2}
  \seeref{huo4}
\end{EntryWithPhonetic}

\begin{EntryWithPhonetic}{和蔼}{he2'ai3}{8,14}{⼝,⾋}[HSK 7-9]
  \definition{adj.}{gentil; afável; amável}
\end{EntryWithPhonetic}

\begin{EntryWithPhonetic}{和解}{he2jie3}{8,13}{⼝,⾓}[HSK 7-9]
  \definition{v.}{reconciliar}
\end{EntryWithPhonetic}

\begin{EntryWithPhonetic}{和睦}{he2mu4}{8,13}{⼝,⽬}[HSK 7-9]
  \definition{adj.}{harmonioso; amigável; amistoso}
  \definition{s.}{harmonia; concórdia}
\end{EntryWithPhonetic}

\begin{EntryWithPhonetic}{和平}{he2ping2}{8,5}{⼝,⼲}[HSK 3]
  \definition{adj.}{pacífico; moderado; não violento | pacífico; tranquilo; sereno}
  \definition{s.}{paz ;ausência de guerra}
\end{EntryWithPhonetic}

\begin{EntryWithPhonetic}{和平共处}{he2ping2 gong4chu3}{8,5,6,5}{⼝,⼲,⼋,⼡}[HSK 7-9]
  \definition{expr.}{coexistência pacífica de nações, sociedades etc.; refere"-se a países com diferentes sistemas sociais que resolvem disputas pacificamente e desenvolvem laços econômicos e culturais com base na igualdade e no benefício mútuo}
\end{EntryWithPhonetic}

\begin{EntryWithPhonetic}{和气}{he2qi5}{8,4}{⼝,⽓}[HSK 7-9]
  \definition{adj.}{gentil; bondoso; educado | amigável; amável; harmonioso}
  \definition{s.}{amizade; relações harmoniosas; atmosfera harmoniosa; sentimentos harmoniosos}
\end{EntryWithPhonetic}

\begin{EntryWithPhonetic}{和尚}{he2shang5}{8,8}{⼝,⼩}[HSK 7-9]
  \definition[个,名,位]{s.}{monge budista; refere"-se aos monges budistas do sexo masculino que praticam o budismo}
\end{EntryWithPhonetic}

\begin{EntryWithPhonetic}{和谐}{he2xie2}{8,11}{⼝,⾔}[HSK 6]
  \definition{adj.}{harmonioso; sem conflitos | em perfeita harmonia; ajuste adequado e simétrico}
  \definition{v.}{(eufemismo) censurar}
\end{EntryWithPhonetic}

%%%%%%%%%% 河 %%%%%%%%%%
\subsection*{河}\addcontentsline{loh}{figure}{河 \dpy{he2}}

\begin{EntryWithPhonetic}{河}{he2}{8}{⽔}[HSK 2]
  \definition*{s.}{Astronomia: o sistema da Via Láctea | O Rio Amarelo; O Rio Huanghe | Sobrenome: He}
  \definition[条,道]{s.}{rio; refere"-se a grandes cursos de água}
\end{EntryWithPhonetic}

\begin{EntryWithPhonetic}{河蚌}{he2bang4}{8,10}{⽔,⾍}
  \definition{s.}{mexilhões | bivalves cultivados em rios e lagos}
\end{EntryWithPhonetic}

\begin{EntryWithPhonetic}{河流}{he2liu2}{8,10}{⽔,⽔}[HSK 7-9]
  \definition[条]{s.}{rio; córrego; um termo geral para grandes fluxos naturais de água (como rios, etc.) na superfície da Terra}
\end{EntryWithPhonetic}

\begin{EntryWithPhonetic}{河南}{he2nan2}{8,9}{⽔,⼗}
  \definition*{s.}{Província de Henan}
\end{EntryWithPhonetic}

\begin{EntryWithPhonetic}{河畔}{he2pan4}{8,10}{⽔,⽥}[HSK 7-9]
  \definition{s.}{planície fluvial | beira-rio}
\end{EntryWithPhonetic}

%%%%%%%%%% 核 %%%%%%%%%%
\subsection*{核}\addcontentsline{loh}{figure}{核 \dpy{he2}}

\begin{EntryWithPhonetic}{核}{he2}{10}{⽊}[HSK 7-9]
  \definition{adj.}{Literário: verdadeiro; fiel}
  \definition{s.}{poço; pedra; caroço | núcleo | núcleo atômico}
  \definition{v.}{examinar; verificar}
  \seeref{hu2}
\end{EntryWithPhonetic}

\begin{EntryWithPhonetic}{核电站}{he2dian4zhan4}{10,5,10}{⽊,⽥,⽴}[HSK 7-9]
  \definition{s.}{usina nuclear; usina que utiliza energia nuclear para gerar eletricidade}
\end{EntryWithPhonetic}

\begin{EntryWithPhonetic}{核对}{he2dui4}{10,5}{⽊,⼨}[HSK 7-9]
  \definition{v.}{verificar; checar; verificar cuidadosamente (para ver se corresponde)}
\end{EntryWithPhonetic}

\begin{EntryWithPhonetic}{核能}{he2neng2}{10,10}{⽊,⾁}[HSK 7-9]
  \definition{s.}{energia nuclear}
\end{EntryWithPhonetic}

\begin{EntryWithPhonetic}{核实}{he2shi2}{10,8}{⽊,⼧}[HSK 7-9]
  \definition{v.}{verificar; checar; verificar se é verdade}
\end{EntryWithPhonetic}

\begin{EntryWithPhonetic}{核桃}{he2tao5}{10,10}{⽊,⽊}[HSK 7-9]
  \definition[颗,个,棵,顆]{s.}{noz | nogueira}
\end{EntryWithPhonetic}

\begin{EntryWithPhonetic}{核武器}{he2wu3qi4}{10,8,16}{⽊,⽌,⼝}[HSK 7-9]
  \definition[个]{s.}{arma nuclear}
\end{EntryWithPhonetic}

\begin{EntryWithPhonetic}{核心}{he2xin1}{10,4}{⽊,⼼}[HSK 6]
  \definition[个]{s.}{núcleo; elite; coração; centro; parte principal (em termos de relacionamento entre as coisas)}
\end{EntryWithPhonetic}

%%%%%%%%%% 荷 %%%%%%%%%%
\subsection*{荷}\addcontentsline{loh}{figure}{荷 \dpy{he2}}

\begin{EntryWithPhonetic}{荷}{he2}{10}{⾋}
  \definition*{s.}{Países Baixos; Holanda, abreviação de 荷兰 | Sobrenome: He}
  \definition{s.}{lótus}
  \seeref{he4}
  \seealsoref{荷兰}{he2lan2}
\end{EntryWithPhonetic}

\begin{EntryWithPhonetic}{荷花}{he2hua1}{10,7}{⾋,⾋}[HSK 7-9]
  \definition[朵,枝,片]{s.}{lótus; flor de lótus}
\end{EntryWithPhonetic}

\begin{EntryWithPhonetic}{荷兰}{he2lan2}{10,5}{⾋,⼋}
  \definition*{s.}{Países Baixos; Holanda}
\end{EntryWithPhonetic}

%%%%%%%%%% 盒 %%%%%%%%%%
\subsection*{盒}\addcontentsline{loh}{figure}{盒 \dpy{he2}}

\begin{EntryWithPhonetic}{盒}{he2}{11}{⽫}[HSK 5]
  \definition{clas.}{caixa (de pequena dimensão)}
  \definition[个]{s.}{caixa; estojo; recipiente; receptáculo}
\end{EntryWithPhonetic}

\begin{EntryWithPhonetic}{盒饭}{he2fan4}{11,7}{⽫,⾷}[HSK 5]
  \definition[份]{s.}{refeição embalada; marmita; \emph{fast-food} vendida em caixas}
\end{EntryWithPhonetic}

\begin{EntryWithPhonetic}{盒子}{he2zi5}{11,3}{⽫,⼦}[HSK 5]
  \definition[个,只,堆]{s.}{caixa; recipiente que têm tampas na parte superior e podem conter coisas dentro, geralmente é pequeno e plano}
\end{EntryWithPhonetic}

%%%%%%%%%% 和 %%%%%%%%%%
\subsection*{和}\addcontentsline{loh}{figure}{和 \dpy{he4}}

\begin{EntryWithPhonetic}{和}{he4}{8}{⼝}
  \definition{v.}{compor um poema em resposta (ao poema de alguém) usando a mesma sequência de rimas | juntar-se à cantoria; cantar junto com outros em harmonia}
  \seeref{he2}
  \seeref{hu2}
  \seeref{huo2}
  \seeref{huo4}
\end{EntryWithPhonetic}

%%%%%%%%%% 贺 %%%%%%%%%%
\subsection*{贺}\addcontentsline{loh}{figure}{贺 \dpy{he4}}

\begin{EntryWithPhonetic}{贺}{he4}{9}{⾙}
  \definition*{s.}{Sobrenome: He}
  \definition{v.}{parabenizar; congratular | celebrar; comemorar}
\end{EntryWithPhonetic}

\begin{EntryWithPhonetic}{贺电}{he4dian4}{9,5}{⾙,⽥}[HSK 7-9]
  \definition[封]{s.}{mensagem de felicitações; telegrama de felicitações}
\end{EntryWithPhonetic}

\begin{EntryWithPhonetic}{贺卡}{he4ka3}{9,5}{⾙,⼘}[HSK 5]
  \definition[张]{s.}{cartão de felicitações; pedaço de papel para parabenizar amigos e parentes em seu casamento, aniversário ou festivais, geralmente impresso com palavras e desenhos de felicitações}
\end{EntryWithPhonetic}

\begin{EntryWithPhonetic}{贺信}{he4xin4}{9,9}{⾙,⼈}[HSK 7-9]
  \definition{s.}{carta de felicitações; carta de congratulações}
\end{EntryWithPhonetic}

%%%%%%%%%% 荷 %%%%%%%%%%
\subsection*{荷}\addcontentsline{loh}{figure}{荷 \dpy{he4}}

\begin{EntryWithPhonetic}{荷}{he4}{10}{⾋}
  \definition{s.}{fardo; responsabilidade}
  \definition{v.}{carregar no ombro ou nas costas | aceitar um favor, frequentemente usado em cartas para expressar cortesia}
  \seeref{he2}
\end{EntryWithPhonetic}

%%%%%%%%%% 喝 %%%%%%%%%%
\subsection*{喝}\addcontentsline{loh}{figure}{喝 \dpy{he4}}

\begin{EntryWithPhonetic}{喝}{he4}{12}{⼝}
  \definition{v.}{gritar bem alto}
  \seeref{he1}
\end{EntryWithPhonetic}

\begin{EntryWithPhonetic}{喝采}{he4/cai3}{12,8}{⼝,⾤}
  \definition{v.+compl.}{aclamar; aplaudir}
\end{EntryWithPhonetic}

\begin{EntryWithPhonetic}{喝彩}{he4cai3}{12,11}{⼝,⼺}
  \definition{s.}{aclamar | torcer}
\end{EntryWithPhonetic}

%%%%%%%%%% 褐 %%%%%%%%%%
\subsection*{褐}\addcontentsline{loh}{figure}{褐 \dpy{he4}}

\begin{EntryWithPhonetic}{褐}{he4}{14}{⾐}
  \definition{adj.}{marrom; castanho; pardo}
  \definition{s.}{pano de cânhamo grosso}
\end{EntryWithPhonetic}

\begin{EntryWithPhonetic}{褐色}{he4 se4}{14,6}{⾐,⾊}
  \definition{s.}{cor marrom}
\end{EntryWithPhonetic}

%%%%%%%%%% 赫 %%%%%%%%%%
\subsection*{赫}\addcontentsline{loh}{figure}{赫 \dpy{he4}}

\begin{EntryWithPhonetic}{赫}{he4}{14}{⾚}
  \definition*{s.}{Sobrenome: He}
  \definition{adj.}{conspícuo; grandioso | vermelho brilhante e flamejante; vermelho como fogo}
  \definition{clas.}{Hz, hertz; abreviação de 赫兹}
  \seealsoref{赫兹}{he4zi1}
\end{EntryWithPhonetic}

\begin{EntryWithPhonetic}{赫然}{he4ran2}{14,12}{⾚,⽕}[HSK 7-9]
  \definition{adj.}{inesperado e chocante/impressionante; descreve algo que é muito marcante ou surpreendente | (raiva, etc.) terrível; violento; descreve o olhar de raiva | grande; eminente; florescente; excepcional; descreve a aparência de ser proeminente}
\end{EntryWithPhonetic}

\begin{EntryWithPhonetic}{赫兹}{he4zi1}{14,9}{⾚,⼋}
  \definition{s.}{hertz (Hz), unidade de frequência}
  \definition{s.}{Heinrich Hertz (1857-1894), físico e meteorologista alemão, pioneiro da radiação eletromagnética}
\end{EntryWithPhonetic}

%%%%%%%%%% 鹤 %%%%%%%%%%
\subsection*{鹤}\addcontentsline{loh}{figure}{鹤 \dpy{he4}}

\begin{EntryWithPhonetic}{鹤}{he4}{15}{⿃}
  \definition*{s.}{Sobrenome: He}
  \definition[只]{s.}{grou (ave)}
\end{EntryWithPhonetic}

\begin{EntryWithPhonetic}{鹤立鸡群}{he4li4ji1qun2}{15,5,7,13}{⿃,⽴,⿃,⽺}[HSK 7-9]
  \definition{expr.}{destaque-se da multidão; manifestamente superior; muito acima do comum; como um guindaste em pé entre galinhas --- fique de pé acima dos outros}
\end{EntryWithPhonetic}

%%%%%%%%%% 黑 %%%%%%%%%%
\subsection*{黑}\addcontentsline{loh}{figure}{黑 \dpy{hei1}}

\begin{EntryWithPhonetic}{黑}{hei1}{12}{⿊}[HSK 2][Kangxi 203]
  \definition*{s.}{Província de Heilongjiang, abreviação de 黑龙江 | Sobrenome: Hei}
  \definition{adj.}{preto; cor semelhante à do carvão | escuro | obscuro; secreto | perverso; sinistro; ruim; cruel | reacionário}
  \definition{s.}{noite}
  \definition{v.}{fazer algo ilegalmente ou de forma desonesta; enganar; desviar dinheiro ilegalmente | invadir (uma rede, sites, computador, etc.)}
  \seealsoref{黑龙江}{hei1long2jiang1}
\end{EntryWithPhonetic}

\begin{EntryWithPhonetic}{黑暗}{hei1'an4}{12,13}{⿊,⽇}[HSK 4]
  \definition{adj.}{escuro; sombrio; sem luz | maligno; corrupto; reacionário}
\end{EntryWithPhonetic}

\begin{EntryWithPhonetic}{黑白}{hei1bai2}{12,5}{⿊,⽩}[HSK 7-9]
  \definition[只]{s.}{preto e branco | certo e errado; metáfora para o certo e o errado, o bem e o mal}
\end{EntryWithPhonetic}

\begin{EntryWithPhonetic}{黑板}{hei1ban3}{12,8}{⿊,⽊}[HSK 2]
  \definition[块,个]{s.}{quadro negro; quadro de giz; uma placa, na qual se pode escrever com giz}
\end{EntryWithPhonetic}

\begin{EntryWithPhonetic}{黑客}{hei1ke4}{12,9}{⿊,⼧}[HSK 7-9]
  \definition[个,些,位,名]{s.}{Empréstimo linguístico: \emph{hacker}; \emph{cracker}; intruso cibernético; gênio da computação; originalmente se refere a pessoas que não são profissionais de informática, mas são muito proficientes em tecnologia de computadores; agora se refere especificamente a pessoas que podem escrever programas de descriptografia para invadir ilegalmente redes de computadores de outras pessoas para interferir ou destruí-las}
\end{EntryWithPhonetic}

\begin{EntryWithPhonetic}{黑龙江}{hei1long2jiang1}{12,5,6}{⿊,⿓,⽔}
  \definition*{s.}{Província de Heilongjiang | Rio Heilong Jiang;  Rio Amur (na Rússia)}
\end{EntryWithPhonetic}

\begin{EntryWithPhonetic}{黑马}{hei1ma3}{12,3}{⿊,⾺}[HSK 7-9]
  \definition[匹,群]{s.}{azarão (cavalo preto) | Figurativo: pessoa pouco conhecida que alcança sucesso inesperado}
\end{EntryWithPhonetic}

\begin{EntryWithPhonetic}{黑色}{hei1se4}{12,6}{⿊,⾊}[HSK 2]
  \definition{adj.}{metafórico: suspeito, ilegal}
  \definition{s.}{cor preta}
\end{EntryWithPhonetic}

\begin{EntryWithPhonetic}{黑手}{hei1shou3}{12,4}{⿊,⼿}[HSK 7-9]
  \definition{s.}{mão negra; manipulador maligno dos bastidores | uma pessoa cruel manipulando alguém ou algo nos bastidores; uma metáfora para pessoas ou forças que secretamente realizam atividades de conspiração}
\end{EntryWithPhonetic}

\begin{EntryWithPhonetic}{黑桃}{hei1tao2}{12,10}{⿊,⽊}
  \definition{s.}{espadas ♠ (em jogos de cartas)}
  \seealsoref{方片}{fang1 pian4}
  \seealsoref{红心}{hong2xin1}
  \seealsoref{梅花}{mei2hua1}
\end{EntryWithPhonetic}

\begin{EntryWithPhonetic}{黑心}{hei1xin1}{12,4}{⿊,⼼}[HSK 7-9]
  \definition{adj.}{malvado; perverso | ganancioso; avarento | (certos bens) de má qualidade | implacável e sem consciência | de mente viciosa cheia de ódio e ciúme}
  \definition{s.}{coração negro; mente maligna | núcleo preto (falha na cerâmica)}
\end{EntryWithPhonetic}

\begin{EntryWithPhonetic}{黑夜}{hei1ye4}{12,8}{⿊,⼣}[HSK 6]
  \definition[个]{s.}{noite ; uma noite muito escura sem luz}
\end{EntryWithPhonetic}

%%%%%%%%%% 嘿 %%%%%%%%%%
\subsection*{嘿}\addcontentsline{loh}{figure}{嘿 \dpy{hei1}}

\begin{EntryWithPhonetic}{嘿}{hei1}{15}{⼝}[HSK 7-9]
  \definition{interj.}{``Ei!''; indicando uma saudação ou para chamar a atenção | expressando orgulho ou satisfação | expressando espanto, surpresa}
  \seeref{mo4}
\end{EntryWithPhonetic}

%%%%%%%%%% 痕 %%%%%%%%%%
\subsection*{痕}\addcontentsline{loh}{figure}{痕 \dpy{hen2}}

\begin{EntryWithPhonetic}{痕}{hen2}{11}{⽧}
  \definition[个]{s.}{marca; traço}
\end{EntryWithPhonetic}

\begin{EntryWithPhonetic}{痕迹}{hen2ji4}{11,9}{⽧,⾡}[HSK 7-9]
  \definition{s.}{marca; traço; a marca deixada por um objeto que entra em contato com outro | traço; rastro; vestígio; coisas ou fenômenos deixados por algo que já existiu}
\end{EntryWithPhonetic}

%%%%%%%%%% 很 %%%%%%%%%%
\subsection*{很}\addcontentsline{loh}{figure}{很 \dpy{hen3}}

\begin{EntryWithPhonetic}{很}{hen3}{9}{⼻}[HSK 1]
  \definition{adv.}{muito; bastante; terrivelmente; indica um grau bastante elevado; definitivo; o mais alto}
\end{EntryWithPhonetic}

\begin{EntryWithPhonetic}{很难说}{hen3 nan2shuo1}{9,10,9}{⼻,⾫,⾔}[HSK 6]
  \definition{adj.}{difícil dizer}
\end{EntryWithPhonetic}

%%%%%%%%%% 狠 %%%%%%%%%%
\subsection*{狠}\addcontentsline{loh}{figure}{狠 \dpy{hen3}}

\begin{EntryWithPhonetic}{狠}{hen3}{9}{⽝}[HSK 6]
  \definition{adj.}{impiedoso; implacável; feroz | firme; resoluto; severo; determinado}
  \definition{adv.}{muito; bastante; bastante | também, frequentemente usado antes de um adjetivo sem intensificar seu significado, ou seja, como um elemento sintático sem sentido}
  \definition{v.}{endurecer (o coração); suprimir (os próprios sentimentos)}
  \variantof{很}
  \seealsoref{很}{hen3}
\end{EntryWithPhonetic}

%%%%%%%%%% 恨 %%%%%%%%%%
\subsection*{恨}\addcontentsline{loh}{figure}{恨 \dpy{hen4}}

\begin{EntryWithPhonetic}{恨}{hen4}{9}{⼼}[HSK 5]
  \definition{s.}{ódio; resentimento}
  \definition{v.}{odiar; ressentir-se}
\end{EntryWithPhonetic}

\begin{EntryWithPhonetic}{恨不得}{hen4bu5de5}{9,4,11}{⼼,⼀,⼻}[HSK 7-9]
  \definition{v.}{estar muito ansioso para; querer poder (fazer algo); mal poder esperar para; expressa um desejo ansioso de realizar algo, geralmente usado para coisas que não podem realmente ser feitas}
\end{EntryWithPhonetic}

%%%%%%%%%% 哼 %%%%%%%%%%
\subsection*{哼}\addcontentsline{loh}{figure}{哼 \dpy{heng1}}

\begin{EntryWithPhonetic}{哼}{heng1}{10}{⼝}[HSK 7-9]
  \definition{v.}{gemer; bufar | cantarolar}
  \seeref{hng5}
\end{EntryWithPhonetic}

%%%%%%%%%% 行 %%%%%%%%%%
\subsection*{行}\addcontentsline{loh}{figure}{行 \dpy{heng2}}

\begin{EntryWithPhonetic}{行}{heng2}{6}{⾏}[Kangxi 144]
  \definition{s.}{usado em 道行}
  \seeref{hang2}
  \seeref{xing2}
  \seealsoref{道行}{dao4 heng2}
\end{EntryWithPhonetic}

%%%%%%%%%% 恒 %%%%%%%%%%
\subsection*{恒}\addcontentsline{loh}{figure}{恒 \dpy{heng2}}

\begin{EntryWithPhonetic}{恒}{heng2}{9}{⼼}
  \definition*{s.}{Sobrenome: Heng}
  \definition{adj.}{permanente; duradouro | usual; comum; constante | usual; frequente; constante}
  \definition{s.}{perseverança; constância}
\end{EntryWithPhonetic}

\begin{EntryWithPhonetic}{恒山}{heng2shan1}{9,3}{⼼,⼭}
  \definition*{s.}{Monte Heng em Shanxi, montanha norte das Cinco Montanhas Sagradas (五岳) | Distrito de Hengshan na cidade de Jixi (鸡西), Heilongjiang}
  \seealsoref{鸡西}{ji1xi1}
  \seealsoref{五岳}{wu3yue4}
\end{EntryWithPhonetic}

\begin{EntryWithPhonetic}{恒星系}{heng2xing1xi4}{9,9,7}{⼼,⽇,⽷}
  \definition{s.}{sistema estelar | galáxia}
\end{EntryWithPhonetic}

%%%%%%%%%% 横 %%%%%%%%%%
\subsection*{横}\addcontentsline{loh}{figure}{横 \dpy{heng2}}

\begin{EntryWithPhonetic}{横}{heng2}{15}{⽊}[HSK 6]
  \definition{adj.}{horizontal; transversal; paralelo ao plano horizontal | em ângulo reto com; direção esquerda-direita | e leste a oeste ou de oeste a leste; direção leste-oeste | desenfreado; turbulento | violento; feroz; irracional}
  \definition{adv.}{de qualquer forma; em qualquer caso | provavelmente; muito provavelmente}
  \definition{s.}{traço horizontal (em caracteres chineses)}
  \definition{v.}{deitar-se transversalmente; estar de lado | colocar algo transversalmente (ou horizontalmente)}
  \seeref{heng4}
  \antonymref{竖}{shu4}
  \antonymref{直}{zhi2}
  \antonymref{纵}{zong4}
\end{EntryWithPhonetic}

\begin{EntryWithPhonetic}{横七竖八}{heng2qi1-shu4ba1}{15,2,9,2}{⽊,⼀,⽴,⼋}[HSK 7-9]
  \definition{expr.}{em desordem; em seis e sete; desorganizado}
\end{EntryWithPhonetic}

\begin{EntryWithPhonetic}{横竖}{heng2shu5}{15,9}{⽊,⽴}
  \definition{adv.}{de qualquer forma; em qualquer maneira; isso significa que não importa o que aconteça, o resultado ou a conclusão não mudará; equivale a 反正}
  \seealsoref{反正}{fan3zheng5}
\end{EntryWithPhonetic}

\begin{EntryWithPhonetic}{横向}{heng2xiang4}{15,6}{⽊,⼝}[HSK 7-9]
  \definition{adj.}{horizontal; transversal | lateral | ortogonal | perpendicular}
  \antonymref{竖向}{shu4xiang4}
  \antonymref{纵向}{zong4xiang4}
\end{EntryWithPhonetic}

%%%%%%%%%% 衡 %%%%%%%%%%
\subsection*{衡}\addcontentsline{loh}{figure}{衡 \dpy{heng2}}

\begin{EntryWithPhonetic}{衡}{heng2}{16}{⾏}
  \definition*{s.}{Sobrenome: Heng}
  \definition[个]{s.}{braço graduado de uma balança | balança; aparelho de pesagem}
  \definition{v.}{pesar; medir; julgar}
\end{EntryWithPhonetic}

\begin{EntryWithPhonetic}{衡量}{heng2liang5}{16,12}{⾏,⾥}[HSK 6]
  \definition{v.}{pesar; medir; comparar; avaliar | considerar; pensar sobre; deliberar}
\end{EntryWithPhonetic}

\begin{EntryWithPhonetic}{衡山}{heng2shan1}{16,3}{⾏,⼭}
  \definition*{s.}{Monte Heng em Hunan, uma cordilheira ao sul das Cinco Montanhas Sagradas (五岳) | Condado de Hengshan em Hengyang (衡阳)}
  \seealsoref{衡阳}{heng2yang2}
  \seealsoref{五岳}{wu3yue4}
\end{EntryWithPhonetic}

\begin{EntryWithPhonetic}{衡阳}{heng2yang2}{16,6}{⾏,⾩}
  \definition*{s.}{Hengyang, cidade de nível de prefeitura em Hunan situada no lado sul do Monte Heng (衡山)}
  \seealsoref{衡山}{heng2shan1}
\end{EntryWithPhonetic}

%%%%%%%%%% 横 %%%%%%%%%%
\subsection*{横}\addcontentsline{loh}{figure}{横 \dpy{heng4}}

\begin{EntryWithPhonetic}{横}{heng4}{15}{⽊}[HSK 7-9]
  \definition{adj.}{chocante e irracional; inesperado}
  \seeref{heng2}
\end{EntryWithPhonetic}

%%%%%%%%%% 哼 %%%%%%%%%%
\subsection*{哼}\addcontentsline{loh}{figure}{哼 \dpy{hng5}}

\begin{EntryWithPhonetic}{哼}{hng5}{10}{⼝}
  \definition{interj.}{``Hmm''; ``Humph''; expressa insatisfação, desprezo, desdém ou indignação}
\end{EntryWithPhonetic}

%%%%%%%%%% 轰 %%%%%%%%%%
\subsection*{轰}\addcontentsline{loh}{figure}{轰 \dpy{hong1}}

\begin{EntryWithPhonetic}{轰}{hong1}{8}{⾞}[HSK 7-9]
  \definition{interj.}{Onomatopéia: ``Bum!''; ``Bang!''; refere"-se aos ruídos altos feitos por trovões, fogo de artilharia, etc.}
  \definition{v.}{retumbar; trovejar; bombardear; explodir | espantar; expulsar}
\end{EntryWithPhonetic}

\begin{EntryWithPhonetic}{轰动}{hong1dong4}{8,6}{⾞,⼒}[HSK 7-9]
  \definition{v.}{causar (criar) uma sensação; fazer um rebuliço; criar um rebuliço}
\end{EntryWithPhonetic}

\begin{EntryWithPhonetic}{轰鸣}{hong1ming2}{8,8}{⾞,⿃}
  \definition{s.}{trovão; rugido}
  \definition{v.}{rosnar; rugir; trovejar}
\end{EntryWithPhonetic}

\begin{EntryWithPhonetic}{轰炸}{hong1zha4}{8,9}{⾞,⽕}[HSK 7-9]
  \definition{v.}{bombardear; lançar bombas de aeronaves sobre vários alvos no solo ou na água}
\end{EntryWithPhonetic}

\begin{EntryWithPhonetic}{轰炸机}{hong1zha4ji1}{8,9,6}{⾞,⽕,⽊}
  \definition{s.}{bombardeiro (aeronave)}
\end{EntryWithPhonetic}

%%%%%%%%%% 哄 %%%%%%%%%%
\subsection*{哄}\addcontentsline{loh}{figure}{哄 \dpy{hong1}}

\begin{EntryWithPhonetic}{哄}{hong1}{9}{⼝}[HSK 7-9]
  \definition{interj.}{Onomatopéia: gargalhadas ou alvoroço}
  \definition{s.}{rugido; clamor; comoção}
  \seeref{hong3}
  \seeref{hong4}
\end{EntryWithPhonetic}

\begin{EntryWithPhonetic}{哄堂大笑}{hong1tang2-da4xiao4}{9,11,3,10}{⼝,⼟,⼤,⽵}[HSK 7-9]
  \definition{expr.}{fazer a sala inteira rugir (em alvoroço); (causar) uma explosão geral de risos; ``Todos caíram na gargalhada.''; uma explosão de risos na plateia; ``A plateia caiu na gargalhada.''; ``As pessoas de toda a casa caíram na gargalhada.''; ``A sala inteira riu (balançando).''}
\end{EntryWithPhonetic}

%%%%%%%%%% 烘 %%%%%%%%%%
\subsection*{烘}\addcontentsline{loh}{figure}{烘 \dpy{hong1}}

\begin{EntryWithPhonetic}{烘}{hong1}{10}{⽕}
  \definition{v.}{secar; assar; aquecer; usar fogo ou vapor para aquecer o corpo ou para cozinhar, aquecer ou secar algo | destacar}
\end{EntryWithPhonetic}

\begin{EntryWithPhonetic}{烘干}{hong1gan1}{10,3}{⽕,⼲}[HSK 7-9]
  \definition{v.}{secar em fogo alto | secar ao lado ou sobre o fogo | assar; secar no forno}
\end{EntryWithPhonetic}

\begin{EntryWithPhonetic}{烘托}{hong1tuo1}{10,6}{⽕,⼿}[HSK 7-9]
  \definition{v.}{adicionar sombreamento ao redor de um objeto para destacá-lo; um dos métodos de pintura chinesa, que utiliza tinta ou cores claras para pontilhar o contorno do objeto e torná-lo mais claro | destacar por contraste; colocar em nítido relevo; fazer com que se destaque}
\end{EntryWithPhonetic}

%%%%%%%%%% 弘 %%%%%%%%%%
\subsection*{弘}\addcontentsline{loh}{figure}{弘 \dpy{hong2}}

\begin{EntryWithPhonetic}{弘}{hong2}{5}{⼸}
  \definition*{s.}{Sobrenome: Hong}
  \definition{adj.}{grande; grandioso; magnífico}
  \definition{v.}{ampliar; expandir | promover}
\end{EntryWithPhonetic}

\begin{EntryWithPhonetic}{弘扬}{hong2yang2}{5,6}{⼸,⼿}[HSK 7-9]
  \definition{v.}{melhorar; levar adiante; desenvolver e expandir; promover; promover vigorosamente}
\end{EntryWithPhonetic}

%%%%%%%%%% 红 %%%%%%%%%%
\subsection*{红}\addcontentsline{loh}{figure}{红 \dpy{hong2}}

\begin{EntryWithPhonetic}{红}{hong2}{6}{⽷}[HSK 2]
  \definition*{s.}{Sobrenome: Hong}
  \definition{adj.}{vermelho | popular; bem-sucedido; símbolo de sucesso ou valorização | vermelho; revolucionário; símbolo da revolução | festivo; símbolo de alegria}
  \definition{s.}{tecido vermelho, bandeirinhas, etc. usados em ocasiões festivas | bônus; dividendo}
\end{EntryWithPhonetic}

\begin{EntryWithPhonetic}{红包}{hong2bao1}{6,5}{⽷,⼓}[HSK 4]
  \definition[个]{s.}{saco de papel vermelho ou envelope contendo dinheiro como presente, gorjeta ou bônus | suborno; propina}
\end{EntryWithPhonetic}

\begin{EntryWithPhonetic}{红宝石}{hong2bao3shi2}{6,8,5}{⽷,⼧,⽯}
  \definition{s.}{rubi}
\end{EntryWithPhonetic}

\begin{EntryWithPhonetic}{红茶}{hong2cha2}{6,9}{⽷,⾋}[HSK 3]
  \definition[杯,壶,斤,种]{s.}{chá preto; chá acabado produzido através de fermentação completa}
\end{EntryWithPhonetic}

\begin{EntryWithPhonetic}{红灯}{hong2deng1}{6,6}{⽷,⽕}[HSK 7-9]
  \definition[盏]{s.}{semáforo vermelho | Figurativo: barreira; proibição | luz vermelha}
\end{EntryWithPhonetic}

\begin{EntryWithPhonetic}{红火}{hong2huo5}{6,4}{⽷,⽕}[HSK 7-9]
  \definition{adj.}{florescente; próspero; descreve prosperidade, prosperidade e agitação | florescente; próspero (meio de vida, carreira)}
\end{EntryWithPhonetic}

\begin{EntryWithPhonetic}{红酒}{hong2jiu3}{6,10}{⽷,⾣}[HSK 3]
  \definition[瓶,杯,壶,斤,箱]{s.}{vinho tinto}
\end{EntryWithPhonetic}

\begin{EntryWithPhonetic}{红绿灯}{hong2lv4deng1}{6,11,6}{⽷,⽷,⽕}
  \definition[个]{s.}{semáforo; sinal de trânsito; os semáforos que orientam os veículos estão localizados principalmente em cruzamentos urbanos; o vermelho significa pare e o verde significa siga}
\end{EntryWithPhonetic}

\begin{EntryWithPhonetic}{红扑扑}{hong2pu1pu1}{6,5,5}{⽷,⼿,⼿}[HSK 7-9]
  \definition{adj.}{corado | vermelho | rosado}
\end{EntryWithPhonetic}

\begin{EntryWithPhonetic}{红润}{hong2run4}{6,10}{⽷,⽔}[HSK 7-9]
  \definition[张]{adj.}{avermelhado; rosado | suave, macio e rosado (pele, bochechas, etc.)}
\end{EntryWithPhonetic}

\begin{EntryWithPhonetic}{红色}{hong2se4}{6,6}{⽷,⾊}[HSK 2]
  \definition{adj.}{vermelho; revolucionário; símbolo da revolução ou da consciência política elevada}
  \definition{s.}{cor vermelha}
\end{EntryWithPhonetic}

\begin{EntryWithPhonetic}{红烧}{hong2shao1}{6,10}{⽷,⽕}
  \definition{s.}{guisado em molho de soja (prato)}
\end{EntryWithPhonetic}

\begin{EntryWithPhonetic}{红薯}{hong2shu3}{6,16}{⽷,⾋}[HSK 7-9]
  \definition[个]{s.}{batata doce}
\end{EntryWithPhonetic}

\begin{EntryWithPhonetic}{红线}{hong2xian4}{6,8}{⽷,⽷}
  \definition{s.}{linha vermelha}
\end{EntryWithPhonetic}

\begin{EntryWithPhonetic}{红心}{hong2xin1}{6,4}{⽷,⼼}
  \definition{s.}{coração vermelho, um coração leal à causa da revolução proletária | alvo | coração ♥ (em jogos de cartas) | red, heart-shaped symbol}
  \seealsoref{方片}{fang1 pian4}
  \seealsoref{黑桃}{hei1tao2}
  \seealsoref{梅花}{mei2hua1}
\end{EntryWithPhonetic}

\begin{EntryWithPhonetic}{红眼}{hong2yan3}{6,11}{⽷,⽬}[HSK 7-9]
  \definition{s.}{conjuntivite | Figurativo: inveja; ciúme}
  \definition{v.}{ficar furioso | Dialeto: ter inveja; ter ciúmes de}
\end{EntryWithPhonetic}

%%%%%%%%%% 宏 %%%%%%%%%%
\subsection*{宏}\addcontentsline{loh}{figure}{宏 \dpy{hong2}}

\begin{EntryWithPhonetic}{宏}{hong2}{7}{⼧}
  \definition*{s.}{Sobrenome: Hong}
  \definition{adj.}{grande; grandioso; magnífico}
  \definition{v.}{divulgar algo; promover algo; atualmente, geralmente é escrito como 弘}
  \seealsoref{弘}{hong2}
\end{EntryWithPhonetic}

\begin{EntryWithPhonetic}{宏大}{hong2da4}{7,3}{⼧,⼤}[HSK 6]
  \definition{adj.}{grande; ótimo | imenso; vasto}
\end{EntryWithPhonetic}

\begin{EntryWithPhonetic}{宏观}{hong2guan1}{7,6}{⼧,⾒}[HSK 7-9]
  \definition{adj.}{macroscópico; grande norma, relacionada ao todo}
  \definition{pref.}{macro-}
\end{EntryWithPhonetic}

\begin{EntryWithPhonetic}{宏伟}{hong2wei3}{7,6}{⼧,⼈}[HSK 7-9]
  \definition{adj.}{grandioso; magnífico; (escala, planta, etc.) magnífico e grandioso}
\end{EntryWithPhonetic}

%%%%%%%%%% 洪 %%%%%%%%%%
\subsection*{洪}\addcontentsline{loh}{figure}{洪 \dpy{hong2}}

\begin{EntryWithPhonetic}{洪}{hong2}{9}{⽔}
  \definition*{s.}{Sobrenome: Hong}
  \definition{adj.}{alto; vasto | grande; grandioso}
  \definition[场]{s.}{enchente; inundação}
\end{EntryWithPhonetic}

\begin{EntryWithPhonetic}{洪亮}{hong2liang4}{9,9}{⽔,⼇}[HSK 7-9]
  \definition{adj.}{alto e claro; ressonante; sonoro}
\end{EntryWithPhonetic}

\begin{EntryWithPhonetic}{洪水}{hong2shui3}{9,4}{⽔,⽔}[HSK 6]
  \definition[场]{s.}{dilúvio; inundação; enchente; um aumento repentino em um rio causado por chuva forte ou derretimento de neve}
\end{EntryWithPhonetic}

%%%%%%%%%% 哄 %%%%%%%%%%
\subsection*{哄}\addcontentsline{loh}{figure}{哄 \dpy{hong3}}

\begin{EntryWithPhonetic}{哄}{hong3}{9}{⼝}[HSK 7-9]
  \definition{v.}{brincar; enganar; tapear | persuadir; agradar os outros com palavras ou ações, especialmente observando ou cuidando de crianças}
  \seeref{hong1}
  \seeref{hong4}
\end{EntryWithPhonetic}

\begin{EntryWithPhonetic}{哄}{hong4}{9}{⼝}[HSK 7-9]
  \definition{s.}{comoção; tumulto}
  \seeref{hong1}
  \seeref{hong3}
\end{EntryWithPhonetic}

%%%%%%%%%% 喉 %%%%%%%%%%
\subsection*{喉}\addcontentsline{loh}{figure}{喉 \dpy{hou2}}

\begin{EntryWithPhonetic}{喉}{hou2}{12}{⼝}
  \definition{s.}{laringe; garganta; a parte do órgão respiratório de humanos e vertebrados terrestres, localizada entre a faringe e a traqueia, tem as funções de ventilação e pronúncia; a faringe e a laringe são geralmente misturadas e chamadas de garganta ou caixa vocal}
\end{EntryWithPhonetic}

\begin{EntryWithPhonetic}{喉咙}{hou2long2}{12,8}{⼝,⼝}[HSK 7-9]
  \definition{s.}{garganta; laringe}
\end{EntryWithPhonetic}

%%%%%%%%%% 猴 %%%%%%%%%%
\subsection*{猴}\addcontentsline{loh}{figure}{猴 \dpy{hou2}}

\begin{EntryWithPhonetic}{猴}{hou2}{12}{⽝}[HSK 5]
  \definition{adj.}{esperto; inteligente; perspicaz | travesso (menino)}
  \definition[只,群]{s.}{macaco}
\end{EntryWithPhonetic}

\begin{EntryWithPhonetic}{猴子}{hou2zi5}{12,3}{⽝,⼦}
  \definition[只]{s.}{macaco}
\end{EntryWithPhonetic}

%%%%%%%%%% 吼 %%%%%%%%%%
\subsection*{吼}\addcontentsline{loh}{figure}{吼 \dpy{hou3}}

\begin{EntryWithPhonetic}{吼}{hou3}{7}{⼝}[HSK 7-9]
  \definition{v.}{rugido; uivo | rugido; gritar alto quando estiver com raiva}
\end{EntryWithPhonetic}

%%%%%%%%%% 后 %%%%%%%%%%
\subsection*{后}\addcontentsline{loh}{figure}{后 \dpy{hou4}}

\begin{EntryWithPhonetic}{后}{hou4}{6}{⼝}[HSK 1]
  \definition*{s.}{Sobrenome: Hou}
  \definition{s.}{atrás; traseiro; a direção oposta àquela para a qual a pessoa está voltada; a direção oposta àquela para a qual a parte de trás de uma casa está voltada | depois; mais tarde no tempo; futuro | último | posteridade; descendência | rainha; imperatriz | governante; soberano; monarca antigo}
  \antonymref{前}{qian2}
  \antonymref{先}{xian1}
\end{EntryWithPhonetic}

\begin{EntryWithPhonetic}{后备}{hou4bei4}{6,8}{⼝,⼡}[HSK 7-9]
  \definition{s.}{reserva; preparado para reabastecimento (pessoal, suprimentos, etc.)}
\end{EntryWithPhonetic}

\begin{EntryWithPhonetic}{后备箱}{hou4bei4xiang1}{6,8,15}{⼝,⼡,⾋}[HSK 7-9]
  \definition{s.}{porta-malas (de um carro)}
\end{EntryWithPhonetic}

\begin{EntryWithPhonetic}{后边}{hou4bian5}{6,5}{⼝,⾡}[HSK 1]
  \definition{adv.}{costas; traseira; atrás}
\end{EntryWithPhonetic}

\begin{EntryWithPhonetic}{后代}{hou4dai4}{6,5}{⼝,⼈}[HSK 7-9]
  \definition{s.}{períodos posteriores (na história); eras posteriores; a era após uma certa era | gerações posteriores; posteridade; descendentes; gerações futuras}
\end{EntryWithPhonetic}

\begin{EntryWithPhonetic}{后盾}{hou4dun4}{6,9}{⼝,⽬}[HSK 7-9]
  \definition{s.}{apoio; força de apoio | suporte; assistência; suporte; apoiador}
\end{EntryWithPhonetic}

\begin{EntryWithPhonetic}{后方}{hou4 fang1}{6,4}{⼝,⽅}
  \definition{s.}{traseira; retaguarda | na parte de trás; na parte traseira}
  \antonymref{前方}{qian2fang1}
  \antonymref{前线}{qian2xian4}
\end{EntryWithPhonetic}

\begin{EntryWithPhonetic}{后顾之忧}{hou4gu4zhi1you1}{6,10,3,7}{⼝,⾴,⼂,⼼}[HSK 7-9]
  \definition{expr.}{preocupações com o que ficou para trás; preocupações com problemas futuros; preocupações persistentes; preocupações não resolvidas; preocupações ou problemas potenciais; ``Ansiedade que exige olhar para trás.''; refere"-se a preocupações com o lar, a família ou o futuro que surgem ao seguir em frente ou sair}
\end{EntryWithPhonetic}

\begin{EntryWithPhonetic}{后果}{hou4guo3}{6,8}{⼝,⽊}[HSK 3]
  \definition{s.}{consequência; resultado (geralmente negativo)}
\end{EntryWithPhonetic}

\begin{EntryWithPhonetic}{后悔}{hou4hui3}{6,10}{⼝,⼼}[HSK 5]
  \definition{v.}{lamentar; ter remorso; arrepender-se}
\end{EntryWithPhonetic}

\begin{EntryWithPhonetic}{后来}{hou4lai2}{6,7}{⼝,⽊}[HSK 2]
  \definition{adv.}{mais tarde; depois; refere"-se a um período posterior a um determinado momento no passado}
\end{EntryWithPhonetic}

\begin{EntryWithPhonetic}{后面}{hou4mian5}{6,9}{⼝,⾯}[HSK 3]
  \definition{adv.}{parte de trás; retaguarda; atrás; a parte posterior do espaço ou localização | mais tarde; depois; no futuro; a parte posterior de um artigo ou discurso em relação ao que está sendo narrado no momento}
\end{EntryWithPhonetic}

\begin{EntryWithPhonetic}{后年}{hou4nian2}{6,6}{⼝,⼲}[HSK 3]
  \definition{s.}{daqui a dois anos; no ano seguinte ao próximo ano}
\end{EntryWithPhonetic}

\begin{EntryWithPhonetic}{后期}{hou4qi1}{6,12}{⼝,⽉}[HSK 7-9]
  \definition{s.}{estágio posterior; período posterior; a última fase de um período}
\end{EntryWithPhonetic}

\begin{EntryWithPhonetic}{后勤}{hou4qin2}{6,13}{⼝,⼒}[HSK 7-9]
  \definition{s.}{logística; serviços de retaguarda; todo o trabalho de fornecimento de áreas distantes da linha de frente para as áreas de linha de frente; trabalho administrativo em agências governamentais, empresas, etc., incluindo finanças, reparos, etc.}
\end{EntryWithPhonetic}

\begin{EntryWithPhonetic}{后人}{hou4ren2}{6,2}{⼝,⼈}[HSK 7-9]
  \definition{s.}{gerações posteriores; gerações futuras | posteridade; descendentes; futuridade}
\end{EntryWithPhonetic}

\begin{EntryWithPhonetic}{后台}{hou4tai2}{6,5}{⼝,⼝}[HSK 7-9]
  \definition{s.}{bastidores; plano de fundo | apoiador de bastidores; apoiador dos bastidores; uma metáfora para uma pessoa ou grupo que manipula ou apoia algo nos bastidores}
\end{EntryWithPhonetic}

\begin{EntryWithPhonetic}{后天}{hou4tian1}{6,4}{⼝,⼤}[HSK 1]
  \definition{s.}{depois de amanhã; período em que uma pessoa ou animal vive e cresce sozinho após deixar o útero materno}
  \antonymref{先天}{xian1tian1}
\end{EntryWithPhonetic}

\begin{EntryWithPhonetic}{后头}{hou4tou5}{6,5}{⼝,⼤}[HSK 4]
  \definition{adv.}{posteriormente; atrás; mais tarde}
  \definition{s.}{a parte de trás; a parte traseira}
\end{EntryWithPhonetic}

\begin{EntryWithPhonetic}{后退}{hou4tui4}{6,9}{⼝,⾡}[HSK 7-9]
  \definition{v.}{recuar; retrocerder; retornar (para um lugar posterior ou para um estágio anterior de desenvolvimento)}
\end{EntryWithPhonetic}

\begin{EntryWithPhonetic}{后续}{hou4xu4}{6,11}{⼝,⽷}[HSK 7-9]
  \definition{adj.}{subsequente; de acompanhamento; decorrente; seguimento}
  \definition{v.}{casar novamente após a morte da esposa}
\end{EntryWithPhonetic}

\begin{EntryWithPhonetic}{后遗症}{hou4yi2zheng4}{6,12,10}{⼝,⾡,⽧}[HSK 7-9]
  \definition{s.}{sequelas; sintomas como defeitos ou disfunções orgânicas que permanecem após a recuperação de certas doenças | ressaca; efeito colateral; consequência; efeito residual}
\end{EntryWithPhonetic}

\begin{EntryWithPhonetic}{后裔}{hou4yi4}{6,13}{⼝,⾐}[HSK 7-9]
  \definition{s.}{descendente (de uma pessoa morta); prole | descendente; posteridade; progênie}
\end{EntryWithPhonetic}

\begin{EntryWithPhonetic}{后者}{hou4zhe3}{6,8}{⼝,⽼}[HSK 7-9]
  \definition{pron.}{o último; a última de duas ou mais pessoas ou coisas mencionadas ou autoevidentes}
  \definition{s.}{o último}
  \antonymref{前者}{qian2zhe3}
\end{EntryWithPhonetic}

%%%%%%%%%% 厚 %%%%%%%%%%
\subsection*{厚}\addcontentsline{loh}{figure}{厚 \dpy{hou4}}

\begin{EntryWithPhonetic}{厚}{hou4}{9}{⼚}[HSK 4]
  \definition*{s.}{Sobrenome: Hou}
  \definition{adj.}{espesso; grosso | profundo | gentil; magnânimo | grande; generoso | rico ou forte em sabor}
  \definition[米,厘米]{s.}{espessura | profundidade}
  \definition{v.}{favorecer; enfatizar}
  \antonymref{薄}{bao2}
\end{EntryWithPhonetic}

\begin{EntryWithPhonetic}{厚道}{hou4dao5}{9,12}{⼚,⾡}[HSK 7-9]
  \definition{adj.}{honesto e gentil; sincero e generoso}
\end{EntryWithPhonetic}

\begin{EntryWithPhonetic}{厚度}{hou4du4}{9,9}{⼚,⼴}[HSK 7-9]
  \definition{s.}{espessura; a distância entre a parte superior e inferior de um objeto plano}
\end{EntryWithPhonetic}

%%%%%%%%%% 候 %%%%%%%%%%
\subsection*{候}\addcontentsline{loh}{figure}{候 \dpy{hou4}}

\begin{EntryWithPhonetic}{候}{hou4}{10}{⼈}
  \definition*{s.}{Sobrenome: Hou}
  \definition{s.}{tempo; estação | condição; estado | situação meteorológica | uma unidade tradicional de tempo no antigo calendário chinês; antigamente, cinco dias constituíam uma estação, o que ainda é usado na meteorologia hoje em dia}
  \definition{v.}{esperar; aguardar | perguntar depois | assistir; observar}
\end{EntryWithPhonetic}

\begin{EntryWithPhonetic}{候选人}{hou4xuan3ren2}{10,9,2}{⼈,⾡,⼈}[HSK 7-9]
  \definition[个,名,位]{s.}{candidato}
\end{EntryWithPhonetic}

%%%%%%%%%% 呼 %%%%%%%%%%
\subsection*{呼}\addcontentsline{loh}{figure}{呼 \dpy{hu1}}

\begin{EntryWithPhonetic}{呼}{hu1}{8}{⼝}
  \definition*{s.}{Sobrenome: Hu}
  \definition{s.}{Onomatopéia: descreve o som do vento}
  \definition{v.}{expirar | gritar; clamar | chamar; ligar; ligar para alguém}
\end{EntryWithPhonetic}

\begin{EntryWithPhonetic}{呼风唤雨}{hu1feng1-huan4yu3}{8,4,10,8}{⼝,⾵,⼝,⾬}[HSK 7-9]
  \definition{expr.}{``Fazer vento e chover.''; refere"-se originalmente ao poder mágico de imortais e taoístas; atualmente, é usado como metáfora para a capacidade de controlar a natureza e, às vezes, como metáfora para a realização de atividades inflamatórias; invocar vento e chuva, exercer poderes mágicos; causar problemas}
\end{EntryWithPhonetic}

\begin{EntryWithPhonetic}{呼唤}{hu1huan4}{8,10}{⼝,⼝}[HSK 7-9]
  \definition{v.}{chamar; gritar para}
\end{EntryWithPhonetic}

\begin{EntryWithPhonetic}{呼救}{hu1jiu4}{8,11}{⼝,⽁}[HSK 7-9]
  \definition{v.}{pedir ajuda; enviar sinais de SOS}
\end{EntryWithPhonetic}

\begin{EntryWithPhonetic}{呼啦啦}{hu1 la1 la1}{8,11,11}{⼝,⼝,⼝}
  \definition{s.}{Onomatopéia: som de bater asas}
\end{EntryWithPhonetic}

\begin{EntryWithPhonetic}{呼声}{hu1sheng1}{8,7}{⼝,⼠}[HSK 7-9]
  \definition[片]{s.}{choro; voz}[良心的呼声。===A voz da consciência.]
\end{EntryWithPhonetic}

\begin{EntryWithPhonetic}{呼吸}{hu1xi1}{8,6}{⼝,⼝}[HSK 4]
  \definition{s.}{um suspiro; metáfora para um período de tempo muito curto}
  \definition{v.}{respirar}
\end{EntryWithPhonetic}

\begin{EntryWithPhonetic}{呼啸}{hu1xiao4}{8,11}{⼝,⼝}
  \definition{v.}{assobiar}
\end{EntryWithPhonetic}

\begin{EntryWithPhonetic}{呼应}{hu1ying4}{8,7}{⼝,⼴}[HSK 7-9]
  \definition{v.}{ecoar; trabalhar em conjunto (com alguém); entrar em contato ou cuidar um do outro um dia de cada vez}
\end{EntryWithPhonetic}

\begin{EntryWithPhonetic}{呼吁}{hu1yu4}{8,6}{⼝,⼝}[HSK 7-9]
  \definition{v.}{apelar; chamar; apelar a um indivíduo ou sociedade, solicitar assistência ou hospedar um apelo a um indivíduo ou sociedade, na esperança de ganhar simpatia e apoio}
\end{EntryWithPhonetic}

%%%%%%%%%% 忽 %%%%%%%%%%
\subsection*{忽}\addcontentsline{loh}{figure}{忽 \dpy{hu1}}

\begin{EntryWithPhonetic}{忽}{hu1}{8}{⼼}
  \definition*{s.}{Sobrenome: Hu}
  \definition{adv.}{agora\dots, agora\dots | de repente; subitamente}[天气忽冷忽热。===O clima está frio em um minuto e quente no outro.]
  \definition{v.}{negligenciar; ignorar; não prestar atenção; não levar a sério}
\end{EntryWithPhonetic}

\begin{EntryWithPhonetic}{忽高忽低}{hu1gao1-hu1di1}{8,10,8,7}{⼼,⾼,⼼,⼈}[HSK 7-9]
  \definition{expr.}{``Altos e baixos.''; ora alto, ora baixo}
\end{EntryWithPhonetic}

\begin{EntryWithPhonetic}{忽略}{hu1lve4}{8,11}{⼼,⽥}[HSK 6]
  \definition{v.}{negligenciar; ignorar; não perceber}
\end{EntryWithPhonetic}

\begin{EntryWithPhonetic}{忽然}{hu1ran2}{8,12}{⼼,⽕}[HSK 2]
  \definition{adv.}{repentinamente; de repente; sem aviso prévio; significa que algo aconteceu de forma rápida e inesperada}
\end{EntryWithPhonetic}

\begin{EntryWithPhonetic}{忽视}{hu1shi4}{8,8}{⼼,⾒}[HSK 4]
  \definition{v.}{ignorar; negligenciar; menosprezar; desprezar; dar de ombros}
\end{EntryWithPhonetic}

\begin{EntryWithPhonetic}{忽悠}{hu1you5}{8,11}{⼼,⼼}[HSK 7-9]
  \definition{v.}{balançar; cintilar; sacudir | enganar; enganar alguém}
\end{EntryWithPhonetic}

%%%%%%%%%% 糊 %%%%%%%%%%
\subsection*{糊}\addcontentsline{loh}{figure}{糊 \dpy{hu1}}

\begin{EntryWithPhonetic}{糊}{hu1}{15}{⽶}
  \definition{v.}{colar; untar; usar uma pasta mais espessa para revestir costuras, furos ou superfícies planas}
  \seeref{hu2}
  \seeref{hu4}
\end{EntryWithPhonetic}

%%%%%%%%%% 和 %%%%%%%%%%
\subsection*{和}\addcontentsline{loh}{figure}{和 \dpy{hu2}}

\begin{EntryWithPhonetic}{和}{hu2}{8}{⼝}
  \definition{v.}{completar um conjunto de Mahjong, 麻将, ou cartas de baralho}
  \seeref{he2}
  \seeref{he4}
  \seeref{huo2}
  \seeref{huo4}
  \seealsoref{麻将}{ma2jiang4}
\end{EntryWithPhonetic}

%%%%%%%%%% 胡 %%%%%%%%%%
\subsection*{胡}\addcontentsline{loh}{figure}{胡 \dpy{hu2}}

\begin{EntryWithPhonetic}{胡}{hu2}{9}{⾁}
  \definition*{s.}{Sobrenome: Hu}
  \definition{adj.}{introduzidos de nacionalidades do norte e do oeste ou do exterior | nos tempos antigos, o termo ``Oriente e Ocidente'' se referia às minorias étnicas do norte e do oeste, e também, de modo geral, às pessoas do exterior}
  \definition{adv.}{imprudentemente; desenfreadamente; escandalosamente; sem lei, ordem ou razão}
  \definition{pron.}{``Por que?''; palavras interrogativas: 为什么, 何故}
  \definition{s.}{nos tempos antigos, geralmente se referia às minorias étnicas do norte e do oeste | violino chinês | barba; bigode}
  \seealsoref{何故}{he2gu4}
  \seealsoref{为什么}{wei4shen2me5}
\end{EntryWithPhonetic}

\begin{EntryWithPhonetic}{胡萝卜}{hu2luo2bo5}{9,11,2}{⾁,⾋,⼘}
  \definition{s.}{cenoura}
\end{EntryWithPhonetic}

\begin{EntryWithPhonetic}{胡闹}{hu2nao4}{9,8}{⾁,⾾}[HSK 7-9]
  \definition{v.}{correr solto; ser travesso; causar problemas; agir de forma irracional | agir intencionalmente; fazer uma cena; agir de forma imprudente; fazer coisas de forma imprudente}
\end{EntryWithPhonetic}

\begin{EntryWithPhonetic}{胡琴}{hu2qin2}{9,12}{⾁,⽟}
  \definition{s.}{huqin, um termo geral para certos instrumentos de arco de duas cordas, como 二胡, 京胡, etc. | família de violinos chineses de duas cordas, com caixa de ressonância de madeira revestida de pele de cobra e arco de bambu com corda de crina de cavalo}
  \seealsoref{二胡}{er4hu2}
  \seealsoref{京胡}{jing1hu2}
\end{EntryWithPhonetic}

\begin{EntryWithPhonetic}{胡说}{hu2shuo1}{9,9}{⾁,⾔}[HSK 7-9]
  \definition{v.}{falar bobagens}
\end{EntryWithPhonetic}

\begin{EntryWithPhonetic}{胡思乱想}{hu2si1-luan4xiang3}{9,9,7,13}{⾁,⼼,⼄,⼼}[HSK 7-9]
  \definition{expr.}{``Tem uma abelha em sua capota.''; deixar-se levar pela fantasia; dar lugar a fantasias tolas; deixar a imaginação correr solta; divagar}
\end{EntryWithPhonetic}

\begin{EntryWithPhonetic}{胡同儿}{hu2tong4r5}{9,6,2}{⾁,⼝,⼉}[HSK 5]
  \definition{s.}{beco}
\end{EntryWithPhonetic}

\begin{EntryWithPhonetic}{胡同}{hu2tong5}{9,6}{⾁,⼝}
  \definition[条,个]{s.}{beco; rua pequena}
\end{EntryWithPhonetic}

\begin{EntryWithPhonetic}{胡子}{hu2zi5}{9,3}{⾁,⼦}[HSK 5]
  \definition[团,根,个,撮]{s.}{barba; bigode | bandido; salteador}
\end{EntryWithPhonetic}

%%%%%%%%%% 壶 %%%%%%%%%%
\subsection*{壶}\addcontentsline{loh}{figure}{壶 \dpy{hu2}}

\begin{EntryWithPhonetic}{壶}{hu2}{10}{⼠}[HSK 6]
  \definition*{s.}{Sobrenome: Hu}
  \definition[个,把]{s.}{chaleira; panela | garrafa; frasco; recipiente para líquidos}
\end{EntryWithPhonetic}

%%%%%%%%%% 核 %%%%%%%%%%
\subsection*{核}\addcontentsline{loh}{figure}{核 \dpy{hu2}}

\begin{EntryWithPhonetic}{核}{hu2}{10}{⽊}
  \definition{s.}{semente; o mesmo que 核}
  \seeref{he2}
\end{EntryWithPhonetic}

%%%%%%%%%% 斛 %%%%%%%%%%
\subsection*{斛}\addcontentsline{loh}{figure}{斛 \dpy{hu2}}

\begin{EntryWithPhonetic}{斛}{hu2}{11}{⽃}
  \definition*{s.}{Sobrenome: Hu}
  \definition{s.}{Arcaico: uma medida seca usada antigamente, originalmente igual a 10 dou (斗), mais tarde 5 dou}
  \seealsoref{斗}{dou4}
\end{EntryWithPhonetic}

%%%%%%%%%% 湖 %%%%%%%%%%
\subsection*{湖}\addcontentsline{loh}{figure}{湖 \dpy{hu2}}

\begin{EntryWithPhonetic}{湖}{hu2}{12}{⽔}[HSK 2]
  \definition*{s.}{Huzhou, abreviação de 湖州 | Um nome que se refere às províncias de Hunan, 湖南,  e Hubei, 湖北}
  \definition[个,片]{s.}{lago}
  \seealsoref{湖北}{hu2bei3}
  \seealsoref{湖南}{hu2nan2}
  \seealsoref{湖州}{hu2zhou1}
\end{EntryWithPhonetic}

\begin{EntryWithPhonetic}{湖北}{hu2bei3}{12,5}{⽔,⼔}
  \definition*{s.}{Província de Hubei (Hupeh), na China central}
\end{EntryWithPhonetic}

\begin{EntryWithPhonetic}{湖南}{hu2nan2}{12,9}{⽔,⼗}
  \definition*{s.}{Província de Hunan}
\end{EntryWithPhonetic}

\begin{EntryWithPhonetic}{湖泊}{hu2po1}{12,8}{⽔,⽔}[HSK 7-9]
  \definition[个,片,些]{s.}{lago; nome geral para lagos}[湖泊中有丰富的鱼类。===O lago é abundante em peixes.]
\end{EntryWithPhonetic}

\begin{EntryWithPhonetic}{湖州}{hu2zhou1}{12,6}{⽔,⼮}
  \definition*{s.}{Cidade de Huzhou, em Zhejiang}
\end{EntryWithPhonetic}

%%%%%%%%%% 葫 %%%%%%%%%%
\subsection*{葫}\addcontentsline{loh}{figure}{葫 \dpy{hu2}}

\begin{EntryWithPhonetic}{葫}{hu2}{12}{⾋}
  \definition{s.}{cabaça}
\end{EntryWithPhonetic}

\begin{EntryWithPhonetic}{葫芦}{hu2lu5}{12,7}{⾋,⾋}
  \definition{adj.}{confuso}
  \definition{s.}{cabaça | termo genérico para bloco e equipamento (ou partes dele)}
\end{EntryWithPhonetic}

%%%%%%%%%% 糊 %%%%%%%%%%
\subsection*{糊}\addcontentsline{loh}{figure}{糊 \dpy{hu2}}

\begin{EntryWithPhonetic}{糊}{hu2}{15}{⽶}[HSK 7-9]
  \definition{adj.}{queimado}
  \definition{s.}{mingau; pasta; papa}
  \definition{v.}{colar com pasta; colar | (comida) ser queimado}
  \seeref{hu1}
  \seeref{hu4}
\end{EntryWithPhonetic}

\begin{EntryWithPhonetic}{糊里糊涂}{hu2 li5 hu2tu5}{15,7,15,10}{⽶,⾥,⽶,⽔}
  \definition{adj.}{desnorteado | perturbado}
\end{EntryWithPhonetic}

\begin{EntryWithPhonetic}{糊涂}{hu2tu5}{15,10}{⽶,⽔}[HSK 7-9]
  \definition{adj.}{confuso; perplexo; desnorteado; com compreensão pouco clara ou confusa das coisas | confuso; com conteúdo confuso}
\end{EntryWithPhonetic}

%%%%%%%%%% 蝴 %%%%%%%%%%
\subsection*{蝴}\addcontentsline{loh}{figure}{蝴 \dpy{hu2}}

\begin{EntryWithPhonetic}{蝴}{hu2}{15}{⾍}
  \definition[对]{s.}{borboleta}
\end{EntryWithPhonetic}

\begin{EntryWithPhonetic}{蝴蝶}{hu2die2}{15,15}{⾍,⾍}
  \definition[只]{s.}{borboleta}
\end{EntryWithPhonetic}

%%%%%%%%%% 虎 %%%%%%%%%%
\subsection*{虎}\addcontentsline{loh}{figure}{虎 \dpy{hu3}}

\begin{EntryWithPhonetic}{虎}{hu3}{8}{⾌}[HSK 5]
  \definition*{s.}{Sobrenome: Hu}
  \definition{adj.}{corajoso; bravo; valente; vigoroso}
  \definition[只]{s.}{tigre}
  \definition{v.}{blefar; o mesmo que 唬 | parecer feroz; mostrar a aparência feroz de alguém}
  \seealsoref{唬}{hu3}
  \seealsoref{老虎}{lao3hu3}
\end{EntryWithPhonetic}

\begin{EntryWithPhonetic}{虎虎}{hu3hu3}{8,8}{⾌,⾌}
  \definition{adj.}{formidável | forte | vigoroso}
\end{EntryWithPhonetic}

\begin{EntryWithPhonetic}{虎口}{hu3kou3}{8,3}{⾌,⼝}
  \definition{s.}{lugar perigoso | cova do tigre}
\end{EntryWithPhonetic}

\begin{EntryWithPhonetic}{虎鼬}{hu3you4}{8,18}{⾌,⿏}
  \definition{s.}{doninha}
\end{EntryWithPhonetic}

%%%%%%%%%% 唬 %%%%%%%%%%
\subsection*{唬}\addcontentsline{loh}{figure}{唬 \dpy{hu3}}

\begin{EntryWithPhonetic}{唬}{hu3}{11}{⼝}
  \definition{v.}{blefar, exagerar para assustar ou confundir}
\end{EntryWithPhonetic}

%%%%%%%%%% 互 %%%%%%%%%%
\subsection*{互}\addcontentsline{loh}{figure}{互 \dpy{hu4}}

\begin{EntryWithPhonetic}{互}{hu4}{4}{⼆}
  \definition{adv.}{mutuamente; um ao outro}
  \definition{pron.}{um ao outro; mútuo}
\end{EntryWithPhonetic}

\begin{EntryWithPhonetic}{互补}{hu4bu3}{4,7}{⼆,⾐}[HSK 7-9]
  \definition{v.}{complementar; complementar-se}
  \antonymref{兼容}{jian1rong2}
\end{EntryWithPhonetic}

\begin{EntryWithPhonetic}{互动}{hu4dong4}{4,6}{⼆,⼒}[HSK 6]
  \definition{v.}{interagir; participar juntos; promover uns aos outros}
  \synonymref{沟通}{gou1tong1}
  \synonymref{交流}{jiao1liu2}
\end{EntryWithPhonetic}

\begin{EntryWithPhonetic}{互访}{hu4fang3}{4,6}{⼆,⾔}[HSK 7-9]
  \definition{s.}{visitas mútuas}
  \definition{v.}{trocar visitas}
\end{EntryWithPhonetic}

\begin{EntryWithPhonetic}{互利}{hu4li4}{4,7}{⼆,⼑}
  \definition{s.}{benefício mútuo}
  \synonymref{双赢}{shuang1ying2}
\end{EntryWithPhonetic}

\begin{EntryWithPhonetic}{互联网}{hu4lian2wang3}{4,12,6}{⼆,⽿,⽹}[HSK 3]
  \definition{s.}{\emph{Internet}; uma enorme rede conectando computadores e redes de computadores}
  \seealsoref{网际网路}{wang3 ji4 wang3 lu4}
  \seealsoref{网际网络}{wang3 ji4 wang3 luo4}
  \seealsoref{网路}{wang3 lu4}
\end{EntryWithPhonetic}

\begin{EntryWithPhonetic}{互相}{hu4xiang1}{4,9}{⼆,⽬}[HSK 3]
  \definition{adv.}{mutuamente; um ao outro; expressa uma relação de igualdade entre as partes}
  \synonymref{彼此}{bi3ci3}
  \synonymref{相互}{xiang1hu4}
  \antonymref{各自}{ge4zi4}
\end{EntryWithPhonetic}

\begin{EntryWithPhonetic}{互信}{hu4xin4}{4,9}{⼆,⼈}[HSK 7-9]
  \definition{s.}{confiança mútua}
  \definition{v.}{confiar um no outro}
\end{EntryWithPhonetic}

\begin{EntryWithPhonetic}{互助}{hu4zhu4}{4,7}{⼆,⼒}[HSK 7-9]
  \definition{v.}{ajudar uns aos outros; ajudar-se mutualmente}[我们应该互助合作。===Devemos ajudar e cooperar uns com os outros.]
  \synonymref{合作}{he2zuo4}
  \synonymref{配合}{pei4he5}
  \synonymref{团结}{tuan2jie2}
\end{EntryWithPhonetic}

%%%%%%%%%% 户 %%%%%%%%%%
\subsection*{户}\addcontentsline{loh}{figure}{户 \dpy{hu4}}

\begin{EntryWithPhonetic}{户}{hu4}{4}{⼾}[HSK 4][Kangxi 63]
  \definition*{s.}{Sobrenome: Hu}
  \definition[个]{s.}{porta com um painel; porta | domicílio; residência; família | status familiar | conta (banco)}
\end{EntryWithPhonetic}

\begin{EntryWithPhonetic}{户外}{hu4wai4}{4,5}{⼾,⼣}[HSK 6]
  \definition{s.}{ao ar livre; espaço aberto ao ar livre}
\end{EntryWithPhonetic}

%%%%%%%%%% 护 %%%%%%%%%%
\subsection*{护}\addcontentsline{loh}{figure}{护 \dpy{hu4}}

\begin{EntryWithPhonetic}{护}{hu4}{7}{⼿}[HSK 6]
  \definition{v.}{proteger; defender | blindar; ser parcial; proteger-se da censura}
\end{EntryWithPhonetic}

\begin{EntryWithPhonetic}{护理}{hu4li3}{7,11}{⼿,⽟}[HSK 7-9]
  \definition{v.}{cuidar; cooperar com médicos para tratar e cuidar de pacientes, idosos e deficientes | cuidar e proteger; proteger e gerenciar para permitir a vida normal ou o crescimento}
\end{EntryWithPhonetic}

\begin{EntryWithPhonetic}{护士}{hu4shi5}{7,3}{⼿,⼠}[HSK 4]
  \definition[名,位]{s.}{enfermeiro; pessoas especializadas em enfermagem em hospitais ou instituições epidemiológicas}
\end{EntryWithPhonetic}

\begin{EntryWithPhonetic}{护照}{hu4zhao4}{7,13}{⼿,⽕}[HSK 2]
  \definition[本,个]{s.}{passaporte; documento emitido pela autoridade competente do país para comprovar a nacionalidade e a identidade dos cidadãos que viajam para o exterior}
\end{EntryWithPhonetic}

%%%%%%%%%% 糊 %%%%%%%%%%
\subsection*{糊}\addcontentsline{loh}{figure}{糊 \dpy{hu4}}

\begin{EntryWithPhonetic}{糊}{hu4}{15}{⽶}
  \definition{s.}{pasta; comida que parece mingau}
  \seeref{hu1}
  \seeref{hu2}
\end{EntryWithPhonetic}

%%%%%%%%%% 化 %%%%%%%%%%
\subsection*{化}\addcontentsline{loh}{figure}{化 \dpy{hua1}}

\begin{EntryWithPhonetic}{化}{hua1}{4}{⼔}
  \variantof{花}
  \seeref{hua4}
\end{EntryWithPhonetic}

%%%%%%%%%% 花 %%%%%%%%%%
\subsection*{花}\addcontentsline{loh}{figure}{花 \dpy{hua1}}

\begin{EntryWithPhonetic}{花}{hua1}{7}{⾋}[HSK 1,2,4]
  \definition*{s.}{Sobrenome: Hua}
  \definition{adj.}{multicolorido; colorido | embaçado; obscuro; deslumbrado e confuso | extravagante; florido; vistoso}
  \definition[朵,支,束,把,盆,簇]{s.}{flor; órgãos de reprodução sexual de plantas com sementes | flor; planta ornamental |  qualquer coisa que se assemelhe a uma flor | fogos de artifício | padrão; design; design decorativo | flor; metáfora para a essência de uma causa | prostituta; cortesã; referindo"-se a prostitutas ou a assuntos relacionados a prostitutas | algodão | varíola | ferimento; ferida; lesões traumáticas sofridas em combate}
  \definition{v.}{gastar; despender; consumir}
\end{EntryWithPhonetic}

\begin{EntryWithPhonetic}{花瓣}{hua1ban4}{7,19}{⾋,⽠}[HSK 7-9]
  \definition[片,个]{s.}{pétala; um componente da corola, semelhante em estrutura a uma folha, mas com células contendo vários pigmentos, resultando em uma variedade de cores}
\end{EntryWithPhonetic}

\begin{EntryWithPhonetic}{花茶}{hua1cha2}{7,9}{⾋,⾋}
  \definition[杯,壶]{s.}{chá perfumado}
\end{EntryWithPhonetic}

\begin{EntryWithPhonetic}{花店}{hua1dian4}{7,8}{⾋,⼴}
  \definition{s.}{floricultura}
\end{EntryWithPhonetic}

\begin{EntryWithPhonetic}{花费}{hua1fei5}{7,9}{⾋,⾙}[HSK 6]
  \definition[笔]{s.}{dinheiro gasto; despesas | custo; gastos; desembolso | despesa}
  \definition{v.}{gastar (tempo ou dinheiro)}
\end{EntryWithPhonetic}

\begin{EntryWithPhonetic}{花卉}{hua1hui4}{7,5}{⾋,⼗}[HSK 7-9]
  \definition{s.}{flores e plantas | pintura de flores e plantas no estilo tradicional chinês}
\end{EntryWithPhonetic}

\begin{EntryWithPhonetic}{花脸}{hua1lian3}{7,11}{⾋,⾁}
  \definition*{s.}{Hualian, personagem do rosto florido, um nome popular para 净 (assim chamado devido à elaborada pintura facial)}
  \seealsoref{净}{jing4}
\end{EntryWithPhonetic}

\begin{EntryWithPhonetic}{花瓶}{hua1ping2}{7,10}{⾋,⽡}[HSK 6]
  \definition[个,对]{s.}{vaso de flores; vaso usado para arranjos florais colocado em ambientes internos como decoração | Figurativo: um ornamento; mulher empregada não por sua habilidade, mas por sua aparência; uma metáfora para uma pessoa ou coisa que é usada apenas para exibição e não tem uso prático}
\end{EntryWithPhonetic}

\begin{EntryWithPhonetic}{花儿}{hua1r5}{7,2}{⾋,⼉}
  \definition[朵,支,束,把,盆,簇]{s.}{flor}
\end{EntryWithPhonetic}

\begin{EntryWithPhonetic}{花生}{hua1sheng1}{7,5}{⾋,⽣}[HSK 6]
  \definition[把,颗,粒,袋]{s.}{amendoim}
  \seealsoref{落花生}{luo4 hua1 sheng1}
\end{EntryWithPhonetic}

\begin{EntryWithPhonetic}{花纹}{hua1wen2}{7,7}{⾋,⽷}[HSK 7-9]
  \definition[条,道,个]{s.}{figura; padrão decorativo; (padrões) várias listras e gráficos}
\end{EntryWithPhonetic}

\begin{EntryWithPhonetic}{花样}{hua1yang4}{7,10}{⾋,⽊}[HSK 7-9]
  \definition[种]{s.}{padrão, geralmente referindo"-se a vários padrões ou tipos | truque | variedade; padrão decorativo; padrão de bordado, geralmente cortado ou esculpido em papel}
\end{EntryWithPhonetic}

\begin{EntryWithPhonetic}{花样游泳}{hua1yang4 you2yong3}{7,10,12,8}{⾋,⽊,⽔,⽔}
  \definition{s.}{nado sincronizado}
\end{EntryWithPhonetic}

\begin{EntryWithPhonetic}{花椰菜}{hua1ye1cai4}{7,12,11}{⾋,⽊,⾋}
  \definition{s.}{couve-flor}
\end{EntryWithPhonetic}

\begin{EntryWithPhonetic}{花园}{hua1yuan2}{7,7}{⾋,⼞}[HSK 2]
  \definition[个,座]{s.}{jardim; um local onde se plantam flores e árvores para passear e descansar}
\end{EntryWithPhonetic}

%%%%%%%%%% 哗 %%%%%%%%%%
\subsection*{哗}\addcontentsline{loh}{figure}{哗 \dpy{hua1}}

\begin{EntryWithPhonetic}{哗}{hua1}{9}{⼝}
  \definition{s.}{(onomatopéia) sons de impacto, batida, fluxo de água, etc.}
  \seeref{hua2}
\end{EntryWithPhonetic}

\begin{EntryWithPhonetic}{哗啦啦}{hua1la1 la5}{9,11,11}{⼝,⼝,⼝}
  \definition{s.}{(onomatopéia) som de colisão, batida}
\end{EntryWithPhonetic}

%%%%%%%%%% 划 %%%%%%%%%%
\subsection*{划}\addcontentsline{loh}{figure}{划 \dpy{hua2}}

\begin{EntryWithPhonetic}{划}{hua2}{6}{⼑}[HSK 4]
  \definition{adj.}{rentável; vale (o esforço); compensa (fazer alguma coisa)}
  \definition{v.}{remar | ser vantajoso para alguém; ser uma pechincha | arranhar; cortar a superfície de; cortar em outra coisa com um objeto pontiagudo | arranhar; golpear;  esfregar uma coisa ou varrer sobre outra}
  \seeref{hua4}
\end{EntryWithPhonetic}

\begin{EntryWithPhonetic}{划船}{hua2 chuan2}{6,11}{⼑,⾈}[HSK 3]
  \definition[次,回]{s.}{remo (ato de remar); passeios de barco; a atividade ou esporte de ``remar um barco com remos''}
  \definition{v.}{remar um barco; a ação ou comportamento de mover um barco na água usando remos}
\end{EntryWithPhonetic}

\begin{EntryWithPhonetic}{划拳}{hua2quan2}{6,10}{⼑,⼿}
  \definition{pron.}{jogo de adivinhação de dedos; ao beber, duas pessoas levantam os dedos e dizem um número, quem disser o número que corresponde ao total de dedos ganha, o perdedor bebe}
  \definition{v.}{jogar o jogo de adivinhação de dedos (jogado em um jantar por duas pessoas)}
\end{EntryWithPhonetic}

\begin{EntryWithPhonetic}{划算}{hua2suan4}{6,14}{⼑,⽵}[HSK 7-9]
  \definition{adj.}{lucrativo; que vale a pena}
  \definition{v.}{calcular; pesar; planejar e esquematizar}
\end{EntryWithPhonetic}

\begin{EntryWithPhonetic}{划艇}{hua2ting3}{6,12}{⼑,⾈}
  \definition{s.}{barco a remo}
\end{EntryWithPhonetic}

%%%%%%%%%% 华 %%%%%%%%%%
\subsection*{华}\addcontentsline{loh}{figure}{华 \dpy{hua2}}

\begin{EntryWithPhonetic}{华}{hua2}{6}{⼗}
  \definition*{s.}{China; refere"-se à China (anteriormente conhecida como Huaxia, 华夏, mais tarde chamada de Zhonghua, 中华, ou simplesmente Hua, 华)}
  \definition{adj.}{esplêndido; magnífico | próspero; florescente | chamativo; extravagante; vaidoso | grisalho}
  \definition{s.}{corona; um halo colorido ao redor do sol ou da lua causado pela difração da luz através das nuvens | creme; melhor parte; a melhor parte das coisas | chinês; refere"-se à nacionalidade Han (língua e escrita) | vezes; anos; refere"-se a (bons) momentos | elixir; essência líquida; substâncias formadas pela sedimentação de minerais na água de nascente | Seu, palavra honorífica, usada para se referir a coisas relacionadas à outra pessoa}
  \seeref{hua4}
  \seealsoref{华夏}{hua2xia4}
  \seealsoref{中华}{zhong1hua2}
\end{EntryWithPhonetic}

\begin{EntryWithPhonetic}{华丽}{hua2li4}{6,7}{⼗,⼀}[HSK 7-9]
  \definition{adj.}{magnífico; resplandecente; deslumbrante; lindo e radiante}
\end{EntryWithPhonetic}

\begin{EntryWithPhonetic}{华侨}{hua2qiao2}{6,8}{⼗,⼈}[HSK 7-9]
  \definition[个,位,名]{s.}{chineses que vivem no exterior}
\end{EntryWithPhonetic}

\begin{EntryWithPhonetic}{华人}{hua2ren2}{6,2}{⼗,⼈}[HSK 3]
  \definition[名,位,个]{s.}{Chinês; chinês étnico | chineses no exterior; refere"-se a cidadãos estrangeiros de ascendência chinesa que obtiveram a nacionalidade do país em que residem}
\end{EntryWithPhonetic}

\begin{EntryWithPhonetic}{华盛顿}{hua2sheng4dun4}{6,11,10}{⼗,⽫,⾴}
  \definition*{s.}{Washington}
\end{EntryWithPhonetic}

\begin{EntryWithPhonetic}{华氏}{hua2shi4}{6,4}{⼗,⽒}
  \definition{s.}{graus Fahrenheit (°F)}
\end{EntryWithPhonetic}

\begin{EntryWithPhonetic}{华夏}{hua2xia4}{6,10}{⼗,⼢}
  \definition*{s.}{Huaxia, nome antigo da China | Catai}
\end{EntryWithPhonetic}

\begin{EntryWithPhonetic}{华裔}{hua2yi4}{6,13}{⼗,⾐}[HSK 7-9]
  \definition[位,名,个]{s.}{etnia chinesa; crianças nascidas de chineses no exterior no país de residência e que adquiriram a nacionalidade do país de residência}
\end{EntryWithPhonetic}

\begin{EntryWithPhonetic}{华语}{hua2yu3}{6,9}{⼗,⾔}[HSK 5]
  \definition*{s.}{Chinês (idioma)}
\end{EntryWithPhonetic}

%%%%%%%%%% 哗 %%%%%%%%%%
\subsection*{哗}\addcontentsline{loh}{figure}{哗 \dpy{hua2}}

\begin{EntryWithPhonetic}{哗}{hua2}{9}{⼝}
  \definition{v.}{ser barulhento; fazer alvoroço}
  \seeref{hua1}
\end{EntryWithPhonetic}

\begin{EntryWithPhonetic}{哗变}{hua2bian4}{9,8}{⼝,⼜}[HSK 7-9]
  \definition{s.}{(um exército) motim | rebelião}
\end{EntryWithPhonetic}

\begin{EntryWithPhonetic}{哗然}{hua2ran2}{9,12}{⼝,⽕}[HSK 7-9]
  \definition{adj.}{Literário: barulhento; em alvoroço; em comoção}[举座哗然。===Todo o público ficou em alvoroço.]
\end{EntryWithPhonetic}

%%%%%%%%%% 滑 %%%%%%%%%%
\subsection*{滑}\addcontentsline{loh}{figure}{滑 \dpy{hua2}}

\begin{EntryWithPhonetic}{滑}{hua2}{12}{⽔}[HSK 5]
  \definition*{s.}{Sobrenome: Hua}
  \definition{adj.}{escorregadio; liso; objetos com superfícies lisas e baixo atrito | astuto; ardiloso; escorregadio}
  \definition{v.}{escorregar; deslizar | se atrapalhar; se safar de algo}
\end{EntryWithPhonetic}

\begin{EntryWithPhonetic}{滑冰}{hua2bing1}{12,6}{⽔,⼎}[HSK 7-9]
  \definition{s.}{patinação no gelo; um evento esportivo em que os atletas usam patins especiais para patinar no gelo, competindo em velocidade ou realizando manobras}
  \definition{v.}{patinar; patinar no gelo; deslizar no gelo}
\end{EntryWithPhonetic}

\begin{EntryWithPhonetic}{滑稽}{hua2ji1}{12,15}{⽔,⽲}[HSK 7-9]
  \definition{adj.}{engraçado; divertido; cômico; (palavras, ações ou gestos) que fazem as pessoas rirem}
  \definition{s.}{conversa cômica; um tipo de arte popular, popular nas áreas de Xangai, Jiangsu e Zhejiang, semelhante ao \emph{crosstalk}}
\end{EntryWithPhonetic}

\begin{EntryWithPhonetic}{滑梯}{hua2ti1}{12,11}{⽔,⽊}[HSK 7-9]
  \definition{s.}{escorregador infantil}
\end{EntryWithPhonetic}

\begin{EntryWithPhonetic}{滑雪}{hua2/xue3}{12,11}{⽔,⾬}[HSK 7-9]
  \definition{v.+compl.}{esquiar; praticar esqui; usar pranchas especiais nos pés para deslizar na neve}
\end{EntryWithPhonetic}

%%%%%%%%%% 豁 %%%%%%%%%%
\subsection*{豁}\addcontentsline{loh}{figure}{豁 \dpy{hua2}}

\begin{EntryWithPhonetic}{豁}{hua2}{17}{⾕}
  \definition{v.}{jogar o jogo de adivinhação de dedos chinês (Huoquan); o mesmo que 划拳}[豁拳规则很简单。===As regras do Huoquan são muito simples.]
  \seeref{huo1}
  \seeref{huo4}
  \seealsoref{划拳}{hua2quan2}
\end{EntryWithPhonetic}

%%%%%%%%%% 化 %%%%%%%%%%
\subsection*{化}\addcontentsline{loh}{figure}{化 \dpy{hua4}}

\begin{EntryWithPhonetic}{化}{hua4}{4}{⼔}[HSK 3]
  \definition*{s.}{Sobrenome: Hua}
  \definition{s.}{química | cultura; costumes e tradições}
  \definition{suf.}{modernizar; modernização; anexado a componentes nominais ou adjetivos para formar verbos, indicando a transformação em algum estado ou qualidade}
  \definition{v.}{mudar; converter; transformar; causasr mudanças | converter; influenciar; influenciar e induzir as pessoas com palavras e ações, levando-as a mudar | derreter; dissolver; fundir | digerir | queimar; reduzir a cinzas | (monge, taoísta) morrer | (de monges budistas ou sacerdotes taoístas) pedir esmolas; arrecadar bens, alimentos, etc.}
  \seeref{hua1}
\end{EntryWithPhonetic}

\begin{EntryWithPhonetic}{化肥}{hua4fei2}{4,8}{⼔,⾁}[HSK 7-9]
  \definition[袋,吨,种]{s.}{fertilizante químico}
\end{EntryWithPhonetic}

\begin{EntryWithPhonetic}{化合}{hua4he2}{4,6}{⼔,⼝}
  \definition{s.}{combinação química}
  \definition{s.}{Química: combinar; sintentizar}
\end{EntryWithPhonetic}

\begin{EntryWithPhonetic}{化解}{hua4jie3}{4,13}{⼔,⾓}[HSK 6]
  \definition{v.}{resolver; eliminar; dissolver; neutralizar}
\end{EntryWithPhonetic}

\begin{EntryWithPhonetic}{化身}{hua4shen1}{4,7}{⼔,⾝}[HSK 7-9]
  \definition{s.}{encarnação; corporificação; personificação}
\end{EntryWithPhonetic}

\begin{EntryWithPhonetic}{化石}{hua4shi2}{4,5}{⼔,⽯}[HSK 5]
  \definition{s.}{fóssil; restos, relíquias ou vestígios de organismos antigos enterrados no solo e transformados em objetos semelhantes a pedras}
\end{EntryWithPhonetic}

\begin{EntryWithPhonetic}{化纤}{hua4xian1}{4,6}{⼔,⽷}[HSK 7-9]
  \definition[吨]{s.}{fibra sintética; abreviação de 化学纤维}[这件衣服是化纤材质。===Este vestido é feito de fibra sintética.]
  \seealsoref{化学纤维}{hua4 xue2 xian1 wei2}
\end{EntryWithPhonetic}

\begin{EntryWithPhonetic}{化险为夷}{hua4xian3wei2yi2}{4,9,4,6}{⼔,⾩,⼂,⼤}[HSK 7-9]
  \definition{expr.}{``Evite o perigo.''; transformar perigo em segurança; evitar um desastre}
\end{EntryWithPhonetic}

\begin{EntryWithPhonetic}{化学}{hua4xue2}{4,8}{⼔,⼦}
  \definition[门]{s.}{química; a ciência que estuda a composição, estrutura, propriedades e leis de mudança da matéria | celuloide}
\end{EntryWithPhonetic}

\begin{EntryWithPhonetic}{化学纤维}{hua4 xue2 xian1 wei2}{4,8,6,11}{⼔,⼦,⽷,⽷}
  \definition{s.}{fibra química | fibra sintética}
\end{EntryWithPhonetic}

\begin{EntryWithPhonetic}{化验}{hua4yan4}{4,10}{⼔,⾺}[HSK 7-9]
  \definition{v.}{fazer um teste de laboratório; conduzir um exame químico; usar métodos físicos ou químicos para examinar a composição e as propriedades das substâncias}
\end{EntryWithPhonetic}

\begin{EntryWithPhonetic}{化妆}{hua4/zhuang1}{4,6}{⼔,⼥}[HSK 7-9]
  \definition{v.+compl.}{maquiar-se; colocar maquiagem; usar algo para deixar o rosto mais bonito}
\end{EntryWithPhonetic}

%%%%%%%%%% 划 %%%%%%%%%%
\subsection*{划}\addcontentsline{loh}{figure}{划 \dpy{hua4}}

\begin{EntryWithPhonetic}{划}{hua4}{6}{⼑}[HSK 4]
  \definition*{s.}{Sobrenome: Hua}
  \definition{s.}{traço de um caracter chinês}
  \definition{v.}{delimitar; diferenciar; delinear | transferir; ceder | planejar; programar | desenhar; marcar; delinear; fazer linhas ou escrever como marcadores com uma caneta ou objeto semelhante a uma caneta}
  \seeref{hua2}
\end{EntryWithPhonetic}

\begin{EntryWithPhonetic}{划分}{hua4fen1}{6,4}{⼑,⼑}[HSK 5]
  \definition{v.}{dividir; particionar; reparticionar | diferenciar; encontrar aspectos diferentes}
\end{EntryWithPhonetic}

\begin{EntryWithPhonetic}{划时代}{hua4shi2dai4}{6,7,5}{⼑,⽇,⼈}[HSK 7-9]
  \definition{adj.}{marcando uma nova época; marcando época}[具有划时代的意义。===É de importância histórica.]
\end{EntryWithPhonetic}

%%%%%%%%%% 华 %%%%%%%%%%
\subsection*{华}\addcontentsline{loh}{figure}{华 \dpy{hua4}}

\begin{EntryWithPhonetic}{华}{hua4}{6}{⼗}
  \definition*{s.}{Huashan Mountain (na província de Shaanxi) | Sobrenome: Hua}
  \seeref{hua2}
\end{EntryWithPhonetic}

\begin{EntryWithPhonetic}{华山}{hua4shan1}{6,3}{⼗,⼭}
  \definition{s.}{Monte Hua em Shaanxi, montanha ocidental das Cinco Montanhas Sagradas (五岳)}
  \seealsoref{五岳}{wu3yue4}
\end{EntryWithPhonetic}

%%%%%%%%%% 画 %%%%%%%%%%
\subsection*{画}\addcontentsline{loh}{figure}{画 \dpy{hua4}}

\begin{EntryWithPhonetic}{画}{hua4}{8}{⽥}[HSK 2]
  \definition*{s.}{Sobrenome: Hua}
  \definition{clas.}{traços (de um caractere chinês)}
  \definition[张,幅]{s.}{desenho; pintura; imagem; figura desenhada | traço horizontal (em caracteres chineses)}
  \definition{v.}{desenhar; pintar | desenhar; marcar; assinar}
  \seealsoref{划}{hua4}
\end{EntryWithPhonetic}

\begin{EntryWithPhonetic}{画册}{hua4ce4}{8,5}{⽥,⼌}[HSK 7-9]
  \definition[部,本]{s.}{álbum de imagens; álbum de pinturas; pinturas ou imagens encadernadas}
\end{EntryWithPhonetic}

\begin{EntryWithPhonetic}{画地为牢}{hua4di4wei2lao2}{8,6,4,7}{⽥,⼟,⼂,⼧}
  \definition{expr.}{desenhar um círculo no chão para servir como uma prisão; restringir as atividades de alguém a uma área ou esfera designada; limitar; restringir | (literário) ser confinado dentro de um círculo desenhado no chão | (figurativo) limitar-se a uma gama restrita de atividades}
\end{EntryWithPhonetic}

\begin{EntryWithPhonetic}{画家}{hua4jia1}{8,10}{⽥,⼧}[HSK 2]
  \definition[个,位,名,些]{s.}{pintor; pessoa com talento para pintura}
\end{EntryWithPhonetic}

\begin{EntryWithPhonetic}{画龙点睛}{hua4long2-dian3jing1}{8,5,9,13}{⽥,⿓,⽕,⽬}[HSK 7-9]
  \definition{expr.}{``Toque final.''; dar vida a um dragão pintado colocando as pupilas dos seus olhos, adicionar o toque que dá vida a uma obra de arte; adicionar o toque final; adicionar uma palavra apropriada para concluir o ponto; adicionar uma ou duas palavras para finalizar o ponto; um toque crucial que reforça um ponto que de outra forma seria difícil de explicar; dar os retoques finais no trabalho de alguém; completar uma imagem}
\end{EntryWithPhonetic}

\begin{EntryWithPhonetic}{画面}{hua4mian4}{8,9}{⽥,⾯}[HSK 5]
  \definition[个,幅,帧]{s.}{quadro; aparência geral de uma imagem; imagem apresentada no quadro, na tela, etc.}
\end{EntryWithPhonetic}

\begin{EntryWithPhonetic}{画儿}{hua4r5}{8,2}{⽥,⼉}[HSK 2]
  \definition[幅,张]{s.}{imagem; desenho; pintura; obra de arte pintada}
\end{EntryWithPhonetic}

\begin{EntryWithPhonetic}{画蛇添足}{hua4she2-tian1zu2}{8,11,11,7}{⽥,⾍,⽔,⾜}[HSK 7-9]
  \definition{expr.}{arruinar o efeito adicionando algo supérfluo; dourar; fazer coisas desnecessárias pode sair pela culatra e piorar as coisas}
\end{EntryWithPhonetic}

\begin{EntryWithPhonetic}{画展}{hua4zhan3}{8,10}{⽥,⼫}[HSK 7-9]
  \definition{s.}{exposição de pinturas; exposição de arte}
\end{EntryWithPhonetic}

%%%%%%%%%% 话 %%%%%%%%%%
\subsection*{话}\addcontentsline{loh}{figure}{话 \dpy{hua4}}

\begin{EntryWithPhonetic}{话}{hua4}{8}{⾔}[HSK 1]
  \definition[句,段,番,种]{s.}{palavra; conversa; a voz que expressa os pensamentos quando falada, ou os caracteres que registram essa voz}
  \definition{v.}{falar sobre; falar a respeito}
\end{EntryWithPhonetic}

\begin{EntryWithPhonetic}{话费}{hua4fei4}{8,9}{⾔,⾙}[HSK 7-9]
  \definition{s.}{uma conta ou taxa telefônica; tarifas de uso do telefone; às vezes se refere simplesmente ao custo de uso do telefone}
\end{EntryWithPhonetic}

\begin{EntryWithPhonetic}{话剧}{hua4ju4}{8,10}{⾔,⼑}[HSK 3]
  \definition[场,幕,部,出,台]{s.}{drama moderno; peça de teatro; peça teatral representada através de diálogos e ações}
\end{EntryWithPhonetic}

\begin{EntryWithPhonetic}{话题}{hua4ti2}{8,15}{⾔,⾴}[HSK 3]
  \definition[个,种,项]{s.}{assunto de uma palestra; tópico de uma conversa; o foco da conversa}
\end{EntryWithPhonetic}

\begin{EntryWithPhonetic}{话筒}{hua4tong3}{8,12}{⾔,⽵}[HSK 7-9]
  \definition[个,只]{s.}{microfone; um termo geral para microfones | megafone; um tubo em forma de cone usado para falar alto com muitas pessoas próximas também é chamado de megafone | receptor (telefone); bocal}
\end{EntryWithPhonetic}

\begin{EntryWithPhonetic}{话语}{hua4yu3}{8,9}{⾔,⾔}[HSK 7-9]
  \definition[句]{s.}{palavras; fala; enunciado; discurso; palavras ditas}
\end{EntryWithPhonetic}

%%%%%%%%%% 怀 %%%%%%%%%%
\subsection*{怀}\addcontentsline{loh}{figure}{怀 \dpy{huai2}}

\begin{EntryWithPhonetic}{怀}{huai2}{7}{⼼}
  \definition*{s.}{Sobrenome: Huai}
  \definition{s.}{seio; peito | mente}
  \definition{v.}{manter em mente; estimar; abrigar | sentir falta; pensar em; ansiar por | conceber (uma criança)}
\end{EntryWithPhonetic}

\begin{EntryWithPhonetic}{怀抱}{huai2bao4}{7,8}{⼼,⼿}[HSK 7-9]
  \definition{s.}{seio; peito | ambição; aspiração; intenção}
  \definition{v.}{abraçar; carregar nos braços; segurar nos braços | estimar; ter em mente}
\end{EntryWithPhonetic}

\begin{EntryWithPhonetic}{怀旧}{huai2jiu4}{7,5}{⼼,⽇}[HSK 7-9]
  \definition{s.}{lembrança afetuosa de tempos passados | nostalgia}
  \definition{v.}{sentir ou demonstrar nostalgia; lembrar de tempos passados ou de velhos conhecidos (geralmente com pensamentos gentis)}
\end{EntryWithPhonetic}

\begin{EntryWithPhonetic}{怀里}{huai2li5}{7,7}{⼼,⾥}[HSK 7-9]
  \definition{s.}{seio; abraço}
\end{EntryWithPhonetic}

\begin{EntryWithPhonetic}{怀念}{huai2nian4}{7,8}{⼼,⼼}[HSK 4]
  \definition{v.}{pensar em; valorizar a memória de}
\end{EntryWithPhonetic}

\begin{EntryWithPhonetic}{怀疑}{huai2yi2}{7,14}{⼼,⽦}[HSK 4]
  \definition{v.}{duvidar; suspeitar | supor}
\end{EntryWithPhonetic}

\begin{EntryWithPhonetic}{怀孕}{huai2/yun4}{7,5}{⼼,⼦}[HSK 7-9]
  \definition{v.+compl.}{estar (ficar) grávida}
\end{EntryWithPhonetic}

\begin{EntryWithPhonetic}{怀着}{huai2zhe5}{7,11}{⼼,⽬}[HSK 7-9]
  \definition{v.}{nutrir; abrigar; ser preenchido com}
\end{EntryWithPhonetic}

%%%%%%%%%% 槐 %%%%%%%%%%
\subsection*{槐}\addcontentsline{loh}{figure}{槐 \dpy{huai2}}

\begin{EntryWithPhonetic}{槐}{huai2}{13}{⽊}
  \definition*{s.}{Sobrenome: Huai}
  \definition{s.}{sophora japonica; alfarrobeira; acácia}
\end{EntryWithPhonetic}

\begin{EntryWithPhonetic}{槐树}{huai2shu4}{13,9}{⽊,⽊}[HSK 7-9]
  \definition[棵,株]{s.}{acácia; árvore de alfarroba; árvore de pagode}
\end{EntryWithPhonetic}

%%%%%%%%%% 坏 %%%%%%%%%%
\subsection*{坏}\addcontentsline{loh}{figure}{坏 \dpy{huai4}}

\begin{EntryWithPhonetic}{坏}{huai4}{7}{⼟}[HSK 1]
  \definition{adj.}{ruim; prejudicial; insatisfatório; péssimo | mal; extremamente; indica um grau profundo, geralmente usado após verbos ou adjetivos que expressam estado psicológico | podre; estragado; impróprio; prejudicial ao uso}
  \definition[种]{s.}{ideia maligna; truque sujo; péssima ideia}
  \definition{v.}{estragar; destruir; corromper}
\end{EntryWithPhonetic}

\begin{EntryWithPhonetic}{坏处}{huai4chu5}{7,5}{⼟,⼡}[HSK 2]
  \definition[个]{s.}{dano; prejuízo; desvantagem; fatores prejudiciais a pessoas ou coisas}
\end{EntryWithPhonetic}

\begin{EntryWithPhonetic}{坏蛋}{huai4dan4}{7,11}{⼟,⾍}
  \definition{s.}{bastardo | canalha | pessoa má}
\end{EntryWithPhonetic}

\begin{EntryWithPhonetic}{坏人}{huai4ren2}{7,2}{⼟,⼈}[HSK 2]
  \definition[个,种]{s.}{malfeitor; canalha; pessoa má; pessoa de má qualidade; pessoa que faz coisas ruins}
\end{EntryWithPhonetic}

\begin{EntryWithPhonetic}{坏事}{huai4shi4}{7,8}{⼟,⼅}[HSK 7-9]
  \definition{s.}{coisa ruim; ação má; coisas que são prejudiciais à sociedade}
  \definition{v.}{arruinar algo; piorar as coisas}
\end{EntryWithPhonetic}

%%%%%%%%%% 欢 %%%%%%%%%%
\subsection*{欢}\addcontentsline{loh}{figure}{欢 \dpy{huan1}}

\begin{EntryWithPhonetic}{欢}{huan1}{6}{⽋}
  \definition*{s.}{Sobrenome: Huan}
  \definition{adj.}{alegre; feliz; jubilante | vigoroso; energético; em pleno andamento; com grande impulso}
  \definition{s.}{amante; querida; um apelido usado por mulheres nos tempos antigos para se referir aos seus amantes; agora, geralmente se refere a alguém de quem você gosta ou com quem tem um relacionamento romântico}
\end{EntryWithPhonetic}

\begin{EntryWithPhonetic}{欢呼}{huan1hu1}{6,8}{⽋,⼝}[HSK 7-9]
  \definition{v.}{saudar; aplaudir; aclamar; dar vivas}
\end{EntryWithPhonetic}

\begin{EntryWithPhonetic}{欢聚}{huan1ju4}{6,14}{⽋,⽿}[HSK 7-9]
  \definition{s.}{celebração | festa}
  \definition{v.}{desfrutar de uma reunião feliz; reunir-se alegremente | celebrar | reunir-se socialmente}
\end{EntryWithPhonetic}

\begin{EntryWithPhonetic}{欢快}{huan1kuai4}{6,7}{⽋,⼼}[HSK 7-9]
  \definition{adj.}{alegre; animado; alegre e despreocupado; feliz e alegre}
\end{EntryWithPhonetic}

\begin{EntryWithPhonetic}{欢乐}{huan1le4}{6,5}{⽋,⼃}[HSK 3]
  \definition{adj.}{feliz; alegre; felicidade (geralmente coletiva)}
\end{EntryWithPhonetic}

\begin{EntryWithPhonetic}{欢声笑语}{huan1sheng1-xiao4yu3}{6,7,10,9}{⽋,⼠,⽵,⾔}[HSK 7-9]
  \definition{expr.}{risos felizes e vozes alegres}
\end{EntryWithPhonetic}

\begin{EntryWithPhonetic}{欢迎}{huan1ying2}{6,7}{⽋,⾡}[HSK 2]
  \definition{adj.}{bem-vindo}
  \definition{v.}{dar as boas-vindas; cumprimentar; receber com alegria | dar as boas-vindas; receber favoravelmente (bem)}
\end{EntryWithPhonetic}

%%%%%%%%%% 还 %%%%%%%%%%
\subsection*{还}\addcontentsline{loh}{figure}{还 \dpy{huan2}}

\begin{EntryWithPhonetic}{还}{huan2}{7}{⾡}
  \definition*{s.}{Sobrenome: Huan}
  \definition{v.}{voltar; retornar; voltar ao lugar original ou restaurar o estado original | retribuir; devolver; reembolsar; devolver o dinheiro ou os bens emprestados ao seu proprietário | dar ou fazer algo em troca; retribuir as ações dos outros}
  \seeref{hai2}
\end{EntryWithPhonetic}

\begin{EntryWithPhonetic}{还款}{huan2 kuan3}{7,12}{⾡,⽋}[HSK 7-9]
  \definition{v.}{reembolsar | devolver dinheiro}
\end{EntryWithPhonetic}

\begin{EntryWithPhonetic}{还原}{huan2/yuan2}{7,10}{⾡,⼚}[HSK 7-9]
  \definition{v.+compl.}{restaurar ao estado ou forma original | desoxidar; reduzir; refere"-se à privação de oxigênio de substâncias que o contêm; também se refere, de modo geral, ao processo pelo qual uma substância ganha elétrons ou pares de elétrons em uma reação química}
\end{EntryWithPhonetic}

%%%%%%%%%% 环 %%%%%%%%%%
\subsection*{环}\addcontentsline{loh}{figure}{环 \dpy{huan2}}

\begin{EntryWithPhonetic}{环}{huan2}{8}{⽟}[HSK 3]
  \definition*{s.}{Sobrenome: Huan}
  \definition{clas.}{usado para anéis}
  \definition[个,串]{s.}{anel; arco | elo; \emph{link}; passo; etapa | anel; objeto em forma de círculo | arredores}
  \definition{v.}{cercar; rodear; circular; circundar}
\end{EntryWithPhonetic}

\begin{EntryWithPhonetic}{环保}{huan2bao3}{8,9}{⽟,⼈}[HSK 3]
  \definition{adj.}{ecológico; benefício para o meio ambiente; não prejudica o meio ambiente}
  \definition{s.}{proteção ambiental}
\end{EntryWithPhonetic}

\begin{EntryWithPhonetic}{环节}{huan2jie2}{8,5}{⽟,⾋}[HSK 5]
  \definition[个]{s.}{\emph{link}; ligação; vínculo; uma das muitas coisas que estão inter-relacionadas | segmento; estrutura anelar de alguns animais inferiores}
\end{EntryWithPhonetic}

\begin{EntryWithPhonetic}{环境}{huan2jing4}{8,14}{⽟,⼟}[HSK 3]
  \definition[个]{s.}{ambiente; os arredores | arredores; circunstâncias; condições políticas, econômicas, culturais, etc., dentro de um determinado âmbito}
\end{EntryWithPhonetic}

\begin{EntryWithPhonetic}{环境卫生}{huan2jing4wei4sheng1}{8,14,3,5}{⽟,⼟,⼙,⽣}
  \definition{s.}{saneamento ambiental; saneamento geral | saneamento}
  \seealsoref{环卫}{huan2wei4}
\end{EntryWithPhonetic}

\begin{EntryWithPhonetic}{环球}{huan2qiu2}{8,11}{⽟,⽟}[HSK 7-9]
  \definition*{s.}{Terra}
  \definition{adj.}{global; mundial}
  \definition{adv.}{ao redor do mundo; circulando a Terra}
  \definition{s.}{mundo inteiro}
\end{EntryWithPhonetic}

\begin{EntryWithPhonetic}{环绕}{huan2rao4}{8,9}{⽟,⽷}[HSK 7-9]
  \definition{v.}{cercar; rodear; envolver}
\end{EntryWithPhonetic}

\begin{EntryWithPhonetic}{环卫}{huan2wei4}{8,3}{⽟,⼙}
  \definition{s.}{limpeza pública; saneamento ambiental; saneamento geral; abreviação de 环境卫生 | Arcaico: guardas imperiais; guardas}
  \seealsoref{环境卫生}{huan2jing4wei4sheng1}
\end{EntryWithPhonetic}

%%%%%%%%%% 缓 %%%%%%%%%%
\subsection*{缓}\addcontentsline{loh}{figure}{缓 \dpy{huan3}}

\begin{EntryWithPhonetic}{缓}{huan3}{12}{⽶}[HSK 7-9]
  \definition{adj.}{lento; sem pressa | sem tensão; relaxado}
  \definition{v.}{atrasar; adiar; protelar | recuperar; reviver; voltar a si}
\end{EntryWithPhonetic}

\begin{EntryWithPhonetic}{缓和}{huan3he2}{12,8}{⽶,⼝}[HSK 7-9]
  \definition{adj.}{relaxado; moderado; suave; pacífico e relaxante; não tenso ou intenso}
  \definition{v.}{relaxar; aliviar; atenuar; facilitar}
\end{EntryWithPhonetic}

\begin{EntryWithPhonetic}{缓缓}{huan3huan3}{12,12}{⽶,⽶}[HSK 7-9]
  \definition{adv.}{lentamente; vagarosamente; gradualmente}
\end{EntryWithPhonetic}

\begin{EntryWithPhonetic}{缓解}{huan3jie3}{12,13}{⽶,⾓}[HSK 4]
  \definition{v.}{facilitar; aliviar; atenuar; amenizar; reduzir}
\end{EntryWithPhonetic}

\begin{EntryWithPhonetic}{缓慢}{huan3man4}{12,14}{⽶,⼼}[HSK 7-9]
  \definition{adj.}{lento; vagaroso}
  \definition{adv.}{lentamente; vagarosamente}
\end{EntryWithPhonetic}

%%%%%%%%%% 幻 %%%%%%%%%%
\subsection*{幻}\addcontentsline{loh}{figure}{幻 \dpy{huan4}}

\begin{EntryWithPhonetic}{幻}{huan4}{4}{⼳}
  \definition{adj.}{irreal; imaginário; ilusório | mágico; mutável}
  \definition{v.}{mudar magicamente}
\end{EntryWithPhonetic}

\begin{EntryWithPhonetic}{幻觉}{huan4jue2}{4,9}{⼳,⾒}[HSK 7-9]
  \definition{s.}{uma ilusão; uma alucinação; sensações falsas na visão, audição, tato, etc., que ocorrem sem estímulo externo}
\end{EntryWithPhonetic}

\begin{EntryWithPhonetic}{幻想}{huan4xiang3}{4,13}{⼳,⼼}[HSK 6]
  \definition[个,种]{s.}{fantasia; visão; arco-íris; ilusão; fruto da imaginação de alguém; imaginar algo que é difícil ou impossível de alcançar}
  \definition{v.}{imaginar; fantasiar; imaginar coisas que ainda não foram realizadas com base em ideais e desejos sociais ou pessoais}
\end{EntryWithPhonetic}

\begin{EntryWithPhonetic}{幻影}{huan4ying3}{4,15}{⼳,⼺}[HSK 7-9]
  \definition{s.}{fantasma; imagem irreal | miragem}
\end{EntryWithPhonetic}

%%%%%%%%%% 唤 %%%%%%%%%%
\subsection*{唤}\addcontentsline{loh}{figure}{唤 \dpy{huan4}}

\begin{EntryWithPhonetic}{唤}{huan4}{10}{⼝}
  \definition{v.}{chamar; fazer um barulho alto para fazer a outra parte acordar, prestar atenção ou vir até você}
\end{EntryWithPhonetic}

\begin{EntryWithPhonetic}{唤起}{huan4qi3}{10,10}{⼝,⾛}[HSK 7-9]
  \definition{v.}{despertar | chamar; evocar}
\end{EntryWithPhonetic}

%%%%%%%%%% 换 %%%%%%%%%%
\subsection*{换}\addcontentsline{loh}{figure}{换 \dpy{huan4}}

\begin{EntryWithPhonetic}{换}{huan4}{10}{⼿}[HSK 2]
  \definition{v.}{negociar; trocar; permutar; dar algo a alguém e, ao mesmo tempo, obter algo dele em troca | mudar; transformar; substituir | trocar dinheiro (câmbio) | transfundir (sangue) | transplantar (um órgão)}
\end{EntryWithPhonetic}

\begin{EntryWithPhonetic}{换成}{huan4cheng2}{10,6}{⼿,⼽}[HSK 7-9]
  \definition{v.}{trocar (algo) por (outro); indica a substituição de um objeto, estado ou situação por outro}
\end{EntryWithPhonetic}

\begin{EntryWithPhonetic}{换钱}{huan4/qian2}{10,10}{⼿,⾦}
  \definition{v.+compl.}{trocar dinheiro (em pequenas valores ou em outra moeda) | trocar (mercadorias) por dinheiro | vender}
\end{EntryWithPhonetic}

\begin{EntryWithPhonetic}{换取}{huan4qu3}{10,8}{⼿,⼜}[HSK 7-9]
  \definition{v.}{trocar (ou escambo) algo por; obter em troca | trocar algo por; obter por troca}
\end{EntryWithPhonetic}

\begin{EntryWithPhonetic}{换位}{huan4wei4}{10,7}{⼿,⼈}[HSK 7-9]
  \definition{v.}{trocar posições; transpor | mudar de posição}
\end{EntryWithPhonetic}

\begin{EntryWithPhonetic}{换言之}{huan4yan2zhi1}{10,7,3}{⼿,⾔,⼂}[HSK 7-9]
  \definition{adv.}{em outras palavras}
\end{EntryWithPhonetic}

%%%%%%%%%% 患 %%%%%%%%%%
\subsection*{患}\addcontentsline{loh}{figure}{患 \dpy{huan4}}

\begin{EntryWithPhonetic}{患}{huan4}{11}{⼼}[HSK 7-9]
  \definition*{s.}{Sobrenome: Huan}
  \definition{s.}{perigo; problema; desastre; flagelo | preocupação; ansiedade}
  \definition{v.}{contrair (doença); sofrer de}
\end{EntryWithPhonetic}

\begin{EntryWithPhonetic}{患病}{huan4bing4}{11,10}{⼼,⽧}[HSK 7-9]
  \definition{v.}{estar doente; ficar doente; adoecer; sofrer de uma doença}
\end{EntryWithPhonetic}

\begin{EntryWithPhonetic}{患有}{huan4you3}{11,6}{⼼,⽉}[HSK 7-9]
  \definition{v.}{sofrer de; refere"-se a alguém que sofre de uma doença ou condição específica}
\end{EntryWithPhonetic}

\begin{EntryWithPhonetic}{患者}{huan4zhe3}{11,8}{⼼,⽼}[HSK 6]
  \definition[个,位,名]{s.}{paciente; sofredor; pessoas com certas doenças}
\end{EntryWithPhonetic}

%%%%%%%%%% 焕 %%%%%%%%%%
\subsection*{焕}\addcontentsline{loh}{figure}{焕 \dpy{huan4}}

\begin{EntryWithPhonetic}{焕}{huan4}{11}{⽕}
  \definition{adj.}{Literário: brilhante; reluzente; radiante}
\end{EntryWithPhonetic}

\begin{EntryWithPhonetic}{焕发}{huan4fa1}{11,5}{⽕,⼜}[HSK 7-9]
  \definition{v.}{brilhar; revigorar; irradiar}
\end{EntryWithPhonetic}

%%%%%%%%%% 荒 %%%%%%%%%%
\subsection*{荒}\addcontentsline{loh}{figure}{荒 \dpy{huang1}}

\begin{EntryWithPhonetic}{荒}{huang1}{9}{⾋}[HSK 7-9]
  \definition*{s.}{Sobrenome: Huang}
  \definition{adj.}{(terra) não utilizada; não cultivada | desolado; estéril | irracional; delirante; fantástico; absurdo | incerto; duvidoso | dissoluto; autoindulgente | grosseiramente processado; bruto}
  \definition[片,块]{s.}{terra devastada; terra inculta; deserto | fome; quebra de safra | escassez | lixo; restos | terra selvagem (floresta)}
  \definition{v.}{(coloquial) negligenciar; estar fora de prática}
\end{EntryWithPhonetic}

\begin{EntryWithPhonetic}{荒诞}{huang1dan4}{9,8}{⾋,⾔}[HSK 7-9]
  \definition{adj.}{fantástico; absurdo; incrível; inacreditável}
\end{EntryWithPhonetic}

\begin{EntryWithPhonetic}{荒凉}{huang1liang2}{9,10}{⾋,⼎}[HSK 7-9]
  \definition{adj.}{selvagem; sombrio e desolado; escassamente povoado; deserto}
\end{EntryWithPhonetic}

\begin{EntryWithPhonetic}{荒谬}{huang1miu4}{9,13}{⾋,⾔}[HSK 7-9]
  \definition{adj.}{absurdo; ridículo; extremamente errado; extremamente irracional}
\end{EntryWithPhonetic}

\begin{EntryWithPhonetic}{荒唐}{huang1tang2}{9,10}{⾋,⼝}
  \definition{adj.}{absurdo; fantástico; grosseiramente exagerado; descreve pensamentos, palavras ou comportamentos anormais, fazendo as pessoas se sentirem estranhas ou ridículas | dissipado; dissoluto; descreve pessoas que não controlam seus desejos, não são limitadas por restrições e fazem as coisas casualmente}
\end{EntryWithPhonetic}

\begin{EntryWithPhonetic}{荒芜}{huang1wu2}{9,7}{⾋,⾋}
  \definition{adj.}{estéril}
\end{EntryWithPhonetic}

%%%%%%%%%% 慌 %%%%%%%%%%
\subsection*{慌}\addcontentsline{loh}{figure}{慌 \dpy{huang1}}

\begin{EntryWithPhonetic}{慌}{huang1}{12}{⼼}[HSK 5]
  \definition{adj.}{agitado; perturbado; confuso; que inspira terror}
  \definition{v.}{estar em estado de pânico; ficar com medo; ficar nervoso | estar com pressa}
\end{EntryWithPhonetic}

\begin{EntryWithPhonetic}{慌乱}{huang1luan4}{12,7}{⼼,⼄}[HSK 7-9]
  \definition{adj.}{agitado; alarmado e confuso; em pânico e ocupado}
\end{EntryWithPhonetic}

\begin{EntryWithPhonetic}{慌忙}{huang1mang2}{12,6}{⼼,⼼}[HSK 5]
  \definition{adj.}{apressado; afobado; com muita pressa}
  \definition{adv.}{apressadamente}
\end{EntryWithPhonetic}

\begin{EntryWithPhonetic}{慌张}{huang1zhang1}{12,7}{⼼,⼸}[HSK 7-9]
  \definition{adj.}{em pânico; agitado; perturbado; confuso}
\end{EntryWithPhonetic}

%%%%%%%%%% 皇 %%%%%%%%%%
\subsection*{皇}\addcontentsline{loh}{figure}{皇 \dpy{huang2}}

\begin{EntryWithPhonetic}{皇}{huang2}{9}{⽩}
  \definition*{s.}{Sobrenome: Huang}
  \definition{adj.}{grandioso; magnífico}
  \definition{s.}{imperador, o governante supremo de uma dinastia feudal após a Dinastia Qin; soberano}
\end{EntryWithPhonetic}

\begin{EntryWithPhonetic}{皇帝}{huang2di4}{9,9}{⽩,⼱}[HSK 6]
  \definition[个,位,任]{s.}{imperador; o título do mais alto governante feudal na China começou com o título de Imperador Qin Shi Huang}
\end{EntryWithPhonetic}

\begin{EntryWithPhonetic}{皇宫}{huang2gong1}{9,9}{⽩,⼧}[HSK 7-9]
  \definition{s.}{palácio (imperial) | palácio imperial; onde o imperador morava}
\end{EntryWithPhonetic}

\begin{EntryWithPhonetic}{皇后}{huang2hou4}{9,6}{⽩,⼝}[HSK 7-9]
  \definition[个,位,任]{s.}{rainha; imperatriz; a esposa do imperador}
\end{EntryWithPhonetic}

\begin{EntryWithPhonetic}{皇上}{huang2shang5}{9,3}{⽩,⼀}[HSK 7-9]
  \definition*{s.}{Sua Majestade; Vossa Majestade | Sua Majestade Imperial | Sua Majestade o Imperador}
  \definition{s.}{imperador; trono; soberano reinante}
\end{EntryWithPhonetic}

\begin{EntryWithPhonetic}{皇室}{huang2shi4}{9,9}{⽩,⼧}[HSK 7-9]
  \definition{s.}{família imperial (ou casa) | governo imperial; corte real | casa imperial | membro da família real}
\end{EntryWithPhonetic}

%%%%%%%%%% 凰 %%%%%%%%%%
\subsection*{凰}\addcontentsline{loh}{figure}{凰 \dpy{huang2}}

\begin{EntryWithPhonetic}{凰}{huang2}{11}{⼏}
  \definition[只]{s.}{Mitologia: fênix fêmea}
\end{EntryWithPhonetic}

%%%%%%%%%% 黄 %%%%%%%%%%
\subsection*{黄}\addcontentsline{loh}{figure}{黄 \dpy{huang2}}

\begin{EntryWithPhonetic}{黄}{huang2}{11}{⿈}[HSK 2][Kangxi 201]
  \definition*{s.}{Rio Huanghe, abreviação de 黄河 | Refere"-se ao Imperador Amarelo, um imperador da mitologia chinesa antiga | Sobrenome: Huang ou Hwang}
  \definition{adj.}{amarelo | obsceno; indecente; pornográfico; símbolo de corrupção e decadência, referindo"-se especificamente à pornografia}
  \definition{s.}{gema; ovas de caranguejo; refere"-se a certas coisas de cor amarela}
  \definition{v.}{fracassar; dar errado}
  \seealsoref{黄河}{huang2he2}
\end{EntryWithPhonetic}

\begin{EntryWithPhonetic}{黄瓜}{huang2gua5}{11,5}{⿈,⽠}[HSK 4]
  \definition[根,棵,株,条]{s.}{pepino}
\end{EntryWithPhonetic}

\begin{EntryWithPhonetic}{黄河}{huang2he2}{11,8}{⿈,⽔}
  \definition*{s.}{Rio Amarelo | Rio Huang He}
\end{EntryWithPhonetic}

\begin{EntryWithPhonetic}{黄昏}{huang2hun1}{11,8}{⿈,⽇}[HSK 7-9]
  \definition[个]{s.}{crepúsculo; refere"-se ao período do pôr do sol ao anoitecer}
\end{EntryWithPhonetic}

\begin{EntryWithPhonetic}{黄金}{huang2jin1}{11,8}{⿈,⾦}[HSK 4]
  \definition{adj.}{de primeira qualidade; dourado;}
  \definition[块,克,两]{s.}{ouro; \emph{aurum}; um tipo de metal, de cor amarela, mais precioso, abreviação de 金, frequentemente falado como 金子}
  \seealsoref{金}{jin1}
  \seealsoref{金子}{jin1zi5}
\end{EntryWithPhonetic}

\begin{EntryWithPhonetic}{黄色}{huang2se4}{11,6}{⿈,⾊}[HSK 2]
  \definition{adj.}{decadente; obsceno; erótico; pornográfico; símbolo de corrupção e decadência, referindo"-se especificamente à pornografia}
  \definition[种]{s.}{cor amarela}
\end{EntryWithPhonetic}

\begin{EntryWithPhonetic}{黄油}{huang2you2}{11,8}{⿈,⽔}
  \definition[盒]{s.}{manteiga}
\end{EntryWithPhonetic}

%%%%%%%%%% 惶 %%%%%%%%%%
\subsection*{惶}\addcontentsline{loh}{figure}{惶 \dpy{huang2}}

\begin{EntryWithPhonetic}{惶}{huang2}{12}{⼼}
  \definition{adj.}{cheio de medo; assustado}
  \definition{s.}{medo; pânico}
  \definition{v.}{temer}
\end{EntryWithPhonetic}

\begin{EntryWithPhonetic}{惶恐}{huang2kong3}{12,10}{⼼,⼼}
  \definition{adj.}{aterrorizado; em pânico; petrificado | inquieto; apreensivo}
\end{EntryWithPhonetic}

%%%%%%%%%% 恍 %%%%%%%%%%
\subsection*{恍}\addcontentsline{loh}{figure}{恍 \dpy{huang3}}

\begin{EntryWithPhonetic}{恍}{huang3}{9}{⼼}
  \definition{adv.}{(junto com 如, 若, etc.) parecer; como se | de repente}
  \seealsoref{如}{ru2}
  \seealsoref{若}{ruo4}
\end{EntryWithPhonetic}

\begin{EntryWithPhonetic}{恍然大悟}{huang3ran2-da4wu4}{9,12,3,10}{⼼,⽕,⼤,⼼}[HSK 7-9]
  \definition{expr.}{de repente ver a luz; de repente perceber o que aconteceu; perceber de repente}
\end{EntryWithPhonetic}

%%%%%%%%%% 晃 %%%%%%%%%%
\subsection*{晃}\addcontentsline{loh}{figure}{晃 \dpy{huang3}}

\begin{EntryWithPhonetic}{晃}{huang3}{10}{⽇}[HSK 7-9]
  \definition*{s.}{Sobrenome: Huang}
  \definition{adj.}{deslumbrante}
  \definition{v.}{passar rapidamente | deslumbrar; cegar}
  \seeref{huang4}
\end{EntryWithPhonetic}

%%%%%%%%%% 谎 %%%%%%%%%%
\subsection*{谎}\addcontentsline{loh}{figure}{谎 \dpy{huang3}}

\begin{EntryWithPhonetic}{谎}{huang3}{11}{⾔}
  \definition[句]{s.}{mentira; falsidade}
  \definition{v.}{contar uma mentira; mentir}
\end{EntryWithPhonetic}

\begin{EntryWithPhonetic}{谎话}{huang3hua4}{11,8}{⾔,⾔}[HSK 7-9]
  \definition[个]{s.}{mentira; falsidade; palavras falsas e enganosas}
\end{EntryWithPhonetic}

\begin{EntryWithPhonetic}{谎言}{huang3yan2}{11,7}{⾔,⾔}[HSK 7-9]
  \definition[个,派]{s.}{mentira; falsidade; é uma declaração falsa e inverídica, frequentemente usada para enganar os outros}
\end{EntryWithPhonetic}

%%%%%%%%%% 晃 %%%%%%%%%%
\subsection*{晃}\addcontentsline{loh}{figure}{晃 \dpy{huang4}}

\begin{EntryWithPhonetic}{晃}{huang4}{10}{⽇}[HSK 7-9]
  \definition{v.}{sacudir; balançar}
  \seeref{huang3}
\end{EntryWithPhonetic}

\begin{EntryWithPhonetic}{晃荡}{huang4dang5}{10,9}{⽇,⾋}[HSK 7-9]
  \definition{v.}{balançar; sacudir | Coloquial: vagar; ficar ocioso; divagar | oscilar}
\end{EntryWithPhonetic}

%%%%%%%%%% 灰 %%%%%%%%%%
\subsection*{灰}\addcontentsline{loh}{figure}{灰 \dpy{hui1}}

\begin{EntryWithPhonetic}{灰}{hui1}{6}{⽕}[HSK 7-9]
  \definition{adj.}{cinza (cor) | desanimado; desencorajado; deprimido}
  \definition[把,堆]{s.}{cinzas; pó que sobra após a queima de um objeto | pó; poeira; substância em pó | cal; argamassa (de cal)}
\end{EntryWithPhonetic}

\begin{EntryWithPhonetic}{灰尘}{hui1chen2}{6,6}{⽕,⼩}[HSK 7-9]
  \definition[层,堆]{s.}{cinza; poeira; sujeira; pó}
\end{EntryWithPhonetic}

\begin{EntryWithPhonetic}{灰色}{hui1se4}{6,6}{⽕,⾊}[HSK 5]
  \definition{adj.}{obscuro; ambíguo | sombrio; pessimista}
  \definition[种]{s.}{cor cinza; acinzentado}
\end{EntryWithPhonetic}

\begin{EntryWithPhonetic}{灰心}{hui1/xin1}{6,4}{⽕,⼼}[HSK 7-9]
  \definition{v.+compl.}{desanimar; ficar desanimado; ficar desapontado; (devido a dificuldades, fracassos) ficar deprimido}
\end{EntryWithPhonetic}

%%%%%%%%%% 恢 %%%%%%%%%%
\subsection*{恢}\addcontentsline{loh}{figure}{恢 \dpy{hui1}}

\begin{EntryWithPhonetic}{恢}{hui1}{9}{⼼}
  \definition{adj.}{extenso; vasto | grande; ótimo}
  \definition{v.}{recuperar; restaurar; restabelecer}
\end{EntryWithPhonetic}

\begin{EntryWithPhonetic}{恢复}{hui1fu4}{9,9}{⼼,⼢}[HSK 5]
  \definition{v.}{retomar; renovar; restaurar; voltar a | reviver; recuperar; reencontrar | restaurar; restabelecer; reabilitar; regenerar; ressurgir; restabelecer alguém em; recuperar o que foi perdido}
\end{EntryWithPhonetic}

%%%%%%%%%% 挥 %%%%%%%%%%
\subsection*{挥}\addcontentsline{loh}{figure}{挥 \dpy{hui1}}

\begin{EntryWithPhonetic}{挥}{hui1}{9}{⼿}[HSK 7-9]
  \definition{v.}{acenar; empunhar; socar | limpar lágrimas, suor, etc. com as mãos | comandar (um exército) | espalhar; dispersar | afastar-se; livrar-se de}
\end{EntryWithPhonetic}

\begin{EntryWithPhonetic}{挥汗如雨}{hui1han4ru2yu3}{9,6,6,8}{⼿,⽔,⼥,⾬}
  \definition{s.}{suor derramado}
  \definition{v.}{pingar com suor}
\end{EntryWithPhonetic}

%%%%%%%%%% 辉 %%%%%%%%%%
\subsection*{辉}\addcontentsline{loh}{figure}{辉 \dpy{hui1}}

\begin{EntryWithPhonetic}{辉}{hui1}{12}{⾞}
  \definition{s.}{brilho; esplendor; fulgor}
  \definition{v.}{brilhar}
\end{EntryWithPhonetic}

\begin{EntryWithPhonetic}{辉煌}{hui1huang2}{12,13}{⾞,⽕}[HSK 7-9]
  \definition{adj.}{brilhante; esplêndido; deslumbrante  | brilhante; glorioso; descreve conquistas notáveis}
\end{EntryWithPhonetic}

%%%%%%%%%% 囘 %%%%%%%%%%
\subsection*{囘}\addcontentsline{loh}{figure}{囘 \dpy{hui2}}

\begin{EntryWithPhonetic}{囘}{hui2}{5}{⼞}
  \variantof{回}
\end{EntryWithPhonetic}

%%%%%%%%%% 回 %%%%%%%%%%
\subsection*{回}\addcontentsline{loh}{figure}{回 \dpy{hui2}}

\begin{EntryWithPhonetic}{回}{hui2}{6}{⼞}[HSK 1,2]
  \definition*{s.}{Sobrenome: Hui}
  \definition*{s.}{Etnia Hui (mulçumanos chineses)}
  \definition{clas.}{usado para coisas, ações, número de vezes |  um trecho de um conto; um capítulo de um romance em capítulos | seção ou capítulo (de um livro clássico)}
  \definition{v.}{circular; enrolar | retornar; voltar; voltar ao lugar de origem | dar meia-volta | responder; contestar | relatar; reportar; responder}
\end{EntryWithPhonetic}

\begin{EntryWithPhonetic}{回报}{hui2bao4}{6,7}{⼞,⼿}[HSK 5]
  \definition{s.}{recompensa; pagamento; benefícios recebidos como resultado de assistência, esforço ou afeto | retornos; benefícios recebidos por meio de investimentos}
  \definition{v.}{pagar de volta; beneficiar pessoas ou organizações que os ajudaram ou cuidaram deles de alguma forma}
\end{EntryWithPhonetic}

\begin{EntryWithPhonetic}{回避}{hui2bi4}{6,16}{⼞,⾌}[HSK 5]
  \definition{v.}{fugir (de um problema); em direito, refere"-se especificamente à não participação nos procedimentos de um caso de um oficial de justiça, etc., que tenha interesse no caso ou nas partes do caso | esquivar"-se; evadir"-se; evitar (encontrar alguém)}
\end{EntryWithPhonetic}

\begin{EntryWithPhonetic}{回答}{hui2da2}{6,12}{⼞,⽵}[HSK 1]
  \definition[个]{s.}{resposta}
  \definition{v.}{responder; explicar a questão; expressar opinião sobre a solicitação}
\end{EntryWithPhonetic}

\begin{EntryWithPhonetic}{回到}{hui2dao4}{6,8}{⼞,⼑}[HSK 1]
  \definition{v.}{retornar para; voltar e chegar (ao lugar onde estava originalmente); (após uma mudança nas circunstâncias) retornar ao estado original}
\end{EntryWithPhonetic}

\begin{EntryWithPhonetic}{回复}{hui2fu4}{6,9}{⼞,⼢}[HSK 4]
  \definition{v.}{responder (a uma carta) | retornar ao estado normal; restaurar algo ao seu estado original}
\end{EntryWithPhonetic}

\begin{EntryWithPhonetic}{回顾}{hui2gu4}{6,10}{⼞,⾴}[HSK 5]
  \definition{v.}{olhar para trás | revisar; fazer uma retrospectiva; olhar para trás, pensar no passado}
\end{EntryWithPhonetic}

\begin{EntryWithPhonetic}{回归}{hui2gui1}{6,5}{⼞,⼹}[HSK 7-9]
  \definition{v.}{retornar; regredir; retornar para (local original, organização, etc.)}
\end{EntryWithPhonetic}

\begin{EntryWithPhonetic}{回国}{hui2 guo2}{6,8}{⼞,⼞}[HSK 2]
  \definition{v.}{regressar ao seu país (terra natal); referindo"-se a voltar do exterior}
\end{EntryWithPhonetic}

\begin{EntryWithPhonetic}{回家}{hui2 jia1}{6,10}{⼞,⼧}[HSK 1]
  \definition{v.}{ir (voltar) para casa; estar em casa; voltar para casa}
\end{EntryWithPhonetic}

\begin{EntryWithPhonetic}{回扣}{hui2kou4}{6,6}{⼞,⼿}[HSK 7-9]
  \definition[笔,的]{s.}{propina; desconto; comissão sobre vendas (para agente)}
\end{EntryWithPhonetic}

\begin{EntryWithPhonetic}{回馈}{hui2kui4}{6,12}{⼞,⾶}[HSK 7-9]
  \definition{v.}{retribuir; recompensar | dar \emph{feedback}; fornecer \emph{feedback}; dar retorno; dar parecer}
\end{EntryWithPhonetic}

\begin{EntryWithPhonetic}{回来}{hui2 lai5}{6,7}{⼞,⽊}[HSK 1]
  \definition{v.}{voltar; regressar (para a minha localização) | retornar; usado após um verbo, significa ``vir ao lugar original''}
\end{EntryWithPhonetic}

\begin{EntryWithPhonetic}{回落}{hui2luo4}{6,12}{⼞,⾋}[HSK 7-9]
  \definition{v.}{(níveis de água, preços, etc.) cair após uma subida; diminuir | recuar | retornar ao nível baixo após uma subida (no nível da água, preço etc.)}
  \antonymref{回升}{hui2sheng1}
\end{EntryWithPhonetic}

\begin{EntryWithPhonetic}{回去}{hui2 qu5}{6,5}{⼞,⼛}[HSK 1]
  \definition{v.}{retornar; voltar; estar de volta; (a partir da minha localização)}
\end{EntryWithPhonetic}

\begin{EntryWithPhonetic}{回升}{hui2sheng1}{6,4}{⼞,⼗}[HSK 7-9]
  \definition{v.}{levantar"-se novamente (após uma queda); levantar-se | recuperar; cair e depois subir novamente}
  \antonymref{回落}{hui2luo4}
\end{EntryWithPhonetic}

\begin{EntryWithPhonetic}{回收}{hui2shou1}{6,6}{⼞,⽁}[HSK 5]
  \definition{v.}{reciclar; reciclar itens (geralmente resíduos ou produtos antigos) para reutilização | recuperar; recolher; recuperar o que foi emitido ou demitido}
\end{EntryWithPhonetic}

\begin{EntryWithPhonetic}{回首}{hui2shou3}{6,9}{⼞,⾸}[HSK 7-9]
  \definition{v.}{virar a cabeça; virar-se (em volta) | olhar para trás; lembrar-se; recordar}
\end{EntryWithPhonetic}

\begin{EntryWithPhonetic}{回头}{hui2tou2}{6,5}{⼞,⼤}[HSK 5]
  \definition{adv.}{mais tarde; depois de um tempo}
  \definition{conj.}{ou então; usado no início da segunda metade de uma frase para indicar o que acontecerá se você não fizer o que fez na primeira metade da frase}
  \definition{v.}{dar a meia-volta; virar a cabeça; virar a cabeça para trás | retornar; voltar | arrepender-se; corrigir seu caminho; reconhecer e corrigir erros}
\end{EntryWithPhonetic}

\begin{EntryWithPhonetic}{回味}{hui2wei4}{6,8}{⼞,⼝}[HSK 7-9]
  \definition{s.}{sabor residual; recordar o sabor agradável de\dots; o gosto residual que fica na boca depois de comer}
  \definition{v.}{recordar e ponderar sobre; reviver coisas que você vivenciou ou com as quais entrou em contato}
\end{EntryWithPhonetic}

\begin{EntryWithPhonetic}{回想}{hui2xiang3}{6,13}{⼞,⼼}[HSK 7-9]
  \definition{v.}{recordar; pensar de novo; pensar (no passado)}
\end{EntryWithPhonetic}

\begin{EntryWithPhonetic}{回信}{hui2/xin4}{6,9}{⼞,⼈}[HSK 5]
  \definition[封]{s.}{uma carta em resposta; uma mensagem verbal em resposta}
  \definition{v.+compl.}{escrever em resposta; escrever de volta; responder uma carta; responder verbalmente uma mensagem}
\end{EntryWithPhonetic}

\begin{EntryWithPhonetic}{回旋}{hui2xuan2}{6,11}{⼞,⽅}
  \definition{v.}{circular | rodar | dar a volta}
\end{EntryWithPhonetic}

\begin{EntryWithPhonetic}{回忆}{hui2yi4}{6,4}{⼞,⼼}[HSK 5]
  \definition[个,段]{s.}{memória; lembrança de eventos ou experiências passadas}
  \definition{v.}{lembrar; recordar}
\end{EntryWithPhonetic}

\begin{EntryWithPhonetic}{回忆录}{hui2yi4lu4}{6,4,8}{⼞,⼼,⼹}[HSK 7-9]
  \definition{s.}{memórias; reminiscências; lembranças; um gênero de escrita que relata experiências pessoais ou eventos históricos familiares}
\end{EntryWithPhonetic}

\begin{EntryWithPhonetic}{回应}{hui2ying4}{6,7}{⼞,⼴}[HSK 6]
  \definition{v.}{responder}
\end{EntryWithPhonetic}

%%%%%%%%%% 廻 %%%%%%%%%%
\subsection*{廻}\addcontentsline{loh}{figure}{廻 \dpy{hui2}}

\begin{EntryWithPhonetic}{廻}{hui2}{8}{⼵}
  \variantof{回}
\end{EntryWithPhonetic}

%%%%%%%%%% 悔 %%%%%%%%%%
\subsection*{悔}\addcontentsline{loh}{figure}{悔 \dpy{hui3}}

\begin{EntryWithPhonetic}{悔}{hui3}{10}{⼼}
  \definition{v.}{lamentar; arrepender-se}
\end{EntryWithPhonetic}

\begin{EntryWithPhonetic}{悔恨}{hui3hen4}{10,9}{⼼,⼼}[HSK 7-9]
  \definition{v.}{arrepender-se profundamente; estar amargamente arrependido}
\end{EntryWithPhonetic}

%%%%%%%%%% 毁 %%%%%%%%%%
\subsection*{毁}\addcontentsline{loh}{figure}{毁 \dpy{hui3}}

\begin{EntryWithPhonetic}{毁}{hui3}{13}{⽎}[HSK 6]
  \definition{v.}{destruir; arruinar; danificar | (dialeto)  transformar, remodelar um item antigo em outra coisa, geralmente roupas | queimar | difamar; caluniar}
\end{EntryWithPhonetic}

\begin{EntryWithPhonetic}{毁坏}{hui3huai4}{13,7}{⽎,⼟}[HSK 7-9]
  \definition{v.}{danificar; devastar; degradar; destroçar; estilhaçar; vandalizar}
\end{EntryWithPhonetic}

\begin{EntryWithPhonetic}{毁灭}{hui3mie4}{13,5}{⽎,⽕}[HSK 7-9]
  \definition{v.}{arruinar; destruir; exterminar; destruir ou eliminar completamente}
\end{EntryWithPhonetic}

%%%%%%%%%% 汇 %%%%%%%%%%
\subsection*{汇}\addcontentsline{loh}{figure}{汇 \dpy{hui4}}

\begin{EntryWithPhonetic}{汇}{hui4}{5}{⽔}[HSK 4]
  \definition{s.}{montagem; coleção; coisas coletadas}
  \definition{v.}{convergir | reunir; coletar | remeter | trocar (câmbio de moedas)}
\end{EntryWithPhonetic}

\begin{EntryWithPhonetic}{汇报}{hui4bao4}{5,7}{⽔,⼿}[HSK 4]
  \definition[份,次]{s.}{relatório; referindo"-se ao conteúdo de declarações escritas ou orais feitas a um superior ou pessoa relevante para apresentar uma situação ou refletir um problema}
  \definition{v.}{relatar; fazer um relato de}
\end{EntryWithPhonetic}

\begin{EntryWithPhonetic}{汇合}{hui4he2}{5,6}{⽔,⼝}[HSK 7-9]
  \definition{s.}{confluência; fusão}
  \definition{v.}{convergir; juntar; reunir; encontrar}
\end{EntryWithPhonetic}

\begin{EntryWithPhonetic}{汇集}{hui4ji2}{5,12}{⽔,⾫}[HSK 7-9]
  \definition{v.}{aduzir; coletar; compilar | reunir-se; congestionar; convergir; reunir; juntar}
\end{EntryWithPhonetic}

\begin{EntryWithPhonetic}{汇聚}{hui4ju4}{5,14}{⽔,⽿}[HSK 7-9]
  \definition{v.}{juntar; montar; reunir; ajuntar; reunir-se; convergência e acumulação (usado principalmente para objetos)}
\end{EntryWithPhonetic}

\begin{EntryWithPhonetic}{汇款}{hui4/kuan3}{5,12}{⽔,⽋}[HSK 5]
  \definition[笔,个]{s.}{remessa; dinheiro enviado ou recebido}
  \definition{v.+compl.}{remeter dinheiro; fazer uma remessa; enviar dinheiro}
\end{EntryWithPhonetic}

\begin{EntryWithPhonetic}{汇率}{hui4lv4}{5,11}{⽔,⽞}[HSK 4]
  \definition[个,种]{s.}{taxa de câmbio; relação entre a moeda de um país e a de outro}
\end{EntryWithPhonetic}

%%%%%%%%%% 会 %%%%%%%%%%
\subsection*{会}\addcontentsline{loh}{figure}{会 \dpy{hui4}}

\begin{EntryWithPhonetic}{会}{hui4}{6}{⼈}[HSK 1,2]
  \definition{adv.}{um momento}
  \definition{clas.}{momento; um curto período de tempo}
  \definition{s.}{reunião; festa; conferência; reunião com um objetivo específico | reunião; reunião no trabalho | feira do templo; festival religioso | associação; sociedade; sindicato; certas organizações públicas | oportunidade; ocasião; momento oportuno | cidade principal; capital; cidade central}
  \definition{suf.}{união; grupo; associação}
  \definition{v.}{ser provável que; ter certeza de; indica que é possível realizar (é possível responder à pergunta separadamente) |  poder; ser capaz de; significa saber como fazer ou ter a capacidade de fazer (geralmente se refere a coisas que precisam ser aprendidas) | saber; compreender; entender | encontrar; ver | reunir-se; reunir; agregar; juntar | destacar-se em; ser bom em; ser hábil em; indica proficiência | pagar (ou custear) contas}
  \seeref{kuai4}
\end{EntryWithPhonetic}

\begin{EntryWithPhonetic}{会场}{hui4chang3}{6,6}{⼈,⼟}[HSK 7-9]
  \definition[个]{s.}{local de encontro; lugar onde as pessoas se reúnem}
\end{EntryWithPhonetic}

\begin{EntryWithPhonetic}{会见}{hui4jian4}{6,4}{⼈,⾒}[HSK 6]
  \definition{v.}{entrevistar; encontrar-se com (especialmente um visitante estrangeiro)}
\end{EntryWithPhonetic}

\begin{EntryWithPhonetic}{会面}{hui4/mian4}{6,9}{⼈,⾯}[HSK 7-9]
  \definition{v.+compl.}{reunir-se; encontrar}
\end{EntryWithPhonetic}

\begin{EntryWithPhonetic}{会首}{hui4shou3}{6,9}{⼈,⾸}
  \definition{s.}{chefe de uma sociedade | patrocinador de uma organização}
\end{EntryWithPhonetic}

\begin{EntryWithPhonetic}{会谈}{hui4tan2}{6,10}{⼈,⾔}[HSK 5]
  \definition{v.}{manter conversações; comumente usado em assuntos internacionais ou atividades diplomáticas}
\end{EntryWithPhonetic}

\begin{EntryWithPhonetic}{会晤}{hui4wu4}{6,11}{⼈,⽇}[HSK 7-9]
  \definition{v.}{reunir-se (com líderes estaduais ou figuras sociais para discutir assuntos importantes)}
\end{EntryWithPhonetic}

\begin{EntryWithPhonetic}{会议}{hui4yi4}{6,5}{⼈,⾔}[HSK 3]
  \definition[次,届,个,场]{s.}{reunião; conferência; reunião organizada pela organização relevante para ouvir opiniões, discutir questões e distribuir tarefas | conselho; congresso; um órgão ou organização permanente que discute e trata frequentemente assuntos importantes}
\end{EntryWithPhonetic}

\begin{EntryWithPhonetic}{会意}{hui4yi4}{6,13}{⼈,⼼}[HSK 7-9]
  \definition{s.}{compreensão; conhecimento | compostos associativos, uma das seis categorias de caracteres chineses, que são formados pela combinação de dois ou mais elementos, cada um com um significado próprio, para criar um novo significado, por exemplo, 信, um caractere composto de 人 (homem) e 言 (palavra), significando uma mensagem ou algo em que se pode acreditar ou confiar}
  \definition{v.}{entender; saber}
\end{EntryWithPhonetic}

\begin{EntryWithPhonetic}{会员}{hui4yuan2}{6,7}{⼈,⼝}[HSK 3]
  \definition[位,名,个,些]{s.}{membro; associado; membros de certos grupos ou organizações}
\end{EntryWithPhonetic}

\begin{EntryWithPhonetic}{会长}{hui4zhang3}{6,4}{⼈,⾧}[HSK 6]
  \definition[位,名,个,些]{s.}{presidente de uma associação ou sociedade | presidente de um clube, comitê etc.}
\end{EntryWithPhonetic}

\begin{EntryWithPhonetic}{会诊}{hui4/zhen3}{6,7}{⼈,⾔}[HSK 7-9]
  \definition{s.}{consulta de médicos; consulta (de grupo)}[医生举行会诊,决定是否需要动手术。===Os médicos realizam uma consulta para decidir se a cirurgia é necessária.]
  \definition{v.+compl.}{(médicos) realizar uma consulta médica; consultar}
\end{EntryWithPhonetic}

%%%%%%%%%% 绘 %%%%%%%%%%
\subsection*{绘}\addcontentsline{loh}{figure}{绘 \dpy{hui4}}

\begin{EntryWithPhonetic}{绘}{hui4}{9}{⽷}
  \definition{v.}{pintar; desenhar}
\end{EntryWithPhonetic}

\begin{EntryWithPhonetic}{绘画}{hui4hua4}{9,8}{⽷,⽥}[HSK 6]
  \definition{s.}{desenho; pintura}
  \definition{v.}{desenhar; pintar}
\end{EntryWithPhonetic}

\begin{EntryWithPhonetic}{绘声绘色}{hui4sheng1-hui4se4}{9,7,9,6}{⽷,⼠,⽷,⾊}[HSK 7-9]
  \definition{expr.}{vívido e colorido; uma descrição animada; vívido; animado}
\end{EntryWithPhonetic}

%%%%%%%%%% 贿 %%%%%%%%%%
\subsection*{贿}\addcontentsline{loh}{figure}{贿 \dpy{hui4}}

\begin{EntryWithPhonetic}{贿}{hui4}{10}{⾙}
  \definition[行]{s.}{bens; riqueza; objetos de valor; propriedade | suborno | Literário: wealth}
  \definition{v.}{subornar}
\end{EntryWithPhonetic}

\begin{EntryWithPhonetic}{贿赂}{hui4lu4}{10,10}{⾙,⾙}[HSK 7-9]
  \definition[笔]{s.}{suborno}
  \definition{v.}{subornar; subornar outros com dinheiro}
\end{EntryWithPhonetic}

%%%%%%%%%% 昏 %%%%%%%%%%
\subsection*{昏}\addcontentsline{loh}{figure}{昏 \dpy{hun1}}

\begin{EntryWithPhonetic}{昏}{hun1}{8}{⽇}[HSK 6]
  \definition*{s.}{Sobrenome: Hun}
  \definition{adj.}{escuro; fraco; embaçado | confuso; embaraçado; inconsciente}
  \definition{s.}{crepúsculo; tarde}
  \definition{v.}{perder a consciência; desmaiar}
\end{EntryWithPhonetic}

\begin{EntryWithPhonetic}{昏迷}{hun1mi2}{8,9}{⽇,⾡}[HSK 7-9]
  \definition{s.}{coma; um estado em que uma pessoa perde a sensibilidade e o conhecimento}
  \definition{v.}{entrar em coma}
\end{EntryWithPhonetic}

%%%%%%%%%% 荤 %%%%%%%%%%
\subsection*{荤}\addcontentsline{loh}{figure}{荤 \dpy{hun1}}

\begin{EntryWithPhonetic}{荤}{hun1}{9}{⾋}[Kangxi 9]
  \definition{adj.}{obsceno; lascivo; vulgar}
  \definition{s.}{carne ou peixe | Budismo: vegetais picantes proibidos aos vegetarianos budistas, como cebola, alho-poró, alho, etc. | alimentos não vegetarianos (carne, peixe etc.) | vegetais com cheiro forte (alho etc.)}
  \antonymref{素}{su4}
\end{EntryWithPhonetic}

%%%%%%%%%% 婚 %%%%%%%%%%
\subsection*{婚}\addcontentsline{loh}{figure}{婚 \dpy{hun1}}

\begin{EntryWithPhonetic}{婚}{hun1}{11}{⼥}
  \definition{s.}{casamento}
  \definition{v.}{casar}
\end{EntryWithPhonetic}

\begin{EntryWithPhonetic}{婚礼}{hun1li3}{11,5}{⼥,⽰}[HSK 4]
  \definition[场]{s.}{casamento; núpcias; cerimônia de casamento}
\end{EntryWithPhonetic}

\begin{EntryWithPhonetic}{婚纱}{hun1sha1}{11,7}{⼥,⽷}[HSK 7-9]
  \definition[件,套,个]{s.}{vestido de noiva; um vestido especial usado pela noiva em seu casamento}
\end{EntryWithPhonetic}

\begin{EntryWithPhonetic}{婚姻}{hun1yin1}{11,9}{⼥,⼥}[HSK 7-9]
  \definition[桩,次,段]{s.}{casamento; matrimônio}
\end{EntryWithPhonetic}

%%%%%%%%%% 浑 %%%%%%%%%%
\subsection*{浑}\addcontentsline{loh}{figure}{浑 \dpy{hun2}}

\begin{EntryWithPhonetic}{浑}{hun2}{9}{⽔}
  \definition*{s.}{Sobrenome: Hun}
  \definition{adj.}{lamacento; turvo | tolo; estúpido | simples e natural; sem sofisticação | inteiro; por toda parte}
  \variantof{混}
\end{EntryWithPhonetic}

\begin{EntryWithPhonetic}{浑身}{hun2shen1}{9,7}{⽔,⾝}[HSK 7-9]
  \definition{s.}{por todo o corpo; da cabeça aos pés; corpo inteiro}
\end{EntryWithPhonetic}

%%%%%%%%%% 混 %%%%%%%%%%
\subsection*{混}\addcontentsline{loh}{figure}{混 \dpy{hun2}}

\begin{EntryWithPhonetic}{混}{hun2}{11}{⽔}
  \definition{adj.}{nublado; o mesmo que 浑, turvo | confuso; embaraçado; irracional}
  \variantof{浑}
  \seeref{hun4}
\end{EntryWithPhonetic}

%%%%%%%%%% 魂 %%%%%%%%%%
\subsection*{魂}\addcontentsline{loh}{figure}{魂 \dpy{hun2}}

\begin{EntryWithPhonetic}{魂}{hun2}{13}{⿁}[HSK 7-9]
  \definition[个]{s.}{alma | humor; espírito | espírito elevado de uma nação}
\end{EntryWithPhonetic}

%%%%%%%%%% 混 %%%%%%%%%%
\subsection*{混}\addcontentsline{loh}{figure}{混 \dpy{hun4}}

\begin{EntryWithPhonetic}{混}{hun4}{11}{⽔}[HSK 6]
  \definition{adj.}{confuso; imundo; turvo; lamacento; impuro}
  \definition{adv.}{de forma imprudente; irresponsável; irrefletidamente}
  \definition{v.}{misturar; confundir; misturar verdadeiro e falso | passar por; esgueirar-se | vagar à deriva; arrastar-se; sobreviver de maneira superficial; contentar-se com | se dar bem com alguém}
  \seeref{hun2}
\end{EntryWithPhonetic}

\begin{EntryWithPhonetic}{混蛋}{hun4dan4}{11,11}{⽔,⾍}
  \definition{s.}{miserável; bastardo; canalha; patife; refere"-se a uma pessoa irracional (um insulto)}
  \synonymref{犯浑}{fan4/hun2}
  \synonymref{王八蛋}{wang2 ba1 dan4}
  \antonymref{绅士}{shen1shi4}
\end{EntryWithPhonetic}

\begin{EntryWithPhonetic}{混饭}{hun4/fan4}{11,7}{⽔,⾷}
  \definition{v.+compl.}{trabalhar para viver | Coloquial: se envolver em um trabalho apenas para ganhar a vida (sem ter nenhum interesse real nele) | comer às custas de outra pessoa}
\end{EntryWithPhonetic}

\begin{EntryWithPhonetic}{混合}{hun4he2}{11,6}{⽔,⼝}[HSK 6]
  \definition{s.}{híbrido; composto; refere"-se a duas ou mais substâncias misturadas sem reação química, mas ainda mantendo suas respectivas propriedades (diferente de 化合)}
  \definition{v.}{misturar; mixar; misturar"-se}
  \seealsoref{化合}{hua4he2}
\end{EntryWithPhonetic}

\begin{EntryWithPhonetic}{混乱}{hun4luan4}{11,7}{⽔,⼄}[HSK 6]
  \definition{adj.}{caótico; confuso; desordenado; desorganizado; fora de ordem}
  \definition[片]{s.}{caos; confusão}
\end{EntryWithPhonetic}

\begin{EntryWithPhonetic}{混凝土}{hun4ning2tu3}{11,16,3}{⽔,⼎,⼟}[HSK 7-9]
  \definition{s.}{concreto; material de construção feito pela mistura de cimento, areia, cascalho e água em uma determinada proporção; após o endurecimento, apresenta propriedades como resistência à pressão, resistência à água e resistência ao fogo}
\end{EntryWithPhonetic}

\begin{EntryWithPhonetic}{混淆}{hun4xiao2}{11,11}{⽔,⽔}[HSK 7-9]
  \definition{v.}{misturar; confundir; colocar duas coisas muito parecidas juntas sem conseguir diferenciá-las}
\end{EntryWithPhonetic}

\begin{EntryWithPhonetic}{混浊}{hun4zhuo2}{11,9}{⽔,⽔}[HSK 7-9]
  \definition{adj.}{lamacento; turvo; nublado | impuro; não transparente; não claro; turvo}
  \definition{s.}{nubécula; opacidade na córnea do olho}
\end{EntryWithPhonetic}

%%%%%%%%%% 豁 %%%%%%%%%%
\subsection*{豁}\addcontentsline{loh}{figure}{豁 \dpy{huo1}}

\begin{EntryWithPhonetic}{豁}{huo1}{17}{⾕}[HSK 7-9]
  \definition{v.}{cortar; quebrar; rachar; dividir | sacrificar; desistir; pagar o preço cruelmente}
  \seeref{hua2}
  \seeref{huo4}
\end{EntryWithPhonetic}

\begin{EntryWithPhonetic}{豁出去}{huo1/chu5qu4}{17,5,5}{⾕,⼐,⼛}[HSK 7-9]
  \definition{v.+compl.}{seguir em frente independentemente; pronto para arriscar tudo}[她豁出去所有,去追逐梦想。===Ela arriscou tudo para perseguir seu sonho.]
\end{EntryWithPhonetic}

%%%%%%%%%% 和 %%%%%%%%%%
\subsection*{和}\addcontentsline{loh}{figure}{和 \dpy{huo2}}

\begin{EntryWithPhonetic}{和}{huo2}{8}{⼝}
  \definition{v.}{combinar uma substância em pó (farinha, gesso, etc.) com água; adicionar líquido ao pó e mexer ou amassar até ficar pegajoso ou espesso}
  \seeref{he2}
  \seeref{he4}
  \seeref{hu2}
  \seeref{huo4}
\end{EntryWithPhonetic}

%%%%%%%%%% 活 %%%%%%%%%%
\subsection*{活}\addcontentsline{loh}{figure}{活 \dpy{huo2}}

\begin{EntryWithPhonetic}{活}{huo2}{9}{⽔}[HSK 3]
  \definition{adj.}{vivo; vivendo; indica que (alguma ação) foi realizada enquanto a pessoa ainda estava viva | vívido; animado; ativo | móvel; em movimento; ativo}
  \definition{adv.}{exatamente; simplesmente; expressa um grau elevado, equivalente a 真正 ou 简直}
  \definition{s.}{emprego; meios de subsistência; trabalho (geralmente refere"-se a trabalho físico) | produto; algo fabricado}
  \definition{v.}{viver; ter vida; sobreviver | salvar (a vida de uma pessoa); fazer sobreviver; manter a vida}
  \seealsoref{简直}{jian3zhi2}
  \seealsoref{真正}{zhen1zheng4}
  \antonymref{死}{si3}
\end{EntryWithPhonetic}

\begin{EntryWithPhonetic}{活动}{huo2dong5}{9,6}{⽔,⼒}[HSK 2]
  \definition{adj.}{móvel; flexível para alterações ou mudanças}
  \definition[些,个,种,类,次]{s.}{atividade; ação tomada com o objetivo de alcançar um determinado objetivo}
  \definition{v.}{fazer exercício; movimentar-se | usar influência pessoal; usar meios irregulares | mover-se}
\end{EntryWithPhonetic}

\begin{EntryWithPhonetic}{活该}{huo2gai1}{9,8}{⽔,⾔}[HSK 7-9]
  \definition{v.aux.}{merecer (uma consequência negativa) | deveria (incluindo o significado de destino); ser decretado pelo destino}
\end{EntryWithPhonetic}

\begin{EntryWithPhonetic}{活力}{huo2li4}{9,2}{⽔,⼒}[HSK 5]
  \definition{s.}{vigor; vitalidade; energia; muito forte, geralmente usado para descrever pessoas, cidades, empresas, economias, etc.}
\end{EntryWithPhonetic}

\begin{EntryWithPhonetic}{活路}{huo2lu4}{9,13}{⽔,⾜}
  \definition{s.}{maneira de sobreviver | meio de subsistência}
  \seeref{huo2lu5}
\end{EntryWithPhonetic}

\begin{EntryWithPhonetic}{活路}{huo2lu5}{9,13}{⽔,⾜}
  \definition{s.}{labor | trabalho físico}
  \seeref{huo2lu4}
\end{EntryWithPhonetic}

\begin{EntryWithPhonetic}{活泼}{huo2po5}{9,8}{⽔,⽔}[HSK 5]
  \definition{adj.}{vívido; ativo; animado; brilhante; vivaz; cheio de vida | Química: reativo; significa que a substância é ativa e reage facilmente com outras substâncias}
\end{EntryWithPhonetic}

\begin{EntryWithPhonetic}{活期}{huo2qi1}{9,12}{⽔,⽉}[HSK 7-9]
  \definition{adj.}{atual; corrente; presente}
\end{EntryWithPhonetic}

\begin{EntryWithPhonetic}{活儿}{huo2r5}{9,2}{⽔,⼉}[HSK 7-9]
  \definition[点]{s.}{emprego; trabalho; geralmente trabalho físico | produto; produtos acabados: artesanato, tecnologia}
\end{EntryWithPhonetic}

\begin{EntryWithPhonetic}{活跃}{huo2yue4}{9,11}{⽔,⾜}[HSK 6]
  \definition{adj.}{ativo; dinâmico; pensamentos, ações ou atividades positivas; ocorrências frequentes | rápido; ativo; dinâmico}
  \definition{v.}{animar; tornar ativo | ser ativo}
\end{EntryWithPhonetic}

\begin{EntryWithPhonetic}{活着}{huo2zhe5}{9,11}{⽔,⽬}
  \definition{adj.}{vivo}
\end{EntryWithPhonetic}

%%%%%%%%%% 火 %%%%%%%%%%
\subsection*{火}\addcontentsline{loh}{figure}{火 \dpy{huo3}}

\begin{EntryWithPhonetic}{火}{huo3}{4}{⽕}[HSK 3,4][Kangxi 86]
  \definition*{s.}{Sobrenome: Huo}
  \definition{adj.}{ardente; flamejante; vermelho como fogo | efervescente; próspero}
  \definition{adv.}{urgentemente}
  \definition{clas.}{usado para unidades militares (antigo)}
  \definition[场,把,团,堆]{s.}{fogo; a luz e as chamas emitidas pela combustão de um objeto | fúria; metáfora para emoções agitadas, irritadas ou raivosas | calor interno (uma das seis causas de doenças) | armas de fogo e munições | a ação de lutar}
  \definition{v.}{ficar com raiva; perder a paciência}
\end{EntryWithPhonetic}

\begin{EntryWithPhonetic}{火暴}{huo3bao4}{4,15}{⽕,⽇}[HSK 7-9]
  \definition{adj.}{impetuoso; impetuoso; irritável; impaciente | próspero; florescente; vigoroso; animado}
\end{EntryWithPhonetic}

\begin{EntryWithPhonetic}{火柴}{huo3chai2}{4,10}{⽕,⽊}[HSK 5]
  \definition[根,盒,包]{s.}{fósforo (palito de fósforo); fósforo de segurança; iniciador de fogo feito de uma tira fina de madeira mergulhada em um composto de fósforo ou enxofre}
\end{EntryWithPhonetic}

\begin{EntryWithPhonetic}{火车}{huo3che1}{4,4}{⽕,⾞}[HSK 1]
  \definition[个,列,节,班,趟]{s.}{trem; comboio}
\end{EntryWithPhonetic}

\begin{EntryWithPhonetic}{火车司机}{huo3che1 si1ji1}{4,4,5,6}{⽕,⾞,⼝,⽊}
  \definition{s.}{maquinista de trem}
\end{EntryWithPhonetic}

\begin{EntryWithPhonetic}{火锅}{huo3guo1}{4,12}{⽕,⾦}[HSK 7-9]
  \definition[顿]{s.}{\emph{hot pot}; uma panela feita de metal ou outro material, que pode ser usada para aquecer a sopa continuamente com eletricidade, álcool, etc., e depois adicionar carne, vegetais, etc. à sopa e comê"-la enquanto cozinha; atualmente, refere"-se principalmente a alimentos cozidos dessa maneira}
\end{EntryWithPhonetic}

\begin{EntryWithPhonetic}{火海}{huo3hai3}{4,10}{⽕,⽔}
  \definition{s.}{um mar de chamas}
\end{EntryWithPhonetic}

\begin{EntryWithPhonetic}{火候}{huo3hou5}{4,10}{⽕,⼈}[HSK 7-9]
  \definition{s.}{duração e grau de aquecimento, cozimento, fundição, etc.  |Coloquial: nível de realização | Coloquial: momento crucial, crítico; emergência}
\end{EntryWithPhonetic}

\begin{EntryWithPhonetic}{火花}{huo3hua1}{4,7}{⽕,⾋}[HSK 7-9]
  \definition[簇]{s.}{faísca; explosão de chamas | um padrão brilhante}
\end{EntryWithPhonetic}

\begin{EntryWithPhonetic}{火箭}{huo3jian4}{4,15}{⽕,⽵}[HSK 6]
  \definition[个,艘,发,枚]{s.}{foguete; uma aeronave de alta velocidade que utiliza força de reação para se impulsionar para a frente; é usado para lançar satélites, naves espaciais, etc.; também pode ser equipado com uma ogiva para fabricar um míssil}
\end{EntryWithPhonetic}

\begin{EntryWithPhonetic}{火炬}{huo3ju4}{4,8}{⽕,⽕}[HSK 7-9]
  \definition[把]{s.}{tocha}
\end{EntryWithPhonetic}

\begin{EntryWithPhonetic}{火辣辣}{huo3la4la4}{4,14,14}{⽕,⾟,⾟}[HSK 7-9]
  \definition{adj.}{ardente; escaldante; abrasador}
\end{EntryWithPhonetic}

\begin{EntryWithPhonetic}{火热}{huo3re4}{4,10}{⽕,⽕}[HSK 7-9]
  \definition{adj.}{ardente; fervente; fervoroso; apaixonado; escaldante; abrasador (opp. 冰冷)}
  \seealsoref{冰冷}{bing1leng3}
\end{EntryWithPhonetic}

\begin{EntryWithPhonetic}{火山}{huo3shan1}{4,3}{⽕,⼭}[HSK 7-9]
  \definition[座]{s.}{vulcão}
\end{EntryWithPhonetic}

\begin{EntryWithPhonetic}{火速}{huo3su4}{4,10}{⽕,⾡}[HSK 7-9]
  \definition{adv.}{em alta velocidade; com pressa; usar a velocidade mais rápida (para fazer coisas urgentes)}
\end{EntryWithPhonetic}

\begin{EntryWithPhonetic}{火腿}{huo3tui3}{4,13}{⽕,⾁}[HSK 5]
  \definition[道,个]{s.}{presunto; as pernas de porco marinadas mais famosas são produzidas em Jinhua, na província de Zhejiang, e em Xuanwei, na província de Yunnan.}
\end{EntryWithPhonetic}

\begin{EntryWithPhonetic}{火焰}{huo3yan4}{4,12}{⽕,⽕}[HSK 7-9]
  \definition[团,缕,股,道]{s.}{chama; labareda; flama}
\end{EntryWithPhonetic}

\begin{EntryWithPhonetic}{火药}{huo3yao4}{4,9}{⽕,⾋}[HSK 7-9]
  \definition[桶,克]{s.}{pólvora; um tipo de explosivo que explode com fumaça, como pólvora preta, ou sem fumaça, como nitrato de celulose}
\end{EntryWithPhonetic}

\begin{EntryWithPhonetic}{火灾}{huo3zai1}{4,7}{⽕,⽕}[HSK 5]
  \definition[起,场]{s.}{fogo (como um desastre); conflagração; desastres causados por incêndios}
\end{EntryWithPhonetic}

%%%%%%%%%% 伙 %%%%%%%%%%
\subsection*{伙}\addcontentsline{loh}{figure}{伙 \dpy{huo3}}

\begin{EntryWithPhonetic}{伙}{huo3}{6}{⼈}[HSK 4]
  \definition{clas.}{grupo; multidão; banda}
  \definition{s.}{iguaria; alimentação; refeições | parceiro; companheiro | coletivo de colegas}
  \definition{v.}{combinar; unir}
\end{EntryWithPhonetic}

\begin{EntryWithPhonetic}{伙伴}{huo3ban4}{6,7}{⼈,⼈}[HSK 4]
  \definition[个,位,群]{s.}{parceiro; companheiro; antigo sistema militar de dez pessoas para uma fogueira, o chefe da fogueira, uma pessoa encarregada de cozinhar, com a fogueira é chamado de parceiro da fogueira, agora se refere à participação comum em uma determinada organização ou engajada em certas atividades}
\end{EntryWithPhonetic}

\begin{EntryWithPhonetic}{伙食}{huo3shi2}{6,9}{⼈,⾷}[HSK 7-9]
  \definition{s.}{angu; rancho; comida; refeições; refere"-se às refeições no refeitório coletivo da unidade}
\end{EntryWithPhonetic}

%%%%%%%%%% 和 %%%%%%%%%%
\subsection*{和}\addcontentsline{loh}{figure}{和 \dpy{huo4}}

\begin{EntryWithPhonetic}{和}{huo4}{8}{⼝}
  \definition{clas.}{usado para enxágues de roupas | usado para fervuras de ervas medicinais}
  \definition{v.}{misturar (ingredientes); misturar pós ou grãos; misturar com água para obter uma consistência mais líquida}
  \seeref{he2}
  \seeref{he4}
  \seeref{hu2}
  \seeref{huo2}
\end{EntryWithPhonetic}

%%%%%%%%%% 或 %%%%%%%%%%
\subsection*{或}\addcontentsline{loh}{figure}{或 \dpy{huo4}}

\begin{EntryWithPhonetic}{或}{huo4}{8}{⼽}[HSK 2]
  \definition{adv.}{talvez; possivelmente; provavelmente | (geralmente na forma negativa) um pouco; ligeiramente}
  \definition{conj.}{ou (indicando escolha); ou\dots ou\dots}
  \definition{pron.}{alguém; algumas pessoas; refere"-se a alguém ou algo, equivalente a 有人 ou 有的}
  \seealsoref{有的}{you3de5}
  \seealsoref{有人}{you3ren2}
\end{EntryWithPhonetic}

\begin{EntryWithPhonetic}{或多或少}{huo4duo1-huo4shao3}{8,6,8,4}{⼽,⼣,⼽,⼩}[HSK 7-9]
  \definition{expr.}{mais ou menos}
\end{EntryWithPhonetic}

\begin{EntryWithPhonetic}{或是}{huo4shi4}{8,9}{⼽,⽇}[HSK 5]
  \definition{adv.}{um ou outro; o outro}
  \definition{conj.}{ou; às vezes, é apenas uma de duas coisas}
\end{EntryWithPhonetic}

\begin{EntryWithPhonetic}{或许}{huo4xu3}{8,6}{⼽,⾔}[HSK 4]
  \definition{adv.}{talvez; possivelmente; receio; não tenho certeza}
\end{EntryWithPhonetic}

\begin{EntryWithPhonetic}{或者}{huo4zhe3}{8,8}{⼽,⽼}[HSK 2]
  \definition{adv.}{talvez; possivelmente}
  \definition{conj.}{ou (usado em expressões afirmativas); ou\dots ou\dots; usado em frases narrativas para indicar uma relação de escolha | ou (usado para indicar equação); indica relação de equivalência, indicando que os objetos anterior e posterior são iguais}
\end{EntryWithPhonetic}

%%%%%%%%%% 货 %%%%%%%%%%
\subsection*{货}\addcontentsline{loh}{figure}{货 \dpy{huo4}}

\begin{EntryWithPhonetic}{货}{huo4}{8}{⾙}[HSK 4]
  \definition{s.}{dinheiro; moeda | bens; mercadorias; \emph{commodity} | refere"-se a uma pessoa com um certo mau caráter (usado como um insulto) | riqueza; fortuna; um termo geral para dinheiro, joias, tecidos, etc.}
  \definition{v.}{vender}
\end{EntryWithPhonetic}

\begin{EntryWithPhonetic}{货币}{huo4bi4}{8,4}{⾙,⼱}[HSK 7-9]
  \definition[种]{s.}{dinheiro; moeda}
\end{EntryWithPhonetic}

\begin{EntryWithPhonetic}{货车}{huo4che1}{8,4}{⾙,⾞}[HSK 7-9]
  \definition[辆]{s.}{trem de mercadorias; trem de carga | vagão de carga; caminhão de carga | caminhão; caminhões e veículos de entrega}
\end{EntryWithPhonetic}

\begin{EntryWithPhonetic}{货机}{huo4ji1}{8,6}{⾙,⽊}
  \definition{s.}{aeronave de carga (ou avião); cargueiro aéreo; aeronaves utilizadas principalmente para o transporte de carga}
  \antonymref{客机}{ke4ji1}
\end{EntryWithPhonetic}

\begin{EntryWithPhonetic}{货物}{huo4wu4}{8,8}{⾙,⽜}[HSK 7-9]
  \definition[件,批,些,吨]{s.}{bens; mercadoria; \emph{commodity}; geralmente se refere a itens para venda}
\end{EntryWithPhonetic}

\begin{EntryWithPhonetic}{货运}{huo4yun4}{8,7}{⾙,⾡}[HSK 7-9]
  \definition{s.}{transporte de carga | carga; frete | mercadorias transportadas}
  \antonymref{客运}{ke4yun4}
\end{EntryWithPhonetic}

%%%%%%%%%% 获 %%%%%%%%%%
\subsection*{获}\addcontentsline{loh}{figure}{获 \dpy{huo4}}

\begin{EntryWithPhonetic}{获}{huo4}{10}{⾋}[HSK 4]
  \definition*{s.}{Sobrenome: Huo}
  \definition{v.}{capturar; pegar | obter; ganhar; colher | colher; ceifar}
\end{EntryWithPhonetic}

\begin{EntryWithPhonetic}{获得}{huo4de2}{10,11}{⾋,⼻}[HSK 4]
  \definition{v.}{adquirir; ganhar; obter; alcançar}
\end{EntryWithPhonetic}

\begin{EntryWithPhonetic}{获奖}{huo4 jiang3}{10,9}{⾋,⼤}[HSK 4]
  \definition{v.}{ganhar prêmio; ser recompensado; ganhar um prêmio; receber um prêmio}
\end{EntryWithPhonetic}

\begin{EntryWithPhonetic}{获取}{huo4qu3}{10,8}{⾋,⼜}[HSK 4]
  \definition{v.}{adquirir; obter; ganhar; colher}
\end{EntryWithPhonetic}

\begin{EntryWithPhonetic}{获胜}{huo4sheng4}{10,9}{⾋,⾁}[HSK 7-9]
  \definition{v.}{vencer; ser vitorioso; triunfar; alcançar a vitória}
\end{EntryWithPhonetic}

\begin{EntryWithPhonetic}{获悉}{huo4xi1}{10,11}{⾋,⼼}[HSK 7-9]
  \definition{v.}{saber (de um evento); receber notícias; ser informado}
\end{EntryWithPhonetic}

%%%%%%%%%% 祸 %%%%%%%%%%
\subsection*{祸}\addcontentsline{loh}{figure}{祸 \dpy{huo4}}

\begin{EntryWithPhonetic}{祸}{huo4}{11}{⽰}
  \definition[场]{s.}{infortúnio; desastre; calamidade | desgraça; catástrofe}
  \definition{v.}{trazer desastre; arruinar | causar problemas}
  \antonymref{福}{fu2}
\end{EntryWithPhonetic}

\begin{EntryWithPhonetic}{祸害}{huo4hai5}{11,10}{⽰,⼧}[HSK 7-9]
  \definition[个]{s.}{desastre; calamidade | maldição; flagelo}
  \definition{v.}{causar desastre; danificar; destruir; arruinar}
\end{EntryWithPhonetic}

%%%%%%%%%% 惑 %%%%%%%%%%
\subsection*{惑}\addcontentsline{loh}{figure}{惑 \dpy{huo4}}

\begin{EntryWithPhonetic}{惑}{huo4}{12}{⼼}
  \definition{v.}{ficar confuso; ficar perplexo | iludir; enganar; confundir}
\end{EntryWithPhonetic}

\begin{EntryWithPhonetic}{惑星}{huo4xing1}{12,9}{⼼,⽇}
  \definition{s.}{planeta}
  \seealsoref{行星}{xing2xing1}
\end{EntryWithPhonetic}

%%%%%%%%%% 霍 %%%%%%%%%%
\subsection*{霍}\addcontentsline{loh}{figure}{霍 \dpy{huo4}}

\begin{EntryWithPhonetic}{霍}{huo4}{16}{⾬}
  \definition*{s.}{Sobrenome: Huo}
  \definition{adv.}{Literário: de repente; rapidamente}
\end{EntryWithPhonetic}

\begin{EntryWithPhonetic}{霍乱}{huo4luan4}{16,7}{⾬,⼄}[HSK 7-9]
  \definition{s.}{cólera; uma doença altamente contagiosa causada pelo Vibrio Cholerae | gastroenterite aguda (geralmente se refere a sintomas como vômitos intensos, diarreia, dor abdominal e cólicas)}
\end{EntryWithPhonetic}

%%%%%%%%%% 豁 %%%%%%%%%%
\subsection*{豁}\addcontentsline{loh}{figure}{豁 \dpy{huo4}}

\begin{EntryWithPhonetic}{豁}{huo4}{17}{⾕}
  \definition{adj.}{claro; aberto; de mente aberta; generoso}
  \definition{v.}{isentar; remeter}
\end{EntryWithPhonetic}

\begin{EntryWithPhonetic}{豁达}{huo4da2}{17,6}{⾕,⾡}[HSK 7-9]
  \definition{adj.}{otimista; de mente aberta; aberto e claro}
\end{EntryWithPhonetic}

%%%%% EOF %%%%%


%%%%%%%%%%%%%%%%%%%%%%%%%%%%%%%% Não existem palavras com pinyin iniciado em "I"
 %%%
%%% J
%%%
\section*{J}\addcontentsline{toc}{section}{J}\addcontentsline{loh}{figure}{\#\#\#\#\#\#\#\# J}

%%%%%%%%%% 几 %%%%%%%%%%
\subsection*{几}\addcontentsline{loh}{figure}{几 \dpy{ji1}}

\begin{EntryWithPhonetic}{几}{ji1}{2}{⼏}[Kangxi 16]
  \definition{adv.}{quase; praticamente}
  \definition{s.}{uma mesa pequena}
  \seeref{ji3}
\end{EntryWithPhonetic}

\begin{EntryWithPhonetic}{几乎}{ji1hu1}{2,5}{⼏,⼃}[HSK 4]
  \definition{adv.}{quase; praticamente; próximo a | perto de; quase; à beira de}
\end{EntryWithPhonetic}

\begin{EntryWithPhonetic}{几率}{ji1lv4}{2,11}{⼏,⽞}[HSK 7-9]
  \definition{s.}{probabilidade; um evento pode ou não ocorrer sob as mesmas condições, a grandeza que indica a possibilidade de ocorrência é chamada de probabilidade}
\end{EntryWithPhonetic}

%%%%%%%%%% 讥 %%%%%%%%%%
\subsection*{讥}\addcontentsline{loh}{figure}{讥 \dpy{ji1}}

\begin{EntryWithPhonetic}{讥}{ji1}{4}{⾔}
  \definition{v.}{ridicularizar; zombar; satirizar}
\end{EntryWithPhonetic}

\begin{EntryWithPhonetic}{讥笑}{ji1xiao4}{4,10}{⾔,⽵}[HSK 7-9]
  \definition{v.}{ridicularizar; zombar; zombar de; tirar sarro de}
\end{EntryWithPhonetic}

%%%%%%%%%% 饥 %%%%%%%%%%
\subsection*{饥}\addcontentsline{loh}{figure}{饥 \dpy{ji1}}

\begin{EntryWithPhonetic}{饥}{ji1}{5}{⾷}
  \definition{adj.}{faminto; com fome}
  \definition[只]{s.}{fome; quebra de safra; colheita pobre ou inexistente}
\end{EntryWithPhonetic}

\begin{EntryWithPhonetic}{饥饿}{ji1'e4}{5,10}{⾷,⾷}[HSK 7-9]
  \definition{adj.}{faminto; esfomeado; estômago vazio; necessidade urgente de comer}[孩子们因饥饿哭泣。===As crianças choravam de fome.]
\end{EntryWithPhonetic}

%%%%%%%%%% 机 %%%%%%%%%%
\subsection*{机}\addcontentsline{loh}{figure}{机 \dpy{ji1}}

\begin{EntryWithPhonetic}{机}{ji1}{6}{⽊}
  \definition*{s.}{Sobrenome: Ji}
  \definition{adj.}{flexível; perspicaz; destreza; agilidade}
  \definition[台]{s.}{máquina; motor | avião; aeronave; aeroplano; refere"-se especificamente a aeronaves | ponto crucial; os fatores"-chave para a ocorrência e mudança das coisas | chance; ocasião; oportunidade; um momento crítico ou oportuno para o desenvolvimento e mudança das coisas | organismo; funções vitais dos organismos | besta; mecanismo de disparo de flechas de madeira em uma besta antiga | assuntos importantes; assuntos extremamente importantes e confidenciais | ideia; intenção}
\end{EntryWithPhonetic}

\begin{EntryWithPhonetic}{机舱}{ji1cang1}{6,10}{⽊,⾈}[HSK 7-9]
  \definition{s.}{sala de máquinas (de um navio) | compartimento de passageiros (de uma aeronave); cabine | espaço de máquinas; cabine de aeronave}
\end{EntryWithPhonetic}

\begin{EntryWithPhonetic}{机场}{ji1chang3}{6,6}{⽊,⼟}[HSK 1]
  \definition[个,家,处,座]{s.}{aeródromo; campo de aviação; aeroporto; campo de voo}
\end{EntryWithPhonetic}

\begin{EntryWithPhonetic}{机动}{ji1dong4}{6,6}{⽊,⼒}[HSK 7-9]
  \definition{adj.}{motorizado; movido a energia | flexível; manobrável; conveniente; móvel | em reserva; para uso emergencial}
\end{EntryWithPhonetic}

\begin{EntryWithPhonetic}{机动车}{ji1dong4che1}{6,6,4}{⽊,⼒,⾞}[HSK 6]
  \definition{s.}{veículo motorizado | veículo automotor; automóvel de passageiros: veículo comercial concebido e tecnicamente adequado para o transporte de passageiros e respetiva bagagem, incluindo o banco do condutor}
  \antonymref{人力车}{ren2li4che1}
\end{EntryWithPhonetic}

\begin{EntryWithPhonetic}{机构}{ji1gou4}{6,8}{⽊,⽊}[HSK 4]
  \definition[所]{s.}{órgão; organização; instituição; instalações; aparelhamento; configuração | mecanismo; funcionamento interno de uma máquina ou unidade | estrutura interna de uma organização}
\end{EntryWithPhonetic}

\begin{EntryWithPhonetic}{机关}{ji1guan1}{6,6}{⽊,⼋}[HSK 6]
  \definition{adj.}{operado por máquina | controlado mecanicamente}
  \definition[个]{s.}{engrenagem; mecanismo; Antigo: refere"-se a certos dispositivos controlados mecanicamente; também se refere às peças de frenagem de dispositivos mecânicos | escritório; órgão; corpo; instituição | esquema; maquinação; estratagema; um plano cuidadoso e inteligente}
\end{EntryWithPhonetic}

\begin{EntryWithPhonetic}{机会}{ji1hui5}{6,6}{⽊,⼈}[HSK 2]
  \definition[个,次,种,些]{s.}{chance; oportunidade; momento favorável raro}
\end{EntryWithPhonetic}

\begin{EntryWithPhonetic}{机甲}{ji1jia3}{6,5}{⽊,⽥}
  \definition{s.}{\emph{mecha} (robôs operados por humanos em mangá japonês)}
\end{EntryWithPhonetic}

\begin{EntryWithPhonetic}{机灵}{ji1ling5}{6,7}{⽊,⽕}[HSK 7-9]
  \definition{adj.}{inteligente; esperto; astuto; espirituoso}
\end{EntryWithPhonetic}

\begin{EntryWithPhonetic}{机密}{ji1mi4}{6,11}{⽊,⼧}[HSK 7-9]
  \definition{adj.}{secreto; classificado; privado; confidencial}
  \definition{s.}{segredo; assuntos confidenciais}
\end{EntryWithPhonetic}

\begin{EntryWithPhonetic}{机票}{ji1piao4}{6,11}{⽊,⽰}[HSK 1]
  \definition[张]{s.}{passagem aérea; passagem de avião}
  \seealsoref{飞机票}{fei1ji1piao4}
\end{EntryWithPhonetic}

\begin{EntryWithPhonetic}{机器}{ji1qi4}{6,16}{⽊,⼝}[HSK 3]
  \definition[台,部,个]{s.}{máquina; maquinário; motor; dispositivos e máquinas que são montados a partir de peças, podem funcionar, transformar energia ou produzir trabalho útil podem ser usados como ferramentas de produção, reduzindo a intensidade do trabalho humano e aumentando a produtividade | aparato; sistema político e econômico}
\end{EntryWithPhonetic}

\begin{EntryWithPhonetic}{机器人}{ji1qi4ren2}{6,16,2}{⽊,⼝,⼈}[HSK 5]
  \definition[个,些]{s.}{androide; golem | pessoa mecânica | robô}
\end{EntryWithPhonetic}

\begin{EntryWithPhonetic}{机械}{ji1xie4}{6,11}{⽊,⽊}[HSK 6]
  \definition{adj.}{rígido; mecânico; inflexível; uma metáfora para uma abordagem rígida e imutável}
  \definition[台,部,个]{s.}{máquina; maquinário; mecanismo; vários dispositivos compostos por princípios mecânicos}
\end{EntryWithPhonetic}

\begin{EntryWithPhonetic}{机遇}{ji1yu4}{6,12}{⽊,⾡}[HSK 4]
  \definition[个]{s.}{chance; oportunidade; circunstâncias favoráveis}
\end{EntryWithPhonetic}

\begin{EntryWithPhonetic}{机制}{ji1zhi4}{6,8}{⽊,⼑}[HSK 5]
  \definition{s.}{mecanismo; processado por máquina; feito por máquina}
\end{EntryWithPhonetic}

\begin{EntryWithPhonetic}{机智}{ji1zhi4}{6,12}{⽊,⽇}[HSK 7-9]
  \definition{adj.}{engenhoso; perspicaz; inteligente e adaptável}
\end{EntryWithPhonetic}

%%%%%%%%%% 肌 %%%%%%%%%%
\subsection*{肌}\addcontentsline{loh}{figure}{肌 \dpy{ji1}}

\begin{EntryWithPhonetic}{肌}{ji1}{6}{⾁}
  \definition[块,片]{s.}{músculo; carne | pele;}
\end{EntryWithPhonetic}

\begin{EntryWithPhonetic}{肌肤}{ji1fu1}{6,8}{⾁,⾁}[HSK 7-9]
  \definition{s.}{músculo e pele (humano)}
\end{EntryWithPhonetic}

\begin{EntryWithPhonetic}{肌肉}{ji1rou4}{6,6}{⾁,⾁}[HSK 5]
  \definition[身,块]{s.}{músculo; um dos tecidos básicos dos músculos humanos e de alguns animais, composto principalmente de células musculares fibrosas, pode se contrair, é o movimento do corpo e o corpo de digestão, respiração, circulação, excreção e outros processos fisiológicos da fonte de energia; pode ser dividido em três tipos: músculo liso, músculo esquelético e músculo cardíaco}
\end{EntryWithPhonetic}

%%%%%%%%%% 鸡 %%%%%%%%%%
\subsection*{鸡}\addcontentsline{loh}{figure}{鸡 \dpy{ji1}}

\begin{EntryWithPhonetic}{鸡}{ji1}{7}{⿃}[HSK 2]
  \definition*{s.}{Sobrenome: Ji}
  \definition[只]{s.}{galo, galinha, frango | palavra ofensiva para uma mulher que ganha dinheiro fazendo sexo com homens}
\end{EntryWithPhonetic}

\begin{EntryWithPhonetic}{鸡蛋}{ji1dan4}{7,11}{⿃,⾍}[HSK 1]
  \definition[个,枚,筐,箱,打]{s.}{ovo de galinha}
\end{EntryWithPhonetic}

\begin{EntryWithPhonetic}{鸡西}{ji1xi1}{7,6}{⿃,⾑}
  \definition*{s.}{Cidade no nível da prefeitura de Jixi, na província de Heilongjiang (黑龙江), no nordeste da China}
  \seealsoref{黑龙江}{hei1long2jiang1}
\end{EntryWithPhonetic}

%%%%%%%%%% 积 %%%%%%%%%%
\subsection*{积}\addcontentsline{loh}{figure}{积 \dpy{ji1}}

\begin{EntryWithPhonetic}{积}{ji1}{10}{⽲}[HSK 7-9]
  \definition{adj.}{de longa data; pendente há muito tempo | antiquíssimo; acumulado ao longo de um longo período de tempo}
  \definition{s.}{Medicina chinesa: indigestão (em bebês e crianças) | Matemática: produro, abreviação de 乘积}
  \definition{v.}{acumular; juntar; amontoar; reunir; coletar}
  \seealsoref{乘积}{cheng2ji1}
\end{EntryWithPhonetic}

\begin{EntryWithPhonetic}{积淀}{ji1dian4}{10,11}{⽲,⽔}[HSK 7-9]
  \definition{s.}{acumulação | depósitos acumulados ao longo de longos períodos | Figurativo: experiência valiosa, sabedoria acumulada}
  \definition{v.}{acumular}
\end{EntryWithPhonetic}

\begin{EntryWithPhonetic}{积极}{ji1ji2}{10,7}{⽲,⽊}[HSK 3]
  \definition{adj.}{ativo; descreve uma atitude proativa e esforçada | positivo; que tem um efeito positivo e ajuda no desenvolvimento das coisas}
\end{EntryWithPhonetic}

\begin{EntryWithPhonetic}{积累}{ji1lei3}{10,11}{⽲,⽷}[HSK 4]
  \definition{s.}{acúmulo; acumulação}
  \definition{v.}{acumular}
\end{EntryWithPhonetic}

\begin{EntryWithPhonetic}{积木}{ji1mu4}{10,4}{⽲,⽊}
  \definition{s.}{blocos de montar (brinquedo)}
\end{EntryWithPhonetic}

\begin{EntryWithPhonetic}{积蓄}{ji1xu4}{10,13}{⽲,⾋}[HSK 7-9]
  \definition[笔]{s.}{poupança; economias}
  \definition{v.}{acumular; poupar; economizar}
\end{EntryWithPhonetic}

%%%%%%%%%% 基 %%%%%%%%%%
\subsection*{基}\addcontentsline{loh}{figure}{基 \dpy{ji1}}

\begin{EntryWithPhonetic}{基}{ji1}{11}{⼟}
  \definition{adj.}{chave; básico; primário; cardinal; fundamental}
  \definition{s.}{base; fundação | base; grupo; radical; (química) uma parte dos átomos contidos na molécula de um composto, quando considerada como uma unidade, é chamada de base}
\end{EntryWithPhonetic}

\begin{EntryWithPhonetic}{基本}{ji1ben3}{11,5}{⼟,⽊}[HSK 3]
  \definition{adj.}{básico; fundamental; elementar | principal}
  \definition{adv.}{basicamente; em geral; no geral; em termos gerais}
  \definition{s.}{fundação}
\end{EntryWithPhonetic}

\begin{EntryWithPhonetic}{基本法}{ji1ben3fa3}{11,5,8}{⼟,⽊,⽔}
  \definition{s.}{lei básica (constituição)}
\end{EntryWithPhonetic}

\begin{EntryWithPhonetic}{基本功}{ji1ben3gong1}{11,5,5}{⼟,⽊,⼒}[HSK 7-9]
  \definition{s.}{treinamento básico; habilidade básica; técnica essencial}
\end{EntryWithPhonetic}

\begin{EntryWithPhonetic}{基本上}{ji1ben3shang5}{11,5,3}{⼟,⽊,⼀}[HSK 3]
  \definition{adv.}{basicamente; principalmente | em geral; de modo geral}
\end{EntryWithPhonetic}

\begin{EntryWithPhonetic}{基层}{ji1ceng2}{11,7}{⼟,⼫}[HSK 7-9]
  \definition{s.}{base; nível local; nível básico; o nível mais baixo de qualquer organização, tem a conexão mais direta com as massas}
\end{EntryWithPhonetic}

\begin{EntryWithPhonetic}{基础}{ji1chu3}{11,10}{⼟,⽯}[HSK 3]
  \definition[个,种,点,层]{s.}{base; fundamento; fundação; a essência ou o ponto de partida do desenvolvimento das coisas | básico; fundamental; refere"-se às condições mínimas | fundação do edifício; base do edifício}
\end{EntryWithPhonetic}

\begin{EntryWithPhonetic}{基地}{ji1di4}{11,6}{⼟,⼟}[HSK 5]
  \definition{s.}{base; como base para alguns negócios | base; um local dedicado à realização de um negócio}
\end{EntryWithPhonetic}

\begin{EntryWithPhonetic}{基督教}{ji1du1jiao4}{11,13,11}{⼟,⽬,⽁}[HSK 6]
  \definition*{s.}{Cristianismo; A Religião Cristã | Cristão}
\end{EntryWithPhonetic}

\begin{EntryWithPhonetic}{基金}{ji1jin1}{11,8}{⼟,⾦}[HSK 5]
  \definition[只,笔]{s.}{fundo; fundos reservados ou destinados ao estabelecimento ou desenvolvimento de uma empresa}
\end{EntryWithPhonetic}

\begin{EntryWithPhonetic}{基因}{ji1yin1}{11,6}{⼟,⼞}[HSK 7-9]
  \definition[段,个]{s.}{gene; unidade básica de um organismo que carrega e transmite informações genéticas; está localizado nos cromossomos do núcleo da célula}
\end{EntryWithPhonetic}

\begin{EntryWithPhonetic}{基于}{ji1yu2}{11,3}{⼟,⼆}[HSK 7-9]
  \definition{prep.}{devido a; por causa de; em vista de; apresentando a premissa ou base de uma ação}
\end{EntryWithPhonetic}

\begin{EntryWithPhonetic}{基准}{ji1zhun3}{11,10}{⼟,⼎}[HSK 7-9]
  \definition{s.}{referência; base; padrão inicial para medição | um critério; um padrão; uma referência; refere"-se a padrões básicos}
\end{EntryWithPhonetic}

%%%%%%%%%% 畸 %%%%%%%%%%
\subsection*{畸}\addcontentsline{loh}{figure}{畸 \dpy{ji1}}

\begin{EntryWithPhonetic}{畸}{ji1}{13}{⽥}
  \definition{adj.}{assimétrico; desequilibrado | irregular; anormal}
  \definition{s.}{Literário: uma quantidade fracionária (acima daquela mencionada em um número redondo); lotes ímpares | deformidade}
\end{EntryWithPhonetic}

\begin{EntryWithPhonetic}{畸形}{ji1xing2}{13,7}{⽥,⼺}[HSK 7-9]
  \definition{adj.}{Medicina: deformado; malformado | desequilibrado; torto; desigual; anormal | deformidade; dismorfia; dismorfose; malformação; disgenesia; monstruosidade; aberração; desenvolvimento anormal de uma parte}
\end{EntryWithPhonetic}

%%%%%%%%%% 激 %%%%%%%%%%
\subsection*{激}\addcontentsline{loh}{figure}{激 \dpy{ji1}}

\begin{EntryWithPhonetic}{激}{ji1}{16}{⽔}
  \definition*{s.}{Sobrenome: Ji}
  \definition{adj.}{afiado; feroz; violento | vívido}
  \definition{adv.}{bruscamente; ferozmente; violentamente}
  \definition{s.}{o impacto de ondas fortes contra a costa}
  \definition{v.}{bater; avançar; correr | despertar; estimular; incitar; excitar | ficar doente por se molhar | esfriar (colocando água gelada, etc.)}
\end{EntryWithPhonetic}

\begin{EntryWithPhonetic}{激动}{ji1dong4}{16,6}{⽔,⼒}[HSK 4]
  \definition{adj.}{animado; entusiasmado; empolgado}
  \definition{v.}{agitar; excitar; tornar fortes os sentimentos de alguém}
\end{EntryWithPhonetic}

\begin{EntryWithPhonetic}{激发}{ji1fa1}{16,5}{⽔,⼜}[HSK 7-9]
  \definition{v.}{despertar; desencadear; estimular; motivar; inspirar | excitar; mudar moléculas e átomos de um estado de energia mais baixo para um estado de energia mais alto}
\end{EntryWithPhonetic}

\begin{EntryWithPhonetic}{激光}{ji1guang1}{16,6}{⽔,⼉}[HSK 7-9]
  \definition*{s.}{LASER; \emph{Light Amplification by Stimulated Emission of Radiation}; amplificação de luz por emissão estimulada de radiação}
\end{EntryWithPhonetic}

\begin{EntryWithPhonetic}{激化}{ji1hua4}{16,4}{⽔,⼔}[HSK 7-9]
  \definition{v.}{aguçar; intensificar; tornar agudo}
\end{EntryWithPhonetic}

\begin{EntryWithPhonetic}{激活}{ji1huo2}{16,9}{⽔,⽔}[HSK 7-9]
  \definition{v.}{ativar; estimular certas substâncias no corpo para torná-las ativas | colocar em jogo; revigorar; estimular metaforicamente, influenciar algo, torná-lo ativo}
\end{EntryWithPhonetic}

\begin{EntryWithPhonetic}{激励}{ji1li4}{16,7}{⽔,⼒}[HSK 7-9]
  \definition{v.}{instar; impelir; inspirar; encorajar; usar palavras ou ações de outras pessoas para encorajar as pessoas a trabalhar mais e fazer melhor | dirigir; excitar; estimular ou excitar uma reação ou atividade}
\end{EntryWithPhonetic}

\begin{EntryWithPhonetic}{激烈}{ji1lie4}{16,10}{⽔,⽕}[HSK 4]
  \definition{adj.}{agudo; afiado; feroz; violento; intenso}
\end{EntryWithPhonetic}

\begin{EntryWithPhonetic}{激起}{ji1qi3}{16,10}{⽔,⾛}[HSK 7-9]
  \definition{v.}{excitar; agitar; despertar; evocar; desencadear}
\end{EntryWithPhonetic}

\begin{EntryWithPhonetic}{激情}{ji1qing2}{16,11}{⽔,⼼}[HSK 6]
  \definition{s.}{paixão; emoções fortes e explosivas, como êxtase, raiva, etc.}
\end{EntryWithPhonetic}

\begin{EntryWithPhonetic}{激素}{ji1su4}{16,10}{⽔,⽷}[HSK 7-9]
  \definition{s.}{Fisiológico: hormônio; uma substância química que regula a atividade celular}
\end{EntryWithPhonetic}

%%%%%%%%%% 鷄 %%%%%%%%%%
\subsection*{鷄}\addcontentsline{loh}{figure}{鷄 \dpy{ji1}}

\begin{EntryWithPhonetic}{鷄}{ji1}{21}{⿃}
  \variantof{鸡}
\end{EntryWithPhonetic}

%%%%%%%%%% 及 %%%%%%%%%%
\subsection*{及}\addcontentsline{loh}{figure}{及 \dpy{ji2}}

\begin{EntryWithPhonetic}{及}{ji2}{3}{⼃}[HSK 7-9]
  \definition*{s.}{Sobrenome: Ji}
  \definition{conj.}{e; bem como; conectando substantivos paralelos ou frases nominais}
  \definition{v.}{alcançar; chegar até | ser comparável a; alcançar (geralmente usado em termos negativos) | chegar a tempo para | estender-se a; cuidar de; envolver | dar}
\end{EntryWithPhonetic}

\begin{EntryWithPhonetic}{及格}{ji2/ge2}{3,10}{⼃,⽊}[HSK 4]
  \definition{v.+compl.}{passar; passar em um teste, exame, etc.}
  \synonymref{合格}{he2ge2}
\end{EntryWithPhonetic}

\begin{EntryWithPhonetic}{及其}{ji2 qi2}{3,8}{⼃,⼋}[HSK 7-9]
  \definition{conj.}{(conjunção que liga dois substantivos) e seu\dots.; e seus\dots.; e dele\dots.; e dela\dots; usado para conectar duas ou mais coisas para indicar que elas são de igual importância ou existem da mesma maneira}[文化及其发展影响社会。===A cultura e seu desenvolvimento influenciam a sociedade.]
\end{EntryWithPhonetic}

\begin{EntryWithPhonetic}{及时}{ji2shi2}{3,7}{⼃,⽇}[HSK 3]
  \definition{adj.}{oportuno; na hora certa; adequado; na ocasião certa}
  \definition{adv.}{prontamente; sem demora; imediatamente}
  \synonymref{随时}{sui2shi2}
  \antonymref{耽误}{dan1wu5}
\end{EntryWithPhonetic}

\begin{EntryWithPhonetic}{及早}{ji2zao3}{3,6}{⼃,⽇}[HSK 7-9]
  \definition{adv.}{o mais cedo possível; antes que seja tarde demais}
  \synonymref{趁早}{chen4zao3}
  \synonymref{赶早}{gan3zao3}
\end{EntryWithPhonetic}

%%%%%%%%%% 吉 %%%%%%%%%%
\subsection*{吉}\addcontentsline{loh}{figure}{吉 \dpy{ji2}}

\begin{EntryWithPhonetic}{吉}{ji2}{6}{⼝}
  \definition*{s.}{Província de Jilin, abreviação de 吉林 | Sobrenome: Ji}
  \definition{adj.}{sortudo; propício; auspicioso}
  \seealsoref{吉林}{ji2lin2}
  \antonymref{凶}{xiong1}
\end{EntryWithPhonetic}

\begin{EntryWithPhonetic}{吉利}{ji2li4}{6,7}{⼝,⼑}[HSK 6]
  \definition{adj.}{sortudo; auspicioso; propício}
\end{EntryWithPhonetic}

\begin{EntryWithPhonetic}{吉林}{ji2lin2}{6,8}{⼝,⽊}
  \definition*{s.}{Província de Jilin}
\end{EntryWithPhonetic}

\begin{EntryWithPhonetic}{吉普}{ji2pu3}{6,12}{⼝,⽇}[HSK 7-9]
  \definition*{s.}{Jeep (marca de carro)}
  \definition[辆]{s.}{Empéstimo linguístico: jipe}[他开着吉普车去沙漠旅行。===Ele fez uma viagem pelo deserto em seu jipe.]
  \seealsoref{吉普车}{ji2pu3che1}
\end{EntryWithPhonetic}

\begin{EntryWithPhonetic}{吉普车}{ji2pu3che1}{6,12,4}{⼝,⽇,⾞}
  \definition[辆]{s.}{Empréstimo linguístico: jipe (veículo militar)}
  \seealsoref{吉普}{ji2pu3}
\end{EntryWithPhonetic}

\begin{EntryWithPhonetic}{吉他}{ji2ta1}{6,5}{⼝,⼈}[HSK 7-9]
  \definition[把]{s.}{Empréstimo linguístico: violão}
\end{EntryWithPhonetic}

\begin{EntryWithPhonetic}{吉祥}{ji2xiang2}{6,10}{⼝,⽰}[HSK 6]
  \definition{adj.}{sortudo; auspicioso; propício}
  \definition[个,种]{s.}{sorte; auspiciosidade; propiciação; um sinal ou símbolo de boa sorte ou fortuna}
\end{EntryWithPhonetic}

\begin{EntryWithPhonetic}{吉祥物}{ji2xiang2wu4}{6,10,8}{⼝,⽰,⽜}[HSK 7-9]
  \definition{s.}{mascote}
\end{EntryWithPhonetic}

%%%%%%%%%% 级 %%%%%%%%%%
\subsection*{级}\addcontentsline{loh}{figure}{级 \dpy{ji2}}

\begin{EntryWithPhonetic}{级}{ji2}{6}{⽷}[HSK 2]
  \definition{clas.}{usado para degraus, escadas, pisos de torres, etc.}
  \definition[个,种]{s.}{nível; classificação; grau; classe | série; turma; qualquer uma das divisões anuais de um curso escolar | degrau}
\end{EntryWithPhonetic}

\begin{EntryWithPhonetic}{级别}{ji2bie2}{6,7}{⽷,⼑}[HSK 7-9]
  \definition[个]{s.}{classificação; nível; escala; a ordem da hierarquia devido às diferenças de identidade, \emph{status} e características}
\end{EntryWithPhonetic}

%%%%%%%%%% 即 %%%%%%%%%%
\subsection*{即}\addcontentsline{loh}{figure}{即 \dpy{ji2}}

\begin{EntryWithPhonetic}{即}{ji2}{7}{⼙}[HSK 7-9]
  \definition{adv.}{no presente; no futuro imediato | prontamente; imediatamente}
  \definition{conj.}{e | mesmo; mesmo que}[她瞟了一眼睡着的孩子,随即匆匆离开了。===Ela olhou para a criança adormecida e então saiu correndo. | 即使下雨我也去。===Eu irei mesmo que chova.]
  \definition{v.}{aproximar-se; alcançar; estar perto | assumir; ascender a; aceitar; começar a se envolver em | ser motivado pela ocasião | estar perto | Literário: ser; significar}
\end{EntryWithPhonetic}

\begin{EntryWithPhonetic}{即便}{ji2bian4}{7,9}{⼙,⼈}[HSK 7-9]
  \definition{conj.}{mesmo; mesmo que; embora; mesmo que isso signifique que, em uma situação hipotética ou extrema, o resultado não mudará}
\end{EntryWithPhonetic}

\begin{EntryWithPhonetic}{即便是}{ji2bian4shi4}{7,9,9}{⼙,⼈,⽇}
  \definition{conj.}{mesmo que seja}
\end{EntryWithPhonetic}

\begin{EntryWithPhonetic}{即或}{ji2huo4}{7,8}{⼙,⼽}
  \definition{conj.}{mesmo se/embora}
\end{EntryWithPhonetic}

\begin{EntryWithPhonetic}{即将}{ji2jiang1}{7,9}{⼙,⼨}[HSK 4]
  \definition{adv.}{em breve; estar prestes a; estar a ponto de}
\end{EntryWithPhonetic}

\begin{EntryWithPhonetic}{即可}{ji2ke3}{7,5}{⼙,⼝}[HSK 7-9]
  \definition{adv.}{equivalente a; pode então; pode imediatamente; e isso será suficiente (fazer algo); significa 就可以了}
  \seealsoref{就可以了}{jiu4 ke3yi3le5}
\end{EntryWithPhonetic}

\begin{EntryWithPhonetic}{即若}{ji2ruo4}{7,8}{⼙,⾋}
  \definition{conj.}{mesmo se/embora}
\end{EntryWithPhonetic}

\begin{EntryWithPhonetic}{即使}{ji2shi3}{7,8}{⼙,⼈}[HSK 5]
  \definition{conj.}{mesmo; mesmo que; mesmo se; apesar de; expressando uma concessão hipotética}
\end{EntryWithPhonetic}

\begin{EntryWithPhonetic}{即是}{ji2shi4}{7,9}{⼙,⽇}
  \definition{conj.}{aquilo é}
\end{EntryWithPhonetic}

%%%%%%%%%% 极 %%%%%%%%%%
\subsection*{极}\addcontentsline{loh}{figure}{极 \dpy{ji2}}

\begin{EntryWithPhonetic}{极}{ji2}{7}{⽊}[HSK 4]
  \definition*{s.}{Sobrenome: Ji}
  \definition{adj.}{máximo; extremo; final; supremo}
  \definition{adv.}{extremamente; excessivamente}
  \definition{s.}{o ponto máximo, mais alto; extremo; ápice; ponto culminante | pólo; as extremidades norte e sul da Terra; as extremidades de um ímã; a extremidade de uma fonte de alimentação ou de um aparelho elétrico onde a corrente entra ou sai do aparelho}
  \definition{v.}{chegar ao fim de; levar a extremos | Literário: fazer o máximo possível}
\end{EntryWithPhonetic}

\begin{EntryWithPhonetic}{极度}{ji2du4}{7,9}{⽊,⼴}[HSK 7-9]
  \definition{adv.}{extremamente; profundamente}
  \definition{s.}{extremo; excedente; o máximo; pólo}
\end{EntryWithPhonetic}

\begin{EntryWithPhonetic}{极端}{ji2duan1}{7,14}{⽊,⽴}[HSK 6]
  \definition{adj.}{extremo; absoluto; sem quaisquer restrições}
  \definition{adv.}{excessivamente; extremamente; alto grau de expressão}
  \definition{s.}{extremo; extremidade; o auge do desenvolvimento}
\end{EntryWithPhonetic}

\begin{EntryWithPhonetic}{……极了}{ji2le5}{7,2}{⽊,⼅}[HSK 3]
  \definition{expr.}{extremamente; alto grau de expressão}
\end{EntryWithPhonetic}

\begin{EntryWithPhonetic}{极力}{ji2li4}{7,2}{⽊,⼒}[HSK 7-9]
  \definition{v.}{fazer o máximo; não poupar esforços; tentar todos os meios possíveis}
\end{EntryWithPhonetic}

\begin{EntryWithPhonetic}{极其}{ji2qi2}{7,8}{⽊,⼋}[HSK 4]
  \definition{adv.}{mais; extremamente; excessivamente}
\end{EntryWithPhonetic}

\begin{EntryWithPhonetic}{极少数}{ji2 shao3shu4}{7,4,13}{⽊,⼩,⽁}[HSK 7-9]
  \definition{num.}{pequena minoria; apenas alguns; um punhado; extremamente poucos}
\end{EntryWithPhonetic}

\begin{EntryWithPhonetic}{极为}{ji2wei2}{7,4}{⽊,⼂}[HSK 7-9]
  \definition{adv.}{extremamente; excessivamente}
\end{EntryWithPhonetic}

\begin{EntryWithPhonetic}{极限}{ji2xian4}{7,8}{⽊,⾩}[HSK 7-9]
  \definition[个]{s.}{limite; extremidade; limite máximo}
\end{EntryWithPhonetic}

%%%%%%%%%% 急 %%%%%%%%%%
\subsection*{急}\addcontentsline{loh}{figure}{急 \dpy{ji2}}

\begin{EntryWithPhonetic}{急}{ji2}{9}{⼼}[HSK 2]
  \definition{adj.}{impaciente; ansioso | irritado; aborrecido; incomodado | rápido e intenso; veloz | urgente; premente}
  \definition{s.}{urgência; emergência; assunto urgente e grave}
  \definition{v.}{preocupar; deixar ansioso | estar ansioso para ajudar; tratar os problemas dos outros como se fossem urgentes e ajudar a resolvê-los imediatamente}
  \antonymref{缓}{huan3}
\end{EntryWithPhonetic}

\begin{EntryWithPhonetic}{急救}{ji2jiu4}{9,11}{⼼,⽁}[HSK 6]
  \definition{s.}{primeiros socorros; tratamento médico de emergência (para pessoas gravemente doentes ou gravemente feridas)}
  \definition{v.}{prestar primeiros socorros; dar tratamento de emergência}
\end{EntryWithPhonetic}

\begin{EntryWithPhonetic}{急剧}{ji2ju4}{9,10}{⼼,⼑}[HSK 7-9]
  \definition{adj.}{rápido; agudo; repentino}
  \definition{adv.}{Formal: rapidamente (geralmente mudando em uma direção ruim ou potencialmente levando a resultados ruins)}
\end{EntryWithPhonetic}

\begin{EntryWithPhonetic}{急忙}{ji2mang2}{9,6}{⼼,⼼}[HSK 4]
  \definition{adv.}{apressadamente; com pressa}
\end{EntryWithPhonetic}

\begin{EntryWithPhonetic}{急迫}{ji2po4}{9,8}{⼼,⾡}[HSK 7-9]
  \definition{adj.}{urgente; premente; imperativo}
\end{EntryWithPhonetic}

\begin{EntryWithPhonetic}{急性}{ji2xing4}{9,8}{⼼,⼼}[HSK 7-9]
  \definition{adj.}{aguda}
  \definition{s.}{pessoa impetuosa; cabeça quente}
  \seealsoref{急性儿}{ji2xing4r5}
  \antonymref{慢性}{man4xing4}
\end{EntryWithPhonetic}

\begin{EntryWithPhonetic}{急性儿}{ji2xing4r5}{9,8,2}{⼼,⼼,⼉}
  \definition{adj.}{impetuoso; temperamental; de temperamento explosivo; de disposição impaciente}
\end{EntryWithPhonetic}

\begin{EntryWithPhonetic}{急需}{ji2xu1}{9,14}{⼼,⾬}[HSK 7-9]
  \definition{v.}{estar em extrema necessidade de}
\end{EntryWithPhonetic}

\begin{EntryWithPhonetic}{急于}{ji2yu2}{9,3}{⼼,⼆}[HSK 7-9]
  \definition{v.}{estar (ser) ansioso; estar (ser) impaciente; estar ansioso para}
\end{EntryWithPhonetic}

\begin{EntryWithPhonetic}{急诊}{ji2zhen3}{9,7}{⼼,⾔}[HSK 7-9]
  \definition{s.}{pronto-socorro; emergência; tratamento de emergência; uma clínica ambulatorial especial em um hospital para pessoas com doenças agudas}
\end{EntryWithPhonetic}

\begin{EntryWithPhonetic}{急转弯}{ji2zhuan3wan1}{9,8,9}{⼼,⾞,⼸}[HSK 7-9]
  \definition{s.}{curva fechada; curva acetuada; cotovelo}
  \definition{v.}{Coloquial: (uma atitude, política, etc.) fazer uma mudança repentina | fazer uma curva repentina | fazer uma mudança radical}
  \seealsoref{急转弯儿}{ji2zhuan3wan1r5}
\end{EntryWithPhonetic}

\begin{EntryWithPhonetic}{急转弯儿}{ji2zhuan3wan1r5}{9,8,9,2}{⼼,⾞,⼸,⼉}
  \definition{s.}{curva fechada}
\end{EntryWithPhonetic}

%%%%%%%%%% 疾 %%%%%%%%%%
\subsection*{疾}\addcontentsline{loh}{figure}{疾 \dpy{ji2}}

\begin{EntryWithPhonetic}{疾}{ji2}{10}{⽧}
  \definition*{s.}{Sobrenome: Ji}
  \definition{s.}{doença; enfermidade; moléstia; padecimento | sofrimento; dor; dificuldade; mazela}
\end{EntryWithPhonetic}

\begin{EntryWithPhonetic}{疾病}{ji2bing4}{10,10}{⽧,⽧}[HSK 6]
  \definition[种]{s.}{doença; enfermidade; termo geral para doença}
\end{EntryWithPhonetic}

%%%%%%%%%% 脊 %%%%%%%%%%
\subsection*{脊}\addcontentsline{loh}{figure}{脊 \dpy{ji2}}

\begin{EntryWithPhonetic}{脊}{ji2}{10}{⾁}
  \definition{s.}{coluna vertebral (de humanos e animais) | espinha; costas; cumeeira; a parte de um objeto em forma de espinha}
  \seeref{ji3}
\end{EntryWithPhonetic}

%%%%%%%%%% 棘 %%%%%%%%%%
\subsection*{棘}\addcontentsline{loh}{figure}{棘 \dpy{ji2}}

\begin{EntryWithPhonetic}{棘}{ji2}{12}{⽊}
  \definition*{s.}{Sobrenome: Ji}
  \definition{s.}{árvore de jujuba | arbustos espinhosos; silvas | espinho}
\end{EntryWithPhonetic}

\begin{EntryWithPhonetic}{棘手}{ji2shou3}{12,4}{⽊,⼿}[HSK 7-9]
  \definition{adj.}{complicado; difícil; espinhoso; difícil de manusear}
\end{EntryWithPhonetic}

%%%%%%%%%% 集 %%%%%%%%%%
\subsection*{集}\addcontentsline{loh}{figure}{集 \dpy{ji2}}

\begin{EntryWithPhonetic}{集}{ji2}{12}{⾫}[HSK 6]
  \definition*{s.}{Sobrenome: Ji}
  \definition{clas.}{parte; volume}
  \definition[个,本]{s.}{mercado; feira rural | coleção; conjunto; antologia | Matemática: conjunto}
  \definition{v.}{reunir; coletar; montar}
\end{EntryWithPhonetic}

\begin{EntryWithPhonetic}{集合}{ji2he2}{12,6}{⾫,⼝}[HSK 4]
  \definition{s.}{conjunto; montagem; coleção; agregação}
  \definition{v.}{reunir-se; juntar-se | reunir, juntar, convocar}
\end{EntryWithPhonetic}

\begin{EntryWithPhonetic}{集会}{ji2hui4}{12,6}{⾫,⼈}[HSK 7-9]
  \definition[个,次]{s.}{assembleia; reunião}
  \definition{v.}{reunir; reunir-se}
\end{EntryWithPhonetic}

\begin{EntryWithPhonetic}{集结}{ji2jie2}{12,9}{⾫,⽷}[HSK 7-9]
  \definition{v.}{(especialmente tropas) reunir; concentrar; estabelecer; fortalecer}
\end{EntryWithPhonetic}

\begin{EntryWithPhonetic}{集体}{ji2ti3}{12,7}{⾫,⼈}[HSK 3]
  \definition{s.}{coletivo; comunidade; grupo; equipe; organizações ou grupos em que muitas pessoas trabalham, estudam e vivem juntas}
\end{EntryWithPhonetic}

\begin{EntryWithPhonetic}{集团}{ji2tuan2}{12,6}{⾫,⼞}[HSK 5]
  \definition[个,家,些]{s.}{anel; bloco; grupo; panelinha; círculo; grupo organizado para agir em conjunto com um determinado objetivo | grupo; entidade econômica com uma direção de negócios especializada, liderada por uma grande empresa com forte poder econômico e alta visibilidade, e formada pela combinação ou fusão de empresas relacionadas}
\end{EntryWithPhonetic}

\begin{EntryWithPhonetic}{集邮}{ji2/you2}{12,7}{⾫,⾢}[HSK 7-9]
  \definition[本]{s.}{filatelia; coleção de selos}
  \definition{v.+compl.}{colecionar selos}
\end{EntryWithPhonetic}

\begin{EntryWithPhonetic}{集中}{ji2zhong1}{12,4}{⾫,⼁}[HSK 3]
  \definition{adj.}{centralizado; concentrado}
  \definition{v.}{concentrar; centralizar; focar; acumular; reunir | reunir pessoas, coisas, forças, etc. dispersas; resumir opiniões, experiências, etc.}
  \antonymref{分散}{fen1san4}
\end{EntryWithPhonetic}

\begin{EntryWithPhonetic}{集装箱}{ji2zhuang1xiang1}{12,12,15}{⾫,⾐,⾋}[HSK 7-9]
  \definition{s.}{\emph{container}}
\end{EntryWithPhonetic}

\begin{EntryWithPhonetic}{集资}{ji2zi1}{12,10}{⾫,⾙}[HSK 7-9]
  \definition{v.}{angariar fundos; recolher dinheiro; reunir recursos | arrecadar (reunir) dinheiro; concentrar fundos; retirar dinheiro (de muitas fontes); arrecadar fundos; solicitar fundos}
\end{EntryWithPhonetic}

%%%%%%%%%% 嫉 %%%%%%%%%%
\subsection*{嫉}\addcontentsline{loh}{figure}{嫉 \dpy{ji2}}

\begin{EntryWithPhonetic}{嫉}{ji2}{13}{⼥}
  \definition{v.}{invejar | odiar | ter ciúmes; ter inveja}
\end{EntryWithPhonetic}

\begin{EntryWithPhonetic}{嫉妒}{ji2du4}{13,7}{⼥,⼥}[HSK 7-9]
  \definition{v.}{invejar; ter ciúmes de}
\end{EntryWithPhonetic}

%%%%%%%%%% 几 %%%%%%%%%%
\subsection*{几}\addcontentsline{loh}{figure}{几 \dpy{ji3}}

\begin{EntryWithPhonetic}{几}{ji3}{2}{⼏}[HSK 1][Kangxi 16]
  \definition{adv.}{quanto?, usado para perguntar sobre quantidade e tempo}
  \definition{num.}{alguns; vários; poucos; indica um número indeterminado maior que um e menor que dez}
  \seeref{ji1}
\end{EntryWithPhonetic}

\begin{EntryWithPhonetic}{几何}{ji3he2}{2,7}{⼏,⼈}
  \definition{s.}{geometria}
\end{EntryWithPhonetic}

%%%%%%%%%% 纪 %%%%%%%%%%
\subsection*{纪}\addcontentsline{loh}{figure}{纪 \dpy{ji3}}

\begin{EntryWithPhonetic}{纪}{ji3}{6}{⽷}
  \definition*{s.}{Sobrenome: Ji}
  \definition{s.}{disciplina | um período de doze anos (na China antiga); um período de anos | (geologia) subdivisão de uma era geológica; período}
  \definition{v.}{colocar por escrito; registrar; mesmo significado de 记, usado principalmente em 记录, 纪年, 纪元, 纪传, etc. | classificar (fios de seda)}
  \seeref{ji4}
  \seealsoref{记}{ji4}
  \seealsoref{纪传}{ji4 zhuan4}
  \seealsoref{记录}{ji4lu4}
  \seealsoref{纪年}{ji4nian2}
  \seealsoref{纪元}{ji4yuan2}
\end{EntryWithPhonetic}

%%%%%%%%%% 挤 %%%%%%%%%%
\subsection*{挤}\addcontentsline{loh}{figure}{挤 \dpy{ji3}}

\begin{EntryWithPhonetic}{挤}{ji3}{9}{⼿}[HSK 5]
  \definition{adj.}{lotado; congestionado; descreve um grande número de pessoas ou coisas e muito pouco espaço}
  \definition{v.}{empacotar; amontoar; aglomerar | sacudir; empurrar contra; empurrar alguém ou algo para longe com seu corpo com toda a força que puder| pressionar; apertar; expulsar por pressão}
\end{EntryWithPhonetic}

\begin{EntryWithPhonetic}{挤压}{ji3ya1}{9,6}{⼿,⼚}[HSK 7-9]
  \definition{v.}{pressionar; espremer; esmagar; aperta | Metalurgia: extrudar}
\end{EntryWithPhonetic}

%%%%%%%%%% 给 %%%%%%%%%%
\subsection*{给}\addcontentsline{loh}{figure}{给 \dpy{ji3}}

\begin{EntryWithPhonetic}{给}{ji3}{9}{⽷}
  \definition{adj.}{abundante; próspero; bem provido para}
  \definition{v.}{fornecer; prover}
  \seeref{gei3}
\end{EntryWithPhonetic}

\begin{EntryWithPhonetic}{给予}{ji3yu3}{9,4}{⽷,⼅}[HSK 6]
  \definition{v.}{dar; conceder; dar em troca}
\end{EntryWithPhonetic}

%%%%%%%%%% 脊 %%%%%%%%%%
\subsection*{脊}\addcontentsline{loh}{figure}{脊 \dpy{ji3}}

\begin{EntryWithPhonetic}{脊}{ji3}{10}{⾁}
  \definition{s.}{espinha dorsal; coluna vertebral | crista; cumeeira; espinhaço | vértebra}
  \seeref{ji2}
\end{EntryWithPhonetic}

\begin{EntryWithPhonetic}{脊梁}{ji3liang2}{10,11}{⾁,⽊}[HSK 7-9]
  \definition{s.}{espinha dorsal | coluna vertebral}
\end{EntryWithPhonetic}

%%%%%%%%%% 计 %%%%%%%%%%
\subsection*{计}\addcontentsline{loh}{figure}{计 \dpy{ji4}}

\begin{EntryWithPhonetic}{计}{ji4}{4}{⾔}[HSK 7-9]
  \definition*{s.}{Sobrenome: Ji}
  \definition{s.}{medidor; aferidor; indicador; um instrumento para medir ou calcular graus, tempo, etc. | ideia; ardil; estratagema; plano}
  \definition{v.}{contar; calcular; numerar | planejar; traçar; imaginar}
\end{EntryWithPhonetic}

\begin{EntryWithPhonetic}{计策}{ji4ce4}{4,12}{⾔,⽵}[HSK 7-9]
  \definition{s.}{estratagema; plano; manobra}
\end{EntryWithPhonetic}

\begin{EntryWithPhonetic}{计划}{ji4hua4}{4,6}{⾔,⼑}[HSK 2]
  \definition[个,项]{s.}{plano; projeto; programa; trabalho, ações, conteúdo e etapas previamente definidos}
  \definition{v.}{planejar; traçar um plano}
\end{EntryWithPhonetic}

\begin{EntryWithPhonetic}{计较}{ji4jiao4}{4,10}{⾔,⾞}[HSK 7-9]
  \definition{v.}{pechinchar; discutir sobre | argumentar; disputar | planejar; pensar sobre}
\end{EntryWithPhonetic}

\begin{EntryWithPhonetic}{计时}{ji4shi2}{4,7}{⾔,⽇}[HSK 7-9]
  \definition{v.}{contar pelo tempo; calcular com base em horas de trabalho e proficiência técnica}
\end{EntryWithPhonetic}

\begin{EntryWithPhonetic}{计算}{ji4suan4}{4,14}{⾔,⽵}[HSK 3]
  \definition{v.}{contar; calcular; computar; enumerar; encontrar a variável desconhecida | planejar; considerar | conspirar secretamente contra os outros; planejar secretamente prejudicar os outros}
\end{EntryWithPhonetic}

\begin{EntryWithPhonetic}{计算机}{ji4suan4ji1}{4,14,6}{⾔,⽵,⽊}[HSK 2]
  \definition[部,台]{s.}{computador; calculadora; máquinas capazes de realizar cálculos matemáticos são feitas com dispositivos mecânicos, como calculadoras manuais, outras são feitas com componentes eletrônicos, como computadores eletrônicos}
\end{EntryWithPhonetic}

\begin{EntryWithPhonetic}{计算机程序}{ji4suan4ji1 cheng2xu4}{4,14,6,12,7}{⾔,⽵,⽊,⽲,⼴}
  \definition{s.}{programa de computador}
\end{EntryWithPhonetic}

%%%%%%%%%% 记 %%%%%%%%%%
\subsection*{记}\addcontentsline{loh}{figure}{记 \dpy{ji4}}

\begin{EntryWithPhonetic}{记}{ji4}{5}{⾔}[HSK 1]
  \definition*{s.}{Sobrenome: Ji}
  \definition{clas.}{tapas, palmadas, bofetadas, etc.; usado para indicar o número de vezes que uma determinada ação é realizada}
  \definition{s.}{assinatura; bloco de notas; livro ou artigo que registra fatos | insígnia; indicação; \& comercial; símbolo | marca de nascença; manchas escuras presentes na pele desde o nascimento}
  \definition{v.}{lembrar; ter em mente; guardar na memória; manter a imagem na mente | escrever (anotar); registrar; inscrever}
\end{EntryWithPhonetic}

\begin{EntryWithPhonetic}{记得}{ji4 de5}{5,11}{⾔,⼻}[HSK 1]
  \definition{v.}{lembrar; recordar; lembrar-se; não esquecer | manter algo em mente; (informal) não se esquecer de fazer algo, usado para lembrar}
\end{EntryWithPhonetic}

\begin{EntryWithPhonetic}{记号}{ji4hao5}{5,5}{⾔,⼝}[HSK 7-9]
  \definition[个]{s.}{marca; sinal; marcas feitas para atrair atenção, auxiliar na identificação e na memória}
\end{EntryWithPhonetic}

\begin{EntryWithPhonetic}{记录}{ji4lu4}{5,8}{⾔,⼹}[HSK 3]
  \definition[份,名,位,个]{s.}{notas; registro | anotador; registrador; a pessoa que faz registros}
  \definition{v.}{tomar notas; registrar; escrever o que ouviu ou o que aconteceu; gravar o som ou a imagem com um gravador ou uma câmera de vídeo e transformar em algum tipo de obra}
\end{EntryWithPhonetic}

\begin{EntryWithPhonetic}{记性}{ji4xing5}{5,8}{⾔,⼼}
  \definition{s.}{memória (habilidade em reter informações)}
\end{EntryWithPhonetic}

\begin{EntryWithPhonetic}{记忆}{ji4yi4}{5,4}{⾔,⼼}[HSK 5]
  \definition[段]{s.}{memória; manter em sua mente uma imagem do passado}
  \definition{v.}{recordar; lembrar; lembrar-se ou recordar alguém ou algo do passado}
\end{EntryWithPhonetic}

\begin{EntryWithPhonetic}{记忆犹新}{ji4yi4-you2xin1}{5,4,7,13}{⾔,⼼,⽝,⽄}[HSK 7-9]
  \definition{expr.}{estar fresco na memória; lembrar vividamente; estar ainda muito vivo na memória; ainda poder lembrar vividamente\dots; permanecer fresco na memória; ainda ter lembranças frescas de\dots; ainda reter memórias de\dots; a memória ainda está fresca.; a coisa está fresca na memória}
\end{EntryWithPhonetic}

\begin{EntryWithPhonetic}{记载}{ji4zai3}{5,10}{⾔,⾞}[HSK 4]
  \definition[段,种,条]{s.}{registro; conta; artigos e materiais que registram eventos}
  \definition{v.}{registrar; colocar por escrito}
\end{EntryWithPhonetic}

\begin{EntryWithPhonetic}{记者}{ji4zhe3}{5,8}{⾔,⽼}[HSK 3]
  \definition[群,名,位]{s.}{repórter; correspondente; jornalista; profissionais dedicados a entrevistar e reportar notícias para a mídia}
\end{EntryWithPhonetic}

\begin{EntryWithPhonetic}{记住}{ji4 zhu5}{5,7}{⾔,⼈}[HSK 1]
  \definition{v.}{lembrar; aprender de cor; ter em mente; guardar na memória}
\end{EntryWithPhonetic}

%%%%%%%%%% 纪 %%%%%%%%%%
\subsection*{纪}\addcontentsline{loh}{figure}{纪 \dpy{ji4}}

\begin{EntryWithPhonetic}{纪}{ji4}{6}{⽷}
  \definition*{s.}{Sobrenome: Ji}
  \definition{s.}{disciplina | idade; época | Geologia: período | um período de doze anos (na China antiga); um período de anos | Geologia: subdivisão de uma era geológica}
  \definition{v.}{colocar por escrito; registrar | registrar, mesmo significado de 记, usado principalmente em 记录, 纪年, 纪元, 纪传, etc. | classificar (fios de seda)}
  \seeref{ji3}
  \seealsoref{记}{ji4}
  \seealsoref{纪传}{ji4 zhuan4}
  \seealsoref{记录}{ji4lu4}
  \seealsoref{纪年}{ji4nian2}
  \seealsoref{纪元}{ji4yuan2}
\end{EntryWithPhonetic}

\begin{EntryWithPhonetic}{纪录}{ji4lu4}{6,8}{⽷,⼹}[HSK 3]
  \definition[项,个]{s.}{recorde (esportes); o número mais alto ou mais baixo registrado em um determinado período de tempo}
\end{EntryWithPhonetic}

\begin{EntryWithPhonetic}{纪录片}{ji4lu4pian4}{6,8,4}{⽷,⼹,⽚}[HSK 7-9]
  \definition[个,部]{s.}{documentário; filme documentário}
\end{EntryWithPhonetic}

\begin{EntryWithPhonetic}{纪律}{ji4lv4}{6,9}{⽷,⼻}[HSK 4]
  \definition{s.}{disciplina; código de conduta que cada membro da vida coletiva deve observar}
\end{EntryWithPhonetic}

\begin{EntryWithPhonetic}{纪年}{ji4nian2}{6,6}{⽷,⼲}
  \definition{s.}{cronologia; uma maneira de numerar os anos | registro cronológico de eventos; anais; um dos gêneros de livros históricos é organizar fatos históricos em ordem cronológica}
\end{EntryWithPhonetic}

\begin{EntryWithPhonetic}{纪念}{ji4nian4}{6,8}{⽷,⼼}[HSK 3]
  \definition[个,次]{s.}{lembrança; recordação; usado para representar uma lembrança (objeto)}
  \definition{v.}{comemorar; expressar saudade por pessoas ou coisas através de objetos ou ações}
\end{EntryWithPhonetic}

\begin{EntryWithPhonetic}{纪念碑}{ji4nian4bei1}{6,8,13}{⽷,⼼,⽯}[HSK 7-9]
  \definition{s.}{monumento; memorial | cenotáfio; placa memorial}
\end{EntryWithPhonetic}

\begin{EntryWithPhonetic}{纪念馆}{ji4nian4guan3}{6,8,11}{⽷,⼼,⾷}[HSK 7-9]
  \definition{s.}{salão memorial; museu em memória de alguém | museu em memória de\dots}
\end{EntryWithPhonetic}

\begin{EntryWithPhonetic}{纪念日}{ji4nian4ri4}{6,8,4}{⽷,⼼,⽇}[HSK 7-9]
  \definition{s.}{dia de comemoração; um dia memorável}
\end{EntryWithPhonetic}

\begin{EntryWithPhonetic}{纪实}{ji4shi2}{6,8}{⽷,⼧}[HSK 7-9]
  \definition{s.}{registro de eventos reais; relatório no local; cobertura ao vivo de eventos ou incidentes}
\end{EntryWithPhonetic}

\begin{EntryWithPhonetic}{纪元}{ji4yuan2}{6,4}{⽷,⼉}
  \definition{s.}{o início de uma era (por exemplo, o reinado de um imperador) | época; era}
\end{EntryWithPhonetic}

\begin{EntryWithPhonetic}{纪传}{ji4 zhuan4}{6,6}{⽷,⼈}
  \definition{s.}{crônica; biografia}
\end{EntryWithPhonetic}

\begin{EntryWithPhonetic}{纪传体}{ji4 zhuan4 ti3}{6,6,7}{⽷,⼈,⼈}
  \definition{s.}{história apresentada em uma série de biografias | gênero histórico baseado em biografia}
\end{EntryWithPhonetic}

%%%%%%%%%% 忌 %%%%%%%%%%
\subsection*{忌}\addcontentsline{loh}{figure}{忌 \dpy{ji4}}

\begin{EntryWithPhonetic}{忌}{ji4}{7}{⼼}[HSK 7-9]
  \definition{s.}{medo; pavor; escrúpulo}
  \definition{v.}{ter ciúmes de; invejar | evitar; afastar-se de; esquivar-se de ; abster-se de | desistir; desistir}
\end{EntryWithPhonetic}

\begin{EntryWithPhonetic}{忌妒}{ji4du4}{7,7}{⼼,⼥}
  \definition{v.}{invejar; ter ciúmes de; estar infeliz ou até mesmo odiar ou ter ciúmes dos outros porque eles são melhores que eles em termos de talento, status, etc.}
\end{EntryWithPhonetic}

\begin{EntryWithPhonetic}{忌讳}{ji4hui4}{7,6}{⼼,⾔}[HSK 7-9]
  \definition[个,种]{s.}{tabu; certas palavras ou ações são tabu devido a costumes ou razões pessoais}
  \definition{v.}{evitar; ser tabu sobre; tentar evitar ou não querer que aconteça (algo que pode ter consequências negativas)}
\end{EntryWithPhonetic}

\begin{EntryWithPhonetic}{忌口}{ji4/kou3}{7,3}{⼼,⼝}[HSK 7-9]
  \definition{v.+compl.}{evitar certos alimentos (como quando se está doente); estar de dieta}
\end{EntryWithPhonetic}

%%%%%%%%%% 技 %%%%%%%%%%
\subsection*{技}\addcontentsline{loh}{figure}{技 \dpy{ji4}}

\begin{EntryWithPhonetic}{技}{ji4}{7}{⼿}
  \definition[门,项]{s.}{destreza; habilidade; estratagema | técnica; tecnologia}
\end{EntryWithPhonetic}

\begin{EntryWithPhonetic}{技俩}{ji4liang3}{7,9}{⼿,⼈}
  \definition{s.}{truque | estratagema | ardil | esquema | estratégia | tática}
\end{EntryWithPhonetic}

\begin{EntryWithPhonetic}{技能}{ji4neng2}{7,10}{⼿,⾁}[HSK 5]
  \definition[种,项]{s.}{habilidade técnica; domínio de uma habilidade ou técnica; capacidade de adquirir e aplicar conhecimento}
\end{EntryWithPhonetic}

\begin{EntryWithPhonetic}{技巧}{ji4qiao3}{7,5}{⼿,⼯}[HSK 4]
  \definition[个,些]{s.}{habilidade; técnica; habilidades engenhosas expressas em artes, artesanato, esportes, etc.}
\end{EntryWithPhonetic}

\begin{EntryWithPhonetic}{技术}{ji4shu4}{7,5}{⼿,⽊}[HSK 3]
  \definition[种,门,项]{s.}{habilidade; técnica; tecnologia; a experiência e o conhecimento acumulados pelo ser humano no processo de utilização e transformação da natureza, e refletidos no trabalho produtivo, também se referem, de maneira geral, a outras habilidades operacionais}
\end{EntryWithPhonetic}

\begin{EntryWithPhonetic}{技艺}{ji4yi4}{7,4}{⼿,⾋}[HSK 7-9]
  \definition{s.}{habilidade; arte; artes cênicas ou artesanato habilidosos}
\end{EntryWithPhonetic}

%%%%%%%%%% 系 %%%%%%%%%%
\subsection*{系}\addcontentsline{loh}{figure}{系 \dpy{ji4}}

\begin{EntryWithPhonetic}{系}{ji4}{7}{⽷}[HSK 4]
  \definition{v.}{amarrar; prender; abotoar; dar um nó}
  \seeref{xi4}
\end{EntryWithPhonetic}

%%%%%%%%%% 剂 %%%%%%%%%%
\subsection*{剂}\addcontentsline{loh}{figure}{剂 \dpy{ji4}}

\begin{EntryWithPhonetic}{剂}{ji4}{8}{⼑}[HSK 7-9]
  \definition*{s.}{Sobrenome: Ji}
  \definition{clas.}{dose}[一剂中药===uma dose de medicina chinesa]
  \definition{s.}{preparação (farmacêutica ou outra química) | pequeno pedaço de massa | agente; certas substâncias químicas}
  \definition{v.}{ajustar; regular}
\end{EntryWithPhonetic}

%%%%%%%%%% 季 %%%%%%%%%%
\subsection*{季}\addcontentsline{loh}{figure}{季 \dpy{ji4}}

\begin{EntryWithPhonetic}{季}{ji4}{8}{⼦}[HSK 4]
  \definition*{s.}{Sobrenome: Ji}
  \definition{s.}{estação; o ano é dividido em quatro estações, primavera, verão, outono e inverno, e uma estação dura três meses | temporada | o fim de uma era | o último mês de uma temporada | o quarto ou mais novo entre irmãos; último na ordem de precedência}
\end{EntryWithPhonetic}

\begin{EntryWithPhonetic}{季度}{ji4du4}{8,9}{⼦,⼴}[HSK 4]
  \definition[个]{s.}{trimestre; período de tempo trimestral}
\end{EntryWithPhonetic}

\begin{EntryWithPhonetic}{季节}{ji4jie2}{8,5}{⼦,⾋}[HSK 4]
  \definition[个]{s.}{estação (clima); um período característico do ano}
\end{EntryWithPhonetic}

%%%%%%%%%% 既 %%%%%%%%%%
\subsection*{既}\addcontentsline{loh}{figure}{既 \dpy{ji4}}

\begin{EntryWithPhonetic}{既}{ji4}{9}{⽆}[HSK 4]
  \definition*{s.}{Sobrenome: Ji}
  \definition{adv.}{já}
  \definition{conj.}{desde; como; agora que | assim como; e também; ambos\dots e\dots; usado em conjunto com advérbios como 且, 又, 也 para indicar uma combinação de ambas as situações}
  \seealsoref{且}{qie3}
  \seealsoref{也}{ye3}
  \seealsoref{又}{you4}
\end{EntryWithPhonetic}

\begin{EntryWithPhonetic}{既不……又不……}{ji4bu4 you4bu4}{9,4,2,4}{⽆,⼀,⼜,⼀}
  \definition{conj.}{nem\dots nem\dots}
\end{EntryWithPhonetic}

\begin{EntryWithPhonetic}{既然}{ji4ran2}{9,12}{⽆,⽕}[HSK 4]
  \definition{conj.}{como; desde; agora que; usado na primeira metade de uma frase, muitas vezes repetido na segunda metade pelos advérbios 就, 也, 还 para indicar que a premissa é primeiro declarada e depois inferida}
  \seealsoref{还}{hai2}
  \seealsoref{就}{jiu4}
  \seealsoref{也}{ye3}
\end{EntryWithPhonetic}

\begin{EntryWithPhonetic}{既又}{ji4you4}{9,2}{⽆,⼜}
  \definition{conj.}{desde | como | agora isso | os dois e | assim como}
\end{EntryWithPhonetic}

%%%%%%%%%% 迹 %%%%%%%%%%
\subsection*{迹}\addcontentsline{loh}{figure}{迹 \dpy{ji4}}

\begin{EntryWithPhonetic}{迹}{ji4}{9}{⾡}
  \definition[点,丝]{s.}{marca; traço; marca deixada para trás | restos; ruínas; vestígio; coisas deixadas por gerações anteriores (principalmente edifícios) | evento importante do passado; coisas feitas; feitos | aparência; ação; figura (escrita)
vestígios}
\end{EntryWithPhonetic}

\begin{EntryWithPhonetic}{迹象}{ji4xiang4}{9,11}{⾡,⾗}[HSK 7-9]
  \definition[种]{s.}{sinal; símbolo; indicação; refere"-se a vestígios e fenômenos que podem ser usados para inferir o passado ou o futuro das coisas}
\end{EntryWithPhonetic}

%%%%%%%%%% 继 %%%%%%%%%%
\subsection*{继}\addcontentsline{loh}{figure}{继 \dpy{ji4}}

\begin{EntryWithPhonetic}{继}{ji4}{10}{⽷}[HSK 7-9]
  \definition{adv.}{então; depois}
  \definition{s.}{filhos; prole}
  \definition{v.}{continuar; ter sucesso; seguir}
\end{EntryWithPhonetic}

\begin{EntryWithPhonetic}{继承}{ji4cheng2}{10,8}{⽷,⼿}[HSK 5]
  \definition{v.}{herdar (o patrimônio de uma pessoa falecida, etc.) de acordo com a lei | continuar; geralmente se refere à aceitação do estilo, da cultura, do conhecimento, etc., daqueles que nos precederam | continuar; os descendentes continuam o trabalho deixado por seus antecessores.}
\end{EntryWithPhonetic}

\begin{EntryWithPhonetic}{继而}{ji4'er2}{10,6}{⽷,⽽}[HSK 7-9]
  \definition{adv.}{então; depois; mais tarde}
\end{EntryWithPhonetic}

\begin{EntryWithPhonetic}{继父}{ji4fu4}{10,4}{⽷,⽗}[HSK 7-9]
  \definition{s.}{padrasto}
\end{EntryWithPhonetic}

\begin{EntryWithPhonetic}{继母}{ji4mu3}{10,5}{⽷,⽏}[HSK 7-9]
  \definition{s.}{madrasta}
\end{EntryWithPhonetic}

\begin{EntryWithPhonetic}{继续}{ji4xu4}{10,11}{⽷,⽷}[HSK 3]
  \definition{s.}{continuação}
  \definition{v.}{continuar; prosseguir | prosseguir; continuar; seguir em frente (com); (atividades, eventos, etc.) continuar após uma pausa ou um determinado período de tempo}
\end{EntryWithPhonetic}

%%%%%%%%%% 寂 %%%%%%%%%%
\subsection*{寂}\addcontentsline{loh}{figure}{寂 \dpy{ji4}}

\begin{EntryWithPhonetic}{寂}{ji4}{11}{⼧}
  \definition{adj.}{quieto; parado; silencioso | solitário}
\end{EntryWithPhonetic}

\begin{EntryWithPhonetic}{寂静}{ji4jing4}{11,14}{⼧,⾭}[HSK 7-9]
  \definition{adj.}{quieto; parado; silencioso; sem som; muito silencioso}
\end{EntryWithPhonetic}

\begin{EntryWithPhonetic}{寂寥}{ji4liao2}{11,14}{⼧,⼧}
  \definition{s.}{solidão | vasto e vazio | quieto e desolado (literário)}
\end{EntryWithPhonetic}

\begin{EntryWithPhonetic}{寂寞}{ji4mo4}{11,13}{⼧,⼧}[HSK 7-9]
  \definition{adj.}{solitário; só; isolado; deserto | quieto; parado; silencioso}
\end{EntryWithPhonetic}

%%%%%%%%%% 寄 %%%%%%%%%%
\subsection*{寄}\addcontentsline{loh}{figure}{寄 \dpy{ji4}}

\begin{EntryWithPhonetic}{寄}{ji4}{11}{⼧}[HSK 4]
  \definition*{s.}{Sobrenome: Ji}
  \definition{adj.}{adotado; fomentado; promovido}
  \definition{v.}{enviar; postar; remeter | confiar; depositar; colocar | depender de; apegar-se a}
\end{EntryWithPhonetic}

\begin{EntryWithPhonetic}{寄存}{ji4cun2}{11,6}{⼧,⼦}
  \definition{v.}{depositar | deixar algo com alguém | armazenar}
\end{EntryWithPhonetic}

\begin{EntryWithPhonetic}{寄递}{ji4di4}{11,10}{⼧,⾡}
  \definition{s.}{entrega de correspondência}
\end{EntryWithPhonetic}

\begin{EntryWithPhonetic}{寄放}{ji4fang4}{11,8}{⼧,⽅}
  \definition{v.}{deixar algo com alguém}
\end{EntryWithPhonetic}

\begin{EntryWithPhonetic}{寄居}{ji4ju1}{11,8}{⼧,⼫}
  \definition{s.}{morar longe de casa}
\end{EntryWithPhonetic}

\begin{EntryWithPhonetic}{寄卖}{ji4mai4}{11,8}{⼧,⼗}
  \definition{v.}{consignar para venda}
\end{EntryWithPhonetic}

\begin{EntryWithPhonetic}{寄生}{ji4sheng1}{11,5}{⼧,⽣}
  \definition{s.}{parasita | parasitismo}
  \definition{v.}{viver tirando vantagem dos outros | viver dentro ou sobre outro organismo como um parasita}
\end{EntryWithPhonetic}

\begin{EntryWithPhonetic}{寄生生活}{ji4sheng1sheng1huo2}{11,5,5,9}{⼧,⽣,⽣,⽔}
  \definition{s.}{parasitismo | vida parasitária}
\end{EntryWithPhonetic}

\begin{EntryWithPhonetic}{寄售}{ji4shou4}{11,11}{⼧,⼝}
  \definition{v.}{venda em consignação}
\end{EntryWithPhonetic}

\begin{EntryWithPhonetic}{寄送}{ji4song4}{11,9}{⼧,⾡}
  \definition{v.}{enviar | transmitir}
\end{EntryWithPhonetic}

\begin{EntryWithPhonetic}{寄宿}{ji4su4}{11,11}{⼧,⼧}
  \definition{s.}{embarque}
  \definition{v.}{embarcar}
\end{EntryWithPhonetic}

\begin{EntryWithPhonetic}{寄托}{ji4tuo1}{11,6}{⼧,⼿}[HSK 7-9]
  \definition{v.}{deixar com alguém; confiar aos cuidados de alguém; confiar | repousar; colocar (esperança, etc.) em; encontrar sustento em; depositar ideais, esperanças, sentimentos, etc. em (alguém ou alguma coisa)}
\end{EntryWithPhonetic}

\begin{EntryWithPhonetic}{寄望}{ji4wang4}{11,11}{⼧,⽉}
  \definition{v.}{depositar esperanças em}
\end{EntryWithPhonetic}

\begin{EntryWithPhonetic}{寄养}{ji4yang3}{11,9}{⼧,⼋}
  \definition{v.}{embarcar | promover | colocar sob os cuidados de alguém (uma criança, animal de estimação, etc.)}
\end{EntryWithPhonetic}

\begin{EntryWithPhonetic}{寄予}{ji4yu3}{11,4}{⼧,⼅}
  \definition{v.}{expressar | colocar (esperança, importância, etc.) em | mostrar}
\end{EntryWithPhonetic}

%%%%%%%%%% 旣 %%%%%%%%%%
\subsection*{旣}\addcontentsline{loh}{figure}{旣 \dpy{ji4}}

\begin{EntryWithPhonetic}{旣}{ji4}{11}{⽆}
  \variantof{既}
\end{EntryWithPhonetic}

%%%%%%%%%% 祭 %%%%%%%%%%
\subsection*{祭}\addcontentsline{loh}{figure}{祭 \dpy{ji4}}

\begin{EntryWithPhonetic}{祭}{ji4}{11}{⽰}[HSK 7-9]
  \definition{v.}{oferecer um sacrifício a | realizar uma cerimônia memorial para | empunhar; usar (arma mágica)}
  \seeref{zhai4}
\end{EntryWithPhonetic}

\begin{EntryWithPhonetic}{祭奠}{ji4dian4}{11,12}{⽰,⼤}[HSK 7-9]
  \definition{v.}{realizar uma cerimônia memorial para; lembrar e mostrar respeito por (os mortos) | realizar ou comparecer a um serviço memorial |  oferecer sacrifícios (aos ancestrais)}
\end{EntryWithPhonetic}

\begin{EntryWithPhonetic}{祭祀}{ji4si4}{11,7}{⽰,⽰}[HSK 7-9]
  \definition{v.}{oferecer sacrifícios aos deuses ou ancestrais; o antigo costume é preparar oferendas aos deuses, Budas ou ancestrais para mostrar respeito e buscar bênçãos; acreditar em (religião)}
\end{EntryWithPhonetic}

%%%%%%%%%% 加 %%%%%%%%%%
\subsection*{加}\addcontentsline{loh}{figure}{加 \dpy{jia1}}

\begin{EntryWithPhonetic}{加}{jia1}{5}{⼒}[HSK 2]
  \definition*{s.}{Canadá, abreviação de 加拿大 | Sobrenome: Jia}
  \definition{v.}{adicionar; somar | aumentar; incrementar; aumentar a quantidade ou o grau em relação ao original | inserir; adicionar; anexar; adicionar o que não existe; colocar no lugar | acrescentar; indica a realização de uma determinada ação | colocar uma coisa em cima da outra | impor ou aplicar algo a outra pessoa; atribuir um determinado comportamento a outra pessoa}
  \seealsoref{加拿大}{jia1na2da4}
\end{EntryWithPhonetic}

\begin{EntryWithPhonetic}{加班}{jia1/ban1}{5,10}{⼒,⽟}[HSK 4]
  \definition{v.+compl.}{fazer horas extras; trabalhar horas extras; aumentar o horário de trabalho ou os turnos além do limite de tempo prescrito}
\end{EntryWithPhonetic}

\begin{EntryWithPhonetic}{加工}{jia1/gong1}{5,3}{⼒,⼯}[HSK 3]
  \definition{s.}{processo | trabalho (de uma máquina)}
  \definition{v.+compl.}{processar; realizar diversos trabalhos em matérias"-primas e produtos semiacabados (como alterar dimensões, formas, propriedades, aumentar a precisão, pureza, etc.) para que atendam aos requisitos especificados | melhorar; polir; refere"-se a todos os tipos de trabalho que tornam o produto final mais perfeito e refinado}
\end{EntryWithPhonetic}

\begin{EntryWithPhonetic}{加紧}{jia1jin3}{5,10}{⼒,⽷}[HSK 7-9]
  \definition{v.}{intensificar; acelerar; aumentar a velocidade ou intensidade}
\end{EntryWithPhonetic}

\begin{EntryWithPhonetic}{加剧}{jia1ju4}{5,10}{⼒,⼑}[HSK 7-9]
  \definition{v.}{agravar; intensificar; exacerbar; tornar-se mais sério do que antes}
\end{EntryWithPhonetic}

\begin{EntryWithPhonetic}{加快}{jia1kuai4}{5,7}{⼒,⼼}[HSK 3]
  \definition{v.}{acelerar; aumentar a velocidade; agilizar}
\end{EntryWithPhonetic}

\begin{EntryWithPhonetic}{加盟}{jia1meng2}{5,13}{⼒,⽫}[HSK 6]
  \definition{v.}{aliar-se a; filiar-se a um sindicato; juntar-se a um grupo ou organização}
\end{EntryWithPhonetic}

\begin{EntryWithPhonetic}{加拿大}{jia1na2da4}{5,10,3}{⼒,⼿,⼤}
  \definition{s.}{Canadá}
\end{EntryWithPhonetic}

\begin{EntryWithPhonetic}{加拿大人}{jia1na2da4ren2}{5,10,3,2}{⼒,⼿,⼤,⼈}
  \definition{s.}{canadense | pessoa ou povo do Canadá}
\end{EntryWithPhonetic}

\begin{EntryWithPhonetic}{加强}{jia1qiang2}{5,12}{⼒,⼸}[HSK 3]
  \definition{v.}{fortalecer; engrandecer; reforçar; tornar mais forte ou mais eficaz}
\end{EntryWithPhonetic}

\begin{EntryWithPhonetic}{加热}{jia1 re4}{5,10}{⼒,⽕}[HSK 5]
  \definition{v.}{aquecer; esquentar; aumentar a temperatura de um objeto}
\end{EntryWithPhonetic}

\begin{EntryWithPhonetic}{加入}{jia1ru4}{5,2}{⼒,⼊}[HSK 4]
  \definition{v.}{juntar-se; unir-se; aderir a; tornar-se um membro de uma organização, grupo | adicionar; colocar em}
\end{EntryWithPhonetic}

\begin{EntryWithPhonetic}{加上}{jia1shang5}{5,3}{⼒,⼀}[HSK 5]
  \definition{conj.}{além disso; em adição}
  \definition{v.}{adicionar; acrescentar; dar; aumentar}
\end{EntryWithPhonetic}

\begin{EntryWithPhonetic}{加深}{jia1shen1}{5,11}{⼒,⽔}[HSK 7-9]
  \definition{v.}{aprofundar; aumentar a profundidade; torne-se mais profundo}
\end{EntryWithPhonetic}

\begin{EntryWithPhonetic}{加速}{jia1su4}{5,10}{⼒,⾡}[HSK 5]
  \definition{v.}{acelerar; agilizar}
\end{EntryWithPhonetic}

\begin{EntryWithPhonetic}{加速度}{jia1su4du4}{5,10,9}{⼒,⾡,⼴}
  \definition{s.}{velocidade acelerada; aceleração | Física: aceleração}
\end{EntryWithPhonetic}

\begin{EntryWithPhonetic}{加以}{jia1yi3}{5,4}{⼒,⼈}[HSK 5]
  \definition{conj.}{além disso; em adição; indica outras razões ou condições}
  \definition{v.aux.}{usado na frente de palavras dissilábicas para indicar como um objeto mencionado deve ser tratado ou descartado | usado antes de um verbo polifônico ou de um substantivo formado a partir de um verbo para indicar como tratar ou lidar com o que foi mencionado anteriormente}
\end{EntryWithPhonetic}

\begin{EntryWithPhonetic}{加油}{jia1/you2}{5,8}{⼒,⽔}[HSK 2]
  \definition{v.+compl.}{abastecer com óleo; reabastecer; adicionar combustível ou óleo lubrificante | fazer um esforço extra; dar o máximo; (Vamos lá!) metáfora para se esforçar ainda mais}
\end{EntryWithPhonetic}

\begin{EntryWithPhonetic}{加油工}{jia1 you2 gong1}{5,8,3}{⼒,⽔,⼯}
  \definition{s.}{frentista}
\end{EntryWithPhonetic}

\begin{EntryWithPhonetic}{加油站}{jia1you2zhan4}{5,8,10}{⼒,⽔,⽴}[HSK 4]
  \definition[个,座,家]{s.}{posto de gasolina; posto de combustível; postos de abastecimento para venda a varejo de gasolina e óleo para carros e outros veículos motorizados}
\end{EntryWithPhonetic}

\begin{EntryWithPhonetic}{加重}{jia1zhong4}{5,9}{⼒,⾥}[HSK 7-9]
  \definition{v.}{tornar mais pesado; aumentar o peso de; aumentar a quantidade de | tornar mais sério; agravar}
\end{EntryWithPhonetic}

%%%%%%%%%% 夹 %%%%%%%%%%
\subsection*{夹}\addcontentsline{loh}{figure}{夹 \dpy{jia1}}

\begin{EntryWithPhonetic}{夹}{jia1}{6}{⼤}[HSK 5]
  \definition{s.}{clipe, grampo, pasta, etc.}
  \definition{v.}{colocar no meio; pressionar de ambos os lados; aplicar força ou ação ao mesmo objeto de ambos os lados ao mesmo tempo | misturar; mesclar; intercalar}
  \seeref{ga1}
  \seeref{jia2}
\end{EntryWithPhonetic}

\begin{EntryWithPhonetic}{夹杂}{jia1 za2}{6,6}{⼤,⽊}
  \definition{v.}{ser misturado com; estar carregado de; adicionar (algo mais)}
\end{EntryWithPhonetic}

\begin{EntryWithPhonetic}{夹肢窝}{jia1 zhi1 wo1}{6,8,12}{⼤,⾁,⽳}
  \definition{s.}{axila; sovaco; também escrito como 胳肢窝}
  \seealsoref{胳肢窝}{ga1 zhi1 wo1}
\end{EntryWithPhonetic}

\begin{EntryWithPhonetic}{夹子}{jia1 zi5}{6,3}{⼤,⼦}
  \definition[个,堆,盒]{s.}{pasta; carteira; algo para guardar dinheiro, papel, etc. | clipe; grampo; pasta; pinça; ferramentas para prender coisas}
\end{EntryWithPhonetic}

%%%%%%%%%% 佳 %%%%%%%%%%
\subsection*{佳}\addcontentsline{loh}{figure}{佳 \dpy{jia1}}

\begin{EntryWithPhonetic}{佳}{jia1}{8}{⼈}
  \definition{adj.}{bom; ótimo; bonito; excelente | o melhor}
\end{EntryWithPhonetic}

\begin{EntryWithPhonetic}{佳节}{jia1jie2}{8,5}{⼈,⾋}[HSK 7-9]
  \definition{s.}{festival; época feliz de festival}[春节是中国人最重要的佳节。===O Festival da Primavera é o festival mais importante para o povo chinês.]
\end{EntryWithPhonetic}

%%%%%%%%%% 茄 %%%%%%%%%%
\subsection*{茄}\addcontentsline{loh}{figure}{茄 \dpy{jia1}}

\begin{EntryWithPhonetic}{茄}{jia1}{8}{⾋}
  \definition{s.}{caracter fonético usado em empréstimos linguísticos para o som ``jia'', embora 夹 seja mais comum}
  \seeref{qie2}
  \seealsoref{夹}{jia1}
\end{EntryWithPhonetic}

%%%%%%%%%% 家 %%%%%%%%%%
\subsection*{家}\addcontentsline{loh}{figure}{家 \dpy{jia1}}

\begin{EntryWithPhonetic}{家}{jia1}{10}{⼧}[HSK 1,2]
  \definition*{s.}{Sobrenome: Jia}
  \definition{adj.}{domado; domesticado; criado; alimentado | interno}
  \definition{clas.}{usado para famílias ou estabelecimentos comerciais; para uso doméstico; lojas; fábricas, etc.}
  \definition{pron.}{Educado: meu (irmã, tio, etc.)}
  \definition[个]{s.}{família; domicílio; clã | lar; casa; residência da família | pessoa ou família envolvida em um determinado comércio; pessoas que trabalham em determinada profissão ou que possuem determinada identidade | especialista em um determinado campo; pessoa que possui conhecimentos especializados ou se dedica a atividades específicas | escola de pensamento; rscola acadêmica | (em cartas de baralho, mah"-jong etc.) festa; lado; refere"-se a jogar xadrez ou cartas, em que uma das partes joga contra a outra | nacionalidade; referindo"-se à etnia | membros da família; parentes; pessoas ou famílias com quem você tem algum tipo de relação | membro do mesmo clã; pessoas com o mesmo sobrenome}
  \definition{suf.}{sufixo substantivo para designar um especialista em alguma atividade, como um músico ou revolucionário, para designar uma profissão como em -eiro, -ista, por exemplo 科学家}
  \seealsoref{科学家}{ke1xue2jia1}
\end{EntryWithPhonetic}

\begin{EntryWithPhonetic}{家电}{jia1dian4}{10,5}{⼧,⽥}[HSK 6]
  \definition[件,台]{s.}{eletrodomésticos, abreviação de 家用电器}
  \seealsoref{家用电器}{jia1yong4 dian4qi4}
\end{EntryWithPhonetic}

\begin{EntryWithPhonetic}{家伙}{jia1huo5}{10,6}{⼧,⼈}[HSK 7-9]
  \definition[些,个,群,帮]{s.}{ferramenta; utensílio; arma; refere"-se a ferramentas ou armas | cara; companheiro; refere"-se a pessoas (com desprezo ou humor)  | gado; animal doméstico}
\end{EntryWithPhonetic}

\begin{EntryWithPhonetic}{家家户户}{jia1jia1hu4hu4}{10,10,4,4}{⼧,⼧,⼾,⼾}[HSK 7-9]
  \definition{expr.}{cada família; cada lar}
\end{EntryWithPhonetic}

\begin{EntryWithPhonetic}{家教}{jia1jiao4}{10,11}{⼧,⽁}[HSK 7-9]
  \definition[个,名,位]{s.}{educação; educação familiar; disciplina doméstica; a educação de moral e etiqueta que os pais dão aos seus filhos |  tutor}
\end{EntryWithPhonetic}

\begin{EntryWithPhonetic}{家境}{jia1jing4}{10,14}{⼧,⼟}[HSK 7-9]
  \definition{s.}{situação financeira familiar; circunstâncias familiares}
\end{EntryWithPhonetic}

\begin{EntryWithPhonetic}{家俱}{jia1ju4}{10,10}{⼧,⼈}
  \definition{s.}{mobília}
\end{EntryWithPhonetic}

\begin{EntryWithPhonetic}{家具}{jia1ju5}{10,8}{⼧,⼋}[HSK 3]
  \definition[件,套,些,个]{s.}{móveis; mobiliário de casa; utensílios domésticos, incluem principalmente camas, mesas, cadeiras, armários, etc.}
\end{EntryWithPhonetic}

\begin{EntryWithPhonetic}{家里}{jia1li3}{10,7}{⼧,⾥}[HSK 1]
  \definition{s.}{(em) casa; (em sua) família | esposa}
\end{EntryWithPhonetic}

\begin{EntryWithPhonetic}{家禽}{jia1qin2}{10,12}{⼧,⽱}[HSK 7-9]
  \definition{s.}{aves domésticas | ave; pássaro doméstico}
\end{EntryWithPhonetic}

\begin{EntryWithPhonetic}{家人}{jia1ren2}{10,2}{⼧,⼈}[HSK 1]
  \definition{s.}{família (de alguém); membro da família; os membros de uma família}
\end{EntryWithPhonetic}

\begin{EntryWithPhonetic}{家属}{jia1shu3}{10,12}{⼧,⼫}[HSK 3]
  \definition{s.}{membros da família; dependentes (familiares); os membros da família que não sejam o próprio chefe da família, ou seja, os membros da família que não sejam o próprio trabalhador}
\end{EntryWithPhonetic}

\begin{EntryWithPhonetic}{家庭}{jia1ting2}{10,9}{⼧,⼴}[HSK 2]
  \definition[个,户]{s.}{família}
\end{EntryWithPhonetic}

\begin{EntryWithPhonetic}{家务}{jia1wu4}{10,5}{⼧,⼒}[HSK 4]
  \definition[堆,次,件]{s.}{trabalho doméstico; tarefas domésticas}
\end{EntryWithPhonetic}

\begin{EntryWithPhonetic}{家乡}{jia1xiang1}{10,3}{⼧,⼄}[HSK 3]
  \definition[片,座]{s.}{cidade natal; o lugar onde sua família vive há gerações}
\end{EntryWithPhonetic}

\begin{EntryWithPhonetic}{家用}{jia1yong4}{10,5}{⼧,⽤}[HSK 7-9]
  \definition{adj.}{doméstico; para uso doméstico}
  \definition[本]{s.}{despesas familiares; dinheiro para manutenção da casa}
  \definition{v.}{ser usado em casa; para uso doméstico}
\end{EntryWithPhonetic}

\begin{EntryWithPhonetic}{家用电器}{jia1yong4 dian4qi4}{10,5,5,16}{⼧,⽤,⽥,⼝}
  \definition{s.}{eletrodoméstico; refere"-se a diversos aparelhos elétricos utilizados na vida doméstica e coletiva}
\end{EntryWithPhonetic}

\begin{EntryWithPhonetic}{家喻户晓}{jia1yu4-hu4xiao3}{10,12,4,10}{⼧,⼝,⼾,⽇}[HSK 7-9]
  \definition{expr.}{nome familiar | conhecido por todas as famílias; amplamente conhecido; conhecido por todos}[他是家喻户晓的演员。===Ele é um ator famoso.]
\end{EntryWithPhonetic}

\begin{EntryWithPhonetic}{家园}{jia1yuan2}{10,7}{⼧,⼞}[HSK 6]
  \definition{s.}{casa; terra natal; um jardim em casa, geralmente referindo"-se à cidade natal ou à família}
\end{EntryWithPhonetic}

\begin{EntryWithPhonetic}{家长}{jia1zhang3}{10,4}{⼧,⾧}[HSK 2]
  \definition[位,名,个]{s.}{pais; patriarca; tutor; guardião; refere"-se aos pais ou outros responsáveis legais}
\end{EntryWithPhonetic}

\begin{EntryWithPhonetic}{家政}{jia1zheng4}{10,9}{⼧,⽁}[HSK 7-9]
  \definition{s.}{tarefas domésticas; trabalho de gestão doméstica}
\end{EntryWithPhonetic}

\begin{EntryWithPhonetic}{家族}{jia1zu2}{10,11}{⼧,⽅}[HSK 7-9]
  \definition[个]{s.}{clã; família; uma organização social baseada em relações de sangue, incluindo várias gerações do mesmo sangue}
\end{EntryWithPhonetic}

%%%%%%%%%% 傢 %%%%%%%%%%
\subsection*{傢}\addcontentsline{loh}{figure}{傢 \dpy{jia1}}

\begin{EntryWithPhonetic}{傢}{jia1}{12}{⼈}
  \definition{s.}{usado em 家伙  e 家俱}
  \variantof{家}
  \seealsoref{傢伙}{jia1huo5}
  \seealsoref{家俱}{jia1ju4}
\end{EntryWithPhonetic}

\begin{EntryWithPhonetic}{傢伙}{jia1huo5}{12,6}{⼈,⼈}
  \variantof{家伙}
\end{EntryWithPhonetic}

\begin{EntryWithPhonetic}{傢俱}{jia1ju4}{12,10}{⼈,⼈}
  \variantof{家俱}
\end{EntryWithPhonetic}

%%%%%%%%%% 嘉 %%%%%%%%%%
\subsection*{嘉}\addcontentsline{loh}{figure}{嘉 \dpy{jia1}}

\begin{EntryWithPhonetic}{嘉}{jia1}{14}{⼝}
  \definition*{s.}{Sobrenome: Jia}
  \definition{adj.}{bom; ótimo | auspicioso | excelente}
  \definition{v.}{elogiar; recomendar}
  \definition{v.}{elogiar}
\end{EntryWithPhonetic}

\begin{EntryWithPhonetic}{嘉宾}{jia1bin1}{14,10}{⼝,⼧}[HSK 6]
  \definition[个,位,名,些]{s.}{convidado}
\end{EntryWithPhonetic}

\begin{EntryWithPhonetic}{嘉年华}{jia1nian2hua2}{14,6,6}{⼝,⼲,⼗}[HSK 7-9]
  \definition{s.}{Empréstimo linguístico: carnaval}
\end{EntryWithPhonetic}

%%%%%%%%%% 夹 %%%%%%%%%%
\subsection*{夹}\addcontentsline{loh}{figure}{夹 \dpy{jia2}}

\begin{EntryWithPhonetic}{夹}{jia2}{6}{⼤}
  \definition{adj.}{forrado; com camada dupla; duas camadas (roupas, colchas, etc.) | pinçado; voz deliberadamente engraçada}
  \seeref{ga1}
  \seeref{jia1}
\end{EntryWithPhonetic}

%%%%%%%%%% 甲 %%%%%%%%%%
\subsection*{甲}\addcontentsline{loh}{figure}{甲 \dpy{jia3}}

\begin{EntryWithPhonetic}{甲}{jia3}{5}{⽥}[HSK 5]
  \definition*{s.}{Sobrenome: Jia}
  \definition{s.}{alfa; primeiro lugar; o primeiro dos caules celestiais, geralmente usado para indicar o primeiro em ordem ou classificação | concha; carapaça; crustáceos | unha; crostas queratinosas nos dedos das mãos e dos pés | armadura; equipamento de proteção feito de metal | Obsoleto: unidade de administração civil composta por 10 residências | uma palavra substituta para uma pessoa ou coisa indefinida; usado como pronome}
  \definition{v.}{ocupar o primeiro lugar; ser melhor do que}
\end{EntryWithPhonetic}

\begin{EntryWithPhonetic}{甲骨文}{jia3gu3wen2}{5,9,4}{⽥,⾻,⽂}
  \definition{s.}{escrituras de oráculos | inscrições em ossos de oráculos (forma original de escritura chinesa)}
\end{EntryWithPhonetic}

%%%%%%%%%% 假 %%%%%%%%%%
\subsection*{假}\addcontentsline{loh}{figure}{假 \dpy{jia3}}

\begin{EntryWithPhonetic}{假}{jia3}{11}{⼈}[HSK 2]
  \definition{adj.}{falso; artificial}
  \definition{conj.}{se; caso; no caso de; conecta frases, expressa relação hipotética, geralmente usada com 如, 若 e 使, equivalente a 如果}
  \definition[个,天]{s.}{falsificação; coisas falsas, irreais ou forjadas}
  \definition{v.}{emprestar | valer-se de; aproveitar; utilizar | supor; presumir; pressupor}
  \seeref{jia4}
  \seealsoref{如}{ru2}
  \seealsoref{如果}{ru2guo3}
  \seealsoref{若}{ruo4}
  \seealsoref{使}{shi3}
\end{EntryWithPhonetic}

\begin{EntryWithPhonetic}{假的}{jia3de5}{11,8}{⼈,⽩}
  \definition{adj.}{falso | substituto | simulado}
\end{EntryWithPhonetic}

\begin{EntryWithPhonetic}{假定}{jia3ding4}{11,8}{⼈,⼧}[HSK 7-9]
  \definition{adj.}{suposto; assim chamado}
  \definition[个,种]{s.}{hipótese; hipótese científica | suposição; postulação; presunção}
  \definition{v.}{supor; assumir; conceder; presumir}
\end{EntryWithPhonetic}

\begin{EntryWithPhonetic}{假冒}{jia3mao4}{11,9}{⼈,⽇}[HSK 7-9]
  \definition{v.}{passar-se por; passar por falso; fingir ser genuíno}
\end{EntryWithPhonetic}

\begin{EntryWithPhonetic}{假如}{jia3ru2}{11,6}{⼈,⼥}[HSK 4]
  \definition{conj.}{se; supondo; no caso}
\end{EntryWithPhonetic}

\begin{EntryWithPhonetic}{假设}{jia3she4}{11,6}{⼈,⾔}[HSK 7-9]
  \definition[个,种,些]{s.}{hipótese; na pesquisa científica, refere"-se à explicação ou conclusão que precisa ser comprovada com base em determinados fenômenos.}
  \definition{v.}{conceder; supor; assumir; presumir | fabricar; inventar; não ser baseado em fatos reais}
\end{EntryWithPhonetic}

\begin{EntryWithPhonetic}{假声}{jia3sheng1}{11,7}{⼈,⼠}
  \definition{s.}{falsete}
  \seealsoref{真声}{zhen1sheng1}
\end{EntryWithPhonetic}

\begin{EntryWithPhonetic}{假使}{jia3shi3}{11,8}{⼈,⼈}[HSK 7-9]
  \definition{conj.}{se; no caso de; supondo que}
\end{EntryWithPhonetic}

\begin{EntryWithPhonetic}{假证件}{jia3zheng4jian4}{11,7,6}{⼈,⾔,⼈}
  \definition{s.}{documentos falsos}
\end{EntryWithPhonetic}

\begin{EntryWithPhonetic}{假装}{jia3zhuang1}{11,12}{⼈,⾐}[HSK 7-9]
  \definition{v.}{fingir; assumir; simular; vestir; tentar fazer de conta; agir deliberadamente de uma forma diferente da situação real para fazer os outros acreditarem}
\end{EntryWithPhonetic}

%%%%%%%%%% 价 %%%%%%%%%%
\subsection*{价}\addcontentsline{loh}{figure}{价 \dpy{jia4}}

\begin{EntryWithPhonetic}{价}{jia4}{6}{⼈}[HSK 5]
  \definition{s.}{preço | valor; (figurativo) valores (éticos, culturais etc.) | Química: valência}
\end{EntryWithPhonetic}

\begin{EntryWithPhonetic}{价格}{jia4ge2}{6,10}{⼈,⽊}[HSK 3]
  \definition[个,种]{s.}{preço; tarifa; o valor monetário da mercadoria}
  \synonymref{代价}{dai4jia4}
  \synonymref{价值}{jia4zhi2}
  \antonymref{价钱}{jia4 qian2}
\end{EntryWithPhonetic}

\begin{EntryWithPhonetic}{价钱}{jia4 qian2}{6,10}{⼈,⾦}[HSK 3]
  \definition[个,种,笔]{s.}{preço}
  \synonymref{代价}{dai4jia4}
  \synonymref{价格}{jia4ge2}
  \synonymref{价值}{jia4zhi2}
\end{EntryWithPhonetic}

\begin{EntryWithPhonetic}{价位}{jia4wei4}{6,7}{⼈,⼈}[HSK 7-9]
  \definition{s.}{preço; nível de preço}
\end{EntryWithPhonetic}

\begin{EntryWithPhonetic}{价值}{jia4zhi2}{6,10}{⼈,⼈}[HSK 3]
  \definition{s.}{valor; o trabalho social necessário condensado nos produtos | valor; importância; efeitos positivos}
  \synonymref{代价}{dai4jia4}
  \synonymref{价钱}{jia4 qian2}
  \synonymref{价格}{jia4ge2}
\end{EntryWithPhonetic}

\begin{EntryWithPhonetic}{价值观}{jia4zhi2guan1}{6,10,6}{⼈,⼈,⾒}[HSK 7-9]
  \definition{s.}{valores; a visão geral sobre economia, política, moralidade, dinheiro, etc; as pessoas têm valores diferentes devido aos seus diferentes níveis sociais}
\end{EntryWithPhonetic}

%%%%%%%%%% 驾 %%%%%%%%%%
\subsection*{驾}\addcontentsline{loh}{figure}{驾 \dpy{jia4}}

\begin{EntryWithPhonetic}{驾}{jia4}{8}{⾺}[HSK 7-9]
  \definition*{s.}{Sobrenome: Jia}
  \definition{pron.}{Cortês: você; você mesmo}
  \definition[点]{s.}{carruagem do imperador; refere"-se especificamente ao carro do imperador, referindo"-se ao imperador | referindo"-se a um veículo, usado como um termo respeitoso para uma pessoa}
  \definition{v.}{atrelar; puxar (uma carroça, etc.) | dirigir (um veículo); pilotar (um avião); velejar (um barco) | montar; cavalgar}
\end{EntryWithPhonetic}

\begin{EntryWithPhonetic}{驾车}{jia4 che1}{8,4}{⾺,⾞}[HSK 7-9]
  \definition{v.}{dirigir um veículo}
\end{EntryWithPhonetic}

\begin{EntryWithPhonetic}{驾驶}{jia4shi3}{8,8}{⾺,⾺}[HSK 5]
  \definition{v.}{dirigir; pilotar; conduzir; guiar; operar (um carro, navio, avião, trator, etc.) para fazê-lo mover}
\end{EntryWithPhonetic}

\begin{EntryWithPhonetic}{驾驭}{jia4yu4}{8,5}{⾺,⾺}[HSK 7-9]
  \definition{v.}{dirigir; conduzir animais ou veículos para a frente | controlar; fazer algo agir de acordo com a vontade de alguém}
\end{EntryWithPhonetic}

\begin{EntryWithPhonetic}{驾照}{jia4zhao4}{8,13}{⾺,⽕}[HSK 5]
  \definition[本,张]{s.}{carteira de motorista}
\end{EntryWithPhonetic}

%%%%%%%%%% 架 %%%%%%%%%%
\subsection*{架}\addcontentsline{loh}{figure}{架 \dpy{jia4}}

\begin{EntryWithPhonetic}{架}{jia4}{9}{⽊}[HSK 3]
  \definition{clas.}{usado para coisas com pilares ou componentes mecânicos | quadrado (usado para montanhas)}
  \definition{s.}{estrutura; organização do corpo humano ou das coisas | prateleira; estante; suporte; componentes que sustentam objetos ou utensílios para colocar objetos, etc.}
  \definition{v.}{colocar para cima; erigir | brigar; discutir | resistir; repelir; afastar | sequestrar; levar alguém à força}
\end{EntryWithPhonetic}

\begin{EntryWithPhonetic}{架式}{jia4shi5}{9,6}{⽊,⼷}
  \variantof{架势}
\end{EntryWithPhonetic}

\begin{EntryWithPhonetic}{架势}{jia4shi5}{9,8}{⽊,⼒}[HSK 7-9]
  \definition{s.}{postura; atitude; posição (sobre um assunto, etc.)}
\end{EntryWithPhonetic}

\begin{EntryWithPhonetic}{架子}{jia4zi5}{9,3}{⽊,⼦}[HSK 7-9]
  \definition[个,种,套]{s.}{estrutura; suporte; um objeto feito de madeira, metal ou outros materiais que pode ser usado para armazenar ou pendurar coisas | esboço; estrutura; a organização e estrutura das coisas | ares; arrogância; maneiras altivas; pensar que você é melhor que os outros e fingir ser de uma certa maneira | postura; posição; pose}
\end{EntryWithPhonetic}

%%%%%%%%%% 假 %%%%%%%%%%
\subsection*{假}\addcontentsline{loh}{figure}{假 \dpy{jia4}}

\begin{EntryWithPhonetic}{假}{jia4}{11}{⼈}
  \definition[个,天]{s.}{feriado; férias; período de suspensão temporária do trabalho ou dos estudos, legal ou aprovado | licença; afastamento temporário; período de licença temporária do trabalho ou dos estudos, após aprovação}
  \seeref{jia3}
\end{EntryWithPhonetic}

\begin{EntryWithPhonetic}{假期}{jia4qi1}{11,12}{⼈,⽉}[HSK 2]
  \definition[个,段,次,种]{s.}{férias; feriados; período de licença}
\end{EntryWithPhonetic}

\begin{EntryWithPhonetic}{假日}{jia4ri4}{11,4}{⼈,⽇}[HSK 6]
  \definition[节]{s.}{feriado; dia de folga}
\end{EntryWithPhonetic}

%%%%%%%%%% 嫁 %%%%%%%%%%
\subsection*{嫁}\addcontentsline{loh}{figure}{嫁 \dpy{jia4}}

\begin{EntryWithPhonetic}{嫁}{jia4}{13}{⼥}[HSK 7-9]
  \definition{v.}{(uma mulher) casar | casar uma filha | transferir (uma culpa, perda, fardo, etc.)}
  \antonymref{娶}{qu3}
\end{EntryWithPhonetic}

\begin{EntryWithPhonetic}{嫁妆}{jia4zhuang5}{13,6}{⼥,⼥}[HSK 7-9]
  \definition{s.}{dote; enxoval}
\end{EntryWithPhonetic}

%%%%%%%%%% 奸 %%%%%%%%%%
\subsection*{奸}\addcontentsline{loh}{figure}{奸 \dpy{jian1}}

\begin{EntryWithPhonetic}{奸}{jian1}{6}{⼥}
  \definition{adj.}{perverso; maligno; traiçoeiro; malicioso}
  \definition{s.}{traidor; espião | pessoa perversa; pessoa traiçoeira | relações sexuais ilícitas; comportamento sexual impróprio}
  \definition{v.}{ter relações sexuais ilícitas}
\end{EntryWithPhonetic}

\begin{EntryWithPhonetic}{奸夫}{jian1fu1}{6,4}{⼥,⼤}
  \definition{s.}{homem adúltero}
\end{EntryWithPhonetic}

\begin{EntryWithPhonetic}{奸诈}{jian1zha4}{6,7}{⼥,⾔}[HSK 7-9]
  \definition{adj.}{fraudulento; astuto; enganoso; traiçoeiro; hipócrita e enganador, não confiável}
\end{EntryWithPhonetic}

%%%%%%%%%% 尖 %%%%%%%%%%
\subsection*{尖}\addcontentsline{loh}{figure}{尖 \dpy{jian1}}

\begin{EntryWithPhonetic}{尖}{jian1}{6}{⼩}[HSK 6]
  \definition{adj.}{pontiagudo; afilado; agudo | agudo; estridente; penetrante | mesquinho; pão-duro | mordaz; cáustico}
  \definition{s.}{ponto; ponta; topo | o melhor do seu tipo; a melhor escolha; a nata da safra; uma pessoa ou coisa notável}
  \definition{v.}{tornar (a voz, etc.) aguda; estridente}
\end{EntryWithPhonetic}

\begin{EntryWithPhonetic}{尖端}{jian1duan1}{6,14}{⼩,⽴}[HSK 7-9]
  \definition{adj.}{mais avançado; sofisticado; o mais alto nível de desenvolvimento (ciência e tecnologia, etc.)}
  \definition{s.}{ponta pontiaguda; pico; a ponta fina e pontiaguda de algo | topo; ápice; pico; uma metáfora para o mais alto nível de ciência e tecnologia}
\end{EntryWithPhonetic}

\begin{EntryWithPhonetic}{尖锐}{jian1rui4}{6,12}{⼩,⾦}[HSK 7-9]
  \definition{adj.}{pontiagudo; a ponta do objeto é fina e pequena, podendo facilmente entrar ou passar por outros objetos | incisivo; penetrante; conhecer e observar as coisas com muita precisão e profundidade | estridente; penetrante; o som é alto e fino, fazendo com que as pessoas se sintam desconfortáveis | afiado; intenso; muito intenso (discussão, luta, etc.)}
\end{EntryWithPhonetic}

%%%%%%%%%% 坚 %%%%%%%%%%
\subsection*{坚}\addcontentsline{loh}{figure}{坚 \dpy{jian1}}

\begin{EntryWithPhonetic}{坚}{jian1}{7}{⼟}
  \definition*{s.}{Sobrenome: Jian}
  \definition{adj.}{duro; firme; sólido; forte | firme; resoluto; constante}
  \definition{s.}{fortaleza; fortificação; um ponto fortemente fortificado; coisas sólidas, principalmente referindo"-se a posições | armadura}
\end{EntryWithPhonetic}

\begin{EntryWithPhonetic}{坚持}{jian1chi2}{7,9}{⼟,⼿}[HSK 3]
  \definition{v.}{persistir em; perseverar em; defender; insistir em; manter-se fiel a; aderir a; persistir com determinação e não desistir quando se depara com dificuldades | aderir a; insistir em; não alterar (os princípios, opiniões, pontos de vista originais, etc.)}
\end{EntryWithPhonetic}

\begin{EntryWithPhonetic}{坚持不懈}{jian1chi2-bu2xie4}{7,9,4,16}{⼟,⼿,⼀,⼼}[HSK 7-9]
  \definition{expr.}{aderir a algo incessantemente; manter-se consistente e persistentemente; incessante; perseverar incessantemente}
\end{EntryWithPhonetic}

\begin{EntryWithPhonetic}{坚定}{jian1ding4}{7,8}{⼟,⼧}[HSK 5]
  \definition{adj.}{firme; inabalável; inamovível; (posição, opinião, vontade, etc.) firme e estável, inabalável}
  \definition{v.}{fortalecer}
\end{EntryWithPhonetic}

\begin{EntryWithPhonetic}{坚固}{jian1gu4}{7,8}{⼟,⼞}[HSK 4]
  \definition{adj.}{firme; sólido; robusto; forte; durável; firmemente unidos e inquebráveis}
\end{EntryWithPhonetic}

\begin{EntryWithPhonetic}{坚决}{jian1jue2}{7,6}{⼟,⼎}[HSK 3]
  \definition{adj.}{firme; resoluto; (atitude, opinião, ação, etc.) determinado e inabalável}
\end{EntryWithPhonetic}

\begin{EntryWithPhonetic}{坚强}{jian1qiang2}{7,12}{⼟,⼸}[HSK 3]
  \definition{adj.}{forte; firme; convicto; (qualidades humanas, personalidade, determinação, etc.) firme e forte, não vacila diante das dificuldades}
  \definition{v.}{fortalecer; tornar forte; é a qualidade, a determinação, etc., que não vacilam}
\end{EntryWithPhonetic}

\begin{EntryWithPhonetic}{坚韧}{jian1ren4}{7,7}{⼟,⾱}[HSK 7-9]
  \definition{adj.}{resistente e elástico; forte e resiliente | firme e tenaz; não impaciente, não entediado}
\end{EntryWithPhonetic}

\begin{EntryWithPhonetic}{坚实}{jian1shi2}{7,8}{⼟,⼧}[HSK 7-9]
  \definition{adj.}{sólido; substancial; forte e resistente}
\end{EntryWithPhonetic}

\begin{EntryWithPhonetic}{坚守}{jian1shou3}{7,6}{⼟,⼧}[HSK 7-9]
  \definition{v.}{segurar; manter-se firme; manter-se firme em; aderir firmemente}
\end{EntryWithPhonetic}

\begin{EntryWithPhonetic}{坚信}{jian1xin4}{7,9}{⼟,⼈}[HSK 7-9]
  \definition{v.}{acreditar firmemente; estar firmemente convencido; estar totalmente confiante de}
\end{EntryWithPhonetic}

\begin{EntryWithPhonetic}{坚硬}{jian1ying4}{7,12}{⼟,⽯}[HSK 7-9]
  \definition{adj.}{duro; sólido; resistente; duro e forte}
\end{EntryWithPhonetic}

%%%%%%%%%% 歼 %%%%%%%%%%
\subsection*{歼}\addcontentsline{loh}{figure}{歼 \dpy{jian1}}

\begin{EntryWithPhonetic}{歼}{jian1}{7}{⽍}
  \definition{v.}{aniquilar; eliminar; destruir}
\end{EntryWithPhonetic}

\begin{EntryWithPhonetic}{歼灭}{jian1mie4}{7,5}{⽍,⽕}[HSK 7-9]
  \definition{v.}{aniquilar; eliminar; destruir}
\end{EntryWithPhonetic}

%%%%%%%%%% 间 %%%%%%%%%%
\subsection*{间}\addcontentsline{loh}{figure}{间 \dpy{jian1}}

\begin{EntryWithPhonetic}{间}{jian1}{7}{⾨}[HSK 1]
  \definition{clas.}{a menor unidade de uma casa; a menor unidade habitacional; cômodo}
  \definition{s.}{espaço entre duas partes  | (em um) tempo ou espaço definido | sala; quarto | uma seção de uma sala ou o espaço lateral entre dois pares de pilares | com um tempo ou espaço definido}
  \seeref{jian4}
\end{EntryWithPhonetic}

%%%%%%%%%% 浅 %%%%%%%%%%
\subsection*{浅}\addcontentsline{loh}{figure}{浅 \dpy{jian1}}

\begin{EntryWithPhonetic}{浅}{jian1}{8}{⽔}
  \definition{adj.}{murmurando, fluindo suavemente, gorgolejando suavemente}
  \definition{s.}{Onomatopéia: som de água em movimento}
  \seeref{qian3}
\end{EntryWithPhonetic}

%%%%%%%%%% 肩 %%%%%%%%%%
\subsection*{肩}\addcontentsline{loh}{figure}{肩 \dpy{jian1}}

\begin{EntryWithPhonetic}{肩}{jian1}{8}{⾁}[HSK 5]
  \definition*{s.}{Sobrenome: Jian}
  \definition{s.}{ombro; torso}
  \definition{v.}{assumir; empreender; carregar; suportar; suportar um fardo}
\end{EntryWithPhonetic}

\begin{EntryWithPhonetic}{肩膀}{jian1bang3}{8,14}{⾁,⾁}[HSK 7-9]
  \definition[个,副]{s.}{ombro}
\end{EntryWithPhonetic}

\begin{EntryWithPhonetic}{肩负}{jian1fu4}{8,6}{⾁,⾙}[HSK 7-9]
  \definition{v.}{assumir; empreender; carregar; suportar; ser confiado a}
\end{EntryWithPhonetic}

%%%%%%%%%% 艰 %%%%%%%%%%
\subsection*{艰}\addcontentsline{loh}{figure}{艰 \dpy{jian1}}

\begin{EntryWithPhonetic}{艰}{jian1}{8}{⾉}
  \definition{adj.}{difícil; duro}
\end{EntryWithPhonetic}

\begin{EntryWithPhonetic}{艰巨}{jian1ju4}{8,4}{⾉,⼯}[HSK 7-9]
  \definition{adj.}{árduo; oneroso; difícil; formidável}
\end{EntryWithPhonetic}

\begin{EntryWithPhonetic}{艰苦}{jian1ku3}{8,8}{⾉,⾋}[HSK 5]
  \definition{adj.}{duro; resistente; árduo; difícil; condições de trabalho ou de vida ruins que tornam as pessoas miseráveis}
\end{EntryWithPhonetic}

\begin{EntryWithPhonetic}{艰苦奋斗}{jian1ku3-fen4dou4}{8,8,8,4}{⾉,⾋,⼤,⽃}[HSK 7-9]
  \definition{expr.}{luta árdua; trabalho duro persistente}
  \definition{v.}{trabalhar duro e perseverantemente; lutar arduamente em meio às dificuldades; trabalhar diligentemente desafiando (apesar) das dificuldades; travar uma luta árdua}
\end{EntryWithPhonetic}

\begin{EntryWithPhonetic}{艰难}{jian1nan2}{8,10}{⾉,⾫}[HSK 5]
  \definition{adj.}{duro; árduo; difícil}
\end{EntryWithPhonetic}

\begin{EntryWithPhonetic}{艰险}{jian1xian3}{8,9}{⾉,⾩}[HSK 7-9]
  \definition{adj.}{perigoso}
  \definition{s.}{dificuldades e perigos}
\end{EntryWithPhonetic}

\begin{EntryWithPhonetic}{艰辛}{jian1xin1}{8,7}{⾉,⾟}[HSK 7-9]
  \definition{adj.}{duro; árduo; difícil}
\end{EntryWithPhonetic}

%%%%%%%%%% 兼 %%%%%%%%%%
\subsection*{兼}\addcontentsline{loh}{figure}{兼 \dpy{jian1}}

\begin{EntryWithPhonetic}{兼}{jian1}{10}{⼋}[HSK 7-9]
  \definition*{s.}{Sobrenome: Jian}
  \definition{adj.}{duplo; dobrado; duplicado | simultâneo; concomitante}
  \definition{adv.}{simultaneamente; concomitivamente; envolve várias coisas ao mesmo tempo.}
  \definition{v.}{ocupar um cargo simultâneo | ter dois ou mais empregos simultaneamente; fazer várias coisas ao mesmo tempo ou possuir várias coisas | Literário: reunir; unir em um só; anexar}
\end{EntryWithPhonetic}

\begin{EntryWithPhonetic}{兼顾}{jian1gu4}{10,10}{⼋,⾴}[HSK 7-9]
  \definition{v.}{levar em consideração duas ou mais coisas; dar atenção a duas ou mais coisas}
\end{EntryWithPhonetic}

\begin{EntryWithPhonetic}{兼任}{jian1ren4}{10,6}{⼋,⼈}[HSK 7-9]
  \definition{v.}{ocupar um cargo simultâneo; ter vários empregos ao mesmo tempo | realizar algo em tempo parcial; trabalhar em tempo parcial}
\end{EntryWithPhonetic}

\begin{EntryWithPhonetic}{兼容}{jian1rong2}{10,10}{⼋,⼧}[HSK 7-9]
  \definition{v.}{abranger a todos; ser compatível; aceitar e acomodar simultaneamente coisas ou aspectos diferentes.}
\end{EntryWithPhonetic}

\begin{EntryWithPhonetic}{兼职}{jian1zhi2}{10,11}{⼋,⽿}[HSK 7-9]
  \definition[份]{pron.}{vaga simultânea; emprego de meio período; cargos ocupados fora da função principal de emprego}
  \definition{v.}{ocupar dois ou mais cargos simultaneamente; exercer outras funções além do trabalho principal}
\end{EntryWithPhonetic}

%%%%%%%%%% 监 %%%%%%%%%%
\subsection*{监}\addcontentsline{loh}{figure}{监 \dpy{jian1}}

\begin{EntryWithPhonetic}{监}{jian1}{10}{⽫}
  \definition{s.}{prisão; cadeia}
  \definition{v.}{supervisionar; inspecionar; observar}
\end{EntryWithPhonetic}

\begin{EntryWithPhonetic}{监测}{jian1ce4}{10,9}{⽫,⽔}[HSK 6]
  \definition{v.}{monitorar; supervisionar e testar}
\end{EntryWithPhonetic}

\begin{EntryWithPhonetic}{监察}{jian1cha2}{10,14}{⽫,⼧}[HSK 7-9]
  \definition{v.}{supervisionar; controlar}
\end{EntryWithPhonetic}

\begin{EntryWithPhonetic}{监督}{jian1du1}{10,13}{⽫,⽬}[HSK 6]
  \definition[个,位,名]{s.}{monitoramento; supervisão; pessoas que supervisionam}
  \definition{v.}{controlar; supervisionar; superintender; monitorar e supervisionar de perto}
\end{EntryWithPhonetic}

\begin{EntryWithPhonetic}{监管}{jian1guan3}{10,14}{⽫,⽵}[HSK 7-9]
  \definition{v.}{monitorar; supervisionar}
\end{EntryWithPhonetic}

\begin{EntryWithPhonetic}{监护}{jian1hu4}{10,7}{⽫,⼿}[HSK 7-9]
  \definition{s.}{tutela}
  \definition{v.}{Lei: desempenhar as funções de tutor; agir como guardião; tutelar | Medicina: cuidar e zelar; vigiar}
\end{EntryWithPhonetic}

\begin{EntryWithPhonetic}{监控}{jian1kong4}{10,11}{⽫,⼿}[HSK 7-9]
  \definition{s.}{monitor}
  \definition{v.}{monitorar}
\end{EntryWithPhonetic}

\begin{EntryWithPhonetic}{监视}{jian1shi4}{10,8}{⽫,⾒}[HSK 7-9]
  \definition{v.}{manter vigilância; ficar de olho em; observar atentamente}
\end{EntryWithPhonetic}

\begin{EntryWithPhonetic}{监狱}{jian1yu4}{10,9}{⽫,⽝}[HSK 7-9]
  \definition[个,所,座]{s.}{cadeia; prisão; instituições estatais responsáveis pela aplicação de penas criminais; locais onde os presos são mantidos}
\end{EntryWithPhonetic}

%%%%%%%%%% 渐 %%%%%%%%%%
\subsection*{渐}\addcontentsline{loh}{figure}{渐 \dpy{jian1}}

\begin{EntryWithPhonetic}{渐}{jian1}{11}{⽔}
  \definition{v.}{encharcar; ficar saturado com | fluir para}
  \seeref{jian4}
\end{EntryWithPhonetic}

%%%%%%%%%% 煎 %%%%%%%%%%
\subsection*{煎}\addcontentsline{loh}{figure}{煎 \dpy{jian1}}

\begin{EntryWithPhonetic}{煎}{jian1}{13}{⽕}[HSK 7-9]
  \definition{s.}{Fitoterapia: decocção}
  \definition{v.}{fritar (em pouco óleo, sem mexer) | ferver (em água)}
\end{EntryWithPhonetic}

\begin{EntryWithPhonetic}{煎饼}{jian1bing3}{13,9}{⽕,⾷}
  \definition[张,块,摞]{s.}{uma panqueca fina feita de farinha de milho-miúdo}
\end{EntryWithPhonetic}

\begin{EntryWithPhonetic}{煎蛋}{jian1dan4}{13,11}{⽕,⾍}
  \definition{s.}{ovo frito}
\end{EntryWithPhonetic}

%%%%%%%%%% 拣 %%%%%%%%%%
\subsection*{拣}\addcontentsline{loh}{figure}{拣 \dpy{jian3}}

\begin{EntryWithPhonetic}{拣}{jian3}{8}{⼿}[HSK 7-9]
  \definition{v.}{escolher; selecionar | pegar; coletar; reunir | o mesmo que 捡}
  \seealsoref{捡}{jian3}
\end{EntryWithPhonetic}

%%%%%%%%%% 俭 %%%%%%%%%%
\subsection*{俭}\addcontentsline{loh}{figure}{俭 \dpy{jian3}}

\begin{EntryWithPhonetic}{俭}{jian3}{9}{⼈}
  \definition*{s.}{Sobrenome: Jian}
  \definition{adj.}{econômico; frugal | querendo; faltando; curto}
\end{EntryWithPhonetic}

\begin{EntryWithPhonetic}{俭省}{jian3sheng3}{9,9}{⼈,⽬}
  \definition{adj.}{econômico}
\end{EntryWithPhonetic}

%%%%%%%%%% 柬 %%%%%%%%%%
\subsection*{柬}\addcontentsline{loh}{figure}{柬 \dpy{jian3}}

\begin{EntryWithPhonetic}{柬}{jian3}{9}{⽊}
  \definition*{s.}{Sobrenome: Jian}
  \definition[张,封]{s.}{cartão; nota; carta; um termo geral para cartas, cartões de visita, postagens, etc.}
\end{EntryWithPhonetic}

\begin{EntryWithPhonetic}{柬埔寨}{jian3pu3zhai4}{9,10,14}{⽊,⼟,⼧}
  \definition*{s.}{Camboja}
\end{EntryWithPhonetic}

%%%%%%%%%% 捡 %%%%%%%%%%
\subsection*{捡}\addcontentsline{loh}{figure}{捡 \dpy{jian3}}

\begin{EntryWithPhonetic}{捡}{jian3}{10}{⼿}[HSK 6]
  \definition{v.}{coletar; reunir; apanhar; pegar coisas do chão}
\end{EntryWithPhonetic}

%%%%%%%%%% 减 %%%%%%%%%%
\subsection*{减}\addcontentsline{loh}{figure}{减 \dpy{jian3}}

\begin{EntryWithPhonetic}{减}{jian3}{11}{⼎}[HSK 4]
  \definition*{s.}{Sobrenome: Jian}
  \definition{v.}{subtrair; remover uma parte da quantidade original | reduzir; diminuir; cortar}
\end{EntryWithPhonetic}

\begin{EntryWithPhonetic}{减肥}{jian3/fei2}{11,8}{⼎,⾁}[HSK 4]
  \definition{v.+compl.}{perder peso; dieta, exercícios, medicamentos, massagem, cirurgia, etc., para reduzir o excesso de gordura corporal, de modo que o grau de obesidade seja reduzido}
\end{EntryWithPhonetic}

\begin{EntryWithPhonetic}{减免}{jian3mian3}{11,7}{⼎,⼉}[HSK 7-9]
  \definition{v.}{mitigar ou anular (uma punição) | reduzir ou isentar (impostos, etc.)}
\end{EntryWithPhonetic}

\begin{EntryWithPhonetic}{减轻}{jian3qing1}{11,9}{⼎,⾞}[HSK 5]
  \definition{v.}{aliviar; remeter; clarear; facilitar; mitigar}
\end{EntryWithPhonetic}

\begin{EntryWithPhonetic}{减弱}{jian3ruo4}{11,10}{⼎,⼸}[HSK 7-9]
  \definition{v.}{reduzir | enfraquecer}
\end{EntryWithPhonetic}

\begin{EntryWithPhonetic}{减少}{jian3shao3}{11,4}{⼎,⼩}[HSK 4]
  \definition{v.}{cair; reduzir; diminuir; subtrair uma parte}
\end{EntryWithPhonetic}

\begin{EntryWithPhonetic}{减速}{jian3/su4}{11,10}{⼎,⾡}[HSK 7-9]
  \definition{v.+compl.}{diminuir a velocidade; desacelerar; retardar | moderar; reduzir a velocidade; reduzir a marcha; atrasar; desacelerar; retardar}
  \antonymref{加速}{jia1su4}
\end{EntryWithPhonetic}

\begin{EntryWithPhonetic}{减压}{jian3ya1}{11,6}{⼎,⼚}[HSK 7-9]
  \definition{v.}{reduzir a pressão; descomprimir | reduzir o fardo | reduzir a pressão; despressurizar; descomprimir | relaxar}
\end{EntryWithPhonetic}

%%%%%%%%%% 剪 %%%%%%%%%%
\subsection*{剪}\addcontentsline{loh}{figure}{剪 \dpy{jian3}}

\begin{EntryWithPhonetic}{剪}{jian3}{11}{⼑}[HSK 5]
  \definition[把]{s.}{tesouras; tesouras de poda; cortadores | pinças; tenazes}
  \definition{v.}{cortar; aparar; tosquiar; cortar (com uma tesoura) | exterminar; eliminar; acabar com}
\end{EntryWithPhonetic}

\begin{EntryWithPhonetic}{剪刀}{jian3dao1}{11,2}{⼑,⼑}[HSK 5]
  \definition[把,个]{s.}{tesoura; tesoura de jardim; instrumento de ferro para cortar tecido, papel, barbante, etc., com duas lâminas interligadas que podem ser abertas e fechadas}
\end{EntryWithPhonetic}

\begin{EntryWithPhonetic}{剪子}{jian3zi5}{11,3}{⼑,⼦}[HSK 5]
  \definition[把]{s.}{tesouras; tesouras de podar; tosquiadeiras}
\end{EntryWithPhonetic}

%%%%%%%%%% 检 %%%%%%%%%%
\subsection*{检}\addcontentsline{loh}{figure}{检 \dpy{jian3}}

\begin{EntryWithPhonetic}{检}{jian3}{11}{⽊}
  \definition*{s.}{Sobrenome: Jian}
  \definition{v.}{verificar; inspecionar; examinar | conter-se; ter cuidado na conduta}
\end{EntryWithPhonetic}

\begin{EntryWithPhonetic}{检测}{jian3ce4}{11,9}{⽊,⽔}[HSK 4]
  \definition{v.}{testar; detectar; verificar}
\end{EntryWithPhonetic}

\begin{EntryWithPhonetic}{检查}{jian3cha2}{11,9}{⽊,⽊}[HSK 2]
  \definition[份,个,次]{s.}{autocrítica; reconhecer e criticar os próprios erros verbais ou escritos}
  \definition{v.}{verificar; inspecionar; examinar; verificar cuidadosamente para descobrir o problema | criticar a si mesmo; identificar seus pontos fracos e erros, e criticar seu próprio comportamento}
\end{EntryWithPhonetic}

\begin{EntryWithPhonetic}{检察}{jian3cha2}{11,14}{⽊,⼧}[HSK 7-9]
  \definition{v.}{realizar trabalho de procuradoria; refere"-se especificamente às atividades de supervisão legal realizadas pelas autoridades de supervisão legal do Estado, em conformidade com a lei}
\end{EntryWithPhonetic}

\begin{EntryWithPhonetic}{检讨}{jian3tao3}{11,5}{⽊,⾔}[HSK 7-9]
  \definition[份,个]{s.}{autocrítica}
  \definition{v.}{fazer uma autocrítica; identificar e reconhecer suas próprias falhas ou erros | examinar; inspecionar}
\end{EntryWithPhonetic}

\begin{EntryWithPhonetic}{检验}{jian3yan4}{11,10}{⽊,⾺}[HSK 5]
  \definition{v.}{testar; examinar; inspecionar}
\end{EntryWithPhonetic}

%%%%%%%%%% 简 %%%%%%%%%%
\subsection*{简}\addcontentsline{loh}{figure}{简 \dpy{jian3}}

\begin{EntryWithPhonetic}{简}{jian3}{13}{⽵}
  \definition*{s.}{Sobrenome: Jian}
  \definition{adj.}{simples; simplificado; breve | breve; em resumo; em poucas palavras}
  \definition{s.}{Arcaico: tiras de bambu (para escrever) | carta; correspondência}
  \definition{v.}{simplificar | (literário) selecionar; escolher}
  \antonymref{繁}{fan2}
\end{EntryWithPhonetic}

\begin{EntryWithPhonetic}{简称}{jian3cheng1}{13,10}{⽵,⽲}[HSK 7-9]
  \definition[个]{s.}{a forma abreviada de um nome; abreviação; forma mais curta; forma simplificada do nome}
  \definition{v.}{chamar algo abreviadamente; abreviar}[科技术语常常被简称。===Termos científicos e técnicos são frequentemente abreviados.]
\end{EntryWithPhonetic}

\begin{EntryWithPhonetic}{简单}{jian3dan1}{13,8}{⽵,⼗}[HSK 3]
  \definition{adj.}{simples; descomplicado; estrutura simples; poucas complicações; fácil de entender, usar ou lidar | comum; lugar-comum; (experiência, capacidade, etc.) comum (usado principalmente em frases negativas) | casual; simplificado; precipitado; pouco cuidadoso}
\end{EntryWithPhonetic}

\begin{EntryWithPhonetic}{简短}{jian3duan3}{13,12}{⽵,⽮}[HSK 7-9]
  \definition{adj.}{breve; curto; conciso; (palavras) não são longas}
\end{EntryWithPhonetic}

\begin{EntryWithPhonetic}{简化}{jian3hua4}{13,4}{⽵,⼔}[HSK 7-9]
  \definition{v.}{simplificar; transformar o complicado em simples}
\end{EntryWithPhonetic}

\begin{EntryWithPhonetic}{简洁}{jian3jie2}{13,9}{⽵,⽔}[HSK 7-9]
  \definition{adj.}{conciso; sucinto; breve; (oratória e escrita) conciso e direto ao ponto, sem palavras desnecessárias}
\end{EntryWithPhonetic}

\begin{EntryWithPhonetic}{简介}{jian3jie4}{13,4}{⽵,⼈}[HSK 6]
  \definition{s.}{breve introdução; sinopse; relato resumido}
  \definition{v.}{fazer um breve relato de (algo)}
\end{EntryWithPhonetic}

\begin{EntryWithPhonetic}{简历}{jian3li4}{13,4}{⽵,⼚}[HSK 4]
  \definition[个,份]{s.}{currículo; \emph{curriculum vitae} (CV); notas biográficas}
\end{EntryWithPhonetic}

\begin{EntryWithPhonetic}{简陋}{jian3lou4}{13,8}{⽵,⾩}[HSK 7-9]
  \definition{adj.}{simples e grosseiro}
\end{EntryWithPhonetic}

\begin{EntryWithPhonetic}{简体字}{jian3ti3zi4}{13,7,6}{⽵,⼈,⼦}[HSK 7-9]
  \definition{s.}{caracteres chineses simplificados (em oposição a 繁体字)}
  \seealsoref{繁体字}{fan2ti3zi4}
\end{EntryWithPhonetic}

\begin{EntryWithPhonetic}{简要}{jian3yao4}{13,9}{⽵,⾑}[HSK 7-9]
  \definition{adj.}{breve; conciso}
\end{EntryWithPhonetic}

\begin{EntryWithPhonetic}{简易}{jian3yi4}{13,8}{⽵,⽇}[HSK 7-9]
  \definition{adj.}{simples e fácil | de construção simples; com equipamentos simples; sem sofisticação}
\end{EntryWithPhonetic}

\begin{EntryWithPhonetic}{简直}{jian3zhi2}{13,8}{⽵,⽬}[HSK 3]
  \definition{adv.}{simplesmente; de forma alguma; praticamente; significa ``exatamente assim'' (tom exagerado)}
\end{EntryWithPhonetic}

%%%%%%%%%% 见 %%%%%%%%%%
\subsection*{见}\addcontentsline{loh}{figure}{见 \dpy{jian4}}

\begin{EntryWithPhonetic}{见}{jian4}{4}{⾒}[HSK 1][Kangxi 147]
  \definition*{s.}{Sobrenome: Jian}
  \definition{part.}{usado antes de um verbo para indicar voz passiva ou para expressar como isso me afeta}
  \definition{s.}{visão; ideia; opinião sobre algo; ponto de vista}
  \definition{v.}{ver; avistar | encontrar-se com; ser exposto a | parecer ser; mostrar evidência de | ver; referir-se a; indicar a fonte ou o local onde deve ser consultado | ver; encontrar; convocar}
  \seeref{xian4}
\end{EntryWithPhonetic}

\begin{EntryWithPhonetic}{见到}{jian4dao4}{4,8}{⾒,⼑}[HSK 2]
  \definition{v.}{ver | encontrar; esbarrar; deparar-se com}
\end{EntryWithPhonetic}

\begin{EntryWithPhonetic}{见过}{jian4guo4}{4,6}{⾒,⾡}[HSK 2]
  \definition{s.}{visto (ver); já viu alguém ou algo; indica um momento no passado; alguém já viu ou encontrou um determinado objeto}
\end{EntryWithPhonetic}

\begin{EntryWithPhonetic}{见解}{jian4jie3}{4,13}{⾒,⾓}[HSK 7-9]
  \definition[种,点,番]{s.}{ponto de vista; opinião; entendimento; compreensão e perspectivas sobre as coisas}
\end{EntryWithPhonetic}

\begin{EntryWithPhonetic}{见面}{jian4/mian4}{4,9}{⾒,⾯}[HSK 1]
  \definition{v.+compl.}{encontrar-se com alguém;  ver um ao outro; ver alguém face-a-face}
\end{EntryWithPhonetic}

\begin{EntryWithPhonetic}{见钱眼开}{jian4qian2-yan3kai1}{4,10,11,4}{⾒,⾦,⽬,⼶}[HSK 7-9]
  \definition{expr.}{os olhos se arregalam de alegria ao ver dinheiro; comover-se ao ver dinheiro; ser tentado pelo dinheiro; não se importar com nada além de dinheiro; arregalar os olhos ao ver dinheiro; ``Mostrar alegria ao ver dinheiro.'', descreve alguém que valoriza o dinheiro em excesso; ganancioso}
\end{EntryWithPhonetic}

\begin{EntryWithPhonetic}{见仁见智}{jian4ren2-jian4zhi4}{4,4,4,12}{⾒,⼈,⾒,⽇}[HSK 7-9]
  \definition{expr.}{opiniões divergentes; O Livro das Mutações, em seus Acréscimos, Parte I, afirma: ``Os benevolentes veem isso como benevolência, os sábios veem isso como sabedoria.'', isso significa que pessoas diferentes têm perspectivas diferentes sobre a mesma questão}
\end{EntryWithPhonetic}

\begin{EntryWithPhonetic}{见识}{jian4shi5}{4,7}{⾒,⾔}[HSK 7-9]
  \definition{s.}{experiência; conhecimento; bom senso; observações}
  \definition{v.}{ampliar o conhecimento; enriquecer a experiência; expor-se a novas experiências e ampliar seus horizontes}
\end{EntryWithPhonetic}

\begin{EntryWithPhonetic}{见外}{jian4wai4}{4,5}{⾒,⼣}[HSK 7-9]
  \definition{v.}{considerar alguém como um estranho; tratar alguém como um estranho; ser excessivamente educado}
\end{EntryWithPhonetic}

\begin{EntryWithPhonetic}{见效}{jian4xiao4}{4,10}{⾒,⽁}[HSK 7-9]
  \definition{v.}{tornar-se eficaz; produzir o resultado desejado; produzir um efeito; ver resultados, surtir efeito}
\end{EntryWithPhonetic}

\begin{EntryWithPhonetic}{见义勇为}{jian4yi4-yong3wei2}{4,3,9,4}{⾒,⼂,⼒,⼂}[HSK 7-9]
  \definition{expr.}{enxergar o que é certo e ter a coragem de fazê-lo; estar pronto para defender uma causa justa com unhas e dentes; faça com ousadia o que é justo; aja com coragem por uma causa justa; ajude um cão manco a passar por cima de uma cerca; nunca hesite quando o bem deve ser feito; nunca hesite em fazer o que é certo; esteja à altura da situação com bravura}
\end{EntryWithPhonetic}

\begin{EntryWithPhonetic}{见证}{jian4zheng4}{4,7}{⾒,⾔}[HSK 7-9]
  \definition{s.}{testemunha; depoimento; testemunhas ou itens que podem servir como prova}
  \definition{v.}{testemunhar; ver algo acontecer; quem presenciou o ocorrido pode confirmar}
\end{EntryWithPhonetic}

%%%%%%%%%% 件 %%%%%%%%%%
\subsection*{件}\addcontentsline{loh}{figure}{件 \dpy{jian4}}

\begin{EntryWithPhonetic}{件}{jian4}{6}{⼈}[HSK 2]
  \definition*{s.}{Sobrenome: Jian}
  \definition{clas.}{item; peça; artigo; usado para coisas individuais}
  \definition{s.}{refere"-se a coisas que podem ser contadas uma a uma | papel; carta; documento; correspondência}
\end{EntryWithPhonetic}

%%%%%%%%%% 间 %%%%%%%%%%
\subsection*{间}\addcontentsline{loh}{figure}{间 \dpy{jian4}}

\begin{EntryWithPhonetic}{间}{jian4}{7}{⾨}
  \definition{s.}{espaço entre as duas partes; abertura; lacuna}
  \definition{v.}{separar | semear a discórdia | desbastar (mudas); podar; remover ou arrancar as mudas em excesso}
  \seeref{jian1}
\end{EntryWithPhonetic}

\begin{EntryWithPhonetic}{间谍}{jian4die2}{7,11}{⾨,⾔}[HSK 7-9]
  \definition[个,名]{s.}{espião; agentes enviados ou recrutados por potências inimigas ou estrangeiras para espionar informações militares, segredos de Estado ou realizar atividades subversivas}
\end{EntryWithPhonetic}

\begin{EntryWithPhonetic}{间断}{jian4duan4}{7,11}{⾨,⽄}[HSK 7-9]
  \definition{s.}{intervalo; salto temporal; disjunção; hiato; interrupção; parada; pulsação; lacuna}
  \definition{v.}{ser desconectado; ser interrompido}
\end{EntryWithPhonetic}

\begin{EntryWithPhonetic}{间隔}{jian4ge2}{7,12}{⾨,⾩}[HSK 7-9]
  \definition{s.}{intervalo; distância das coisas no espaço ou no tempo}
\end{EntryWithPhonetic}

\begin{EntryWithPhonetic}{间或}{jian4huo4}{7,8}{⾨,⼽}
  \definition{adv.}{às vezes | ocasionalmente | de vez em quando}
\end{EntryWithPhonetic}

\begin{EntryWithPhonetic}{间接}{jian4jie1}{7,11}{⾨,⼿}[HSK 5]
  \definition{adj.}{indireto; de segunda mão}
  \antonymref{直接}{zhi2jie1}
\end{EntryWithPhonetic}

\begin{EntryWithPhonetic}{间隙}{jian4xi4}{7,12}{⾨,⾩}[HSK 7-9]
  \definition{s.}{lacuna; espaço; intervalo; tempo ou espaço não utilizado}
\end{EntryWithPhonetic}

%%%%%%%%%% 建 %%%%%%%%%%
\subsection*{建}\addcontentsline{loh}{figure}{建 \dpy{jian4}}

\begin{EntryWithPhonetic}{建}{jian4}{8}{⼵}[HSK 3]
  \definition*{s.}{Província de Fujian | Rio Jian Jiang (na província de Fujian) | Sobrenome: Jian}
  \definition{v.}{construir; construir; erigir | estabelecer; configurar; fundar | propor; defender; apresentar (suas próprias opiniões)}
\end{EntryWithPhonetic}

\begin{EntryWithPhonetic}{建成}{jian4cheng2}{8,6}{⼵,⼽}[HSK 3]
  \definition{v.}{terminar a construção}
\end{EntryWithPhonetic}

\begin{EntryWithPhonetic}{建交}{jian4/jiao1}{8,6}{⼵,⼇}[HSK 7-9]
  \definition{v.+compl.}{estabelecer relações diplomáticas}
\end{EntryWithPhonetic}

\begin{EntryWithPhonetic}{建立}{jian4li4}{8,5}{⼵,⽴}[HSK 3]
  \definition{v.}{estabelecer; construir; começar a construir | vir a ser; começar a surgir; começar a se formar}
\end{EntryWithPhonetic}

\begin{EntryWithPhonetic}{建立者}{jian4li4zhe3}{8,5,8}{⼵,⽴,⽼}
  \definition{s.}{fundador; construtor}
\end{EntryWithPhonetic}

\begin{EntryWithPhonetic}{建设}{jian4she4}{8,6}{⼵,⾔}[HSK 3]
  \definition{s.}{reconstrução; desenvolvimento; trabalhos relacionados com a construção}
  \definition{v.}{construir; edificar; (Estado ou coletividade) criar novos empreendimentos ou aumento de novas instalações}
\end{EntryWithPhonetic}

\begin{EntryWithPhonetic}{建设性}{jian4she4xing4}{8,6,8}{⼵,⾔,⼼}
  \definition{adj.}{construtivo}
  \definition{s.}{construtividade}
\end{EntryWithPhonetic}

\begin{EntryWithPhonetic}{建设者}{jian4she4zhe3}{8,6,8}{⼵,⾔,⽼}
  \definition{s.}{construtor}
\end{EntryWithPhonetic}

\begin{EntryWithPhonetic}{建树}{jian4shu4}{8,9}{⼵,⽊}[HSK 7-9]
  \definition{s.}{realização; conquista}
  \definition{v.}{dar uma contribuição; contribuir | Literário: alcançar uma conquista}
\end{EntryWithPhonetic}

\begin{EntryWithPhonetic}{建议}{jian4yi4}{8,5}{⼵,⾔}[HSK 3]
  \definition[个,点,条]{s.}{proposta; sugestão; recomendação; para que alguém ou alguma coisa evolua para melhor, para o coletivo; pontos de vista e opiniões apresentados pelos líderes, etc.}
  \definition{v.}{propor; sugerir; recomendar; em relação a determinada pessoa ou situação, apresentar seus pontos de vista e opiniões ao coletivo, aos líderes ou a indivíduos, para que as coisas evoluam para melhor}
\end{EntryWithPhonetic}

\begin{EntryWithPhonetic}{建造}{jian4zao4}{8,10}{⼵,⾡}[HSK 5]
  \definition{v.}{construir; edificar}
\end{EntryWithPhonetic}

\begin{EntryWithPhonetic}{建筑}{jian4zhu4}{8,12}{⼵,⽵}[HSK 5]
  \definition[座,幢,排]{s.}{construção; estrutura; edifício; prédio}
  \definition{v.}{construir; erguer; edificar; construir casas, estradas, pontes, etc.}
\end{EntryWithPhonetic}

\begin{EntryWithPhonetic}{建筑师}{jian4zhu4shi1}{8,12,6}{⼵,⽵,⼱}[HSK 7-9]
  \definition{s.}{arquiteto; profissionais de engenharia e técnicos atuantes na indústria da construção}
\end{EntryWithPhonetic}

\begin{EntryWithPhonetic}{建筑物}{jian4zhu4wu4}{8,12,8}{⼵,⽵,⽜}[HSK 7-9]
  \definition[栋]{s.}{edifício; estrutura; projetos de engenharia civil realizados pelo homem, como casas e pontes}
\end{EntryWithPhonetic}

%%%%%%%%%% 剑 %%%%%%%%%%
\subsection*{剑}\addcontentsline{loh}{figure}{剑 \dpy{jian4}}

\begin{EntryWithPhonetic}{剑}{jian4}{9}{⼑}[HSK 6]
  \definition[把,口]{s.}{espada; sabre; florete}
\end{EntryWithPhonetic}

\begin{EntryWithPhonetic}{剑客}{jian4ke4}{9,9}{⼑,⼧}
  \definition{s.}{espada | esgrimista, espadachim}
\end{EntryWithPhonetic}

%%%%%%%%%% 贱 %%%%%%%%%%
\subsection*{贱}\addcontentsline{loh}{figure}{贱 \dpy{jian4}}

\begin{EntryWithPhonetic}{贱}{jian4}{9}{⾙}[HSK 7-9]
  \definition*{s.}{Sobrenome: Jian}
  \definition{adj.}{baixo preço; barato | humilde | baixo; básico; desprezível | humilde; baixa posição social}
  \definition{pron.}{meu (autodepreciativo)}
  \antonymref{贵}{gui4}
\end{EntryWithPhonetic}

%%%%%%%%%% 健 %%%%%%%%%%
\subsection*{健}\addcontentsline{loh}{figure}{健 \dpy{jian4}}

\begin{EntryWithPhonetic}{健}{jian4}{10}{⼈}
  \definition{adj.}{forte; saudável; bem definido | ser forte em; ser bom em; apresentar um grau superior à média em determinado aspecto}
  \definition{v.}{fortalecer; endurecer; revigorar}
\end{EntryWithPhonetic}

\begin{EntryWithPhonetic}{健康}{jian4kang1}{10,11}{⼈,⼴}[HSK 2]
  \definition{adj.}{em forma; saudável; descreve que a pessoa está em ótimo estado físico ou mental, sem nenhum problema | sudável; tudo está normal, sem problemas | saudável; livre de doenças; bom para a saúde}
  \definition{s.}{saúde; físico; estado de saúde}
\end{EntryWithPhonetic}

\begin{EntryWithPhonetic}{健美}{jian4mei3}{10,9}{⼈,⽺}[HSK 7-9]
  \definition{adj.}{forte e bonito; vigoroso e gracioso; robusto e elegante}
  \definition[次]{s.}{fisiculturismo; exercícios que desenvolvem os músculos e o físico}
\end{EntryWithPhonetic}

\begin{EntryWithPhonetic}{健全}{jian4quan2}{10,6}{⼈,⼊}[HSK 5]
  \definition{adj.}{saudável; íntegro; capaz; apto; robusto e sem mácula | sólido; completo; perfeito}
  \definition{v.}{aperfeiçoar; melhorar; fortalecer; reforçar}
\end{EntryWithPhonetic}

\begin{EntryWithPhonetic}{健身}{jian4/shen1}{10,7}{⼈,⾝}[HSK 4]
  \definition{s.}{exercício físico | \emph{fitness}}
  \definition{v.+compl.}{exercitar-se; manter a forma; praticar um esporte, especialmente a ginástica, inclusive em aparelhos, para desenvolver força, flexibilidade, aumentar a resistência, melhorar a coordenação e o controle de todas as partes do corpo}
\end{EntryWithPhonetic}

\begin{EntryWithPhonetic}{健壮}{jian4zhuang4}{10,6}{⼈,⼠}[HSK 7-9]
  \definition{adj.}{robusto; saudável e forte}
\end{EntryWithPhonetic}

%%%%%%%%%% 渐 %%%%%%%%%%
\subsection*{渐}\addcontentsline{loh}{figure}{渐 \dpy{jian4}}

\begin{EntryWithPhonetic}{渐}{jian4}{11}{⽔}
  \definition{adv.}{gradualmente; por graus}
  \seeref{jian1}
\end{EntryWithPhonetic}

\begin{EntryWithPhonetic}{渐渐}{jian4jian4}{11,11}{⽔,⽔}[HSK 4]
  \definition{adv.}{gradualmente; pouco a pouco; passo a passo; indica um aumento ou diminuição gradual em grau ou quantidade}
\end{EntryWithPhonetic}

%%%%%%%%%% 溅 %%%%%%%%%%
\subsection*{溅}\addcontentsline{loh}{figure}{溅 \dpy{jian4}}

\begin{EntryWithPhonetic}{溅}{jian4}{12}{⽔}[HSK 7-9]
  \definition{s.}{respingo; o líquido foi ejetado em todas as direções devido ao impacto}
  \definition{v.}{respingar; salpicar}
\end{EntryWithPhonetic}

%%%%%%%%%% 鉴 %%%%%%%%%%
\subsection*{鉴}\addcontentsline{loh}{figure}{鉴 \dpy{jian4}}

\begin{EntryWithPhonetic}{鉴}{jian4}{13}{⾦}
  \definition*{s.}{Sobrenome: Jian}
  \definition{expr.}{uma expressão idiomática antiga usada para escrever cartas, depois da saudação inicial para pedir que alguém leia a carta}
  \definition{s.}{espelho (feito de bronze ou latão); espelho de bronze antigo | advertência; lição objetiva}
  \definition{v.}{Literário: refletir; espelhar | inspecionar; examinar; escrutinar; olhar cuidadosamente}
\end{EntryWithPhonetic}

\begin{EntryWithPhonetic}{鉴别}{jian4bie2}{13,7}{⾦,⼑}[HSK 7-9]
  \definition{v.}{distinguir; diferenciar; discernir; discriminar; distinguir entre o genuíno e o falso, o bom e o ruim}
\end{EntryWithPhonetic}

\begin{EntryWithPhonetic}{鉴定}{jian4ding4}{13,8}{⾦,⼧}[HSK 6]
  \definition{s.}{avaliação dos pontos fortes e fracos de uma pessoa; avaliação de pessoas ou coisas}
  \definition{v.}{avaliar; identificar; autenticar; determinar; identificar e determinar (a autenticidade e a qualidade das coisas) | conduzir uma avaliação; avaliar o desempenho de uma pessoa ao longo de um determinado período de tempo}
\end{EntryWithPhonetic}

\begin{EntryWithPhonetic}{鉴赏}{jian4shang3}{13,12}{⾦,⾙}[HSK 7-9]
  \definition{v.}{apreciar; avaliar e apreciar (obras de arte, relíquias culturais, etc.)}
\end{EntryWithPhonetic}

\begin{EntryWithPhonetic}{鉴于}{jian4yu2}{13,3}{⾦,⼆}[HSK 7-9]
  \definition{conj.}{considerando; com base em; tendo em vista; vendo que; usado no início da primeira oração de uma frase causal complexa para indicar causa ou fundamento}[鉴于群众反映,服务改善了。===Considerando o feedback do público, o serviço foi aprimorado.]
  \definition{prep.}{tendo em vista; à luz de; apresente uma situação observada ou considerada como base para a ação}
\end{EntryWithPhonetic}

%%%%%%%%%% 键 %%%%%%%%%%
\subsection*{键}\addcontentsline{loh}{figure}{键 \dpy{jian4}}

\begin{EntryWithPhonetic}{键}{jian4}{13}{⾦}[HSK 5]
  \definition[个]{s.}{chave | tecla (de uma máquina de escrever, piano, etc.) | Química: ligação | Literário: ferrolho (de uma porta) | pino (para máquinas)  | etapa crucial}
\end{EntryWithPhonetic}

\begin{EntryWithPhonetic}{键盘}{jian4pan2}{13,11}{⾦,⽫}[HSK 5]
  \definition[台,个]{s.}{teclado; cravo; painel de teclas}
\end{EntryWithPhonetic}

%%%%%%%%%% 箭 %%%%%%%%%%
\subsection*{箭}\addcontentsline{loh}{figure}{箭 \dpy{jian4}}

\begin{EntryWithPhonetic}{箭}{jian4}{15}{⽵}[HSK 6]
  \definition[支]{s.}{seta | distância percorrida por uma flecha}
\end{EntryWithPhonetic}

%%%%%%%%%% 江 %%%%%%%%%%
\subsection*{江}\addcontentsline{loh}{figure}{江 \dpy{jiang1}}

\begin{EntryWithPhonetic}{江}{jiang1}{6}{⽔}[HSK 4]
  \definition*{s.}{Rio Changjiang | Sobrenome: Jiang}
  \definition[条,道]{s.}{rio grande}
\end{EntryWithPhonetic}

\begin{EntryWithPhonetic}{江南水乡}{jiang1nan2shui3xiang1}{6,9,4,3}{⽔,⼗,⽔,⼄}
  \definition*{s.}{Vila Aquática de Jiangnan | Cidades Aquáticas}
\end{EntryWithPhonetic}

\begin{EntryWithPhonetic}{江水}{jiang1shui3}{6,4}{⽔,⽔}
  \definition{s.}{água do rio}
\end{EntryWithPhonetic}

\begin{EntryWithPhonetic}{江苏}{jiang1su1}{6,7}{⽔,⾋}
  \definition*{s.}{Província de Jiangsu}
\end{EntryWithPhonetic}

\begin{EntryWithPhonetic}{江西}{jiang1xi1}{6,6}{⽔,⾑}
  \definition*{s.}{Jiangxi}
\end{EntryWithPhonetic}

%%%%%%%%%% 姜 %%%%%%%%%%
\subsection*{姜}\addcontentsline{loh}{figure}{姜 \dpy{jiang1}}

\begin{EntryWithPhonetic}{姜}{jiang1}{9}{⼥}[HSK 7-9]
  \definition*{s.}{Sobrenome: Jiang}
  \definition[磅,斤,两]{s.}{gengibre; rizoma de gengibre}
\end{EntryWithPhonetic}

%%%%%%%%%% 将 %%%%%%%%%%
\subsection*{将}\addcontentsline{loh}{figure}{将 \dpy{jiang1}}

\begin{EntryWithPhonetic}{将}{jiang1}{9}{⼨}[HSK 5]
  \definition*{s.}{Sobrenome: Jiang}
  \definition{adv.}{estar indo para; parcialmente\dots parcialmente\dots}
  \definition{part.}{expressar uma direção, como 进来, 出去; usado no meio de verbos e complementos que indicam tendência, como 进来, 出去, etc.}
  \definition{prep.}{com; por meio de; por | usado da mesma forma que 把}
  \definition{v.}{fazer algo; lidar com (um assunto) | dar um cheque-mate | cuidar (da saúde) | incitar alguém a agir; desafiar; estimular | segurar; pegar | colocar; tirar | levar; trazer | dar suporte; dar apoio}
  \seeref{jiang4}
  \seeref{qiang1}
  \seealsoref{把}{ba3}
  \seealsoref{出去}{chu1 qu5}
  \seealsoref{进来}{jin4 lai5}
\end{EntryWithPhonetic}

\begin{EntryWithPhonetic}{将近}{jiang1jin4}{9,7}{⼨,⾡}[HSK 3]
  \definition{adv.}{quase}
\end{EntryWithPhonetic}

\begin{EntryWithPhonetic}{将军}{jiang1/jun1}{9,6}{⼨,⼍}[HSK 6]
  \definition[位,名]{s.}{general; geralmente se refere a generais seniores}
  \definition{v.+compl.}{dar xeque-mate; atacar o general ou rei do oponente no xadrez; colocar alguém em grandes apuros; metáfora para dar a alguém um problema difícil ou dificultar a tarefa para essa pessoa}
\end{EntryWithPhonetic}

\begin{EntryWithPhonetic}{将来}{jiang1lai2}{9,7}{⼨,⽊}[HSK 3]
  \definition[个]{s.}{no futuro (geralmente se refere a um período mais longo)}
\end{EntryWithPhonetic}

\begin{EntryWithPhonetic}{将要}{jiang1yao4}{9,9}{⼨,⾑}[HSK 5]
  \definition{adv.}{irá; deverá; estará prestes a; irá a; indica que um ato ou situação ocorre logo em seguida}
\end{EntryWithPhonetic}

%%%%%%%%%% 僵 %%%%%%%%%%
\subsection*{僵}\addcontentsline{loh}{figure}{僵 \dpy{jiang1}}

\begin{EntryWithPhonetic}{僵}{jiang1}{15}{⼈}[HSK 7-9]
  \definition{adj.}{rígido; paralisado; congelado | rígido; austero; duro | Dialeto: em impasse; tenso}
  \definition{v.}{Dialeto: parar de sorrir; ficar com uma expressão séria}
\end{EntryWithPhonetic}

\begin{EntryWithPhonetic}{僵化}{jiang1hua4}{15,4}{⼈,⼔}[HSK 7-9]
  \definition{v.}{tornar-se rígido; ossificar; tornar-se estereotipado; parar de se desenvolver; petrificar; inativar}
\end{EntryWithPhonetic}

\begin{EntryWithPhonetic}{僵局}{jiang1ju2}{15,7}{⼈,⼫}[HSK 7-9]
  \definition[个,种]{s.}{impasse; beco sem saída; a questão é difícil de resolver e o progresso está paralisado}
\end{EntryWithPhonetic}

%%%%%%%%%% 讲 %%%%%%%%%%
\subsection*{讲}\addcontentsline{loh}{figure}{讲 \dpy{jiang3}}

\begin{EntryWithPhonetic}{讲}{jiang3}{6}{⾔}[HSK 2]
  \definition[种]{s.}{palestra; discurso}
  \definition{v.}{contar; falar | explicar; transmitir oralmente; esclarecer | negociar; barganhar | ser exigente com; valorizar; dar importância}
\end{EntryWithPhonetic}

\begin{EntryWithPhonetic}{讲话}{jiang3hua4}{6,8}{⾔,⾔}[HSK 2]
  \definition[个]{s.}{discurso; palestra | guia; introdução}
  \definition{v.}{falar; conversar; dirigir-se a alguém | criticar}
\end{EntryWithPhonetic}

\begin{EntryWithPhonetic}{讲解}{jiang3jie3}{6,13}{⾔,⾓}[HSK 7-9]
  \definition{v.}{explicar; interpretar; expor}
\end{EntryWithPhonetic}

\begin{EntryWithPhonetic}{讲究}{jiang3jiu5}{6,7}{⾔,⽳}[HSK 4]
  \definition{adj.}{requintado; elegante; de bom gosto; exigente com a vida e com outros aspectos, buscando alto nível, qualidade e detalhes}
  \definition{s.}{estudo cuidadoso; algo que merece atenção; elementos e aspectos que merecem atenção especial}
  \definition{v.}{dar ênfase a; ser específico sobre; prestar atenção a}
\end{EntryWithPhonetic}

\begin{EntryWithPhonetic}{讲课}{jiang3 ke4}{6,10}{⾔,⾔}[HSK 6]
  \definition{v.}{ensinar; dar palestras; proferir uma palestra | dar uma lição (palestra)}
\end{EntryWithPhonetic}

\begin{EntryWithPhonetic}{讲述}{jiang3shu4}{6,8}{⾔,⾡}[HSK 7-9]
  \definition{v.}{narrar; relatar; contar sobre; dar um relato de}
\end{EntryWithPhonetic}

\begin{EntryWithPhonetic}{讲学}{jiang3/xue2}{6,8}{⾔,⼦}[HSK 7-9]
  \definition{v.+compl.}{ministrar palestras; discorrer sobre um assunto acadêmico}
\end{EntryWithPhonetic}

\begin{EntryWithPhonetic}{讲座}{jiang3zuo4}{6,10}{⾔,⼴}[HSK 4]
  \definition[场,次]{s.}{palestra; um curso de palestras; a forma de instrução usada para ensinar um determinado assunto ou tópico, geralmente por meio de palestras ao vivo, seriados de rádio ou televisão ou seriados de jornal.}
\end{EntryWithPhonetic}

%%%%%%%%%% 奖 %%%%%%%%%%
\subsection*{奖}\addcontentsline{loh}{figure}{奖 \dpy{jiang3}}

\begin{EntryWithPhonetic}{奖}{jiang3}{9}{⼤}[HSK 4]
  \definition[个,次]{s.}{prêmio; recompensa | elogio; loa}
  \definition{v.}{elogiar; recompensar; recomendar; incentivar}
\end{EntryWithPhonetic}

\begin{EntryWithPhonetic}{奖杯}{jiang3bei1}{9,8}{⼤,⽊}[HSK 7-9]
  \definition[个,座]{s.}{taça (como prêmio); copa; troféu; os prêmios em formato de taça, concedidos aos vencedores em competições esportivas, geralmente são feitos de ouro ou prata}
\end{EntryWithPhonetic}

\begin{EntryWithPhonetic}{奖金}{jiang3jin1}{9,8}{⼤,⾦}[HSK 4]
  \definition[个,笔]{s.}{bônus; recompensa; prêmio; prêmio em dinheiro; dinheiro de recompensa, dinheiro dado às pessoas para incentivá-las ou elogiá-las por terem se saído bem em alguma coisa}
\end{EntryWithPhonetic}

\begin{EntryWithPhonetic}{奖励}{jiang3li4}{9,7}{⼤,⼒}[HSK 5]
  \definition{s.}{prêmio; recompensa; dinheiro ou honras dadas em troca de elogios ou incentivos}
  \definition{v.}{recompensar; incentivar; encorajar}
\end{EntryWithPhonetic}

\begin{EntryWithPhonetic}{奖牌}{jiang3pai2}{9,12}{⼤,⽚}[HSK 7-9]
  \definition{s.}{medalha (concedida como prêmio); as medalhas são divididas em ouro, prata e bronze; os vencedores em competições esportivas recebem prêmios de acordo com esses três níveis}
\end{EntryWithPhonetic}

\begin{EntryWithPhonetic}{奖品}{jiang3pin3}{9,9}{⼤,⼝}[HSK 7-9]
  \definition[个,些,份]{s.}{prêmio; itens para recompensa}
\end{EntryWithPhonetic}

\begin{EntryWithPhonetic}{奖项}{jiang3xiang4}{9,9}{⼤,⾴}[HSK 7-9]
  \definition[项]{s.}{prêmio; projetos premiados}
\end{EntryWithPhonetic}

\begin{EntryWithPhonetic}{奖学金}{jiang3xue2jin1}{9,8,8}{⼤,⼦,⾦}[HSK 4]
  \definition[笔]{s.}{bolsa de estudos; exposição; prêmios concedidos por escolas, organizações ou indivíduos a alunos com bom desempenho acadêmico}
\end{EntryWithPhonetic}

%%%%%%%%%% 匠 %%%%%%%%%%
\subsection*{匠}\addcontentsline{loh}{figure}{匠 \dpy{jiang4}}

\begin{EntryWithPhonetic}{匠}{jiang4}{6}{⼕}
  \definition*{s.}{Sobrenome: Jiang}
  \definition{s.}{artesão | pessoa de realizações notáveis em um campo específico; mestre}
\end{EntryWithPhonetic}

%%%%%%%%%% 降 %%%%%%%%%%
\subsection*{降}\addcontentsline{loh}{figure}{降 \dpy{jiang4}}

\begin{EntryWithPhonetic}{降}{jiang4}{8}{⾩}[HSK 4]
  \definition*{s.}{Sobrenome: Jiang}
  \definition{v.}{cair; descer; quedar-se | diminuir; reduzir; cair; abaixar | nascer}
  \antonymref{升}{sheng1}
\end{EntryWithPhonetic}

\begin{EntryWithPhonetic}{降低}{jiang4di1}{8,7}{⾩,⼈}[HSK 4]
  \definition{v.}{reduzir; cortar; diminuir; rebaixar; cair; abaixar}
\end{EntryWithPhonetic}

\begin{EntryWithPhonetic}{降价}{jiang4 jia4}{8,6}{⾩,⼈}[HSK 4]
  \definition{v.}{ficar mais barato; cortar o preço; reduzir o preço}
\end{EntryWithPhonetic}

\begin{EntryWithPhonetic}{降临}{jiang4lin2}{8,9}{⾩,⼁}[HSK 7-9]
  \definition{v.}{acontecer; chegar; vir}[春天降临,万物复苏。===A primavera chega e tudo renasce.]
\end{EntryWithPhonetic}

\begin{EntryWithPhonetic}{降落}{jiang4luo4}{8,12}{⾩,⾋}[HSK 4]
  \definition{v.}{aterrissar; descer; descer do céu}
\end{EntryWithPhonetic}

\begin{EntryWithPhonetic}{降温}{jiang4 wen1}{8,12}{⾩,⽔}[HSK 4]
  \definition{v.}{baixar a temperatura (como em uma oficina);  recusar | cair a temperatura | esfriar; resfriar; metáfora para um declínio no entusiasmo ou uma diminuição no ímpeto de algo}
\end{EntryWithPhonetic}

%%%%%%%%%% 将 %%%%%%%%%%
\subsection*{将}\addcontentsline{loh}{figure}{将 \dpy{jiang4}}

\begin{EntryWithPhonetic}{将}{jiang4}{9}{⼨}
  \definition{s.}{general; nome do posto; abaixo de marechal de campo; acima de coronel}
  \definition{v.}{comandar; liderar}
  \seeref{jiang1}
  \seeref{qiang1}
\end{EntryWithPhonetic}

%%%%%%%%%% 强 %%%%%%%%%%
\subsection*{强}\addcontentsline{loh}{figure}{强 \dpy{jiang4}}

\begin{EntryWithPhonetic}{强}{jiang4}{12}{⼸}
  \definition{adj.}{teimoso; inflexível}
  \seeref{qiang2}
  \seeref{qiang3}
\end{EntryWithPhonetic}

%%%%%%%%%% 酱 %%%%%%%%%%
\subsection*{酱}\addcontentsline{loh}{figure}{酱 \dpy{jiang4}}

\begin{EntryWithPhonetic}{酱}{jiang4}{13}{⾣}[HSK 6]
  \definition{adj.}{marinado em molho de soja; cozido em molho de soja}
  \definition{s.}{molho espesso feito de soja, farinha, etc. | molho; pasta; geleia | um condimento pastoso feito de feijão, trigo fermentados e sal}
  \definition{v.}{cozinhar ou conservar em molho de soja}
\end{EntryWithPhonetic}

\begin{EntryWithPhonetic}{酱油}{jiang4you2}{13,8}{⾣,⽔}[HSK 6]
  \definition[袋,瓶,壶,桶]{s.}{molho de soja}
\end{EntryWithPhonetic}

%%%%%%%%%% 犟 %%%%%%%%%%
\subsection*{犟}\addcontentsline{loh}{figure}{犟 \dpy{jiang4}}

\begin{EntryWithPhonetic}{犟}{jiang4}{16}{⽜}
  \variantof{强}
\end{EntryWithPhonetic}

%%%%%%%%%% 交 %%%%%%%%%%
\subsection*{交}\addcontentsline{loh}{figure}{交 \dpy{jiao1}}

\begin{EntryWithPhonetic}{交}{jiao1}{6}{⼇}[HSK 2]
  \definition*{s.}{Sobrenome: Jiao}
  \definition{adv.}{mutuamente; recíprocamente; um ao outro | juntos; simultaneamente}
  \definition{s.}{amigo; conhecido; amizade; relacionamento | transação comercial; negócio; barganha | queda}
  \definition{v.}{entregar | (de lugares ou períodos de tempo) cruzar; encontrar; unir | chegar (a uma determinada hora ou estação); estabelecer-se; vir | cruzar; intersectar | associar-se a | ter relações sexuais | acasalar; reproduzir-se | transferir as coisas para as partes interessadas | unir (lugares ou períodos de tempo)}
  \antonymref{接}{jie1}
  \antonymref{收}{shou1}
\end{EntryWithPhonetic}

\begin{EntryWithPhonetic}{交班}{jiao1ban1}{6,10}{⼇,⽟}
  \definition{v.}{passar para o próximo turno de trabalho}
  \synonymref{接班}{jie1/ban1}
\end{EntryWithPhonetic}

\begin{EntryWithPhonetic}{交杯酒}{jiao1bei1jiu3}{6,8,10}{⼇,⽊,⾣}
  \definition{s.}{copo de vinho nupcial}
\end{EntryWithPhonetic}

\begin{EntryWithPhonetic}{交叉}{jiao1cha1}{6,3}{⼇,⼜}[HSK 7-9]
  \definition{v.}{cruzar; entrecruzar; interseccionar | coincidir | revezar-se}
\end{EntryWithPhonetic}

\begin{EntryWithPhonetic}{交叉点}{jiao1cha1dian3}{6,3,9}{⼇,⼜,⽕}
  \definition{s.}{interseção; cruzamento; encruzilhada; ponto de interseção; junção}
\end{EntryWithPhonetic}

\begin{EntryWithPhonetic}{交叉口}{jiao1cha1kou3}{6,3,3}{⼇,⼜,⼝}
  \definition{s.}{intersecção (rodovia)}
\end{EntryWithPhonetic}

\begin{EntryWithPhonetic}{交代}{jiao1dai4}{6,5}{⼇,⼈}[HSK 5]
  \definition{v.}{contar; entregar | ordenar; insistir; contar aos outros sobre suas intenções, instruções | contar; admitir}
  \synonymref{打发}{da3fa5}
  \synonymref{叮嘱}{ding1zhu3}
  \synonymref{吩咐}{fen1fu4}
  \synonymref{交接}{jiao1jie1}
  \synonymref{派遣}{pai4qian3}
  \synonymref{嘱托}{zhu3tuo1}
  \synonymref{嘱咐}{zhu3fu5}
  \antonymref{抗拒}{kang4ju4}
\end{EntryWithPhonetic}

\begin{EntryWithPhonetic}{交叠}{jiao1die2}{6,13}{⼇,⼜}
  \definition{s.}{sobreposição}
  \synonymref{重叠}{chong2die2}
  \synonymref{交集}{jiao1ji2}
\end{EntryWithPhonetic}

\begin{EntryWithPhonetic}{交费}{jiao1fei4}{6,9}{⼇,⾙}[HSK 3]
  \definition{v.}{pagar taxas ou impostos; pagar uma taxa ou imposto}
\end{EntryWithPhonetic}

\begin{EntryWithPhonetic}{交锋}{jiao1/feng1}{6,12}{⼇,⾦}[HSK 7-9]
  \definition{v.+compl.}{entrar em conflito; cruzar espadas; confrontar; participar de uma batalha ou disputa; ter um confronto (com alguém)}
  \synonymref{打仗}{da3/zhang4}
  \synonymref{接触}{jie1chu4}
  \synonymref{战争}{zhan4zheng1}
  \antonymref{逃避}{tao2bi4}
\end{EntryWithPhonetic}

\begin{EntryWithPhonetic}{交付}{jiao1fu4}{6,5}{⼇,⼈}[HSK 7-9]
  \definition{v.}{entregar; passar a responsabilidade; transferir a responsabilidade; pagar a}
\end{EntryWithPhonetic}

\begin{EntryWithPhonetic}{交给}{jiao1gei3}{6,9}{⼇,⽷}[HSK 2]
  \definition{v.}{entregar para | dar para}
\end{EntryWithPhonetic}

\begin{EntryWithPhonetic}{交媾}{jiao1gou4}{6,13}{⼇,⼥}
  \definition{v.}{copular | ter relações sexuais}
\end{EntryWithPhonetic}

\begin{EntryWithPhonetic}{交换}{jiao1huan4}{6,10}{⼇,⼿}[HSK 4]
  \definition{v.}{trocar; permutar; comutar; intercambiar}
  \synonymref{撤换}{che4huan4}
  \synonymref{兑换}{dui4huan4}
  \synonymref{交流}{jiao1liu2}
  \synonymref{替换}{ti4huan4}
  \antonymref{换取}{huan4qu3}
\end{EntryWithPhonetic}

\begin{EntryWithPhonetic}{交集}{jiao1ji2}{6,12}{⼇,⾫}[HSK 7-9]
  \definition{s.}{sobreposição; conexão; terreno comum; pontos em comum; intersecção; convergência}
  \definition{v.}{(diferentes sentimentos) estar misturado; ocorrer simultaneamente}
  \synonymref{暴躁}{bao4zao4}
  \synonymref{烦躁}{fan2zao4}
  \synonymref{慌张}{huang1zhang1}
  \synonymref{焦虑}{jiao1lv4}
  \synonymref{焦躁}{jiao1zao4}
  \synonymref{恐慌}{kong3huang1}
  \synonymref{着急}{zhao2/ji2}
  \antonymref{分散}{fen1san4}
\end{EntryWithPhonetic}

\begin{EntryWithPhonetic}{交际}{jiao1ji4}{6,7}{⼇,⾩}[HSK 4]
  \definition{s.}{contato; comunicação; relações sociais; contato interpessoal, socialização}
  \synonymref{社交}{she4jiao1}
  \synonymref{外交}{wai4jiao1}
\end{EntryWithPhonetic}

\begin{EntryWithPhonetic}{交接}{jiao1jie1}{6,11}{⼇,⼿}[HSK 7-9]
  \definition{v.}{juntar-se; conectar-se; conectar | entregar e assumir o controle; transferir | associar-se a; fazer amizade com; fazer amigos}
  \synonymref{交代}{jiao1dai4}
\end{EntryWithPhonetic}

\begin{EntryWithPhonetic}{交界}{jiao1jie4}{6,9}{⼇,⽥}[HSK 7-9]
  \definition{v.}{ter uma fronteira comum; ter um limite comum; fazer fronteira com}
\end{EntryWithPhonetic}

\begin{EntryWithPhonetic}{交警}{jiao1jing3}{6,19}{⼇,⾔}[HSK 3]
  \definition{s.}{policial de trânsito, abreviação de 交通警察}
  \seealsoref{交通警察}{jiao1tong1 jing3cha2}
\end{EntryWithPhonetic}

\begin{EntryWithPhonetic}{交流}{jiao1liu2}{6,10}{⼇,⽔}[HSK 3]
  \definition{v.}{trocar; interagir; comunicar-se; compartilhar o que cada um tem com o outro}
  \synonymref{对话}{dui4hua4}
  \synonymref{沟通}{gou1tong1}
  \synonymref{互动}{hu4dong4}
  \synonymref{换取}{huan4qu3}
  \synonymref{交换}{jiao1huan4}
  \antonymref{封闭}{feng1bi4}
\end{EntryWithPhonetic}

\begin{EntryWithPhonetic}{交纳}{jiao1na4}{6,7}{⼇,⽷}[HSK 7-9]
  \definition{v.}{pagar (ao estado ou a uma organização); entregar uma quantia predeterminada de dinheiro ou bens a um governo ou órgão público}
  \synonymref{缴纳}{jiao3na4}
\end{EntryWithPhonetic}

\begin{EntryWithPhonetic}{交朋友}{jiao1 peng2you3}{6,8,4}{⼇,⽉,⼜}[HSK 2]
  \definition{v.}{fazer amizade com alguém; fazer amigos}
  \synonymref{打交道}{da3 jiao1dao5}
\end{EntryWithPhonetic}

\begin{EntryWithPhonetic}{交情}{jiao1qing5}{6,11}{⼇,⼼}[HSK 7-9]
  \definition{s.}{amizade; relação amigável}
  \synonymref{情谊}{qing2yi4}
  \synonymref{友谊}{you3yi4}
\end{EntryWithPhonetic}

\begin{EntryWithPhonetic}{交涉}{jiao1she4}{6,10}{⼇,⽔}[HSK 7-9]
  \definition{v.}{negociar; entrar em contato com; fazer representações; discutir soluções para questões relacionadas com a outra parte}
  \synonymref{对话}{dui4hua4}
  \synonymref{谈判}{tan2pan4}
  \synonymref{协商}{xie2shang1}
\end{EntryWithPhonetic}

\begin{EntryWithPhonetic}{交谈}{jiao1tan2}{6,10}{⼇,⾔}[HSK 7-9]
  \definition{v.}{conversar; bater papo; falar um com o outro}
\end{EntryWithPhonetic}

\begin{EntryWithPhonetic}{交替}{jiao1ti4}{6,12}{⼇,⽈}[HSK 7-9]
  \definition{v.}{dar lugar a; substituir; suplantar; substituir coisas antigas por coisas novas | alternar; revezar-se}
  \synonymref{轮流}{lun2liu2}
  \antonymref{维持}{wei2chi2}
\end{EntryWithPhonetic}

\begin{EntryWithPhonetic}{交通}{jiao1tong1}{6,10}{⼇,⾡}[HSK 2]
  \definition{s.}{tráfego | ligação; conexão | transporte; termo genérico para todos os tipos de transporte, como ferroviário e rodoviário}
  \definition{v.}{conspirar; fazer amizades; conchavar | estar conectado; estar ligado; estar vinculado | associar-se a; conspirar com}
  \synonymref{通行}{tong1xing2}
  \synonymref{运输}{yun4shu1}
  \antonymref{堵塞}{du3se4}
\end{EntryWithPhonetic}

\begin{EntryWithPhonetic}{交通警察}{jiao1tong1 jing3cha2}{6,10,19,14}{⼇,⾡,⾔,⼧}
  \definition{s.}{policial de trânsito}
  \seealsoref{交警}{jiao1jing3}
\end{EntryWithPhonetic}

\begin{EntryWithPhonetic}{交头接耳}{jiao1tou2-jie1'er3}{6,5,11,6}{⼇,⼤,⼿,⽿}[HSK 7-9]
  \definition{expr.}{falar ao ouvido um do outro; sussurrar um para o outro; trocar sussurros confidenciais; sussurrar um ao ouvido do outro; cochichar um com o outro}
\end{EntryWithPhonetic}

\begin{EntryWithPhonetic}{交往}{jiao1wang3}{6,8}{⼇,⼻}[HSK 3]
  \definition{v.}{estar em contato com; associar-se a; interagir}
  \synonymref{交易}{jiao1yi4}
  \synonymref{来往}{lai2wang3}
  \synonymref{往来}{wang3lai2}
  \synonymref{相处}{xiang1chu3}
  \antonymref{断交}{duan4/jiao1}
\end{EntryWithPhonetic}

\begin{EntryWithPhonetic}{交响}{jiao1xiang3}{6,9}{⼇,⼝}
  \definition{s.}{sinfonia}
  \synonymref{共鸣}{gong4ming2}
\end{EntryWithPhonetic}

\begin{EntryWithPhonetic}{交响乐}{jiao1xiang3yue4}{6,9,5}{⼇,⼝,⼃}[HSK 7-9]
  \definition{s.}{sinfonia; música sinfônica; as peças musicais de grande escala executadas por uma orquestra normalmente consistem em quatro movimentos e são capazes de expressar pensamentos e sentimentos diversos e complexos}
\end{EntryWithPhonetic}

\begin{EntryWithPhonetic}{交易}{jiao1yi4}{6,8}{⼇,⽇}[HSK 3]
  \definition[笔,桩,个,场]{s.}{negócio; comércio; transação comercial; transação; atividades de compra e venda de mercadorias}
  \definition{v.}{negociar; comprar e vender mercadorias}
  \synonymref{成交}{cheng2/jiao1}
  \synonymref{交往}{jiao1wang3}
  \synonymref{来往}{lai2wang3}
  \synonymref{买卖}{mai3mai4}
  \synonymref{买卖}{mai3mai5}
  \synonymref{贸易}{mao4yi4}
  \synonymref{生意}{sheng1yi5}
  \synonymref{业务}{ye4wu4}
  \synonymref{营业}{ying2ye4}
  \antonymref{赠送}{zeng4song4}
\end{EntryWithPhonetic}

\begin{EntryWithPhonetic}{交运}{jiao1yun4}{6,7}{⼇,⾡}
  \definition{v.}{despachar (bagagem em um aeroporto, etc.) | entregar para transporte}
\end{EntryWithPhonetic}

%%%%%%%%%% 郊 %%%%%%%%%%
\subsection*{郊}\addcontentsline{loh}{figure}{郊 \dpy{jiao1}}

\begin{EntryWithPhonetic}{郊}{jiao1}{8}{⾢}
  \definition*{s.}{Sobrenome: Jiao}
  \definition{s.}{subúrbios; periferias; áreas ao redor da cidade}
\end{EntryWithPhonetic}

\begin{EntryWithPhonetic}{郊区}{jiao1qu1}{8,4}{⾢,⼖}[HSK 5]
  \definition[个,片,块]{s.}{subúrbios; arredores; periferia; área ao redor da cidade que está administrativamente sob a jurisdição da cidade}
\end{EntryWithPhonetic}

\begin{EntryWithPhonetic}{郊外}{jiao1wai4}{8,5}{⾢,⼣}[HSK 7-9]
  \definition{s.}{subúrbio; periferia; a zona rural ao redor de uma cidade; a área fora da cidade (referindo"-se a uma cidade específica)}
\end{EntryWithPhonetic}

\begin{EntryWithPhonetic}{郊游}{jiao1you2}{8,12}{⾢,⽔}[HSK 7-9]
  \definition{s.}{passeio; excursão}
\end{EntryWithPhonetic}

%%%%%%%%%% 娇 %%%%%%%%%%
\subsection*{娇}\addcontentsline{loh}{figure}{娇 \dpy{jiao1}}

\begin{EntryWithPhonetic}{娇}{jiao1}{9}{⼥}
  \definition{adj.}{terno; adorável; encantador | frágil; delicado | melindroso; exigente}
  \definition{v.}{mimar; estragar}
\end{EntryWithPhonetic}

\begin{EntryWithPhonetic}{娇惯}{jiao1guan4}{9,11}{⼥,⼼}[HSK 7-9]
  \definition{v.}{mimar; paparicar; estragar}
\end{EntryWithPhonetic}

\begin{EntryWithPhonetic}{娇气}{jiao1qi4}{9,4}{⼥,⽓}[HSK 7-9]
  \definition{adj.}{exigente; melindroso; com personalidade frágil, incapaz de suportar dificuldades ou injustiças | terno; frágil; delicado; (os itens) são facilmente danificados; (as plantas) não são fáceis de cultivar}
  \definition{s.}{delicadeza; fragilidade; personalidade e estilo frágeis}
\end{EntryWithPhonetic}

%%%%%%%%%% 浇 %%%%%%%%%%
\subsection*{浇}\addcontentsline{loh}{figure}{浇 \dpy{jiao1}}

\begin{EntryWithPhonetic}{浇}{jiao1}{9}{⽔}[HSK 7-9]
  \definition{adj.}{decadente; precipitado e pérfido}
  \definition{v.}{derramar líquido sobre; borrifar água sobre | aguar; regar; irrigar | injetar fluido no molde}
\end{EntryWithPhonetic}

%%%%%%%%%% 骄 %%%%%%%%%%
\subsection*{骄}\addcontentsline{loh}{figure}{骄 \dpy{jiao1}}

\begin{EntryWithPhonetic}{骄}{jiao1}{9}{⾺}
  \definition{adj.}{orgulhoso; arrogante; vaidoso | Literário: feroz; intenso; forte; violento}
\end{EntryWithPhonetic}

\begin{EntryWithPhonetic}{骄傲}{jiao1'ao4}{9,12}{⾺,⼈}[HSK 6]
  \definition{adj.}{arrogante; vaidoso; orgulhoso}
  \definition{s.}{orgulho; pessoas ou coisas das quais se orgulhar}
\end{EntryWithPhonetic}

%%%%%%%%%% 胶 %%%%%%%%%%
\subsection*{胶}\addcontentsline{loh}{figure}{胶 \dpy{jiao1}}

\begin{EntryWithPhonetic}{胶}{jiao1}{10}{⾁}
  \definition*{s.}{Sobrenome: Jiao}
  \definition{adj.}{pegajoso; viscoso; grudento}
  \definition{s.}{cola; goma; adesivo | borracha | gel; colóide}
  \definition{v.}{colar com cola | colar; grudar}
\end{EntryWithPhonetic}

\begin{EntryWithPhonetic}{胶带}{jiao1dai4}{10,9}{⾁,⼱}[HSK 5]
  \definition[卷,条,段]{s.}{fita de embalagem transparente; fita adesiva | fita magnética de plástico; fita de gravação | fita emborrachada; cinta de borracha}
\end{EntryWithPhonetic}

\begin{EntryWithPhonetic}{胶卷}{jiao1juan3}{10,8}{⾁,⼙}
  \definition{s.}{filme | rolo de filme}
\end{EntryWithPhonetic}

\begin{EntryWithPhonetic}{胶囊}{jiao1nang2}{10,22}{⾁,⾐}[HSK 7-9]
  \definition{s.}{Medicina: cápsula; refere"-se a uma cápsula de gelatina usada para encapsular medicamentos em pó ou granulados, facilitando a ingestão}
\end{EntryWithPhonetic}

\begin{EntryWithPhonetic}{胶片}{jiao1pian4}{10,4}{⾁,⽚}[HSK 7-9]
  \definition[卷]{s.}{filme; cartucho; filme fotográfico}
\end{EntryWithPhonetic}

\begin{EntryWithPhonetic}{胶水}{jiao1shui3}{10,4}{⾁,⽔}[HSK 5]
  \definition[瓶]{s.}{cola; mucilagem; cola líquida}
\end{EntryWithPhonetic}

%%%%%%%%%% 教 %%%%%%%%%%
\subsection*{教}\addcontentsline{loh}{figure}{教 \dpy{jiao1}}

\begin{EntryWithPhonetic}{教}{jiao1}{11}{⽁}
  \definition*{s.}{Sobrenome: Jiao}
  \definition{prep.}{em uma frase passiva para introduzir o executor da ação}
  \definition{s.}{religião | professor; referência à educação ou aos professores}
  \definition{v.}{ensinar; instruir |  pedir; ordenar; dizer | permitir; possibilitar}
  \seeref{jiao4}
  \synonymref{授}{shou4}
  \antonymref{学}{xue2}
\end{EntryWithPhonetic}

\begin{EntryWithPhonetic}{教会}{jiao1hui4}{11,6}{⽁,⼈}
  \definition{v.}{mostrar | ensinar}
  \seeref{jiao4hui4}
\end{EntryWithPhonetic}

%%%%%%%%%% 焦 %%%%%%%%%%
\subsection*{焦}\addcontentsline{loh}{figure}{焦 \dpy{jiao1}}

\begin{EntryWithPhonetic}{焦}{jiao1}{12}{⽕}[HSK 7-9]
  \definition*{s.}{Sobrenome: Jiao}
  \definition{adj.}{queimado; chamuscado; carbonizado | preocupado; ansioso}
  \definition{clas.}{J; Joule, abreviação}
  \definition{pref.}{(química) piro-}
  \definition{s.}{Metalurgia: coque}
\end{EntryWithPhonetic}

\begin{EntryWithPhonetic}{焦点}{jiao1dian3}{12,9}{⽕,⽕}[HSK 6]
  \definition{s.}{foco; ponto focal; Matemática: refere"-se a um ponto que tem uma relação especial com uma elipse, hipérbole, parábola, etc. | foco; ponto focal; Óptica: refere"-se à intersecção de feixes de luz paralelos após serem refratados por uma lente ou refletidos por um espelho curvo | foco; questão central; metaforicamente, uma coisa ou princípio que chama a atenção para o foco}
\end{EntryWithPhonetic}

\begin{EntryWithPhonetic}{焦急}{jiao1ji2}{12,9}{⽕,⼼}[HSK 7-9]
  \definition{adj.}{ansioso; preocupado; com pressa}
\end{EntryWithPhonetic}

\begin{EntryWithPhonetic}{焦距}{jiao1ju4}{12,11}{⽕,⾜}[HSK 7-9]
  \definition{s.}{Ótica: distância focal; comprimento focal}
\end{EntryWithPhonetic}

\begin{EntryWithPhonetic}{焦虑}{jiao1lv4}{12,10}{⽕,⾌}[HSK 7-9]
  \definition{adj.}{ansioso; preocupado; apreensivo}
\end{EntryWithPhonetic}

\begin{EntryWithPhonetic}{焦躁}{jiao1zao4}{12,20}{⽕,⾜}[HSK 7-9]
  \definition{adj.}{impaciente; inquieto de ansiedade; ansioso e irritadiço}
\end{EntryWithPhonetic}

%%%%%%%%%% 礁 %%%%%%%%%%
\subsection*{礁}\addcontentsline{loh}{figure}{礁 \dpy{jiao1}}

\begin{EntryWithPhonetic}{礁}{jiao1}{17}{⽯}
  \definition{s.}{recife; rocha}
\end{EntryWithPhonetic}

\begin{EntryWithPhonetic}{礁石}{jiao1shi2}{17,5}{⽯,⽯}[HSK 7-9]
  \definition{s.}{recife; rocha}
\end{EntryWithPhonetic}

%%%%%%%%%% 矫 %%%%%%%%%%
\subsection*{矫}\addcontentsline{loh}{figure}{矫 \dpy{jiao2}}

\begin{EntryWithPhonetic}{矫}{jiao2}{11}{⽮}
  \definition{s.}{usado em 矫情}
  \seeref{jiao3}
  \seealsoref{矫情}{jiao2qing5}
\end{EntryWithPhonetic}

\begin{EntryWithPhonetic}{矫情}{jiao2qing2}{11,11}{⽮,⼼}
  \definition{v.}{ser afetadamente não convencional; fingir ser incomum; ir deliberadamente contra o senso comum para demonstrar superioridade ou ser diferente dos outros}
  \seeref{jiao2qing5}
\end{EntryWithPhonetic}

\begin{EntryWithPhonetic}{矫情}{jiao2qing5}{11,11}{⽮,⼼}
  \definition{adj.}{briguento; contencioso; irracional; isso se refere a apresentar argumentos descabidos e causar problemas.}
  \seeref{jiao2qing2}
\end{EntryWithPhonetic}

%%%%%%%%%% 嚼 %%%%%%%%%%
\subsection*{嚼}\addcontentsline{loh}{figure}{嚼 \dpy{jiao2}}

\begin{EntryWithPhonetic}{嚼}{jiao2}{20}{⼝}[HSK 7-9]
  \definition{v.}{mastigar; mascar; limitado para uso em 过屠门而大嚼}
  \seeref{jiao4}
  \seeref{jue2}
  \seealsoref{过屠门而大嚼}{guo4 tu2men2 er2 da4 jiao2}
\end{EntryWithPhonetic}

%%%%%%%%%% 角 %%%%%%%%%%
\subsection*{角}\addcontentsline{loh}{figure}{角 \dpy{jiao3}}

\begin{EntryWithPhonetic}{角}{jiao3}{7}{⾓}[HSK 2][Kangxi 148]
  \definition*{s.}{Jiao, uma das mansões lunares}
  \definition{clas.}{uma unidade monetária fracionária na China (=1/10 de um yuan ou 10 fen)}
  \definition[个,只,对]{s.}{chifre; o objeto duro que cresce na cabeça de bovinos, ovinos, veados, etc. | buzina; corneta; instrumentos musicais tocados no exército antigo | algo com a forma de um chifre | cabo; promontório; península | esquina; canto; a junção entre duas arestas de um objeto | ângulo}
  \seeref{jue2}
\end{EntryWithPhonetic}

\begin{EntryWithPhonetic}{角度}{jiao3du4}{7,9}{⾓,⼴}[HSK 2]
  \definition[个,种]{s.}{perspectiva; ponto de vista; o ponto de partida para ver as coisas | ângulo; o tamanho do ângulo; normalmente expresso em graus ou radianos}
\end{EntryWithPhonetic}

\begin{EntryWithPhonetic}{角落}{jiao3luo4}{7,12}{⾓,⾋}[HSK 7-9]
  \definition[个,处]{s.}{canto; recanto; o ângulo côncavo na junção de duas paredes ou estruturas semelhantes | um lugar remoto}
\end{EntryWithPhonetic}

%%%%%%%%%% 狡 %%%%%%%%%%
\subsection*{狡}\addcontentsline{loh}{figure}{狡 \dpy{jiao3}}

\begin{EntryWithPhonetic}{狡}{jiao3}{9}{⽝}
  \definition{adj.}{astuto; esperto; ardiloso}
\end{EntryWithPhonetic}

\begin{EntryWithPhonetic}{狡猾}{jiao3hua2}{9,12}{⽝,⽜}[HSK 7-9]
  \definition{adj.}{astuto; ardiloso; manhoso; esperto; escorregadio; astuto e indigno de confiança}
\end{EntryWithPhonetic}

%%%%%%%%%% 绞 %%%%%%%%%%
\subsection*{绞}\addcontentsline{loh}{figure}{绞 \dpy{jiao3}}

\begin{EntryWithPhonetic}{绞}{jiao3}{9}{⽷}[HSK 7-9]
  \definition{clas.}{meada; novelo; utilizado para fios, lã, etc.}
  \definition{v.}{torcer; espremer; emaranhar | dar corda | picar; moer | pendurar pelo pescoço; estrangular | entrelaçar}
\end{EntryWithPhonetic}

%%%%%%%%%% 饺 %%%%%%%%%%
\subsection*{饺}\addcontentsline{loh}{figure}{饺 \dpy{jiao3}}

\begin{EntryWithPhonetic}{饺}{jiao3}{9}{⾷}
  \definition[盘,碗,顿,个]{s.}{bolinho de massa; \emph{dumpling}}
\end{EntryWithPhonetic}

\begin{EntryWithPhonetic}{饺子}{jiao3zi5}{9,3}{⾷,⼦}[HSK 2]
  \definition[个,盘,碗,锅]{s.}{jiaozi; bolinho chinês; bolinho de massa}
\end{EntryWithPhonetic}

%%%%%%%%%% 矫 %%%%%%%%%%
\subsection*{矫}\addcontentsline{loh}{figure}{矫 \dpy{jiao3}}

\begin{EntryWithPhonetic}{矫}{jiao3}{11}{⽮}
  \definition*{s.}{Sobrenome: Jiao}
  \definition{adj.}{forte; corajoso}
  \definition{v.}{retificar; corrigir; resolver | fingir; simular; dissimular}
  \seeref{jiao2}
\end{EntryWithPhonetic}

\begin{EntryWithPhonetic}{矫正}{jiao3zheng4}{11,5}{⽮,⽌}[HSK 7-9]
  \definition{v.}{corrigir; retificar}
\end{EntryWithPhonetic}

%%%%%%%%%% 脚 %%%%%%%%%%
\subsection*{脚}\addcontentsline{loh}{figure}{脚 \dpy{jiao3}}

\begin{EntryWithPhonetic}{脚}{jiao3}{11}{⾁}[HSK 2]
  \definition{clas.}{usado para chutes}
  \definition[只,双]{s.}{pé; a parte inferior das pernas de pessoas ou animais, que entra em contato com o solo | base; pé; a parte inferior do objeto | antigamente, referia"-se ao trabalho físico de transporte de cargas | resíduos; sobras}
\end{EntryWithPhonetic}

\begin{EntryWithPhonetic}{脚步}{jiao3bu4}{11,7}{⾁,⽌}[HSK 5]
  \definition{s.}{pé; passo; pisada; refere"-se ao movimento das pernas ao caminhar | ritmo; passo; distância entre os pés dianteiros e traseiros ao caminhar}
\end{EntryWithPhonetic}

\begin{EntryWithPhonetic}{脚印}{jiao3yin4}{11,5}{⾁,⼙}[HSK 6]
  \definition{s.}{trilha; pegada; marca de pé; os rastros deixados pelos passos}
\end{EntryWithPhonetic}

%%%%%%%%%% 搅 %%%%%%%%%%
\subsection*{搅}\addcontentsline{loh}{figure}{搅 \dpy{jiao3}}

\begin{EntryWithPhonetic}{搅}{jiao3}{12}{⼿}[HSK 7-9]
  \definition{v.}{mexer; misturar | perturbar; incomodar; interromper}
\end{EntryWithPhonetic}

\begin{EntryWithPhonetic}{搅拌}{jiao3ban4}{12,8}{⼿,⼿}[HSK 7-9]
  \definition{v.}{misturar; mexer; agitar; usar uma colher, um palito ou um utensílio semelhante para girar a mistura e homogeneizá-la}
\end{EntryWithPhonetic}

%%%%%%%%%% 缴 %%%%%%%%%%
\subsection*{缴}\addcontentsline{loh}{figure}{缴 \dpy{jiao3}}

\begin{EntryWithPhonetic}{缴}{jiao3}{16}{⽷}[HSK 7-9]
  \definition{v.}{pagar | capturar | entregar (em)}
  \seeref{zhuo2}
\end{EntryWithPhonetic}

\begin{EntryWithPhonetic}{缴费}{jiao3fei4}{16,9}{⽷,⾙}[HSK 7-9]
  \definition{v.}{pagar taxas; a taxa exigida para pagar por um serviço ou produto}
\end{EntryWithPhonetic}

\begin{EntryWithPhonetic}{缴纳}{jiao3na4}{16,7}{⽷,⽷}[HSK 7-9]
  \definition{v.}{pagar algo ao público; pagar; receber; colocar em}
\end{EntryWithPhonetic}

%%%%%%%%%% 叫 %%%%%%%%%%
\subsection*{叫}\addcontentsline{loh}{figure}{叫 \dpy{jiao4}}

\begin{EntryWithPhonetic}{叫}{jiao4}{5}{⼝}[HSK 1,3]
  \definition{adj.}{macho (animal)}
  \definition{prep.}{usado em frases passivas; introduz o agente da ação; equivalente a 被 | combinado com 看, 说; usado para expressar suas ideias e pontos de vista}
  \definition{v.}{chorar; gritar; berrar | nomear; chamar | chamar; chamar a atenção | cumprimentar; saudar; dizer olá | pedir; ordenar; licitar | permitir; concordar com algo; concordar em fazer algo | contratar; encomendar; comprar o que você precisa}
  \seealsoref{被}{bei4}
  \seealsoref{看}{kan4}
  \seealsoref{说}{shuo1}
\end{EntryWithPhonetic}

\begin{EntryWithPhonetic}{叫板}{jiao4/ban3}{5,8}{⼝,⽊}[HSK 7-9]
  \definition{v.+compl.}{Coloquial: desafiar | sinalizar aos músicos (na ópera chinesa, prolongando uma palavra falada antes de iniciar uma canção)}
\end{EntryWithPhonetic}

\begin{EntryWithPhonetic}{叫好}{jiao4/hao3}{5,6}{⼝,⼥}[HSK 7-9]
  \definition{v.+compl.}{aplaudir; gritar ``Bravo!''; gritar ``Muito bem!'' | aplaudir | torcer}
\end{EntryWithPhonetic}

\begin{EntryWithPhonetic}{叫作}{jiao4zuo4}{5,7}{⼝,⼈}[HSK 2]
  \definition{v.}{ser chamado de; ser conhecido como}
\end{EntryWithPhonetic}

%%%%%%%%%% 觉 %%%%%%%%%%
\subsection*{觉}\addcontentsline{loh}{figure}{觉 \dpy{jiao4}}

\begin{EntryWithPhonetic}{觉}{jiao4}{9}{⾒}[HSK 6]
  \definition[个]{s.}{sono; o processo desde adormecer até acordar}
  \seeref{jue2}
\end{EntryWithPhonetic}

%%%%%%%%%% 校 %%%%%%%%%%
\subsection*{校}\addcontentsline{loh}{figure}{校 \dpy{jiao4}}

\begin{EntryWithPhonetic}{校}{jiao4}{10}{⽊}
  \definition{v.}{verificar | comparar | revisar}
  \seeref{xiao4}
\end{EntryWithPhonetic}

%%%%%%%%%% 轿 %%%%%%%%%%
\subsection*{轿}\addcontentsline{loh}{figure}{轿 \dpy{jiao4}}

\begin{EntryWithPhonetic}{轿}{jiao4}{10}{⾞}
  \definition{s.}{liteira; palanquim; cadeira de arruar}
\end{EntryWithPhonetic}

\begin{EntryWithPhonetic}{轿车}{jiao4che1}{10,4}{⾞,⾞}[HSK 7-9]
  \definition[辆]{s.}{carruagem (puxada por cavalos); carruagens puxadas por animais com cortinas cobrindo os compartimentos de passageiros antigamente | ônibus; carro; sedã; um carro relativamente luxuoso e confortável, com teto e assentos para passageiros}
\end{EntryWithPhonetic}

%%%%%%%%%% 较 %%%%%%%%%%
\subsection*{较}\addcontentsline{loh}{figure}{较 \dpy{jiao4}}

\begin{EntryWithPhonetic}{较}{jiao4}{10}{⾞}[HSK 3]
  \definition{adj.}{claro; óbvio; evidente}
  \definition{adv.}{comparativamente; relativamente; razoavelmente; bastante; bastante}
  \definition{prep.}{usado para comparar características e graus; introduzir o objeto de comparação; equivalente a 比}
  \definition{v.}{comparar | disputar}
  \seealsoref{比}{bi3}
\end{EntryWithPhonetic}

\begin{EntryWithPhonetic}{较劲}{jiao4/jin4}{10,7}{⾞,⼒}[HSK 7-9]
  \definition{v.+compl.}{exigir esforço extra | adequar a própria força a (competição de força; disputa de habilidade) | colocar-se contra alguém; ir um contra o outro}
\end{EntryWithPhonetic}

\begin{EntryWithPhonetic}{较量}{jiao4liang4}{10,12}{⾞,⾥}[HSK 7-9]
  \definition{v.}{realizar uma competição; medir a própria força com; determinar quem é superior ou inferior através da competição, da luta ou de outros meios | regatear; discutir; disputar; calcular}
\end{EntryWithPhonetic}

%%%%%%%%%% 敎 %%%%%%%%%%
\subsection*{敎}\addcontentsline{loh}{figure}{敎 \dpy{jiao4}}

\begin{EntryWithPhonetic}{敎}{jiao4}{11}{⽁}
  \variantof{教}
\end{EntryWithPhonetic}

%%%%%%%%%% 教 %%%%%%%%%%
\subsection*{教}\addcontentsline{loh}{figure}{教 \dpy{jiao4}}

\begin{EntryWithPhonetic}{教}{jiao4}{11}{⽁}[HSK 1]
  \definition*{s.}{Sobrenome: Jiao}
  \definition{prep.}{em uma frase passiva para apresentar o autor da ação}
  \definition{s.}{religião | educação; professor}
  \definition{v.}{ensinar; instruir | perguntar; ordenar; contar | permitir; permitir}
  \seeref{jiao1}
  \synonymref{授}{shou4}
  \antonymref{学}{xue2}
\end{EntryWithPhonetic}

\begin{EntryWithPhonetic}{教材}{jiao4cai2}{11,7}{⽁,⽊}[HSK 3]
  \definition[本,套]{s.}{livro didático; materiais didáticos, incluindo livros didáticos, apostilas, materiais de referência, vídeos, imagens, etc.}
\end{EntryWithPhonetic}

\begin{EntryWithPhonetic}{教导}{jiao4dao3}{11,6}{⽁,⼨}
  \definition{s.}{instrução | orientação | ensino}
  \definition{v.}{instruir | orientar | ensinar}
\end{EntryWithPhonetic}

\begin{EntryWithPhonetic}{教官}{jiao4guan1}{11,8}{⽁,⼧}
  \definition{s.}{instrutor militar; um oficial que serviu como treinador no antigo exército ou escola}
\end{EntryWithPhonetic}

\begin{EntryWithPhonetic}{教会}{jiao4hui4}{11,6}{⽁,⼈}
  \definition{s.}{igreja cristã}
  \seeref{jiao1hui4}
\end{EntryWithPhonetic}

\begin{EntryWithPhonetic}{教科书}{jiao4ke1shu1}{11,9,4}{⽁,⽲,⼄}[HSK 7-9]
  \definition[本]{s.}{livro didático; livro do aluno | livro escolar; um livro escrito especialmente para alunos usarem em sala de aula e para revisão}
\end{EntryWithPhonetic}

\begin{EntryWithPhonetic}{教练}{jiao4lian4}{11,8}{⽁,⽷}[HSK 3]
  \definition[个,位,名]{s.}{instrutor; treinador (esportes); pessoas que trabalham como treinadores}
  \definition{v.}{treinar; treinar outras pessoas para dominarem uma determinada técnica (como esportes, dirigir carros, pilotar aviões, etc.)}
\end{EntryWithPhonetic}

\begin{EntryWithPhonetic}{教师}{jiao4shi1}{11,6}{⽁,⼱}[HSK 2]
  \definition[个,位,名]{s.}{professor; professor de escola}
\end{EntryWithPhonetic}

\begin{EntryWithPhonetic}{教室}{jiao4shi4}{11,9}{⽁,⼧}[HSK 2]
  \definition[间]{s.}{sala de aula}
\end{EntryWithPhonetic}

\begin{EntryWithPhonetic}{教授}{jiao4shou4}{11,11}{⽁,⼿}[HSK 4]
  \definition[个,位,名]{s.}{professor (universitário); o professor com a classificação mais alta em uma universidade}
  \definition{v.}{ensinar; instruir; dar aulas; dar palestras}
\end{EntryWithPhonetic}

\begin{EntryWithPhonetic}{教堂}{jiao4tang2}{11,11}{⽁,⼟}[HSK 6]
  \definition[座,所,间]{s.}{igreja; capela; catedral; casa de deus; um lugar onde os cristãos realizam cerimônias religiosas}
\end{EntryWithPhonetic}

\begin{EntryWithPhonetic}{教条}{jiao4tiao2}{11,7}{⽁,⽊}[HSK 7-9]
  \definition{adj.}{dogmático; opinativo}
  \definition{s.}{dogma; doutrina; credo | princípio; chavão}
\end{EntryWithPhonetic}

\begin{EntryWithPhonetic}{教学}{jiao4xue2}{11,8}{⽁,⼦}[HSK 2]
  \definition[个,门]{s.}{ensino; educação; o processo de transmissão de conhecimentos e habilidades}
\end{EntryWithPhonetic}

\begin{EntryWithPhonetic}{教学楼}{jiao4xue2lou2}{11,8,13}{⽁,⼦,⽊}[HSK 1]
  \definition{s.}{prédio da escola; bloco de ensino; edifícios utilizados para atividades educacionais, geralmente incluindo salas de aula, laboratórios, auditórios, etc.}
\end{EntryWithPhonetic}

\begin{EntryWithPhonetic}{教训}{jiao4xun5}{11,5}{⽁,⾔}[HSK 4]
  \definition[个,次,番,顿]{s.}{moral; lição}
  \definition{v.}{repreender; educar; ensinar uma lição a alguém; dar uma bronca em alguém; dar um sermão em alguém (por ter cometido um erro, etc.)}
\end{EntryWithPhonetic}

\begin{EntryWithPhonetic}{教养}{jiao4yang3}{11,9}{⽁,⼋}[HSK 7-9]
  \definition{s.}{criação; educação; formação}
\end{EntryWithPhonetic}

\begin{EntryWithPhonetic}{教育}{jiao4yu4}{11,8}{⽁,⾁}[HSK 2]
  \definition{s.}{educação; refere"-se a atividades sociais cujo objetivo direto é influenciar o desenvolvimento físico e mental das pessoas; refere"-se principalmente ao processo de formação dos alunos nas escolas}
  \definition{v.}{ensinar; educar; inspirar, fazer compreender a razão}
\end{EntryWithPhonetic}

\begin{EntryWithPhonetic}{教育部}{jiao4yu4bu4}{11,8,10}{⽁,⾁,⾢}[HSK 6]
  \definition*{s.}{Ministério da Educação}
\end{EntryWithPhonetic}

\begin{EntryWithPhonetic}{教长}{jiao4zhang3}{11,4}{⽁,⾧}
  \definition{s.}{imã (Islã) | mulá}
\end{EntryWithPhonetic}

%%%%%%%%%% 嚼 %%%%%%%%%%
\subsection*{嚼}\addcontentsline{loh}{figure}{嚼 \dpy{jiao4}}

\begin{EntryWithPhonetic}{嚼}{jiao4}{20}{⼝}
  \definition{v.}{mascar; ruminar}
\end{EntryWithPhonetic}

%%%%%%%%%% 节 %%%%%%%%%%
\subsection*{节}\addcontentsline{loh}{figure}{节 \dpy{jie1}}

\begin{EntryWithPhonetic}{节}{jie1}{5}{⾋}
  \definition{adj.}{momento crucial; momento crítico; momento decisivo; metáfora para algo importante, decisivo ou oportuno}
  \seeref{jie2}
\end{EntryWithPhonetic}

%%%%%%%%%% 阶 %%%%%%%%%%
\subsection*{阶}\addcontentsline{loh}{figure}{阶 \dpy{jie1}}

\begin{EntryWithPhonetic}{阶}{jie1}{6}{⾩}
  \definition{s.}{degrau; escada; escadaria | classificação | escala | ordem | estágio}
\end{EntryWithPhonetic}

\begin{EntryWithPhonetic}{阶层}{jie1ceng2}{6,7}{⾩,⼫}[HSK 7-9]
  \definition{s.}{posição; seção; estrato (social); isso se refere à estratificação dentro da mesma classe com base em diferentes status socioeconômicos, como a divisão da classe camponesa em camponeses pobres, camponeses médios, etc.}
\end{EntryWithPhonetic}

\begin{EntryWithPhonetic}{阶段}{jie1duan4}{6,9}{⾩,⽎}[HSK 4]
  \definition[个,段]{s.}{estágio; fase; período; bancada; gradação}
\end{EntryWithPhonetic}

\begin{EntryWithPhonetic}{阶级}{jie1ji2}{6,6}{⾩,⽷}[HSK 7-9]
  \definition[个,种]{s.}{classe (social); grupos sociais divididos de acordo com o \emph{status} socioeconômico das pessoas | degraus; escadas; passos | classificação; número de passos}
\end{EntryWithPhonetic}

\begin{EntryWithPhonetic}{阶梯}{jie1ti1}{6,11}{⾩,⽊}[HSK 7-9]
  \definition{s.}{lance de escadas; escada; degraus e escadas são metáforas para meios ou caminhos de ascensão social; equipamentos que funcionam de maneira semelhante a escadas}
\end{EntryWithPhonetic}

%%%%%%%%%% 皆 %%%%%%%%%%
\subsection*{皆}\addcontentsline{loh}{figure}{皆 \dpy{jie1}}

\begin{EntryWithPhonetic}{皆}{jie1}{9}{⽩}[HSK 7-9]
  \definition{adv.}{todos; em todos os casos; cada um e todos}
\end{EntryWithPhonetic}

%%%%%%%%%% 结 %%%%%%%%%%
\subsection*{结}\addcontentsline{loh}{figure}{结 \dpy{jie1}}

\begin{EntryWithPhonetic}{结}{jie1}{9}{⽷}[HSK 7-9]
  \definition{v.}{dar (frutos); formar (sementes); produzir frutos ou sementes (uma planta)}
  \seeref{jie2}
\end{EntryWithPhonetic}

\begin{EntryWithPhonetic}{结果}{jie1/guo3}{9,8}{⽷,⽊}[HSK 7-9]
  \definition{v.+compl.}{frutificar; dar frutos}
  \seeref{jie2/guo3}
\end{EntryWithPhonetic}

\begin{EntryWithPhonetic}{结实}{jie1shi5}{9,8}{⽷,⼧}[HSK 3]
  \definition{adj.}{sólido; resistente; durável | forte; resistente; robusto}
\end{EntryWithPhonetic}

%%%%%%%%%% 接 %%%%%%%%%%
\subsection*{接}\addcontentsline{loh}{figure}{接 \dpy{jie1}}

\begin{EntryWithPhonetic}{接}{jie1}{11}{⼿}[HSK 2]
  \definition*{s.}{Sobrenome: Jie}
  \definition{v.}{entrar em contato com; aproximar-se de | conectar; unir; juntar | continuar; prosseguir | assumir o controle; assumir o trabalho de outra pessoa e continuar a fazê-lo | pegar; agarrar; segurar ou sustentar com as mãos | receber; aceitar | encontrar; dar as boas-vindas}
\end{EntryWithPhonetic}

\begin{EntryWithPhonetic}{接班}{jie1/ban1}{11,10}{⼿,⽟}[HSK 7-9]
  \definition{v.}{assumir o turno de alguém; substituir alguém; assumir o lugar de; dar continuidade a; (sucessor) Assumir o trabalho do turno anterior | ter sucesso; dar continuidade a algo iniciado por seu antecessor}
  \seealsoref{接班儿}{jie1ban1r5}
\end{EntryWithPhonetic}

\begin{EntryWithPhonetic}{接班儿}{jie1ban1r5}{11,10,2}{⼿,⽟,⼉}
  \definition{v.}{assumir o turno de alguém; substituir alguém | ter sucesso; dar continuidade a algo iniciado por seu antecessor}
  \seealsoref{接班}{jie1/ban1}
\end{EntryWithPhonetic}

\begin{EntryWithPhonetic}{接班人}{jie1ban1ren2}{11,10,2}{⼿,⽟,⼈}[HSK 7-9]
  \definition{s.}{sucessor; a pessoa que assume o trabalho do turno anterior é frequentemente usada metaforicamente}
\end{EntryWithPhonetic}

\begin{EntryWithPhonetic}{接触}{jie1chu4}{11,13}{⼿,⾓}[HSK 5]
  \definition{v.}{entrar em contato com | entrar em contato; tocar; interagir | engajar; o termo militar refere"-se a fogo cruzado}
\end{EntryWithPhonetic}

\begin{EntryWithPhonetic}{接待}{jie1dai4}{11,9}{⼿,⼻}[HSK 3]
  \definition{v.}{receber (alguém); acolher; recepcionar; receber com cordialidade e generosidade}
\end{EntryWithPhonetic}

\begin{EntryWithPhonetic}{接到}{jie1dao4}{11,8}{⼿,⼑}[HSK 2]
  \definition{v.}{receber (carta, etc.)}
\end{EntryWithPhonetic}

\begin{EntryWithPhonetic}{接(电话)}{jie1(dian4hua4)}{11,5,8}{⼿,⽥,⾔}
  \definition{v.}{atender (o telefone) | receber (uma ligação telefônica)}
\end{EntryWithPhonetic}

\begin{EntryWithPhonetic}{接二连三}{jie1'er4-lian2san1}{11,2,7,3}{⼿,⼆,⾡,⼀}[HSK 7-9]
  \definition{expr.}{um após o outro; em rápida sucessão}
\end{EntryWithPhonetic}

\begin{EntryWithPhonetic}{接轨}{jie1/gui3}{11,6}{⼿,⾞}[HSK 7-9]
  \definition{s.}{junção; integração; ligação}
  \definition{v.+compl.}{ligar; juntar; conectar os trilhos | integrar; juntar-se a; mudar para; entrar na onda; alinhar-se; alinhar a; essa metáfora descreve como sistemas e métodos podem ser interconectados e consistentes}
\end{EntryWithPhonetic}

\begin{EntryWithPhonetic}{接济}{jie1ji4}{11,9}{⼿,⽔}[HSK 7-9]
  \definition{v.}{prestar assistência material a; dar ajuda financeira a; prestar auxílio material a}
\end{EntryWithPhonetic}

\begin{EntryWithPhonetic}{接见}{jie1jian4}{11,4}{⼿,⾒}[HSK 7-9]
  \definition{v.}{receber alguém; conceder uma entrevista a; reunir-se com as pessoas que vieram}
\end{EntryWithPhonetic}

\begin{EntryWithPhonetic}{接近}{jie1jin4}{11,7}{⼿,⾡}[HSK 3]
  \definition{adj.}{perto; próximo; a diferença entre os dois é mínima}
  \definition{v.}{estar perto de; aproximar; aproximar-se}
\end{EntryWithPhonetic}

\begin{EntryWithPhonetic}{接力}{jie1li4}{11,2}{⼿,⼒}[HSK 7-9]
  \definition{s.}{relé; trabalho por revezamento; revezamento}
\end{EntryWithPhonetic}

\begin{EntryWithPhonetic}{接连}{jie1lian2}{11,7}{⼿,⾡}[HSK 5]
  \definition{adv.}{no final; em sucessão; em uma fileira; um após o outro; seguindo o anterior; continuando}
\end{EntryWithPhonetic}

\begin{EntryWithPhonetic}{接纳}{jie1na4}{11,7}{⼿,⽷}[HSK 7-9]
  \definition{v.}{ser admitido (em uma organização); aceitar (como membro); incluir (indivíduos ou grupos que ingressam na organização) | adotar; aceitar; tomar}
\end{EntryWithPhonetic}

\begin{EntryWithPhonetic}{接收}{jie1shou1}{11,6}{⼿,⽁}[HSK 6]
  \definition{v.}{aceitar; receber | assumir; expropriar; tomar posse (de uma instituição, propriedade, etc.) de acordo com a lei | admitir; aceitar; absorver}
\end{EntryWithPhonetic}

\begin{EntryWithPhonetic}{接手}{jie1shou3}{11,4}{⼿,⼿}[HSK 7-9]
  \definition{v.}{assumir (responsabilidades, etc.); assumir problemas; assumir o trabalho de outra pessoa.}
\end{EntryWithPhonetic}

\begin{EntryWithPhonetic}{接受}{jie1shou4}{11,8}{⼿,⼜}[HSK 2]
  \definition{v.}{aceitar; não recusar (o que os outros oferecem) | concordar; não recusar (opiniões/sugestões/críticas/convites de outras pessoas, etc.)}
\end{EntryWithPhonetic}

\begin{EntryWithPhonetic}{接送}{jie1song4}{11,9}{⼿,⾡}[HSK 7-9]
  \definition{v.}{buscar e levar}
\end{EntryWithPhonetic}

\begin{EntryWithPhonetic}{接替}{jie1ti4}{11,12}{⼿,⽈}[HSK 7-9]
  \definition{v.}{assumir o controle; substituir | suceder; ocupar o lugar de}
\end{EntryWithPhonetic}

\begin{EntryWithPhonetic}{接听}{jie1ting1}{11,7}{⼿,⼝}[HSK 7-9]
  \definition{v.}{atender (o telefone)}
\end{EntryWithPhonetic}

\begin{EntryWithPhonetic}{接通}{jie1tong1}{11,10}{⼿,⾡}[HSK 7-9]
  \definition{v.}{transmitir; fazer ligação telefônica | conectar; completar a ligação; conseguir passar | fechar; encerrar; interromper; inserir; ativar; ligar; completar}
\end{EntryWithPhonetic}

\begin{EntryWithPhonetic}{接下来}{jie1xia4lai2}{11,3,7}{⼿,⼀,⽊}[HSK 2]
  \definition{expr.}{próximo; seguinte; indica uma sequência temporal subsequente}
\end{EntryWithPhonetic}

\begin{EntryWithPhonetic}{接着}{jie1zhe5}{11,11}{⼿,⽬}[HSK 2]
  \definition{adv.}{por sua vez; um após o outro; sucessivamente; conectado (à frase anterior); imediatamente após (a ação anterior)}
  \definition{v.}{seguir; prosseguir; continuar; seguir em frente; ficar ao lado | pegar com as mãos; apanhar}
\end{EntryWithPhonetic}

%%%%%%%%%% 揭 %%%%%%%%%%
\subsection*{揭}\addcontentsline{loh}{figure}{揭 \dpy{jie1}}

\begin{EntryWithPhonetic}{揭}{jie1}{12}{⼿}[HSK 6]
  \definition*{s.}{Sobrenome: Jie}
  \definition{v.}{rasgar; arrancar; tirar | descobrir; levantar (a tampa, etc.) | expor; mostrar; trazer à luz | (literário) levantar; içar}
\end{EntryWithPhonetic}

\begin{EntryWithPhonetic}{揭发}{jie1fa1}{12,5}{⼿,⼜}[HSK 7-9]
  \definition{v.}{expor; desmascarar; trazer à luz; expor e denunciar (pessoas más e más ações)}
\end{EntryWithPhonetic}

\begin{EntryWithPhonetic}{揭露}{jie1lu4}{12,21}{⼿,⾬}[HSK 7-9]
  \definition{v.}{expor; desmascarar; descobrir; revelar o que estava oculto}
\end{EntryWithPhonetic}

\begin{EntryWithPhonetic}{揭示}{jie1shi4}{12,5}{⼿,⽰}[HSK 7-9]
  \definition{v.}{anunciar; promulgar; exibir publicamente | revelar; desvendar; trazer à luz; apontar ou esclarecer a essência de coisas que não são facilmente visíveis}
\end{EntryWithPhonetic}

\begin{EntryWithPhonetic}{揭晓}{jie1xiao3}{12,10}{⼿,⽇}[HSK 7-9]
  \definition{v.}{revelar; anunciar; tornar conhecido; divulgar publicamente os resultados da investigação para que todos fiquem cientes}
\end{EntryWithPhonetic}

%%%%%%%%%% 街 %%%%%%%%%%
\subsection*{街}\addcontentsline{loh}{figure}{街 \dpy{jie1}}

\begin{EntryWithPhonetic}{街}{jie1}{12}{⾏}[HSK 2]
  \definition[条]{s.}{rua; avenida com prédios dos dois lados | mercado; feira rural}
\end{EntryWithPhonetic}

\begin{EntryWithPhonetic}{街道}{jie1dao4}{12,12}{⾏,⾡}[HSK 4]
  \definition[条]{s.}{caminho; rua; estrada; via pública com casas em ambos os lados, relativamente larga | escritório do subdistrito; tipo de organização responsável por gerenciar determinados aspectos da rua}
\end{EntryWithPhonetic}

\begin{EntryWithPhonetic}{街头}{jie1tou2}{12,5}{⾏,⼤}[HSK 6]
  \definition{s.}{rua; esquina da rua}
\end{EntryWithPhonetic}

\begin{EntryWithPhonetic}{街舞}{jie1wu3}{12,14}{⾏,⾇}
  \definition{s.}{dança de rua, \emph{street dance} (por exemplo, \emph{breakdance})}
\end{EntryWithPhonetic}

%%%%%%%%%% 楷 %%%%%%%%%%
\subsection*{楷}\addcontentsline{loh}{figure}{楷 \dpy{jie1}}

\begin{EntryWithPhonetic}{楷}{jie1}{13}{⽊}
  \definition{s.}{árvore de pistache chinês}
  \seeref{kai3}
\end{EntryWithPhonetic}

%%%%%%%%%% 节 %%%%%%%%%%
\subsection*{节}\addcontentsline{loh}{figure}{节 \dpy{jie2}}

\begin{EntryWithPhonetic}{节}{jie2}{5}{⾋}[HSK 2,6]
  \definition*{s.}{Sobrenome: Jie}
  \definition{clas.}{nó (kn), velocidade de um barco | para seções, comprimentos}
  \definition[个]{s.}{junta; botão; nó; geralmente se refere à parte da grama ou caule da grama onde as folhas crescem ou à parte onde os galhos e troncos das plantas são conectados | parte; divisão; um trecho de algo interligado; uma parte do todo | festival; feriado; dia memorável; um período de tempo ou um dia com características específicas | item; assunto | castidade; integridade ética e moral | articulação; o local onde os ossos humanos ou animais se conectam | etiqueta; cerimonial | batida; ritmo | registro; documento utilizado na antiguidade para comprovar a identidade | estação do ano | sílaba}
  \definition{v.}{economizar; conservar; poupar | resumir; extrair; retirar uma parte do todo | controlar; restringir; moderar}
  \seeref{jie1}
\end{EntryWithPhonetic}

\begin{EntryWithPhonetic}{节假日}{jie2jia4ri4}{5,11,4}{⾋,⼈,⽇}[HSK 6]
  \definition[个]{s.}{feriados; festivais e feriados}
\end{EntryWithPhonetic}

\begin{EntryWithPhonetic}{节俭}{jie2jian3}{5,9}{⾋,⼈}[HSK 7-9]
  \definition{adj.}{econômico; frugal}
\end{EntryWithPhonetic}

\begin{EntryWithPhonetic}{节目}{jie2mu4}{5,5}{⾋,⽬}[HSK 2]
  \definition[个,场,项,台]{s.}{programa; item (em um programa); programas artísticos ou projetos transmitidos por rádios e televisões}
\end{EntryWithPhonetic}

\begin{EntryWithPhonetic}{节能}{jie2 neng2}{5,10}{⾋,⾁}[HSK 6]
  \definition{v.}{economizar no consumo de energia; conservar energia}
\end{EntryWithPhonetic}

\begin{EntryWithPhonetic}{节气}{jie2qi4}{5,4}{⾋,⽓}[HSK 7-9]
  \definition{s.}{termo solar (é qualquer um dos 24 períodos nos calendários lunisolares tradicionais chineses); com base na duração do dia e da noite, na altura da sombra do meio-dia e em outros fatores, vários pontos são designados ao longo do ano, cada ponto é chamado de termo solar; os termos solares indicam a posição da Terra em sua órbita, ou seja, a posição do Sol na eclíptica; geralmente, também se referem ao dia da semana em que cada ponto ocorre}
\end{EntryWithPhonetic}

\begin{EntryWithPhonetic}{节日}{jie2ri4}{5,4}{⾋,⽇}[HSK 2]
  \definition[个,种,类]{s.}{festival; feriado; dia de comemoração tradicional; dia comemorativo estabelecido por lei}
\end{EntryWithPhonetic}

\begin{EntryWithPhonetic}{节省}{jie2sheng3}{5,9}{⾋,⽬}[HSK 4]
  \definition{adj.}{econômico; parcimonioso}
  \definition{v.}{economizar; conservar; usar com moderação; reduzir; eliminar ou minimizar o esgotamento de itens potencialmente esgotáveis}
\end{EntryWithPhonetic}

\begin{EntryWithPhonetic}{节水}{jie2shui3}{5,4}{⾋,⽔}[HSK 7-9]
  \definition{v.}{economizar água}
\end{EntryWithPhonetic}

\begin{EntryWithPhonetic}{节衣缩食}{jie2yi1-suo1shi2}{5,6,14,9}{⾋,⾐,⽷,⾷}[HSK 7-9]
  \definition{expr.}{``Reduza os gastos com comida e roupas.''; economizar em comida e roupas; ser mais econômico; viver frugalmente; praticar austeridade; praticar uma economia rigorosa}
\end{EntryWithPhonetic}

\begin{EntryWithPhonetic}{节约}{jie2yue1}{5,6}{⾋,⽷}[HSK 3]
  \definition{adj.}{econômico; sem luxo}
  \definition{v.}{guardar; economizar; usar com moderação; economizar gastos desnecessários}
\end{EntryWithPhonetic}

\begin{EntryWithPhonetic}{节奏}{jie2zou4}{5,9}{⾋,⼤}[HSK 6]
  \definition[个,种]{s.}{ritmo; o fenômeno da alternância regular de comprimento, força e fraqueza das notas na música | padrão regular; uma metáfora para um processo de ajuste adequado com tensão e relaxamento}
\end{EntryWithPhonetic}

%%%%%%%%%% 劫 %%%%%%%%%%
\subsection*{劫}\addcontentsline{loh}{figure}{劫 \dpy{jie2}}

\begin{EntryWithPhonetic}{劫}{jie2}{7}{⼒}[HSK 7-9]
  \definition{s.}{calamidade; desastre; infortúnio}
  \definition{v.}{roubar; saquear; invadir | coagir; compelir; intimidar}
\end{EntryWithPhonetic}

\begin{EntryWithPhonetic}{劫持}{jie2chi2}{7,9}{⼒,⼿}[HSK 7-9]
  \definition{v.}{sequestrar; manter sob coação; raptar; ameaçar; manter como refém}
\end{EntryWithPhonetic}

%%%%%%%%%% 杰 %%%%%%%%%%
\subsection*{杰}\addcontentsline{loh}{figure}{杰 \dpy{jie2}}

\begin{EntryWithPhonetic}{杰}{jie2}{8}{⽊}
  \definition{adj.}{notável; proeminente; fora do comum}
  \definition[位,名,个,些]{s.}{pessoa excepcional; herói; uma pessoa com talentos excepcionais}
\end{EntryWithPhonetic}

\begin{EntryWithPhonetic}{杰出}{jie2chu1}{8,5}{⽊,⼐}[HSK 6]
  \definition{adj.}{notável; proeminente; (talento, realização) excepcional}
\end{EntryWithPhonetic}

%%%%%%%%%% 拮 %%%%%%%%%%
\subsection*{拮}\addcontentsline{loh}{figure}{拮 \dpy{jie2}}

\begin{EntryWithPhonetic}{拮}{jie2}{9}{⼿}
  \definition{adj.}{trabalhoso | sem dinheiro | antagônico | trabalhando duro | pressionado}
\end{EntryWithPhonetic}

\begin{EntryWithPhonetic}{拮据}{jie2ju1}{9,11}{⼿,⼿}
  \definition{adj.}{em circunstâncias difíceis; sem dinheiro; em dificuldades}
\end{EntryWithPhonetic}

%%%%%%%%%% 洁 %%%%%%%%%%
\subsection*{洁}\addcontentsline{loh}{figure}{洁 \dpy{jie2}}

\begin{EntryWithPhonetic}{洁}{jie2}{9}{⽔}
  \definition{adj.}{limpo; arrumado | honesto; íntegro}
  \definition{v.}{limpar; purificar; tornar limpo | tornar inocente}
\end{EntryWithPhonetic}

\begin{EntryWithPhonetic}{洁净}{jie2jing4}{9,8}{⽔,⼎}[HSK 7-9]
  \definition{adj.}{limpo; impecável}
\end{EntryWithPhonetic}

%%%%%%%%%% 结 %%%%%%%%%%
\subsection*{结}\addcontentsline{loh}{figure}{结 \dpy{jie2}}

\begin{EntryWithPhonetic}{结}{jie2}{9}{⽷}[HSK 4]
  \definition*{s.}{Sobrenome: Jie}
  \definition{s.}{nó | declaração juramentada; garantia por escrito; documento que, antigamente, significava um reconhecimento de encerramento ou uma garantia de responsabilidade}
  \definition{v.}{amarrar; tricotar; dar nó; tecer | formar; forjar; cimentar; solidificar | resolver; concluir | combinar; formar um relacionamento}
  \seeref{jie1}
\end{EntryWithPhonetic}

\begin{EntryWithPhonetic}{结冰}{jie2bing1}{9,6}{⽷,⼎}[HSK 7-9]
  \definition{v.}{congelar; cobrir com gelo}
\end{EntryWithPhonetic}

\begin{EntryWithPhonetic}{结构}{jie2gou4}{9,8}{⽷,⽊}[HSK 4]
  \definition[个]{s.}{estrutura; composição; construção; formação; constituição; tecido; forma; sistematização; mecânica; organização | arquitetura; estrutura; construção; construção de partes de edifícios com suporte de carga ou com carga externa | Geologia: textura}[这些矿物质具有致密结构。===Esses minerais têm uma estrutura densa.]
\end{EntryWithPhonetic}

\begin{EntryWithPhonetic}{结果}{jie2/guo3}{9,8}{⽷,⽊}[HSK 2]
  \definition{conj.}{como resultado; no final}
  \definition{v.}{despachar | matar}
  \definition{v.+compl.}{resultado; conclusão; consequência}
  \seeref{jie1/guo3}
\end{EntryWithPhonetic}

\begin{EntryWithPhonetic}{结合}{jie2he2}{9,6}{⽷,⼝}[HSK 3]
  \definition{v.}{ligar; unir; combinar; integrar; formar uma relação estreita entre pessoas ou coisas | casar-se; unir-se em matrimônio; referir-se especificamente a casais}
\end{EntryWithPhonetic}

\begin{EntryWithPhonetic}{结婚}{jie2/hun1}{9,11}{⽷,⼥}[HSK 3]
  \definition{v.+compl.}{casar; casar"-se; casar"-se bem}
\end{EntryWithPhonetic}

\begin{EntryWithPhonetic}{结婚礼服}{jie2hun1 li3 fu2}{9,11,5,8}{⽷,⼥,⽰,⽉}
  \definition{s.}{vestido de casamento; vestido de noiva}
\end{EntryWithPhonetic}

\begin{EntryWithPhonetic}{结晶}{jie2jing1}{9,12}{⽷,⽇}[HSK 7-9]
  \definition{s.}{cristal; substâncias cristalinas | Figurativo: os frutos (do trabalho ou da atividade); metáfora para conquistas preciosas}
  \definition{v.}{cristalizar; as substâncias podem formar cristais a partir de um estado líquido (solução ou fundido) ou gasoso}
\end{EntryWithPhonetic}

\begin{EntryWithPhonetic}{结局}{jie2ju2}{9,7}{⽷,⼫}[HSK 7-9]
  \definition[个]{s.}{final; desfecho; resultado final; resultado final; conclusão}
\end{EntryWithPhonetic}

\begin{EntryWithPhonetic}{结论}{jie2lun4}{9,6}{⽷,⾔}[HSK 4]
  \definition[个]{s.}{conclusão; palavra final sobre uma pessoa ou coisa após investigação e pesquisa | veredito; julgamento deduzido de premissas também é chamado de conclusão}
\end{EntryWithPhonetic}

\begin{EntryWithPhonetic}{结社自由}{jie2she4zi4you2}{9,7,6,5}{⽷,⽰,⾃,⽥}
  \definition{s.}{(constitucional) liberdade de associação}
\end{EntryWithPhonetic}

\begin{EntryWithPhonetic}{结识}{jie2shi2}{9,7}{⽷,⾔}[HSK 7-9]
  \definition{v.}{conhecer alguém; fazer amizade com; familiarizar-se com alguém}
\end{EntryWithPhonetic}

\begin{EntryWithPhonetic}{结束}{jie2shu4}{9,7}{⽷,⽊}[HSK 3]
  \definition{v.}{finalizar; fechar; terminar; concluir; encerrar; desenvolver ou avançar até a fase final, sem continuidade}
\end{EntryWithPhonetic}

\begin{EntryWithPhonetic}{结束辩论}{jie2shu4 bian4 lun4}{9,7,16,6}{⽷,⽊,⾟,⾔}
  \definition{s.}{debate de encerramento}
\end{EntryWithPhonetic}

\begin{EntryWithPhonetic}{结束工作}{jie2shu4gong1zuo4}{9,7,3,7}{⽷,⽊,⼯,⼈}
  \definition{s.}{trabalho final}
  \definition{v.}{terminar o trabalho}
\end{EntryWithPhonetic}

\begin{EntryWithPhonetic}{结束剂}{jie2shu4 ji4}{9,7,8}{⽷,⽊,⼑}
  \definition{s.}{finalizador}
\end{EntryWithPhonetic}

\begin{EntryWithPhonetic}{结束区}{jie2shu4 qu1}{9,7,4}{⽷,⽊,⼖}
  \definition{s.}{zona final}
\end{EntryWithPhonetic}

\begin{EntryWithPhonetic}{结束文本}{jie2shu4 wen2ben3}{9,7,4,5}{⽷,⽊,⽂,⽊}
  \definition{s.}{texto final}
\end{EntryWithPhonetic}

\begin{EntryWithPhonetic}{结束语}{jie2shu4yu3}{9,7,9}{⽷,⽊,⾔}
  \definition{s.}{conclusões finais | considerações finais}
\end{EntryWithPhonetic}

\begin{EntryWithPhonetic}{结尾}{jie2wei3}{9,7}{⽷,⼫}[HSK 7-9]
  \definition{s.}{final; fase de encerramento; fim de fase ou parte}
  \definition{v.}{encerrar; concluir a etapa final}
\end{EntryWithPhonetic}

%%%%%%%%%% 捷 %%%%%%%%%%
\subsection*{捷}\addcontentsline{loh}{figure}{捷 \dpy{jie2}}

\begin{EntryWithPhonetic}{捷}{jie2}{11}{⼿}
  \definition*{s.}{Sobrenome: Jie}
  \definition{adj.}{rápido; ágil}
  \definition{s.}{vitória; triunfo; sucesso}
\end{EntryWithPhonetic}

\begin{EntryWithPhonetic}{捷径}{jie2jing4}{11,8}{⼿,⼻}
  \definition{s.}{atalho}
\end{EntryWithPhonetic}

%%%%%%%%%% 截 %%%%%%%%%%
\subsection*{截}\addcontentsline{loh}{figure}{截 \dpy{jie2}}

\begin{EntryWithPhonetic}{截}{jie2}{14}{⼽}[HSK 7-9]
  \definition{clas.}{seção; pedaço; comprimento}
  \definition{prep.}{por (um tempo especificado); até}
  \definition{v.}{cortar; romper | parar; verificar; interromper; interceptar}
\end{EntryWithPhonetic}

\begin{EntryWithPhonetic}{截然不同}{jie2ran2-bu4tong2}{14,12,4,6}{⼽,⽕,⼀,⼝}[HSK 7-9]
  \definition{expr.}{``Completamente diferente.''; tão diferente quanto preto e branco; polos opostos}
\end{EntryWithPhonetic}

\begin{EntryWithPhonetic}{截止}{jie2zhi3}{14,4}{⼽,⽌}[HSK 6]
  \definition{adv.}{até (um certo limite de tempo); por (um tempo especificado)}
\end{EntryWithPhonetic}

\begin{EntryWithPhonetic}{截至}{jie2zhi4}{14,6}{⼽,⾄}[HSK 6]
  \definition{adv.}{a partir de; até (um certo limite de tempo); por (um tempo especificado)}
\end{EntryWithPhonetic}

%%%%%%%%%% 竭 %%%%%%%%%%
\subsection*{竭}\addcontentsline{loh}{figure}{竭 \dpy{jie2}}

\begin{EntryWithPhonetic}{竭}{jie2}{14}{⽴}
  \definition*{s.}{Sobrenome: Jie}
  \definition{v.}{esgotar; consumir | Literário: secar; drenar}
\end{EntryWithPhonetic}

\begin{EntryWithPhonetic}{竭尽全力}{jie2jin4-quan2li4}{14,6,6,2}{⽴,⼫,⼊,⼒}[HSK 7-9]
  \definition{expr.}{``Dê o seu melhor.''; não poupar esforços; fazer o máximo possível; com todas as forças; usar todas as suas forças para descrever o ato de fazer o máximo esforço; fazer o máximo possível; fazer tudo o que estiver ao seu alcance}
\end{EntryWithPhonetic}

\begin{EntryWithPhonetic}{竭力}{jie2li4}{14,2}{⽴,⼒}[HSK 7-9]
  \definition{v.}{fazer o máximo; fazer o máximo; não poupar esforços; tentar por todos os meios possíveis; dar o melhor de si; usar todos os esforços do corpo e da mente para\dots; usar cada grama de sua energia}
\end{EntryWithPhonetic}

%%%%%%%%%% 姐 %%%%%%%%%%
\subsection*{姐}\addcontentsline{loh}{figure}{姐 \dpy{jie3}}

\begin{EntryWithPhonetic}{姐}{jie3}{8}{⼥}
  \definition[个,位]{s.}{irmã mais velha; irmã | termo genérico para mulheres jovens | mulheres da mesma geração que são mais velhas do que você (geralmente não inclui aquelas que podem ser chamadas de cunhadas) | um título respeitoso para mulheres jovens profissionais em determinados cargos}
  \seealsoref{姐姐}{jie3jie5}
\end{EntryWithPhonetic}

\begin{EntryWithPhonetic}{姐夫}{jie3fu5}{8,4}{⼥,⼤}
  \definition{s.}{marido da irmã mais velha}
\end{EntryWithPhonetic}

\begin{EntryWithPhonetic}{姐姐}{jie3jie5}{8,8}{⼥,⼥}[HSK 1]
  \definition[个]{s.}{irmã mais velha}
\end{EntryWithPhonetic}

\begin{EntryWithPhonetic}{姐妹}{jie3mei4}{8,8}{⼥,⼥}[HSK 4]
  \definition[个]{s.}{irmãs}
\end{EntryWithPhonetic}

%%%%%%%%%% 解 %%%%%%%%%%
\subsection*{解}\addcontentsline{loh}{figure}{解 \dpy{jie3}}

\begin{EntryWithPhonetic}{解}{jie3}{13}{⾓}[HSK 6]
  \definition{s.}{solução; o valor de uma variável desconhecida em uma equação algébrica}
  \definition{v.}{dividir; separar | desfazer; desatar; abrir algo que esteja amarrado ou encadernado | acalmar; dissipar; dispensar; eliminar | resolver; explicar; interpretar | entender; compreender | aliviar-se (excreção de urina e fezes) | dissolver; desintegrar | (cálculo analítico) resolver; solucionar}
\end{EntryWithPhonetic}

\begin{EntryWithPhonetic}{解除}{jie3chu2}{13,9}{⾓,⾩}[HSK 5]
  \definition{v.}{remover; aliviar; livrar"-se de; eliminar}
\end{EntryWithPhonetic}

\begin{EntryWithPhonetic}{解答}{jie3da2}{13,12}{⾓,⽵}[HSK 7-9]
  \definition{v.}{responder; explicar}
\end{EntryWithPhonetic}

\begin{EntryWithPhonetic}{解读}{jie3du2}{13,10}{⾓,⾔}[HSK 7-9]
  \definition{v.}{decodificar; interpretar; explicar; compreender por meio da análise}
\end{EntryWithPhonetic}

\begin{EntryWithPhonetic}{解放}{jie3fang4}{13,8}{⾓,⽅}[HSK 5]
  \definition*{s.}{Libertação (que significou o fim do domínio do regime reacionário Kuomintang em 1949 e ao estabelecimento da República Popular da China)}
  \definition{v.}{libertar; emancipar; eliminar as restrições para permitir o desenvolvimento da liberdade}
\end{EntryWithPhonetic}

\begin{EntryWithPhonetic}{解雇}{jie3gu4}{13,12}{⾓,⾫}[HSK 7-9]
  \definition{v.}{demitir; dispensar; exonerar}
\end{EntryWithPhonetic}

\begin{EntryWithPhonetic}{解救}{jie3jiu4}{13,11}{⾓,⽁}[HSK 7-9]
  \definition{v.}{salvar; resgatar; sair do perigo ou da dificuldade}
\end{EntryWithPhonetic}

\begin{EntryWithPhonetic}{解决}{jie3jue2}{13,6}{⾓,⼎}[HSK 3]
  \definition{v.}{solucionar; resolver; liquidar; resolver problemas com resultados | acabar com; descartar; eliminar (o inimigo)}
\end{EntryWithPhonetic}

\begin{EntryWithPhonetic}{解开}{jie3 kai1}{13,4}{⾓,⼶}[HSK 3]
  \definition{v.}{desatar; desamarrar; desabotoar; desamarrar ou desfazer nós}
\end{EntryWithPhonetic}

\begin{EntryWithPhonetic}{解剖}{jie3pou1}{13,10}{⾓,⼑}[HSK 7-9]
  \definition{v.}{dissecar | analisar; metaforicamente, refere"-se à observação e análise aprofundada das coisas}
\end{EntryWithPhonetic}

\begin{EntryWithPhonetic}{解散}{jie3san4}{13,12}{⾓,⽁}[HSK 7-9]
  \definition{v.}{dispensar | dissolver; desmantelar; cancelar}
\end{EntryWithPhonetic}

\begin{EntryWithPhonetic}{解释}{jie3shi4}{13,12}{⾓,⾤}[HSK 4]
  \definition{v.}{explicar; expor; interpretar | analisar; explicaro significado, razões, justificativas, etc.}
\end{EntryWithPhonetic}

\begin{EntryWithPhonetic}{解说}{jie3shuo1}{13,9}{⾓,⾔}[HSK 6]
  \definition{v.}{narrar; comentar; fazer um comentário; explicar oralmente}
\end{EntryWithPhonetic}

\begin{EntryWithPhonetic}{解体}{jie3ti3}{13,7}{⾓,⼈}[HSK 7-9]
  \definition{v.}{(corpo orgânico) decompor-se | (sistema social, organização, etc.) desintegrar-se | desmontar; desintegrar; quebrar; desmantelar}
\end{EntryWithPhonetic}

\begin{EntryWithPhonetic}{解脱}{jie3tuo1}{13,11}{⾓,⾁}[HSK 7-9]
  \definition{v.}{libertar"-se das preocupações mundanas; no budismo,  se refere à libertação do sofrimento e à conquista da liberdade | libertar"-se (ou desvencilhar"-se); livrar"-se de | absolver; exonerar}
\end{EntryWithPhonetic}

\begin{EntryWithPhonetic}{解围}{jie3/wei2}{13,7}{⾓,⼞}[HSK 7-9]
  \definition{v.+compl.}{forçar um inimigo a levantar um cerco; resgatar de um cerco; vir em socorro dos sitiados | ajudar a sair de uma situação difícil; evitar constrangimento; livrar alguém de um constrangimento; livrar alguém de uma enrascada; amenizar o constrangimento de alguém}
\end{EntryWithPhonetic}

\begin{EntryWithPhonetic}{解析}{jie3xi1}{13,8}{⾓,⽊}[HSK 7-9]
  \definition{v.}{analisar; examinar minuciosamente; resolver}
\end{EntryWithPhonetic}

\begin{EntryWithPhonetic}{解压}{jie3ya1}{13,6}{⾓,⼚}
  \definition{v.}{aliviar o estresse | Computação: descomprimir}
\end{EntryWithPhonetic}

%%%%%%%%%% 介 %%%%%%%%%%
\subsection*{介}\addcontentsline{loh}{figure}{介 \dpy{jie4}}

\begin{EntryWithPhonetic}{介}{jie4}{4}{⼈}
  \definition*{s.}{Sobrenome: Jie}
  \definition{adj.}{direto; honesto e franco; correto}
  \definition{s.}{armadura | concha (crustáceos e criaturas aquáticas) | preposição}
  \definition{v.}{estar situado entre; interpor | levar a sério; levar em conta; ter em mente}
\end{EntryWithPhonetic}

\begin{EntryWithPhonetic}{介入}{jie4ru4}{4,2}{⼈,⼊}[HSK 7-9]
  \definition{v.}{intervir; interpor-se; envolver-se}
  \synonymref{参与}{can1yu4}
  \antonymref{旁观}{pang2guan1}
\end{EntryWithPhonetic}

\begin{EntryWithPhonetic}{介绍}{jie4shao4}{4,8}{⼈,⽷}[HSK 1]
  \definition{s.}{introdução; apresentação}
  \definition{v.}{introduzir; apresentar | recomendar; sugerir | dar a conhecer; informar}
  \seealsoref{说明}{shuo1ming2}
\end{EntryWithPhonetic}

\begin{EntryWithPhonetic}{介意}{jie4/yi4}{4,13}{⼈,⼼}[HSK 7-9]
  \definition{v.+compl.}{(geralmente na forma negativa) incomodar; ofender-se; guardar coisas desagradáveis no coração; preocupar-se com elas}
  \synonymref{当心}{dang1xin1}
  \synonymref{留心}{liu2/xin1}
  \synonymref{留意}{liu2/yi4}
  \synonymref{小心}{xiao3xin5}
  \synonymref{在意}{zai4/yi4}
  \synonymref{在乎}{zai4hu5}
\end{EntryWithPhonetic}

\begin{EntryWithPhonetic}{介于}{jie4yu2}{4,3}{⼈,⼆}[HSK 7-9]
  \definition{v.}{estar no meio (entre os dois)}
\end{EntryWithPhonetic}

%%%%%%%%%% 戒 %%%%%%%%%%
\subsection*{戒}\addcontentsline{loh}{figure}{戒 \dpy{jie4}}

\begin{EntryWithPhonetic}{戒}{jie4}{7}{⼽}[HSK 5]
  \definition[个,枚]{s.}{advertência; exortação | disciplina monástica budista; preceitos budistas | anel (dedo)}
  \definition{v.}{proteger-se contra; estar preparado; estar atento | advertir; exortar; admoestar | abandonar; parar; desistir; desistir (de um hábito ruim)}
\end{EntryWithPhonetic}

\begin{EntryWithPhonetic}{戒备}{jie4bei4}{7,8}{⼽,⼡}[HSK 7-9]
  \definition{v.}{tomar precauções; estar em alerta; são muito cuidadosos em suas palavras e ações para evitar que algo de ruim aconteça temendo que outros possam lhes fazer mal | guardar; proteger-se de; os militares ou a polícia protegem um local para evitar que coisas ruins aconteçam}
\end{EntryWithPhonetic}

\begin{EntryWithPhonetic}{戒烟}{jie4 yan1}{7,10}{⼽,⽕}[HSK 7-9]
  \definition{v.}{deixar de fumar; parar de fumar; refere"-se a abandonar o hábito de fumar ópio.; refere"-se também a abandonar o hábito de fumar cigarros}
\end{EntryWithPhonetic}

\begin{EntryWithPhonetic}{戒指}{jie4zhi5}{7,9}{⼽,⼿}[HSK 7-9]
  \definition[个]{s.}{anel}
  \seealsoref{戒指儿}{jie4zhi5r5}
\end{EntryWithPhonetic}

\begin{EntryWithPhonetic}{戒指儿}{jie4zhi5r5}{7,9,2}{⼽,⼿,⼉}
  \definition{s.}{anel}
  \seealsoref{戒指}{jie4zhi5}
\end{EntryWithPhonetic}

%%%%%%%%%% 芥 %%%%%%%%%%
\subsection*{芥}\addcontentsline{loh}{figure}{芥 \dpy{jie4}}

\begin{EntryWithPhonetic}{芥}{jie4}{7}{⾋}
  \definition{s.}{mostarda}
  \seeref{gai4}
\end{EntryWithPhonetic}

\begin{EntryWithPhonetic}{芥兰}{jie4lan2}{7,5}{⾋,⼋}
  \definition{s.}{couve}
\end{EntryWithPhonetic}

%%%%%%%%%% 届 %%%%%%%%%%
\subsection*{届}\addcontentsline{loh}{figure}{届 \dpy{jie4}}

\begin{EntryWithPhonetic}{届}{jie4}{8}{⼫}[HSK 5]
  \definition{clas.}{sessões (de uma conferência); anos (de graduação); quantificador, ligeiramente equivalente a 次, usado para reuniões regulares ou turmas de formandos, etc.}
  \definition{v.}{vencer o prazo}
  \seealsoref{次}{ci4}
\end{EntryWithPhonetic}

\begin{EntryWithPhonetic}{届时}{jie4shi2}{8,7}{⼫,⽇}[HSK 7-9]
  \definition{adv.}{na ocasião; quando chegar a hora; no momento determinado; no horário combinado}
\end{EntryWithPhonetic}

%%%%%%%%%% 界 %%%%%%%%%%
\subsection*{界}\addcontentsline{loh}{figure}{界 \dpy{jie4}}

\begin{EntryWithPhonetic}{界}{jie4}{9}{⽥}[HSK 6]
  \definition{s.}{fronteira; limite | escopo; extensão | círculos | divisão primária; reino | era geológica | (matemática) limite | mundo; faixa dividida por ocupação, emprego ou gênero, etc. | grupo}
\end{EntryWithPhonetic}

\begin{EntryWithPhonetic}{界碑}{jie4bei1}{9,13}{⽥,⽯}
  \definition{s.}{marco de fronteira}
\end{EntryWithPhonetic}

\begin{EntryWithPhonetic}{界定}{jie4ding4}{9,8}{⽥,⼧}[HSK 7-9]
  \definition{v.}{definir; delimitar; especificar os limites; definir escopo | definir; dar uma definição}
\end{EntryWithPhonetic}

\begin{EntryWithPhonetic}{界线}{jie4xian4}{9,8}{⽥,⽷}[HSK 7-9]
  \definition{s.}{fronteira; limite; a fronteira entre coisas diferentes; a linha que divide duas regiões}
\end{EntryWithPhonetic}

\begin{EntryWithPhonetic}{界限}{jie4xian4}{9,8}{⽥,⾩}[HSK 7-9]
  \definition{s.}{linha; limites; fronteiras; demarcação (ou divisão); a fronteira entre coisas diferentes | limite; fim}
\end{EntryWithPhonetic}

%%%%%%%%%% 借 %%%%%%%%%%
\subsection*{借}\addcontentsline{loh}{figure}{借 \dpy{jie4}}

\begin{EntryWithPhonetic}{借}{jie4}{10}{⼈}[HSK 2]
  \definition{adv.}{por meio de}
  \definition{v.}{emprestar | pedir emprestado | usar como pretexto | aproveitar; tirar proveito (de uma oportunidade, etc.)}
\end{EntryWithPhonetic}

\begin{EntryWithPhonetic}{借鉴}{jie4jian4}{10,13}{⼈,⾦}[HSK 6]
  \definition{s.}{tirar lições de; aproveitar a experiência de; ganhar experiência e lições com o passado ou com as experiências de outras pessoas}
\end{EntryWithPhonetic}

\begin{EntryWithPhonetic}{借口}{jie4kou3}{10,3}{⼈,⼝}[HSK 7-9]
  \definition[个,种]{s.}{desculpa; pretexto; razões falsas apresentadas para atingir um objetivo}
  \definition{v.}{usar como desculpa; usar sob o pretexto de; usar com a justificativa de; usar (algo) como motivo (que não seja um motivo real)}
\end{EntryWithPhonetic}

\begin{EntryWithPhonetic}{借书证}{jie4shu1zheng4}{10,4,7}{⼈,⼄,⾔}
  \definition{s.}{cartão da biblioteca; comprovante de solicitação}
  \seealsoref{借书证卡}{jie4shu1zheng4 ka3}
\end{EntryWithPhonetic}

\begin{EntryWithPhonetic}{借书证卡}{jie4shu1zheng4 ka3}{10,4,7,5}{⼈,⼄,⾔,⼘}
  \definition{s.}{cartão da biblioteca}
  \seealsoref{借书证}{jie4shu1zheng4}
\end{EntryWithPhonetic}

\begin{EntryWithPhonetic}{借条}{jie4tiao2}{10,7}{⼈,⽊}[HSK 7-9]
  \definition{s.}{recibo de empréstimo; nota promissória}
  \seealsoref{借条儿}{jie4tiao2r5}
\end{EntryWithPhonetic}

\begin{EntryWithPhonetic}{借条儿}{jie4tiao2r5}{10,7,2}{⼈,⽊,⼉}
  \definition{s.}{nota promissória}
\end{EntryWithPhonetic}

\begin{EntryWithPhonetic}{借用}{jie4yong4}{10,5}{⼈,⽤}[HSK 7-9]
  \definition{v.}{tomar emprestado; ter o empréstimo de | usar algo para outro propósito}
\end{EntryWithPhonetic}

\begin{EntryWithPhonetic}{借助}{jie4zhu4}{10,7}{⼈,⼒}[HSK 7-9]
  \definition{v.}{contar com a ajuda de; obter apoio de}
\end{EntryWithPhonetic}

%%%%%%%%%% 今 %%%%%%%%%%
\subsection*{今}\addcontentsline{loh}{figure}{今 \dpy{jin1}}

\begin{EntryWithPhonetic}{今}{jin1}{4}{⼈}
  \definition*{s.}{Sobrenome: Jin}
  \definition{s.}{agora; o presente | moderno | de hoje; deste ano | isso; isto}
  \antonymref{古}{gu3}
  \antonymref{昔}{xi1}
\end{EntryWithPhonetic}

\begin{EntryWithPhonetic}{今后}{jin1hou4}{4,6}{⼈,⼝}[HSK 2]
  \definition{s.}{a partir de agora; doravante; no futuro; desde o momento em que falamos}
  \synonymref{此后}{ci3hou4}
  \synonymref{从此}{cong2ci3}
  \synonymref{往后}{wang3hou4}
  \synonymref{以后}{yi3hou4}
  \synonymref{以来}{yi3lai2}
  \antonymref{从来}{cong2lai2}
  \antonymref{当前}{dang1qian2}
  \antonymref{以前}{yi3qian2}
\end{EntryWithPhonetic}

\begin{EntryWithPhonetic}{今年}{jin1nian2}{4,6}{⼈,⼲}[HSK 1]
  \definition{adv.}{este ano}
  \synonymref{来年}{lai2nian2}
  \antonymref{去年}{qu4nian2}
  \antonymref{往年}{wang3nian2}
\end{EntryWithPhonetic}

\begin{EntryWithPhonetic}{今人}{jin1ren2}{4,2}{⼈,⼈}
  \definition{s.}{contemporâneos; pessoas da nossa era; pessoas de hoje | pessoas modernas; pessoas contemporâneas}
  \antonymref{古人}{gu3ren2}
\end{EntryWithPhonetic}

\begin{EntryWithPhonetic}{今日}{jin1ri4}{4,4}{⼈,⽇}[HSK 5]
  \definition{s.}{hoje}
  \antonymref{当初}{dang1chu1}
  \antonymref{往日}{wang3ri4}
  \antonymref{昔日}{xi1ri4}
\end{EntryWithPhonetic}

\begin{EntryWithPhonetic}{今天}{jin1tian1}{4,4}{⼈,⼤}[HSK 1]
  \definition{adv.}{hoje; neste dia | agora; o momento ou a época atual}
  \synonymref{今日}{jin1ri4}
  \antonymref{昨天}{zuo2tian1}
\end{EntryWithPhonetic}

%%%%%%%%%% 斤 %%%%%%%%%%
\subsection*{斤}\addcontentsline{loh}{figure}{斤 \dpy{jin1}}

\begin{EntryWithPhonetic}{斤}{jin1}{4}{⽄}[HSK 2][Kangxi 69]
  \definition{clas.}{uma unidade de peso (=500 gramas)}
  \definition{s.}{machado; cutelo; ferramentas antigas para cortar árvores}
\end{EntryWithPhonetic}

%%%%%%%%%% 金 %%%%%%%%%%
\subsection*{金}\addcontentsline{loh}{figure}{金 \dpy{jin1}}

\begin{EntryWithPhonetic}{金}{jin1}{8}{⾦}[HSK 3][Kangxi 167]
  \definition*{s.}{Dinastia Jin (1115-1234) | Sobrenome: Jin}
  \definition{adj.}{dourado | altamente respeitado; precioso. metáfora de nobreza}
  \definition[锭,块]{s.}{ouro | metal | dinheiro | instrumento antigo de percussão de metal}
\end{EntryWithPhonetic}

\begin{EntryWithPhonetic}{金额}{jin1'e2}{8,15}{⾦,⾴}[HSK 6]
  \definition[份,笔]{s.}{quantidade de dinheiro; soma de dinheiro}
\end{EntryWithPhonetic}

\begin{EntryWithPhonetic}{金刚石}{jin1gang1shi2}{8,6,5}{⾦,⼑,⽯}
  \definition{s.}{diamante, também chamado de 钻石}[金刚石比什么金属都硬。===O diamante é mais duro que qualquer metal.]
  \seealsoref{钻石}{zuan4shi2}
\end{EntryWithPhonetic}

\begin{EntryWithPhonetic}{金牌}{jin1pai2}{8,12}{⾦,⽚}[HSK 3]
  \definition[枚]{s.}{medalha de ouro; refere"-se à medalha conquistada pelo campeão em uma competição esportiva | ficha de ouro; placa de ouro usada como símbolo}
\end{EntryWithPhonetic}

\begin{EntryWithPhonetic}{金钱}{jin1qian2}{8,10}{⾦,⾦}[HSK 6]
  \definition[沓,笔,堆]{s.}{dinheiro; moeda}
\end{EntryWithPhonetic}

\begin{EntryWithPhonetic}{金融}{jin1rong2}{8,16}{⾦,⿀}[HSK 6]
  \definition{s.}{finanças; serviços bancários; refere"-se a atividades econômicas como a emissão, circulação e retirada de moeda, a concessão e retirada de empréstimos, o depósito e retirada de depósitos e transações de câmbio}
\end{EntryWithPhonetic}

\begin{EntryWithPhonetic}{金色}{jin1 se4}{8,6}{⾦,⾊}
  \definition{s.}{cor ouro; dourado}
\end{EntryWithPhonetic}

\begin{EntryWithPhonetic}{金属}{jin1shu3}{8,12}{⾦,⼫}[HSK 7-9]
  \definition[种,块,片]{s.}{metal; um tipo de substância com superfície relativamente lisa e brilhante, porém opaca, capaz de conduzir eletricidade e calor}
\end{EntryWithPhonetic}

\begin{EntryWithPhonetic}{金字塔}{jin1zi4ta3}{8,6,12}{⾦,⼦,⼟}[HSK 7-9]
  \definition[座]{s.}{pirâmide (edifício ou estrutura); as pirâmides egípcias, um tipo de estrutura utilizada por alguns povos antigos, são pirâmides de pedra com três ou mais lados, que, à distância, lembram o caractere chinês 金 (ouro); elas serviam de túmulo para antigos imperadores}
\end{EntryWithPhonetic}

\begin{EntryWithPhonetic}{金子}{jin1zi5}{8,3}{⾦,⼦}[HSK 7-9]
  \definition{s.}{ouro; elemento metálico, símbolo Au (aurum) amarelo-avermelhado, macio, dúctil, quimicamente estável é um metal precioso, usado para fabricar dinheiro, ornamentos etc.}
\end{EntryWithPhonetic}

%%%%%%%%%% 津 %%%%%%%%%%
\subsection*{津}\addcontentsline{loh}{figure}{津 \dpy{jin1}}

\begin{EntryWithPhonetic}{津}{jin1}{9}{⽔}
  \definition*{s.}{abreviação de Tianjin, 天津}
  \definition{adj.}{úmido; molhado; hidratado}
  \definition{s.}{suor | travessia de balsa; vau; balsa | metáfora para cargos importantes | saliva}
  \seealsoref{天津}{tian1jin1}
\end{EntryWithPhonetic}

\begin{EntryWithPhonetic}{津津有味}{jin1jin1-you3wei4}{9,9,6,8}{⽔,⽔,⽉,⼝}[HSK 7-9]
  \definition{expr.}{com grande prazer; o sabor é delicioso; com entusiasmo; com grande satisfação}
\end{EntryWithPhonetic}

\begin{EntryWithPhonetic}{津贴}{jin1tie1}{9,9}{⽔,⾙}[HSK 7-9]
  \definition{s.}{subsídio; auxílio financeiro; abono; pensão; além dos salários, os subsídios também se referem aos auxílios de custo de vida para funcionários do sistema de fornecimento}
  \definition{v.}{subsidiar; conceder auxílio financeiro; subsidiar pessoas com dinheiro ou bens}
\end{EntryWithPhonetic}

%%%%%%%%%% 矜 %%%%%%%%%%
\subsection*{矜}\addcontentsline{loh}{figure}{矜 \dpy{jin1}}

\begin{EntryWithPhonetic}{矜}{jin1}{9}{⽭}
  \definition{adj.}{presunçoso; vaidoso | contido; reservado; determinado}
  \definition{v.}{ter pena; simpatizar com; compadecer-se}
\end{EntryWithPhonetic}

%%%%%%%%%% 筋 %%%%%%%%%%
\subsection*{筋}\addcontentsline{loh}{figure}{筋 \dpy{jin1}}

\begin{EntryWithPhonetic}{筋}{jin1}{12}{⽵}[HSK 7-9]
  \definition*{s.}{Sobrenome: Jin}
  \definition[根,条]{s.}{músculo; Coloquial: tendão; ligamento; Coloquial: veias salientes sob a pele; qualquer coisa que se assemelhe a um tendão ou veia}
  \seealsoref{筋儿}{jin1r5}
\end{EntryWithPhonetic}

\begin{EntryWithPhonetic}{筋儿}{jin1r5}{12,2}{⽵,⼉}
  \definition{s.}{Coloquial: tendão; ligamento | qualquer coisa que se assemelhe a um tendão ou veia}
  \seealsoref{筋}{jin1}
\end{EntryWithPhonetic}

%%%%%%%%%% 禁 %%%%%%%%%%
\subsection*{禁}\addcontentsline{loh}{figure}{禁 \dpy{jin1}}

\begin{EntryWithPhonetic}{禁不住}{jin1bu5zhu4}{13,4,7}{⽰,⼀,⼈}[HSK 7-9]
  \definition{v.}{ser incapaz de resistir; ser incapaz de suportar ou aguentar; aplicado tanto a pessoas quanto a coisas | não consigo evitar (fazer algo); não consigo me conter; incapaz de suprimir, incontrolável; aplica-se apenas a humanos}
\end{EntryWithPhonetic}

%%%%%%%%%% 仅 %%%%%%%%%%
\subsection*{仅}\addcontentsline{loh}{figure}{仅 \dpy{jin3}}

\begin{EntryWithPhonetic}{仅}{jin3}{4}{⼈}[HSK 3]
  \definition{adv.}{somente; meramente; por muito pouco}
\end{EntryWithPhonetic}

\begin{EntryWithPhonetic}{仅此而已}{jin3ci3'er2yi3}{4,6,6,3}{⼈,⽌,⽽,⼰}
  \definition{adv.}{apenas isso e nada mais | isso é tudo}
  \antonymref{不计其数}{bu2 ji4 qi2 shu4}
\end{EntryWithPhonetic}

\begin{EntryWithPhonetic}{仅次于}{jin3 ci4 yu2}{4,6,3}{⼈,⽋,⼆}[HSK 7-9]
  \definition{adv.}{(em segundo lugar) precedido apenas por\dots; quanto menor o nível, mais tarde a ordem}
  \synonymref{差不多}{cha4bu5duo1}
\end{EntryWithPhonetic}

\begin{EntryWithPhonetic}{仅仅}{jin3jin3}{4,4}{⼈,⼈}[HSK 3]
  \definition{adv.}{somente; meramente; por muito pouco; indica que está limitado a um determinado âmbito}
  \synonymref{只是}{zhi3shi4}
  \antonymref{处处}{chu4chu4}
  \antonymref{甚至}{shen4zhi4}
  \antonymref{统统}{tong3tong3}
\end{EntryWithPhonetic}

%%%%%%%%%% 尽 %%%%%%%%%%
\subsection*{尽}\addcontentsline{loh}{figure}{尽 \dpy{jin3}}

\begin{EntryWithPhonetic}{尽}{jin3}{6}{⼫}[HSK 7-9]
  \definition{adv.}{na maior extensão possível | na extremidade mais distante de | usado antes de palavras que indicam direção, o mesmo que 最 | de agora em diante}
  \definition{prep.}{dentro dos limites de}
  \definition{v.}{dar prioridade a; deixar que certas pessoas ou coisas tenham precedência}
  \seeref{jin4}
  \seealsoref{最}{zui4}
\end{EntryWithPhonetic}

\begin{EntryWithPhonetic}{尽管}{jin3guan3}{6,14}{⼫,⽵}[HSK 5]
  \definition{adv.}{justo; livremente; faça o que quiser, não se preocupe, não há restrições de movimento ou comportamento}
  \definition{conj.}{no entanto; embora; apesar de ; normalmente usado no início de uma frase anterior para introduzir um fato, seguido de 但是, etc. para introduzir um resultado que o fato não deveria ter; às vezes, também pode ser usado no início de uma frase posterior.}
  \seealsoref{但是}{dan4shi4}
\end{EntryWithPhonetic}

\begin{EntryWithPhonetic}{尽可能}{jin3ke3neng2}{6,5,10}{⼫,⼝,⾁}[HSK 5]
  \definition{adv.}{na medida do possível; com o melhor de sua capacidade; tentar fazer algo, atingir um determinado nível ou extensão}
\end{EntryWithPhonetic}

\begin{EntryWithPhonetic}{尽快}{jin3kuai4}{6,7}{⼫,⼼}[HSK 4]
  \definition{adv.}{com toda a velocidade; o mais rápido possível; o mais breve possível}
\end{EntryWithPhonetic}

\begin{EntryWithPhonetic}{尽量}{jin3liang4}{6,12}{⼫,⾥}[HSK 3]
  \definition{adv.}{tanto quanto possível; da melhor maneira possível}
\end{EntryWithPhonetic}

\begin{EntryWithPhonetic}{尽早}{jin3zao3}{6,6}{⼫,⽇}[HSK 7-9]
  \definition{adv.}{o mais cedo possível; assim que possível; indica que deve ser feito o mais cedo possível}
\end{EntryWithPhonetic}

%%%%%%%%%% 紧 %%%%%%%%%%
\subsection*{紧}\addcontentsline{loh}{figure}{紧 \dpy{jin3}}

\begin{EntryWithPhonetic}{紧}{jin3}{10}{⽷}[HSK 3]
  \definition{adj.}{tenso; apertado; o estado em que um objeto se encontra após ser submetido a uma grande força de tração ou pressão.| seguro; firme | cerrado; apertado | urgente; premente; tenso | rigoroso; rígido; severo | difícil; sem dinheiro}
  \definition{v.}{apertar; tornar mais apertado}
\end{EntryWithPhonetic}

\begin{EntryWithPhonetic}{紧凑}{jin3cou4}{10,11}{⽷,⼎}[HSK 7-9]
  \definition{adj.}{compacto; conciso; bem estruturado; rígido; sucinto}
\end{EntryWithPhonetic}

\begin{EntryWithPhonetic}{紧急}{jin3ji2}{10,9}{⽷,⼼}[HSK 3]
  \definition{adj./adj.}{urgente; premente; crítico}
\end{EntryWithPhonetic}

\begin{EntryWithPhonetic}{紧接着}{jin3 jie1zhe5}{10,11,11}{⽷,⼿,⽬}[HSK 7-9]
  \definition{expr.}{imediatamente depois; uma coisa aconteceu após a outra}
\end{EntryWithPhonetic}

\begin{EntryWithPhonetic}{紧紧}{jin3jin3}{10,10}{⽷,⽷}[HSK 5]
  \definition{adv.}{firmemente; estreitamente; apertadamente; prestar muita atenção (em algo)}
\end{EntryWithPhonetic}

\begin{EntryWithPhonetic}{紧密}{jin3mi4}{10,11}{⽷,⼧}[HSK 4]
  \definition{adj.}{próximos; inseparáveis | incessante; rápido e intenso}
\end{EntryWithPhonetic}

\begin{EntryWithPhonetic}{紧迫}{jin3po4}{10,8}{⽷,⾡}[HSK 7-9]
  \definition{adj.}{urgente; premente; iminente; sem margem para manobras}
\end{EntryWithPhonetic}

\begin{EntryWithPhonetic}{紧缺}{jin3que1}{10,10}{⽷,⽸}[HSK 7-9]
  \definition{adj.}{em falta; extremamente necessário | escasso}
\end{EntryWithPhonetic}

\begin{EntryWithPhonetic}{紧缩}{jin3suo1}{10,14}{⽷,⽷}[HSK 7-9]
  \definition{v.}{reduzir; cortar; desmantelar; encolher}
\end{EntryWithPhonetic}

\begin{EntryWithPhonetic}{紧张}{jin3zhang1}{10,7}{⽷,⼸}[HSK 3]
  \definition{adj.}{nervoso; tenso; mentalmente em estado de alerta, excitado e inquieto | apertado; em falta; o que está disponível não satisfaz os requisitos| tenso; intenso; intenso ou urgente, causando tensão mental}
\end{EntryWithPhonetic}

%%%%%%%%%% 谨 %%%%%%%%%%
\subsection*{谨}\addcontentsline{loh}{figure}{谨 \dpy{jin3}}

\begin{EntryWithPhonetic}{谨}{jin3}{13}{⾔}
  \definition{adj.}{cuidadoso; cauteloso; circunspecto | solene; sincero; respeitoso}
\end{EntryWithPhonetic}

\begin{EntryWithPhonetic}{谨慎}{jin3shen4}{13,13}{⾔,⼼}[HSK 7-9]
  \definition{adj.}{prudente; cuidadoso; cauteloso; circunspecto}
\end{EntryWithPhonetic}

%%%%%%%%%% 锦 %%%%%%%%%%
\subsection*{锦}\addcontentsline{loh}{figure}{锦 \dpy{jin3}}

\begin{EntryWithPhonetic}{锦}{jin3}{13}{⾦}
  \definition*{s.}{Sobrenome: Jin}
  \definition{adj.}{brilhante e bonito (cores brilhantes e lindas)}
  \definition[块]{s.}{brocado; tecidos de seda com padrões coloridos}
\end{EntryWithPhonetic}

\begin{EntryWithPhonetic}{锦旗}{jin3qi2}{13,14}{⾦,⽅}[HSK 7-9]
  \definition{s.}{estandarte de seda (como prêmio ou presente) | flâmula; bandeiras feitas de seda colorida são concedidas aos vencedores de competições ou trabalhos produtivos, ou dadas a grupos ou indivíduos como sinal de respeito ou gratidão}
\end{EntryWithPhonetic}

\begin{EntryWithPhonetic}{锦上添花}{jin3 shang4 tian1 hua1}{13,3,11,7}{⾦,⼀,⽔,⾋}
  \definition{expr.}{adicionar flores ao brocado, tornar o que é bom ainda melhor; melhorar | dourando o lírio}
\end{EntryWithPhonetic}

%%%%%%%%%% 尽 %%%%%%%%%%
\subsection*{尽}\addcontentsline{loh}{figure}{尽 \dpy{jin4}}

\begin{EntryWithPhonetic}{尽}{jin4}{6}{⼫}[HSK 6]
  \definition*{s.}{Sobrenome: Jin}
  \definition{adj.}{exausto; acabado | ao máximo; ao limite | tudo; exaustivo}
  \definition{v.}{esgotar | tentar o seu melhor; fazer o melhor uso possível | morrer; falecer | terminar | chegar ao fim ao máximo; alcançar extremos}
  \seeref{jin3}
\end{EntryWithPhonetic}

\begin{EntryWithPhonetic}{尽力}{jin4/li4}{6,2}{⼫,⼒}[HSK 4]
  \definition{v.+compl.}{esforçar-se ao máximo; esforçar-se ao máximo; usar toda a sua força; fazer algo com seu melhor esforço}
\end{EntryWithPhonetic}

\begin{EntryWithPhonetic}{尽情}{jin4qing2}{6,11}{⼫,⼼}[HSK 7-9]
  \definition{v.}{expressar os próprios sentimentos de forma plena e livre; significa agir de acordo com os próprios sentimentos, na medida do possível}
\end{EntryWithPhonetic}

\begin{EntryWithPhonetic}{尽头}{jin4tou2}{6,5}{⼫,⼤}[HSK 7-9]
  \definition[台]{s.}{fim}[小路尽头是一片树林。===No fim do caminho havia um bosque de árvores.]
\end{EntryWithPhonetic}

%%%%%%%%%% 劲 %%%%%%%%%%
\subsection*{劲}\addcontentsline{loh}{figure}{劲 \dpy{jin4}}

\begin{EntryWithPhonetic}{劲}{jin4}{7}{⼒}
  \definition{s.}{força; energia | vigor; espírito; ímpeto; fervor | ar; modo; expressão | interesse; prazer; entusiasmo}
  \seeref{jing4}
  \seealsoref{劲儿}{jin4r5}
\end{EntryWithPhonetic}

\begin{EntryWithPhonetic}{劲儿}{jin4r5}{7,2}{⼒,⼉}
  \definition{s.}{energia; força}
  \seealsoref{劲}{jin4}
\end{EntryWithPhonetic}

\begin{EntryWithPhonetic}{劲头}{jin4tou2}{7,5}{⼒,⼤}[HSK 7-9]
  \definition{s.}{força; energia; poder | vigor; espírito; ímpeto; fervor}
\end{EntryWithPhonetic}

%%%%%%%%%% 近 %%%%%%%%%%
\subsection*{近}\addcontentsline{loh}{figure}{近 \dpy{jin4}}

\begin{EntryWithPhonetic}{近}{jin4}{7}{⾡}[HSK 2]
  \definition{adj.}{próximo; perto; distância espacial ou temporal curta | íntimo; intimamente relacionado; relação estreita | fácil de entender}
  \antonymref{远}{yuan3}
\end{EntryWithPhonetic}

\begin{EntryWithPhonetic}{近代}{jin4dai4}{7,5}{⾡,⼈}[HSK 4]
  \definition{s.}{tempos modernos; era passada relativamente próxima à era moderna, geralmente referida na história chinesa como 1840 a 1919 | o tempo ou era do capitalismo}
\end{EntryWithPhonetic}

\begin{EntryWithPhonetic}{近来}{jin4lai2}{7,7}{⾡,⽊}[HSK 5]
  \definition{adv.}{ultimamente; recentemente; de tarde; refere"-se a um período de tempo entre o passado imediato e o presente}
\end{EntryWithPhonetic}

\begin{EntryWithPhonetic}{近年来}{jin4nian2 lai2}{7,6,7}{⾡,⼲,⽊}[HSK 7-9]
  \definition{expr.}{nos últimos anos}
\end{EntryWithPhonetic}

\begin{EntryWithPhonetic}{近期}{jin4qi1}{7,12}{⾡,⽉}[HSK 3]
  \definition{adv.}{num futuro próximo; brevemente}
\end{EntryWithPhonetic}

\begin{EntryWithPhonetic}{近日}{jin4ri4}{7,4}{⾡,⽇}[HSK 6]
  \definition{s.}{recentemente; nos últimos dias; apontando para o passado | nos próximos dias; refere"-se ao futuro}
\end{EntryWithPhonetic}

\begin{EntryWithPhonetic}{近视}{jin4shi4}{7,8}{⾡,⾒}[HSK 6]
  \definition{adj.}{miopia; uma deficiência visual em que a visão próxima é clara, mas a visão distante é turva | míope (figurativo); metáfora para miopia}
\end{EntryWithPhonetic}

%%%%%%%%%% 进 %%%%%%%%%%
\subsection*{进}\addcontentsline{loh}{figure}{进 \dpy{jin4}}

\begin{EntryWithPhonetic}{进}{jin4}{7}{⾡}[HSK 1]
  \definition*{s.}{Sobrenome: Jin}
  \definition{clas.}{para seções em um edifício ou complexo residencial; qualquer uma das várias fileiras de casas em um complexo residencial de estilo antigo}
  \definition{s.}{Matemática: base de um sistema numérico}
  \definition{v.}{avançar; ir adiante; seguir em frente; oposto a 退 | entrar; entrar em; entrar ou sair; oposto a 出 | receber | comer; tomar; beber | submeter; apresentar | marcar um gol}
  \definition{v.aux.}{usado após um verbo, significa ``para dentro''}
  \seealsoref{出}{chu1}
  \seealsoref{退}{tui4}
\end{EntryWithPhonetic}

\begin{EntryWithPhonetic}{进步}{jin4bu4}{7,7}{⾡,⽌}[HSK 3]
  \definition{adj.}{progressivo; adequado às tendências da época; que impulsiona o desenvolvimento social}
  \definition{v.}{avançar; progredir; melhorar}
  \antonymref{落后}{luo4/hou4}
\end{EntryWithPhonetic}

\begin{EntryWithPhonetic}{进场}{jin4chang3}{7,6}{⾡,⼟}[HSK 7-9]
  \definition{v.}{entrar na arena; marchar | Aeroespacial: abordagem}
  \antonymref{退场}{tui4chang3}
\end{EntryWithPhonetic}

\begin{EntryWithPhonetic}{进程}{jin4cheng2}{7,12}{⾡,⽲}[HSK 7-9]
  \definition{s.}{processo; progresso; procedimento; o processo de mudança ou progresso das coisas}
\end{EntryWithPhonetic}

\begin{EntryWithPhonetic}{进出}{jin4chu1}{7,5}{⾡,⼐}[HSK 7-9]
  \definition{s.}{(faturamento) empresarial | entrada e saída | recibos e pagamentos}
  \definition{v.}{passar para dentro e para fora;  entrar e sair}
\end{EntryWithPhonetic}

\begin{EntryWithPhonetic}{进出口}{jin4-chu1kou3}{7,5,3}{⾡,⼐,⼝}[HSK 7-9]
  \definition{s.}{importação e exportação | saída; saídas e entradas}
\end{EntryWithPhonetic}

\begin{EntryWithPhonetic}{进度}{jin4du4}{7,9}{⾡,⼴}[HSK 7-9]
  \definition{s.}{taxa de progresso; taxa de avanço; a velocidade com que (o trabalho, o estudo, etc.) progride}
\end{EntryWithPhonetic}

\begin{EntryWithPhonetic}{进而}{jin4'er2}{7,6}{⾡,⽽}[HSK 7-9]
  \definition{conj.}{em seguida; além disso; e então; depois disso; continue em frente; vá mais longe}
\end{EntryWithPhonetic}

\begin{EntryWithPhonetic}{进攻}{jin4gong1}{7,7}{⾡,⽁}[HSK 6]
  \definition{s.}{ofensiva}
  \definition{v.}{atacar; assaltar; tomar a ofensiva}
  \antonymref{防守}{fang2shou3}
\end{EntryWithPhonetic}

\begin{EntryWithPhonetic}{进化}{jin4hua4}{7,4}{⾡,⼔}[HSK 5]
  \definition[个]{s.}{evolução; os organismos se desenvolvem e evoluem do simples para o complexo e de níveis baixos para altos}
  \definition{v.}{evoluir; um termo geral usado para descrever uma mudança gradual para melhor}
\end{EntryWithPhonetic}

\begin{EntryWithPhonetic}{进口}{jin4/kou3}{7,3}{⾡,⼝}[HSK 4]
  \definition{adj.}{importado}
  \definition{s.}{importação; entrada de um edifício ou local, também chamada de 入口}
  \definition{v.+compl.}{importar; comprar ou transportar mercadorias de outro país ou região | entrar no porto; navegar em direção a um porto}
  \seealsoref{入口}{ru4/kou3}
\end{EntryWithPhonetic}

\begin{EntryWithPhonetic}{进来}{jin4 lai5}{7,7}{⾡,⽊}[HSK 1]
  \definition{v.}{entrar (para a minha localização)}
\end{EntryWithPhonetic}

\begin{EntryWithPhonetic}{进去}{jin4 qu5}{7,5}{⾡,⼛}[HSK 1]
  \definition{v.}{entrar (a partir da minha localização)}
  \definition{v.aux.}{usado depois de um verbo, significa ``ir para dentro''; para um determinado intervalo ou período de tempo}
\end{EntryWithPhonetic}

\begin{EntryWithPhonetic}{进入}{jin4ru4}{7,2}{⾡,⼊}[HSK 2]
  \definition{v.}{entrar; entrar em}
\end{EntryWithPhonetic}

\begin{EntryWithPhonetic}{进行}{jin4xing2}{7,6}{⾡,⾏}[HSK 2]
  \definition{v.}{continuar; estar em andamento; estar em progresso | fazer; conduzir; realizar; executar | marchar; avançar; prosseguir; estar em marcha}
\end{EntryWithPhonetic}

\begin{EntryWithPhonetic}{进行编程}{jin4xing2bian1cheng2}{7,6,12,12}{⾡,⾏,⽷,⽲}
  \definition{s.}{programa de computador executável}
\end{EntryWithPhonetic}

\begin{EntryWithPhonetic}{进修}{jin4xiu1}{7,9}{⾡,⼈}[HSK 7-9]
  \definition{v.}{participar de estudos avançados; fazer um curso de atualização; envolver-se em estudos avançados; aprimorar as competências profissionais com formação complementar}
\end{EntryWithPhonetic}

\begin{EntryWithPhonetic}{进一步}{jin4yi2bu4}{7,1,7}{⾡,⼀,⽌}[HSK 3]
  \definition{adv.}{mais; dar um passo adiante; avançar um passo; indica que as coisas estão progredindo em um nível mais alto do que antes}
\end{EntryWithPhonetic}

\begin{EntryWithPhonetic}{进展}{jin4zhan3}{7,10}{⾡,⼫}[HSK 3]
  \definition{v.}{fazer progresso; progredir; avançar no desenvolvimento}
\end{EntryWithPhonetic}

%%%%%%%%%% 晋 %%%%%%%%%%
\subsection*{晋}\addcontentsline{loh}{figure}{晋 \dpy{jin4}}

\begin{EntryWithPhonetic}{晋}{jin4}{10}{⽇}
  \definition*{s.}{Estado da Dinastia Zhou (1046-256 a.C.), ocupando partes do que hoje são Shanxi, Shaanxi, Hebei e Henan | Dinastia Jin Ocidental (265-316), Dinastia Jin Oriental (317-420) e Dinastia Jin Posterior (936-946) | Nome abreviado da província de Shanxi: 山西 | Sobrenome: Jin}
  \definition{v.}{avançar | promover}
  \seealsoref{山西}{shan1xi1}
\end{EntryWithPhonetic}

\begin{EntryWithPhonetic}{晋升}{jin4sheng1}{10,4}{⽇,⼗}[HSK 7-9]
  \definition{v.}{elevar; promover (a um cargo superior)}
\end{EntryWithPhonetic}

%%%%%%%%%% 浸 %%%%%%%%%%
\subsection*{浸}\addcontentsline{loh}{figure}{浸 \dpy{jin4}}

\begin{EntryWithPhonetic}{浸}{jin4}{10}{⽔}
  \definition{adv.}{gradualmente; passo a passo; pouco a pouco | Literário: gradualmente; cada vez mais}
  \definition{v.}{deixar de molho; imergir; mergulhar | saturar}
\end{EntryWithPhonetic}

\begin{EntryWithPhonetic}{浸泡}{jin4pao4}{10,8}{⽔,⽔}[HSK 7-9]
  \definition{v.}{mergulhar; banhar; imergir; deixar de molho em líquido}
\end{EntryWithPhonetic}

%%%%%%%%%% 禁 %%%%%%%%%%
\subsection*{禁}\addcontentsline{loh}{figure}{禁 \dpy{jin4}}

\begin{EntryWithPhonetic}{禁}{jin4}{13}{⽰}
  \definition*{s.}{Sobrenome: Jin}
  \definition{s.}{um tabu; assuntos não permitidos por lei ou costume | área proibida | residência real; o lugar onde o imperador viveu nos tempos antigos}
  \definition{v.}{proibir; banir | aprisionar; deter}
\end{EntryWithPhonetic}

\begin{EntryWithPhonetic}{禁忌}{jin4ji4}{13,7}{⽰,⼼}[HSK 7-9]
  \definition[个]{s.}{tabu}
  \definition{v.}{evitar; abster-se de; contraindicar; evitar certos comportamentos geralmente se refere à abstinência de certos alimentos ou medicamentos}
\end{EntryWithPhonetic}

\begin{EntryWithPhonetic}{禁区}{jin4qu1}{13,4}{⽰,⼖}[HSK 7-9]
  \definition{s.}{zona proibida; zona restrita; faixa proibida | reserva (de vida selvagem ou vegetal); parque natural | (futebol) área de penalidade, grande área; (basquetebol) área restrita | área fora dos limites; áreas que são proibidas}
\end{EntryWithPhonetic}

\begin{EntryWithPhonetic}{禁止}{jin4zhi3}{13,4}{⽰,⽌}[HSK 4]
  \definition{v.}{banir; proibir; interditar}
\end{EntryWithPhonetic}

%%%%%%%%%% 京 %%%%%%%%%%
\subsection*{京}\addcontentsline{loh}{figure}{京 \dpy{jing1}}

\begin{EntryWithPhonetic}{京}{jing1}{8}{⼇}
  \definition*{s.}{Pequim (Beijing), abreviação de 北京 | Sobrenome: Jing}
  \definition{num.}{dez milhões (um numeral antigo); 10.000.000; 1000.0000}
  \definition{s.}{capital de um país}
  \seealsoref{北京}{bei3jing1}
\end{EntryWithPhonetic}

\begin{EntryWithPhonetic}{京二胡}{jing1'er4hu2}{8,2,9}{⼇,⼆,⾁}
  \definition{s.}{um tipo de violino chinês semelhante ao 二胡 de duas cordas, usado principalmente para acompanhamento do canto da ópera de Pequim | também chamado de 京胡 | jing'erhu, um violino de duas cordas, intermediário em tamanho e tom entre o 京胡 e o 二胡, usado para acompanhar a ópera chinesa}
  \seealsoref{二胡}{er4hu2}
  \seealsoref{京胡}{jing1hu2}
\end{EntryWithPhonetic}

\begin{EntryWithPhonetic}{京胡}{jing1hu2}{8,9}{⼇,⾁}
  \definition{s.}{jinghu, um instrumento de arco de duas cordas com registro agudo; violino da ópera de Pequim | também chamado de 京二胡 | jinghu, um 二胡 (violino de duas cordas) menor e mais agudo, usado para acompanhar a ópera chinesa}
  \seealsoref{二胡}{er4hu2}
  \seealsoref{胡琴}{hu2qin2}
  \seealsoref{京二胡}{jing1'er4hu2}
\end{EntryWithPhonetic}

\begin{EntryWithPhonetic}{京剧}{jing1ju4}{8,10}{⼇,⼑}[HSK 3]
  \definition*[场,段]{s.}{Ópera de Pequim}
  \synonymref{戏剧}{xi4ju4}
\end{EntryWithPhonetic}

%%%%%%%%%% 经 %%%%%%%%%%
\subsection*{经}\addcontentsline{loh}{figure}{经 \dpy{jing1}}

\begin{EntryWithPhonetic}{经}{jing1}{8}{⽷}[HSK 7-9]
  \definition*{s.}{Sobrenome: Jing}
  \definition{adj.}{constante; regular}
  \definition{prep.}{como resultado de; depois; através de}
  \definition{s.}{urdidura, os fios longitudinais de um tecido | Medicina chinesa: canais principais e colaterais | Geografia: longitude | escritura; sutra; cânone; clássico | menstruação}
  \definition{v.}{Literário: gerenciar; lidar com; envolver-se em | enforcar-se | suportar; ficar de pé; aguentar; resistir | passar por; sofrer; experimentar}
  \seeref{jing4}
  \antonymref{纬}{wei3}
\end{EntryWithPhonetic}

\begin{EntryWithPhonetic}{经常}{jing1chang2}{8,11}{⽷,⼱}[HSK 2]
  \definition{adj.}{habitual; cotidiano; diário; do dia a dia}
  \definition{adv.}{frequentemente; regularmente; constantemente; com frequência; indica que a ação ocorre repetidamente}
\end{EntryWithPhonetic}

\begin{EntryWithPhonetic}{经典}{jing1dian3}{8,8}{⽷,⼋}[HSK 4]
  \definition{adj.}{clássico; (escritos ou obras, etc.) que são típicos, autorizados}
  \definition{s.}{clássicos; escritos tradicionais e valiosos; os livros mais importantes e fundamentais da religião | escrituras; escritos de doutrinas religiosas}
\end{EntryWithPhonetic}

\begin{EntryWithPhonetic}{经度}{jing1du4}{8,9}{⽷,⼴}[HSK 7-9]
  \definition{s.}{longitude}
\end{EntryWithPhonetic}

\begin{EntryWithPhonetic}{经费}{jing1fei4}{8,9}{⽷,⾙}[HSK 5]
  \definition[笔]{s.}{fundos; desembolso; gastos | despesas; gastos}
\end{EntryWithPhonetic}

\begin{EntryWithPhonetic}{经过}{jing1guo4}{8,6}{⽷,⾡}[HSK 2]
  \definition{prep.}{depois; através; como resultado de; passar por uma atividade ou evento que traz novas mudanças para pessoas ou coisas}
  \definition[个,段,番]{s.}{processo; curso; experiência}
  \definition{v.}{passar; atravessar; passar por; através de (local, tempo, ação, etc.)}
\end{EntryWithPhonetic}

\begin{EntryWithPhonetic}{经济}{jing1ji4}{8,9}{⽷,⽔}[HSK 3]
  \definition{adj.}{econômico; parcimonioso; descreve algo que custa pouco e rende muito; preço acessível}
  \definition{s.}{economia; a soma das relações de produção social em um determinado período histórico|econômico; de valor industrial ou econômico; refere"-se à economia nacional; também se refere a um determinado setor da economia nacional | economia; refere"-se às atividades econômicas, incluindo produção, circulação, distribuição e consumo, bem como atividades ou processos financeiros, de seguros, etc. | renda; situação financeira; refere"-se à situação financeira de uma pessoa}
  \definition{v.}{governar o país e beneficiar o povo}
\end{EntryWithPhonetic}

\begin{EntryWithPhonetic}{经久不息}{jing1jiu3-bu4xi1}{8,3,4,10}{⽷,⼃,⼀,⼼}[HSK 7-9]
  \definition{expr.}{prolongado; duradouro}
\end{EntryWithPhonetic}

\begin{EntryWithPhonetic}{经理}{jing1li3}{8,11}{⽷,⽟}[HSK 2]
  \definition[个,位,名]{s.}{gerente; diretor; pessoas responsáveis pela gestão e administração de empresas ou corporações}
\end{EntryWithPhonetic}

\begin{EntryWithPhonetic}{经历}{jing1li4}{8,4}{⽷,⼚}[HSK 3]
  \definition[个,次,段,种]{s.}{experiência; coisas que você viu, fez ou sofreu pessoalmente}
  \definition{v.}{passar por; atravessar; ter visto, feito ou sofrido pessoalmente}
\end{EntryWithPhonetic}

\begin{EntryWithPhonetic}{经贸}{jing1mao4}{8,9}{⽷,⾙}[HSK 7-9]
  \definition{s.}{economia e comércio; o termo coletivo para economia e comércio}
\end{EntryWithPhonetic}

\begin{EntryWithPhonetic}{经商}{jing1/shang1}{8,11}{⽷,⼝}[HSK 7-9]
  \definition{v.+compl.}{dedicar-se ao comércio; estar em atividade comercial; envolver-se em atividades comerciais}
\end{EntryWithPhonetic}

\begin{EntryWithPhonetic}{经受}{jing1shou4}{8,8}{⽷,⼜}[HSK 7-9]
  \definition{v.}{experimentar; suportar; resistir; aguentar; passar por}
\end{EntryWithPhonetic}

\begin{EntryWithPhonetic}{经线}{jing1xian4}{8,8}{⽷,⽷}
  \definition{s.}{urdidura | Geografia: meridiano | linha de longitude}
\end{EntryWithPhonetic}

\begin{EntryWithPhonetic}{经验}{jing1yan4}{8,10}{⽷,⾺}[HSK 3]
  \definition[个,次,种]{s.}{experiência; conhecimento ou habilidades adquiridos através da prática}
  \definition{v.}{experimentar; passar por; ter visto, feito ou sofrido pessoalmente}
\end{EntryWithPhonetic}

\begin{EntryWithPhonetic}{经营}{jing1ying2}{8,11}{⽷,⾋}[HSK 3]
  \definition{v.}{executar; gerenciar; operar; envolver"-se em; planejar e gerenciar (empresas, etc.) | gerenciar; refere"-se a planos e organizações em geral}
\end{EntryWithPhonetic}

%%%%%%%%%% 茎 %%%%%%%%%%
\subsection*{茎}\addcontentsline{loh}{figure}{茎 \dpy{jing1}}

\begin{EntryWithPhonetic}{茎}{jing1}{8}{⾋}[HSK 7-9]
  \definition{clas.}{Literário: utilizado como classificador para indicar um objeto em forma de tira}[几茎小草。===Algumas folhas de grama.]
  \definition[根]{s.}{caule (de uma planta); talo; tronco | algo como um caule ou haste}
\end{EntryWithPhonetic}

%%%%%%%%%% 荆 %%%%%%%%%%
\subsection*{荆}\addcontentsline{loh}{figure}{荆 \dpy{jing1}}

\begin{EntryWithPhonetic}{荆}{jing1}{9}{⾋}
  \definition*{s.}{Sobrenome: Jing}
  \definition{s.}{árvore-da-castidade; vitex | uma vara para açoitar; varas de punição antigas feitas de galhos de espinho | a própria esposa; uma forma humilde de se referir à esposa antigamente}
\end{EntryWithPhonetic}

\begin{EntryWithPhonetic}{荆棘}{jing1ji2}{9,12}{⾋,⽊}[HSK 7-9]
  \definition{s.}{silvas; cardos e espinhos; vegetação rasteira espinhosa; geralmente se refere a arbustos espinhosos que crescem nas montanhas e nos campos}
\end{EntryWithPhonetic}

%%%%%%%%%% 惊 %%%%%%%%%%
\subsection*{惊}\addcontentsline{loh}{figure}{惊 \dpy{jing1}}

\begin{EntryWithPhonetic}{惊}{jing1}{11}{⼼}[HSK 7-9]
  \definition{v.}{assustar; ficar assustado; ficar nervoso devido a estímulo repentino; ficar com medo | surpreender; chocar; alarmar}
\end{EntryWithPhonetic}

\begin{EntryWithPhonetic}{惊诧}{jing1cha4}{11,8}{⼼,⾔}[HSK 7-9]
  \definition{adj.}{surpreso; admirado; estupefato | muito surpreso}
\end{EntryWithPhonetic}

\begin{EntryWithPhonetic}{惊呆}{jing1dai1}{11,7}{⼼,⼝}
  \definition{adj.}{atordoado; estupefato; chocado}
\end{EntryWithPhonetic}

\begin{EntryWithPhonetic}{惊慌}{jing1huang1}{11,12}{⼼,⼼}[HSK 7-9]
  \definition{adj.}{assustado; alarmado; amedrontado; atemorizado; em pânico}
\end{EntryWithPhonetic}

\begin{EntryWithPhonetic}{惊慌失措}{jing1huang1-shi1cuo4}{11,12,5,11}{⼼,⼼,⼤,⼿}[HSK 7-9]
  \definition{expr.}{apavorado; tomado pelo pânico; em pânico; perder a cabeça de medo; no Capítulo 11 da Parte 4 de ``O Oriente'', de Wei Wei: ``Guo Xiang e seus homens lançaram um ataque feroz, e o inimigo, em pânico, esqueceu-se de resistir, preocupando-se apenas em subir nos tanques.''; confusão aterrorizante; ficar apavorado; ficar em pânico}
\end{EntryWithPhonetic}

\begin{EntryWithPhonetic}{惊奇}{jing1qi2}{11,8}{⼼,⼤}[HSK 7-9]
  \definition{v.}{maravilhar-se; ficar surpreso; ficar boquiaberto}
\end{EntryWithPhonetic}

\begin{EntryWithPhonetic}{惊人}{jing1ren2}{11,2}{⼼,⼈}[HSK 6]
  \definition{adj.}{surpreso; espantado; atônito; surpreendente}
\end{EntryWithPhonetic}

\begin{EntryWithPhonetic}{惊叹}{jing1tan4}{11,5}{⼼,⼝}[HSK 7-9]
  \definition{v.}{maravilhar-se com; admirar-se com; exclamar (com admiração)}
\end{EntryWithPhonetic}

\begin{EntryWithPhonetic}{惊天动地}{jing1tian1-dong4di4}{11,4,6,6}{⼼,⼤,⼒,⼟}[HSK 7-9]
  \definition{expr.}{que abala os céus e a terra; que faz a terra tremer; assustar os céus e mover a terra; sacudir os céus e assustar a terra; que abala o mundo; de proporções sísmicas; de impacto mundial}
\end{EntryWithPhonetic}

\begin{EntryWithPhonetic}{惊喜}{jing1xi3}{11,12}{⼼,⼝}[HSK 6]
  \definition{s.}{boa surpresa; agradavelmente surpreso}
\end{EntryWithPhonetic}

\begin{EntryWithPhonetic}{惊险}{jing1xian3}{11,9}{⼼,⾩}[HSK 7-9]
  \definition{adj.}{emocionante; de tirar o fôlego; alarmantemente perigoso}
\end{EntryWithPhonetic}

\begin{EntryWithPhonetic}{惊心动魄}{jing1xin1-dong4po4}{11,4,6,14}{⼼,⼼,⼒,⿁}[HSK 7-9]
  \definition{expr.}{comovente; profundamente impactante; ficar apavorado (horror); de tirar o fôlego; de arrepiar os cabelos; emocionante; fazer o coração de alguém disparar; abalar alguém profundamente; deixar alguém sem fôlego}
\end{EntryWithPhonetic}

\begin{EntryWithPhonetic}{惊醒}{jing1xing3}{11,16}{⼼,⾣}[HSK 7-9]
  \definition{v.}{acordar sobressaltado; despertado pelo susto; despertado pelo choque}
\end{EntryWithPhonetic}

\begin{EntryWithPhonetic}{惊讶}{jing1ya4}{11,6}{⼼,⾔}[HSK 7-9]
  \definition{adj.}{surpreso; admirado; estupefato; atônito; sentindo-me surpreso e estranho}
\end{EntryWithPhonetic}

\begin{EntryWithPhonetic}{惊异}{jing1yi4}{11,6}{⼼,⼶}
  \definition{adj.}{surpreso; admirado; estupefato; atônito}
\end{EntryWithPhonetic}

%%%%%%%%%% 晶 %%%%%%%%%%
\subsection*{晶}\addcontentsline{loh}{figure}{晶 \dpy{jing1}}

\begin{EntryWithPhonetic}{晶}{jing1}{12}{⽇}
  \definition{adj.}{brilhante; cristalino}
  \definition{s.}{cristal de rocha; cristal; quartzo}
\end{EntryWithPhonetic}

\begin{EntryWithPhonetic}{晶莹}{jing1ying2}{12,10}{⽇,⾋}[HSK 7-9]
  \definition{adj.}{brilhante e cristalino; cintilante e translúcido; brilhante e transparente}
\end{EntryWithPhonetic}

%%%%%%%%%% 兢 %%%%%%%%%%
\subsection*{兢}\addcontentsline{loh}{figure}{兢 \dpy{jing1}}

\begin{EntryWithPhonetic}{兢}{jing1}{14}{⼉}
  \definition{adj.}{medroso | cauteloso | forte}
  \definition{v.}{mover}
\end{EntryWithPhonetic}

\begin{EntryWithPhonetic}{兢兢业业}{jing1jing1ye4ye4}{14,14,5,5}{⼉,⼉,⼀,⼀}[HSK 7-9]
  \definition{expr.}{cauteloso e consciencioso; zeloso; aplicado; descreve alguém que é muito cuidadoso, cauteloso e responsável ao fazer as coisas}
\end{EntryWithPhonetic}

%%%%%%%%%% 精 %%%%%%%%%%
\subsection*{精}\addcontentsline{loh}{figure}{精 \dpy{jing1}}

\begin{EntryWithPhonetic}{精}{jing1}{14}{⽶}[HSK 6]
  \definition{adj.}{refinado; escolhido; purificado ou selecionado | perfeito; excelente; melhor | fino; preciso; meticuloso | inteligente; astuto; esperto | habilidoso; versado; proficiente}
  \definition{adv.}{muito; extremamente; antes de certos adjetivos, significa 十分 ou 非常}
  \definition{s.}{extrato; essência; essência refinada ou selecionada; extraída | energia; espírito | semente; esperma; sêmen | \emph{goblin}; espírito; elfo; demônio}
  \seealsoref{非常}{fei1chang2}
  \seealsoref{十分}{shi2fen1}
  \antonymref{粗}{cu1}
\end{EntryWithPhonetic}

\begin{EntryWithPhonetic}{精彩}{jing1cai3}{14,11}{⽶,⼺}[HSK 3]
  \definition{adj.}{brilhante; esplêndido; maravilhoso}
\end{EntryWithPhonetic}

\begin{EntryWithPhonetic}{精打细算}{jing1da3-xi4suan4}{14,5,8,14}{⽶,⼿,⽷,⽵}[HSK 7-9]
  \definition{expr.}{seja muito cuidadoso nos cálculos; contagem precisa; seja preciso nos cálculos; faça um orçamento rigoroso; cálculo cuidadoso e detalhado (meticuloso); cálculo cuidadoso e orçamento rigoroso; conte cada centavo e faça cada centavo valer a pena; corte os gastos com precisão; planejar com meticulosidade e cuidado, isso significa calcular com precisão o uso de mão de obra, recursos materiais e recursos financeiros para evitar desperdícios}
\end{EntryWithPhonetic}

\begin{EntryWithPhonetic}{精华}{jing1hua2}{14,6}{⽶,⼗}[HSK 7-9]
  \definition{s.}{elite; creme; escolha; essência; quintessência; a melhor e mais refinada parte de tudo | glória; esplendor; brilho; luz (do Sol e da Lua)}
\end{EntryWithPhonetic}

\begin{EntryWithPhonetic}{精简}{jing1jian3}{14,13}{⽶,⽵}[HSK 7-9]
  \definition{v.}{reduzir; simplificar; cortar; simplificar; eliminar o desnecessário e conservar o necessário}
\end{EntryWithPhonetic}

\begin{EntryWithPhonetic}{精力}{jing1li4}{14,2}{⽶,⼒}[HSK 4]
  \definition[些]{s.}{energia; vigor; força mental e física}
\end{EntryWithPhonetic}

\begin{EntryWithPhonetic}{精练}{jing1lian4}{14,8}{⽶,⽷}[HSK 7-9]
  \definition{adj.}{conciso; sucinto; lacônico | refinado; palavras e frases redundantes eliminadas}
  \definition{v.}{praticar intensivamente}
  \synonymref{干脆}{gan1cui4}
  \synonymref{简单}{jian3dan1}
  \synonymref{简洁}{jian3jie2}
  \synonymref{精炼}{jing1lian4}
  \antonymref{冗长}{rong3chang2}
\end{EntryWithPhonetic}

\begin{EntryWithPhonetic}{精炼}{jing1lian4}{14,9}{⽶,⽕}
  \definition{adj.}{conciso; sucinto; lacônico}
  \definition{s.}{refino; remoção de impurezas}
  \definition{v.}{Metalurgia: refinar; purificar; fundir}
\end{EntryWithPhonetic}

\begin{EntryWithPhonetic}{精灵}{jing1ling2}{14,7}{⽶,⽕}
  \definition{s.}{espírito | fada | elfo | duende | gênio}
\end{EntryWithPhonetic}

\begin{EntryWithPhonetic}{精美}{jing1mei3}{14,9}{⽶,⽺}[HSK 6]
  \definition{adj.}{elegante; requintado}
\end{EntryWithPhonetic}

\begin{EntryWithPhonetic}{精密}{jing1mi4}{14,11}{⽶,⼧}
  \definition{adj.}{preciso; preciso e meticuloso}
\end{EntryWithPhonetic}

\begin{EntryWithPhonetic}{精妙}{jing1miao4}{14,7}{⽶,⼥}[HSK 7-9]
  \definition{adj.}{requintado | fino e delicado (geralmente de obras de arte)}
\end{EntryWithPhonetic}

\begin{EntryWithPhonetic}{精明}{jing1ming2}{14,8}{⽶,⽇}[HSK 7-9]
  \definition{adj.}{astuto; sagaz; perspicaz; inteligente e brilhante}
\end{EntryWithPhonetic}

\begin{EntryWithPhonetic}{精疲力竭}{jing1pi2-li4jie2}{14,10,2,14}{⽶,⽧,⼒,⽴}[HSK 7-9]
  \definition{expr.}{esgotado; exausto; desgastado; descrevendo fadiga extrema e completa falta de energia}
\end{EntryWithPhonetic}

\begin{EntryWithPhonetic}{精品}{jing1pin3}{14,9}{⽶,⼝}[HSK 6]
  \definition[个]{s.}{belas obras (de arte); objetos de arte | produtos de qualidade; artigos de excelente qualidade; produto \emph{premium}}
\end{EntryWithPhonetic}

\begin{EntryWithPhonetic}{精确}{jing1que4}{14,12}{⽶,⽯}[HSK 7-9]
  \definition{adj.}{exato; preciso; acurado; muito preciso e correto}
\end{EntryWithPhonetic}

\begin{EntryWithPhonetic}{精神}{jing1shen2}{14,9}{⽶,⽰}[HSK 3]
  \definition[种,个,类,股]{s.}{espírito; mente; estado mental; refere"-se à consciência, às atividades mentais e ao estado psicológico geral de uma pessoa | substância; espírito; essência; propósito; significado principal}
  \seeref{jing1shen5}
\end{EntryWithPhonetic}

\begin{EntryWithPhonetic}{精神病}{jing1shen2bing4}{14,9,10}{⽶,⽰,⽧}[HSK 7-9]
  \definition{s.}{doença mental; transtorno mental; psicose}[这是幻想型精神病的体现。===Isso é uma manifestação de psicose delirante.]
\end{EntryWithPhonetic}

\begin{EntryWithPhonetic}{精神}{jing1shen5}{14,9}{⽶,⽰}[HSK 3]
  \definition{adj.}{animado; espirituoso; vigoroso; descreve uma pessoa como cheia de energia | muito bonito; boa aparência, bom físico}
  \definition[种,个,类,股]{s.}{impulso; vigor; vitalidade}
  \seeref{jing1shen2}
\end{EntryWithPhonetic}

\begin{EntryWithPhonetic}{精髓}{jing1sui3}{14,21}{⽶,⾻}[HSK 7-9]
  \definition{s.}{medula; medula óssea; quintessência; essência metafórica das coisas}
\end{EntryWithPhonetic}

\begin{EntryWithPhonetic}{精通}{jing1tong1}{14,10}{⽶,⾡}[HSK 7-9]
  \definition{v.}{dominar; ser proficiente em; ter um bom domínio de; ter um profundo entendimento e conhecimento abrangente de uma área específica de estudo, tecnologia ou negócios}
\end{EntryWithPhonetic}

\begin{EntryWithPhonetic}{精细}{jing1xi4}{14,8}{⽶,⽷}[HSK 7-9]
  \definition{adj.}{fino; cuidadoso; meticuloso; muito delicado | astuto; perspicaz e cuidadoso; muito meticuloso}
\end{EntryWithPhonetic}

\begin{EntryWithPhonetic}{精心}{jing1xin1}{14,4}{⽶,⼼}[HSK 7-9]
  \definition{adv.}{meticulosamente; cuidadosamente; elaboradamente; preste muita atenção; concentre-se totalmente}[他们精心设计了这个项目。===Eles planejaram este projeto meticulosamente.]
\end{EntryWithPhonetic}

\begin{EntryWithPhonetic}{精益求精}{jing1yi4qiu2jing1}{14,10,7,14}{⽶,⽫,⽔,⽶}[HSK 7-9]
  \definition{expr.}{``Busque a excelência.''; esforçar-se pela perfeição; buscar a melhoria constante; perseguir a excelência; almejar a perfeição; já está muito bom, mas você ainda quer que fique ainda melhor; melhorar algo constantemente; continuar melhorando}
\end{EntryWithPhonetic}

\begin{EntryWithPhonetic}{精英}{jing1ying1}{14,8}{⽶,⾋}[HSK 7-9]
  \definition{s.}{creme; essência; quintessência | escolhido; elite; pessoa de habilidade excepcional}
\end{EntryWithPhonetic}

\begin{EntryWithPhonetic}{精致}{jing1zhi4}{14,10}{⽶,⾄}[HSK 7-9]
  \definition{adj.}{fino; requintado; delicado}[我们欣赏她精致的手工艺品。===Admiramos seu trabalho artesanal requintado.]
\end{EntryWithPhonetic}

\begin{EntryWithPhonetic}{精子}{jing1zi3}{14,3}{⽶,⼦}
  \definition{s.}{espermatozoide; célula germinativa}
\end{EntryWithPhonetic}

%%%%%%%%%% 鲸 %%%%%%%%%%
\subsection*{鲸}\addcontentsline{loh}{figure}{鲸 \dpy{jing1}}

\begin{EntryWithPhonetic}{鲸}{jing1}{16}{⿂}
  \definition[头,只,条]{s.}{baleia; cetáceo}
\end{EntryWithPhonetic}

\begin{EntryWithPhonetic}{鲸鲨}{jing1sha1}{16,15}{⿂,⿂}
  \definition{s.}{tubarão baleia}
\end{EntryWithPhonetic}

\begin{EntryWithPhonetic}{鲸鱼}{jing1yu2}{16,8}{⿂,⿂}
  \definition{s.}{baleia}
\end{EntryWithPhonetic}

%%%%%%%%%% 井 %%%%%%%%%%
\subsection*{井}\addcontentsline{loh}{figure}{井 \dpy{jing3}}

\begin{EntryWithPhonetic}{井}{jing3}{4}{⼆}[HSK 6]
  \definition*{s.}{Jing, uma das mansões lunares | Sobrenome: Jing}
  \definition{adj.}{limpo; organizado}
  \definition[口]{s.}{poço; um buraco profundo cavado no chão para tirar água | algo em forma de poço | vila natal ou cidade natal}
\end{EntryWithPhonetic}

%%%%%%%%%% 颈 %%%%%%%%%%
\subsection*{颈}\addcontentsline{loh}{figure}{颈 \dpy{jing3}}

\begin{EntryWithPhonetic}{颈}{jing3}{11}{⾴}
  \definition{s.}{pescoço}
  \seeref{geng3}
\end{EntryWithPhonetic}

\begin{EntryWithPhonetic}{颈部}{jing3bu4}{11,10}{⾴,⾢}[HSK 7-9]
  \definition{s.}{pescoço}
\end{EntryWithPhonetic}

%%%%%%%%%% 景 %%%%%%%%%%
\subsection*{景}\addcontentsline{loh}{figure}{景 \dpy{jing3}}

\begin{EntryWithPhonetic}{景}{jing3}{12}{⽇}[HSK 6]
  \definition*{s.}{Sobrenome: Jing}
  \definition{adj.}{grandioso; elevado; grande}
  \definition{s.}{vista; cenário; cena | situação; condição | cenário (de uma peça ou filme) | cena (de uma peça)}
  \definition{v.}{admirar; reverenciar; respeitar}
\end{EntryWithPhonetic}

\begin{EntryWithPhonetic}{景点}{jing3dian3}{12,9}{⽇,⽕}[HSK 6]
  \definition[个,处]{s.}{local cênico; atração turística; um lugar onde se concentram as atrações turísticas, incluindo atrações naturais e culturais}
\end{EntryWithPhonetic}

\begin{EntryWithPhonetic}{景观}{jing3guan1}{12,6}{⽇,⾒}[HSK 7-9]
  \definition{s.}{cenário; paisagem; paisagem formada naturalmente; também se refere a paisagem criada artificialmente}
\end{EntryWithPhonetic}

\begin{EntryWithPhonetic}{景区}{jing3qu1}{12,4}{⽇,⼖}[HSK 7-9]
  \definition{s.}{ponto turístico | área cênica}
\end{EntryWithPhonetic}

\begin{EntryWithPhonetic}{景色}{jing3se4}{12,6}{⽇,⾊}[HSK 3]
  \definition[片,幅,道,处]{s.}{vista; cena; cenário; paisagem}
\end{EntryWithPhonetic}

\begin{EntryWithPhonetic}{景象}{jing3xiang4}{12,11}{⽇,⾗}[HSK 5]
  \definition[个,种]{s.}{cena; visão; vista; quadro}
\end{EntryWithPhonetic}

%%%%%%%%%% 警 %%%%%%%%%%
\subsection*{警}\addcontentsline{loh}{figure}{警 \dpy{jing3}}

\begin{EntryWithPhonetic}{警}{jing3}{19}{⾔}
  \definition{s.}{policial}
  \definition{v.}{alertar | avisar}
\end{EntryWithPhonetic}

\begin{EntryWithPhonetic}{警察}{jing3cha2}{19,14}{⾔,⼧}[HSK 3]
  \definition[个,位,名,群,队]{s.}{polícia; policial; oficial de polícia; as forças armadas que mantêm a segurança social do país são uma parte importante do aparato estatal; também se refere aos membros dessas forças armadas}
\end{EntryWithPhonetic}

\begin{EntryWithPhonetic}{警车}{jing3che1}{19,4}{⾔,⾞}[HSK 7-9]
  \definition{s.}{carro (ou van) da polícia}
\end{EntryWithPhonetic}

\begin{EntryWithPhonetic}{警告}{jing3gao4}{19,7}{⾔,⼝}[HSK 5]
  \definition[个]{s.}{advertência (como medida disciplinar); uma forma de punição}
  \definition{v.}{avisar; advertir; admoestar}
\end{EntryWithPhonetic}

\begin{EntryWithPhonetic}{警官}{jing3guan1}{19,8}{⾔,⼧}[HSK 7-9]
  \definition[名]{s.}{policial; guarda; agente; oficial de polícia}
\end{EntryWithPhonetic}

\begin{EntryWithPhonetic}{警惕}{jing3ti4}{19,11}{⾔,⼼}[HSK 7-9]
  \definition{v.}{estar vigilante; ficar atento; estar em alerta; estar em guarda contra; estar muito atento aos perigos potenciais ou tendências errôneas}
\end{EntryWithPhonetic}

\begin{EntryWithPhonetic}{警钟}{jing3zhong1}{19,9}{⾔,⾦}[HSK 7-9]
  \definition{s.}{sino de alarme; toque de alarme; sirene}
\end{EntryWithPhonetic}

%%%%%%%%%% 劲 %%%%%%%%%%
\subsection*{劲}\addcontentsline{loh}{figure}{劲 \dpy{jing4}}

\begin{EntryWithPhonetic}{劲}{jing4}{7}{⼒}
  \definition{adj.}{forte; poderoso; resistente; resoluto}
\end{EntryWithPhonetic}

%%%%%%%%%% 净 %%%%%%%%%%
\subsection*{净}\addcontentsline{loh}{figure}{净 \dpy{jing4}}

\begin{EntryWithPhonetic}{净}{jing4}{8}{⼎}[HSK 6]
  \definition{adj.}{limpo | (depois de um verbo) terminado; sem nada sobrando | líquido | vazio; oco; nu}
  \definition{adv.}{todo; o tempo todo | somente; meramente; nada além de | inteiramente; indica puro e nada mais}
  \definition{s.}{o ``rosto pintado'', comumente conhecido como Hualian (花脸), um tipo de personagem da ópera de Pequim, etc.}
  \definition{v.}{tornar limpo | limpar; lavar; esfregar para limpar}
  \seealsoref{花脸}{hua1lian3}
\end{EntryWithPhonetic}

\begin{EntryWithPhonetic}{净化}{jing4hua4}{8,4}{⼎,⼔}[HSK 7-9]
  \definition{v.}{purificar; remover impurezas para purificar o objeto}
\end{EntryWithPhonetic}

%%%%%%%%%% 经 %%%%%%%%%%
\subsection*{经}\addcontentsline{loh}{figure}{经 \dpy{jing4}}

\begin{EntryWithPhonetic}{经}{jing4}{8}{⽷}
  \definition{s.}{fio de urdidura na tecelagem}
  \seeref{jing1}
\end{EntryWithPhonetic}

%%%%%%%%%% 竞 %%%%%%%%%%
\subsection*{竞}\addcontentsline{loh}{figure}{竞 \dpy{jing4}}

\begin{EntryWithPhonetic}{竞}{jing4}{10}{⽴}
  \definition{adj.}{forte; poderoso}
  \definition{v.}{competir; contender; disputar | contestar}
\end{EntryWithPhonetic}

\begin{EntryWithPhonetic}{竞技}{jing4ji4}{10,7}{⽴,⼿}[HSK 7-9]
  \definition{s.}{atletismo; provas de atletismo; esportes; pista e campo}
  \definition{v.}{competir; desafiar; geralmente referindo"-se a competições atléticas}
\end{EntryWithPhonetic}

\begin{EntryWithPhonetic}{竞赛}{jing4sai4}{10,14}{⽴,⾙}[HSK 5]
  \definition[个]{s.}{concurso; competição; partida; corrida}
  \definition{v.}{correr; competir; competir uns com os outros por superioridade; em esportes, produção e outras atividades, para comparar competência, habilidade etc., usado principalmente na linguagem falada}
\end{EntryWithPhonetic}

\begin{EntryWithPhonetic}{竞相}{jing4xiang1}{10,9}{⽴,⽬}[HSK 7-9]
  \definition{adv.}{ansiosamente}
  \definition{s.}{competição}
  \definition{v.}{competir; disputar}
\end{EntryWithPhonetic}

\begin{EntryWithPhonetic}{竞选}{jing4xuan3}{10,9}{⽴,⾡}[HSK 7-9]
  \definition{s.}{eleição; campanha eleitoral}
  \definition{v.}{participar de uma disputa eleitoral; fazer campanha para (um cargo); candidatar-se a}
\end{EntryWithPhonetic}

\begin{EntryWithPhonetic}{竞争}{jing4zheng1}{10,6}{⽴,⼑}[HSK 5]
  \definition{v.}{competir; disputar; lutar; entre duas ou mais partes; em prol de seus próprios interesses; lutar pela vitória por meio de uma disputa de sua própria força contra outra}
\end{EntryWithPhonetic}

%%%%%%%%%% 竟 %%%%%%%%%%
\subsection*{竟}\addcontentsline{loh}{figure}{竟 \dpy{jing4}}

\begin{EntryWithPhonetic}{竟}{jing4}{11}{⾳}[HSK 7-9]
  \definition{adj.}{todo; por toda parte; do começo ao fim}
  \definition{adv.}{no final; eventualmente | na verdade; inesperadamente; significa algo inesperado, equivalente a 居然}
  \definition{v.}{terminar; completar | investigar}
  \seealsoref{居然}{ju1ran2}
\end{EntryWithPhonetic}

\begin{EntryWithPhonetic}{竟敢}{jing4gan3}{11,11}{⾳,⽁}[HSK 7-9]
  \definition{v.}{ousar de fato; ter a audácia; ter a impertinência | ousar}
\end{EntryWithPhonetic}

\begin{EntryWithPhonetic}{竟然}{jing4ran2}{11,12}{⾳,⽕}[HSK 4]
  \definition{adv.}{de fato; inesperadamente; para surpresa de alguém; chegar ao ponto de; indica que algo é um pouco inesperado}
\end{EntryWithPhonetic}

%%%%%%%%%% 敬 %%%%%%%%%%
\subsection*{敬}\addcontentsline{loh}{figure}{敬 \dpy{jing4}}

\begin{EntryWithPhonetic}{敬}{jing4}{12}{⽁}[HSK 7-9]
  \definition*{s.}{Sobrenome: Jing}
  \definition{adj.}{respeitoso; reverente}
  \definition{adv.}{respeitosamente}
  \definition{v.}{respeitar; honrar; estimar | oferecer educadamente | envolver-se em; dedicar-se a}
\end{EntryWithPhonetic}

\begin{EntryWithPhonetic}{敬爱}{jing4'ai4}{12,10}{⽁,⽖}[HSK 7-9]
  \definition{v.}{amar; estimar; acarinhar; respeitar e amar}
\end{EntryWithPhonetic}

\begin{EntryWithPhonetic}{敬而远之}{jing4'er2yuan3zhi1}{12,6,7,3}{⽁,⽽,⾡,⼂}[HSK 7-9]
  \definition{expr.}{``Respeite, mas mantenha distância.''; mantenha uma distância respeitosa de alguém; afaste-se de; demonstrar respeito à distância}
\end{EntryWithPhonetic}

\begin{EntryWithPhonetic}{敬酒}{jing4/jiu3}{12,10}{⽁,⾣}[HSK 7-9]
  \definition{v.+compl.}{brindar; propor um brinde; levantar seu copo respeitosamente para convidar a outra pessoa a beber}
\end{EntryWithPhonetic}

\begin{EntryWithPhonetic}{敬礼}{jing4/li3}{12,5}{⽁,⽰}[HSK 7-9]
  \definition{s.}{honoríficos, usados no final de uma carta}[结尾都是``此致敬礼''。===Todas elas terminam com ``Atenciosamente''.]
  \definition{v.+compl.}{saudar; prestar continência; demonstrar respeito através de gestos como olhar para alguém, levantar a mão, ficar em posição de sentido e fazer uma reverência}
\end{EntryWithPhonetic}

\begin{EntryWithPhonetic}{敬佩}{jing4pei4}{12,8}{⽁,⼈}[HSK 7-9]
  \definition{v.}{estimar; admirar; respeitar e admirar}
\end{EntryWithPhonetic}

\begin{EntryWithPhonetic}{敬请}{jing4qing3}{12,10}{⽁,⾔}[HSK 7-9]
  \definition{v.}{solicitar; convidar respeitosamente; termos educados usados para convidar ou pedir (que alguém faça algo)}
\end{EntryWithPhonetic}

\begin{EntryWithPhonetic}{敬业}{jing4ye4}{12,5}{⽁,⼀}[HSK 7-9]
  \definition{v.}{ser dedicado ao próprio trabalho; refere"-se a um espírito louvável, dedicado aos estudos ou ao trabalho}
\end{EntryWithPhonetic}

\begin{EntryWithPhonetic}{敬意}{jing4yi4}{12,13}{⽁,⼼}[HSK 7-9]
  \definition{s.}{respeito; tributo; homenagem}
\end{EntryWithPhonetic}

\begin{EntryWithPhonetic}{敬重}{jing4zhong4}{12,9}{⽁,⾥}[HSK 7-9]
  \definition{v.}{reverenciar; honrar; estimar; respeitar profundamente}
\end{EntryWithPhonetic}

%%%%%%%%%% 靓 %%%%%%%%%%
\subsection*{靓}\addcontentsline{loh}{figure}{靓 \dpy{jing4}}

\begin{EntryWithPhonetic}{靓}{jing4}{12}{⾭}
  \definition{v.}{(referindo"-se a vestimenta de alguém) ficar bonito | vestir"-se | maquiar (o rosto)}
  \seeref{liang4}
\end{EntryWithPhonetic}

%%%%%%%%%% 境 %%%%%%%%%%
\subsection*{境}\addcontentsline{loh}{figure}{境 \dpy{jing4}}

\begin{EntryWithPhonetic}{境}{jing4}{14}{⼟}
  \definition{s.}{fronteira; limite | lugar; área; território; região | condição; situação; circunstâncias}
\end{EntryWithPhonetic}

\begin{EntryWithPhonetic}{境地}{jing4di4}{14,6}{⼟,⼟}[HSK 7-9]
  \definition{s.}{situação; circunstâncias; as circunstâncias ou a situação encontradas (geralmente usadas em um sentido negativo) | reino; estado}
\end{EntryWithPhonetic}

\begin{EntryWithPhonetic}{境界}{jing4jie4}{14,9}{⼟,⽥}[HSK 7-9]
  \definition{s.}{limite; limites de terra | estado; nível; extensão alcançada; o grau em que algo é alcançado ou o estado em que se manifesta}
\end{EntryWithPhonetic}

\begin{EntryWithPhonetic}{境内}{jing4nei4}{14,4}{⼟,⼌}[HSK 7-9]
  \definition{s.}{área dentro das fronteiras | doméstico | interno (para um país, província, cidade etc.) | dentro das fronteiras}
  \antonymref{境外}{jing4wai4}
\end{EntryWithPhonetic}

\begin{EntryWithPhonetic}{境外}{jing4wai4}{14,5}{⼟,⼣}[HSK 7-9]
  \definition{s.}{área fora das fronteiras (ou do território) de um país; além das fronteiras de um país ou região}
\end{EntryWithPhonetic}

\begin{EntryWithPhonetic}{境遇}{jing4yu4}{14,12}{⼟,⾡}[HSK 7-9]
  \definition{s.}{a sorte de alguém; as circunstâncias; circunstâncias e encontros}
\end{EntryWithPhonetic}

%%%%%%%%%% 静 %%%%%%%%%%
\subsection*{静}\addcontentsline{loh}{figure}{静 \dpy{jing4}}

\begin{EntryWithPhonetic}{静}{jing4}{14}{⾭}[HSK 3]
  \definition*{s.}{Sobrenome: Jing}
  \definition{adj.}{tranquilo;  sossegado; calmo; imóvel | silencioso; quieto; sem emitir nenhum som | calmo, sereno; serenidade; (interior) paz}
  \definition{v.}{acalmar-se; aquietar-se; tranquilizar (o coração)}
\end{EntryWithPhonetic}

\begin{EntryWithPhonetic}{静止}{jing4zhi3}{14,4}{⾭,⽌}[HSK 7-9]
  \definition{adj.}{estático; imóvel; parado; estacionário}
\end{EntryWithPhonetic}

%%%%%%%%%% 镜 %%%%%%%%%%
\subsection*{镜}\addcontentsline{loh}{figure}{镜 \dpy{jing4}}

\begin{EntryWithPhonetic}{镜}{jing4}{16}{⾦}
  \definition*{s.}{Sobrenome: Jing}
  \definition[面,副]{s.}{espelho | lente; vidro; dispositivos para auxiliar a visão ou conduzir experimentos ópticos}
  \definition{v.}{espelhar | perceber | usar como referência}
\end{EntryWithPhonetic}

\begin{EntryWithPhonetic}{镜头}{jing4tou2}{16,5}{⾦,⼤}[HSK 4]
  \definition[个,组]{s.}{lente de câmera; objetiva; combinação de várias lentes, usada para formar uma imagem | foto; cena}
\end{EntryWithPhonetic}

\begin{EntryWithPhonetic}{镜子}{jing4zi5}{16,3}{⾦,⼦}[HSK 4]
  \definition[面,个]{s.}{espelho; instrumento de reflexão de imagem liso e plano, antigamente esmerilhado a partir de um disco grosso de cobre fundido, atualmente feito de vidro plano revestido de prata ou alumínio | óculos; óculos de grau}
\end{EntryWithPhonetic}

%%%%%%%%%% 窘 %%%%%%%%%%
\subsection*{窘}\addcontentsline{loh}{figure}{窘 \dpy{jiong3}}

\begin{EntryWithPhonetic}{窘}{jiong3}{12}{⽳}
  \definition{adj.}{em situação financeira precária; sem dinheiro; pobre | desajeitado; constrangido; desconfortável; difícil}
  \definition{v.}{constranger; desconcertar; dificultar as coisas}
\end{EntryWithPhonetic}

\begin{EntryWithPhonetic}{窘迫}{jiong3po4}{12,8}{⽳,⾡}[HSK 7-9]
  \definition{adj.}{muito pobre; miserável; extremamente ruim | envergonhado; pressionado; em apuros; descreve uma situação que causa constrangimento}
\end{EntryWithPhonetic}

%%%%%%%%%% 纠 %%%%%%%%%%
\subsection*{纠}\addcontentsline{loh}{figure}{纠 \dpy{jiu1}}

\begin{EntryWithPhonetic}{纠}{jiu1}{5}{⽷}
  \definition*{s.}{Sobrenome: Jiu}
  \definition{v.}{emaranhar | reunir-se | corrigir; retificar | supervisionar; superintender}
\end{EntryWithPhonetic}

\begin{EntryWithPhonetic}{纠缠}{jiu1chan2}{5,13}{⽷,⽷}[HSK 7-9]
  \definition{v.}{enredar-se; estar em apuros; entrelaçar | importunar; preocupar; perturbar; causar problemas}
\end{EntryWithPhonetic}

\begin{EntryWithPhonetic}{纠纷}{jiu1fen1}{5,7}{⽷,⽷}[HSK 6]
  \definition[个,次]{s.}{questão; disputa; existem contradições ou conflitos de interesse entre as duas partes que precisam ser resolvidos}
\end{EntryWithPhonetic}

\begin{EntryWithPhonetic}{纠葛}{jiu1ge2}{5,12}{⽷,⾋}
  \definition{s.}{emaranhado | disputa}
\end{EntryWithPhonetic}

\begin{EntryWithPhonetic}{纠正}{jiu1zheng4}{5,5}{⽷,⽌}[HSK 6]
  \definition{v.}{fazer certo; corrigir (deficiências ou erros em pensamentos, ações, métodos, etc.)}
\end{EntryWithPhonetic}

%%%%%%%%%% 究 %%%%%%%%%%
\subsection*{究}\addcontentsline{loh}{figure}{究 \dpy{jiu1}}

\begin{EntryWithPhonetic}{究}{jiu1}{7}{⽳}
  \definition{adv.}{na verdade; realmente; afinal}
  \definition{v.}{estudar cuidadosamente; aprofundar; investigar; rastrear}
\end{EntryWithPhonetic}

\begin{EntryWithPhonetic}{究竟}{jiu1jing4}{7,11}{⽳,⾳}[HSK 4]
  \definition{adv.}{de fato; exatamente; usado em frases interrogativas para buscar | afinal de contas, no final; ênfase em fatos ou motivos}
  \definition{s.}{resultado; desfecho; a causa, o efeito ou a história completa do que aconteceu}
\end{EntryWithPhonetic}

%%%%%%%%%% 揪 %%%%%%%%%%
\subsection*{揪}\addcontentsline{loh}{figure}{揪 \dpy{jiu1}}

\begin{EntryWithPhonetic}{揪}{jiu1}{12}{⼿}[HSK 7-9]
  \definition{v.}{segurar com firmeza; agarrar e puxar}
\end{EntryWithPhonetic}

%%%%%%%%%% 九 %%%%%%%%%%
\subsection*{九}\addcontentsline{loh}{figure}{九 \dpy{jiu3}}

\begin{EntryWithPhonetic}{九}{jiu3}{2}{⼄}[HSK 1]
  \definition*{s.}{Sobrenome: Jiu}
  \definition{adj.}{muitos; numerosos; indica várias vezes ou a maioria das vezes}
  \definition{num.}{nove; 9}
  \definition{s.}{cada um dos nove períodos de nove dias começando no dia seguinte ao solstício de inverno}
\end{EntryWithPhonetic}

%%%%%%%%%% 久 %%%%%%%%%%
\subsection*{久}\addcontentsline{loh}{figure}{久 \dpy{jiu3}}

\begin{EntryWithPhonetic}{久}{jiu3}{3}{⼃}[HSK 3]
  \definition{adj.}{por muito tempo; longo período de tempo | duração de tempo especificada}
  \antonymref{暂}{zan4}
\end{EntryWithPhonetic}

\begin{EntryWithPhonetic}{久违}{jiu3wei2}{3,7}{⼃,⾡}[HSK 7-9]
  \definition{v.}{não ver há muito tempo; fazer muito tempo desde o último encontro; apenas um comentário educado, muito tempo sem ver}
  \synonymref{重逢}{chong2feng2}
  \antonymref{经常}{jing1chang2}
\end{EntryWithPhonetic}

\begin{EntryWithPhonetic}{久仰}{jiu3yang3}{3,6}{⼃,⼈}[HSK 7-9]
  \definition{expr.}{``É um prazer conhecê-lo(a).''; ``Há muito tempo que desejo conhecê-lo(a).''; ``Há muito tempo que aguardo ansiosamente o nosso encontro.''; ``Já ouvi falar muito de você.''}
\end{EntryWithPhonetic}

%%%%%%%%%% 韭 %%%%%%%%%%
\subsection*{韭}\addcontentsline{loh}{figure}{韭 \dpy{jiu3}}

\begin{EntryWithPhonetic}{韭}{jiu3}{9}{⾲}[Kangxi 179]
  \definition{s.}{alho de flor perfumada; cebolinha chinesa}
\end{EntryWithPhonetic}

\begin{EntryWithPhonetic}{韭菜}{jiu3cai4}{9,11}{⾲,⾋}
  \definition{s.}{cebolinha-de-alho; cebolinha chinesa | Coloquial: pessoas ingênuas ou facilmente exploráveis, especialmente pequenos investidores ou consumidores, comparadas à cebolinha, que pode ser colhida repetidamente para obter lucro}
\end{EntryWithPhonetic}

%%%%%%%%%% 酒 %%%%%%%%%%
\subsection*{酒}\addcontentsline{loh}{figure}{酒 \dpy{jiu3}}

\begin{EntryWithPhonetic}{酒}{jiu3}{10}{⾣}[HSK 2]
  \definition*{s.}{Sobrenome: Jiu}
  \definition[口,杯,瓶,罐,桶,缸]{s.}{bebida alcoólica; vinho; licor; bebidas destiladas}
\end{EntryWithPhonetic}

\begin{EntryWithPhonetic}{酒吧}{jiu3ba1}{10,7}{⾣,⼝}[HSK 4]
  \definition[家,间]{s.}{bar; \emph{pub}; um local onde são vendidas bebidas alcoólicas e onde as pessoas podem beber e conversar, referindo"-se principalmente a um restaurante ou hotel de estilo ocidental especializado na venda de bebidas alcoólicas}
\end{EntryWithPhonetic}

\begin{EntryWithPhonetic}{酒店}{jiu3dian4}{10,8}{⾣,⼴}[HSK 2]
  \definition[家,个]{s.}{hotel; Estabelecimento comercial que oferece hospedagem e alimentação aos hóspedes | restaurante}
\end{EntryWithPhonetic}

\begin{EntryWithPhonetic}{酒馆}{jiu3guan3}{10,11}{⾣,⾷}
  \definition{s.}{bar | taverna | adega}
\end{EntryWithPhonetic}

\begin{EntryWithPhonetic}{酒鬼}{jiu3gui3}{10,9}{⾣,⿁}[HSK 5]
  \definition[个]{s.}{bebedor de vinho; beberrão; ébrio | alcoólatra}
\end{EntryWithPhonetic}

\begin{EntryWithPhonetic}{酒精}{jiu3jing1}{10,14}{⾣,⽶}[HSK 7-9]
  \definition{s.}{álcool; álcool etílico; etanol}
\end{EntryWithPhonetic}

\begin{EntryWithPhonetic}{酒楼}{jiu3lou2}{10,13}{⾣,⽊}[HSK 7-9]
  \definition[座,家]{s.}{restaurante (em nomes de restaurantes)}[广东酒楼===Restaurante Guangdong]
\end{EntryWithPhonetic}

\begin{EntryWithPhonetic}{酒水}{jiu3shui3}{10,4}{⾣,⽔}[HSK 6]
  \definition{s.}{bebidas; bebidas e álcool | Dialeto: festa; banquete}
\end{EntryWithPhonetic}

%%%%%%%%%% 旧 %%%%%%%%%%
\subsection*{旧}\addcontentsline{loh}{figure}{旧 \dpy{jiu4}}

\begin{EntryWithPhonetic}{旧}{jiu4}{5}{⽇}[HSK 3]
  \definition{adj.}{passado; antigo; velho; ultrapassado | usado; desgastado; velho; descolorido ou deformado devido ao uso prolongado ou ao tempo | antigo; único; que já existiu; anterior}
  \definition{s.}{velha amizade; velho amigo}
  \antonymref{新}{xin1}
\end{EntryWithPhonetic}

%%%%%%%%%% 救 %%%%%%%%%%
\subsection*{救}\addcontentsline{loh}{figure}{救 \dpy{jiu4}}

\begin{EntryWithPhonetic}{救}{jiu4}{11}{⽁}[HSK 3]
  \definition*{s.}{Sobrenome: Jiu}
  \definition{v.}{resgatar; salvar | salvar de; aliviar (angústia, etc.) | resgatar; livrar alguém de um desastre ou perigo | ajudar; aliviar; socorrer; livrar pessoas e coisas de desastres e perigos}
\end{EntryWithPhonetic}

\begin{EntryWithPhonetic}{救出}{jiu4chu1}{11,5}{⽁,⼐}
  \definition{v.}{resgatar | tirar do perigo}
\end{EntryWithPhonetic}

\begin{EntryWithPhonetic}{救护车}{jiu4hu4che1}{11,7,4}{⽁,⼿,⾞}[HSK 7-9]
  \definition[辆]{s.}{ambulância; os veículos que transportam os feridos estão equipados com instalações que permitem à equipe médica prestar primeiros socorros temporários, cuidados médicos e serviços de enfermagem aos feridos}
\end{EntryWithPhonetic}

\begin{EntryWithPhonetic}{救济}{jiu4ji4}{11,9}{⽁,⽔}[HSK 7-9]
  \definition{v.}{fornecer ajuda com dinheiro ou bens; usar dinheiro e bens para ajudar vítimas de desastres ou outras pessoas que vivem em situação de pobreza}
\end{EntryWithPhonetic}

\begin{EntryWithPhonetic}{救命}{jiu4/ming4}{11,8}{⽁,⼝}[HSK 6]
  \definition{interj.}{``Socorro!''; ``Salve-me!''}
  \definition{v.+compl.}{ajudar; salvar a vida de alguém}
\end{EntryWithPhonetic}

\begin{EntryWithPhonetic}{救援}{jiu4yuan2}{11,12}{⽁,⼿}[HSK 6]
  \definition{v.}{resgatar; socorrer; vir em auxílio de alguém (resgate)}
\end{EntryWithPhonetic}

\begin{EntryWithPhonetic}{救灾}{jiu4 zai1}{11,7}{⽁,⽕}[HSK 5]
  \definition{v.}{ajudar as vítimas de desastres, aliviar o desastre; resgatar pessoas afetadas por desastres; recuperar danos causados por desastres}
\end{EntryWithPhonetic}

\begin{EntryWithPhonetic}{救治}{jiu4zhi4}{11,8}{⽁,⽔}[HSK 7-9]
  \definition{v.}{retirar um paciente do perigo; tratar e curar}
\end{EntryWithPhonetic}

\begin{EntryWithPhonetic}{救助}{jiu4zhu4}{11,7}{⽁,⼒}[HSK 6]
  \definition{v.}{ajudar alguém em perigo ou dificuldade; socorrer; resgatar e ajudar}
\end{EntryWithPhonetic}

%%%%%%%%%% 就 %%%%%%%%%%
\subsection*{就}\addcontentsline{loh}{figure}{就 \dpy{jiu4}}

\begin{EntryWithPhonetic}{就}{jiu4}{12}{⼪}[HSK 1]
  \definition{adv.}{de imediato; imediatamente; indica que algo ocorrerá em breve | tão cedo quanto; já; há muito tempo; indica que a ação ocorreu há muito tempo | assim que; logo depois; indica que os eventos se sucedem imediatamente | nesse caso; então; indica que, sob determinadas condições, ocorre naturalmente um determinado resultado | exatamente; precisamente; indica que é exatamente assim | apenas; meramente; somente | tantos quanto; enfatiza a quantidade | apenas; simplesmente; reforço da afirmação | colocado entre dois componentes idênticos, significa tolerância ou indiferença}
  \definition{prep.}{tirar proveito de alguém (algo); expressa condições, oportunidades, etc., equivalente a 趁 | quando se trata de alguém (algo); relativo a; com relação a; sobre; objeto ou escopo da introdução da ação |no local; introduz o local próximo ao qual a ação ocorreu}
  \definition{v.}{ser comido com; ir com; pratos, frutas, etc., acompanhados de alimentos básicos ou bebidas alcoólicas | aproximar-se; mover-se em direção a | ir para; assumir; empreender; envolver-se em; entrar em | realizar; fazer | tirar proveito de; acomodar-se a; adequar-se; encaixar-se | assumir; começar a entrar ou a exercer | seguir; acompanhar}
  \seealsoref{趁}{chen4}
\end{EntryWithPhonetic}

\begin{EntryWithPhonetic}{就餐}{jiu4can1}{12,16}{⼪,⾷}[HSK 7-9]
  \definition{v.}{comer; jantar; fazer uma refeição; ir comer}
\end{EntryWithPhonetic}

\begin{EntryWithPhonetic}{就地}{jiu4di4}{12,6}{⼪,⼟}[HSK 7-9]
  \definition{adv.}{no local; no próprio local | localmente}
\end{EntryWithPhonetic}

\begin{EntryWithPhonetic}{就读}{jiu4du2}{12,10}{⼪,⾔}[HSK 7-9]
  \definition{v.}{fazer um curso; frequentar a escola; ir à escola; ingressar na escola}
\end{EntryWithPhonetic}

\begin{EntryWithPhonetic}{就近}{jiu4jin4}{12,7}{⼪,⾡}[HSK 7-9]
  \definition{adv.}{(fazer ou obter algo) nas proximidades; na vizinhança; sem ter que ir longe; significa que está por perto}
\end{EntryWithPhonetic}

\begin{EntryWithPhonetic}{就可以了}{jiu4 ke3yi3le5}{12,5,4,2}{⼪,⼝,⼈,⼅}
  \definition{expr.}{é isso; é o suficiente}
\end{EntryWithPhonetic}

\begin{EntryWithPhonetic}{就任}{jiu4ren4}{12,6}{⼪,⼈}[HSK 7-9]
  \definition{v.}{assumir o cargo; tomar posse | assumir o próprio cargo}
\end{EntryWithPhonetic}

\begin{EntryWithPhonetic}{就是}{jiu4shi4}{12,9}{⼪,⽇}[HSK 3]
  \definition{adv.}{exatamente; precisamente; expressar concordância com a afirmação da outra pessoa ou confirmar que a afirmação da outra pessoa está correta | apenas; simplesmente; expressa afirmação, determinação ou ênfase, o significado específico deve ser determinado com base no contexto anterior ou posterior | usado para indicar escolha}
  \definition{conj.}{ainda que; mesmo que se reconheça que essa situação é verdadeira, a situação posterior não mudará}
  \definition{part.}{usado no final de uma frase para expressar afirmação}
\end{EntryWithPhonetic}

\begin{EntryWithPhonetic}{就是说}{jiu4shi4shuo1}{12,9,9}{⼪,⽇,⾔}[HSK 6]
  \definition{expr.}{ou seja; isto é; em outras palavras; é frequentemente usado como uma interjeição em uma frase para indicar que as palavras seguintes são uma explicação ou esclarecimento das anteriores}
\end{EntryWithPhonetic}

\begin{EntryWithPhonetic}{就算}{jiu4suan4}{12,14}{⼪,⽵}[HSK 6]
  \definition{conj.}{mesmo que; concedido que; expressam uma relação hipotética e concessiva, frequentemente usadas com 也, equivalente a 即使}
  \seealsoref{即使}{ji2shi3}
  \seealsoref{也}{ye3}
\end{EntryWithPhonetic}

\begin{EntryWithPhonetic}{就要}{jiu4yao4}{12,9}{⼪,⾑}[HSK 2]
  \definition{adv.}{estar prestes a; estar indo para; estar no ponto de}
\end{EntryWithPhonetic}

\begin{EntryWithPhonetic}{就业}{jiu4/ye4}{12,5}{⼪,⼀}[HSK 3]
  \definition{v.+compl.}{conseguir um emprego; obter emprego; assumir uma ocupação; começar a trabalhar}
\end{EntryWithPhonetic}

\begin{EntryWithPhonetic}{就医}{jiu4/yi1}{12,7}{⼪,⼖}[HSK 7-9]
  \definition{v.+compl.}{consultar um médico; ir ao médico; buscar aconselhamento médico}
\end{EntryWithPhonetic}

\begin{EntryWithPhonetic}{就诊}{jiu4/zhen3}{12,7}{⼪,⾔}[HSK 7-9]
  \definition{v.+compl.}{consultar um médico; procurar aconselhamento médico}
\end{EntryWithPhonetic}

\begin{EntryWithPhonetic}{就职}{jiu4/zhi2}{12,11}{⼪,⽿}
  \definition{v.+compl.}{assumir o cargo; assumir oficialmente um cargo (geralmente referindo"-se a uma posição de maior hierarquia)}
\end{EntryWithPhonetic}

\begin{EntryWithPhonetic}{就坐}{jiu4zuo4}{12,7}{⼪,⼟}
  \definition{v.}{sentar-se; estar sentado}
\end{EntryWithPhonetic}

\begin{EntryWithPhonetic}{就座}{jiu4/zuo4}{12,10}{⼪,⼴}[HSK 7-9]
  \definition{v.+compl.}{sentar-se; estar sentado | ocupar o próprio lugar (assento)}
  \seealsoref{就坐}{jiu4zuo4}
\end{EntryWithPhonetic}

%%%%%%%%%% 舅 %%%%%%%%%%
\subsection*{舅}\addcontentsline{loh}{figure}{舅 \dpy{jiu4}}

\begin{EntryWithPhonetic}{舅}{jiu4}{13}{⾅}
  \definition*{s.}{Sobrenome: Jiu}
  \definition[个,位,名,些]{s.}{irmão da mãe; tio materno | irmão da esposa; cunhado | Literário: pai do marido; sogro}
\end{EntryWithPhonetic}

\begin{EntryWithPhonetic}{舅舅}{jiu4jiu5}{13,13}{⾅,⾅}[HSK 7-9]
  \definition[个,位,名]{s.}{tio; irmão da mãe; é assim que você se dirige ao irmão mais velho ou mais novo de sua mãe; você também pode chamá-lo de 舅}
  \seealsoref{舅}{jiu4}
\end{EntryWithPhonetic}

%%%%%%%%%% 车 %%%%%%%%%%
\subsection*{车}\addcontentsline{loh}{figure}{车 \dpy{ju1}}

\begin{EntryWithPhonetic}{车}{ju1}{4}{⾞}[Kangxi 159]
  \definition{s.}{torre; castelo; carruagem, uma das peças do xadrez chinês}
  \seeref{che1}
\end{EntryWithPhonetic}

%%%%%%%%%% 居 %%%%%%%%%%
\subsection*{居}\addcontentsline{loh}{figure}{居 \dpy{ju1}}

\begin{EntryWithPhonetic}{居}{ju1}{8}{⼫}
  \definition*{s.}{Sobrenome: Ju}
  \definition{s.}{residência; casa | restaurante (em nomes de restaurantes)}
  \definition{v.}{residir; morar; viver | ocupar uma determinada posição; ocupar (um lugar); estar (em uma determinada posição) | reivindicar; afirmar | armazenar; guardar | ficar parado; estar parado}
\end{EntryWithPhonetic}

\begin{EntryWithPhonetic}{居高临下}{ju1gao1-lin2xia4}{8,10,9,3}{⼫,⾼,⼁,⼀}[HSK 7-9]
  \definition{expr.}{``Olhando para baixo.''; ocupar uma posição (ou altura) dominante; olhar de cima para baixo; viver no alto e olhar para baixo; ocupar o terreno elevado; ignorar; elevar-se acima | Figurativo: arrogância baseada na posição social de alguém}
\end{EntryWithPhonetic}

\begin{EntryWithPhonetic}{居民}{ju1min2}{8,5}{⼫,⽒}[HSK 4]
  \definition[个,户,位]{s.}{residente; habitante; pessoas que estão fixas em um único lugar}
\end{EntryWithPhonetic}

\begin{EntryWithPhonetic}{居民楼}{ju1min2lou2}{8,5,13}{⼫,⽒,⽊}[HSK 7-9]
  \definition{s.}{edifício residencial}
\end{EntryWithPhonetic}

\begin{EntryWithPhonetic}{居然}{ju1ran2}{8,12}{⼫,⽕}[HSK 5]
  \definition{adv.}{inesperadamente; para surpresa de alguém; além da expectativa (expressão idiomática)}
  \definition{v.}{ir tão longe a ponto de; ter a impudência de; ter o descaramento de}
\end{EntryWithPhonetic}

\begin{EntryWithPhonetic}{居住}{ju1zhu4}{8,7}{⼫,⼈}[HSK 4]
  \definition{v.}{viver; residir; morar; habitar}
\end{EntryWithPhonetic}

%%%%%%%%%% 拘 %%%%%%%%%%
\subsection*{拘}\addcontentsline{loh}{figure}{拘 \dpy{ju1}}

\begin{EntryWithPhonetic}{拘}{ju1}{8}{⼿}
  \definition{adj.}{inflexível; sem flexibilidade}
  \definition{v.}{prender; deter | restringir; limitar; constranger | aderir rigidamente; ser inflexível |limitar}
\end{EntryWithPhonetic}

\begin{EntryWithPhonetic}{拘留}{ju1liu2}{8,10}{⼿,⽥}[HSK 7-9]
  \definition{v.}{deter; manter sob custódia; colocar em prisão provisória; restringir a liberdade pessoal}
\end{EntryWithPhonetic}

\begin{EntryWithPhonetic}{拘束}{ju1shu4}{8,7}{⼿,⽊}[HSK 7-9]
  \definition{adj.}{desajeitado; desconfortável; constrangido; reservado; não natural}
  \definition{v.}{restringir; limitar; restringir excessivamente as palavras e ações de outras pessoas}
\end{EntryWithPhonetic}

%%%%%%%%%% 据 %%%%%%%%%%
\subsection*{据}\addcontentsline{loh}{figure}{据 \dpy{ju1}}

\begin{EntryWithPhonetic}{据}{ju1}{11}{⼿}
  \definition{part.}{elemento formador de palavras}
  \seeref{ju4}
  \seealsoref{拮据}{jie2ju1}
\end{EntryWithPhonetic}

%%%%%%%%%% 鞠 %%%%%%%%%%
\subsection*{鞠}\addcontentsline{loh}{figure}{鞠 \dpy{ju1}}

\begin{EntryWithPhonetic}{鞠}{ju1}{17}{⾰}
  \definition*{s.}{Sobrenome: Ju}
  \definition{s.}{arco | Arcaico: uma bola de futebol; um tipo antigo de bola; bola usada para jogar}
  \definition{v.}{Literátio: criar; educar; nutrir | obrar; curvar}
\end{EntryWithPhonetic}

\begin{EntryWithPhonetic}{鞠躬}{ju1/gong1}{17,10}{⾰,⾝}[HSK 7-9]
  \definition{v.+compl.}{curvar"-se; inclinar"-se}
\end{EntryWithPhonetic}

%%%%%%%%%% 局 %%%%%%%%%%
\subsection*{局}\addcontentsline{loh}{figure}{局 \dpy{ju2}}

\begin{EntryWithPhonetic}{局}{ju2}{7}{⼫}[HSK 4,6]
  \definition{adj.}{limitado; confinado}
  \definition{clas.}{\emph{set}; jogo; turno}
  \definition{s.}{tabuleiro de xadrez | situação; estado de coisas | generosidade de espírito; extensão da tolerância de alguém | festa; reunião; refere"-se a certas reuniões | ardil; armadilha | parte; porção; papel | escritório; agência; agências governamentais divididas por negócios | significa ``loja'' em nomes de lojas | departamento; agência; nomes de certas entidades empresariais | escritório; usado como nome de uma instituição ou outro local de negócios}
\end{EntryWithPhonetic}

\begin{EntryWithPhonetic}{局部}{ju2bu4}{7,10}{⼫,⾢}[HSK 7-9]
  \definition{s.}{parte; uma parte; não o todo}
\end{EntryWithPhonetic}

\begin{EntryWithPhonetic}{局面}{ju2mian4}{7,9}{⼫,⾯}[HSK 5]
  \definition[种]{s.}{aspecto; fase; situação; o estado das coisas em um período de tempo, em sua maior parte abstraído | escopo; escala}
\end{EntryWithPhonetic}

\begin{EntryWithPhonetic}{局势}{ju2shi4}{7,8}{⼫,⼒}[HSK 7-9]
  \definition{s.}{situação; estado (de coisas); (político, militar, etc.) desenvolvimentos ao longo de um período de tempo}
\end{EntryWithPhonetic}

\begin{EntryWithPhonetic}{局限}{ju2xian4}{7,8}{⼫,⾩}[HSK 7-9]
  \definition{v.}{limitar; confinar; limitar a um determinado intervalo}
\end{EntryWithPhonetic}

\begin{EntryWithPhonetic}{局长}{ju2zhang3}{7,4}{⼫,⾧}[HSK 5]
  \definition[位,名,个,些]{s.}{comissário; diretor; principais chefes de gabinete do governo}
\end{EntryWithPhonetic}

%%%%%%%%%% 菊 %%%%%%%%%%
\subsection*{菊}\addcontentsline{loh}{figure}{菊 \dpy{ju2}}

\begin{EntryWithPhonetic}{菊}{ju2}{11}{⾋}
  \definition*{s.}{Sobrenome: Ju}
  \definition[朵]{s.}{crisântemo}
\end{EntryWithPhonetic}

\begin{EntryWithPhonetic}{菊花}{ju2hua1}{11,7}{⾋,⾋}[HSK 7-9]
  \definition[朵,枝,把,束,棵,株]{s.}{crisântemo | Gíria: ânus; uma metáfora para o ânus (ou reto) humano}
\end{EntryWithPhonetic}

%%%%%%%%%% 橘 %%%%%%%%%%
\subsection*{橘}\addcontentsline{loh}{figure}{橘 \dpy{ju2}}

\begin{EntryWithPhonetic}{橘}{ju2}{16}{⽊}
  \definition[只,棵]{s.}{tangerina}
\end{EntryWithPhonetic}

\begin{EntryWithPhonetic}{橘子}{ju2zi5}{16,3}{⽊,⼦}[HSK 7-9]
  \definition[个,堆,箱,筐]{s.}{tangerina; laranja mandarina}
\end{EntryWithPhonetic}

\begin{EntryWithPhonetic}{橘子汁}{ju2zi5zhi1}{16,3,5}{⽊,⼦,⽔}
  \definition[瓶,杯,罐,盒]{s.}{suco de laranja}
  \seealsoref{橙汁}{cheng2zhi1}
  \seealsoref{柳橙汁}{liu3cheng2zhi1}
\end{EntryWithPhonetic}

%%%%%%%%%% 咀 %%%%%%%%%%
\subsection*{咀}\addcontentsline{loh}{figure}{咀 \dpy{ju3}}

\begin{EntryWithPhonetic}{咀}{ju3}{8}{⼝}
  \definition{v.}{mastigar; mascar}
  \seeref{zui3}
\end{EntryWithPhonetic}

\begin{EntryWithPhonetic}{咀嚼}{ju3jue2}{8,20}{⼝,⼝}
  \definition{v.}{mastigar; triturar alimentos com os dentes | refletir sobre; essa metáfora descreve o ato de ponderar e compreender algo repetidamente}
\end{EntryWithPhonetic}

%%%%%%%%%% 柜 %%%%%%%%%%
\subsection*{柜}\addcontentsline{loh}{figure}{柜 \dpy{ju3}}

\begin{EntryWithPhonetic}{柜}{ju3}{8}{⽊}
  \definition{s.}{faia; salgueiro}
  \seeref{gui4}
\end{EntryWithPhonetic}

%%%%%%%%%% 沮 %%%%%%%%%%
\subsection*{沮}\addcontentsline{loh}{figure}{沮 \dpy{ju3}}

\begin{EntryWithPhonetic}{沮}{ju3}{8}{⽔}
  \definition{v.}{Literário: parar; prevenir; evitar | ficar melancólico; ficar triste}
  \seeref{ju4}
\end{EntryWithPhonetic}

\begin{EntryWithPhonetic}{沮丧}{ju3sang4}{8,8}{⽔,⼗}[HSK 7-9]
  \definition{adj.}{desanimado; abatido; decepcionado}
\end{EntryWithPhonetic}

%%%%%%%%%% 举 %%%%%%%%%%
\subsection*{举}\addcontentsline{loh}{figure}{举 \dpy{ju3}}

\begin{EntryWithPhonetic}{举}{ju3}{9}{⼂}[HSK 2]
  \definition*{s.}{Sobrenome: Ju}
  \definition{adj.}{inteiro; completo}
  \definition{s.}{ato; ação; movimento; comportamento | (nas dinastias Ming e Qing) candidato aprovado nos exames imperiais a nível provincial}
  \definition{v.}{levantar; erguer; sustentar | começar; iniciar; surgir | eleger; escolher; recomendar; selecionar | citar; enumerar; propor; revelar}
\end{EntryWithPhonetic}

\begin{EntryWithPhonetic}{举办}{ju3ban4}{9,4}{⼂,⼒}[HSK 3]
  \definition{v.}{conduzir; organizar; realizar}
  \antonymref{进行}{jin4xing2}
  \antonymref{举行}{ju3xing2}
  \antonymref{召开}{zhao4kai1}
\end{EntryWithPhonetic}

\begin{EntryWithPhonetic}{举报}{ju3bao4}{9,7}{⼂,⼿}[HSK 7-9]
  \definition{v.}{relatar; denunciar}[我决定举报不法行为。===Decidi denunciar as atividades ilegais.]
  \synonymref{揭发}{jie1fa1}
  \synonymref{投诉}{tou2su4}
\end{EntryWithPhonetic}

\begin{EntryWithPhonetic}{举措}{ju3cuo4}{9,11}{⼂,⼿}[HSK 7-9]
  \definition{v.}{mover; agir; medir}
  \synonymref{办法}{ban4fa3}
  \synonymref{步骤}{bu4zhou4}
  \synonymref{措施}{cuo4shi1}
  \synonymref{动作}{dong4zuo4}
  \synonymref{方法}{fang1fa3}
  \synonymref{举动}{ju3dong4}
  \synonymref{设施}{she4shi1}
  \synonymref{行动}{xing2dong4}
\end{EntryWithPhonetic}

\begin{EntryWithPhonetic}{举动}{ju3dong4}{9,6}{⼂,⼒}[HSK 5]
  \definition{s.}{ato; atividade; movimento; ação}
  \synonymref{举措}{ju3cuo4}
  \synonymref{举止}{ju3zhi3}
  \synonymref{行为}{xing2wei2}
  \synonymref{作为}{zuo4wei2}
  \antonymref{内心}{nei4xin1}
\end{EntryWithPhonetic}

\begin{EntryWithPhonetic}{举例}{ju3/li4}{9,8}{⼂,⼈}[HSK 7-9]
  \definition{v.+compl.}{dar um exemplo; citar um caso}
  \synonymref{例如}{li4ru2}
\end{EntryWithPhonetic}

\begin{EntryWithPhonetic}{举世闻名}{ju3shi4-wen2ming2}{9,5,9,6}{⼂,⼀,⾨,⼝}[HSK 7-9]
  \definition{expr.}{``De renome mundial.''; mundialmente famoso}
  \synonymref{大名鼎鼎}{da4ming2-ding3ding3}
  \synonymref{举世瞩目}{ju3shi4-zhu3mu4}
  \antonymref{不为人知}{bu4wei2ren2zhi1}
  \antonymref{默默无闻}{mo4mo4-wu2wen2}
\end{EntryWithPhonetic}

\begin{EntryWithPhonetic}{举世无双}{ju3shi4-wu2shuang1}{9,5,4,4}{⼂,⼀,⽆,⼜}[HSK 7-9]
  \definition{expr.}{``Inigualável no mundo.''; inigualável; incomparável; sem igual; único; número um do mundo}
  \synonymref{独一无二}{du2yi1-wu2'er4}
\end{EntryWithPhonetic}

\begin{EntryWithPhonetic}{举世瞩目}{ju3shi4-zhu3mu4}{9,5,17,5}{⼂,⼀,⽬,⽬}[HSK 7-9]
  \definition{expr.}{``De renome mundial.''; atrair a atenção mundial; tornar-se o centro das atenções mundiais; o mundo inteiro está assistindo}
  \synonymref{举世闻名}{ju3shi4-wen2ming2}
  \antonymref{默默无闻}{mo4mo4-wu2wen2}
\end{EntryWithPhonetic}

\begin{EntryWithPhonetic}{举手}{ju3 shou3}{9,4}{⼂,⼿}[HSK 2]
  \definition{v.}{levantar a mão ou as mãos; levantar a mão para sinalizar ou responder a uma pergunta}
\end{EntryWithPhonetic}

\begin{EntryWithPhonetic}{举行}{ju3xing2}{9,6}{⼂,⾏}[HSK 2]
  \definition{v.}{realizar (uma reunião, cerimônia, etc.); realizar (atividades formais ou solenes)}
  \synonymref{进行}{jin4xing2}
  \synonymref{举办}{ju3ban4}
  \synonymref{实行}{shi2xing2}
  \synonymref{召开}{zhao4kai1}
  \antonymref{保留}{bao3liu2}
  \antonymref{取消}{qu3xiao1}
\end{EntryWithPhonetic}

\begin{EntryWithPhonetic}{举一反三}{ju3yi1-fan3san1}{9,1,4,3}{⼂,⼀,⼜,⼀}[HSK 7-9]
  \definition{expr.}{aprender por analogia; inferir outras coisas a partir de um fato; aprender muitas coisas por analogia a partir de uma única coisa}
\end{EntryWithPhonetic}

\begin{EntryWithPhonetic}{举止}{ju3zhi3}{9,4}{⼂,⽌}[HSK 7-9]
  \definition{s.}{maneira; comportamento; porte; postura; refere"-se à postura e ao comportamento}
  \synonymref{举动}{ju3dong4}
  \synonymref{行动}{xing2dong4}
  \synonymref{行为}{xing2wei2}
\end{EntryWithPhonetic}

\begin{EntryWithPhonetic}{举重}{ju3zhong4}{9,9}{⼂,⾥}[HSK 7-9]
  \definition{s.}{levantamento de peso}
  \definition{v.}{levantar pesos}
\end{EntryWithPhonetic}

%%%%%%%%%% 巨 %%%%%%%%%%
\subsection*{巨}\addcontentsline{loh}{figure}{巨 \dpy{ju4}}

\begin{EntryWithPhonetic}{巨}{ju4}{4}{⼯}
  \definition*{s.}{Sobrenome: Ju}
  \definition{adj.}{enorme; tremendo; gigantesco}
\end{EntryWithPhonetic}

\begin{EntryWithPhonetic}{巨大}{ju4da4}{4,3}{⼯,⼤}[HSK 4]
  \definition{adj.}{enorme; tremendo; enorme; gigantesco; imenso}
\end{EntryWithPhonetic}

\begin{EntryWithPhonetic}{巨额}{ju4'e2}{4,15}{⼯,⾴}[HSK 7-9]
  \definition{adj.}{enorme; imenso; soma gigantesca; grande quantidade; (dinheiro, etc.) uma grande quantia}
\end{EntryWithPhonetic}

\begin{EntryWithPhonetic}{巨人}{ju4ren2}{4,2}{⼯,⼈}[HSK 7-9]
  \definition{s.}{gigante; colosso | \emph{homo monstrosus}; arranha-céu}
\end{EntryWithPhonetic}

\begin{EntryWithPhonetic}{巨头}{ju4tou2}{4,5}{⼯,⼤}[HSK 7-9]
  \definition{s.}{magnata; pessoa ilustre; líderes com influência considerável nas esferas política e econômica}
\end{EntryWithPhonetic}

\begin{EntryWithPhonetic}{巨星}{ju4xing1}{4,9}{⼯,⽇}[HSK 7-9]
  \definition{s.}{Astronomia: estrela gigante | Figurativo: figura excepcional; superestrela; megaestrela (da ópera, do basquete etc.)}
\end{EntryWithPhonetic}

\begin{EntryWithPhonetic}{巨型}{ju4xing2}{4,9}{⼯,⼟}[HSK 7-9]
  \definition{adj.}{gigante; gigantesco; enorme; tamanho extra grande}
\end{EntryWithPhonetic}

%%%%%%%%%% 句 %%%%%%%%%%
\subsection*{句}\addcontentsline{loh}{figure}{句 \dpy{ju4}}

\begin{EntryWithPhonetic}{句}{ju4}{5}{⼝}[HSK 2]
  \definition{clas.}{para sentenças, frases ou linhas de versos}
  \definition{s.}{frase; sentença}
  \seeref{gou4}
\end{EntryWithPhonetic}

\begin{EntryWithPhonetic}{句子}{ju4zi5}{5,3}{⼝,⼦}[HSK 2]
  \definition[个,句]{s.}{sentença; uma unidade linguística composta por palavras ou frases que expressa um significado completo}
\end{EntryWithPhonetic}

%%%%%%%%%% 拒 %%%%%%%%%%
\subsection*{拒}\addcontentsline{loh}{figure}{拒 \dpy{ju4}}

\begin{EntryWithPhonetic}{拒}{ju4}{7}{⼿}
  \definition{v.}{resistir; repelir | recusar; rejeitar}
\end{EntryWithPhonetic}

\begin{EntryWithPhonetic}{拒绝}{ju4jue2}{7,9}{⼿,⽷}[HSK 5]
  \definition{v.}{recusar; rejeitar; declinar; não aceitar (pedidos, sugestões ou presentes)}
\end{EntryWithPhonetic}

%%%%%%%%%% 足 %%%%%%%%%%
\subsection*{足}\addcontentsline{loh}{figure}{足 \dpy{ju4}}

\begin{EntryWithPhonetic}{足}{ju4}{7}{⾜}[Kangxi 157]
  \definition{adj.}{excessivo}
  \seeref{zu2}
\end{EntryWithPhonetic}

%%%%%%%%%% 具 %%%%%%%%%%
\subsection*{具}\addcontentsline{loh}{figure}{具 \dpy{ju4}}

\begin{EntryWithPhonetic}{具}{ju4}{8}{⼋}
  \definition*{s.}{Sobrenome: Ju}
  \definition{clas.}{(literário) usado para caixões, cadáveres e certos objetos}
  \definition{s.}{utensílio; ferramenta; implemento | capacidade; habilidade}
  \definition{v.}{possuir; ter | fornecer; prover | declarar; enumerar}
\end{EntryWithPhonetic}

\begin{EntryWithPhonetic}{具备}{ju4bei4}{8,8}{⼋,⼡}[HSK 4]
  \definition{v.}{ter; possuir; ser provido de}
\end{EntryWithPhonetic}

\begin{EntryWithPhonetic}{具体}{ju4ti3}{8,7}{⼋,⼈}[HSK 3]
  \definition{adj.}{específico; particular | concreto; específico; mais detalhado; muito detalhado; muito claro | concreto; real; não é abstrato, tem uma forma definida; pode ser visto ou sentido}
  \definition{v.}{incorporar; objetivar; combinar teorias, princípios, padrões, etc. com pessoas ou coisas específicas}
\end{EntryWithPhonetic}

\begin{EntryWithPhonetic}{具有}{ju4you3}{8,6}{⼋,⽉}[HSK 3]
  \definition{v.}{ter; possuir; ser provido de}
\end{EntryWithPhonetic}

%%%%%%%%%% 沮 %%%%%%%%%%
\subsection*{沮}\addcontentsline{loh}{figure}{沮 \dpy{ju4}}

\begin{EntryWithPhonetic}{沮}{ju4}{8}{⽔}
  \definition{s.}{lama e folhas em decomposição}
\end{EntryWithPhonetic}

%%%%%%%%%% 俱 %%%%%%%%%%
\subsection*{俱}\addcontentsline{loh}{figure}{俱 \dpy{ju4}}

\begin{EntryWithPhonetic}{俱}{ju4}{10}{⼈}
  \definition{adv.}{(literário) tudo; completamente; inteiramente}
\end{EntryWithPhonetic}

\begin{EntryWithPhonetic}{俱乐部}{ju4le4bu4}{10,5,10}{⼈,⼃,⾢}[HSK 5]
  \definition[个,家,间]{s.}{clube; grupos e locais para atividades sociais, políticas, literárias, recreativas e outras}
\end{EntryWithPhonetic}

%%%%%%%%%% 剧 %%%%%%%%%%
\subsection*{剧}\addcontentsline{loh}{figure}{剧 \dpy{ju4}}

\begin{EntryWithPhonetic}{剧}{ju4}{10}{⼑}[HSK 6]
  \definition*{s.}{Sobrenome: Ju}
  \definition{adj.}{agudo; grave; intenso; violento}
  \definition[部,个,种]{s.}{obra teatral; drama; peça; ópera}
\end{EntryWithPhonetic}

\begin{EntryWithPhonetic}{剧本}{ju4ben3}{10,5}{⼑,⽊}[HSK 5]
  \definition[部,个]{s.}{cenário; roteiro (para drama, filme, etc.); gênero de obra literária que consiste em diálogos entre personagens (às vezes cantados) e indicações de palco}
\end{EntryWithPhonetic}

\begin{EntryWithPhonetic}{剧场}{ju4chang3}{10,6}{⼑,⼟}[HSK 3]
  \definition[个,坐]{s.}{teatro; local para apresentações teatrais, musicais, etc.}
\end{EntryWithPhonetic}

\begin{EntryWithPhonetic}{剧烈}{ju4lie4}{10,10}{⼑,⽕}[HSK 7-9]
  \definition{adj.}{violento; agudo; severo; feroz; rápido e intenso}
\end{EntryWithPhonetic}

\begin{EntryWithPhonetic}{剧目}{ju4mu4}{10,5}{⼑,⽬}[HSK 7-9]
  \definition{s.}{repertório; programa; lista de peças teatrais ou óperas}
\end{EntryWithPhonetic}

\begin{EntryWithPhonetic}{剧情}{ju4qing2}{10,11}{⼑,⼼}[HSK 7-9]
  \definition[个,段]{s.}{a história; o enredo (de uma peça ou ópera)}
\end{EntryWithPhonetic}

\begin{EntryWithPhonetic}{剧团}{ju4tuan2}{10,6}{⼑,⼞}[HSK 7-9]
  \definition[家,个]{s.}{companhia teatral; grupo de ópera; trupe | grupo de teatro}
\end{EntryWithPhonetic}

\begin{EntryWithPhonetic}{剧院}{ju4yuan4}{10,9}{⼑,⾩}[HSK 7-9]
  \definition[家,座]{s.}{teatro; casa de espetáculos | companhias teatrais maiores e de classe mais alta}
\end{EntryWithPhonetic}

\begin{EntryWithPhonetic}{剧组}{ju4zu3}{10,8}{⼑,⽷}[HSK 7-9]
  \definition{s.}{equipe de produção; elenco e equipe técnica; um grupo composto por diretores, atores e membros da equipe com o objetivo de apresentar uma peça teatral ou filmar um filme ou série de televisão}
\end{EntryWithPhonetic}

%%%%%%%%%% 据 %%%%%%%%%%
\subsection*{据}\addcontentsline{loh}{figure}{据 \dpy{ju4}}

\begin{EntryWithPhonetic}{据}{ju4}{11}{⼿}[HSK 6]
  \definition*{s.}{Sobrenome: Ju}
  \definition{prep.}{de acordo com; com base em}
  \definition{s.}{evidência; certificado; prova}
  \definition{v.}{ocupar; apreender | confiar em; depender de}
  \seeref{ju1}
\end{EntryWithPhonetic}

\begin{EntryWithPhonetic}{据此}{ju4ci3}{11,6}{⼿,⽌}[HSK 7-9]
  \definition{v.}{se basear nesses fundamentos; ter em vista o exposto acima; fazer algo em conformidade; se basear nas circunstâncias ou razões já mencionadas}
\end{EntryWithPhonetic}

\begin{EntryWithPhonetic}{据说}{ju4shuo1}{11,9}{⼿,⾔}[HSK 3]
  \definition{v.}{é o que dizem; é o que se diz}
\end{EntryWithPhonetic}

\begin{EntryWithPhonetic}{据悉}{ju4xi1}{11,11}{⼿,⼼}[HSK 7-9]
  \definition{adv.}{é relatado (que); de acordo com o que aprendi}
\end{EntryWithPhonetic}

%%%%%%%%%% 距 %%%%%%%%%%
\subsection*{距}\addcontentsline{loh}{figure}{距 \dpy{ju4}}

\begin{EntryWithPhonetic}{距}{ju4}{11}{⾜}[HSK 7-9]
  \definition{s.}{distância | espora (de um galo, etc.)}
  \definition{v.}{estar separado (longe) de; estar distante de}
\end{EntryWithPhonetic}

\begin{EntryWithPhonetic}{距离}{ju4li2}{11,10}{⾜,⼇}[HSK 4]
  \definition[个,段]{s.}{distância}
  \definition{v.}{estar distante de}
\end{EntryWithPhonetic}

%%%%%%%%%% 锯 %%%%%%%%%%
\subsection*{锯}\addcontentsline{loh}{figure}{锯 \dpy{ju4}}

\begin{EntryWithPhonetic}{锯}{ju4}{13}{⾦}[HSK 7-9]
  \definition[把,个]{s.}{serra; serrote; ferramentas para cortar madeira, etc.}
  \definition{v.}{cortar com uma serra; serrar}
\end{EntryWithPhonetic}

%%%%%%%%%% 聚 %%%%%%%%%%
\subsection*{聚}\addcontentsline{loh}{figure}{聚 \dpy{ju4}}

\begin{EntryWithPhonetic}{聚}{ju4}{14}{⽿}[HSK 4]
  \definition*{s.}{Sobrenome: Ju}
  \definition{v.}{reunir-se; juntar-se}
\end{EntryWithPhonetic}

\begin{EntryWithPhonetic}{聚会}{ju4hui4}{14,6}{⽿,⼈}[HSK 4]
  \definition[个,次]{s.}{reunião; encontro; confraternização; festa}
  \definition{v.}{encontrar-se; reunir-se}
\end{EntryWithPhonetic}

\begin{EntryWithPhonetic}{聚集}{ju4ji2}{14,12}{⽿,⾫}[HSK 7-9]
  \definition{v.}{reunir; juntar; coletar; reunir-se; juntar-se}
\end{EntryWithPhonetic}

\begin{EntryWithPhonetic}{聚精会神}{ju4jing1-hui4shen2}{14,14,6,9}{⽿,⽶,⼈,⽰}[HSK 7-9]
  \definition{expr.}{concentrado; concentrar a atenção; focar a mente; estar absorto em; estar profundamente concentrado; estar totalmente concentrado}
\end{EntryWithPhonetic}

\begin{EntryWithPhonetic}{聚散}{ju4san4}{14,12}{⽿,⽁}
  \definition{s.}{juntos e separados | agregação e dissipação}
\end{EntryWithPhonetic}

%%%%%%%%%% 捐 %%%%%%%%%%
\subsection*{捐}\addcontentsline{loh}{figure}{捐 \dpy{juan1}}

\begin{EntryWithPhonetic}{捐}{juan1}{10}{⼿}[HSK 6]
  \definition{s.}{imposto}
  \definition{v.}{renunciar; abandonar | contribuir; doar; assinar}
\end{EntryWithPhonetic}

\begin{EntryWithPhonetic}{捐款}{juan1/kuan3}{10,12}{⼿,⽋}[HSK 6]
  \definition[笔]{s.}{doação; contribuição (de dinheiro); valor doado}
  \definition{v.+compl.}{doar; contribuir com dinheiro}
\end{EntryWithPhonetic}

\begin{EntryWithPhonetic}{捐献}{juan1xian4}{10,13}{⼿,⽝}[HSK 7-9]
  \definition{v.}{doar; apresentar; contribuir (para uma organização); doar bens ao (estado, a uma cooperativa, etc.)}
\end{EntryWithPhonetic}

\begin{EntryWithPhonetic}{捐赠}{juan1zeng4}{10,16}{⼿,⾙}[HSK 6]
  \definition{v.}{apresentar; contribuir (como um presente); doar (itens para um país ou grupo)}
\end{EntryWithPhonetic}

\begin{EntryWithPhonetic}{捐助}{juan1zhu4}{10,7}{⼿,⼒}[HSK 6]
  \definition{v.}{oferecer (assistência financeira ou material); contribuir; doar}
\end{EntryWithPhonetic}

%%%%%%%%%% 圈 %%%%%%%%%%
\subsection*{圈}\addcontentsline{loh}{figure}{圈 \dpy{juan1}}

\begin{EntryWithPhonetic}{圈}{juan1}{11}{⼞}
  \definition{v.}{prender aves e animais de criação | Coloquial: prender os criminosos; colocar na cadeia, prisão | confinar; encarcerar}
  \seeref{juan4}
  \seeref{quan1}
\end{EntryWithPhonetic}

%%%%%%%%%% 卷 %%%%%%%%%%
\subsection*{卷}\addcontentsline{loh}{figure}{卷 \dpy{juan3}}

\begin{EntryWithPhonetic}{卷}{juan3}{8}{⼙}[HSK 4]
  \definition{clas.}{usado para pequenas coisas enroladas (maço de papel dinheiro, carretel de filme, etc.) | usado para rolos, carretéis, bobinas, etc.}
  \definition[张]{s.}{rolo; carretel; bobina}
  \definition{v.}{enrolar; dobrar algo em um cilindro ou semicírculo | varrer; carregar; levar junto | envolver-se; participar}
  \seeref{juan4}
\end{EntryWithPhonetic}

\begin{EntryWithPhonetic}{卷入}{juan3ru4}{8,2}{⼙,⼊}[HSK 7-9]
  \definition{v.}{ser envolvido; estar envolvido; ser atraído para; estar envolvido em}
\end{EntryWithPhonetic}

\begin{EntryWithPhonetic}{卷子}{juan3zi5}{8,3}{⼙,⼦}
  \definition{s.}{rolinho primavera; um tipo de prato de massa, feito amassando em folhas finas, cobrindo um lado com óleo e sal, enrolando-a e cozinhando-a no vapor}
  \seeref{juan4zi5}
\end{EntryWithPhonetic}

\begin{EntryWithPhonetic}{卷}{juan4}{8}{⼙}[HSK 4]
  \definition{clas.}{usado para capítulos, seções ou volumes; fascículos}
  \definition[大,小]{s.}{livro; livros e pinturas que são enrolados para coleção; geralmente se refere a pinturas e caligrafia | papel de exame | arquivo; dossiê}
  \seeref{juan3}
\end{EntryWithPhonetic}

\begin{EntryWithPhonetic}{卷子}{juan4zi5}{8,3}{⼙,⼦}[HSK 7-9]
  \definition{s.}{prova; prova de exame; um caderno fino ou uma única folha de papel para anotar as respostas durante as provas}
  \seeref{juan3zi5}
\end{EntryWithPhonetic}

%%%%%%%%%% 圈 %%%%%%%%%%
\subsection*{圈}\addcontentsline{loh}{figure}{圈 \dpy{juan4}}

\begin{EntryWithPhonetic}{圈}{juan4}{11}{⼞}[HSK 7-9]
  \definition*{s.}{Sobrenome: Juan}
  \definition{s.}{curral; local onde o gado ou as aves são mantidos, geralmente cercado ou murado, alguns com galpões}
  \seeref{juan1}
  \seeref{quan1}
\end{EntryWithPhonetic}

%%%%%%%%%% 决 %%%%%%%%%%
\subsection*{决}\addcontentsline{loh}{figure}{决 \dpy{jue2}}

\begin{EntryWithPhonetic}{决}{jue2}{6}{⼎}
  \definition{v.}{decidir; determinar | executar uma pessoa | (de um dique, etc.) romper; desabar}
\end{EntryWithPhonetic}

\begin{EntryWithPhonetic}{决不}{jue2bu4}{6,4}{⼎,⼀}[HSK 5]
  \definition{adv.}{em hipótese alguma; nunca | definitivamente não; certamente não; sob nenhuma circunstância; de forma alguma}
\end{EntryWithPhonetic}

\begin{EntryWithPhonetic}{决策}{jue2ce4}{6,12}{⼎,⽵}[HSK 6]
  \definition{s.}{decisão política; decisão de importância estratégica; estratégia ou método de decisão}
  \definition{v.}{formular políticas; tomar uma decisão estratégica; decidir sobre uma estratégia ou abordagem}
\end{EntryWithPhonetic}

\begin{EntryWithPhonetic}{决定}{jue2ding4}{6,8}{⼎,⼧}[HSK 3]
  \definition{adj.}{decisivo; as leis objetivas levam as coisas a se desenvolverem e mudarem em determinada direção}
  \definition[项,个]{s.}{decisão; resolução; assuntos decididos}
  \definition{v.}{decidir; determinar; algo se torna a base ou o pré-requisito para outra coisa; desempenha um papel dominante | decidir; resolver; tomar uma decisão; propor uma forma de agir}
\end{EntryWithPhonetic}

\begin{EntryWithPhonetic}{决赛}{jue2sai4}{6,14}{⼎,⾙}[HSK 3]
  \definition[场]{s.}{finais (de uma competição); em competições esportivas, a última partida ou rodada disputada para determinar a classificação}
\end{EntryWithPhonetic}

\begin{EntryWithPhonetic}{决心}{jue2xin1}{6,4}{⼎,⼼}[HSK 3]
  \definition{s.}{resolução; determinação; determinação inabalável}
  \definition{v.}{secidir-se; decidir fazer algo e não vacilar nem mudar de ideia}
\end{EntryWithPhonetic}

\begin{EntryWithPhonetic}{决议}{jue2yi4}{6,5}{⼎,⾔}[HSK 7-9]
  \definition[项]{s.}{resolução | resultado}
\end{EntryWithPhonetic}

%%%%%%%%%% 诀 %%%%%%%%%%
\subsection*{诀}\addcontentsline{loh}{figure}{诀 \dpy{jue2}}

\begin{EntryWithPhonetic}{诀}{jue2}{6}{⾔}
  \definition[条,个]{s.}{rimas; mnemônicos | jeito; truques do ofício; boas maneiras de resolver problemas}
  \definition{v.}{dar adeus; partir}
\end{EntryWithPhonetic}

\begin{EntryWithPhonetic}{诀别}{jue2bie2}{6,7}{⾔,⼑}[HSK 7-9]
  \definition{v.}{se despedir | separar-se (geralmente com pouca esperança de um novo encontro)}
\end{EntryWithPhonetic}

\begin{EntryWithPhonetic}{诀窍}{jue2qiao4}{6,10}{⾔,⽳}[HSK 7-9]
  \definition{s.}{talento; habilidade;  segredo do sucesso; truques do ofício}
  \seealsoref{诀窍儿}{jue2qiao4r5}
\end{EntryWithPhonetic}

\begin{EntryWithPhonetic}{诀窍儿}{jue2qiao4r5}{6,10,2}{⾔,⽳,⼉}
  \definition{s.}{segredo do sucesso; chave para o sucesso; truques do ofício; jeito para a coisa}
  \seealsoref{诀窍}{jue2qiao4}
\end{EntryWithPhonetic}

%%%%%%%%%% 角 %%%%%%%%%%
\subsection*{角}\addcontentsline{loh}{figure}{角 \dpy{jue2}}

\begin{EntryWithPhonetic}{角}{jue2}{7}{⾓}[Kangxi 148]
  \definition*{s.}{Sobrenome: Jue}
  \definition[个,只,对]{s.}{papel; parte; personagem | tipo de papel (no drama tradicional chinês); categorias de divisão profissional do trabalho entre atores de ópera | ator ou atriz | uma antiga taça de vinho com três pernas | uma nota da antiga escala chinesa de cinco tons, correspondente a 3 na notação musical numerada}
  \definition{v.}{competir; contender; lutar}
  \seeref{jiao3}
\end{EntryWithPhonetic}

\begin{EntryWithPhonetic}{角色}{jue2se4}{7,6}{⾓,⾊}[HSK 4]
  \definition{s.}{papel; personagem em uma peça; personagem representado por um ator | papel; função; parte; uma metáfora para um certo tipo de pessoas na vida social}
\end{EntryWithPhonetic}

\begin{EntryWithPhonetic}{角逐}{jue2zhu2}{7,10}{⾓,⾡}[HSK 7-9]
  \definition{v.}{contender; disputar; entrar em rivalidade; fazer malabarismos por}
\end{EntryWithPhonetic}

%%%%%%%%%% 绝 %%%%%%%%%%
\subsection*{绝}\addcontentsline{loh}{figure}{绝 \dpy{jue2}}

\begin{EntryWithPhonetic}{绝}{jue2}{9}{⽷}[HSK 6]
  \definition{adj.}{exausto; esgotado; acabado | desesperado; sem esperança | único; soberbo; incomparável | não deixar margem de manobra; não fazer concessões; intransigente}
  \definition{adv.}{extremamente; mais | (antes de uma negativa) absolutamente; no mínimo; por qualquer meio; em qualquer conta}
  \definition{s.}{(literário) jueju, um poema de quatro linhas}
  \definition{v.}{cortar; romper | parar de respirar; morrer}
\end{EntryWithPhonetic}

\begin{EntryWithPhonetic}{绝版}{jue2ban3}{9,8}{⽷,⽚}
  \definition{adj.}{esgotado | fora de catálogo}
\end{EntryWithPhonetic}

\begin{EntryWithPhonetic}{绝不}{jue2bu4}{9,4}{⽷,⼀}
  \definition{adv.}{definitivamente não | de forma alguma | sob nenhuma circunstância}
\end{EntryWithPhonetic}

\begin{EntryWithPhonetic}{绝大多数}{jue2da4duo1shu4}{9,3,6,13}{⽷,⼤,⼣,⽁}[HSK 6]
  \definition{expr.}{maioria absoluta | uma maioria esmagadora}
\end{EntryWithPhonetic}

\begin{EntryWithPhonetic}{绝对}{jue2dui4}{9,5}{⽷,⼨}[HSK 3]
  \definition{adj.}{absoluto; sem condições; sem restrições | absoluto; extremo; incompleto; sem margem para negociação ou alteração}
  \definition{adv.}{absolutamente; completamente; com certeza}
\end{EntryWithPhonetic}

\begin{EntryWithPhonetic}{绝技}{jue2ji4}{9,7}{⽷,⼿}[HSK 7-9]
  \definition{s.}{habilidade única; habilidade consumada | \emph{tour-de-force}; uma atuação ou conquista impressionante que foi realizada ou gerenciada com grande habilidade | façanha | feito supremo}
\end{EntryWithPhonetic}

\begin{EntryWithPhonetic}{绝望}{jue2/wang4}{9,11}{⽷,⽉}[HSK 5]
  \definition{v.+compl.}{desesperar; desistir de toda esperança; perder toda esperança de}
\end{EntryWithPhonetic}

\begin{EntryWithPhonetic}{绝缘}{jue2yuan2}{9,12}{⽷,⽷}[HSK 7-9]
  \definition{v.}{isolar; utilizar materiais como borracha ou madeira para bloquear a eletricidade, impedindo sua passagem | ser separado de; ser isolado de; completamente isolado do mundo exterior ou de um objeto específico}
\end{EntryWithPhonetic}

\begin{EntryWithPhonetic}{绝招}{jue2zhao1}{9,8}{⽷,⼿}[HSK 7-9]
  \definition{s.}{habilidade única; movimento delicado inesperado (como último recurso); habilidades únicas e magníficas; métodos engenhosos}
\end{EntryWithPhonetic}

%%%%%%%%%% 觉 %%%%%%%%%%
\subsection*{觉}\addcontentsline{loh}{figure}{觉 \dpy{jue2}}

\begin{EntryWithPhonetic}{觉}{jue2}{9}{⾒}
  \definition{s.}{sentimento; senso; percepção e discriminação de estímulos externos}
  \definition{v.}{sentir; perceber | acordar | tornar-se consciente; tornar-se desperto; despertar; entender}
  \seeref{jiao4}
\end{EntryWithPhonetic}

\begin{EntryWithPhonetic}{觉得}{jue2de5}{9,11}{⾒,⼻}[HSK 1]
  \definition{v.}{sentir; estar ciente; pressentir; causar uma sensação | pensar; sentir; encontrar; considerar (tom menos assertivo)}
\end{EntryWithPhonetic}

\begin{EntryWithPhonetic}{觉悟}{jue2wu4}{9,10}{⾒,⼼}[HSK 6]
  \definition{s.}{consciência; percepção; compreensão; nível de consciência}
  \definition{v.}{vir a compreender; tornar-se consciente de; tornar-se politicamente desperto; despertar}
\end{EntryWithPhonetic}

\begin{EntryWithPhonetic}{觉醒}{jue2xing3}{9,16}{⾒,⾣}[HSK 7-9]
  \definition{v.}{despertar; acordar; perceber; passar da confusão à clareza, do erro à correção, após reconhecer os erros e problemas}
\end{EntryWithPhonetic}

%%%%%%%%%% 倔 %%%%%%%%%%
\subsection*{倔}\addcontentsline{loh}{figure}{倔 \dpy{jue2}}

\begin{EntryWithPhonetic}{倔}{jue2}{10}{⼈}
  \definition{adj.}{rude; mal-humorado; abrupto; curto (uso limitado em 倔犟) | teimoso; direto e franco}
  \seeref{jue4}
\end{EntryWithPhonetic}

\begin{EntryWithPhonetic}{倔强}{jue2jiang4}{10,12}{⼈,⼸}[HSK 7-9]
  \definition{adj.}{teimoso; rígido; inflexível; de personalidade forte e teimosa}
\end{EntryWithPhonetic}

%%%%%%%%%% 崛 %%%%%%%%%%
\subsection*{崛}\addcontentsline{loh}{figure}{崛 \dpy{jue2}}

\begin{EntryWithPhonetic}{崛}{jue2}{11}{⼭}
  \definition{v.}{Literário: subir abruptamente; levantar-se abruptamente; empinar}
\end{EntryWithPhonetic}

\begin{EntryWithPhonetic}{崛起}{jue2qi3}{11,10}{⼭,⾛}[HSK 7-9]
  \definition{v.}{surgir abruptamente; subir repentinamente; aparecer de repente no horizonte | ganhar destaque; ascender}
\end{EntryWithPhonetic}

%%%%%%%%%% 脚 %%%%%%%%%%
\subsection*{脚}\addcontentsline{loh}{figure}{脚 \dpy{jue2}}

\begin{EntryWithPhonetic}{脚}{jue2}{11}{⾁}
  \variantof{角}
\end{EntryWithPhonetic}

%%%%%%%%%% 爵 %%%%%%%%%%
\subsection*{爵}\addcontentsline{loh}{figure}{爵 \dpy{jue2}}

\begin{EntryWithPhonetic}{爵}{jue2}{17}{⽖}
  \definition{s.}{grau de nobreza; aristocracia | o título de nobreza | Arcaico: um antigo recipiente para vinho com três pés e uma alça em forma de laço}
\end{EntryWithPhonetic}

\begin{EntryWithPhonetic}{爵士}{jue2shi4}{17,3}{⽖,⼠}[HSK 7-9]
  \definition*{s.}{\emph{Sir} (Senhor); Título honorífico para cavalheiros da Ordem do Império Britânico}
  \definition{s.}{cavaleiro; o título mais baixo de uma monarquia europeia | Empréstimo linguístico: jazz}[它赞助大型爵士音乐会。===A empresa patrocina concertos de jazz de grande escala.]
\end{EntryWithPhonetic}

%%%%%%%%%% 嚼 %%%%%%%%%%
\subsection*{嚼}\addcontentsline{loh}{figure}{嚼 \dpy{jue2}}

\begin{EntryWithPhonetic}{嚼}{jue2}{20}{⼝}
  \definition{v.}{mastigar; morder; mastigar completamente; é usado em algumas palavras compostas e expressões idiomáticas; usado em 咀嚼}
  \seeref{jiao2}
  \seeref{jiao4}
  \seealsoref{咀嚼}{ju3jue2}
\end{EntryWithPhonetic}

%%%%%%%%%% 倔 %%%%%%%%%%
\subsection*{倔}\addcontentsline{loh}{figure}{倔 \dpy{jue4}}

\begin{EntryWithPhonetic}{倔}{jue4}{10}{⼈}[HSK 7-9]
  \definition{adj.}{teimoso; direto; rude; grosseiro; de natureza direta, com uma atitude severa em relação aos outros}
  \seeref{jue2}
\end{EntryWithPhonetic}

%%%%%%%%%% 军 %%%%%%%%%%
\subsection*{军}\addcontentsline{loh}{figure}{军 \dpy{jun1}}

\begin{EntryWithPhonetic}{军}{jun1}{6}{⼍}
  \definition*{s.}{Sobrenome: Jun}
  \definition{s.}{forças armadas; exército; tropas | exército; contingente; muitas pessoas participando de uma atividade | exército; unidades militares}
\end{EntryWithPhonetic}

\begin{EntryWithPhonetic}{军队}{jun1dui4}{6,4}{⼍,⾩}[HSK 6]
  \definition[支,个]{s.}{forças armadas; exército; tropas}
\end{EntryWithPhonetic}

\begin{EntryWithPhonetic}{军官}{jun1guan1}{6,8}{⼍,⼧}[HSK 7-9]
  \definition{s.}{oficial; militares com patente igual ou superior a tenente, também se refere a oficiais com patente igual ou superior a comandante de pelotão nas forças armadas}
\end{EntryWithPhonetic}

\begin{EntryWithPhonetic}{军舰}{jun1jian4}{6,10}{⼍,⾈}[HSK 6]
  \definition[艘,只]{s.}{navio de guerra; embarcação naval | \emph{warcraft}; um termo geral para embarcações militares equipadas com armas e equipamentos que podem executar missões de combate, incluindo principalmente navios de guerra, cruzadores, contratorpedeiros, porta-aviões, submarinos, torpedeiros, etc.}
\end{EntryWithPhonetic}

\begin{EntryWithPhonetic}{军人}{jun1ren2}{6,2}{⼍,⼈}[HSK 5]
  \definition[名,位,个]{s.}{soldado; militar; pessoal militar; pessoas com status militar; pessoas servindo nas forças armadas}
\end{EntryWithPhonetic}

\begin{EntryWithPhonetic}{军事}{jun1shi4}{6,8}{⼍,⼅}[HSK 6]
  \definition{s.}{militar; assuntos militares; assuntos relativos aos militares e à guerra}
\end{EntryWithPhonetic}

\begin{EntryWithPhonetic}{军装}{jun1zhuang1}{6,12}{⼍,⾐}
  \definition{s.}{uniforme militar}
\end{EntryWithPhonetic}

%%%%%%%%%% 君 %%%%%%%%%%
\subsection*{君}\addcontentsline{loh}{figure}{君 \dpy{jun1}}

\begin{EntryWithPhonetic}{君}{jun1}{7}{⼝}
  \definition*{s.}{Sobrenome: Jun}
  \definition[个,位,名,些]{s.}{monarca; soberano; governante supremo | (como título) Senhor; Sr. | (literário) (em trato direto) você; senhor | cavalheiro | governante}
\end{EntryWithPhonetic}

\begin{EntryWithPhonetic}{君主立宪制}{jun1zhu3li4xian4zhi4}{7,5,5,9,8}{⼝,⼂,⽴,⼧,⼑}
  \definition{s.}{monarquia constitucional}
\end{EntryWithPhonetic}

\begin{EntryWithPhonetic}{君子}{jun1zi3}{7,3}{⼝,⼦}[HSK 7-9]
  \definition{s.}{cavalheiro; nobre; pessoa de caráter nobre; refere"-se a uma pessoa de elevado caráter moral}
\end{EntryWithPhonetic}

%%%%%%%%%% 均 %%%%%%%%%%
\subsection*{均}\addcontentsline{loh}{figure}{均 \dpy{jun1}}

\begin{EntryWithPhonetic}{均}{jun1}{7}{⼟}
  \definition{adj.}{igual; equilibrado; uniforme; igual em quantidade}
  \definition{adv.}{todos; sem exceção}
  \definition{v.}{ser igual a; igualar as quantidades; dividir igualmente}
\end{EntryWithPhonetic}

\begin{EntryWithPhonetic}{均衡}{jun1heng2}{7,16}{⼟,⾏}[HSK 7-9]
  \definition{adj.}{uniforme; equilibrado; proporcional}
\end{EntryWithPhonetic}

\begin{EntryWithPhonetic}{均匀}{jun1yun2}{7,4}{⼟,⼓}[HSK 7-9]
  \definition{adj.}{uniforme; homogêneo; regular; as quantidades são distribuídas ou alocadas igualmente em cada parte; os intervalos de tempo são iguais}
\end{EntryWithPhonetic}

%%%%%%%%%% 俊 %%%%%%%%%%
\subsection*{俊}\addcontentsline{loh}{figure}{俊 \dpy{jun4}}

\begin{EntryWithPhonetic}{俊}{jun4}{9}{⼈}[HSK 7-9]
  \definition*{s.}{Sobrenome: Jun}
  \definition{adj.}{bonito; lindo; encantador; atraente; delicado e bonito | talentoso; inteligente; brilhante; excepcional; excepcionalmente inteligente}
  \definition{s.}{uma pessoa de talento excepcional; pessoas com inteligência excepcional}
\end{EntryWithPhonetic}

\begin{EntryWithPhonetic}{俊俏}{jun4qiao4}{9,9}{⼈,⼈}[HSK 7-9]
  \definition{adj.}{Coloquial: bonito e encantador}
\end{EntryWithPhonetic}

%%%%%%%%%% 骏 %%%%%%%%%%
\subsection*{骏}\addcontentsline{loh}{figure}{骏 \dpy{jun4}}

\begin{EntryWithPhonetic}{骏}{jun4}{10}{⾺}
  \definition{s.}{belo cavalo; corcel; animal de montaria}
\end{EntryWithPhonetic}

\begin{EntryWithPhonetic}{骏马}{jun4ma3}{10,3}{⾺,⾺}[HSK 7-9]
  \definition[匹,群]{s.}{belo cavalo; corcel; animal de montaria}
\end{EntryWithPhonetic}

%%%%%%%%%% 竣 %%%%%%%%%%
\subsection*{竣}\addcontentsline{loh}{figure}{竣 \dpy{jun4}}

\begin{EntryWithPhonetic}{竣}{jun4}{12}{⽴}
  \definition{v.}{concluir; terminar; finalizar}
\end{EntryWithPhonetic}

\begin{EntryWithPhonetic}{竣工}{jun4gong1}{12,3}{⽴,⼯}[HSK 7-9]
  \definition{v.}{ser concluído, finalizado (projetos)}
\end{EntryWithPhonetic}

%%%%% EOF %%%%%


 %%%
%%% K
%%%
\section*{K}\addcontentsline{toc}{section}{K}\addcontentsline{loh}{figure}{\#\#\#\#\#\#\#\# K}

%%%%%%%%%% 咖 %%%%%%%%%%
\subsection*{咖}\addcontentsline{loh}{figure}{咖 \dpy{ka1}}

\begin{EntryWithPhonetic}{咖}{ka1}{8}{⼝}
  \definition[杯]{s.}{classe | café | graduação}
\end{EntryWithPhonetic}

\begin{EntryWithPhonetic}{咖啡}{ka1fei1}{8,11}{⼝,⼝}[HSK 3]
  \definition[杯,瓶,罐,壶,包,袋,盒]{s.}{(empréstimo linguístico) café}
\end{EntryWithPhonetic}

\begin{EntryWithPhonetic}{咖啡馆}{ka1fei1guan3}{8,11,11}{⼝,⼝,⾷}
  \definition[家]{s.}{cafeteria}
\end{EntryWithPhonetic}

\begin{EntryWithPhonetic}{咖啡色}{ka1fei1 se4}{8,11,6}{⼝,⼝,⾊}
  \definition{s.}{cor café}
\end{EntryWithPhonetic}

%%%%%%%%%% 卡 %%%%%%%%%%
\subsection*{卡}\addcontentsline{loh}{figure}{卡 \dpy{ka3}}

\begin{EntryWithPhonetic}{卡}{ka3}{5}{⼘}[HSK 2]
  \definition{clas.}{calorias (cal)}
  \definition[张,片]{s.}{cartão; documento semelhante a um cartão | cassete; dispositivo tipo compartimento para colocar fitas cassete no gravador | caminhão}
  \seeref{qia3}
\end{EntryWithPhonetic}

\begin{EntryWithPhonetic}{卡车}{ka3che1}{5,4}{⼘,⾞}[HSK 7-9]
  \definition[辆]{s.}{caminhão; caminhões pesados para transporte de mercadorias, equipamentos, etc.}
\end{EntryWithPhonetic}

\begin{EntryWithPhonetic}{卡车司机}{ka3che1 si1ji1}{5,4,5,6}{⼘,⾞,⼝,⽊}
  \definition{s.}{motorista de caminhão}
\end{EntryWithPhonetic}

\begin{EntryWithPhonetic}{卡片}{ka3pian4}{5,4}{⼘,⽚}[HSK 7-9]
  \definition[张,盒,套]{s.}{cartão; pedaços de papel usados para registrar diversas informações para comparação, verificação e referência}
\end{EntryWithPhonetic}

\begin{EntryWithPhonetic}{卡片游戏}{ka3pian4 you2xi4}{5,4,12,6}{⼘,⽚,⽔,⼽}
  \definition{s.}{carta de baralho; jogos de cartas}
\end{EntryWithPhonetic}

\begin{EntryWithPhonetic}{卡通}{ka3tong1}{5,10}{⼘,⾡}[HSK 7-9]
  \definition[本]{s.}{Empréstimo linguístico: \emph{cartoon}; desenho animado}
\end{EntryWithPhonetic}

%%%%%%%%%% 开 %%%%%%%%%%
\subsection*{开}\addcontentsline{loh}{figure}{开 \dpy{kai1}}

\begin{EntryWithPhonetic}{开}{kai1}{4}{⼶}[HSK 1]
  \definition*{s.}{Sobrenome: Kai}
  \definition{clas.}{divisão do papel de impressão de tamanho padrão (uma parte da folha inteira) | quilate; unidade de cálculo da quantidade de ouro puro contida no ouro}
  \definition{s.}{porcentagem; percentual}
  \definition{v.}{abrir; estar ligado; ligar | recuperar; abrir; fazer uma abertura; escavar; abrir caminho; desbravar | abrir para fora; soltar-se | descongelar (rios); tornar-se navegável | levantar; libertar | iniciar; operar; manobrar | mover; estabelecer | executar; configurar | começar; iniciar | manter | escrever; fazer uma lista de | pagamento (salários, tarifas, etc.) | ferver}
  \definition{v.aux.}{usado após um verbo, indica ampliação ou expansão | usado após um verbo, indica o início e a continuidade}
  \seealsoref{开尔文}{kai1'er3wen2}
\end{EntryWithPhonetic}

\begin{EntryWithPhonetic}{开办}{kai1ban4}{4,4}{⼶,⼒}[HSK 7-9]
  \definition{v.}{abrir; iniciar; criar; configurar; estabelecer (uma empresa ou entidade comercial)}
\end{EntryWithPhonetic}

\begin{EntryWithPhonetic}{开采}{kai1cai3}{4,8}{⼶,⾤}[HSK 7-9]
  \definition{v.}{minerar; extrair; explorar; recuperar}
\end{EntryWithPhonetic}

\begin{EntryWithPhonetic}{开场}{kai1/chang3}{4,6}{⼶,⼟}[HSK 7-9]
  \definition{v.+compl.}{(apresentação, etc.) começar; iniciar; estrear; o início de uma apresentação teatral ou de uma apresentação cultural em geral também pode ser usado metaforicamente para descrever o início de uma atividade geral}
\end{EntryWithPhonetic}

\begin{EntryWithPhonetic}{开场白}{kai1chang3bai2}{4,6,5}{⼶,⼟,⽩}[HSK 7-9]
  \definition{s.}{prólogo (de uma peça); introdução; discurso de abertura; comentários iniciais (ou introdutórios); as linhas iniciais de uma peça teatral que introduzem o tema; metaforicamente, a seção inicial de um artigo ou discurso que apresenta a ideia principal}
\end{EntryWithPhonetic}

\begin{EntryWithPhonetic}{开车}{kai1/che1}{4,4}{⼶,⾞}[HSK 1]
  \definition{v.+compl.}{dirigir um carro, trem, etc. | colocar uma máquina em funcionamento | (de um trem, etc.) partida | dirigir veículos motorizados}
\end{EntryWithPhonetic}

\begin{EntryWithPhonetic}{开除}{kai1chu2}{4,9}{⼶,⾩}[HSK 7-9]
  \definition{v.}{expulsar; demitir; dispensar; despedir}
\end{EntryWithPhonetic}

\begin{EntryWithPhonetic}{开创}{kai1chuang4}{4,6}{⼶,⼑}[HSK 6]
  \definition{v.}{começar; iniciar; fundar; ser pioneiro; estabelecer; criar}
\end{EntryWithPhonetic}

\begin{EntryWithPhonetic}{开动}{kai1dong4}{4,6}{⼶,⼒}[HSK 7-9]
  \definition{v.}{iniciar; pôr em movimento; pôr em funcionamento; iniciar operação | mover-se; marchar; estar em movimento; zarpar}
\end{EntryWithPhonetic}

\begin{EntryWithPhonetic}{开尔文}{kai1'er3wen2}{4,5,4}{⼶,⼩,⽂}
  \definition{s.}{Kelvin, temperatura absoluta | K, escala de temperatura}
\end{EntryWithPhonetic}

\begin{EntryWithPhonetic}{开发区}{kai1fa1qu1}{4,5,4}{⼶,⼜,⼖}[HSK 7-9]
  \definition*{s.}{Zona Econômica Aberta; Zona Econômica Especial}
  \definition{s.}{zona de desenvolvimento}
\end{EntryWithPhonetic}

\begin{EntryWithPhonetic}{开发商}{kai1fa1shang1}{4,5,11}{⼶,⼜,⼝}[HSK 7-9]
  \definition{s.}{incorporador (de imóveis, de um produto comercial, etc.)}
\end{EntryWithPhonetic}

\begin{EntryWithPhonetic}{开发}{kai1fa5}{4,5}{⼶,⼜}[HSK 3]
  \definition{v.}{explorar; trabalhar com recursos naturais como terras baldias, minas, florestas e energia hidráulica para fins de aproveitamento | tornar acessível; descobrir ou explorar talentos, tecnologias, etc. para aproveitamento}
\end{EntryWithPhonetic}

\begin{EntryWithPhonetic}{开放}{kai1fang4}{4,8}{⼶,⽅}[HSK 3]
  \definition{adj.}{de mente aberta; sem restrições por convenções; pensamento e ambiente não conservadores, disposto a aceitar coisas novas e novas ideias; personalidade alegre}
  \definition{v.}{florescer | abrir (para o público); levantar bloqueios, proibições, restrições, etc. | diminuir uma proibição, restrição, etc. (de política); (economia) reduzir as restrições políticas, com justificativas específicas}
\end{EntryWithPhonetic}

\begin{EntryWithPhonetic}{开工}{kai1/gong1}{4,3}{⼶,⼯}[HSK 7-9]
  \definition{v.+compl.}{(fábrica, etc.) entrar em operação; iniciar a construção}
\end{EntryWithPhonetic}

\begin{EntryWithPhonetic}{开关}{kai1guan1}{4,6}{⼶,⼋}[HSK 6]
  \definition[个,种,些]{s.}{interruptor; um dispositivo que conecta e desconecta o circuito de um dispositivo elétrico | registro; um dispositivo instalado em uma tubulação de fluido para controlar o fluxo}
\end{EntryWithPhonetic}

\begin{EntryWithPhonetic}{开花}{kai1/hua1}{4,7}{⼶,⾋}[HSK 4]
  \definition{v.+compl.}{florescer; desabrochar; estar em flor; entrar em flor;  metáfora para um coração feliz ou um rosto sorridente | explodir; quebrar; dividir | sentir-se feliz ou sorrir alegremente | (experiência) espalhar-se; (empreendimento) surgir; surgir | (cabeça) ser ferido e sangrar profusamente}
\end{EntryWithPhonetic}

\begin{EntryWithPhonetic}{开会}{kai1/hui4}{4,6}{⼶,⼈}[HSK 1]
  \definition{v.+compl.}{realizar uma reunião; ter uma reunião; participar de uma reunião (conferência)}
\end{EntryWithPhonetic}

\begin{EntryWithPhonetic}{开机}{kai1 ji1}{4,6}{⼶,⽊}[HSK 2]
  \definition{v.}{começar a filmar um filme ou programa de TV; refere"-se ao início das filmagens (de filmes, séries de TV, etc.) | ligar uma máquina}
\end{EntryWithPhonetic}

\begin{EntryWithPhonetic}{开垦}{kai1ken3}{4,9}{⼶,⼟}[HSK 7-9]
  \definition{v.}{abrir (ou recuperar) terrenos baldios; trazer para o cultivo; abrir terrenos baldios para a agricultura; preparar o terreno; cultivar}
\end{EntryWithPhonetic}

\begin{EntryWithPhonetic}{开口}{kai1/kou3}{4,3}{⼶,⼝}[HSK 7-9]
  \definition{s.}{abertura; um corte ou rachadura}
  \definition{v.+compl.}{abrir a boca; começar a falar; abrir a boca e falar | pedir algo a alguém; fazer um pedido ou uma exigência |  afiar (uma faca nova ou uma tesoura nova) | cavar uma brecha; encontrar uma brecha; abrir uma fenda}
\end{EntryWithPhonetic}

\begin{EntryWithPhonetic}{开阔}{kai1kuo4}{4,12}{⼶,⾨}[HSK 7-9]
  \definition{adj.}{aberto; amplo; largo | de mente aberta}
  \definition{v.}{ampliar; abrir}
\end{EntryWithPhonetic}

\begin{EntryWithPhonetic}{开朗}{kai1lang3}{4,10}{⼶,⽉}[HSK 7-9]
  \definition{adj.}{otimista; alegre; despreocupado; (pensamentos, personalidade e mentalidade) otimista, alegre e não melancólico ou deprimido | espaçoso; bem iluminado; aberto e desimpedido; aberto e luminoso}
\end{EntryWithPhonetic}

\begin{EntryWithPhonetic}{开幕}{kai1 mu4}{4,13}{⼶,⼱}[HSK 5]
  \definition{v.}{começar a apresentação; iniciar o espetáculo; levantar das cortinas | abrir; inaugurar; iniciar (uma conferência, exposição, etc.)}
\end{EntryWithPhonetic}

\begin{EntryWithPhonetic}{开幕式}{kai1mu4shi4}{4,13,6}{⼶,⼱,⼷}[HSK 5]
  \definition[场,次,届]{s.}{cerimônia de abertura; cerimônias e apresentações antes de eventos esportivos ou grandes eventos}
\end{EntryWithPhonetic}

\begin{EntryWithPhonetic}{开辟}{kai1pi4}{4,13}{⼶,⾟}[HSK 7-9]
  \definition{v.}{abrir; talhar; romper; abrir passagem | abrir; iniciar; desenvolver; explorar; ampliar; expandir | abrir; iniciar; fundar; estabelecer; criar}
\end{EntryWithPhonetic}

\begin{EntryWithPhonetic}{开启}{kai1qi3}{4,7}{⼶,⼝}[HSK 7-9]
  \definition{v.}{abrir | iniciar | Computação: ativar}
\end{EntryWithPhonetic}

\begin{EntryWithPhonetic}{开枪}{kai1 qiang1}{4,8}{⼶,⽊}[HSK 7-9]
  \definition{v.}{disparar com um rifle, pistola, etc.; disparar um tiro; atirar}
\end{EntryWithPhonetic}

\begin{EntryWithPhonetic}{开设}{kai1she4}{4,6}{⼶,⾔}[HSK 6]
  \definition{v.}{montar; estabelecer; abrir (uma loja, fábrica, etc.); estabelecer novas instituições ou campos | oferecer (um curso na faculdade, etc.)}
\end{EntryWithPhonetic}

\begin{EntryWithPhonetic}{开始}{kai1shi3}{4,8}{⼶,⼥}[HSK 3]
  \definition[个]{s.}{começo; início; estágio inicial}
  \definition{v.}{começar; iniciar; começar a fazer algo}
\end{EntryWithPhonetic}

\begin{EntryWithPhonetic}{开水}{kai1shui3}{4,4}{⼶,⽔}[HSK 4]
  \definition[杯,瓶]{s.}{água fervida; água fervente}
\end{EntryWithPhonetic}

\begin{EntryWithPhonetic}{开锁}{kai1suo3}{4,12}{⼶,⾦}
  \definition{v.}{desbloquear | destravar}
\end{EntryWithPhonetic}

\begin{EntryWithPhonetic}{开天辟地}{kai1tian1-pi4di4}{4,4,13,6}{⼶,⼤,⾟,⼟}[HSK 7-9]
  \definition{expr.}{``Criação do Céu e da Terra.''; a criação do céu e da terra; quando o céu se separou da terra; a criação do mundo; no princípio do céu e da terra (Gênesis) | desde o alvorecer da história; desde o início da história | que marca época; inovador, revolucionário, que inaugura uma nova era}
\end{EntryWithPhonetic}

\begin{EntryWithPhonetic}{开通}{kai1tong1}{4,10}{⼶,⾡}[HSK 6]
  \definition{v.}{limpar; dragar; remover obstáculos de; abrir o canal; desbloquear}
  \seeref{kai1tong5}
\end{EntryWithPhonetic}

\begin{EntryWithPhonetic}{开通}{kai1tong5}{4,10}{⼶,⾡}
  \definition{adj.}{liberal; mente aberta; mente moderna; mente liberal; sábio e sensato; não conservador ou teimoso}
  \seeref{kai1tong1}
\end{EntryWithPhonetic}

\begin{EntryWithPhonetic}{开头}{kai1/tou2}{4,5}{⼶,⼤}[HSK 6]
  \definition{s.}{início; começo; o momento ou estágio do início; antecedente no tempo}
  \definition{v.+compl.}{começar, iniciar; a primeira ocorrência de um evento, ação, fenômeno, etc. | pôr-se a pé; começar}
\end{EntryWithPhonetic}

\begin{EntryWithPhonetic}{开拓}{kai1tuo4}{4,8}{⼶,⼿}[HSK 7-9]
  \definition{v.}{desenvolver; ser pioneiro; abrir caminho; expandir; abrir}
\end{EntryWithPhonetic}

\begin{EntryWithPhonetic}{开玩笑}{kai1 wan2xiao4}{4,8,10}{⼶,⽟,⽵}[HSK 1]
  \definition{v.}{fazer (ou brincar, fazer) uma piada; gracejar; zombar de; provocar; fazer uma brincadeira; zombar de alguém | tratar casualmente; dar pouca importância a; considerar como um assunto insignificante; insignificante | fazer uma brincadeira; pregar uma peça; brincar; em tom de brincadeira}
\end{EntryWithPhonetic}

\begin{EntryWithPhonetic}{开销}{kai1xiao1}{4,12}{⼶,⾦}[HSK 7-9]
  \definition[笔,项]{s.}{despesas; taxas pagas}
  \definition{v.}{pagar despesas (taxas)}
\end{EntryWithPhonetic}

\begin{EntryWithPhonetic}{开心}{kai1/xin1}{4,4}{⼶,⼼}[HSK 2]
  \definition{adj.}{feliz; alegre; exultante; encantado}
  \definition{v.+compl.}{provocar; brincar; tirar sarro de alguém; zombar; divertir-se}
\end{EntryWithPhonetic}

\begin{EntryWithPhonetic}{开学}{kai1 xue2}{4,8}{⼶,⼦}[HSK 2]
  \definition{v.}{iniciar as aulas; iniciar o semestre; começar as aulas}
\end{EntryWithPhonetic}

\begin{EntryWithPhonetic}{开业}{kai1 ye4}{4,5}{⼶,⼀}[HSK 3]
  \definition[场]{v.}{iniciar um negócio; abrir para negócios | abrir um consultório particular}
\end{EntryWithPhonetic}

\begin{EntryWithPhonetic}{开夜车}{kai1/ye4che1}{4,8,4}{⼶,⼣,⾞}[HSK 6]
  \definition{v.+compl.}{``Dirigir à noite.''; ``Conduzir carro à noite.''; trabalhar até tarde da noite; ficar acordado até tarde da noite estudando ou trabalhando para cumprir prazos}
\end{EntryWithPhonetic}

\begin{EntryWithPhonetic}{开展}{kai1zhan3}{4,10}{⼶,⼫}[HSK 3]
  \definition{v.}{lançar; desenvolver | abrir; inaugurar}
\end{EntryWithPhonetic}

\begin{EntryWithPhonetic}{开张}{kai1/zhang1}{4,7}{⼶,⼸}[HSK 7-9]
  \definition{v.+compl.}{estrear; abrir um negócio; começar a fazer negócios; lojas, hotéis, etc., recém-construídos, começam a abrir as portas | fazer a primeira transação do dia; para empresários, isso se refere à primeira transação do dia}
  \antonymref{关张}{guan1zhang1}
\end{EntryWithPhonetic}

\begin{EntryWithPhonetic}{开支}{kai1zhi1}{4,4}{⼶,⽀}[HSK 7-9]
  \definition{v.}{pagar (despesas); gastar | pagar salários; receber o pagamento}
\end{EntryWithPhonetic}

%%%%%%%%%% 凯 %%%%%%%%%%
\subsection*{凯}\addcontentsline{loh}{figure}{凯 \dpy{kai3}}

\begin{EntryWithPhonetic}{凯}{kai3}{8}{⼏}
  \definition*{s.}{Sobrenome: Kai}
  \definition{adj.}{vitorioso; triunfante}
  \definition{s.}{canção da vitória; canção triunfal | vitória; triunfo}
\end{EntryWithPhonetic}

\begin{EntryWithPhonetic}{凯歌}{kai3ge1}{8,14}{⼏,⽋}[HSK 7-9]
  \definition{s.}{canção de triunfo (ou vitória); hino; canções cantadas após uma vitória}
\end{EntryWithPhonetic}

%%%%%%%%%% 楷 %%%%%%%%%%
\subsection*{楷}\addcontentsline{loh}{figure}{楷 \dpy{kai3}}

\begin{EntryWithPhonetic}{楷}{kai3}{13}{⽊}
  \definition*{s.}{Sobrenome: Kai}
  \definition{s.}{modelo; padrão | escrita regular (em caligrafia chinesa)}
\end{EntryWithPhonetic}

\begin{EntryWithPhonetic}{楷模}{kai3mo2}{13,14}{⽊,⽊}[HSK 7-9]
  \definition{s.}{modelo; exemplo a seguir; modelo a ser seguido; exemplos notáveis}
\end{EntryWithPhonetic}

%%%%%%%%%% 刊 %%%%%%%%%%
\subsection*{刊}\addcontentsline{loh}{figure}{刊 \dpy{kan1}}

\begin{EntryWithPhonetic}{刊}{kan1}{5}{⼑}
  \definition{s.}{periódico; publicação (jornais, revistas, etc., excluindo livros) | (geralmente em um jornal) coluna especial}
  \definition{v.}{Literário: cortar; picar | Literário: esculpir; gravar | apagar ou corrigir | imprimir; publicar; publicar em um jornal ou revista}
\end{EntryWithPhonetic}

\begin{EntryWithPhonetic}{刊登}{kan1deng1}{5,12}{⼑,⽨}[HSK 7-9]
  \definition{v.}{lançar; publicar}
\end{EntryWithPhonetic}

\begin{EntryWithPhonetic}{刊物}{kan1wu4}{5,8}{⼑,⽜}[HSK 7-9]
  \definition[份,本,家,期]{s.}{jornal; periódico; publicação; revistas publicadas regularmente ou irregularmente geralmente contêm artigos, imagens, etc.}
\end{EntryWithPhonetic}

%%%%%%%%%% 看 %%%%%%%%%%
\subsection*{看}\addcontentsline{loh}{figure}{看 \dpy{kan1}}

\begin{EntryWithPhonetic}{看}{kan1}{9}{⽬}[HSK 6]
  \definition{v.}{cuidar de; tomar conta de; cuidar de; proteger | manter sob vigilância}
  \seeref{kan4}
\end{EntryWithPhonetic}

\begin{EntryWithPhonetic}{看管}{kan1guan3}{9,14}{⽬,⽵}[HSK 6]
  \definition{v.}{cuidar; atender | guardar; vigiar; ficar de olho em | assumir o comando; estar no comando}
\end{EntryWithPhonetic}

\begin{EntryWithPhonetic}{看护}{kan1hu4}{9,7}{⽬,⼿}[HSK 7-9]
  \definition{s.}{Obsoleto: enfermeira hospitalar}
  \definition{v.}{cuidar; zelar por; tratar de}
\end{EntryWithPhonetic}

%%%%%%%%%% 勘 %%%%%%%%%%
\subsection*{勘}\addcontentsline{loh}{figure}{勘 \dpy{kan1}}

\begin{EntryWithPhonetic}{勘}{kan1}{11}{⼒}
  \definition{v.}{ler e corrigir; conferir | investigar; realizar levantamento | ler e corrigir o texto de; revisar}
\end{EntryWithPhonetic}

\begin{EntryWithPhonetic}{勘探}{kan1tan4}{11,11}{⼒,⼿}[HSK 7-9]
  \definition{v.}{examinar; prospectar; investigar a distribuição de depósitos minerais e determinar a localização, forma, tamanho, regularidade metalogenética, propriedades das rochas e estrutura geológica dos corpos de minério}
\end{EntryWithPhonetic}

%%%%%%%%%% 堪 %%%%%%%%%%
\subsection*{堪}\addcontentsline{loh}{figure}{堪 \dpy{kan1}}

\begin{EntryWithPhonetic}{堪}{kan1}{12}{⼟}
  \definition*{s.}{Sobrenome: Kan}
  \definition{v.}{pode; consegue | suportar; resistir; aguentar}
\end{EntryWithPhonetic}

\begin{EntryWithPhonetic}{堪称}{kan1cheng1}{12,10}{⼟,⽲}[HSK 7-9]
  \definition{v.}{pode ser classificado como; pode ser chamado assim; merece ser chamado assim}
\end{EntryWithPhonetic}

%%%%%%%%%% 侃 %%%%%%%%%%
\subsection*{侃}\addcontentsline{loh}{figure}{侃 \dpy{kan3}}

\begin{EntryWithPhonetic}{侃}{kan3}{8}{⼈}
  \definition{adj.}{íntegro e honesto; reto e franco; direto | amável; agradável | animado; alegre}
  \definition{v.}{Coloquial: bater papo ociosamente; conversar à toa; fofocar | gabar-se | conversar fluentemente}
\end{EntryWithPhonetic}

\begin{EntryWithPhonetic}{侃大山}{kan3 da4shan1}{8,3,3}{⼈,⼤,⼭}[HSK 7-9]
  \definition{v.}{fofocar; dedurar; bater papo; bater um papo; conversar fiado}
\end{EntryWithPhonetic}

%%%%%%%%%% 砍 %%%%%%%%%%
\subsection*{砍}\addcontentsline{loh}{figure}{砍 \dpy{kan3}}

\begin{EntryWithPhonetic}{砍}{kan3}{9}{⽯}[HSK 7-9]
  \definition{v.}{cortar; picar; talhar; desbastar; derrubar; podar | reduzir; diminuir; remover | Dialeto: atirar algo em | Coloquial: conversar à toa; fofocar}
\end{EntryWithPhonetic}

\begin{EntryWithPhonetic}{砍刀}{kan3dao1}{9,2}{⽯,⼑}
  \definition{s.}{facão | machete}
\end{EntryWithPhonetic}

\begin{EntryWithPhonetic}{砍掉}{kan3diao4}{9,11}{⽯,⼿}
  \definition{v.}{amputar}
\end{EntryWithPhonetic}

\begin{EntryWithPhonetic}{砍断}{kan3duan4}{9,11}{⽯,⽄}
  \definition{v.}{cortar}
\end{EntryWithPhonetic}

\begin{EntryWithPhonetic}{砍价}{kan3jia4}{9,6}{⽯,⼈}
  \definition{v.}{barganhar | cortar ou derrubar um preço}
\end{EntryWithPhonetic}

\begin{EntryWithPhonetic}{砍杀}{kan3sha1}{9,6}{⽯,⽊}
  \definition{v.}{atacar com arma branca}
\end{EntryWithPhonetic}

\begin{EntryWithPhonetic}{砍伤}{kan3shang1}{9,6}{⽯,⼈}
  \definition{v.}{ferir com lâmina ou machado}
\end{EntryWithPhonetic}

\begin{EntryWithPhonetic}{砍树}{kan3shu4}{9,9}{⽯,⽊}
  \definition{v.}{derrubar árvores}
\end{EntryWithPhonetic}

\begin{EntryWithPhonetic}{砍死}{kan3si3}{9,6}{⽯,⽍}
  \definition{v.}{matar com um machado}
\end{EntryWithPhonetic}

\begin{EntryWithPhonetic}{砍头}{kan3tou2}{9,5}{⽯,⼤}
  \definition{v.}{decapitar}
\end{EntryWithPhonetic}

%%%%%%%%%% 看 %%%%%%%%%%
\subsection*{看}\addcontentsline{loh}{figure}{看 \dpy{kan4}}

\begin{EntryWithPhonetic}{看}{kan4}{9}{⽬}[HSK 1]
  \definition{interj.}{``Cuidado!'' (para um perigo)}
  \definition{part.}{tentar, usado depois de outros verbos}
  \definition{v.}{ver; olhar para; observar; fazer contato visual com pessoas ou objetos | pensar; considerar; observar; julgar; observar e analisar | visitar; ver; fazer uma visita | olhar para; considerar; tratar | tratar (um paciente ou uma doença) | cuidar | ficar atento; ficar de olho | depender de; ser dependente de | ler}
  \seeref{kan1}
\end{EntryWithPhonetic}

\begin{EntryWithPhonetic}{看病}{kan4/bing4}{9,10}{⽬,⽧}[HSK 1]
  \definition{v.+compl.}{(de um médico) ver um paciente | (de um paciente) ver (consultar) um médico}
\end{EntryWithPhonetic}

\begin{EntryWithPhonetic}{看不起}{kan4bu5qi3}{9,4,10}{⽬,⼀,⾛}[HSK 4]
  \definition{v.}{desprezar; desdenhar; menosprezar; ter desprezo; olhar de cima para baixo}
\end{EntryWithPhonetic}

\begin{EntryWithPhonetic}{看成}{kan4cheng2}{9,6}{⽬,⼽}[HSK 5]
  \definition{v.}{ser capaz de ver ou assistir | tomar como; olhar como; considerar como | tratar como; considerar como; pensar como; ter como}
\end{EntryWithPhonetic}

\begin{EntryWithPhonetic}{看出}{kan4 chu1}{9,5}{⽬,⼐}[HSK 5]
  \definition{v.}{decifrar; ver; sondar; encontrar; discernir; perceber | descobrir; estar ciente de}
\end{EntryWithPhonetic}

\begin{EntryWithPhonetic}{看待}{kan4dai4}{9,9}{⽬,⼻}[HSK 5]
  \definition{v.}{tratar; considerar; olhar com atenção; ter uma certa atitude ou visão em relação a alguém ou alguma coisa}
\end{EntryWithPhonetic}

\begin{EntryWithPhonetic}{看淡}{kan4dan4}{9,11}{⽬,⽔}
  \definition{v.}{considerar sem importância | ser indiferente a (fama, riqueza, etc.) | (de uma economia ou mercado) enfraquecer, ficar mais lento, diminuir a velocidade}
\end{EntryWithPhonetic}

\begin{EntryWithPhonetic}{看到}{kan4 dao4}{9,8}{⽬,⼑}[HSK 1]
  \definition{v.}{ver; avistar}
\end{EntryWithPhonetic}

\begin{EntryWithPhonetic}{看得出}{kan4de5chu1}{9,11,5}{⽬,⼻,⼐}[HSK 7-9]
  \definition{expr.}{ser evidente; poder ver; poder ser visto}
\end{EntryWithPhonetic}

\begin{EntryWithPhonetic}{看得见}{kan4de5jian4}{9,11,4}{⽬,⼻,⾒}[HSK 6]
  \definition{adj.}{perceptível; visível; tangível}
\end{EntryWithPhonetic}

\begin{EntryWithPhonetic}{看得起}{kan4de5qi3}{9,11,10}{⽬,⼻,⾛}[HSK 6]
  \definition{v.}{ter uma boa opinião sobre; pensar muito (ou muito) sobre}
\end{EntryWithPhonetic}

\begin{EntryWithPhonetic}{看法}{kan4fa5}{9,8}{⽬,⽔}[HSK 2]
  \definition[个,种,点]{s.}{opinião; perspectiva; (ponto de) vista; uma maneira de ver uma coisa | opinião desfavorável (ou crítica) sobre alguém}
\end{EntryWithPhonetic}

\begin{EntryWithPhonetic}{看好}{kan4hao3}{9,6}{⽬,⼥}[HSK 6]
  \definition{v.}{elogiar; apreciar; encorajar; acreditar que pessoas ou coisas terão uma boa tendência | estar prestes a surgir uma boa tendência}
\end{EntryWithPhonetic}

\begin{EntryWithPhonetic}{看见}{kan4 jian5}{9,4}{⽬,⾒}[HSK 1]
  \definition{v.}{ver; avistar; ao olhar, descobrir alguém ou algo}
\end{EntryWithPhonetic}

\begin{EntryWithPhonetic}{看来}{kan4lai5}{9,7}{⽬,⽊}[HSK 4]
  \definition{adv.}{parecer; parecer como se (ou embora); refere"-se a um julgamento aproximado; expressa um julgamento por observação}
  \definition{v.}{ser considerado; na visão de alguém; na opinião de alguém; expressar a ideia aproximada que o locutor tem da situação}
\end{EntryWithPhonetic}

\begin{EntryWithPhonetic}{看起来}{kan4qi3lai5}{9,10,7}{⽬,⾛,⽊}[HSK 3]
  \definition{v.}{parecer; aparentar; dar a impressão de (ou como se)}
\end{EntryWithPhonetic}

\begin{EntryWithPhonetic}{看热闹}{kan4 re4nao5}{9,10,8}{⽬,⽕,⾾}[HSK 7-9]
  \definition{v.}{observar a emoção (ou a diversão) | regozijar"-se com (ou sobre); observar de braços cruzados; ficar de fora | observar a cena movimentada; ser um espectador; observar; ver (assistir) a diversão}
\end{EntryWithPhonetic}

\begin{EntryWithPhonetic}{看上去}{kan4shang4qu5}{9,3,5}{⽬,⼀,⼛}[HSK 3]
  \definition{adv.}{parece que}
\end{EntryWithPhonetic}

\begin{EntryWithPhonetic}{看似}{kan4si4}{9,6}{⽬,⼈}[HSK 7-9]
  \definition{v.}{parecer; dar a impressão de ser}
\end{EntryWithPhonetic}

\begin{EntryWithPhonetic}{看台}{kan4tai2}{9,5}{⽬,⼝}[HSK 7-9]
  \definition{s.}{arquibancada | arquibancada para espectadores | terraço | plataforma de visualização}
\end{EntryWithPhonetic}

\begin{EntryWithPhonetic}{看望}{kan4wang5}{9,11}{⽬,⽉}[HSK 4]
  \definition{v.}{ver; visitar; ligar; dar uma olhada; ir até os pais, idosos, professores ou amigos para cumprimentá-los}
\end{EntryWithPhonetic}

\begin{EntryWithPhonetic}{看样子}{kan4 yang4zi5}{9,10,3}{⽬,⽊,⼦}[HSK 7-9]
  \definition{v.}{parecer; aparentar; dar a impressão de ser; estimar com base na situação (frequentemente usado como elemento inserido em uma frase)}
\end{EntryWithPhonetic}

\begin{EntryWithPhonetic}{看中}{kan4/zhong4}{9,4}{⽬,⼁}[HSK 7-9]
  \definition{v.+compl.}{optar por; gostar de; sentir-se satisfeito com}
\end{EntryWithPhonetic}

\begin{EntryWithPhonetic}{看重}{kan4zhong4}{9,9}{⽬,⾥}[HSK 7-9]
  \definition{v.}{ter em alta consideração; considerar importante; atribuir importância a}
\end{EntryWithPhonetic}

\begin{EntryWithPhonetic}{看作}{kan4zuo4}{9,7}{⽬,⼈}[HSK 6]
  \definition{v.}{considerar como; olhar como}
\end{EntryWithPhonetic}

%%%%%%%%%% 康 %%%%%%%%%%
\subsection*{康}\addcontentsline{loh}{figure}{康 \dpy{kang1}}

\begin{EntryWithPhonetic}{康}{kang1}{11}{⼴}
  \definition*{s.}{Sobrenome: Kang}
  \definition{adj.}{saudável |  fácil; pacífico; abundante | amplo; largo | Dialeto: de baixa qualidade; inferior}
  \definition{s.}{bem-estar; saúde | palha; farelo; casca}
  \definition{v.}{(normalmente de um rabanete) tornar-se esponjoso}
\end{EntryWithPhonetic}

\begin{EntryWithPhonetic}{康复}{kang1fu4}{11,9}{⼴,⼢}[HSK 6]
  \definition{v.}{Saúde: estaurar; recuperar; reabilitar}
\end{EntryWithPhonetic}

%%%%%%%%%% 慷 %%%%%%%%%%
\subsection*{慷}\addcontentsline{loh}{figure}{慷 \dpy{kang1}}

\begin{EntryWithPhonetic}{慷}{kang1}{14}{⼼}
  \definition{adj.}{generoso | magnânimo}
\end{EntryWithPhonetic}

\begin{EntryWithPhonetic}{慷慨}{kang1kai3}{14,12}{⼼,⼼}[HSK 7-9]
  \definition{adj.}{generoso; descreve alguém como alguém que não é mesquinho; disposto a ajudar os outros com dinheiro ou bens | veemente; fervoroso; apaixonado; descreve alguém como alguém repleto de senso de justiça e emocionalmente intenso}
\end{EntryWithPhonetic}

%%%%%%%%%% 扛 %%%%%%%%%%
\subsection*{扛}\addcontentsline{loh}{figure}{扛 \dpy{kang2}}

\begin{EntryWithPhonetic}{扛}{kang2}{6}{⼿}[HSK 7-9]
  \definition{v.}{carregar objetos nos ombros |  suportar; aguentar | lidar; assumir}
  \seeref{gang1}
\end{EntryWithPhonetic}

%%%%%%%%%% 抗 %%%%%%%%%%
\subsection*{抗}\addcontentsline{loh}{figure}{抗 \dpy{kang4}}

\begin{EntryWithPhonetic}{抗}{kang4}{7}{⼿}
  \definition*{s.}{Sobrenome: Kang}
  \definition{pref.}{anti-}
  \definition{v.}{resistir; combater; lutar | recusar; desafiar}
\end{EntryWithPhonetic}

\begin{EntryWithPhonetic}{抗衡}{kang4heng2}{7,16}{⼿,⾏}[HSK 7-9]
  \definition{v.}{competir com; desafiar; rivalizar | servir de contrapeso a; competir com; rivalizar com; estar em pé de igualdade com}
\end{EntryWithPhonetic}

\begin{EntryWithPhonetic}{抗拒}{kang4ju4}{7,7}{⼿,⼿}[HSK 7-9]
  \definition{v.}{resistir; suportar}
\end{EntryWithPhonetic}

\begin{EntryWithPhonetic}{抗生素}{kang4sheng1su4}{7,5,10}{⼿,⽣,⽷}[HSK 7-9]
  \definition{s.}{antibiótico}
\end{EntryWithPhonetic}

\begin{EntryWithPhonetic}{抗议}{kang4yi4}{7,5}{⼿,⾔}[HSK 6]
  \definition{v.}{protestar; reconsiderar; levantar objeções fortes}
\end{EntryWithPhonetic}

\begin{EntryWithPhonetic}{抗争}{kang4zheng1}{7,6}{⼿,⼑}[HSK 7-9]
  \definition{v.}{resistir; opor-se; posicionar-se contra; confrontar; lutar}
\end{EntryWithPhonetic}

%%%%%%%%%% 考 %%%%%%%%%%
\subsection*{考}\addcontentsline{loh}{figure}{考 \dpy{kao3}}

\begin{EntryWithPhonetic}{考}{kao3}{6}{⽼}[HSK 1]
  \definition*{s.}{Sobrenome: Kao}
  \definition{adj.}{antigo; velho; com idade avançada}
  \definition{s.}{o pai falecido de alguém}
  \definition{v.}{examinar; dar (fazer) um exame, teste ou questionário | verificar; inspecionar | estudar; verificar; investigar | perguntar; testar; fazer perguntas para que o outro responda, a fim de testar suas habilidades em determinada área}
\end{EntryWithPhonetic}

\begin{EntryWithPhonetic}{考察}{kao3cha2}{6,14}{⽼,⼧}[HSK 4]
  \definition{v.}{inspecionar; investigar; observar e estudar}
\end{EntryWithPhonetic}

\begin{EntryWithPhonetic}{考场}{kao3chang3}{6,6}{⽼,⼟}[HSK 6]
  \definition{s.}{sala de exames}
\end{EntryWithPhonetic}

\begin{EntryWithPhonetic}{考核}{kao3he2}{6,10}{⽼,⽊}[HSK 5]
  \definition{v.}{examinar; checar; avaliar; avaliar (a proficiência de alguém)}
\end{EntryWithPhonetic}

\begin{EntryWithPhonetic}{考量}{kao3liang2}{6,12}{⽼,⾥}[HSK 7-9]
  \definition{v.}{considerar; examinar e medir}
\end{EntryWithPhonetic}

\begin{EntryWithPhonetic}{考虑}{kao3lv4}{6,10}{⽼,⾌}[HSK 4]
  \definition{v.}{considerar; refletir sobre; levar em conta}
\end{EntryWithPhonetic}

\begin{EntryWithPhonetic}{考生}{kao3sheng1}{6,5}{⽼,⽣}[HSK 2]
  \definition{s.}{candidato a exame; alunos inscritos para o exame de admissão}
\end{EntryWithPhonetic}

\begin{EntryWithPhonetic}{考试}{kao3/shi4}{6,8}{⽼,⾔}[HSK 1]
  \definition[次]{s.}{teste; exame; prova; atividades realizadas para verificar conhecimentos ou habilidades}
  \definition{v.+compl.}{testar; avaliar; avaliar conhecimentos e habilidades por meio de perguntas escritas ou orais.}
\end{EntryWithPhonetic}

\begin{EntryWithPhonetic}{考题}{kao3ti2}{6,15}{⽼,⾴}[HSK 6]
  \definition{s.}{questões de exame; prova de exame; tópicos de exame}
\end{EntryWithPhonetic}

\begin{EntryWithPhonetic}{考验}{kao3yan4}{6,10}{⽼,⾺}[HSK 3]
  \definition[场,个,种]{s.}{teste; julgamento; atividade realizada para verificar se as habilidades, ideias, moral e qualidades de uma pessoa atendem aos requisitos}
  \definition{v.}{testar; testar as capacidades, ideias, moral e qualidades de uma pessoa através de situações, ações ou ambientes difíceis, para verificar se elas atendem aos requisitos}
\end{EntryWithPhonetic}

%%%%%%%%%% 拷 %%%%%%%%%%
\subsection*{拷}\addcontentsline{loh}{figure}{拷 \dpy{kao3}}

\begin{EntryWithPhonetic}{拷}{kao3}{9}{⼿}
  \definition{v.}{açoitar; bater; torturar | copiar | Dialeto, Empréstimo linguístico: \emph{call}, ligar | vencer | interrogar sob tortura}
\end{EntryWithPhonetic}

%%%%%%%%%% 烤 %%%%%%%%%%
\subsection*{烤}\addcontentsline{loh}{figure}{烤 \dpy{kao3}}

\begin{EntryWithPhonetic}{烤}{kao3}{10}{⽕}
  \definition{v.}{assar | grelhar}
\end{EntryWithPhonetic}

\begin{EntryWithPhonetic}{烤肉}{kao3rou4}{10,6}{⽕,⾁}[HSK 5]
  \definition[块,串,片,盘]{s.}{churrasco (literalmente carne assada)}
\end{EntryWithPhonetic}

\begin{EntryWithPhonetic}{烤鸭}{kao3ya1}{10,10}{⽕,⿃}[HSK 5]
  \definition[只,盘]{s.}{pato assado; pato recheado e assado em um forno especial após ser abatido}
\end{EntryWithPhonetic}

%%%%%%%%%% 靠 %%%%%%%%%%
\subsection*{靠}\addcontentsline{loh}{figure}{靠 \dpy{kao4}}

\begin{EntryWithPhonetic}{靠}{kao4}{15}{⾮}[HSK 2]
  \definition{prep.}{manter (em); aproximar"-se (de); ao longo de | por; graças a; com base em; de acordo com}
  \definition{s.}{armadura de palco (feita de seda bordada); armadura usada pelos generais militares antigos nas peças teatrais}
  \definition{v.}{inclinar"-se; sentado ou em pé, deixar parte do peso do corpo ser suportado por outra pessoa ou objeto (pessoa) | encostar"-se (em); apoiar"-se ou levantar-se com a ajuda de alguma coisa | aproximar"-se; estar perto de | confiar em; depender de | confiar}
\end{EntryWithPhonetic}

\begin{EntryWithPhonetic}{靠近}{kao4jin4}{15,7}{⾮,⾡}[HSK 5]
  \definition{adv.}{próximo; perto de; ao lado de}
  \definition{v.}{aproximar"-se; chegar perto; avançar em direção a um determinado objetivo de modo que a distância fique cada vez menor}
\end{EntryWithPhonetic}

\begin{EntryWithPhonetic}{靠拢}{kao4long3}{15,8}{⾮,⼿}[HSK 7-9]
  \definition{v.}{aproximar-se de; encostar-se; reunir-se; aconchegar-se}
\end{EntryWithPhonetic}

%%%%%%%%%% 苛 %%%%%%%%%%
\subsection*{苛}\addcontentsline{loh}{figure}{苛 \dpy{ke1}}

\begin{EntryWithPhonetic}{苛}{ke1}{8}{⾋}
  \definition{adj.}{duro; severo; exigente; opressivo | excessivamente elaborado; exorbitante; diverso; variado}
\end{EntryWithPhonetic}

\begin{EntryWithPhonetic}{苛刻}{ke1ke4}{8,8}{⾋,⼑}[HSK 7-9]
  \definition{adj.}{severo; rigoroso; os requisitos são muito rigorosos ou as condições são muito elevadas}
\end{EntryWithPhonetic}

%%%%%%%%%% 科 %%%%%%%%%%
\subsection*{科}\addcontentsline{loh}{figure}{科 \dpy{ke1}}

\begin{EntryWithPhonetic}{科}{ke1}{9}{⽲}[HSK 2]
  \definition*{s.}{Sobrenome: Ke}
  \definition{s.}{um ramo de estudo acadêmico ou profissional |uma divisão ou subdivisão de uma unidade administrativa | família | instruções de palco no drama chinês clássico; nos roteiros de peças clássicas, termos usados para indicar as ações dos personagens | nível; classificação; categoria | sessão de exames; refere"-se às disciplinas, notas e anos das provas para a seleção de candidatos a cargos públicos militares e civis na antiguidade | tecnológico | assunto | lei; regulamento; decreto | penalidade; pena; punição | treinamento profissional ou formal; curso profissionalizante}
  \definition{v.}{proferir uma sentença (penal)}
\end{EntryWithPhonetic}

\begin{EntryWithPhonetic}{科幻}{ke1huan4}{9,4}{⽲,⼳}[HSK 7-9]
  \definition[个]{s.}{ficção científica}
\end{EntryWithPhonetic}

\begin{EntryWithPhonetic}{科技}{ke1ji4}{9,7}{⽲,⼿}[HSK 3]
  \definition{s.}{ciência e tecnologia}
\end{EntryWithPhonetic}

\begin{EntryWithPhonetic}{科目}{ke1mu4}{9,5}{⽲,⽬}[HSK 7-9]
  \definition[个]{s.}{curso; disciplina (em um currículo); categorias como disciplinas acadêmicas, classificadas de acordo com suas diferentes naturezas | cabeçalhos em um livro de contas; registros contábeis}
\end{EntryWithPhonetic}

\begin{EntryWithPhonetic}{科普}{ke1pu3}{9,12}{⽲,⽇}[HSK 7-9]
  \definition{v.}{popularizar a ciência}
\end{EntryWithPhonetic}

\begin{EntryWithPhonetic}{科学}{ke1xue2}{9,8}{⽲,⼦}[HSK 2]
  \definition{adj.}{científico; em conformidade com as leis da ciência}
  \definition[门,个,种]{s.}{ciência; um conjunto de conhecimentos que reflete as leis objetivas da natureza, da sociedade, do pensamento, etc.}
\end{EntryWithPhonetic}

\begin{EntryWithPhonetic}{科学家}{ke1xue2jia1}{9,8,10}{⽲,⼦,⼧}
  \definition[位,名,个]{s.}{cientista; pessoas com realizações significativas no campo da pesquisa científica}
\end{EntryWithPhonetic}

\begin{EntryWithPhonetic}{科研}{ke1yan2}{9,9}{⽲,⽯}[HSK 6]
  \definition{s.}{pesquisa científica}
  \definition{v.}{envolver-se em pesquisa científica}
\end{EntryWithPhonetic}

%%%%%%%%%% 棵 %%%%%%%%%%
\subsection*{棵}\addcontentsline{loh}{figure}{棵 \dpy{ke1}}

\begin{EntryWithPhonetic}{棵}{ke1}{12}{⽊}[HSK 4]
  \definition{clas.}{usado para plantas, árvores}
\end{EntryWithPhonetic}

%%%%%%%%%% 颗 %%%%%%%%%%
\subsection*{颗}\addcontentsline{loh}{figure}{颗 \dpy{ke1}}

\begin{EntryWithPhonetic}{颗}{ke1}{14}{⾴}[HSK 5]
  \definition{clas.}{usado para grãos, pérolas, dentes, corações, satelites, pequenas esferas, etc.}
  \definition{s.}{grão; partícula; pequenas coisas redondas}
\end{EntryWithPhonetic}

%%%%%%%%%% 磕 %%%%%%%%%%
\subsection*{磕}\addcontentsline{loh}{figure}{磕 \dpy{ke1}}

\begin{EntryWithPhonetic}{磕}{ke1}{15}{⽯}[HSK 7-9]
  \definition{v.}{bater (com força em algo); bater em algo duro | derrubar algo de um recipiente, vaso, etc.}
\end{EntryWithPhonetic}

%%%%%%%%%% 蝌 %%%%%%%%%%
\subsection*{蝌}\addcontentsline{loh}{figure}{蝌 \dpy{ke1}}

\begin{EntryWithPhonetic}{蝌}{ke1}{15}{⾍}
  \definition[只]{s.}{girino}
\end{EntryWithPhonetic}

\begin{EntryWithPhonetic}{蝌蚪}{ke1dou3}{15,10}{⾍,⾍}
  \definition{s.}{girino}
\end{EntryWithPhonetic}

%%%%%%%%%% 壳 %%%%%%%%%%
\subsection*{壳}\addcontentsline{loh}{figure}{壳 \dpy{ke2}}

\begin{EntryWithPhonetic}{壳}{ke2}{7}{⼠}[HSK 7-9]
  \definition[层,个]{s.}{casca, concha; significa o mesmo que 壳 | concha; revestimento externo | empresa de fachada}
  \seeref{qiao4}
  \seealsoref{壳儿}{ke2r5}
\end{EntryWithPhonetic}

\begin{EntryWithPhonetic}{壳儿}{ke2r5}{7,2}{⼠,⼉}
  \definition{s.}{crosta | concha}
  \seealsoref{壳}{ke2}
\end{EntryWithPhonetic}

%%%%%%%%%% 咳 %%%%%%%%%%
\subsection*{咳}\addcontentsline{loh}{figure}{咳 \dpy{ke2}}

\begin{EntryWithPhonetic}{咳}{ke2}{9}{⼝}[HSK 5]
  \definition{v.}{tossir}
  \seeref{hai1}
\end{EntryWithPhonetic}

\begin{EntryWithPhonetic}{咳嗽}{ke2sou5}{9,14}{⼝,⼝}[HSK 7-9]
  \definition[次,声,阵,回]{s.}{tosse}
  \definition{v.}{ter tosse; tossir}
\end{EntryWithPhonetic}

%%%%%%%%%% 可 %%%%%%%%%%
\subsection*{可}\addcontentsline{loh}{figure}{可 \dpy{ke3}}

\begin{EntryWithPhonetic}{可}{ke3}{5}{⼝}[HSK 5]
  \definition*{s.}{Sobrenome: Ke}
  \definition{adv.}{indica ênfase | indica o fortalecimento de perguntas retóricas | indica um tom de questionamento mais forte | sobre; a respeito de}
  \definition{conj.}{mas; ainda}
  \definition{v.}{aprovar; concordar com | poder; permitir; ser capaz de | precisar (fazer); valer a pena (fazer); merecer | ajustar; adequar | estar pronto para; estar disposto a; pretender}
  \seeref{ke4}
\end{EntryWithPhonetic}

\begin{EntryWithPhonetic}{可爱}{ke3'ai4}{5,10}{⼝,⽖}[HSK 2]
  \definition{adj.}{adorável; simpático; encantador | bonitinho; adorável | amado; querido; encantador; cativante; relacionamento próximo, sentimentos profundos | fofo; bonito}
\end{EntryWithPhonetic}

\begin{EntryWithPhonetic}{可悲}{ke3bei1}{5,12}{⼝,⽕}[HSK 7-9]
  \definition{adj.}{triste; lamentável; deplorável; de partir o coração}
\end{EntryWithPhonetic}

\begin{EntryWithPhonetic}{可编程}{ke3bian1cheng2}{5,12,12}{⼝,⽷,⽲}
  \definition{adj.}{programável}
\end{EntryWithPhonetic}

\begin{EntryWithPhonetic}{可不是}{ke3bu2shi4}{5,4,9}{⼝,⼀,⽇}[HSK 7-9]
  \definition{adv.}{certamente; exatamente; é assim mesmo; para expressar concordância, é frequentemente usado como uma frase independente, mas também pode ser expresso como 可不}
  \seealsoref{可不}{ke3bu4}
\end{EntryWithPhonetic}

\begin{EntryWithPhonetic}{可不}{ke3bu4}{5,4}{⼝,⼀}
  \definition{adv.}{certamente; exatamente; em uma conversa, significa concordar plenamente com o que a outra pessoa diz}
  \seealsoref{可不是}{ke3bu2shi4}
\end{EntryWithPhonetic}

\begin{EntryWithPhonetic}{可擦写可编程只读存储器}{ke3 ca1 xie3 ke3 bian1cheng2 zhi1 du2 cun2chu3qi4}{5,17,5,5,12,12,5,10,6,12,16}{⼝,⼿,⼍,⼝,⽷,⽲,⼝,⾔,⼦,⼈,⼝}
  \definition{s.}{EPROM (\emph{erasable programmable read-only memory})}
\end{EntryWithPhonetic}

\begin{EntryWithPhonetic}{可乘之机}{ke3cheng2zhi1ji1}{5,10,3,6}{⼝,⽲,⼂,⽊}[HSK 7-9]
  \definition{expr.}{uma oportunidade para aproveitar; oportunidade a explorar; abertura}
\end{EntryWithPhonetic}

\begin{EntryWithPhonetic}{可耻}{ke3chi3}{5,10}{⼝,⽿}[HSK 7-9]
  \definition{adj.}{vergonhoso; ignominioso; desonroso}
\end{EntryWithPhonetic}

\begin{EntryWithPhonetic}{可歌可泣}{ke3ge1-ke3qi4}{5,14,5,8}{⼝,⽋,⼝,⽔}[HSK 7-9]
  \definition{expr.}{``Uma história digna de elogios e lágrimas.''; capaz de evocar elogios e lágrimas; tanto alegre quanto trágico; comovente e digno de uma canção; tocante e digno de uma canção; digno de elogios, comovente até às lágrimas refere"-se a feitos trágicos e heroicos que tocam profundamente as pessoas; digno de ser louvado e de comover até às lágrimas}
\end{EntryWithPhonetic}

\begin{EntryWithPhonetic}{可观}{ke3guan1}{5,6}{⼝,⾒}[HSK 7-9]
  \definition{adj.}{que vale a pena ver; que vale a pena assistir | considerável; impressionante; substancial; de nível alcançado; de grau relativamente alto}
\end{EntryWithPhonetic}

\begin{EntryWithPhonetic}{可贵}{ke3gui4}{5,9}{⼝,⾙}[HSK 7-9]
  \definition{adj.}{estimado; valioso; valorizado; louvável; recomendável}
\end{EntryWithPhonetic}

\begin{EntryWithPhonetic}{可见}{ke3jian4}{5,4}{⼝,⾒}[HSK 4]
  \definition{adj.}{visível; concebível; algo que é óbvio ou evidente}
  \definition{conj.}{isso mostra; isto prova; é, portanto, claro (ou evidente, óbvio) que}
  \definition{v.}{ser ou estar visível ; ser ou estar claro}
\end{EntryWithPhonetic}

\begin{EntryWithPhonetic}{可卡因}{ke3ka3yin1}{5,5,6}{⼝,⼘,⼞}
  \definition{s.}{Empréstimo linguístico: cocaína; eritroxilon}
\end{EntryWithPhonetic}

\begin{EntryWithPhonetic}{可靠}{ke3kao4}{5,15}{⼝,⾮}[HSK 3]
  \definition{adj.}{confiável; digno de confiança | verdadeiro; autêntico; descrever notícias e outras informações como verdadeiras, de modo que as pessoas possam acreditar nelas}
\end{EntryWithPhonetic}

\begin{EntryWithPhonetic}{可口}{ke3kou3}{5,3}{⼝,⼝}[HSK 7-9]
  \definition{adj.}{bom; saboroso; apetitoso; gostoso de comer; (alimento ou bebida) tem uma textura agradável e um sabor bom}
\end{EntryWithPhonetic}

\begin{EntryWithPhonetic}{可口可乐}{ke3kou3ke3le4}{5,3,5,5}{⼝,⼝,⼝,⼃}
  \definition*{s.}{Empréstimo linguístico: Coca-Cola}
\end{EntryWithPhonetic}

\begin{EntryWithPhonetic}{可乐}{ke3le4}{5,5}{⼝,⼃}[HSK 3]
  \definition*[罐,杯,瓶,听,口]{s.}{\emph{coke}; coca; coca-cola}
  \definition{adj.}{engraçado; divertido; risível}
\end{EntryWithPhonetic}

\begin{EntryWithPhonetic}{可怜}{ke3lian2}{5,8}{⼝,⼼}[HSK 5]
  \definition{adj.}{pobre; lamentável; lastimável | miserável (de quantidade ou qualidade); descreve um número pequeno ou um lugar tão pequeno que não vale a pena falar sobre ele}
  \definition{v.}{ter pena; ter piedade de; ter simpatia por pessoas que tiveram coisas muito ruins acontecendo com elas}
\end{EntryWithPhonetic}

\begin{EntryWithPhonetic}{可能}{ke3neng2}{5,10}{⼝,⾁}[HSK 2]
  \definition{adj.}{possível}
  \definition{adv.}{possivelmente}
  \definition[种]{s.}{possibilidade; tendências ou oportunidades que podem se tornar realidade}
\end{EntryWithPhonetic}

\begin{EntryWithPhonetic}{可怕}{ke3pa4}{5,8}{⼝,⼼}[HSK 2]
  \definition{adj.}{assustador; terrível; hediondo; medonho; horrível; aterrorizante}
  \definition{adv.}{terrivelmente}
\end{EntryWithPhonetic}

\begin{EntryWithPhonetic}{可是}{ke3shi4}{5,9}{⼝,⽇}[HSK 2]
  \definition{adv.}{de fato (usado para dar ênfase), equivalente a 的确}
  \definition{conj.}{mas; no entanto; contudo; conecta frases, expressa uma relação de transição, equivalente a 但是}
  \seealsoref{但是}{dan4shi4}
  \seealsoref{的确}{di2que4}
\end{EntryWithPhonetic}

\begin{EntryWithPhonetic}{可谓}{ke3wei4}{5,11}{⼝,⾔}[HSK 7-9]
  \definition{s.}{poder ser dito; poder ser considerado; poder ser chamado de}[这机会可谓是千载难逢。===Pode-se dizer que esta é a oportunidade da sua vida.]
\end{EntryWithPhonetic}

\begin{EntryWithPhonetic}{可恶}{ke3wu4}{5,10}{⼝,⼼}[HSK 7-9]
  \definition{adj.}{odioso; abominável; detestável; repugnante; extremamente irritante}
\end{EntryWithPhonetic}

\begin{EntryWithPhonetic}{可惜}{ke3xi1}{5,11}{⼝,⼼}[HSK 5]
  \definition{adj.}{é uma pena; é muito ruim; é lamentável}
  \definition{adv.}{infelizmente}
\end{EntryWithPhonetic}

\begin{EntryWithPhonetic}{可想而知}{ke3xiang3'er2zhi1}{5,13,6,8}{⼝,⼼,⽽,⽮}[HSK 7-9]
  \definition{expr.}{é fácil de imaginar; é óbvio; claramente; como você pode imaginar; você pode imaginar isso sem precisar de explicações; pode-se muito bem imaginar}
\end{EntryWithPhonetic}

\begin{EntryWithPhonetic}{可笑}{ke3xiao4}{5,10}{⼝,⽵}[HSK 7-9]
  \definition{adj.}{absurdo; risível; ridículo; hilário; engraçado}
\end{EntryWithPhonetic}

\begin{EntryWithPhonetic}{可信}{ke3xin4}{5,9}{⼝,⼈}[HSK 7-9]
  \definition{adj.}{confiável; em que (quem) se pode acreditar}
\end{EntryWithPhonetic}

\begin{EntryWithPhonetic}{可行}{ke3xing2}{5,6}{⼝,⾏}[HSK 7-9]
  \definition{adj.}{viável; praticável; funcional}
\end{EntryWithPhonetic}

\begin{EntryWithPhonetic}{可疑}{ke3yi2}{5,14}{⼝,⽦}[HSK 7-9]
  \definition{adj.}{duvidoso; suspeito; questionável}
\end{EntryWithPhonetic}

\begin{EntryWithPhonetic}{可以}{ke3yi3}{5,4}{⼝,⼈}[HSK 2]
  \definition{adj.}{aceitável; nada mal; muito bom | impressionante; espantoso; tremendo}
  \definition{v.}{poder; ter condições, capacidade e tempo para fazer algo ou ter alguma utilidade | permitir; poder | valer a pena fazer; considerar que vale a pena, recomendar fazer algo}
\end{EntryWithPhonetic}

%%%%%%%%%% 渴 %%%%%%%%%%
\subsection*{渴}\addcontentsline{loh}{figure}{渴 \dpy{ke3}}

\begin{EntryWithPhonetic}{渴}{ke3}{12}{⽔}[HSK 1]
  \definition{adj.}{sedento}
  \definition{adv.}{ansiosamente; metáfora de urgência}
  \definition{v.}{desejar; ansiar por}
\end{EntryWithPhonetic}

\begin{EntryWithPhonetic}{渴望}{ke3wang4}{12,11}{⽔,⽉}[HSK 5]
  \definition{v.}{aspirar; (ter sede, ansiar, desejar) por}
\end{EntryWithPhonetic}

%%%%%%%%%% 可 %%%%%%%%%%
\subsection*{可}\addcontentsline{loh}{figure}{可 \dpy{ke4}}

\begin{EntryWithPhonetic}{可}{ke4}{5}{⼝}
  \definition{s.}{governante supremo de uma tribo nômade do norte; Khan (可汗), título do governante supremo dos antigos grupos étnicos xianbei, turco, uigur e mongol}
  \seeref{ke3}
  \seealsoref{可汗}{ke4han2}
\end{EntryWithPhonetic}

\begin{EntryWithPhonetic}{可汗}{ke4han2}{5,6}{⼝,⽔}
  \definition{s.}{khan (empréstimo linguístico); cham}
\end{EntryWithPhonetic}

%%%%%%%%%% 克 %%%%%%%%%%
\subsection*{克}\addcontentsline{loh}{figure}{克 \dpy{ke4}}

\begin{EntryWithPhonetic}{克}{ke4}{7}{⼗}[HSK 2]
  \definition*{s.}{Sobrenome: Ke}
  \definition{clas.}{g, grama, unidade de peso | unidade tibetana de volume ou medida seca (com capacidade para cerca de 25 斤, de cevada) | unidade tibetana de área de terra equivalente a cerca de 1 亩}
  \definition{v.}{poder; ser capaz de | tolerar; conter; restringir; suprimir| subjugar; capturar; conquistar (uma cidade, etc.) | digerir (alimentos) | reduzir; diminuir | definir um limite de tempo}
  \seealsoref{斤}{jin1}
  \seealsoref{亩}{mu3}
\end{EntryWithPhonetic}

\begin{EntryWithPhonetic}{克服}{ke4fu2}{7,8}{⼗,⽉}[HSK 3]
  \definition{v.}{sobrepujar; superar; conquistar; vencer com força de vontade e determinação (deficiências, erros, fenômenos negativos, condições desfavoráveis, etc.) | aguentar; suportar (dificuldades, inconveniências, etc.)}
\end{EntryWithPhonetic}

\begin{EntryWithPhonetic}{克隆}{ke4long2}{7,11}{⼗,⾩}[HSK 7-9]
  \definition{v.}{clonar; geralmente, refere"-se à reprodução assexuada induzida artificialmente | copiar; imitar; metaforicamente, significa copiar (frequentemente usado em um sentido humorístico ou depreciativo)}
\end{EntryWithPhonetic}

\begin{EntryWithPhonetic}{克制}{ke4zhi4}{7,8}{⼗,⼑}[HSK 7-9]
  \definition{v.}{restringir; exercer contenção; exercer autocontrole nas emoções, palavras e ações}
\end{EntryWithPhonetic}

%%%%%%%%%% 刻 %%%%%%%%%%
\subsection*{刻}\addcontentsline{loh}{figure}{刻 \dpy{ke4}}

\begin{EntryWithPhonetic}{刻}{ke4}{8}{⼑}[HSK 2,5]
  \definition{adj.}{cruel; severo; rude; indelicado | no mais alto grau}
  \definition{clas.}{um quarto (de uma hora, 15min)}
  \definition[件]{s.}{quarto (de hora); momento}
  \definition{v.}{esculpir; inscrever; gravar; talhar com uma faca (padrões, texto, etc.) | definir um limite de tempo | imprimir (CD)}
\end{EntryWithPhonetic}

\begin{EntryWithPhonetic}{刻画}{ke4hua4}{8,8}{⼑,⽥}
  \definition{v.}{retratar | tirar um retrato}
\end{EntryWithPhonetic}

\begin{EntryWithPhonetic}{刻苦}{ke4ku3}{8,8}{⼑,⾋}[HSK 7-9]
  \definition{adj.}{assíduo; trabalhador; meticuloso; diligente e trabalhador, capaz de se dedicar ao trabalho árduo | simples e econômico}
\end{EntryWithPhonetic}

\begin{EntryWithPhonetic}{刻意}{ke4yi4}{8,13}{⼑,⼼}[HSK 7-9]
  \definition{adv.}{diligentemente; assiduamente; fazendo tudo o que se pode; estar completamente absorto; dedicar-se inteiramente a algo; isso enfatiza ações tomadas para atrair a atenção dos outros}
\end{EntryWithPhonetic}

\begin{EntryWithPhonetic}{刻钟}{ke4 zhong1}{8,9}{⼑,⾦}
  \definition{s.}{um quarto de hora}
\end{EntryWithPhonetic}

\begin{EntryWithPhonetic}{刻舟求剑}{ke4zhou1-qiu2jian4}{8,6,7,9}{⼑,⾈,⽔,⼑}[HSK 7-9]
  \definition{expr.}{``Marcando o barco para encontrar a espada.''; um entalhe na lateral de um barco para localizar uma espada que caiu ao mar; Figurativo: uma ação que se torna inútil devido a circunstâncias alteradas; tomar medidas sem levar em conta mudanças nas circunstâncias; ``Um homem do estado de Chu deixou cair sua espada no rio enquanto o atravessava. Ele marcou o local onde a espada havia caído na lateral do barco. Quando o barco parou, ele entrou na água a partir do ponto marcado para procurar sua espada, mas, naturalmente, não a encontrou.'' de Lüshi Chunqiu (吕氏春秋), Observando o Presente (察今)}
\end{EntryWithPhonetic}

%%%%%%%%%% 客 %%%%%%%%%%
\subsection*{客}\addcontentsline{loh}{figure}{客 \dpy{ke4}}

\begin{EntryWithPhonetic}{客}{ke4}{9}{⼧}
  \definition*{s.}{Sobrenome: Ke}
  \definition{adj.}{objetivo; independente da consciência humana | estrangeiro; não desta região, unidade ou indústria}
  \definition{clas.}{porção (de comida, bebida, etc.); em algumas áreas, é usado para vender alimentos e bebidas em porções}
  \definition[个,位,名,些]{s.}{convidado; visitante; aquele que é convidado; aquele que vem visitar | viajante; passageiro | comerciante viajante; refere"-se especificamente a comerciantes que transportam mercadorias de um lugar para o outro | cliente; patrono; consumidor | uma pessoa envolvida em alguma atividade específica; pessoas que viajam fazendo algum tipo de atividade}
  \definition{v.}{ser um estranho; estabelecer-se (ou viver) em um lugar estranho; estar longe de casa ou morar no exterior}
  \antonymref{主}{zhu3}
\end{EntryWithPhonetic}

\begin{EntryWithPhonetic}{客车}{ke4che1}{9,4}{⼧,⾞}[HSK 6]
  \definition[辆,列,次,趟]{s.}{ônibus; veículo de passageiros; veículos que transportam passageiros em ferrovias e estradas}
\end{EntryWithPhonetic}

\begin{EntryWithPhonetic}{客房}{ke4fang2}{9,8}{⼧,⼾}[HSK 7-9]
  \definition{s.}{quarto de hóspedes; quartos para viajantes ou hóspedes}
\end{EntryWithPhonetic}

\begin{EntryWithPhonetic}{客观}{ke4guan1}{9,6}{⼧,⾒}[HSK 3]
  \definition{adj.}{objetivo; justo e razoável; imparcial; com base na situação real, sem preconceitos pessoais}
  \definition{s.}{objetivo; existe fora da consciência, sem depender da consciência subjetiva}
\end{EntryWithPhonetic}

\begin{EntryWithPhonetic}{客户}{ke4hu4}{9,4}{⼧,⼾}[HSK 5]
  \definition[位,个,家,批]{s.}{cliente; consumidor}
\end{EntryWithPhonetic}

\begin{EntryWithPhonetic}{客机}{ke4ji1}{9,6}{⼧,⽊}[HSK 7-9]
  \definition[架]{s.}{avião de passageiros; avião comercial}
  \antonymref{货机}{huo4ji1}
\end{EntryWithPhonetic}

\begin{EntryWithPhonetic}{客流}{ke4liu2}{9,10}{⼧,⽔}[HSK 7-9]
  \definition{s.}{fluxo de passageiros; o setor de transportes refere"-se ao fluxo de passageiros em uma determinada direção dentro de um determinado período de tempo | fluxo de clientes (frequentando uma loja, etc.)}
\end{EntryWithPhonetic}

\begin{EntryWithPhonetic}{客气}{ke4qi5}{9,4}{⼧,⽓}[HSK 5]
  \definition{adj.}{educado; modesto; cortês}
  \definition{v.}{ser educado; ser cortês; fazer comentários educados ou agir educadamente}
\end{EntryWithPhonetic}

\begin{EntryWithPhonetic}{客人}{ke4ren5}{9,2}{⼧,⼈}[HSK 2]
  \definition[位,个,桌,拨,批]{s.}{visitante; convidado | cliente; passageiro; hóspede; viajante}
\end{EntryWithPhonetic}

\begin{EntryWithPhonetic}{客厅}{ke4ting1}{9,4}{⼧,⼚}[HSK 5]
  \definition[间,个]{s.}{sala de estar; sala de visitas; sala para receber convidados}
\end{EntryWithPhonetic}

\begin{EntryWithPhonetic}{客运}{ke4yun4}{9,7}{⼧,⾡}[HSK 7-9]
  \definition{s.}{transporte de passageiros; (setor de transporte) o negócio de transporte de passageiros}
\end{EntryWithPhonetic}

%%%%%%%%%% 课 %%%%%%%%%%
\subsection*{课}\addcontentsline{loh}{figure}{课 \dpy{ke4}}

\begin{EntryWithPhonetic}{课}{ke4}{10}{⾔}[HSK 1]
  \definition{clas.}{aula; lição; unidade de tempo de ensino; parágrafo do material didático}
  \definition[门,节]{s.}{classe; aula; ensino por etapas planejado | disciplina; curso | imposto; antiga referência a impostos | seção; departamentos de escritório criados no antigo governo}
  \definition{v.}{cobrar; impor; taxar}
\end{EntryWithPhonetic}

\begin{EntryWithPhonetic}{课本}{ke4ben3}{10,5}{⾔,⽊}[HSK 1]
  \definition[本]{s.}{livro didático; livro-texto}
\end{EntryWithPhonetic}

\begin{EntryWithPhonetic}{课程}{ke4cheng2}{10,12}{⾔,⽲}[HSK 3]
  \definition[个,堂,节,门]{s.}{curso; currículo; as disciplinas e o programa letivo da escola}
\end{EntryWithPhonetic}

\begin{EntryWithPhonetic}{课堂}{ke4tang2}{10,11}{⾔,⼟}[HSK 2]
  \definition[间,节,个]{s.}{sala de aula; local onde se realizam as aulas; local onde se realizam as atividades de ensino}
\end{EntryWithPhonetic}

\begin{EntryWithPhonetic}{课题}{ke4ti2}{10,15}{⾔,⾴}[HSK 5]
  \definition[组]{s.}{uma questão para estudo ou discussão; principais questões a serem pesquisadas ou discutidas, ou assuntos importantes que precisam ser resolvidos com urgência | tarefa; problema; questões a serem resolvidas}
\end{EntryWithPhonetic}

\begin{EntryWithPhonetic}{课文}{ke4wen2}{10,4}{⾔,⽂}[HSK 1]
  \definition[篇,段]{s.}{texto (de uma lição); texto principal do livro didático (diferente das notas de rodapé, exercícios, etc.)}
\end{EntryWithPhonetic}

%%%%%%%%%% 肯 %%%%%%%%%%
\subsection*{肯}\addcontentsline{loh}{figure}{肯 \dpy{ken3}}

\begin{EntryWithPhonetic}{肯}{ken3}{8}{⾁}[HSK 6]
  \definition{s.}{carne presa ao osso}
  \definition{v.}{concordar; consentir}
  \definition{v.aux.}{estar disposto a; estar pronto para; para expressar vontade subjetiva; vontade de aceitar}
\end{EntryWithPhonetic}

\begin{EntryWithPhonetic}{肯定}{ken3ding4}{8,8}{⾁,⼧}[HSK 5]
  \definition{adj.}{certo; definitivo; positivo; afirmativo | positivo; afirmativo; aceitável}
  \definition{adv.}{certamente; definitivamente; sem dúvida; sem dúvida alguma}
  \definition{v.}{afirmar; aprovar; confirmar; considerar positivo; reconhecer a existência de algo ou sua autenticidade ou racionalidade}
  \antonymref{否定}{fou3ding4}
\end{EntryWithPhonetic}

%%%%%%%%%% 恳 %%%%%%%%%%
\subsection*{恳}\addcontentsline{loh}{figure}{恳 \dpy{ken3}}

\begin{EntryWithPhonetic}{恳}{ken3}{10}{⼼}
  \definition{adj.}{sério; sincero | cordial; honesto}
  \definition{v.}{pedir; suplicar; implorar; rogar}
\end{EntryWithPhonetic}

\begin{EntryWithPhonetic}{恳求}{ken3qiu2}{10,7}{⼼,⽔}[HSK 7-9]
  \definition{v.}{implorar; suplicar; rogar; solicitar encarecidamente}
\end{EntryWithPhonetic}

%%%%%%%%%% 啃 %%%%%%%%%%
\subsection*{啃}\addcontentsline{loh}{figure}{啃 \dpy{ken3}}

\begin{EntryWithPhonetic}{啃}{ken3}{11}{⼝}[HSK 7-9]
  \definition{v.}{roer; mordiscar | Figurativo: estudar}
\end{EntryWithPhonetic}

%%%%%%%%%% 坑 %%%%%%%%%%
\subsection*{坑}\addcontentsline{loh}{figure}{坑 \dpy{keng1}}

\begin{EntryWithPhonetic}{坑}{keng1}{7}{⼟}[HSK 7-9]
  \definition[个]{s.}{poço; buraco; cavidade | poço; túnel; caverna subterrânea}
  \definition{v.}{enredar; enganar; trapacear | nos tempos antigos, significava enterrar as pessoas vivas}
\end{EntryWithPhonetic}

\begin{EntryWithPhonetic}{坑人}{keng1/ren2}{7,2}{⼟,⼈}
  \definition{v.+compl.}{enganar; ludibriar | Dialeto: ficar chateado (com uma grande perda) | enganar alguém}
\end{EntryWithPhonetic}

%%%%%%%%%% 空 %%%%%%%%%%
\subsection*{空}\addcontentsline{loh}{figure}{空 \dpy{kong1}}

\begin{EntryWithPhonetic}{空}{kong1}{8}{⽳}[HSK 3]
  \definition*{s.}{Sobrenome: Kong}
  \definition{adj.}{vazio; oco; nulo; não inclui nada; não contém nada ou não tem conteúdo; irrealista}
  \definition{adv.}{por nada; em vão; sem efeito}
  \definition{s.}{céu; ar | vazio; vazio do mundo dos sentidos}
  \seeref{kong4}
\end{EntryWithPhonetic}

\begin{EntryWithPhonetic}{空荡荡}{kong1dang4dang4}{8,9,9}{⽳,⾋,⾋}[HSK 7-9]
  \definition{adj.}{vazio; deserto; descreve uma casa, terreno, etc., como estando muito vazio | vazio; desolado; descreve um estado de vazio espiritual e falta de plenitude}
\end{EntryWithPhonetic}

\begin{EntryWithPhonetic}{空地}{kong1di4}{8,6}{⽳,⼟}
  \definition{s.}{abertura; espaço vazio; área; gramado; terreno baldio; espaço aberto}
  \seeref{kong4di4}
\end{EntryWithPhonetic}

\begin{EntryWithPhonetic}{空间}{kong1jian1}{8,7}{⽳,⾨}[HSK 4]
  \definition[个]{s.}{espaço; recinto; cômodo; espaço em branco; interespaço}
\end{EntryWithPhonetic}

\begin{EntryWithPhonetic}{空间站}{kong1jian1zhan4}{8,7,10}{⽳,⾨,⽴}
  \definition{s.}{estação espacial}
\end{EntryWithPhonetic}

\begin{EntryWithPhonetic}{空姐}{kong1jie3}{8,8}{⽳,⼥}
  \definition[名,位,个]{s.}{aeromoça; comissária de bordo; abreviação de 空中小姐}
  \seealsoref{空中小姐}{kong1zhong1xiao3jie3}
\end{EntryWithPhonetic}

\begin{EntryWithPhonetic}{空军}{kong1jun1}{8,6}{⽳,⼍}[HSK 6]
  \definition[名,位,个,支]{s.}{força aérea; um exército que luta no ar, geralmente composto por várias unidades de aviação e unidades terrestres da força aérea}
\end{EntryWithPhonetic}

\begin{EntryWithPhonetic}{空难}{kong1nan4}{8,10}{⽳,⾫}[HSK 7-9]
  \definition{s.}{desastre aéreo; acidente aéreo; incidente aéreo; acidente de aviação}
\end{EntryWithPhonetic}

\begin{EntryWithPhonetic}{空气}{kong1qi4}{8,4}{⽳,⽓}[HSK 2]
  \definition[缕,股,份,阵]{s.}{ar; gases que compõe a atmosfera terrestre | atmosfera}
\end{EntryWithPhonetic}

\begin{EntryWithPhonetic}{空前}{kong1qian2}{8,9}{⽳,⼑}[HSK 7-9]
  \definition{adj.}{sem precedentes; nunca antes}
\end{EntryWithPhonetic}

\begin{EntryWithPhonetic}{空调}{kong1tiao2}{8,10}{⽳,⾔}[HSK 3]
  \definition[台,个]{s.}{ar-condicionado;  condicionador de ar}
\end{EntryWithPhonetic}

\begin{EntryWithPhonetic}{空想}{kong1xiang3}{8,13}{⽳,⼼}[HSK 7-9]
  \definition{s.}{pensamento irrealista; fantasia; devaneio | fantasia; sonho vão; esperança vã}
  \definition{v.}{entregar-se à fantasia; sonhar acordado}
\end{EntryWithPhonetic}

\begin{EntryWithPhonetic}{空心菜}{kong1xin1cai4}{8,4,11}{⽳,⼼,⾋}
  \definition{s.}{espinafre aquático | \emph{ong choy} | repolho do pântano | convolvulus aquático | glória-da-manhã aquática}
  \seealsoref{蕹菜}{weng4cai4}
\end{EntryWithPhonetic}

\begin{EntryWithPhonetic}{空虚}{kong1xu1}{8,11}{⽳,⾌}[HSK 7-9]
  \definition{adj.}{vazio; oco; não contém nada de substancial; não é substancial}
\end{EntryWithPhonetic}

\begin{EntryWithPhonetic}{空中}{kong1zhong1}{8,4}{⽳,⼁}[HSK 5]
  \definition{adj.}{aéreo; aerotransportado; refere"-se à transmissão de sinais de rádio}
  \definition{s.}{no céu; no ar}
\end{EntryWithPhonetic}

\begin{EntryWithPhonetic}{空中小姐}{kong1zhong1xiao3jie3}{8,4,3,8}{⽳,⼁,⼩,⼥}
  \definition{s.}{aeromoça}
\end{EntryWithPhonetic}

%%%%%%%%%% 孔 %%%%%%%%%%
\subsection*{孔}\addcontentsline{loh}{figure}{孔 \dpy{kong3}}

\begin{EntryWithPhonetic}{孔}{kong3}{4}{⼦}
  \definition*{s.}{Abreviação de Confúcio, 孔子 | Sobrenome: Kong}
  \definition{adj.}{Clássico: muito; bastante; razoavelmente}
  \definition{adj.}{aberto; desimpedido; claro; desobstruído}
  \definition{clas.}{usado para habitações em cavernas}
  \definition[个,排]{s.}{buraco; abertura; poro}
  \seealsoref{孔子}{kong3zi3}
\end{EntryWithPhonetic}

\begin{EntryWithPhonetic}{孔夫子}{kong3fu1zi3}{4,4,3}{⼦,⼤,⼦}
  \definition*{s.}{Confúcio (551-479 aC), pensador e filósofo social chinês}
  \seealsoref{孔子}{kong3zi3}
\end{EntryWithPhonetic}

\begin{EntryWithPhonetic}{孔雀}{kong3que4}{4,11}{⼦,⾫}
  \definition{s.}{pavão}
\end{EntryWithPhonetic}

\begin{EntryWithPhonetic}{孔子}{kong3zi3}{4,3}{⼦,⼦}
  \definition*{s.}{Confúcio (551-479 aC), pensador e filósofo social chinês}
  \seealsoref{孔夫子}{kong3fu1zi3}
\end{EntryWithPhonetic}

\begin{EntryWithPhonetic}{孔子学院}{kong3zi3 xue2yuan4}{4,3,8,9}{⼦,⼦,⼦,⾩}
  \definition*{s.}{Instituto Confúcio, organização estabelecida internacionalmente pela República Popular da China, que promove a língua e a cultura chinesas}
\end{EntryWithPhonetic}

%%%%%%%%%% 恐 %%%%%%%%%%
\subsection*{恐}\addcontentsline{loh}{figure}{恐 \dpy{kong3}}

\begin{EntryWithPhonetic}{恐}{kong3}{10}{⼼}
  \definition{adv.}{talvez; provavelmente}
  \definition{v.}{temer; recear; ter medo de | ameaçar; aterrorizar; intimidar}
\end{EntryWithPhonetic}

\begin{EntryWithPhonetic}{恐怖}{kong3bu4}{10,8}{⼼,⼼}[HSK 7-9]
  \definition[部]{adj.}{terrível; aterrador; horripilante; medo causado por ameaças à vida ou por presenciar violência ou derramamento de sangue | assustador; aterrorizante | terroristas; o comportamento ou os métodos utilizados são extremamente cruéis e perversos, causando choque e medo}
\end{EntryWithPhonetic}

\begin{EntryWithPhonetic}{恐怖主义}{kong3bu4zhu3yi4}{10,8,5,3}{⼼,⼼,⼂,⼂}
  \definition{adj.}{terrorista}
  \definition{s.}{terrorismo}
\end{EntryWithPhonetic}

\begin{EntryWithPhonetic}{恐吓}{kong3he4}{10,6}{⼼,⼝}[HSK 7-9]
  \definition{v.}{ameaçar; assustar; intimidar; ameaçar alguém com palavras ou meios ameaçadores}
\end{EntryWithPhonetic}

\begin{EntryWithPhonetic}{恐慌}{kong3huang1}{10,12}{⼼,⼼}[HSK 7-9]
  \definition{adj.}{pânico; em pânico; pânico devido ao medo}
  \definition{s.}{pânico; medo}
\end{EntryWithPhonetic}

\begin{EntryWithPhonetic}{恐惧}{kong3ju4}{10,11}{⼼,⼼}[HSK 7-9]
  \definition{adj.}{assustado; com medo; muito assustado}
\end{EntryWithPhonetic}

\begin{EntryWithPhonetic}{恐龙}{kong3long2}{10,5}{⼼,⿓}[HSK 7-9]
  \definition[只,头]{s.}{dinossauro | garota feia (gíria da \emph{Internet}, ofensiva)}
\end{EntryWithPhonetic}

\begin{EntryWithPhonetic}{恐怕}{kong3pa4}{10,8}{⼼,⼼}[HSK 3]
  \definition{adv.}{talvez; provavelmente; pode ser; expressa suposição; estimativa. | por medo de; expressar estimativa e preocupação}
  \definition{v.}{ter medo de; temer; recear}
\end{EntryWithPhonetic}

%%%%%%%%%% 空 %%%%%%%%%%
\subsection*{空}\addcontentsline{loh}{figure}{空 \dpy{kong4}}

\begin{EntryWithPhonetic}{空}{kong4}{8}{⽳}[HSK 4]
  \definition*{s.}{Sobrenome: Kong}
  \definition{adj.}{vazio; oco; nulo; que não contém nada; que não tem nada ou nenhum conteúdo; impraticável}
  \definition{adv.}{para nada; em vão; sem efeito}
  \definition{s.}{céu; ar | vazio; ausência do mundo dos sentidos}
  \seeref{kong1}
\end{EntryWithPhonetic}

\begin{EntryWithPhonetic}{空白}{kong4bai2}{8,5}{⽳,⽩}[HSK 7-9]
  \definition[块,片,个]{s.}{espaço; margem; espaço em branco; (na diagramação da página, páginas do livro, ilustrações, etc.) partes vazias, não preenchidas ou não utilizadas}
\end{EntryWithPhonetic}

\begin{EntryWithPhonetic}{空地}{kong4di4}{8,6}{⽳,⼟}[HSK 7-9]
  \definition{s.}{abertura; espaço vazio; área; gramado; terreno baldio; espaço aberto}
  \seeref{kong1di4}
\end{EntryWithPhonetic}

\begin{EntryWithPhonetic}{空儿}{kong4r5}{8,2}{⽳,⼉}[HSK 3]
  \definition[个]{s.}{tempo livre; sem horário específico | sala; espaço (não utilizado); área ainda não utilizada}
  \definition{v.}{ter tempo livre}
\end{EntryWithPhonetic}

\begin{EntryWithPhonetic}{空隙}{kong4xi4}{8,12}{⽳,⾩}[HSK 7-9]
  \definition{s.}{lacuna; vazio; espaço; folga; o espaço vazio no meio | intervalo; interstício; tempo livre não utilizado | chance; ocasião; oportunidade; lacunas; oportunidades a explorar}
\end{EntryWithPhonetic}

%%%%%%%%%% 控 %%%%%%%%%%
\subsection*{控}\addcontentsline{loh}{figure}{控 \dpy{kong4}}

\begin{EntryWithPhonetic}{控}{kong4}{11}{⼿}
  \definition{v.}{acusar; cobrar | controlar; dominar | manter (parte do corpo em uma determinada posição) sem apoio | virar (um recipiente) de cabeça para baixo para deixar o líquido escorrer}
\end{EntryWithPhonetic}

\begin{EntryWithPhonetic}{控告}{kong4gao4}{11,7}{⼿,⼝}[HSK 7-9]
  \definition{v.}{acusar; denunciar; incriminar; indiciar; processar alguém; apresentar uma queixa legal contra alguém}
\end{EntryWithPhonetic}

\begin{EntryWithPhonetic}{控制}{kong4zhi4}{11,8}{⼿,⼑}[HSK 5]
  \definition{v.}{controlar; restringir; dominar; fazer com que não ultrapasse um determinado limite | controlar; dominar; comandar; ocupar, fazer com que não se perca}
\end{EntryWithPhonetic}

%%%%%%%%%% 抠 %%%%%%%%%%
\subsection*{抠}\addcontentsline{loh}{figure}{抠 \dpy{kou1}}

\begin{EntryWithPhonetic}{抠}{kou1}{7}{⼿}[HSK 7-9]
  \definition{adj.}{Dialeto: mesquinho; avarento}
  \definition{v.}{escolher; cavar ou desenterrar com o dedo ou algo pontiagudo; arranhar | esculpir; cortar | aprofundar-se em; estudar meticulosamente; ir desnecessariamente ao âmago de}
\end{EntryWithPhonetic}

%%%%%%%%%% 口 %%%%%%%%%%
\subsection*{口}\addcontentsline{loh}{figure}{口 \dpy{kou3}}

\begin{EntryWithPhonetic}{口}{kou3}{3}{⼝}[HSK 1][Kangxi 30]
  \definition*{s.}{Sobrenome: Kou}
  \definition{clas.}{usado para coisas com bocas (pessoas, animais domésticos, canhões, etc.) | usado para mordidas ou bocados | usado para idiomas}
  \definition{s.}{boca | borda; boca; o espaço externo ao recipiente | saída; entrada; local de entrada e saída | o gosto de alguém | corte; buraco; ferida |  a borda de uma faca; lâminas de facas, espadas, tesouras, etc. | a idade de um animal de tração | seção; departamento; sistema integrado de departamentos relacionados | conversa, discurso; pronunciamento; referência à fala | um portão da Grande Muralha (frequentemente usado em nomes de lugares)}
\end{EntryWithPhonetic}

\begin{EntryWithPhonetic}{口碑}{kou3bei1}{3,13}{⼝,⽯}[HSK 7-9]
  \definition{s.}{elogio público; isso se refere à avaliação verbal que as pessoas fazem de alguém (antigamente, elogios a uma pessoa eram frequentemente gravados em tábuas de pedra)}
\end{EntryWithPhonetic}

\begin{EntryWithPhonetic}{口才}{kou3cai2}{3,3}{⼝,⼿}[HSK 7-9]
  \definition{s.}{eloquência; persuasão; a capacidade de se expressar verbalmente; o talento para a fala}
\end{EntryWithPhonetic}

\begin{EntryWithPhonetic}{口吃}{kou3chi1}{3,6}{⼝,⼝}[HSK 7-9]
  \definition{s.}{gagueira; espasmofemia; balbucinato; mogilalia; battarismo; battarismo; iscnofonia; pselismo; o fenômeno de repetir palavras ou interromper frases ao falar é um defeito habitual de linguagem comumente conhecido como gagueira}
\end{EntryWithPhonetic}

\begin{EntryWithPhonetic}{口吃病}{kou3chi1 bing4}{3,6,10}{⼝,⼝,⽧}
  \definition{s.}{doença da gagueira}
\end{EntryWithPhonetic}

\begin{EntryWithPhonetic}{口袋妖怪}{kou3dai4 yao1guai4}{3,11,7,8}{⼝,⾐,⼥,⼼}
  \definition*{s.}{Pokémon (franquia de mídia japonesa)}
\end{EntryWithPhonetic}

\begin{EntryWithPhonetic}{口袋}{kou3dai5}{3,11}{⼝,⾐}[HSK 4]
  \definition[个,只]{s.}{bolso | saco; sacola; artigos de tecido ou couro}
\end{EntryWithPhonetic}

\begin{EntryWithPhonetic}{口感}{kou3gan3}{3,13}{⼝,⼼}[HSK 7-9]
  \definition{s.}{textura (dos alimentos); sabor; sensação que o alimento proporciona na boca}
\end{EntryWithPhonetic}

\begin{EntryWithPhonetic}{口号}{kou3hao4}{3,5}{⼝,⼝}[HSK 5]
  \definition[个,条,些]{s.}{\emph{slogan}; palavra de ordem; lema}
\end{EntryWithPhonetic}

\begin{EntryWithPhonetic}{口径}{kou3jing4}{3,8}{⼝,⼻}[HSK 7-9]
  \definition{s.}{calibre; diâmetro; diâmetro da boca circular do vaso | requisitos; especificações; geralmente se refere às especificações, desempenho, etc., exigidos | declaração; pensamento; linha de ação ou fala; conteúdo metafórico da fala}
\end{EntryWithPhonetic}

\begin{EntryWithPhonetic}{口令}{kou3ling4}{3,5}{⼝,⼈}[HSK 7-9]
  \definition[串]{s.}{palavra de comando | senha; palavra-chave; contra-senha | palavra; comando; lema; comando verbal | códigos verbais para identificar amigo ou inimigo}
\end{EntryWithPhonetic}

\begin{EntryWithPhonetic}{口气}{kou3qi4}{3,4}{⼝,⽓}[HSK 7-9]
  \definition{s.}{maneira de falar; a força e o ritmo do tom; a dinâmica do discurso | implicação; o que realmente se quer dizer; o significado não dito nas palavras | tom; nota; tom emocional na fala}
\end{EntryWithPhonetic}

\begin{EntryWithPhonetic}{口腔}{kou3qiang1}{3,12}{⼝,⾁}[HSK 7-9]
  \definition{s.}{cavidade oral; a cavidade oral é um espaço oco composto pelos lábios, bochechas, palato duro e palato mole; contém órgãos como dentes, língua e glândulas salivares}
\end{EntryWithPhonetic}

\begin{EntryWithPhonetic}{口哨}{kou3shao4}{3,10}{⼝,⼝}[HSK 7-9]
  \definition{s.}{apito; assobio}
  \definition{v.}{assobiar}
\end{EntryWithPhonetic}

\begin{EntryWithPhonetic}{口试}{kou3shi4}{3,8}{⼝,⾔}[HSK 6]
  \definition{s.}{exame oral (ou teste); um tipo de exame que exige que os candidatos respondam a perguntas oralmente}
  \definition{v.}{examinar oralmente}
  \antonymref{笔试}{bi3shi4}
\end{EntryWithPhonetic}

\begin{EntryWithPhonetic}{口水}{kou3shui3}{3,4}{⼝,⽔}[HSK 7-9]
  \definition{s.}{saliva; baba; o termo geral para saliva}
\end{EntryWithPhonetic}

\begin{EntryWithPhonetic}{口头}{kou3tou2}{3,5}{⼝,⼤}[HSK 7-9]
  \definition{s.}{oral; verbal; expressa-se através da fala | boca (referindo"-se à boca ao falar); lábios}
  \synonymref{表面}{biao3mian4}
  \antonymref{思想}{si1xiang3}
  \antonymref{行动}{xing2dong4}
  \antonymref{行为}{xing2wei2}
\end{EntryWithPhonetic}

\begin{EntryWithPhonetic}{口味}{kou3wei4}{3,8}{⼝,⼝}[HSK 7-9]
  \definition[个,种]{s.}{sabor da comida; o gosto da comida | o gosto de uma pessoa; preferência de cada um em termos de sabor | gostos; metáfora para interesses e hobbies pessoais}
\end{EntryWithPhonetic}

\begin{EntryWithPhonetic}{口香糖}{kou3xiang1tang2}{3,9,16}{⼝,⾹,⽶}[HSK 7-9]
  \definition{s.}{goma de mascar; chiclete; um tipo de doce feito da seiva viscosa que escorre do tronco da sapotilha (uma árvore perene cujos frutos têm o tamanho de peras e formato de coração, daí o nome), juntamente com açúcar e aromatizantes; é para ser mastigado apenas e não deve ser engolido}
\end{EntryWithPhonetic}

\begin{EntryWithPhonetic}{口音}{kou3yin1}{3,9}{⼝,⾳}[HSK 7-9]
  \definition[种]{s.}{sotaques; sons da fala oral (linguística); os hábitos de fala de uma pessoa, especialmente seus hábitos de pronúncia, podem revelar sua origem linguística}
  \seeref{kou3yin5}
\end{EntryWithPhonetic}

\begin{EntryWithPhonetic}{口音}{kou3yin5}{3,9}{⼝,⾳}
  \definition[种]{s.}{voz | sotaque; dialeto}
  \seeref{kou3yin1}
\end{EntryWithPhonetic}

\begin{EntryWithPhonetic}{口语}{kou3yu3}{3,9}{⼝,⾔}[HSK 4]
  \definition[门]{s.}{linguagem oral; linguagem falada; linguagem coloquial; linguagem usada em conversas}
\end{EntryWithPhonetic}

\begin{EntryWithPhonetic}{口罩}{kou3zhao4}{3,13}{⼝,⽹}[HSK 7-9]
  \definition[副]{s.}{máscara cirúrgica; produtos de higiene; feitos de gaze, etc.; usados sobre a boca e o nariz para evitar a entrada de poeira e germes}
\end{EntryWithPhonetic}

\begin{EntryWithPhonetic}{口子}{kou3zi5}{3,3}{⼝,⼦}[HSK 7-9]
  \definition{clas.}{referente a pessoas}[你家有几口子?===Quantas pessoas há na sua família?]
  \definition{s.}{Coloquial: pessoa | marido ou esposa | abertura; buraco; corte; rasgo; uma grande lacuna; uma fenda | Figurativo: abertura; oportunidade}
\end{EntryWithPhonetic}

%%%%%%%%%% 扣 %%%%%%%%%%
\subsection*{扣}\addcontentsline{loh}{figure}{扣 \dpy{kou4}}

\begin{EntryWithPhonetic}{扣}{kou4}{6}{⼿}[HSK 6]
  \definition*{s.}{Sobrenome: Kou}
  \definition{clas.}{giro; volta; uma volta de uma rosca}
  \definition[个,颗,粒]{s.}{nó | fivela; botão | círculo de rosca (em um parafuso)}
  \definition{v.}{fivela; abotoar; amarrar ou prender com um laço ou anel | colocar uma xícara, tigela etc. de cabeça para baixo; cobrir com uma xícara, tigela etc. invertida; colocar a boca do recipiente para baixo | deter; prender; levar sob custódia | cravar; esmagar (a bola); arremessar ou bater (em uma bola) com força de cima para baixo | atracar; deduzir; descontar; subtrair uma parte do valor original | puxar; pressionar | impor; marcar sem fundamento; acusar injustamente; impor ou atribuir (um crime ou má fama) a alguém}
\end{EntryWithPhonetic}

\begin{EntryWithPhonetic}{扣除}{kou4chu2}{6,9}{⼿,⾩}[HSK 7-9]
  \definition{v.}{deduzir; tirar; subtrair do total}
\end{EntryWithPhonetic}

\begin{EntryWithPhonetic}{扣留}{kou4liu2}{6,10}{⼿,⽥}[HSK 7-9]
  \definition{s.}{apreensão; detenção}
  \definition{v.}{deter; manter sob custódia; pôr em prisão domiciliar; prender}
\end{EntryWithPhonetic}

\begin{EntryWithPhonetic}{扣人心弦}{kou4ren2xin1xian2}{6,2,4,8}{⼿,⼈,⼼,⼸}[HSK 7-9]
  \definition{expr.}{emocionante; cativar alguém; conquistar o coração de; comovente; tocar os sentimentos de alguém; tocar o coração de alguém; tocar profundamente; muito tocante; descrever poesia, prosa, performances, etc., como tendo uma qualidade contagiante que desperta emoções; eletrizante; de tirar o fôlego}
\end{EntryWithPhonetic}

\begin{EntryWithPhonetic}{扣押}{kou4ya1}{6,8}{⼿,⼿}[HSK 7-9]
  \definition{s.}{detenção}
  \definition{v.}{deter; apreender; manter sob custódia | apreender; confiscar; embargar | sequestrar (ou tomar, reter) alguém como refém}
\end{EntryWithPhonetic}

%%%%%%%%%% 枯 %%%%%%%%%%
\subsection*{枯}\addcontentsline{loh}{figure}{枯 \dpy{ku1}}

\begin{EntryWithPhonetic}{枯}{ku1}{9}{⽊}
  \definition{adj.}{murcho | (de um poço, rio, etc.) seco | chato; desinteressante | magro e abatido; emaciado}
  \definition[片]{s.}{borra; resíduo}
\end{EntryWithPhonetic}

\begin{EntryWithPhonetic}{枯木}{ku1mu4}{9,4}{⽊,⽊}
  \definition{s.}{árvore morta | madeira morta}
\end{EntryWithPhonetic}

\begin{EntryWithPhonetic}{枯燥}{ku1zao4}{9,17}{⽊,⽕}[HSK 7-9]
  \definition{adj.}{sem graça; monótono e sem vida; monótono; desinteressante}
\end{EntryWithPhonetic}

%%%%%%%%%% 哭 %%%%%%%%%%
\subsection*{哭}\addcontentsline{loh}{figure}{哭 \dpy{ku1}}

\begin{EntryWithPhonetic}{哭}{ku1}{10}{⼝}[HSK 2]
  \definition{v.}{chorar; soluçar; lamentar-se; chorar de dor ou de emoção}
\end{EntryWithPhonetic}

\begin{EntryWithPhonetic}{哭泣}{ku1qi4}{10,8}{⼝,⽔}[HSK 7-9]
  \definition{v.}{chorar; soluçar; chorar copiosamente}
\end{EntryWithPhonetic}

\begin{EntryWithPhonetic}{哭墙}{ku1qiang2}{10,14}{⼝,⼟}
  \definition*{s.}{Muro das Lamentações (Jerusalém)}
\end{EntryWithPhonetic}

\begin{EntryWithPhonetic}{哭笑不得}{ku1xiao4-bu4de2}{10,10,4,11}{⼝,⽵,⼀,⼻}[HSK 7-9]
  \definition{expr.}{``Sem saber se ria ou chorava.''; a incapacidade de chorar ou rir descreve uma situação constrangedora em que a pessoa não sabe o que fazer}
\end{EntryWithPhonetic}

%%%%%%%%%% 窟 %%%%%%%%%%
\subsection*{窟}\addcontentsline{loh}{figure}{窟 \dpy{ku1}}

\begin{EntryWithPhonetic}{窟}{ku1}{13}{⽳}
  \definition{s.}{buraco; caverna; cavidade; toca; gruta}
\end{EntryWithPhonetic}

\begin{EntryWithPhonetic}{窟窿}{ku1long5}{13,16}{⽳,⽳}[HSK 7-9]
  \definition[个]{s.}{furo; cavidade; buraco | déficit (figurativo); débito; dívida; essa metáfora se refere a déficits financeiros ou brechas no trabalho}
\end{EntryWithPhonetic}

%%%%%%%%%% 苦 %%%%%%%%%%
\subsection*{苦}\addcontentsline{loh}{figure}{苦 \dpy{ku3}}

\begin{EntryWithPhonetic}{苦}{ku3}{8}{⾋}[HSK 4]
  \definition{adj.}{amargo; descreve um sabor parecido com o de melão amargo ou raiz de coptis | difícil; doloroso; sofrido}
  \definition{adv.}{meticulosamente; diligentemente; pacientemente}
  \definition{v.}{causar sofrimento a alguém; dificultar a vida de alguém; causar dor; tornar desconfortável | sofrer de; ser incomodado por; sentir-se angustiado com uma situação | estar desgastado; cortar demais; descrever a superação de um certo nível em algum aspecto}
  \antonymref{甘}{gan1}
  \antonymref{甜}{tian2}
\end{EntryWithPhonetic}

\begin{EntryWithPhonetic}{苦瓜}{ku3gua1}{8,5}{⾋,⽠}
  \definition{s.}{melão amargo (cabaça amarga, pêra bálsamo, maçã bálsamo, pepino amargo)}
\end{EntryWithPhonetic}

\begin{EntryWithPhonetic}{苦力}{ku3li4}{8,2}{⾋,⼒}[HSK 7-9]
  \definition{s.}{trabalhador hindu ou chinês; carregador | Empréstimo linguístico: \emph{coolie}, trabalhador chinês não qualificado nos tempos coloniais | trabalho amargo | trabalho árduo}
\end{EntryWithPhonetic}

\begin{EntryWithPhonetic}{苦练}{ku3 lian4}{8,8}{⾋,⽷}[HSK 7-9]
  \definition{v.}{treinar diligentemente; praticar bastante}
\end{EntryWithPhonetic}

\begin{EntryWithPhonetic}{苦难}{ku3nan4}{8,10}{⾋,⾫}[HSK 7-9]
  \definition{adj.}{sofrimento; dificuldade}
  \definition{s.}{sofrimento; miséria; angústia; tribulação; dor e desastre}
\end{EntryWithPhonetic}

\begin{EntryWithPhonetic}{苦恼}{ku3nao3}{8,9}{⾋,⼼}[HSK 7-9]
  \definition{adj.}{aflito; preocupado; atormentado}
  \definition{v.}{atormentar; causar dor e sofrimento}
\end{EntryWithPhonetic}

\begin{EntryWithPhonetic}{苦笑}{ku3xiao4}{8,10}{⾋,⽵}[HSK 7-9]
  \definition{s.}{sorriso forçado; sorriso irônico; sorriso amargo}
  \definition{v.}{forçar um sorriso; esboçar um sorriso irônico; dar uma risada amarga; produzir um sorriso forçado; forçar um sorriso quando você não está de bom humor}
\end{EntryWithPhonetic}

\begin{EntryWithPhonetic}{苦心}{ku3xin1}{8,4}{⾋,⼼}[HSK 7-9]
  \definition{adv.}{com esmero; assiduamente; meticulosamente; com esforço ou dedicação; isso demonstra que você dedicou muito esforço e energia}
  \definition{s.}{dores; dificuldade tomada; o esforço e a energia investidos em trabalhar arduamente por algo}
\end{EntryWithPhonetic}

%%%%%%%%%% 库 %%%%%%%%%%
\subsection*{库}\addcontentsline{loh}{figure}{库 \dpy{ku4}}

\begin{EntryWithPhonetic}{库}{ku4}{7}{⼴}[HSK 5]
  \definition{s.}{depósito; tesouraria; armazém; almoxarifado; edifícios e equipamentos para armazenamento de mercadorias | Computação: banco de dados}
\end{EntryWithPhonetic}

%%%%%%%%%% 裤 %%%%%%%%%%
\subsection*{裤}\addcontentsline{loh}{figure}{裤 \dpy{ku4}}

\begin{EntryWithPhonetic}{裤}{ku4}{12}{⾐}
  \definition[条]{s.}{calças}
\end{EntryWithPhonetic}

\begin{EntryWithPhonetic}{裤子}{ku4zi5}{12,3}{⾐,⼦}[HSK 3]
  \definition[条]{s.}{calças; calções; roupas usadas abaixo da cintura, com cós, virilha e duas pernas}
\end{EntryWithPhonetic}

%%%%%%%%%% 酷 %%%%%%%%%%
\subsection*{酷}\addcontentsline{loh}{figure}{酷 \dpy{ku4}}

\begin{EntryWithPhonetic}{酷}{ku4}{14}{⾣}[HSK 6]
  \definition{adj.}{cruel; opressivo | feroz; escaldante | brutal | Empréstimo linguístico: \emph{cool}; legal; excelente; moderno; ótimo | elegante e sóbrio; gracioso e severo}
  \definition{adv.}{muito; extremamente}
\end{EntryWithPhonetic}

\begin{EntryWithPhonetic}{酷斯拉}{ku4si1la1}{14,12,8}{⾣,⽄,⼿}
  \definition*{s.}{Godzilla. do Japonês Gojira, ゴジラ}
  \seealsoref{哥斯拉}{ge1si1la1}
\end{EntryWithPhonetic}

\begin{EntryWithPhonetic}{酷似}{ku4si4}{14,6}{⾣,⼈}[HSK 7-9]
  \definition{v.}{ser a própria imagem de; ser exatamente igual a; apresentar forte semelhança com}
\end{EntryWithPhonetic}

%%%%%%%%%% 夸 %%%%%%%%%%
\subsection*{夸}\addcontentsline{loh}{figure}{夸 \dpy{kua1}}

\begin{EntryWithPhonetic}{夸}{kua1}{6}{⼤}[HSK 7-9]
  \definition{v.}{exagerar; superestimar; vangloriar-se | elogiar; destacar as qualidades positivas de uma pessoa ou coisa}
\end{EntryWithPhonetic}

\begin{EntryWithPhonetic}{夸大}{kua1da4}{6,3}{⼤,⼤}[HSK 7-9]
  \definition{v.}{exagerar; superestimar; magnificar; engrandecer}
\end{EntryWithPhonetic}

\begin{EntryWithPhonetic}{夸奖}{kua1jiang3}{6,9}{⼤,⼤}[HSK 7-9]
  \definition{v.}{louvar; elogiar; cumprimentar; demonstrar apreço e incentivar alguém por suas qualidades ou boas ações}
\end{EntryWithPhonetic}

\begin{EntryWithPhonetic}{夸夸其谈}{kua1kua1-qi2tan2}{6,6,8,10}{⼤,⼤,⼋,⾔}[HSK 7-9]
  \definition{expr.}{entregar-se à verborragia; conversa empolgante e jactanciosa sem muito significado; cheio de ar quente; um longo discurso cheio de pompa; tagarelice; falar demais; uma grande conversa; discursar; conversa pomposa; conversa ou escrita pomposa, mas sem sentido; discurso pomposo; exagerar; desabafar; exagerar (ou usar verborragia); tagarelar; falar pelos cotovelos}
\end{EntryWithPhonetic}

\begin{EntryWithPhonetic}{夸耀}{kua1yao4}{6,20}{⼤,⽻}[HSK 7-9]
  \definition{v.}{ostentar; gabar"-se de; vangloriar"-se de; exibir (as próprias habilidades, conquistas, status e poder)}
\end{EntryWithPhonetic}

\begin{EntryWithPhonetic}{夸张}{kua1zhang1}{6,7}{⼤,⼸}[HSK 7-9]
  \definition{adj.}{exagerado; superestimado; pomposo}
  \definition{s.}{hipérbole; uma figura de linguagem que utiliza palavras exageradas para descrever coisas}
\end{EntryWithPhonetic}

%%%%%%%%%% 垮 %%%%%%%%%%
\subsection*{垮}\addcontentsline{loh}{figure}{垮 \dpy{kua3}}

\begin{EntryWithPhonetic}{垮}{kua3}{9}{⼟}[HSK 7-9]
  \definition{v.}{colapsar; cair (quebrar); desmoronar | desmoronar (declínio mental e físico)}
\end{EntryWithPhonetic}

%%%%%%%%%% 挎 %%%%%%%%%%
\subsection*{挎}\addcontentsline{loh}{figure}{挎 \dpy{kua4}}

\begin{EntryWithPhonetic}{挎}{kua4}{9}{⼿}[HSK 7-9]
  \definition{v.}{carregar no braço | carregar algo sobre o ombro, ao redor do pescoço ou ao lado do corpo | pendurar coisas no ombro, pescoço ou cintura}
\end{EntryWithPhonetic}

%%%%%%%%%% 跨 %%%%%%%%%%
\subsection*{跨}\addcontentsline{loh}{figure}{跨 \dpy{kua4}}

\begin{EntryWithPhonetic}{跨}{kua4}{13}{⾜}[HSK 6]
  \definition{adj.}{localizado ao lado de; anexo a}
  \definition{v.}{dar um passo; andar a passos largos | disputar; ficar de pernas abertas | atravessar; ir além (dos limites de uma certa quantidade, tempo, região, etc.)}
\end{EntryWithPhonetic}

\begin{EntryWithPhonetic}{跨国}{kua4guo2}{13,8}{⾜,⼞}[HSK 7-9]
  \definition{adj.}{transnacional; que transcende fronteiras nacionais, envolvendo dois ou mais países}
\end{EntryWithPhonetic}

\begin{EntryWithPhonetic}{跨越}{kua4yue4}{13,12}{⾜,⾛}[HSK 7-9]
  \definition{v.}{passar por cima; saltar por cima; atravessar; dar um passo largo; cruzar fronteiras regionais ou de período}
\end{EntryWithPhonetic}

%%%%%%%%%% 会 %%%%%%%%%%
\subsection*{会}\addcontentsline{loh}{figure}{会 \dpy{kuai4}}

\begin{EntryWithPhonetic}{会}{kuai4}{6}{⼈}
  \definition[个,场,次]{s.}{contabilidade}
  \definition{v.}{computar; calcular; equilibrar uma conta}
  \seeref{hui4}
\end{EntryWithPhonetic}

\begin{EntryWithPhonetic}{会计}{kuai4ji5}{6,4}{⼈,⾔}[HSK 4]
  \definition[个,位,名]{s.}{contabilidade | contador; contabilista; guarda-livros; pessoal que trabalha como contador}
\end{EntryWithPhonetic}

%%%%%%%%%% 块 %%%%%%%%%%
\subsection*{块}\addcontentsline{loh}{figure}{块 \dpy{kuai4}}

\begin{EntryWithPhonetic}{块}{kuai4}{7}{⼟}[HSK 1]
  \definition{clas.}{usado para coisas em pedaços | usado para coisas em pedaços ou em algumas formas de folhas | usado para moedas de prata ou notas de papel equivalentes a 圆}
  \definition{s.}{pedaço; pedaço (de terra); peça; algo que forma um pedaço ou massa}
  \seealsoref{圆}{yuan2}
\end{EntryWithPhonetic}

%%%%%%%%%% 快 %%%%%%%%%%
\subsection*{快}\addcontentsline{loh}{figure}{快 \dpy{kuai4}}

\begin{EntryWithPhonetic}{快}{kuai4}{7}{⼼}[HSK 1]
  \definition*{s.}{Sobrenome: Kuai}
  \definition{adj.}{rápido; veloz | apressado | perspicaz; ágil; inteligente; de mente rápida | (faca, espada, etc.) afiado | direto; franco; sem rodeios | satisfeito; feliz; gratificado | rápido; veloz; alta velocidade; tempo de execução curto | satisfeito; feliz; contente | engenhoso; ágil | afiado; facas, tesouras, machados e outros objetos afiados | sincero}
  \definition{adv.}{em breve; antes de muito tempo; estar prestes a | rapidamente}
  \definition{s.}{policial; polícia | (antigo) oficial encarregado de efetuar prisões}
  \antonymref{钝}{dun4}
  \antonymref{慢}{man4}
\end{EntryWithPhonetic}

\begin{EntryWithPhonetic}{快餐}{kuai4can1}{7,16}{⼼,⾷}[HSK 2]
  \definition[份,顿]{s.}{pedido (comida) rápido; \emph{fast food}; refere"-se a refeições simples preparadas com antecedência e que podem ser servidas rapidamente}
\end{EntryWithPhonetic}

\begin{EntryWithPhonetic}{快车}{kuai4che1}{7,4}{⼼,⾞}[HSK 6]
  \definition{s.}{trem ou ônibus expresso; um trem ou ônibus com menos paradas e tempos de viagem mais curtos (usado principalmente para transporte de passageiros)}
  \antonymref{慢车}{man4che1}
\end{EntryWithPhonetic}

\begin{EntryWithPhonetic}{快递}{kuai4di4}{7,10}{⼼,⾡}[HSK 4]
  \definition[个,件,批]{s.}{correio rápido; entrega expressa; entrega rápida}
  \definition{v.}{entregar (serviço de entrega rápida por transportadoras especializadas)}
\end{EntryWithPhonetic}

\begin{EntryWithPhonetic}{快点儿}{kuai4dian3r5}{7,9,2}{⼼,⽕,⼉}[HSK 2]
  \definition{v.}{apressar-se}
\end{EntryWithPhonetic}

\begin{EntryWithPhonetic}{快活}{kuai4huo5}{7,9}{⼼,⽔}[HSK 5]
  \definition{adj.}{feliz; alegre; contente; animado}
\end{EntryWithPhonetic}

\begin{EntryWithPhonetic}{快捷}{kuai4jie2}{7,11}{⼼,⼿}[HSK 7-9]
  \definition{adj.}{rápido; veloz (em relação à velocidade); ágil (em relação a pessoas)}
\end{EntryWithPhonetic}

\begin{EntryWithPhonetic}{快乐}{kuai4le4}{7,5}{⼼,⼃}[HSK 2]
  \definition{adj.}{feliz; alegre; animado; prazeiroso}
  \definition{s.}{felicidade | alegria}
\end{EntryWithPhonetic}

\begin{EntryWithPhonetic}{快速}{kuai4su4}{7,10}{⼼,⾡}[HSK 3]
  \definition{adj.}{rápido; veloz; de alta velocidade; descreve o tempo curto gasto para caminhar, fazer algo, etc.}
\end{EntryWithPhonetic}

\begin{EntryWithPhonetic}{快要}{kuai4yao4}{7,9}{⼼,⾑}[HSK 2]
  \definition{adv.}{estar prestes a; estar indo para; estar à beira de; em breve; em pouco tempo; indica que a situação está prestes a ocorrer}
\end{EntryWithPhonetic}

%%%%%%%%%% 筷 %%%%%%%%%%
\subsection*{筷}\addcontentsline{loh}{figure}{筷 \dpy{kuai4}}

\begin{EntryWithPhonetic}{筷}{kuai4}{13}{⽵}
  \definition[双,根,个]{s.}{pauzinhos para comer}
\end{EntryWithPhonetic}

\begin{EntryWithPhonetic}{筷子}{kuai4zi5}{13,3}{⽵,⼦}[HSK 2]
  \definition[根,双,副,把,对]{s.}{pauzinhos; \emph{chopsticks}; dois bastôes finos feitos de bambu, madeira, metal ou outro material, usados para segurar comida ou outros objetos}
\end{EntryWithPhonetic}

%%%%%%%%%% 宽 %%%%%%%%%%
\subsection*{宽}\addcontentsline{loh}{figure}{宽 \dpy{kuan1}}

\begin{EntryWithPhonetic}{宽}{kuan1}{10}{⼧}[HSK 4]
  \definition*{s.}{Sobrenome: Kuan}
  \definition{adj.}{largo; amplo; espaçoso; grandes distâncias horizontais | leniente; generoso; indulgente | bem de vida; rico; confortável}
  \definition[米]{s.}{largura; amplitude}[桌子有一米宽。===A mesa tem um metro de largura.]
  \definition{v.}{relaxar; aliviar}
  \antonymref{窄}{zhai3}
\end{EntryWithPhonetic}

\begin{EntryWithPhonetic}{宽敞}{kuan1chang5}{10,12}{⼧,⽁}[HSK 7-9]
  \definition{adj.}{amplo; espaçoso; descreve um espaço ou área grande}
\end{EntryWithPhonetic}

\begin{EntryWithPhonetic}{宽度}{kuan1du4}{10,9}{⼧,⼴}[HSK 5]
  \definition{s.}{largura; amplitude; duração; o grau de largura e estreiteza; a distância horizontal (no caso de um retângulo, a distância entre os dois lados mais longos)}
\end{EntryWithPhonetic}

\begin{EntryWithPhonetic}{宽泛}{kuan1fan4}{10,7}{⼧,⽔}[HSK 7-9]
  \definition{adj.}{abrangente; (conteúdo, significado) abrange uma ampla gama de aspectos}
\end{EntryWithPhonetic}

\begin{EntryWithPhonetic}{宽广}{kuan1guang3}{10,3}{⼧,⼴}[HSK 4]
  \definition{adj.}{vasto; amplo; espaçoso; extenso}
\end{EntryWithPhonetic}

\begin{EntryWithPhonetic}{宽厚}{kuan1hou4}{10,9}{⼧,⼚}[HSK 7-9]
  \definition{adj.}{largo e espesso; amplo e sólido | tolerante; gentil e generoso; tolerância e bondade | simples; sincero; (voz) profunda e ressonante}
\end{EntryWithPhonetic}

\begin{EntryWithPhonetic}{宽阔}{kuan1kuo4}{10,12}{⼧,⾨}[HSK 6]
  \definition{adj.}{amplo; largo; espaçoso | tolerante; mente aberta; descreve uma mente alegre e ampla}
\end{EntryWithPhonetic}

\begin{EntryWithPhonetic}{宽容}{kuan1rong2}{10,10}{⼧,⼧}[HSK 7-9]
  \definition{adj.}{tolerante; generoso e magnânimo, não mesquinho ou oportunista}
  \definition{v.}{tolerar; ter paciência com; ser tolerante com os outros, não guardar rancor nem insistir no assunto}
\end{EntryWithPhonetic}

\begin{EntryWithPhonetic}{宽恕}{kuan1shu4}{10,10}{⼧,⼼}[HSK 7-9]
  \definition{v.}{perdoar; desculpar; absolver}
\end{EntryWithPhonetic}

\begin{EntryWithPhonetic}{宽松}{kuan1song1}{10,8}{⼧,⽊}[HSK 7-9]
  \definition{adj.}{(roupas) folgado e confortável; espaçoso e sem aglomeração; relaxante e sem aperto | (estado mental, atmosfera, etc.) relaxado; aliviado; sem tensão; livre de preocupações; descontraído; não há tensão | Economia: abundante; que tem dinheiro suficiente para viver bem e sem problemas, mas sem ser excessivamente rico}
\end{EntryWithPhonetic}

\begin{EntryWithPhonetic}{宽影片}{kuan1ying3pian4}{10,15,4}{⼧,⼺,⽚}
  \definition{s.}{filme \emph{widescreen}}
\end{EntryWithPhonetic}

%%%%%%%%%% 款 %%%%%%%%%%
\subsection*{款}\addcontentsline{loh}{figure}{款 \dpy{kuan3}}

\begin{EntryWithPhonetic}{款}{kuan3}{12}{⽋}
  \definition{adj.}{sincero | lento; sem pressa | vazio; oco; irreal; é intercambiável com 窾}
  \definition{s.}{parágrafo; seção (de um artigo em um documento legal, etc.); os itens listados de acordo com as disposições de leis, regulamentos, tratados, etc. | dinheiro; fundo; uma quantia de dinheiro | o nome do remetente ou destinatário inscrito em uma pintura ou obra de caligrafia oferecida como presente; inscrições fundidas em recipientes de bronze, como sinos e tripés; títulos em pinturas e caligrafia | forma; estilo; especificações}
  \definition[笔,个]{v.}{entreter; receber com hospitalidade | Literário: bater; prostrar-se}
  \seealsoref{窾}{kuan3}
\end{EntryWithPhonetic}

\begin{EntryWithPhonetic}{款式}{kuan3shi4}{12,6}{⽋,⼷}[HSK 7-9]
  \definition[种,个,款,类]{s.}{modelo; estilo; design; moda; padrão; formato}
\end{EntryWithPhonetic}

\begin{EntryWithPhonetic}{款项}{kuan3xiang4}{12,9}{⽋,⾴}[HSK 7-9]
  \definition[宗]{s.}{soma de dinheiro; refere"-se a uma grande quantia de dinheiro com um propósito específico | cláusula (ordem, regra, tratado); artigos (em leis, regulamentos, tratados, etc.); itens dentro de artigos de leis, regulamentos, tratados, etc., são geralmente divididos em cláusulas, e cláusulas em subitens}
\end{EntryWithPhonetic}

%%%%%%%%%% 窾 %%%%%%%%%%
\subsection*{窾}\addcontentsline{loh}{figure}{窾 \dpy{kuan3}}

\begin{EntryWithPhonetic}{窾}{kuan3}{17}{⽳}
  \definition{adj.}{oco}
  \definition{s.}{rachadura; cavidade | (onomatopéia) água batendo na rocha}
  \definition{v.}{escavar um buraco}
  \seeref{cuan4}
\end{EntryWithPhonetic}

%%%%%%%%%% 筐 %%%%%%%%%%
\subsection*{筐}\addcontentsline{loh}{figure}{筐 \dpy{kuang1}}

\begin{EntryWithPhonetic}{筐}{kuang1}{12}{⽵}[HSK 7-9]
  \definition[个,只]{s.}{cesto; cestaria; caixa; recipientes trançados com tiras de bambu, galhos de salgueiro e sarças}
\end{EntryWithPhonetic}

%%%%%%%%%% 狂 %%%%%%%%%%
\subsection*{狂}\addcontentsline{loh}{figure}{狂 \dpy{kuang2}}

\begin{EntryWithPhonetic}{狂}{kuang2}{7}{⽝}[HSK 5]
  \definition*{s.}{Sobrenome: Kuang}
  \definition{adj.}{louco; maluco | violento; selvagem | selvagem; delirante; furioso; desenfreado; desinibido; sem restrições | arrogante; autoritário}
\end{EntryWithPhonetic}

\begin{EntryWithPhonetic}{狂欢}{kuang2huan1}{7,6}{⽝,⽋}[HSK 7-9]
  \definition{s.}{festejar; fazer folia; carnavalear; aproveitar a diversão}
  \definition{s.}{folia; carnaval}
\end{EntryWithPhonetic}

\begin{EntryWithPhonetic}{狂欢节}{kuang2huan1jie2}{7,6,5}{⽝,⽋,⾋}[HSK 7-9]
  \definition*{s.}{Carnaval}
\end{EntryWithPhonetic}

\begin{EntryWithPhonetic}{狂热}{kuang2re4}{7,10}{⽝,⽕}[HSK 7-9]
  \definition{adj.}{fanático; febril; o entusiasmo extremo que foi despertado em um determinado momento}
  \definition{s.}{febre; emoções extremamente intensas}
\end{EntryWithPhonetic}

%%%%%%%%%% 况 %%%%%%%%%%
\subsection*{况}\addcontentsline{loh}{figure}{况 \dpy{kuang4}}

\begin{EntryWithPhonetic}{况}{kuang4}{7}{⼎}
  \definition*{s.}{Sobrenome: Kuang}
  \definition{conj.}{além disso | mesmo; muito menos; sem mencionar}
  \definition{s.}{condição; situação}
  \definition{v.}{comparar}
\end{EntryWithPhonetic}

\begin{EntryWithPhonetic}{况且}{kuang4qie3}{7,5}{⼎,⼀}
  \definition{conj.}{além disso; além do mais; orações de conexão para expressar uma relação progressiva}
\end{EntryWithPhonetic}

%%%%%%%%%% 旷 %%%%%%%%%%
\subsection*{旷}\addcontentsline{loh}{figure}{旷 \dpy{kuang4}}

\begin{EntryWithPhonetic}{旷}{kuang4}{7}{⽇}
  \definition*{s.}{Sobrenome: Kuang}
  \definition{adj.}{vasto; espaçoso | livre de preocupações e ideias mesquinhas | folgado}
  \definition{v.}{negligenciar ou desperdiçar | estar ausente de | desperdiçar; abandonar; negligenciar}
\end{EntryWithPhonetic}

\begin{EntryWithPhonetic}{旷课}{kuang4/ke4}{7,10}{⽇,⾔}[HSK 7-9]
  \definition{v.+compl.}{matar aula; cabular aula; faltar às aulas}
\end{EntryWithPhonetic}

\begin{EntryWithPhonetic}{旷野}{kuang4ye3}{7,11}{⽇,⾥}
  \definition{s.}{região selvagem; as planícies abertas}
\end{EntryWithPhonetic}

%%%%%%%%%% 矿 %%%%%%%%%%
\subsection*{矿}\addcontentsline{loh}{figure}{矿 \dpy{kuang4}}

\begin{EntryWithPhonetic}{矿}{kuang4}{8}{⽯}[HSK 6]
  \definition[个,座]{s.}{depósito de minério | minério | mina}
\end{EntryWithPhonetic}

\begin{EntryWithPhonetic}{矿藏}{kuang4cang2}{8,17}{⽯,⾋}[HSK 7-9]
  \definition{s.}{depósito mineral; recurso mineral; termo genérico para todos os tipos de minerais enterrados no subsolo}
\end{EntryWithPhonetic}

\begin{EntryWithPhonetic}{矿泉水}{kuang4quan2shui3}{8,9,4}{⽯,⽔,⽔}[HSK 4]
  \definition[瓶,杯,口]{s.}{água mineral de nascente}
\end{EntryWithPhonetic}

%%%%%%%%%% 框 %%%%%%%%%%
\subsection*{框}\addcontentsline{loh}{figure}{框 \dpy{kuang4}}

\begin{EntryWithPhonetic}{框}{kuang4}{10}{⽊}[HSK 7-9]
  \definition{s.}{moldura; estojo | caixa; bloco}
  \definition{v.}{Obsoleto: desenhar uma moldura ao redor; adicionar linhas ao redor do texto e das imagens | Obsoleto: restringir; confinar; conter; amarrar; colocar em uma camisa de força}
\end{EntryWithPhonetic}

\begin{EntryWithPhonetic}{框架}{kuang4jia4}{10,9}{⽊,⽊}[HSK 7-9]
  \definition[个,副,种,套]{s.}{moldura; estrutura; na construção civil, as estruturas são formadas por conexões como vigas e colunas | estrutura; estrutura básica de um sistema, texto, etc.; metáfora para a organização e estrutura das coisas}
\end{EntryWithPhonetic}

%%%%%%%%%% 亏 %%%%%%%%%%
\subsection*{亏}\addcontentsline{loh}{figure}{亏 \dpy{kui1}}

\begin{EntryWithPhonetic}{亏}{kui1}{3}{⼆}[HSK 5]
  \definition{adv.}{felizmente; por sorte; graças a | contrariamente, expressando sarcasmo}
  \definition{s.}{prejuízo; perda; déficit | perda; dano; ferida}
  \definition{v.}{perder dinheiro, etc.; ter um déficit; ter prejuízo | ter falta de; ser deficiente; carecer de | tratar injustamente; causar prejuízo; trair a confiança}
\end{EntryWithPhonetic}

\begin{EntryWithPhonetic}{亏本}{kui1/ben3}{3,5}{⼆,⽊}[HSK 7-9]
  \definition{v.+compl.}{estar no vermelho; perder o capital; perder dinheiro (nos negócios)}
\end{EntryWithPhonetic}

\begin{EntryWithPhonetic}{亏损}{kui1sun3}{3,10}{⼆,⼿}[HSK 7-9]
  \definition{v.}{perder; esgotar; ter déficit; as despesas excedem as receitas | desidratar; enfraquecer; o corpo está enfraquecido devido a danos ou falta de nutrição}
\end{EntryWithPhonetic}

%%%%%%%%%% 葵 %%%%%%%%%%
\subsection*{葵}\addcontentsline{loh}{figure}{葵 \dpy{kui2}}

\begin{EntryWithPhonetic}{葵}{kui2}{12}{⾋}
  \definition*{s.}{Sobrenome: Kui}
  \definition[朵]{s.}{certas ervas de flores grandes}
\end{EntryWithPhonetic}

\begin{EntryWithPhonetic}{葵花}{kui2hua1}{12,7}{⾋,⾋}
  \definition{s.}{girassol (flor)}
\end{EntryWithPhonetic}

%%%%%%%%%% 昆 %%%%%%%%%%
\subsection*{昆}\addcontentsline{loh}{figure}{昆 \dpy{kun1}}

\begin{EntryWithPhonetic}{昆}{kun1}{8}{⽇}
  \definition*{s.}{Sobrenome: Kun}
  \definition{s.}{irmão mais velho | descendentes; filhos}
\end{EntryWithPhonetic}

\begin{EntryWithPhonetic}{昆虫}{kun1chong2}{8,6}{⽇,⾍}[HSK 7-9]
  \definition[只,种,个,群,堆]{s.}{inseto; uma classe de artrópodes}
\end{EntryWithPhonetic}

%%%%%%%%%% 捆 %%%%%%%%%%
\subsection*{捆}\addcontentsline{loh}{figure}{捆 \dpy{kun3}}

\begin{EntryWithPhonetic}{捆}{kun3}{10}{⼿}[HSK 7-9]
  \definition{clas.}{feixe; maço; materiais usados para amarrar}
  \definition{s.}{coisas que estão agrupadas}
  \definition{v.}{amarrar; prender; agrupar | amarrar; acorrentar; algemar | agrupar; enfardar}
  \seealsoref{捆儿}{kun3r5}
\end{EntryWithPhonetic}

\begin{EntryWithPhonetic}{捆儿}{kun3r5}{10,2}{⼿,⼉}
  \definition{s.}{coisas que estão agrupadas}
  \seealsoref{捆}{kun3}
\end{EntryWithPhonetic}

%%%%%%%%%% 困 %%%%%%%%%%
\subsection*{困}\addcontentsline{loh}{figure}{困 \dpy{kun4}}

\begin{EntryWithPhonetic}{困}{kun4}{7}{⼞}[HSK 3]
  \definition{adj.}{cansado; exausto; fatigado | difícil; complicado; difícil e penoso; pobre e miserável | sonolento; com sono; cansado, com vontade de dormir}
  \definition{v.}{ficar encalhado; estar em apuros; preso em dificuldades e sofrimentos ou limitado por circunstâncias e condições que não pode escapar | cercar; envolver; imobilizar; controlar dentro de um determinado limite | dormir}
\end{EntryWithPhonetic}

\begin{EntryWithPhonetic}{困惑}{kun4huo4}{7,12}{⼞,⼼}[HSK 7-9]
  \definition{adj.}{em um quebra-cabeça; me sinto confuso e sem saber o que fazer}
  \definition{v.}{sentir-se perplexo; ser intrigante o porquê, e eu não saber o que fazer}
\end{EntryWithPhonetic}

\begin{EntryWithPhonetic}{困境}{kun4jing4}{7,14}{⼞,⼟}[HSK 7-9]
  \definition{s.}{dilema; situação difícil; apuro}
\end{EntryWithPhonetic}

\begin{EntryWithPhonetic}{困难}{kun4nan5}{7,10}{⼞,⾫}[HSK 3]
  \definition{adj.}{dificuldades financeiras; circunstâncias difíceis | complicado; complexo; difícil; árduo; a situação é complexa e há muitos obstáculos}
  \definition[种]{s.}{dificuldade; situação difícil; problemas ou situações difíceis de resolver no trabalho e na vida}
\end{EntryWithPhonetic}

\begin{EntryWithPhonetic}{困扰}{kun4rao3}{7,7}{⼞,⼿}[HSK 5]
  \definition{v.}{perturbar; deixar perplexo; perseguir}
\end{EntryWithPhonetic}

%%%%%%%%%% 扩 %%%%%%%%%%
\subsection*{扩}\addcontentsline{loh}{figure}{扩 \dpy{kuo4}}

\begin{EntryWithPhonetic}{扩}{kuo4}{6}{⼿}[HSK 7-9]
  \definition{v.}{expandir; ampliar; estender; alargar}
\end{EntryWithPhonetic}

\begin{EntryWithPhonetic}{扩大}{kuo4da4}{6,3}{⼿,⼤}[HSK 4]
  \definition{v.}{ampliar; expandir; estender; alargar}
\end{EntryWithPhonetic}

\begin{EntryWithPhonetic}{扩建}{kuo4jian4}{6,8}{⼿,⼵}[HSK 7-9]
  \definition{v.}{expandir; ampliar; ampliar a escala original do edifício ou a escala da área}
\end{EntryWithPhonetic}

\begin{EntryWithPhonetic}{扩散}{kuo4san4}{6,12}{⼿,⽁}[HSK 7-9]
  \definition{v.}{espalhar; difundir; dispersar; proliferar}
\end{EntryWithPhonetic}

\begin{EntryWithPhonetic}{扩展}{kuo4zhan3}{6,10}{⼿,⼫}[HSK 4]
  \definition{v.}{esticar; expandir; estender; espalhar}
\end{EntryWithPhonetic}

\begin{EntryWithPhonetic}{扩张}{kuo4zhang1}{6,7}{⼿,⼸}[HSK 7-9]
  \definition{v.}{expandir; aumentar; estender; espalhar; engrandecer | dilatar (dilatação vascular)}
\end{EntryWithPhonetic}

%%%%%%%%%% 括 %%%%%%%%%%
\subsection*{括}\addcontentsline{loh}{figure}{括 \dpy{kuo4}}

\begin{EntryWithPhonetic}{括}{kuo4}{9}{⼿}
  \definition{v.}{unir (músculos, etc.); contrair | incluir | adicionar colchetes a | amarrar; empacotar}
\end{EntryWithPhonetic}

\begin{EntryWithPhonetic}{括号}{kuo4hao4}{9,5}{⼿,⼝}[HSK 4]
  \definition{s.}{chaves, colchetes e parênteses (em fórmulas aritméticas ou algébricas, os símbolos que indicam a combinação e a ordem de vários números ou termos) | colchetes e parênteses usados como um tipo de sinal de pontuação para mostrar a parte explicativa de uma passagem em um texto}
\end{EntryWithPhonetic}

\begin{EntryWithPhonetic}{括弧}{kuo4hu2}{9,8}{⼿,⼸}[HSK 7-9]
  \definition{s.}{parênteses; também podem se referir a indicadores}
\end{EntryWithPhonetic}

%%%%%%%%%% 阔 %%%%%%%%%%
\subsection*{阔}\addcontentsline{loh}{figure}{阔 \dpy{kuo4}}

\begin{EntryWithPhonetic}{阔}{kuo4}{12}{⾨}[HSK 6]
  \definition{adj.}{amplo; amplo; vasto | rico | longo, no sentido de ``há muito tempo'' | vazio; impraticável}
\end{EntryWithPhonetic}

\begin{EntryWithPhonetic}{阔绰}{kuo4chuo4}{12,11}{⾨,⽷}[HSK 7-9]
  \definition{adj.}{ostentoso; generoso com dinheiro; extravagante; luxuoso}
\end{EntryWithPhonetic}

%%%%% EOF %%%%%


 %%%
%%% L
%%%

\section*{L}\addcontentsline{toc}{section}{L}

\begin{EntryWithPhonetic}{垃}{la1}{8}{⼟}
  \definition[堆]{s.}{lixo}
\end{EntryWithPhonetic}

\begin{EntryWithPhonetic}{垃圾}{la1 ji1}{8,6}{⼟、⼟}[HSK 4]
  \definition{adj.}{lixo; inútil, ruim ou prejudicial}
  \definition[袋,桶,堆,车,片]{s.}{entulho; lixo; refugo; rejeito; resíduo; coisa inútil que é jogada fora; metáfora para alguém ou algo que perdeu seu valor ou serve a um propósito ruim}
\end{EntryWithPhonetic}

\begin{EntryWithPhonetic}{垃圾车}{la1ji1che1}{8,6,4}{⼟、⼟、⾞}
  \definition{s.}{caminhão de lixo}
\end{EntryWithPhonetic}

\begin{EntryWithPhonetic}{垃圾电邮}{la1ji1 dian4you2}{8,6,5,7}{⼟、⼟、⽥、⾢}
  \definition{s.}{\emph{e-mail} de \emph{spam}}
  \seealsoref{垃圾邮件}{la1ji1 you2jian4}
\end{EntryWithPhonetic}

\begin{EntryWithPhonetic}{垃圾堆}{la1ji1dui1}{8,6,11}{⼟、⼟、⼟}
  \definition{s.}{depósito de lixo}
\end{EntryWithPhonetic}

\begin{EntryWithPhonetic}{垃圾工}{la1ji1gong1}{8,6,3}{⼟、⼟、⼯}
  \definition{s.}{lixeiro | gari}
\end{EntryWithPhonetic}

\begin{EntryWithPhonetic}{垃圾食品}{la1ji1shi2pin3}{8,6,9,9}{⼟、⼟、⾷、⼝}
  \definition{s.}{\emph{junk food}}
\end{EntryWithPhonetic}

\begin{EntryWithPhonetic}{垃圾筒}{la1ji1tong3}{8,6,12}{⼟、⼟、⽵}
  \definition{s.}{cesto de lixo}
\end{EntryWithPhonetic}

\begin{EntryWithPhonetic}{垃圾箱}{la1ji1xiang1}{8,6,15}{⼟、⼟、⾋}
  \definition{s.}{cesto de lixo}
\end{EntryWithPhonetic}

\begin{EntryWithPhonetic}{垃圾邮件}{la1ji1 you2jian4}{8,6,7,6}{⼟、⼟、⾢、⼈}
  \definition{s.}{\emph{spam}, \emph{e-mail} não solicitado}
  \seealsoref{垃圾电邮}{la1ji1 dian4you2}
\end{EntryWithPhonetic}

\begin{EntryWithPhonetic}{拉}{la1}{8}{⼿}[HSK 2]
  \definition{s.}{abreviação de América Latina, 拉丁美洲}
  \definition{v.}{puxar; arrastar; rebocar | transportar por veículo; rebocar | arrastar (ou puxar) para fora | mover (tropas para um lugar) | dar uma mãozinha; ajudar | arrastar para dentro; implicar; envolver | criar (criança) | atrair; conquistar; solicitar; angariar votos | bater-papo | organizar; preparar | ter dívidas; estar endividado | pressionar; recrutar à força | (no tênis, tênis de mesa, etc.) levantar (a bola) | tocar (certos instrumentos musicais); puxar uma parte do instrumento para que ele emita som | prolongar; espaçar | envolver-se em | (coloquial) esvaziar os intestinos | levantar, uma das técnicas do tênis de mesa | destruir; esmagar; quebrar}
  \seeref{la4}
  \seealsoref{拉丁美洲}{la1ding1 mei3zhou1}
\end{EntryWithPhonetic}

\begin{EntryWithPhonetic}{拉布布}{la1bu4bu4}{8,5,5}{⼿、⼱、⼱}
  \definition*{s.}{Labubu}
\end{EntryWithPhonetic}

\begin{EntryWithPhonetic}{拉丁美洲}{la1ding1 mei3zhou1}{8,2,9,9}{⼿、⼀、⽺、⽔}
  \definition*{s.}{América Latina, nome coletivo dos países da América Central e do Sul, devido ao fato de a maioria de seus habitantes ser descendente de povos latinos e de a língua falada ser do grupo latino}
\end{EntryWithPhonetic}

\begin{EntryWithPhonetic}{拉开}{la1 kai1}{8,4}{⼿、⼶}[HSK 4]
  \definition{v.}{puxar para abrir; recuar| ampliar; espaçar; distanciar; afastar; separar}
\end{EntryWithPhonetic}

\begin{EntryWithPhonetic}{拉拉队}{la1la1dui4}{8,8,4}{⼿、⼿、⾩}
  \definition{s.}{claque | torcida}
\end{EntryWithPhonetic}

\begin{EntryWithPhonetic}{拉萨}{la1sa4}{8,11}{⼿、⾋}
  \definition*{s.}{Lhasa, capital da Região Autônoma do Tibete, 西藏自治区}
  \seealsoref{西藏自治区}{xi1zang4 zi4zhi4qu1}
\end{EntryWithPhonetic}

\begin{EntryWithPhonetic}{啦}{la1}{11}{⼝}
  \definition{s.}{(onomatoméia) som de canto, aplausos etc.; usado para palavras como 呼啦啦, 哗啦啦, 哩哩啦啦, etc.}
  \seeref{la5}
  \seealsoref{呼啦啦}{hu1 la1 la1}
  \seealsoref{哗啦啦}{hua1la1 la5}
  \seealsoref{哩哩啦啦}{li1 li1 la1 la1}
\end{EntryWithPhonetic}

\begin{EntryWithPhonetic}{拉}{la4}{8}{⼿}
  \definition{s.}{usado em 拉拉蛄 \dpy{la4la4gu3}}
  \seeref{la1}
  \seealsoref{拉拉蛄}{la4la4gu3}
\end{EntryWithPhonetic}

\begin{EntryWithPhonetic}{拉拉蛄}{la4la4gu3}{8,8,11}{⼿、⼿、⾍}
  \variantof{蝲蝲蛄}
\end{EntryWithPhonetic}

\begin{EntryWithPhonetic}{落}{la4}{12}{⾋}[HSK 5]
  \definition{v.}{deixar de fora; estar ausente | deixar para trás; esquecer de trazer; deixar algo em algum lugar e esquecer de levar| ficar para trás (ou cair); não conseguir acompanhar}
  \seeref{lao4}
  \seeref{luo4}
\end{EntryWithPhonetic}

\begin{EntryWithPhonetic}{蜡}{la4}{14}{⾍}
  \definition{s.}{cera; óleos produzidos por animais, minerais ou plantas | vela}
\end{EntryWithPhonetic}

\begin{EntryWithPhonetic}{蜡烛}{la4zhu2}{14,10}{⾍、⽕}
  \definition[根,支]{s.}{vela | círio | peça, geralmente de cera, que possui um pavio e se utiliza para iluminar}
\end{EntryWithPhonetic}

\begin{EntryWithPhonetic}{辣}{la4}{14}{⾟}[HSK 4]
  \definition{adj.}{apimentado; picante; pungente; quente | cruel; implacável; venenoso; vicioso}
  \definition{v.}{queimar; picar; formigar; ter uma irritação picante (boca, nariz ou olhos)}
\end{EntryWithPhonetic}

\begin{EntryWithPhonetic}{蝲}{la4}{15}{⾍}
  \definition{s.}{lagostim de água doce}
  \seealsoref{蝲蛄}{la4gu3}
\end{EntryWithPhonetic}

\begin{EntryWithPhonetic}{蝲蛄}{la4gu3}{15,11}{⾍、⾍}
  \definition{s.}{lagostim; lagostim de água doce}
\end{EntryWithPhonetic}

\begin{EntryWithPhonetic}{蝲蝲蛄}{la4la4gu3}{15,15,11}{⾍、⾍、⾍}
  \definition{s.}{grilo toupeira}
\end{EntryWithPhonetic}

\begin{EntryWithPhonetic}{啦}{la5}{11}{⼝}[HSK 6]
  \definition{part.}{uma palavra composta de 了 e 啊, que tem o significado de ambos}
  \seeref{la1}
  \seealsoref{啊}{a5}
  \seealsoref{了}{le5}
\end{EntryWithPhonetic}

\begin{EntryWithPhonetic}{来}{lai2}{7}{⽊}[HSK 1]
  \definition*{s.}{Sobrenome Lai}
  \definition{part.}{usado após uma palavra numérica ou de quantidade; indica uma quantidade aproximada | usado depois de numerais como 一, 二, 三; para listar razões ou fatos, etc.}
  \definition{s.}{usado após uma expressão de tempo para indicar uma duração que vai do passado ao presente}
  \definition{v.}{vir; chegar; de outro lugar para o lugar onde o interlocutor se encontra | aparecer; acontecer; vir; (problemas, coisas, etc.) ocorrerem; surgirem | substitui um verbo com significado específico, indicando a realização de uma ação específica | estar indo para; usado antes de outro verbo, indica que algo será feito | vir para fazer algo; usado após outro verbo, indica que se vai fazer algo | usado para indicar um propósito; expressar o objetivo, fazer algo usando o método, a atitude ou a direção anteriores | usado com 得 ou 不 para indicar possibilidade, capacidade ou hábito}
  \seealsoref{不}{bu4}
  \seealsoref{得}{de5}
\end{EntryWithPhonetic}

\begin{EntryWithPhonetic}{来不及}{lai2bu5ji2}{7,4,3}{⽊、⼀、⼃}[HSK 4]
  \definition{v.}{ser tarde demais; não ter tempo; não ter tempo suficiente (para fazer algo); não ser possível participar ou se atualizar devido a restrições de tempo}
\end{EntryWithPhonetic}

\begin{EntryWithPhonetic}{来到}{lai2 dao4}{7,8}{⽊、⼑}[HSK 1]
  \definition{v.}{chegar; vir}
\end{EntryWithPhonetic}

\begin{EntryWithPhonetic}{来得及}{lai2de5ji2}{7,11,3}{⽊、⼻、⼃}[HSK 4]
  \definition{v.}{ainda ter tempo; ser capaz de fazer isso; ser capaz de fazer algo a tempo; ainda ter tempo de cuidar ou de colocar em dia}
\end{EntryWithPhonetic}

\begin{EntryWithPhonetic}{来往}{lai2 wang3}{7,8}{⽊、⼻}[HSK 6]
  \definition{s.}{negociação; contato com alguém; interações sociais}
  \definition{v.}{ir e vir | ter negócios com alguém}
\end{EntryWithPhonetic}

\begin{EntryWithPhonetic}{来信}{lai2 xin4}{7,9}{⽊、⼈}[HSK 5]
  \definition[封]{s.}{sua carta; carta recebida; carta ao interlocutor}
  \definition{v.}{enviar uma carta para aqui; enviar uma carta para o remetente}
\end{EntryWithPhonetic}

\begin{EntryWithPhonetic}{来源}{lai2yuan2}{7,13}{⽊、⽔}[HSK 4]
  \definition{s.}{origem; causa; fonte; tabula rasa (ou seja, o lugar de onde as coisas vêm)}
  \definition{v.}{originar-se; surgir; ter origem; (algo) originar (seguido de 于)}
  \seealsoref{于}{yu2}
\end{EntryWithPhonetic}

\begin{EntryWithPhonetic}{来自}{lai2zi4}{7,6}{⽊、⾃}[HSK 2]
  \definition{v.}{vir de (um local) | \emph{From:} (cabeçalho de \emph{e -mail})}
\end{EntryWithPhonetic}

\begin{EntryWithPhonetic}{赖}{lai4}{13}{⾙}[HSK 6]
  \definition*{s.}{Sobrenome Lai}
  \definition{adj.}{ruim; pobre; não é bom}
  \definition{v.}{confiar em; depender de | permanecer em um lugar; prolongar a permanência de alguém em um lugar; ficar e recusar-se a sair | negar o próprio erro ou responsabilidade; voltar atrás na palavra; repudiar; negar; não admitir culpa; não assumir responsabilidade | colocar a culpa nos outros; incriminar falsamente (acusar); acusar alguém de algo errado; acusar alguém falsamente | culpar}
\end{EntryWithPhonetic}

\begin{EntryWithPhonetic}{兰}{lan2}{5}{⼋}
  \definition*{s.}{Sobrenome Lan}
  \definition{s.}{orquídea | lírio magnólia}
\end{EntryWithPhonetic}

\begin{EntryWithPhonetic}{兰花}{lan2hua1}{5,7}{⼋、⾋}
  \definition{s.}{orquídea}
\end{EntryWithPhonetic}

\begin{EntryWithPhonetic}{兰州}{lan2zhou1}{5,6}{⼋、⼮}
  \definition*{s.}{Lanzhou. capital da província de Gansu, 甘肃}
  \seealsoref{甘肃}{gan1su4}
\end{EntryWithPhonetic}

\begin{EntryWithPhonetic}{栏}{lan2}{9}{⽊}
  \definition{s.}{cerca; corrimão; balaustrada | curral; galpão; celeiro; chiqueiro | coluna (de uma página ou tabela, ou de um jornal) | quadro (de avisos); prancha; tabuleiro | Esporte: obstáculo}
\end{EntryWithPhonetic}

\begin{EntryWithPhonetic}{栏目}{lan2mu4}{9,5}{⽊、⽬}[HSK 6]
  \definition[个,档]{s.}{coluna; programa; seções nomeadas de jornais, revistas, etc. divididas de acordo com a natureza de seu conteúdo}
\end{EntryWithPhonetic}

\begin{EntryWithPhonetic}{蓝}{lan2}{13}{⾋}[HSK 2]
  \definition*{s.}{Sobrenome Lan}
  \definition{adj.}{azul}
  \definition{s.}{planta índigo; anil | plantas azuis; refere-se a certas plantas que podem ser usadas como corante azul ou certas plantas cujas folhas são azul-esverdeadas}
\end{EntryWithPhonetic}

\begin{EntryWithPhonetic}{蓝领}{lan2 ling3}{13,11}{⾋、⾴}[HSK 6]
  \definition[名,位,个]{s.}{trabalhador braçal}
\end{EntryWithPhonetic}

\begin{EntryWithPhonetic}{蓝色}{lan2 se4}{13,6}{⾋、⾊}[HSK 2]
  \definition[抹,片,缕,团,块]{s.}{cor azul}
\end{EntryWithPhonetic}

\begin{EntryWithPhonetic}{篮}{lan2}{16}{⽵}
  \definition[个]{s.}{cesto | o anel de ferro e a rede na cesta de basquete}
\end{EntryWithPhonetic}

\begin{EntryWithPhonetic}{篮球}{lan2qiu2}{16,11}{⽵、⽟}[HSK 2]
  \definition[个,只]{s.}{basquetebol | bola de basquete; refere-se à bola utilizada no basquetebol}
\end{EntryWithPhonetic}

\begin{EntryWithPhonetic}{懒}{lan3}{16}{⼼}[HSK 6]
  \definition{adj.}{indolente; preguiçoso (oposto de 勤) | lento; lânguido | ocioso; preguiçoso}
  \seealsoref{勤}{qin2}
\end{EntryWithPhonetic}

\begin{EntryWithPhonetic}{懒虫}{lan3chong2}{16,6}{⼼、⾍}
  \definition{s.}{desleixado ocioso | (insulto) sujeito preguiçoso}
\end{EntryWithPhonetic}

\begin{EntryWithPhonetic}{懒怠}{lan3dai4}{16,9}{⼼、⼼}
  \definition{s.}{preguiça}
\end{EntryWithPhonetic}

\begin{EntryWithPhonetic}{懒得}{lan3de5}{16,11}{⼼、⼻}
  \definition{adv.}{demasiado preguiçoso}
  \definition{v.}{não sentir vontade (de fazer algo)}
\end{EntryWithPhonetic}

\begin{EntryWithPhonetic}{懒惰}{lan3duo4}{16,12}{⼼、⼼}
  \definition{adj.}{preguiçoso}
\end{EntryWithPhonetic}

\begin{EntryWithPhonetic}{懒鬼}{lan3gui3}{16,9}{⼼、⿁}
  \definition{s.}{cara preguiçoso}
\end{EntryWithPhonetic}

\begin{EntryWithPhonetic}{懒汉}{lan3han4}{16,5}{⼼、⽔}
  \definition{s.}{sujeito ocioso | vagabundo | preguiçosos}
\end{EntryWithPhonetic}

\begin{EntryWithPhonetic}{懒人}{lan3ren2}{16,2}{⼼、⼈}
  \definition{s.}{pessoa preguiçosa}
\end{EntryWithPhonetic}

\begin{EntryWithPhonetic}{懒散}{lan3san3}{16,12}{⼼、⽁}
  \definition{adj.}{inativo | indolente | preguiçoso | negligente}
\end{EntryWithPhonetic}

\begin{EntryWithPhonetic}{懒腰}{lan3yao1}{16,13}{⼼、⾁}
  \definition[个]{s.}{alongamento (do corpo)}
\end{EntryWithPhonetic}

\begin{EntryWithPhonetic}{烂}{lan4}{9}{⽕}[HSK 5]
  \definition{adj.}{macio; pastoso; amassado | podre; deteriorado | quebrado; esfarrapado; gasto | desorganizado; indigno}
  \definition{adv.}{totalmente; extremamente; completamente; expressa um grau muito profundo}
  \definition{v.}{apodrecer; infeccionar; decompor-se}
\end{EntryWithPhonetic}

\begin{EntryWithPhonetic}{廊}{lang2}{11}{⼴}
  \definition[个]{s.}{varanda; corredor}
\end{EntryWithPhonetic}

\begin{EntryWithPhonetic}{廊坊}{lang2fang2}{11,7}{⼴、⼟}
  \definition*{s.}{Cidade de Langfang em Hebei}
\end{EntryWithPhonetic}

\begin{EntryWithPhonetic}{朗}{lang3}{10}{⽉}
  \definition*{s.}{Sobrenome Lang}
  \definition{adj.}{claro; brilhante | alto e claro (som)}
\end{EntryWithPhonetic}

\begin{EntryWithPhonetic}{朗读}{lang3du2}{10,10}{⽉、⾔}[HSK 5]
  \definition{v.}{ler em voz alta; recitar com voz clara e alta}
\end{EntryWithPhonetic}

\begin{EntryWithPhonetic}{浪}{lang4}{10}{⽔}
  \definition*{s.}{Sobrenome Lang}
  \definition{adj.}{desenfreado; perdulário}
  \definition{adv.}{livremente}
  \definition[朵,阵,波]{s.}{onda; vagalhão; rebentação | algo ondulatório | coisas ondulando como ondas}
  \definition{v.}{passear; divagar}
\end{EntryWithPhonetic}

\begin{EntryWithPhonetic}{浪费}{lang4fei4}{10,9}{⽔、⾙}[HSK 3]
  \definition{adj.}{desperdiçado; extravagante; não econômico}
  \definition{v.}{desperdiçar; dissipar; esbanjar; ser extravagante; uso excessivo ou inadequado de bens, recursos humanos, tempo, etc.}
\end{EntryWithPhonetic}

\begin{EntryWithPhonetic}{浪花}{lang4hua1}{10,7}{⽔、⾋}
  \definition[朵]{s.}{\emph{spray} | \emph{spray} do oceano | (figurativo) acontecimentos de sua vida}
\end{EntryWithPhonetic}

\begin{EntryWithPhonetic}{浪漫}{lang4man4}{10,14}{⽔、⽔}[HSK 5]
  \definition{adj.}{romântico; poético | não convencional; boêmio; abandonado; libertino; devasso; comportar-se de maneira descuidada e descuidada (geralmente se referindo a relacionamentos entre pessoas) | irrealista; impraticável}
\end{EntryWithPhonetic}

\begin{EntryWithPhonetic}{捞}{lao1}{10}{⼿}
  \definition{v.}{pescar | dragar}
\end{EntryWithPhonetic}

\begin{EntryWithPhonetic}{劳}{lao2}{7}{⼒}
  \definition*{s.}{Sobrenome Lao}
  \definition{adj.}{difícil; cansativo; cansado}
  \definition{s.}{fadiga; trabalho árduo | ação meritória; serviço; conquistas | trabalhador | mérito | trabalhador braçal}
  \definition{v.}{trabalho; labor | esforço; exercício intenso | (pedir um favor a alguém, também 有劳) colocar alguém no trabalho de | expressar apreço (ao executor de uma tarefa); recompensar | colocar alguém no trabalho de; incomodar alguém com algo | trazer presentes para}
  \seealsoref{有劳}{you3lao2}
\end{EntryWithPhonetic}

\begin{EntryWithPhonetic}{劳动}{lao2dong4}{7,6}{⼒、⼒}[HSK 5]
  \definition[次]{s.}{trabalho; mão de obra; atividades intelectuais ou físicas que podem criar valor | trabalho físico; trabalho manual; referindo-se especificamente ao trabalho físico}
  \definition{v.}{realizar trabalho físico}
\end{EntryWithPhonetic}

\begin{EntryWithPhonetic}{劳工同事}{lao2gong1 tong2shi4}{7,3,6,8}{⼒、⼯、⼝、⼅}
  \definition{s.}{colaborador | colega de trabalho}
\end{EntryWithPhonetic}

\begin{EntryWithPhonetic}{牢}{lao2}{7}{⼧}[HSK 6]
  \definition*{s.}{Sobrenome Lao}
  \definition{adj.}{firme; durável}
  \definition{s.}{prisão; cadeia | (cercado para animais) curral; baia; galinheiro; estábulo; estrebaria; cocheira | (arcaico) animal de sacrifício}
\end{EntryWithPhonetic}

\begin{EntryWithPhonetic}{老}{lao3}{6}{⽼}[HSK 1,2][Kangxi 125]
  \definition*{s.}{Sobrenome Lao}
  \definition{adj.}{velho; envelhecido; idade avançada | antigo; de longa data; existe há muito tempo | antigo; desatualizado; obsoleto; ultrapassado  | antigo; tradicional; original | coberto de vegetação; os vegetais cresceram além do período ideal para serem consumidos | resistente; endurecido; alimentos muito cozidos | escuro; profundo; (sobre cores) | último nascido; o mais novo | veterano; experiente; sofisticado}
  \definition{adv.}{longo; por muito tempo | sempre (fazendo algo) | muito}
  \definition{pref.}{usado para designar pessoas, ordem de classificação, certos nomes de animais e plantas}
  \definition{s.}{idosos; pessoas mais velhas | ancião; sênior; um título respeitoso para pessoas mais velhas}
  \definition{v.}{envelhecer | morrer; referindo-se à morte de um idoso}
\end{EntryWithPhonetic}

\begin{EntryWithPhonetic}{老百姓}{lao3bai3xing4}{6,6,8}{⽼、⽩、⼥}[HSK 3]
  \definition[些]{s.}{povo; civis; pessoas comuns; residentes (em contraste com militares e funcionários públicos)}
\end{EntryWithPhonetic}

\begin{EntryWithPhonetic}{老板}{lao3ban3}{6,8}{⽼、⽊}[HSK 3]
  \definition[个,位]{s.}{chefe; dono; líder; refere-se ao gerente de uma empresa comercial ou industrial | antigo título honorífico dado a atores famosos de ópera ou atores que também eram diretores de companhias de ópera}
\end{EntryWithPhonetic}

\begin{EntryWithPhonetic}{老兵}{lao3bing1}{6,7}{⽼、⼋}
  \definition{s.}{velho soldado | veterano de guerra | veterano (alguém que tem muita experiência em algum domínio)}
\end{EntryWithPhonetic}

\begin{EntryWithPhonetic}{老公}{lao3 gong1}{6,4}{⽼、⼋}[HSK 4]
  \definition[个,位,名]{s.}{marido; esposo}
\end{EntryWithPhonetic}

\begin{EntryWithPhonetic}{老虎}{lao3hu3}{6,8}{⽼、⾌}
  \definition[只]{s.}{tigre}
  \seealsoref{虎}{hu3}
\end{EntryWithPhonetic}

\begin{EntryWithPhonetic}{老家}{lao3 jia1}{6,10}{⽼、⼧}[HSK 4]
  \definition{s.}{cidade natal; local de origem | lugar nativo; refere-se às gerações anteriores da família ou ao local onde a pessoa nasceu ou viveu}
\end{EntryWithPhonetic}

\begin{EntryWithPhonetic}{老年}{lao3 nian2}{6,6}{⽼、⼲}[HSK 2]
  \definition[个]{s.}{idoso; velhice; idade acima de 60 ou 70 anos}
\end{EntryWithPhonetic}

\begin{EntryWithPhonetic}{老朋友}{lao3 peng2 you3}{6,8,4}{⽼、⽉、⼜}[HSK 2]
  \definition[个,位,名]{s.}{velho amigo; refere-se a amigos que conhecemos há muito tempo e com quem temos uma relação íntima}
\end{EntryWithPhonetic}

\begin{EntryWithPhonetic}{老婆}{lao3po2}{6,11}{⽼、⼥}[HSK 4]
  \definition[个,位,名]{s.}{esposa}
\end{EntryWithPhonetic}

\begin{EntryWithPhonetic}{老人}{lao3 ren2}{6,2}{⽼、⼈}[HSK 1]
  \definition[位]{s.}{homem ou mulher de idade avançada; o idoso; o velho}
\end{EntryWithPhonetic}

\begin{EntryWithPhonetic}{老人家}{lao3 ren2 jia1}{6,2,10}{⽼、⼈、⼧}
  \definition[位,名,个]{s.}{avô; avó; pessoa idosa venerável; um título respeitoso para os idosos | maneira de chamar o pai ou a mãe idosos na frente dos outros; referir-se aos próprios pais ou aos pais, professores, etc. de outras pessoas}
\end{EntryWithPhonetic}

\begin{EntryWithPhonetic}{老师}{lao3shi1}{6,6}{⽼、⼱}[HSK 1]
  \definition[个,位]{s.}{professor; título honorífico para professores; refere-se, de maneira geral, a pessoas que transmitem cultura e tecnologia ou que são dignas de admiração em termos de ideias, moralidade e conhecimentos profissionais}
\end{EntryWithPhonetic}

\begin{EntryWithPhonetic}{老是}{lao3 shi4}{6,9}{⽼、⽇}[HSK 2]
  \definition{adv.}{sempre; indica que a ação continua ou que o estado permanece inalterado, equivalente a 一直}
  \seealsoref{一直}{yi4zhi2}
\end{EntryWithPhonetic}

\begin{EntryWithPhonetic}{老实}{lao3shi5}{6,8}{⽼、⼧}[HSK 4]
  \definition{adj.}{franco; sincero; honesto | bom; bem-comportado | ingênuo; simplório; meio bobo; facilmente enganado; eufemismo para pouco inteligente}
\end{EntryWithPhonetic}

\begin{EntryWithPhonetic}{老太太}{lao3 tai4 tai5}{6,4,4}{⽼、⼤、⼤}[HSK 3]
  \definition[位,名,个]{s.}{velha senhora; (em tratamento direto)Venerável Senhora; uma maneira respeitosa de chamar uma senhora idosa; título honorífico para mulheres idosas | (forma de tratamento) sua velha mãe; minha velha mãe, avó ou sogra; referindo-se à própria mãe, à mãe do outro ou à mãe de outra pessoa, à sogra ou à sogra política}
\end{EntryWithPhonetic}

\begin{EntryWithPhonetic}{老头儿}{lao3 tou2r5}{6,5,2}{⽼、⼤、⼉}[HSK 3]
  \definition{s.}{(coloquial) (com um tom de intimidade) velho; velho amigo}
  \seealsoref{老头子}{lao3 tou2zi5}
\end{EntryWithPhonetic}

\begin{EntryWithPhonetic}{老头子}{lao3 tou2zi5}{6,5,3}{⽼、⼤、⼦}
  \definition{s.}{velho antiquado (ou velho rabugento) | (referindo-se ao marido idoso) meu velho | chefe de uma sociedade secreta | (coloquial) velho; velho rabugento}
  \seealsoref{老头儿}{lao3 tou2r5}
\end{EntryWithPhonetic}

\begin{EntryWithPhonetic}{老乡}{lao3 xiang1}{6,3}{⽼、⼄}[HSK 6]
  \definition[个,位]{s.}{conterrâneo; conterrâneo | uma maneira de chamar um fazendeiro cujo nome você não conhece}
\end{EntryWithPhonetic}

\begin{EntryWithPhonetic}{落}{lao4}{12}{⾋}
  \definition{v.}{cair; cair de uma altura elevada | se abaixar; descer; ir para baixo | permanecer; fazer uma parada; deixar para trás | obter; ter; receber}
  \seeref{la4}
  \seeref{luo4}
\end{EntryWithPhonetic}

\begin{EntryWithPhonetic}{乐}{le4}{5}{⼃}[HSK 3]
  \definition*{s.}{Sobrenome Le}
  \definition{adj.}{feliz; contente; rejubilante; animado; bem disposto}
  \definition{s.}{prazer; diversão; felicidade}
  \definition{v.}{desfrutar; ficar feliz em; amar; encontrar prazer em | rir; divertir-se}
  \seeref{yue4}
\end{EntryWithPhonetic}

\begin{EntryWithPhonetic}{乐高}{le4gao1}{5,10}{⼃、⾼}
  \definition*{s.}{Lego (brinquedo)}
\end{EntryWithPhonetic}

\begin{EntryWithPhonetic}{乐观}{le4guan1}{5,6}{⼃、⾒}[HSK 3]
  \definition{adj.}{esperançoso; otimista; confiante; espírito alegre, confiante no futuro (oposto a 悲观)}
  \seealsoref{悲观}{bei1guan1}
\end{EntryWithPhonetic}

\begin{EntryWithPhonetic}{乐趣}{le4qu4}{5,15}{⼃、⾛}[HSK 4]
  \definition[个,种,些,点]{s.}{alegria; deleite; prazer; implicação de fazer alguém se sentir feliz; um humor de preferência}
\end{EntryWithPhonetic}

\begin{EntryWithPhonetic}{乐园}{le4yuan2}{5,7}{⼃、⼞}
  \definition{s.}{paraíso}
\end{EntryWithPhonetic}

\begin{EntryWithPhonetic}{了}{le5}{2}{⼅}[HSK 1,3]
  \definition{part.}{usada após verbos ou adjetivos para indicar a conclusão de uma ação, em um momento no passado ou antes do início de outra ação, ou uma ação esperada ou presumida | usada para indicar uma mudança de situação ou estado, seja real ou prevista | comandos ou solicitações em resposta a uma situação alterada; usada para xpressar urgência ou dissuadir | usada para indicar que algo chegou ao extremo; usada no final da frase ou em pausas no meio da frase, para expressar um tom de exclamação}
  \seeref{liao3}
\end{EntryWithPhonetic}

\begin{EntryWithPhonetic}{累}{lei2}{11}{⽷}
  \definition*{s.}{Sobrenome Lei}
  \definition{adj.}{incômodo; complicado}
  \definition{s.}{corda; cordão | touro na época de acasalamento}
  \definition{v.}{amarrar; prender; atar | copular}
  \seeref{lei3}
  \seeref{lei4}
\end{EntryWithPhonetic}

\begin{EntryWithPhonetic}{雷}{lei2}{13}{⾬}
  \definition*{s.}{Sobrenome Lei}
  \definition[声,个,颗]{s.}{trovão | (militar) mina}
\end{EntryWithPhonetic}

\begin{EntryWithPhonetic}{雷电}{lei2dian4}{13,5}{⾬、⽥}
  \definition{s.}{trovão e relâmpago; raio}
\end{EntryWithPhonetic}

\begin{EntryWithPhonetic}{雷亚尔}{lei2ya4'er3}{13,6,5}{⾬、⼆、⼩}
  \definition*{s.}{Real Brasileiro}
\end{EntryWithPhonetic}

\begin{EntryWithPhonetic}{累}{lei3}{11}{⽷}
  \definition*{s.}{Sobrenome Lei}
  \definition{adj.}{em andamento; repetido; contínuo}
  \definition{v.}{acumular; empilhar; colocar em cima de outro | envolver; implicar | construir empilhando tijolos, pedras, terra, etc.}
  \seeref{lei2}
  \seeref{lei4}
\end{EntryWithPhonetic}

\begin{EntryWithPhonetic}{絫}{lei3}{12}{⽷}
  \variantof{累}
\end{EntryWithPhonetic}

\begin{EntryWithPhonetic}{泪}{lei4}{8}{⽔}[HSK 4]
  \definition[滴,行]{s.}{lágrima | algo semelhante a uma lágrima}
\end{EntryWithPhonetic}

\begin{EntryWithPhonetic}{泪水}{lei4 shui3}{8,4}{⽔、⽔}[HSK 4]
  \definition[滴,行]{s.}{lágrima}
\end{EntryWithPhonetic}

\begin{EntryWithPhonetic}{类}{lei4}{9}{⽶}[HSK 3]
  \definition*{s.}{Sobrenome Lei}
  \definition{clas.}{tipo; espécie; categoria usada para pessoas ou coisas}
  \definition{s.}{classe; categoria; tipo; variedade; a combinação de muitas coisas semelhantes ou iguais}
  \definition{v.}{assemelhar-se a; ser semelhante a}
\end{EntryWithPhonetic}

\begin{EntryWithPhonetic}{类似}{lei4si4}{9,6}{⽶、⼈}[HSK 3]
  \definition{adj.}{semelhante; análogo}
\end{EntryWithPhonetic}

\begin{EntryWithPhonetic}{类型}{lei4xing2}{9,9}{⽶、⼟}[HSK 4]
  \definition[种,个]{s.}{tipo; espécie; categoria; tipos formados por coisas com características comuns}
\end{EntryWithPhonetic}

\begin{EntryWithPhonetic}{累}{lei4}{11}{⽷}[HSK 1]
  \definition{adj.}{cansado; exausto; fatigado}
  \definition{v.}{cansar; desgastar; fatigar; esgotar | labutar; trabalhar duro}
  \seeref{lei2}
  \seeref{lei3}
\end{EntryWithPhonetic}

\begin{EntryWithPhonetic}{冷}{leng3}{7}{⼎}[HSK 1]
  \definition*{s.}{Sobrenome Leng}
  \definition{adj.}{frio; baixa temperatura; sensação de frio | gelado; frio por natureza; sem entusiasmo; sem gentileza | desolado; pouco frequentado; quieto; sem agitação | negligenciado; indesejável; ignorado por todos | raro; estranho; incomum | feito em segredo; filmado de forma escondida; lançado secretamente}
  \definition{v.}{esfriar; resfriar | esfriar; congelar; tornar-se indiferente, apático | ignorar}
\end{EntryWithPhonetic}

\begin{EntryWithPhonetic}{冷静}{leng3jing4}{7,14}{⼎、⾭}[HSK 4]
  \definition{adj.}{calmo; descreve uma pessoa que consegue ficar atenta em uma situação importante ou de emergência e não toma decisões aleatórias por causa de seus sentimentos no momento | (lugar) tranquilo; quieto; deserto}
\end{EntryWithPhonetic}

\begin{EntryWithPhonetic}{冷门}{leng3men2}{7,3}{⼎、⾨}
  \definition{s.}{uma profissão, ofício ou ramo de aprendizagem que recebe pouca atenção | um vencedor inesperado; azarão}
\end{EntryWithPhonetic}

\begin{EntryWithPhonetic}{冷气}{leng3 qi4}{7,4}{⼎、⽓}[HSK 6]
  \definition[股,阵]{s.}{ar frio (ou fresco); correntes de ar frio | ar condicionado; ar resfriado por equipamento de refrigeração | ar condicionado; equipamentos de ar condicionado}
\end{EntryWithPhonetic}

\begin{EntryWithPhonetic}{冷水}{leng3 shui3}{7,4}{⼎、⽔}[HSK 6]
  \definition[杯,瓶]{s.}{água fria | água não fervida}
\end{EntryWithPhonetic}

\begin{EntryWithPhonetic}{哩哩啦啦}{li1 li1 la1 la1}{10,10,11,11}{⼝、⼝、⼝、⼝}
  \definition{adj.}{espalhado; disperso; disseminado; difuso; esporádico; aqui e ali}
\end{EntryWithPhonetic}

\begin{EntryWithPhonetic}{厘}{li2}{9}{⼚}
  \definition*{s.}{Sobrenome Li}
  \definition{clas.}{li, uma unidade tradicional de comprimento, igual a 0,001 chi (市尺), e equivalente a 0,333 milímetro ou 0,013 polegada | li, uma unidade tradicional de peso, igual a 0,0001 jin (市斤), e equivalente a 5 centigramas ou 0,771 grãos | li, uma unidade tradicional de área, igual a 0,01 mu (市亩), e equivalente a 0,667 metro quadrado ou 0,797 jarda quadrada | li, unidade monetária chinesa, igual a 0,1 fen ou 0,001 yuan | li, unidade de taxa de juros, igual a 0,1\% de juros mensais ou 1\% de juros anuais | quantidade muito pequena; fração; o mínimo}
  \definition{v.}{regular; retificar | administrar}
  \seealsoref{市尺}{shi4 chi3}
  \seealsoref{市斤}{shi4jin1}
  \seealsoref{市亩}{shi4mu3}
\end{EntryWithPhonetic}

\begin{EntryWithPhonetic}{厘米}{li2mi3}{9,6}{⼚、⽶}[HSK 4]
  \definition{clas.}{centímetro; unidade de comprimento, símbolo cm, 1 metro é igual a 100 centímetros}
\end{EntryWithPhonetic}

\begin{EntryWithPhonetic}{离}{li2}{10}{⼇}[HSK 2]
  \definition*{s.}{Um dos Oito Diagramas | Sobrenome Li}
  \definition{prep.}{(ser longe) de\dots até\dots}
  \definition{v.}{partir; separar-se; afastar-se; estar longe de | prescindir; dispensar; ser independente de | mudar de; desviar-se de | mudar de; desviar-se de; trair; ser incompatível}
\end{EntryWithPhonetic}

\begin{EntryWithPhonetic}{离不开}{li2 bu4 kai1}{10,4,4}{⼇、⼀、⼶}[HSK 4]
  \definition{v.}{não pode prescindir; ser inseparável de; não ser capaz de se separar ou deixar uma pessoa, coisa ou circunstância}
\end{EntryWithPhonetic}

\begin{EntryWithPhonetic}{离婚}{li2/hun1}{10,11}{⼇、⼥}[HSK 3]
  \definition{v.+compl.}{divórciar; romper um casamento; obter o divórcio}
\end{EntryWithPhonetic}

\begin{EntryWithPhonetic}{离开}{li2kai1}{10,4}{⼇、⼶}[HSK 2]
  \definition{v.}{deixar; partir; desviar-se; separar-se das pessoas, dos lugares e das coisas}
\end{EntryWithPhonetic}

\begin{EntryWithPhonetic}{梨}{li2}{11}{⽊}[HSK 5]
  \definition*{s.}{Sobrenome Li}
  \definition[个,只,斤,棵,种]{s.}{perira; árvore de pera | pera}
\end{EntryWithPhonetic}

\begin{EntryWithPhonetic}{黎}{li2}{15}{⿉}
  \definition*{s.}{Etnia Li, uma das minorias nacionais da província de Hainan | Sobrenome Li}
  \definition{adj.}{Literário: preto; escuro | Literário: numeroso}
  \definition{s.}{multidão; as massas; a população}
\end{EntryWithPhonetic}

\begin{EntryWithPhonetic}{礼}{li3}{5}{⽰}[HSK 5]
  \definition*{s.}{Sobrenome Li}
  \definition[份]{s.}{observâncias cerimoniais em geral; cerimônia; rito | cortesia; etiqueta; boas maneiras | presente; oferta}
\end{EntryWithPhonetic}

\begin{EntryWithPhonetic}{礼拜}{li3 bai4}{5,9}{⽰、⼿}[HSK 5]
  \definition[个]{s.}{dia da semana; usado em conjunto com 一, 二, 三, 四, 五, 六, 日(或天, indica um dia específico da semana | semana; referência à semana | domingo}
  \definition{v.}{prestar homenagem aos deuses que veneram; rezar; orar}
\end{EntryWithPhonetic}

\begin{EntryWithPhonetic}{礼节}{li3jie2}{5,5}{⽰、⾋}
  \definition{s.}{protocolo | cerimônia | etiqueta}
\end{EntryWithPhonetic}

\begin{EntryWithPhonetic}{礼貌}{li3mao4}{5,14}{⽰、⾘}[HSK 5]
  \definition{adj.}{educado; descreve uma pessoa que fala e age respeitando os outros, sem arrogância, de acordo com as exigências das relações sociais}
  \definition{s.}{cortesia; educação; boas maneiras}
\end{EntryWithPhonetic}

\begin{EntryWithPhonetic}{礼让}{li3rang4}{5,5}{⽰、⾔}
  \definition{s.}{cortesia}
  \definition{v.}{mostrar consideração por (outros) | ceder a (outro veículo, etc.)}
\end{EntryWithPhonetic}

\begin{EntryWithPhonetic}{礼堂}{li3 tang2}{5,11}{⽰、⼟}[HSK 6]
  \definition[个,座,处]{s.}{auditórios; salão de assembleias; um salão para reuniões ou cerimônias}
\end{EntryWithPhonetic}

\begin{EntryWithPhonetic}{礼物}{li3wu4}{5,8}{⽰、⽜}[HSK 2]
  \definition[份,件,个,分,些]{s.}{presente; lembrança; itens oferecidos como forma de respeito ou celebração, referindo-se de maneira geral a itens oferecidos como presente}
\end{EntryWithPhonetic}

\begin{EntryWithPhonetic}{李}{li3}{7}{⽊}
  \definition*{s.}{Sobrenome Li}
  \definition[棵]{s.}{ameixa | ameixeira}
\end{EntryWithPhonetic}

\begin{EntryWithPhonetic}{李四}{li3si4}{7,5}{⽊、⼞}
  \definition*{s.}{Li Si | Zé Ninguém | Nome para uma pessoa não especificada, 2 de 3}
  \seealsoref{王五}{wang2wu3}
  \seealsoref{张三}{zhang1san1}
\end{EntryWithPhonetic}

\begin{EntryWithPhonetic}{李子}{li3zi5}{7,3}{⽊、⼦}
  \definition[个]{s.}{ameixa}
\end{EntryWithPhonetic}

\begin{EntryWithPhonetic}{里}{li3}{7}{⾥}[HSK 1][Kangxi 166]
  \definition*{s.}{Sobrenome Li}
  \definition{clas.}{li, uma unidade chinesa de comprimento (= 1/2 quilômetro)}
  \definition{s.}{forro; revestimento; interior; parte de trás do tecido | interno; dentro; no interior | vizinhança; vizinhos | cidade natal; local de origem}
\end{EntryWithPhonetic}

\begin{EntryWithPhonetic}{里边}{li3 bian5}{7,5}{⾥、⾡}[HSK 1]
  \definition{s.}{em; dentro; no interior}
\end{EntryWithPhonetic}

\begin{EntryWithPhonetic}{里面}{li3 mian4}{7,9}{⾥、⾯}[HSK 3]
  \definition{s.}{dentro; interior}
\end{EntryWithPhonetic}

\begin{EntryWithPhonetic}{里斯本}{li3si1ben3}{7,12,5}{⾥、⽄、⽊}
  \definition*{s.}{Lisboa}
\end{EntryWithPhonetic}

\begin{EntryWithPhonetic}{里斯本大学}{li3si1ben3 da4xue2}{7,12,5,3,8}{⾥、⽄、⽊、⼤、⼦}
  \definition*{s.}{Universidade de Lisboa}
\end{EntryWithPhonetic}

\begin{EntryWithPhonetic}{里头}{li3 tou5}{7,5}{⾥、⼤}[HSK 2]
  \definition{s.}{dentro}
\end{EntryWithPhonetic}

\begin{EntryWithPhonetic}{哩}{li3}{10}{⼝}
  \definition{clas.}{milha (unidade de comprimento igual a 1.609,344 m)}
  \seeref{li5}
\end{EntryWithPhonetic}

\begin{EntryWithPhonetic}{理}{li3}{11}{⽟}[HSK 6]
  \definition*{s.}{Sobrenome Li}
  \definition{s.}{textura; grão (em madeira, pele, etc.) | ordem; sequência | razão; lógica; verdade | ciências naturais (especialmente física)}
  \definition{v.}{gerenciar; executar | colocar em ordem; arrumar | (geralmente no negativo) prestar atenção a; fazer um gesto ou falar com | tratar | colocar em ordem; limpar | tomar conhecimento de; prestar atenção a; expressar uma atitude; expressar uma opinião}
\end{EntryWithPhonetic}

\begin{EntryWithPhonetic}{理财}{li3 cai2}{11,7}{⽟、⾙}[HSK 6]
  \definition{v.}{administrar questões financeiras; conduzir transações financeiras; administrar propriedade; ser responsável pelo trabalho financeiro}
\end{EntryWithPhonetic}

\begin{EntryWithPhonetic}{理发}{li3/fa4}{11,5}{⽟、⼜}[HSK 3]
  \definition{v.+compl.}{cortar e aparar o cabelo; ter (dar) um corte de cabelo}
\end{EntryWithPhonetic}

\begin{EntryWithPhonetic}{理解}{li3jie3}{11,13}{⽟、⾓}[HSK 3]
  \definition{v.}{entender; compreender; compreender o significado por trás de algo através da reflexão e do aprendizado | entender com empatia; achar que os outros não conseguem fazer determinada coisa e demonstrar compaixão, perdão e não crítica}
\end{EntryWithPhonetic}

\begin{EntryWithPhonetic}{理论}{li3lun4}{11,6}{⽟、⾔}[HSK 3]
  \definition[套,个]{s.}{teoria; uma série de conclusões tiradas pelas pessoas sobre atividades naturais ou sociais}
  \definition{v.}{argumentar; raciocinar com alguém; discutir com outras pessoas sobre quem está certo ou errado}
\end{EntryWithPhonetic}

\begin{EntryWithPhonetic}{理想}{li3xiang3}{11,13}{⽟、⼼}[HSK 2]
  \definition{adj.}{ideal; perfeito | conforme o desejado; satisfatório}
  \definition{adv.}{idealmente}
  \definition[个,种]{s.}{ideal; sonho; aspiração}
\end{EntryWithPhonetic}

\begin{EntryWithPhonetic}{理由}{li3you2}{11,5}{⽟、⽥}[HSK 3]
  \definition[个,条,种,堆]{s.}{razão; justificativa; fundamento; a razão pela qual as coisas são feitas desta ou daquela maneira}
\end{EntryWithPhonetic}

\begin{EntryWithPhonetic}{理智}{li3zhi4}{11,12}{⽟、⽇}[HSK 6]
  \definition{adj.}{racional; sensato; cabeça fria; sóbrio; calmo}
  \definition{s.}{sentido; razão; intelecto; a capacidade de distinguir o certo do errado, analisar e julgar e controlar as emoções e o comportamento de acordo}
\end{EntryWithPhonetic}

\begin{EntryWithPhonetic}{力}{li4}{2}{⼒}[HSK 3][Kangxi 19]
  \definition*{s.}{Sobrenome Li}
  \definition{adj.}{forte; eficiente; capaz | forte; poderoso; referência geral à função das coisas}
  \definition{adv.}{energicamente; arduamente; vigorosamente; com todo o esforço; com toda a dedicação}
  \definition{s.}{força; energia; poder; (física) refere-se à ação de alterar o estado de movimento ou a forma de um objeto |poder; força; habilidade; capacidade; funções dos órgãos do corpo humano | força física; resistência física}
\end{EntryWithPhonetic}

\begin{EntryWithPhonetic}{力量}{li4liang5}{2,12}{⼒、⾥}[HSK 3]
  \definition[出]{s.}{força física; força espiritual | habilidade; capacidade | eficácia; efeito | força (pessoa ou grupo que tem muito poder ou influência); referência a uma pessoa ou grupo que pode desempenhar um papel importante}
\end{EntryWithPhonetic}

\begin{EntryWithPhonetic}{力气}{li4qi5}{2,4}{⼒、⽓}[HSK 4]
  \definition[把]{s.}{força física; eficiência muscular; força | esforço}
\end{EntryWithPhonetic}

\begin{EntryWithPhonetic}{历}{li4}{4}{⼚}
  \definition{adj.}{todas as anteriores (ocasiões, sessões, etc.)}
  \definition{adv.}{por toda parte; um por um}
  \definition{s.}{experiência; registro | almanaque; anuário; calendário}
  \definition{v.}{passar por; sofrer; experimentar | passar através; atravessar}
\end{EntryWithPhonetic}

\begin{EntryWithPhonetic}{历史}{li4shi3}{4,5}{⼚、⼝}[HSK 4]
  \definition[段]{s.}{história; registro do passado; processo de desenvolvimento da natureza e da sociedade humana; processo de desenvolvimento de uma coisa ou pessoa | história; eventos passados; experiência | história; refere-se ao tema da história}
\end{EntryWithPhonetic}

\begin{EntryWithPhonetic}{厉}{li4}{5}{⼚}
  \definition*{s.}{Sobrenome Li}
  \definition{adj.}{rigoroso; estrito | severo; sombrio; sério}
\end{EntryWithPhonetic}

\begin{EntryWithPhonetic}{厉害}{li4hai5}{5,10}{⼚、⼧}[HSK 5]
  \definition{adj.}{feroz; severo; descreve uma situação como sendo muito grave | severo; duro; descreve uma pessoa que é exigente com os outros, muito severa, muitas vezes deixando os outros um pouco assustados | incrível; talentoso; impressionante; usado para avaliar a capacidade de uma pessoa ou algo que ela fez que é notável | aterrorizante; assustador; descreve animais ferozes e assustadores}
\end{EntryWithPhonetic}

\begin{EntryWithPhonetic}{立}{li4}{5}{⽴}[HSK 5][Kangxi 117]
  \definition{adj.}{ereto; vertical; na vertical}
  \definition{adv.}{imediatamente; instantaneamente}
  \definition{v.}{ficar em pé, com os pés no chão ou apoiados em algum objeto; o objeto deve estar na vertical | erguer; colocar (ou levantar) algo; colocar em pé | encontrar; criar; elaborar; formular; estabelecer | configurar; fundar; estabelecer | viver; existir | ascender ao trono; antigamente, referia-se à ascensão ao trono de um monarca | nomear; designar; antigamente, significava estabelecer uma determinada posição ou status}
\end{EntryWithPhonetic}

\begin{EntryWithPhonetic}{立场}{li4chang3}{5,6}{⽴、⼟}[HSK 5]
  \definition[个]{s.}{posição; postura; a posição e a atitude adotadas ao reconhecer e lidar com os problemas | ponto de vista; refere-se especificamente à atitude de reconhecer e lidar com questões a partir dos interesses de uma determinada classe, ou seja, a posição de classe}
\end{EntryWithPhonetic}

\begin{EntryWithPhonetic}{立法}{li4fa3}{5,8}{⽴、⽔}
  \definition{s.}{legislação}
  \definition{v.}{promulgar leis | legislar}
\end{EntryWithPhonetic}

\begin{EntryWithPhonetic}{立即}{li4ji2}{5,7}{⽴、⼙}[HSK 4]
  \definition{adv.}{prontamente; imediatamente; de imediato}
\end{EntryWithPhonetic}

\begin{EntryWithPhonetic}{立刻}{li4ke4}{5,8}{⽴、⼑}[HSK 3]
  \definition{adv.}{imediatamente; de ​​uma vez; indica que algo acontecerá imediatamente após um determinado momento}
\end{EntryWithPhonetic}

\begin{EntryWithPhonetic}{利}{li4}{7}{⼑}[HSK 6]
  \definition*{s.}{Sobrenome Li}
  \definition{adj.}{afiado; cortante | favorável; conveniente; sem dificuldades; sem ou com poucas dificuldades}
  \definition{s.}{benefício; vantagem | lucro; ganhos; juros}
  \definition{v.}{beneficiar; tornar vantajoso}
\end{EntryWithPhonetic}

\begin{EntryWithPhonetic}{利润}{li4run4}{7,10}{⼑、⽔}[HSK 5]
  \definition[笔,份]{s.}{lucro; o dinheiro ganho com atividades comerciais e industriais}
\end{EntryWithPhonetic}

\begin{EntryWithPhonetic}{利息}{li4xi1}{7,10}{⼑、⼼}[HSK 4]
  \definition{s.}{acréscimo; juros; dinheiro recebido além do valor principal como resultado de depósitos ou empréstimos (diferenciado de 本金)}
  \seealsoref{本金}{ben3 jin1}
\end{EntryWithPhonetic}

\begin{EntryWithPhonetic}{利益}{li4yi4}{7,10}{⼑、⽫}[HSK 4]
  \definition[个,种]{s.}{ganho; lucro; juros; benefício}
\end{EntryWithPhonetic}

\begin{EntryWithPhonetic}{利用}{li4yong4}{7,5}{⼑、⽤}[HSK 3]
  \definition{v.}{usar; utilizar; fazer uso de; fazer com que algo ou alguém funcione bem| explorar; tirar vantagem de; usar meios para fazer com que pessoas ou coisas sirvam aos seus interesses}
\end{EntryWithPhonetic}

\begin{EntryWithPhonetic}{例}{li4}{8}{⼈}
  \definition{adj.}{regular; rotineiro}
  \definition{s.}{exemplo; instância | precedente | caso; instância | regras; estatutos; regulamentos}
  \definition{v.}{analogizar}
\end{EntryWithPhonetic}

\begin{EntryWithPhonetic}{例如}{li4ru2}{8,6}{⼈、⼥}[HSK 2]
  \definition{conj.}{por exemplo; tal como; como por exemplo; colocado antes do exemplo, indica que o exemplo vem a seguir}
\end{EntryWithPhonetic}

\begin{EntryWithPhonetic}{例外}{li4wai4}{8,5}{⼈、⼣}[HSK 5]
  \definition[个,种]{s.}{exceção; situações que não se enquadram nas regras gerais ou nas leis comuns}
  \definition{v.}{ser excepcional; ser uma exceção}
\end{EntryWithPhonetic}

\begin{EntryWithPhonetic}{例子}{li4 zi5}{8,3}{⼈、⼦}[HSK 2]
  \definition[个]{s.}{exemplo; algo usado para ajudar a explicar ou provar uma determinada situação ou afirmação}
\end{EntryWithPhonetic}

\begin{EntryWithPhonetic}{隶}{li4}{8}{⾪}[Kangxi 171]
  \definition*{s.}{Sobrenome Li}
  \definition{s.}{escravo; pessoa em servidão; pessoas escravizadas | Arcaico: corredor de cargo governamental na China feudal | um dos estilos antigos da caligrafia chinesa}
  \definition{v.}{estar subordinado a; estar afiliado a (ou com)}
\end{EntryWithPhonetic}

\begin{EntryWithPhonetic}{荔}{li4}{9}{⾋}
  \definition[颗]{s.}{lichia | (arcaico) uma espécie de grama semelhante à taboa}
\end{EntryWithPhonetic}

\begin{EntryWithPhonetic}{荔枝}{li4zhi1}{9,8}{⾋、⽊}
  \definition{s.}{lichia}
\end{EntryWithPhonetic}

\begin{EntryWithPhonetic}{鬲}{li4}{10}{⿀}[Kangxi 193]
  \definition{s.}{recipiente de cerâmica antigo com três pernas usado para cozinhar, com marcas de cordão na parte externa e pernas ocas}
  \seeref{ge2}
\end{EntryWithPhonetic}

\begin{EntryWithPhonetic}{詈}{li4}{12}{⾔}
  \definition{v.}{xingar; usar linguagem severa}
\end{EntryWithPhonetic}

\begin{EntryWithPhonetic}{詈骂}{li4ma4}{12,9}{⾔、⾺}
  \definition{v.}{xingar | abusar}
\end{EntryWithPhonetic}

\begin{EntryWithPhonetic}{哩}{li5}{10}{⼝}
  \definition{part.}{(dialeto) final modal semelhante a 呢 ou 啦, usado em um tom definido, mas um tanto exagerado}
  \seeref{li3}
  \seealsoref{啦}{la5}
  \seealsoref{呢}{ne5}
\end{EntryWithPhonetic}

\begin{EntryWithPhonetic}{俩}{lia3}{9}{⼈}[HSK 4]
  \definition{num.}{ambos; dois; contração de 两个 | alguns; vários; refere-se a um pequeno número}
\end{EntryWithPhonetic}

\begin{EntryWithPhonetic}{俩钱}{lia3qian2}{9,10}{⼈、⾦}
  \definition{s.}{uma pequena quantia de dinheiro}
\end{EntryWithPhonetic}

\begin{EntryWithPhonetic}{连}{lian2}{7}{⾡}[HSK 3]
  \definition*{s.}{Sobrenome Lian}
  \definition{adv.}{em sucessão; um após o outro; repetidamente}
  \definition{prep.}{incluindo; incluido | até mesmo}
  \definition[个]{s.}{companhia; unidades organizacionais das forças armadas}
  \definition{v.}{ligar; juntar; conectar | envolver-se (em problemas); implicar; incriminar | costurar; coser}
\end{EntryWithPhonetic}

\begin{EntryWithPhonetic}{连接}{lian2 jie1}{7,11}{⾡、⼿}[HSK 5]
  \definition[条]{s.}{conexão}
  \definition{v.}{ligar; unir; relacionar, conectar; anexar}
\end{EntryWithPhonetic}

\begin{EntryWithPhonetic}{连忙}{lian2mang2}{7,6}{⾡、⼼}[HSK 3]
  \definition{adv.}{imediatamente; de imediato; com pressa; apressadamente}
\end{EntryWithPhonetic}

\begin{EntryWithPhonetic}{连锁反应}{lian2suo3fan3ying4}{7,12,4,7}{⾡、⾦、⼜、⼴}
  \definition{s.}{reação em cadeia}
\end{EntryWithPhonetic}

\begin{EntryWithPhonetic}{连续}{lian2xu4}{7,11}{⾡、⽷}[HSK 3]
  \definition{adv.}{continuamente; sucessivamente; em uma fileira; um após o outro}
\end{EntryWithPhonetic}

\begin{EntryWithPhonetic}{连续剧}{lian2 xu4 ju4}{7,11,10}{⾡、⽷、⼑}[HSK 3]
  \definition[部,集]{s.}{série; novela; drama dividido em vários episódios, transmitido continuamente pela rádio ou televisão, com enredo contínuo}
\end{EntryWithPhonetic}

\begin{EntryWithPhonetic}{帘}{lian2}{8}{⼱}
  \definition[块,个]{s.}{bandeira em mastro sobre adega; bandeira como placa de loja | cortina; tela de bambu ou tecido; objetos para cobrir portas e janelas}
\end{EntryWithPhonetic}

\begin{EntryWithPhonetic}{莲}{lian2}{10}{⾋}
  \definition*{s.}{Sobrenome Lian}
  \definition[粒]{s.}{lótus}
\end{EntryWithPhonetic}

\begin{EntryWithPhonetic}{莲花}{lian2hua1}{10,7}{⾋、⾋}
  \definition{s.}{flor de lótus | lírio aquático}
\end{EntryWithPhonetic}

\begin{EntryWithPhonetic}{莲藕}{lian2'ou3}{10,18}{⾋、⾋}
  \definition{s.}{raiz de Lotus}
\end{EntryWithPhonetic}

\begin{EntryWithPhonetic}{联}{lian2}{12}{⽿}
  \definition{s.}{dísticos (antitéticos)}
  \definition{v.}{aliar-se a; unir-se; juntar-se a}
\end{EntryWithPhonetic}

\begin{EntryWithPhonetic}{联合}{lian2he2}{12,6}{⽿、⼝}[HSK 3]
  \definition{adj.}{conjunto; unido; federal; combinado}
  \definition{s.}{aliado; união; aliança; conectar-se ou unir-se para agir em conjunto}
\end{EntryWithPhonetic}

\begin{EntryWithPhonetic}{联合国}{lian2 he2 guo2}{12,6,8}{⽿、⼝、⼞}[HSK 3]
  \definition*{s.}{Nações Unidas; Organização internacional fundada em 1945, após o fim da Segunda Guerra Mundial, com sede em Nova Iorque, Estados Unidos ; as suas principais instituições são a Assembleia Geral, o Conselho de Segurança, o Conselho Econômico e Social e o Secretariado; de acordo com a Carta das Nações Unidas, os seus principais objetivos são manter a paz e a segurança internacionais, desenvolver relações amigáveis entre os países e promover a cooperação internacional nas áreas econômica e cultural}
\end{EntryWithPhonetic}

\begin{EntryWithPhonetic}{联合会}{lian2he2hui4}{12,6,6}{⽿、⼝、⼈}
  \definition{s.}{federação}
\end{EntryWithPhonetic}

\begin{EntryWithPhonetic}{联络}{lian2luo4}{12,9}{⽿、⽷}[HSK 5]
  \definition{v.}{entrar em contato; comunicar-se; entrar em contato com}
\end{EntryWithPhonetic}

\begin{EntryWithPhonetic}{联盟}{lian2meng2}{12,13}{⽿、⽫}[HSK 6]
  \definition{s.}{aliança; coalizão; liga; união}
\end{EntryWithPhonetic}

\begin{EntryWithPhonetic}{联赛}{lian2 sai4}{12,14}{⽿、⾙}[HSK 6]
  \definition{s.}{jogos da liga | liga (esportiva) | torneio da liga}
\end{EntryWithPhonetic}

\begin{EntryWithPhonetic}{联手}{lian2 shou3}{12,4}{⽿、⼿}[HSK 6]
  \definition{v.}{dar as mãos; cooperar | Literário: dar as mãos | agir em conjunto}
\end{EntryWithPhonetic}

\begin{EntryWithPhonetic}{联系}{lian2xi4}{12,7}{⽿、⽷}[HSK 3]
  \definition[个,种,层]{s.}{relacionamento; relacionamento entre duas coisas}
  \definition{v.}{entrar em contato; contatar; comunicar-se com alguém por telefone, e-mail ou carta | agendar; entrar em contato com; estabelecer algum tipo de relação com a outra parte | relacionar; combinar; integrar}
\end{EntryWithPhonetic}

\begin{EntryWithPhonetic}{联想}{lian2xiang3}{12,13}{⽿、⼼}[HSK 5]
  \definition*{s.}{Lenovo (empresa)}
  \definition{v.}{associar-se a; estabelecer uma conexão mental; lembrar-se de algo; lembrar-se de outras pessoas ou coisas relacionadas devido a alguém ou algo; evocar outros conceitos relacionados devido a um determinado conceito}
\end{EntryWithPhonetic}

\begin{EntryWithPhonetic}{脸}{lian3}{11}{⾁}[HSK 2]
  \definition[张,个]{s.}{rosto (de pessoas ou animais); a parte frontal da cabeça, da testa ao queixo | parte frontal de algo | cara; autoestima; aparência | rosto; expressões faciais}
\end{EntryWithPhonetic}

\begin{EntryWithPhonetic}{脸盆}{lian3 pen2}{11,9}{⾁、⽫}[HSK 5]
  \definition[个]{s.}{lavatório; bacia para lavar as mãos e o rosto}
\end{EntryWithPhonetic}

\begin{EntryWithPhonetic}{脸色}{lian3 se4}{11,6}{⾁、⾊}[HSK 5]
  \definition{s.}{aparência; tez; cor da pele | aparência; expressão facial | (indicando a condição física de alguém) aparência; cor}
\end{EntryWithPhonetic}

\begin{EntryWithPhonetic}{练}{lian4}{8}{⽷}[HSK 2]
  \definition*{s.}{Sobrenome Lian}
  \definition{adj.}{habilidoso; experiente; bem treinado}
  \definition{s.}{seda branca}
  \definition{v.}{tratar, amaciar e branquear a seda por meio de fervura; cozinhar seda crua ou tecidos de seda crua | treinar; praticar; exercitar}
\end{EntryWithPhonetic}

\begin{EntryWithPhonetic}{练习}{lian4xi2}{8,3}{⽷、⼄}[HSK 2]
  \definition[项,次]{s.}{exercício (em livros); tarefas ou exercícios organizados para consolidar os resultados da aprendizagem}
  \definition{v.}{praticar; exercitar; repitir várias vezes até ficar bem treinado}
\end{EntryWithPhonetic}

\begin{EntryWithPhonetic}{炼}{lian4}{9}{⽕}
  \definition{v.}{fundir; refinar | temperar (um metal) com fogo | pesar a palavra; procurar a frase certa; polir | trabalhar; tornar uma substância pura ou resistente por aquecimento, etc. | polir; fazer as palavras bonitas e concisas}
\end{EntryWithPhonetic}

\begin{EntryWithPhonetic}{恋}{lian4}{10}{⼼}
  \definition*{s.}{Sobrenome Lian}
  \definition{v.}{amor (romântico) | ansiar por; sentir-se apegado a | amar; apaixonar-se por | não querendo se separar de; sentir sua falta para sempre; não suportar ficar separado}
\end{EntryWithPhonetic}

\begin{EntryWithPhonetic}{恋爱}{lian4'ai4}{10,10}{⼼、⽖}[HSK 5]
  \definition[个,场,段]{s.}{namoro; afeto; amor romântico; ações que demonstram o amor mútuo}
  \definition{v.}{amar; estar apaixonado}
\end{EntryWithPhonetic}

\begin{EntryWithPhonetic}{良}{liang2}{7}{⾉}
  \definition*{s.}{Sobrenome Liang}
  \definition{adj.}{bom; ótimo; agradável}
  \definition{adv.}{muito; muito mesmo; de fato}
  \definition{s.}{boas pessoas; pessoas gentis; talentos excepcionais}
\end{EntryWithPhonetic}

\begin{EntryWithPhonetic}{良好}{liang2hao3}{7,6}{⾉、⼥}[HSK 4]
  \definition{adj.}{bom; ótimo; bem; satisfatório}
\end{EntryWithPhonetic}

\begin{EntryWithPhonetic}{良田}{liang2tian2}{7,5}{⾉、⽥}
  \definition{s.}{terra agrícola boa | terra fértil}
\end{EntryWithPhonetic}

\begin{EntryWithPhonetic}{良心}{liang2xin1}{7,4}{⾉、⼼}
  \definition{s.}{consciência}
\end{EntryWithPhonetic}

\begin{EntryWithPhonetic}{凉}{liang2}{10}{⼎}[HSK 2]
  \definition{adj.}{frio; gelado; ligeiramente fria (menos do que 冷) | sombrio; desolado; sem animação | desanimado; desapontado | usado para prevenir o calor e manter a temperatura amena; para proteção contra o calor}
  \definition{s.}{frio; refere-se a um ambiente fresco ou a uma brisa fresca}
  \seeref{liang4}
  \seealsoref{冷}{leng3}
\end{EntryWithPhonetic}

\begin{EntryWithPhonetic}{凉快}{liang2kuai5}{10,7}{⼎、⼼}[HSK 2]
  \definition{adj.}{fresco; refrescante}
  \definition{v.}{refrescar; refrescar-se; deixar o corpo fresco e revigorado}
\end{EntryWithPhonetic}

\begin{EntryWithPhonetic}{凉水}{liang2 shui3}{10,4}{⼎、⽔}[HSK 3]
  \definition{s.}{água fria; água não aquecida | água não fervida}
\end{EntryWithPhonetic}

\begin{EntryWithPhonetic}{凉鞋}{liang2 xie2}{10,15}{⼎、⾰}[HSK 6]
  \definition[双,只]{s.}{sandália; alpargata; alpercata; alparca ; sapatos de verão com cabedal ventilado}
\end{EntryWithPhonetic}

\begin{EntryWithPhonetic}{量}{liang2}{12}{⾥}[HSK 4]
  \definition{v.}{medir | estimar; dimensionar}
  \seeref{liang4}
\end{EntryWithPhonetic}

\begin{EntryWithPhonetic}{粮}{liang2}{13}{⽶}
  \definition[斤,粒]{s.}{grãos; alimentos; provisões | imposto sobre grãos | nutrição | imposto agrícola; grãos como imposto agrícola}
\end{EntryWithPhonetic}

\begin{EntryWithPhonetic}{粮食}{liang2shi5}{13,9}{⽶、⾷}[HSK 4]
  \definition[种,吨,袋,颗,粒]{s.}{alimentos; grãos; termo geral para os vários tipos de arroz, feijão, etc. que podem ser consumidos}
\end{EntryWithPhonetic}

\begin{EntryWithPhonetic}{两}{liang3}{7}{⼀}[HSK 1,2]
  \definition*{s.}{Sobrenome Liang}
  \definition{clas.}{liang, uma unidade de peso (=50 gramas)}
  \definition{num.}{dois (sempre usado antes de classificadores) | poucos; alguns; indica um número indeterminado}
  \definition{s.}{ambos (lados); qualquer (lado)}
\end{EntryWithPhonetic}

\begin{EntryWithPhonetic}{两岸}{liang3 an4}{7,8}{⼀、⼭}[HSK 5]
  \definition{s.}{ambos os lados; ambas as margens; ambas as costas; entre os dois lados do estreito; bilateral}
\end{EntryWithPhonetic}

\begin{EntryWithPhonetic}{两边}{liang3 bian1}{7,5}{⼀、⾡}[HSK 4]
  \definition{s.}{ambos os lados; ambas as direções; ambos os lugares | ambas as partes; ambos os lados}
\end{EntryWithPhonetic}

\begin{EntryWithPhonetic}{两侧}{liang3 ce4}{7,8}{⼀、⼈}[HSK 6]
  \definition{s.}{dois flancos; dois (ambos) lados; ambos}
\end{EntryWithPhonetic}

\begin{EntryWithPhonetic}{两码事}{liang3ma3shi4}{7,8,8}{⼀、⽯、⼅}
  \definition{expr.}{duas coisas completamente diferentes; dois assuntos diferentes}
\end{EntryWithPhonetic}

\begin{EntryWithPhonetic}{两手}{liang3 shou3}{7,4}{⼀、⼿}[HSK 6]
  \definition{s.}{ambas as mãos | ambos os aspectos; táticas duplas | Coloquial: habilidade; capacidade}
\end{EntryWithPhonetic}

\begin{EntryWithPhonetic}{亮}{liang4}{9}{⼇}[HSK 2]
  \definition*{s.}{Sobrenome Lian}
  \definition{adj.}{brilhante; claro | alto e claro; retumbante | esclarecido; aberto e claro}
  \definition{s.}{luz}
  \definition{v.}{iluminar; clarear; brilhar | elevar a voz; ressoar; tornar o som mais alto | revelar; mostrar; aparecer; exibir}
\end{EntryWithPhonetic}

\begin{EntryWithPhonetic}{凉}{liang4}{10}{⼎}
  \definition{v.}{deixar algo esfriar; deixar um objeto quente descansar por um tempo para que a temperatura diminua}
  \seeref{liang2}
\end{EntryWithPhonetic}

\begin{EntryWithPhonetic}{辆}{liang4}{11}{⾞}[HSK 2]
  \definition{clas.}{usado para automóveis, veículos, etc.}
\end{EntryWithPhonetic}

\begin{EntryWithPhonetic}{量}{liang4}{12}{⾥}
  \definition{s.}{instrumento de medida; antigamente, o termo se referia a objetos como baldes e litros, que medem o volume | capacidade (para tolerância ou ingestão de alimentos ou bebidas); refere-se ao limite do que pode ser acomodado | quantidade; valor; volume; número}
  \definition{v.}{estimar; medir; pesar}
  \seeref{liang2}
\end{EntryWithPhonetic}

\begin{EntryWithPhonetic}{疗}{liao2}{7}{⽧}
  \definition{v.}{tratar; curar | recuperar}
\end{EntryWithPhonetic}

\begin{EntryWithPhonetic}{疗养}{liao2 yang3}{7,9}{⽧、⼋}[HSK 4]
  \definition{v.}{recuperar; convalescer; tratar pessoas com doenças crônicas ou debilitantes em instituições médicas especializadas com foco na recuperação}
\end{EntryWithPhonetic}

\begin{EntryWithPhonetic}{聊}{liao2}{11}{⽿}[HSK 6]
  \definition*{s.}{Sobrenome Liao}
  \definition{adv.}{apenas; meramente; provisoriamente; por enquanto | um pouco; ligeiramente}
  \definition{v.}{tagarelar; conversar; bater papo | confiar (ou depender, recorrer) a}
\end{EntryWithPhonetic}

\begin{EntryWithPhonetic}{聊天}{liao2/tian1}{11,4}{⽿、⼤}
  \definition{v.+compl.}{papear | bater papo}
\end{EntryWithPhonetic}

\begin{EntryWithPhonetic}{聊天儿}{liao2/tian1r5}{11,4,2}{⽿、⼤、⼉}[HSK 6]
  \definition{v.+compl.}{conversar; fofocar; bater papo; duas ou mais pessoas conversando sem um tópico ou propósito específico}
\end{EntryWithPhonetic}

\begin{EntryWithPhonetic}{了}{liao3}{2}{⼅}
  \definition*{s.}{Sobrenome Liao}
  \definition{adv.}{inteiramente; um pouco; totalmente (mais usado em negativas)}
  \definition{v.}{terminar; concluir; encerrar; cumprir; eliminar; resolver | compreender; saber; perceber; saber claramente | expressar possibilidade ou impossibilidade; usado com 得 ou 不 após o verbo, indica possibilidade ou impossibilidade}
  \seeref{le5}
  \seealsoref{不}{bu4}
  \seealsoref{得}{de5}
\end{EntryWithPhonetic}

\begin{EntryWithPhonetic}{了不起}{liao3bu5qi3}{2,4,10}{⼅、⼀、⾛}[HSK 4]
  \definition{adj.}{incrível; fantástico; extraordinário | sério; grave}
\end{EntryWithPhonetic}

\begin{EntryWithPhonetic}{了解}{liao3jie3}{2,13}{⼅、⾓}[HSK 4]
  \definition{v.}{entender; compreender | investigar; indagar sobre}
\end{EntryWithPhonetic}

\begin{EntryWithPhonetic}{料}{liao4}{10}{⽃}[HSK 6]
  \definition{clas.}{usado na medicina tradicional chinesa para preparar pílulas | unidade usada para calcular um pedaço de madeira, é a seção transversal em ambas as extremidades, que é de 1 pé (quadrado) com 7 pés de comprimento}
  \definition{s.}{material; coisa | (grão) alimento; forragem | artigos de vidro; vidros coloridos opacos | (para pílulas de medicina chinesa) prescrição}
  \definition{v.}{supor; esperar; antecipar | gerenciar; cuidar de | prever}
\end{EntryWithPhonetic}

\begin{EntryWithPhonetic}{列}{lie4}{6}{⼑}[HSK 4]
  \definition*{s.}{Sobrenome Lie}
  \definition{clas.}{usado para coisas em linhas e colunas}
  \definition{pron.}{cada um e todos; cada; muito}
  \definition{s.}{linha; arquivo; classificação (oposto a 行) | classificação; escopo | ranque | tipo}
  \definition{v.}{organizar; alinhar; colocar em ordem | listar; inserir em uma lista; classificar | formar uma linha}
  \seealsoref{行}{hang2}
\end{EntryWithPhonetic}

\begin{EntryWithPhonetic}{列车}{lie4che1}{6,4}{⼑、⾞}[HSK 4]
  \definition[列,班,趟,辆,节]{s.}{trem; trem em uma composição contínua, puxado por uma locomotiva e equipado com uma tripulação e marcações prescritas; geralmente um trem de passageiros}
\end{EntryWithPhonetic}

\begin{EntryWithPhonetic}{列入}{lie4 ru4}{6,2}{⼑、⼊}[HSK 4]
  \definition{v.}{listar; entrar em uma lista; ser incluído em | incluir em uma lista; juntar-se; registrar-se}
\end{EntryWithPhonetic}

\begin{EntryWithPhonetic}{列为}{lie4 wei2}{6,4}{⼑、⼂}[HSK 4]
  \definition{v.}{ser classificado como; ser listado como}
\end{EntryWithPhonetic}

\begin{EntryWithPhonetic}{烈}{lie4}{10}{⽕}
  \definition*{s.}{Sobrenome Lie}
  \definition{adj.}{forte; violento; intenso; feroz | justo; severo | firme; convicto; rigoroso}
  \definition{s.}{pessoa que morreu por uma causa justa | conquistas; façanhas | mártir sacrificando-se por uma causa justa}
\end{EntryWithPhonetic}

\begin{EntryWithPhonetic}{烈士}{lie4shi4}{10,3}{⽕、⼠}
  \definition{s.}{mártir}
\end{EntryWithPhonetic}

\begin{EntryWithPhonetic}{猎}{lie4}{11}{⽝}
  \definition[个]{s.}{traje de caça}
  \definition{v.}{caçar | procurar; perseguir}
\end{EntryWithPhonetic}

\begin{EntryWithPhonetic}{猎物}{lie4wu4}{11,8}{⽝、⽜}
  \definition{s.}{presa (vítima de um predador)}
\end{EntryWithPhonetic}

\begin{EntryWithPhonetic}{裂}{lie4}{12}{⾐}[HSK 6]
  \definition{s.}{entalhe; incisão; entalhes grandes e profundos nas bordas das folhas ou corolas | brecha; lacuna; rachadura; refere-se à rachadura ou divisão que aparece na superfície ou no interior de um objeto}
  \definition{v.}{dividir; rachar; rasgar | (figurativo) quebrar; esmagar; arruinar}
\end{EntryWithPhonetic}

\begin{EntryWithPhonetic}{邻}{lin2}{7}{⾢}
  \definition{adj.}{vizinho; perto; adjacente; perto; próximo}
  \definition{s.}{vizinho | bairro; vizinhança}
\end{EntryWithPhonetic}

\begin{EntryWithPhonetic}{邻居}{lin2ju1}{7,8}{⾢、⼫}[HSK 5]
  \definition[个,位,名,家]{s.}{vizinho; pessoas ou famílias que moram muito perto}
\end{EntryWithPhonetic}

\begin{EntryWithPhonetic}{临}{lin2}{9}{⼁}
  \definition*{s.}{Sobrenome Lin}
  \definition{adv.}{pouco antes; prestes a; no ponto de; indica que uma ação está prestes a ocorrer}
  \definition{v.}{encarar; enfrentar; aproximar-se | chegar; estar presente | copiar (um modelo de caligrafia ou pintura); traçar sobre as palavras ou figuras | olhar de cima para baixo | ir de cima para baixo}
\end{EntryWithPhonetic}

\begin{EntryWithPhonetic}{临近}{lin2jin4}{9,7}{⼁、⾡}
  \definition{v.}{aproximar-se; estar perto de}
\end{EntryWithPhonetic}

\begin{EntryWithPhonetic}{临时}{lin2shi2}{9,7}{⼁、⽇}[HSK 4]
  \definition{adj.}{temporário; provisório; por um breve período}
  \definition{adv.}{no momento em que algo acontece (quando as coisas dão errado)}
\end{EntryWithPhonetic}

\begin{EntryWithPhonetic}{淋}{lin2}{11}{⽔}
  \definition{v.}{borrifar | pingar | derramar | encharcar}
  \seeref{lin4}
\end{EntryWithPhonetic}

\begin{EntryWithPhonetic}{淋}{lin4}{11}{⽔}
  \definition{s.}{gonorréia}
  \definition{v.}{filtrar | coar | drenar}
  \seeref{lin2}
\end{EntryWithPhonetic}

\begin{EntryWithPhonetic}{令}{ling2}{5}{⼈}
  \definition*{s.}{Antigo nome geográfico, na região atual de Linyi, província de Shanxi | Sobrenome Ling}
  \seeref{ling3}
  \seeref{ling4}
\end{EntryWithPhonetic}

\begin{EntryWithPhonetic}{灵}{ling2}{7}{⽕}
  \definition*{s.}{Sobrenome Ling}
  \definition{adj.}{rápido; inteligente; afiado | eficaz; efetivo | flexível; hábil}
  \definition{s.}{espírito; alma | inteligência; mente | fada; duende; elfo | restos mortais do falecido; esquife | carro funerário; caixão ou algo relacionado aos mortos}
\end{EntryWithPhonetic}

\begin{EntryWithPhonetic}{灵感}{ling2gan3}{7,13}{⽕、⼼}
  \definition{s.}{inspiração | explosão de criatividade em empreendimento científico ou artístico}
\end{EntryWithPhonetic}

\begin{EntryWithPhonetic}{灵魂}{ling2hun2}{7,13}{⽕、⿁}
  \definition{s.}{alma | espírito}
\end{EntryWithPhonetic}

\begin{EntryWithPhonetic}{灵活}{ling2huo2}{7,9}{⽕、⽔}[HSK 6]
  \definition[种,点,些]{adj.}{ágil; rápido; ligeiro; descreve a capacidade de fazer rapidamente mudanças apropriadas com base na situação ao lidar com as coisas | flexível; elástico; descreve reações rápidas, como movimentos e funções cerebrais}
\end{EntryWithPhonetic}

\begin{EntryWithPhonetic}{铃}{ling2}{10}{⾦}[HSK 5]
  \definition[串,个]{s.}{sino; instrumento musical feito de metal | objetos em forma de sino | cápsula; botão; broto}
\end{EntryWithPhonetic}

\begin{EntryWithPhonetic}{铃声}{ling2 sheng1}{10,7}{⾦、⼠}[HSK 5]
  \definition{s.}{o tilintar de sinos; o som de um sino tocando}
\end{EntryWithPhonetic}

\begin{EntryWithPhonetic}{陵}{ling2}{10}{⾩}
  \definition*{s.}{Sobrenome Ling}
  \definition{s.}{colina; monte | túmulo imperial; mausoléu}
  \definition{v.}{(literário) intimidar; violar}
\end{EntryWithPhonetic}

\begin{EntryWithPhonetic}{陵园}{ling2yuan2}{10,7}{⾩、⼞}
  \definition{s.}{cemitério}
\end{EntryWithPhonetic}

\begin{EntryWithPhonetic}{菱}{ling2}{11}{⾋}
  \definition{s.}{maruca; caltrop aquático; castanha d'água}
\end{EntryWithPhonetic}

\begin{EntryWithPhonetic}{菱角}{ling2jiao5}{11,7}{⾋、⾓}
  \definition{s.}{castanha d'água}
\end{EntryWithPhonetic}

\begin{EntryWithPhonetic}{零}{ling2}{13}{⾬}[HSK 1]
  \definition*{s.}{Sobrenome Ling}
  \definition{adj.}{ímpar; dispersos; fragmentados (em oposição a 整)}
  \definition{num.}{zero; 0; também grafado como 〇; representa um número menor que qualquer número positivo e maior que qualquer número negativo; representa a ausência de quantidade | zero grau no termômetro | usado para indicar qualidade, comprimento, tempo, idade, etc. Entre dois dígitos, indica que a quantidade da unidade mais alta é acompanhada pela quantidade da unidade mais baixa | sinal de zero (0); nulo; espaço em branco para indicar números em caracteres chineses maiúsculos}
  \definition{s.}{fragmento; fração; lote ímpar; um número fracionário que não é suficiente para uma determinada unidade; um ponto decimal diferente de um inteiro}
  \definition{v.}{(de chuva, lágrimas, etc.) cair | murchar e cair}
  \seealsoref{整}{zheng3}
\end{EntryWithPhonetic}

\begin{EntryWithPhonetic}{零散}{ling2san3}{13,12}{⾬、⽁}
  \definition{adj.}{espalhado; disperso}
\end{EntryWithPhonetic}

\begin{EntryWithPhonetic}{零食}{ling2shi2}{13,9}{⾬、⾷}[HSK 4]
  \definition[包,袋,盒,箱,堆]{s.}{lanches; refrescos; petiscos entre as refeições; alimentação esporádica, além das refeições normais}
\end{EntryWithPhonetic}

\begin{EntryWithPhonetic}{零下}{ling2 xia4}{13,3}{⾬、⼀}[HSK 2]
  \definition{s.}{abaixo de zero; negativo}
\end{EntryWithPhonetic}

\begin{EntryWithPhonetic}{令}{ling3}{5}{⼈}
  \definition{clas.}{resma (de papel); unidade de medida de papel: 500 folhas inteiras de papel original produzidas mecanicamente equivalem a 1 resma}
  \seeref{ling2}
  \seeref{ling4}
\end{EntryWithPhonetic}

\begin{EntryWithPhonetic}{岭}{ling3}{8}{⼭}
  \definition{s.}{cordilheira}
\end{EntryWithPhonetic}

\begin{EntryWithPhonetic}{领}{ling3}{11}{⾴}[HSK 3]
  \definition{clas.}{usado para roupas, mantos, esteiras, tapetes, telas, etc.}
  \definition{s.}{pescoço; gargalo | gola; colarinho; faixa de pescoço | esboço; ponto principal; essência}
  \definition{v.}{conduzir; guiar; orientar | possuir; ser o possuidor de; ter jurisdição sobre | obter; conseguir; receber (o que foi distribuído) | aceitar; tomar |entender; compreender (o significado)}
\end{EntryWithPhonetic}

\begin{EntryWithPhonetic}{领带}{ling3 dai4}{11,9}{⾴、⼱}[HSK 5]
  \definition[条]{s.}{colar; gargantilha; gravata}
\end{EntryWithPhonetic}

\begin{EntryWithPhonetic}{领导}{ling3dao3}{11,6}{⾴、⼨}[HSK 3]
  \definition[个,位,名,些]{s.}{líder; liderança; pessoa que ocupa uma posição de liderança}
  \definition{v.}{liderar; exercer liderança; (elogio) liderar, gerenciar outras pessoas;  trabalhar com outras pessoas ou avançar em direção a um objetivo}
\end{EntryWithPhonetic}

\begin{EntryWithPhonetic}{领情}{ling3/qing2}{11,11}{⾴、⼼}
  \definition{v.+compl.}{sentir-se grato a alguém}
\end{EntryWithPhonetic}

\begin{EntryWithPhonetic}{领取}{ling3 qu3}{11,8}{⾴、⼜}[HSK 6]
  \definition{v.}{sacar; receber; obter; receber o que lhe é enviado}
\end{EntryWithPhonetic}

\begin{EntryWithPhonetic}{领先}{ling3xian1}{11,6}{⾴、⼉}[HSK 3]
  \definition{v.}{liderar; assumir a liderança; estar na liderança; (velocidade, desempenho, etc.) superar pessoas ou coisas semelhantes, estar na vanguarda}
\end{EntryWithPhonetic}

\begin{EntryWithPhonetic}{领袖}{ling3xiu4}{11,10}{⾴、⾐}[HSK 6]
  \definition[个,位,名]{s.}{líder de estados, grupos políticos, organizações de massa, etc.}
\end{EntryWithPhonetic}

\begin{EntryWithPhonetic}{令}{ling4}{5}{⼈}[HSK 5]
  \definition{adj.}{bom; excelente | termos de cortesia usados para se referir aos familiares e parentes da outra pessoa}
  \definition{s.}{ordem; decreto; comando; ordem emitida pela autoridade superior | um título oficial; administradores de certos departamentos governamentais na antiguidade | temporada; estação; clima e fenologia de uma determinada estação | poema-canção; letra curta}
  \definition{v.}{ordenar; comandar | fazer com que alguém; fazer com que; permitir que}
  \seeref{ling2}
  \seeref{ling3}
\end{EntryWithPhonetic}

\begin{EntryWithPhonetic}{令人}{ling4ren2}{5,2}{⼈、⼈}
  \definition{v.}{causar alguém (a fazer alguma coisa) | fazer alguém ficar zangado, encantado, etc.}
\end{EntryWithPhonetic}

\begin{EntryWithPhonetic}{另}{ling4}{5}{⼝}[HSK 6]
  \definition*{s.}{Sobrenome Ling}
  \definition{adv.}{além disso; indica que está fora do escopo da declaração | no lugar de; em vez de}
  \definition{pron.}{(com substantivos) outro; diferente; refere-se a pessoas ou coisas fora do escopo do que é dito}
\end{EntryWithPhonetic}

\begin{EntryWithPhonetic}{另外}{ling4wai4}{5,5}{⼝、⼣}[HSK 3]
  \definition{adv.}{além disso; em adição; ademais; além do mais; além de que; além do que já foi dito}
  \definition{conj.}{além disso; usada entre duas ou mais frases, indica algo além do que foi mencionado anteriormente}
  \definition{pron.}{outro; além das pessoas ou coisas mencionadas anteriormente}
\end{EntryWithPhonetic}

\begin{EntryWithPhonetic}{另一方面}{ling4 yi4 fang1 mian4}{5,1,4,9}{⼝、⼀、⽅、⾯}[HSK 3]
  \definition{adv./conj.}{outro aspecto | por outro lado; por sua vez; em contrapartida}
\end{EntryWithPhonetic}

\begin{EntryWithPhonetic}{刘}{liu2}{6}{⼑}
  \definition*{s.}{Sobrenome Liu}
  \definition{s.}{Clássico: um tipo de machado de batalha}
  \definition{v.}{matar; massacrar}
\end{EntryWithPhonetic}

\begin{EntryWithPhonetic}{流}{liu2}{10}{⽔}[HSK 2]
  \definition*{s.}{Sobrenome Liu}
  \definition{adj.}{fluente; tão suave quanto a água corrente}
  \definition{clas.}{lúmen; abreviação de lumens, 流明}
  \definition[名,个]{s.}{corrente de água | corrente; algo que se assemelha a um fluxo de água | razão; taxa; classe; grau; ramificação; facção; hierarquia}
  \definition{v.}{(de líquido) fluir | vaguear; vagar; mover-se de um lugar para outro; movimentar-se sem direção fixa | espalhar; circular; transmitir; divulgar | degenerar; mudar para pior; tender (aspecto negativo) | banir; enviar para o exílio | correr (ou fluir) como líquido; refere-se à parte do rio após deixar sua nascente (em contraste com a 源)}
  \seealsoref{流明}{liu2ming2}
  \seealsoref{源}{yuan2}
\end{EntryWithPhonetic}

\begin{EntryWithPhonetic}{流传}{liu2chuan2}{10,6}{⽔、⼈}[HSK 4]
  \definition[间]{v.}{espalhar; circular; passar adiante}
\end{EntryWithPhonetic}

\begin{EntryWithPhonetic}{流动}{liu2 dong4}{10,6}{⽔、⼒}[HSK 5]
  \definition{v.}{(água, ar, etc.) fluir; correr; circular | ir de um lugar para outro; estar em movimento; ser móvel (oposto a 固定)}
  \seealsoref{固定}{gu4ding4}
\end{EntryWithPhonetic}

\begin{EntryWithPhonetic}{流感}{liu2 gan3}{10,13}{⽔、⼼}[HSK 6]
  \definition{s.}{gripe; influenza; abreviação de 流行性感冒}
  \seealsoref{流行性感冒}{liu2xing2 xing4 gan3mao4}
\end{EntryWithPhonetic}

\begin{EntryWithPhonetic}{流利}{liu2li4}{10,7}{⽔、⼑}[HSK 2]
  \definition{adj.}{fluente; suave; lúcido; falar e escrever com fluência e clareza | com fluência; sem dificuldades}
\end{EntryWithPhonetic}

\begin{EntryWithPhonetic}{流明}{liu2ming2}{10,8}{⽔、⽇}
  \definition{s.}{(empréstimo linguístico) lúmen (unidade de fluxo luminoso)}
\end{EntryWithPhonetic}

\begin{EntryWithPhonetic}{流水}{liu2shui3}{10,4}{⽔、⽔}
  \definition{s.}{água corrente | (negócio) rotatividade}
\end{EntryWithPhonetic}

\begin{EntryWithPhonetic}{流通}{liu2tong1}{10,10}{⽔、⾡}[HSK 5]
  \definition{v.}{(ar, dinheiro, mercadorias, etc.) fluir; circular}
\end{EntryWithPhonetic}

\begin{EntryWithPhonetic}{流星}{liu2xing1}{10,9}{⽔、⽇}
  \definition{s.}{meteoro | estrela cadente}
\end{EntryWithPhonetic}

\begin{EntryWithPhonetic}{流行}{liu2xing2}{10,6}{⽔、⾏}[HSK 2]
  \definition{adj.}{popular; na moda; muito popular}
  \definition{v.}{ser popular; prevalecer; espalhar-se amplamente; divulgar amplamente}
\end{EntryWithPhonetic}

\begin{EntryWithPhonetic}{流行性感冒}{liu2xing2 xing4 gan3mao4}{10,6,8,13,9}{⽔、⾏、⼼、⼼、⽇}
  \definition{s.}{gripe muito forte; influenza}
\end{EntryWithPhonetic}

\begin{EntryWithPhonetic}{留}{liu2}{10}{⽥}[HSK 2]
  \definition*{s.}{Sobrenome Liu}
  \definition{v.}{ficar; permanecer; parar em um determinado local ou posição; não se afastar | estudar no exterior (geralmente seguido pelo nome de um país com uma sílaba) | pedir a alguém para ficar; manter alguém onde está | concentrar-se em; concentrar a atenção em algo | manter; guardar; reservar; não joger fora | acumular; deixar crescer | aceitar; receber | transmitir (legado); deixar para trás}
\end{EntryWithPhonetic}

\begin{EntryWithPhonetic}{留神}{liu2/shen2}{10,9}{⽥、⽰}
  \definition{v.+compl.}{tomar cuidado | prestar atenção | manter os olhos abertos}
\end{EntryWithPhonetic}

\begin{EntryWithPhonetic}{留下}{liu2 xia4}{10,3}{⽥、⼀}[HSK 2]
  \definition{v.}{deixar; parar em algum lugar}
\end{EntryWithPhonetic}

\begin{EntryWithPhonetic}{留学}{liu2xue2}{10,8}{⽥、⼦}[HSK 3]
  \definition{v.}{estudar no exterior; permanecer no estrangeiro para estudar ou pesquisar}
\end{EntryWithPhonetic}

\begin{EntryWithPhonetic}{留学生}{liu2 xue2 sheng1}{10,8,5}{⽥、⼦、⽣}[HSK 2]
  \definition[个,位,名,批,帮]{s.}{estudante estrangeiro; estudante que retornou; estudante que estuda no exterior}
\end{EntryWithPhonetic}

\begin{EntryWithPhonetic}{留言}{liu2 yan2}{10,7}{⽥、⾔}[HSK 6]
  \definition[条]{s.}{mensagem; recado}
  \definition{v.}{deixar uma mensagem; deixar seus comentários}
\end{EntryWithPhonetic}

\begin{EntryWithPhonetic}{柳}{liu3}{9}{⽊}
  \definition*{s.}{Liu, a vigésima quarta das vinte e oito constelações, consistindo de oito estrelas em Hydra | Liu, uma das mansões lunares | Sobrenome Liu}
  \definition[棵]{s.}{salgueiro}
\end{EntryWithPhonetic}

\begin{EntryWithPhonetic}{柳橙汁}{liu3cheng2zhi1}{9,16,5}{⽊、⽊、⽔}
  \definition[瓶,杯,罐,盒]{s.}{suco de laranja}
  \seealsoref{橙汁}{cheng2zhi1}
  \seealsoref{橘子汁}{ju2zi5zhi1}
\end{EntryWithPhonetic}

\begin{EntryWithPhonetic}{六}{liu4}{4}{⼋}[HSK 1]
  \definition*{s.}{Sobrenome Liu}
  \definition{num.}{seis; 6}
  \definition{s.}{símbolo musical utilizado na partitura da música tradicional chinesa, representando o primeiro grau da escala musical, equivalente ao ``5'' da notação musical simplificada}
\end{EntryWithPhonetic}

\begin{EntryWithPhonetic}{陆}{liu4}{7}{⾩}
  \definition{num.}{seis, usado para o numeral 六 em cheques, etc. para evitar erros ou alterações}
  \seeref{lu4}
  \seealsoref{六}{liu4}
\end{EntryWithPhonetic}

\begin{EntryWithPhonetic}{遛}{liu4}{13}{⾡}
  \definition{v.}{passear | andar (um animal) | caminhar conduzindo um animal doméstico}
\end{EntryWithPhonetic}

\begin{EntryWithPhonetic}{遛狗}{liu4/gou3}{13,8}{⾡、⽝}
  \definition{v.+compl.}{passear com um cachorro}
\end{EntryWithPhonetic}

\begin{EntryWithPhonetic}{龙}{long2}{5}{⿓}[HSK 3][Kangxi 212]
  \definition*{s.}{Sobrenome Long}
  \definition[条]{s.}{dragão; animal mítico e sobrenatural, com chifres, escamas, garras e bigodes, capaz de voar e mergulhar na água, provocar nuvens e chuva | dinossauro; um enorme réptil extinto; referência a certos répteis gigantes da antiguidade | do imperador; dragão como símbolo do imperador; usado na era feudal como símbolo do imperador; também se refere a coisas pertencentes ao imperador | em forma de dragão; com um desenho de dragão; refere-se a certos objetos que formam uma sequência semelhante a um dragão ou decorados com motivos de dragões}
\end{EntryWithPhonetic}

\begin{EntryWithPhonetic}{龙山}{long2shan1}{5,3}{⿓、⼭}
  \definition*{s.}{Longshan}
\end{EntryWithPhonetic}

\begin{EntryWithPhonetic}{龙虾}{long2xia1}{5,9}{⿓、⾍}
  \definition{s.}{lagosta}
\end{EntryWithPhonetic}

\begin{EntryWithPhonetic}{笼}{long2}{11}{⽵}
  \definition{s.}{armação fechada de bambu, arame, etc. | jaula | gaiola}
  \seeref{long3}
\end{EntryWithPhonetic}

\begin{EntryWithPhonetic}{笼子}{long2zi5}{11,3}{⽵、⼦}
  \definition{s.}{jaula | cesta | gaiola | recipiente}
  \seeref{long3zi5}
\end{EntryWithPhonetic}

\begin{EntryWithPhonetic}{笼}{long3}{11}{⽵}
  \definition{v.}{envolver | cobrir}
  \seeref{long2}
\end{EntryWithPhonetic}

\begin{EntryWithPhonetic}{笼子}{long3zi5}{11,3}{⽵、⼦}
  \definition{s.}{caixa grande | porta-malas}
  \seeref{long2zi5}
\end{EntryWithPhonetic}

\begin{EntryWithPhonetic}{弄}{long4}{7}{⼶}
  \definition{s.}{rua estreita; beco; viela; travessa}
  \seeref{nong4}
\end{EntryWithPhonetic}

\begin{EntryWithPhonetic}{楼}{lou2}{13}{⽊}[HSK 1]
  \definition*{s.}{Sobrenome Lou}
  \definition{clas.}{andar, piso}
  \definition[层,座,栋]{s.}{um prédio com muitos andares | piso; andar | superestrutura; uma estrutura com um convés superior; um andar adicional construído sobre uma casa ou outro edifício | nome usado para certas lojas ou locais de entretenimento | arco ornamental; certas construções decorativas altas com passagens por baixo}
\end{EntryWithPhonetic}

\begin{EntryWithPhonetic}{楼道}{lou2 dao4}{13,12}{⽊、⾡}[HSK 6]
  \definition[个]{s.}{corredor; passagem | passagem (em edifício de vários andares)}
\end{EntryWithPhonetic}

\begin{EntryWithPhonetic}{楼房}{lou2 fang2}{13,8}{⽊、⼾}[HSK 6]
  \definition[栋,幢,座,套,层]{s.}{um edifício de dois ou mais andares}
\end{EntryWithPhonetic}

\begin{EntryWithPhonetic}{楼上}{lou2 shang4}{13,3}{⽊、⼀}[HSK 1]
  \definition{s.}{no andar de cima | autor anterior em um tópico do fórum; em plataformas como fóruns na internet, refere-se à pessoa que se manifesta antes de você.}
\end{EntryWithPhonetic}

\begin{EntryWithPhonetic}{楼梯}{lou2 ti1}{13,11}{⽊、⽊}[HSK 4]
  \definition[个,层,段,阶]{s.}{escada; escadaria; degraus no meio de dois andares para permitir que as pessoas subam ou desçam as escadas}
\end{EntryWithPhonetic}

\begin{EntryWithPhonetic}{楼下}{lou2 xia4}{13,3}{⽊、⼀}[HSK 1]
  \definition{s.}{no andar de baixo}
\end{EntryWithPhonetic}

\begin{EntryWithPhonetic}{漏}{lou4}{14}{⽔}[HSK 5]
  \definition{s.}{relógio de água; ampulheta | falha; ponto fraco | gonorreia; a medicina tradicional chinesa refere-se a certas doenças que causam secreção de pus, sangue e muco | unidade de tempo medida por um relógio de água durante a noite}
  \definition{v.}{(líquido, gás, etc.) pingar; vazar; escorrer; cair (de um buraco ou fenda) | vazar; deixar escapar; divulgar | perder; deixar de fora por engano | vazar; o objeto tem poros e pode vazar coisas | há uma fuga de ar}
\end{EntryWithPhonetic}

\begin{EntryWithPhonetic}{漏电}{lou4dian4}{14,5}{⽔、⽥}
  \definition{v.}{vazar eletricidade}
\end{EntryWithPhonetic}

\begin{EntryWithPhonetic}{漏洞}{lou4 dong4}{14,9}{⽔、⽔}[HSK 5]
  \definition[个,点]{s.}{vazamento; rachadura; lacunas ou buracos desnecessários que permitem que coisas vazem | falha; defeito; lacuna; (fala, ação, método, etc.) imperfeições}
\end{EntryWithPhonetic}

\begin{EntryWithPhonetic}{露}{lou4}{21}{⾬}[HSK 6]
  \definition{v.}{mostrar; apresentar (uma certa emoção ou olhar no rosto) | mostrar; aparentar; fazer algo visível; as pessoas podem ver}
  \seeref{lu4}
\end{EntryWithPhonetic}

\begin{EntryWithPhonetic}{卢}{lu2}{5}{⼘}
  \definition*{s.}{Luxemburgo, abreviação de 卢森堡 | Sobrenome Lu}
  \definition{s.}{Aarcaico: preta (cor)}
  \seealsoref{卢森堡}{lu2sen1bao3}
\end{EntryWithPhonetic}

\begin{EntryWithPhonetic}{卢森堡}{lu2sen1bao3}{5,12,12}{⼘、⽊、⼟}
  \definition*{s.}{Luxemburgo}
\end{EntryWithPhonetic}

\begin{EntryWithPhonetic}{卢旺达}{lu2wang4da2}{5,8,6}{⼘、⽇、⾡}
  \definition*{s.}{Ruanda}
\end{EntryWithPhonetic}

\begin{EntryWithPhonetic}{芦}{lu2}{7}{⾋}
  \definition*{s.}{Sobrenome Lu}
  \definition{s.}{junco}
\end{EntryWithPhonetic}

\begin{EntryWithPhonetic}{芦笋}{lu2sun3}{7,10}{⾋、⽵}
  \definition{s.}{aspargos}
\end{EntryWithPhonetic}

\begin{EntryWithPhonetic}{陆}{lu4}{7}{⾩}
  \definition*{s.}{Sobrenome Lu}
  \definition[个]{s.}{terra; terreno | rota terrestre; por terra}
  \seeref{liu4}
\end{EntryWithPhonetic}

\begin{EntryWithPhonetic}{陆地}{lu4di4}{7,6}{⾩、⼟}[HSK 4]
  \definition[块,片]{s.}{terra; terra seca (em oposição ao mar); superfície da Terra, excluindo os oceanos (e, às vezes, rios e lagos)}
\end{EntryWithPhonetic}

\begin{EntryWithPhonetic}{陆军}{lu4 jun1}{7,6}{⾩、⼍}[HSK 6]
  \definition{s.}{força terrestre; exército}
\end{EntryWithPhonetic}

\begin{EntryWithPhonetic}{陆路}{lu4lu4}{7,13}{⾩、⾜}
  \definition{s.}{rota terrestre}
\end{EntryWithPhonetic}

\begin{EntryWithPhonetic}{陆续}{lu4xu4}{7,11}{⾩、⽷}[HSK 4]
  \definition{adv.}{sucessivamente; um após o outro; intermitentemente}
\end{EntryWithPhonetic}

\begin{EntryWithPhonetic}{录}{lu4}{8}{⼹}[HSK 3]
  \definition{s.}{registro; cadastro; coleção; seleções}
  \definition{v.}{copiar; gravar; escrever; copiar; registrar | contratar; selecionar; empregar; adotar ou nomear | gravar em fita magnética}
\end{EntryWithPhonetic}

\begin{EntryWithPhonetic}{录取}{lu4qu3}{8,8}{⼹、⼜}[HSK 4]
  \definition{v.}{aceitar; admitir; recrutar; entrar; matricular (os aprovados no exame)}
\end{EntryWithPhonetic}

\begin{EntryWithPhonetic}{录像}{lu4/xiang4}{8,13}{⼹、⼈}[HSK 6]
  \definition[段,个,些,盘]{s.}{vídeo; gravação; fita de vídeo; imagens gravadas com celulares, câmeras, etc.}
  \definition{v.+compl.}{gravar bídeo; gravar em fita de vídeo | usar celulares, câmeras e outros dispositivos para salvar registros de vídeo}
\end{EntryWithPhonetic}

\begin{EntryWithPhonetic}{录像带}{lu4xiang4dai4}{8,13,9}{⼹、⼈、⼱}
  \definition[盘]{s.}{video-cassete}
\end{EntryWithPhonetic}

\begin{EntryWithPhonetic}{录像机}{lu4xiang4ji1}{8,13,6}{⼹、⼈、⽊}
  \definition[台]{s.}{gravador de vídeo | VCR}
\end{EntryWithPhonetic}

\begin{EntryWithPhonetic}{录音}{lu4/yin1}{8,9}{⼹、⾳}[HSK 3]
  \definition[段,个]{s.}{gravação de som; som gravado com equipamento especializado}
  \definition{v.+compl.}{gravar; converter o som em sinal elétrico e, em seguida, gravá-lo por meios mecânicos, ópticos ou eletromagnéticos}
\end{EntryWithPhonetic}

\begin{EntryWithPhonetic}{录音机}{lu4 yin1 ji1}{8,9,6}{⼹、⾳、⽊}[HSK 6]
  \definition[台]{s.}{gravador de som; máquina de gravação (de fita)}
\end{EntryWithPhonetic}

\begin{EntryWithPhonetic}{鹿}{lu4}{11}{⿅}[Kangxi 198]
  \definition*{s.}{Sobrenome Lu}
  \definition[只,头,群]{s.}{cervo | veado}
\end{EntryWithPhonetic}

\begin{EntryWithPhonetic}{路}{lu4}{13}{⾜}[HSK 1]
  \definition*{s.}{Sobrenome Lu}
  \definition{clas.}{tipo; classe | linha; coluna; usado para um grupo de pessoas ou uma equipe; para organizar em ordem}
  \definition[条]{s.}{estrada; caminho; via | viagem; jornada; distância | maneira; meios | sequência; linha; lógica | região; distrito | rota | classe; classificação; grau | linha; fileira}
\end{EntryWithPhonetic}

\begin{EntryWithPhonetic}{路边}{lu4 bian1}{13,5}{⾜、⾡}[HSK 2]
  \definition{s.}{calçada; beira da estrada; margem da rua}
\end{EntryWithPhonetic}

\begin{EntryWithPhonetic}{路过}{lu4 guo4}{13,6}{⾜、⾡}[HSK 6]
  \definition{v.}{passar por (algum lugar); atravessar}
\end{EntryWithPhonetic}

\begin{EntryWithPhonetic}{路口}{lu4 kou3}{13,3}{⾜、⼝}[HSK 1]
  \definition[个]{s.}{cruzamento; intersecção; onde as estradas se encontram}
\end{EntryWithPhonetic}

\begin{EntryWithPhonetic}{路上}{lu4 shang5}{13,3}{⾜、⼀}[HSK 1]
  \definition{s.}{na estrada | a caminho; na rota; em processo de mudança de um lugar para outro}
\end{EntryWithPhonetic}

\begin{EntryWithPhonetic}{路线}{lu4 xian4}{13,8}{⾜、⽷}[HSK 3]
  \definition[条]{s.}{rota; caminho; linha; a estrada percorrida de um lugar a outro | linha; diretriz (de política, ideologia, campo de trabalho); a via fundamental a seguir em termos ideológicos, políticos ou profissionais}
\end{EntryWithPhonetic}

\begin{EntryWithPhonetic}{露}{lu4}{21}{⾬}[HSK 6]
  \definition{adj.}{fora de uma casa, tenda, etc., sem cobertura}
  \definition{s.}{orvalho; gotas de água condensadas | xarope; suco de fruta; bebida destilada de flores, folhas ou frutos}
  \definition{v.}{revelar; expor; mostrar; trair}
  \seeref{lou4}
\end{EntryWithPhonetic}

\begin{EntryWithPhonetic}{露珠}{lu4zhu1}{21,10}{⾬、⽟}
  \definition{s.}{orvalho}
\end{EntryWithPhonetic}

\begin{EntryWithPhonetic}{乱}{luan4}{7}{⼄}[HSK 3]
  \definition{adj.}{em desordem; em confusão; em desarrumação; sem ordem nem organização | em um estado mental confuso | (de uma sociedade) turbulento; agitado | (de relações sexuais) impróprio; promíscuo}
  \definition{adv.}{aleatoriamente; arbitrariamente; indiscriminadamente; sem restrições; à vontade}
  \definition{s.}{motim; agitação; tumulto; revolta; guerra; calamidade}
  \definition{v.}{confundir; embaralhar; misturar; causar desordem}
\end{EntryWithPhonetic}

\begin{EntryWithPhonetic}{伦}{lun2}{6}{⼈}
  \definition*{s.}{Sobrenome Lun}
  \definition{s.}{relações humanas (especialmente como concebidas pela ética feudal) | lógica; ordem | par; correspondência; (mesma) classe | ética; relações humanas | sequência lógica; ordem | o mesmo tipo; semelhante; igual}
\end{EntryWithPhonetic}

\begin{EntryWithPhonetic}{伦敦}{lun2dun1}{6,12}{⼈、⽁}
  \definition*{s.}{Londres}
\end{EntryWithPhonetic}

\begin{EntryWithPhonetic}{论}{lun2}{6}{⾔}
  \definition*{s.}{Os Analectos de Confúcio, registro dos ditos e feitos de Confúcio e seus discípulos}
  \seeref{lun4}
\end{EntryWithPhonetic}

\begin{EntryWithPhonetic}{轮}{lun2}{8}{⾞}[HSK 4]
  \definition{clas.}{usado para sol vermelho, lua brilhante, etc. | usado para rodadas | doze anos de idade (os doze ramos terrestres são usados para lembrar o gênero humano e cada doze anos de idade é um ciclo)}
  \definition{s.}{roda | anel; disco; objeto semelhante a uma roda | navio a vapor; barco a vapor}
  \definition{v.}{revezar; substituir um ao outro em sequência (para fazer algo)}
\end{EntryWithPhonetic}

\begin{EntryWithPhonetic}{轮船}{lun2chuan2}{8,11}{⾞、⾈}[HSK 4]
  \definition[艘,班]{s.}{vapor; navio a vapor; barco a vapor}
\end{EntryWithPhonetic}

\begin{EntryWithPhonetic}{轮回}{lun2hui2}{8,6}{⾞、⼞}
  \definition[个]{s.}{reencarnação (Budismo) | ciclo}
  \definition{v.}{reencarnar}
\end{EntryWithPhonetic}

\begin{EntryWithPhonetic}{轮椅}{lun2 yi3}{8,12}{⾞、⽊}[HSK 4]
  \definition{s.}{cadeira de rodas; dispositivo de assento especialmente projetado com rodas para pessoas com dificuldade de locomoção, que pode ser acionado por um disco de roda ou manivela operados manualmente}
\end{EntryWithPhonetic}

\begin{EntryWithPhonetic}{轮子}{lun2 zi5}{8,3}{⾞、⼦}[HSK 4]
  \definition[个,只]{s.}{roda; peças circulares de veículos ou máquinas com capacidade de rotação}
\end{EntryWithPhonetic}

\begin{EntryWithPhonetic}{论}{lun4}{6}{⾔}
  \definition*{s.}{Sobrenome Lun}
  \definition{prep.}{por (uma certa unidade de medida) | de acordo com (um certo sistema ou princípio)}
  \definition{s.}{visão; opinião; declaração | (frequentemente em títulos) dissertação; ensaio; tratado | teoria; doutrina | ideia; palavras ou artigos que analisam e explicam coisas}
  \definition{v.}{discutir; falar sobre; discursar sobre; comentar | mencionar; considerar; falar de | decidir sobre; determinar | decidir sobre a natureza da culpa; punir | argumentar; analisar e explicar coisas | considerar; ponderar; medir; avaliar}
  \seeref{lun2}
\end{EntryWithPhonetic}

\begin{EntryWithPhonetic}{论文}{lun4wen2}{6,4}{⾔、⽂}[HSK 4]
  \definition[篇]{s.}{tese; redação; artigo; artigo que discute ou examina uma questão}
\end{EntryWithPhonetic}

\begin{EntryWithPhonetic}{罗}{luo2}{8}{⽹}
  \definition*{s.}{Sobrenome Luo}
  \definition{clas.}{uma grosa; uma bruta; doze dúzias; 144 unidades}
  \definition{s.}{uma rede para capturar pássaros | peneira; tela | uma espécie de gaze de seda}
  \definition{v.}{pegar pássaros com uma rede | espalhar; exibir; mostrar | coletar; reunir; recrutar | peneirar}
\end{EntryWithPhonetic}

\begin{EntryWithPhonetic}{逻}{luo2}{11}{⾡}
  \definition{s.}{patrulha | (literário) a beira de um riacho de montanha}
  \definition{v.}{patrulhar; fazer rondas}
\end{EntryWithPhonetic}

\begin{EntryWithPhonetic}{逻辑}{luo2ji5}{11,13}{⾡、⾞}[HSK 5]
  \definition[套,条,种]{s.}{lógica; lei objetiva; a objetividade das leis que regem o desenvolvimento das coisas | lógica; razão; regras para o pensamento | lógica como ciência do raciocínio, do pensamento; disciplina que estuda a lógica}
\end{EntryWithPhonetic}

\begin{EntryWithPhonetic}{螺}{luo2}{17}{⾍}
  \definition{s.}{concha em espiral | caracol | búzio}
\end{EntryWithPhonetic}

\begin{EntryWithPhonetic}{螺丝}{luo2si1}{17,5}{⾍、⼀}
  \definition{s.}{parafuso}
\end{EntryWithPhonetic}

\begin{EntryWithPhonetic}{骆}{luo4}{9}{⾺}
  \definition*{s.}{Sobrenome Luo}
  \definition[只]{s.}{Arcaico: um cavalo branco com crina preta, mencionado em antigos livros chineses}
\end{EntryWithPhonetic}

\begin{EntryWithPhonetic}{骆驼}{luo4tuo5}{9,8}{⾺、⾺}
  \definition[头,只,匹]{s.}{camelo | coloquial: cabeça-dura, idiota}
\end{EntryWithPhonetic}

\begin{EntryWithPhonetic}{落}{luo4}{12}{⾋}[HSK 4]
  \definition*{s.}{Sobrenome Luo}
  \definition{s.}{paradeiro; lugar para ficar; local de descanso | assentamento; local de reunião | parte curta; área pequena; refere-se a um pequeno lugar ou área}
  \definition{v.}{cair; cair de uma altura elevada | se abaixar; descer; ir para baixo | abaixar; deixar cair (ou descer); fazer descer | afundar; declinar; descer | ficar para trás; ficar para trás ou ficar de fora | permanecer; fazer uma parada; deixar para trás | cair sobre; repousar com | obter; ter; receber | anotar; escrever no papel | cair em; entrar em; ficar preso}
  \seeref{la4}
  \seeref{lao4}
\end{EntryWithPhonetic}

\begin{EntryWithPhonetic}{落后}{luo4hou4}{12,6}{⾋、⼝}[HSK 3]
  \definition{adj.}{atrasado; trabalho em atraso, nível de desenvolvimento ou grau de reconhecimento (em oposição a 进步)}
  \definition{v.}{ficar para trás; ficar atrasado; ficar para trás em relação aos outros durante o avanço ou o progresso do trabalho}
  \seealsoref{进步}{jin4bu4}
\end{EntryWithPhonetic}

\begin{EntryWithPhonetic}{落花生}{luo4 hua1 sheng1}{12,7,5}{⾋、⾋、⽣}
  \definition{s.}{amendoim | noz de macaco}
\end{EntryWithPhonetic}

\begin{EntryWithPhonetic}{落日}{luo4ri4}{12,4}{⾋、⽇}
  \definition{s.}{pôr do sol}
\end{EntryWithPhonetic}

\begin{EntryWithPhonetic}{落实}{luo4shi2}{12,8}{⾋、⼧}[HSK 5]
  \definition{adj.}{sentimento de tranquilidade; (humor) estável; seguro}
  \definition{v.}{implementar; ser praticável; tornar os planos, políticas, medidas, etc. específicos e compreensíveis, de modo a que possam ser realizados | implementar; colocar em prática; pôr em prática significa que os planos, políticas e medidas são específicos e claros, e podem ser realizados}
\end{EntryWithPhonetic}

\begin{EntryWithPhonetic}{落汤鸡}{luo4tang1ji1}{12,6,7}{⾋、⽔、⿃}
  \definition{s.}{uma pessoa que parece encharcada e acamada| sofrimento profundo}
\end{EntryWithPhonetic}

\begin{EntryWithPhonetic}{驴}{lv2}{7}{⾺}
  \definition[头,只]{s.}{burro; asno; jumento; jegue}
\end{EntryWithPhonetic}

\begin{EntryWithPhonetic}{旅}{lv3}{10}{⽅}
  \definition{adv.}{juntos; conjuntamente}
  \definition[个]{s.}{brigada; unidade organizacional militar, abaixo do nível de divisão e acima do nível de regimento ou batalhão | força; tropas; geralmente se refere aos militares | viajante; passageiro; turista | viagem; jornada | pessoas}
  \definition{v.}{viajar; ficar longe de casa; ir para longe; morar longe de casa}
\end{EntryWithPhonetic}

\begin{EntryWithPhonetic}{旅程}{lv3cheng2}{10,12}{⽅、⽲}
  \definition{s.}{jornada | viagem}
\end{EntryWithPhonetic}

\begin{EntryWithPhonetic}{旅店}{lv3 dian5}{10,8}{⽅、⼴}[HSK 6]
  \definition[家,个]{s.}{pousada; albergue; hotel}
\end{EntryWithPhonetic}

\begin{EntryWithPhonetic}{旅馆}{lv3 guan3}{10,11}{⽅、⾷}[HSK 3]
  \definition[家,个,所]{s.}{pousada; hotel; local comercial destinado ao alojamento de viajantes}
\end{EntryWithPhonetic}

\begin{EntryWithPhonetic}{旅客}{lv3 ke4}{10,9}{⽅、⼧}[HSK 2]
  \definition[名,位,个,些]{s.}{viajante; passageiro; as agências de transporte e turismo referem-se às pessoas que viajam}
\end{EntryWithPhonetic}

\begin{EntryWithPhonetic}{旅行}{lv3xing2}{10,6}{⽅、⾏}[HSK 2]
  \definition{v.}{viajar; passear; para tratar de assuntos ou passear, ir de um lugar para outro (geralmente se refere a distâncias longas)}
\end{EntryWithPhonetic}

\begin{EntryWithPhonetic}{旅行社}{lv3 xing2 she4}{10,6,7}{⽅、⾏、⽰}[HSK 3]
  \definition[家]{s.}{agência de viagens; agência especializada em serviços relacionados a viagens, que providencia hospedagem, transporte e outros serviços para viajantes}
\end{EntryWithPhonetic}

\begin{EntryWithPhonetic}{旅游}{lv3you2}{10,12}{⽅、⽔}[HSK 2]
  \definition{v.}{viajar para outros lugares para passear e fazer turismo}
\end{EntryWithPhonetic}

\begin{EntryWithPhonetic}{屡}{lv3}{12}{⼫}
  \definition{adv.}{uma e outra vez; repetidamente | frequentemente}
\end{EntryWithPhonetic}

\begin{EntryWithPhonetic}{屡次}{lv3ci4}{12,6}{⼫、⽋}
  \definition{adv.}{repetidamente | uma e outra vez | muitas vezes}
\end{EntryWithPhonetic}

\begin{EntryWithPhonetic}{律}{lv4}{9}{⼻}
  \definition*{s.}{Sobrenome Lü}
  \definition{s.}{lei; regra; estatuto; regulamento}
  \definition{v.}{restringir; disciplinar; manter sob controle}
\end{EntryWithPhonetic}

\begin{EntryWithPhonetic}{律师}{lv4shi1}{9,6}{⼻、⼱}[HSK 4]
  \definition[名,个,位]{s.}{advogado; procurador; profissionais encarregados pelas partes ou nomeados pelo tribunal para auxiliar as partes no litígio, para comparecer ao tribunal para defesa e para tratar de assuntos jurídicos relacionados, de acordo com a lei}
\end{EntryWithPhonetic}

\begin{EntryWithPhonetic}{率}{lv4}{11}{⽞}
  \definition{s.}{taxa; razão; proporção; a relação proporcional entre duas grandezas relacionadas}
  \seeref{shuai4}
\end{EntryWithPhonetic}

\begin{EntryWithPhonetic}{绿}{lv4}{11}{⽷}[HSK 2]
  \definition*{s.}{Sobrenome Lü}
  \definition{adj.}{verde}
  \definition{v.}{tornar-se verde; ficar verde}
\end{EntryWithPhonetic}

\begin{EntryWithPhonetic}{绿茶}{lv4 cha2}{11,9}{⽷、⾋}[HSK 3]
  \definition{s.}{chá verde; chá produzido apenas através dos processos de maturação, enrolamento (ou sem enrolamento) e secagem, sem passar por fermentação, com cor verde-claro}
\end{EntryWithPhonetic}

\begin{EntryWithPhonetic}{绿豆}{lv4dou4}{11,7}{⽷、⾖}
  \definition{s.}{vagens}
\end{EntryWithPhonetic}

\begin{EntryWithPhonetic}{绿豆芽}{lv4dou4 ya2}{11,7,7}{⽷、⾖、⾋}
  \definition{s.}{broto de feijão verde}
\end{EntryWithPhonetic}

\begin{EntryWithPhonetic}{绿化}{lv4 hua4}{11,4}{⽷、⼔}[HSK 6]
  \definition{v.}{tornar verde plantando árvores, flores, etc.; arborizar; reflorestar; plantar árvores, flores e plantas para embelezar o ambiente ou prevenir a erosão do solo}
\end{EntryWithPhonetic}

\begin{EntryWithPhonetic}{绿色}{lv4 se4}{11,6}{⽷、⾊}[HSK 2]
  \definition{adj.}{verde; ecológico; sem poluição; em conformidade com os requisitos ambientais}
  \definition{s.}{cor verde}
\end{EntryWithPhonetic}

\begin{EntryWithPhonetic}{略}{lve4}{11}{⽥}
  \definition{adv.}{ligeiramente | marginalmente | aproximadamente}
\end{EntryWithPhonetic}

\begin{EntryWithPhonetic}{略微}{lve4wei1}{11,13}{⽥、⼻}
  \definition{adv.}{ligeiramente | marginalmente | aproximadamente}
\end{EntryWithPhonetic}

%%%%% EOF %%%%%


 %%%
%%% M
%%%
\section*{M}\addcontentsline{toc}{section}{M}\addcontentsline{loh}{figure}{\#\#\#\#\#\#\#\# M}

%%%%%%%%%% 妈 %%%%%%%%%%
\subsection*{妈}\addcontentsline{loh}{figure}{妈 \dpy{ma1}}

\begin{EntryWithPhonetic}{妈}{ma1}{6}{⼥}[HSK 1]
  \definition[个,位]{s.}{mãe; mamãe | uma forma de tratamento para uma mulher casada uma geração mais velha | (antigo) uma forma de tratamento para uma empregada doméstica de meia-idade ou velha}
  \seealsoref{妈妈}{ma1 ma5}
\end{EntryWithPhonetic}

\begin{EntryWithPhonetic}{妈妈}{ma1 ma5}{6,6}{⼥,⼥}[HSK 1]
  \definition[个,位]{s.}{mamãe; mãe | uma forma de chamar uma mulher de meia-idade; títulos de respeito para mulheres mais velhas}
\end{EntryWithPhonetic}

%%%%%%%%%% 抹 %%%%%%%%%%
\subsection*{抹}\addcontentsline{loh}{figure}{抹 \dpy{ma1}}

\begin{EntryWithPhonetic}{抹}{ma1}{8}{⼿}
  \definition{v.}{esfregar; limpar | deslizar algo para fora; tirar}
  \seeref{mo3}
  \seeref{mo4}
\end{EntryWithPhonetic}

%%%%%%%%%% 蚂 %%%%%%%%%%
\subsection*{蚂}\addcontentsline{loh}{figure}{蚂 \dpy{ma1}}

\begin{EntryWithPhonetic}{蚂}{ma1}{9}{⾍}
  \definition{part.}{caracter formador de palavras}
  \definition[只]{s.}{libélula}
  \seeref{ma3}
  \seeref{ma4}
\end{EntryWithPhonetic}

%%%%%%%%%% 麻 %%%%%%%%%%
\subsection*{麻}\addcontentsline{loh}{figure}{麻 \dpy{ma1}}

\begin{EntryWithPhonetic}{麻}{ma1}{11}{⿇}[Kangxi 200]
  \definition{adj.}{sombrio; escuro; completamente escuro}
  \seeref{ma2}
\end{EntryWithPhonetic}

%%%%%%%%%% 吗 %%%%%%%%%%
\subsection*{吗}\addcontentsline{loh}{figure}{吗 \dpy{ma2}}

\begin{EntryWithPhonetic}{吗}{ma2}{6}{⼝}
  \definition{adv.}{(coloquial) que?}
  \seeref{ma3}
  \seeref{ma5}
\end{EntryWithPhonetic}

%%%%%%%%%% 麻 %%%%%%%%%%
\subsection*{麻}\addcontentsline{loh}{figure}{麻 \dpy{ma2}}

\begin{EntryWithPhonetic}{麻}{ma2}{11}{⿇}[HSK 7-9][Kangxi 200]
  \definition*{s.}{Sobrenome: Ma}
  \definition{adj.}{áspero; grosseiro | marcado; manchado | espinhas; manchas ásperas; cicatrizes deixadas após a varíola}
  \definition[棵,株]{s.}{nome geral para cânhamo, linho, etc. | fibra de cânhamo, linho, etc. para têxteis | sésamo; gergelim | marcas de varíola; um rosto com marcas de varíola}
  \definition{v.}{anestesiar | corromper (a mente de alguém); envenenar}
  \seeref{ma1}
\end{EntryWithPhonetic}

\begin{EntryWithPhonetic}{麻痹}{ma2bi4}{11,13}{⿇,⽧}[HSK 7-9]
  \definition{adj.}{insensível; descuidado; negligente; essa metáfora descreve um estado de entorpecimento mental e perda de vigilância}
  \definition{v.}{ficar paralisado; ficar insensível; perder total ou parcialmente as funções sensoriais e motoras em uma parte do corpo | estar insensível; baixar a guarda; entorpecer a mente}
\end{EntryWithPhonetic}

\begin{EntryWithPhonetic}{麻烦}{ma2fan5}{11,10}{⿇,⽕}[HSK 3]
  \definition{adj.}{incômodo; inconveniente; complicado; trabalhoso; burocrático | incômodo; inconveniente; (a situação) é confusa e complicada}
  \definition[个,些,点,堆]{s.}{problema; inconveniência; assuntos complicados e difíceis de resolver}
  \definition{v.}{incomodar; perturbar; incomodar alguém; irritar; aborrecer; causar incômodo ou sobrecarregar outras pessoas}
\end{EntryWithPhonetic}

\begin{EntryWithPhonetic}{麻将}{ma2jiang4}{11,9}{⿇,⼨}[HSK 7-9]
  \definition*[副]{s.}{Mahjong}
\end{EntryWithPhonetic}

\begin{EntryWithPhonetic}{麻辣}{ma2la4}{11,14}{⿇,⾟}[HSK 7-9]
  \definition{adj.}{picante; quente e dormente; descreve um sabor como dormente e picante, semelhante à pimenta-de-Sichuan ou à pimenta malagueta}
\end{EntryWithPhonetic}

\begin{EntryWithPhonetic}{麻辣豆腐}{ma2la4 dou4fu5}{11,14,7,14}{⿇,⾟,⾖,⾁}
  \definition{s.}{tofú guisado em molho picante (prato)}
\end{EntryWithPhonetic}

\begin{EntryWithPhonetic}{麻木}{ma2mu4}{11,4}{⿇,⽊}[HSK 7-9]
  \definition{adj.}{dormente; descreve uma sensação de dormência ou perda de sensibilidade em uma parte do corpo devido ao frio ou à inatividade prolongada | entorpecido; insensível; sem vida; apático; de raciocínio lento}
\end{EntryWithPhonetic}

\begin{EntryWithPhonetic}{麻醉}{ma2zui4}{11,15}{⿇,⾣}[HSK 7-9]
  \definition{v.}{narcotizar; anestesiar; utilizar drogas, acupuntura ou outros métodos para deixar temporariamente todo ou parte de um organismo inconsciente | corromper (a mente de alguém); desgastar (a força de vontade de alguém); essa metáfora descreve o uso de certos métodos para desgastar a vontade de uma pessoa, fazendo com que ela perca a capacidade de distinguir o certo do errado}
\end{EntryWithPhonetic}

%%%%%%%%%% 马 %%%%%%%%%%
\subsection*{马}\addcontentsline{loh}{figure}{马 \dpy{ma3}}

\begin{EntryWithPhonetic}{马}{ma3}{3}{⾺}[HSK 3][Kangxi 187]
  \definition*{s.}{Sobrenome: Ma}
  \definition{adj.}{grande; extenso; amplo}
  \definition[匹,头,只,群]{s.}{cavalo | a peça do cavalo no xadrez chinês}
\end{EntryWithPhonetic}

\begin{EntryWithPhonetic}{马车}{ma3 che1}{3,4}{⾺,⾞}[HSK 6]
  \definition[辆]{s.}{carruagem (puxada por cavalo); carroça; charrete}
\end{EntryWithPhonetic}

\begin{EntryWithPhonetic}{马耳他}{ma3'er3ta1}{3,6,5}{⾺,⽿,⼈}
  \definition*{s.}{Malta}
\end{EntryWithPhonetic}

\begin{EntryWithPhonetic}{马后炮}{ma3hou4pao4}{3,6,9}{⾺,⼝,⽕}[HSK 7-9]
  \definition{s.}{(termo do xadrez chinês) ação ou conselho tardio; esforço tardio; tarde demais | Figurativo: ação tardia | visão retrospectiva}
\end{EntryWithPhonetic}

\begin{EntryWithPhonetic}{马虎}{ma3hu5}{3,8}{⾺,⾌}[HSK 7-9]
  \definition{adj.}{descuidado; casual | superficial; apressado; descuidado}
  \definition{v.}{encarar de forma leviana; fazer um trabalho malfeito}
\end{EntryWithPhonetic}

\begin{EntryWithPhonetic}{马克思列宁主义}{ma3ke4si1 lie4ning2 zhu3yi4}{3,7,9,6,5,5,3}{⾺,⼗,⼼,⼑,⼧,⼂,⼂}
  \definition*{s.}{Marxismo-Leninismo}
\end{EntryWithPhonetic}

\begin{EntryWithPhonetic}{马力}{ma3li4}{3,2}{⾺,⼒}[HSK 7-9]
  \definition{clas.}{Física: cavalos de potência, cavalo-vapor (cv)}
\end{EntryWithPhonetic}

\begin{EntryWithPhonetic}{马列主义}{ma3 lie4 zhu3yi4}{3,6,5,3}{⾺,⼑,⼂,⼂}
  \definition*{s.}{Marxismo-Leninismo}
\end{EntryWithPhonetic}

\begin{EntryWithPhonetic}{马路}{ma3lu4}{3,13}{⾺,⾜}[HSK 1]
  \definition[条]{s.}{estrada; rua; avenida; estradas largas e planas para o tráfego de carros e cavalos nas cidades ou nos subúrbios}
\end{EntryWithPhonetic}

\begin{EntryWithPhonetic}{马马虎虎}{ma3ma3hu3hu3}{3,3,8,8}{⾺,⾺,⾌,⾌}
  \definition{adj.}{tolerável; aceitável; mais ou menos; razoável; não tão ruim | descuidado; casual; vago; descuidado e negligente na execução das tarefas}
\end{EntryWithPhonetic}

\begin{EntryWithPhonetic}{马上}{ma3shang4}{3,3}{⾺,⼀}[HSK 1]
  \definition{adv.}{imediatamente; de uma só vez; em um piscar de olhos | em breve; em um futuro próximo; em um curto espaço de tempo}
\end{EntryWithPhonetic}

\begin{EntryWithPhonetic}{马桶}{ma3tong3}{3,11}{⾺,⽊}[HSK 7-9]
  \definition[个,台]{s.}{vaso sanitário}[小心别把手机掉进马桶。===Tenha cuidado para não deixar seu celular cair no vaso sanitário.]
\end{EntryWithPhonetic}

\begin{EntryWithPhonetic}{马尾}{ma3wei3}{3,7}{⾺,⼫}
  \definition{s.}{(penteado) rabo de cavalo | cauda de cavalo}
\end{EntryWithPhonetic}

\begin{EntryWithPhonetic}{马戏}{ma3xi4}{3,6}{⾺,⼽}[HSK 7-9]
  \definition[场,次]{s.}{circo | espetáculo de circo (apresentação)}
\end{EntryWithPhonetic}

%%%%%%%%%% 吗 %%%%%%%%%%
\subsection*{吗}\addcontentsline{loh}{figure}{吗 \dpy{ma3}}

\begin{EntryWithPhonetic}{吗}{ma3}{6}{⼝}
  \definition{s.}{usada em 吗啡, morfina}
  \seeref{ma2}
  \seeref{ma5}
  \seealsoref{吗啡}{ma3fei1}
\end{EntryWithPhonetic}

\begin{EntryWithPhonetic}{吗啡}{ma3fei1}{6,11}{⼝,⼝}
  \definition{s.}{morfina (empréstimo linguístico)}
\end{EntryWithPhonetic}

%%%%%%%%%% 码 %%%%%%%%%%
\subsection*{码}\addcontentsline{loh}{figure}{码 \dpy{ma3}}

\begin{EntryWithPhonetic}{码}{ma3}{8}{⽯}[HSK 7-9]
  \definition{clas.}{refere-se a um assunto específico ou a uma categoria de assuntos; refere-se a uma coisa ou a uma classe de coisas | jarda; unidade de comprimento britânica e americana}
  \definition{s.}{um sinal ou objeto que indica número; código; símbolos ou ferramentas que indicam números, como códigos de barras ou \emph{QR code}}
  \definition{v.}{empilhar; acumular}
\end{EntryWithPhonetic}

\begin{EntryWithPhonetic}{码头}{ma3tou2}{8,5}{⽯,⼤}[HSK 5]
  \definition[个,座]{s.}{doca; cais; píer; molhe; edifícios à beira-mar ou à beira do rio destinados exclusivamente à atracação de embarcações, embarque e desembarque de passageiros e carga e descarga de mercadorias | cidade portuária; centro comercial e de transportes; refere-se a uma cidade comercial com transporte terrestre e marítimo bem desenvolvido.}
\end{EntryWithPhonetic}

%%%%%%%%%% 蚂 %%%%%%%%%%
\subsection*{蚂}\addcontentsline{loh}{figure}{蚂 \dpy{ma3}}

\begin{EntryWithPhonetic}{蚂}{ma3}{9}{⾍}
  \definition{part.}{caracter formador de palavras}
  \seeref{ma1}
  \seeref{ma4}
\end{EntryWithPhonetic}

\begin{EntryWithPhonetic}{蚂蚁}{ma3yi3}{9,9}{⾍,⾍}
  \definition{s.}{formiga}
\end{EntryWithPhonetic}

\begin{EntryWithPhonetic}{蚂}{ma4}{9}{⾍}
  \definition{part.}{caracter formador de palavras}
  \seeref{ma1}
  \seeref{ma3}
\end{EntryWithPhonetic}

%%%%%%%%%% 骂 %%%%%%%%%%
\subsection*{骂}\addcontentsline{loh}{figure}{骂 \dpy{ma4}}

\begin{EntryWithPhonetic}{骂}{ma4}{9}{⾺}[HSK 5]
  \definition{v.}{abusar; xingar; insultar; insultar alguém com palavras grosseiras ou maliciosas | repreender; censurar; condenar}
\end{EntryWithPhonetic}

\begin{EntryWithPhonetic}{骂街}{ma4jie1}{9,12}{⾺,⾏}
  \definition{v.}{gritar abusos na rua}
\end{EntryWithPhonetic}

\begin{EntryWithPhonetic}{骂名}{ma4ming2}{9,6}{⾺,⼝}
  \definition{s.}{infâmia}
\end{EntryWithPhonetic}

%%%%%%%%%% 吗 %%%%%%%%%%
\subsection*{吗}\addcontentsline{loh}{figure}{吗 \dpy{ma5}}

\begin{EntryWithPhonetic}{吗}{ma5}{6}{⼝}[HSK 1]
  \definition{part.}{usado no final de uma pergunta | como uma pausa em uma frase antes de introduzir o ponto principal | usado no final de uma pergunta retórica}
  \seeref{ma2}
  \seeref{ma3}
\end{EntryWithPhonetic}

%%%%%%%%%% 嘛 %%%%%%%%%%
\subsection*{嘛}\addcontentsline{loh}{figure}{嘛 \dpy{ma5}}

\begin{EntryWithPhonetic}{嘛}{ma5}{14}{⼝}[HSK 6]
  \definition{part.}{usado no final de uma declaração para expressar que é claro que é verdade que é óbvio | usado no final de uma frase imperativa para expressar expectativa ou dissuasão | usado em uma frase para indicar uma pausa e chamar a atenção da outra pessoa}
\end{EntryWithPhonetic}

%%%%%%%%%% 埋 %%%%%%%%%%
\subsection*{埋}\addcontentsline{loh}{figure}{埋 \dpy{mai2}}

\begin{EntryWithPhonetic}{埋}{mai2}{10}{⼟}[HSK 6]
  \definition{v.}{cobrir (com terra, neve, etc.); enterrar | esconder | enterrar (uma pessoa morta)}
  \seeref{man2}
\end{EntryWithPhonetic}

\begin{EntryWithPhonetic}{埋藏}{mai2cang2}{10,17}{⼟,⾋}[HSK 7-9]
  \definition{v.}{ocultar os próprios sentimentos e pensamentos; esconder as próprias emoções; não deixar que seus pensamentos e sentimentos transpareçam | enterrar algo no chão para que outros não consigam encontrar; esconder o objeto debaixo da terra para que ninguém o encontre | (minerais) estar oculto no solo ou nas montanhas; (minerais, etc.) enterrados no subsolo ou em montanhas}
\end{EntryWithPhonetic}

\begin{EntryWithPhonetic}{埋伏}{mai2fu2}{10,6}{⼟,⼈}[HSK 7-9]
  \definition{v.}{emboscar; armar uma emboscada; desdobrar tropas secretamente em áreas por onde se espera que o inimigo passe e aguarde o momento oportuno para atacar | esconder-se; ficar debaixo de uma cobertura; espreitar}
\end{EntryWithPhonetic}

\begin{EntryWithPhonetic}{埋没}{mai2mo4}{10,7}{⼟,⽔}[HSK 7-9]
  \definition{v.}{enterrar; cobrir (com terra, neve, etc.) | negligenciar; sufocar; suprimir; tornar invisível}
\end{EntryWithPhonetic}

%%%%%%%%%% 买 %%%%%%%%%%
\subsection*{买}\addcontentsline{loh}{figure}{买 \dpy{mai3}}

\begin{EntryWithPhonetic}{买}{mai3}{6}{⼄}[HSK 1]
  \definition*{s.}{Sobrenome: Mai}
  \definition{v.}{comprar; adquirir | comprar; subornar; usar dinheiro ou outros meios para angariar apoio| pedir; obter; trocar dinheiro por coisas}
\end{EntryWithPhonetic}

\begin{EntryWithPhonetic}{买不起}{mai3 bu5 qi3}{6,4,10}{⼄,⼀,⾛}[HSK 7-9]
  \definition{v.}{não ter condições de comprar | não ter condições de pagar}
\end{EntryWithPhonetic}

\begin{EntryWithPhonetic}{买东西}{mai3 dong1xi5}{6,5,6}{⼄,⼀,⾑}
  \definition{v.}{fazer compras; comprar bens ou serviços}
\end{EntryWithPhonetic}

\begin{EntryWithPhonetic}{买卖}{mai3 mai4}{6,8}{⼄,⼗}[HSK 5]
  \definition[笔,桩,宗,家]{s.}{negócio; compra e venda; transação | Privado: loja; armazém}
\end{EntryWithPhonetic}

%%%%%%%%%% 迈 %%%%%%%%%%
\subsection*{迈}\addcontentsline{loh}{figure}{迈 \dpy{mai4}}

\begin{EntryWithPhonetic}{迈}{mai4}{6}{⾡}[HSK 7-9]
  \definition{adj.}{velho; idoso}
  \definition{clas.}{milha}
  \definition{v.}{dar um passo; passar; avançar; levantar o pé e caminhar para a frente; dar um passo largo}
\end{EntryWithPhonetic}

\begin{EntryWithPhonetic}{迈进}{mai4jin4}{6,7}{⾡,⾡}[HSK 7-9]
  \definition{v.}{avançar; seguir em frente; prosseguir com passos largos}
\end{EntryWithPhonetic}

%%%%%%%%%% 麦 %%%%%%%%%%
\subsection*{麦}\addcontentsline{loh}{figure}{麦 \dpy{mai4}}

\begin{EntryWithPhonetic}{麦}{mai4}{7}{⿆}[Kangxi 199]
  \definition*{s.}{Sobrenome: Mai}
  \definition[袋,筐,车]{s.}{um termo geral para trigo, cevada, etc.}
\end{EntryWithPhonetic}

\begin{EntryWithPhonetic}{麦当劳}{mai4dang1lao2}{7,6,7}{⿆,⼹,⼒}
  \definition*{s.}{McDonald's, restaurante de \emph{fast-food}}
  \seealsoref{麦当劳叔叔}{mai4dang1lao2 shu1shu5}
\end{EntryWithPhonetic}

\begin{EntryWithPhonetic}{麦当劳叔叔}{mai4dang1lao2 shu1shu5}{7,6,7,8,8}{⿆,⼹,⼒,⼜,⼜}
  \definition*{s.}{Ronald McDonald}
  \seealsoref{麦当劳}{mai4dang1lao2}
\end{EntryWithPhonetic}

\begin{EntryWithPhonetic}{麦淇淋}{mai4qi2lin2}{7,11,11}{⿆,⽔,⽔}
  \definition{s.}{Empréstimo linguístico: margarina}
\end{EntryWithPhonetic}

%%%%%%%%%% 卖 %%%%%%%%%%
\subsection*{卖}\addcontentsline{loh}{figure}{卖 \dpy{mai4}}

\begin{EntryWithPhonetic}{卖}{mai4}{8}{⼗}[HSK 2]
  \definition*{s.}{Sobrenome: Mai}
  \definition{clas.}{um prato (nos tempos antigos); antigamente, os restaurantes chamavam cada prato vendido de 一卖 (uma porção)}
  \definition{v.}{vender (oposto de 买) | trair (o próprio país ou amigos); alcançar objetivos pessoais à custa dos interesses do país, da nação e dos outros | não poupar esforços; esforçar-se ao máximo; tentar fazer o máximo possível | mostrar-se intencionalmente; exibir-se | vender o próprio trabalho; trabalhar em troca de dinheiro}
  \seealsoref{买}{mai3}
\end{EntryWithPhonetic}

\begin{EntryWithPhonetic}{卖弄}{mai4nong5}{8,7}{⼗,⼶}[HSK 7-9]
  \definition{v.}{exibir-se; desfilar; exibir ou ostentar intencionalmente (as próprias habilidades)}
\end{EntryWithPhonetic}

%%%%%%%%%% 脉 %%%%%%%%%%
\subsection*{脉}\addcontentsline{loh}{figure}{脉 \dpy{mai4}}

\begin{EntryWithPhonetic}{脉}{mai4}{9}{⾁}
  \definition{s.}{artérias e veias | pulso | nervura (de uma folha, asa de inseto, etc.) | rede; sistema; malha | Coloquial: milhas}
  \seeref{mo4}
\end{EntryWithPhonetic}

\begin{EntryWithPhonetic}{脉搏}{mai4bo2}{9,13}{⾁,⼿}[HSK 7-9]
  \definition{s.}{pulso; o fenômeno das artérias pulsarem ritmicamente à medida que o sangue bombeado impacta as paredes arteriais durante a contração cardíaca | uma metáfora para o desenvolvimento, as mudanças ou as tendências da sociedade, da vida, etc.}
\end{EntryWithPhonetic}

\begin{EntryWithPhonetic}{脉络}{mai4luo4}{9,9}{⾁,⽷}[HSK 7-9]
  \definition{s.}{termo geral para artérias, veias e canais por onde circula a energia vital; a medicina tradicional chinesa se refere aos vasos sanguíneos e meridianos por todo o corpo | uma linha de raciocínio; uma sequência de ideias; metaforicamente, refere-se à ordem ou estrutura das coisas ou da escrita | as nervuras (de uma folha, etc.); veios nas folhas das plantas ou em outras estruturas}
\end{EntryWithPhonetic}

%%%%%%%%%% 嫚 %%%%%%%%%%
\subsection*{嫚}\addcontentsline{loh}{figure}{嫚 \dpy{man1}}

\begin{EntryWithPhonetic}{嫚}{man1}{14}{⼥}
  \definition{s.}{menina bem-comportada}
  \seealsoref{嫚子}{man1zi5}
\end{EntryWithPhonetic}

\begin{EntryWithPhonetic}{嫚子}{man1zi5}{14,3}{⼥,⼦}
  \definition{s.}{Dialeto: menina}
\end{EntryWithPhonetic}

%%%%%%%%%% 埋 %%%%%%%%%%
\subsection*{埋}\addcontentsline{loh}{figure}{埋 \dpy{man2}}

\begin{EntryWithPhonetic}{埋}{man2}{10}{⼟}
  \definition{part.}{caracter formador de palavras}
  \seeref{mai2}
\end{EntryWithPhonetic}

\begin{EntryWithPhonetic}{埋怨}{man2yuan4}{10,9}{⼟,⼼}[HSK 7-9]
  \definition{v.}{reclamar; culpar; resmungar; estar insatisfeiro com alguém ou algo que se acredita ser a causa do desconforto da situação}
\end{EntryWithPhonetic}

%%%%%%%%%% 蛮 %%%%%%%%%%
\subsection*{蛮}\addcontentsline{loh}{figure}{蛮 \dpy{man2}}

\begin{EntryWithPhonetic}{蛮}{man2}{12}{⾍}[HSK 7-9]
  \definition{adj.}{grosseiro; rude; feroz; irracional; cruel | imprudente; implacável}
  \definition{adv.}{muito; bastante; razoavelmente}
  \definition{s.}{um nome antigo para os grupos étnicos do sul}
\end{EntryWithPhonetic}

%%%%%%%%%% 谩 %%%%%%%%%%
\subsection*{谩}\addcontentsline{loh}{figure}{谩 \dpy{man2}}

\begin{EntryWithPhonetic}{谩}{man2}{13}{⾔}
  \definition{v.}{enganar; ludibriar; iludir}
  \seeref{man4}
\end{EntryWithPhonetic}

%%%%%%%%%% 蔓 %%%%%%%%%%
\subsection*{蔓}\addcontentsline{loh}{figure}{蔓 \dpy{man2}}

\begin{EntryWithPhonetic}{蔓}{man2}{14}{⾋}
  \definition{s.}{couve-chinesa | nabo}
  \seeref{man4}
  \seeref{wan4}
\end{EntryWithPhonetic}

%%%%%%%%%% 馒 %%%%%%%%%%
\subsection*{馒}\addcontentsline{loh}{figure}{馒 \dpy{man2}}

\begin{EntryWithPhonetic}{馒}{man2}{14}{⾷}
  \definition{s.}{pão cozido no vapor}
\end{EntryWithPhonetic}

\begin{EntryWithPhonetic}{馒头}{man2tou5}{14,5}{⾷,⼤}[HSK 6]
  \definition[个,锅,屉,筐]{s.}{pão cozido no vapor; um alimento cozido no vapor feito de farinha fermentada, geralmente redondo na parte superior e plano na parte inferior, sem recheio}
\end{EntryWithPhonetic}

%%%%%%%%%% 瞒 %%%%%%%%%%
\subsection*{瞒}\addcontentsline{loh}{figure}{瞒 \dpy{man2}}

\begin{EntryWithPhonetic}{瞒}{man2}{15}{⽬}[HSK 7-9]
  \definition*{s.}{Sobrenome: Man}
  \definition{v.}{ocultar a verdade de; esconder; esconder a verdade de alguém}
\end{EntryWithPhonetic}

%%%%%%%%%% 满 %%%%%%%%%%
\subsection*{满}\addcontentsline{loh}{figure}{满 \dpy{man3}}

\begin{EntryWithPhonetic}{满}{man3}{13}{⽔}[HSK 2]
  \definition*{s.}{Etnia Manchu | Sobrenome: Man}
  \definition{adj.}{cheio; repleto; lotado; totalmente cheio; atingindo o limite da capacidade | tudo; inteiro; completo | presunçoso; complacente; orgulhoso}
  \definition{adv.}{muito; um tanto; bastante | completamente; inteiramente; perfeitamente}
  \definition{v.}{encher | sentir-se satisfeito; sentir que já é o suficiente | expirar; atingir o limite; atingir um determinado prazo ou limite}
\end{EntryWithPhonetic}

\begin{EntryWithPhonetic}{满分}{man3fen1}{13,4}{⽔,⼑}
  \definition{s.}{pontuação completa}
\end{EntryWithPhonetic}

\begin{EntryWithPhonetic}{满怀}{man3huai2}{13,7}{⽔,⼼}[HSK 7-9]
  \definition{s.}{peito; refere-se a toda a área frontal do tórax}
  \definition{v.}{estar cheio de; estar impregnado de}
\end{EntryWithPhonetic}

\begin{EntryWithPhonetic}{满满}{man3man3}{13,13}{⽔,⽔}
  \definition{adj.}{completo | densamente empacotado}
\end{EntryWithPhonetic}

\begin{EntryWithPhonetic}{满意}{man3yi4}{13,13}{⽔,⼼}[HSK 2]
  \definition{adj.}{satisfeito; contente; gratificado}
  \definition{v.}{estar satisfeito; sentir-se contente; satisfazer os seus desejos; estar de acordo com os seus desejos}
\end{EntryWithPhonetic}

\begin{EntryWithPhonetic}{满足}{man3zu2}{13,7}{⽔,⾜}[HSK 3]
  \definition{v.}{estar satisfeito; contentar-se; sentir-se satisfeito | satisfazer}
\end{EntryWithPhonetic}

%%%%%%%%%% 谩 %%%%%%%%%%
\subsection*{谩}\addcontentsline{loh}{figure}{谩 \dpy{man4}}

\begin{EntryWithPhonetic}{谩}{man4}{13}{⾔}
  \definition{v.}{ser desrespeitoso | caluniar | desconsiderar}
  \seeref{man2}
\end{EntryWithPhonetic}

\begin{EntryWithPhonetic}{谩骂}{man4ma4}{13,9}{⾔,⾺}
  \definition{v.}{proferir injúrias; insultar; criticar; xingar | lançar abusos; xingar aleatoriamente}
  \variantof{嫚骂}
\end{EntryWithPhonetic}

%%%%%%%%%% 嫚 %%%%%%%%%%
\subsection*{嫚}\addcontentsline{loh}{figure}{嫚 \dpy{man4}}

\begin{EntryWithPhonetic}{嫚}{man4}{14}{⼥}
  \definition*{s.}{Sobrenome: Man}
  \definition{s.}{Dialeto: menina}
  \definition{v.}{Literário: desprezar; menosprezar; insultar; humilhar}
  \seeref{man1}
\end{EntryWithPhonetic}

\begin{EntryWithPhonetic}{嫚骂}{man4ma4}{14,9}{⼥,⾺}
  \definition{s.}{insultar; repreender; xingar}
\end{EntryWithPhonetic}

%%%%%%%%%% 慢 %%%%%%%%%%
\subsection*{慢}\addcontentsline{loh}{figure}{慢 \dpy{man4}}

\begin{EntryWithPhonetic}{慢}{man4}{14}{⼼}[HSK 1]
  \definition*{s.}{Sobrenome: Man}
  \definition{adj.}{lento; devagar; baixa velocidade; longa duração (em oposição a 快) | rude; arrogante; sem educação com as pessoas | frouxo; lento}
  \definition{adv.}{lentamente}
  \seealsoref{快}{kuai4}
\end{EntryWithPhonetic}

\begin{EntryWithPhonetic}{慢车}{man4 che1}{14,4}{⼼,⾞}[HSK 6]
  \definition{s.}{trem lento com muitas paradas (oposto a 快车) | ônibus ou trem local; parada do trem}
  \seealsoref{快车}{kuai4 che1}
\end{EntryWithPhonetic}

\begin{EntryWithPhonetic}{慢动作}{man4dong4zuo4}{14,6,7}{⼼,⼒,⼈}
  \definition{s.}{(cinema) câmera lenta}
\end{EntryWithPhonetic}

\begin{EntryWithPhonetic}{慢慢}{man4 man4}{14,14}{⼼,⼼}[HSK 3]
  \definition{adv.}{lentamente; vagarosamente; gradualmente | lentamente; vagarosamente; gradualmente; depois de um longo período de tempo}
\end{EntryWithPhonetic}

\begin{EntryWithPhonetic}{慢慢来}{man4man4 lai2}{14,14,7}{⼼,⼼,⽊}[HSK 7-9]
  \definition{v.}{ir com calma; não ter pressa; significa não ter impaciência ao fazer as coisas e prosseguir no seu próprio ritmo}
\end{EntryWithPhonetic}

\begin{EntryWithPhonetic}{慢性}{man4xing4}{14,8}{⼼,⼼}[HSK 7-9]
  \definition{adj.}{crônico; duradouro | lento (em fazer efeito)}
\end{EntryWithPhonetic}

%%%%%%%%%% 漫 %%%%%%%%%%
\subsection*{漫}\addcontentsline{loh}{figure}{漫 \dpy{man4}}

\begin{EntryWithPhonetic}{漫}{man4}{14}{⽔}[HSK 7-9]
  \definition{adj.}{livre; desimpedido; casual; sem restrições; arbitrário | em todo lugar; por toda parte | longo; extenso; distante}
  \definition{adv.}{não; não há necessidade de; expressa negação, equivalente a 不要}
  \definition{v.}{transbordar; inundar; alagar | estar em todo lugar; estar em todos os lugares}
  \seealsoref{不要}{bu2 yao4}
\end{EntryWithPhonetic}

\begin{EntryWithPhonetic}{漫长}{man4chang2}{14,4}{⽔,⾧}[HSK 5]
  \definition{adj.}{muito longo; interminável; (tempo, espaço) dura muito tempo}
\end{EntryWithPhonetic}

\begin{EntryWithPhonetic}{漫画}{man4hua4}{14,8}{⽔,⽥}[HSK 5]
  \definition[幅,本,张,套]{s.}{desenho animado; caricatura; \emph{cartoon}}
\end{EntryWithPhonetic}

\begin{EntryWithPhonetic}{漫骂}{man4ma4}{14,9}{⽔,⾺}
  \definition{v.}{usar linguagem ofensiva contra; insultar; difamar}
  \variantof{谩骂}
\end{EntryWithPhonetic}

\begin{EntryWithPhonetic}{漫游}{man4you2}{14,12}{⽔,⽔}[HSK 7-9]
  \definition{v.}{vagar; perambular; dar voltas; fazer uma viagem de lazer | vagar; navegar; isso se refere à capacidade de telefones celulares e outros dispositivos se conectarem a qualquer terminal em outra área de serviço por meio da rede, após entrarem em uma área de serviço não registrada | (peixes) mover-se livremente; nadar livremente na água}
\end{EntryWithPhonetic}

%%%%%%%%%% 蔓 %%%%%%%%%%
\subsection*{蔓}\addcontentsline{loh}{figure}{蔓 \dpy{man4}}

\begin{EntryWithPhonetic}{蔓}{man4}{14}{⾋}
  \definition{s.}{uma videira com gavinhas; caule fino que não consegue ficar em pé}
  \definition{v.}{rastejar; espalhar; estender}
  \seeref{man2}
  \seeref{wan4}
\end{EntryWithPhonetic}

\begin{EntryWithPhonetic}{蔓草}{man4cao3}{14,9}{⾋,⾋}
  \definition{s.}{videira | trepadeira}
\end{EntryWithPhonetic}

\begin{EntryWithPhonetic}{蔓延}{man4yan2}{14,6}{⾋,⼵}[HSK 7-9]
  \definition{v.}{espalhar; esticar; estender | infestar; espalhar; essa metáfora descreve coisas que se estendem e se expandem para fora como trepadeiras rastejantes}
\end{EntryWithPhonetic}

%%%%%%%%%% 忙 %%%%%%%%%%
\subsection*{忙}\addcontentsline{loh}{figure}{忙 \dpy{mang2}}

\begin{EntryWithPhonetic}{忙}{mang2}{6}{⼼}[HSK 1]
  \definition*{s.}{Sobrenome: Mang}
  \definition{adj.}{ocupado; movimentado; totalmente ocupado; muitas coisas para fazer, sem tempo livre (oposto de 闲) | imperativo; ansioso; urgente}
  \definition{v.}{apressar-se; agitar-se; fazer algo com urgência e constantemente | trabalhar; fazer}
  \seealsoref{闲}{xian2}
\end{EntryWithPhonetic}

\begin{EntryWithPhonetic}{忙得}{mang2de2}{6,11}{⼼,⼻}
  \definition{adj.}{muito ocupado}
\end{EntryWithPhonetic}

\begin{EntryWithPhonetic}{忙活}{mang2huo2}{6,9}{⼼,⽔}
  \definition{s.}{tarefa urgente}
  \definition{v.}{estar ocupado com o trabalho; estar envolvido com tarefas; estar com pressa para terminar as coisas}
  \seeref{mang2huo5}
\end{EntryWithPhonetic}

\begin{EntryWithPhonetic}{忙活}{mang2huo5}{6,9}{⼼,⽔}[HSK 7-9]
  \definition{adj.}{Coloquial: ocupado; movimentado}
  \definition{v.}{estar ocupado; movimentar-se freneticamente}
  \seeref{mang2huo2}
\end{EntryWithPhonetic}

\begin{EntryWithPhonetic}{忙碌}{mang2lu4}{6,13}{⼼,⽯}[HSK 7-9]
  \definition{adj.}{ocupado; atarefado; agitado; movimentado; ocupado com várias coisas, sem tempo livre}
\end{EntryWithPhonetic}

\begin{EntryWithPhonetic}{忙乱}{mang2luan4}{6,7}{⼼,⼄}[HSK 7-9]
  \definition{adj.}{apressado e desordenado; às pressas e em meio à confusão}
  \definition{v.}{estar com pressa e desorganizado; realizar uma tarefa de forma apressada e desordenada; agir de forma apressada e desordenada}
\end{EntryWithPhonetic}

%%%%%%%%%% 盲 %%%%%%%%%%
\subsection*{盲}\addcontentsline{loh}{figure}{盲 \dpy{mang2}}

\begin{EntryWithPhonetic}{盲}{mang2}{8}{⽬}
  \definition{adj.}{cego | incapaz de distinguir coisas | totalmente incompetente}
  \definition{adv.}{cegamente}
\end{EntryWithPhonetic}

\begin{EntryWithPhonetic}{盲目}{mang2mu4}{8,5}{⽬,⽬}[HSK 7-9]
  \definition{adj.}{cego; sem rumo; essa metáfora descreve a falta de compreensão clara; consideração incompleta ou descuidada; objetivos pouco claros}
\end{EntryWithPhonetic}

\begin{EntryWithPhonetic}{盲人}{mang2 ren2}{8,2}{⽬,⼈}[HSK 6]
  \definition[个,位,名]{s.}{cego; pessoa cega; pessoas com deficiência visual}
\end{EntryWithPhonetic}

%%%%%%%%%% 茫 %%%%%%%%%%
\subsection*{茫}\addcontentsline{loh}{figure}{茫 \dpy{mang2}}

\begin{EntryWithPhonetic}{茫}{mang2}{9}{⾋}
  \definition{adj.}{ilimitado e indistinto | ignorante; no escuro | disseminado e pouco claro; descreve a água ou outras coisas como ilimitadas e incertas}
\end{EntryWithPhonetic}

\begin{EntryWithPhonetic}{茫然}{mang2ran2}{9,12}{⾋,⽕}[HSK 7-9]
  \definition{adj.}{vazio; vago; ignorante; no escuro; descreve um estado de completa ignorância ou perplexidade | frustrado; decepcionado; descreve uma aparência atordoada ou distraída devido à decepção}
\end{EntryWithPhonetic}

%%%%%%%%%% 猫 %%%%%%%%%%
\subsection*{猫}\addcontentsline{loh}{figure}{猫 \dpy{mao1}}

\begin{EntryWithPhonetic}{猫}{mao1}{11}{⽝}[HSK 2]
  \definition*[只,种,群,窝,个]{s.}{gato |  Empréstimo linguístico: MODEM}
  \definition{v.}{esconder-se; entrar em esconderijo | inclinar-se para a frente; curvar-se}
  \seeref{mao2}
\end{EntryWithPhonetic}

\begin{EntryWithPhonetic}{猫熊}{mao1xiong2}{11,14}{⽝,⽕}
  \definition[把,只]{s.}{panda gigante}
  \seealsoref{熊猫}{xiong2mao1}
\end{EntryWithPhonetic}

%%%%%%%%%% 毛 %%%%%%%%%%
\subsection*{毛}\addcontentsline{loh}{figure}{毛 \dpy{mao2}}

\begin{EntryWithPhonetic}{毛}{mao2}{4}{⽑}[HSK 1,3][Kangxi 82]
  \definition*{s.}{Sobrenome: Mao}
  \definition{adj.}{bruto; semiacabado | grosseiro | pequeno | fino | descuidado; rude; precipitado | assustado; nervoso; em pânico | impetuoso | rústico; sem acabamento | impuro | (de moeda) que não vale mais seu valor nominal; depreciado}
  \definition{clas.}{mao, uma unidade fracionária de dinheiro na China; dez centavos; uma peça de dez centavos}
  \definition[根]{s.}{(de um animal, planta, etc.) cabelo; pena; penugem | (de humanos) cabelo; barba | planta; colheita | lã | mofo; bolor}
  \definition{v.}{depreciar; desvalorizar; refere-se à desvalorização da moeda | (de cavalos, gado, etc.) assustar-se; sentir medo}
\end{EntryWithPhonetic}

\begin{EntryWithPhonetic}{毛笔}{mao2 bi3}{4,10}{⽑,⽵}[HSK 5]
  \definition[支,枝,根,管]{s.}{pincel para escrever; pincel chinês; canetas feitas com pelos de coelho, carneiro, doninha, etc., são materiais tradicionais utilizados para escrever caracteres chineses e pintar pinturas tradicionais chinesas}
\end{EntryWithPhonetic}

\begin{EntryWithPhonetic}{毛病}{mao2bing4}{4,10}{⽑,⽧}[HSK 3]
  \definition[个,点,种,些]{s.}{doença ou deficiência; condição de saúde precária ou deficiência física | problema; fracasso; onde o produto está com defeito ou não funciona corretamente | mau hábito; deficiência; falhas no comportamento humano}
\end{EntryWithPhonetic}

\begin{EntryWithPhonetic}{毛巾}{mao2jin1}{4,3}{⽑,⼱}[HSK 4]
  \definition[条,块]{s.}{toalha; toalha de banho}
\end{EntryWithPhonetic}

\begin{EntryWithPhonetic}{毛衣}{mao2 yi1}{4,6}{⽑,⾐}[HSK 4]
  \definition[件,个]{s.}{suéter; blusa feita de lã}
\end{EntryWithPhonetic}

%%%%%%%%%% 矛 %%%%%%%%%%
\subsection*{矛}\addcontentsline{loh}{figure}{矛 \dpy{mao2}}

\begin{EntryWithPhonetic}{矛}{mao2}{5}{⽭}[Kangxi 110]
  \definition{s.}{Arcaico: lança; lanceta}
\end{EntryWithPhonetic}

\begin{EntryWithPhonetic}{矛盾}{mao2dun4}{5,9}{⽭,⽬}[HSK 5]
  \definition{adj.}{contraditório; descreve pessoas ou coisas que se opõem ou se repelem mutuamente}
  \definition[对,个,种]{s.}{problema; contradição; discrepância; inconsistência | disputas e conflitos; relacionamento de oposição entre as duas partes devido a diferenças de opinião ou abordagem}
  \definition{v.}{opor-se; entrar em conflito; contradizer; nesta situação, apenas uma das opções está correta ou é verdadeira; não é possível que ambas estejam corretas ao mesmo tempo}
\end{EntryWithPhonetic}

\begin{EntryWithPhonetic}{矛头}{mao2tou2}{5,5}{⽭,⼤}[HSK 7-9]
  \definition{s.}{ponta de lança; lança}
\end{EntryWithPhonetic}

%%%%%%%%%% 牦 %%%%%%%%%%
\subsection*{牦}\addcontentsline{loh}{figure}{牦 \dpy{mao2}}

\begin{EntryWithPhonetic}{牦}{mao2}{8}{⽜}
  \definition[头]{s.}{iaque; boi da Tartária}
\end{EntryWithPhonetic}

\begin{EntryWithPhonetic}{牦牛}{mao2niu2}{8,4}{⽜,⽜}
  \definition{s.}{iaque}
\end{EntryWithPhonetic}

%%%%%%%%%% 茅 %%%%%%%%%%
\subsection*{茅}\addcontentsline{loh}{figure}{茅 \dpy{mao2}}

\begin{EntryWithPhonetic}{茅}{mao2}{8}{⾋}
  \definition*{s.}{Sobrenome: Mao}
  \definition[座]{s.}{capim-cogon | planta semelhante ao capim-cogon (como palha)}
\end{EntryWithPhonetic}

\begin{EntryWithPhonetic}{茅厕}{mao2ce4}{8,8}{⾋,⼚}
  \definition{s.}{(dialeto) latrina}
\end{EntryWithPhonetic}

\begin{EntryWithPhonetic}{茅台}{mao2tai2}{8,5}{⾋,⼝}
  \definition*{s.}{Moutai (bebida alcoólica)}
  \seealsoref{茅台酒}{mao2tai2 jiu3}
\end{EntryWithPhonetic}

\begin{EntryWithPhonetic}{茅台酒}{mao2tai2 jiu3}{8,5,10}{⾋,⼝,⾣}[HSK 7-9]
  \definition*[瓶,斤,箱,口,杯]{s.}{Maotai; Moutai (espírito forte) | Maotai (um licor chinês); Mao Tai}
\end{EntryWithPhonetic}

%%%%%%%%%% 猫 %%%%%%%%%%
\subsection*{猫}\addcontentsline{loh}{figure}{猫 \dpy{mao2}}

\begin{EntryWithPhonetic}{猫}{mao2}{11}{⽝}
  \definition{v.}{utilizado em 猫腰 \dpy{mao2yao1}}
  \seeref{mao1}
  \seealsoref{猫腰}{mao2yao1}
\end{EntryWithPhonetic}

\begin{EntryWithPhonetic}{猫腰}{mao2yao1}{11,13}{⽝,⾁}
  \definition{v.}{curvar-se}
\end{EntryWithPhonetic}

%%%%%%%%%% 茂 %%%%%%%%%%
\subsection*{茂}\addcontentsline{loh}{figure}{茂 \dpy{mao4}}

\begin{EntryWithPhonetic}{茂}{mao4}{8}{⾋}
  \definition*{s.}{Sobrenome: Mao}
  \definition{adj.}{luxuriante; exuberante; profuso | rico e esplêndido; rico e requintado}
  \definition[种]{s.}{ciclopentadieno; composto orgânico, fórmula molecular $C_5H_6$, líquido incolor, usado na fabricação de pesticidas, plásticos, etc.}
\end{EntryWithPhonetic}

\begin{EntryWithPhonetic}{茂密}{mao4mi4}{8,11}{⾋,⼧}[HSK 7-9]
  \definition{adj.}{(grama ou árvores) denso; espesso; exuberante e denso}
\end{EntryWithPhonetic}

\begin{EntryWithPhonetic}{茂盛}{mao4sheng4}{8,11}{⾋,⽫}[HSK 7-9]
  \definition{adj.}{exuberante; viçoso; abundante; florescente | exuberante; florescente; próspero}
\end{EntryWithPhonetic}

%%%%%%%%%% 冒 %%%%%%%%%%
\subsection*{冒}\addcontentsline{loh}{figure}{冒 \dpy{mao4}}

\begin{EntryWithPhonetic}{冒}{mao4}{9}{⽇}[HSK 5]
  \definition*{s.}{Sobrenome: Mao}
  \definition{adv.}{com ousadia; precipitadamente | fingidamente; falsamente; fraudulentamente}
  \definition{v.}{emitir; liberar; enviar (para cima) | arriscar; ser corajoso}
\end{EntryWithPhonetic}

\begin{EntryWithPhonetic}{冒充}{mao4chong1}{9,6}{⽇,⼉}[HSK 7-9]
  \definition{v.}{fingir ser; fazer passar alguém/algo por; confundir o falso com o verdadeiro; confundir o mau com o bom}
\end{EntryWithPhonetic}

\begin{EntryWithPhonetic}{冒犯}{mao4fan4}{9,5}{⽇,⽝}[HSK 7-9]
  \definition{v.}{ofender; afrontar}
\end{EntryWithPhonetic}

\begin{EntryWithPhonetic}{冒昧}{mao4mei4}{9,9}{⽇,⽇}[HSK 7-9]
  \definition{v.}{ter a ousadia de fazer algo; tomar a liberdade de; descreve as palavras e ações de alguém como frívolas, desconsiderando seu status, posição ou a ocasião (frequentemente usado como uma expressão de humildade)}
\end{EntryWithPhonetic}

\begin{EntryWithPhonetic}{冒险}{mao4/xian3}{9,9}{⽇,⾩}[HSK 7-9]
  \definition{v.+compl.}{arriscar; aventurar-se; correr um risco; assumir um risco}
\end{EntryWithPhonetic}

%%%%%%%%%% 贸 %%%%%%%%%%
\subsection*{贸}\addcontentsline{loh}{figure}{贸 \dpy{mao4}}

\begin{EntryWithPhonetic}{贸}{mao4}{9}{⾙}
  \definition*{s.}{Sobrenome: Mao}
  \definition{s.}{comércio; negociação}
\end{EntryWithPhonetic}

\begin{EntryWithPhonetic}{贸易}{mao4yi4}{9,8}{⾙,⽇}[HSK 5]
  \definition[笔,宗,项,个]{s.}{comércio; troca; negócios; refere-se a atividades comerciais, como a troca de mercadorias}
  \definition{v.}{fazer uma transação comercial}
\end{EntryWithPhonetic}

%%%%%%%%%% 帽 %%%%%%%%%%
\subsection*{帽}\addcontentsline{loh}{figure}{帽 \dpy{mao4}}

\begin{EntryWithPhonetic}{帽}{mao4}{12}{⼱}
  \definition[个,顶]{s.}{chapéu; boné | capa; uma coisa que cobre um objeto e tem a função ou formato de um chapéu | elmo; capacete}
\end{EntryWithPhonetic}

\begin{EntryWithPhonetic}{帽子}{mao4zi5}{12,3}{⼱,⼦}[HSK 4]
  \definition[顶,个,种]{s.}{boné; chapéu; capacete | etiqueta; rótulo; marca}
\end{EntryWithPhonetic}

%%%%%%%%%% 没 %%%%%%%%%%
\subsection*{没}\addcontentsline{loh}{figure}{没 \dpy{mei2}}

\begin{EntryWithPhonetic}{没}{mei2}{7}{⽔}[HSK 1]
  \definition{adv.}{não; nunca; negar que uma ação ou situação tenha ocorrido, com o significado de 不曾}
  \definition{pref.}{não (prefixo negativo para verbos, traduzido para outras línguas com verbos no pretérito)}
  \definition{v.}{não possuir; não ter | não existe; não há | ninguém; usado antes de 谁, 什么, 哪个, significa 全都不 | não ser tão bom quanto; ser inferior a; não chega a; não é tão bom quanto | menor que; insuficiente}
  \seeref{mo4}
  \seealsoref{不曾}{bu4 ceng2}
  \seealsoref{哪个}{na3ge5}
  \seealsoref{全都不}{quan2dou1 bu4}
  \seealsoref{谁}{shei2}
  \seealsoref{什么}{shen2me5}
\end{EntryWithPhonetic}

\begin{EntryWithPhonetic}{没错}{mei2 cuo4}{7,13}{⽔,⾦}[HSK 4]
  \definition{adv.}{está certo; é isso mesmo; não há como errar}
\end{EntryWithPhonetic}

\begin{EntryWithPhonetic}{没法儿}{mei2 fa3r5}{7,8,2}{⽔,⽔,⼉}[HSK 4]
  \definition{adv.}{não pode; sem chance}
\end{EntryWithPhonetic}

\begin{EntryWithPhonetic}{没关系}{mei2guan1xi5}{7,6,7}{⽔,⼋,⽷}[HSK 1]
  \definition{v.}{está tudo bem; não é nada; não importa; não se preocupe}
  \seealsoref{没有关系}{mei2you3guan1xi5}
\end{EntryWithPhonetic}

\begin{EntryWithPhonetic}{没劲}{mei2jin4}{7,7}{⽔,⼒}[HSK 7-9]
  \definition{adj.}{chato; sem graça; entediante; desinteressante (expressando insatisfação)}
  \definition{v.}{cansar; entediar-se; não se sentir bem}
\end{EntryWithPhonetic}

\begin{EntryWithPhonetic}{没了}{mei2le5}{7,2}{⽔,⼅}
  \definition{v.}{estar morto | deixar de existir}
\end{EntryWithPhonetic}

\begin{EntryWithPhonetic}{没什么}{mei2 shen2 me5}{7,4,3}{⽔,⼈,⼃}[HSK 1]
  \definition{expr.}{não é nada; está tudo bem; não importa}
\end{EntryWithPhonetic}

\begin{EntryWithPhonetic}{没事儿}{mei2 shi4r5}{7,8,2}{⽔,⼅,⼉}[HSK 1]
  \definition{expr.}{fora de perigo; nada sério | não importa; não é nada; está tudo bem; não importa | está tudo bem; sem problemas; não se preocupe com isso; não é grande coisa; não há nada errado}
  \definition{v.}{não ter nada para fazer; ser livre; estar perdido | estar desempregado; estar sem trabalho | não ter responsabilidade}
\end{EntryWithPhonetic}

\begin{EntryWithPhonetic}{没说的}{mei2shuo1de5}{7,9,8}{⽔,⾔,⽩}[HSK 7-9]
  \definition{adj.}{impecável; perfeito | óbvio; sem dúvida alguma; não há necessidade de dizer mais nada sobre isso | naturalmente; é claro | inquestionável; realmente bom; sem defeitos a criticar | sem problemas}
\end{EntryWithPhonetic}

\begin{EntryWithPhonetic}{没完没了}{mei2wan2-mei2liao3}{7,7,7,2}{⽔,⼧,⽔,⼅}[HSK 7-9]
  \definition{expr.}{``Isso nunca acaba.''; interminável; sem fim; ininterrupto; que continua indefinidamente; descreve alguém que fala ou age sem parar}
\end{EntryWithPhonetic}

\begin{EntryWithPhonetic}{没想到}{mei2 xiang3 dao4}{7,13,8}{⽔,⼼,⼑}[HSK 4]
  \definition{expr.}{não esperava; inesperado}
\end{EntryWithPhonetic}

\begin{EntryWithPhonetic}{没意思}{mei2 yi4si5}{7,13,9}{⽔,⼼,⼼}[HSK 7-9]
  \definition{adj.}{entediante; tedioso; sem graça nenhuma}
\end{EntryWithPhonetic}

\begin{EntryWithPhonetic}{没用}{mei2 yong4}{7,5}{⽔,⽤}[HSK 3]
  \definition{adj.}{inútil; imprestável; sem valor; sem préstimo; vão; que não serve para nada}
\end{EntryWithPhonetic}

\begin{EntryWithPhonetic}{没有}{mei2 you3}{7,6}{⽔,⽉}[HSK 1]
  \definition{adv.}{ainda não; (usado com o pretérito) não; ação ou estado negativo ocorreu}
  \definition{v.}{não há; não tem; não existe}
\end{EntryWithPhonetic}

\begin{EntryWithPhonetic}{没有次序}{mei2you3 ci4xu4}{7,6,6,7}{⽔,⽉,⽋,⼴}
  \definition{adj.}{sem ordem; nenhuma ordem}
\end{EntryWithPhonetic}

\begin{EntryWithPhonetic}{没有关系}{mei2you3guan1xi5}{7,6,6,7}{⽔,⽉,⼋,⽷}
  \definition{expr.}{Está tudo bem; sem problemas}
  \seealsoref{没关系}{mei2guan1xi5}
\end{EntryWithPhonetic}

\begin{EntryWithPhonetic}{没有哪一种东西}{mei2you3 na3 yi4 zhong3 dong1xi1}{7,6,9,1,9,5,6}{⽔,⽉,⼝,⼀,⽲,⼀,⾑}
  \definition{pron.}{nada; não existe tal coisa}
\end{EntryWithPhonetic}

\begin{EntryWithPhonetic}{没有谁}{mei2you3 shei2}{7,6,10}{⽔,⽉,⾔}
  \definition{pron.}{ninguém}
\end{EntryWithPhonetic}

\begin{EntryWithPhonetic}{没有意思}{mei2you3yi4si5}{7,6,13,9}{⽔,⽉,⼼,⼼}
  \definition{adj.}{entediante; desinteressante}
\end{EntryWithPhonetic}

\begin{EntryWithPhonetic}{没辙}{mei2/zhe2}{7,16}{⽔,⾞}[HSK 7-9]
  \definition{v.+compl.}{sem saber o que fazer; ser incapaz de encontrar uma saída; não ter jeito}
\end{EntryWithPhonetic}

\begin{EntryWithPhonetic}{没准儿}{mei2/zhun3r5}{7,10,2}{⽔,⼎,⼉}[HSK 7-9]
  \definition{adj.}{não confiável}
  \definition{adv.}{talvez; quem sabe}
  \definition{v.+compl.}{não ter certeza; estar incerto; não se saber ao certo}
\end{EntryWithPhonetic}

%%%%%%%%%% 枚 %%%%%%%%%%
\subsection*{枚}\addcontentsline{loh}{figure}{枚 \dpy{mei2}}

\begin{EntryWithPhonetic}{枚}{mei2}{8}{⽊}[HSK 7-9]
  \definition*{s.}{Sobrenome: Mei}
  \definition{clas.}{utilizado para objetos pequenos | utilizado para moedas, anéis, distintivos, pérolas, medalhas esportivas, foguetes, satélites etc.}
  \definition{s.}{Arcaico: tronco de árvore (significado original) | Arcaico: chicote | Arcaico: pino de madeira, usado como mordaça para soldados em marcha}
\end{EntryWithPhonetic}

%%%%%%%%%% 玫 %%%%%%%%%%
\subsection*{玫}\addcontentsline{loh}{figure}{玫 \dpy{mei2}}

\begin{EntryWithPhonetic}{玫}{mei2}{8}{⽟}
  \definition[朵]{s.}{rosa | Arcaico: um tipo de jade bonito}
\end{EntryWithPhonetic}

\begin{EntryWithPhonetic}{玫瑰}{mei2gui5}{8,13}{⽟,⽟}[HSK 7-9]
  \definition[束,朵,棵,株]{s.}{rosa; um arbusto decíduo, seus ramos são espinhosos, suas folhas são ovais e suas flores, que podem ser vermelho-púrpura, brancas e de outras cores, são perfumadas e ornamentais}
\end{EntryWithPhonetic}

%%%%%%%%%% 眉 %%%%%%%%%%
\subsection*{眉}\addcontentsline{loh}{figure}{眉 \dpy{mei2}}

\begin{EntryWithPhonetic}{眉}{mei2}{9}{⽬}
  \definition*{s.}{Sobrenome: Mei}
  \definition[个]{s.}{sobrancelha | a margem superior de uma página; o espaço em branco na parte superior da página}
\end{EntryWithPhonetic}

\begin{EntryWithPhonetic}{眉开眼笑}{mei2kai1-yan3xiao4}{9,4,11,10}{⽬,⼶,⽬,⽵}[HSK 7-9]
  \definition{expr.}{parecer alegre; um semblante radiante; estar sempre sorrindo; irradiar alegria; estar cheio de felicidade; rosto transbordando de sorrisos; sentir-se feliz e sorrir; sorrir de orelha a orelha; os olhos brilhando de alegria; o rosto se iluminando de sorrisos; os olhos brilhantes dançando de alegria; sorrir alegremente (de orelha a orelha); muito feliz}
\end{EntryWithPhonetic}

\begin{EntryWithPhonetic}{眉毛}{mei2mao5}{9,4}{⽬,⽑}[HSK 7-9]
  \definition[根]{s.}{sobrancelha; pelos que crescem ao longo da borda superior da órbita ocular humana}
\end{EntryWithPhonetic}

\begin{EntryWithPhonetic}{眉头}{mei2tou2}{9,5}{⽬,⼤}
  \definition{s.}{testa; área próxima às sobrancelhas}
\end{EntryWithPhonetic}

%%%%%%%%%% 梅 %%%%%%%%%%
\subsection*{梅}\addcontentsline{loh}{figure}{梅 \dpy{mei2}}

\begin{EntryWithPhonetic}{梅}{mei2}{11}{⽊}
  \definition*{s.}{Sobrenome: Mei}
  \definition{s.}{ameixa | flor de ameixa | ameixeira | estação chuvosa}
\end{EntryWithPhonetic}

\begin{EntryWithPhonetic}{梅花}{mei2 hua1}{11,7}{⽊,⾋}[HSK 6]
  \definition[朵,枝,片,瓣,束,株]{s.}{paus ♣ (um naipe em jogos de cartas) | flor de ameixa | doçura-de-inverno; refere-se especificamente à flor-de-inverno ; também se refere a algo que se parece com esta flor}
  \seealsoref{方片}{fang1 pian4}
  \seealsoref{黑桃}{hei1 tao2}
  \seealsoref{红心}{hong2 xin1}
\end{EntryWithPhonetic}

\begin{EntryWithPhonetic}{梅赛德斯-奔驰}{mei2sai4de2si1-ben1chi2}{11,14,15,12,8,6}{⽊,⾙,⼻,⽄,⼤,⾺}
  \definition*{s.}{Mercedes-Benz}
\end{EntryWithPhonetic}

%%%%%%%%%% 媒 %%%%%%%%%%
\subsection*{媒}\addcontentsline{loh}{figure}{媒 \dpy{mei2}}

\begin{EntryWithPhonetic}{媒}{mei2}{12}{⼥}
  \definition{s.}{casamenteiro; intermediário | intermediário; médio}
  \definition{v.}{fazer uma combinação}
\end{EntryWithPhonetic}

\begin{EntryWithPhonetic}{媒体}{mei2ti3}{12,7}{⼥,⼈}[HSK 3]
  \definition[家,个,种]{s.}{mídia; mídia de massa; vários meios de comunicação e transmissão de informações, como televisão, rádio, jornais, etc.}
\end{EntryWithPhonetic}

%%%%%%%%%% 煤 %%%%%%%%%%
\subsection*{煤}\addcontentsline{loh}{figure}{煤 \dpy{mei2}}

\begin{EntryWithPhonetic}{煤}{mei2}{13}{⽕}[HSK 5]
  \definition[块,吨,斤,堆]{s.}{carvão; carvão vegetal; minério sólido preto}
\end{EntryWithPhonetic}

\begin{EntryWithPhonetic}{煤矿}{mei2kuang4}{13,8}{⽕,⽯}[HSK 7-9]
  \definition{s.}{mina de carvão}
\end{EntryWithPhonetic}

\begin{EntryWithPhonetic}{煤气}{mei2 qi4}{13,4}{⽕,⽓}[HSK 5]
  \definition[罐,瓶]{s.}{gás; gás de carvão; gás obtido a partir do processamento do carvão não tem cor nem odor, é tóxico e pode ser queimado ou utilizado como matéria-prima na indústria química | envenenamento por monóxido de carbono}
\end{EntryWithPhonetic}

\begin{EntryWithPhonetic}{煤炭}{mei2tan4}{13,9}{⽕,⽕}[HSK 7-9]
  \definition{s.}{carvão}
\end{EntryWithPhonetic}

%%%%%%%%%% 每 %%%%%%%%%%
\subsection*{每}\addcontentsline{loh}{figure}{每 \dpy{mei3}}

\begin{EntryWithPhonetic}{每}{mei3}{7}{⽏}[HSK 3]
  \definition{adv.}{cada um; cada qual; indica qualquer uma das repetições ou um conjunto de repetições de um movimento}
  \definition{pron.}{cada; cada um; cada qual; refere-se a qualquer indivíduo do grupo, enfatizando as semelhanças entre os indivíduos}
\end{EntryWithPhonetic}

\begin{EntryWithPhonetic}{每次}{mei3ci4}{7,6}{⽏,⽋}
  \definition{adv.}{toda vez | cada vez}
\end{EntryWithPhonetic}

\begin{EntryWithPhonetic}{每当}{mei3dang1}{7,6}{⽏,⼹}[HSK 7-9]
  \definition{adv.}{sempre que; todas as vezes; toda vez que isso acontece; em qualquer momento}
\end{EntryWithPhonetic}

\begin{EntryWithPhonetic}{每逢}{mei3feng2}{7,10}{⽏,⾡}[HSK 7-9]
  \definition{adv.}{sempre que; em todas as ocasiões; em todas as situações; toda vez que me deparo com isso; toda vez que chego}
\end{EntryWithPhonetic}

\begin{EntryWithPhonetic}{每个}{mei3ge4}{7,3}{⽏,⼈}
  \definition{pron.}{cada; cada um}
\end{EntryWithPhonetic}

\begin{EntryWithPhonetic}{每个人}{mei3ge5ren2}{7,3,2}{⽏,⼈,⼈}
  \definition{pron.}{todo mundo | todos}
\end{EntryWithPhonetic}

\begin{EntryWithPhonetic}{每天}{mei3tian1}{7,4}{⽏,⼤}
  \definition{adv.}{todo dia | cada dia}
\end{EntryWithPhonetic}

%%%%%%%%%% 美 %%%%%%%%%%
\subsection*{美}\addcontentsline{loh}{figure}{美 \dpy{mei3}}

\begin{EntryWithPhonetic}{美}{mei3}{9}{⽺}[HSK 3]
  \definition*{s.}{Abreviatura de América, 美洲 | Abreviatura de Estados Unidos da América, 美国 | As Américas, 美洲}
  \definition{adj.}{belo; bonito (oposto de 丑) | muito satisfatório; bom; agradável}
  \definition{s.}{beleza (oposto de 丑)}
  \definition{v.}{embelezar; tornar mais bonito | estar satisfeito consigo mesmo; orgulhar-se; sentir-se presunçoso}
  \seealsoref{丑}{chou3}
  \seealsoref{美国}{mei3guo2}
  \seealsoref{美洲}{mei3zhou1}
\end{EntryWithPhonetic}

\begin{EntryWithPhonetic}{美德}{mei3de2}{9,15}{⽺,⼻}[HSK 7-9]
  \definition*{s.}{My Duc District (Hanoi)}
  \definition[种,些]{s.}{virtude; excelência moral; bom caráter}
\end{EntryWithPhonetic}

\begin{EntryWithPhonetic}{美观}{mei3guan1}{9,6}{⽺,⾒}[HSK 7-9]
  \definition{adj.}{artístico; belo; agradável à vista; que agrada aos olhos}
\end{EntryWithPhonetic}

\begin{EntryWithPhonetic}{美国}{mei3guo2}{9,8}{⽺,⼞}
  \definition*{s.}{Estados Unidos da América}
\end{EntryWithPhonetic}

\begin{EntryWithPhonetic}{美国人}{mei3guo2ren2}{9,8,2}{⽺,⼞,⼈}
  \definition{s.}{americano | pessoa ou povo dos Estados Unidos da América}
\end{EntryWithPhonetic}

\begin{EntryWithPhonetic}{美好}{mei3 hao3}{9,6}{⽺,⼥}[HSK 3]
  \definition{adj.}{bem; feliz; glorioso; descreve a vida, os desejos, etc. como sendo muito bons e satisfatórios}
\end{EntryWithPhonetic}

\begin{EntryWithPhonetic}{美化}{mei3hua4}{9,4}{⽺,⼔}[HSK 7-9]
  \definition{v.}{embelezar; enfeitar; adornar; estética através da decoração e do embelezamento | embelezar; retratar o mal como bem e o feio como belo}
\end{EntryWithPhonetic}

\begin{EntryWithPhonetic}{美甲}{mei3jia3}{9,5}{⽺,⽥}
  \definition{s.}{manicure e/ou pedicure}
\end{EntryWithPhonetic}

\begin{EntryWithPhonetic}{美金}{mei3 jin1}{9,8}{⽺,⾦}[HSK 4]
  \definition{s.}{USD; dólar americano: a moeda local dos Estados Unidos}
\end{EntryWithPhonetic}

\begin{EntryWithPhonetic}{美景}{mei3jing3}{9,12}{⽺,⽇}[HSK 7-9]
  \definition{s.}{paisagem deslumbrante; paisagens belíssimas (como o mar, a terra ou o céu)}
\end{EntryWithPhonetic}

\begin{EntryWithPhonetic}{美丽}{mei3li4}{9,7}{⽺,⼀}[HSK 3]
  \definition{adj.}{bonito; lindo; capaz de proporcionar uma sensação de beleza}
\end{EntryWithPhonetic}

\begin{EntryWithPhonetic}{美满}{mei3man3}{9,13}{⽺,⽔}[HSK 7-9]
  \definition{adj.}{feliz; perfeitamente satisfatório; lindo e perfeito}
\end{EntryWithPhonetic}

\begin{EntryWithPhonetic}{美妙}{mei3miao4}{9,7}{⽺,⼥}[HSK 7-9]
  \definition{adj.}{esplêndido; belo; maravilhoso; descreve obras literárias, música, sentimentos, etc., como belos, únicos ou engenhosos}
\end{EntryWithPhonetic}

\begin{EntryWithPhonetic}{美女}{mei3 nv3}{9,3}{⽺,⼥}[HSK 4]
  \definition[个,位,名,些]{s.}{beldade; mulher bonita; uma jovem linda}
\end{EntryWithPhonetic}

\begin{EntryWithPhonetic}{美人}{mei3ren2}{9,2}{⽺,⼈}[HSK 7-9]
  \definition[个,位,名]{s.}{beleza; mulher bonita}
\end{EntryWithPhonetic}

\begin{EntryWithPhonetic}{美容}{mei3 rong2}{9,10}{⽺,⼧}[HSK 6]
  \definition{v.}{embelezar; melhorar a aparência de alguém; deixar seu rosto bonito retocando, cuidando, etc.}
\end{EntryWithPhonetic}

\begin{EntryWithPhonetic}{美食}{mei3 shi2}{9,9}{⽺,⾷}[HSK 3]
  \definition[种,道,桌]{s.}{iguaria; (gastronomia) comida saborosa}
\end{EntryWithPhonetic}

\begin{EntryWithPhonetic}{美术}{mei3shu4}{9,5}{⽺,⽊}[HSK 3]
  \definition[种]{s.}{arte; artes plásticas: arte que ocupa um determinado espaço, compõe imagens estéticas e permite que as pessoas apreciem visualmente, incluindo pintura, escultura, arquitetura, etc. | pintura; pintura tradicional chinesa}
\end{EntryWithPhonetic}

\begin{EntryWithPhonetic}{美味}{mei3wei4}{9,8}{⽺,⼝}[HSK 7-9]
  \definition[顿]{s.}{comida deliciosa; iguaria (\emph{delicacy}); sabor delicioso}
\end{EntryWithPhonetic}

\begin{EntryWithPhonetic}{美学}{mei3xue2}{9,8}{⽺,⼦}
  \definition{s.}{estética; a ciência que estuda as leis e os princípios gerais da beleza na natureza, na sociedade e na arte explora principalmente a natureza da beleza, a relação entre arte e realidade e as leis gerais da criação artística}
\end{EntryWithPhonetic}

\begin{EntryWithPhonetic}{美元}{mei3yuan2}{9,4}{⽺,⼉}[HSK 3]
  \definition*[元,笔,沓]{s.}{Dólar Americano; a moeda dos Estados Unidos}
\end{EntryWithPhonetic}

\begin{EntryWithPhonetic}{美中不足}{mei3zhong1-bu4zu2}{9,4,4,7}{⽺,⼁,⼀,⾜}[HSK 7-9]
  \definition{expr.}{belo, porém incompleto (que carece de perfeição); uma imperfeição em algo perfeito; uma falha em algo aparentemente perfeito; um problema; a parte desagradável de algo agradável; algumas imperfeições em algo aparentemente perfeito; alguma pequena falta de perfeição; algo que não atinge a perfeição; há uma falha no ato; é bom, mas ainda tem falhas; um obstáculo; uma falha que prejudica a beleza; um defeito em algo aparentemente perfeito}
\end{EntryWithPhonetic}

\begin{EntryWithPhonetic}{美洲}{mei3zhou1}{9,9}{⽺,⽔}
  \definition*{s.}{América (incluindo Norte, Central e Sul)}
\end{EntryWithPhonetic}

\begin{EntryWithPhonetic}{美洲人}{mei3zhou1ren2}{9,9,2}{⽺,⽔,⼈}
  \definition{s.}{americano | pessoa ou povo do continente Americano}
\end{EntryWithPhonetic}

\begin{EntryWithPhonetic}{美滋滋}{mei3zi1zi1}{9,12,12}{⽺,⽔,⽔}[HSK 7-9]
  \definition{interj.}{Me sentindo ótimo!; exultante; muito feliz; muito satisfeito consigo mesmo}
\end{EntryWithPhonetic}

%%%%%%%%%% 妹 %%%%%%%%%%
\subsection*{妹}\addcontentsline{loh}{figure}{妹 \dpy{mei4}}

\begin{EntryWithPhonetic}{妹}{mei4}{8}{⼥}[HSK 1]
  \definition*{s.}{Sobrenome: Mei}
  \definition[个]{s.}{irmã mais nova | parente do sexo feminino da mesma geração | jovem garota; jovem mulher ou menina}
  \seealsoref{妹妹}{mei4 mei5}
\end{EntryWithPhonetic}

\begin{EntryWithPhonetic}{妹夫}{mei4fu5}{8,4}{⼥,⼤}
  \definition{s.}{marido da irmã mais nova}
\end{EntryWithPhonetic}

\begin{EntryWithPhonetic}{妹妹}{mei4 mei5}{8,8}{⼥,⼥}[HSK 1]
  \definition[个]{s.}{irmã mais nova}
\end{EntryWithPhonetic}

%%%%%%%%%% 谜 %%%%%%%%%%
\subsection*{谜}\addcontentsline{loh}{figure}{谜 \dpy{mei4}}

\begin{EntryWithPhonetic}{谜}{mei4}{11}{⾔}
  \definition[个]{s.}{enigma}
  \seeref{mi2}
  \seealsoref{谜儿}{mei4r5}
\end{EntryWithPhonetic}

\begin{EntryWithPhonetic}{谜儿}{mei4r5}{11,2}{⾔,⼉}
  \definition{s.}{enigma; mistério}
\end{EntryWithPhonetic}

%%%%%%%%%% 魅 %%%%%%%%%%
\subsection*{魅}\addcontentsline{loh}{figure}{魅 \dpy{mei4}}

\begin{EntryWithPhonetic}{魅}{mei4}{14}{⿁}
  \definition{s.}{espírito maligno; demônio | \emph{goblin}; trasgo; gnomo; duende maléfico}
  \definition{v.}{atormentar; cativar}
\end{EntryWithPhonetic}

\begin{EntryWithPhonetic}{魅力}{mei4li4}{14,2}{⿁,⼒}[HSK 7-9]
  \definition[种]{s.}{charme; feitiço; glamour; bruxaria; carisma; feitiçaria; encanto; fascínio; encantamento; o poder de atrair e motivar pessoas}
\end{EntryWithPhonetic}

%%%%%%%%%% 闷 %%%%%%%%%%
\subsection*{闷}\addcontentsline{loh}{figure}{闷 \dpy{men1}}

\begin{EntryWithPhonetic}{闷}{men1}{7}{⾨}[HSK 7-9]
  \definition{adj.}{abafado; fechado; sufocante; baixa pressão de ar ou má circulação de ar | abafado; som baixo ou opaco}
  \definition{v.}{cubrir bem; fazer algo hermético | ficar sem fala; parar de falar | fechar a si mesmo ou alguém dentro de casa; ficar em casa e não sair}
  \seeref{men4}
\end{EntryWithPhonetic}

\begin{EntryWithPhonetic}{闷热}{men1re4}{7,10}{⾨,⽕}
  \definition{adj.}{abafado | quente e abafado | sufocantemente quente | quente e sensual}
\end{EntryWithPhonetic}

%%%%%%%%%% 门 %%%%%%%%%%
\subsection*{门}\addcontentsline{loh}{figure}{门 \dpy{men2}}

\begin{EntryWithPhonetic}{门}{men2}{3}{⾨}[HSK 1][Kangxi 169]
  \definition*{s.}{Sobrenome: Men}
  \definition{clas.}{para equipamentos de artilharia (por exemplo: canhões) | para trabalhos escolares, ciência e tecnologia, etc. | para idiomas | para casamentos | para parentes}
  \definition[个,把,道,扇]{s.}{entradas e saídas de edifícios, veículos, navios, aviões, etc. | válvula; interruptor; algo que funciona como um interruptor ou como uma porta | habilidade; método; acesso; maneira de fazer algo | família; ramo de uma família ou clã | seita (religiosa); escola (de pensamento); faculdades acadêmicas, ideológicas ou religiosas | classe; categoria; ramo de estudo; refere-se à categoria geral de coisas | filo; segundo nível da classificação biológica | (computador) \emph{gate}; porta (lógica) | porta; portão; entrada; refere-se a uma porta que pode ser aberta e fechada, instalada na entrada e saída | qualquer abertura; partes de objetos que podem ser abertas e fechadas | orifício no corpo humano; refere-se especificamente aos orifícios do corpo humano | estudar com o mesmo professor; refere-se especificamente ao professor ou mestre | posição em um jogo de apostas (em relação ao local onde se senta ou onde se faz uma aposta)}
\end{EntryWithPhonetic}

\begin{EntryWithPhonetic}{门当户对}{men2dang1-hu4dui4}{3,6,4,5}{⾨,⼹,⼾,⼨}[HSK 7-9]
  \definition{expr.}{``Casar com alguém de posição social equivalente.''; compatibilidade social e econômica adequada (para casamento); (um possível parceiro para casamento) uma combinação adequada; as famílias são bem compatíveis em termos de status social}
\end{EntryWithPhonetic}

\begin{EntryWithPhonetic}{门道}{men2dao4}{3,12}{⾨,⾡}
  \definition{s.}{porta | portal}
  \seeref{men2dao5}
\end{EntryWithPhonetic}

\begin{EntryWithPhonetic}{门道}{men2dao5}{3,12}{⾨,⾡}
  \definition{s.}{talento | a maneira de fazer algo}
  \seeref{men2dao4}
\end{EntryWithPhonetic}

\begin{EntryWithPhonetic}{门槛}{men2kan3}{3,14}{⾨,⽊}[HSK 7-9]
  \definition{s.}{soleira; batente da porta; a viga horizontal ou faixa de pedra na parte inferior do batente da porta, próxima ao chão, etc. | o requisito para entrar em um determinado campo; metaforicamente, refere-se aos padrões ou condições para entrar em um determinado intervalo}
\end{EntryWithPhonetic}

\begin{EntryWithPhonetic}{门口}{men2 kou3}{3,3}{⾨,⼝}[HSK 1]
  \definition[个]{s.}{porta; portão; entrada; porta de entrada}
\end{EntryWithPhonetic}

\begin{EntryWithPhonetic}{门铃}{men2ling2}{3,10}{⾨,⾦}[HSK 7-9]
  \definition{s.}{campainha; uma campainha ou campainha elétrica para bater à porta}
\end{EntryWithPhonetic}

\begin{EntryWithPhonetic}{门路}{men2lu5}{3,13}{⾨,⾜}[HSK 7-9]
  \definition{s.}{maneira de fazer algo; jeito | conexões sociais (para conseguir empregos, etc.) | jeito; saber fazer; truque do ofício; o segredo para fazer as coisas acontecerem}
  \seealsoref{门道}{men2dao5}
\end{EntryWithPhonetic}

\begin{EntryWithPhonetic}{门票}{men2 piao4}{3,11}{⾨,⽰}[HSK 1]
  \definition{s.}{bilhete de entrada; bilhete de admissão; ingressos para locais de turismo, entretenimento, etc.}
\end{EntryWithPhonetic}

\begin{EntryWithPhonetic}{门诊}{men2 zhen3}{3,7}{⾨,⾔}[HSK 5]
  \definition{s.}{(no hospital) clínica ambulatorial; seção para pacientes ambulatoriais; local onde os médicos atendem pacientes que não estão internados no hospital}
\end{EntryWithPhonetic}

%%%%%%%%%% 闷 %%%%%%%%%%
\subsection*{闷}\addcontentsline{loh}{figure}{闷 \dpy{men4}}

\begin{EntryWithPhonetic}{闷}{men4}{7}{⾨}[HSK 7-9]
  \definition{adj.}{entediado; deprimido; irritado; desanimado | hermeticamente fechado; selado | triste e silencioso; chateado | hermético}
  \definition{s.}{desânimo}
  \seeref{men1}
\end{EntryWithPhonetic}

%%%%%%%%%% 们 %%%%%%%%%%
\subsection*{们}\addcontentsline{loh}{figure}{们 \dpy{men5}}

\begin{EntryWithPhonetic}{们}{men5}{5}{⼈}[HSK 1]
  \definition{suf.}{usado após pronomes ou substantivos que se referem a pessoas para indicar pluralidade}
\end{EntryWithPhonetic}

%%%%%%%%%% 蒙 %%%%%%%%%%
\subsection*{蒙}\addcontentsline{loh}{figure}{蒙 \dpy{meng1}}

\begin{EntryWithPhonetic}{蒙}{meng1}{13}{⾋}[HSK 6]
  \definition{adj.}{inconsciente; sem sentido;  em coma | confuso; perplexo}
  \definition{v.}{enganar; enganar; trapacear; iludir; trair | fazer um palpite ousado; dar um palpite ousado; arriscar-se}
  \seeref{meng2}
  \seeref{meng3}
\end{EntryWithPhonetic}

%%%%%%%%%% 萌 %%%%%%%%%%
\subsection*{萌}\addcontentsline{loh}{figure}{萌 \dpy{meng2}}

\begin{EntryWithPhonetic}{萌}{meng2}{11}{⾋}
  \definition*{s.}{Sobrenome: Meng}
  \definition{s.}{broto; rebento | Arcaico: o povo comum}
  \definition{v.}{(plantas) brotar; surgir; brotar; germinar | começar; surgir; ocorrer; emergir}
\end{EntryWithPhonetic}

\begin{EntryWithPhonetic}{萌发}{meng2fa1}{11,5}{⾋,⼜}[HSK 7-9]
  \definition{v.}{brotar; germinar; germinação de sementes ou esporos | emergir; vir à tona; isso se refere metaforicamente ao início de um determinado pensamento, ideia ou sentimento}
\end{EntryWithPhonetic}

\begin{EntryWithPhonetic}{萌芽}{meng2ya2}{11,7}{⾋,⾋}[HSK 7-9]
  \definition{s.}{semente; germe; rudimento; uma metáfora para algo novo e ainda não totalmente desenvolvido}
  \definition{v.}{brotar; germinar; desabrochar; o brotar de plantas é uma metáfora para algo que está apenas começando a acontecer}
\end{EntryWithPhonetic}

%%%%%%%%%% 盟 %%%%%%%%%%
\subsection*{盟}\addcontentsline{loh}{figure}{盟 \dpy{meng2}}

\begin{EntryWithPhonetic}{盟}{meng2}{13}{⽫}
  \definition{adj.}{jurados (irmãos); aliados | jurado; antigamente, referia-se a uma irmandade jurada}
  \definition{s.}{aliança; coligação | liga (uma divisão administrativa da Região Autônoma da Mongólia Interior, correspondente a uma prefeitura)}
  \definition{v.}{aliar-se | fazer um juramento; jurar}
  \seeref{ming2}
\end{EntryWithPhonetic}

\begin{EntryWithPhonetic}{盟友}{meng2you3}{13,4}{⽫,⼜}[HSK 7-9]
  \definition{s.}{aliado; amigo leal | país (ou estado) aliado}
\end{EntryWithPhonetic}

%%%%%%%%%% 蒙 %%%%%%%%%%
\subsection*{蒙}\addcontentsline{loh}{figure}{蒙 \dpy{meng2}}

\begin{EntryWithPhonetic}{蒙}{meng2}{13}{⾋}[HSK 6]
  \definition*{s.}{Sobrenome: Meng}
  \definition{adj.}{ignorância; analfabetismo; falta de instrução | nebuloso; aparência pequena e pouco clara, como chuva ou neblina}
  \definition{s.}{aberto; inicial}
  \definition{v.}{cobrir; espalhar | receber apoio | receber; encontrar-se com; encontrar-se; palavras respeitosas; expressam os benefícios recebidos de outros | sofrer; incorrer}
  \seeref{meng1}
  \seeref{meng3}
\end{EntryWithPhonetic}

\begin{EntryWithPhonetic}{蒙面}{meng2mian4}{13,9}{⾋,⾯}
  \definition{adj.}{descarado | desavergonhado | mascarado}
  \definition{v.}{cobrir o rosto | usar uma máscara}
\end{EntryWithPhonetic}

%%%%%%%%%% 朦 %%%%%%%%%%
\subsection*{朦}\addcontentsline{loh}{figure}{朦 \dpy{meng2}}

\begin{EntryWithPhonetic}{朦}{meng2}{17}{⽉}
  \definition{adj.}{indistinto | pouco claro}
  \definition{v.}{enganar}
\end{EntryWithPhonetic}

\begin{EntryWithPhonetic}{朦胧}{meng2long2}{17,9}{⽉,⾁}[HSK 7-9]
  \definition{adj.}{fraco; nebuloso; obscuro; pouco claro; vago}
\end{EntryWithPhonetic}

%%%%%%%%%% 猛 %%%%%%%%%%
\subsection*{猛}\addcontentsline{loh}{figure}{猛 \dpy{meng3}}

\begin{EntryWithPhonetic}{猛}{meng3}{11}{⽝}[HSK 6]
  \definition*{s.}{Sobrenome: Meng}
  \definition{adj.}{feroz; violento | enérgico; vigoroso | valente}
  \definition{adv.}{de repente; abruptamente | vigorosamente; com força repentina | (coloquial) ao contentamento do coração; de todo o coração | ferozmente; violentamente}
\end{EntryWithPhonetic}

\begin{EntryWithPhonetic}{猛烈}{meng3lie4}{11,10}{⽝,⽕}[HSK 7-9]
  \definition{adj.}{feroz; violento; vigoroso}
\end{EntryWithPhonetic}

\begin{EntryWithPhonetic}{猛然}{meng3ran2}{11,12}{⽝,⽕}[HSK 7-9]
  \definition{adv.}{de repente; abruptamente; indica ação repentina e rápida}
\end{EntryWithPhonetic}

%%%%%%%%%% 蒙 %%%%%%%%%%
\subsection*{蒙}\addcontentsline{loh}{figure}{蒙 \dpy{meng3}}

\begin{EntryWithPhonetic}{蒙}{meng3}{13}{⾋}
  \definition{s.}{grupo étnico mongol; mongol}
  \seeref{meng1}
  \seeref{meng2}
\end{EntryWithPhonetic}

%%%%%%%%%% 懵 %%%%%%%%%%
\subsection*{懵}\addcontentsline{loh}{figure}{懵 \dpy{meng3}}

\begin{EntryWithPhonetic}{懵}{meng3}{18}{⼼}
  \definition{adj.}{confuso; ignorante; irracional | inconsciente; entorpecido}
\end{EntryWithPhonetic}

\begin{EntryWithPhonetic}{懵懂}{meng3dong3}{18,15}{⼼,⼼}
  \definition{adj.}{confuso | ignorante}
\end{EntryWithPhonetic}

%%%%%%%%%% 梦 %%%%%%%%%%
\subsection*{梦}\addcontentsline{loh}{figure}{梦 \dpy{meng4}}

\begin{EntryWithPhonetic}{梦}{meng4}{11}{⼣}[HSK 4]
  \definition*{s.}{Sobrenome: Meng}
  \definition[个,场]{s.}{sonho; atividade de representação no cérebro durante o sono}
  \definition{v.}{sonhar; ter um sonho}
\end{EntryWithPhonetic}

\begin{EntryWithPhonetic}{梦幻}{meng4huan4}{11,4}{⼣,⼳}[HSK 7-9]
  \definition[种,片]{s.}{ilusão; sonho; devaneio; cenas e situações estranhas que aparecem nos sonhos}
\end{EntryWithPhonetic}

\begin{EntryWithPhonetic}{梦见}{meng4 jian4}{11,4}{⼣,⾒}[HSK 4]
  \definition{v.}{sonhar; sonhar com; ver em um sonho}
\end{EntryWithPhonetic}

\begin{EntryWithPhonetic}{梦想}{meng4xiang3}{11,13}{⼣,⼼}[HSK 4]
  \definition[个,种,些,番]{s.}{sonho; esperança vã; sonho irreal; divagação; um desejo ou ideia que você espera particularmente realizar}
  \definition{v.}{sonhar; desejar sinceramente; ansiar}
\end{EntryWithPhonetic}

%%%%%%%%%% 眯 %%%%%%%%%%
\subsection*{眯}\addcontentsline{loh}{figure}{眯 \dpy{mi1}}

\begin{EntryWithPhonetic}{眯}{mi1}{11}{⽬}
  \definition{v.}{estreitar os olhos | esmagar | (dialeto) tirar uma soneca}
  \seeref{mi2}
\end{EntryWithPhonetic}

%%%%%%%%%% 弥 %%%%%%%%%%
\subsection*{弥}\addcontentsline{loh}{figure}{弥 \dpy{mi2}}

\begin{EntryWithPhonetic}{弥}{mi2}{8}{⼸}
  \definition*{s.}{Sobrenome: Mi}
  \definition{adj.}{cheio; inteiro}
  \definition{adv.}{Literário: mais; ainda mais}
  \definition{v.}{transbordar; encher | cobrir; encher}
\end{EntryWithPhonetic}

\begin{EntryWithPhonetic}{弥补}{mi2bu3}{8,7}{⼸,⾐}[HSK 7-9]
  \definition{v.}{remediar; compensar; reparar; restituir; consertar; preencher; inventar}
\end{EntryWithPhonetic}

\begin{EntryWithPhonetic}{弥漫}{mi2man4}{8,14}{⼸,⽔}[HSK 7-9]
  \definition{v.}{encher; transbordar; preencher o ar; espalhar-se por toda parte}
\end{EntryWithPhonetic}

%%%%%%%%%% 迷 %%%%%%%%%%
\subsection*{迷}\addcontentsline{loh}{figure}{迷 \dpy{mi2}}

\begin{EntryWithPhonetic}{迷}{mi2}{9}{⾡}[HSK 3]
  \definition[个]{s.}{fã; entusiasta; aficionado; pessoa que gosta excessivamente de algo}
  \definition{v.}{estar confuso; perder o rumo; se perder-se; perda da capacidade de discernimento e julgamento | ficar fascinado por; entregar-se a; ficar encantado com (por); ser louco por | confundir; desorientar; fascinar; encantar; tornar indistinto; deixar encantado e fascinado}
\end{EntryWithPhonetic}

\begin{EntryWithPhonetic}{迷宫}{mi2gong1}{9,9}{⾡,⼧}
  \definition{s.}{labirinto}
\end{EntryWithPhonetic}

\begin{EntryWithPhonetic}{迷惑}{mi2huo4}{9,12}{⾡,⼼}[HSK 7-9]
  \definition{adj.}{perplexo; confuso; desnorteado; pouco claro; incompreesível}
  \definition{v.}{intrigar; confundir; deixar perplexo; desconcertar}
\end{EntryWithPhonetic}

\begin{EntryWithPhonetic}{迷惑不解}{mi2huo4-bu4jie3}{9,12,4,13}{⾡,⼼,⼀,⾓}[HSK 7-9]
  \definition{expr.}{sentir-se perplexo; estar confuso; ficar intrigado}
\end{EntryWithPhonetic}

\begin{EntryWithPhonetic}{迷恋}{mi2lian4}{9,10}{⾡,⼼}[HSK 7-9]
  \definition{v.}{ser obcecado por; estar apaixonado por; agarrar-se loucamente a; ter um carinho excessivo por algo e achar difícil abrir mão disso}
\end{EntryWithPhonetic}

\begin{EntryWithPhonetic}{迷路}{mi2/lu4}{9,13}{⾡,⾜}[HSK 7-9]
  \definition{v.+compl.}{desviar-se; perder-se; errar o caminho; perder a noção de direção; ir pelo caminho errado; não conseguir encontrar o caminho | perder-se; estar perdido na vida; essa metáfora descreve a perda da direção correta}
\end{EntryWithPhonetic}

\begin{EntryWithPhonetic}{迷你}{mi2ni3}{9,7}{⾡,⼈}
  \definition{adj.}{(empréstimo linguístico) mini, como em minissaia ou \emph{Mini Cooper}}
\end{EntryWithPhonetic}

\begin{EntryWithPhonetic}{迷人}{mi2ren2}{9,2}{⾡,⼈}[HSK 5]
  \definition{adj.}{encantador; fascinante; sedutor; hipnotizante}
  \definition{v.}{confundir; intrigar; enganar}
\end{EntryWithPhonetic}

\begin{EntryWithPhonetic}{迷失}{mi2shi1}{9,5}{⾡,⼤}[HSK 7-9]
  \definition{v.}{perder-se; ser incapaz de distinguir (direções, estradas, etc.)}
\end{EntryWithPhonetic}

\begin{EntryWithPhonetic}{迷信}{mi2xin4}{9,9}{⾡,⼈}[HSK 5]
  \definition{s.}{superstição; crença supersticiosa | fé cega; adoração cega}
  \definition{v.}{ter fé cega em; ter um fetiche de}
\end{EntryWithPhonetic}

%%%%%%%%%% 眯 %%%%%%%%%%
\subsection*{眯}\addcontentsline{loh}{figure}{眯 \dpy{mi2}}

\begin{EntryWithPhonetic}{眯}{mi2}{11}{⽬}
  \definition{v.}{cegar (como com poeira)}
  \seeref{mi1}
\end{EntryWithPhonetic}

%%%%%%%%%% 谜 %%%%%%%%%%
\subsection*{谜}\addcontentsline{loh}{figure}{谜 \dpy{mi2}}

\begin{EntryWithPhonetic}{谜}{mi2}{11}{⾔}[HSK 7-9]
  \definition[个]{s.}{enigma; charada | enigma; mistério; quebra-cabeça}
  \seeref{mei4}
\end{EntryWithPhonetic}

\begin{EntryWithPhonetic}{谜底}{mi2di3}{11,8}{⾔,⼴}[HSK 7-9]
  \definition[个]{s.}{resposta a um enigma | verdade; metáfora para a verdade dos fatos}
\end{EntryWithPhonetic}

\begin{EntryWithPhonetic}{谜团}{mi2tuan2}{11,6}{⾔,⼞}[HSK 7-9]
  \definition{s.}{dúvidas e suspeitas | assuntos elusivos | enigma | situação imprevisível}
\end{EntryWithPhonetic}

\begin{EntryWithPhonetic}{谜语}{mi2yu3}{11,9}{⾔,⾔}[HSK 7-9]
  \definition[条,则]{s.}{enigma; charada; um enigma é uma mensagem críptica que alude a coisas ou palavras, deixando ao leitor a tarefa de adivinhar; consiste principalmente em duas partes: o próprio enigma e a resposta}
\end{EntryWithPhonetic}

%%%%%%%%%% 米 %%%%%%%%%%
\subsection*{米}\addcontentsline{loh}{figure}{米 \dpy{mi3}}

\begin{EntryWithPhonetic}{米}{mi3}{6}{⽶}[HSK 2,3][Kangxi 119]
  \definition*{s.}{Sobrenome: Mi}
  \definition{clas.}{m, metro; unidade principal de comprimento do sistema métrico}
  \definition[粒,斤]{s.}{arroz | sementes descascadas; refere-se a sementes comestíveis descascadas ou sem casca | qualquer coisa que se assemelhe a um grão de arroz}
\end{EntryWithPhonetic}

\begin{EntryWithPhonetic}{米饭}{mi3fan4}{6,7}{⽶,⾷}[HSK 1]
  \definition{s.}{arroz (cozido)}
\end{EntryWithPhonetic}

%%%%%%%%%% 秘 %%%%%%%%%%
\subsection*{秘}\addcontentsline{loh}{figure}{秘 \dpy{mi4}}

\begin{EntryWithPhonetic}{秘}{mi4}{10}{⽲}
  \definition{adj.}{secreto; misterioso | raro; raramente visto; estranho}
  \definition{adv.}{secretamente; privadamente}
  \definition{s.}{secretário}
  \definition{v.}{manter algo em segredo; esconder algo; guardar segredos | bloquear; obstruir; ter dificuldade para defecar}
  \seeref{bi4}
\end{EntryWithPhonetic}

\begin{EntryWithPhonetic}{秘方}{mi4fang1}{10,4}{⽲,⽅}[HSK 7-9]
  \definition{s.}{receita secreta; prescrição secreta; prescrições não divulgadas com efeitos médicos significativos}
\end{EntryWithPhonetic}

\begin{EntryWithPhonetic}{秘诀}{mi4jue2}{10,6}{⽲,⾔}[HSK 7-9]
  \definition[个,条]{s.}{código mágico; segredo (do sucesso); uma boa maneira de resolver o problema sem torná-lo público}
\end{EntryWithPhonetic}

\begin{EntryWithPhonetic}{秘密}{mi4mi4}{10,11}{⽲,⼧}[HSK 4]
  \definition{adj.}{secreto}
  \definition[个,条,些]{s.}{segredo; algo secreto; coisas que você não quer que as pessoas saibam}
\end{EntryWithPhonetic}

\begin{EntryWithPhonetic}{秘书}{mi4shu1}{10,4}{⽲,⼄}[HSK 4]
  \definition[个,位,名]{s.}{o cargo de secretário; funções de secretariado | secretário; pessoas encarregadas da correspondência e que auxiliam o chefe do órgão ou departamento na condução diária de seu trabalho}
\end{EntryWithPhonetic}

\begin{EntryWithPhonetic}{秘书长}{mi4 shu1 zhang3}{10,4,4}{⽲,⼄,⾧}[HSK 6]
  \definition{s.}{secretário-geral}
\end{EntryWithPhonetic}

%%%%%%%%%% 密 %%%%%%%%%%
\subsection*{密}\addcontentsline{loh}{figure}{密 \dpy{mi4}}

\begin{EntryWithPhonetic}{密}{mi4}{11}{⼧}[HSK 4]
  \definition*{s.}{Sobrenome: Mi}
  \definition{adj.}{fechado; denso; espesso | íntimo; próximo; afetuoso | delicado; fino; cuidadoso; meticuloso}
  \definition{adv.}{secretamente}
  \definition{s.}{segredo | densidade | senha; \emph{password}}
\end{EntryWithPhonetic}

\begin{EntryWithPhonetic}{密不可分}{mi4bu4ke3fen1}{11,4,5,4}{⼧,⼀,⼝,⼑}[HSK 7-9]
  \definition{expr.}{inextricavelmente ligados (expressão idiomática) | inseparáveis}
\end{EntryWithPhonetic}

\begin{EntryWithPhonetic}{密度}{mi4du4}{11,9}{⼧,⼴}[HSK 7-9]
  \definition[口]{s.}{densidade; espessura | Física: densidade; a gravidade específica é a razão entre a massa de um objeto e seu volume; anteriormente, era conhecida como densidade específica}
\end{EntryWithPhonetic}

\begin{EntryWithPhonetic}{密封}{mi4feng1}{11,9}{⼧,⼨}[HSK 7-9]
  \definition{v.}{selar; vedar; selar hermeticamente; selar completamente; lacrar rigorosamente}
\end{EntryWithPhonetic}

\begin{EntryWithPhonetic}{密集}{mi4ji2}{11,12}{⼧,⾫}[HSK 7-9]
  \definition{adj.}{denso; intensivo; concentrado}
  \definition{v.}{concentrar-se; aglomerar-se}
\end{EntryWithPhonetic}

\begin{EntryWithPhonetic}{密码}{mi4ma3}{11,8}{⼧,⽯}[HSK 4]
  \definition[个,种]{s.}{código; senha; um código secreto especialmente formulado usado entre as partes acordadas (diferente do 明码)}
  \seealsoref{明码}{ming2ma3}
\end{EntryWithPhonetic}

\begin{EntryWithPhonetic}{密切}{mi4qie4}{11,4}{⼧,⼑}[HSK 4]
  \definition{adj.}{próximo; íntimo; relacionamento próximo}
  \definition{adv.}{cuidadosamente; atentamente; intimamente}
  \definition{v.}{tornar-se próximo; tornar-se íntimo; conectar-se}
\end{EntryWithPhonetic}

%%%%%%%%%% 蜜 %%%%%%%%%%
\subsection*{蜜}\addcontentsline{loh}{figure}{蜜 \dpy{mi4}}

\begin{EntryWithPhonetic}{蜜}{mi4}{14}{⾍}[HSK 7-9]
  \definition{adj.}{melado; doce}
  \definition{s.}{mel | semelhante ao mel | coisas parecidas com mel; melaço}
\end{EntryWithPhonetic}

\begin{EntryWithPhonetic}{蜜蜂}{mi4feng1}{14,13}{⾍,⾍}[HSK 7-9]
  \definition[只,群,箱,窝]{s.}{abelha; abelha-melífera}
\end{EntryWithPhonetic}

\begin{EntryWithPhonetic}{蜜桃}{mi4tao2}{14,10}{⾍,⽊}
  \definition{s.}{nectarina | pêssego | pêssego suculento}
\end{EntryWithPhonetic}

\begin{EntryWithPhonetic}{蜜月}{mi4yue4}{14,4}{⾍,⽉}[HSK 7-9]
  \definition{s.}{lua de mel; o primeiro mês após o casamento}
\end{EntryWithPhonetic}

%%%%%%%%%% 棉 %%%%%%%%%%
\subsection*{棉}\addcontentsline{loh}{figure}{棉 \dpy{mian2}}

\begin{EntryWithPhonetic}{棉}{mian2}{12}{⽊}
  \definition{adj.}{almofadado com algodão; acolchoado}
  \definition[些,种,类]{s.}{termo genérico para algodão ou paina | algodão | material semelhante ao algodão | acolchoado ou estofado de algodão}
\end{EntryWithPhonetic}

\begin{EntryWithPhonetic}{棉花}{mian2hua1}{12,7}{⽊,⾋}[HSK 7-9]
  \definition[团,斤,种]{s.}{algodão; as fibras dos caroços de algodão são utilizadas para fiar fios, enchimento de roupas e roupas de cama, etc. | algodão; o nome comum para o capim-algodão}
\end{EntryWithPhonetic}

%%%%%%%%%% 免 %%%%%%%%%%
\subsection*{免}\addcontentsline{loh}{figure}{免 \dpy{mian3}}

\begin{EntryWithPhonetic}{免}{mian3}{7}{⼉}[HSK 7-9]
  \definition*{s.}{Sobrenome: Mian}
  \definition{v.}{desculpar alguém de algo; isentar; dispensar; renunciar | remover do cargo; demitir | evitar; desviar; escapar | não deveria ser permitido; não precisar fazer algo | remover; livrar-se de | isentar; dispensar | não permitir}
\end{EntryWithPhonetic}

\begin{EntryWithPhonetic}{免不了}{mian3bu5liao3}{7,4,2}{⼉,⼀,⼅}[HSK 7-9]
  \definition{v.}{ser inevitável; ser fadado a ser}
\end{EntryWithPhonetic}

\begin{EntryWithPhonetic}{免除}{mian3chu2}{7,9}{⼉,⾩}[HSK 7-9]
  \definition{v.}{prevenir; evitar; afugentar | dispensar; desculpar; isentar; aliviar | imunizar; isentar}
\end{EntryWithPhonetic}

\begin{EntryWithPhonetic}{免得}{mian3de5}{7,11}{⼉,⼻}[HSK 6]
  \definition{conj.}{de modo a não; para evitar; para que não; indica evitar uma situação que não é desejável e é frequentemente usado no início da oração seguinte}
\end{EntryWithPhonetic}

\begin{EntryWithPhonetic}{免费}{mian3/fei4}{7,9}{⼉,⾙}[HSK 4]
  \definition{v.+compl.}{isentar de taxas; tonar grátis}
\end{EntryWithPhonetic}

\begin{EntryWithPhonetic}{免税}{mian3/shui4}{7,12}{⼉,⽲}
  \definition{adj.}{isento de impostos (tributação)}
  \definition{s.}{livre de impostos | isenção de impostos}
  \definition{v.+compl.}{isentar impostos}
\end{EntryWithPhonetic}

\begin{EntryWithPhonetic}{免疫}{mian3yi4}{7,9}{⼉,⽧}[HSK 7-9]
  \definition{adj.}{imune; imunológico}
  \definition{s.}{imunidade; por possuírem resistência a certas doenças infecciosas, são imunes a elas; existem dois tipos: imunidade inata e imunidade adquirida}
\end{EntryWithPhonetic}

\begin{EntryWithPhonetic}{免职}{mian3/zhi2}{7,11}{⼉,⽿}[HSK 7-9]
  \definition{v.+compl.}{remover alguém do cargo | destituir alguém do seu cargo | demitir alguém do seu cargo}
\end{EntryWithPhonetic}

%%%%%%%%%% 勉 %%%%%%%%%%
\subsection*{勉}\addcontentsline{loh}{figure}{勉 \dpy{mian3}}

\begin{EntryWithPhonetic}{勉}{mian3}{9}{⼒}
  \definition{s.}{Sobrenome: Mian}
  \definition{v.}{esforçar-se; esforçar-se para; lutar | encorajar; instar; exortar | forçar-se a fazer algo; esforçar-se para fazer o que está além de suas capacidades; fazer algo contra a vontade; esforçar-se para trabalhar arduamente; empenhar-se em realizar trabalhos árduos; dar o seu melhor, mesmo que não tenha forças suficientes}
\end{EntryWithPhonetic}

\begin{EntryWithPhonetic}{勉强}{mian3qiang3}{9,12}{⼒,⼸}[HSK 7-9]
  \definition{adj.}{relutante; indisposto; a contragosto; não muito disposto | inadequado; pouco convincente; forçado; inverossímil | mal o suficiente; mal dá para\dots}[这些钱勉强够用一周的。===Esse dinheiro mal dá para uma semana.]
  \definition{v.}{empurrar; forçar alguém a fazer algo; obrigar alguém a fazer algo que não quer fazer}
\end{EntryWithPhonetic}

%%%%%%%%%% 缅 %%%%%%%%%%
\subsection*{缅}\addcontentsline{loh}{figure}{缅 \dpy{mian3}}

\begin{EntryWithPhonetic}{缅}{mian3}{12}{⽷}
  \definition*{s.}{Mianmar (antiga Birmânia), abreviação de 缅甸}
  \definition{adj.}{remoto; muito distante | detalhado}
  \definition{s.}{Literário: filamento fino}
  \definition{v.}{Dialeto: enrolar; virar | ponderar; refletir}
  \seealsoref{缅甸}{mian3dian4}
\end{EntryWithPhonetic}

\begin{EntryWithPhonetic}{缅甸}{mian3dian4}{12,7}{⽷,⽥}
  \definition*{s.}{Birmânia; Mianmar}
\end{EntryWithPhonetic}

\begin{EntryWithPhonetic}{缅怀}{mian3huai2}{12,7}{⽷,⼼}[HSK 7-9]
  \definition{v.}{recordar (atos passados); guardar com carinho a memória de}
\end{EntryWithPhonetic}

%%%%%%%%%% 靣 %%%%%%%%%%
\subsection*{靣}\addcontentsline{loh}{figure}{靣 \dpy{mian4}}

\begin{EntryWithPhonetic}{靣}{mian4}{8}{⼀}[Kangxi 176]
  \variantof{面}
\end{EntryWithPhonetic}

%%%%%%%%%% 面 %%%%%%%%%%
\subsection*{面}\addcontentsline{loh}{figure}{面 \dpy{mian4}}

\begin{EntryWithPhonetic}{面}{mian4}{9}{⾯}[HSK 2][Kangxi 176]
  \definition*{s.}{Sobrenome: Mian}
  \definition{adj.}{macio e farinhento; descreve algo que é muito macio ao comer | superficial}
  \definition{adv.}{diretamente; pessoalmente; na frente de alguém; cara a cara}
  \definition{clas.}{usado para objetos planos | usado para indicar o número de vezes que as pessoas se encontram}
  \definition[斤,两,碗]{s.}{face; parte frontal da cabeça; rosto | topo; superfície | capa; exterior; a parte externa de um objeto ou a face frontal de um tecido (em oposição à 里) | (matemática) superfície | cara; sentimento; emoção | geral; área total; abrangente; toda a região | lado; aspecto | escopo; escala; extensão; alcance; âmbito | farinha; farinha de trigo | pó; algo em pó | macarrão; \emph{noodle}}
  \definition{suf.}{sufixo para localização ou direção; anexado ao final de palavras que indicam localização, equivalente a 边}
  \definition{v.}{encarar algo | encontrar; revelar-se}
  \seealsoref{边}{bian1}
  \seealsoref{里}{li3}
\end{EntryWithPhonetic}

\begin{EntryWithPhonetic}{面包}{mian4bao1}{9,5}{⾯,⼓}[HSK 1]
  \definition[个,片,袋,块]{s.}{pão}[我买八个面包了。===Comprei oito pães. | 他在吃两片面包。===Ele está comendo duas fatias de pão. | 我在家里带了一袋面包。===Trouxe um saco de pão para casa. | 我拿了一块面包。===Peguei um pedaço de pão.]
\end{EntryWithPhonetic}

\begin{EntryWithPhonetic}{面部}{mian4bu4}{9,10}{⾯,⾢}[HSK 7-9]
  \definition{s.}{rosto; face}
\end{EntryWithPhonetic}

\begin{EntryWithPhonetic}{面对}{mian4dui4}{9,5}{⾯,⼨}[HSK 3]
  \definition{v.}{enfrentar; defrontar; olhar para (uma pessoa ou um objeto específico) | confrontar (problema); problemas, dificuldades e outras questões que precisam ser resolvidas e que merecem atenção}
\end{EntryWithPhonetic}

\begin{EntryWithPhonetic}{面对面}{mian4 dui4 mian4}{9,5,9}{⾯,⼨,⾯}[HSK 6]
  \definition{adj./expr.}{frente a frente; cara a cara; vis"-à"-vis}
\end{EntryWithPhonetic}

\begin{EntryWithPhonetic}{面对面吃面}{mian4dui4mian4 chi1 mian4}{9,5,9,6,9}{⾯,⼨,⾯,⼝,⾯}
  \definition{expr.}{Comer macarrão cara a cara; indica que o seu estado atual, ou algumas das posições em que você está, ou algumas das coisas que você fez são muito claras}
\end{EntryWithPhonetic}

\begin{EntryWithPhonetic}{面粉}{mian4fen3}{9,10}{⾯,⽶}[HSK 7-9]
  \definition[份]{s.}{farinha; farinha de trigo}
\end{EntryWithPhonetic}

\begin{EntryWithPhonetic}{面红耳赤}{mian4hong2-er3chi4}{9,6,6,7}{⾯,⽷,⽿,⾚}[HSK 7-9]
  \definition{expr.}{ficar corado; ficar ruborizado de raiva; descreve um rosto corado devido à impaciência ou timidez}
\end{EntryWithPhonetic}

\begin{EntryWithPhonetic}{面积}{mian4ji1}{9,10}{⾯,⽲}[HSK 3]
  \definition{s.}{área (de um andar, pedaço de terreno, etc.); área de uma superfície; o tamanho de uma superfície plana ou da superfície de um objeto}
\end{EntryWithPhonetic}

\begin{EntryWithPhonetic}{面临}{mian4lin2}{9,9}{⾯,⼁}[HSK 4]
  \definition{v.}{ser confrontado com; encontrar (uma situação) na frente de}
\end{EntryWithPhonetic}

\begin{EntryWithPhonetic}{面貌}{mian4mao4}{9,14}{⾯,⾘}[HSK 5]
  \definition[种,个]{s.}{rosto; traços faciais; formato do rosto; aparência | aparência; aspecto; aparência (das coisas)}
\end{EntryWithPhonetic}

\begin{EntryWithPhonetic}{面面俱到}{mian4mian4-ju4dao4}{9,9,10,8}{⾯,⾯,⼈,⼑}[HSK 7-9]
  \definition{expr.}{cuidar de tudo; resolver tudo; contemplar todos os aspectos e não deixa nada de fora; dar atenção a todos os aspectos de uma questão}
\end{EntryWithPhonetic}

\begin{EntryWithPhonetic}{面目全非}{mian4mu4-quan2fei1}{9,5,6,8}{⾯,⽬,⼊,⾮}[HSK 7-9]
  \definition{expr.}{perder a própria identidade; uma mudança completa; tudo parece errado ou diferente; ser alterado (distorcido) a ponto de ficar irreconhecível; não ser mais como era antes; ser muito diferente do original; os originais não existem mais; a aparência das coisas mudou drasticamente (frequentemente com uma conotação negativa); mudança além do reconhecimento (frequentemente com uma conotação pejorativa)}
\end{EntryWithPhonetic}

\begin{EntryWithPhonetic}{面前}{mian4 qian2}{9,9}{⾯,⼑}[HSK 2]
  \definition{s.}{antes; na frente de; diante de}
\end{EntryWithPhonetic}

\begin{EntryWithPhonetic}{面试}{mian4 shi4}{9,8}{⾯,⾔}[HSK 4]
  \definition{v.}{entrevistar (é realizado na forma de perguntas e respostas orais presenciais)}
\end{EntryWithPhonetic}

\begin{EntryWithPhonetic}{面条}{mian4tiao2}{9,7}{⾯,⽊}
  \definition{s.}{macarrão | espaguete}
\end{EntryWithPhonetic}

\begin{EntryWithPhonetic}{面条儿}{mian4 tiao2r5}{9,7,2}{⾯,⽊,⼉}[HSK 1]
  \definition{s.}{macarrão; \emph{noodles}}
\end{EntryWithPhonetic}

\begin{EntryWithPhonetic}{面团}{mian4tuan2}{9,6}{⾯,⼞}
  \definition{s.}{massa | pasta}
\end{EntryWithPhonetic}

\begin{EntryWithPhonetic}{面向}{mian4 xiang4}{9,6}{⾯,⼝}[HSK 6]
  \definition{v.}{virar o rosto para; virar na direção de; defrontar; voltado para algum lugar | estar orientado para as necessidades de; atender a; principalmente para um certo tipo de pessoas}
\end{EntryWithPhonetic}

\begin{EntryWithPhonetic}{面子}{mian4zi5}{9,3}{⾯,⼦}[HSK 5]
  \definition{s.}{face; exterior; parte externa; superfície do objeto | imagem; reputação; prestígio; decência; vaidade superficial | sentimentos; sensibilidades | pó}
\end{EntryWithPhonetic}

%%%%%%%%%% 糆 %%%%%%%%%%
\subsection*{糆}\addcontentsline{loh}{figure}{糆 \dpy{mian4}}

\begin{EntryWithPhonetic}{糆}{mian4}{15}{⽶}
  \variantof{面}
\end{EntryWithPhonetic}

%%%%%%%%%% 麫 %%%%%%%%%%
\subsection*{麫}\addcontentsline{loh}{figure}{麫 \dpy{mian4}}

\begin{EntryWithPhonetic}{麫}{mian4}{15}{⿆}
  \variantof{面}
\end{EntryWithPhonetic}

%%%%%%%%%% 苗 %%%%%%%%%%
\subsection*{苗}\addcontentsline{loh}{figure}{苗 \dpy{miao2}}

\begin{EntryWithPhonetic}{苗}{miao2}{8}{⾋}[HSK 7-9]
  \definition*{s.}{Miao, grupo étnico, abreviação de 苗族 | Sobrenome: Miao}
  \definition[棵,株,尾,头,些]{s.}{planta jovem; muda; broto; plantas recém-germinadas | descendente; filho | filhotes de alguns animais; animais domésticos recém-nascidos | vacina | algo que se assemelha a uma planta jovem}
  \seealsoref{苗族}{miao2zu2}
\end{EntryWithPhonetic}

\begin{EntryWithPhonetic}{苗条}{miao2tiao5}{8,7}{⾋,⽊}[HSK 7-9]
  \definition{adj.}{(figura feminina) magra; esbelta; esbelta e graciosa}
\end{EntryWithPhonetic}

\begin{EntryWithPhonetic}{苗头}{miao2tou5}{8,5}{⾋,⼤}[HSK 7-9]
  \definition{s.}{sintoma de uma tendência; indício de um novo desenvolvimento; uma tendência ou situação de desenvolvimento ligeiramente emergente}
\end{EntryWithPhonetic}

\begin{EntryWithPhonetic}{苗族}{miao2zu2}{8,11}{⾋,⽅}
  \definition*{s.}{Grupo étnico Hmong ou Miao do sudoeste da China; uma das minorias étnicas da China, distribuída por Guizhou 贵州, Hunan 湖南, Yunnan 云南, Guangxi 广西, Sichuan 四川, Guangdong 广东 e Hubei 湖北}
  \seealsoref{广东}{guang3dong1}
  \seealsoref{广西}{guang3xi1}
  \seealsoref{贵州}{gui4zhou1}
  \seealsoref{湖北}{hu2bei3}
  \seealsoref{湖南}{hu2nan2}
  \seealsoref{四川}{si4chuan1}
  \seealsoref{云南}{yun2nan2}
\end{EntryWithPhonetic}

%%%%%%%%%% 描 %%%%%%%%%%
\subsection*{描}\addcontentsline{loh}{figure}{描 \dpy{miao2}}

\begin{EntryWithPhonetic}{描}{miao2}{11}{⼿}
  \definition{v.}{traçar; copiar | retocar; retocar | traçar um desenho | retratar | esboçar}
\end{EntryWithPhonetic}

\begin{EntryWithPhonetic}{描绘}{miao2hui4}{11,9}{⼿,⽷}[HSK 7-9]
  \definition{v.}{descrever; retratar; representar; desenhar}
\end{EntryWithPhonetic}

\begin{EntryWithPhonetic}{描述}{miao2 shu4}{11,8}{⼿,⾡}[HSK 4]
  \definition[段,种]{s.}{descrição; trecho que descreve um evento ou uma cena}
  \definition{v.}{descrever; representar}
\end{EntryWithPhonetic}

\begin{EntryWithPhonetic}{描写}{miao2xie3}{11,5}{⼿,⼍}[HSK 4]
  \definition{v.}{representar; retratar; descrever; usar a linguagem e as palavras para transmitir uma imagem concreta de uma pessoa, evento ou situação}
\end{EntryWithPhonetic}

%%%%%%%%%% 瞄 %%%%%%%%%%
\subsection*{瞄}\addcontentsline{loh}{figure}{瞄 \dpy{miao2}}

\begin{EntryWithPhonetic}{瞄}{miao2}{13}{⽬}
  \definition{v.}{concentrar o olhar em; mirar | olhar fixamente para; mirar em; prestar atenção em | olhar; dar uma olhada rápida}
\end{EntryWithPhonetic}

\begin{EntryWithPhonetic}{瞄准}{miao2/zhun3}{13,10}{⽬,⼎}[HSK 7-9]
  \definition{v.+compl.}{mirar; apontar para}
\end{EntryWithPhonetic}

%%%%%%%%%% 秒 %%%%%%%%%%
\subsection*{秒}\addcontentsline{loh}{figure}{秒 \dpy{miao3}}

\begin{EntryWithPhonetic}{秒}{miao3}{9}{⽲}[HSK 5]
  \definition{adv.}{instantaneamente}
  \definition{s.}{segundo (unidade de tempo) | segundo (unidade de medida angular)}
\end{EntryWithPhonetic}

%%%%%%%%%% 渺 %%%%%%%%%%
\subsection*{渺}\addcontentsline{loh}{figure}{渺 \dpy{miao3}}

\begin{EntryWithPhonetic}{渺}{miao3}{12}{⽔}
  \definition{adj.}{(uma vasta extensão de água) vasta; que se estende ao longe | distante e indistinto; vago | minúsculo; insignificante}
\end{EntryWithPhonetic}

\begin{EntryWithPhonetic}{渺小}{miao3xiao3}{12,3}{⽔,⼩}[HSK 7-9]
  \definition{adj.}{minúsculo; reles; desprezível; insignificante}
\end{EntryWithPhonetic}

%%%%%%%%%% 妙 %%%%%%%%%%
\subsection*{妙}\addcontentsline{loh}{figure}{妙 \dpy{miao4}}

\begin{EntryWithPhonetic}{妙}{miao4}{7}{⼥}[HSK 6]
  \definition*{s.}{Sobrenome: Miao}
  \definition{adj.}{maravilhoso; excelente; bom | engenhoso; esperto; sutil | extraordinário | requintado; mágico; engenhoso; misterioso}
\end{EntryWithPhonetic}

\begin{EntryWithPhonetic}{妙招}{miao4zhao1}{7,8}{⼥,⼿}
  \definition{adj.}{escorregadio}
  \definition{s.}{movimento inteligente | maneira inteligente de fazer algo}
\end{EntryWithPhonetic}

%%%%%%%%%% 庙 %%%%%%%%%%
\subsection*{庙}\addcontentsline{loh}{figure}{庙 \dpy{miao4}}

\begin{EntryWithPhonetic}{庙}{miao4}{8}{⼴}[HSK 7-9]
  \definition[座,个,间]{s.}{templo; santuário | feira do templo | Literário: corte imperial; corte real | Literário: imperador falecido | casa de incenso; locais onde, no passado, foram consagradas tábuas ancestrais, divindades ou figuras históricas}
\end{EntryWithPhonetic}

\begin{EntryWithPhonetic}{庙会}{miao4hui4}{8,6}{⼴,⼈}[HSK 7-9]
  \definition{s.}{feira; feira do templo; festival feira do templo; mercados montados em templos ou próximos a eles; geralmente realizados em festivais ou dias específicos}
\end{EntryWithPhonetic}

%%%%%%%%%% 灭 %%%%%%%%%%
\subsection*{灭}\addcontentsline{loh}{figure}{灭 \dpy{mie4}}

\begin{EntryWithPhonetic}{灭}{mie4}{5}{⽕}[HSK 6]
  \definition{v.}{extinguir-se | extinguir; apagar; desligar | afogar; inundar; submergir | perecer; destruir | exterminar; apagar; acabar com; tornar inexistente}
\end{EntryWithPhonetic}

\begin{EntryWithPhonetic}{灭火}{mie4huo3}{5,4}{⽕,⽕}
  \definition{s.}{combate a incêndios}
  \definition{v.}{extinguir um incêndio}
\end{EntryWithPhonetic}

\begin{EntryWithPhonetic}{灭绝}{mie4jue2}{5,9}{⽕,⽷}[HSK 7-9]
  \definition{v.}{extinguir-se; ser extinto; eliminar completamente | perder completamente; perder totalmente}
\end{EntryWithPhonetic}

\begin{EntryWithPhonetic}{灭亡}{mie4wang2}{5,3}{⽕,⼇}[HSK 7-9]
  \definition{v.}{ser destruído; extinguir-se; perecer; desaparecer; isso se refere à eliminação de uma nação, grupo étnico ou grupo político, que deixa de existir | destruir; exterminar}
\end{EntryWithPhonetic}

%%%%%%%%%% 民 %%%%%%%%%%
\subsection*{民}\addcontentsline{loh}{figure}{民 \dpy{min2}}

\begin{EntryWithPhonetic}{民}{min2}{5}{⽒}
  \definition*{s.}{Sobrenome: Min}
  \definition{adj.}{folclórico ; civil (não militar)}
  \definition{s.}{pessoa | membro de um grupo étnico | uma pessoa de uma determinada ocupação | do povo; folclore | civil; cidadão | o povo | um membro de uma nacionalidade}
\end{EntryWithPhonetic}

\begin{EntryWithPhonetic}{民办}{min2ban4}{5,4}{⽒,⼒}[HSK 7-9]
  \definition{adj.}{administrado pela população local; administrado por civis (escola) | privado; administrado de forma privada (oposto de 公办)}
  \seealsoref{公办}{gong1ban4}
\end{EntryWithPhonetic}

\begin{EntryWithPhonetic}{民歌}{min2 ge1}{5,14}{⽒,⽋}[HSK 6]
  \definition[支,首]{s.}{canção folclórica; os nomes dos autores das canções transmitidas oralmente são muitas vezes desconhecidos}
\end{EntryWithPhonetic}

\begin{EntryWithPhonetic}{民工}{min2 gong1}{5,3}{⽒,⼯}[HSK 6]
  \definition{s.}{trabalhador trabalhando em um projeto público | trabalhador temporário alistado em um projeto público | agricultor que trabalha em empregos temporários na cidade | trabalhador migrante}
\end{EntryWithPhonetic}

\begin{EntryWithPhonetic}{民间}{min2jian1}{5,7}{⽒,⾨}[HSK 3]
  \definition{s.}{entre o povo | não governamental; de pessoa para pessoa}
\end{EntryWithPhonetic}

\begin{EntryWithPhonetic}{民警}{min2 jing3}{5,19}{⽒,⾔}[HSK 6]
  \definition{s.}{polícia; policial}
\end{EntryWithPhonetic}

\begin{EntryWithPhonetic}{民俗}{min2su2}{5,9}{⽒,⼈}[HSK 7-9]
  \definition{s.}{costumes populares; tradição popular}
\end{EntryWithPhonetic}

\begin{EntryWithPhonetic}{民意}{min2 yi4}{5,13}{⽒,⼼}[HSK 6]
  \definition{s.}{vontade do povo; vontade popular | opinião pública}
\end{EntryWithPhonetic}

\begin{EntryWithPhonetic}{民用}{min2yong4}{5,5}{⽒,⽤}[HSK 7-9]
  \definition{adj.}{para uso civil; civil | usado no dia a dia das pessoas}
\end{EntryWithPhonetic}

\begin{EntryWithPhonetic}{民众}{min2zhong4}{5,6}{⽒,⼈}[HSK 7-9]
  \definition{s.}{a população; o povo comum; as massas populares}
\end{EntryWithPhonetic}

\begin{EntryWithPhonetic}{民主}{min2zhu3}{5,5}{⽒,⼂}[HSK 6]
  \definition{adj.}{democrático; em consonância com os princípios democráticos}
  \definition[个]{s.}{democracia; direitos democráticos; refere-se ao direito do povo de participar da vida política e dos assuntos do Estado e de expressar livremente suas opiniões}
\end{EntryWithPhonetic}

\begin{EntryWithPhonetic}{民族}{min2zu2}{5,11}{⽒,⽅}[HSK 3]
  \definition[个]{s.}{nação; uma comunidade estável formada ao longo da história pela humanidade, com uma língua comum, uma região comum, uma vida econômica comum e uma mentalidade comum expressa em uma cultura comum | grupo étnico; refere-se, de maneira geral, às comunidades formadas ao longo da história por pessoas em diferentes estágios de desenvolvimento social}
\end{EntryWithPhonetic}

%%%%%%%%%% 敏 %%%%%%%%%%
\subsection*{敏}\addcontentsline{loh}{figure}{敏 \dpy{min3}}

\begin{EntryWithPhonetic}{敏}{min3}{11}{⽁}
  \definition*{s.}{Sobrenome: Min}
  \definition{adj.}{rápido; ágil | perspicaz; inteligente; rápido | inteligente; esperto}
\end{EntryWithPhonetic}

\begin{EntryWithPhonetic}{敏感}{min3gan3}{11,13}{⽁,⼼}[HSK 5]
  \definition{adj.}{sensível; descreve pessoas ou animais que rapidamente percebem mudanças ou estímulos externos | reativo; sensível; fácil de causar reações intensas}
\end{EntryWithPhonetic}

\begin{EntryWithPhonetic}{敏捷}{min3jie2}{11,11}{⽁,⼿}[HSK 7-9]
  \definition{adj.}{ágil; rápido; descreve reações rápidas em ações, pensamentos, etc.}
\end{EntryWithPhonetic}

\begin{EntryWithPhonetic}{敏锐}{min3rui4}{11,12}{⽁,⾦}[HSK 7-9]
  \definition{adj.}{agudo; perspicaz; aguçado; (pensamento) rápido de raciocínio, (intuição) aguçado}
\end{EntryWithPhonetic}

%%%%%%%%%% 名 %%%%%%%%%%
\subsection*{名}\addcontentsline{loh}{figure}{名 \dpy{ming2}}

\begin{EntryWithPhonetic}{名}{ming2}{6}{⼝}[HSK 2]
  \definition*{s.}{Sobrenome: Ming}
  \definition{adj.}{notável; famoso; conhecido; renomado}
  \definition{clas.}{usado para pessoas | usado para classificação por ordem}
  \definition{s.}{nome; denominação | desculpa; pretexto | fama; reputação}
  \definition{v.}{nome próprio (é) | expressar; descrever | possuir; tomar; ter}
\end{EntryWithPhonetic}

\begin{EntryWithPhonetic}{名称}{ming2 cheng1}{6,10}{⼝,⽲}[HSK 2]
  \definition[个,种]{s.}{nomes, apelidos e formas de se referir a pessoas ou coisas}
\end{EntryWithPhonetic}

\begin{EntryWithPhonetic}{名单}{ming2 dan1}{6,8}{⼝,⼗}[HSK 2]
  \definition[个,份]{s.}{lista com nomes de pessoas ou nomes de organizações}
\end{EntryWithPhonetic}

\begin{EntryWithPhonetic}{名额}{ming2'e2}{6,15}{⼝,⾴}[HSK 6]
  \definition[个]{s.}{cota de pessoas; número de pessoas designadas ou permitidas; número necessário de pessoal}
\end{EntryWithPhonetic}

\begin{EntryWithPhonetic}{名副其实}{ming2fu4qi2shi2}{6,11,8,8}{⼝,⼑,⼋,⼧}[HSK 7-9]
  \definition{expr.}{``Faz jus ao seu nome.''; ser digno do nome; ser algo na realidade, bem como no nome; ser digno da própria reputação; em nome e de fato; no verdadeiro sentido do termo; a reputação de alguém é justificada; o nome corresponde à realidade; merecer verdadeiramente o seu nome; fiel ao próprio nome}
\end{EntryWithPhonetic}

\begin{EntryWithPhonetic}{名贵}{ming2gui4}{6,9}{⼝,⾙}[HSK 7-9]
  \definition{adj.}{famoso e precioso; raro}
\end{EntryWithPhonetic}

\begin{EntryWithPhonetic}{名利}{ming2li4}{6,7}{⼝,⼑}[HSK 7-9]
  \definition{s.}{fama e ganho; fama e riqueza | fama e dinheiro; refere-se ao status e aos interesses de um indivíduo}
\end{EntryWithPhonetic}

\begin{EntryWithPhonetic}{名牌儿}{ming2 pai2r5}{6,12,2}{⼝,⽚,⼉}[HSK 4]
  \definition{s.}{marca famosa}
\end{EntryWithPhonetic}

\begin{EntryWithPhonetic}{名片}{ming2pian4}{6,4}{⼝,⽚}[HSK 4]
  \definition[张,盒,叠]{s.}{cartão de visita; um pedaço de papel retangular com o nome, o cargo, o endereço etc. impressos}
\end{EntryWithPhonetic}

\begin{EntryWithPhonetic}{名气}{ming2qi5}{6,4}{⼝,⽓}[HSK 7-9]
  \definition{s.}{nome; fama; reputação}
\end{EntryWithPhonetic}

\begin{EntryWithPhonetic}{名人}{ming2 ren2}{6,2}{⼝,⼈}[HSK 4]
  \definition[位,个]{s.}{celebridade; pessoa famosa}
\end{EntryWithPhonetic}

\begin{EntryWithPhonetic}{名声}{ming2sheng1}{6,7}{⼝,⼠}[HSK 7-9]
  \definition{s.}{reputação; renome; prestígio; comentários amplamente divulgados na sociedade}
\end{EntryWithPhonetic}

\begin{EntryWithPhonetic}{名胜}{ming2 sheng4}{6,9}{⼝,⾁}[HSK 6]
  \definition[处,个]{s.}{pontos turísticos; atrações famosas; lugares famosos com locais históricos ou belas paisagens}
\end{EntryWithPhonetic}

\begin{EntryWithPhonetic}{名言}{ming2yan2}{6,7}{⼝,⾔}[HSK 7-9]
  \definition{s.}{ditados; comentário famoso; frases ou expressões famosas}
\end{EntryWithPhonetic}

\begin{EntryWithPhonetic}{名义}{ming2 yi4}{6,3}{⼝,⼂}[HSK 6]
  \definition{s.}{nominal; em nome (geralmente seguido por 上); um nome ou título usado como base para fazer algo}[有人盗用我名义申请贷款。===Alguém solicitou um empréstimo em meu nome. | 他们只是名义上的夫妻。===Eles são marido e mulher apenas no nome.]
  \seealsoref{上}{shang4}
\end{EntryWithPhonetic}

\begin{EntryWithPhonetic}{名誉}{ming2yu4}{6,13}{⼝,⾔}[HSK 6]
  \definition{adj.}{honorário; nominal (geralmente se refere ao nome de um presente, com um sentido de respeito)}[他是学校的名誉教授。===Ele é professor honorário da escola.]
  \definition{s.}{fama; reputação; honra}[名誉才是最神圣的。===Reputação é a coisa mais sagrada. | 我用自己的名誉发誓。===Juro pela minha honra.]
\end{EntryWithPhonetic}

\begin{EntryWithPhonetic}{名著}{ming2zhu4}{6,11}{⼝,⽬}[HSK 7-9]
  \definition{s.}{clássico; livro famoso; obra famosa; obra-prima; obras valiosas e famosas}
\end{EntryWithPhonetic}

\begin{EntryWithPhonetic}{名字}{ming2zi5}{6,6}{⼝,⼦}[HSK 1]
  \definition[个]{s.}{nome; nome próprio | nome (de uma coisa)}
\end{EntryWithPhonetic}

%%%%%%%%%% 明 %%%%%%%%%%
\subsection*{明}\addcontentsline{loh}{figure}{明 \dpy{ming2}}

\begin{EntryWithPhonetic}{明}{ming2}{8}{⽇}
  \definition*{s.}{Dinastia Ming (1368-1644) | Sobrenome: Ming}
  \definition{adj.}{claro; brilhante; brilhante | claro; distinto; de fácil entendimento | aberto; evidente; explícito; exposto | de ​​olhos aguçados; boa visão; visão nítida | honesto}
  \definition{adv.}{claramente; definitivamente; aparentemente; de fato}
  \definition{s.}{imediatamente a seguir no tempo; ao lado deste ano e hoje; visão}
  \definition{v.}{mostrar; revelar; tornar conhecido; deixar claro | entender; compreender}
\end{EntryWithPhonetic}

\begin{EntryWithPhonetic}{明白}{ming2bai5}{8,5}{⽇,⽩}[HSK 1]
  \definition{adj.}{claro; óbvio; evidente; inequívoco | sensato; razoável | aberto; franco; inequívoco; explícito}
  \definition{v.}{entender; compreender; saber}
\end{EntryWithPhonetic}

\begin{EntryWithPhonetic}{明朗}{ming2lang3}{8,10}{⽇,⽉}[HSK 7-9]
  \definition{adj.}{brilhante e claro; bem iluminado (geralmente se referindo a ambientes externos) | claro; óbvio; inequívoco | franco; direto; alegre e otimista; íntegro e honesto; (pensamentos, mentalidade, caráter, etc.) otimista, alegre e não melancólico ou deprimido}
\end{EntryWithPhonetic}

\begin{EntryWithPhonetic}{明亮}{ming2 liang4}{8,9}{⽇,⼇}[HSK 5]
  \definition{adj.}{claro; bem iluminado | brilhante; resplandecente | claro; simples; compreensível}
\end{EntryWithPhonetic}

\begin{EntryWithPhonetic}{明码}{ming2ma3}{8,8}{⽇,⽯}
  \definition{s.}{código simples, em claro (oposto a 密码) | preço claramente marcado}
  \seealsoref{密码}{mi4ma3}
\end{EntryWithPhonetic}

\begin{EntryWithPhonetic}{明媚}{ming2mei4}{8,12}{⽇,⼥}[HSK 7-9]
  \definition{adj.}{brilhante e belo; radiante e encantador}
\end{EntryWithPhonetic}

\begin{EntryWithPhonetic}{明明}{ming2ming2}{8,8}{⽇,⽇}[HSK 5]
  \definition{adv.}{obviamente; claramente; sem dúvida; indica que o fenômeno ou princípio é evidente}
\end{EntryWithPhonetic}

\begin{EntryWithPhonetic}{明年}{ming2 nian2}{8,6}{⽇,⼲}[HSK 1]
  \definition{s.}{próximo ano}
\end{EntryWithPhonetic}

\begin{EntryWithPhonetic}{明确}{ming2que4}{8,12}{⽇,⽯}[HSK 3]
  \definition{adj.}{claro; definido; específico}
  \definition{v.}{deixar claro; tornar definitivo; tornar um ponto de vista, uma tarefa, etc. claro, compreensível e definitivo}
\end{EntryWithPhonetic}

\begin{EntryWithPhonetic}{明日}{ming2 ri4}{8,4}{⽇,⽇}[HSK 6]
  \definition{s.}{amanhã}
  \seealsoref{明天}{ming2tian1}
\end{EntryWithPhonetic}

\begin{EntryWithPhonetic}{明天}{ming2tian1}{8,4}{⽇,⼤}[HSK 1]
  \definition{s.}{amanhã | futuro próximo}
\end{EntryWithPhonetic}

\begin{EntryWithPhonetic}{明显}{ming2xian3}{8,9}{⽇,⽇}[HSK 3]
  \definition{adj.}{claro; óbvio; distinto; claramente visível}
\end{EntryWithPhonetic}

\begin{EntryWithPhonetic}{明星}{ming2xing1}{8,9}{⽇,⽇}[HSK 2]
  \definition[个,位,颗,名]{s.}{estrela; ator, atleta, cantor famosos, etc. | talento de ponta; profissional de destaque; também é usado como metáfora para pessoas ou grupos que se destacam pelo seu bom desempenho ou excelência | estrela brilhante; estrela resplandecente; referindo-se a estrelas muito brilhantes}
\end{EntryWithPhonetic}

\begin{EntryWithPhonetic}{明智}{ming2zhi4}{8,12}{⽇,⽇}[HSK 7-9]
  \definition{adj.}{sábio; sensato; sagaz; previdente; ponderado}
\end{EntryWithPhonetic}

\begin{EntryWithPhonetic}{明珠}{ming2zhu1}{8,10}{⽇,⽟}
  \definition{s.}{pérola | jóia (de grande valor)}
\end{EntryWithPhonetic}

%%%%%%%%%% 鸣 %%%%%%%%%%
\subsection*{鸣}\addcontentsline{loh}{figure}{鸣 \dpy{ming2}}

\begin{EntryWithPhonetic}{鸣}{ming2}{8}{⿃}
  \definition{v.}{chorar (pássaros, animais e insetos) | fazer um som | dar voz (gratidão, queixas, etc.)}
\end{EntryWithPhonetic}

%%%%%%%%%% 铭 %%%%%%%%%%
\subsection*{铭}\addcontentsline{loh}{figure}{铭 \dpy{ming2}}

\begin{EntryWithPhonetic}{铭}{ming2}{11}{⾦}
  \definition*{s.}{Sobrenome: Ming}
  \definition{s.}{inscrição; textos antigos fundidos ou gravados em objetos e estelas para registrar fatos, realizações ou para servir de advertência}
  \definition{v.}{gravar; lembrar; inscrever; inscrever textos comemorativos em objetos; uma metáfora para recordar profundamente}
\end{EntryWithPhonetic}

\begin{EntryWithPhonetic}{铭记}{ming2ji4}{11,5}{⾦,⾔}[HSK 7-9]
  \definition{v.}{lembrar sempre; gravar na mente; guardar algo na memória com muito carinho}
\end{EntryWithPhonetic}

%%%%%%%%%% 盟 %%%%%%%%%%
\subsection*{盟}\addcontentsline{loh}{figure}{盟 \dpy{ming2}}

\begin{EntryWithPhonetic}{盟}{ming2}{13}{⽫}
  \definition{v.}{jurar; prometer; fazer um juramento}
  \seeref{meng2}
\end{EntryWithPhonetic}

%%%%%%%%%% 命 %%%%%%%%%%
\subsection*{命}\addcontentsline{loh}{figure}{命 \dpy{ming4}}

\begin{EntryWithPhonetic}{命}{ming4}{8}{⼝}[HSK 6-9]
  \definition[条]{s.}{vida | sorte; destino; fado | ordem; comando; instrução | atribuição de um nome, título etc.}
  \definition{v.}{ordenar; nomear | atribuir (um nome etc.)}
\end{EntryWithPhonetic}

\begin{EntryWithPhonetic}{命令}{ming4ling4}{8,5}{⼝,⼈}[HSK 5]
  \definition[条,项,道,个]{s.}{ordem; comando; instruções emitidas pelos superiores aos subordinados}
  \definition{v.}{ordenar; comandar}
\end{EntryWithPhonetic}

\begin{EntryWithPhonetic}{命名}{ming4/ming2}{8,6}{⼝,⼝}[HSK 7-9]
  \definition{v.+compl.}{dar nome a alguém ou a alguma coisa; atribuir um nome a; dar nome; conferir um nome; geralmente não usado para fins pessoais}
\end{EntryWithPhonetic}

\begin{EntryWithPhonetic}{命题}{ming4/ti2}{8,15}{⼝,⾴}[HSK 7-9]
  \definition[个]{s.}{proposição; afirmação; tese; em lógica, refere-se à forma linguística usada para expressar juízos}['地球是圆的' 是一个命题。===A afirmação de que ``a Terra é redonda'' é uma proposição.]
  \definition{v.}{atribuir um tema; formular uma pergunta}
\end{EntryWithPhonetic}

\begin{EntryWithPhonetic}{命运}{ming4yun4}{8,7}{⼝,⾡}[HSK 3]
  \definition[个]{s.}{tendência de desenvolvimento; tendência de futuro; metáfora para a direção e tendência do desenvolvimento e das mudanças | destino; sina; sorte; refere-se à vida e à morte, à riqueza e à pobreza e a todas as experiências da vida}
\end{EntryWithPhonetic}

%%%%%%%%%% 摸 %%%%%%%%%%
\subsection*{摸}\addcontentsline{loh}{figure}{摸 \dpy{mo1}}

\begin{EntryWithPhonetic}{摸}{mo1}{13}{⼿}[HSK 4]
  \definition{v.}{sentir; acariciar; tocar; tocar (um objeto) levemente com a mão e depois removê-lo ou mover a mão suavemente sobre a superfície do objeto | sentir para; tatear para; sentir algo com as mãos | descobrir; sentir; sondar; explorar; tentar fazer ou entender | sentir o caminho; tatear no escuro; andar por estradas que você não consegue reconhecer | furtar; roubar}
\end{EntryWithPhonetic}

\begin{EntryWithPhonetic}{摸索}{mo1suo3}{13,10}{⼿,⽷}[HSK 7-9]
  \definition{v.}{tatear; apalpar; explorar; sentir; procurar | investigar; estudar; explorar; descobrir; procurar; buscar; pesquisar}
\end{EntryWithPhonetic}

%%%%%%%%%% 模 %%%%%%%%%%
\subsection*{模}\addcontentsline{loh}{figure}{模 \dpy{mo2}}

\begin{EntryWithPhonetic}{模}{mo2}{14}{⽊}
  \definition{s.}{padrão | modelo; exemplo | modelo (pessoa) | exame simulado | módulo}
  \definition{v.}{imitar | copiar; emular}
  \seeref{mu2}
\end{EntryWithPhonetic}

\begin{EntryWithPhonetic}{模范}{mo2fan4}{14,9}{⽊,⾋}[HSK 5]
  \definition{adj.}{exemplar}
  \definition{s.}{modelo; exemplo excelente; pessoa exemplar; coisa exemplar; pessoas ou coisas exemplares que servem de modelo}
\end{EntryWithPhonetic}

\begin{EntryWithPhonetic}{模仿}{mo2fang3}{14,6}{⽊,⼈}[HSK 5]
  \definition{v.}{copiar; imitar; aprender a fazer algo seguindo um modelo pronto}
\end{EntryWithPhonetic}

\begin{EntryWithPhonetic}{模糊}{mo2hu5}{14,15}{⽊,⽶}[HSK 5]
  \definition{adj.}{vago; confuso; indistinto}
  \definition{v.}{confundir; desorientar}
\end{EntryWithPhonetic}

\begin{EntryWithPhonetic}{模拟}{mo2ni3}{14,7}{⽊,⼿}[HSK 7-9]
  \definition{v.}{ser análogo; imitar; simular; fazer de maneira formal}
\end{EntryWithPhonetic}

\begin{EntryWithPhonetic}{模式}{mo2shi4}{14,6}{⽊,⼷}[HSK 5]
  \definition{s.}{modelo; modo; padrão; a forma padrão de algo ou o modelo padrão que as pessoas podem seguir}
\end{EntryWithPhonetic}

\begin{EntryWithPhonetic}{模特儿}{mo2 te4r5}{14,10,2}{⽊,⽜,⼉}[HSK 4]
  \definition[个,名,位]{s.}{modelo (pessoa que posa para um fotógrafo ou pintor ou escultor); objeto de representação ou referência usado por artistas para esboços e esculturas, como o corpo humano, objetos, modelos etc.; também se refere aos arquétipos que os estudiosos da literatura usam para retratar seus personagens | modelo (uma pessoa que usa roupas para exibir modas); pessoa ou manequim usado para exibir estilos de roupas}
\end{EntryWithPhonetic}

\begin{EntryWithPhonetic}{模型}{mo2xing2}{14,9}{⽊,⼟}[HSK 4]
  \definition[个]{s.}{modelo; padrão; itens feitos em escala com base em objetos ou desenhos | molde; padrão; molde para fundir máquinas, objetos, etc.}
\end{EntryWithPhonetic}

%%%%%%%%%% 膜 %%%%%%%%%%
\subsection*{膜}\addcontentsline{loh}{figure}{膜 \dpy{mo2}}

\begin{EntryWithPhonetic}{膜}{mo2}{14}{⾁}[HSK 6]
  \definition[张]{s.}{membrana | filme; revestimento fino}
\end{EntryWithPhonetic}

\begin{EntryWithPhonetic}{膜拜}{mo2bai4}{14,9}{⾁,⼿}
  \definition{v.}{ajoelhar-se e se curvar com as mãos unidas no nível da testa | ter ou mostrar sentimentos fortes de respeito e admiração por um deus}
\end{EntryWithPhonetic}

%%%%%%%%%% 摩 %%%%%%%%%%
\subsection*{摩}\addcontentsline{loh}{figure}{摩 \dpy{mo2}}

\begin{EntryWithPhonetic}{摩}{mo2}{15}{⼿}
  \definition{v.}{esfregar; raspar; tocar | refletir; estudar | afagar}
\end{EntryWithPhonetic}

\begin{EntryWithPhonetic}{摩擦}{mo2ca1}{15,17}{⼿,⼿}[HSK 5]
  \definition{s.}{atrito; desacordo; conflito (entre duas partes); a ação de impedir o movimento relativo entre dois objetos em contato, produzida na superfície de contato | atrito; metáfora para o conflito entre as duas partes}
  \definition{v.}{esfregar}
\end{EntryWithPhonetic}

\begin{EntryWithPhonetic}{摩托}{mo2 tuo1}{15,6}{⼿,⼿}[HSK 5]
  \definition[辆]{s.}{Empréstimo linguístico: motor; motor de combustão interna | Empréstimo linguístico: motocicleta, abreviação de 摩托车}
  \seealsoref{摩托车}{mo2tuo1che1}
\end{EntryWithPhonetic}

\begin{EntryWithPhonetic}{摩托车}{mo2tuo1che1}{15,6,4}{⼿,⼿,⾞}
  \definition[辆,部]{s.}{(empréstimo linguístico) motocicleta}
\end{EntryWithPhonetic}

%%%%%%%%%% 磨 %%%%%%%%%%
\subsection*{磨}\addcontentsline{loh}{figure}{磨 \dpy{mo2}}

\begin{EntryWithPhonetic}{磨}{mo2}{16}{⽯}[HSK 6]
  \definition{v.}{esfregar; desgastar | moer; refletir; polir | desgastar; esgotar; cansar; exaurir | incomodar; causar problemas | destruir; obliterar; extinguir-se | ficar ocioso; perder tempo; perder tempo; procrastinar}
  \seeref{mo4}
\end{EntryWithPhonetic}

\begin{EntryWithPhonetic}{磨菇}{mo2gu5}{16,11}{⽯,⾋}
  \variantof{蘑菇}
\end{EntryWithPhonetic}

\begin{EntryWithPhonetic}{磨合}{mo2he2}{16,6}{⽯,⼝}[HSK 7-9]
  \definition{v.}{amaciar; máquinas e veículos novos ou reformados, após um período de operação, têm suas marcas de usinagem suavizadas, tornando as superfícies de fricção mais bem vedadas | (pessoas) conviver em harmonia; aprender a se dar bem; acomodar-se mutuamente}
\end{EntryWithPhonetic}

\begin{EntryWithPhonetic}{磨难}{mo2nan4}{16,10}{⽯,⾫}[HSK 7-9]
  \definition{s.}{tribulação; dificuldade; sofrimento; o tormento sofrido em circunstâncias difíceis também é chamado de tribulação}
\end{EntryWithPhonetic}

\begin{EntryWithPhonetic}{磨损}{mo2sun3}{16,10}{⽯,⼿}[HSK 7-9]
  \definition{v.}{desgastar; causar abrasão; desgastar por fricção e uso}
\end{EntryWithPhonetic}

%%%%%%%%%% 蘑 %%%%%%%%%%
\subsection*{蘑}\addcontentsline{loh}{figure}{蘑 \dpy{mo2}}

\begin{EntryWithPhonetic}{蘑}{mo2}{19}{⾋}
  \definition{s.}{cogumelo}
\end{EntryWithPhonetic}

\begin{EntryWithPhonetic}{蘑菇}{mo2gu5}{19,11}{⾋,⾋}[HSK 7-9]
  \definition[个,朵,斤,种]{s.}{cogumelo; termo genérico para fungos em forma de guarda-chuva; referindo-se especificamente a cogumelos champignon ou cogumelos shiitake}
  \definition{v.}{afligir; importunar; insistir em | demorar; enrolar; movimentar-se lenta e arrastadamente}
\end{EntryWithPhonetic}

%%%%%%%%%% 魔 %%%%%%%%%%
\subsection*{魔}\addcontentsline{loh}{figure}{魔 \dpy{mo2}}

\begin{EntryWithPhonetic}{魔}{mo2}{20}{⿁}
  \definition{adj.}{místico; misterioso; mágico}
  \definition{s.}{espírito maligno; demônio; diabo; monstro | mágico; místico}
\end{EntryWithPhonetic}

\begin{EntryWithPhonetic}{魔鬼}{mo2gui3}{20,9}{⿁,⿁}[HSK 7-9]
  \definition[个,些,群]{s.}{diabo; demônio; monstro; na religião ou mitologia, refere-se a fantasmas ou monstros malignos; metaforicamente, também pode se referir a pessoas perversas que cometem atos malignos}
\end{EntryWithPhonetic}

\begin{EntryWithPhonetic}{魔术}{mo2shu4}{20,5}{⿁,⽊}[HSK 7-9]
  \definition[个,场]{s.}{magia; ilusionismo; prestidigitação; truques; utilizando princípios físicos e químicos ou dispositivos especiais, os objetos podem aparecer, desaparecer ou sofrer mudanças maravilhosas de maneira sutil e imperceptível}
\end{EntryWithPhonetic}

\begin{EntryWithPhonetic}{魔头}{mo2tou2}{20,5}{⿁,⼤}
  \definition[个,些]{s.}{diabo; demônio; monstro; espírito maligno}
\end{EntryWithPhonetic}

%%%%%%%%%% 抹 %%%%%%%%%%
\subsection*{抹}\addcontentsline{loh}{figure}{抹 \dpy{mo3}}

\begin{EntryWithPhonetic}{抹}{mo3}{8}{⼿}[HSK 7-9]
  \definition{v.}{colocar; aplicar; untar; engessar | limpar | anular; apagar | (para nuvem, etc.) irradiar; raiar; riscar; traçar | riscar; cancelar; marcar; remover; excluir}
  \seeref{ma1}
  \seeref{mo4}
\end{EntryWithPhonetic}

\begin{EntryWithPhonetic}{抹泪}{mo3lei4}{8,8}{⼿,⽔}
  \definition{v.}{limpar as lágrimas | Figurativo: derramar lágrimas}
\end{EntryWithPhonetic}

%%%%%%%%%% 末 %%%%%%%%%%
\subsection*{末}\addcontentsline{loh}{figure}{末 \dpy{mo4}}

\begin{EntryWithPhonetic}{末}{mo4}{5}{⽊}[HSK 4]
  \definition{adj.}{último; final}
  \definition{s.}{ponta; terminal; extremidade; o final de algo | não essenciais; detalhes secundários | fim; final | pó; poeira | um papel na ópera tradicional}
\end{EntryWithPhonetic}

\begin{EntryWithPhonetic}{末日}{mo4ri4}{5,4}{⽊,⽇}[HSK 7-9]
  \definition{s.}{último dia; dia do juízo final; no cristianismo, refere-se ao último dia do mundo, geralmente significando o dia da morte ou da destruição}
\end{EntryWithPhonetic}

%%%%%%%%%% 没 %%%%%%%%%%
\subsection*{没}\addcontentsline{loh}{figure}{没 \dpy{mo4}}

\begin{EntryWithPhonetic}{没}{mo4}{7}{⽔}
  \definition{adj.}{último; final}
  \definition{v.}{afundar na água; submergir | transbordar; subir além; exceder ou ultrapassar | esconder-se; desaparecer; sumir; ocultar-se | confiscar; expropriar | morrer}
  \variantof{没}
  \seeref{mei2}
\end{EntryWithPhonetic}

\begin{EntryWithPhonetic}{没落}{mo4luo4}{7,12}{⽔,⾋}[HSK 7-9]
  \definition{v.}{declinar; diminuir; estar em declínio; afundar}
\end{EntryWithPhonetic}

\begin{EntryWithPhonetic}{没收}{mo4 shou1}{7,6}{⽔,⽁}[HSK 6]
  \definition{v.}{confiscar; expropriar; os bens e pertences de pessoas ou grupos que violem leis ou proibições serão tornados propriedade pública, de acordo com a lei}
\end{EntryWithPhonetic}

%%%%%%%%%% 抹 %%%%%%%%%%
\subsection*{抹}\addcontentsline{loh}{figure}{抹 \dpy{mo4}}

\begin{EntryWithPhonetic}{抹}{mo4}{8}{⼿}
  \definition{v.}{rebocar; engessar; alisar a massa ou o gesso com uma espátula | virar; contornar; dar uma volta de perto}
  \seeref{ma1}
  \seeref{mo3}
\end{EntryWithPhonetic}

%%%%%%%%%% 陌 %%%%%%%%%%
\subsection*{陌}\addcontentsline{loh}{figure}{陌 \dpy{mo4}}

\begin{EntryWithPhonetic}{陌}{mo4}{8}{⾩}
  \definition[个]{s.}{Literário: caminho entre campos (indo de leste a oeste); trilhas entre campos que correm de leste a oeste; geralmente se refere a estradas nos campos | Obsoleto: estrada}
\end{EntryWithPhonetic}

\begin{EntryWithPhonetic}{陌生}{mo4sheng1}{8,5}{⾩,⽣}[HSK 7-9]
  \definition{adj.}{estranho; desconhecido; inexperiente; indica algo desconhecido ou não familiar e é frequentemente usado como um atributo ou predicado}
\end{EntryWithPhonetic}

%%%%%%%%%% 脉 %%%%%%%%%%
\subsection*{脉}\addcontentsline{loh}{figure}{脉 \dpy{mo4}}

\begin{EntryWithPhonetic}{脉}{mo4}{9}{⾁}
  \definition{adv.}{afetuosamente; amorosamente; carinhosamente; expressar afeto silenciosamente através dos olhos ou ações}
  \seeref{mai4}
\end{EntryWithPhonetic}

%%%%%%%%%% 莫 %%%%%%%%%%
\subsection*{莫}\addcontentsline{loh}{figure}{莫 \dpy{mo4}}

\begin{EntryWithPhonetic}{莫}{mo4}{10}{⾋}
  \definition*{s.}{Sobrenome: Mo}
  \definition{adv.}{não, frequentemente usado em frases imperativas | não; não pode | pode ser que; não pode ser que; é possível que}
  \definition{pron.}{nenhum; nada; ninguém; significa 没有谁 ou 没有哪一种东西}
  \seealsoref{没有哪一种东西}{mei2you3 na3 yi4 zhong3 dong1xi1}
  \seealsoref{没有谁}{mei2you3 shei2}
\end{EntryWithPhonetic}

\begin{EntryWithPhonetic}{莫非}{mo4fei1}{10,8}{⾋,⾮}[HSK 7-9]
  \definition{adv.}{pode ser que; é possível que; será que; frequentemente usado em conjunto com 不成}
  \seealsoref{不成}{bu4 cheng2}
\end{EntryWithPhonetic}

\begin{EntryWithPhonetic}{莫过于}{mo4guo4yu2}{10,6,3}{⾋,⾡,⼆}[HSK 7-9]
  \definition{conj.}{nada é mais\dots do que; nada é melhor do que; nada pode superar}
\end{EntryWithPhonetic}

\begin{EntryWithPhonetic}{莫名其妙}{mo4ming2qi2miao4}{10,6,8,7}{⾋,⼝,⼋,⼥}[HSK 7-9]
  \definition{adj.}{desconcertante; enigmático; bizarro; sem lógica; inexplicável; ninguém consegue explicar seu mistério (razão), indicando que as coisas são estranhas e incompreensíveis; o caractere 名 também é escrito como 明}
\end{EntryWithPhonetic}

%%%%%%%%%% 漠 %%%%%%%%%%
\subsection*{漠}\addcontentsline{loh}{figure}{漠 \dpy{mo4}}

\begin{EntryWithPhonetic}{漠}{mo4}{13}{⽔}
  \definition{adj.}{indiferente; desinteressado | distante; frio; indiferente; despreocupado}
  \definition{s.}{deserto}
\end{EntryWithPhonetic}

\begin{EntryWithPhonetic}{漠然}{mo4ran2}{13,12}{⽔,⽕}[HSK 7-9]
  \definition{adj.}{indiferente; apático; desinteressado; sem qualquer preocupação}
  \definition{adv.}{indiferentemente; apaticamente}
\end{EntryWithPhonetic}

%%%%%%%%%% 嘿 %%%%%%%%%%
\subsection*{嘿}\addcontentsline{loh}{figure}{嘿 \dpy{mo4}}

\begin{EntryWithPhonetic}{嘿}{mo4}{15}{⼝}
  \definition{adj.}{quieto; silencioso; tácito}
  \seeref{hei1}
\end{EntryWithPhonetic}

%%%%%%%%%% 墨 %%%%%%%%%%
\subsection*{墨}\addcontentsline{loh}{figure}{墨 \dpy{mo4}}

\begin{EntryWithPhonetic}{墨}{mo4}{15}{⿊}[HSK 7-9]
  \definition*{s.}{Escola Moísta; Moísmo | México, abreviação de 墨西哥}
  \definition{adj.}{preto; escuro como breu | corrupto | escuro}
  \definition{s.}{tinta chinesa; bastão de tinta | pigmento; tinta | caligrafia ou pintura | aprendizagem; alfabetização | marcador de linha de carpinteiro; marcador de tinta | tatuar o rosto (um castigo); uma punição na China antiga | corrupção; peculato; fraude}
  \seealsoref{墨西哥}{mo4xi1ge1}
\end{EntryWithPhonetic}

\begin{EntryWithPhonetic}{墨镜}{mo4jing4}{15,16}{⿊,⾦}
  \definition[只,双,副]{s.}{óculos escuros}
\end{EntryWithPhonetic}

\begin{EntryWithPhonetic}{墨水}{mo4 shui3}{15,4}{⿊,⽔}[HSK 6]
  \definition[瓶]{s.}{tinta chinesa preparada; tinta (para caneta-tinteiro) | aprendizagem; alfabetização; uma metáfora para o conhecimento ou a capacidade de ler e escrever}
\end{EntryWithPhonetic}

\begin{EntryWithPhonetic}{墨西哥}{mo4xi1ge1}{15,6,10}{⿊,⾑,⼝}
  \definition*{s.}{México; Planalto no México}
\end{EntryWithPhonetic}

%%%%%%%%%% 磨 %%%%%%%%%%
\subsection*{磨}\addcontentsline{loh}{figure}{磨 \dpy{mo4}}

\begin{EntryWithPhonetic}{磨}{mo4}{16}{⽯}
  \definition[盘]{s.}{mó (pedra pesada e redonda para moinho)}
  \definition{v.}{moer; esfarelar; triturar | virar; inverter a marcha}
  \seeref{mo2}
\end{EntryWithPhonetic}

%%%%%%%%%% 默 %%%%%%%%%%
\subsection*{默}\addcontentsline{loh}{figure}{默 \dpy{mo4}}

\begin{EntryWithPhonetic}{默}{mo4}{16}{⿊}
  \definition*{s.}{Sobrenome: Mo}
  \definition{adj.}{taciturno; reservado | silencioso}
  \definition{v.}{escrever de memória}
\end{EntryWithPhonetic}

\begin{EntryWithPhonetic}{默读}{mo4du2}{16,10}{⿊,⾔}[HSK 7-9]
  \definition{v.}{ler em silêncio (oposto de 朗读) | subvocalizar}
  \seealsoref{朗读}{lang3du2}
\end{EntryWithPhonetic}

\begin{EntryWithPhonetic}{默默}{mo4mo4}{16,16}{⿊,⿊}[HSK 4]
  \definition{adj.}{mudo; quieto; silencioso}
  \definition{adv.}{silenciosamente}
\end{EntryWithPhonetic}

\begin{EntryWithPhonetic}{默默无闻}{mo4mo4-wu2wen2}{16,16,4,9}{⿊,⿊,⽆,⾨}[HSK 7-9]
  \definition{expr.}{obscuro; quieto e desconhecido; desconhecido}
\end{EntryWithPhonetic}

\begin{EntryWithPhonetic}{默契}{mo4qi4}{16,9}{⿊,⼤}[HSK 7-9]
  \definition{adj.}{bem coordenado; mutuamente e tacitamente compreendido/acordado; descreve uma conexão profunda entre duas pessoas que transcende as palavras}
  \definition[些,种,份,点]{s.}{acordo ou contrato secreto; entendimento tácito}
\end{EntryWithPhonetic}

%%%%%%%%%% 谋 %%%%%%%%%%
\subsection*{谋}\addcontentsline{loh}{figure}{谋 \dpy{mou2}}

\begin{EntryWithPhonetic}{谋}{mou2}{11}{⾔}
  \definition*{s.}{Sobrenome: Mou}
  \definition[个]{s.}{estratagema; plano; esquema | estratégia; ideia; esquema; plano}
  \definition{v.}{trabalhar para; buscar | consultar | planejar; traçar | conferir; discutir}
\end{EntryWithPhonetic}

\begin{EntryWithPhonetic}{谋害}{mou2hai4}{11,10}{⾔,⼧}[HSK 7-9]
  \definition{v.}{planejar um assassinato; conspirar para matar; tramar para matar | planejar uma conspiração contra; conspirar contra alguém; tramar para incriminar}
\end{EntryWithPhonetic}

\begin{EntryWithPhonetic}{谋求}{mou2qiu2}{11,7}{⾔,⽔}[HSK 7-9]
  \definition{v.}{buscar; esforçar-se por; estar em questão de; tentar obter}
\end{EntryWithPhonetic}

\begin{EntryWithPhonetic}{谋生}{mou2sheng1}{11,5}{⾔,⽣}[HSK 7-9]
  \definition{v.}{ganhar a vida; obter renda; tentar encontrar uma maneira de ganhar a vida}
\end{EntryWithPhonetic}

%%%%%%%%%% 某 %%%%%%%%%%
\subsection*{某}\addcontentsline{loh}{figure}{某 \dpy{mou3}}

\begin{EntryWithPhonetic}{某}{mou3}{9}{⽊}[HSK 3]
  \definition{pron.}{alguém ou algo indefinido; refere-se a pessoas ou coisas incertas | referindo-se a si mesmo; em vez do seu próprio nome | alguns; certos; refere-se a uma pessoa ou coisa específica cujo nome não se sabe ou não se pode revelar | tal e tal; substituir o nome de outra pessoa (geralmente com um tom rude)}
\end{EntryWithPhonetic}

\begin{EntryWithPhonetic}{某些}{mou3 xie1}{9,8}{⽊,⼆}
  \definition{pron.}{certos; alguns; uns poucos; refere-se a pessoas ou coisas que são conhecidas, mas das quais não se fala}
\end{EntryWithPhonetic}

%%%%%%%%%% 模 %%%%%%%%%%
\subsection*{模}\addcontentsline{loh}{figure}{模 \dpy{mu2}}

\begin{EntryWithPhonetic}{模}{mu2}{14}{⽊}
  \definition*{s.}{Sobrenome: Mu}
  \definition{s.}{molde; padrão; matriz}
  \seeref{mo2}
\end{EntryWithPhonetic}

\begin{EntryWithPhonetic}{模具}{mu2ju4}{14,8}{⽊,⼋}
  \definition{s.}{molde | matriz | padrão}
\end{EntryWithPhonetic}

\begin{EntryWithPhonetic}{模样}{mu2yang4}{14,10}{⽊,⽊}[HSK 5]
  \definition[副,种]{s.}{aparência; a aparência ou o estilo de vestir de uma pessoa | indicando uma estimativa aproximada de tempo ou idade; expressão de estimativas relativas a tempo, idade, etc. | tendência; situação; inclinação}
\end{EntryWithPhonetic}

%%%%%%%%%% 母 %%%%%%%%%%
\subsection*{母}\addcontentsline{loh}{figure}{母 \dpy{mu3}}

\begin{EntryWithPhonetic}{母}{mu3}{5}{⽏}[HSK 6][Kangxi 80]
  \definition*{s.}{Sobrenome: Mu}
  \definition{adj.}{fêmea}
  \definition[位,名,个,些]{s.}{mãe | fêmea (animal) (oposto a 公) | origem; pais | parentes idosas; geralmente se refere a mulheres idosas | côncavo | fonte; algo que tem a capacidade ou função de produzir outras coisas}
  \seealsoref{公}{gong1}
\end{EntryWithPhonetic}

\begin{EntryWithPhonetic}{母鸡}{mu3ji1}{5,7}{⽏,⿃}[HSK 6]
  \definition{s.}{galinha}
\end{EntryWithPhonetic}

\begin{EntryWithPhonetic}{母女}{mu3 nv3}{5,3}{⽏,⼥}[HSK 6]
  \definition{s.}{mãe e filha}
\end{EntryWithPhonetic}

\begin{EntryWithPhonetic}{母亲}{mu3qin1}{5,9}{⽏,⼇}[HSK 3]
  \definition[位,名,个,些]{s.}{mãe}
\end{EntryWithPhonetic}

\begin{EntryWithPhonetic}{母语}{mu3yu3}{5,9}{⽏,⾔}
  \definition{s.}{língua materna | língua nativa}
\end{EntryWithPhonetic}

\begin{EntryWithPhonetic}{母子}{mu3 zi3}{5,3}{⽏,⼦}[HSK 6]
  \definition{s.}{mãe e filho}
\end{EntryWithPhonetic}

%%%%%%%%%% 亩 %%%%%%%%%%
\subsection*{亩}\addcontentsline{loh}{figure}{亩 \dpy{mu3}}

\begin{EntryWithPhonetic}{亩}{mu3}{7}{⼇}[HSK 7-9]
  \definition{clas.}{usado para campos | acre; unidade de área igual a um décimo quinto de um hectare}
\end{EntryWithPhonetic}

%%%%%%%%%% 牡 %%%%%%%%%%
\subsection*{牡}\addcontentsline{loh}{figure}{牡 \dpy{mu3}}

\begin{EntryWithPhonetic}{牡}{mu3}{7}{⽜}
  \definition[些]{s.}{macho (de certas aves e animais) (oposto de 牝) | colinas | peônia}
  \seealsoref{牝}{pin4}
\end{EntryWithPhonetic}

\begin{EntryWithPhonetic}{牡丹江}{mu3dan1jiang1}{7,4,6}{⽜,⼂,⽔}
  \definition*{s.}{Cidade de Mudanjiang na província de Heilongjiang, 黑龙江 no nordeste da China}
  \seealsoref{黑龙江}{hei1long2jiang1}
\end{EntryWithPhonetic}

\begin{EntryWithPhonetic}{牡丹}{mu3dan5}{7,4}{⽜,⼂}[HSK 7-9]
  \definition{s.}{peônia; peônia arbórea}
\end{EntryWithPhonetic}

%%%%%%%%%% 姥 %%%%%%%%%%
\subsection*{姥}\addcontentsline{loh}{figure}{姥 \dpy{mu3}}

\begin{EntryWithPhonetic}{姥}{mu3}{9}{⼥}
  \definition*{s.}{Sobrenome: Mu}
  \definition{s.}{mulher idosa; Literário: velha senhora}
  \seeref{lao3}
\end{EntryWithPhonetic}

%%%%%%%%%% 木 %%%%%%%%%%
\subsection*{木}\addcontentsline{loh}{figure}{木 \dpy{mu4}}

\begin{EntryWithPhonetic}{木}{mu4}{4}{⽊}[Kangxi 75]
  \definition{adj.}{de madeira; feito de madeira | estúpido; de raciocínio lento; atordoado; lento para reagir | simplório; chato | entorpecido; de madeira; dormência localizada ou perda de sensibilidade}
  \definition{s.}{árvore | madeira; madeiramento | caixão}
\end{EntryWithPhonetic}

\begin{EntryWithPhonetic}{木板}{mu4ban3}{4,8}{⽊,⽊}[HSK 7-9]
  \definition[张,块]{s.}{prancha; tábua; ripa}
\end{EntryWithPhonetic}

\begin{EntryWithPhonetic}{木材}{mu4cai2}{4,7}{⽊,⽊}[HSK 7-9]
  \definition[批,根]{s.}{madeira; tábuas; madeira serrada; materiais após processamento preliminar de árvores abatidas}
\end{EntryWithPhonetic}

\begin{EntryWithPhonetic}{木匠}{mu4jiang4}{4,6}{⽊,⼕}[HSK 7-9]
  \definition[名,位,个]{s.}{carpinteiro; trabalhadores que fabricam ou reparam peças de madeira, e que produzem e instalam componentes de madeira para casas}
\end{EntryWithPhonetic}

\begin{EntryWithPhonetic}{木偶}{mu4'ou3}{4,11}{⽊,⼈}[HSK 7-9]
  \definition{s.}{fantoche, marionete | imagem de madeira; figura esculpida}
\end{EntryWithPhonetic}

\begin{EntryWithPhonetic}{木头}{mu4tou5}{4,5}{⽊,⼤}[HSK 3]
  \definition[根,块,堆,截]{s.}{tronco; madeira; lenha; denominação genérica para madeira e materiais de madeira}
\end{EntryWithPhonetic}

%%%%%%%%%% 目 %%%%%%%%%%
\subsection*{目}\addcontentsline{loh}{figure}{目 \dpy{mu4}}

\begin{EntryWithPhonetic}{目}{mu4}{5}{⽬}[Kangxi 109]
  \definition*{s.}{Sobrenome: Mu}
  \definition{s.}{olho | item | (biologia) ordem | lista de coisas; catálogo; sumário | buraco em uma rede; malha (abertura)  | (de documentos, teses, etc.) nome; título | ponto; ponto de território, um termo do Go; refere-se à intersecção das linhas verticais e horizontais no tabuleiro, uma intersecção é chamada de 一目, \dpy{yi2 mu4}}
  \definition{v.}{(literário) olhar; considerar}
\end{EntryWithPhonetic}

\begin{EntryWithPhonetic}{目标}{mu4biao1}{5,9}{⽬,⽊}[HSK 3]
  \definition[个,项]{s.}{alvo; objetivo; objeto de tiro, ataque ou busca | objetivo; meta; destino; a situação ou padrão que se deseja alcançar}
\end{EntryWithPhonetic}

\begin{EntryWithPhonetic}{目不转睛}{mu4bu4zhuan3jing1}{5,4,8,13}{⽬,⼀,⾞,⽬}[HSK 7-9]
  \definition{expr.}{``Olhos fixos.''; as pupilas não se movem; um olhar fixo; observar com a máxima concentração; estar totalmente atento a\dots; concentrar o olhar em; contemplar\dots; manter o olhar fixo em\dots; olhar para algo (ou alguém) atentamente; olhar com olhos fixos; não desviar o olhar de\dots; fixar os olhos em\dots; encarar\dots fixamente; olhar para\dots sem piscar; observar\dots sem pestanejar; com todos os olhos; com toda a atenção voltada para a pessoa; descreve alguém que está muito concentrado e absorto naquilo que está observando}
\end{EntryWithPhonetic}

\begin{EntryWithPhonetic}{目瞪口呆}{mu4deng4-kou3dai1}{5,17,3,7}{⽬,⽬,⼝,⼝}[HSK 7-9]
  \definition{expr.}{estupefato; atônito; atordoado; descreve a sensação de estar assustado e atordoado}
\end{EntryWithPhonetic}

\begin{EntryWithPhonetic}{目的}{mu4di4}{5,8}{⽬,⽩}[HSK 2]
  \definition[个,些,种]{s.}{objetivo; meta; alvo; finalidade; propósito; o lugar ou situação que se deseja alcançar; o resultado que se deseja obter; o centro do alvo}
\end{EntryWithPhonetic}

\begin{EntryWithPhonetic}{目的地}{mu4di4di4}{5,8,6}{⽬,⽩,⼟}[HSK 7-9]
  \definition{s.}{destino; o lugar aonde quero chegar}
\end{EntryWithPhonetic}

\begin{EntryWithPhonetic}{目睹}{mu4du3}{5,13}{⽬,⽬}[HSK 7-9]
  \definition{v.}{testemunhar; ver com os próprios olhos}
\end{EntryWithPhonetic}

\begin{EntryWithPhonetic}{目光}{mu4guang1}{5,6}{⽬,⼉}[HSK 5]
  \definition[道,束,种]{s.}{olhar fixo; a expressão e atitude reveladas pelos olhos | visão; vista; percepção visual; a linha imaginária formada entre os olhos e o objeto quando se olha para ele | perspicácia (capacidade de observar e reconhecer coisas); conhecimento adquirido através do contato com as coisas, capacidade de observar as coisas}
\end{EntryWithPhonetic}

\begin{EntryWithPhonetic}{目录}{mu4lu4}{5,8}{⽬,⼹}[HSK 7-9]
  \definition[个]{s.}{lista; conteúdo; catálogo; listar os itens em uma determinada ordem para referência | conteúdo; sumário; os títulos dos capítulos listados em livros e periódicos (geralmente colocados antes do texto principal)}
\end{EntryWithPhonetic}

\begin{EntryWithPhonetic}{目前}{mu4qian2}{5,9}{⽬,⼑}[HSK 3]
  \definition{adv.}{agora; recentemente; no momento; no presente}
\end{EntryWithPhonetic}

\begin{EntryWithPhonetic}{目中无人}{mu4zhong1-wu2ren2}{5,4,4,2}{⽬,⼁,⽆,⼈}[HSK 7-9]
  \definition{expr.}{considerar todos inferiores a si; ​​arrogante; presunçoso; aos seus olhos, não há outro; não se importar com ninguém; tratar os outros com desprezo; se achar superior e menosprezar os outros; não ter respeito por ninguém; olhar para todos com desdém; ser arrogante; ser presunçoso; ficar convencido demais; orgulhoso demais; com o nariz empinado; orgulhoso e altivo}
\end{EntryWithPhonetic}

%%%%%%%%%% 沐 %%%%%%%%%%
\subsection*{沐}\addcontentsline{loh}{figure}{沐 \dpy{mu4}}

\begin{EntryWithPhonetic}{沐}{mu4}{7}{⽔}
  \definition*{s.}{Sobrenome: Mu}
  \definition{v.}{Significado original: lavar o cabelo | Literário: receber (ou ser dado) (bondade, favor, etc.)}
  \definition{v.}{Transliteração: banhar-se}
\end{EntryWithPhonetic}

\begin{EntryWithPhonetic}{沐浴露}{mu4yu4lu4}{7,10,21}{⽔,⽔,⾬}[HSK 7-9]
  \definition[瓶]{s.}{gel de banho; sabonete líquido; creme de banho}
\end{EntryWithPhonetic}

%%%%%%%%%% 牧 %%%%%%%%%%
\subsection*{牧}\addcontentsline{loh}{figure}{牧 \dpy{mu4}}

\begin{EntryWithPhonetic}{牧}{mu4}{8}{⽜}
  \definition*{s.}{Sobrenome: Mu}
  \definition{v.}{cuidar (de ovelhas, gado, etc.); pastorear}
\end{EntryWithPhonetic}

\begin{EntryWithPhonetic}{牧场}{mu4chang3}{8,6}{⽜,⼟}[HSK 7-9]
  \definition[个,片]{s.}{pastagem; campo de pastoreio; área de pastagem; pastos}
\end{EntryWithPhonetic}

\begin{EntryWithPhonetic}{牧民}{mu4min2}{8,5}{⽜,⽒}[HSK 7-9]
  \definition[个]{s.}{pastor; pessoas em áreas pastoris que ganham a vida criando gado}
\end{EntryWithPhonetic}

%%%%%%%%%% 募 %%%%%%%%%%
\subsection*{募}\addcontentsline{loh}{figure}{募 \dpy{mu4}}

\begin{EntryWithPhonetic}{募}{mu4}{12}{⼒}
  \definition*{s.}{Sobrenome: Mu}
  \definition{v.}{arrecadar; coletar | alistar; recrutar}
\end{EntryWithPhonetic}

\begin{EntryWithPhonetic}{募捐}{mu4/juan1}{12,10}{⼒,⼿}[HSK 7-9]
  \definition[场,次]{v.+compl.}{solicitar contribuições; arrecadar doações; passar o chapéu}
\end{EntryWithPhonetic}

%%%%%%%%%% 墓 %%%%%%%%%%
\subsection*{墓}\addcontentsline{loh}{figure}{墓 \dpy{mu4}}

\begin{EntryWithPhonetic}{墓}{mu4}{13}{⼟}
  \definition[座,个,号]{s.}{sepultura; túmulo; mausoléu}
\end{EntryWithPhonetic}

\begin{EntryWithPhonetic}{墓碑}{mu4bei1}{13,13}{⼟,⽯}[HSK 7-9]
  \definition[块,座]{s.}{lápide; túmulo}
\end{EntryWithPhonetic}

\begin{EntryWithPhonetic}{墓地}{mu4di4}{13,6}{⼟,⼟}[HSK 7-9]
  \definition{s.}{cemitério; local de sepultamento}
\end{EntryWithPhonetic}

%%%%%%%%%% 幕 %%%%%%%%%%
\subsection*{幕}\addcontentsline{loh}{figure}{幕 \dpy{mu4}}

\begin{EntryWithPhonetic}{幕}{mu4}{13}{⼱}[HSK 7-9]
  \definition{s.}{tenda | cortina; tela | ato (de uma peça); cena; um trecho mais completo da peça | Obsoleto: gabinete de um governador ou de um general | materiais de cobertura como tecido, seda e feltro}
\end{EntryWithPhonetic}

\begin{EntryWithPhonetic}{幕后}{mu4hou4}{13,6}{⼱,⼝}[HSK 7-9]
  \definition{s.}{nos bastidores; por trás das cenas; por trás da cortina do palco; uma metáfora para aqueles que permanecem ocultos e seus lugares escondidos}
\end{EntryWithPhonetic}

%%%%%%%%%% 穆 %%%%%%%%%%
\subsection*{穆}\addcontentsline{loh}{figure}{穆 \dpy{mu4}}

\begin{EntryWithPhonetic}{穆}{mu4}{16}{⽲}
  \definition*{s.}{Sobrenome: Mu}
  \definition{adj.}{solene; reverente | respeitoso}
\end{EntryWithPhonetic}

\begin{EntryWithPhonetic}{穆棱}{mu4ling2}{16,12}{⽲,⽊}
  \definition*{s.}{Cidade no nível do condado de Muling, na província de Mudanjiang 牡丹江,Heilongjiang}
  \seealsoref{牡丹江}{mu3dan1jiang1}
\end{EntryWithPhonetic}

\begin{EntryWithPhonetic}{穆斯林}{mu4si1lin2}{16,12,8}{⽲,⽄,⽊}[HSK 7-9]
  \definition[个,位,名]{s.}{muçulmano}
\end{EntryWithPhonetic}

%%%%% EOF %%%%%


 %%%
%%% N
%%%
\section*{N}\addcontentsline{toc}{section}{N}\addcontentsline{loh}{figure}{\#\#\#\#\#\#\#\# N}

%%%%%%%%%% 那 %%%%%%%%%%
\subsection*{那}\addcontentsline{loh}{figure}{那 \dpy{na1}}

\begin{EntryWithPhonetic}{那}{na1}{6}{⾢}
  \definition*{s.}{Sobrenome: Na}
  \seeref{na3}
  \seeref{na4}
  \seeref{ne4}
  \seeref{nei4}
  \seeref{nuo2}
\end{EntryWithPhonetic}

%%%%%%%%%% 拿 %%%%%%%%%%
\subsection*{拿}\addcontentsline{loh}{figure}{拿 \dpy{na2}}

\begin{EntryWithPhonetic}{拿}{na2}{10}{⼿}[HSK 1]
  \definition{part.}{usado da mesma forma que 把: para marcar o seguinte substantivo seguinte como objeto direto}
  \definition{prep.}{ferramentas, materiais, métodos, etc. utilizados para a introdução | os objetos que estão sendo manipulados para introdução}
  \definition{v.}{segurar; pegar; pegar ou mover objetos com as mãos ou de outra forma | apreender; capturar; prender; usar força bruta para capturar | ter certeza de; ser capaz de fazer; ter uma compreensão firme de | tornar as coisas difíceis para alguém; colocar alguém em uma situação difícil; obstruir; chantagear; coagir; causar dificuldades intencionalmente | fingir ou fazer (algum tipo de postura ou aparência) | ter certeza de; tomar uma decisão | obter; ganhar; receber}
\end{EntryWithPhonetic}

\begin{EntryWithPhonetic}{拿出}{na2 chu1}{10,5}{⼿,⼐}[HSK 2]
  \definition{v.}{apresentar (evidências) | fornecer | apresentar (uma proposta) | oferecer; servir | retirar; tirar}
\end{EntryWithPhonetic}

\begin{EntryWithPhonetic}{拿到}{na2 dao4}{10,8}{⼿,⼑}[HSK 2]
  \definition{v.}{pegar; obter, conseguir}
\end{EntryWithPhonetic}

\begin{EntryWithPhonetic}{拿手}{na2shou3}{10,4}{⼿,⼿}[HSK 7-9]
  \definition{adj.}{hábil; especialista; bom em; proficiente em determinada tecnologia}
\end{EntryWithPhonetic}

\begin{EntryWithPhonetic}{拿走}{na2 zou3}{10,7}{⼿,⾛}[HSK 6]
  \definition{v.}{tirar; remover}
\end{EntryWithPhonetic}

%%%%%%%%%% 那 %%%%%%%%%%
\subsection*{那}\addcontentsline{loh}{figure}{那 \dpy{na3}}

\begin{EntryWithPhonetic}{那}{na3}{6}{⾢}
  \definition{adv.}{expressa negação em perguntas retóricas}
  \definition{pron.}{qual? | qualquer que seja; qualquer que; para expressar incerteza em uma declaração | variante de 哪}
  \seeref{na1}
  \seeref{na4}
  \seeref{ne4}
  \seeref{nei4}
  \seeref{nuo2}
  \seealsoref{哪}{na3}
\end{EntryWithPhonetic}

%%%%%%%%%% 哪 %%%%%%%%%%
\subsection*{哪}\addcontentsline{loh}{figure}{哪 \dpy{na3}}

\begin{EntryWithPhonetic}{哪}{na3}{9}{⼝}[HSK 1,4]
  \definition{adv.}{para expressar uma pergunta retórica, indicando que é impossível}
  \definition{pron.}{qual?; o que?; expressa a necessidade de determinar um entre várias pessoas ou coisas | qualquer; ser usado em um sentido geral | qual?; o que?; (usado sozinho, o mesmo que 什么, frequentemente usado de forma intercambiável com 什么) | qualquer; qualquer que seja; refere"-se a qualquer um, geralmente seguido por 都 ou 也, ou usando dois 哪 antes e depois | qual (indica algo incerto)}
  \seeref{na5}
  \seeref{nei3}
  \seealsoref{都}{dou1}
  \seealsoref{什么}{shen2me5}
  \seealsoref{也}{ye3}
\end{EntryWithPhonetic}

\begin{EntryWithPhonetic}{哪个}{na3ge5}{9,3}{⼝,⼈}
  \definition{pron.}{qual deles (pergunta sobre o objeto) | quem (perguntar a alguém ou indicar qualquer pessoa)}
\end{EntryWithPhonetic}

\begin{EntryWithPhonetic}{哪国人}{na3 guo2ren2}{9,8,2}{⼝,⼞,⼈}
  \definition{expr.}{de qual país?}
\end{EntryWithPhonetic}

\begin{EntryWithPhonetic}{哪里}{na3 li3}{9,7}{⼝,⾥}[HSK 1]
  \definition{adv.}{usado em perguntas retóricas para expressar um significado negativo}
  \definition{pron.}{onde?; em que lugar? | onde quer que seja; em qualquer lugar | usado como uma resposta educada a um elogio}
\end{EntryWithPhonetic}

\begin{EntryWithPhonetic}{哪怕}{na3pa4}{9,8}{⼝,⼼}[HSK 4]
  \definition{conj.}{mesmo; mesmo se; mesmo que; não importa o quão}
\end{EntryWithPhonetic}

\begin{EntryWithPhonetic}{哪儿}{na3r5}{9,2}{⼝,⼉}[HSK 1]
  \definition{adv.}{usado para perguntas retóricas, indicando negação}
  \definition{pron.}{onde? | onde quer que seja; em qualquer lugar | usado como uma resposta educada a um elogio}
\end{EntryWithPhonetic}

\begin{EntryWithPhonetic}{哪些}{na3xie1}{9,8}{⼝,⼆}[HSK 1]
  \definition{pron.}{quais?}
\end{EntryWithPhonetic}

\begin{EntryWithPhonetic}{哪知道}{na3 zhi1dao4}{9,8,12}{⼝,⽮,⾡}[HSK 7-9]
  \definition{expr.}{Quem sabe?}[我哪知道 你挑起来的。===Como eu ia saber que você tinha começado?]
\end{EntryWithPhonetic}

%%%%%%%%%% 那 %%%%%%%%%%
\subsection*{那}\addcontentsline{loh}{figure}{那 \dpy{na4}}

\begin{EntryWithPhonetic}{那}{na4}{6}{⾢}[HSK 1,2]
  \definition{conj.}{então; nessa situação; nesse caso; o mesmo que 那么}
  \definition{pron.}{aquele; aquilo; indica pessoas ou coisas distantes | aquele; aquilo; expressa muitas coisas, sem se referir especificamente a uma pessoa ou coisa, e é frequentemente usado em conjunto com 这}
  \seeref{na1}
  \seeref{na3}
  \seeref{ne4}
  \seeref{nei4}
  \seeref{nuo2}
  \seealsoref{那么}{na4 me5}
  \seealsoref{这}{zhe4}
\end{EntryWithPhonetic}

\begin{EntryWithPhonetic}{那边}{na4 bian5}{6,5}{⾢,⾡}[HSK 1]
  \definition{pron.}{ali; acolá; aquele lado}
\end{EntryWithPhonetic}

\begin{EntryWithPhonetic}{那个}{na4ge5}{6,3}{⾢,⼈}
  \definition{pron.}{aquele | usado antes de verbos e adjetivos para indicar exagero | para substituir o discurso direto inconveniente}
\end{EntryWithPhonetic}

\begin{EntryWithPhonetic}{那会儿}{na4 hui4r5}{6,6,2}{⾢,⼈,⼉}[HSK 2]
  \definition{pron.}{então; naquela época; refere"-se ao passado ou ao futuro}
\end{EntryWithPhonetic}

\begin{EntryWithPhonetic}{那里}{na4 li3}{6,7}{⾢,⾥}[HSK 1]
  \definition{pron./s.}{lá; ali; aquele lugar; indica um lugar distante}
\end{EntryWithPhonetic}

\begin{EntryWithPhonetic}{那么}{na4 me5}{6,3}{⾢,⼃}[HSK 2]
  \definition{conj.}{então; nesse caso; afirmar o resultado esperado ou fazer um julgamento}
  \definition{pron.}{assim; dessa maneira; indica a natureza, o estado, a forma, o grau, etc. | assim; sobre; colocado antes do numeral, indica uma estimativa}
\end{EntryWithPhonetic}

\begin{EntryWithPhonetic}{那麽}{na4 me5}{6,14}{⾢,⿇}
  \variantof{那么}
\end{EntryWithPhonetic}

\begin{EntryWithPhonetic}{那儿}{na4r5}{6,2}{⾢,⼉}[HSK 1]
  \definition{pron.}{lá; ali; naquele lugar | então; naquela época (usado após 打, 从 e 由)}
  \seealsoref{从}{cong2}
  \seealsoref{打}{da3}
  \seealsoref{由}{you2}
\end{EntryWithPhonetic}

\begin{EntryWithPhonetic}{那时}{na4 shi2}{6,7}{⾢,⽇}
  \definition{pron.}{então; naquela época; naqueles dias; geralmente se refere a um período de tempo distante do presente}
  \seealsoref{那时候}{na4 shi2 hou5}
\end{EntryWithPhonetic}

\begin{EntryWithPhonetic}{那时候}{na4 shi2 hou5}{6,7,10}{⾢,⽇,⼈}[HSK 2]
  \definition{adv.}{naquela hora; em algum momento no passado}
  \seealsoref{那时}{na4 shi2}
\end{EntryWithPhonetic}

\begin{EntryWithPhonetic}{那些}{na4 xie1}{6,8}{⾢,⼆}[HSK 1]
  \definition{pron.}{aqueles; indica duas ou mais pessoas ou coisas}
\end{EntryWithPhonetic}

\begin{EntryWithPhonetic}{那样}{na4 yang4}{6,10}{⾢,⽊}[HSK 2]
  \definition{pron.}{assim; tal; desse tipo; desse gênero; dessa natureza; desse tipo; indica a natureza, o estado, a maneira, o grau ou refere"-se a uma ação ou situação específica}
\end{EntryWithPhonetic}

\begin{EntryWithPhonetic}{那咱}{na4 zan5}{6,9}{⾢,⼝}
  \definition{s.}{(informal) naquela época; então | (antigo) naquela época}
\end{EntryWithPhonetic}

%%%%%%%%%% 呐 %%%%%%%%%%
\subsection*{呐}\addcontentsline{loh}{figure}{呐 \dpy{na4}}

\begin{EntryWithPhonetic}{呐}{na4}{7}{⼝}
  \definition{interj./v.}{elemento formador de palavras}
  \seeref{ne4}
\end{EntryWithPhonetic}

\begin{EntryWithPhonetic}{呐喊}{na4han3}{7,12}{⼝,⼝}[HSK 7-9]
  \definition{v.}{gritar bem alto; berrar bem alto}
\end{EntryWithPhonetic}

%%%%%%%%%% 纳 %%%%%%%%%%
\subsection*{纳}\addcontentsline{loh}{figure}{纳 \dpy{na4}}

\begin{EntryWithPhonetic}{纳}{na4}{7}{⽷}
  \definition*{s.}{Sobrenome: Na}
  \definition{v.}{receber; admitir; trazer para dentro; deixar entrar | aceitar | aproveitar | pagar impostos; pagar taxas | trazer para; colocar em | costurar com pontos próximos (sobre um remendo, etc.); remendar}
\end{EntryWithPhonetic}

\begin{EntryWithPhonetic}{纳闷儿}{na4/men4r5}{7,7,2}{⽷,⾨,⼉}[HSK 7-9]
  \definition{v.+compl.}{maravilhar-se; sentir-se perplexo; ficar confuso}
\end{EntryWithPhonetic}

\begin{EntryWithPhonetic}{纳入}{na4ru4}{7,2}{⽷,⼊}[HSK 7-9]
  \definition{v.}{incluir; incorporar; trazer para; inserir; classificar (geralmente usado para conceitos abstratos)}
\end{EntryWithPhonetic}

\begin{EntryWithPhonetic}{纳税}{na4/shui4}{7,12}{⽷,⽲}[HSK 7-9]
  \definition{v.+compl.}{pagar impostos; pagar impostos ao estado}
\end{EntryWithPhonetic}

\begin{EntryWithPhonetic}{纳税人}{na4shui4ren2}{7,12,2}{⽷,⽲,⼈}[HSK 7-9]
  \definition{s.}{contribuinte; pagador de impostos}[我们不能这样浪费纳税人的钱。===Não podemos desperdiçar o dinheiro dos contribuintes dessa forma.]
\end{EntryWithPhonetic}

%%%%%%%%%% 哪 %%%%%%%%%%
\subsection*{哪}\addcontentsline{loh}{figure}{哪 \dpy{na5}}

\begin{EntryWithPhonetic}{哪}{na5}{9}{⼝}
  \definition{part.}{usado depois de uma palavra com a terminação -n, é equivalente a 啊}
  \seeref{na3}
  \seeref{nei3}
  \seealsoref{啊}{a5}
\end{EntryWithPhonetic}

%%%%%%%%%% 乃 %%%%%%%%%%
\subsection*{乃}\addcontentsline{loh}{figure}{乃 \dpy{nai3}}

\begin{EntryWithPhonetic}{乃}{nai3}{2}{⼃}[HSK 7-9]
  \definition{adv.}{então; portanto | somente então}
  \definition{pron.}{você; seu}
  \definition{v.}{ser | ser realmente; ser de fato}[失败乃成功之母。===O fracasso é a mãe do sucesso.]
\end{EntryWithPhonetic}

\begin{EntryWithPhonetic}{乃至}{nai3zhi4}{2,6}{⼃,⾄}[HSK 7-9]
  \definition{conj.}{até mesmo; usado para enfatizar que algo excede o alcance ou a extensão esperada}
\end{EntryWithPhonetic}

%%%%%%%%%% 奶 %%%%%%%%%%
\subsection*{奶}\addcontentsline{loh}{figure}{奶 \dpy{nai3}}

\begin{EntryWithPhonetic}{奶}{nai3}{5}{⼥}[HSK 1]
  \definition{adj.}{bebê; infância; infantil}
  \definition[杯,滴,瓶,只,桶]{s.}{seios; mama | leite; produtos lácteos}
  \definition{v.}{amamentar; mamar}
\end{EntryWithPhonetic}

\begin{EntryWithPhonetic}{奶茶}{nai3 cha2}{5,9}{⼥,⾋}[HSK 3]
  \definition[杯]{s.}{chá com leite; chá com leite de vaca ou de ovelha}
\end{EntryWithPhonetic}

\begin{EntryWithPhonetic}{奶粉}{nai3 fen3}{5,10}{⼥,⽶}[HSK 6]
  \definition[袋,桶,罐,勺]{s.}{leite em pó}
\end{EntryWithPhonetic}

\begin{EntryWithPhonetic}{奶奶}{nai3nai5}{5,5}{⼥,⼥}[HSK 1]
  \definition[位]{s.}{avó (paterna) | vovó; avó; mulheres mais velhas | jovem senhora da casa}
\end{EntryWithPhonetic}

\begin{EntryWithPhonetic}{奶牛}{nai3 niu2}{5,4}{⼥,⽜}[HSK 6]
  \definition{s.}{vaca leiteira (ou leiteira); vaca}
\end{EntryWithPhonetic}

%%%%%%%%%% 耐 %%%%%%%%%%
\subsection*{耐}\addcontentsline{loh}{figure}{耐 \dpy{nai4}}

\begin{EntryWithPhonetic}{耐}{nai4}{9}{⽽}[HSK 7-9]
  \definition{v.}{ser capaz de suportar (ou tolerar); poder resistir; poder suportar | suportar; aguentar; resistir}
\end{EntryWithPhonetic}

\begin{EntryWithPhonetic}{耐人寻味}{nai4ren2xun2wei4}{9,2,6,8}{⽽,⼈,⼨,⼝}[HSK 7-9]
  \definition{expr.}{intrigante; instigante; é algo profundo e que merece uma reflexão cuidadosa}
\end{EntryWithPhonetic}

\begin{EntryWithPhonetic}{耐心}{nai4xin1}{9,4}{⽽,⼼}[HSK 5]
  \definition{adj.}{paciente}
  \definition[些]{s.}{paciência; uma pessoa que não se importa com problemas e é paciente}
\end{EntryWithPhonetic}

\begin{EntryWithPhonetic}{耐性}{nai4xing4}{9,8}{⽽,⼼}[HSK 7-9]
  \definition{s.}{paciência; resistência | tolerância; uma personalidade paciente e sem pressa}
\end{EntryWithPhonetic}

%%%%%%%%%% 男 %%%%%%%%%%
\subsection*{男}\addcontentsline{loh}{figure}{男 \dpy{nan2}}

\begin{EntryWithPhonetic}{男}{nan2}{7}{⽥}[HSK 1]
  \definition{adj.}{homem; macho; masculino}
  \definition[个,位]{s.}{filho; menino | homem | barão (o mais baixo de cinco ordens de nobreza)}
  \antonymref{女}{nv3}
\end{EntryWithPhonetic}

\begin{EntryWithPhonetic}{男孩儿}{nan2hai2r5}{7,9,2}{⽥,⼦,⼉}[HSK 1]
  \definition{s.}{menino; rapaz}
\end{EntryWithPhonetic}

\begin{EntryWithPhonetic}{男女}{nan2 nv3}{7,3}{⽥,⼥}[HSK 4]
  \definition{s.}{homens e mulheres; masculino e feminino}
\end{EntryWithPhonetic}

\begin{EntryWithPhonetic}{男朋友}{nan2 peng2 you5}{7,8,4}{⽥,⽉,⼜}[HSK 1]
  \definition{s.}{namorado}
\end{EntryWithPhonetic}

\begin{EntryWithPhonetic}{男人}{nan2 ren2}{7,2}{⽥,⼈}[HSK 1]
  \definition[个]{s.}{homem adulto; macho; cavalheiro | marido}
\end{EntryWithPhonetic}

\begin{EntryWithPhonetic}{男生}{nan2 sheng1}{7,5}{⽥,⽣}[HSK 1]
  \definition[个]{s.}{menino; estudante; estudante do sexo masculino; aluno do sexo masculino}
\end{EntryWithPhonetic}

\begin{EntryWithPhonetic}{男士}{nan2 shi4}{7,3}{⽥,⼠}[HSK 4]
  \definition{s.}{cavalheiro; \emph{gentleman}}
\end{EntryWithPhonetic}

\begin{EntryWithPhonetic}{男性}{nan2 xing4}{7,8}{⽥,⼼}[HSK 5]
  \definition{s.}{masculino; homem; masculinidade}
  \antonymref{女性}{nv3 xing4}
\end{EntryWithPhonetic}

\begin{EntryWithPhonetic}{男子}{nan2zi3}{7,3}{⽥,⼦}[HSK 3]
  \definition[个,位]{s.}{uma pessoa do sexo masculino; um homem}
\end{EntryWithPhonetic}

%%%%%%%%%% 南 %%%%%%%%%%
\subsection*{南}\addcontentsline{loh}{figure}{南 \dpy{nan2}}

\begin{EntryWithPhonetic}{南}{nan2}{9}{⼗}[HSK 1]
  \definition*{s.}{Sobrenome: Nan}
  \definition{s.}{sul; uma das quatro direções básicas, o lado direito quando se está de frente para o sol pela manhã | especificamente no sul da China}
  \antonymref{北}{bei3}
\end{EntryWithPhonetic}

\begin{EntryWithPhonetic}{南北}{nan2 bei3}{9,5}{⼗,⼔}[HSK 5]
  \definition{s.}{(território) norte e sul | (distância) de norte a sul}
\end{EntryWithPhonetic}

\begin{EntryWithPhonetic}{南边}{nan2 bian5}{9,5}{⼗,⾡}[HSK 1]
  \definition{s.}{sul; lado sul}
\end{EntryWithPhonetic}

\begin{EntryWithPhonetic}{南部}{nan2 bu4}{9,10}{⼗,⾢}[HSK 3]
  \definition{s.}{parte sul; sul | a parte sul}
\end{EntryWithPhonetic}

\begin{EntryWithPhonetic}{南方}{nan2 fang1}{9,4}{⼗,⽅}[HSK 2]
  \definition{s.}{sul; indica a direção sul | o sul; a região sul}
\end{EntryWithPhonetic}

\begin{EntryWithPhonetic}{南瓜}{nan2gua1}{9,5}{⼗,⽠}[HSK 7-9]
  \definition[个,斤,块]{s.}{abóbora}
\end{EntryWithPhonetic}

\begin{EntryWithPhonetic}{南极}{nan2ji2}{9,7}{⼗,⽊}[HSK 5]
  \definition*{s.}{Polo Sul; Polo Antártico | Polo sul magnético}
  \definition{s.}{polo sul magnético}
\end{EntryWithPhonetic}

\begin{EntryWithPhonetic}{南京}{nan2jing1}{9,8}{⼗,⼇}
  \definition*{s.}{Nanquim, capital da província de Jiangsu, 江苏}
  \seealsoref{江苏}{jiang1su1}
\end{EntryWithPhonetic}

\begin{EntryWithPhonetic}{南面}{nan2mian4}{9,9}{⼗,⾯}
  \definition{s.}{sul | lado sul}
\end{EntryWithPhonetic}

%%%%%%%%%% 难 %%%%%%%%%%
\subsection*{难}\addcontentsline{loh}{figure}{难 \dpy{nan2}}

\begin{EntryWithPhonetic}{难}{nan2}{10}{⾫}[HSK 1]
  \definition{adj.}{difícil; duro; problemático | dificilmente possível; inevitável | ruim; desagradável | problemático; improvável}
  \definition{s.}{dificuldade}
  \definition{v.}{colocar alguém em uma situação difícil}
  \seeref{nan4}
  \antonymref{易}{yi4}
\end{EntryWithPhonetic}

\begin{EntryWithPhonetic}{难处}{nan2chu3}{10,5}{⾫,⼡}
  \definition{adj.}{dificuldade de convivência; difícil de conviver; difícil de lidar}
  \seeref{nan2chu4}
  \seeref{nan2chu5}
\end{EntryWithPhonetic}

\begin{EntryWithPhonetic}{难处}{nan2chu4}{10,5}{⾫,⼡}[HSK 7-9]
  \definition{s.}{dificuldade; problema; assunto difícil}
  \seeref{nan2chu3}
  \seeref{nan2chu5}
\end{EntryWithPhonetic}

\begin{EntryWithPhonetic}{难处}{nan2chu5}{10,5}{⾫,⼡}
  \definition{s.}{dificuldade; problema; assunto difícil}
  \seeref{nan2chu3}
  \seeref{nan2chu4}
\end{EntryWithPhonetic}

\begin{EntryWithPhonetic}{难道}{nan2dao4}{10,12}{⾫,⾡}[HSK 3]
  \definition{adv.}{certamente não significa que\dots?; é possível que\dots?; não me diga\dots; poderia ser que\dots?; usado em frases interrogativas para reforçar o tom interrogativo; frequentemente usado com palavras como 吗 e 不成}
  \seealsoref{不成}{bu4 cheng2}
  \seealsoref{吗}{ma5}
\end{EntryWithPhonetic}

\begin{EntryWithPhonetic}{难得}{nan2de2}{10,11}{⾫,⼻}[HSK 5]
  \definition{adj.}{raro; difícil de encontrar; difícil de obter ou realizar, indicando que é valioso}
  \definition{adv.}{raramente; com pouca frequência}
\end{EntryWithPhonetic}

\begin{EntryWithPhonetic}{难得一见}{nan2de2 yi2 jian4}{10,11,1,4}{⾫,⼻,⼀,⾒}[HSK 7-9]
  \definition{expr.}{``Uma visão rara.''; raramente visto}
\end{EntryWithPhonetic}

\begin{EntryWithPhonetic}{难点}{nan2dian3}{10,9}{⾫,⽕}[HSK 7-9]
  \definition[个]{s.}{ponto difícil; osso duro de roer | dificuldade; áreas onde o problema não é fácil de resolver}
\end{EntryWithPhonetic}

\begin{EntryWithPhonetic}{难度}{nan2 du4}{10,9}{⾫,⼴}[HSK 3]
  \definition{s.}{dificuldade; grau de dificuldade}
\end{EntryWithPhonetic}

\begin{EntryWithPhonetic}{难怪}{nan2guai4}{10,8}{⾫,⼼}[HSK 7-9]
  \definition{adv.}{não é de admirar; não me admira}
  \definition{v.}{ser compreensível; ser perdoável}
\end{EntryWithPhonetic}

\begin{EntryWithPhonetic}{难关}{nan2guan1}{10,6}{⾫,⼋}[HSK 7-9]
  \definition[道,个]{s.}{crise; barreira; apertado; dificuldade; obstáculo; aperto; uma metáfora para uma dificuldade difícil de superar}
\end{EntryWithPhonetic}

\begin{EntryWithPhonetic}{难过}{nan2guo4}{10,6}{⾫,⾡}[HSK 2]
  \definition{adj.}{triste; ruim; psicologicamente desconfortável | difícil; árduo}
\end{EntryWithPhonetic}

\begin{EntryWithPhonetic}{难堪}{nan2kan1}{10,12}{⾫,⼟}[HSK 7-9]
  \definition{adj.}{constrangedor; sem graça}
  \definition{v.}{ser difícil de suportar}
\end{EntryWithPhonetic}

\begin{EntryWithPhonetic}{难看}{nan2 kan4}{10,9}{⾫,⽬}[HSK 2]
  \definition{adj.}{feio; desagradável à vista | vergonhoso; embaraçoso; desonroso; sem glória; sem dignidade}
\end{EntryWithPhonetic}

\begin{EntryWithPhonetic}{难受}{nan2shou4}{10,8}{⾫,⼜}[HSK 2]
  \definition{adj.}{sentir dor; sentir-se mal; sentir-se desconfortável | sentir-se mal; sentir-se infeliz; de mau humor; triste}
\end{EntryWithPhonetic}

\begin{EntryWithPhonetic}{难说}{nan2shuo1}{10,9}{⾫,⾔}[HSK 7-9]
  \definition{v.}{ser difícil dizer; nunca se saber ao certo}[他什么时候回来还很难说。===É difícil dizer quando ele voltará.]
\end{EntryWithPhonetic}

\begin{EntryWithPhonetic}{难题}{nan2 ti2}{10,15}{⾫,⾴}[HSK 2]
  \definition[个,道]{s.}{desafio; problema difícil; questão difícil; questões difíceis de responder ou resolver}
\end{EntryWithPhonetic}

\begin{EntryWithPhonetic}{难听}{nan2 ting1}{10,7}{⾫,⼝}[HSK 2]
  \definition{adj.}{desagradável de ouvir | ofensivo; grosseiro; vulgar e desagradável | escandaloso; indigno}
\end{EntryWithPhonetic}

\begin{EntryWithPhonetic}{难忘}{nan2 wang4}{10,7}{⾫,⼼}[HSK 6]
  \definition{adj.}{memorável; inesquecível}
\end{EntryWithPhonetic}

\begin{EntryWithPhonetic}{难为情}{nan2wei2qing2}{10,4,11}{⾫,⼂,⼼}[HSK 7-9]
  \definition{adj.}{tímido; envergonhado; constrangido; descreve o sentimento de constrangimento ou vergonha que alguém experimenta ao se deparar com uma situação embaraçosa ou constrangedora | desconcertante; embaraçoso; difícil de lidar; descreve a sensação de estar preso em uma situação difícil ou de se sentir impotente ao se deparar com um problema que é difícil de decidir, lidar ou enfrentar}
\end{EntryWithPhonetic}

\begin{EntryWithPhonetic}{难以}{nan2 yi3}{10,4}{⾫,⼈}[HSK 5]
  \definition{adj.}{difícil; complicado}
\end{EntryWithPhonetic}

\begin{EntryWithPhonetic}{难以想象}{nan2yi3-xiang3xiang4}{10,4,13,11}{⾫,⼈,⼼,⾗}[HSK 7-9]
  \definition{expr.}{inimaginável; difícil imaginar}
\end{EntryWithPhonetic}

\begin{EntryWithPhonetic}{难以置信}{nan2yi3-zhi4xin4}{10,4,13,9}{⾫,⼈,⽹,⼈}[HSK 7-9]
  \definition{expr.}{difícil de acreditar; incrível; inacreditável; ser quase inacreditável; ser inacreditável; ser difícil de acreditar; ser bastante difícil de engolir}
\end{EntryWithPhonetic}

\begin{EntryWithPhonetic}{难}{nan4}{10}{⾫}
  \definition{s.}{catástrofe; calamidade; desastre; adversidade; grande infortúnio}
  \definition{v.}{acusar; culpar}
  \seeref{nan2}
\end{EntryWithPhonetic}

\begin{EntryWithPhonetic}{难免}{nan4mian3}{10,7}{⾫,⼉}[HSK 4]
  \definition{adj.}{inevitável; difícil de evitar}
\end{EntryWithPhonetic}

%%%%%%%%%% 孬 %%%%%%%%%%
\subsection*{孬}\addcontentsline{loh}{figure}{孬 \dpy{nao1}}

\begin{EntryWithPhonetic}{孬}{nao1}{10}{⼥}
  \definition{adj.}{ruim | covarde | Dialeto: não (é) bom (contração de 不 + 好)}
  \seealsoref{不}{bu4}
  \seealsoref{好}{hao3}
\end{EntryWithPhonetic}

%%%%%%%%%% 呶 %%%%%%%%%%
\subsection*{呶}\addcontentsline{loh}{figure}{呶 \dpy{nao2}}

\begin{EntryWithPhonetic}{呶}{nao2}{8}{⼝}
  \definition{interj.}{Onomatopéia: ruído alto e contínuo}
  \definition{v.}{Literário: gritar; clamar; falar ruidosamente}
  \seealsoref{努}{nu3}
\end{EntryWithPhonetic}

%%%%%%%%%% 挠 %%%%%%%%%%
\subsection*{挠}\addcontentsline{loh}{figure}{挠 \dpy{nao2}}

\begin{EntryWithPhonetic}{挠}{nao2}{9}{⼿}[HSK 7-9]
  \definition{v.}{coçar; (usar os dedos) para segurar delicadamente | dificultar; obstruir; impedir que os outros façam as coisas sem problemas | recuar; ceder; dar a mão; dobrar, metaforicamente significando ceder}
\end{EntryWithPhonetic}

%%%%%%%%%% 恼 %%%%%%%%%%
\subsection*{恼}\addcontentsline{loh}{figure}{恼 \dpy{nao3}}

\begin{EntryWithPhonetic}{恼}{nao3}{9}{⼼}
  \definition{adj.}{infeliz; preocupado; aflito; angustiado}
  \definition{v.}{perturbar; irritar; incomodar | ficar com raiva; ficar irritado}
\end{EntryWithPhonetic}

\begin{EntryWithPhonetic}{恼羞成怒}{nao3xiu1-cheng2nu4}{9,10,6,9}{⼼,⽺,⼽,⼼}[HSK 7-9]
  \definition{expr.}{ficar furioso de vergonha; ser envergonhado a ponto de ficar com raiva; sentir vergonha a ponto de ficar com raiva; irritação; ficar furioso por causa da humilhação}
\end{EntryWithPhonetic}

%%%%%%%%%% 脑 %%%%%%%%%%
\subsection*{脑}\addcontentsline{loh}{figure}{脑 \dpy{nao3}}

\begin{EntryWithPhonetic}{脑}{nao3}{10}{⾁}
  \definition{s.}{(fisiologia) cérebro | tofu;  substância branca semelhante ao cérebro ou à medula espinhal cerebral | cabeça | a essência de um objeto}
\end{EntryWithPhonetic}

\begin{EntryWithPhonetic}{脑袋}{nao3dai5}{10,11}{⾁,⾐}[HSK 4]
  \definition[颗,个]{s.}{cabeça; a parte mais alta do corpo humano ou a parte mais alta de um animal que contém órgãos como a boca, o nariz, os olhos etc. | mente; cérebro; capacidade de pensar, lembrar, etc.}
\end{EntryWithPhonetic}

\begin{EntryWithPhonetic}{脑瓜}{nao3gua1}{10,5}{⾁,⽠}
  \definition{s.}{crânio | cérebro | cabeça | mente | mentalidade | ideia}
  \seealsoref{脑瓜子}{nao3gua1zi5}
\end{EntryWithPhonetic}

\begin{EntryWithPhonetic}{脑瓜子}{nao3gua1zi5}{10,5,3}{⾁,⽠,⼦}
  \definition{s.}{Coloquial: crânio; cérebro; cabeça; mente; mentalidade; ideia}
  \seealsoref{脑瓜}{nao3gua1}
\end{EntryWithPhonetic}

\begin{EntryWithPhonetic}{脑海}{nao3hai3}{10,10}{⾁,⽔}[HSK 7-9]
  \definition{s.}{cérebro (referindo"-se principalmente às suas funções de pensamento e memória); mente; olho da mente}
\end{EntryWithPhonetic}

\begin{EntryWithPhonetic}{脑筋}{nao3jin1}{10,12}{⾁,⽵}[HSK 7-9]
  \definition[个]{s.}{cérebro; mente; cabeça; refere"-se a habilidades como raciocínio e memória | ideias; conceitos; refere"-se a métodos ou hábitos de pensamento}
\end{EntryWithPhonetic}

\begin{EntryWithPhonetic}{脑子}{nao3 zi5}{10,3}{⾁,⼦}[HSK 5]
  \definition[个]{s.}{cérebro | mente; cabeça; cérebro; inteligência; poder mental; refere"-se à capacidade de pensar, memorizar, raciocinar, etc.; inteligência}
\end{EntryWithPhonetic}

%%%%%%%%%% 闹 %%%%%%%%%%
\subsection*{闹}\addcontentsline{loh}{figure}{闹 \dpy{nao4}}

\begin{EntryWithPhonetic}{闹}{nao4}{8}{⾾}[HSK 4]
  \definition{adj.}{barulhento}
  \definition{v.}{fazer barulho; provocar problemas | dar vazão (à sua raiva, ressentimento, etc.) | sofrer de; ser incomodado por; ocorrer (um desastre ou coisa ruim) | fazer;  entrar em ação | agitar; perturbar | brincar; fazer bagunça}
\end{EntryWithPhonetic}

\begin{EntryWithPhonetic}{闹事}{nao4/shi4}{8,8}{⾾,⼅}[HSK 7-9]
  \definition{v.+compl.}{criar perturbação; causar problemas}
\end{EntryWithPhonetic}

\begin{EntryWithPhonetic}{闹着玩儿}{nao4zhe5wan2r5}{8,11,8,2}{⾾,⽬,⽟,⼉}[HSK 7-9]
  \definition{expr.}{``Estou brincando.''; piada; brincar; fazer algo por diversão; estar brincando}
\end{EntryWithPhonetic}

\begin{EntryWithPhonetic}{闹钟}{nao4 zhong1}{8,9}{⾾,⾦}[HSK 4]
  \definition[个,台,只,款]{s.}{despertador; relógios capazes de tocar alarmes em horários predeterminados}
\end{EntryWithPhonetic}

%%%%%%%%%% 那 %%%%%%%%%%
\subsection*{那}\addcontentsline{loh}{figure}{那 \dpy{ne4}}

\begin{EntryWithPhonetic}{那}{ne4}{6}{⾢}
  \definition{conj.}{então; nesse caso; o mesmo que 那么}
  \definition{pron.}{aquele; aquilo; pronúncia coloquial de 那 (\dpy{na4})}
  \seeref{na1}
  \seeref{na3}
  \seeref{na4}
  \seeref{nei4}
  \seeref{nuo2}
  \seealsoref{那么}{na4 me5}
\end{EntryWithPhonetic}

%%%%%%%%%% 呐 %%%%%%%%%%
\subsection*{呐}\addcontentsline{loh}{figure}{呐 \dpy{ne4}}

\begin{EntryWithPhonetic}{呐}{ne4}{7}{⼝}
  \definition{adj.}{lento; de ritmo lento; hesitante em falar}
  \definition{v.}{sussurrar; falar baixinho}
\end{EntryWithPhonetic}

%%%%%%%%%% 呢 %%%%%%%%%%
\subsection*{呢}\addcontentsline{loh}{figure}{呢 \dpy{ne5}}

\begin{EntryWithPhonetic}{呢}{ne5}{8}{⼝}[HSK 1]
  \definition{part.}{usada no final de frases interrogativas (especificamente perguntas, perguntas de escolha e perguntas retóricas) para indicar um tom interrogativo | usada no final de uma frase declarativa, indica que uma ação ou situação está em andamento | usada em frases para indicar uma pausa (muitas vezes em pares) | usada no final de uma frase declarativa para confirmar um fato e convencer o interlocutor (com um tom de indicação e exagero)}
  \seeref{ni2}
\end{EntryWithPhonetic}

%%%%%%%%%% 哪 %%%%%%%%%%
\subsection*{哪}\addcontentsline{loh}{figure}{哪 \dpy{nei3}}

\begin{EntryWithPhonetic}{哪}{nei3}{9}{⼝}
  \definition{part.}{qual? (interrogativo, seguido de classificador ou numeral-classificador)}
  \seeref{na3}
  \seeref{na5}
\end{EntryWithPhonetic}

%%%%%%%%%% 内 %%%%%%%%%%
\subsection*{内}\addcontentsline{loh}{figure}{内 \dpy{nei4}}

\begin{EntryWithPhonetic}{内}{nei4}{4}{⼌}[HSK 3]
  \definition*{s.}{Sobrenome: Nei}
  \definition{s.}{dentro; interior; parte interna ou lateral |  (antes de um substantivo ou verbo na formação de uma palavra composta) interno | (depois de um substantivo para indicar lugar, tempo, escopo ou limites) dentro; em | coração; mente | esposa ou parentes da esposa}
  \antonymref{外}{wai4}
\end{EntryWithPhonetic}

\begin{EntryWithPhonetic}{内部}{nei4bu4}{4,10}{⼌,⾢}[HSK 4]
  \definition{s.}{interior; dentro; interno; dentro de um determinado intervalo}
\end{EntryWithPhonetic}

\begin{EntryWithPhonetic}{内存}{nei4cun2}{4,6}{⼌,⼦}[HSK 7-9]
  \definition{s.}{RAM (\emph{random access memory}) | capacidade da RAM | memória do computador; memória interna, abreviação de 内存储器 | armazenamento interno}
  \seealsoref{内存储器}{nei4cun2chu3qi4}
  \seealsoref{随机存取存储器}{sui2ji1cun2qu3cun2chu3qi4}
  \seealsoref{随机存取记忆体}{sui2ji1cun2qu3ji4yi4ti3}
\end{EntryWithPhonetic}

\begin{EntryWithPhonetic}{内存储器}{nei4cun2chu3qi4}{4,6,12,16}{⼌,⼦,⼈,⼝}
  \definition{s.}{memória interna}
\end{EntryWithPhonetic}

\begin{EntryWithPhonetic}{内地}{nei4 di4}{4,6}{⼌,⼟}[HSK 6]
  \definition{s.}{interior; sertão | China continental (RPC excluindo Hong Kong e Macau, mas incluindo ilhas como Hainan) | Japão (usado em Taiwan durante a colonização japonesa)}
\end{EntryWithPhonetic}

\begin{EntryWithPhonetic}{内阁}{nei4ge2}{4,9}{⼌,⾨}[HSK 7-9]
  \definition{s.}{gabinete; em alguns países, o órgão executivo máximo é composto pelo primeiro-ministro e vários membros do gabinete (ministros, chefes de gabinete ou outros funcionários de alto escalão)}
\end{EntryWithPhonetic}

\begin{EntryWithPhonetic}{内涵}{nei4han2}{4,11}{⼌,⽔}[HSK 7-9]
  \definition{s.}{intenção; conotação; os atributos essenciais das coisas objetivas refletidos por um conceito}
\end{EntryWithPhonetic}

\begin{EntryWithPhonetic}{内行}{nei4hang2}{4,6}{⼌,⾏}[HSK 7-9]
  \definition{adj.}{habilidoso; proficiente; especialista em; amplo conhecimento e experiência em determinado assunto ou função}
  \definition{s.}{especialista; perito; craque; profissional}
\end{EntryWithPhonetic}

\begin{EntryWithPhonetic}{内科}{nei4ke1}{4,9}{⼌,⽲}[HSK 4]
  \definition{s.}{medicina geral; clínica geral; clínica médica}
\end{EntryWithPhonetic}

\begin{EntryWithPhonetic}{内幕}{nei4mu4}{4,13}{⼌,⼱}[HSK 7-9]
  \definition{s.}{o que acontece nos bastidores; histórias internas}
\end{EntryWithPhonetic}

\begin{EntryWithPhonetic}{内燃机}{nei4ran2ji1}{4,16,6}{⼌,⽕,⽊}
  \definition{s.}{motor de combustão interna}
\end{EntryWithPhonetic}

\begin{EntryWithPhonetic}{内容}{nei4rong2}{4,10}{⼌,⼧}[HSK 3]
  \definition[份,个,项]{s.}{conteúdo; substância; a substância ou significado contido em algo}
\end{EntryWithPhonetic}

\begin{EntryWithPhonetic}{内外}{nei4 wai4}{4,5}{⼌,⼣}[HSK 6]
  \definition{s.}{dentro e fora; nacional e estrangeiro; interno e externo | ao redor; aproximadamente; número aproximado de exibição}
\end{EntryWithPhonetic}

\begin{EntryWithPhonetic}{内向}{nei4xiang4}{4,6}{⼌,⼝}[HSK 7-9]
  \definition{adj.}{(economia, etc.) voltado para o mercado interno; introvertido; descreve a personalidade de uma pessoa como sendo quieta e relutante em expressar seus pensamentos}
\end{EntryWithPhonetic}

\begin{EntryWithPhonetic}{内心}{nei4 xin1}{4,4}{⼌,⼼}[HSK 3]
  \definition{s.}{coração; interior; íntimo do ser}
\end{EntryWithPhonetic}

\begin{EntryWithPhonetic}{内省}{nei4xing3}{4,9}{⼌,⽬}
  \definition{s.}{introspecção}
  \definition{v.}{refletir sobre si mesmo}
\end{EntryWithPhonetic}

\begin{EntryWithPhonetic}{内需}{nei4xu1}{4,14}{⼌,⾬}[HSK 7-9]
  \definition{s.}{Economia: demanda interna}
  \antonymref{外需}{wai4xu1}
\end{EntryWithPhonetic}

\begin{EntryWithPhonetic}{内衣}{nei4 yi1}{4,6}{⼌,⾐}[HSK 6]
  \definition[件,个]{s.}{roupa íntima}
\end{EntryWithPhonetic}

\begin{EntryWithPhonetic}{内在}{nei4zai4}{4,6}{⼌,⼟}[HSK 5]
  \definition{adj.}{intrínseco; algo que existe em si mesmo, mas que não pode ser descoberto através da observação direta | interno; imanente; difícil de perceber}
\end{EntryWithPhonetic}

\begin{EntryWithPhonetic}{内资}{nei4 zi1}{4,10}{⼌,⾙}
  \definition{s.}{capital nacional; financiamento interno; investimento de fontes nacionais}
  \antonymref{外资}{wai4 zi1}
\end{EntryWithPhonetic}

%%%%%%%%%% 那 %%%%%%%%%%
\subsection*{那}\addcontentsline{loh}{figure}{那 \dpy{nei4}}

\begin{EntryWithPhonetic}{那}{nei4}{6}{⾢}
  \definition{conj.}{então; o mesmo que 那么}
  \definition{pron.}{aquele; aquilo; A pronúncia coloquial de 那 (\dpy{na4})}
  \seeref{na1}
  \seeref{na3}
  \seeref{na4}
  \seeref{ne4}
  \seeref{nuo2}
  \seealsoref{那么}{na4 me5}
\end{EntryWithPhonetic}

%%%%%%%%%% 嫩 %%%%%%%%%%
\subsection*{嫩}\addcontentsline{loh}{figure}{嫩 \dpy{nen4}}

\begin{EntryWithPhonetic}{嫩}{nen4}{14}{⼥}[HSK 7-9]
  \definition{adj.}{terno; delicado; recém-nascido e frágil | macio; malpassado; alguns alimentos são rápidos de cozinhar e fáceis de mastigar | claro; suave; algumas cores são claras | sem habilidade; inexperiente}
\end{EntryWithPhonetic}

%%%%%%%%%% 能 %%%%%%%%%%
\subsection*{能}\addcontentsline{loh}{figure}{能 \dpy{neng2}}

\begin{EntryWithPhonetic}{能}{neng2}{10}{⾁}[HSK 1]
  \definition*{s.}{Sobrenome: Neng}
  \definition{adv.}{talvez}
  \definition{s.}{habilidade; capacidade; competência | potência; energia; em física, refere"-se à energia}
  \definition{v.}{poder fazer; ser capaz de | ser possível | entre 不 \dots 不 para expressar obrigação, certeza ou grande probabilidade | poder; ter permissão para | ser bom em fazer algo | permitir}
\end{EntryWithPhonetic}

\begin{EntryWithPhonetic}{能不能}{neng2 bu4 neng2}{10,4,10}{⾁,⼀,⾁}[HSK 3]
  \definition{adv.}{pode ou não pode\dots?}
\end{EntryWithPhonetic}

\begin{EntryWithPhonetic}{能否}{neng2 fou3}{10,7}{⾁,⼝}[HSK 6]
  \definition{adv.}{é possível; se ou não; pode ou não pode; Você consegue?; expressa dúvida, frequentemente usado em perguntas de sim ou não}
\end{EntryWithPhonetic}

\begin{EntryWithPhonetic}{能干}{neng2gan4}{10,3}{⾁,⼲}[HSK 4]
  \definition{adj.}{apto; capaz; competente}
\end{EntryWithPhonetic}

\begin{EntryWithPhonetic}{能够}{neng2 gou4}{10,11}{⾁,⼣}[HSK 2]
  \definition{v.}{poder; ser capaz de; indica que possui uma determinada capacidade ou que atingiu um determinado nível de eficiência | poder; ser capaz de; indica que algo é permitido sob certas condições ou por motivos razoáveis}
\end{EntryWithPhonetic}

\begin{EntryWithPhonetic}{能耗}{neng2hao4}{10,10}{⾁,⽾}[HSK 7-9]
  \definition{s.}{consumo de energia}
\end{EntryWithPhonetic}

\begin{EntryWithPhonetic}{能力}{neng2li4}{10,2}{⾁,⼒}[HSK 3]
  \definition[个,种]{s.}{habilidade; capacidade; aptidão; as condições subjetivas para ser competente para uma tarefa}
\end{EntryWithPhonetic}

\begin{EntryWithPhonetic}{能量}{neng2liang4}{10,12}{⾁,⾥}[HSK 5]
  \definition[种]{s.}{energia; quantidade de energia; Uma grandeza física que mede a capacidade da matéria de realizar trabalho | capacidade; competências; capacidade e papel que uma pessoa pode desempenhar}
\end{EntryWithPhonetic}

\begin{EntryWithPhonetic}{能耐}{neng2nai5}{10,9}{⾁,⽽}[HSK 7-9]
  \definition{adj.}{habilidade; capacidade; destreza; competência | talento}
\end{EntryWithPhonetic}

\begin{EntryWithPhonetic}{能人}{neng2ren2}{10,2}{⾁,⼈}[HSK 7-9]
  \definition{s.}{pessoa capaz | mentes capazes | pessoa de grande calibre}
\end{EntryWithPhonetic}

\begin{EntryWithPhonetic}{能上能下}{neng2shang4neng2xia4}{10,3,10,3}{⾁,⼀,⾁,⼀}
  \definition{s.}{pronto para aceitar qualquer trabalho, alto ou baixo}
\end{EntryWithPhonetic}

\begin{EntryWithPhonetic}{能源}{neng2yuan2}{10,13}{⾁,⽔}[HSK 7-9]
  \definition[种,个]{s.}{energia; fonte de energia; recursos que geram diversas formas de energia, incluindo energia mecânica, térmica, luminosa, eletromagnética e química; exemplos incluem combustíveis, energia hidrelétrica, energia solar e energia eólica}
\end{EntryWithPhonetic}

%%%%%%%%%% 尼 %%%%%%%%%%
\subsection*{尼}\addcontentsline{loh}{figure}{尼 \dpy{ni2}}

\begin{EntryWithPhonetic}{尼}{ni2}{5}{⼫}
  \definition[个,名,位]{s.}{freira budista | freira; convento de freiras}
\end{EntryWithPhonetic}

\begin{EntryWithPhonetic}{尼龙}{ni2long2}{5,5}{⼫,⿓}[HSK 7-9]
  \definition{s.}{Empréstimo linguístico: \emph{nylon}; náilon; poliamida; resinas que contêm ligações amida em suas moléculas, incluindo plásticos feitos com essas resinas, apresentam-se em diversas variedades}
\end{EntryWithPhonetic}

%%%%%%%%%% 呢 %%%%%%%%%%
\subsection*{呢}\addcontentsline{loh}{figure}{呢 \dpy{ni2}}

\begin{EntryWithPhonetic}{呢}{ni2}{8}{⼝}
  \definition{s.}{(tecido feito de) lã; tecido de lã (para roupas pesadas); tecido de lã pesada; revestimento ou roupa de lã}
  \seeref{ne5}
\end{EntryWithPhonetic}

%%%%%%%%%% 泥 %%%%%%%%%%
\subsection*{泥}\addcontentsline{loh}{figure}{泥 \dpy{ni2}}

\begin{EntryWithPhonetic}{泥}{ni2}{8}{⽔}[HSK 6]
  \definition*{s.}{Sobrenome: Ni}
  \definition{s.}{lama; atoleiro | pasta ou polpa; amassado | qualquer matéria pastosa; purê de vegetais ou frutas}
  \seeref{ni4}
\end{EntryWithPhonetic}

\begin{EntryWithPhonetic}{泥潭}{ni2tan2}{8,15}{⽔,⽔}[HSK 7-9]
  \definition{s.}{pântano; charco; atoleiro | poça de lama}
\end{EntryWithPhonetic}

\begin{EntryWithPhonetic}{泥土}{ni2tu3}{8,3}{⽔,⼟}[HSK 7-9]
  \definition[块,堆]{s.}{lama; solo; argila; terra}
\end{EntryWithPhonetic}

%%%%%%%%%% 你 %%%%%%%%%%
\subsection*{你}\addcontentsline{loh}{figure}{你 \dpy{ni3}}

\begin{EntryWithPhonetic}{你}{ni3}{7}{⼈}[HSK 1]
  \definition{pron.}{você (segunda pessoa do singular); refere"-se à pessoa com quem se está conversando | (referindo"-se a qualquer pessoa) você; um; qualquer um | com 我 ou 你 em estruturas paralelas para indicar várias ou muitas pessoas se comportando da mesma maneira}
  \seealsoref{您}{nin2}
  \seealsoref{我}{wo3}
\end{EntryWithPhonetic}

\begin{EntryWithPhonetic}{你的}{ni3 de5}{7,8}{⼈,⽩}
  \definition{pron.}{seu}
\end{EntryWithPhonetic}

\begin{EntryWithPhonetic}{你好}{ni3hao3}{7,6}{⼈,⼥}
  \definition{interj.}{``Olá!''; ``Oi!''}
\end{EntryWithPhonetic}

\begin{EntryWithPhonetic}{你们}{ni3men5}{7,5}{⼈,⼈}[HSK 1]
  \definition{pron.}{você (segunda pessoa do plural); refere"-se a mais de uma pessoa ou a várias pessoas, incluindo a outra parte}
\end{EntryWithPhonetic}

\begin{EntryWithPhonetic}{你们的}{ni3men5 de5}{7,5,8}{⼈,⼈,⽩}
  \definition{pron.}{vossos}
\end{EntryWithPhonetic}

%%%%%%%%%% 拟 %%%%%%%%%%
\subsection*{拟}\addcontentsline{loh}{figure}{拟 \dpy{ni3}}

\begin{EntryWithPhonetic}{拟}{ni3}{7}{⼿}[HSK 7-9]
  \definition{v.}{elaborar; projetar; conceber; rascunhar | Literário: pretender; planejar | imitar; reproduzir; simular | comparar; traçar um paralelo; com a intenção de ser ilícito | conjecturar; imaginar}
\end{EntryWithPhonetic}

\begin{EntryWithPhonetic}{拟定}{ni3ding4}{7,8}{⼿,⼧}[HSK 7-9]
  \definition{v.}{elaborar; redigir; planejar; formular | adivinhar; conjecturar; especular}
\end{EntryWithPhonetic}

%%%%%%%%%% 伲 %%%%%%%%%%
\subsection*{伲}\addcontentsline{loh}{figure}{伲 \dpy{ni4}}

\begin{EntryWithPhonetic}{伲}{ni4}{7}{⼈}
  \definition{pron.}{Dialeto: eu; nós; meu; nosso}
  \seealsoref{你}{ni3}
\end{EntryWithPhonetic}

%%%%%%%%%% 泥 %%%%%%%%%%
\subsection*{泥}\addcontentsline{loh}{figure}{泥 \dpy{ni4}}

\begin{EntryWithPhonetic}{泥}{ni4}{8}{⽔}
  \definition{adj.}{fanático; teimoso; obstinado; cabeçudo}
  \definition{v.}{cobrir ou rebocar com gesso, massa de vidraceiro, etc.}
  \seeref{ni2}
\end{EntryWithPhonetic}

%%%%%%%%%% 逆 %%%%%%%%%%
\subsection*{逆}\addcontentsline{loh}{figure}{逆 \dpy{ni4}}

\begin{EntryWithPhonetic}{逆}{ni4}{9}{⾡}[HSK 7-9]
  \definition{adj.}{contrário; contra; oposto; inverso | traidor; rebelde}
  \definition{adv.}{antecipadamente; com antecedência}
  \definition{s.}{traidor; rebelde}
  \definition{v.}{ir contra; opor-se; desobedecer; resistir; desafiar | Lliterário: saudar; cumprimentar}
  \antonymref{顺}{shun4}
\end{EntryWithPhonetic}

\begin{EntryWithPhonetic}{逆境}{ni4jing4}{9,14}{⾡,⼟}
  \definition[对]{s.}{adversidade; tribulação; circunstâncias adversas; circunstâncias desfavoráveis}
\end{EntryWithPhonetic}

%%%%%%%%%% 匿 %%%%%%%%%%
\subsection*{匿}\addcontentsline{loh}{figure}{匿 \dpy{ni4}}

\begin{EntryWithPhonetic}{匿}{ni4}{10}{⼖}
  \definition{v.}{esconder; ocultar; manter em segredo}
\end{EntryWithPhonetic}

\begin{EntryWithPhonetic}{匿名}{ni4ming2}{10,6}{⼖,⼝}[HSK 7-9]
  \definition{v.}{permanecer anônimo; não revelar o próprio nome}
\end{EntryWithPhonetic}

%%%%%%%%%% 年 %%%%%%%%%%
\subsection*{年}\addcontentsline{loh}{figure}{年 \dpy{nian2}}

\begin{EntryWithPhonetic}{年}{nian2}{6}{⼲}[HSK 1]
  \definition*{s.}{Sobrenome: Nian}
  \definition{clas.}{ano; usado para calcular o número de anos}
  \definition{s.}{ano | idade | um período (época) da história | colheita anual | Ano Novo | artigos para o dia de Ano Novo | um período da vida de uma pessoa; fases da vida humana divididas por idade}
\end{EntryWithPhonetic}

\begin{EntryWithPhonetic}{年初}{nian2 chu1}{6,7}{⼲,⾐}[HSK 3]
  \definition{s.}{o começo do ano; os primeiros dias do ano}
\end{EntryWithPhonetic}

\begin{EntryWithPhonetic}{年代}{nian2dai4}{6,5}{⼲,⼈}[HSK 3]
  \definition[个]{s.}{idade; anos; tempo; um período de tempo com características distintas na história | uma década de um século; período de dez anos}
\end{EntryWithPhonetic}

\begin{EntryWithPhonetic}{年底}{nian2 di3}{6,8}{⼲,⼴}[HSK 3]
  \definition[个]{s.}{fim de ano; o fim do ano; geralmente os últimos dias de dezembro ou o fim do ano}
\end{EntryWithPhonetic}

\begin{EntryWithPhonetic}{年度}{nian2du4}{6,9}{⼲,⼴}[HSK 5]
  \definition{s.}{ano; de acordo com a natureza e as necessidades de um negócio, há um prazo de doze meses com data de início e término definidas}
\end{EntryWithPhonetic}

\begin{EntryWithPhonetic}{年画}{nian2hua4}{6,8}{⼲,⽥}[HSK 7-9]
  \definition{s.}{fotos de Ano Novo (ou Festival da Primavera); durante o Ano Novo Lunar, são afixadas imagens que retratam alegria e prosperidade}
  \seealsoref{年画儿}{nian2hua4r5}
\end{EntryWithPhonetic}

\begin{EntryWithPhonetic}{年画儿}{nian2hua4r5}{6,8,2}{⼲,⽥,⼉}
  \definition{s.}{foto de Ano Novo (Festival da Primavera)}
\end{EntryWithPhonetic}

\begin{EntryWithPhonetic}{年货}{nian2huo4}{6,8}{⼲,⾙}
  \definition{s.}{mercadorias vendidas no Ano Novo Chinês}
\end{EntryWithPhonetic}

\begin{EntryWithPhonetic}{年级}{nian2ji2}{6,6}{⼲,⽷}[HSK 2]
  \definition[个]{s.}{série; ano; níveis divididos de acordo com o tempo de estudo dos alunos na escola}
\end{EntryWithPhonetic}

\begin{EntryWithPhonetic}{年纪}{nian2ji4}{6,6}{⼲,⽷}[HSK 3]
  \definition[把,个]{s.}{idade (de uma pessoa)}
\end{EntryWithPhonetic}

\begin{EntryWithPhonetic}{年龄}{nian2ling2}{6,13}{⼲,⿒}[HSK 5]
  \definition[个,段]{s.}{idade; animais, plantas e outros seres vivos vivem e crescem no mundo durante um determinado número de anos}
\end{EntryWithPhonetic}

\begin{EntryWithPhonetic}{年迈}{nian2mai4}{6,6}{⼲,⾡}[HSK 7-9]
  \definition{adj.}{velho; idoso}
\end{EntryWithPhonetic}

\begin{EntryWithPhonetic}{年前}{nian2 qian2}{6,9}{⼲,⼑}[HSK 5]
  \definition{s.}{(pouco) antes da virada do ano | antes do final do ano | antes do ano novo}
\end{EntryWithPhonetic}

\begin{EntryWithPhonetic}{年轻}{nian2qing1}{6,9}{⼲,⾞}[HSK 2]
  \definition{adj.}{jovem; não muito velho (geralmente se refere a pessoas entre 10 e 20 anos)}
\end{EntryWithPhonetic}

\begin{EntryWithPhonetic}{年限}{nian2xian4}{6,8}{⼲,⾩}[HSK 7-9]
  \definition{s.}{limite de idade; número fixo de anos; o número de anos especificado ou usado como padrão geral}
\end{EntryWithPhonetic}

\begin{EntryWithPhonetic}{年薪}{nian2xin1}{6,16}{⼲,⾋}[HSK 7-9]
  \definition{s.}{salário anual; remuneração anual}[他的年薪已经达到了五十万元。===Seu salário anual chegou a 500.000 yuans.]
\end{EntryWithPhonetic}

\begin{EntryWithPhonetic}{年夜饭}{nian2ye4fan4}{6,8,7}{⼲,⼣,⾷}[HSK 7-9]
  \definition[顿,桌]{s.}{jantar de família na véspera de Ano Novo; um jantar especial realizado na véspera de Ano Novo (na noite de 31 de dezembro)}
\end{EntryWithPhonetic}

\begin{EntryWithPhonetic}{年终}{nian2zhong1}{6,8}{⼲,⽷}[HSK 7-9]
  \definition{s.}{fim de ano; fim do ano}
\end{EntryWithPhonetic}

%%%%%%%%%% 粘 %%%%%%%%%%
\subsection*{粘}\addcontentsline{loh}{figure}{粘 \dpy{nian2}}

\begin{EntryWithPhonetic}{粘}{nian2}{11}{⽶}
  \variantof{黏}
\end{EntryWithPhonetic}

%%%%%%%%%% 黏 %%%%%%%%%%
\subsection*{黏}\addcontentsline{loh}{figure}{黏 \dpy{nian2}}

\begin{EntryWithPhonetic}{黏}{nian2}{17}{⿉}[HSK 7-9]
  \definition{adj.}{pegajoso; adesivo; glutinoso}
\end{EntryWithPhonetic}

%%%%%%%%%% 碾 %%%%%%%%%%
\subsection*{碾}\addcontentsline{loh}{figure}{碾 \dpy{nian3}}

\begin{EntryWithPhonetic}{碾}{nian3}{15}{⽯}
  \definition[台,个]{s.}{rolo e mó; rolo de pedra | rolo compressor}
  \definition{v.}{moer ou descascar com um rolo; esmagar | (literário) cortar e polir (jade, vidro, etc.) | achatar | pisar; pisotear, 轧}
  \seealsoref{辗}{zhan3}
\end{EntryWithPhonetic}

\begin{EntryWithPhonetic}{碾碎}{nian3sui4}{15,13}{⽯,⽯}
  \definition{v.}{pulverizar | esmagar}
\end{EntryWithPhonetic}

%%%%%%%%%% 廿 %%%%%%%%%%
\subsection*{廿}\addcontentsline{loh}{figure}{廿 \dpy{nian4}}

\begin{EntryWithPhonetic}{廿}{nian4}{4}{⼶}
  \definition{num.}{(dialeto) vinte; 20}
\end{EntryWithPhonetic}

%%%%%%%%%% 念 %%%%%%%%%%
\subsection*{念}\addcontentsline{loh}{figure}{念 \dpy{nian4}}

\begin{EntryWithPhonetic}{念}{nian4}{8}{⼼}[HSK 3]
  \definition*{s.}{Sobrenome: Nian}
  \definition{num.}{vinte; 20; capitalização do número 廿}
  \definition{s.}{ideia; pensamento; pensamentos ou intenções internas}
  \definition{v.}{ler em voz alta | estudar; frequentar a escola | considerar; levar em conta | sentir falta; pensar em; pensar sobre; pensar frequentemente sobre}
  \seealsoref{廿}{nian4}
\end{EntryWithPhonetic}

\begin{EntryWithPhonetic}{念念不忘}{nian4nian4-bu2wang4}{8,8,4,7}{⼼,⼼,⼀,⼼}[HSK 7-9]
  \definition{expr.}{inesquecível; ter sempre em mente (expressão idiomática); nunca se esqueça}
\end{EntryWithPhonetic}

\begin{EntryWithPhonetic}{念书}{nian4/shu1}{8,4}{⼼,⼄}[HSK 7-9]
  \definition{v.+compl.}{estudar; ir à escola; frequentar a escola | ler livros}
\end{EntryWithPhonetic}

\begin{EntryWithPhonetic}{念头}{nian4tou5}{8,5}{⼼,⼤}[HSK 7-9]
  \definition[个,种]{s.}{ideia; pensamento; intenção; plano}
\end{EntryWithPhonetic}

%%%%%%%%%% 娘 %%%%%%%%%%
\subsection*{娘}\addcontentsline{loh}{figure}{娘 \dpy{niang2}}

\begin{EntryWithPhonetic}{娘}{niang2}{10}{⼥}[HSK 7-9]
  \definition{adj.}{efeminado; maricas}
  \definition{s.}{mãe | forma de tratamento para uma mulher idosa casada; dirigindo-se a uma mulher casada mais velha ou de idade mais avançada | jovem mulher}
\end{EntryWithPhonetic}

%%%%%%%%%% 酿 %%%%%%%%%%
\subsection*{酿}\addcontentsline{loh}{figure}{酿 \dpy{niang2}}

\begin{EntryWithPhonetic}{酿}{niang2}{14}{⾣}
  \definition{v.}{fermentar; preparar (vinho de arroz)}
  \seeref{niang4}
\end{EntryWithPhonetic}

\begin{EntryWithPhonetic}{酿}{niang4}{14}{⾣}
  \definition{s.}{vinho}
  \definition{v.}{fazer (vinho); fabricar (cerveja) | produzir (mel) | levar a; resultar em; formar gradualmente}
  \seeref{niang2}
\end{EntryWithPhonetic}

\begin{EntryWithPhonetic}{酿造}{niang4zao4}{14,10}{⾣,⾡}[HSK 7-9]
  \definition{v.}{fabricar (cerveja, etc.); produzir (vinho, vinagre, etc.); fabricar utilizando fermentação}
\end{EntryWithPhonetic}

%%%%%%%%%% 鸟 %%%%%%%%%%
\subsection*{鸟}\addcontentsline{loh}{figure}{鸟 \dpy{niao3}}

\begin{EntryWithPhonetic}{鸟}{niao3}{5}{⿃}[HSK 2][Kangxi 196]
  \definition*{s.}{Sobrenome: Niao}
  \definition[只,群]{s.}{pássaro; ave}
  \seeref{diao3}
\end{EntryWithPhonetic}

\begin{EntryWithPhonetic}{鸟巢}{niao3chao2}{5,11}{⿃,⼮}[HSK 7-9]
  \definition[个,些]{s.}{ninho de pássaro; ninhos usados por pássaros para pôr ovos e incubar seus filhotes}
\end{EntryWithPhonetic}

\begin{EntryWithPhonetic}{鸟儿}{niao3r5}{5,2}{⿃,⼉}
  \definition[只]{s.}{pássaro | ave}
\end{EntryWithPhonetic}

%%%%%%%%%% 尿 %%%%%%%%%%
\subsection*{尿}\addcontentsline{loh}{figure}{尿 \dpy{niao4}}

\begin{EntryWithPhonetic}{尿}{niao4}{7}{⼫}[HSK 7-9]
  \definition[泡]{s.}{urina}
  \definition{v.}{urinar; fazer xixi | Coloquial: fazer água; urinar; mijar}
  \seeref{sui1}
\end{EntryWithPhonetic}

%%%%%%%%%% 溺 %%%%%%%%%%
\subsection*{溺}\addcontentsline{loh}{figure}{溺 \dpy{niao4}}

\begin{EntryWithPhonetic}{溺}{niao4}{13}{⽔}
  \variantof{尿}
\end{EntryWithPhonetic}

%%%%%%%%%% 捏 %%%%%%%%%%
\subsection*{捏}\addcontentsline{loh}{figure}{捏 \dpy{nie1}}

\begin{EntryWithPhonetic}{捏}{nie1}{10}{⼿}[HSK 7-9]
  \definition{v.}{beliscar; segurar entre os dedos; usar o polegar e os outros dedos para pinçar | moldar; amassar com os dedos; usar os dedos para moldar o objeto macio | fabricar; compor; apresentar deliberadamente uma declaração falsa como se fosse um fato | juntar; unir; unir duas coisas ou duas pessoas | beliscar; usar as mãos para empurrar, pressionar, beliscar e amassar o corpo pode promover a circulação sanguínea, aumentar a resistência da pele e regular a função nervosa; pressionar firmemente com a palma da mão}
\end{EntryWithPhonetic}

%%%%%%%%%% 您 %%%%%%%%%%
\subsection*{您}\addcontentsline{loh}{figure}{您 \dpy{nin2}}

\begin{EntryWithPhonetic}{您}{nin2}{11}{⼼}[HSK 1]
  \definition{pron.}{você; a forma de tratamento respeitosa da segunda pessoa do singular 你}
  \seealsoref{你}{ni3}
\end{EntryWithPhonetic}

%%%%%%%%%% 宁 %%%%%%%%%%
\subsection*{宁}\addcontentsline{loh}{figure}{宁 \dpy{ning2}}

\begin{EntryWithPhonetic}{宁}{ning2}{5}{⼧}
  \definition*{s.}{Região Autônoma de Ningxia Hui, abreviação de 宁夏回族自治区 | outro nome para Nanquim, 南京 | Sobrenome: Ning}
  \definition{adj.}{calmo, pacífico, sereno; tranquilo}
  \definition{v.}{Literário: fazer uma visita (aos pais ou aos mais velhos); | Literário: pacificar; apaziguar}
  \seeref{ning4}
  \seealsoref{南京}{nan2jing1}
  \seealsoref{宁夏回族自治区}{ning2xia4 hui2zu2 zi4zhi4qu1}
\end{EntryWithPhonetic}

\begin{EntryWithPhonetic}{宁静}{ning2 jing4}{5,14}{⼧,⾭}[HSK 4]
  \definition{adj.}{calmo; tranquilo; pacífico}
\end{EntryWithPhonetic}

\begin{EntryWithPhonetic}{宁夏回族自治区}{ning2xia4 hui2zu2 zi4zhi4qu1}{5,10,6,11,6,8,4}{⼧,⼢,⼞,⽅,⾃,⽔,⼖}
  \definition*{s.}{Região Autônoma de Ningxia Hui}
\end{EntryWithPhonetic}

%%%%%%%%%% 拧 %%%%%%%%%%
\subsection*{拧}\addcontentsline{loh}{figure}{拧 \dpy{ning2}}

\begin{EntryWithPhonetic}{拧}{ning2}{8}{⼿}[HSK 7-9]
  \definition{v.}{torcer | beliscar; torcer a pele com os dedos e virá-la com força}
  \seeref{ning3}
  \seeref{ning4}
\end{EntryWithPhonetic}

%%%%%%%%%% 柠 %%%%%%%%%%
\subsection*{柠}\addcontentsline{loh}{figure}{柠 \dpy{ning2}}

\begin{EntryWithPhonetic}{柠}{ning2}{9}{⽊}
  \definition{s.}{limão}
\end{EntryWithPhonetic}

\begin{EntryWithPhonetic}{柠檬}{ning2meng2}{9,17}{⽊,⽊}
  \definition[个,片,只]{s.}{limão}
\end{EntryWithPhonetic}

%%%%%%%%%% 凝 %%%%%%%%%%
\subsection*{凝}\addcontentsline{loh}{figure}{凝 \dpy{ning2}}

\begin{EntryWithPhonetic}{凝}{ning2}{16}{⼎}
  \definition{adv.}{atentamente; atenção fixa}
  \definition{v.}{congelar; coagular; coalhar; condensar | contemplar; olhar pensativamente}
\end{EntryWithPhonetic}

\begin{EntryWithPhonetic}{凝固}{ning2gu4}{16,8}{⼎,⼞}[HSK 7-9]
  \definition{v.}{estagnar; parar, não se move mais nem muda de direção}
\end{EntryWithPhonetic}

\begin{EntryWithPhonetic}{凝聚}{ning2ju4}{16,14}{⼎,⽿}[HSK 7-9]
  \definition{v.}{condensar (o vapor); coagular; coalhar (os fluidos) | reunir; acumular; juntar}
\end{EntryWithPhonetic}

%%%%%%%%%% 拧 %%%%%%%%%%
\subsection*{拧}\addcontentsline{loh}{figure}{拧 \dpy{ning3}}

\begin{EntryWithPhonetic}{拧}{ning3}{8}{⼿}[HSK 7-9]
  \definition{adj.}{errado; equivocado; de cabeça para baixo; oposto}
  \definition{v.}{torcer; parafusar | divergir; discordar; estar em desacordo}
  \seeref{ning2}
  \seeref{ning4}
\end{EntryWithPhonetic}

\begin{EntryWithPhonetic}{拧开}{ning3kai1}{8,4}{⼿,⼶}
  \definition{v.}{ligar ou desligar (girando um botão) | girar (a maçaneta de uma porta) | abrir (uma torneira) | desenroscar (uma tampa) | desaparafusar | arrancar à força}
\end{EntryWithPhonetic}

%%%%%%%%%% 宁 %%%%%%%%%%
\subsection*{宁}\addcontentsline{loh}{figure}{宁 \dpy{ning4}}

\begin{EntryWithPhonetic}{宁}{ning4}{5}{⼧}
  \definition*{s.}{Sobrenome: Ning}
  \definition{adv.}{preferir; preferiria; melhor | Literário: como pôde ser; poderia haver}
  \seeref{ning2}
\end{EntryWithPhonetic}

\begin{EntryWithPhonetic}{宁可}{ning4ke3}{5,5}{⼧,⼝}[HSK 7-9]
  \definition{adv.}{melhor; preferiria; isso indica a escolha feita após comparar as vantagens e desvantagens de dois lados (frequentemente repetindo 与其 no texto anterior ou 也不 no texto seguinte)}
  \seealsoref{宁可…也不…}{ning4ke3 ye3bu4}
  \seealsoref{与其…宁可…}{yu3qi2 ning4ke3}
\end{EntryWithPhonetic}

\begin{EntryWithPhonetic}{宁可…也不…}{ning4ke3 ye3bu4}{5,5,3,4}{⼧,⼝,⼄,⼀}
  \definition{conj.}{preferiria\dots do que\dots}
\end{EntryWithPhonetic}

\begin{EntryWithPhonetic}{宁可…也要…}{ning4ke3 ye3yao4}{5,5,3,9}{⼧,⼝,⼄,⾑}
  \definition{conj.}{mesmo que tenhamos que\dots nós iremos\dots}
\end{EntryWithPhonetic}

\begin{EntryWithPhonetic}{宁肯}{ning4ken3}{5,8}{⼧,⾁}
  \definition{conj.}{mais\dots do que\dots, melhor\dots do que\dots}
\end{EntryWithPhonetic}

\begin{EntryWithPhonetic}{宁愿}{ning4yuan4}{5,14}{⼧,⽕}[HSK 7-9]
  \definition{adv.}{em vez disso; melhor; mais dispostos e mais ansiosos para escolher uma determinada situação ou abordagem}[他宁愿看书,也不看电视。===Ele prefere ler livros a assistir televisão.]
\end{EntryWithPhonetic}

%%%%%%%%%% 拧 %%%%%%%%%%
\subsection*{拧}\addcontentsline{loh}{figure}{拧 \dpy{ning4}}

\begin{EntryWithPhonetic}{拧}{ning4}{8}{⼿}
  \definition{adj.}{teimoso}
  \seeref{ning2}
  \seeref{ning3}
\end{EntryWithPhonetic}

%%%%%%%%%% 牛 %%%%%%%%%%
\subsection*{牛}\addcontentsline{loh}{figure}{牛 \dpy{niu2}}

\begin{EntryWithPhonetic}{牛}{niu2}{4}{⽜}[HSK 3,5][Kangxi 93]
  \definition*{s.}{Sobrenome: Niu}
  \definition{adj.}{muito capaz ou bom; descreve pessoas ou coisas como sendo muito capazes, muito competentes | teimoso; arrogante; descreve uma pessoa que é muito orgulhosa ou muito insistente em suas opiniões, difícil de mudar}
  \definition{clas.}{Newton (medida física de força)}
  \definition[头]{s.}{gado; boi | niu (nona das vinte e oito constelações em que a esfera celeste foi dividida, consistindo de seis estrelas, três em Áries e três em Sagitário)}
\end{EntryWithPhonetic}

\begin{EntryWithPhonetic}{牛顿}{niu2dun4}{4,10}{⽜,⾴}
  \definition*{s.}{Newton (nome) | N; Newton, unidade de força do SI}
\end{EntryWithPhonetic}

\begin{EntryWithPhonetic}{牛郎织女}{niu2 lang2 zhi1nv3}{4,8,8,3}{⽜,⾢,⽷,⼥}
  \definition*{s.}{Vaqueiro e Tecelã (personagens de contos folclóricos) | Altair e Vega (estrelas)}[我们这些牛郎织女都恨透了那条无情的``天河''。===Nós, o Vaqueiro e a Tecelã, odiamos a implacável ``Via Láctea''.]
  \definition{s.}{marido e mulher que vivem longe um do outro}
\end{EntryWithPhonetic}

\begin{EntryWithPhonetic}{牛奶}{niu2nai3}{4,5}{⽜,⼥}[HSK 1]
  \definition[杯,袋,瓶,盒,箱,桶]{s.}{leite}
\end{EntryWithPhonetic}

\begin{EntryWithPhonetic}{牛人}{niu2ren2}{4,2}{⽜,⼈}
  \definition{s.}{(coloquial) o cara | verdadeiro especialista | \emph{badass}}
\end{EntryWithPhonetic}

\begin{EntryWithPhonetic}{牛肉}{niu2rou4}{4,6}{⽜,⾁}
  \definition{s.}{carne de vaca | bife}
\end{EntryWithPhonetic}

\begin{EntryWithPhonetic}{牛仔裤}{niu2zai3ku4}{4,5,12}{⽜,⼈,⾐}[HSK 5]
  \definition[条]{s.}{calças jeans; calças geralmente feitas de tecido jeans azul grosso}
\end{EntryWithPhonetic}

%%%%%%%%%% 扭 %%%%%%%%%%
\subsection*{扭}\addcontentsline{loh}{figure}{扭 \dpy{niu3}}

\begin{EntryWithPhonetic}{扭}{niu3}{7}{⼿}[HSK 6]
  \definition{v.}{virar-se; girar | torcer; girar | torcer; luxar | rolar; balançar (ao caminhar) | agarrar; pegar;  lutar com}
\end{EntryWithPhonetic}

\begin{EntryWithPhonetic}{扭曲}{niu3qu1}{7,6}{⼿,⽈}[HSK 7-9]
  \definition{v.}{torcer; distorcer; deformação torsional | deformar; distorcer; deturpar; a distorção causa deformação e distorção}
\end{EntryWithPhonetic}

\begin{EntryWithPhonetic}{扭头}{niu3/tou2}{7,5}{⼿,⼤}[HSK 7-9]
  \definition{v.+compl.}{desviar o olhar; virar as costas | virar (em volta) | dar meia-volta | virar a cabeça}
\end{EntryWithPhonetic}

\begin{EntryWithPhonetic}{扭转}{niu3zhuan3}{7,8}{⼿,⾞}[HSK 7-9]
  \definition{v.}{virar-se; inverter a marcha | dar meia-volta; reverter; corrigir situações anormais ou alterar circunstâncias desfavoráveis}
\end{EntryWithPhonetic}

%%%%%%%%%% 纽 %%%%%%%%%%
\subsection*{纽}\addcontentsline{loh}{figure}{纽 \dpy{niu3}}

\begin{EntryWithPhonetic}{纽}{niu3}{7}{⽷}
  \definition*{s.}{Sobrenome: Niu}
  \definition[条]{s.}{alça; puxador | botão | vínculo; amarra | fruto ou melão recém-formado em uma trepadeira | Obsoleto: consoante inicial (linguística) | ligação; eixo}
  \definition{v.}{Dialeto: abotoar}
\end{EntryWithPhonetic}

\begin{EntryWithPhonetic}{纽带}{niu3dai4}{7,9}{⽷,⼱}[HSK 7-9]
  \definition[根]{s.}{laço; elo; vínculo; ligação; conexão; pessoas ou coisas que podem servir como elo}
\end{EntryWithPhonetic}

\begin{EntryWithPhonetic}{纽扣}{niu3kou4}{7,6}{⽷,⼿}[HSK 7-9]
  \definition[粒,颗,枚,个]{s.}{botão; objetos pequenos, redondos ou planos, que podem ser usados para fechar roupas, etc.}
\end{EntryWithPhonetic}

%%%%%%%%%% 农 %%%%%%%%%%
\subsection*{农}\addcontentsline{loh}{figure}{农 \dpy{nong2}}

\begin{EntryWithPhonetic}{农}{nong2}{6}{⼍}
  \definition*{s.}{Sobrenome: Nong}
  \definition{s.}{agricultura; criação de animais | camponês; fazendeiro}
\end{EntryWithPhonetic}

\begin{EntryWithPhonetic}{农产品}{nong2 chan3 pin3}{6,6,9}{⼍,⼇,⼝}[HSK 5]
  \definition[批]{s.}{produtos agrícolas}
\end{EntryWithPhonetic}

\begin{EntryWithPhonetic}{农场}{nong2chang3}{6,6}{⼍,⼟}[HSK 7-9]
  \definition[座,家]{s.}{fazenda; empresas que utilizam máquinas e se dedicam à produção agrícola em larga escala}
\end{EntryWithPhonetic}

\begin{EntryWithPhonetic}{农村}{nong2cun1}{6,7}{⼍,⽊}[HSK 3]
  \definition{s.}{aldeia; campo; área rural; locais onde vivem os trabalhadores principalmente dedicados à produção agrícola}
\end{EntryWithPhonetic}

\begin{EntryWithPhonetic}{农历}{nong2li4}{6,4}{⼍,⼚}[HSK 7-9]
  \definition{s.}{o calendário lunar; o calendário tradicional chinês}
\end{EntryWithPhonetic}

\begin{EntryWithPhonetic}{农民}{nong2min2}{6,5}{⼍,⽒}[HSK 3]
  \definition[个,位,名,些]{s.}{fazendeiro; camponês; campesinato; trabalhadores que participam da produção agrícola há muito tempo}
\end{EntryWithPhonetic}

\begin{EntryWithPhonetic}{农民工}{nong2min2gong1}{6,5,3}{⼍,⽒,⼯}[HSK 7-9]
  \definition{s.}{trabalhadores migrantes; trabalhadores com registro de domicílio agrícola que exercem atividades não agrícolas em áreas urbanas}
\end{EntryWithPhonetic}

\begin{EntryWithPhonetic}{农业}{nong2ye4}{6,5}{⼍,⼀}[HSK 3]
  \definition{s.}{agricultura}
\end{EntryWithPhonetic}

\begin{EntryWithPhonetic}{农作物}{nong2zuo4wu4}{6,7,8}{⼍,⼈,⽜}[HSK 7-9]
  \definition{s.}{culturas agrícolas; plantas agrícolas; termo genérico para diversas culturas agrícolas, como grãos, oleaginosas, hortaliças, algodão e tabaco}
\end{EntryWithPhonetic}

%%%%%%%%%% 浓 %%%%%%%%%%
\subsection*{浓}\addcontentsline{loh}{figure}{浓 \dpy{nong2}}

\begin{EntryWithPhonetic}{浓}{nong2}{9}{⽔}[HSK 4]
  \definition{adj.}{denso; espesso; concentrado; um líquido ou gás que contém mais de um determinado ingrediente | grande; forte; profundo (de grau ou extensão) | profundo; (algumas cores) escuro}
\end{EntryWithPhonetic}

\begin{EntryWithPhonetic}{浓厚}{nong2hou4}{9,9}{⽔,⼚}[HSK 7-9]
  \definition{adj.}{muito forte (de cor, interesse, intenção, atmosfera, etc.); descreve algo como sendo de grande interesse ou significativo em termos de cor, visibilidade, atmosfera, etc. | espesso; denso; descreve fumaça, neblina ou nuvens como abundantes e densas}
\end{EntryWithPhonetic}

\begin{EntryWithPhonetic}{浓缩}{nong2suo1}{9,14}{⽔,⽷}[HSK 7-9]
  \definition{v.}{concentrar; condensar; a concentração de uma solução é aumentada pela evaporação do solvente através de métodos como o aquecimento | enriquecer; geralmente se refere à redução das partes desnecessárias de algo, de modo que o conteúdo das partes necessárias aumente relativamente}
\end{EntryWithPhonetic}

\begin{EntryWithPhonetic}{浓郁}{nong2yu4}{9,8}{⽔,⾢}[HSK 7-9]
  \definition{adj.}{(perfume, fragrância, aroma, etc.) forte | denso; espesso; exuberante | (interesse, cor, etc.) forte; rico}
\end{EntryWithPhonetic}

\begin{EntryWithPhonetic}{浓重}{nong2zhong4}{9,9}{⽔,⾥}[HSK 7-9]
  \definition{adj.}{denso; espesso; forte; (fumaça, cheiro, cor, etc.) muito denso e pesado}
\end{EntryWithPhonetic}

%%%%%%%%%% 弄 %%%%%%%%%%
\subsection*{弄}\addcontentsline{loh}{figure}{弄 \dpy{nong4}}

\begin{EntryWithPhonetic}{弄}{nong4}{7}{⼶}[HSK 2]
  \definition{v.}{fazer, realizar; tratar; organizar | obter; buscar; tentar conseguir; encontrar uma maneira de conseguir | brincar com; enganar | pregar uma peça; brincar; manipular | mexer com; perturbar}
  \seeref{long4}
\end{EntryWithPhonetic}

\begin{EntryWithPhonetic}{弄虚作假}{nong4xu1-zuo4jia3}{7,11,7,11}{⼶,⾌,⼈,⼈}[HSK 7-9]
  \definition{expr.}{recorrer ao engano; empregar artimanhas; usar de artifícios; pregar peças; praticar fraude; humilhar-se; fazer truques e enganar as pessoas}
\end{EntryWithPhonetic}

%%%%%%%%%% 奴 %%%%%%%%%%
\subsection*{奴}\addcontentsline{loh}{figure}{奴 \dpy{nu2}}

\begin{EntryWithPhonetic}{奴}{nu2}{5}{⼥}
  \definition{pron.}{Obsoleto: eu, mim, sua humilde serva, esta garota (nome autoproclamado de uma garota, frequentemente encontrado na literatura vernácula antiga)}
  \definition[个,位]{s.}{servo; escravo; pessoas da sociedade antiga que eram oprimidas, exploradas e escravizadas, e que não tinham liberdade pessoal nem direitos políticos}
  \definition{v.}{escravizar; tornar alguém escravo | escravizar;  tratar e usar como escravo}
  \seealsoref{主}{zhu3}
  \antonymref{主}{zhu3}
\end{EntryWithPhonetic}

\begin{EntryWithPhonetic}{奴隶}{nu2li4}{5,8}{⼥,⾪}[HSK 7-9]
  \definition[个]{s.}{escravo; as pessoas que trabalhavam para donos de escravos e não tinham liberdade pessoal eram frequentemente compradas, vendidas ou mortas arbitrariamente por esses donos}
\end{EntryWithPhonetic}

%%%%%%%%%% 努 %%%%%%%%%%
\subsection*{努}\addcontentsline{loh}{figure}{努 \dpy{nu3}}

\begin{EntryWithPhonetic}{努}{nu3}{7}{⼒}
  \definition{v.}{(coloquial) aplicar (a força de alguém); exercer (o esforço de alguém) | (dialeto) machucar-se por esforço excessivo | projetar-se; inchar | aplicar (força); exercer (esforço); usar}
  \seealsoref{呶}{nao2}
\end{EntryWithPhonetic}

\begin{EntryWithPhonetic}{努力}{nu3li4}{7,2}{⼒,⼒}[HSK 2]
  \definition{adj.}{extenuante; árduo | diligente; trabalhador; quem faz as coisas com o máximo de capacidade ou esforço possível}
  \definition{s.}{esforço; tentativa; fazer o melhor possível}
  \definition{v.}{fazer grandes esforços; esforçar"-se; empenhar"-se | esforçar"-se; usar toda a força possível}
\end{EntryWithPhonetic}

%%%%%%%%%% 怒 %%%%%%%%%%
\subsection*{怒}\addcontentsline{loh}{figure}{怒 \dpy{nu4}}

\begin{EntryWithPhonetic}{怒}{nu4}{9}{⼼}
  \definition{adj.}{zangado; furioso | feroz; forte; descreve um forte impulso}
  \definition{adv.}{com força; vigorosamente; dinamicamente | com raiva}
  \definition{s.}{raiva; fúria}
  \definition{v.}{enfurecer-se; ficar com raiva}
\end{EntryWithPhonetic}

\begin{EntryWithPhonetic}{怒放}{nu4fang4}{9,8}{⼼,⽅}
  \definition{v.}{florescer em plena floração}
\end{EntryWithPhonetic}

\begin{EntryWithPhonetic}{怒骂}{nu4ma4}{9,9}{⼼,⾺}
  \definition{v.}{praguejar de raiva}
\end{EntryWithPhonetic}

%%%%%%%%%% 暖 %%%%%%%%%%
\subsection*{暖}\addcontentsline{loh}{figure}{暖 \dpy{nuan3}}

\begin{EntryWithPhonetic}{暖}{nuan3}{13}{⽇}[HSK 5]
  \definition{adj.}{caloroso; cordial}
  \definition{v.}{aquecer; esquentar; aquecer algo ou aquecer o corpo}
\end{EntryWithPhonetic}

\begin{EntryWithPhonetic}{暖烘烘}{nuan3hong1hong1}{13,10,10}{⽇,⽕,⽕}[HSK 7-9]
  \definition{adj.}{agradável e quentinho; aconchegante; acolhedor | emocionante}
\end{EntryWithPhonetic}

\begin{EntryWithPhonetic}{暖和}{nuan3huo5}{13,8}{⽇,⼝}[HSK 3]
  \definition{adj.}{morno; nem frio nem quente}
  \definition{v.}{aquecer; esquentar}
\end{EntryWithPhonetic}

\begin{EntryWithPhonetic}{暖气}{nuan3qi4}{13,4}{⽇,⽓}[HSK 4]
  \definition[个,种]{s.}{aquecedor; aquecimento; aquecimento central}
\end{EntryWithPhonetic}

%%%%%%%%%% 那 %%%%%%%%%%
\subsection*{那}\addcontentsline{loh}{figure}{那 \dpy{nuo2}}

\begin{EntryWithPhonetic}{那}{nuo2}{6}{⾢}
  \definition*{s.}{Sobrenome: Nuo}
  \seeref{na1}
  \seeref{na3}
  \seeref{na4}
  \seeref{ne4}
  \seeref{nei4}
\end{EntryWithPhonetic}

%%%%%%%%%% 挪 %%%%%%%%%%
\subsection*{挪}\addcontentsline{loh}{figure}{挪 \dpy{nuo2}}

\begin{EntryWithPhonetic}{挪}{nuo2}{9}{⼿}[HSK 7-9]
  \definition{v.}{mover; deslocar; transportar}
\end{EntryWithPhonetic}

%%%%%%%%%% 诺 %%%%%%%%%%
\subsection*{诺}\addcontentsline{loh}{figure}{诺 \dpy{nuo4}}

\begin{EntryWithPhonetic}{诺}{nuo4}{10}{⾔}
  \definition*{s.}{Sobrenome: Nuo}
  \definition{interj.}{`Sim!''}
  \definition{v.}{prometer}
\end{EntryWithPhonetic}

\begin{EntryWithPhonetic}{诺贝尔奖}{nuo4bei4'er3 jiang3}{10,4,5,9}{⾔,⾙,⼩,⼤}
  \definition*{s.}{Prêmio Nobel}
\end{EntryWithPhonetic}

\begin{EntryWithPhonetic}{诺奖}{nuo4jiang3}{10,9}{⾔,⼤}
  \definition*{s.}{Prêmio Nobel, abreviação de 诺贝尔奖}
  \seealsoref{诺贝尔奖}{nuo4bei4'er3 jiang3}
\end{EntryWithPhonetic}

\begin{EntryWithPhonetic}{诺言}{nuo4yan2}{10,7}{⾔,⾔}[HSK 7-9]
  \definition[个]{s.}{promessa}[他总是兑现他的诺言。===Ele sempre cumpre suas promessas.]
\end{EntryWithPhonetic}

%%%%%%%%%% 女 %%%%%%%%%%
\subsection*{女}\addcontentsline{loh}{figure}{女 \dpy{nv3}}

\begin{EntryWithPhonetic}{女}{nv3}{3}{⼥}[HSK 1][Kangxi 38]
  \definition{adj.}{mulher; feminino | fêmea (de certos animais)}
  \definition{s.}{menina; filha | nü, uma das mansões lunares | mulher}
  \antonymref{男}{nan2}
\end{EntryWithPhonetic}

\begin{EntryWithPhonetic}{女儿}{nv3'er2}{3,2}{⼥,⼉}[HSK 1]
  \definition[个]{s.}{menina; filha}
  \seealsoref{儿子}{er2zi5}
\end{EntryWithPhonetic}

\begin{EntryWithPhonetic}{女孩}{nv3hai2}{3,9}{⼥,⼦}
  \definition{s.}{menina | garota}
\end{EntryWithPhonetic}

\begin{EntryWithPhonetic}{女孩儿}{nv3 hai2r5}{3,9,2}{⼥,⼦,⼉}[HSK 1]
  \definition{s.}{garota; menina; atualmente também se refere a mulher adolescente | filha}
\end{EntryWithPhonetic}

\begin{EntryWithPhonetic}{女朋友}{nv3 peng2 you5}{3,8,4}{⼥,⽉,⼜}[HSK 1]
  \definition{s.}{namorada}
\end{EntryWithPhonetic}

\begin{EntryWithPhonetic}{女人}{nv3 ren2}{3,2}{⼥,⼈}[HSK 1]
  \definition[个,位]{s.}{mulher adulta}
\end{EntryWithPhonetic}

\begin{EntryWithPhonetic}{女生}{nv3 sheng1}{3,5}{⼥,⽣}[HSK 1]
  \definition[个]{s.}{estudante; aluna; estudante do sexo feminino | menina; jovem mulher}
\end{EntryWithPhonetic}

\begin{EntryWithPhonetic}{女士}{nv3shi4}{3,3}{⼥,⼠}[HSK 4]
  \definition{pron.}{Sra.; Senhorita; Senhora; título honorífico para mulheres (agora usado em contextos diplomáticos)}
  \definition[位,名,个,些]{s.}{senhora; madame}
\end{EntryWithPhonetic}

\begin{EntryWithPhonetic}{女王}{nv3wang2}{3,4}{⼥,⽟}
  \definition{s.}{rainha}
\end{EntryWithPhonetic}

\begin{EntryWithPhonetic}{女性}{nv3 xing4}{3,8}{⼥,⼼}[HSK 5]
  \definition[个,位,名]{s.}{mulher; feminino; feminilidade}
  \antonymref{男性}{nan2 xing4}
\end{EntryWithPhonetic}

\begin{EntryWithPhonetic}{女婿}{nv3xu5}{3,12}{⼥,⼥}[HSK 7-9]
  \definition[个,位]{s.}{genro; marido da filha | em algumas regiões, refere"-se ao marido}
\end{EntryWithPhonetic}

\begin{EntryWithPhonetic}{女子}{nv3 zi3}{3,3}{⼥,⼦}[HSK 3]
  \definition[位,名,个]{s.}{mulher; feminino; pessoa do sexo feminino}
\end{EntryWithPhonetic}

%%%%%%%%%% 虐 %%%%%%%%%%
\subsection*{虐}\addcontentsline{loh}{figure}{虐 \dpy{nve4}}

\begin{EntryWithPhonetic}{虐}{nve4}{9}{⾌}
  \definition{adj.}{cruel; tirânico; brutal e cruel}
\end{EntryWithPhonetic}

\begin{EntryWithPhonetic}{虐待}{nve4dai4}{9,9}{⾌,⼻}[HSK 7-9]
  \definition{v.}{maltratar; tiranizar; tratar com métodos cruéis e impiedosos}
\end{EntryWithPhonetic}

%%%%% EOF %%%%%


 %%%
%%% O
%%%
\section*{O}\addcontentsline{toc}{section}{O}\addcontentsline{loh}{figure}{\#\#\#\#\#\#\#\# O}

%%%%%%%%%% 喔 %%%%%%%%%%
\subsection*{喔}\addcontentsline{loh}{figure}{喔 \dpy{o1}}

\begin{EntryWithPhonetic}{喔}{o1}{12}{⼝}
  \definition{interj.}{``Oh!'', ``Entendi!'', usado para indicar realização, compreensão}
\end{EntryWithPhonetic}

%%%%%%%%%% 哦 %%%%%%%%%%
\subsection*{哦}\addcontentsline{loh}{figure}{哦 \dpy{o2}}

\begin{EntryWithPhonetic}{哦}{o2}{10}{⼝}
  \definition{interj.}{``Oh!''; usado para indicar dúvida ou surpresa}
  \seeref{e2}
  \seeref{o4}
  \seeref{o5}
\end{EntryWithPhonetic}

\begin{EntryWithPhonetic}{哦}{o4}{10}{⼝}[HSK 7-9]
  \definition{interj.}{``Oh!''; usado para indicar que acabou de aprender algo}
  \seeref{e2}
  \seeref{o2}
  \seeref{o5}
\end{EntryWithPhonetic}

\begin{EntryWithPhonetic}{哦}{o5}{10}{⼝}
  \definition{part.}{usado no final da frase para indicar que uma pessoa está afirmando um fato que a outra pessoa não sabe | usado no final da frase para transmitir informalidade, calor, simpatia ou intimidade}
  \seeref{e2}
  \seeref{o2}
  \seeref{o4}
\end{EntryWithPhonetic}

%%%%%%%%%% 区 %%%%%%%%%%
\subsection*{区}\addcontentsline{loh}{figure}{区 \dpy{ou1}}

\begin{EntryWithPhonetic}{区}{ou1}{4}{⼖}
  \definition*{s.}{Sobrenome: Ou}
  \seeref{qu1}
\end{EntryWithPhonetic}

%%%%%%%%%% 欧 %%%%%%%%%%
\subsection*{欧}\addcontentsline{loh}{figure}{欧 \dpy{ou1}}

\begin{EntryWithPhonetic}{欧}{ou1}{8}{⽋}
  \definition*{s.}{Europa, abreviação de 欧洲 | Sobrenome: Ou}
  \seealsoref{欧洲}{ou1zhou1}
\end{EntryWithPhonetic}

\begin{EntryWithPhonetic}{欧盟}{ou1meng2}{8,13}{⽋,⽫}
  \definition*{s.}{União Europeia (EU)}
\end{EntryWithPhonetic}

\begin{EntryWithPhonetic}{欧阳询}{ou1yang2 xun2}{8,6,8}{⽋,⾩,⾔}
  \definition*{s.}{Ouyang Xun (557--641), um dos quatro grandes calígrafos do início da dinastia Tang (唐初四大家)}
  \seealsoref{唐初四大家}{tang2 chu1 si4 da4jia1}
\end{EntryWithPhonetic}

\begin{EntryWithPhonetic}{欧洲}{ou1zhou1}{8,9}{⽋,⽔}
  \definition*{s.}{Europa}
\end{EntryWithPhonetic}

\begin{EntryWithPhonetic}{欧洲共同体}{ou1zhou1 gong4tong2ti3}{8,9,6,6,7}{⽋,⽔,⼋,⼝,⼈}
  \definition*{s.}{Comunidade Europeia}
\end{EntryWithPhonetic}

\begin{EntryWithPhonetic}{欧洲人}{ou1zhou1ren2}{8,9,2}{⽋,⽔,⼈}
  \definition{s.}{europeu | pessoa ou povo da Europa}
\end{EntryWithPhonetic}

%%%%%%%%%% 殴 %%%%%%%%%%
\subsection*{殴}\addcontentsline{loh}{figure}{殴 \dpy{ou1}}

\begin{EntryWithPhonetic}{殴}{ou1}{8}{⽎}
  \definition{v.}{espancar; bater; acertar}
\end{EntryWithPhonetic}

\begin{EntryWithPhonetic}{殴打}{ou1da3}{8,5}{⽎,⼿}[HSK 7-9]
  \definition{v.}{espancar; bater; bater em alguém}
\end{EntryWithPhonetic}

%%%%%%%%%% 呕 %%%%%%%%%%
\subsection*{呕}\addcontentsline{loh}{figure}{呕 \dpy{ou3}}

\begin{EntryWithPhonetic}{呕}{ou3}{7}{⼝}
  \definition{v.}{vomitar}
\end{EntryWithPhonetic}

\begin{EntryWithPhonetic}{呕吐}{ou3tu4}{7,6}{⼝,⼝}[HSK 7-9]
  \definition{v.}{vomitar; enjoar; jorrar pela boca a comida proveniente do estômago}
\end{EntryWithPhonetic}

%%%%%%%%%% 偶 %%%%%%%%%%
\subsection*{偶}\addcontentsline{loh}{figure}{偶 \dpy{ou3}}

\begin{EntryWithPhonetic}{偶}{ou3}{11}{⼈}
  \definition{adv.}{por acaso; por acidente; de vez em quando; ocasionalmente | par; número par; pareado}
  \definition{s.}{imagem; ídolo; figuras feitas de madeira, barro, etc. | companheiro; cônjuge; parceiro; refere"-se a um casal ou a um dos casais}
  \antonymref{奇}{qi2}
\end{EntryWithPhonetic}

\begin{EntryWithPhonetic}{偶尔}{ou3'er3}{11,5}{⼈,⼩}[HSK 5]
  \definition{adj.}{ocasional}
  \definition{adv.}{ocasionalmente; de vez em quando; às vezes}
\end{EntryWithPhonetic}

\begin{EntryWithPhonetic}{偶然}{ou3ran2}{11,12}{⼈,⽕}[HSK 5]
  \definition{adj.}{acidental; ocasional}
  \definition{adv.}{por acaso; acidentalmente; sem querer; inesperadamente | ocasionalmente; de vez em quando; às vezes}
\end{EntryWithPhonetic}

\begin{EntryWithPhonetic}{偶像}{ou3xiang4}{11,13}{⼈,⼈}[HSK 5]
  \definition[位,个,名]{s.}{ídolo; pessoa amada pelas pessoas; refere"-se a uma pessoa que é apreciada por todos e que, em certos aspectos, é digna de admiração e respeito}
\end{EntryWithPhonetic}

%%%%% EOF %%%%%


 %%%
%%% P
%%%
\section*{P}\addcontentsline{toc}{section}{P}\addcontentsline{loh}{figure}{\#\#\#\#\#\#\#\# P}

%%%%%%%%%% 趴 %%%%%%%%%%
\subsection*{趴}\addcontentsline{loh}{figure}{趴 \dpy{pa1}}

\begin{EntryWithPhonetic}{趴}{pa1}{9}{⾜}[HSK 7-9]
  \definition{v.}{deitar"-se de bruços; arriar; espreguiçar"-se | curvar"-se; apoiar"-se em; inclinar"-se para a frente apoiando"-se em um objeto}
\end{EntryWithPhonetic}

%%%%%%%%%% 扒 %%%%%%%%%%
\subsection*{扒}\addcontentsline{loh}{figure}{扒 \dpy{pa2}}

\begin{EntryWithPhonetic}{扒}{pa2}{5}{⼿}
  \definition{v.}{reunir; juntar; reunir ou espalhar coisas com as mãos ou com um ancinho | roubar; furtar | arranhar; coçar com as mãos | cozinhar; refogar; cozinhar os alimentos em fogo baixo}
  \seeref{ba1}
\end{EntryWithPhonetic}

\begin{EntryWithPhonetic}{扒犁}{pa2li2}{5,11}{⼿,⽜}
  \definition{s.}{Dialeto: trenó; arado}
  \seealsoref{爬犁}{pa2li2}
\end{EntryWithPhonetic}

%%%%%%%%%% 爬 %%%%%%%%%%
\subsection*{爬}\addcontentsline{loh}{figure}{爬 \dpy{pa2}}

\begin{EntryWithPhonetic}{爬}{pa2}{8}{⽖}[HSK 2]
  \definition{v.}{rastejar; arrastar"-se; engatinhar | escalar; trepar; subir com dificuldade | sentar"-se; levantar"-se; levantar"-se da posição deitada ou sentada}
\end{EntryWithPhonetic}

\begin{EntryWithPhonetic}{爬杆}{pa2gan1}{8,7}{⽖,⽊}
  \definition{s.}{escalada em poste}
  \definition{v.}{escalar um poste}
\end{EntryWithPhonetic}

\begin{EntryWithPhonetic}{爬竿}{pa2gan1}{8,9}{⽖,⽵}
  \definition{s.}{poste de escalada | escalada em poste (como ginástica ou ato de circo)}
\end{EntryWithPhonetic}

\begin{EntryWithPhonetic}{爬犁}{pa2li2}{8,11}{⽖,⽜}
  \definition{s.}{trenó}
  \seealsoref{扒犁}{pa2li2}
\end{EntryWithPhonetic}

\begin{EntryWithPhonetic}{爬墙}{pa2qiang2}{8,14}{⽖,⼟}
  \definition{v.}{escalar uma parede}
\end{EntryWithPhonetic}

\begin{EntryWithPhonetic}{爬山}{pa2/shan1}{8,3}{⽖,⼭}[HSK 2]
  \definition{v.+compl.}{escalar uma montanha}
\end{EntryWithPhonetic}

\begin{EntryWithPhonetic}{爬上}{pa2shang4}{8,3}{⽖,⼀}
  \definition{v.}{escalar}
\end{EntryWithPhonetic}

\begin{EntryWithPhonetic}{爬升}{pa2sheng1}{8,4}{⽖,⼗}
  \definition{v.}{ascender | ganhar promoção | subir (números de vendas, etc.) | aumentar}
\end{EntryWithPhonetic}

\begin{EntryWithPhonetic}{爬梳}{pa2shu1}{8,11}{⽖,⽊}
  \definition{v.}{vasculhar (documentos históricos, etc.) | desvendar}
\end{EntryWithPhonetic}

\begin{EntryWithPhonetic}{爬行}{pa2xing2}{8,6}{⽖,⾏}
  \definition{v.}{rastejar | arrastar | engatinhar}
\end{EntryWithPhonetic}

%%%%%%%%%% 怕 %%%%%%%%%%
\subsection*{怕}\addcontentsline{loh}{figure}{怕 \dpy{pa4}}

\begin{EntryWithPhonetic}{怕}{pa4}{8}{⼼}[HSK 2]
  \definition{adv.}{(expressando suposição, julgamento, estimativa, etc.) talvez; suponho; receio (que)}
  \definition{adv.}{por medo; talvez; suponho}
  \definition{v.}{temer; ter medo; recear; sentir medo, ficar nervoso | estar preocupado com; estar preocupado por (ou sobre); ter medo de que algo possa acontecer | ser afetado por; não conseguir suportar; não aguentar mais}
\end{EntryWithPhonetic}

%%%%%%%%%% 拍 %%%%%%%%%%
\subsection*{拍}\addcontentsline{loh}{figure}{拍 \dpy{pai1}}

\begin{EntryWithPhonetic}{拍}{pai1}{8}{⼿}[HSK 3]
  \definition[个,副,对]{s.}{bastão; raquete | batida; tempo; (música) uma unidade para medir a duração de uma nota musical}
  \definition{v.}{tirar (uma foto); usar uma câmera para capturar imagens de pessoas e objetos em filme | dar um tapinha; bater suavemente com as mãos ou ferramentas | bater asas | bater (ondas do mar) | enviar (um telegrama, etc.) | bajular}
\end{EntryWithPhonetic}

\begin{EntryWithPhonetic}{拍板}{pai1/ban3}{8,8}{⼿,⽊}[HSK 7-9]
  \definition{s.}{aplausos}
  \definition{v.+compl.}{marcar o tempo com palmas | bater o martelo | ter a palavra final; dar o veredicto final; tomar a decisão final}
\end{EntryWithPhonetic}

\begin{EntryWithPhonetic}{拍马}{pai1ma3}{8,3}{⼿,⾺}
  \definition{v.}{instigar um cavalo dando tapinhas em seu traseiro | lisonjear | bajular}
  \seealsoref{拍马屁}{pai1ma3pi4}
\end{EntryWithPhonetic}

\begin{EntryWithPhonetic}{拍马屁}{pai1ma3pi4}{8,3,7}{⼿,⾺,⼫}
  \definition{s.}{puxa-saco | bajulador}
  \definition{v.}{puxar o saco | bajular}
  \seealsoref{拍马}{pai1ma3}
\end{EntryWithPhonetic}

\begin{EntryWithPhonetic}{拍卖}{pai1mai4}{8,8}{⼿,⼗}[HSK 7-9]
  \definition{s.}{leilão; uma forma pública de venda de produtos onde todos oferecem publicamente o seu preço, e o item é vendido para quem oferecer o preço mais alto}
  \definition{v.}{leiloar; realizar atividades de leilão | vender mercadorias a preços reduzidos; baixar o preço para vender as mercadorias rapidamente}
\end{EntryWithPhonetic}

\begin{EntryWithPhonetic}{拍摄}{pai1she4}{8,13}{⼿,⼿}[HSK 5]
  \definition{s.}{fotografar; tirar (uma foto); usar uma câmera fotográfica para capturar imagens de pessoas e objetos}
\end{EntryWithPhonetic}

\begin{EntryWithPhonetic}{拍戏}{pai1/xi4}{8,6}{⼿,⼽}[HSK 7-9]
  \definition{v.+compl.}{fazer um filme ou peça de televisão; filmar uma cena | filmar}
\end{EntryWithPhonetic}

\begin{EntryWithPhonetic}{拍照}{pai1/zhao4}{8,13}{⼿,⽕}[HSK 4]
  \definition{v.+compl.}{fotografar; tirar uma foto}
\end{EntryWithPhonetic}

%%%%%%%%%% 徘 %%%%%%%%%%
\subsection*{徘}\addcontentsline{loh}{figure}{徘 \dpy{pai2}}

\begin{EntryWithPhonetic}{徘}{pai2}{11}{⼻}
  \definition{adj.}{irresoluto; indeciso}
  \definition{v.}{vagar}
\end{EntryWithPhonetic}

\begin{EntryWithPhonetic}{徘徊}{pai2huai2}{11,9}{⼻,⼻}[HSK 7-9]
  \definition{v.}{andar de um lado para o outro no mesmo lugar | Figurativo: vacilar; hesitar; uma metáfora para hesitação e indecisão | flutuar; essa metáfora descreve as coisas mudando para cima e para baixo dentro de uma determinada faixa}
\end{EntryWithPhonetic}

%%%%%%%%%% 排 %%%%%%%%%%
\subsection*{排}\addcontentsline{loh}{figure}{排 \dpy{pai2}}

\begin{EntryWithPhonetic}{排}{pai2}{11}{⼿}[HSK 2,3]
  \definition{clas.}{usado para linhas, filas; coisas usadas para formar filas}
  \definition{s.}{linha; fileira; fileiras horizontais | pelotão; unidade militar, abaixo do nível de companhia, acima do nível de pelotão | jangada; balsa; um meio de transporte aquático feito de bambu e madeira unidos lado a lado; também se refere a bambu e madeira amarrados em fileiras para facilitar o transporte aquático | torta; bolo de carne; bolinho assado; comida cozida no vapor}
  \definition{v.}{organizar; alinhar; colocar em ordem; posicionar ou organizar em uma determinada ordem; ordenar | ensaiar | ejetar; excluir; dispensar; remover; eliminar | empurrar o obstáculo para fora do caminho}
\end{EntryWithPhonetic}

\begin{EntryWithPhonetic}{排斥}{pai2chi4}{11,5}{⼿,⽄}[HSK 7-9]
  \definition{v.}{repelir; rejeitar; excluir; fazer com que (uma pessoa ou coisa) se afaste do seu próprio grupo}
\end{EntryWithPhonetic}

\begin{EntryWithPhonetic}{排除}{pai2chu2}{11,9}{⼿,⾩}[HSK 5]
  \definition{v.}{remover; superar; excluir; eliminar; livrar"-se de}
\end{EntryWithPhonetic}

\begin{EntryWithPhonetic}{排队}{pai2/dui4}{11,4}{⼿,⾩}[HSK 2]
  \definition{v.+compl.}{formar uma fila; alinhar"-se; enfileirar"-se; organizar em sequência | listar; classificar}
\end{EntryWithPhonetic}

\begin{EntryWithPhonetic}{排放}{pai2fang4}{11,8}{⼿,⽅}[HSK 7-9]
  \definition{v.}{colocar (as coisas) em ordem adequada | emitir; descarregar (gases de escape, águas residuais, etc.); deixar sair; drenar}
\end{EntryWithPhonetic}

\begin{EntryWithPhonetic}{排行榜}{pai2hang2bang3}{11,6,14}{⼿,⾏,⽊}[HSK 6]
  \definition{s.}{lista; classificação; lista de classificação; (de registros) os gráficos; uma lista em uma determinada ordem publicada com base em certos resultados estatísticos}
\end{EntryWithPhonetic}

\begin{EntryWithPhonetic}{排挤}{pai2ji3}{11,9}{⼿,⼿}
  \definition{v.}{ostracizar; afastar; expulsar; espremer; excluir; marginalizar; usar o poder ou os meios para fazer com que aqueles que lhe são desfavoráveis percam seu status ou seus interesses}
\end{EntryWithPhonetic}

\begin{EntryWithPhonetic}{排练}{pai2lian4}{11,8}{⼿,⽷}[HSK 7-9]
  \definition{v.}{ensaiar; ensaiar ou praticar uma determinada cerimônia ou apresentação}
\end{EntryWithPhonetic}

\begin{EntryWithPhonetic}{排列}{pai2lie4}{11,6}{⼿,⼑}[HSK 4]
  \definition{v.}{classificar; colocar; variar; organizar; pôr em ordem}
\end{EntryWithPhonetic}

\begin{EntryWithPhonetic}{排名}{pai2 ming2}{11,6}{⼿,⼝}[HSK 3]
  \definition{s.}{classificação; resultado; organizado de acordo com determinados critérios}
\end{EntryWithPhonetic}

\begin{EntryWithPhonetic}{排球}{pai2qiu2}{11,11}{⼿,⽟}[HSK 2]
  \definition[场,只,个]{s.}{voleibol; bola de voleibol}
\end{EntryWithPhonetic}

\begin{EntryWithPhonetic}{排水}{pai2shui3}{11,4}{⼿,⽔}
  \definition{v.}{drenar}
\end{EntryWithPhonetic}

%%%%%%%%%% 牌 %%%%%%%%%%
\subsection*{牌}\addcontentsline{loh}{figure}{牌 \dpy{pai2}}

\begin{EntryWithPhonetic}{牌}{pai2}{12}{⽚}[HSK 4]
  \definition[块,副,张,个,种]{s.}{placa; tabuleta; quadro; placar | marca; marca registrada; marca comercial; \emph{trademark} | cartas, dominó, etc. | a tonalidade de uma música}
\end{EntryWithPhonetic}

\begin{EntryWithPhonetic}{牌照}{pai2zhao4}{12,13}{⽚,⽕}[HSK 7-9]
  \definition{s.}{placa de matrícula; certificado de licenciamento; certificado de registro de veículo ou licença comercial emitida pelo departamento administrativo competente}
\end{EntryWithPhonetic}

\begin{EntryWithPhonetic}{牌子}{pai2zi5}{12,3}{⽚,⼦}[HSK 3]
  \definition[个,种,块]{s.}{sinal; placa; placas feitas de madeira ou outros materiais, geralmente com texto nelas | marca; marca registrada; um nome especial dado por uma empresa ao seu próprio produto}
\end{EntryWithPhonetic}

%%%%%%%%%% 派 %%%%%%%%%%
\subsection*{派}\addcontentsline{loh}{figure}{派 \dpy{pai4}}

\begin{EntryWithPhonetic}{派}{pai4}{9}{⽔}[HSK 3]
  \definition{adj.}{elegante; bonito; imponente}
  \definition{clas.}{usado para grupos, escolas de pensamento ou arte, etc. | usado para um discursos, situações, cenas, etc.}
  \definition[个,块,种]{s.}{panelinha; facção; pessoas com ideias, visões e estilos semelhantes | torta; um alimento recheado comumente consumido pelos ocidentais, geralmente doce | maneira e ar; estilo ou comportamento | afluente; braço de rio}
  \definition{v.}{enviar; despachar; arranjar ou ordenar que uma pessoa faça algo; providenciar transporte | alocar; repartir; distribuir}
\end{EntryWithPhonetic}

\begin{EntryWithPhonetic}{派别}{pai4bie2}{9,7}{⽔,⼑}[HSK 7-9]
  \definition{s.}{grupo; seita; escola; facção | categorias; panelinha}
\end{EntryWithPhonetic}

\begin{EntryWithPhonetic}{派出}{pai4chu1}{9,5}{⽔,⼐}[HSK 6]
  \definition{v.}{despachar; expedi | enviar}
\end{EntryWithPhonetic}

\begin{EntryWithPhonetic}{派遣}{pai4qian3}{9,13}{⽔,⾡}[HSK 7-9]
  \definition{v.}{despachar; enviar alguém em missão (governo, organização, etc.)}
\end{EntryWithPhonetic}

%%%%%%%%%% 扳 %%%%%%%%%%
\subsection*{扳}\addcontentsline{loh}{figure}{扳 \dpy{pan1}}

\begin{EntryWithPhonetic}{扳}{pan1}{7}{⼿}
  \definition{v.}{segurar; agarrar; puxar; escalar | confiar em; buscar ajuda; associar"-se a pessoas de status superior; refere"-se a formar um relacionamento ou estabelecer um relacionamento com alguém de alto \emph{status} | envolver; relacionar"-se com}
  \seeref{ban1}
\end{EntryWithPhonetic}

%%%%%%%%%% 攀 %%%%%%%%%%
\subsection*{攀}\addcontentsline{loh}{figure}{攀 \dpy{pan1}}

\begin{EntryWithPhonetic}{攀}{pan1}{19}{⼿}[HSK 7-9]
  \definition{v.}{escalar; escalar | buscar conexões em altos cargos | envolver; implicar | agarrar; agarrar-se; segurar-se a}
\end{EntryWithPhonetic}

\begin{EntryWithPhonetic}{攀爬}{pan1pa2}{19,8}{⼿,⽖}
  \definition{v.}{escalar; escalada em rocha; refere"-se ao movimento em uma determinada direção usando apenas as mãos e os pés, com o mínimo uso de ferramentas}
\end{EntryWithPhonetic}

\begin{EntryWithPhonetic}{攀升}{pan1sheng1}{19,4}{⼿,⼗}[HSK 7-9]
  \definition{v.}{subir para um ponto mais alto | (preços, quantidade, etc.) subir; aumentar; escalar | subir}
\end{EntryWithPhonetic}

\begin{EntryWithPhonetic}{攀岩}{pan1yan2}{19,8}{⼿,⼭}
  \definition{s.}{escalada em rocha; isso se refere a esse tipo de esporte}
  \definition{v.}{escalar uma parede rochosa íngreme com equipamento mínimo}
\end{EntryWithPhonetic}

%%%%%%%%%% 爿 %%%%%%%%%%
\subsection*{爿}\addcontentsline{loh}{figure}{爿 \dpy{pan2}}

\begin{EntryWithPhonetic}{爿}{pan2}{4}{⽙}[Kangxi 90]
  \definition{clas.}{usado para faixas de terra ou bambu, lojas, fábricas etc.}
\end{EntryWithPhonetic}

%%%%%%%%%% 胖 %%%%%%%%%%
\subsection*{胖}\addcontentsline{loh}{figure}{胖 \dpy{pan2}}

\begin{EntryWithPhonetic}{胖}{pan2}{9}{⾁}
  \definition{adj.}{saudável}
  \seeref{pang4}
\end{EntryWithPhonetic}

%%%%%%%%%% 般 %%%%%%%%%%
\subsection*{般}\addcontentsline{loh}{figure}{般 \dpy{pan2}}

\begin{EntryWithPhonetic}{般}{pan2}{10}{⾈}
  \definition{adj.}{feliz; bem-aventurado}
  \seeref{ban1}
  \seeref{bo1}
\end{EntryWithPhonetic}

\begin{EntryWithPhonetic}{般乐}{pan2le4}{10,5}{⾈,⼃}
  \definition{v.}{jogar | divertir"-se}
\end{EntryWithPhonetic}

%%%%%%%%%% 盘 %%%%%%%%%%
\subsection*{盘}\addcontentsline{loh}{figure}{盘 \dpy{pan2}}

\begin{EntryWithPhonetic}{盘}{pan2}{11}{⽫}[HSK 4,7-9]
  \definition*{s.}{Sobrenome: Pan}
  \definition{clas.}{usado para pratos, pedras de moer, etc. | usado para jogos de xadrez e de bola | usado para as coisas que estão entrelaçadas, emaranhadas}
  \definition[套,只]{s.}{bandeja; tabuleiro | recipiente plano e raso, como uma bandeja, prato, travessa etc.  | preço atual; cotação de mercado; refere"-se ao preço básico pelo qual as commodities são negociadas}
  \definition{v.}{enrolar; torcer; enrolar (para cima); entrelaçar; cercar | construir (assentando tijolos, pedras, etc.) | checar; examinar; interrogar; verificar um por um ou repetidamente (quantidade, situação, etc.) | transferir a propriedade de; passar para outra pessoa | carregar; transportar}
\end{EntryWithPhonetic}

\begin{EntryWithPhonetic}{盘算}{pan2suan5}{11,14}{⽫,⽵}[HSK 7-9]
  \definition{v.}{calcular; determinar; planejar | considerar e ponderar; premeditar; deliberar}
\end{EntryWithPhonetic}

\begin{EntryWithPhonetic}{盘子}{pan2zi5}{11,3}{⽫,⼦}[HSK 4]
  \definition[个,叠,套,只]{s.}{prato; utensílio de fundo raso para guardar objetos, maior do que um pires, geralmente de formato redondo | situação de mercado; taxa de mercado; transação comercial}
\end{EntryWithPhonetic}

%%%%%%%%%% 槃 %%%%%%%%%%
\subsection*{槃}\addcontentsline{loh}{figure}{槃 \dpy{pan2}}

\begin{EntryWithPhonetic}{槃}{pan2}{14}{⽊}
  \variantof{盘}
\end{EntryWithPhonetic}

%%%%%%%%%% 判 %%%%%%%%%%
\subsection*{判}\addcontentsline{loh}{figure}{判 \dpy{pan4}}

\begin{EntryWithPhonetic}{判}{pan4}{7}{⼑}[HSK 6]
  \definition{adv.}{obviamente há uma diferença}
  \definition{v.}{distinguir; discriminar; separar | julgar; decidir; avaliar | sentenciar; condenar}
\end{EntryWithPhonetic}

\begin{EntryWithPhonetic}{判处}{pan4chu3}{7,5}{⼑,⼡}[HSK 7-9]
  \definition{s.}{sentença; condenação; o julgamento e a sentença, pelo tribunal, daqueles que violam a lei penal}
  \definition{v.}{sentenciar (alguém) a; condenar (alguém) a}
\end{EntryWithPhonetic}

\begin{EntryWithPhonetic}{判定}{pan4ding4}{7,8}{⼑,⼧}[HSK 7-9]
  \definition{v.}{julgar; decidir; determinar; a análise leva a uma conclusão ou a uma decisão}
\end{EntryWithPhonetic}

\begin{EntryWithPhonetic}{判断}{pan4duan4}{7,11}{⼑,⽄}[HSK 3]
  \definition[个,项]{s.}{julgamento; conclusões tiradas após reflexão e análise}
  \definition{v.}{julgar; decidir}
\end{EntryWithPhonetic}

\begin{EntryWithPhonetic}{判决}{pan4jue2}{7,6}{⼑,⼎}[HSK 7-9]
  \definition[个]{s.}{decisão judicial; julgamento}
  \definition{v.}{proferir um veredicto; adjudicar; condenar; emitir um julgamento; pronunciar (julgamento)}
\end{EntryWithPhonetic}

%%%%%%%%%% 叛 %%%%%%%%%%
\subsection*{叛}\addcontentsline{loh}{figure}{叛 \dpy{pan4}}

\begin{EntryWithPhonetic}{叛}{pan4}{9}{⼜}
  \definition{adj.}{rebelde}
  \definition{s.}{rebelião}
  \definition{v.}{trair | rebelar"-se | revoltar"-se}
\end{EntryWithPhonetic}

\begin{EntryWithPhonetic}{叛逆}{pan4ni4}{9,9}{⼜,⾡}[HSK 7-9]
  \definition{s.}{rebelde; pessoas que traem}
  \definition{v.}{rebelar"-se/revoltar"-se contra; trair}
\end{EntryWithPhonetic}

%%%%%%%%%% 盼 %%%%%%%%%%
\subsection*{盼}\addcontentsline{loh}{figure}{盼 \dpy{pan4}}

\begin{EntryWithPhonetic}{盼}{pan4}{9}{⽬}[HSK 7-9]
  \definition*{s.}{Sobrenome: Pan}
  \definition{adj.}{(olhos) com preto e branco fortemente contrastados; olhos claros}
  \definition{v.}{olhar | esperar por; ansiar por | sentir falta de ; continuar pensando sobre}
\end{EntryWithPhonetic}

\begin{EntryWithPhonetic}{盼望}{pan4wang4}{9,11}{⽬,⽉}[HSK 6]
  \definition{v.}{esperar por; ansiar por; esperar que algo aconteça em breve}
\end{EntryWithPhonetic}

%%%%%%%%%% 乓 %%%%%%%%%%
\subsection*{乓}\addcontentsline{loh}{figure}{乓 \dpy{pang1}}

\begin{EntryWithPhonetic}{乓}{pang1}{6}{⼃}
  \definition{interj.}{Onomatopéia: barulho repentino feito por tiros, uma porta batendo, coisas quebrando, etc.; estrondo; estouro; batida; colisão}
\end{EntryWithPhonetic}

%%%%%%%%%% 庞 %%%%%%%%%%
\subsection*{庞}\addcontentsline{loh}{figure}{庞 \dpy{pang2}}

\begin{EntryWithPhonetic}{庞}{pang2}{8}{⼴}
  \definition*{s.}{Sobrenome: Pang}
  \definition{adj.}{enorme | inúmeros e desordenados; numerosos e desorganizados}
  \definition{s.}{molde do rosto de alguém | rosto; placa frontal}
\end{EntryWithPhonetic}

\begin{EntryWithPhonetic}{庞大}{pang2da4}{8,3}{⼴,⼤}[HSK 7-9]
  \definition{adj.}{enorme; colossal; gigantesco; imenso; (em termos de forma, estrutura, quantidade, etc.) é muito grande; excessivamente grande}
\end{EntryWithPhonetic}

%%%%%%%%%% 旁 %%%%%%%%%%
\subsection*{旁}\addcontentsline{loh}{figure}{旁 \dpy{pang2}}

\begin{EntryWithPhonetic}{旁}{pang2}{10}{⽅}[HSK 5]
  \definition{adj.}{outro | abundante; abrangente}
  \definition{s.}{lado | radical lateral de um caractere chinês}
\end{EntryWithPhonetic}

\begin{EntryWithPhonetic}{旁边}{pang2bian1}{10,5}{⽅,⾡}[HSK 1]
  \definition{s.}{junto a; próximo de; ao lado}
\end{EntryWithPhonetic}

\begin{EntryWithPhonetic}{旁观}{pang2guan1}{10,6}{⽅,⾒}[HSK 7-9]
  \definition{v.}{observar; ser um espectador | observar de fora}
\end{EntryWithPhonetic}

%%%%%%%%%% 磅 %%%%%%%%%%
\subsection*{磅}\addcontentsline{loh}{figure}{磅 \dpy{pang2}}

\begin{EntryWithPhonetic}{磅}{pang2}{15}{⽯}
  \definition{adj.}{majestoso; abundante; cheio de energia; magnífico}
  \seeref{bang4}
\end{EntryWithPhonetic}

%%%%%%%%%% 胖 %%%%%%%%%%
\subsection*{胖}\addcontentsline{loh}{figure}{胖 \dpy{pang4}}

\begin{EntryWithPhonetic}{胖}{pang4}{9}{⾁}[HSK 3]
  \definition{adj.}{gordo; robusto; rechonchudo; (corpo humano) com muita gordura ou carne}
  \seeref{pan2}
  \antonymref{瘦}{shou4}
\end{EntryWithPhonetic}

\begin{EntryWithPhonetic}{胖子}{pang4zi5}{9,3}{⾁,⼦}[HSK 4]
  \definition[个]{s.}{obeso; gordo; pessoa gorda}
\end{EntryWithPhonetic}

%%%%%%%%%% 抛 %%%%%%%%%%
\subsection*{抛}\addcontentsline{loh}{figure}{抛 \dpy{pao1}}

\begin{EntryWithPhonetic}{抛}{pao1}{7}{⼿}[HSK 7-9]
  \definition{v.}{atirar; lançar; arremessar | deixar para trás; descartar; abandonar | vender por um preço inferior ao real; vender em excesso; vender em grande quantidade | mostrar; expor}
\end{EntryWithPhonetic}

\begin{EntryWithPhonetic}{抛开}{pao1kai1}{7,4}{⼿,⼶}[HSK 7-9]
  \definition{v.}{livrar"-se de; revogar; afastar"-se | jogar fora}
\end{EntryWithPhonetic}

\begin{EntryWithPhonetic}{抛弃}{pao1qi4}{7,7}{⼿,⼶}[HSK 7-9]
  \definition{v.}{abandonar; deixar de lado; renunciar; jogar fora}
\end{EntryWithPhonetic}

%%%%%%%%%% 泡 %%%%%%%%%%
\subsection*{泡}\addcontentsline{loh}{figure}{泡 \dpy{pao1}}

\begin{EntryWithPhonetic}{泡}{pao1}{8}{⽔}
  \definition{adj.}{esponjoso; oco e macio; não duro}
  \definition{clas.}{usado para fezes e urina}
  \definition[串,个]{s.}{algo fofo e macio | pequeno lago}
  \seeref{pao4}
\end{EntryWithPhonetic}

%%%%%%%%%% 刨 %%%%%%%%%%
\subsection*{刨}\addcontentsline{loh}{figure}{刨 \dpy{pao2}}

\begin{EntryWithPhonetic}{刨}{pao2}{7}{⼑}[HSK 7-9]
  \definition{v.}{cavar; escavar | cortar; remover; remover ou subtrair de algo que já existe}
  \seeref{bao4}
\end{EntryWithPhonetic}

%%%%%%%%%% 炮 %%%%%%%%%%
\subsection*{炮}\addcontentsline{loh}{figure}{炮 \dpy{pao2}}

\begin{EntryWithPhonetic}{炮}{pao2}{9}{⽕}
  \definition{v.}{Medicina tradicional chinesa: preparar a medicina chinesa assando-a em uma panela de ferro quente até dourar e estalar}
  \seeref{bao1}
  \seeref{pao4}
\end{EntryWithPhonetic}

%%%%%%%%%% 跑 %%%%%%%%%%
\subsection*{跑}\addcontentsline{loh}{figure}{跑 \dpy{pao2}}

\begin{EntryWithPhonetic}{跑}{pao2}{12}{⾜}
  \definition{v.}{(animais) bater com a pata (no chão); (animais) escavar o solo com suas garras ou cascos}
  \seeref{pao3}
\end{EntryWithPhonetic}

\begin{EntryWithPhonetic}{跑}{pao3}{12}{⾜}[HSK 1]
  \definition{v.}{correr; pessoas ou animais que se movem rapidamente para a frente com as pernas e os pés | caminhar; passear | fugir; escapar | correr de um lado para outro; fazer rondas; correr atrás de algo | de um líquido ou gás) vazar; evaporar | (como complemento de um verbo) fora; longe | participar de uma corrida}
  \seeref{pao2}
\end{EntryWithPhonetic}

\begin{EntryWithPhonetic}{跑步}{pao3/bu4}{12,7}{⾜,⽌}[HSK 3]
  \definition{v.+compl.}{correr; trotar}
\end{EntryWithPhonetic}

\begin{EntryWithPhonetic}{跑车}{pao3che1}{12,4}{⾜,⾞}[HSK 7-9]
  \definition{s.}{bicicleta de corrida | \emph{roadster}; carro de corrida | carrinho para transportar toras em uma floresta}
  \definition{v.}{Coloquial: (condutores de trem) estar em serviço | (vagões de carvão em uma mina) deslizar acidentalmente para baixo (desgovernado) | Dialeto: dirigir um veículo de transporte | trabalhar em um trem; atendente de trem trabalhando no trem | escorregar acidentalmente para baixo; isso se refere a um acidente em um poço inclinado de mina, onde o cabo de aço se rompe repentinamente durante o içamento ou o guincho escorrega por outros motivos}
\end{EntryWithPhonetic}

\begin{EntryWithPhonetic}{跑道}{pao3dao4}{12,12}{⾜,⾡}[HSK 7-9]
  \definition[条]{s.}{pista de decolagem; a pista de taxiagem utilizada pelas aeronaves durante a decolagem e o pouso | pista; pista de atletismo; as linhas brancas desenhadas na pista são usadas para corridas de corrida ou ciclismo}
\end{EntryWithPhonetic}

\begin{EntryWithPhonetic}{跑调}{pao3diao4}{12,10}{⾜,⾔}
  \definition{v.}{Coloquial: estar fora do tom ou desafinado (enquanto canta)}
\end{EntryWithPhonetic}

\begin{EntryWithPhonetic}{跑掉}{pao3diao4}{12,11}{⾜,⼿}
  \definition{v.}{fugir}
\end{EntryWithPhonetic}

\begin{EntryWithPhonetic}{跑肚}{pao3du4}{12,7}{⾜,⾁}
  \definition{v.}{Coloquial: ter diarréia}
\end{EntryWithPhonetic}

\begin{EntryWithPhonetic}{跑酷}{pao3ku4}{12,14}{⾜,⾣}
  \definition*{s.}{Empréstimo linguístico: Parkour}
\end{EntryWithPhonetic}

\begin{EntryWithPhonetic}{跑龙套}{pao3 long2tao4}{12,5,10}{⾜,⿓,⼤}[HSK 7-9]
  \definition{v.}{Teatro: interpretar um papel secundário | desempenhar um papel secundário; não ser ninguém | desempenhar um papel pequeno}
\end{EntryWithPhonetic}

\begin{EntryWithPhonetic}{跑马}{pao3ma3}{12,3}{⾜,⾺}
  \definition{s.}{corrida de cavalos}
  \definition{v.}{andar a cavalo em ritmo acelerado}
\end{EntryWithPhonetic}

\begin{EntryWithPhonetic}{跑题}{pao3ti2}{12,15}{⾜,⾴}
  \definition{v.}{divagar | fugir do assunto | tergiversar}
\end{EntryWithPhonetic}

\begin{EntryWithPhonetic}{跑腿}{pao3tui3}{12,13}{⾜,⾁}
  \definition{v.}{realizar tarefas}
\end{EntryWithPhonetic}

%%%%%%%%%% 泡 %%%%%%%%%%
\subsection*{泡}\addcontentsline{loh}{figure}{泡 \dpy{pao4}}

\begin{EntryWithPhonetic}{泡}{pao4}{8}{⽔}[HSK 6]
  \definition[串,个]{s.}{bolha | algo em forma de bolha}
  \definition{v.}{mergulhar; encharcar | despejar água fervente em (chá, sopa instantânea, etc.) | enrolar; demorar-se; ficar por aí | (coloquial) (de um homem) brincar no campo; brincar com uma mulher | perder tempo; matar o tempo deliberadamente}
  \seeref{pao1}
\end{EntryWithPhonetic}

\begin{EntryWithPhonetic}{泡沫}{pao4mo4}{8,8}{⽔,⽔}[HSK 7-9]
  \definition{s.}{espuma; pequenas bolhas se aglomeraram na superfície do líquido | ilusão; bolha econômica; essa metáfora descreve a prosperidade superficial e o florescimento de algo que, na realidade, é vazio e irreal}
\end{EntryWithPhonetic}

%%%%%%%%%% 炮 %%%%%%%%%%
\subsection*{炮}\addcontentsline{loh}{figure}{炮 \dpy{pao4}}

\begin{EntryWithPhonetic}{炮}{pao4}{9}{⽕}[HSK 6]
  \definition{s.}{arma grande; canhão; peça de artilharia | fogo de artifício | buraco de explosão cheio de dinamite | canhão, uma das peças do xadrez chinês}
  \seeref{bao1}
  \seeref{pao2}
\end{EntryWithPhonetic}

\begin{EntryWithPhonetic}{炮车}{pao4che1}{9,4}{⽕,⾞}
  \definition{s.}{veículo de artilharia; tanque de guerra}
\end{EntryWithPhonetic}

%%%%%%%%%% 胚 %%%%%%%%%%
\subsection*{胚}\addcontentsline{loh}{figure}{胚 \dpy{pei1}}

\begin{EntryWithPhonetic}{胚}{pei1}{9}{⾁}
  \definition{s.}{embrião}
\end{EntryWithPhonetic}

\begin{EntryWithPhonetic}{胚胎}{pei1tai1}{9,9}{⾁,⾁}[HSK 7-9]
  \definition{s.}{embrião}[子宫里的胚胎。===Um embrião no útero.]
\end{EntryWithPhonetic}

%%%%%%%%%% 陪 %%%%%%%%%%
\subsection*{陪}\addcontentsline{loh}{figure}{陪 \dpy{pei2}}

\begin{EntryWithPhonetic}{陪}{pei2}{10}{⾩}[HSK 5]
  \definition{v.}{servir; acompanhar; cuidar; fazer companhia a alguém | auxiliar; ajudar}
\end{EntryWithPhonetic}

\begin{EntryWithPhonetic}{陪伴}{pei2ban4}{10,7}{⾩,⼈}[HSK 7-9]
  \definition{v.}{acompanhar; fazer companhia a alguém}
\end{EntryWithPhonetic}

\begin{EntryWithPhonetic}{陪同}{pei2tong2}{10,6}{⾩,⼝}[HSK 6]
  \definition{v.}{acompanhar; acompanhar alguém para fazer uma atividade ou trabalhar junto}
\end{EntryWithPhonetic}

\begin{EntryWithPhonetic}{陪葬}{pei2zang4}{10,12}{⾩,⾋}[HSK 7-9]
  \definition{v.}{ser enterrado com ou ao lado de uma pessoa falecida (do cônjuge ou companheiro(a) do falecido, ou de objetos funerários) | Obsoleto: (esposa, concubina ou escrava) ser enterrada viva com os mortos | (estatuetas ou objetos) enterrar com os mortos | (esposa ou concubina após a sua morte) ser enterrada com o marido ou junto ao seu túmulo}
\end{EntryWithPhonetic}

%%%%%%%%%% 培 %%%%%%%%%%
\subsection*{培}\addcontentsline{loh}{figure}{培 \dpy{pei2}}

\begin{EntryWithPhonetic}{培}{pei2}{11}{⼟}
  \definition{v.}{aterrar com terra; aterrar | fomentar; treinar | cultivar; crescer e desenvolver-se propositalmente}
\end{EntryWithPhonetic}

\begin{EntryWithPhonetic}{培训}{pei2xun4}{11,5}{⼟,⾔}[HSK 4]
  \definition{v.}{treinar (trabalhadores técnicos, quadros profissionais, etc.)}
\end{EntryWithPhonetic}

\begin{EntryWithPhonetic}{培训班}{pei2xun4ban1}{11,5,10}{⼟,⾔,⽟}[HSK 4]
  \definition{s.}{aula de treinamento; curso de treinamento}
\end{EntryWithPhonetic}

\begin{EntryWithPhonetic}{培养}{pei2yang3}{11,9}{⼟,⼋}[HSK 4]
  \definition{v.}{cultivar (plantas, microorganismos) | promover; treinar ou desenvolver; educar e treinar para um determinado propósito durante um longo período de tempo; fazer crescer | progredir gradualmente; desenvolver ou cultivar gradualmente (hábito, qualidade, caráter, emoção, estilo, interesse, habilidade, etc.)}
\end{EntryWithPhonetic}

\begin{EntryWithPhonetic}{培育}{pei2yu4}{11,8}{⼟,⾁}[HSK 4]
  \definition{v.}{criar; fomentar; educar; procriar; nutrir; cultivar}
\end{EntryWithPhonetic}

%%%%%%%%%% 赔 %%%%%%%%%%
\subsection*{赔}\addcontentsline{loh}{figure}{赔 \dpy{pei2}}

\begin{EntryWithPhonetic}{赔}{pei2}{12}{⾙}[HSK 5]
  \definition{v.}{compensar; pagar por; indenizar | sofrer uma perda; fazer negócios e perder dinheiro | desculpar-se | suportar uma perda}
\end{EntryWithPhonetic}

\begin{EntryWithPhonetic}{赔偿}{pei2chang2}{12,11}{⾙,⼈}[HSK 5]
  \definition{v.}{indenizar; compensar; pagar por; indenizar outras pessoas ou grupos por perdas causadas por suas próprias ações}
\end{EntryWithPhonetic}

\begin{EntryWithPhonetic}{赔钱}{pei2/qian2}{12,10}{⾙,⾦}[HSK 7-9]
  \definition{v.+compl.}{perder dinheiro | compensar; compensar com dinheiro os prejuízos causados a terceiros}
\end{EntryWithPhonetic}

%%%%%%%%%% 佩 %%%%%%%%%%
\subsection*{佩}\addcontentsline{loh}{figure}{佩 \dpy{pei4}}

\begin{EntryWithPhonetic}{佩}{pei4}{8}{⼈}
  \definition{s.}{um ornamento usado como pingente amarrados em cintos nos tempos antigos}
  \definition{v.}{vestir (na cintura, etc.) | (arcaico) admirar | (arcaico) usar, especialmente uma pistola ou espada, na cintura}
\end{EntryWithPhonetic}

\begin{EntryWithPhonetic}{佩服}{pei4fu2}{8,8}{⼈,⽉}[HSK 7-9]
  \definition{v.}{admirar; respeitar; dar os parabéns a alguém; ter uma alta opinião de alguém; considerar respeitáveis e adoráveis}
\end{EntryWithPhonetic}

%%%%%%%%%% 配 %%%%%%%%%%
\subsection*{配}\addcontentsline{loh}{figure}{配 \dpy{pei4}}

\begin{EntryWithPhonetic}{配}{pei4}{10}{⾣}[HSK 3]
  \definition{adj.}{adequado; bem combinado}
  \definition{s.}{cônjuge (geralmente referindo"-se a uma esposa)}
  \definition{v.}{unir"-se em matrimônio | (animais) acasalar; copular | compor; combinar; mesclar; amalgamar; misturar | distribuir de forma planejada; repartir | encontrar algo para encaixar ou substituir outra coisa; compensar as partes faltantes de acordo com certos padrões | combinar; harmonizar com; estar em harmonia com | exilar; banir; nos tempos antigos, referia"-se ao exílio de criminosos}
  \definition{v.aux.}{adequar"-se a; merecer; ser qualificado; ser digno de}
\end{EntryWithPhonetic}

\begin{EntryWithPhonetic}{配备}{pei4bei4}{10,8}{⾣,⼡}[HSK 5]
  \definition{s.}{equipamento; material; conjunto completo de utensílios, etc.}
  \definition{v.}{fornecer; alocar; equipar; distribuir conforme necessário | posicionar; dispor (tropas, etc.)}
\end{EntryWithPhonetic}

\begin{EntryWithPhonetic}{配合}{pei4he5}{10,6}{⾣,⼝}[HSK 3]
  \definition{v.}{cooperar; coordenar; todas as partes trabalham juntas para concluir tarefas comuns}
\end{EntryWithPhonetic}

\begin{EntryWithPhonetic}{配件}{pei4jian4}{10,6}{⾣,⼈}[HSK 7-9]
  \definition{s.}{peças; acessórios; complementos; peças ou componentes usados para montar máquinas | peças de substituição; peças ou componentes que são reinstalados após terem sido danificados}
\end{EntryWithPhonetic}

\begin{EntryWithPhonetic}{配偶}{pei4'ou3}{10,11}{⾣,⼈}[HSK 7-9]
  \definition{s.}{cônjuge; refere"-se ao marido ou à esposa (frequentemente usado em documentos legais)}
\end{EntryWithPhonetic}

\begin{EntryWithPhonetic}{配送}{pei4song4}{10,9}{⾣,⾡}[HSK 7-9]
  \definition{s.}{entrega; distribuição; \emph{delivery}}
\end{EntryWithPhonetic}

\begin{EntryWithPhonetic}{配套}{pei4/tao4}{10,10}{⾣,⼤}[HSK 5]
  \definition{v.+compl.}{formar um conjunto ou sistema completo; combinar vários elementos relacionados em um conjunto completo}
\end{EntryWithPhonetic}

\begin{EntryWithPhonetic}{配音}{pei4/yin1}{10,9}{⾣,⾳}[HSK 7-9]
  \definition{s.}{dublagem (de um filme, etc.); locução}[这部电影有配音版本。===Este filme possui uma versão dublada.]
  \definition{v.+compl.}{dublar; narrar}
\end{EntryWithPhonetic}

\begin{EntryWithPhonetic}{配置}{pei4zhi4}{10,13}{⾣,⽹}[HSK 6]
  \definition{s.}{configuração; refere"-se especificamente à seleção e combinação de software e hardware em várias partes de computadores, carros, etc.}
  \definition{v.}{implantar; alocar; dispor (tropas, etc.); equipar e configurar}
\end{EntryWithPhonetic}

%%%%%%%%%% 喷 %%%%%%%%%%
\subsection*{喷}\addcontentsline{loh}{figure}{喷 \dpy{pen1}}

\begin{EntryWithPhonetic}{喷}{pen1}{12}{⼝}[HSK 5]
  \definition{v.}{jorrar; esguichar; expelir sob pressão | borrifar; espalhar; pulverizar}
  \seeref{pen4}
\end{EntryWithPhonetic}

\begin{EntryWithPhonetic}{喷泉}{pen1quan2}{12,9}{⼝,⽔}[HSK 7-9]
  \definition[个,处,注]{s.}{fonte; fonte que jorra água}
\end{EntryWithPhonetic}

%%%%%%%%%% 盆 %%%%%%%%%%
\subsection*{盆}\addcontentsline{loh}{figure}{盆 \dpy{pen2}}

\begin{EntryWithPhonetic}{盆}{pen2}{9}{⽫}[HSK 5]
  \definition*{s.}{Sobrenome: Pen}
  \definition{s.}{bacia; banheira; panela; utensílios para guardar ou lavar coisas}
\end{EntryWithPhonetic}

\begin{EntryWithPhonetic}{盆友}{pen2you3}{9,4}{⽫,⼜}
  \definition{s.}{Gíria da \emph{Internet}: amigo (trocadilho com 朋友)}
  \seealsoref{朋友}{peng2you5}
\end{EntryWithPhonetic}

%%%%%%%%%% 喷 %%%%%%%%%%
\subsection*{喷}\addcontentsline{loh}{figure}{喷 \dpy{pen4}}

\begin{EntryWithPhonetic}{喷}{pen4}{12}{⼝}
  \definition{s.}{na época; tempo no mercado; época em que frutas, peixes e camarões são comercializados em grande quantidade | colheita; número de vezes que floresceu e frutificou; número de vezes que foi colhido na maturação}
  \seeref{pen1}
\end{EntryWithPhonetic}

%%%%%%%%%% 抨 %%%%%%%%%%
\subsection*{抨}\addcontentsline{loh}{figure}{抨 \dpy{peng1}}

\begin{EntryWithPhonetic}{抨}{peng1}{8}{⼿}
  \definition{s.}{Literário: \emph{impeachment}; censura}
  \definition{v.}{atacar; criticar; destituir; censurar}
\end{EntryWithPhonetic}

\begin{EntryWithPhonetic}{抨击}{peng1ji1}{8,5}{⼿,⼐}[HSK 7-9]
  \definition{v.}{atacar (falando ou escrevendo); bombardear (com palavras); criticar}[他把电视采访作为一个机会,向反对党进行猛烈抨击。===Ele aproveitou a entrevista na televisão para lançar um ataque feroz contra o partido da oposição.]
\end{EntryWithPhonetic}

%%%%%%%%%% 烹 %%%%%%%%%%
\subsection*{烹}\addcontentsline{loh}{figure}{烹 \dpy{peng1}}

\begin{EntryWithPhonetic}{烹}{peng1}{11}{⽕}
  \definition{v.}{ferver; cozinhar | saltear; refogar; fritar}
\end{EntryWithPhonetic}

\begin{EntryWithPhonetic}{烹调}{peng1tiao2}{11,10}{⽕,⾔}[HSK 7-9]
  \definition{v.}{cozinhar; cozinhar e temperar}
\end{EntryWithPhonetic}

%%%%%%%%%% 朋 %%%%%%%%%%
\subsection*{朋}\addcontentsline{loh}{figure}{朋 \dpy{peng2}}

\begin{EntryWithPhonetic}{朋}{peng2}{8}{⽉}
  \definition*{s.}{Sobrenome: Peng}
  \definition{s.}{amigo}
  \definition{v.}{Literário: rivalizar; igualar; comparar | Literário: reunir"-se em grupo; juntar"-se em grupo}
\end{EntryWithPhonetic}

\begin{EntryWithPhonetic}{朋友}{peng2you5}{8,4}{⽉,⼜}[HSK 1]
  \definition[个,位,帮,群]{s.}{amigo; pessoas que têm um bom relacionamento, uma boa relação, se entendem e se ajudam mutuamente | namorado; namorada}
\end{EntryWithPhonetic}

%%%%%%%%%% 蓬 %%%%%%%%%%
\subsection*{蓬}\addcontentsline{loh}{figure}{蓬 \dpy{peng2}}

\begin{EntryWithPhonetic}{蓬}{peng2}{13}{⾋}
  \definition*{s.}{Sobrenome: Peng}
  \definition{adj.}{fofo; desgrenhado}
  \definition{clas.}{touceira; aglomerado; utilizado para flores e plantas exuberantes, etc.}
  \definition{s.}{pulga amarga; erva"-de"-pulga; absinto selvagem; erva"-de"-santa"-luzia}
\end{EntryWithPhonetic}

\begin{EntryWithPhonetic}{蓬勃}{peng2bo2}{13,9}{⾋,⼒}[HSK 7-9]
  \definition{adj.}{vigoroso; próspero; florescente}
\end{EntryWithPhonetic}

%%%%%%%%%% 鹏 %%%%%%%%%%
\subsection*{鹏}\addcontentsline{loh}{figure}{鹏 \dpy{peng2}}

\begin{EntryWithPhonetic}{鹏}{peng2}{13}{⿃}
  \definition*{s.}{Peng, um pássaro gigante lendário da mitologia chinesa}
\end{EntryWithPhonetic}

\begin{EntryWithPhonetic}{鹏程万里}{peng2cheng2-wan4li3}{13,12,3,7}{⿃,⽲,⼀,⾥}[HSK 7-9]
  \definition{suf.}{``Um futuro brilhante o aguarda!'';  ``O lendário Roca voa dez mil milhas.''; tenha um futuro brilhante; as perspectivas futuras de alguém são brilhantes}
\end{EntryWithPhonetic}

%%%%%%%%%% 膨 %%%%%%%%%%
\subsection*{膨}\addcontentsline{loh}{figure}{膨 \dpy{peng2}}

\begin{EntryWithPhonetic}{膨}{peng2}{16}{⾁}
  \definition{v.}{inchar; inflar | expandir; aumentar o comprimento ou o volume de um objeto}
\end{EntryWithPhonetic}

\begin{EntryWithPhonetic}{膨胀}{peng2zhang4}{16,8}{⾁,⾁}[HSK 7-9]
  \definition{v.}{inchar; dilatar; expandir; o volume do objeto aumenta | inflar; essa metáfora descreve algo que foi expandido ou crescido de forma inadequada}
\end{EntryWithPhonetic}

%%%%%%%%%% 捧 %%%%%%%%%%
\subsection*{捧}\addcontentsline{loh}{figure}{捧 \dpy{peng3}}

\begin{EntryWithPhonetic}{捧}{peng3}{11}{⼿}[HSK 7-9]
  \definition{clas.}{utilizado para coisas que podem ser seguradas}
  \definition{v.}{segurar ou carregar com ambas as mãos; apoiar com ambas as mãos | impulsionar; elogiar; exaltar; lisonjear; vangloriar}
\end{EntryWithPhonetic}

\begin{EntryWithPhonetic}{捧场}{peng3/chang3}{11,6}{⼿,⼟}[HSK 7-9]
  \definition{v.+compl.}{aplaudir; comparecer e apoiar (uma reunião, apresentação, etc.) | promover; elogiar; bajular | ser membro de uma claque | ser membro de um grupo de apoio; elogiar profusamente; prestar homenagem pública a alguém; promover alguém no programa; gabar"-se para os outros}
\end{EntryWithPhonetic}

%%%%%%%%%% 碰 %%%%%%%%%%
\subsection*{碰}\addcontentsline{loh}{figure}{碰 \dpy{peng4}}

\begin{EntryWithPhonetic}{碰}{peng4}{13}{⽯}[HSK 2]
  \definition{v.}{tocar; bater; esbarrar | encontrar; esbarrar | arriscar; tentar | tentar a sorte | reunir"-se para discutir; ter uma reunião curta}
\end{EntryWithPhonetic}

\begin{EntryWithPhonetic}{碰到}{peng4dao4}{13,8}{⽯,⼑}[HSK 2]
  \definition{v.}{encontrar (com); esbarrar; cruzar}
\end{EntryWithPhonetic}

\begin{EntryWithPhonetic}{碰钉子}{peng4 ding1zi5}{13,7,3}{⽯,⾦,⼦}[HSK 7-9]
  \definition{v.}{encontrar uma rejeição; deparar"-se com um obstáculo; essa metáfora descreve o encontro com contratempos, a rejeição ou a repreensão}
\end{EntryWithPhonetic}

\begin{EntryWithPhonetic}{碰见}{peng4jian4}{13,4}{⽯,⾒}[HSK 2]
  \definition{v.}{encontrar; encontrar"-se; sem combinar, encontrar"-se por acaso}
\end{EntryWithPhonetic}

\begin{EntryWithPhonetic}{碰巧}{peng4qiao3}{13,5}{⽯,⼯}[HSK 7-9]
  \definition{adv.}{por acaso; por coincidência}
\end{EntryWithPhonetic}

\begin{EntryWithPhonetic}{碰上}{peng4shang5}{13,3}{⽯,⼀}[HSK 7-9]
  \definition{v.}{encontrar; deparar"-se com; esbarrar em}[我在路上碰上了老朋友。===Encontrei um velho amigo na estrada.]
\end{EntryWithPhonetic}

\begin{EntryWithPhonetic}{碰头}{peng4/tou2}{13,5}{⽯,⼤}
  \definition{s.}{colisão | conflito}
  \definition{v.}{colidir}
  \definition{v.+compl.}{conhecer e discutir | juntar ideias | ver-se}
\end{EntryWithPhonetic}

\begin{EntryWithPhonetic}{碰运气}{peng4yun4qi5}{13,7,4}{⽯,⾡,⽓}
  \definition{v.}{deixar algo ao acaso | tentar a sorte}
\end{EntryWithPhonetic}

\begin{EntryWithPhonetic}{碰撞}{peng4zhuang4}{13,15}{⽯,⼿}[HSK 7-9]
  \definition{v.}{colidir; esbarrar em | ofender; afrontar}
\end{EntryWithPhonetic}

%%%%%%%%%% 批 %%%%%%%%%%
\subsection*{批}\addcontentsline{loh}{figure}{批 \dpy{pi1}}

\begin{EntryWithPhonetic}{批}{pi1}{7}{⼿}[HSK 4]
  \definition{adj.}{(compra ou venda) atacado; a granel; em grandes quantidades}
  \definition{clas.}{usado para mercadorias a granel, grande número de pessoas}
  \definition{s.}{fibras de algodão, linho, etc., prontas para serem estiradas e torcidas | anotação; comentário}
  \definition{v.}{escrever comentários ou críticas sobre documentos subordinados, textos de outras pessoas, tarefas etc. | refutar; criticar | dar um tapa}
\end{EntryWithPhonetic}

\begin{EntryWithPhonetic}{批发}{pi1fa1}{7,5}{⼿,⼜}[HSK 7-9]
  \definition{v.}{verder no atacado; vender mercadorias a granel; comprar e vender mercadorias a granel}
\end{EntryWithPhonetic}

\begin{EntryWithPhonetic}{批判}{pi1pan4}{7,7}{⼿,⼑}[HSK 7-9]
  \definition[个]{s.}{crítica}[批判性地思考问题。===Analise os problemas de forma crítica.]
  \definition{v.}{criticar; repudiar; analisar e refutar sistematicamente pensamentos, afirmações ou ações errôneas}
\end{EntryWithPhonetic}

\begin{EntryWithPhonetic}{批评}{pi1ping2}{7,7}{⼿,⾔}[HSK 3]
  \definition{v.}{criticar; comentar sobre deficiências e erros | criticar; apontar vantagens e desvantagens; comentar sobre o que é bom e o que é ruim}
\end{EntryWithPhonetic}

\begin{EntryWithPhonetic}{批准}{pi1/zhun3}{7,10}{⼿,⼎}[HSK 3]
  \definition{v.+compl.}{aprovar}
\end{EntryWithPhonetic}

%%%%%%%%%% 披 %%%%%%%%%%
\subsection*{披}\addcontentsline{loh}{figure}{披 \dpy{pi1}}

\begin{EntryWithPhonetic}{披}{pi1}{8}{⼿}[HSK 5]
  \definition{v.}{colocar sobre os ombros; enrolar em volta; cobrir ou colocar sobre os ombros | abrir; desenrolar; espalhar | abrir"-se; rachar}
\end{EntryWithPhonetic}

\begin{EntryWithPhonetic}{披露}{pi1lu4}{8,21}{⼿,⾬}[HSK 7-9]
  \definition{v.}{publicar; anunciar; tornar público | revelar; mostrar; divulgar}
\end{EntryWithPhonetic}

%%%%%%%%%% 劈 %%%%%%%%%%
\subsection*{劈}\addcontentsline{loh}{figure}{劈 \dpy{pi1}}

\begin{EntryWithPhonetic}{劈}{pi1}{15}{⼑}[HSK 7-9]
  \definition{s.}{máquina simples, cunha}
  \definition{v.}{dividir; picar; partir | ir contra (o rosto de alguém, etc.) | (raio) atingir | (madeira, etc.) rachar; fender | Dialeto: (voz) ficar rouco}
  \seeref{pi3}
\end{EntryWithPhonetic}

%%%%%%%%%% 皮 %%%%%%%%%%
\subsection*{皮}\addcontentsline{loh}{figure}{皮 \dpy{pi2}}

\begin{EntryWithPhonetic}{皮}{pi2}{5}{⽪}[HSK 3][Kangxi 107]
  \definition*{s.}{Sobrenome: Pi}
  \definition{adj.}{macios e encharcados; não mais crocantes | malandro; travesso | apático; endurecido; indiferente devido a repetidas repreensões | pegajoso; tenaz; resiliente}
  \definition{pref.}{pico- (um trilhonésimo)}
  \definition[层,块,张,个]{s.}{pele; casca; uma camada de tecido na superfície dos organismos animais e vegetais | pele; couro; couro processado | capa; embalagem; a camada externa que envolve algo | superfície do objeto | folha; peça larga e plana (de algum material fino) | borracha}
\end{EntryWithPhonetic}

\begin{EntryWithPhonetic}{皮包}{pi2bao1}{5,5}{⽪,⼓}[HSK 3]
  \definition[个,只,款]{s.}{bolsa; pasta; portfólio; bolsas de couro}
\end{EntryWithPhonetic}

\begin{EntryWithPhonetic}{皮带}{pi2dai4}{5,9}{⽪,⼱}[HSK 7-9]
  \definition[条,根]{s.}{cinto; cinto de couro; uma faixa de couro para a cintura}
\end{EntryWithPhonetic}

\begin{EntryWithPhonetic}{皮肤}{pi2fu1}{5,8}{⽪,⾁}[HSK 5]
  \definition{adj.}{superficial}
  \definition[种,块,片,层]{s.}{pele; couro; derme}
\end{EntryWithPhonetic}

\begin{EntryWithPhonetic}{皮卡}{pi2ka3}{5,5}{⽪,⼘}
  \definition{s.}{Empréstimo linguístico: \emph{pick-up} | caminhonete}
\end{EntryWithPhonetic}

\begin{EntryWithPhonetic}{皮卡丘}{pi2ka3qiu1}{5,5,5}{⽪,⼘,⼀}
  \definition*{s.}{Pikachu (Pokémon, 口袋妖怪)}
  \seealsoref{口袋妖怪}{kou3dai4 yao1guai4}
\end{EntryWithPhonetic}

\begin{EntryWithPhonetic}{皮球}{pi2qiu2}{5,11}{⽪,⽟}[HSK 6]
  \definition{s.}{bola (feita de borracha, couro etc.)}
\end{EntryWithPhonetic}

\begin{EntryWithPhonetic}{皮下}{pi2xia4}{5,3}{⽪,⼀}
  \definition{adj.}{(injeção) subcutâneo | sob a pele}
\end{EntryWithPhonetic}

\begin{EntryWithPhonetic}{皮鞋}{pi2xie2}{5,15}{⽪,⾰}[HSK 5]
  \definition[双,只,款]{s.}{sapatos feitos de couro}
\end{EntryWithPhonetic}

%%%%%%%%%% 疲 %%%%%%%%%%
\subsection*{疲}\addcontentsline{loh}{figure}{疲 \dpy{pi2}}

\begin{EntryWithPhonetic}{疲}{pi2}{10}{⽧}
  \definition{adj.}{cansado; exausto; fatigado}
\end{EntryWithPhonetic}

\begin{EntryWithPhonetic}{疲惫}{pi2bei4}{10,12}{⽧,⼼}[HSK 7-9]
  \definition{adj.}{exausto; esgotado; extremamente cansado}
\end{EntryWithPhonetic}

\begin{EntryWithPhonetic}{疲惫不堪}{pi2bei4-bu4kan1}{10,12,4,12}{⽧,⼼,⼀,⼟}[HSK 7-9]
  \definition{expr.}{completamente exausto; totalmente esgotado; descrevendo fadiga extrema}
\end{EntryWithPhonetic}

\begin{EntryWithPhonetic}{疲倦}{pi2juan4}{10,10}{⽧,⼈}[HSK 7-9]
  \definition{adj.}{cansado; exausto; fatigado; descreve uma pessoa que apresenta falta de energia e ânimo devido à falta prolongada de descanso, doença ou medicação}
\end{EntryWithPhonetic}

\begin{EntryWithPhonetic}{疲劳}{pi2lao2}{10,7}{⽧,⼒}[HSK 7-9]
  \definition{adj.}{cansado; necessidade de repouso devido a esforço físico ou mental excessivo | exaustivo; fatigado; exercícios físicos em excesso ou estímulos intensos podem enfraquecer a função ou a capacidade de resposta das células, tecidos ou órgãos | fatigado; isso descreve materiais que não podem ser usados normalmente devido ao uso prolongado ou à força externa excessiva}
\end{EntryWithPhonetic}

%%%%%%%%%% 啤 %%%%%%%%%%
\subsection*{啤}\addcontentsline{loh}{figure}{啤 \dpy{pi2}}

\begin{EntryWithPhonetic}{啤}{pi2}{11}{⼝}
  \definition{s.}{cerveja}
\end{EntryWithPhonetic}

\begin{EntryWithPhonetic}{啤酒}{pi2jiu3}{11,10}{⼝,⾣}[HSK 3]
  \definition[杯,瓶,罐,桶,缸]{s.}{(empréstimo linguístico) cerveja; uma bebida de baixo teor alcoólico feita de malte de cevada e lúpulo, com espuma e aroma especial}
\end{EntryWithPhonetic}

\begin{EntryWithPhonetic}{啤酒馆}{pi2jiu3guan3}{11,10,11}{⼝,⾣,⾷}
  \definition{s.}{cervejaria}
\end{EntryWithPhonetic}

%%%%%%%%%% 脾 %%%%%%%%%%
\subsection*{脾}\addcontentsline{loh}{figure}{脾 \dpy{pi2}}

\begin{EntryWithPhonetic}{脾}{pi2}{12}{⾁}[HSK 7-9]
  \definition{s.}{baço (órgão interno)}
\end{EntryWithPhonetic}

\begin{EntryWithPhonetic}{脾气}{pi2qi5}{12,4}{⾁,⽓}[HSK 5]
  \definition[股]{s.}{temperamento; disposição; referindo"-se ao caráter de uma pessoa | mau humor; temperamento irascível}
\end{EntryWithPhonetic}

%%%%%%%%%% 匹 %%%%%%%%%%
\subsection*{匹}\addcontentsline{loh}{figure}{匹 \dpy{pi3}}

\begin{EntryWithPhonetic}{匹}{pi3}{4}{⼖}[HSK 5]
  \definition{adj.}{solitário}
  \definition{clas.}{usado para cavalos, mulas, etc. | usado para rolos inteiros de seda ou tecido}
  \definition{v.}{ser igual a; ser compatível com}
\end{EntryWithPhonetic}

\begin{EntryWithPhonetic}{匹配}{pi3pei4}{4,10}{⼖,⾣}[HSK 7-9]
  \definition{v.}{unir; casar; corresponder; combinar; ser compatível}
\end{EntryWithPhonetic}

%%%%%%%%%% 否 %%%%%%%%%%
\subsection*{否}\addcontentsline{loh}{figure}{否 \dpy{pi3}}

\begin{EntryWithPhonetic}{否}{pi3}{7}{⼝}
  \definition{adj.}{ruim; maligno; perverso}
  \definition{v.}{censurar}
  \seeref{fou3}
\end{EntryWithPhonetic}

%%%%%%%%%% 劈 %%%%%%%%%%
\subsection*{劈}\addcontentsline{loh}{figure}{劈 \dpy{pi3}}

\begin{EntryWithPhonetic}{劈}{pi3}{15}{⼑}
  \definition{v.}{dividir; separar | romper; arrancar; ato de separar do objeto original | machucar as pernas ou os dedos abrindo"-os demais}
\end{EntryWithPhonetic}

%%%%%%%%%% 屁 %%%%%%%%%%
\subsection*{屁}\addcontentsline{loh}{figure}{屁 \dpy{pi4}}

\begin{EntryWithPhonetic}{屁}{pi4}{7}{⼫}
  \definition{s.}{vento (ou gás) (dos intestinos); peido | (vulgar) bobagem; merda; lixo | quadril; bunda}
\end{EntryWithPhonetic}

\begin{EntryWithPhonetic}{屁股}{pi4gu5}{7,8}{⼫,⾁}
  \definition{s.}{nádega | quadris}
\end{EntryWithPhonetic}

\begin{EntryWithPhonetic}{屁话}{pi4hua4}{7,8}{⼫,⾔}
  \definition{s.}{absurdo | tolice | besteira}
\end{EntryWithPhonetic}

%%%%%%%%%% 媲 %%%%%%%%%%
\subsection*{媲}\addcontentsline{loh}{figure}{媲 \dpy{pi4}}

\begin{EntryWithPhonetic}{媲}{pi4}{13}{⼥}
  \definition{v.}{Literário: ser igual a; ser compatível com; ser comparável a}
\end{EntryWithPhonetic}

\begin{EntryWithPhonetic}{媲美}{pi4mei3}{13,9}{⼥,⽺}[HSK 7-9]
  \definition{adj.}{comparável a; rival}
  \definition{v.}{comparar"-se favoravelmente com; rivalizar com; estar em pé de igualdade com; ser equivalente um ao outro}
\end{EntryWithPhonetic}

%%%%%%%%%% 僻 %%%%%%%%%%
\subsection*{僻}\addcontentsline{loh}{figure}{僻 \dpy{pi4}}

\begin{EntryWithPhonetic}{僻}{pi4}{15}{⼈}
  \definition{adj.}{afastado; isolado | excêntrico; estranho | raro}
\end{EntryWithPhonetic}

\begin{EntryWithPhonetic}{僻静}{pi4jing4}{15,14}{⼈,⾭}[HSK 7-9]
  \definition{adj.}{isolado; solitário; tranquilo}
\end{EntryWithPhonetic}

%%%%%%%%%% 譬 %%%%%%%%%%
\subsection*{譬}\addcontentsline{loh}{figure}{譬 \dpy{pi4}}

\begin{EntryWithPhonetic}{譬}{pi4}{20}{⾔}
  \definition{s.}{exemplo; analogia; metáfora}
  \definition{v.}{dar um exemplo; fazer uma analogia}
\end{EntryWithPhonetic}

\begin{EntryWithPhonetic}{譬如}{pi4ru2}{20,6}{⾔,⼥}[HSK 7-9]
  \definition{adv./conj.}{por exemplo, tal como}
  \definition{v.}{tomar como exemplo | Literário: ser semelhante a; ser exatamente igual a}
\end{EntryWithPhonetic}

\begin{EntryWithPhonetic}{譬如说}{pi4ru2 shuo1}{20,6,9}{⾔,⼥,⾔}[HSK 7-9]
  \definition{adv.}{por exemplo}
\end{EntryWithPhonetic}

%%%%%%%%%% 片 %%%%%%%%%%
\subsection*{片}\addcontentsline{loh}{figure}{片 \dpy{pian1}}

\begin{EntryWithPhonetic}{片}{pian1}{4}{⽚}[Kangxi 91]
  \definition{s.}{película; filme; refere"-se a filmes com imagens, paisagens ou imagens gravadas com som}
  \seeref{pian4}
\end{EntryWithPhonetic}

\begin{EntryWithPhonetic}{片儿}{pian1r5}{4,2}{⽚,⼉}
  \definition{s.}{folha | película; filme}
\end{EntryWithPhonetic}

\begin{EntryWithPhonetic}{片子}{pian1zi5}{4,3}{⽚,⼦}[HSK 7-9]
  \definition{s.}{imagem de raio-X; negativos fotográficos de raios X | rolo de filme | filme; cinema | disco; disco de gramofone}
  \seeref{pian4zi5}
\end{EntryWithPhonetic}

%%%%%%%%%% 扁 %%%%%%%%%%
\subsection*{扁}\addcontentsline{loh}{figure}{扁 \dpy{pian1}}

\begin{EntryWithPhonetic}{扁}{pian1}{9}{⼾}
  \definition{adj.}{pequeno | fora do caminho; remoto}
  \seeref{bian3}
\end{EntryWithPhonetic}

\begin{EntryWithPhonetic}{扁舟}{pian1 zhou1}{9,6}{⼾,⾈}
  \definition[叶,艘]{s.}{pequeno barco; esquife}
\end{EntryWithPhonetic}

%%%%%%%%%% 偏 %%%%%%%%%%
\subsection*{偏}\addcontentsline{loh}{figure}{偏 \dpy{pian1}}

\begin{EntryWithPhonetic}{偏}{pian1}{11}{⼈}[HSK 6]
  \definition{adj.}{parcial; preconceituoso; injusto; focando apenas em um lado | torto; inclinado | não dominante; auxiliar | remoto; periférico; longe do centro; incomum}
  \definition{adv.}{intencionalmente; insistentemente; persistentemente; indica ir intencionalmente contra o senso comum ou a solicitação de outra pessoa}
  \definition{expr.}{uma expressão educada para indicar que alguém já tomou chá ou comeu}
  \definition{v.}{divergir; não ser igual a; ser diferente de; exceder ou ficar aquém dos padrões normais | desviar-se; afastar-se; sair na direção certa}
  \antonymref{正}{zheng4}
\end{EntryWithPhonetic}

\begin{EntryWithPhonetic}{偏差}{pian1cha1}{11,9}{⼈,⼯}[HSK 7-9]
  \definition{s.}{(ângulo de) desvio; declinação; o ângulo em que um objeto em movimento se afasta de uma direção definida | desvio; erro; erros excessivos ou insuficientes no trabalho}
\end{EntryWithPhonetic}

\begin{EntryWithPhonetic}{偏方}{pian1fang1}{11,4}{⼈,⽅}[HSK 7-9]
  \definition{s.}{receita popular | remédio popular | remédio caseiro}
  \seealsoref{偏方儿}{pian1fang1r5}
\end{EntryWithPhonetic}

\begin{EntryWithPhonetic}{偏方儿}{pian1fang1r5}{11,4,2}{⼈,⽅,⼉}
  \definition{s.}{remédio popular}
\end{EntryWithPhonetic}

\begin{EntryWithPhonetic}{偏见}{pian1jian4}{11,4}{⼈,⾒}[HSK 7-9]
  \definition[种]{s.}{preconceito; viés; visões limitadas a um determinado aspecto}
\end{EntryWithPhonetic}

\begin{EntryWithPhonetic}{偏僻}{pian1pi4}{11,15}{⼈,⼈}[HSK 7-9]
  \definition{adj.}{remoto; fora de mão; fica longe da área movimentada e o transporte é inconveniente}
\end{EntryWithPhonetic}

\begin{EntryWithPhonetic}{偏偏}{pian1pian1}{11,11}{⼈,⼈}[HSK 7-9]
  \definition{adv.}{de forma deliberada; persistentemente; isso indica um ato deliberado contrário aos requisitos ou circunstâncias objetivas | infelizmente; isso indica que os fatos são exatamente o oposto do que se esperava ou desejava | somente; sozinho; indica um intervalo, semelhante a 单单}
  \seealsoref{单单}{dan1dan1}
\end{EntryWithPhonetic}

\begin{EntryWithPhonetic}{偏向}{pian1xiang4}{11,6}{⼈,⼝}[HSK 7-9]
  \definition{s.}{desvio; tendência errônea; tendências incorretas ou incompletas}
  \definition{v.}{proteger; ter predileção por; favorecer alguém em detrimento de outro; dar apoio ou proteção sem princípios a | preferir; inclinar"-se; favorecer}
\end{EntryWithPhonetic}

\begin{EntryWithPhonetic}{偏远}{pian1yuan3}{11,7}{⼈,⾡}[HSK 7-9]
  \definition{adj.}{remoto; distante}
\end{EntryWithPhonetic}

%%%%%%%%%% 篇 %%%%%%%%%%
\subsection*{篇}\addcontentsline{loh}{figure}{篇 \dpy{pian1}}

\begin{EntryWithPhonetic}{篇}{pian1}{15}{⽵}[HSK 2]
  \definition*{s.}{Sobrenome: Pian}
  \definition{clas.}{usado para folhas de papel, páginas de livros, artigos, etc.}
  \definition{s.}{um pedaço de escrita | folha (de papel, etc.) | (para papel, folhas de livros, artigos, etc.) folha; página; pedaço}
\end{EntryWithPhonetic}

\begin{EntryWithPhonetic}{篇幅}{pian1fu2}{15,12}{⽵,⼱}[HSK 7-9]
  \definition{s.}{extensão (de um texto); o comprimento de um artigo | número de páginas; número de páginas e seções em livros, jornais, etc.}
\end{EntryWithPhonetic}

%%%%%%%%%% 便 %%%%%%%%%%
\subsection*{便}\addcontentsline{loh}{figure}{便 \dpy{pian2}}

\begin{EntryWithPhonetic}{便}{pian2}{9}{⼈}
  \definition*{s.}{Sobrenome: Pian}
  \definition{adj.}{silencioso e confortável}
  \seeref{bian4}
\end{EntryWithPhonetic}

\begin{EntryWithPhonetic}{便宜}{pian2yi5}{9,8}{⼈,⼧}[HSK 2]
  \definition{adj.}{barato; acessível}
  \definition[个,份,件]{s.}{vantagem em algum aspecto | ganho; lucro; vantagem; benefício indevido}
  \definition{v.}{deixar alguém escapar impune; obter algum benefício}
  \seeref{bian4yi2}
\end{EntryWithPhonetic}

%%%%%%%%%% 片 %%%%%%%%%%
\subsection*{片}\addcontentsline{loh}{figure}{片 \dpy{pian4}}

\begin{EntryWithPhonetic}{片}{pian4}{4}{⽚}[HSK 2][Kangxi 91]
  \definition*{s.}{Sobrenome: Pian}
  \definition{adj.}{breve; parcial; incompleto; fragmentário; esporádico; breve | unilateral}
  \definition{clas.}{usado para coisas em forma de lâminas | usado para terrenos ou superfícies aquáticas com a mesma paisagem e que estão conectados entre si | usado para paisagens, clima, sons, linguagem, intenções, etc. (usado em conjunto com o numeral 一)}
  \definition{s.}{plano, fatia; floco; pedaço fino; algo plano e fino | seção; parte de uma grande área; uma pequena parte do todo ou uma área menor dividida dentro de uma área maior | filme; peça de TV; referência ao filme}
  \definition{v.}{fatiar; cortar em fatias; cortar em fatias finas com uma faca | abrir; cortar; separar}
  \seeref{pian1}
  \seealsoref{一}{yi1}
\end{EntryWithPhonetic}

\begin{EntryWithPhonetic}{片段}{pian4duan4}{4,9}{⽚,⽎}[HSK 7-9]
  \definition{s.}{parte; trecho; fragmento; excerto; segmento; episódio; seção; um segmento de um todo (geralmente referindo"-se a um artigo, romance, peça de teatro, experiência de vida, etc.)}
\end{EntryWithPhonetic}

\begin{EntryWithPhonetic}{片面}{pian4mian4}{4,9}{⽚,⾯}[HSK 4]
  \definition{adj.}{unilateral; tendencioso para um lado}
  \antonymref{全面}{quan2mian4}
\end{EntryWithPhonetic}

\begin{EntryWithPhonetic}{片子}{pian4zi5}{4,3}{⽚,⼦}[HSK 7-9]
  \definition{s.}{fatia; lasca; pedaço; fatia fina e plana; coisas planas e finas | cartão de visita}
  \seeref{pian1zi5}
\end{EntryWithPhonetic}

%%%%%%%%%% 骗 %%%%%%%%%%
\subsection*{骗}\addcontentsline{loh}{figure}{骗 \dpy{pian4}}

\begin{EntryWithPhonetic}{骗}{pian4}{12}{⾺}[HSK 5]
  \definition{v.}{enganar; trapacear; iludir; ludibriar; usar mentiras ou meios fraudulentos para fazer alguém acreditar ou ser enganado | enganar; fraudar | montar (um cavalo); balançar (ou saltar) para a sela}
\end{EntryWithPhonetic}

\begin{EntryWithPhonetic}{骗人}{pian4 ren2}{12,2}{⾺,⼈}[HSK 7-9]
  \definition{v.}{enganar; ludibriar alguém; enganar alguém usando mentiras ou truques}
\end{EntryWithPhonetic}

\begin{EntryWithPhonetic}{骗子}{pian4zi5}{12,3}{⾺,⼦}[HSK 5]
  \definition[个]{s.}{trapaceiro; vigarista; fraudador; impostor; golpista; pessoa que obtém bens de forma fraudulenta}
\end{EntryWithPhonetic}

%%%%%%%%%% 漂 %%%%%%%%%%
\subsection*{漂}\addcontentsline{loh}{figure}{漂 \dpy{piao1}}

\begin{EntryWithPhonetic}{漂}{piao1}{14}{⽔}[HSK 7-9]
  \definition{v.}{flutuar; derivar; flutuar na superfície de um líquido; flutuar na superfície da água e mover"-se com a corrente; mover"-se com o vento}
  \seeref{piao3}
  \seeref{piao4}
\end{EntryWithPhonetic}

\begin{EntryWithPhonetic}{漂流}{piao1liu2}{14,10}{⽔,⽔}
  \definition{s.}{\emph{rafting}}
  \definition{v.}{derivar; flutuar; flutuar na água e à deriva com a corrente | deixar"-se levar; ter vida errante; estar à deriva | praticar \emph{rafting} em um pequeno barco ou jangada inflável, geralmente como atividade recreativa; descer as corredeiras em pequenos barcos ou jangadas é hoje considerado, em grande parte, uma atividade de recreação aquática}
\end{EntryWithPhonetic}

%%%%%%%%%% 飘 %%%%%%%%%%
\subsection*{飘}\addcontentsline{loh}{figure}{飘 \dpy{piao1}}

\begin{EntryWithPhonetic}{飘}{piao1}{15}{⾵}[HSK 7-9]
  \definition{adj.}{superficial; frívolo; autossatisfeito; complacente | instável; fraco}
  \definition{v.}{balançar para lá e para cá; flutuar (no ar); tremular}
\end{EntryWithPhonetic}

%%%%%%%%%% 朴 %%%%%%%%%%
\subsection*{朴}\addcontentsline{loh}{figure}{朴 \dpy{piao2}}

\begin{EntryWithPhonetic}{朴}{piao2}{6}{⽊}
  \definition*{s.}{Sobrenome coreano: Park, Pak ou Bak | Sobrenome: Piao}
  \definition{adj.}{simples; despretensioso}
  \seeref{po1}
  \seeref{po4}
  \seeref{pu3}
\end{EntryWithPhonetic}

%%%%%%%%%% 漂 %%%%%%%%%%
\subsection*{漂}\addcontentsline{loh}{figure}{漂 \dpy{piao3}}

\begin{EntryWithPhonetic}{漂}{piao3}{14}{⽔}
  \definition{v.}{branquear (com água sanitária) | enxaguar; enxaguar com água para remover as impurezas}
  \seeref{piao1}
  \seeref{piao4}
\end{EntryWithPhonetic}

%%%%%%%%%% 票 %%%%%%%%%%
\subsection*{票}\addcontentsline{loh}{figure}{票 \dpy{piao4}}

\begin{EntryWithPhonetic}{票}{piao4}{11}{⽰}[HSK 1]
  \definition{clas.}{para grupos, lotes, transações comerciais}
  \definition[张]{s.}{bilhete; passagem; ingresso | cédula | nota bancária; conta | pessoa mantida em cativeiro por sequestradores para obter resgate; refém | apresentação amadora (de ópera de Pequim, etc.); peças teatrais amadoras}
  \definition{v.}{atuar como amador (na ópera de Pequim)}
\end{EntryWithPhonetic}

\begin{EntryWithPhonetic}{票房}{piao4fang2}{11,8}{⽰,⼾}[HSK 7-9]
  \definition{s.}{Coloquial: bilheteria (em uma estação ferroviária, aeroporto, etc.); bilheteria (em um teatro, estádio, etc.) | valor de bilheteria; receita de bilheteria | Obsoleto: clube para artistas amadores de ópera de Pequim, etc.}
  \seealsoref{票房儿}{piao4fang2r5}
\end{EntryWithPhonetic}

\begin{EntryWithPhonetic}{票房儿}{piao4fang2r5}{11,8,2}{⽰,⼾,⼉}
  \definition{s.}{bilheteria}
\end{EntryWithPhonetic}

\begin{EntryWithPhonetic}{票价}{piao4jia4}{11,6}{⽰,⼈}[HSK 3]
  \definition[个]{s.}{o preço de um ingresso; taxa de admissão; taxa de entrada}
\end{EntryWithPhonetic}

%%%%%%%%%% 漂 %%%%%%%%%%
\subsection*{漂}\addcontentsline{loh}{figure}{漂 \dpy{piao4}}

\begin{EntryWithPhonetic}{漂}{piao4}{14}{⽔}
  \definition{adj.}{bonita; usado em 漂亮}
  \definition{v.}{falhar; terminar em fracasso}[这笔投资的钱全都漂了。===Todo o dinheiro desse investimento foi perdido.]
  \seeref{piao1}
  \seeref{piao3}
  \seealsoref{漂亮}{piao4liang5}
\end{EntryWithPhonetic}

\begin{EntryWithPhonetic}{漂亮}{piao4liang5}{14,9}{⽔,⼇}[HSK 2]
  \definition{adj.}{bonito; lindo; atraente; de boa aparência; esteticamente agradável | excelente; notável | não pode ser utilizado para descrever homens}
\end{EntryWithPhonetic}

%%%%%%%%%% 撇 %%%%%%%%%%
\subsection*{撇}\addcontentsline{loh}{figure}{撇 \dpy{pie1}}

\begin{EntryWithPhonetic}{撇}{pie1}{14}{⼿}
  \definition{v.}{descartar; jogar ao mar; abandonar | desnatar; retirar delicadamente o líquido da superfície}
  \seeref{pie3}
\end{EntryWithPhonetic}

\begin{EntryWithPhonetic}{撇}{pie3}{14}{⼿}[HSK 7-9]
  \definition{clas.}{utilizado para coisas sobrancelhas e barbas}
  \definition{s.}{traço descendente à esquerda 丿(em caracteres chineses)}
  \definition{v.}{atirar; arremessar; lançar}
  \seeref{pie1}
\end{EntryWithPhonetic}

%%%%%%%%%% 拼 %%%%%%%%%%
\subsection*{拼}\addcontentsline{loh}{figure}{拼 \dpy{pin1}}

\begin{EntryWithPhonetic}{拼}{pin1}{9}{⼿}[HSK 5]
  \definition{v.}{montar; juntar as peças | dar tudo de si no trabalho; estar disposto a arriscar a vida (em lutas, no trabalho, etc.); fazer tudo o que for preciso; arriscar tudo}
\end{EntryWithPhonetic}

\begin{EntryWithPhonetic}{拼搏}{pin1bo2}{9,13}{⼿,⼿}[HSK 7-9]
  \definition{v.}{dar tudo de si; encarar (um desafio) de frente; utilizar todas as suas forças para obter algo ou atingir um objetivo}
\end{EntryWithPhonetic}

\begin{EntryWithPhonetic}{拼命}{pin1/ming4}{9,8}{⼿,⼝}[HSK 7-9]
  \definition{adv.}{desesperadamente; arriscar a vida; com todas as forças; por favor, faça algo com o máximo empenho e energia}
  \definition{v.+compl.}{arriscar a própria vida; esforçar"-se ao máximo; brigar com alguém ou fazer coisas sem considerar a segurança}
\end{EntryWithPhonetic}

\begin{EntryWithPhonetic}{拼音}{pin1yin1}{9,9}{⼿,⾳}
  \definition{s.}{escrita fonética | pinyin (romanização chinesa)}
\end{EntryWithPhonetic}

%%%%%%%%%% 贫 %%%%%%%%%%
\subsection*{贫}\addcontentsline{loh}{figure}{贫 \dpy{pin2}}

\begin{EntryWithPhonetic}{贫}{pin2}{8}{⾙}
  \definition{adj.}{pobre; empobrecido | inadequado; deficiente; insuficiente | tagarela; loquaz; falante; chato e irritante}
\end{EntryWithPhonetic}

\begin{EntryWithPhonetic}{贫富}{pin2 fu4}{8,12}{⾙,⼧}[HSK 7-9]
  \definition{s.}{pobreza e riqueza; pobres e ricos}
\end{EntryWithPhonetic}

\begin{EntryWithPhonetic}{贫困}{pin2kun4}{8,7}{⾙,⼞}[HSK 6]
  \definition{adj.}{pobre; indigente; necessitado; empobrecido; assolado pela pobreza; em circunstâncias difíceis}
\end{EntryWithPhonetic}

\begin{EntryWithPhonetic}{贫民窟}{pin2min2ku1}{8,5,13}{⾙,⽒,⽳}
  \definition{s.}{favela}
\end{EntryWithPhonetic}

\begin{EntryWithPhonetic}{贫穷}{pin2qiong2}{8,7}{⾙,⽳}[HSK 7-9]
  \definition{adj.}{pobre; empobrecido; falta de meios de produção e de meios de subsistência}
\end{EntryWithPhonetic}

%%%%%%%%%% 频 %%%%%%%%%%
\subsection*{频}\addcontentsline{loh}{figure}{频 \dpy{pin2}}

\begin{EntryWithPhonetic}{频}{pin2}{13}{⾴}
  \definition*{s.}{Sobrenome: Pin}
  \definition{adj.}{frequente}
  \definition{adv.}{frequentemente; repetidamente}
  \definition{s.}{Física: frequência; o número de vezes que um objeto vibra por segundo}
\end{EntryWithPhonetic}

\begin{EntryWithPhonetic}{频道}{pin2dao4}{13,12}{⾴,⾡}[HSK 5]
  \definition[个]{s.}{canal; canal de frequência; televisão e rádio, os sinais de som e imagem ocupam um determinado canal de frequência}
\end{EntryWithPhonetic}

\begin{EntryWithPhonetic}{频繁}{pin2fan2}{13,17}{⾴,⽷}[HSK 5]
  \definition{adj.}{frequentemente}
  \definition{adj.}{frequente}
\end{EntryWithPhonetic}

\begin{EntryWithPhonetic}{频率}{pin2lv4}{13,11}{⾴,⽞}[HSK 7-9]
  \definition{s.}{frequência; número de vibrações por segundo; o número de vezes que um determinado evento ocorre dentro de uma unidade de tempo}
\end{EntryWithPhonetic}

\begin{EntryWithPhonetic}{频频}{pin2pin2}{13,13}{⾴,⾴}[HSK 7-9]
  \definition{adv.}{repetidamente; várias e várias vezes; continuamente}
\end{EntryWithPhonetic}

%%%%%%%%%% 品 %%%%%%%%%%
\subsection*{品}\addcontentsline{loh}{figure}{品 \dpy{pin3}}

\begin{EntryWithPhonetic}{品}{pin3}{9}{⼝}[HSK 5]
  \definition*{s.}{Sobrenome: Pin}
  \definition{s.}{artigo; produto | grau; classe; classificação; nível | caráter; qualidade | classificação; os graus dos funcionários públicos antigos, num total de nove graus}
  \definition{v.}{provar; saborear; degustar algo com discernimento | soprar; tocar (instrumentos de sopro) | avaliar; distinguir o bom do ruim}
\end{EntryWithPhonetic}

\begin{EntryWithPhonetic}{品尝}{pin3chang2}{9,9}{⼝,⼩}[HSK 7-9]
  \definition{v.}{provar; degustar; saborear; saborear a comida devagar e com atenção}
\end{EntryWithPhonetic}

\begin{EntryWithPhonetic}{品德}{pin3de2}{9,15}{⼝,⼻}[HSK 7-9]
  \definition[种]{s.}{caráter moral; qualidade; qualidade e moralidade}
\end{EntryWithPhonetic}

\begin{EntryWithPhonetic}{品牌}{pin3pai2}{9,12}{⼝,⽚}[HSK 6]
  \definition[个,种]{s.}{marca registrada; nome de marca}
\end{EntryWithPhonetic}

\begin{EntryWithPhonetic}{品位}{pin3wei4}{9,7}{⼝,⼈}[HSK 7-9]
  \definition{s.}{qualidade; gosto; refere"-se à qualidade ou ao valor de uma pessoa ou coisa | grau; qualidade; a quantidade de um elemento desejado ou de seu composto contida em um minério (geralmente expressa em porcentagem) | classificação; na antiguidade, referia"-se à posição e ao cargo oficial}
\end{EntryWithPhonetic}

\begin{EntryWithPhonetic}{品行}{pin3xing2}{9,6}{⼝,⾏}[HSK 7-9]
  \definition[位]{s.}{conduta moral; comportamento | caráter; conduta}
\end{EntryWithPhonetic}

\begin{EntryWithPhonetic}{品质}{pin3zhi4}{9,8}{⼝,⾙}[HSK 4]
  \definition[个,种]{s.}{qualidade; caráter; natureza do pensamento, da compreensão, do caráter, etc., conforme expresso no comportamento, no estilo, etc. | qualidade (de produtos, mercadorias, etc.)}
\end{EntryWithPhonetic}

\begin{EntryWithPhonetic}{品种}{pin3zhong3}{9,9}{⼝,⽲}[HSK 5]
  \definition[个,些]{s.}{raça; linhagem; variedade; refere"-se a um grupo de organismos com características genéticas comuns, formados por meio da seleção e cultivo artificiais de culturas, gado, aves, etc. | variedade; sortimento; referência geral ao tipo de item}
\end{EntryWithPhonetic}

%%%%%%%%%% 牝 %%%%%%%%%%
\subsection*{牝}\addcontentsline{loh}{figure}{牝 \dpy{pin4}}

\begin{EntryWithPhonetic}{牝}{pin4}{6}{⽜}
  \definition{adj.}{(de certas aves e animais) fêmea}
  \definition{s.}{fêmea (de algumas aves e animais)}
  \antonymref{牡}{mu3}
\end{EntryWithPhonetic}

%%%%%%%%%% 聘 %%%%%%%%%%
\subsection*{聘}\addcontentsline{loh}{figure}{聘 \dpy{pin4}}

\begin{EntryWithPhonetic}{聘}{pin4}{13}{⽿}[HSK 7-9]
  \definition{v.}{contratar | visitar como enviado; visitar como representante da família para fins de arranjo matrimonial (cultura tradicional) | (uma menina) casar"-se ou ser dada em casamento; ficar noivos}
\end{EntryWithPhonetic}

\begin{EntryWithPhonetic}{聘请}{pin4qing3}{13,10}{⽿,⾔}[HSK 6]
  \definition{v.}{convidar; empregar; envolver; chamar; contratar alguém para assumir uma posição}
\end{EntryWithPhonetic}

\begin{EntryWithPhonetic}{聘任}{pin4ren4}{13,6}{⽿,⼈}[HSK 7-9]
  \definition{v.}{contratar alguém para; nomear alguém para um cargo; contratar; recrutar por convite; contratar para atuar como (cargo)}
\end{EntryWithPhonetic}

\begin{EntryWithPhonetic}{聘用}{pin4yong4}{13,5}{⽿,⽤}[HSK 7-9]
  \definition{v.}{contratar; nomear alguém para um cargo}
\end{EntryWithPhonetic}

%%%%%%%%%% 乒 %%%%%%%%%%
\subsection*{乒}\addcontentsline{loh}{figure}{乒 \dpy{ping1}}

\begin{EntryWithPhonetic}{乒}{ping1}{6}{⼃}
  \definition{interj.}{Onomatopéia: estalo; estouro; estrondo | Onomatopéia: ``ping''}
  \definition{s.}{tênis de mesa; pingue-pongue; abreviação de 乒乓球 | bola de tênis de mesa; bola de pingue-pongue; abreviação de 乒乓球}
  \seealsoref{乒乓球}{ping1pang1qiu2}
\end{EntryWithPhonetic}

\begin{EntryWithPhonetic}{乒乓球}{ping1pang1qiu2}{6,6,11}{⼃,⼃,⽟}[HSK 7-9]
  \definition[个,只]{s.}{pingue-pongue; tênis de mesa | bola de tênis de mesa; bola de pingue-pongue}
\end{EntryWithPhonetic}

%%%%%%%%%% 平 %%%%%%%%%%
\subsection*{平}\addcontentsline{loh}{figure}{平 \dpy{ping2}}

\begin{EntryWithPhonetic}{平}{ping2}{5}{⼲}[HSK 2]
  \definition*{s.}{Sobrenome: Ping}
  \definition{adj.}{plano; nivelado; uniforme; liso | igual; justo | mesma pontuação; empatado | médio; comum | silencioso; tranquilo | no mesmo nível; altura igual; sem diferença | imparcial; médio; equitativo | calmo; estável; tranquilo | comum;  frequente}
  \definition{s.}{no mesmo nível; em pé de igualdade com; igual | tom nivelado, um dos quatro tons do chinês clássico}
  \definition{v.}{tornar nivelado ou uniforme; nivelar | reprimir; suprimir | acalmar; tornar pacífico; silenciar (acalmar); conter a raiva | estar no mesmo nível | acalmar; amenizar; controlar a raiva}
\end{EntryWithPhonetic}

\begin{EntryWithPhonetic}{平安}{ping2'an1}{5,6}{⼲,⼧}[HSK 2]
  \definition{s.}{seguro; bem; sem contratempos; sem acidentes; são e salvo}
\end{EntryWithPhonetic}

\begin{EntryWithPhonetic}{平常}{ping2chang2}{5,11}{⼲,⼱}[HSK 2]
  \definition{adj.}{comum; normal; ordinário; nada de especial}
  \definition{adv.}{normalmente; geralmente; como regra geral}
\end{EntryWithPhonetic}

\begin{EntryWithPhonetic}{平常心}{ping2chang2xin1}{5,11,4}{⼲,⼱,⼼}[HSK 7-9]
  \definition[个]{s.}{mente calma; atitude calma e serena; compostura}
\end{EntryWithPhonetic}

\begin{EntryWithPhonetic}{平淡}{ping2dan4}{5,11}{⼲,⽔}[HSK 7-9]
  \definition{adj.}{sem graça; monótono; entediante; insípido; prosaico; banal; desinteressante; insosso e sem graça; ordinário; sem qualquer reviravolta}
\end{EntryWithPhonetic}

\begin{EntryWithPhonetic}{平等}{ping2deng3}{5,12}{⼲,⽵}[HSK 2]
  \definition{adj.}{igual; igualdade; refere"-se ao fato de as pessoas gozarem de tratamento igualitário nos aspectos sociais, políticos, econômicos e jurídicos}
\end{EntryWithPhonetic}

\begin{EntryWithPhonetic}{平地}{ping2di4}{5,6}{⼲,⼟}
  \definition{v.}{nivelar a terra | aplanar}
\end{EntryWithPhonetic}

\begin{EntryWithPhonetic}{平凡}{ping2fan2}{5,3}{⼲,⼏}[HSK 6]
  \definition{adj.}{comum; ordinário; normal; não surpreendente}
\end{EntryWithPhonetic}

\begin{EntryWithPhonetic}{平方}{ping2fang1}{5,4}{⼲,⽅}[HSK 4]
  \definition{s.}{Matemática: segunda potência (de uma quantidade); quadrado | metro quadrado (m²)}[那间房有十二平方。===O quarto tem doze metros quadrados.]
\end{EntryWithPhonetic}

\begin{EntryWithPhonetic}{平方米}{ping2fang1mi3}{5,4,6}{⼲,⽅,⽶}[HSK 6]
  \definition{s.}{metro quadrado; a unidade legal de medida de área, 1 metro quadrado é igual a 10.000 centímetros quadrados}
\end{EntryWithPhonetic}

\begin{EntryWithPhonetic}{平方市丈}{ping2fang1 shi4 zhang4}{5,4,5,3}{⼲,⽅,⼱,⼀}
  \definition{clas.}{pés quadrados}
\end{EntryWithPhonetic}

\begin{EntryWithPhonetic}{平和}{ping2he2}{5,8}{⼲,⼝}[HSK 7-9]
  \definition{adj.}{suave; ameno; moderado; plácido | (medicamento) leve}[这种药的药性平和。===Este medicamento tem um efeito leve.]
\end{EntryWithPhonetic}

\begin{EntryWithPhonetic}{平衡}{ping2heng2}{5,16}{⼲,⾏}[HSK 6]
  \definition{adj.}{balanceado; equilibrado; os aspectos opostos são iguais ou compensados em quantidade ou qualidade | equilibrado; várias forças atuam sobre um objeto com magnitude igual e direções opostas para manter o objeto estável}
  \definition{v.}{equilibrar; trazer ou manter em equilíbrio; tornar as coisas ou alimentos iguais em quantidade, qualidade ou força}
\end{EntryWithPhonetic}

\begin{EntryWithPhonetic}{平价}{ping2jia4}{5,6}{⼲,⼈}[HSK 7-9]
  \definition{s.}{preços estabilizados (ou normalizados, moderados) | paridade; preço justo; preço de paridade | preço razoável; preço baixo; preço justo | preço justo (estatal); preço fixado pelo Estado; preço normal, racional; o padrão da moeda oficial de um país; a taxa de câmbio padrão entre as moedas do padrão"-ouro de dois países}
  \definition{v.}{estabilizar os preços}
\end{EntryWithPhonetic}

\begin{EntryWithPhonetic}{平静}{ping2jing4}{5,14}{⼲,⾭}[HSK 4]
  \definition{adj.}{(humor, ambiente, etc.) calmo; quieto; pacífico; tranquilo}
\end{EntryWithPhonetic}

\begin{EntryWithPhonetic}{平均}{ping2jun1}{5,7}{⼲,⼟}[HSK 4]
  \definition{adj.}{igual; médio}
  \definition{s.}{média}
  \definition{v.}{calcular a média de um conjunto de números}
\end{EntryWithPhonetic}

\begin{EntryWithPhonetic}{平面}{ping2mian4}{5,9}{⼲,⾯}[HSK 7-9]
  \definition[个]{s.}{plano; superfície plana; duas dimensões}
\end{EntryWithPhonetic}

\begin{EntryWithPhonetic}{平民}{ping2min2}{5,5}{⼲,⽒}[HSK 7-9]
  \definition[个]{s.}{civis; pessoas comuns; refere"-se geralmente a pessoas comuns (em oposição a nobres ou classes privilegiadas)}
\end{EntryWithPhonetic}

\begin{EntryWithPhonetic}{平日}{ping2ri4}{5,4}{⼲,⽇}[HSK 7-9]
  \definition{adv.}{normalmente; em tempos normais; em dias comuns; em circunstâncias normais}
  \definition{s.}{dias úteis; dias comuns; dias da semana}
\end{EntryWithPhonetic}

\begin{EntryWithPhonetic}{平时}{ping2shi2}{5,7}{⼲,⽇}[HSK 2]
  \definition{s.}{em tempos normais; em tempos comuns | em tempo de paz; refere"-se a períodos normais}
\end{EntryWithPhonetic}

\begin{EntryWithPhonetic}{平台}{ping2tai2}{5,5}{⼲,⼝}[HSK 6]
  \definition[个]{s.}{casa com telhado plano rebocado | terraço | plataforma móvel; metaforicamente, refere"-se às áreas, oportunidades, ambientes, espaços, etc. que fornecem suporte e garantia para algo | plataforma; um sistema em um computador eletrônico que consiste em software e hardware básicos; tal sistema pode suportar a execução de programas aplicativos e softwares aplicativos podem ser desenvolvidos nesse sistema | plataforma; lugar; falando metaforicamente, o mesmo nível ou grau}
\end{EntryWithPhonetic}

\begin{EntryWithPhonetic}{平坦}{ping2tan3}{5,8}{⼲,⼟}[HSK 5]
  \definition{adj.}{plano; uniforme; nivelado; liso; sem elevações ou depressões (referindo"-se principalmente ao relevo)}
\end{EntryWithPhonetic}

\begin{EntryWithPhonetic}{平稳}{ping2wen3}{5,14}{⼲,⽲}[HSK 4]
  \definition{adj.}{firme; estável; suave e constante; sem oscilações ou flutuações}
\end{EntryWithPhonetic}

\begin{EntryWithPhonetic}{平息}{ping2xi1}{5,10}{⼲,⼼}[HSK 7-9]
  \definition{v.}{acalmar"-se; aquietar"-se; emoções negativas como conflito, disputa ou raiva diminuem ou cessam | suprimir; abafar; sufocar (uma rebelião, etc.); usar a força para resolver problemas}
\end{EntryWithPhonetic}

\begin{EntryWithPhonetic}{平原}{ping2yuan2}{5,10}{⼲,⼚}[HSK 5]
  \definition[片,个]{s.}{campo; planície; terreno plano e extenso}
\end{EntryWithPhonetic}

%%%%%%%%%% 评 %%%%%%%%%%
\subsection*{评}\addcontentsline{loh}{figure}{评 \dpy{ping2}}

\begin{EntryWithPhonetic}{评}{ping2}{7}{⾔}[HSK 6]
  \definition*{s.}{Sobrenome: Ping}
  \definition{v.}{comentar; criticar; revisar | julgar; avaliar}
\end{EntryWithPhonetic}

\begin{EntryWithPhonetic}{评定}{ping2ding4}{7,8}{⾔,⼧}[HSK 7-9]
  \definition{v.}{emitir juízo sobre; avaliar; analisar; apreciar; julgar; as decisões são tomadas com base em avaliação ou análise}
\end{EntryWithPhonetic}

\begin{EntryWithPhonetic}{评估}{ping2gu1}{7,7}{⾔,⼈}[HSK 5]
  \definition{v.}{estimar; avaliar; apreciar; avaliar e estimar (coisas abstratas)}
\end{EntryWithPhonetic}

\begin{EntryWithPhonetic}{评价}{ping2jia4}{7,6}{⾔,⼈}[HSK 3]
  \definition[个,项,条,份]{s.}{avaliação; apreciação; comentários ou opiniões de pessoas sobre alguém ou algo}
  \definition{v.}{estimar valor; avaliar valor}
\end{EntryWithPhonetic}

\begin{EntryWithPhonetic}{评论}{ping2lun4}{7,6}{⾔,⾔}[HSK 5]
  \definition[篇,些]{s.}{revisão; comentário; artigos ou comentários críticos}
  \definition{v.}{discutir; comentar sobre algo ou alguém}
\end{EntryWithPhonetic}

\begin{EntryWithPhonetic}{评论员}{ping2lun4yuan2}{7,6,7}{⾔,⾔,⼝}[HSK 7-9]
  \definition[名,位,个]{s.}{comentarista}
\end{EntryWithPhonetic}

\begin{EntryWithPhonetic}{评判}{ping2pan4}{7,7}{⾔,⼑}[HSK 7-9]
  \definition{v.}{julgar; decidir; emitir juízo sobre; determinar o vencedor ou a superioridade/inferioridade}
\end{EntryWithPhonetic}

\begin{EntryWithPhonetic}{评审}{ping2shen3}{7,8}{⾔,⼧}[HSK 7-9]
  \definition[名,位]{s.}{juiz; revisor; pessoas que realizam o trabalho de revisão}
  \definition{v.}{avaliar; examinar e verificar}
\end{EntryWithPhonetic}

\begin{EntryWithPhonetic}{评委}{ping2wei3}{7,8}{⾔,⼥}[HSK 7-9]
  \definition[位,名,个,些]{s.}{membro de uma comissão de revisão; é um termo geral que se refere àqueles que são nomeados, implementam as regras e avaliam o pessoal}
\end{EntryWithPhonetic}

\begin{EntryWithPhonetic}{评选}{ping2xuan3}{7,9}{⾔,⾡}[HSK 6]
  \definition{v.}{escolher por meio de avaliação pública; avaliar e eleger}
\end{EntryWithPhonetic}

%%%%%%%%%% 凭 %%%%%%%%%%
\subsection*{凭}\addcontentsline{loh}{figure}{凭 \dpy{ping2}}

\begin{EntryWithPhonetic}{凭}{ping2}{8}{⼏}[HSK 5]
  \definition{conj.}{não importa (o que, como, etc.); conecta frases complexas condicionais para expressar incondicionalidade, equivalente a 任凭 ou 不论}
  \definition{prep.}{introduzir a ação ou o comportamento com base em algo; quando a frase nominal após 凭 é longa, pode"-se adicionar 着 após 凭}
  \definition[张]{s.}{prova; evidência}
  \definition{v.}{apoiar"-se; encostar"-se | confiar em; depender de | basear"-se em; tomar como base}
  \seealsoref{不论}{bu2lun4}
  \seealsoref{任凭}{ren4 ping2}
  \seealsoref{着}{zhe5}
\end{EntryWithPhonetic}

\begin{EntryWithPhonetic}{凭借}{ping2jie4}{8,10}{⼏,⼈}[HSK 7-9]
  \definition{v.}{confiar em; depender de}
\end{EntryWithPhonetic}

\begin{EntryWithPhonetic}{凭着}{ping2zhe5}{8,11}{⼏,⽬}[HSK 7-9]
  \definition{prep.}{em virtude de; se baseia em}[她凭着多年的经验做事。===Ela se baseia em seus anos de experiência para realizar as coisas.]
\end{EntryWithPhonetic}

\begin{EntryWithPhonetic}{凭证}{ping2zheng4}{8,7}{⼏,⾔}[HSK 7-9]
  \definition{s.}{prova; certificado; comprovante; evidência}
\end{EntryWithPhonetic}

%%%%%%%%%% 苹 %%%%%%%%%%
\subsection*{苹}\addcontentsline{loh}{figure}{苹 \dpy{ping2}}

\begin{EntryWithPhonetic}{苹}{ping2}{8}{⾋}
  \definition[个]{s.}{uma espécie de artemísia | maçã | lentilha"-d'água}
\end{EntryWithPhonetic}

\begin{EntryWithPhonetic}{苹果}{ping2guo3}{8,8}{⾋,⽊}[HSK 3]
  \definition[个,斤,筐,箱,棵,种]{s.}{maçã}
\end{EntryWithPhonetic}

%%%%%%%%%% 屏 %%%%%%%%%%
\subsection*{屏}\addcontentsline{loh}{figure}{屏 \dpy{ping2}}

\begin{EntryWithPhonetic}{屏}{ping2}{9}{⼫}
  \definition{s.}{tela | um conjunto de pergaminhos; tiras de tela}
  \definition{v.}{proteger alguém ou algo; resguardar}
  \seeref{bing1}
  \seeref{bing3}
\end{EntryWithPhonetic}

\begin{EntryWithPhonetic}{屏幕}{ping2mu4}{9,13}{⼫,⼱}[HSK 6]
  \definition[个,块]{s.}{tela; a parte dos computadores, televisores, celulares, etc. que exibe texto, imagens, etc.}
\end{EntryWithPhonetic}

%%%%%%%%%% 瓶 %%%%%%%%%%
\subsection*{瓶}\addcontentsline{loh}{figure}{瓶 \dpy{ping2}}

\begin{EntryWithPhonetic}{瓶}{ping2}{10}{⽡}[HSK 2]
  \definition*{s.}{Sobrenome: Ping}
  \definition{clas.}{usado para coisas que são engarrafadas; quantidade contida em um frasco, vaso, garrafa}
  \definition[个]{s.}{jarra; vaso; frasco; garrafa}
\end{EntryWithPhonetic}

\begin{EntryWithPhonetic}{瓶盖}{ping2gai4}{10,11}{⽡,⽫}
  \definition{s.}{tampa de garrafa}
\end{EntryWithPhonetic}

\begin{EntryWithPhonetic}{瓶颈}{ping2jing3}{10,11}{⽡,⾴}[HSK 7-9]
  \definition[个]{s.}{gargalo; gargalo de uma garrafa | gargalo; uma metáfora para um elo crucial que impede o desenvolvimento de algo}
\end{EntryWithPhonetic}

\begin{EntryWithPhonetic}{瓶装}{ping2zhuang1}{10,12}{⽡,⾐}
  \definition{adj.}{engarrafado}
\end{EntryWithPhonetic}

\begin{EntryWithPhonetic}{瓶子}{ping2zi5}{10,3}{⽡,⼦}[HSK 2]
  \definition[个,只,种]{s.}{garrafa; recipientes com gargalo feitos de cerâmica, vidro, plástico, etc., geralmente em forma cilíndrica}
\end{EntryWithPhonetic}

%%%%%%%%%% 萍 %%%%%%%%%%
\subsection*{萍}\addcontentsline{loh}{figure}{萍 \dpy{ping2}}

\begin{EntryWithPhonetic}{萍}{ping2}{11}{⾋}
  \definition{s.}{lentilha"-d'água}
\end{EntryWithPhonetic}

\begin{EntryWithPhonetic}{萍水相逢}{ping2shui3-xiang1feng2}{11,4,9,10}{⾋,⽔,⽬,⾡}[HSK 7-9]
  \definition{expr.}{(estranhos) encontram"-se por acaso como tufos de lentilha"-d'água à deriva; um conhecido casual; realizar uma reunião informal e temporária}
\end{EntryWithPhonetic}

%%%%%%%%%% 甁 %%%%%%%%%%
\subsection*{甁}\addcontentsline{loh}{figure}{甁 \dpy{ping2}}

\begin{EntryWithPhonetic}{甁}{ping2}{12}{⽡}
  \variantof{瓶}
\end{EntryWithPhonetic}

%%%%%%%%%% 朴 %%%%%%%%%%
\subsection*{朴}\addcontentsline{loh}{figure}{朴 \dpy{po1}}

\begin{EntryWithPhonetic}{朴}{po1}{6}{⽊}
  \definition{s.}{uma arma tradicional de haste com uma lâmina longa e estreita, usada com ambas as mãos}
  \seeref{piao2}
  \seeref{po4}
  \seeref{pu3}
\end{EntryWithPhonetic}

%%%%%%%%%% 坡 %%%%%%%%%%
\subsection*{坡}\addcontentsline{loh}{figure}{坡 \dpy{po1}}

\begin{EntryWithPhonetic}{坡}{po1}{8}{⼟}[HSK 6]
  \definition{adj.}{inclinado}
  \definition{s.}{declive | encosta}
\end{EntryWithPhonetic}

%%%%%%%%%% 泼 %%%%%%%%%%
\subsection*{泼}\addcontentsline{loh}{figure}{泼 \dpy{po1}}

\begin{EntryWithPhonetic}{泼}{po1}{8}{⽔}[HSK 5]
  \definition{adj.}{rude e irracional; mal"-humorado | Dialeto: ousado e vigoroso; ousado e resoluto}
  \definition{v.}{espalhar; salpicar; derramar; derramar ou espalhar o líquido com força para fora}
\end{EntryWithPhonetic}

\begin{EntryWithPhonetic}{泼冷水}{po1 leng3shui3}{8,7,4}{⽔,⼎,⽔}[HSK 7-9]
  \definition{v.}{desencorajar; jogar um balde de água fria em; arrefecer o entusiasmo (ou o ânimo) de alguém; isso se refere metaforicamente a suprimir ou limitar o entusiasmo de alguém, ou a fazê-lo cair em si}
\end{EntryWithPhonetic}

%%%%%%%%%% 颇 %%%%%%%%%%
\subsection*{颇}\addcontentsline{loh}{figure}{颇 \dpy{po1}}

\begin{EntryWithPhonetic}{颇}{po1}{11}{⽪}[HSK 7-9]
  \definition*{s.}{Sobrenome: Po}
  \definition{adj.}{oblíquo; inclinado para um lado | Literário: tendencioso; incorreto}
  \definition{adv.}{muito; bastante; consideravelmente}
\end{EntryWithPhonetic}

%%%%%%%%%% 朴 %%%%%%%%%%
\subsection*{朴}\addcontentsline{loh}{figure}{朴 \dpy{po4}}

\begin{EntryWithPhonetic}{朴}{po4}{6}{⽊}
  \definition{s.}{\emph{hackberry} chinês; celtis}
  \seeref{piao2}
  \seeref{po1}
  \seeref{pu3}
\end{EntryWithPhonetic}

%%%%%%%%%% 迫 %%%%%%%%%%
\subsection*{迫}\addcontentsline{loh}{figure}{迫 \dpy{po4}}

\begin{EntryWithPhonetic}{迫}{po4}{8}{⾡}
  \definition{adj.}{urgente; premente}
  \definition{s.}{morteiro; artilharia}
  \definition{v.}{compelir; forçar; pressionar | aproximar"-se; ir em direção a (ou perto de)}
\end{EntryWithPhonetic}

\begin{EntryWithPhonetic}{迫不及待}{po4bu4ji2dai4}{8,4,3,9}{⾡,⼀,⼃,⼻}[HSK 7-9]
  \definition{expr.}{``Mal posso esperar!''; incapaz de se conter; ansioso para fazer algo sem demora; urgente demais para esperar mais}
\end{EntryWithPhonetic}

\begin{EntryWithPhonetic}{迫害}{po4hai4}{8,10}{⾡,⼧}[HSK 7-9]
  \definition{v.}{perseguir; oprimir cruelmente (frequentemente referindo"-se à opressão política)}
\end{EntryWithPhonetic}

\begin{EntryWithPhonetic}{迫切}{po4qie4}{8,4}{⾡,⼑}[HSK 4]
  \definition{adj.}{urgente; premente; muito ansiosamente, a ponto de ser difícil esperar}
\end{EntryWithPhonetic}

\begin{EntryWithPhonetic}{迫使}{po4shi3}{8,8}{⾡,⼈}[HSK 7-9]
  \definition{v.}{forçar; compelir; obrigar; impor; coagir (por meio de poder político ou econômico)}
\end{EntryWithPhonetic}

%%%%%%%%%% 破 %%%%%%%%%%
\subsection*{破}\addcontentsline{loh}{figure}{破 \dpy{po4}}

\begin{EntryWithPhonetic}{破}{po4}{10}{⽯}[HSK 3]
  \definition{adj.}{quebrado; danificado; rasgado; desgastado | insignificante; péssimo; medíocre}
  \definition{v.}{quebrar; danificar | dividir; cortar; separar | trocar (dinheiro) | livrar"-se de; destruir; romper com | derrotar; capturar (uma cidade, etc.) | gastar dinheiro | revelar a verdade sobre; expor | mudar; romper; quebrar (regras, hábitos, ideias, etc.)}
\end{EntryWithPhonetic}

\begin{EntryWithPhonetic}{破案}{po4'an4}{10,10}{⽯,⽊}[HSK 7-9]
  \definition{v.}{resolver um caso; elucidar um caso; desvendar um caso criminal; localizar o criminoso}
\end{EntryWithPhonetic}

\begin{EntryWithPhonetic}{破产}{po4/chan3}{10,6}{⽯,⼇}[HSK 4]
  \definition{v.+compl.}{falir; ir à falência; tornar"-se insolvente; entrar em liquidação; perder todo o patrimônio | falhar; fracassar; não dar em nada; figura de linguagem (geralmente com uma conotação depreciativa)}
\end{EntryWithPhonetic}

\begin{EntryWithPhonetic}{破除}{po4chu2}{10,9}{⽯,⾩}[HSK 7-9]
  \definition{v.}{livrar"-se de; erradicar; romper com; eliminar; abolir; remover (as coisas ruins que eram originalmente respeitadas ou acreditadas)}
\end{EntryWithPhonetic}

\begin{EntryWithPhonetic}{破坏}{po4huai4}{10,7}{⽯,⼟}[HSK 3]
  \definition{v.}{demolir; naufragar; soçobrar; destruir; obliterar | quebrar; violar (um acordo, regulamento, etc.); não cumprir (disposições legais, regras, acordos, princípios, etc.) | prejudicar; perturbar; sabotar; causar grande dano; causar danos às coisas | reverter; mudar (um sistema social, costume, etc.) completamente ou violentamente | destruir; decompor; danificar o tecido ou a estrutura de um objeto}
\end{EntryWithPhonetic}

\begin{EntryWithPhonetic}{破坏性}{po4huai4xing4}{10,7,8}{⽯,⼟,⼼}
  \definition{adj.}{destrutivo}
  \definition{s.}{poder destrutivo}
\end{EntryWithPhonetic}

\begin{EntryWithPhonetic}{破解}{po4jie3}{10,13}{⽯,⾓}[HSK 7-9]
  \definition{v.}{analisar e explicar (um texto antigo, escrita pré-histórica, etc.) | decodificar; revelar; desvendar; decifrar | resolver; solucionar; chegar a um acordo | afastar calamidades, azar, etc., por meio de magia; agir como um talismã contra | romper (um vínculo, restrição etc.) | crackear (software)}
\end{EntryWithPhonetic}

\begin{EntryWithPhonetic}{破旧}{po4jiu4}{10,5}{⽯,⽇}[HSK 7-9]
  \definition{adj.}{desgastado; dilapidado; velho e surrado; deteriorado}
\end{EntryWithPhonetic}

\begin{EntryWithPhonetic}{破裂}{po4lie4}{10,12}{⽯,⾐}[HSK 7-9]
  \definition{v.}{rachar; estalar; quebrar; romper; fraturar | quebrar; os sentimentos e os relacionamentos não conseguem mais continuar devido a conflitos e contradições}
\end{EntryWithPhonetic}

\begin{EntryWithPhonetic}{破灭}{po4mie4}{10,5}{⽯,⽕}[HSK 7-9]
  \definition{v.}{ser despedaçado; desmoronar; evaporar; ficar desiludido | ser destruído}
\end{EntryWithPhonetic}

\begin{EntryWithPhonetic}{破碎}{po4sui4}{10,13}{⽯,⽯}[HSK 7-9]
  \definition{adj.}{esfarrapado; quebrado; fragmentado}
  \definition{v.}{esmagar; despedaçar; quebrar algo em pedaços}
\end{EntryWithPhonetic}

%%%%%%%%%% 魄 %%%%%%%%%%
\subsection*{魄}\addcontentsline{loh}{figure}{魄 \dpy{po4}}

\begin{EntryWithPhonetic}{魄}{po4}{14}{⿁}
  \definition{s.}{alma | vigor; espírito; coragem; energia}
  \seeref{bo2}
  \seeref{tuo4}
\end{EntryWithPhonetic}

\begin{EntryWithPhonetic}{魄力}{po4li4}{14,2}{⿁,⼒}[HSK 7-9]
  \definition[种]{s.}{coragem; ousadia; audácia e resolução; refere"-se à coragem e à determinação com que alguém lida com as situações}
\end{EntryWithPhonetic}

%%%%%%%%%% 扑 %%%%%%%%%%
\subsection*{扑}\addcontentsline{loh}{figure}{扑 \dpy{pu1}}

\begin{EntryWithPhonetic}{扑}{pu1}{5}{⼿}[HSK 6]
  \definition{s.}{sopro; refere"-se a gases, fragrâncias, cinzas, areia, etc. que se apresentam | espanador}
  \definition{v.}{atacar; lançar"-se sobre; correr para frente com toda a sua força e, de repente, jogar todo o seu corpo em um objeto | dedicar; dedicar todas as energias a uma causa; colocar toda a sua energia em (trabalho, carreira, etc.) | bater asas; esvoaçar | inclinar"-se}
\end{EntryWithPhonetic}

\begin{EntryWithPhonetic}{扑克}{pu1ke4}{5,7}{⼿,⼗}[HSK 7-9]
  \definition[副,张]{s.}{Eempréstimo linguístico: \emph{poker}  | baralho; cartas de baralho}
\end{EntryWithPhonetic}

\begin{EntryWithPhonetic}{扑面而来}{pu1mian4-er2lai2}{5,9,6,7}{⼿,⾯,⽽,⽊}[HSK 7-9]
  \definition{expr.}{``Uma rajada de ar.''; vir de frente; bem na sua cara; está vindo direto na sua direção; indica uma situação em que uma ação, coisa, respiração ou palavra ocorre e chega diretamente}
\end{EntryWithPhonetic}

%%%%%%%%%% 铺 %%%%%%%%%%
\subsection*{铺}\addcontentsline{loh}{figure}{铺 \dpy{pu1}}

\begin{EntryWithPhonetic}{铺}{pu1}{12}{⾦}[HSK 6]
  \definition{clas.}{usado para kang, etc.; kang, uma plataforma de alvenaria ou de barro em uma extremidade de um cômodo, aquecida no inverno por fogueiras embaixo e coberta com esteiras para dormir}
  \definition{v.}{espalhar; estender; desdobrar | colocar; pavimentar}
  \seeref{pu4}
\end{EntryWithPhonetic}

\begin{EntryWithPhonetic}{铺垫}{pu1dian4}{12,9}{⾦,⼟}
  \definition{s.}{cobre leito | colcha | roupa de cama}
  \definition{v.}{pavimentar}
\end{EntryWithPhonetic}

\begin{EntryWithPhonetic}{铺路}{pu1/lu4}{12,13}{⾦,⾜}[HSK 7-9]
  \definition{v.+compl.}{pavimentar (uma estrada); construir uma estrada | Figurativo: preparar o terreno (para algo); criar condições para fazer algo; lançar as bases para | dar um presente a alguém para garantir o sucesso}
\end{EntryWithPhonetic}

%%%%%%%%%% 菩 %%%%%%%%%%
\subsection*{菩}\addcontentsline{loh}{figure}{菩 \dpy{pu2}}

\begin{EntryWithPhonetic}{菩}{pu2}{11}{⾋}
  \definition*[个]{s.}{Buda; Bodhisattva | Estátua de Bodhisattva}
\end{EntryWithPhonetic}

\begin{EntryWithPhonetic}{菩萨}{pu2sa4}{11,11}{⾋,⾋}[HSK 7-9]
  \definition*{s.}{Bodhisattva Guanyin; Buda; Deus}
  \definition[个]{s.}{divindade; deus | Figurativo: pessoa bondosa | ídolo budista}
\end{EntryWithPhonetic}

%%%%%%%%%% 葡 %%%%%%%%%%
\subsection*{葡}\addcontentsline{loh}{figure}{葡 \dpy{pu2}}

\begin{EntryWithPhonetic}{葡}{pu2}{12}{⾋}
  \definition*{s.}{Portugal, abreviação de 葡萄牙}
  \seealsoref{葡萄牙}{pu2tao2ya2}
\end{EntryWithPhonetic}

\begin{EntryWithPhonetic}{葡汉词典}{pu2-han4 ci2dian3}{12,5,7,8}{⾋,⽔,⾔,⼋}
  \definition{s.}{dicionário português-chinês}
  \seealsoref{汉葡词典}{han4-pu2 ci2dian3}
\end{EntryWithPhonetic}

\begin{EntryWithPhonetic}{葡萄酒}{pu2tao2jiu3}{12,11,10}{⾋,⾋,⾣}[HSK 5]
  \definition[瓶,杯,口,桶]{s.}{vinho (de uva)}
\end{EntryWithPhonetic}

\begin{EntryWithPhonetic}{葡萄牙}{pu2tao2ya2}{12,11,4}{⾋,⾋,⽛}
  \definition{s.}{Portugal}
\end{EntryWithPhonetic}

\begin{EntryWithPhonetic}{葡萄牙文}{pu2tao2ya2wen2}{12,11,4,4}{⾋,⾋,⽛,⽂}
  \definition{s.}{português, língua portuguesa}
  \seealsoref{葡文}{pu2wen2}
\end{EntryWithPhonetic}

\begin{EntryWithPhonetic}{葡萄牙语}{pu2tao2ya2yu3}{12,11,4,9}{⾋,⾋,⽛,⾔}
  \definition{s.}{português, língua portuguesa}
  \seealsoref{葡语}{pu2yu3}
\end{EntryWithPhonetic}

\begin{EntryWithPhonetic}{葡萄}{pu2tao5}{12,11}{⾋,⾋}[HSK 5]
  \definition[串,颗,粒,棵,种]{s.}{parreira | uva}
\end{EntryWithPhonetic}

\begin{EntryWithPhonetic}{葡文}{pu2wen2}{12,4}{⾋,⽂}
  \definition{s.}{português, língua portuguesa}
  \seealsoref{葡萄牙文}{pu2tao2ya2wen2}
\end{EntryWithPhonetic}

\begin{EntryWithPhonetic}{葡语}{pu2yu3}{12,9}{⾋,⾔}
  \definition{s.}{português, língua portuguesa}
  \seealsoref{葡萄牙语}{pu2tao2ya2yu3}
\end{EntryWithPhonetic}

%%%%%%%%%% 朴 %%%%%%%%%%
\subsection*{朴}\addcontentsline{loh}{figure}{朴 \dpy{pu3}}

\begin{EntryWithPhonetic}{朴}{pu3}{6}{⽊}
  \definition{adj.}{simples e direto; simples e honesto}
  \seeref{piao2}
  \seeref{po1}
  \seeref{po4}
\end{EntryWithPhonetic}

\begin{EntryWithPhonetic}{朴实}{pu3shi2}{6,8}{⽊,⼧}[HSK 7-9]
  \definition{adj.}{simples; descomplicado; despretensioso | sincero e honesto; ingênuo | realista; com os pés no chão}
\end{EntryWithPhonetic}

\begin{EntryWithPhonetic}{朴素}{pu3su4}{6,10}{⽊,⽷}[HSK 7-9]
  \definition{adj.}{(cor, estilo, linguagem, etc.) simples; sem graça; despretensioso; sem adornos; refere"-se à linguagem e às emoções que são sinceras e não exageradas | (estilo de vida) frugal; econômico; simples e modesto; não extravagante | ingênuo; subdesenvolvido; embrionário}
\end{EntryWithPhonetic}

%%%%%%%%%% 普 %%%%%%%%%%
\subsection*{普}\addcontentsline{loh}{figure}{普 \dpy{pu3}}

\begin{EntryWithPhonetic}{普}{pu3}{12}{⽇}
  \definition*{s.}{Sobrenome: Pu}
  \definition{adj.}{geral; universal}
\end{EntryWithPhonetic}

\begin{EntryWithPhonetic}{普遍}{pu3bian4}{12,12}{⽇,⾡}[HSK 3]
  \definition{adj.}{geral; comum; universal; difundido; a existência é muito ampla; tem semelhança}
\end{EntryWithPhonetic}

\begin{EntryWithPhonetic}{普及}{pu3ji2}{12,3}{⽇,⼃}[HSK 3]
  \definition{adj.}{popular; universal; onipresente; amplamente compreendido, aceito ou utilizado}
  \definition[种]{v.}{popularizar; disseminar; espalhar entre as pessoas; promover amplamente o conhecimento, a educação, a tecnologia, etc. para popularizá-los}
\end{EntryWithPhonetic}

\begin{EntryWithPhonetic}{普快}{pu3 kuai4}{12,7}{⽇,⼼}
  \definition*{s.}{Expresso Comum}
\end{EntryWithPhonetic}

\begin{EntryWithPhonetic}{普通}{pu3tong1}{12,10}{⽇,⾡}[HSK 2]
  \definition{adj.}{comum; normal; geral; médio; em geral, nada de especial, como a maioria das pessoas ou coisas}
\end{EntryWithPhonetic}

\begin{EntryWithPhonetic}{普通话}{pu3tong1hua4}{12,10,8}{⽇,⾡,⾔}[HSK 2]
  \definition*{s.}{Mandarim (``linguagem comum'') | Putonghua (fala comum da língua chinesa) | Língua oficial da China}
\end{EntryWithPhonetic}

\begin{EntryWithPhonetic}{普通人}{pu3tong1 ren2}{12,10,2}{⽇,⾡,⼈}[HSK 7-9]
  \definition{s.}{pessoa comum; cidadão comum}
\end{EntryWithPhonetic}

%%%%%%%%%% 谱 %%%%%%%%%%
\subsection*{谱}\addcontentsline{loh}{figure}{谱 \dpy{pu3}}

\begin{EntryWithPhonetic}{谱}{pu3}{14}{⾔}[HSK 7-9]
  \definition{s.}{registro ou documento de fácil consulta (na forma de gráficos, tabelas, listas, etc.); cronologia; um livro compilado para consulta, organizado de acordo com a categoria ou sistema do assunto e utilizando tabelas ou outros formatos claros | manual; guia; formatos ou diagramas que podem ser usados para orientar a prática | partitura; partitura musical; partitura para música}
  \definition{v.}{compor música | exibir"-se; mostrar"-se arrogante}
\end{EntryWithPhonetic}

%%%%%%%%%% 堡 %%%%%%%%%%
\subsection*{堡}\addcontentsline{loh}{figure}{堡 \dpy{pu4}}

\begin{EntryWithPhonetic}{堡}{pu4}{12}{⼟}
  \definition{s.}{cidade ou rua (frequentemente usado em nomes de lugares)}
  \seeref{bao3}
  \seeref{bu3}
\end{EntryWithPhonetic}

%%%%%%%%%% 铺 %%%%%%%%%%
\subsection*{铺}\addcontentsline{loh}{figure}{铺 \dpy{pu4}}

\begin{EntryWithPhonetic}{铺}{pu4}{12}{⾦}
  \definition{s.}{pequena loja; depósito | uma cama feita de tábuas de madeira; geralmente se refere a uma cama | estação de correios; antiga estação de correios (usada principalmente em nomes de lugares)}
  \seeref{pu1}
\end{EntryWithPhonetic}

%%%%%%%%%% 瀑 %%%%%%%%%%
\subsection*{瀑}\addcontentsline{loh}{figure}{瀑 \dpy{pu4}}

\begin{EntryWithPhonetic}{瀑}{pu4}{18}{⽔}
  \definition{s.}{cachoeira; catarata}
  \seeref{bao4}
\end{EntryWithPhonetic}

\begin{EntryWithPhonetic}{瀑布}{pu4bu4}{18,5}{⽔,⼱}[HSK 7-9]
  \definition[道,条]{s.}{queda de água; cachoeira; cascata; catarata}
\end{EntryWithPhonetic}

%%%%%%%%%% 曝 %%%%%%%%%%
\subsection*{曝}\addcontentsline{loh}{figure}{曝 \dpy{pu4}}

\begin{EntryWithPhonetic}{曝}{pu4}{19}{⽇}
  \definition{v.}{expor ao sol}
  \seeref{bao4}
\end{EntryWithPhonetic}

%%%%% EOF %%%%%


 %%%
%%% Q
%%%
\section*{Q}\addcontentsline{toc}{section}{Q}\addcontentsline{loh}{figure}{\#\#\#\#\#\#\#\# Q}

%%%%%%%%%% 七 %%%%%%%%%%
\subsection*{七}\addcontentsline{loh}{figure}{七 \dpy{qi1}}

\begin{EntryWithPhonetic}{七}{qi1}{2}{⼀}[HSK 1]
  \definition*{s.}{Sobrenome: Qi}
  \definition{num.}{sete; 7}
  \definition{s.}{antigamente, os mortos eram homenageados a cada sete dias, chamados de 七, até o quadragésimo nono dia, num total de sete 七}
\end{EntryWithPhonetic}

\begin{EntryWithPhonetic}{七夕}{qi1xi1}{2,3}{⼀,⼣}
  \definition*{s.}{Dia dos Namorados Chinês, quando o vaqueiro e a tecelã (牛郎织女) têm permissão para se reunirem anualmente | Festival das Meninas | Festival Duplo Sete, noite do sétimo mês lunar}
  \seealsoref{牛郎织女}{niu2 lang2 zhi1nv3}
\end{EntryWithPhonetic}

\begin{EntryWithPhonetic}{七嘴八舌}{qi1zui3-ba1she2}{2,16,2,6}{⼀,⼝,⼋,⾆}[HSK 7-9]
  \definition{expr.}{``Uma cacofonia de vozes.''; todos falando ao mesmo tempo; falando uns por cima dos outros; isso descreve uma situação em que muitas pessoas estão falando ao mesmo tempo, com opiniões conflitantes; também descreve alguém que é falante e fofoqueiro}
\end{EntryWithPhonetic}

%%%%%%%%%% 沏 %%%%%%%%%%
\subsection*{沏}\addcontentsline{loh}{figure}{沏 \dpy{qi1}}

\begin{EntryWithPhonetic}{沏}{qi1}{7}{⽔}[HSK 7-9]
  \definition{v.}{infundir (com água fervente); dissolver (em água fervente)}[妈妈喜欢沏一壶绿茶。===Minha mãe gosta de infundir um bule de chá verde.]
\end{EntryWithPhonetic}

%%%%%%%%%% 妻 %%%%%%%%%%
\subsection*{妻}\addcontentsline{loh}{figure}{妻 \dpy{qi1}}

\begin{EntryWithPhonetic}{妻}{qi1}{8}{⼥}
  \definition{s.}{esposa}
  \seeref{qi4}
\end{EntryWithPhonetic}

\begin{EntryWithPhonetic}{妻子}{qi1zi3}{8,3}{⼥,⼦}
  \definition[个]{s.}{esposa e filhos; (chinês antigo) refere-se a esposas, filhos e filhas}
  \seeref{qi1zi5}
\end{EntryWithPhonetic}

\begin{EntryWithPhonetic}{妻子}{qi1zi5}{8,3}{⼥,⼦}[HSK 4]
  \definition[个]{s.}{esposa (não é usado como um termo carinhoso)}
  \seeref{qi1zi3}
\end{EntryWithPhonetic}

%%%%%%%%%% 凄 %%%%%%%%%%
\subsection*{凄}\addcontentsline{loh}{figure}{凄 \dpy{qi1}}

\begin{EntryWithPhonetic}{凄}{qi1}{10}{⼎}
  \definition{adj.}{frio; gelado | sombrio e desolado | triste; miserável; infeliz}
\end{EntryWithPhonetic}

\begin{EntryWithPhonetic}{凄凉}{qi1liang2}{10,10}{⼎,⼎}[HSK 7-9]
  \definition{adj.}{sombrio; desolador; solitário e desolado; miserável (frequentemente usado para descrever um ambiente ou cena) | miserável; sombrio e desolado; desolado e trágico}[夜晚的街道显得很凄凉。===As ruas estavam sombrias à noite.|他的眼神看起来很凄凉。===Seus olhos pareciam muito desolados.]
\end{EntryWithPhonetic}

%%%%%%%%%% 期 %%%%%%%%%%
\subsection*{期}\addcontentsline{loh}{figure}{期 \dpy{qi1}}

\begin{EntryWithPhonetic}{期}{qi1}{12}{⽉}[HSK 3]
  \definition{clas.}{questão; número; termo; coisas usadas para parcelamento}
  \definition{s.}{um período de tempo; fase; estágio | horário agendado; data agendada | tempo designado (programado)}
  \definition{v.}{marcar uma consulta | esperar; aguardar | esperar; ter esperança}
\end{EntryWithPhonetic}

\begin{EntryWithPhonetic}{期待}{qi1dai4}{12,9}{⽉,⼻}[HSK 4]
  \definition{v.}{aguardar; esperar; aguardar ansiosamente; ter em mente a realização de um determinado fim ou a ocorrência de uma determinada situação}
\end{EntryWithPhonetic}

\begin{EntryWithPhonetic}{期间}{qi1jian1}{12,7}{⽉,⾨}[HSK 4]
  \definition{s.}{prazo; tempo; período}
\end{EntryWithPhonetic}

\begin{EntryWithPhonetic}{期末}{qi1 mo4}{12,5}{⽉,⽊}[HSK 4]
  \definition{s.}{terminal; final do prazo; fim do período}
\end{EntryWithPhonetic}

\begin{EntryWithPhonetic}{期盼}{qi1pan4}{12,9}{⽉,⽬}[HSK 7-9]
  \definition{v.}{esperar; aguardar}
\end{EntryWithPhonetic}

\begin{EntryWithPhonetic}{期望}{qi1wang4}{12,11}{⽉,⽉}[HSK 5]
  \definition{s.}{esperança; expectativa}
  \definition{v.}{esperar; ter esperança}
\end{EntryWithPhonetic}

\begin{EntryWithPhonetic}{期限}{qi1xian4}{12,8}{⽉,⾩}[HSK 4]
  \definition{s.}{prazo; limite de tempo; tempo alocado; período de tempo limitado, também o limite final do limite de tempo; \emph{deadline}}
\end{EntryWithPhonetic}

\begin{EntryWithPhonetic}{期中}{qi1 zhong1}{12,4}{⽉,⼁}[HSK 4]
  \definition{adj.}{provisório; interino; intermediário}
\end{EntryWithPhonetic}

%%%%%%%%%% 欺 %%%%%%%%%%
\subsection*{欺}\addcontentsline{loh}{figure}{欺 \dpy{qi1}}

\begin{EntryWithPhonetic}{欺}{qi1}{12}{⽋}
  \definition{v.}{enganar; trapacear | intimidar; tirar vantagem de alguém; tirar vantagem da fraqueza de (alguém, etc.)}
\end{EntryWithPhonetic}

\begin{EntryWithPhonetic}{欺负}{qi1fu5}{12,6}{⽋,⾙}[HSK 6]
  \definition{v.}{violar, oprimir ou insultar com meios irracionais; \emph{bully}}
\end{EntryWithPhonetic}

\begin{EntryWithPhonetic}{欺骗}{qi1pian4}{12,12}{⽋,⾺}[HSK 7-9]
  \definition{v.}{enganar; ludibriar; usar palavras ou ações falsas para encobrir a verdade e enganar as pessoas}
\end{EntryWithPhonetic}

\begin{EntryWithPhonetic}{欺诈}{qi1zha4}{12,7}{⽋,⾔}[HSK 7-9]
  \definition{v.}{trapacear; enganar; usar métodos astutos para enganar as pessoas para obter lucro}
\end{EntryWithPhonetic}

%%%%%%%%%% 漆 %%%%%%%%%%
\subsection*{漆}\addcontentsline{loh}{figure}{漆 \dpy{qi1}}

\begin{EntryWithPhonetic}{漆}{qi1}{14}{⽔}[HSK 7-9]
  \definition*{s.}{Sobrenome: Qi}
  \definition[桶,种,层]{s.}{laca; tinta}
  \definition{v.}{aplicar verniz; pintar}
\end{EntryWithPhonetic}

%%%%%%%%%% 齐 %%%%%%%%%%
\subsection*{齐}\addcontentsline{loh}{figure}{齐 \dpy{qi2}}

\begin{EntryWithPhonetic}{齐}{qi2}{6}{⿑}[HSK 3][Kangxi 210]
  \definition*{s.}{Qi, um estado da Dinastia Zhou | Dinastia Qi do Sul (479-502), uma das Dinastias do Sul | Dinastia Qi do Norte (550-577), uma das Dinastias do Norte | Sobrenome: Qi}
  \definition{adj.}{arrumado; uniforme; regular; comprimento, tamanho, etc. são praticamente iguais; uniformes | semelhante; similar; da mesma forma; de acordo| tudo pronto; todos presentes; completo; perfeito}
  \definition{adv.}{juntos; simultaneamente; ao mesmo tempo}
  \definition{v.}{estar no mesmo nível que; alcançar o mesmo nível | estar nivelado em um ponto ou ao longo de uma linha; tornar consistente; harmonizar}
\end{EntryWithPhonetic}

\begin{EntryWithPhonetic}{齐国}{qi2 guo2}{6,8}{⿑,⼞}
  \definition*{s.}{Estado Qi de Zhou Ocidental e os Estados Combatentes (1122-265 a.C.), centrado em Shandong}
\end{EntryWithPhonetic}

\begin{EntryWithPhonetic}{齐全}{qi2quan2}{6,6}{⿑,⼊}[HSK 5]
  \definition{adj.}{completo; tudo pronto}
\end{EntryWithPhonetic}

\begin{EntryWithPhonetic}{齐心协力}{qi2xin1-xie2li4}{6,4,6,2}{⿑,⼼,⼗,⼒}[HSK 7-9]
  \definition{expr.}{trabalhar em conjunto; fazer esforços concertados; reunir; trabalhar como um; trabalhar com um propósito comum; fazer esforços conjuntos}
\end{EntryWithPhonetic}

%%%%%%%%%% 其 %%%%%%%%%%
\subsection*{其}\addcontentsline{loh}{figure}{其 \dpy{qi2}}

\begin{EntryWithPhonetic}{其}{qi2}{8}{⼋}[HSK 5]
  \definition*{s.}{Sobrenome: Qi}
  \definition{adv.}{fazer uma suposição ou uma réplica | expressar comando, ordem}
  \definition{pron.}{dele (dela, deles, delas) | ele, ela, isso, eles; elas | isso; tal | isso (referindo-se a nenhuma pessoa ou coisa específica)}
  \definition{suf.}{sufixo de palavra, anexado ao advérbio}
\end{EntryWithPhonetic}

\begin{EntryWithPhonetic}{其次}{qi2ci4}{8,6}{⼋,⽋}[HSK 3]
  \definition{adj.}{secundário}
  \definition{conj.}{próximo; então; em segundo lugar; mais tarde na ordem}
\end{EntryWithPhonetic}

\begin{EntryWithPhonetic}{其后}{qi2hou4}{8,6}{⼋,⼝}[HSK 7-9]
  \definition{adv.}{mais tarde; depois; posteriormente | depois disso | próximo}
\end{EntryWithPhonetic}

\begin{EntryWithPhonetic}{其间}{qi2jian1}{8,7}{⼋,⾨}[HSK 7-9]
  \definition{s.}{nele; deles; entre eles; no meio | durante este (ou aquele) período; dentro de um determinado período de tempo}
\end{EntryWithPhonetic}

\begin{EntryWithPhonetic}{其实}{qi2shi2}{8,8}{⼋,⼧}[HSK 3]
  \definition{adv.}{na verdade; na realidade; a primeira parte é a situação aparente, e 其实 é usado para introduzir a situação real}
\end{EntryWithPhonetic}

\begin{EntryWithPhonetic}{其他}{qi2ta1}{8,5}{⼋,⼈}[HSK 2]
  \definition{pron.}{outra pessoa/outra coisa | outras coisas; outras pessoas; em substituição de outras pessoas ou coisas}
\end{EntryWithPhonetic}

\begin{EntryWithPhonetic}{其余}{qi2yu2}{8,7}{⼋,⼈}[HSK 4]
  \definition{pron.}{o resto; os outros; o restante}
\end{EntryWithPhonetic}

\begin{EntryWithPhonetic}{其中}{qi2zhong1}{8,4}{⼋,⼁}[HSK 2]
  \definition{pron.}{dentro; entre (os quais, eles, etc.); em (o qual, ele, etc.); nas pessoas ou coisas mencionadas anteriormente}
\end{EntryWithPhonetic}

%%%%%%%%%% 奇 %%%%%%%%%%
\subsection*{奇}\addcontentsline{loh}{figure}{奇 \dpy{qi2}}

\begin{EntryWithPhonetic}{奇}{qi2}{8}{⼤}
  \definition{adj.}{ímpar (número); singular; solteiro; não em pares (ao contrário de 偶)}
  \definition{s.}{lotes ímpares; quantidade fracionária (acima daquela mencionada em um número redondo)}
  \seealsoref{偶}{ou3}
\end{EntryWithPhonetic}

\begin{EntryWithPhonetic}{奇怪}{qi2guai4}{8,8}{⼤,⼼}[HSK 3]
  \definition{adj.}{estranho; diferente do habitual; raramente visto, até um pouco irracional | estranho; esquisito; a descrição é diferente do imaginado e é difícil de entender}
  \definition{v.}{ficar perplexo; maravilhar-se; sentir-se surpreso; sentir-se estranho; sentir-se incompreensível}
\end{EntryWithPhonetic}

\begin{EntryWithPhonetic}{奇花异草}{qi2hua1-yi4cao3}{8,7,6,9}{⼤,⾋,⼶,⾋}[HSK 7-9]
  \definition{expr.}{``Flores exóticas e ervas raras.''; vista espetacular do mundo botânico; muito raramente visto, incomum}
\end{EntryWithPhonetic}

\begin{EntryWithPhonetic}{奇迹}{qi2ji4}{8,9}{⼤,⾡}[HSK 7-9]
  \definition[个,种]{s.}{milagre; maravilha; coisas extraordinárias inimagináveis}
\end{EntryWithPhonetic}

\begin{EntryWithPhonetic}{奇妙}{qi2miao4}{8,7}{⼤,⼥}[HSK 6]
  \definition{adj.}{maravilhoso; milagroso; intrigante; muito inteligente e engenhoso (usado principalmente para descrever coisas interessantes e novas)}
\end{EntryWithPhonetic}

\begin{EntryWithPhonetic}{奇特}{qi2te4}{8,10}{⼤,⽜}[HSK 7-9]
  \definition{adj.}{estranho; peculiar; singular; incomum; extraordinário}
\end{EntryWithPhonetic}

%%%%%%%%%% 歧 %%%%%%%%%%
\subsection*{歧}\addcontentsline{loh}{figure}{歧 \dpy{qi2}}

\begin{EntryWithPhonetic}{歧}{qi2}{8}{⽌}
  \definition{adj.}{divergente; diferente; ambíguo; inconsistente}
  \definition{s.}{bifurcação; ramificação; bifurcação da estrada; uma estrada que se ramifica de uma estrada principal}
\end{EntryWithPhonetic}

\begin{EntryWithPhonetic}{歧视}{qi2shi4}{8,8}{⽌,⾒}[HSK 7-9]
  \definition[种,些,点]{s.}{discriminação}
  \definition{v.}{discriminar contra; discriminar; tratar alguém ou um grupo de forma desigual, com uma atitude injusta ou desproporcional}
\end{EntryWithPhonetic}

%%%%%%%%%% 祈 %%%%%%%%%%
\subsection*{祈}\addcontentsline{loh}{figure}{祈 \dpy{qi2}}

\begin{EntryWithPhonetic}{祈}{qi2}{8}{⽰}
  \definition{v.}{orar; rezar}
\end{EntryWithPhonetic}

\begin{EntryWithPhonetic}{祈祷}{qi2dao3}{8,11}{⽰,⽰}[HSK 7-9]
  \definition{v.}{orar; realizar um ritual religioso; expressar silenciosamente seus desejos a Deus}
\end{EntryWithPhonetic}

%%%%%%%%%% 骑 %%%%%%%%%%
\subsection*{骑}\addcontentsline{loh}{figure}{骑 \dpy{qi2}}

\begin{EntryWithPhonetic}{骑}{qi2}{11}{⾺}[HSK 2]
  \definition{s.}{cavalos ou outros animais para montaria | cavalaria; cavaleiro, também se refere genericamente a qualquer pessoa que monta a cavalo}
  \definition{v.}{montar (um animal ou bicicleta); sentar-se na parte de trás de | montar; abranger ambos os lados}
\end{EntryWithPhonetic}

\begin{EntryWithPhonetic}{骑车}{qi2 che1}{11,4}{⾺,⾞}[HSK 2]
  \definition{v.}{andar de bicicleta; pedalar}
\end{EntryWithPhonetic}

%%%%%%%%%% 棋 %%%%%%%%%%
\subsection*{棋}\addcontentsline{loh}{figure}{棋 \dpy{qi2}}

\begin{EntryWithPhonetic}{棋}{qi2}{12}{⽊}[HSK 7-9]
  \definition[盘]{s.}{xadrez | jogo semelhante ao xadrez | uma partida de xadrez}
  \definition[个,颗]{s.}{peça de xadrez}
\end{EntryWithPhonetic}

\begin{EntryWithPhonetic}{棋子儿}{qi2zi3r5}{12,3,2}{⽊,⼦,⼉}
  \definition{s.}{peça de xadrez}
\end{EntryWithPhonetic}

\begin{EntryWithPhonetic}{棋子}{qi2zi5}{12,3}{⽊,⼦}[HSK 7-9]
  \definition[个,颗]{s.}{peça (em um jogo de tabuleiro); peça de xadrez}
  \seealsoref{棋子儿}{qi2zi3r5}
\end{EntryWithPhonetic}

%%%%%%%%%% 旗 %%%%%%%%%%
\subsection*{旗}\addcontentsline{loh}{figure}{旗 \dpy{qi2}}

\begin{EntryWithPhonetic}{旗}{qi2}{14}{⽅}
  \definition[面]{s.}{bandeira}
\end{EntryWithPhonetic}

\begin{EntryWithPhonetic}{旗袍}{qi2pao2}{14,10}{⽅,⾐}[HSK 7-9]
  \definition[件,个]{s.}{qipao; cheongsam; uma túnica longa usada por mulheres, originalmente usada por mulheres manchus}
\end{EntryWithPhonetic}

\begin{EntryWithPhonetic}{旗帜}{qi2zhi4}{14,8}{⽅,⼱}[HSK 7-9]
  \definition[面]{s.}{bandeira; estandarte | modelo; bom exemplo; metáfora para modelo ou exemplo a seguir | bandeira (de um pensamento representativo ou posição política); essa metáfora se refere a uma ideologia, doutrina ou força política representativa ou influente}
\end{EntryWithPhonetic}

%%%%%%%%%% 乞 %%%%%%%%%%
\subsection*{乞}\addcontentsline{loh}{figure}{乞 \dpy{qi3}}

\begin{EntryWithPhonetic}{乞}{qi3}{3}{⼄}
  \definition*{s.}{Sobrenome: Qi}
  \definition{v.}{implorar (por esmolas, etc.); suplicar}
\end{EntryWithPhonetic}

\begin{EntryWithPhonetic}{乞丐}{qi3gai4}{3,4}{⼄,⼀}[HSK 7-9]
  \definition[个,位,群]{s.}{mendigo; pessoas que não têm meios de subsistência e dependem exclusivamente da mendicância para conseguir comida e dinheiro para sobreviver}
\end{EntryWithPhonetic}

\begin{EntryWithPhonetic}{乞求}{qi3qiu2}{3,7}{⼄,⽔}[HSK 7-9]
  \definition{v.}{implorar; suplicar; mendigar | cair de joelhos}
\end{EntryWithPhonetic}

\begin{EntryWithPhonetic}{乞讨}{qi3tao3}{3,5}{⼄,⾔}[HSK 7-9]
  \definition{v.}{implorar; pedir dinheiro, pedir comida, etc.}
\end{EntryWithPhonetic}

%%%%%%%%%% 企 %%%%%%%%%%
\subsection*{企}\addcontentsline{loh}{figure}{企 \dpy{qi3}}

\begin{EntryWithPhonetic}{企}{qi3}{6}{⼈}
  \definition{v.}{ficar na ponta dos pés | esperar ansiosamente por algo; ansiar por | planejar um projeto}
\end{EntryWithPhonetic}

\begin{EntryWithPhonetic}{企图}{qi3tu2}{6,8}{⼈,⼞}[HSK 6]
  \definition[种]{s.}{plano; tentativa; intenção (principalmente negativa)}
  \definition{v.}{procurar; tentar; pretender}
\end{EntryWithPhonetic}

\begin{EntryWithPhonetic}{企业}{qi3ye4}{6,5}{⼈,⼀}[HSK 4]
  \definition[家,个]{s.}{empresa; estabelecimento; empreendimento; negócio; setores envolvidos em atividades econômicas como produção, transporte, comércio, etc., como fábricas, minas, ferrovias, empresas comerciais, etc.}
\end{EntryWithPhonetic}

%%%%%%%%%% 岂 %%%%%%%%%%
\subsection*{岂}\addcontentsline{loh}{figure}{岂 \dpy{qi3}}

\begin{EntryWithPhonetic}{岂}{qi3}{6}{⼭}
  \definition*{s.}{Sobrenome: Qi}
  \definition{adv.}{Litarário: expressa uma pergunta retórica, equivalente a 哪里, 怎么 e 难道}
  \seealsoref{哪里}{na3 li3}
  \seealsoref{难道}{nan2dao4}
  \seealsoref{怎么}{zen3me5}
\end{EntryWithPhonetic}

\begin{EntryWithPhonetic}{岂有此理}{qi3you3ci3li3}{6,6,6,11}{⼭,⽉,⽌,⽟}[HSK 7-9]
  \definition{expr./interj.}{Isso é um absurdo!; Que exorbitante!; Absurdo!; Como isso pode ser assim?; Ridículo!}
\end{EntryWithPhonetic}

%%%%%%%%%% 启 %%%%%%%%%%
\subsection*{启}\addcontentsline{loh}{figure}{启 \dpy{qi3}}

\begin{EntryWithPhonetic}{启}{qi3}{7}{⼝}
  \definition*{s.}{Sobrenome: Qi}
  \definition{s.}{nota; carta; um dos antigos estilos literários, uma carta relativamente curta}
  \definition{v.}{abrir | despertar; iluminar | começar; iniciar | declarar; informar}
\end{EntryWithPhonetic}

\begin{EntryWithPhonetic}{启程}{qi3cheng2}{7,12}{⼝,⽲}
  \definition{v.}{partir; iniciar uma viagem; começar sua longa jornada}
\end{EntryWithPhonetic}

\begin{EntryWithPhonetic}{启迪}{qi3di2}{7,8}{⼝,⾡}[HSK 7-9]
  \definition{v.}{inspirar; iluminar; incentivar;}
\end{EntryWithPhonetic}

\begin{EntryWithPhonetic}{启动}{qi3 dong4}{7,6}{⼝,⼒}[HSK 5]
  \definition{v.}{ligar (uma máquina); acionar; ligar máquinas, equipamentos elétricos, etc., para começar a trabalhar | entrar em vigor; começar a vigorar e a ser implementados planos, projetos, documentos jurídicos, etc.}
\end{EntryWithPhonetic}

\begin{EntryWithPhonetic}{启发}{qi3fa1}{7,5}{⼝,⼜}[HSK 5]
  \definition{s.}{iluminação; esclarecimento; fenômenos e princípios que levam as pessoas a refletir e a abrir suas mentes}
  \definition{v.}{despertar; inspirar; esclarecer; orientar, fazer com que compreendam}
\end{EntryWithPhonetic}

\begin{EntryWithPhonetic}{启蒙}{qi3meng2}{7,13}{⼝,⾋}[HSK 7-9]
  \definition{v.}{iniciar; transmitir conhecimento rudimentar a iniciantes; ensinar conhecimentos ou habilidades básicas a iniciantes | esclarecer; libertar alguém de preconceitos ou superstições; por meio da publicidade e da educação, a sociedade pode aceitar coisas novas e se livrar da ignorância e do atraso}
\end{EntryWithPhonetic}

\begin{EntryWithPhonetic}{启示}{qi3shi4}{7,5}{⼝,⽰}[HSK 7-9]
  \definition[个,条,种]{s.}{revelação; inspiração; iluminação; orientação inspiradora leva à compreensão; dicas e sugestões para ajudar você a entender}
  \definition{v.}{revelar; inspirar}
\end{EntryWithPhonetic}

\begin{EntryWithPhonetic}{启事}{qi3shi4}{7,8}{⼝,⼅}[HSK 5]
  \definition[个,则,份,张,条]{s.}{aviso; anúncio; texto publicado em jornais ou afixado em paredes com o objetivo de divulgar publicamente algo}
\end{EntryWithPhonetic}

%%%%%%%%%% 起 %%%%%%%%%%
\subsection*{起}\addcontentsline{loh}{figure}{起 \dpy{qi3}}

\begin{EntryWithPhonetic}{起}{qi3}{10}{⾛}[HSK 1]
  \definition{clas.}{caso; instância | lote; grupo}
  \definition{prep.}{de; colocado antes de uma palavra de tempo ou lugar, indica um ponto de partida | por; colocado antes de uma palavra de lugar, indica um lugar por onde passou}
  \definition{v.}{levantar-se; ficar de pé| iniciar; lançar; deicar a posição original | subir; ascender | aparecer; levantar; crescer (bolhas, protuberâncias, brotoeja) | puxar para cima; puxar para fora; tirar o que está guardado ou incorporado | crescer; aumentar | esboçar; elaborar | construir; montar; estabelecer | receber (comprovante) | começar; iniciar; combina com 从 e 由; indica quando, onde e quem começou | buscar; pegar; usado após um verbo, indica movimento para cima | indicar se alguém tem força suficiente ou não; usado após um verbo, indica que a força é suficiente ou insuficiente | indicar que a ação envolve alguém ou algo; equivalente a 及 ou 到 | começar; iniciar; usado depois de um verbo, indica o início de uma ação | juntar; implodir; (informal) usado depois de um verbo, para unir coisas ou fechá-las}
  \seealsoref{从}{cong2}
  \seealsoref{到}{dao4}
  \seealsoref{及}{ji2}
  \seealsoref{由}{you2}
\end{EntryWithPhonetic}

\begin{EntryWithPhonetic}{起步}{qi3bu4}{10,7}{⾛,⽌}[HSK 7-9]
  \definition{v.}{começar; mover-se; começar a dar os primeiros passos | começar; dar início; metaforicamente, significa o início de (uma carreira, emprego, etc.)}
\end{EntryWithPhonetic}

\begin{EntryWithPhonetic}{起草}{qi3/cao3}{10,9}{⾛,⾋}[HSK 7-9]
  \definition{v.+compl.}{elaborar; redigir; preparar um rascunho}
\end{EntryWithPhonetic}

\begin{EntryWithPhonetic}{起程}{qi3cheng2}{10,12}{⾛,⽲}[HSK 7-9]
  \definition{v.}{partir; sair; iniciar uma viagem}
  \seealsoref{启程}{qi3cheng2}
\end{EntryWithPhonetic}

\begin{EntryWithPhonetic}{起初}{qi3chu1}{10,7}{⾛,⾐}[HSK 7-9]
  \definition{adv.}{princípio; início; inicial (frequentemente contrastado com 后来 ou 现在)}
  \seealsoref{后来}{hou4lai2}
  \seealsoref{现在}{xian4zai4}
\end{EntryWithPhonetic}

\begin{EntryWithPhonetic}{起床}{qi3/chuang2}{10,7}{⾛,⼴}[HSK 1]
  \definition{v.+compl.}{levantar-se; sair da cama; acordar e sair da cama (geralmente pela manhã); levantar-se da posição sentada, deitada ou deitada de bruços, ou sentar-se a partir da posição deitada}
\end{EntryWithPhonetic}

\begin{EntryWithPhonetic}{起到}{qi3 dao4}{10,8}{⾛,⼑}[HSK 5]
  \definition{v.}{ter (um efeito motivador, etc.); desempenhar (um papel estabilizador, etc.)}
\end{EntryWithPhonetic}

\begin{EntryWithPhonetic}{起点}{qi3 dian3}{10,9}{⾛,⽕}[HSK 6]
  \definition[个]{s.}{ponto de partida (para o tempo ou local do início de algo); o lugar ou hora de início | ponto de partida (para o nível ou base de algo feito inicialmente); refere-se especificamente ao ponto de partida designado em um evento de pista}
\end{EntryWithPhonetic}

\begin{EntryWithPhonetic}{起飞}{qi3fei1}{10,3}{⾛,⾶}[HSK 2]
  \definition{v.}{decolar; levantar voo | crescer rapidamente; decolar; disparar; metáfora para o rápido desenvolvimento de negócios, economia, etc.}
\end{EntryWithPhonetic}

\begin{EntryWithPhonetic}{起伏}{qi3fu2}{10,6}{⾛,⼈}[HSK 7-9]
  \definition{v.}{subir e descer; ondular;  subida e descida contínuas | (emoções, relações, etc.) flutuar; oscilar; aumentar e diminuir; essa metáfora descreve a oscilação de emoções, relacionamentos, etc.}
\end{EntryWithPhonetic}

\begin{EntryWithPhonetic}{起劲}{qi3jin4}{10,7}{⾛,⼒}[HSK 7-9]
  \definition{adj.}{vigoroso; animado; entusiasmado; enérgico; descreve um alto nível de entusiasmo por fazer algo}
  \seealsoref{起劲儿}{qi3jin4r5}
\end{EntryWithPhonetic}

\begin{EntryWithPhonetic}{起劲儿}{qi3jin4r5}{10,7,2}{⾛,⼒,⼉}
  \definition{v.}{ser enérgico}
\end{EntryWithPhonetic}

\begin{EntryWithPhonetic}{起来}{qi3/lai2}{10,7}{⾛,⽊}[HSK 1]
  \definition{v.+compl.}{levantar-se; passar de posições como deitado, sentado ou ajoelhado para ficar em pé | levantar-se; sair da cama | levantar-se; revoltar-se; rebelar-se; refere-se a ascensão, surgimento, levantamento, etc.}
  \seeref{qi3lai5}
  \seeref{qi5lai2}
\end{EntryWithPhonetic}

\begin{EntryWithPhonetic}{起来}{qi3lai5}{10,7}{⾛,⽊}
  \definition{v.aux.}{usado depois de um verbo para indicar movimento ascendente}
  \seeref{qi3/lai2}
  \seeref{qi5lai2}
\end{EntryWithPhonetic}

\begin{EntryWithPhonetic}{起码}{qi3ma3}{10,8}{⾛,⽯}[HSK 5]
  \definition{adj.}{mínimo; elementar; rudimentar}
  \definition{adv.}{mínimamente; pelo menos;}
\end{EntryWithPhonetic}

\begin{EntryWithPhonetic}{起跑线}{qi3pao3xian4}{10,12,8}{⾛,⾜,⽷}[HSK 7-9]
  \definition{s.}{linha de partida (em uma corrida de revezamento); a linha de largada (de uma corrida); a linha de partida em provas de atletismo; metaforicamente, o ponto de partida para o trabalho, estudo, etc.}
\end{EntryWithPhonetic}

\begin{EntryWithPhonetic}{起诉}{qi3 su4}{10,7}{⾛,⾔}[HSK 6]
  \definition{v.}{processar; entrar com uma ação judicial}
\end{EntryWithPhonetic}

\begin{EntryWithPhonetic}{起跳}{qi3tiao4}{10,13}{⾛,⾜}
  \definition{v.}{(atletismo) decolar (no início de um salto) | (de preço, salário, etc.) começar (de um determinado nível)}
\end{EntryWithPhonetic}

\begin{EntryWithPhonetic}{起源}{qi3yuan2}{10,13}{⾛,⽔}[HSK 7-9]
  \definition{s.}{a origem; a origem mais remota das coisas}
  \definition{v.}{originar-se; ter origem em; começar a acontecer; começar a aparecer}
\end{EntryWithPhonetic}

%%%%%%%%%% 气 %%%%%%%%%%
\subsection*{气}\addcontentsline{loh}{figure}{气 \dpy{qi4}}

\begin{EntryWithPhonetic}{气}{qi4}{4}{⽓}[HSK 2][Kangxi 84]
  \definition*{s.}{Sobrenome: Qi}
  \definition[口]{s.}{gás; gás em geral | ar; especificamente, o ar | respiração | clima; refere-se a fenômenos naturais como sol, chuva, frio e calor | cheiro; odor; o cheiro que o nariz sente | ânimo; moral; estado mental | ares; estilo; maneiras; refere-se ao estilo e aos hábitos de uma pessoa | raiva; irritação; aborrecimento; sentimento de irritação | energia vital; energia da vida; na medicina tradicional chinesa refere-se às substâncias sutis que circulam no corpo humano e permitem que os vários órgãos funcionem normalmente | certos sintomas (de doenças); na medicina tradicional chinesa refere-se a um determinado quadro clínico}
  \definition{v.}{ficar com raiva; ficar furioso; ficar irritado | irritar; enfurecer; deixar com raiva | ser intimidado; sofrer injustiça; intimidar}
\end{EntryWithPhonetic}

\begin{EntryWithPhonetic}{气氛}{qi4fen1}{4,8}{⽓,⽓}[HSK 6]
  \definition{s.}{atmosfera; sensação circundante; uma certa emoção ou cena que existe em um determinado ambiente e pode fazer as pessoas sentirem}
\end{EntryWithPhonetic}

\begin{EntryWithPhonetic}{气愤}{qi4fen4}{4,12}{⽓,⼼}[HSK 7-9]
  \definition{adj.}{furioso; indignado; irritado e ressentido}
\end{EntryWithPhonetic}

\begin{EntryWithPhonetic}{气管}{qi4guan3}{4,14}{⽓,⽵}[HSK 7-9]
  \definition[个,条,根]{s.}{traqueia; tubo respiratório}
\end{EntryWithPhonetic}

\begin{EntryWithPhonetic}{气候}{qi4hou4}{4,10}{⽓,⼈}[HSK 3]
  \definition[种]{s.}{clima; tempo; condições meteorológicas gerais obtidas após muitos anos de observação em uma determinada região, estão relacionadas com correntes de ar, latitude, altitude acima do nível do mar, relevo, etc. | tendência; situação; metáfora do ambiente social, de uma determinada tendência | resultado; influência; conquista; realização; metáfora para algum tipo de resultado, conquista, influência significativa ou potencial de desenvolvimento}
\end{EntryWithPhonetic}

\begin{EntryWithPhonetic}{气馁}{qi4nei3}{4,10}{⽓,⾷}[HSK 7-9]
  \definition{adj.}{desencorajado}
\end{EntryWithPhonetic}

\begin{EntryWithPhonetic}{气派}{qi4pai4}{4,9}{⽓,⽔}[HSK 7-9]
  \definition{adj.}{de maneira marcante; de ​​aparência impressionante; presunçoso; arrogante}
  \definition{s.}{ar digno; maneira imponente; refere-se à atitude de uma pessoa, ao ímpeto ou à aura emanada por certas coisas}
\end{EntryWithPhonetic}

\begin{EntryWithPhonetic}{气泡}{qi4pao4}{4,8}{⽓,⽔}[HSK 7-9]
  \definition{s.}{bolha; bolha de ar; um corpo esférico ou hemisférico formado por um gás que ocupa um determinado espaço dentro ou abaixo da superfície de um sólido ou líquido}
\end{EntryWithPhonetic}

\begin{EntryWithPhonetic}{气魄}{qi4po4}{4,14}{⽓,⿁}[HSK 7-9]
  \definition{s.}{ousadia de visão; amplitude de espírito; audácia; coragem para fazer as coisas | maneira imponente; impulso}
\end{EntryWithPhonetic}

\begin{EntryWithPhonetic}{气球}{qi4qiu2}{4,11}{⽓,⽟}[HSK 4]
  \definition[个,只]{s.}{balão; bolas feitas de borracha, plástico, etc., que podem ser aumentadas soprando ar nelas e podem ser usadas como brinquedos, decorações ou meios de transporte}
\end{EntryWithPhonetic}

\begin{EntryWithPhonetic}{气势}{qi4shi4}{4,8}{⽓,⼒}[HSK 7-9]
  \definition{s.}{ímpeto; maneira imponente; um certo poder e posição exibidos por (uma pessoa ou coisa)}
\end{EntryWithPhonetic}

\begin{EntryWithPhonetic}{气体}{qi4 ti3}{4,7}{⽓,⼈}[HSK 5]
  \definition[种,瓶,升]{s.}{gás; não têm forma nem volume definidos e podem fluir.; o ar, o oxigênio, o gás metano e outros são gases}
\end{EntryWithPhonetic}

\begin{EntryWithPhonetic}{气味}{qi4wei4}{4,8}{⽓,⼝}[HSK 7-9]
  \definition[种,股]{s.}{cheiro; o cheiro que impregna o ar | caráter e hábito (geralmente ruins); metaforicamente, refere-se ao temperamento; gosto (geralmente usado em sentido pejorativo)}
\end{EntryWithPhonetic}

\begin{EntryWithPhonetic}{气温}{qi4 wen1}{4,12}{⽓,⽔}[HSK 2]
  \definition[个]{s.}{temperatura do ar}
\end{EntryWithPhonetic}

\begin{EntryWithPhonetic}{气息}{qi4xi1}{4,10}{⽓,⼼}[HSK 7-9]
  \definition[种,丝,缕,股]{s.}{respiração; o ar que entra e sai quando respiramos | cheiro; estilo; metaforicamente, refere-se a uma forte sensação ou característica que algo transmite}
\end{EntryWithPhonetic}

\begin{EntryWithPhonetic}{气象}{qi4xiang4}{4,11}{⽓,⾗}[HSK 5]
  \definition[种,派]{s.}{fenômenos meteorológicos; condições e fenômenos atmosféricos, como vento, relâmpagos, trovões, geadas, neve, etc. | meteorologia | situação; atmosfera; cena; circunstância | maneira imponente}
\end{EntryWithPhonetic}

\begin{EntryWithPhonetic}{气质}{qi4zhi4}{4,8}{⽓,⾙}[HSK 7-9]
  \definition{s.}{temperamento; disposição; traços de personalidade | qualidades; componentes; refere-se aos traços de personalidade e estilo relativamente estáveis ​​de uma pessoa}
\end{EntryWithPhonetic}

%%%%%%%%%% 迄 %%%%%%%%%%
\subsection*{迄}\addcontentsline{loh}{figure}{迄 \dpy{qi4}}

\begin{EntryWithPhonetic}{迄}{qi4}{6}{⾡}
  \definition{adv.}{até agora; ao longo de todo o processo; sempre; constantemente; usado antes de 未 ou 无}
  \definition{prep.}{até; para}
  \seealsoref{未}{wei4}
  \seealsoref{无}{wu2}
\end{EntryWithPhonetic}

\begin{EntryWithPhonetic}{迄今}{qi4jin1}{6,4}{⾡,⼈}[HSK 7-9]
  \definition{adv.}{até agora; até o momento; até hoje}
\end{EntryWithPhonetic}

\begin{EntryWithPhonetic}{迄今为止}{qi4jin1-wei2zhi3}{6,4,4,4}{⾡,⼈,⼂,⽌}[HSK 7-9]
  \definition{adv.}{até hoje; até agora; até o momento}
\end{EntryWithPhonetic}

%%%%%%%%%% 汽 %%%%%%%%%%
\subsection*{汽}\addcontentsline{loh}{figure}{汽 \dpy{qi4}}

\begin{EntryWithPhonetic}{汽}{qi4}{7}{⽔}
  \definition{s.}{vapor | vaporizador}
\end{EntryWithPhonetic}

\begin{EntryWithPhonetic}{汽车}{qi4 che1}{7,4}{⽔,⾞}[HSK 1]
  \definition[辆,种,款]{s.}{automóvel; carro; veículo motorizado; veículo movido a motor de combustão interna, que circula principalmente em rodovias ou ruas, geralmente com quatro ou mais pneus de borracha, usado para transportar pessoas ou mercadorias}
\end{EntryWithPhonetic}

\begin{EntryWithPhonetic}{汽水}{qi4 shui3}{7,4}{⽔,⽔}[HSK 4]
  \definition[罐,杯,瓶,听,口]{s.}{refrigerante; refrigerante gaseificado; bebida refrescante, feita com a pressão de dióxido de carbono para dissolver na água e adicionar açúcar, suco de frutas, especiarias etc.}
\end{EntryWithPhonetic}

\begin{EntryWithPhonetic}{汽油}{qi4you2}{7,8}{⽔,⽔}[HSK 4]
  \definition[桶,升,吨]{s.}{gasolina; mistura líquida de hidrocarbonetos com volatilidade e combustibilidade, que é usada como combustível a partir do fracionamento ou craqueamento do petróleo}
\end{EntryWithPhonetic}

%%%%%%%%%% 妻 %%%%%%%%%%
\subsection*{妻}\addcontentsline{loh}{figure}{妻 \dpy{qi4}}

\begin{EntryWithPhonetic}{妻}{qi4}{8}{⼥}
  \definition{v.}{casar uma mulher com (alguém)}
  \seeref{qi1}
\end{EntryWithPhonetic}

%%%%%%%%%% 契 %%%%%%%%%%
\subsection*{契}\addcontentsline{loh}{figure}{契 \dpy{qi4}}

\begin{EntryWithPhonetic}{契}{qi4}{9}{⼤}
  \definition{s.}{contrato; escritura | Arcaico: personagens esculpidos}
  \definition{v.}{Literário: gravar; esculpir | concordar; dar-se bem}
  \seeref{xie4}
\end{EntryWithPhonetic}

\begin{EntryWithPhonetic}{契机}{qi4ji1}{9,6}{⼤,⽊}[HSK 7-9]
  \definition{s.}{oportunidade; ponto de virada; a chave para mudar as coisas numa direção favorável}
\end{EntryWithPhonetic}

\begin{EntryWithPhonetic}{契约}{qi4yue1}{9,6}{⼤,⽷}[HSK 7-9]
  \definition{s.}{escritura; carta; contrato; documentos comprovativos de vendas, hipotecas, arrendamentos, etc.}
\end{EntryWithPhonetic}

%%%%%%%%%% 器 %%%%%%%%%%
\subsection*{器}\addcontentsline{loh}{figure}{器 \dpy{qi4}}

\begin{EntryWithPhonetic}{器}{qi4}{16}{⼝}
  \definition[台]{s.}{dispositivo | ferramenta | utensílio}
\end{EntryWithPhonetic}

\begin{EntryWithPhonetic}{器材}{qi4cai2}{16,7}{⼝,⽊}[HSK 7-9]
  \definition[种,批,套,件]{s.}{material; aparelho; equipamento; ferramentas e materiais}
\end{EntryWithPhonetic}

\begin{EntryWithPhonetic}{器官}{qi4guan1}{16,8}{⼝,⼧}[HSK 4]
  \definition[个,种]{s.}{órgão; aparelho; parte de um organismo que consiste em vários tipos de tecidos celulares que podem desempenhar uma função fisiológica separada}
\end{EntryWithPhonetic}

\begin{EntryWithPhonetic}{器械}{qi4xie4}{16,11}{⼝,⽊}[HSK 7-9]
  \definition[种,批,些]{s.}{aparelho; dispositivo; instrumento; equipamento; instrumentos com finalidades especiais ou com construção relativamente precisa | arma; armamento; instrumentos e dispositivos usados ​​diretamente para matar pessoal inimigo e destruir instalações de combate inimigas, como facas, armas de fogo, artilharia e mísseis}
\end{EntryWithPhonetic}

%%%%%%%%%% 起 %%%%%%%%%%
\subsection*{起}\addcontentsline{loh}{figure}{起 \dpy{qi5}}

\begin{EntryWithPhonetic}{起来}{qi5lai2}{10,7}{⾛,⽊}
  \definition{v.}{descrever resultados, retratar comportamentos, transmitir movimento}
  \seeref{qi3/lai2}
  \seeref{qi3lai5}
\end{EntryWithPhonetic}

%%%%%%%%%% 掐 %%%%%%%%%%
\subsection*{掐}\addcontentsline{loh}{figure}{掐 \dpy{qia1}}

\begin{EntryWithPhonetic}{掐}{qia1}{11}{⼿}[HSK 7-9]
  \definition{s.}{Dialeto: um punhado, maço, pitada, etc. de}
  \definition{v.}{beliscar; dar uma mordidinha | agarrar}
  \seealsoref{掐儿}{qia1r5}
\end{EntryWithPhonetic}

\begin{EntryWithPhonetic}{掐儿}{qia1r5}{11,2}{⼿,⼉}
  \definition{s.}{Dialeto: um punhado, maço, pitada, etc. de}
\end{EntryWithPhonetic}

%%%%%%%%%% 卡 %%%%%%%%%%
\subsection*{卡}\addcontentsline{loh}{figure}{卡 \dpy{qia3}}

\begin{EntryWithPhonetic}{卡}{qia3}{5}{⼘}[HSK 7-9]
  \definition*{s.}{Sobrenome: Qia}
  \definition[张,片]{s.}{clipe; prendedor; pinça; utensílio para prender objetos | posto de controle; posto de guarda ou posto de controle localizado em vias de comunicação importantes ou em locais com terreno acidentado}
  \definition{v.}{encravar; ficar preso; impedir de se mover | parar; controlar; impedir | pressionar firmemente com a palma da mão}
  \seeref{ka3}
\end{EntryWithPhonetic}

\begin{EntryWithPhonetic}{卡子}{qia3zi5}{5,3}{⼘,⼦}[HSK 7-9]
  \definition[个,种,把]{s.}{presilha; grampo de cabelo; prendedor; ferramenta de fixação | posto de controle; postos de controle ou áreas de fiscalização estabelecidos para fins de arrecadação de impostos ou segurança}
\end{EntryWithPhonetic}

%%%%%%%%%% 恰 %%%%%%%%%%
\subsection*{恰}\addcontentsline{loh}{figure}{恰 \dpy{qia4}}

\begin{EntryWithPhonetic}{恰}{qia4}{9}{⼼}
  \definition{adv.}{exatamente | apenas}
\end{EntryWithPhonetic}

\begin{EntryWithPhonetic}{恰当}{qia4dang4}{9,6}{⼼,⼹}[HSK 6]
  \definition{adj.}{adequado; apropriado; conveniente; apropriado; a linguagem ou abordagem é muito apropriada}
\end{EntryWithPhonetic}

\begin{EntryWithPhonetic}{恰到好处}{qia4dao4-hao3chu4}{9,8,6,5}{⼼,⼑,⼥,⼡}[HSK 7-9]
  \definition{expr.}{``Na medida certa.''; perfeito (para o propósito ou ocasião); significa que as palavras e ações de alguém atingiram o ponto mais apropriado}
\end{EntryWithPhonetic}

\begin{EntryWithPhonetic}{恰好}{qia4 hao3}{9,6}{⼼,⼥}[HSK 6]
  \definition{adv.}{na medida certa; como a sorte quis}
\end{EntryWithPhonetic}

\begin{EntryWithPhonetic}{恰恰}{qia4 qia4}{9,9}{⼼,⼼}[HSK 6]
  \definition{adv.}{justamente; exatamente; precisamente; bem na hora}
\end{EntryWithPhonetic}

\begin{EntryWithPhonetic}{恰恰相反}{qia4qia4 xiang1fan3}{9,9,9,4}{⼼,⼼,⽬,⼜}[HSK 7-9]
  \definition{expr.}{pelo contrário}
\end{EntryWithPhonetic}

\begin{EntryWithPhonetic}{恰巧}{qia4qiao3}{9,5}{⼼,⼯}[HSK 7-9]
  \definition{adv.}{por acaso; felizmente ou infelizmente; perfeitamente; por coincidência}
\end{EntryWithPhonetic}

\begin{EntryWithPhonetic}{恰如其分}{qia4ru2-qi2fen4}{9,6,8,4}{⼼,⼥,⼋,⼑}[HSK 7-9]
  \definition{expr.}{``Na medida certa.''; apropriado; agir ou falar de maneira diplomática}
\end{EntryWithPhonetic}

%%%%%%%%%% 洽 %%%%%%%%%%
\subsection*{洽}\addcontentsline{loh}{figure}{洽 \dpy{qia4}}

\begin{EntryWithPhonetic}{洽}{qia4}{9}{⽔}
  \definition{adj.}{em harmonia; em acordo | extenso; amplo}
  \definition{v.}{consultar; combinar com}
\end{EntryWithPhonetic}

\begin{EntryWithPhonetic}{洽谈}{qia4tan2}{9,10}{⽔,⾔}[HSK 7-9]
  \definition{v.}{negociar; negociação e consulta geralmente se referem às conversas ou discussões realizadas em atividades comerciais relacionadas a negócios, transações de mercadorias e compra e venda}
\end{EntryWithPhonetic}

%%%%%%%%%% 千 %%%%%%%%%%
\subsection*{千}\addcontentsline{loh}{figure}{千 \dpy{qian1}}

\begin{EntryWithPhonetic}{千}{qian1}{3}{⼗}[HSK 2]
  \definition*{s.}{Sobrenome: Qian}
  \definition{num.}{mil; 1.000; 1000 | a grande quantidade de; um grande número de}
\end{EntryWithPhonetic}

\begin{EntryWithPhonetic}{千变万化}{qian1bian4-wan4hua4}{3,8,3,4}{⼗,⼜,⼀,⼔}[HSK 7-9]
  \definition{expr.}{``Sempre em mudança.''; as miríades de mudanças; mudança caleidoscópica; mudanças intermináveis; em constante transformação; ser infinito em variedade; mudanças infinitas; em constante mudança}
\end{EntryWithPhonetic}

\begin{EntryWithPhonetic}{千方百计}{qian1fang1-bai3ji4}{3,4,6,4}{⼗,⽅,⽩,⾔}[HSK 7-9]
  \definition{expr.}{por todos os meios; fazer tudo o que for possível; descreve alguém que esgotou todos os meios ou métodos}
\end{EntryWithPhonetic}

\begin{EntryWithPhonetic}{千古}{qian1gu3}{3,5}{⼗,⼝}
  \definition{adv.}{por toda a eternidade | em todas as idades}
  \definition{s.}{eternidade (usada em um dístico elegíaco, coroa de flores, etc., dedicada aos mortos)}
\end{EntryWithPhonetic}

\begin{EntryWithPhonetic}{千家万户}{qian1jia1-wan4hu4}{3,10,3,4}{⼗,⼧,⼀,⼾}[HSK 7-9]
  \definition{expr.}{``Milhares de famílias.''; inúmeras famílias; todas as famílias}
\end{EntryWithPhonetic}

\begin{EntryWithPhonetic}{千军万马}{qian1jun1-wan4ma3}{3,6,3,3}{⼗,⼍,⼀,⾺}[HSK 7-9]
  \definition{expr.}{``Milhares de soldados.''; milhares e milhares de homens e cavalos; um exército poderoso; uma força imensa; todos os cavalos do rei e todos os homens do rei; exército magnífico com milhares de homens e cavalos; demonstração impressionante de força humana}
\end{EntryWithPhonetic}

\begin{EntryWithPhonetic}{千钧一发}{qian1jun1-yi1fa4}{3,9,1,5}{⼗,⾦,⼀,⼜}[HSK 7-9]
  \definition{expr.}{``Por pouco não deu certo.''; cem pesos pendurados por um fio; em perigo iminente; uma questão de vida ou morte}
\end{EntryWithPhonetic}

\begin{EntryWithPhonetic}{千克}{qian1 ke4}{3,7}{⼗,⼗}[HSK 2]
  \definition{clas.}{kg; quilo; quilograma; 1 quilograma equivale a 1.000 gramas, ou 2 jin (斤)}
  \seealsoref{斤}{jin1}
\end{EntryWithPhonetic}

\begin{EntryWithPhonetic}{千年}{qian1nian2}{3,6}{⼗,⼲}
  \definition{s.}{milênio}
\end{EntryWithPhonetic}

\begin{EntryWithPhonetic}{千千万万}{qian1qian1wan4wan4}{3,3,3,3}{⼗,⼗,⼀,⼀}
  \definition{num.}{inumerável | números incontáveis | milhares e milhares}
\end{EntryWithPhonetic}

\begin{EntryWithPhonetic}{千万}{qian1wan4}{3,3}{⼗,⼀}[HSK 3]
  \definition{adv.}{(usado para indicar desejos fortes) por todos os meios; sob quaisquer circunstâncias; expressa uma exortação sincera, equivalente a 务必}
  \definition{num.}{dez milhões; 10.000.000; 1000.0000; milhões e milhões; um número aproximado, indicando um grande número}
  \seealsoref{务必}{wu4bi4}
\end{EntryWithPhonetic}

%%%%%%%%%% 迁 %%%%%%%%%%
\subsection*{迁}\addcontentsline{loh}{figure}{迁 \dpy{qian1}}

\begin{EntryWithPhonetic}{迁}{qian1}{6}{⾡}[HSK 7-9]
  \definition{v.}{mover algo para algum lugar; migrar | mudar}
\end{EntryWithPhonetic}

\begin{EntryWithPhonetic}{迁就}{qian1jiu4}{6,12}{⾡,⼪}[HSK 7-9]
  \definition{v.}{ceder a; acomodar-se a; atender aos interesses dos outros}
\end{EntryWithPhonetic}

\begin{EntryWithPhonetic}{迁移}{qian1yi2}{6,11}{⾡,⽲}[HSK 7-9]
  \definition{v.}{mover; migrar; mudar-se de sua localização original para outro lugar}
\end{EntryWithPhonetic}

%%%%%%%%%% 牵 %%%%%%%%%%
\subsection*{牵}\addcontentsline{loh}{figure}{牵 \dpy{qian1}}

\begin{EntryWithPhonetic}{牵}{qian1}{9}{⽜}[HSK 6]
  \definition{v.}{conduzir (segurando a mão, o cabresto, etc.); puxar | envolver-se | sentir falta; preocupar-se com | controlar; restringir; ser retido; ser constrangido}
\end{EntryWithPhonetic}

\begin{EntryWithPhonetic}{牵扯}{qian1che3}{9,7}{⽜,⼿}[HSK 7-9]
  \definition{v.}{envolver; arrastar para}
\end{EntryWithPhonetic}

\begin{EntryWithPhonetic}{牵挂}{qian1gua4}{9,9}{⽜,⼿}[HSK 7-9]
  \definition{v.}{preocupar-se; estar preocupado; perder}
\end{EntryWithPhonetic}

\begin{EntryWithPhonetic}{牵涉}{qian1she4}{9,10}{⽜,⽔}[HSK 7-9]
  \definition{v.}{preocupar-se com; envolver; arrastar para; uma coisa está relacionada a outras coisas ou pessoas}
\end{EntryWithPhonetic}

\begin{EntryWithPhonetic}{牵头}{qian1/tou2}{9,5}{⽜,⼤}[HSK 7-9]
  \definition{v.+compl.}{intermediar (por exemplo: casamenteiro) | coordenar (uma operação combinada) | conduzir (um animal pela cabeça) | mediar | assumir a liderança}
\end{EntryWithPhonetic}

\begin{EntryWithPhonetic}{牵制}{qian1zhi4}{9,8}{⽜,⼑}[HSK 7-9]
  \definition{v.}{conter; imobilizar; amarrar; restringir ou impedir a livre circulação (frequentemente usado em contextos militares)}
\end{EntryWithPhonetic}

%%%%%%%%%% 铅 %%%%%%%%%%
\subsection*{铅}\addcontentsline{loh}{figure}{铅 \dpy{qian1}}

\begin{EntryWithPhonetic}{铅}{qian1}{10}{⾦}[HSK 7-9]
  \definition[根,盒]{s.}{chumbo (Pb) | grafite (em um lápis); grafite preta}
\end{EntryWithPhonetic}

\begin{EntryWithPhonetic}{铅笔}{qian1bi3}{10,10}{⾦,⽵}[HSK 6]
  \definition[支,盒,种,枝,杆]{s.}{lápis; canetas com pontas de grafite ou argila pigmentada}
\end{EntryWithPhonetic}

%%%%%%%%%% 谦 %%%%%%%%%%
\subsection*{谦}\addcontentsline{loh}{figure}{谦 \dpy{qian1}}

\begin{EntryWithPhonetic}{谦}{qian1}{12}{⾔}
  \definition*{s.}{Sobrenome: Qian}
  \definition{adj.}{modesto}
  \definition{s.}{modéstia}
\end{EntryWithPhonetic}

\begin{EntryWithPhonetic}{谦虚}{qian1xu1}{12,11}{⾔,⾌}[HSK 6]
  \definition{adj.}{modesto; não se orgulhe de suas próprias conquistas e esteja disposto a aceitar críticas e opiniões de outras pessoas}
  \definition{v.}{falar modestamente; quando recebo elogios e cumprimentos de outras pessoas, sinto que não sou tão bom}
\end{EntryWithPhonetic}

\begin{EntryWithPhonetic}{谦逊}{qian1xun4}{12,9}{⾔,⾡}[HSK 7-9]
  \definition{adj.}{humilde; modesto; despretensioso; sem afetação}
\end{EntryWithPhonetic}

%%%%%%%%%% 签 %%%%%%%%%%
\subsection*{签}\addcontentsline{loh}{figure}{签 \dpy{qian1}}

\begin{EntryWithPhonetic}{签}{qian1}{13}{⽵}[HSK 5,7-9]
  \definition[个,根,支]{s.}{tiras de bambu usadas para adivinhação ou sorteio; pPequenas tiras de bambu ou varas finas com caracteres e símbolos gravados, usadas para adivinhação, jogos de azar ou como fichas para contagem, etc. | etiqueta; adesivo; pequena tira usada como marca | um pedaço fino e pontiagudo de bambu ou madeira; pequeno bastão pontiagudo}
  \definition{v.}{assinar; autografar; escrever o nome, palavras ou fazer marcas em documentos ou recibos | fazer comentários breves em um documento; escrever brevemente (pontos principais ou opiniões) | (em costura) alinhavar; costura grosseira}
\end{EntryWithPhonetic}

\begin{EntryWithPhonetic}{签订}{qian1 ding4}{13,4}{⽵,⾔}[HSK 5]
  \definition{v.}{concluir e assinar (um tratado, etc.)}
\end{EntryWithPhonetic}

\begin{EntryWithPhonetic}{签名}{qian1/ming2}{13,6}{⽵,⼝}[HSK 5]
  \definition[个,次]{s.}{assinatura; autógrafo}
  \definition{v.+compl.}{assinar o próprio nome; autografar; escrever seu nome para indicar concordância, apoio ou homenagem, etc.}
\end{EntryWithPhonetic}

\begin{EntryWithPhonetic}{签署}{qian1shu3}{13,13}{⽵,⽹}[HSK 7-9]
  \definition{v.}{assinar; assinar formalmente e apor o próprio nome em documentos e tratados importantes}
\end{EntryWithPhonetic}

\begin{EntryWithPhonetic}{签约}{qian1 yue1}{13,6}{⽵,⽷}[HSK 5]
  \definition{v.}{assinar um contrato; assinar contratos e tratados, frequentemente utilizado no trabalho e em cooperações comerciais}
\end{EntryWithPhonetic}

\begin{EntryWithPhonetic}{签证}{qian1zheng4}{13,7}{⽵,⾔}[HSK 5]
  \definition[张,个,份]{s.}{visto; visto de entrada em um país}
\end{EntryWithPhonetic}

\begin{EntryWithPhonetic}{签字}{qian1 zi4}{13,6}{⽵,⼦}[HSK 5]
  \definition{v.}{assinar; colocar a assinatura; escrever seu nome à mão em documentos, recibos, etc., para demonstrar responsabilidade}
\end{EntryWithPhonetic}

%%%%%%%%%% 前 %%%%%%%%%%
\subsection*{前}\addcontentsline{loh}{figure}{前 \dpy{qian2}}

\begin{EntryWithPhonetic}{前}{qian2}{9}{⼑}[HSK 1]
  \definition*{s.}{Sobrenome: Qian}
  \definition{s.}{frente | futuro; perspectiva | atrás; antes; mais cedo do que uma coisa ou um momento | à frente; para a frente; na parte frontal (referindo-se ao espaço, em oposição a 后) | precedente; antes que algo aconteça | antigo; antigamente | topo; primeiro; primeiro na ordem | frente; campo de batalha | A.C. (Antes de~Cristo)}[前293年===293 a.C.]
  \definition{v.}{seguir em frente; ir em frente}
  \seealsoref{公元}{gong1yuan2}
  \seealsoref{后}{hou4}
\end{EntryWithPhonetic}

\begin{EntryWithPhonetic}{前辈}{qian2bei4}{9,12}{⼑,⾞}[HSK 7-9]
  \definition{s.}{idoso; sênior; geração mais velha; refere-se a pessoas mais velhas ou com mais experiência na mesma indústria ou área de atuação | predecessor; antepassados; ancestrais; a geração anterior ou gerações anteriores, também se referindo aos ancestrais}
\end{EntryWithPhonetic}

\begin{EntryWithPhonetic}{前边}{qian2 bian5}{9,5}{⼑,⾡}[HSK 1]
  \definition{adv.}{à frente; na frente}
\end{EntryWithPhonetic}

\begin{EntryWithPhonetic}{前不久}{qian2bu4jiu3}{9,4,3}{⼑,⼀,⼃}[HSK 7-9]
  \definition{adv.}{não faz muito tempo | não muito tempo antes}
\end{EntryWithPhonetic}

\begin{EntryWithPhonetic}{前方}{qian2 fang1}{9,4}{⼑,⽅}[HSK 6]
  \definition{s.}{frente; o espaço à frente; a direção voltada para a frente; a frente (em oposição à 后方) | linha de frente; frente de batalha; áreas onde os exércitos de ambos os lados estão se aproximando ou lutando}
  \seealsoref{后方}{hou4 fang1}
\end{EntryWithPhonetic}

\begin{EntryWithPhonetic}{前赴后继}{qian2fu4-hou4ji4}{9,9,6,10}{⼑,⾛,⼝,⽷}[HSK 7-9]
  \definition{expr.}{``Um após o outro.''; avançar destemidamente em onda após onda; os que estão na frente sobem, e os que estão atrás os seguem, o que demonstra um espírito de progresso entusiasmado e contínuo}
\end{EntryWithPhonetic}

\begin{EntryWithPhonetic}{前后}{qian2 hou4}{9,6}{⼑,⼝}[HSK 3]
  \definition{s.}{em volta; sobre; um período de tempo ligeiramente anterior ou posterior a um horário específico| do início ao fim; refere-se ao período de tempo do início ao fim de algo | frente e verso; na frente e atrás de algo}
\end{EntryWithPhonetic}

\begin{EntryWithPhonetic}{前进}{qian2 jin4}{9,7}{⼑,⾡}[HSK 3]
  \definition{v.}{marchar; avançar; para ir em frente; seguir em frente; geralmente se refere ao desenvolvimento futuro}
\end{EntryWithPhonetic}

\begin{EntryWithPhonetic}{前景}{qian2jing3}{9,12}{⼑,⽇}[HSK 5]
  \definition{s.}{primeiro plano (de uma vista, imagem, foto, etc.); as imagens que parecem mais próximas do espectador em pinturas, palcos e telas | vista; perspectiva; prospecto; ponto de vista; situações que podem ocorrer no trabalho, na carreira, etc.}
\end{EntryWithPhonetic}

\begin{EntryWithPhonetic}{前来}{qian2 lai2}{9,7}{⼑,⽊}[HSK 6]
  \definition{v.}{vir; em direção à localização e direção do falante}
\end{EntryWithPhonetic}

\begin{EntryWithPhonetic}{前面}{qian2mian4}{9,9}{⼑,⾯}[HSK 3]
  \definition{s.}{frente; a parte frontal do espaço ou posição | parte anterior; acima; a parte que vem primeiro na ordem; a parte de um artigo ou discurso que precede a narração atual}
\end{EntryWithPhonetic}

\begin{EntryWithPhonetic}{前年}{qian2 nian2}{9,6}{⼑,⼲}[HSK 2]
  \definition{adv.}{há dois anos; dois anos atrás}
\end{EntryWithPhonetic}

\begin{EntryWithPhonetic}{前期}{qian2qi1}{9,12}{⼑,⽉}[HSK 7-9]
  \definition{s.}{estágio inicial; primeiros dias; prófase; a etapa anterior de um determinado período}
\end{EntryWithPhonetic}

\begin{EntryWithPhonetic}{前任}{qian2ren4}{9,6}{⼑,⼈}[HSK 7-9]
  \definition[个,位,名]{s.}{predecessor; a pessoa que ocupava este cargo anteriormente}
\end{EntryWithPhonetic}

\begin{EntryWithPhonetic}{前所未有}{qian2suo3wei4you3}{9,8,5,6}{⼑,⼾,⽊,⽉}[HSK 7-9]
  \definition{expr.}{``Sem precedentes.''; nunca existiu antes; até então desconhecido; nunca antes na história}
\end{EntryWithPhonetic}

\begin{EntryWithPhonetic}{前台}{qian2tai2}{9,5}{⼑,⼝}[HSK 7-9]
  \definition[个]{s.}{proscênio; trabalho diverso para uma apresentação; refere-se a diversas tarefas administrativas relacionadas ao desempenho | palco; frente do palco; a parte da frente do palco, voltada para a plateia, é onde os atores se apresentam; geralmente, ela é separada da área dos bastidores por cortinas ou outras barreiras | (em um hotel) balcão de recepção; balcões de atendimento em restaurantes, casas noturnas, hotéis, etc., responsáveis ​​pela recepção, cadastro e pagamento | lugar público; referindo-se metaforicamente a uma ocasião pública (frequentemente usado em sentido pejorativo)}
\end{EntryWithPhonetic}

\begin{EntryWithPhonetic}{前提}{qian2ti2}{9,12}{⼑,⼿}[HSK 5]
  \definition[个,项]{s.}{premissa; pressuposto | pré-requisito; pressuposição; condições prévias para que algo aconteça ou se desenvolva}
\end{EntryWithPhonetic}

\begin{EntryWithPhonetic}{前天}{qian2 tian1}{9,4}{⼑,⼤}[HSK 1]
  \definition{adv.}{anteontem; dia anterior a ontem}
\end{EntryWithPhonetic}

\begin{EntryWithPhonetic}{前头}{qian2 tou5}{9,5}{⼑,⼤}[HSK 4]
  \definition{s.}{à frente; na frente; adiante}
\end{EntryWithPhonetic}

\begin{EntryWithPhonetic}{前途}{qian2tu2}{9,10}{⼑,⾡}[HSK 4]
  \definition[片,段,种]{s.}{futuro; perspectiva; prospecto; originalmente, refere-se à jornada à frente, mas, metaforicamente, refere-se ao futuro.}
\end{EntryWithPhonetic}

\begin{EntryWithPhonetic}{前往}{qian2 wang3}{9,8}{⼑,⼻}[HSK 3]
  \definition{v.}{ir para; prosseguir para; partir para; ir em frente}
\end{EntryWithPhonetic}

\begin{EntryWithPhonetic}{前无古人}{qian2wu2gu3ren2}{9,4,5,2}{⼑,⽆,⼝,⼈}[HSK 7-9]
  \definition[位,名,个,些]{expr.}{``Sem precedentes.''; refere-se a algo que nunca foi possuído ou alcançado antes; algo sem precedentes; sem paralelo na história}
\end{EntryWithPhonetic}

\begin{EntryWithPhonetic}{前夕}{qian2xi1}{9,3}{⼑,⼣}[HSK 7-9]
  \definition{s.}{véspera; na noite anterior | a véspera; geralmente se refere ao período de tempo imediatamente anterior à ocorrência de um evento ou ao momento em que um evento está prestes a ocorrer}
\end{EntryWithPhonetic}

\begin{EntryWithPhonetic}{前线}{qian2xian4}{9,8}{⼑,⽷}[HSK 7-9]
  \definition{s.}{linha de frente; frente (oposto à 后方) | frente de batalha; a área onde os dois exércitos se aproximam durante uma batalha (em oposição à 后方)}
  \seealsoref{后方}{hou4 fang1}
\end{EntryWithPhonetic}

\begin{EntryWithPhonetic}{前沿}{qian2yan2}{9,8}{⼑,⽔}[HSK 7-9]
  \definition{s.}{Militar: posição avançada | Figurativo: fronteira na pesquisa científica | fronteira (da ciência, tecnologia etc.) | posto avançado}
\end{EntryWithPhonetic}

\begin{EntryWithPhonetic}{前仰后合}{qian2yang3-hou4he2}{9,6,6,6}{⼑,⼈,⼝,⼝}[HSK 7-9]
  \definition{expr.}{``Inclinando-se para a frente e para trás.''; balançar (com risos); balançar para frente e para trás}
\end{EntryWithPhonetic}

\begin{EntryWithPhonetic}{前者}{qian2zhe3}{9,8}{⼑,⽼}[HSK 7-9]
  \definition{adj.}{antigo; anterior; primeiro; precedente; refere-se à primeira das duas coisas ou pessoas listadas acima (distinguindo-se de 后者)}
  \seealsoref{后者}{hou4zhe3}
\end{EntryWithPhonetic}

%%%%%%%%%% 虔 %%%%%%%%%%
\subsection*{虔}\addcontentsline{loh}{figure}{虔 \dpy{qian2}}

\begin{EntryWithPhonetic}{虔}{qian2}{10}{⾌}
  \definition*{s.}{Sobrenome: Qian}
  \definition{adj.}{piedoso; sincero}
\end{EntryWithPhonetic}

\begin{EntryWithPhonetic}{虔诚}{qian2cheng2}{10,8}{⾌,⾔}[HSK 7-9]
  \definition{adj.}{piedoso; devoto; devotado; respeitoso e sincero (frequentemente referindo-se à religião ou à fé)}
\end{EntryWithPhonetic}

%%%%%%%%%% 钱 %%%%%%%%%%
\subsection*{钱}\addcontentsline{loh}{figure}{钱 \dpy{qian2}}

\begin{EntryWithPhonetic}{钱}{qian2}{10}{⾦}[HSK 1]
  \definition*{s.}{Sobrenome: Qian}
  \definition{clas.}{qian, uma unidade de peso (=5 gramas) | qian, uma unidade de peso (um décimo de um tael 两)}
  \definition[笔]{s.}{dinheiro; riqueza; bens | moeda de cobre; dinheiro | objeto em forma de moeda de cobre | fundo; montante | dinheiro guardado ou gasto para algum fim específico (geralmente se refere a quantias significativas de dinheiro que entram e saem de órgãos públicos, organizações, etc.)}
  \seealsoref{两}{liang3}
\end{EntryWithPhonetic}

\begin{EntryWithPhonetic}{钱包}{qian2 bao1}{10,5}{⾦,⼓}[HSK 1]
  \definition[个]{s.}{carteira; bolsa; bolsa de dinheiro}
\end{EntryWithPhonetic}

\begin{EntryWithPhonetic}{钱财}{qian2cai2}{10,7}{⾦,⾙}[HSK 7-9]
  \definition{s.}{riqueza; dinheiro}
\end{EntryWithPhonetic}

%%%%%%%%%% 钳 %%%%%%%%%%
\subsection*{钳}\addcontentsline{loh}{figure}{钳 \dpy{qian2}}

\begin{EntryWithPhonetic}{钳}{qian2}{10}{⾦}
  \definition{s.}{pinças; alicates; tenazes}
  \definition{v.}{agarrar (com pinças); prender | restringir; limitar}
\end{EntryWithPhonetic}

\begin{EntryWithPhonetic}{钳子}{qian2zi5}{10,3}{⾦,⼦}[HSK 7-9]
  \definition[把,个]{s.}{alicate; tenaz; ferramentas usadas para prender ou cortar coisas}
\end{EntryWithPhonetic}

%%%%%%%%%% 潜 %%%%%%%%%%
\subsection*{潜}\addcontentsline{loh}{figure}{潜 \dpy{qian2}}

\begin{EntryWithPhonetic}{潜}{qian2}{15}{⽔}
  \definition*{s.}{Sobrenome: Qian}
  \definition{adj.}{latente; oculto}
  \definition{adv.}{furtivamente; secretamente; às escondidas}
  \definition{v.}{ir para debaixo d'água; esconder-se debaixo d'água; mergulhar | esconder | vadear (atravessar) na água | enterrar | fugir de casa}
\end{EntryWithPhonetic}

\begin{EntryWithPhonetic}{潜力}{qian2li4}{15,2}{⽔,⼒}[HSK 6]
  \definition{s.}{potencial; potencialidade; capacidade latente; as habilidades e possibilidades de desenvolvimento que as pessoas e as coisas ainda não demonstraram}
\end{EntryWithPhonetic}

\begin{EntryWithPhonetic}{潜能}{qian2neng2}{15,10}{⽔,⾁}[HSK 7-9]
  \definition{s.}{proficiência; potencial de habilidades ou energia inexploradas}
\end{EntryWithPhonetic}

\begin{EntryWithPhonetic}{潜水}{qian2/shui3}{15,4}{⽔,⽔}[HSK 7-9]
  \definition{s.}{água freática; lençol freático; água subterrânea escondida na primeira camada impermeável abaixo do solo}
  \definition{v.+compl.}{mergulhar; ir para debaixo d'água; entrar abaixo da superfície da água}
\end{EntryWithPhonetic}

\begin{EntryWithPhonetic}{潜艇}{qian2ting3}{15,12}{⽔,⾈}[HSK 7-9]
  \definition{s.}{submarino; navios de guerra que se dedicam principalmente a operações de combate subaquáticas, usando torpedos ou mísseis para atacar navios inimigos e alvos costeiros, e que também servem como embarcações de reconhecimento}
\end{EntryWithPhonetic}

\begin{EntryWithPhonetic}{潜移默化}{qian2yi2-mo4hua4}{15,11,16,4}{⽔,⽲,⿊,⼔}[HSK 7-9]
  \definition{expr.}{``Influência sutil.''; de maneiras sutis; influência gradual e imperceptível; influência imperceptível; influência lenta e despercebida; isso se refere a uma mudança nos pensamentos ou no caráter de uma pessoa causada por influência ou influência inconsciente; exercer uma influência sutil no caráter, pensamento, etc. de alguém; influenciar imperceptivelmente}
\end{EntryWithPhonetic}

\begin{EntryWithPhonetic}{潜在}{qian2zai4}{15,6}{⽔,⼟}[HSK 7-9]
  \definition{adj.}{latente; oculto; potencial; ela existe dentro das coisas e não é facilmente descoberta ou detectada}[这个问题有潜在风险。===Essa questão apresenta riscos potenciais.]
\end{EntryWithPhonetic}

%%%%%%%%%% 浅 %%%%%%%%%%
\subsection*{浅}\addcontentsline{loh}{figure}{浅 \dpy{qian3}}

\begin{EntryWithPhonetic}{浅}{qian3}{8}{⽔}[HSK 4]
  \definition{adj.}{raso; superficial;  (em oposição a 深) | fácil; simples; redação, conteúdo, etc. simples e fáceis de entender | superficial; não é profundo em aprendizado, percepção e sabedoria | não próximo; não íntimo; sentimentos não profundos | (cor) claro; pálido;  cor pouco intensa; leve |experiência breve; duração de tempo breve | baixo grau; peso leve; nível baixo}
  \seeref{jian1}
  \seealsoref{深}{shen1}
\end{EntryWithPhonetic}

%%%%%%%%%% 谴 %%%%%%%%%%
\subsection*{谴}\addcontentsline{loh}{figure}{谴 \dpy{qian3}}

\begin{EntryWithPhonetic}{谴}{qian3}{15}{⾔}
  \definition*{s.}{Sobrenome: Qian}
  \definition{s.}{falha | Literário: pecado}
  \definition{v.}{condenar; denunciar; censurar | rebaixar; antigamente, os funcionários eram rebaixados ou exilados}
\end{EntryWithPhonetic}

\begin{EntryWithPhonetic}{谴责}{qian3ze2}{15,8}{⾔,⾙}[HSK 7-9]
  \definition{v.}{culpar; condenar; censurar; denunciar}
\end{EntryWithPhonetic}

%%%%%%%%%% 欠 %%%%%%%%%%
\subsection*{欠}\addcontentsline{loh}{figure}{欠 \dpy{qian4}}

\begin{EntryWithPhonetic}{欠}{qian4}{4}{⽋}[HSK 5][Kangxi 76]
  \definition{v.}{bocejar | levantar ligeiramente (uma parte do corpo) | estar em dívida; estar atrasado com; não devolver o que pediu emprestado a outra pessoa, ou não dar o que deveria ter dado a outra pessoa | faltar; não ser suficiente}
\end{EntryWithPhonetic}

\begin{EntryWithPhonetic}{欠缺}{qian4que1}{4,10}{⽋,⽸}[HSK 7-9]
  \definition{s.}{deficiência; falha}
  \definition{v.}{ter falta de; ser deficiente/carente/inadequado em}
\end{EntryWithPhonetic}

\begin{EntryWithPhonetic}{欠条}{qian4tiao2}{4,7}{⽋,⽊}[HSK 7-9]
  \definition{s.}{uma fatura assinada como reconhecimento de dívida; recibo de dívida | fichas; recibos | certificado de dívida}
\end{EntryWithPhonetic}

%%%%%%%%%% 歉 %%%%%%%%%%
\subsection*{歉}\addcontentsline{loh}{figure}{歉 \dpy{qian4}}

\begin{EntryWithPhonetic}{歉}{qian4}{14}{⽋}
  \definition{adj.}{pobre, ruim (colheita) ruim; baixa produtividade (agrícola)}
  \definition{s.}{pedido de desculpas; apologia | quebra de safra}
  \definition{v.}{pedir desculpas; sentir pena dos outros}
\end{EntryWithPhonetic}

\begin{EntryWithPhonetic}{歉意}{qian4yi4}{14,13}{⽋,⼼}[HSK 7-9]
  \definition{s.}{desculpas; arrependimento; pedido de desculpas}
\end{EntryWithPhonetic}

%%%%%%%%%% 呛 %%%%%%%%%%
\subsection*{呛}\addcontentsline{loh}{figure}{呛 \dpy{qiang1}}

\begin{EntryWithPhonetic}{呛}{qiang1}{7}{⼝}[HSK 7-9]
  \definition{v.}{sufocar; engasgar}
  \seeref{qiang4}
\end{EntryWithPhonetic}

%%%%%%%%%% 抢 %%%%%%%%%%
\subsection*{抢}\addcontentsline{loh}{figure}{抢 \dpy{qiang1}}

\begin{EntryWithPhonetic}{抢}{qiang1}{7}{⼿}
  \definition{prep.}{contra; direção relativa inversa}
  \definition{v.}{bater; tocar}
  \seeref{qiang3}
\end{EntryWithPhonetic}

%%%%%%%%%% 枪 %%%%%%%%%%
\subsection*{枪}\addcontentsline{loh}{figure}{枪 \dpy{qiang1}}

\begin{EntryWithPhonetic}{枪}{qiang1}{8}{⽊}[HSK 5]
  \definition*{s.}{Sobrenome: Qiang}
  \definition[把,杆,支,挺]{s.}{lança | arma; rifle; arma de fogo | uma coisa em forma de arma | enxada; ferramenta para cavar a terra}
  \definition{v.}{escrever artigos ou responder perguntas para outras pessoas}
\end{EntryWithPhonetic}

\begin{EntryWithPhonetic}{枪毙}{qiang1bi4}{8,10}{⽊,⽐}[HSK 7-9]
  \definition{v.}{executar por disparo; matar | Humor figurativo: rejeitar; vetar; recusar (uma proposta, manuscrito, etc.) | matar a tiros}
\end{EntryWithPhonetic}

%%%%%%%%%% 将 %%%%%%%%%%
\subsection*{将}\addcontentsline{loh}{figure}{将 \dpy{qiang1}}

\begin{EntryWithPhonetic}{将}{qiang1}{9}{⼨}
  \definition{v.}{pedir; apelar para}
  \seeref{jiang1}
  \seeref{jiang4}
\end{EntryWithPhonetic}

%%%%%%%%%% 腔 %%%%%%%%%%
\subsection*{腔}\addcontentsline{loh}{figure}{腔 \dpy{qiang1}}

\begin{EntryWithPhonetic}{腔}{qiang1}{12}{⾁}[HSK 7-9]
  \definition{clas.}{Arcaico: utilizado para carcaças de animais abatidos}
  \definition{s.}{cavidade; câmara (em corpos humanos ou animais) | afinação; tom; tom de voz | sotaque (na fala) | discurso | (geralmente no vernáculo antigo para cabra ou ovelha abatida) carcaça}
  \seealsoref{腔儿}{qiang1r5}
\end{EntryWithPhonetic}

\begin{EntryWithPhonetic}{腔儿}{qiang1r5}{12,2}{⾁,⼉}
  \definition{s.}{afinação; tom | sotaque | fala}
\end{EntryWithPhonetic}

%%%%%%%%%% 强 %%%%%%%%%%
\subsection*{强}\addcontentsline{loh}{figure}{强 \dpy{qiang2}}

\begin{EntryWithPhonetic}{强}{qiang2}{12}{⼸}[HSK 3]
  \definition*{s.}{Sobrenome: Qiang}
  \definition{adj.}{forte; poderoso  (em oposição a 弱) | melhor; superior | mais; extra; adicional; um pouco mais que; usado após uma fração ou decimal para indicar que é um pouco maior que o número | resoluto; firme | violento | alto padrão}
  \definition{v.}{fortalecer; tornar forte; tornar poderoso}
  \seeref{jiang4}
  \seeref{qiang3}
  \seealsoref{弱}{ruo4}
\end{EntryWithPhonetic}

\begin{EntryWithPhonetic}{强大}{qiang2 da4}{12,3}{⼸,⼤}[HSK 3]
  \definition{adj.}{forte; poderoso; potente; possante; descreve força forte e grande poder}
\end{EntryWithPhonetic}

\begin{EntryWithPhonetic}{强盗}{qiang2 dao4}{12,11}{⼸,⽫}[HSK 6]
  \definition[个,群,伙,帮]{s.}{ladrão; bandido; uma pessoa que usa violência para confiscar a propriedade de outros; também se refere a uma pessoa ou força que se envolve em comportamento semelhante}
\end{EntryWithPhonetic}

\begin{EntryWithPhonetic}{强调}{qiang2diao4}{12,10}{⼸,⾔}[HSK 3]
  \definition{v.}{salientar; sublinhar; enfatizar; dar ênfase a; vincar}
\end{EntryWithPhonetic}

\begin{EntryWithPhonetic}{强度}{qiang2 du4}{12,9}{⼸,⼴}[HSK 5]
  \definition[个,种]{s.}{intensidade; força | magnitude; rigor; avidez}
\end{EntryWithPhonetic}

\begin{EntryWithPhonetic}{强化}{qiang2 hua4}{12,4}{⼸,⼔}[HSK 6]
  \definition{v.}{intensificar; fortalecer; consolidar; tornar mais forte, melhorar sua habilidade e nível}
\end{EntryWithPhonetic}

\begin{EntryWithPhonetic}{强加}{qiang2jia1}{12,5}{⼸,⼒}[HSK 7-9]
  \definition{v.}{forçar; impor; forçar alguém a aceitar uma determinada opinião ou prática}
\end{EntryWithPhonetic}

\begin{EntryWithPhonetic}{强劲}{qiang2jing4}{12,7}{⼸,⼒}[HSK 7-9]
  \definition{adj.}{poderoso}
\end{EntryWithPhonetic}

\begin{EntryWithPhonetic}{强烈}{qiang2lie4}{12,10}{⼸,⽕}[HSK 3]
  \definition{adj.}{muito forte; intenso; poderoso | violento; impetuoso; nível muito alto; atitude muito firme, sem espaço para mudanças | afiado; marcante; mostrado em contraste; muito claro}
\end{EntryWithPhonetic}

\begin{EntryWithPhonetic}{强势}{qiang2 shi4}{12,8}{⼸,⼒}[HSK 6]
  \definition*{adj.}{forte; poderoso; dominante}
  \definition{s.}{momento; ímpeto; grande impulso; forte impulso | força; influência dominante; forças poderosas}
\end{EntryWithPhonetic}

\begin{EntryWithPhonetic}{强项}{qiang2xiang4}{12,9}{⼸,⾴}[HSK 7-9]
  \definition{adj.}{Literário: resoluto e inflexível; reto e inabalável}
  \definition{s.}{ponto forte (de um atleta ou de uma equipe) | jogo, evento ou assunto no qual alguém é forte | principal ponto forte | especialidade}
\end{EntryWithPhonetic}

\begin{EntryWithPhonetic}{强行}{qiang2xing2}{12,6}{⼸,⾏}[HSK 7-9]
  \definition{adv.}{à força; com força}
\end{EntryWithPhonetic}

\begin{EntryWithPhonetic}{强硬}{qiang2ying4}{12,12}{⼸,⽯}[HSK 7-9]
  \definition{adj.}{forte; resistente; inflexível; poderoso; não disposto a recuar}
\end{EntryWithPhonetic}

\begin{EntryWithPhonetic}{强占}{qiang2zhan4}{12,5}{⼸,⼘}[HSK 7-9]
  \definition{v.}{ocupar à força; apoderar-se | ocupar à força; confiscar; anexar}
\end{EntryWithPhonetic}

\begin{EntryWithPhonetic}{强制}{qiang2zhi4}{12,8}{⼸,⼑}[HSK 7-9]
  \definition{v.}{forçar; compelir; coagir; utilizar o poder legal, político e econômico para forçar}
\end{EntryWithPhonetic}

\begin{EntryWithPhonetic}{强壮}{qiang2 zhuang4}{12,6}{⼸,⼠}[HSK 6]
  \definition{s.}{(corpo) forte, poderoso, robusto, resistente}
  \definition{v.}{fortalecer; construir}
\end{EntryWithPhonetic}

%%%%%%%%%% 墙 %%%%%%%%%%
\subsection*{墙}\addcontentsline{loh}{figure}{墙 \dpy{qiang2}}

\begin{EntryWithPhonetic}{墙}{qiang2}{14}{⼟}[HSK 2]
  \definition[面,堵,道]{s.}{parede; barreira ou perímetro construído com tijolos, pedras, etc. | qualquer coisa com a forma ou função de uma parede; a parte de um objeto que funciona como parede ou divisória}
  \definition{v.}{(gíria) bloquear (um website) (usado geralmente na voz passiva: 被墙)}
\end{EntryWithPhonetic}

\begin{EntryWithPhonetic}{墙壁}{qiang2 bi4}{14,16}{⼟,⼟}[HSK 5]
  \definition[面,堵,道]{s.}{parede; barreira ou perímetro construído com tijolos, pedras ou terra}
\end{EntryWithPhonetic}

\begin{EntryWithPhonetic}{墙纸}{qiang2zhi3}{14,7}{⼟,⽷}
  \definition{s.}{papel de parede}
\end{EntryWithPhonetic}

%%%%%%%%%% 抢 %%%%%%%%%%
\subsection*{抢}\addcontentsline{loh}{figure}{抢 \dpy{qiang3}}

\begin{EntryWithPhonetic}{抢}{qiang3}{7}{⼿}[HSK 5]
  \definition{v.}{roubar; saquear | agarrar; apanhar; arrebatar | disputar; lutar por; ser o primeiro; competir para ser o primeiro | correr; apressar-se; fazer uma incursão | raspar; arranhar; raspar ou esfregar uma camada da superfície de um objeto}
  \seeref{qiang1}
\end{EntryWithPhonetic}

\begin{EntryWithPhonetic}{抢夺}{qiang3duo2}{7,6}{⼿,⼤}[HSK 7-9]
  \definition{v.}{agarrar; arrebatar; tomar}
\end{EntryWithPhonetic}

\begin{EntryWithPhonetic}{抢劫}{qiang3jie2}{7,7}{⼿,⼒}[HSK 7-9]
  \definition{v.}{roubar; saquear; pilhar; usar violência ilegalmente para se apropriar da propriedade alheia}
\end{EntryWithPhonetic}

\begin{EntryWithPhonetic}{抢救}{qiang3jiu4}{7,11}{⼿,⽁}[HSK 5]
  \definition{v.}{salvar; resgatar; prestar de socorro ou assistência rápidos em situações de emergência | salvar; tomar medidas rápidas para evitar ou minimizar perdas iminentes.}
\end{EntryWithPhonetic}

\begin{EntryWithPhonetic}{抢掠}{qiang3lve4}{7,11}{⼿,⼿}
  \definition{s.}{saque | pilhagem}
  \definition{v.}{saquear | pilhar}
\end{EntryWithPhonetic}

\begin{EntryWithPhonetic}{抢眼}{qiang3yan3}{7,11}{⼿,⽬}[HSK 7-9]
  \definition{adj.}{chamativo; significa ser muito chamativo e atrair a atenção do público}
\end{EntryWithPhonetic}

%%%%%%%%%% 强 %%%%%%%%%%
\subsection*{强}\addcontentsline{loh}{figure}{强 \dpy{qiang3}}

\begin{EntryWithPhonetic}{强}{qiang3}{12}{⼸}
  \definition{v.}{fazer um esforço; esforçar-se}
  \seeref{jiang4}
  \seeref{qiang2}
\end{EntryWithPhonetic}

\begin{EntryWithPhonetic}{强迫}{qiang3po4}{12,8}{⼸,⾡}[HSK 5]
  \definition{v.}{impelir; forçar; impor; compelir; aplicar pessão para obedecer}
\end{EntryWithPhonetic}

%%%%%%%%%% 呛 %%%%%%%%%%
\subsection*{呛}\addcontentsline{loh}{figure}{呛 \dpy{qiang4}}

\begin{EntryWithPhonetic}{呛}{qiang4}{7}{⼝}
  \definition{v.}{causar asfixia ou sufocamento; irritar; sensação de mal-estar devido à entrada de gases irritantes no sistema respiratório}
  \seeref{qiang1}
\end{EntryWithPhonetic}

%%%%%%%%%% 悄 %%%%%%%%%%
\subsection*{悄}\addcontentsline{loh}{figure}{悄 \dpy{qiao1}}

\begin{EntryWithPhonetic}{悄}{qiao1}{10}{⼼}
  \definition{adj.}{quieto; silencioso}
  \seeref{qiao3}
\end{EntryWithPhonetic}

\begin{EntryWithPhonetic}{悄悄}{qiao1qiao1}{10,10}{⼼,⼼}[HSK 5]
  \definition{adv.}{silenciosamente; em silêncio; aos sussuros; sem som ou em voz baixa; com o mínimo de ruído possível}
\end{EntryWithPhonetic}

%%%%%%%%%% 敲 %%%%%%%%%%
\subsection*{敲}\addcontentsline{loh}{figure}{敲 \dpy{qiao1}}

\begin{EntryWithPhonetic}{敲}{qiao1}{14}{⽁}[HSK 5]
  \definition{v.}{bater; dar uma pancada; golpear | explorar alguém; cobrar a mais; extorquir; chantagear | lembrar; criticar; alertar; advertir}
\end{EntryWithPhonetic}

\begin{EntryWithPhonetic}{敲边鼓}{qiao1 bian1gu3}{14,5,13}{⽁,⾡,⿎}[HSK 7-9]
  \definition{v.}{``Tocar trompa.'' | Coloquial: falar ou agir para ajudar alguém à margem; apoiar alguém; apoiar alguém em uma discussão}
\end{EntryWithPhonetic}

\begin{EntryWithPhonetic}{敲门}{qiao1 men2}{14,3}{⽁,⾨}[HSK 5]
  \definition{v.}{bater na porta}
\end{EntryWithPhonetic}

\begin{EntryWithPhonetic}{敲诈}{qiao1zha4}{14,7}{⽁,⾔}[HSK 7-9]
  \definition{v.}{extorquir; chantagear; usar o poder, a intimidação e as ameaças para extorquir dinheiro}
\end{EntryWithPhonetic}

%%%%%%%%%% 乔 %%%%%%%%%%
\subsection*{乔}\addcontentsline{loh}{figure}{乔 \dpy{qiao2}}

\begin{EntryWithPhonetic}{乔}{qiao2}{6}{⼃}
  \definition*{s.}{Sobrenome: Qiao}
  \definition{adj.}{alto, imponente; orgulhoso, imponente}
\end{EntryWithPhonetic}

\begin{EntryWithPhonetic}{乔装}{qiao2zhuang1}{6,12}{⼃,⾐}[HSK 7-9]
  \definition{v.}{disfarçar; vestir-se}
\end{EntryWithPhonetic}

%%%%%%%%%% 桥 %%%%%%%%%%
\subsection*{桥}\addcontentsline{loh}{figure}{桥 \dpy{qiao2}}

\begin{EntryWithPhonetic}{桥}{qiao2}{10}{⽊}[HSK 3]
  \definition*{s.}{Sobrenome: Qiao}
  \definition[座]{s.}{ponte; construção que atravessa a água conectando as duas margens}
\end{EntryWithPhonetic}

\begin{EntryWithPhonetic}{桥梁}{qiao2liang2}{10,11}{⽊,⽊}[HSK 6]
  \definition[座]{s.}{ponte; acesso; uma obra construída na superfície do rio, conectando as duas margens | ponte; metáfora para pessoas ou coisas que podem se comunicar}
\end{EntryWithPhonetic}

%%%%%%%%%% 翘 %%%%%%%%%%
\subsection*{翘}\addcontentsline{loh}{figure}{翘 \dpy{qiao2}}

\begin{EntryWithPhonetic}{翘}{qiao2}{12}{⽻}
  \definition{v.}{levantar (a cabeça) | empenar; tornar"-se deformado}
  \seeref{qiao4}
\end{EntryWithPhonetic}

%%%%%%%%%% 瞧 %%%%%%%%%%
\subsection*{瞧}\addcontentsline{loh}{figure}{瞧 \dpy{qiao2}}

\begin{EntryWithPhonetic}{瞧}{qiao2}{17}{⽬}[HSK 5]
  \definition{v.}{ver; olhar | tratar; diagnosticar e tratar | ver; visitar; fazer uma visita}
\end{EntryWithPhonetic}

\begin{EntryWithPhonetic}{瞧不起}{qiao2bu5qi3}{17,4,10}{⽬,⼀,⾛}[HSK 7-9]
  \definition{v.}{olhar com desdém para alguém; menosprezar alguém; torcer o nariz para alguém; desprezar}
\end{EntryWithPhonetic}

%%%%%%%%%% 巧 %%%%%%%%%%
\subsection*{巧}\addcontentsline{loh}{figure}{巧 \dpy{qiao3}}

\begin{EntryWithPhonetic}{巧}{qiao3}{5}{⼯}[HSK 3]
  \definition{adj.}{habilidoso; engenhoso; esperto | oportuno; coincidente; fortuito | astuto; enganoso; enganador; traiçoeiro; ardiloso | (de mão, língua) hábil; loquaz}
  \definition{s.}{(tecnologia, artesanato) habilidade; destreza}
\end{EntryWithPhonetic}

\begin{EntryWithPhonetic}{巧合}{qiao3he2}{5,6}{⼯,⼝}[HSK 7-9]
  \definition{s.}{coincidência; (coisas) coincidentes ou idênticas}
\end{EntryWithPhonetic}

\begin{EntryWithPhonetic}{巧克力}{qiao3ke4li4}{5,7,2}{⼯,⼗,⼒}[HSK 4]
  \definition[块,颗,盒,包]{s.}{Empréstimo linguístico: chocolate; alimentos feitos com cacau em pó como principal matéria-prima, açúcar e especiarias}
\end{EntryWithPhonetic}

\begin{EntryWithPhonetic}{巧妙}{qiao3miao4}{5,7}{⼯,⼥}[HSK 6]
  \definition{adj.}{inteligente; engenhoso; (método ou técnica, etc.) inteligente, além do comum}
\end{EntryWithPhonetic}

%%%%%%%%%% 悄 %%%%%%%%%%
\subsection*{悄}\addcontentsline{loh}{figure}{悄 \dpy{qiao3}}

\begin{EntryWithPhonetic}{悄}{qiao3}{10}{⼼}
  \definition{adj.}{quieto; silencioso | triste; preocupado; aflito}
  \seeref{qiao1}
\end{EntryWithPhonetic}

%%%%%%%%%% 壳 %%%%%%%%%%
\subsection*{壳}\addcontentsline{loh}{figure}{壳 \dpy{qiao4}}

\begin{EntryWithPhonetic}{壳}{qiao4}{7}{⼠}
  \definition[层,个]{s.}{Coloquial: concha | invólucro; caixa; carapaça | empresa de fachada (ou corporação) | superfície dura}
  \seeref{ke2}
\end{EntryWithPhonetic}

%%%%%%%%%% 窍 %%%%%%%%%%
\subsection*{窍}\addcontentsline{loh}{figure}{窍 \dpy{qiao4}}

\begin{EntryWithPhonetic}{窍}{qiao4}{10}{⽳}
  \definition{s.}{abertura; furo | chave para algo; a chave para a questão}
\end{EntryWithPhonetic}

\begin{EntryWithPhonetic}{窍门}{qiao4men2}{10,3}{⽳,⾨}[HSK 7-9]
  \definition[个]{s.}{habilidade; chave para um problema; segredo para fazer algo; um método inteligente que resolve o problema e é simples e fácil de implementar}
\end{EntryWithPhonetic}

%%%%%%%%%% 翘 %%%%%%%%%%
\subsection*{翘}\addcontentsline{loh}{figure}{翘 \dpy{qiao4}}

\begin{EntryWithPhonetic}{翘}{qiao4}{12}{⽻}[HSK 7-9]
  \definition{v.}{manter (segurar) erguido; dobrar (virar) para cima; enrolar-se}
  \seeref{qiao2}
\end{EntryWithPhonetic}

%%%%%%%%%% 撬 %%%%%%%%%%
\subsection*{撬}\addcontentsline{loh}{figure}{撬 \dpy{qiao4}}

\begin{EntryWithPhonetic}{撬}{qiao4}{15}{⼿}[HSK 7-9]
  \definition{v.}{arrombar; arrancar; forçar com alavanca}[钥匙丢了,他只好把门撬开。===Ele perdeu a chave, então teve que arrombar a porta.]
\end{EntryWithPhonetic}

%%%%%%%%%% 切 %%%%%%%%%%
\subsection*{切}\addcontentsline{loh}{figure}{切 \dpy{qie1}}

\begin{EntryWithPhonetic}{切}{qie1}{4}{⼑}[HSK 4]
  \definition{v.}{cortar; fatiar; separar itens com uma faca | cortar ou romper; truncar | Geometria: refere-se a quando uma linha, círculo ou superfície intercepta um círculo, arco ou esfera em apenas um ponto}
  \seeref{qie4}
\end{EntryWithPhonetic}

\begin{EntryWithPhonetic}{切除}{qie1chu2}{4,9}{⼑,⾩}[HSK 7-9]
  \definition[次]{s.}{excisão; ressecção; abscisão; corte; seccionamento}
  \definition{v.}{excisar; ressectar; remover cirurgicamente uma parte de uma estrutura ou órgão do corpo}
\end{EntryWithPhonetic}

\begin{EntryWithPhonetic}{切断}{qie1duan4}{4,11}{⼑,⽄}[HSK 7-9]
  \definition{v.}{cortar; cortar algo ao meio com uma faca; uma metáfora para usar a força para separar coisas que estão conectadas}
\end{EntryWithPhonetic}

\begin{EntryWithPhonetic}{切割}{qie1ge1}{4,12}{⼑,⼑}[HSK 7-9]
  \definition{v.}{esculpir; cortar algo com uma faca | cortar algo com máquina, fogo, arco voltaico; refere-se especificamente ao corte de materiais metálicos com máquinas-ferramentas ou à queima deles com arco elétrico, laser, etc.}
\end{EntryWithPhonetic}

%%%%%%%%%% 茄 %%%%%%%%%%
\subsection*{茄}\addcontentsline{loh}{figure}{茄 \dpy{qie2}}

\begin{EntryWithPhonetic}{茄}{qie2}{8}{⾋}
  \definition[只]{s.}{berinjela}
  \seeref{jia1}
\end{EntryWithPhonetic}

\begin{EntryWithPhonetic}{茄子}{qie2 zi5}{8,3}{⾋,⼦}[HSK 6]
  \definition{interj.}{Onomatopéia: ``xis'' fonético (ao ser fotografado), equivale ao ``diga xis''}
  \definition[个,根]{s.}{berinjela (fruto e planta)}
\end{EntryWithPhonetic}

%%%%%%%%%% 且 %%%%%%%%%%
\subsection*{且}\addcontentsline{loh}{figure}{且 \dpy{qie3}}

\begin{EntryWithPhonetic}{且}{qie3}{5}{⼀}[HSK 7-9]
  \definition*{s.}{Sobrenome: Qie}
  \definition{adv.}{apenas; por enquanto | por um longo tempo; indica algo duradouro e resistente}
  \definition{conj.}{mesmo; até; até mesmo; usado na primeira cláusula de uma frase complexa para expressar concessão, equivalente a 尚且 | ambos\dots e\dots; conecta adjetivos ou verbos para expressar relacionamento paralelo, equivalente a 而且 e 又…又…}
  \seealsoref{而且}{er2 qie3}
  \seealsoref{尚且}{shang4 qie3}
  \seealsoref{又…又…}{you4 you4}
\end{EntryWithPhonetic}

%%%%%%%%%% 切 %%%%%%%%%%
\subsection*{切}\addcontentsline{loh}{figure}{切 \dpy{qie4}}

\begin{EntryWithPhonetic}{切}{qie4}{4}{⼑}
  \definition{adj.}{ansioso; sério | duro; severo; rude; áspero}
  \definition{adv.}{com certeza; certamente}
  \definition{s.}{limiar; degrau}
  \definition{v.}{ser prático ou realista | ajustar-se ou corresponder | ser próximo ou íntimo | cortar algo em pedaços com uma faca | tomar o pulso (medicina tradicional chinesa)}
  \seeref{qie1}
\end{EntryWithPhonetic}

\begin{EntryWithPhonetic}{切身}{qie4shen1}{4,7}{⼑,⾝}[HSK 7-9]
  \definition{adj.}{de interesse imediato para si próprio ou para alguém | pessoal; de primeira mão | de preocupação imediata para si mesmo; estreitamente relacionado a si mesmo}
\end{EntryWithPhonetic}

\begin{EntryWithPhonetic}{切实}{qie4shi2}{4,8}{⼑,⼧}[HSK 6]
  \definition{adj.}{prático; viável; realista}
\end{EntryWithPhonetic}

%%%%%%%%%% 窃 %%%%%%%%%%
\subsection*{窃}\addcontentsline{loh}{figure}{窃 \dpy{qie4}}

\begin{EntryWithPhonetic}{窃}{qie4}{9}{⽳}
  \definition{adv.}{secretamente; sorrateiramente; furtivamente | usado antes de verbos para demonstrar modéstia, frequentemente significando ``Eu humildemente penso\dots'' ou ``Eu acredito em particular\dots''}[臣窃谓此策虽妙,实难实行。===Acredito humildemente que, embora essa estratégia seja engenhosa, na realidade é difícil de implementar.]
  \definition{pron.}{Literário: (referindo-se às próprias opiniões) meu; minha}
  \definition{v.}{roubar; furtar | apoderar-se ou ocupar ilegitimamente; tomar posse sem direito}
\end{EntryWithPhonetic}

\begin{EntryWithPhonetic}{窃取}{qie4qu3}{9,8}{⽳,⼜}[HSK 7-9]
  \definition{v.}{usurpar; apoderar-se; roubar (frequentemente usado metaforicamente)}
\end{EntryWithPhonetic}

%%%%%%%%%% 亲 %%%%%%%%%%
\subsection*{亲}\addcontentsline{loh}{figure}{亲 \dpy{qin1}}

\begin{EntryWithPhonetic}{亲}{qin1}{9}{⼇}[HSK 3]
  \definition{adj.}{parente próximo; relacionado por sangue; de ​​parentesco consanguíneo; parente consanguíneo mais próximo | querido; próximo; íntimo; relações próximas entre pessoas; sentimentos profundos (em oposição a 疏) | em si mesmo; pessoalmente}
  \definition[位]{s.}{pais; refere-se aos pais; também se refere apenas ao pai ou à mãe | parente; refere-se a pessoas que são relacionadas por sangue ou casamento| casal; casamento; refere-se ao casamento ou relacionamento conjugal | noiva; refere-se especificamente à noiva}
  \definition{v.}{beijar | (de países, partidos, etc.) a favor de; apoiar; estar perto de}
  \seeref{qing4}
  \seealsoref{疏}{shu1}
\end{EntryWithPhonetic}

\begin{EntryWithPhonetic}{亲爱}{qin1'ai4}{9,10}{⼇,⽖}[HSK 4]
  \definition{adj.}{querido; amado; termo carinhoso que expressa intimidade e afeto}
\end{EntryWithPhonetic}

\begin{EntryWithPhonetic}{亲和力}{qin1he2li4}{9,8,2}{⼇,⼝,⼒}[HSK 7-9]
  \definition{s.}{afinidade; as forças que interagem quando duas ou mais substâncias se combinam para formar um composto | amabilidade; sociabilidade; forte atração; metaforicamente falando, refere-se a uma força que faz as pessoas se sentirem confortáveis, amigáveis ​​e dispostas a se aproximar}
\end{EntryWithPhonetic}

\begin{EntryWithPhonetic}{亲近}{qin1jin4}{9,7}{⼇,⾡}[HSK 7-9]
  \definition{adj.}{íntimo e próximo}
  \definition{v.}{ser próximo de; ter intimidade com}
\end{EntryWithPhonetic}

\begin{EntryWithPhonetic}{亲密}{qin1mi4}{9,11}{⼇,⼧}[HSK 4]
  \definition{adj.}{próximo; íntimo; relacionamento afetuoso e próximo}
\end{EntryWithPhonetic}

\begin{EntryWithPhonetic}{亲朋好友}{qin1peng2-hao3you3}{9,8,6,4}{⼇,⽉,⼥,⼜}[HSK 7-9]
  \definition{s.}{``Amigos e familiares.''; amigos e família; parentes e amigos}
\end{EntryWithPhonetic}

\begin{EntryWithPhonetic}{亲戚}{qin1qi5}{9,11}{⼇,⼽}[HSK 7-9]
  \definition[门,个,位]{s.}{parentes; pessoas com laços matrimoniais ou consanguíneos em sua própria família}
\end{EntryWithPhonetic}

\begin{EntryWithPhonetic}{亲切}{qin1qie4}{9,4}{⼇,⼑}[HSK 3]
  \definition{adj.}{gentil; cordial; cheio de sinceridade e cuidado, fazendo com que as pessoas se sintam acolhidas e acessíveis | próximo; íntimo; por familiaridade e afeição}
\end{EntryWithPhonetic}

\begin{EntryWithPhonetic}{亲情}{qin1qing2}{9,11}{⼇,⼼}[HSK 7-9]
  \definition{s.}{afeto; laços familiares; vínculo emocional entre membros da família; o afeto entre membros da família}
\end{EntryWithPhonetic}

\begin{EntryWithPhonetic}{亲热}{qin1re4}{9,10}{⼇,⽕}[HSK 7-9]
  \definition{adj.}{afetuoso; íntimo; caloroso; íntimo e acolhedor}
  \definition{v.}{comportar-se afetuosamente; demonstrar intimidade e entusiasmo}
\end{EntryWithPhonetic}

\begin{EntryWithPhonetic}{亲人}{qin1 ren2}{9,2}{⼇,⼈}[HSK 3]
  \definition[个,位]{s.}{um membro da família; os pais, o cônjuge, os filhos, etc.; refere-se a parentes ou cônjuges | queridos; entes queridos; aqueles queridos para alguém; uma metáfora para pessoas que têm um relacionamento próximo e sentimentos profundos}
\end{EntryWithPhonetic}

\begin{EntryWithPhonetic}{亲身}{qin1shen1}{9,7}{⼇,⾝}[HSK 7-9]
  \definition{adj.}{pessoal; em primeira mão}
  \definition{adv.}{pessoalmente}
\end{EntryWithPhonetic}

\begin{EntryWithPhonetic}{亲生}{qin1sheng1}{9,5}{⼇,⽣}[HSK 7-9]
  \definition{adj.}{próprio; biológico (filhos, pais); aquelas que dão à luz a si mesmas ou têm filhos próprios}
  \definition{v.}{ser filho biológico de alguém (ou seja, não adotado)}
\end{EntryWithPhonetic}

\begin{EntryWithPhonetic}{亲手}{qin1shou3}{9,4}{⼇,⼿}[HSK 7-9]
  \definition{adv.}{si mesmo; pessoalmente; com as próprias mãos}
\end{EntryWithPhonetic}

\begin{EntryWithPhonetic}{亲属}{qin1 shu3}{9,12}{⼇,⼫}[HSK 6]
  \definition{s.}{parentes; cognatos}
\end{EntryWithPhonetic}

\begin{EntryWithPhonetic}{亲眼}{qin1 yan3}{9,11}{⼇,⽬}[HSK 6]
  \definition{adv.}{pessoalmente; com os próprios olhos}
\end{EntryWithPhonetic}

\begin{EntryWithPhonetic}{亲友}{qin1you3}{9,4}{⼇,⼜}[HSK 7-9]
  \definition[位,个]{s.}{amigos e parentes; parentes próximos}
\end{EntryWithPhonetic}

\begin{EntryWithPhonetic}{亲自}{qin1zi4}{9,6}{⼇,⾃}[HSK 3]
  \definition{adv.}{pessoalmente; em pessoa; si mesmo; fazer algo diretamente por si mesmo}
\end{EntryWithPhonetic}

%%%%%%%%%% 侵 %%%%%%%%%%
\subsection*{侵}\addcontentsline{loh}{figure}{侵 \dpy{qin1}}

\begin{EntryWithPhonetic}{侵}{qin1}{9}{⼈}
  \definition*{s.}{Sobrenome: Qin}
  \definition{prep.}{aproximando-se; aproximar}
  \definition{v.}{invadir; intrometer-se em; infringir | aproximar-se (amanhecer)}
\end{EntryWithPhonetic}

\begin{EntryWithPhonetic}{侵犯}{qin1fan4}{9,5}{⼈,⽝}[HSK 6]
  \definition{v.}{violar; invadir; infringir; interferência ilegal com terceiros e violação de seus direitos | violar; fazer incursões; invadir o território de outro país}
\end{EntryWithPhonetic}

\begin{EntryWithPhonetic}{侵害}{qin1hai4}{9,10}{⼈,⼧}[HSK 7-9]
  \definition{v.}{prejudicar; violar; infringir}
\end{EntryWithPhonetic}

\begin{EntryWithPhonetic}{侵略}{qin1lve4}{9,11}{⼈,⽥}[HSK 7-9]
  \definition{v.}{invadir; agredir; violar o território e a soberania de outro país por meio de invasão armada, interferência política ou infiltração econômica e cultural, prejudicando assim os interesses desse outro país}
\end{EntryWithPhonetic}

\begin{EntryWithPhonetic}{侵权}{qin1quan2}{9,6}{⼈,⽊}[HSK 7-9]
  \definition{s.}{infração}
  \definition{v.}{infringir os direitos de}
\end{EntryWithPhonetic}

\begin{EntryWithPhonetic}{侵入}{qin1ru4}{9,2}{⼈,⼊}
  \definition{v.}{invadir; intrometer-se em; fazer incursões em; (o inimigo) entra no território; (coisas estranhas ou nocivas) entram no interior}
\end{EntryWithPhonetic}

\begin{EntryWithPhonetic}{侵占}{qin1zhan4}{9,5}{⼈,⼘}[HSK 7-9]
  \definition{v.}{ocupar à força; tomar posse ilegal da propriedade alheia | invadir e ocupar o território de outro país; ocupar o território de outro país por meio de agressão}
\end{EntryWithPhonetic}

%%%%%%%%%% 钦 %%%%%%%%%%
\subsection*{钦}\addcontentsline{loh}{figure}{钦 \dpy{qin1}}

\begin{EntryWithPhonetic}{钦}{qin1}{9}{⾦}
  \definition*{s.}{Sobrenome: Qin}
  \definition{adv.}{pelo próprio imperador}
  \definition{v.}{admirar; respeitar}
\end{EntryWithPhonetic}

\begin{EntryWithPhonetic}{钦佩}{qin1pei4}{9,8}{⾦,⼈}[HSK 7-9]
  \definition{v.}{admirar; respeitar; ter em alta consideração alguém; sentir respeito e afeto por alguém}
\end{EntryWithPhonetic}

%%%%%%%%%% 芹 %%%%%%%%%%
\subsection*{芹}\addcontentsline{loh}{figure}{芹 \dpy{qin2}}

\begin{EntryWithPhonetic}{芹}{qin2}{7}{⾋}
  \definition[把,棵]{s.}{aipo | aipo chinês}
\end{EntryWithPhonetic}

\begin{EntryWithPhonetic}{芹菜}{qin2cai4}{7,11}{⾋,⾋}
  \definition{s.}{salsão}
\end{EntryWithPhonetic}

%%%%%%%%%% 琴 %%%%%%%%%%
\subsection*{琴}\addcontentsline{loh}{figure}{琴 \dpy{qin2}}

\begin{EntryWithPhonetic}{琴}{qin2}{12}{⽟}[HSK 5]
  \definition*{s.}{Sobrenome: Qin}
  \definition[架,台]{s.}{cítara; qin; guqin (um instrumento de cordas dedilhadas com sete cordas, em alguns aspectos semelhante à cítara)  | nome genérico para certos instrumentos musicais}
\end{EntryWithPhonetic}

\begin{EntryWithPhonetic}{琴键}{qin2jian4}{12,13}{⽟,⾦}
  \definition{s.}{tecla de piano}
\end{EntryWithPhonetic}

%%%%%%%%%% 禽 %%%%%%%%%%
\subsection*{禽}\addcontentsline{loh}{figure}{禽 \dpy{qin2}}

\begin{EntryWithPhonetic}{禽}{qin2}{12}{⽱}
  \definition*{s.}{Sobrenome: Qin}
  \definition[只]{s.}{aves; pássaros | termo genérico para aves e animais}
\end{EntryWithPhonetic}

%%%%%%%%%% 勤 %%%%%%%%%%
\subsection*{勤}\addcontentsline{loh}{figure}{勤 \dpy{qin2}}

\begin{EntryWithPhonetic}{勤}{qin2}{13}{⼒}
  \definition*{s.}{Sobrenome: Qin}
  \definition{adj.}{diligente; industrial; trabalhador}
  \definition{adv.}{frequentemente}
  \definition{s.}{dever; serviço | presença; trabalhadores que chegam ao trabalho no horário especificado}
\end{EntryWithPhonetic}

\begin{EntryWithPhonetic}{勤奋}{qin2fen4}{13,8}{⼒,⼤}[HSK 5]
  \definition{adj.}{diligente; assíduo; trabalhador; descreve alguém que se esforça continuamente nos estudos ou no trabalho}
\end{EntryWithPhonetic}

\begin{EntryWithPhonetic}{勤工俭学}{qin2gong1-jian3xue2}{13,3,9,8}{⼒,⼯,⼈,⼦}[HSK 7-9]
  \definition{expr.}{``Programa de trabalho e estudo.''; estudar em um programa de trabalho e estudo; trabalho em tempo parcial e estudo em tempo parcial; administrar uma escola com base na autossustentabilidade por meio de trabalho árduo; estudar em regime de trabalho e estudo; a prática de trabalhar enquanto estuda}
\end{EntryWithPhonetic}

\begin{EntryWithPhonetic}{勤快}{qin2kuai5}{13,7}{⼒,⼼}[HSK 7-9]
  \definition{adj.}{diligente; trabalhador; gosta de fazer coisas e não tem medo de se cansar}
\end{EntryWithPhonetic}

\begin{EntryWithPhonetic}{勤劳}{qin2lao2}{13,7}{⼒,⼒}[HSK 7-9]
  \definition{adj.}{diligente; trabalhador; esforçado; trabalhador e corajoso; trabalhe duro e não tenha medo das dificuldades}
\end{EntryWithPhonetic}

%%%%%%%%%% 擒 %%%%%%%%%%
\subsection*{擒}\addcontentsline{loh}{figure}{擒 \dpy{qin2}}

\begin{EntryWithPhonetic}{擒}{qin2}{15}{⼿}
  \definition{v.}{capturar; pegar; apreender}
\end{EntryWithPhonetic}

\begin{EntryWithPhonetic}{擒获}{qin2huo4}{15,10}{⼿,⾋}
  \definition{v.}{apreender | capturar}
\end{EntryWithPhonetic}

%%%%%%%%%% 寝 %%%%%%%%%%
\subsection*{寝}\addcontentsline{loh}{figure}{寝 \dpy{qin3}}

\begin{EntryWithPhonetic}{寝}{qin3}{13}{⼧}
  \definition{s.}{quarto | túmulo; tumba}
  \definition{v.}{dormir | parar; terminar}
\end{EntryWithPhonetic}

\begin{EntryWithPhonetic}{寝室}{qin3shi4}{13,9}{⼧,⼧}[HSK 7-9]
  \definition[间]{s.}{quarto (em um dormitório); quarto de dormir}
\end{EntryWithPhonetic}

%%%%%%%%%% 青 %%%%%%%%%%
\subsection*{青}\addcontentsline{loh}{figure}{青 \dpy{qing1}}

\begin{EntryWithPhonetic}{青}{qing1}{8}{⾭}[HSK 5][Kangxi 174]
  \definition*{s.}{Província de Qinghai, abreviação de 青海 | Sobrenome: Qing}
  \definition{adj.}{azul ou verde | preto | jovens (pessoas)}
  \definition{s.}{grama verde | colheitas jovens (não maduras) | tiras de bambu verde}
  \seealsoref{青海}{qing1hai3}
\end{EntryWithPhonetic}

\begin{EntryWithPhonetic}{青菜}{qing1cai4}{8,11}{⾭,⾋}
  \definition{s.}{verduras}
\end{EntryWithPhonetic}

\begin{EntryWithPhonetic}{青春}{qing1chun1}{8,9}{⾭,⽇}[HSK 4]
  \definition[个]{s.}{juventude; jovialidade}
\end{EntryWithPhonetic}

\begin{EntryWithPhonetic}{青春期}{qing1chun1qi1}{8,9,12}{⾭,⽇,⽉}[HSK 7-9]
  \definition{s.}{puberdade; adolescência; refere-se ao período em que os órgãos sexuais masculinos e femininos se desenvolvem rapidamente até a maturidade completa, tipicamente entre os 14 e 16 anos para os meninos e entre os 13 e 14 anos para as meninas}
\end{EntryWithPhonetic}

\begin{EntryWithPhonetic}{青海}{qing1hai3}{8,10}{⾭,⽔}
  \definition*{s.}{Província de Qinghai}
\end{EntryWithPhonetic}

\begin{EntryWithPhonetic}{青椒}{qing1jiao1}{8,12}{⾭,⽊}
  \definition{s.}{pimenta verde}
\end{EntryWithPhonetic}

\begin{EntryWithPhonetic}{青年}{qing1 nian2}{8,6}{⾭,⼲}[HSK 2]
  \definition[个,位,名,些]{s.}{juventude; jovem; refere-se ao período entre os 15 e os 30 anos de idade.}
\end{EntryWithPhonetic}

\begin{EntryWithPhonetic}{青年节}{qing1nian2jie2}{8,6,5}{⾭,⼲,⾋}
  \definition*{s.}{Dia da Juventude (4 de maio)}
\end{EntryWithPhonetic}

\begin{EntryWithPhonetic}{青少年}{qing1shao4nian2}{8,4,6}{⾭,⼩,⼲}[HSK 2]
  \definition[位,名,个,些]{s.}{adolescentes}
\end{EntryWithPhonetic}

\begin{EntryWithPhonetic}{青天}{qing1tian1}{8,4}{⾭,⼤}
  \definition{s.}{céu claro, limpo ou azul}
\end{EntryWithPhonetic}

\begin{EntryWithPhonetic}{青铜}{qing1tong2}{8,11}{⾭,⾦}
  \definition{s.}{bronze (liga de cobre, 銅, e estanho, 锡)}
\end{EntryWithPhonetic}

\begin{EntryWithPhonetic}{青蛙}{qing1wa1}{8,12}{⾭,⾍}[HSK 7-9]
  \definition[只]{s.}{sapo}
\end{EntryWithPhonetic}

\begin{EntryWithPhonetic}{青玉米}{qing1yu4mi3}{8,5,6}{⾭,⽟,⽶}
  \definition{s.}{milho verde}
\end{EntryWithPhonetic}

%%%%%%%%%% 轻 %%%%%%%%%%
\subsection*{轻}\addcontentsline{loh}{figure}{轻 \dpy{qing1}}

\begin{EntryWithPhonetic}{轻}{qing1}{9}{⾞}[HSK 2]
  \definition{adj.}{de pouco peso; leve (oposto de 重) | (de carga, equipamento, etc.) pequeno; simples | pequeno em número, grau, etc. | não sério; relaxante; leve | sem importância | suave; delicado | levianos, crédulos | leve; peso leve; densidade baixa | leve; descontraído; fácil | imprudente; descuidado | inconstante; frívolo}
  \definition{v.}{menosprezar; subestimar}
  \seealsoref{重}{zhong4}
\end{EntryWithPhonetic}

\begin{EntryWithPhonetic}{轻而易举}{qing1'er2yi4ju3}{9,6,8,9}{⾞,⽽,⽇,⼂}[HSK 7-9]
  \definition{expr.}{pode ser feito de forma descuidada; com facilidade; leve e fácil de levantar; descreve algo como fácil de fazer, que exige pouco esforço}
\end{EntryWithPhonetic}

\begin{EntryWithPhonetic}{轻蔑}{qing1mie4}{9,14}{⾞,⾋}[HSK 7-9]
  \definition{v.}{desprezar; menosprezar; ignorar}
\end{EntryWithPhonetic}

\begin{EntryWithPhonetic}{轻松}{qing1song1}{9,8}{⾞,⽊}[HSK 4]
  \definition{adj.}{leve; relaxado; livre de fardos; não nervoso; não cansado}
  \definition{v.}{sentir-se livre de fardos; não se sentir nervoso ou cansado}
\end{EntryWithPhonetic}

\begin{EntryWithPhonetic}{轻微}{qing1wei1}{9,13}{⾞,⼻}[HSK 7-9]
  \definition{adj.}{leve; insignificante; banal; trivial}
\end{EntryWithPhonetic}

\begin{EntryWithPhonetic}{轻型}{qing1xing2}{9,9}{⾞,⼟}[HSK 7-9]
  \definition{adj.}{leve (máquinas, aeronaves etc.); leve (oposto a 重型)}
  \seealsoref{重型}{zhong4xing2}
\end{EntryWithPhonetic}

\begin{EntryWithPhonetic}{轻易}{qing1yi4}{9,8}{⾞,⽇}[HSK 4]
  \definition{adv.}{facilmente; prontamente | facilmente; precipitadamente; indica que uma ação é realizada casualmente, geralmente usado em frases negativas}
\end{EntryWithPhonetic}

%%%%%%%%%% 倾 %%%%%%%%%%
\subsection*{倾}\addcontentsline{loh}{figure}{倾 \dpy{qing1}}

\begin{EntryWithPhonetic}{倾}{qing1}{10}{⼈}
  \definition{s.}{desvio; tendência}
  \definition{v.}{inclinar; inclinar-se; dobrar-se | colapsar | virar e despejar; esvaziar | fazer tudo o que puder; usar todos os recursos | sobrecarregar; dominar; dominar | admirar | superar}
\end{EntryWithPhonetic}

\begin{EntryWithPhonetic}{倾城}{qing1cheng2}{10,9}{⼈,⼟}
  \definition{adj.}{sedutora (mulher)}
  \definition{adv.}{de todo o lugar | vindo de todos os lugares}
  \definition{v.}{arruinar e derrubar o estado}
\end{EntryWithPhonetic}

\begin{EntryWithPhonetic}{倾家荡产}{qing1jia1-dang4chan3}{10,10,9,6}{⼈,⼧,⾋,⼇}[HSK 7-9]
  \definition{expr.}{``Perder a fortuna da família.''; todos os bens da família foram perdidos; ser reduzido à pobreza e à ruína}
\end{EntryWithPhonetic}

\begin{EntryWithPhonetic}{倾诉}{qing1su4}{10,7}{⼈,⾔}[HSK 7-9]
  \definition{v.}{confessar; desabafar; confidenciar}
\end{EntryWithPhonetic}

\begin{EntryWithPhonetic}{倾听}{qing1ting1}{10,7}{⼈,⼝}[HSK 7-9]
  \definition{v.}{escutar atentamente; dar ouvidos com atenção; é frequentemente usado quando um superior se dirige a um subordinado}
\end{EntryWithPhonetic}

\begin{EntryWithPhonetic}{倾向}{qing1xiang4}{10,6}{⼈,⼝}[HSK 6]
  \definition{s.}{tendência; desvio; inclinação; direção do desenvolvimento}
  \definition{v.}{preferir; estar inclinado a; concordar com uma determinada opinião}
\end{EntryWithPhonetic}

\begin{EntryWithPhonetic}{倾销}{qing1xiao1}{10,12}{⼈,⾦}[HSK 7-9]
  \definition{v.}{despejar; praticar dumping; os capitalistas monopolistas vendem grandes quantidades de mercadorias a preços abaixo do mercado para derrotar os concorrentes, conquistar participação de mercado, monopolizar os preços das commodities e obter lucros exorbitantes}
\end{EntryWithPhonetic}

\begin{EntryWithPhonetic}{倾斜}{qing1xie2}{10,11}{⼈,⽃}[HSK 7-9]
  \definition{v.}{inclinar; inclinar-se; pender | ser a favor de; dar tratamento preferencial a; a metáfora descreve uma orientação política que enfatiza um aspecto específico}
\end{EntryWithPhonetic}

%%%%%%%%%% 清 %%%%%%%%%%
\subsection*{清}\addcontentsline{loh}{figure}{清 \dpy{qing1}}

\begin{EntryWithPhonetic}{清}{qing1}{11}{⽔}[HSK 6]
  \definition*{s.}{Dinastia Qing (1644-1911) | Sobrenome: Qing}
  \definition{adj.}{claro; não misturado; (líquido ou gasoso) puro e sem mistura (em oposição a 浊) | silencioso; quieto | justo e honesto | distinto; claro; esclarecido | simples; puro, sem qualquer adulteração ou combinação | limpo; puro}
  \definition{v.}{limpar; tornar limpo | resolver; esclarecer; pagar; liquidar | contar; inspecionar}
  \seealsoref{浊}{zhuo2}
\end{EntryWithPhonetic}

\begin{EntryWithPhonetic}{清唱}{qing1chang4}{11,11}{⽔,⼝}
  \definition{v.}{cantar à capela}
\end{EntryWithPhonetic}

\begin{EntryWithPhonetic}{清彻}{qing1che4}{11,7}{⽔,⼻}
  \variantof{清澈}
\end{EntryWithPhonetic}

\begin{EntryWithPhonetic}{清澈}{qing1che4}{11,15}{⽔,⽔}
  \definition{adj.}{claro | límpido}
\end{EntryWithPhonetic}

\begin{EntryWithPhonetic}{清晨}{qing1chen2}{11,11}{⽔,⽇}[HSK 5]
  \definition{s.}{matinal; manhã cedo; geralmente se refere ao período do amanhecer até logo após o nascer do sol}
\end{EntryWithPhonetic}

\begin{EntryWithPhonetic}{清除}{qing1chu2}{11,9}{⽔,⾩}[HSK 7-9]
  \definition{v.}{eliminar; remover; livrar-se de; remover completamente}
\end{EntryWithPhonetic}

\begin{EntryWithPhonetic}{清楚}{qing1chu5}{11,13}{⽔,⽊}[HSK 2]
  \definition{adj.}{claro; distinto; compreensível; organizado; fácil de identificar e entender | plenamente consciente de; claro sobre}
  \definition{v.}{ter clareza sobre; compreender; ação que expressa compreensão e conhecimento}
\end{EntryWithPhonetic}

\begin{EntryWithPhonetic}{清脆}{qing1cui4}{11,10}{⽔,⾁}[HSK 7-9]
  \definition{adj.}{claro e melodioso; nítido e agradável ao ouvido, não abafado | (comida) crocante e refrescante}
\end{EntryWithPhonetic}

\begin{EntryWithPhonetic}{清单}{qing1dan1}{11,8}{⽔,⼗}[HSK 7-9]
  \definition[张]{s.}{lista detalhada; relato detalhado; um catálogo; um inventário; formulário de inscrição detalhado para projetos relevantes}
\end{EntryWithPhonetic}

\begin{EntryWithPhonetic}{清淡}{qing1dan4}{11,11}{⽔,⽔}[HSK 7-9]
  \definition{adj.}{leve; fraco; suave; delicado; (cor, cheiro) leve e suave; não forte | leve; não gorduroso ou com sabor forte; (alimento) com baixo teor de gordura | suave; simples; descreve uma vida ou ritmo de vida como simples e descomplicado}
\end{EntryWithPhonetic}

\begin{EntryWithPhonetic}{清洁}{qing1jie2}{11,9}{⽔,⽔}[HSK 6]
  \definition{adj.}{limpo; sem poeira, gordura, etc.}
  \definition{v.}{limpar}
\end{EntryWithPhonetic}

\begin{EntryWithPhonetic}{清洁工}{qing1 jie2 gong1}{11,9,3}{⽔,⽔,⼯}[HSK 6]
  \definition{s.}{coletor de lixo; trabalhador de saneamento; limpador de rua; trabalhadores envolvidos na limpeza do ambiente, remoção de lixo e fezes, etc.}
\end{EntryWithPhonetic}

\begin{EntryWithPhonetic}{清静}{qing1jing4}{11,14}{⽔,⾭}[HSK 7-9]
  \definition{adj.}{(ambiente) calmo; pacífico; tranquilo; isolado; silencioso}
\end{EntryWithPhonetic}

\begin{EntryWithPhonetic}{清理}{qing1li3}{11,11}{⽔,⽟}[HSK 5]
  \definition{v.}{esclarecer; resolver; verificar; colocar em ordem; organizar tudo e jogar fora o que não for útil}
\end{EntryWithPhonetic}

\begin{EntryWithPhonetic}{清凉}{qing1liang2}{11,10}{⽔,⼎}[HSK 7-9]
  \definition{adj.}{fresco e refrescante; agradavelmente fresco; fresco e agradável; refrescante e gelado}
\end{EntryWithPhonetic}

\begin{EntryWithPhonetic}{清明}{qing1ming2}{11,8}{⽔,⽇}[HSK 7-9]
  \definition*{s.}{Festival de Qingming (nos dias 4, 5 e 6 de abril); é costume limpar os túmulos nos dias 4, 5 ou 6 de abril, de acordo com a tradição popular}
  \definition{adj.}{limpo e em pé; (política) é legal e ordeira | calmo; sereno; (mental) claro e calmo | claro; brilhante}
\end{EntryWithPhonetic}

\begin{EntryWithPhonetic}{清明节}{qing1 ming2 jie2}{11,8,5}{⽔,⽇,⾋}[HSK 6]
  \definition*{s.}{Qingming ou Festival do Brilho Puro ou Dia da Varredura de Túmulos, Dia dos Finados (uma das 24~divisões do ano solar no calendário lunar chinês:~dia~4 ou 5~de~abril solar)}
\end{EntryWithPhonetic}

\begin{EntryWithPhonetic}{清爽}{qing1shuang3}{11,11}{⽔,⽘}
  \definition{adj.}{refrescante | relaxado}
\end{EntryWithPhonetic}

\begin{EntryWithPhonetic}{清晰}{qing1xi1}{11,12}{⽔,⽇}[HSK 7-9]
  \definition{adj.}{claro; distinto; consegue ver e ouvir com clareza, compreender com clareza; tem um processo de pensamento claro}
\end{EntryWithPhonetic}

\begin{EntryWithPhonetic}{清洗}{qing1 xi3}{11,9}{⽔,⽔}[HSK 6]
  \definition{v.}{enxaguar; lavar; limpar | purgar; limpar | eliminar}
\end{EntryWithPhonetic}

\begin{EntryWithPhonetic}{清新}{qing1xin1}{11,13}{⽔,⽄}[HSK 7-9]
  \definition{adj.}{fresco; puro e fresco; fresco e limpo; fresco e refrescante | (estilo) romance; original; inovador e único}
\end{EntryWithPhonetic}

\begin{EntryWithPhonetic}{清醒}{qing1xing3}{11,16}{⽔,⾣}[HSK 4]
  \definition{adj.}{sóbrio; lúcido}
  \definition{v.}{recuperar a consciência; recuperar-se de um coma}
\end{EntryWithPhonetic}

\begin{EntryWithPhonetic}{清真寺}{qing1zhen1si4}{11,10,6}{⽔,⼗,⼨}[HSK 7-9]
  \definition[所,座,个]{s.}{mesquita; as mesquitas islâmicas também são chamadas de salas de oração}
\end{EntryWithPhonetic}

%%%%%%%%%% 蜻 %%%%%%%%%%
\subsection*{蜻}\addcontentsline{loh}{figure}{蜻 \dpy{qing1}}

\begin{EntryWithPhonetic}{蜻}{qing1}{14}{⾍}
  \definition[只]{s.}{libélula, 蜻蜓}
  \seealsoref{蜻蜓}{qing1ting2}
\end{EntryWithPhonetic}

\begin{EntryWithPhonetic}{蜻蜓}{qing1ting2}{14,12}{⾍,⾍}
  \definition{s.}{libélula}
\end{EntryWithPhonetic}

\begin{EntryWithPhonetic}{蜻蝏}{qing1ting2}{14,15}{⾍,⾍}
  \variantof{蜻蜓}
\end{EntryWithPhonetic}

%%%%%%%%%% 情 %%%%%%%%%%
\subsection*{情}\addcontentsline{loh}{figure}{情 \dpy{qing2}}

\begin{EntryWithPhonetic}{情}{qing2}{11}{⼼}[HSK 7-9]
  \definition{s.}{sentimento; afeição | amor; paixão | paixão sexual; luxúria | favor; gentileza | situação; circunstâncias; condição | razão; sentido | sensibilidades; sentimentos}
\end{EntryWithPhonetic}

\begin{EntryWithPhonetic}{情报}{qing2bao4}{11,7}{⼼,⼿}[HSK 7-9]
  \definition[个,份]{s.}{inteligência; informação; notícias e reportagens sobre determinada situação são frequentemente classificadas como confidenciais}
\end{EntryWithPhonetic}

\begin{EntryWithPhonetic}{情不自禁}{qing2bu2zi4jin1}{11,4,6,13}{⼼,⼀,⾃,⽰}[HSK 7-9]
  \definition{expr.}{``Não consigo ajudar.''; não conseguir se conter; não conseguir evitar (fazer algo); ser tomado por um impulso repentino de; dominado pela emoção; enfatizando o controle total sobre as próprias emoções}
\end{EntryWithPhonetic}

\begin{EntryWithPhonetic}{情调}{qing2diao4}{11,10}{⼼,⾔}[HSK 7-9]
  \definition[出]{s.}{sentimento; apelo emocional; tom afetivo; o estilo expresso através de pensamentos e sentimentos; a natureza das coisas que podem evocar diversas emoções diferentes nas pessoas}
\end{EntryWithPhonetic}

\begin{EntryWithPhonetic}{情感}{qing2 gan3}{11,13}{⼼,⼼}[HSK 3]
  \definition[份]{s.}{emoção; sentimento | afeição; apego; reações psicológicas positivas ou negativas a estímulos externos, como gosto, raiva, tristeza, medo, amor, nojo, etc.}
\end{EntryWithPhonetic}

\begin{EntryWithPhonetic}{情怀}{qing2huai2}{11,7}{⼼,⼼}[HSK 7-9]
  \definition{s.}{sentimentos; um estado de espírito que contém uma determinada emoção}
\end{EntryWithPhonetic}

\begin{EntryWithPhonetic}{情节}{qing2jie2}{11,5}{⼼,⾋}[HSK 5]
  \definition[个,段]{s.}{enredo; trama; desenrolar específico dos acontecimentos | circunstância; detalhes do crime ou erro | enredo; roteiro; refere-se especificamente ao processo de desenvolvimento e evolução dos conflitos e contradições em obras literárias narrativas}
\end{EntryWithPhonetic}

\begin{EntryWithPhonetic}{情结}{qing2jie2}{11,9}{⼼,⽷}[HSK 7-9]
  \definition{s.}{complexidade; a turbulência emocional em meu coração; um certo sentimento que frequentemente persiste em minha mente}
\end{EntryWithPhonetic}

\begin{EntryWithPhonetic}{情景}{qing2jing3}{11,12}{⼼,⽇}[HSK 4]
  \definition[个,幕,种]{s.}{cena; vista; circunstâncias}
\end{EntryWithPhonetic}

\begin{EntryWithPhonetic}{情况}{qing2kuang4}{11,7}{⼼,⼎}[HSK 3]
  \definition[种,个,些]{s.}{condição; situação; circunstâncias; estado das coisas | mudanças notáveis e impactantes}
\end{EntryWithPhonetic}

\begin{EntryWithPhonetic}{情侣}{qing2lv3}{11,8}{⼼,⼈}[HSK 7-9]
  \definition[对,双,群]{s.}{amantes; namorados; um casal apaixonado ou um deles}
\end{EntryWithPhonetic}

\begin{EntryWithPhonetic}{情人}{qing2ren2}{11,2}{⼼,⼈}[HSK 7-9]
  \definition[对,个,位]{s.}{amante; namorado(a) | concubina; amante}
\end{EntryWithPhonetic}

\begin{EntryWithPhonetic}{情形}{qing2xing2}{11,7}{⼼,⼺}[HSK 5]
  \definition[个,种]{s.}{situação; condição; circunstâncias; estado de coisas; a situação específica das coisas}
\end{EntryWithPhonetic}

\begin{EntryWithPhonetic}{情绪}{qing2xu4}{11,11}{⼼,⽷}[HSK 6]
  \definition[种,片,股,丝]{s.}{mau humor; depressão; um sentimento ruim no coração, especialmente um estado mental desagradável quando se sente injusto | emoção; humor; moral; sentimento; o estado mental de uma pessoa ao longo de um período de tempo}
\end{EntryWithPhonetic}

\begin{EntryWithPhonetic}{情谊}{qing2yi4}{11,10}{⼼,⾔}[HSK 7-9]
  \definition{s.}{amizade; sentimentos amigáveis; emoções amistosas; os sentimentos de carinho e amor entre pessoas}
\end{EntryWithPhonetic}

\begin{EntryWithPhonetic}{情愿}{qing2yuan4}{11,14}{⼼,⽕}[HSK 7-9]
  \definition{v.}{estar disposto a; estar genuinamente disposto a fazer algo que outros não estão dispostos a fazer}
\end{EntryWithPhonetic}

%%%%%%%%%% 晴 %%%%%%%%%%
\subsection*{晴}\addcontentsline{loh}{figure}{晴 \dpy{qing2}}

\begin{EntryWithPhonetic}{晴}{qing2}{12}{⽇}[HSK 2]
  \definition{adj.}{ensolarado; bom; claro; não há nuvens no céu ou há poucas nuvens}
\end{EntryWithPhonetic}

\begin{EntryWithPhonetic}{晴朗}{qing2lang3}{12,10}{⽇,⽉}[HSK 5]
  \definition{adj.}{bom; claro; ensolarado; céu limpo e sem nuvens}
\end{EntryWithPhonetic}

\begin{EntryWithPhonetic}{晴天}{qing2 tian1}{12,4}{⽇,⼤}[HSK 2]
  \definition[个]{s.}{dia ensolarado; tempo sem nuvens ou com poucas nuvens; em meteorologia, refere-se a um tempo em que a cobertura de nuvens no céu é inferior a 10\%}
\end{EntryWithPhonetic}

%%%%%%%%%% 请 %%%%%%%%%%
\subsection*{请}\addcontentsline{loh}{figure}{请 \dpy{qing3}}

\begin{EntryWithPhonetic}{请}{qing3}{10}{⾔}[HSK 1]
  \definition*{s.}{Sobrenome: Qing}
  \definition{v.}{solicitar; perguntar | convidar; envolver | por favor; uma expressão educada usada quando você quer que alguém faça algo | comprar coisas sagradas para sacrifício, como incenso, velas, cavalos de papel e santuários de Buda; superstição se refere à compra de estátuas de Buda, santuários, etc. | entreter}
\end{EntryWithPhonetic}

\begin{EntryWithPhonetic}{请假}{qing3/jia4}{10,11}{⾔,⼈}[HSK 1]
  \definition{v.+compl.}{pedir licença para sair; solicitar permissão para não trabalhar ou estudar por um determinado período de tempo devido a doença ou outros motivos}
\end{EntryWithPhonetic}

\begin{EntryWithPhonetic}{请假条}{qing3jia4tiao2}{10,11,7}{⾔,⼈,⽊}
  \definition{s.}{pedido de licença de ausência (do trabalho ou da escola)}
\end{EntryWithPhonetic}

\begin{EntryWithPhonetic}{请柬}{qing3jian3}{10,9}{⾔,⽊}[HSK 7-9]
  \definition[封,张,份]{s.}{cartão de convite; convite}
\end{EntryWithPhonetic}

\begin{EntryWithPhonetic}{请教}{qing3jiao4}{10,11}{⾔,⽁}[HSK 3]
  \definition{v.}{consultar; pedir conselho}
\end{EntryWithPhonetic}

\begin{EntryWithPhonetic}{请进}{qing3 jin4}{10,7}{⾔,⾡}[HSK 1]
  \definition{v.}{por favor entre; convidar alguém para um espaço ou lugar}
\end{EntryWithPhonetic}

\begin{EntryWithPhonetic}{请客}{qing3/ke4}{10,9}{⾔,⼧}[HSK 2]
  \definition{v.+compl.}{receber convidados; hospedar convidados | oferecer; convidar; pagar a conta; arcar com os custos; convidar alguém para comer, tomar chá, etc.}
\end{EntryWithPhonetic}

\begin{EntryWithPhonetic}{请求}{qing3qiu2}{10,7}{⾔,⽔}[HSK 2]
  \definition[个,次]{s.}{pedido; petição; solicitação; refere-se à exigência apresentada}
  \definition{v.}{pedir; solicitar; requerer; peticionar; fazer uma solicitação e pedir que a outra parte concorde com ela}
\end{EntryWithPhonetic}

\begin{EntryWithPhonetic}{请帖}{qing3tie3}{10,8}{⾔,⼱}[HSK 7-9]
  \definition[张,份]{s.}{cartão; convite; cartão de convite; notificação enviada ao convidar convidados}
\end{EntryWithPhonetic}

\begin{EntryWithPhonetic}{请问}{qing3 wen4}{10,6}{⾔,⾨}[HSK 1]
  \definition{expr.}{Com licença, posso perguntar\dots? (para perguntar por qualquer coisa); uma maneira educada de pedir para alguém responder a uma pergunta}
\end{EntryWithPhonetic}

\begin{EntryWithPhonetic}{请坐}{qing3 zuo4}{10,7}{⾔,⼟}[HSK 1]
  \definition{v.}{por favor, sente-se; convidar outras pessoas para sentar ou descansar}
\end{EntryWithPhonetic}

%%%%%%%%%% 庆 %%%%%%%%%%
\subsection*{庆}\addcontentsline{loh}{figure}{庆 \dpy{qing4}}

\begin{EntryWithPhonetic}{庆}{qing4}{6}{⼴}
  \definition*{s.}{Sobrenome: Qing}
  \definition{s.}{celebração | ocasião para celebração; um aniversário que vale a pena comemorar}
  \definition{v.}{celebrar; felicitar; comemorar}
\end{EntryWithPhonetic}

\begin{EntryWithPhonetic}{庆典}{qing4dian3}{6,8}{⼴,⼋}[HSK 7-9]
  \definition{s.}{cerimônia; celebração; uma cerimônia de celebração muito grandiosa}
\end{EntryWithPhonetic}

\begin{EntryWithPhonetic}{庆贺}{qing4he4}{6,9}{⼴,⾙}[HSK 7-9]
  \definition{v.}{parabenizar; celebrar; celebrar uma ocasião alegre compartilhada ou parabenizar alguém que está recebendo boas notícias}
\end{EntryWithPhonetic}

\begin{EntryWithPhonetic}{庆幸}{qing4xing4}{6,8}{⼴,⼲}[HSK 7-9]
  \definition{v.}{alegrar-se; ficar contente; ficar feliz por uma situação inesperadamente boa}
\end{EntryWithPhonetic}

\begin{EntryWithPhonetic}{庆祝}{qing4zhu4}{6,9}{⼴,⽰}[HSK 3]
  \definition{v.}{celebrar; comemorar; festejar; realizar atividades para comemorar ou celebrar festivais comuns e eventos felizes}
\end{EntryWithPhonetic}

%%%%%%%%%% 亲 %%%%%%%%%%
\subsection*{亲}\addcontentsline{loh}{figure}{亲 \dpy{qing4}}

\begin{EntryWithPhonetic}{亲}{qing4}{9}{⼇}
  \definition{s.}{parentes por afinidade; parentes por casamento}
  \seeref{qin1}
\end{EntryWithPhonetic}

%%%%%%%%%% 穷 %%%%%%%%%%
\subsection*{穷}\addcontentsline{loh}{figure}{穷 \dpy{qiong2}}

\begin{EntryWithPhonetic}{穷}{qiong2}{7}{⽳}[HSK 4]
  \definition{adj.}{remoto; isolado; de difícil acesso | pobre; atingido pela pobreza | situação difícil, sem saída}
  \definition{adv.}{completamente | extremamente}
  \definition{v.}{exaurir; esgotar; consmir | ir até o fim; perseguir completamente perseguido; sondar profundamente | gastar}
\end{EntryWithPhonetic}

\begin{EntryWithPhonetic}{穷人}{qiong2 ren2}{7,2}{⽳,⼈}[HSK 4]
  \definition[个]{s.}{os pobres; pessoas pobres}
\end{EntryWithPhonetic}

%%%%%%%%%% 丘 %%%%%%%%%%
\subsection*{丘}\addcontentsline{loh}{figure}{丘 \dpy{qiu1}}

\begin{EntryWithPhonetic}{丘}{qiu1}{5}{⼀}
  \definition*{s.}{Sobrenome: Qiu}
  \definition[个]{s.}{monte; outeiro | (literário) sepultura}
\end{EntryWithPhonetic}

\begin{EntryWithPhonetic}{丘陵}{qiu1ling2}{5,10}{⼀,⾩}[HSK 7-9]
  \definition[个,片]{s.}{colinas; colinas baixas contínuas}
\end{EntryWithPhonetic}

%%%%%%%%%% 秋 %%%%%%%%%%
\subsection*{秋}\addcontentsline{loh}{figure}{秋 \dpy{qiu1}}

\begin{EntryWithPhonetic}{秋}{qiu1}{9}{⽲}
  \definition*{s.}{Sobrenome: Qiu}
  \definition{s.}{outono | época da colheita; a estação em que as colheitas amadurecem; colheitas maduras no outono | ano; refere-se a um ano | um período de tempo (geralmente conturbado)}
\end{EntryWithPhonetic}

\begin{EntryWithPhonetic}{秋季}{qiu1 ji4}{9,8}{⽲,⼦}[HSK 4]
  \definition[个]{s.}{outono; terceiro trimestre do ano, segundo o costume chinês, refere-se ao período de três meses entre o outono e o inverno, também se refere aos sétimo, oitavo e nono meses do calendário lunar}
\end{EntryWithPhonetic}

\begin{EntryWithPhonetic}{秋天}{qiu1 tian1}{9,4}{⽲,⼤}[HSK 2]
  \definition[个,段,季,番]{s.}{outono}
\end{EntryWithPhonetic}

%%%%%%%%%% 仇 %%%%%%%%%%
\subsection*{仇}\addcontentsline{loh}{figure}{仇 \dpy{qiu2}}

\begin{EntryWithPhonetic}{仇}{qiu2}{4}{⼈}
  \definition*{s.}{Sobrenome: Qiu}
  \definition{s.}{Literário: cônjuge; esposa; companheira}
  \seeref{chou2}
\end{EntryWithPhonetic}

%%%%%%%%%% 囚 %%%%%%%%%%
\subsection*{囚}\addcontentsline{loh}{figure}{囚 \dpy{qiu2}}

\begin{EntryWithPhonetic}{囚}{qiu2}{5}{⼞}
  \definition[个,群,位,名,些,批]{s.}{prisioneiro; condenado}
  \definition{v.}{aprisionar}
\end{EntryWithPhonetic}

\begin{EntryWithPhonetic}{囚犯}{qiu2fan4}{5,5}{⼞,⽝}[HSK 7-9]
  \definition[名]{s.}{prisioneiro; condenado}
\end{EntryWithPhonetic}

%%%%%%%%%% 求 %%%%%%%%%%
\subsection*{求}\addcontentsline{loh}{figure}{求 \dpy{qiu2}}

\begin{EntryWithPhonetic}{求}{qiu2}{7}{⽔}[HSK 2]
  \definition*{s.}{Sobrenome: Qiu}
  \definition{v.}{implorar; solicitar; suplicar; rogar | lutar por; buscar; investigar | tentar; procurar; tentar obter | demandar}
\end{EntryWithPhonetic}

\begin{EntryWithPhonetic}{求婚}{qiu2/hun1}{7,11}{⽔,⼥}[HSK 7-9]
  \definition{v.+compl.}{propor; fazer uma oferta de casamento; pedir em casamento}
\end{EntryWithPhonetic}

\begin{EntryWithPhonetic}{求救}{qiu2jiu4}{7,11}{⽔,⽁}[HSK 7-9]
  \definition{v.}{pedir socorro; solicitar que alguém venha em socorro; solicitar ajuda (geralmente usado em situações de desastre e perigo)}
\end{EntryWithPhonetic}

\begin{EntryWithPhonetic}{求学}{qiu2xue2}{7,8}{⽔,⼦}[HSK 7-9]
  \definition{v.}{estudar; estudar na escola | buscar conhecimento; dedicar-se aos estudos; explorar o conhecimento}
\end{EntryWithPhonetic}

\begin{EntryWithPhonetic}{求医}{qiu2yi1}{7,7}{⽔,⼖}[HSK 7-9]
  \definition{v.}{consultar um médico | procurar tratamento médico}
\end{EntryWithPhonetic}

\begin{EntryWithPhonetic}{求证}{qiu2zheng4}{7,7}{⽔,⾔}[HSK 7-9]
  \definition{v.}{procurar provar; procurar evidências (ou verificação) | buscar confirmação | buscar provas}
\end{EntryWithPhonetic}

\begin{EntryWithPhonetic}{求职}{qiu2 zhi2}{7,11}{⽔,⽿}[HSK 6]
  \definition{v.}{procurar emprego; candidatar-se a um emprego; encontrar um emprego}
\end{EntryWithPhonetic}

\begin{EntryWithPhonetic}{求助}{qiu2zhu4}{7,7}{⽔,⼒}[HSK 7-9]
  \definition{v.}{recorrer a alguém em busca de ajuda; pedir ajuda; solicitar assistência}
\end{EntryWithPhonetic}

%%%%%%%%%% 球 %%%%%%%%%%
\subsection*{球}\addcontentsline{loh}{figure}{球 \dpy{qiu2}}

\begin{EntryWithPhonetic}{球}{qiu2}{11}{⽟}[HSK 1]
  \definition[个,颗,筐]{s.}{esfera; globo; equipamento de jogo antigo, objeto tridimensional circular, feito de couro, recheado com penas, para ser chutado com os pés ou batido com um bastão | qualquer coisa com formato de bola; algo esférico ou quase esférico | bola; refere-se a certos artigos esportivos (geralmente redondos e tridimensionais) | jogo; partida; referência a esportes com bola | o Globo; a Terra; referindo-se especificamente à Terra}
\end{EntryWithPhonetic}

\begin{EntryWithPhonetic}{球场}{qiu2 chang3}{11,6}{⽟,⼟}[HSK 2]
  \definition[个,座]{s.}{quadra; campo; terreno para jogos com bola; campos para a prática de esportes com bola, como basquete, futebol, tênis e vôlei, cuja forma, tamanho e equipamentos variam de acordo com as exigências de cada esporte}
\end{EntryWithPhonetic}

\begin{EntryWithPhonetic}{球队}{qiu2 dui4}{11,4}{⽟,⾩}[HSK 2]
  \definition[个,支]{s.}{equipe (basquete, futebol, etc.); equipe de atletas formada para competições esportivas com bola, como times de basquete, futebol, etc.}
\end{EntryWithPhonetic}

\begin{EntryWithPhonetic}{球迷}{qiu2mi2}{11,9}{⽟,⾡}[HSK 3]
  \definition[个,位,名,些]{s.}{fã (de esportes de bola); pessoas obcecadas por jogar ou assistir jogos de bola}
\end{EntryWithPhonetic}

\begin{EntryWithPhonetic}{球拍}{qiu2 pai1}{11,8}{⽟,⼿}[HSK 6]
  \definition[支]{s.}{(tênis, badminton, etc.) raquete}
\end{EntryWithPhonetic}

\begin{EntryWithPhonetic}{球鞋}{qiu2 xie2}{11,15}{⽟,⾰}[HSK 2]
  \definition[双,只,款]{s.}{tênis de ginástica; tênis de tênis; tênis esportivos}
\end{EntryWithPhonetic}

\begin{EntryWithPhonetic}{球星}{qiu2 xing1}{11,9}{⽟,⽇}[HSK 6]
  \definition[位,名]{s.}{estrela do esporte (esporte com bola)}
\end{EntryWithPhonetic}

\begin{EntryWithPhonetic}{球衣}{qiu2yi1}{11,6}{⽟,⾐}
  \definition{s.}{uniforme, roupa (de uma equipe específica); camisa; camisa polo}
\end{EntryWithPhonetic}

\begin{EntryWithPhonetic}{球员}{qiu2 yuan2}{11,7}{⽟,⼝}[HSK 6]
  \definition[名,位,个]{s.}{Esporte: jogador | membro do clube esportivo}
\end{EntryWithPhonetic}

%%%%%%%%%% 区 %%%%%%%%%%
\subsection*{区}\addcontentsline{loh}{figure}{区 \dpy{qu1}}

\begin{EntryWithPhonetic}{区}{qu1}{4}{⼖}[HSK 3]
  \definition{s.}{área; distrito; região; zona; uma determinada área em terra, água ou ar | uma divisão administrativa; as divisões administrativas incluem regiões autônomas étnicas de nível provincial, distritos municipais e de condado e distritos de condado; grandes regiões administrativas, regiões, zonas especiais e regiões administrativas especiais}
  \definition{v.}{classificar; subdividir; distinguir}
  \seeref{ou1}
\end{EntryWithPhonetic}

\begin{EntryWithPhonetic}{区别}{qu1bie2}{4,7}{⼖,⼑}[HSK 3]
  \definition[种,个]{s.}{diferença; distinção; discriminação}
  \definition{v.}{distinguir; diferenciar; fazer distinção entre}
\end{EntryWithPhonetic}

\begin{EntryWithPhonetic}{区分}{qu1fen1}{4,4}{⼖,⼑}[HSK 6]
  \definition{v.}{discriminar; diferenciar; distinguir; comparar dois ou mais objetos; reconhecer suas diferenças}
\end{EntryWithPhonetic}

\begin{EntryWithPhonetic}{区域}{qu1yu4}{4,11}{⼖,⼟}[HSK 5]
  \definition[片,块,个]{s.}{área; setor; região; faixa; inclui áreas regionais com condições naturais, culturais, administrativas, etc.}
\end{EntryWithPhonetic}

%%%%%%%%%% 曲 %%%%%%%%%%
\subsection*{曲}\addcontentsline{loh}{figure}{曲 \dpy{qu1}}

\begin{EntryWithPhonetic}{曲}{qu1}{6}{⽈}
  \definition*{s.}{Sobrenome: Qu}
  \definition{adj.}{dobrado; curvo; sinuoso; oposto a 直 | errado; injustificável}
  \definition{s.}{curva (de um rio, etc.) | fermento; levedura}
  \definition{v.}{dobrar; torcer}
  \seeref{qu3}
  \seealsoref{直}{zhi2}
\end{EntryWithPhonetic}

\begin{EntryWithPhonetic}{曲棍球}{qu1gun4qiu2}{6,12,11}{⽈,⽊,⽟}
  \definition{s.}{hóquei em campo; hóquei | bola de hóquei}
\end{EntryWithPhonetic}

\begin{EntryWithPhonetic}{曲线}{qu1xian4}{6,8}{⽈,⽷}[HSK 7-9]
  \definition[条]{s.}{curva; em geometria, refere-se à trajetória de um ponto que se move sob certas condições em um plano ou no espaço | corpo curvado; linhas onduladas; também se refere às linhas do corpo humano}
\end{EntryWithPhonetic}

\begin{EntryWithPhonetic}{曲折}{qu1zhe2}{6,7}{⽈,⼿}[HSK 7-9]
  \definition{adj.}{sinuoso; tortuoso; não reto | complicado; intrincado; a situação e o enredo são complexos}
  \definition{s.}{complicações; enredo complexo e frustrante}
\end{EntryWithPhonetic}

%%%%%%%%%% 驱 %%%%%%%%%%
\subsection*{驱}\addcontentsline{loh}{figure}{驱 \dpy{qu1}}

\begin{EntryWithPhonetic}{驱}{qu1}{7}{⾺}
  \definition{v.}{dirigir (um cavalo, um carro, etc.) | expulsar; dispersar | correr rápido}
\end{EntryWithPhonetic}

\begin{EntryWithPhonetic}{驱动}{qu1dong4}{7,6}{⾺,⼒}[HSK 7-9]
  \definition{v.}{acionar; alimentar; ser movido | acionar; atuar; dirigir; ser motivado}
\end{EntryWithPhonetic}

\begin{EntryWithPhonetic}{驱逐}{qu1zhu2}{7,10}{⾺,⾡}[HSK 7-9]
  \definition{v.}{expulsar; deportar; banir; expelir}
\end{EntryWithPhonetic}

%%%%%%%%%% 屈 %%%%%%%%%%
\subsection*{屈}\addcontentsline{loh}{figure}{屈 \dpy{qu1}}

\begin{EntryWithPhonetic}{屈}{qu1}{8}{⼫}
  \definition*{s.}{Sobrenome: Qu}
  \definition[个]{s.}{injustiça; tratamento injusto | erro; queixa; injustiça}
  \definition{v.}{dobrar; curvar; encurvar | subjugar; submeter | tratar mal; tratar injustamente (ou deslealmente) | estar errado}
\end{EntryWithPhonetic}

\begin{EntryWithPhonetic}{屈服}{qu1fu2}{8,8}{⼫,⽉}[HSK 7-9]
  \definition{v.}{subjugar; submeter-se; ceder; dobrar-se; ceder e recuar diante da pressão externa, desistir da luta}
\end{EntryWithPhonetic}

\begin{EntryWithPhonetic}{屈原}{qu1yuan2}{8,10}{⼫,⼚}
  \definition*{s.}{Qu Yuan, poeta, é uma figura histórica famosa na cultura chinesa que viveu durante o Período dos Reinos Combatentes (340-278 a.C.).}
\end{EntryWithPhonetic}

%%%%%%%%%% 趋 %%%%%%%%%%
\subsection*{趋}\addcontentsline{loh}{figure}{趋 \dpy{qu1}}

\begin{EntryWithPhonetic}{趋}{qu1}{12}{⾛}
  \definition{v.}{apressar-se | tender para; tender a se tornar | (ganso, cobra, etc.) estalar a cabeça e morder as pessoas}
\end{EntryWithPhonetic}

\begin{EntryWithPhonetic}{趋势}{qu1shi4}{12,8}{⾛,⼒}[HSK 4]
  \definition{s.}{rumo; tendência; direção; impulso das coisas que se movem em uma direção ou outra}
\end{EntryWithPhonetic}

\begin{EntryWithPhonetic}{趋于}{qu1yu2}{12,3}{⾛,⼆}[HSK 7-9]
  \definition{v.}{tender a}
\end{EntryWithPhonetic}

%%%%%%%%%% 渠 %%%%%%%%%%
\subsection*{渠}\addcontentsline{loh}{figure}{渠 \dpy{qu2}}

\begin{EntryWithPhonetic}{渠}{qu2}{11}{⽊}
  \definition*{s.}{Sobrenome: Qu}
  \definition{adj.}{Literário: grande}
  \definition{pron.}{Dialeto: ele; ela}
  \definition[条]{s.}{canal; vala; fosso; trincheira | borda externa da roda | escudo}
\end{EntryWithPhonetic}

\begin{EntryWithPhonetic}{渠道}{qu2dao4}{11,12}{⽊,⾡}[HSK 6]
  \definition[条,个,种]{s.}{vala de irrigação; os cursos de água escavados pelos trabalhadores para drenagem e irrigação | maneira; meio; caminho}
\end{EntryWithPhonetic}

%%%%%%%%%% 曲 %%%%%%%%%%
\subsection*{曲}\addcontentsline{loh}{figure}{曲 \dpy{qu3}}

\begin{EntryWithPhonetic}{曲}{qu3}{6}{⽈}[HSK 7-9]
  \definition{s.}{canção; melodia; partitura}
  \seeref{qu1}
\end{EntryWithPhonetic}

%%%%%%%%%% 取 %%%%%%%%%%
\subsection*{取}\addcontentsline{loh}{figure}{取 \dpy{qu3}}

\begin{EntryWithPhonetic}{取}{qu3}{8}{⼜}[HSK 2]
  \definition{v.}{pegar; obter; buscar; pegar de um lugar; pegar nas mãos | visar; procurar; obter; provocar | adotar; assumir; escolher; selecionar}
\end{EntryWithPhonetic}

\begin{EntryWithPhonetic}{取代}{qu3dai4}{8,5}{⼜,⼈}[HSK 7-9]
  \definition{v.}{deslocar; substituir; suplantar; substituir por; assumir o controle; tomar o lugar de}
\end{EntryWithPhonetic}

\begin{EntryWithPhonetic}{取得}{qu3 de2}{8,11}{⼜,⼻}[HSK 2]
  \definition{v.}{ganhar; adquirir; obter; ser o primeiro a conseguir}
\end{EntryWithPhonetic}

\begin{EntryWithPhonetic}{取缔}{qu3di4}{8,12}{⼜,⽷}[HSK 7-9]
  \definition{v.}{proibir; criminalizar; suprimir; cancelar, encerrar ou proibir explicitamente}
\end{EntryWithPhonetic}

\begin{EntryWithPhonetic}{取而代之}{qu3'er2dai4zhi1}{8,6,5,3}{⼜,⽽,⼈,⼂}[HSK 7-9]
  \definition{expr.}{substituir alguém; suplantar alguém; tomar o lugar de alguém ou de algo; assumir o controle}
\end{EntryWithPhonetic}

\begin{EntryWithPhonetic}{取经}{qu3/jing1}{8,8}{⼜,⽷}[HSK 7-9]
  \definition{v.+compl.}{fazer uma peregrinação em busca de escrituras budistas | buscar experiência; aprender com a experiência de outra pessoa}
\end{EntryWithPhonetic}

\begin{EntryWithPhonetic}{取决于}{qu3jue2 yu2}{8,6,3}{⼜,⼎,⼆}[HSK 7-9]
  \definition{v.}{depender de; ser determinado por (algo)}
\end{EntryWithPhonetic}

\begin{EntryWithPhonetic}{取款}{qu3kuan3}{8,12}{⼜,⽋}[HSK 6]
  \definition{v.}{sacar dinheiro (de um banco); retirar o dinheiro que você depositou (geralmente se refere a retirar dinheiro do banco)}
\end{EntryWithPhonetic}

\begin{EntryWithPhonetic}{取款机}{qu3 kuan3 ji1}{8,12,6}{⼜,⽋,⽊}[HSK 6]
  \definition{s.}{ATM; caixa eletrônico; um caixa eletrônico é uma máquina que pode concluir automaticamente operações bancárias, como saques e consultas de saldo}
\end{EntryWithPhonetic}

\begin{EntryWithPhonetic}{取暖}{qu3nuan3}{8,13}{⼜,⽇}[HSK 7-9]
  \definition{v.}{aquecer-se; utilizar a energia térmica para aquecer o corpo}
\end{EntryWithPhonetic}

\begin{EntryWithPhonetic}{取胜}{qu3sheng4}{8,9}{⼜,⾁}[HSK 7-9]
  \definition{v.}{obter a vitória; alcançar o sucesso; alcançar a vitória}
\end{EntryWithPhonetic}

\begin{EntryWithPhonetic}{取水}{qu3shui3}{8,4}{⼜,⽔}
  \definition{v.}{obter água (de um poço, etc.)}
\end{EntryWithPhonetic}

\begin{EntryWithPhonetic}{取现}{qu3xian4}{8,8}{⼜,⾒}
  \definition{v.}{sacar dinheiro}
\end{EntryWithPhonetic}

\begin{EntryWithPhonetic}{取消}{qu3xiao1}{8,10}{⼜,⽔}[HSK 3]
  \definition{v.}{cancelar; suspender; anular; abolir; revogar; rescindir; tornar o sistema original, regulamentos, qualificações, direitos, etc. inválidos}
\end{EntryWithPhonetic}

\begin{EntryWithPhonetic}{取笑}{qu3xiao4}{8,10}{⼜,⽵}[HSK 7-9]
  \definition{v.}{ridicularizar; zombar de; fazer alarde de; buscar diversão}
\end{EntryWithPhonetic}

\begin{EntryWithPhonetic}{取悦}{qu3yue4}{8,10}{⼜,⼼}
  \definition{v.}{tentar agradar}
\end{EntryWithPhonetic}

%%%%%%%%%% 娶 %%%%%%%%%%
\subsection*{娶}\addcontentsline{loh}{figure}{娶 \dpy{qu3}}

\begin{EntryWithPhonetic}{娶}{qu3}{11}{⼥}[HSK 7-9]
  \definition{v.}{casar (com uma mulher); tomar por esposa}
\end{EntryWithPhonetic}

%%%%%%%%%% 厺 %%%%%%%%%%
\subsection*{厺}\addcontentsline{loh}{figure}{厺 \dpy{qu4}}

\begin{EntryWithPhonetic}{厺}{qu4}{5}{⼤}
  \variantof{去}
\end{EntryWithPhonetic}

%%%%%%%%%% 去 %%%%%%%%%%
\subsection*{去}\addcontentsline{loh}{figure}{去 \dpy{qu4}}

\begin{EntryWithPhonetic}{去}{qu4}{5}{⼛}[HSK 1]
  \definition{adj.}{passado; último; refere-se ao tempo passado (um ano)}
  \definition{adv.}{muito; extremamente; usado depois de adjetivos como 大, 多 e 远, significa 极 ou 非常}
  \definition{s.}{tom descendente, um dos quatro tons do chinês clássico e o quarto tom na pronúncia padrão do chinês moderno}
  \definition{v.}{ir; partir; sair | estar separado de | perder | remover; livrar-se de | ir (a algum lugar) para fazer algo; sair do local onde o interlocutor se encontra para outro lugar (oposto a 来) | ir para; estar indo para (fazer algo lá); usado antes de outro verbo para indicar fazer algo | desempenhar o papel de; representar o papel de; interpretar papéis em óperas | enviar; fazer ir; despachar}
  \definition{v.aux.}{usado entre uma frase verbal (ou frase preposicional) e um verbo para indicar que o primeiro é um método ou atitude e o último é um propósito | usado depois de um verbo para indicar que a ação está longe da localização do falante}
  \seealsoref{大}{da4}
  \seealsoref{多}{duo1}
  \seealsoref{非常}{fei1chang2}
  \seealsoref{极}{ji2}
  \seealsoref{来}{lai2}
  \seealsoref{远}{yuan3}
\end{EntryWithPhonetic}

\begin{EntryWithPhonetic}{去除}{qu4chu2}{5,9}{⼛,⾩}[HSK 7-9]
  \definition{v.}{remover; eliminar; livrar-se de | desalojar; expulsar}
\end{EntryWithPhonetic}

\begin{EntryWithPhonetic}{去处}{qu4chu4}{5,5}{⼛,⼡}[HSK 7-9]
  \definition{s.}{lugar para ir; paradeiro | lugar; local; localização}
\end{EntryWithPhonetic}

\begin{EntryWithPhonetic}{去掉}{qu4 diao4}{5,11}{⼛,⼿}[HSK 6]
  \definition{v.}{livrar-se de; tirar; acabar com; abandonar; erradicar}
\end{EntryWithPhonetic}

\begin{EntryWithPhonetic}{去年}{qu4nian2}{5,6}{⼛,⼲}[HSK 1]
  \definition{s.}{ano passado}
\end{EntryWithPhonetic}

\begin{EntryWithPhonetic}{去世}{qu4shi4}{5,5}{⼛,⼀}[HSK 3]
  \definition{v.}{(usado apenas para adultos, com conotações solenes) morrer; falecer; deixar este mundo}
\end{EntryWithPhonetic}

\begin{EntryWithPhonetic}{去死}{qu4si3}{5,6}{⼛,⽍}
  \definition{interj.}{Caia morto! | Vá para o Inferno!}
\end{EntryWithPhonetic}

\begin{EntryWithPhonetic}{去向}{qu4xiang4}{5,6}{⼛,⼝}[HSK 7-9]
  \definition{s.}{direção em que alguém ou algo se moveu}
\end{EntryWithPhonetic}

%%%%%%%%%% 趣 %%%%%%%%%%
\subsection*{趣}\addcontentsline{loh}{figure}{趣 \dpy{qu4}}

\begin{EntryWithPhonetic}{趣}{qu4}{15}{⾛}
  \definition{adj.}{interessante}
  \definition{s.}{interesse; deleite; diversão; passatempo | propósito; inclinação}
\end{EntryWithPhonetic}

\begin{EntryWithPhonetic}{趣味}{qu4wei4}{15,8}{⾛,⼝}[HSK 7-9]
  \definition{adj.}{agradável; interessante; atraente; simpático}
  \definition[种]{s.}{interesse; deleite; passatempo}
\end{EntryWithPhonetic}

%%%%%%%%%% 圈 %%%%%%%%%%
\subsection*{圈}\addcontentsline{loh}{figure}{圈 \dpy{quan1}}

\begin{EntryWithPhonetic}{圈}{quan1}{11}{⼞}[HSK 4]
  \definition[个]{s.}{anel; círculo; refere-se a algo em forma de anel | domínio; grupo; escopo; círculo(s)}
  \definition{v.}{cercar; rodear; circundar | marcar com um círculo}
  \seeref{juan1}
  \seeref{juan4}
\end{EntryWithPhonetic}

\begin{EntryWithPhonetic}{圈粉}{quan1fen3}{11,10}{⼞,⽶}
  \definition{s.}{(neologismo, coloquial) ganhar alguém como fã, obter novos fãs}
\end{EntryWithPhonetic}

\begin{EntryWithPhonetic}{圈套}{quan1tao4}{11,10}{⼞,⼤}[HSK 7-9]
  \definition[个]{s.}{malha; armadilha; laço; esquemas para enganar pessoas}
\end{EntryWithPhonetic}

\begin{EntryWithPhonetic}{圈子}{quan1zi5}{11,3}{⼞,⼦}[HSK 7-9]
  \definition[个]{s.}{anel; círculo; uma figura plana, redonda e oca; um objeto em forma de anel | círculo; panelinha; grupo; âmbito; refere-se ao âmbito das atividades humanas ou à área de um grupo}
\end{EntryWithPhonetic}

%%%%%%%%%% 全 %%%%%%%%%%
\subsection*{全}\addcontentsline{loh}{figure}{全 \dpy{quan2}}

\begin{EntryWithPhonetic}{全}{quan2}{6}{⼊}[HSK 2]
  \definition*{s.}{Sobrenome: Quan}
  \definition{adj.}{completo; total; inteiro}
  \definition{adv.}{inteiramente; totalmente; completamente; significa 100\%; equivalente a 完全 ou 全然}
  \definition{v.}{manter intacto; tornar perfeito ou completo; completar}
  \seealsoref{全然}{quan2ran2}
  \seealsoref{完全}{wan2quan2}
\end{EntryWithPhonetic}

\begin{EntryWithPhonetic}{全部}{quan2bu4}{6,10}{⼊,⾢}[HSK 2]
  \definition{adv.}{tudo; total; inteiro; completo; aplica-se a todos, sem exceção}
  \definition{s.}{totalidade; total; integridade; a soma de todas as partes; o todo}
\end{EntryWithPhonetic}

\begin{EntryWithPhonetic}{全长}{quan2chang2}{6,4}{⼊,⾧}[HSK 7-9]
  \definition{s.}{comprimento total | extensão; alcance}
\end{EntryWithPhonetic}

\begin{EntryWithPhonetic}{全场}{quan2 chang3}{6,6}{⼊,⼟}[HSK 3]
  \definition{s.}{toda a audiência; todos os presentes; todo o público}
\end{EntryWithPhonetic}

\begin{EntryWithPhonetic}{全称特命全权大使}{quan2cheng1 te4ming4 quan2quan2 da4shi3}{6,10,10,8,6,6,3,8}{⼊,⽲,⽜,⼝,⼊,⽊,⼤,⼈}
  \definition*{s.}{Embaixador Extraordinário e Plenipotenciário}
\end{EntryWithPhonetic}

\begin{EntryWithPhonetic}{全程}{quan2cheng2}{6,12}{⼊,⽲}[HSK 7-9]
  \definition{s.}{toda a jornada; todo o percurso}
\end{EntryWithPhonetic}

\begin{EntryWithPhonetic}{全都}{quan2 dou1}{6,10}{⼊,⾢}[HSK 5]
  \definition{adv.}{tudo; todos; sem exceção}
\end{EntryWithPhonetic}

\begin{EntryWithPhonetic}{全都不}{quan2dou1 bu4}{6,10,4}{⼊,⾢,⼀}
  \definition{adj.}{nada; nenhum; nenhum deles; nada disso}
\end{EntryWithPhonetic}

\begin{EntryWithPhonetic}{全方位}{quan2fang1wei4}{6,4,7}{⼊,⽅,⼈}[HSK 7-9]
  \definition{adj.}{versátil | em todo o redor | completo | abrangente | holístico | omnidirecional}
\end{EntryWithPhonetic}

\begin{EntryWithPhonetic}{全国}{quan2 guo2}{6,8}{⼊,⼞}[HSK 2]
  \definition{s.}{toda a nação (ou país); em todo o país; em todo o território nacional | toda a nação; todo o país}
\end{EntryWithPhonetic}

\begin{EntryWithPhonetic}{全家}{quan2 jia1}{6,10}{⼊,⼧}[HSK 2]
  \definition{s.}{toda a família; a família inteira}
\end{EntryWithPhonetic}

\begin{EntryWithPhonetic}{全局}{quan2ju2}{6,7}{⼊,⼫}[HSK 7-9]
  \definition{s.}{situação geral; situação como um todo}
\end{EntryWithPhonetic}

\begin{EntryWithPhonetic}{全力}{quan2 li4}{6,2}{⼊,⼒}[HSK 6]
  \definition{s.}{exercendo todos os seus esforços; energia ou força total; toda força ou energia}
\end{EntryWithPhonetic}

\begin{EntryWithPhonetic}{全力以赴}{quan2li4yi3fu4}{6,2,4,9}{⼊,⼒,⼈,⾛}[HSK 7-9]
  \definition{expr.}{``Dê tudo de si.''; fazer a todo custo; dar o máximo de si; prosseguir; dedicar todas as suas forças a algo}
\end{EntryWithPhonetic}

\begin{EntryWithPhonetic}{全面}{quan2mian4}{6,9}{⼊,⾯}[HSK 3]
  \definition{adj.}{geral; completo; abrangente; onipotente}
  \definition{s.}{todos os aspectos; cada aspecto}
  \seealsoref{片面}{pian4mian4}
\end{EntryWithPhonetic}

\begin{EntryWithPhonetic}{全能}{quan2neng2}{6,10}{⼊,⾁}[HSK 7-9]
  \definition{adj.}{todo-poderoso; onipotente | Esporte: versátil | pluripotente}
\end{EntryWithPhonetic}

\begin{EntryWithPhonetic}{全年}{quan2 nian2}{6,6}{⼊,⼲}[HSK 2]
  \definition{s.}{ano inteiro | anual; todo ano}
\end{EntryWithPhonetic}

\begin{EntryWithPhonetic}{全球}{quan2 qiu2}{6,11}{⼊,⽟}[HSK 3]
  \definition[门]{s.}{o mundo inteiro; a Terra inteira}
\end{EntryWithPhonetic}

\begin{EntryWithPhonetic}{全然}{quan2ran2}{6,12}{⼊,⽕}
  \definition{adv.}{completamente; inteiramente}
\end{EntryWithPhonetic}

\begin{EntryWithPhonetic}{全身}{quan2 shen1}{6,7}{⼊,⾝}[HSK 2]
  \definition{s.}{corpo inteiro; por todo o corpo; todo o corpo}
\end{EntryWithPhonetic}

\begin{EntryWithPhonetic}{全世界}{quan2 shi4 jie4}{6,5,9}{⼊,⼀,⽥}[HSK 5]
  \definition[种]{s.}{mundo inteiro; mundo todo | em todo o mundo}
\end{EntryWithPhonetic}

\begin{EntryWithPhonetic}{全体}{quan2 ti3}{6,7}{⼊,⼈}[HSK 2]
  \definition{s.}{(frequentemente referido a pessoas) todos; número total; todos | por todo o corpo | todos; inteiro; a soma de todas as partes; a soma de todos os indivíduos (geralmente se refere a pessoas)}
\end{EntryWithPhonetic}

\begin{EntryWithPhonetic}{全文}{quan2wen2}{6,4}{⼊,⽂}[HSK 7-9]
  \definition{s.}{texto completo}
\end{EntryWithPhonetic}

\begin{EntryWithPhonetic}{全心全意}{quan2xin1-quan2yi4}{6,4,6,13}{⼊,⼼,⼊,⼼}[HSK 7-9]
  \definition{expr.}{``De todo o coração.''; dedicar-se de corpo e alma a; de corpo e alma; com todo o coração}
\end{EntryWithPhonetic}

\begin{EntryWithPhonetic}{全新}{quan2 xin1}{6,13}{⼊,⽄}[HSK 6]
  \definition{adj.}{totalmente novo; inteiramente/completamente novo; refere-se a algo completamente novo, especialmente algo que não foi usado}
\end{EntryWithPhonetic}

\begin{EntryWithPhonetic}{全职}{quan2zhi2}{6,11}{⼊,⽿}
  \definition{s.}{período integral | tempo inteiro | (trabalho) \emph{full-time}}
\end{EntryWithPhonetic}

%%%%%%%%%% 权 %%%%%%%%%%
\subsection*{权}\addcontentsline{loh}{figure}{权 \dpy{quan2}}

\begin{EntryWithPhonetic}{权}{quan2}{6}{⽊}[HSK 6]
  \definition*{s.}{Sobrenome: Quan}
  \definition{adv.}{provisoriamente; por enquanto}
  \definition{s.}{Lliterário: contrapeso; peso deslizante de uma balança romana | poder; autoridade | direito | posição vantajosa | conveniência}
  \definition{v.}{pesar; medir o peso}
\end{EntryWithPhonetic}

\begin{EntryWithPhonetic}{权衡}{quan2heng2}{6,16}{⽊,⾏}[HSK 7-9]
  \definition{s.}{peso; peso de pesagem e balança de pesagem}
  \definition{v.}{pesar; equilibrar; calcular; refere-se a pesar, comparar e considerar}
\end{EntryWithPhonetic}

\begin{EntryWithPhonetic}{权力}{quan2li4}{6,2}{⽊,⼒}[HSK 6]
  \definition[种]{s.}{poder; autoridade; o poder de liderança no âmbito da responsabilidade | poder; coerção política; o poder coercitivo do status social e político}
\end{EntryWithPhonetic}

\begin{EntryWithPhonetic}{权利}{quan2li4}{6,7}{⽊,⼑}[HSK 4]
  \definition[项,种,个,条,份]{s.}{direito; interesse; os poderes e benefícios (em oposição a 义务) exercidos por um cidadão ou pessoa jurídica de acordo com a lei}
  \seealsoref{义务}{yi4wu4}
\end{EntryWithPhonetic}

\begin{EntryWithPhonetic}{权威}{quan2wei1}{6,9}{⽊,⼥}[HSK 7-9]
  \definition{adj.}{autoritativo; tem o poder e o prestígio para convencer as pessoas}
  \definition{s.}{autoridade; poder de decisão; o poder e o prestígio para inspirar a fé | autoridade; pessoa de autoridade; a pessoa ou coisa mais influente e relevante em um determinado âmbito ou área}
\end{EntryWithPhonetic}

\begin{EntryWithPhonetic}{权益}{quan2yi4}{6,10}{⽊,⽫}[HSK 7-9]
  \definition{s.}{direitos; interesses; direito legal; direitos e interesses; os direitos invioláveis ​​que devem ser desfrutados}
\end{EntryWithPhonetic}

%%%%%%%%%% 泉 %%%%%%%%%%
\subsection*{泉}\addcontentsline{loh}{figure}{泉 \dpy{quan2}}

\begin{EntryWithPhonetic}{泉}{quan2}{9}{⽔}[HSK 5]
  \definition*{s.}{Sobrenome: Quan}
  \definition[股,眼,汪]{s.}{fonte (de água mineral) | a nascente de um rio | termo antigo para moeda}
\end{EntryWithPhonetic}

%%%%%%%%%% 拳 %%%%%%%%%%
\subsection*{拳}\addcontentsline{loh}{figure}{拳 \dpy{quan2}}

\begin{EntryWithPhonetic}{拳}{quan2}{10}{⼿}[HSK 7-9]
  \definition*{s.}{Sobrenome: Quan}
  \definition[个,记,套]{s.}{punho | boxe; pugilismo}
  \definition{v.}{enrolar}
\end{EntryWithPhonetic}

\begin{EntryWithPhonetic}{拳法}{quan2fa3}{10,8}{⼿,⽔}
  \definition{s.}{boxe | luta}
\end{EntryWithPhonetic}

\begin{EntryWithPhonetic}{拳头}{quan2tou2}{10,5}{⼿,⼤}[HSK 7-9]
  \definition{adj.}{nocaute; (de produtos) de boa qualidade e competitividade; uma metáfora para ter uma vantagem e uma forte competitividade}
  \definition[个]{s.}{punho; mãos com os dedos dobrados para dentro e entrelaçados}
\end{EntryWithPhonetic}

\begin{EntryWithPhonetic}{拳王}{quan2wang2}{10,4}{⼿,⽟}
  \definition{s.}{pugilista | boxeador}
\end{EntryWithPhonetic}

%%%%%%%%%% 犬 %%%%%%%%%%
\subsection*{犬}\addcontentsline{loh}{figure}{犬 \dpy{quan3}}

\begin{EntryWithPhonetic}{犬}{quan3}{4}{⽝}[Kangxi 94]
  \definition{s.}{cachorro}
\end{EntryWithPhonetic}

%%%%%%%%%% 劝 %%%%%%%%%%
\subsection*{劝}\addcontentsline{loh}{figure}{劝 \dpy{quan4}}

\begin{EntryWithPhonetic}{劝}{quan4}{4}{⼒}[HSK 5]
  \definition*{s.}{Sobrenome: Quan}
  \definition{v.}{insistir; aconselhar; tentar persuadir; persuadir, argumentar para que as pessoas obedeçam | incentivar; encorajar}
\end{EntryWithPhonetic}

\begin{EntryWithPhonetic}{劝告}{quan4gao4}{4,7}{⼒,⼝}[HSK 7-9]
  \definition[席]{s.}{conselho; advertência; exortação; palavras ditas na esperança de que alguém corrija seus erros ou aceite conselhos}
  \definition{v.}{aconselhar; exortar; insistir; persuadir as pessoas com argumentos racionais; ajudá-las a corrigir seus erros ou a aceitar conselhos}
\end{EntryWithPhonetic}

\begin{EntryWithPhonetic}{劝说}{quan4shuo1}{4,9}{⼒,⾔}[HSK 7-9]
  \definition{v.}{aconselhar; persuadir; persuadir alguém a fazer algo ou fazer com que alguém concorde com algo}
\end{EntryWithPhonetic}

\begin{EntryWithPhonetic}{劝阻}{quan4zu3}{4,7}{⼒,⾩}[HSK 7-9]
  \definition{v.}{dissuadir alguém de; convencer alguém a não fazer algo; persuadir e parar}
\end{EntryWithPhonetic}

%%%%%%%%%% 券 %%%%%%%%%%
\subsection*{券}\addcontentsline{loh}{figure}{券 \dpy{quan4}}

\begin{EntryWithPhonetic}{券}{quan4}{8}{⼑}[HSK 6]
  \definition[张]{s.}{certificado; bilhete; ingresso; uma conta ou pedaço de papel que serve como recibo}
\end{EntryWithPhonetic}

%%%%%%%%%% 缺 %%%%%%%%%%
\subsection*{缺}\addcontentsline{loh}{figure}{缺 \dpy{que1}}

\begin{EntryWithPhonetic}{缺}{que1}{10}{⽸}[HSK 3]
  \definition{adj.}{incompleto; imperfeito}
  \definition[种]{s.}{vaga; abertura; falta}
  \definition{v.}{estar com falta de; faltar | estar ausente}
\end{EntryWithPhonetic}

\begin{EntryWithPhonetic}{缺点}{que1dian3}{10,9}{⽸,⽕}[HSK 3]
  \definition[个,些]{s.}{desvantagem; deficiência; inconveniência; ponto fraco; uma deficiência ou imperfeição (em oposição a 优点)}
  \seealsoref{优点}{you1dian3}
\end{EntryWithPhonetic}

\begin{EntryWithPhonetic}{缺乏}{que1fa2}{10,4}{⽸,⼃}[HSK 5]
  \definition{v.}{faltar; estar em falta de; não ter ou não ter totalmente (algo que deveria possuir ou é desejaria possuir)}
\end{EntryWithPhonetic}

\begin{EntryWithPhonetic}{缺口}{que1kou3}{10,3}{⽸,⼝}[HSK 7-9]
  \definition[个]{s.}{fenda; entalhe; brecha; reentrância; uma lacuna formada pela falta de uma parte de um objeto | déficit; falta de fundos, materiais, etc.; suprimentos, fundos, etc. insuficientes}
  \seealsoref{缺口儿}{que1kou3r5}
\end{EntryWithPhonetic}

\begin{EntryWithPhonetic}{缺口儿}{que1kou3r5}{10,3,2}{⽸,⼝,⼉}
  \definition{s.}{brecha; lacuna}
\end{EntryWithPhonetic}

\begin{EntryWithPhonetic}{缺勤}{que1/qin2}{10,13}{⽸,⼒}
  \definition{v.+compl.}{ausentar-se do dever (trabalho)}
\end{EntryWithPhonetic}

\begin{EntryWithPhonetic}{缺少}{que1shao3}{10,4}{⽸,⼩}[HSK 3]
  \definition{v.}{falta; estar com falta de; estar em falta de; geralmente se refere à falta de pessoas ou coisas}
\end{EntryWithPhonetic}

\begin{EntryWithPhonetic}{缺失}{que1shi1}{10,5}{⽸,⼤}[HSK 7-9]
  \definition{s.}{defeito; desvantagem; deficiência}
  \definition{v.}{perder; faltar; ter pouca}
\end{EntryWithPhonetic}

\begin{EntryWithPhonetic}{缺席}{que1/xi2}{10,10}{⽸,⼱}[HSK 7-9]
  \definition{v.+compl.}{ausentar-se; estar ausente (de uma reunião, etc.)}
\end{EntryWithPhonetic}

\begin{EntryWithPhonetic}{缺陷}{que1xian4}{10,10}{⽸,⾩}[HSK 6]
  \definition[个,处,项]{pron.}{defeito; falha; inconveniência; mancha; um lugar onde uma pessoa ou coisa está incompleta ou tem falhas porque algo está faltando}
\end{EntryWithPhonetic}

%%%%%%%%%% 却 %%%%%%%%%%
\subsection*{却}\addcontentsline{loh}{figure}{却 \dpy{que4}}

\begin{EntryWithPhonetic}{却}{que4}{7}{⼙}[HSK 4]
  \definition{adv.}{mas; contudo; no entanto; enquanto; indica um ponto de virada}
  \definition{v.}{recuar; retroceder | afastar; repelir; desencorajar | declinar; recusar; rejeitar}
  \definition{v.aux.}{usado depois de certos verbos para indicar a conclusão de uma ação, resultado, equivalente a 去 ou 掉}
  \seealsoref{掉}{diao4}
  \seealsoref{去}{qu4}
\end{EntryWithPhonetic}

\begin{EntryWithPhonetic}{却是}{que4 shi4}{7,9}{⼙,⽇}[HSK 6]
  \definition{conj.}{na verdade; no entanto; o fato é\dots; indica um ponto de virada, contrário às suas expectativas anteriores}
\end{EntryWithPhonetic}

%%%%%%%%%% 确 %%%%%%%%%%
\subsection*{确}\addcontentsline{loh}{figure}{确 \dpy{que4}}

\begin{EntryWithPhonetic}{确}{que4}{12}{⽯}
  \definition{adj.}{autenticado | sólido | firme | real | verdadeiro}
\end{EntryWithPhonetic}

\begin{EntryWithPhonetic}{确保}{que4bao3}{12,9}{⽯,⼈}[HSK 3]
  \definition{v.}{assegurar; garantir; manter ou garantir com certeza}
\end{EntryWithPhonetic}

\begin{EntryWithPhonetic}{确定}{que4ding4}{12,8}{⽯,⼧}[HSK 3]
  \definition{adj.}{definido; certo; claro}
  \definition{v.}{firmar; definir; determinar; tomar uma decisão clara e não mudar}
\end{EntryWithPhonetic}

\begin{EntryWithPhonetic}{确立}{que4li4}{12,5}{⽯,⽴}[HSK 5]
  \definition{v.}{estabelecer; criar; construir; estabelecer ou consolidar firmemente}
\end{EntryWithPhonetic}

\begin{EntryWithPhonetic}{确切}{que4qie4}{12,4}{⽯,⼑}[HSK 7-9]
  \definition{adj.}{verdadeiro; exato; definido; preciso; apropriado | verdadeiro; seguro; confiável; credível; digno de confiança}
\end{EntryWithPhonetic}

\begin{EntryWithPhonetic}{确认}{que4ren4}{12,4}{⽯,⾔}[HSK 4]
  \definition{v.}{afirmar; confirmar; reconhecer; confirmar explicitamente (fatos, princípios, etc.)}
\end{EntryWithPhonetic}

\begin{EntryWithPhonetic}{确实}{que4shi2}{12,8}{⽯,⼧}[HSK 3]
  \definition{adj.}{verdadeiro; confiável; autêntico}
  \definition{adv.}{verdadeiramente; realmente; de ​​fato; afirmar a autenticidade de fatos objetivos}
\end{EntryWithPhonetic}

\begin{EntryWithPhonetic}{确信}{que4xin4}{12,9}{⽯,⼈}[HSK 7-9]
  \definition{s.}{confirmação; informações autênticas e confiáveis}
  \definition{v.}{ter certeza; acreditar firmemente; estar convencido; acreditar plenamente, sem dúvida alguma}
\end{EntryWithPhonetic}

\begin{EntryWithPhonetic}{确凿}{que4zao2}{12,12}{⽯,⼐}[HSK 7-9]
  \definition{adj.}{autêntico; irrefutável; inegável; absolutamente verdadeiro; inegavelmente verdadeiro}
\end{EntryWithPhonetic}

\begin{EntryWithPhonetic}{确诊}{que4zhen3}{12,7}{⽯,⾔}[HSK 7-9]
  \definition{v.}{diagnosticar; fazer um diagnóstico definitivo; fazer um diagnóstico preciso (da doença)}
\end{EntryWithPhonetic}

%%%%%%%%%% 裙 %%%%%%%%%%
\subsection*{裙}\addcontentsline{loh}{figure}{裙 \dpy{qun2}}

\begin{EntryWithPhonetic}{裙}{qun2}{12}{⾐}
  \definition[条]{s.}{saia | avental | algo como uma saia}
\end{EntryWithPhonetic}

\begin{EntryWithPhonetic}{裙子}{qun2zi5}{12,3}{⾐,⼦}[HSK 3]
  \definition[条,件]{s.}{saia (peça de vestuário); uma vestimenta usada abaixo da cintura}
\end{EntryWithPhonetic}

%%%%%%%%%% 群 %%%%%%%%%%
\subsection*{群}\addcontentsline{loh}{figure}{群 \dpy{qun2}}

\begin{EntryWithPhonetic}{群}{qun2}{13}{⽺}[HSK 3]
  \definition*{s.}{Sobrenome: Qun}
  \definition{adj.}{em grupos; numerosos}
  \definition{clas.}{usado para grupos de pessoas ou coisas; grupo; rebanho; manada}
  \definition{s.}{multidão; grupo; muitas pessoas ou coisas reunidas | as massas; um grupo de pessoas; refere-se a um grande número de pessoas}
\end{EntryWithPhonetic}

\begin{EntryWithPhonetic}{群山}{qun2shan1}{13,3}{⽺,⼭}
  \definition{s.}{montanhas | uma cadeia de colinas}
\end{EntryWithPhonetic}

\begin{EntryWithPhonetic}{群体}{qun2 ti3}{13,7}{⽺,⼈}[HSK 5]
  \definition[个]{s.}{colônia; um conjunto composto por muitos indivíduos da mesma espécie que estão fisicamente conectados, exemplos incluem corais entre os animais e certas algas entre as plantas | grupos; refere-se, de maneira geral, ao conjunto formado por muitos indivíduos interligados que compartilham características essenciais em comum}
\end{EntryWithPhonetic}

\begin{EntryWithPhonetic}{群众}{qun2zhong4}{13,6}{⽺,⼈}[HSK 5]
  \definition[个,名,位]{s.}{as massas; refere-se ao povo em geral | não filiado; apartidário; refere-se a pessoas que não são membros do Partido Comunista Chinês nem da Liga da Juventude Comunista | alguém que não ocupa uma posição de liderança}
\end{EntryWithPhonetic}

%%%%% EOF %%%%%


 %%%
%%% R
%%%
\section*{R}\addcontentsline{toc}{section}{R}\addcontentsline{loh}{figure}{\#\#\#\#\#\#\#\# R}

%%%%%%%%%% 儿 %%%%%%%%%%
\subsection*{儿}\addcontentsline{loh}{figure}{儿 \dpy{r5}}

\begin{EntryWithPhonetic}{儿}{r5}{2}{⼉}[Kangxi 10]
  \definition{suf.}{sufixo diminutivo não silábico | final retroflexo, pronunciado como ``r'' | adicionado a substantivos para expressar pequenez  | adicionado a verbos, adjetivos e classificadores para formar substantivos | adicionado a substantivos para formar substantivos com significados diferentes | sufixos de alguns verbos | anexado após adjetivos duplicados}
  \seeref{er2}
\end{EntryWithPhonetic}

%%%%%%%%%% 然 %%%%%%%%%%
\subsection*{然}\addcontentsline{loh}{figure}{然 \dpy{ran2}}

\begin{EntryWithPhonetic}{然}{ran2}{12}{⽕}
  \definition{conj.}{mas | no entanto}
\end{EntryWithPhonetic}

\begin{EntryWithPhonetic}{然而}{ran2'er2}{12,6}{⽕,⽽}[HSK 4]
  \definition{conj.}{ainda; mas; contudo; todavia; usado no início de uma frase para indicar uma transição; para indicar uma transição, geralmente é precedido por uma conjunção como 虽然 para indicar concessão}
  \seealsoref{虽然}{sui1ran2}
\end{EntryWithPhonetic}

\begin{EntryWithPhonetic}{然后}{ran2hou4}{12,6}{⽕,⼝}[HSK 2]
  \definition{conj.}{então; depois disso; posteriormente; indica que algo segue após uma ação ou situação}
\end{EntryWithPhonetic}

%%%%%%%%%% 燃 %%%%%%%%%%
\subsection*{燃}\addcontentsline{loh}{figure}{燃 \dpy{ran2}}

\begin{EntryWithPhonetic}{燃}{ran2}{16}{⽕}
  \definition{v.}{queimar | acender; inflamar}
\end{EntryWithPhonetic}

\begin{EntryWithPhonetic}{燃放}{ran2fang4}{16,8}{⽕,⽅}[HSK 7-9]
  \definition{v.}{acender (fogos de artifício, etc.); acender fogos de artifício, etc., para causar uma explosão}
\end{EntryWithPhonetic}

\begin{EntryWithPhonetic}{燃料}{ran2liao4}{16,10}{⽕,⽃}[HSK 4]
  \definition[种]{s.}{combustível; carburante; substâncias que podem gerar calor e energia luminosa quando queimadas podem ser divididas em três tipos de acordo com sua forma: combustível sólido (como carvão, carvão vegetal, madeira), combustível líquido (como gasolina, querosene) e combustível gasoso (como gás de carvão, biogás); também se refere a substâncias que podem gerar energia nuclear, como urânio, plutônio, etc.}
\end{EntryWithPhonetic}

\begin{EntryWithPhonetic}{燃气}{ran2qi4}{16,4}{⽕,⽓}[HSK 7-9]
  \definition{s.}{combustível gasoso, como gás de carvão, biogás, gás natural e gás liquefeito de petróleo}
\end{EntryWithPhonetic}

\begin{EntryWithPhonetic}{燃烧}{ran2shao1}{16,10}{⽕,⽕}[HSK 4]
  \definition{v.}{queimar; acender | arder; inflamar; ferver; metáfora para as emoções de uma pessoa serem muito fortes, como um fogo ardente}
\end{EntryWithPhonetic}

\begin{EntryWithPhonetic}{燃油}{ran2you2}{16,8}{⽕,⽔}[HSK 7-9]
  \definition{s.}{óleo combustível}[燃油价格持续上涨了。===Os preços dos combustíveis continuaram a subir.]
\end{EntryWithPhonetic}

%%%%%%%%%% 染 %%%%%%%%%%
\subsection*{染}\addcontentsline{loh}{figure}{染 \dpy{ran3}}

\begin{EntryWithPhonetic}{染}{ran3}{9}{⽊}[HSK 5]
  \definition*{s.}{Sobrenome: Ran}
  \definition{s.}{soja fermentada e temperada em forma de pasta}
  \definition{v.}{tingir; pintar | pegar (uma doença); cair em (um mau hábito, etc.) | sujar; contaminar | pegar (contrair) (uma doença) | adquirir (um mau hábito, etc.); contaminar}
\end{EntryWithPhonetic}

%%%%%%%%%% 嚷 %%%%%%%%%%
\subsection*{嚷}\addcontentsline{loh}{figure}{嚷 \dpy{rang1}}

\begin{EntryWithPhonetic}{嚷}{rang1}{20}{⼝}
  \definition{v.}{gritar; berrar; fazer barulho | gritar; fazer barulho; revelar; falar em voz alta; dizer algo}
  \seeref{rang3}
\end{EntryWithPhonetic}

\begin{EntryWithPhonetic}{嚷}{rang3}{20}{⼝}[HSK 7-9]
  \definition{v.}{gritar; berrar; causar alvoroço | Coloquial: fazer barulho; causar alvoroço | Dialeto: repreender; humilhar; culpar}
  \seeref{rang1}
\end{EntryWithPhonetic}

%%%%%%%%%% 壤 %%%%%%%%%%
\subsection*{壤}\addcontentsline{loh}{figure}{壤 \dpy{rang3}}

\begin{EntryWithPhonetic}{壤}{rang3}{20}{⼟}
  \definition{s.}{solo | terra | (literário) a terra (em contraste com o céu 天)}
\end{EntryWithPhonetic}

%%%%%%%%%% 让 %%%%%%%%%%
\subsection*{让}\addcontentsline{loh}{figure}{让 \dpy{rang4}}

\begin{EntryWithPhonetic}{让}{rang4}{5}{⾔}[HSK 2]
  \definition*{s.}{Sobrenome: Rang}
  \definition{prep.}{em uma frase passiva para introduzir o executor da ação | de acordo com; em conformidade com; à luz de; com base em; usado para expressar a opinião subjetiva de alguém}
  \definition{v.}{ceder; recuar; render"-se; desistir; admitir | convidar; oferecer | deixar; permitir; fazer | deixar alguém ter algo por um preço justo | ser inferior a; não ser tão bom quanto | ceder; afastar"-se | expressar desejos | esquivar"-se; evitar; fugir | usado antes de 我们, indica uma ordem ou sugestão para que todos façam algo juntos}
  \seealsoref{我们}{wo3men5}
\end{EntryWithPhonetic}

\begin{EntryWithPhonetic}{让步}{rang4/bu4}{5,7}{⾔,⽌}[HSK 7-9]
  \definition{v.+compl.}{ceder; fazer uma concessão; comprometer; em uma disputa, significa renunciar parcial ou totalmente às próprias opiniões ou interesses}
\end{EntryWithPhonetic}

\begin{EntryWithPhonetic}{让座}{rang4 zuo4}{5,10}{⾔,⼴}[HSK 6]
  \definition{v.}{oferecer seu lugar a alguém; ceder seu lugar a alguém | convidar os convidados para se sentarem}
\end{EntryWithPhonetic}

%%%%%%%%%% 饶 %%%%%%%%%%
\subsection*{饶}\addcontentsline{loh}{figure}{饶 \dpy{rao2}}

\begin{EntryWithPhonetic}{饶}{rao2}{9}{⾷}[HSK 7-9]
  \definition{adj.}{rico; abundante; farto}
  \definition{conj.}{embora; apesar do fato de que; indica concessão, com significado semelhante a 虽然 ou 尽管}
  \definition{v.}{ter misericórdia de; absolver alguém; perdoar; absolver | dar algo a mais; permitir que alguém receba algo em troca | desculpar; perdoar; tolerar}
  \seealsoref{尽管}{jin3guan3}
  \seealsoref{虽然}{sui1ran2}
\end{EntryWithPhonetic}

\begin{EntryWithPhonetic}{饶恕}{rao2shu4}{9,10}{⾷,⼼}[HSK 7-9]
  \definition{v.}{desculpar; absolver; perdoar; não punir quando a punição é devida}
\end{EntryWithPhonetic}

%%%%%%%%%% 扰 %%%%%%%%%%
\subsection*{扰}\addcontentsline{loh}{figure}{扰 \dpy{rao3}}

\begin{EntryWithPhonetic}{扰}{rao3}{7}{⼿}
  \definition*{s.}{Sobrenome: Rao}
  \definition{adj.}{desordenado; bagunçado}
  \definition{v.}{perturbar; importunar; causar problemas | abusar da hospitalidade de alguém}
\end{EntryWithPhonetic}

\begin{EntryWithPhonetic}{扰乱}{rao3luan4}{7,7}{⼿,⼄}[HSK 7-9]
  \definition{v.}{importunar; perturbar; causar confusão; usar palavras ou ações para interromper ou causar caos em um processo em andamento}
\end{EntryWithPhonetic}

%%%%%%%%%% 绕 %%%%%%%%%%
\subsection*{绕}\addcontentsline{loh}{figure}{绕 \dpy{rao4}}

\begin{EntryWithPhonetic}{绕}{rao4}{9}{⽷}[HSK 5]
  \definition*{s.}{Sobrenome: Rao}
  \definition{v.}{enrolar; bobinar; rebobinar | mover"-se em círculo; girar; revolver | fazer um desvio; contornar; dar a volta | confundir; desorientar}
\end{EntryWithPhonetic}

\begin{EntryWithPhonetic}{绕行}{rao4xing2}{9,6}{⽷,⾏}[HSK 7-9]
  \definition{v.}{desviar; contornar | mover"-se em círculo; circular}
\end{EntryWithPhonetic}

%%%%%%%%%% 惹 %%%%%%%%%%
\subsection*{惹}\addcontentsline{loh}{figure}{惹 \dpy{re3}}

\begin{EntryWithPhonetic}{惹}{re3}{12}{⼼}[HSK 7-9]
  \definition{v.}{provocar; convidar ou pedir (algo indesejável); (características de uma pessoa ou coisa) evocar sentimentos de amor ou ódio | ofender; provocar; instigar; (palavras e ações) tocar na outra pessoa | atrair; causar (algo ruim)}
\end{EntryWithPhonetic}

%%%%%%%%%% 热 %%%%%%%%%%
\subsection*{热}\addcontentsline{loh}{figure}{热 \dpy{re4}}

\begin{EntryWithPhonetic}{热}{re4}{10}{⽕}[HSK 1]
  \definition{adj.}{quente; temperatura elevada | ardente; caloroso; profundamente afetuoso | ansioso; invejoso; descreve inveja e desejo de possuir algo | térmico; altamente radioativo | popular; muito procurado; muito apreciado por muitas pessoas}
  \definition{s.}{calor; energia liberada pelo movimento irregular das moléculas dentro de um objeto | febre; febre alta causada por doença | moda passageira; mania; febre}
  \definition{v.}{aquecer (geralmente se refere a alimentos)}
\end{EntryWithPhonetic}

\begin{EntryWithPhonetic}{热爱}{re4'ai4}{10,10}{⽕,⽖}[HSK 3]
  \definition{v.}{amar ardentemente; amar de coração; ter amor profundo por; amar apaixonadamente}
\end{EntryWithPhonetic}

\begin{EntryWithPhonetic}{热潮}{re4chao2}{10,15}{⽕,⽔}[HSK 7-9]
  \definition{s.}{surto; mania; onda de entusiasmo; descreve uma situação próspera e agitada}
\end{EntryWithPhonetic}

\begin{EntryWithPhonetic}{热带}{re4dai4}{10,9}{⽕,⼱}[HSK 7-9]
  \definition{s.}{zona tropical; os trópicos; localizada em ambos os lados da linha do Equador, entre o Trópico de Câncer e o Trópico de Capricórnio, a região apresenta condições climáticas onde a duração do dia e da noite varia pouco com as estações do ano, o clima é quente durante todo o ano e as chuvas são abundantes}
\end{EntryWithPhonetic}

\begin{EntryWithPhonetic}{热点}{re4dian3}{10,9}{⽕,⽕}[HSK 6]
  \definition{s.}{ponto de acesso; \emph{hotspot}}
\end{EntryWithPhonetic}

\begin{EntryWithPhonetic}{热泪盈眶}{re4lei4ying2kuang4}{10,8,9,11}{⽕,⽔,⽫,⽬}
  \definition{expr.}{olhos cheios de lágrimas de emoção | extremamente emocionado}
\end{EntryWithPhonetic}

\begin{EntryWithPhonetic}{热量}{re4liang4}{10,12}{⽕,⾥}[HSK 5]
  \definition{s.}{calor; quantidade de calor; calorias; em física, refere"-se à energia transferida entre objetos com temperaturas diferentes, do objeto com temperatura mais alta para o objeto com temperatura mais baixa}
\end{EntryWithPhonetic}

\begin{EntryWithPhonetic}{热烈}{re4lie4}{10,10}{⽕,⽕}[HSK 3]
  \definition{adj.}{caloroso; fervoroso; ardente; entusiasmado; excitado}
\end{EntryWithPhonetic}

\begin{EntryWithPhonetic}{热门}{re4men2}{10,3}{⽕,⾨}[HSK 5]
  \definition{adj.}{popular; durante um período de tempo, foi algo que interessava a todos}
  \definition{s.}{algo que desperta o interesse popular; metáfora para algo que está na moda e recebe a atenção de todos (em contraste com 冷门)}
  \seealsoref{冷门}{leng3men2}
\end{EntryWithPhonetic}

\begin{EntryWithPhonetic}{热闹}{re4nao5}{10,8}{⽕,⾾}[HSK 4]
  \definition{adj.}{animado; agitado; movimentado com barulho e excitação; descreve uma cena animada com uma atmosfera calorosa}
  \definition{s.}{uma vista emocionante; uma cena de agitação e excitação; atmosfera acolhedora}
  \definition{v.}{animar; divertir"-se}
\end{EntryWithPhonetic}

\begin{EntryWithPhonetic}{热气}{re4qi4}{10,4}{⽕,⽓}[HSK 7-9]
  \definition[股]{s.}{vapor; calor | Coloquial: entusiasmo | gás quente}
\end{EntryWithPhonetic}

\begin{EntryWithPhonetic}{热气球}{re4qi4qiu2}{10,4,11}{⽕,⽓,⽟}[HSK 7-9]
  \definition{s.}{balão de ar quente}[我们坐了热气球升空。===Fizemos um passeio de balão de ar quente.]
\end{EntryWithPhonetic}

\begin{EntryWithPhonetic}{热情}{re4qing2}{10,11}{⽕,⼼}[HSK 2]
  \definition{adj.}{caloroso; fervoroso; entusiasmado; cordial; descreve sentimentos calorosos por alguém}
  \definition{s.}{entusiasmo; ardor; devoção; calor humano; zelo; sentimentos calorosos}
\end{EntryWithPhonetic}

\begin{EntryWithPhonetic}{热水}{re4shui3}{10,4}{⽕,⽔}[HSK 6]
  \definition{s.}{água quente; água em temperatura mais alta}
\end{EntryWithPhonetic}

\begin{EntryWithPhonetic}{热水器}{re4shui3qi4}{10,4,16}{⽕,⽔,⼝}[HSK 6]
  \definition[台]{s.}{aquecedor de água; aparelhos que aquecem água usando eletricidade, gás natural, gás liquefeito de petróleo ou energia solar}
\end{EntryWithPhonetic}

\begin{EntryWithPhonetic}{热腾腾}{re4teng2teng2}{10,13,13}{⽕,⾁,⾁}[HSK 7-9]
  \definition{adj.}{bem quente; fumegando}[桌上放着一碗热腾腾的汤。===Uma tigela de sopa fumegante está sobre a mesa.]
\end{EntryWithPhonetic}

\begin{EntryWithPhonetic}{热线}{re4xian4}{10,8}{⽕,⽷}[HSK 6]
  \definition[条]{s.}{raio infravermelho | linha direta; \emph{hot line}; uma linha telefônica ou telegráfica direta; uma linha para um ponto de acesso | rota quente (ou movimentada, popular) | raio de calor}
\end{EntryWithPhonetic}

\begin{EntryWithPhonetic}{热心}{re4xin1}{10,4}{⽕,⼼}[HSK 4]
  \definition{adj.}{ardente; sincero; entusiasmado; afetuoso; apaixonado; interessado}
  \definition{v.}{ser entusiasmado com alguma coisa}
\end{EntryWithPhonetic}

\begin{EntryWithPhonetic}{热血沸腾}{re4xue4-fei4teng2}{10,6,8,13}{⽕,⾎,⽔,⾁}
  \definition{expr.}{estar animado; ter o sangue correndo}
\end{EntryWithPhonetic}

\begin{EntryWithPhonetic}{热衷}{re4zhong1}{10,10}{⽕,⾐}[HSK 7-9]
  \definition{v.}{gostar de; ter grande apreço por; gostar muito de (uma determinada atividade) | ansiar; desejar ardentemente; buscar avidamente por (fama, fortuna e poder)}
\end{EntryWithPhonetic}

%%%%%%%%%% 人 %%%%%%%%%%
\subsection*{人}\addcontentsline{loh}{figure}{人 \dpy{ren2}}

\begin{EntryWithPhonetic}{人}{ren2}{2}{⼈}[HSK 1][Kangxi 9]
  \definition*{s.}{Sobrenome: Ren}
  \definition[个,名,位]{s.}{homem; pessoa; pessoas; ser humano | todos; cada um; todo mundo | adulto; crescido | uma pessoa envolvida em uma atividade específica | pessoas; outras pessoas | caráter; personalidade; qualidade, caráter ou reputação de uma pessoa | como alguém se sente; estado de saúde de alguém | mão de obra; força de trabalho}
  \synonymref{样}{yang4}
  \antonymref{神}{shen2}
  \antonymref{我}{wo3}
\end{EntryWithPhonetic}

\begin{EntryWithPhonetic}{人才}{ren2cai2}{2,3}{⼈,⼿}[HSK 3]
  \definition{adj.}{aparência bonita, elegante}
  \definition[个,些,位]{s.}{talento; pessoal qualificado; pessoa com capacidade; uma pessoa com capacidade e integridade política; uma pessoa com talentos especiais | aparência bonita; refere"-se à aparência; especialmente à aparência bonita}
\end{EntryWithPhonetic}

\begin{EntryWithPhonetic}{人材}{ren2cai2}{2,7}{⼈,⽊}
  \variantof{人才}
\end{EntryWithPhonetic}

\begin{EntryWithPhonetic}{人次}{ren2ci4}{2,6}{⼈,⽋}[HSK 7-9]
  \definition{clas.}{visitantes; utilizado para o número de total participantes em várias visitas}[参观展览的总共二十万人次。===A exposição atraiu um total de 200.000 visitantes.]
\end{EntryWithPhonetic}

\begin{EntryWithPhonetic}{人道}{ren2dao4}{2,12}{⼈,⾡}[HSK 7-9]
  \definition{s.}{solidariedade humana; humanitarismo | humano | Budismo: ``a maneira humana'', um dos estágios do ciclo de reencarnação | relação sexual}
  \synonymref{人性}{ren2xing4}
\end{EntryWithPhonetic}

\begin{EntryWithPhonetic}{人格}{ren2ge2}{2,10}{⼈,⽊}[HSK 7-9]
  \definition{s.}{caráter; personalidade; individualidade; a soma do caráter, temperamento, habilidades e outras características de uma pessoa | qualidade moral; caráter moral pessoal | entidade jurídica; dignidade humana; as qualificações de uma pessoa para agir como sujeito de direitos e obrigações}
  \synonymref{品德}{pin3de2}
  \synonymref{品行}{pin3xing2}
  \synonymref{人品}{ren2pin3}
\end{EntryWithPhonetic}

\begin{EntryWithPhonetic}{人工}{ren2gong1}{2,3}{⼈,⼯}[HSK 3]
  \definition{adj.}{feito pelo homem; artificial}
  \definition[个]{s.}{trabalho manual; trabalho feito à mão | mão de obra; homem-dia; uma unidade de cálculo da quantidade de trabalho realizado}
  \synonymref{人力}{ren2li4}
  \synonymref{人为}{ren2wei2}
  \synonymref{人造}{ren2zao4}
  \antonymref{华山}{hua4shan1}
  \antonymref{天然}{tian1ran2}
  \antonymref{野生}{ye3sheng1}
\end{EntryWithPhonetic}

\begin{EntryWithPhonetic}{人工智能}{ren2gong1-zhi4neng2}{2,3,12,10}{⼈,⼯,⽇,⾁}[HSK 7-9]
  \definition*{s.}{Inteligência Artificial (IA)}
\end{EntryWithPhonetic}

\begin{EntryWithPhonetic}{人海}{ren2hai3}{2,10}{⼈,⽔}
  \definition{s.}{uma multidão | um mar de pessoas}
\end{EntryWithPhonetic}

\begin{EntryWithPhonetic}{人家}{ren2jia1}{2,10}{⼈,⼧}
  \definition[户,个]{s.}{lar; família; família do noivo; casa do futuro marido}
  \seeref{ren2jia5}
\end{EntryWithPhonetic}

\begin{EntryWithPhonetic}{人家}{ren2jia5}{2,10}{⼈,⼧}[HSK 4]
  \definition{pron.}{outros; uma pessoa ou pessoas diferentes do falante ou ouvinte; refere"-se a alguém diferente de si mesmo ou de outra pessoa | certa pessoa ou pessoas (a pessoa ou pessoas mencionadas em um contexto próximo, aproximadamente equivalente ao pronome de terceira pessoa);  refere"-se a uma pessoa ou algumas pessoas, com significado semelhante a 他 | eu; mim (usado retoricamente no lugar do primeiro pronome pessoal, muitas vezes expressando descontentamento de forma jocosa; geralmente usado quando se fala com pessoas próximas, para significar 自己, usado principamente por meninas)}
  \seeref{ren2jia1}
  \seealsoref{他}{ta1}
  \seealsoref{自己}{zi4ji3}
\end{EntryWithPhonetic}

\begin{EntryWithPhonetic}{人间}{ren2jian1}{2,7}{⼈,⾨}[HSK 5]
  \definition{s.}{o mundo humano; o Mundo; a Terra}
  \antonymref{地狱}{di4yu4}
  \antonymref{天堂}{tian1tang2}
\end{EntryWithPhonetic}

\begin{EntryWithPhonetic}{人均}{ren2jun1}{2,7}{⼈,⼟}[HSK 7-9]
  \definition{adj.}{per capita (ou por pessoa, cabeça)}[人均收入今年有所增长。===A renda per capita aumentou este ano.]
\end{EntryWithPhonetic}

\begin{EntryWithPhonetic}{人口}{ren2kou3}{2,3}{⼈,⼝}[HSK 2]
  \definition[个,群]{s.}{população; o número total de pessoas que vivem em uma determinada região durante um determinado período de tempo | número de membros da família; o número total de pessoas em uma família | pessoas; público; população; referência geral a pessoas | rumores do povo; referindo"-se à opinião pública}
  \synonymref{人手}{ren2shou3}
\end{EntryWithPhonetic}

\begin{EntryWithPhonetic}{人类}{ren2lei4}{2,9}{⼈,⽶}[HSK 3]
  \definition[种]{s.}{humano; humanidade; raça humana; um termo geral para pessoas}
  \synonymref{动物}{dong4wu4}
  \synonymref{人们}{ren2men5}
\end{EntryWithPhonetic}

\begin{EntryWithPhonetic}{人力}{ren2li4}{2,2}{⼈,⼒}[HSK 5]
  \definition{s.}{mão de obra; trabalho manual; força de trabalho}
  \synonymref{人工}{ren2gong1}
\end{EntryWithPhonetic}

\begin{EntryWithPhonetic}{人力车}{ren2li4che1}{2,2,4}{⼈,⼒,⾞}
  \definition{s.}{veículo de duas rodas puxado ou empurrado por um homem | Obsoleto: riquixá | uma carroça puxada ou empurrada por humanos}
  \synonymref{自行车}{zi4xing2che1}
  \antonymref{机动车}{ji1dong4che1}
  \antonymref{兽力车}{shou4li4che1}
\end{EntryWithPhonetic}

\begin{EntryWithPhonetic}{人们}{ren2men5}{2,5}{⼈,⼈}[HSK 2]
  \definition{s.}{homens; pessoas; o público; referindo"-se a muitas pessoas; todos}
  \synonymref{大家}{da4jia1}
  \synonymref{人家}{ren2jia5}
  \synonymref{人类}{ren2lei4}
  \synonymref{人群}{ren2qun2}
\end{EntryWithPhonetic}

\begin{EntryWithPhonetic}{人民}{ren2min2}{2,5}{⼈,⽒}[HSK 3]
  \definition[群,批,个,国]{s.}{o povo; refere"-se a um certo tipo de pessoas; membros básicos da sociedade com as massas trabalhadoras como o corpo principal}
  \synonymref{群众}{qun2zhong4}
  \antonymref{敌人}{di2ren2}
\end{EntryWithPhonetic}

\begin{EntryWithPhonetic}{人民币}{ren2min2bi4}{2,5,4}{⼈,⽒,⼱}[HSK 3]
  \definition*[块,张,元]{s.}{Renminbi (RMB); Yuan Chinês (CYN); nome da moeda chinesa}
\end{EntryWithPhonetic}

\begin{EntryWithPhonetic}{人品}{ren2pin3}{2,9}{⼈,⼝}[HSK 7-9]
  \definition{s.}{caráter; força moral; integridade | Coloquial: aparência; porte; atitude}
  \synonymref{品德}{pin3de2}
  \synonymref{品行}{pin3xing2}
  \synonymref{人格}{ren2ge2}
  \synonymref{为人}{wei2ren2}
\end{EntryWithPhonetic}

\begin{EntryWithPhonetic}{人气}{ren2qi4}{2,4}{⼈,⽓}[HSK 7-9]
  \definition[股,点,些]{s.}{fama; humor; popularidade; sentimento público/aclamação/apoio/confiança/entusiasmo; o grau em que uma pessoa ou coisa é popular | atmosfera animada | qualidade e estilo excelentes; boa personalidade; refere"-se ao caráter de uma pessoa}
\end{EntryWithPhonetic}

\begin{EntryWithPhonetic}{人情}{ren2qing2}{2,11}{⼈,⼼}[HSK 7-9]
  \definition{s.}{sensibilidades; compaixão humana; emoções humanas; as emoções que as pessoas deveriam ter em circunstâncias normais | sentimentos; sensibilidades; relacionamento | etiqueta; costume; cortesia e costumes nas interações interpessoais | bondade; favor | presente; dádiva; um presente oferecido para expressar um determinado sentimento}
\end{EntryWithPhonetic}

\begin{EntryWithPhonetic}{人权}{ren2quan2}{2,6}{⼈,⽊}[HSK 6]
  \definition{s.}{direitos humanos}[最基本的人权是生存权。===O direito humano mais básico é o direito à vida.]
  \seealsoref{人权法}{ren2quan2fa3}
\end{EntryWithPhonetic}

\begin{EntryWithPhonetic}{人权法}{ren2quan2fa3}{2,6,8}{⼈,⽊,⽔}
  \definition*{s.}{Direitos Humanos}
  \seealsoref{人权}{ren2quan2}
\end{EntryWithPhonetic}

\begin{EntryWithPhonetic}{人群}{ren2qun2}{2,13}{⼈,⽺}[HSK 3]
  \definition[个,类]{s.}{multidão; ajuntamento; torpel; aglomeração; um grupo de pessoas}
  \synonymref{人们}{ren2men5}
\end{EntryWithPhonetic}

\begin{EntryWithPhonetic}{人身}{ren2shen1}{2,7}{⼈,⾝}[HSK 7-9]
  \definition[个]{s.}{corpo vivo de um ser humano; pessoa | corpo humano | pessoal}
\end{EntryWithPhonetic}

\begin{EntryWithPhonetic}{人生}{ren2sheng1}{2,5}{⼈,⽣}[HSK 3]
  \definition{s.}{vida; sobrevivência e vida humana}
  \synonymref{存在}{cun2zai4}
  \synonymref{经历}{jing1li4}
  \synonymref{历程}{li4cheng2}
  \synonymref{生命}{sheng1ming4}
  \antonymref{灭亡}{mie4wang2}
  \antonymref{没落}{mo4luo4}
  \antonymref{死亡}{si3wang2}
\end{EntryWithPhonetic}

\begin{EntryWithPhonetic}{人士}{ren2shi4}{2,3}{⼈,⼠}[HSK 5]
  \definition{s.}{pessoa; figura; personalidade; figura pública; pessoas com certa influência social}
\end{EntryWithPhonetic}

\begin{EntryWithPhonetic}{人事}{ren2shi4}{2,8}{⼈,⼅}[HSK 7-9]
  \definition{s.}{assuntos humanos; acontecimentos na vida humana; separação, reencontro, circunstâncias, sobrevivência e morte dos seres humanos | assuntos pessoais; questões relativas a alterações de pessoal dentro de uma unidade, como recrutamento, demissão, promoção, rebaixamento, recompensas e punições, treinamento e transferências | modos de vida; relações humanas e princípios | consciência do mundo exterior; o objeto da consciência humana | o que é humanamente possível; o que os humanos podem fazer | relações de recursos humanos; refere"-se às relações entre pessoas | pessoal}
\end{EntryWithPhonetic}

\begin{EntryWithPhonetic}{人手}{ren2shou3}{2,4}{⼈,⼿}[HSK 7-9]
  \definition{s.}{mão de obra; mão; pessoas que fazem coisas}
  \synonymref{人口}{ren2kou3}
  \synonymref{忍耐}{ren3nai4}
  \synonymref{容忍}{rong2ren3}
\end{EntryWithPhonetic}

\begin{EntryWithPhonetic}{人数}{ren2shu4}{2,13}{⼈,⽁}[HSK 2]
  \definition{s.}{número de pessoas; significa o número total de pessoas, uma quantidade de pessoas; normalmente, usa-se números para fazer estatísticas específicas, mas às vezes também se usa um intervalo aproximado para fazer estimativas}
\end{EntryWithPhonetic}

\begin{EntryWithPhonetic}{人体}{ren2ti3}{2,7}{⼈,⼈}[HSK 7-9]
  \definition{s.}{corpo humano}
  \seealsoref{人身}{ren2shen1}
  \antonymref{雕塑}{diao1su4}
\end{EntryWithPhonetic}

\begin{EntryWithPhonetic}{人为}{ren2wei2}{2,4}{⼈,⼂}[HSK 7-9]
  \definition{adj.}{artificial; feito pelo homem; causado por pessoas (usado para descrever coisas desagradáveis)}
  \definition{v.}{fazer pelo homem; fazer esforço humano; fazer isso com força humana}
  \synonymref{报酬}{bao4chou5}
  \synonymref{人工}{ren2gong1}
  \synonymref{人造}{ren2zao4}
  \antonymref{天然}{tian1ran2}
\end{EntryWithPhonetic}

\begin{EntryWithPhonetic}{人文}{ren2wen2}{2,4}{⼈,⽂}[HSK 7-9]
  \definition{s.}{humanidades; atividades culturais na sociedade humana; originalmente referindo"-se à poesia, livros, ritos e música, posteriormente passou a se referir a vários fenômenos culturais na sociedade humana}
  \synonymref{文化}{wen2hua4}
  \synonymref{文明}{wen2ming2}
  \antonymref{地理}{di4li3}
\end{EntryWithPhonetic}

\begin{EntryWithPhonetic}{人物}{ren2wu4}{2,8}{⼈,⽜}[HSK 5]
  \definition[个,位,名]{s.}{personagem; personagens criados em obras literárias e artísticas | figura; personalidade; homem influente; refere"-se a pessoas com grande talento e status; também se refere a pessoas com certas características ou que são representativas em algum aspecto | pintura figurativa; um tipo de pintura tradicional chinesa com personagens como tema}
  \synonymref{角色}{jue2se4}
  \synonymref{形象}{xing2xiang4}
\end{EntryWithPhonetic}

\begin{EntryWithPhonetic}{人像}{ren2xiang4}{2,13}{⼈,⼈}
  \definition{s.}{``retrato'' de uma pessoa (esboço, foto, escultura, etc.)}
\end{EntryWithPhonetic}

\begin{EntryWithPhonetic}{人行道}{ren2xing2dao4}{2,6,12}{⼈,⾏,⾡}[HSK 7-9]
  \definition{s.}{calçada; as calçadas em ambos os lados da rua são exclusivas para pedestres}
\end{EntryWithPhonetic}

\begin{EntryWithPhonetic}{人性}{ren2xing4}{2,8}{⼈,⼼}[HSK 7-9]
  \definition{s.}{humanidade; natureza humana; as emoções e a razão normais que os seres humanos possuem}
  \synonymref{本性}{ben3xing4}
  \synonymref{人道}{ren2dao4}
\end{EntryWithPhonetic}

\begin{EntryWithPhonetic}{人选}{ren2xuan3}{2,9}{⼈,⾡}[HSK 7-9]
  \definition{s.}{candidato; pessoa selecionada de acordo com determinados critérios}
\end{EntryWithPhonetic}

\begin{EntryWithPhonetic}{人鱼}{ren2yu2}{2,8}{⼈,⿂}
  \definition{s.}{sereia | peixe-boi | salamandra gigante}
\end{EntryWithPhonetic}

\begin{EntryWithPhonetic}{人员}{ren2yuan2}{2,7}{⼈,⼝}[HSK 3]
  \definition[个,位,名]{s.}{funcionários ; uma pessoa que ocupa uma determinada posição | pessoal; membros de um grupo}
  \synonymref{职员}{zhi2yuan2}
\end{EntryWithPhonetic}

\begin{EntryWithPhonetic}{人缘儿}{ren2yuan2r5}{2,12,2}{⼈,⽷,⼉}[HSK 7-9]
  \definition{s.}{relações com as pessoas; popularidade}
\end{EntryWithPhonetic}

\begin{EntryWithPhonetic}{人造}{ren2zao4}{2,10}{⼈,⾡}[HSK 7-9]
  \definition{adj.}{feito pelo homem; artificial; imitação | sintético}
  \synonymref{人工}{ren2gong1}
  \synonymref{人为}{ren2wei2}
  \antonymref{天然}{tian1ran2}
\end{EntryWithPhonetic}

\begin{EntryWithPhonetic}{人质}{ren2zhi4}{2,8}{⼈,⾙}[HSK 7-9]
  \definition[个,名]{s.}{refém; uma das partes detém os pertences da outra parte para obrigá-la a cumprir uma promessa ou aceitar uma condição}
\end{EntryWithPhonetic}

%%%%%%%%%% 仁 %%%%%%%%%%
\subsection*{仁}\addcontentsline{loh}{figure}{仁 \dpy{ren2}}

\begin{EntryWithPhonetic}{仁}{ren2}{4}{⼈}
  \definition*{s.}{Sobrenome: Ren}
  \definition{adj.}{sensível}
  \definition{s.}{benevolência; bondade; generosidade; humanidade | ideal | amêndoa; caroço | carne de camarão | amor; bondade}
  \seealsoref{仁儿}{ren2r5}
  \synonymref{慈}{ci2}
\end{EntryWithPhonetic}

\begin{EntryWithPhonetic}{仁慈}{ren2ci2}{4,13}{⼈,⼼}[HSK 7-9]
  \definition{adj.}{bondoso; misericordioso; benevolente; descreve alguém como muito educado, compassivo e capaz de cuidar e ajudar os outros}
  \synonymref{慈善}{ci2shan4}
  \synonymref{慈祥}{ci2xiang2}
  \synonymref{善良}{shan4liang2}
  \antonymref{残酷}{can2ku4}
  \antonymref{残忍}{can2ren3}
  \antonymref{冷酷}{leng3ku4}
\end{EntryWithPhonetic}

\begin{EntryWithPhonetic}{仁儿}{ren2r5}{4,2}{⼈,⼉}
  \definition{s.}{amêndoa; caroço | carne de camarão}
\end{EntryWithPhonetic}

%%%%%%%%%% 任 %%%%%%%%%%
\subsection*{任}\addcontentsline{loh}{figure}{任 \dpy{ren2}}

\begin{EntryWithPhonetic}{任}{ren2}{6}{⼈}
  \definition*{s.}{Condado de Ren em Hebei (河北) | usado em nomes de lugares, por exemplo, Renxian (任县) e Renqi (任丘) ficam na província de Hebei (河北) | Sobrenome: Ren}
  \seeref{ren4}
\end{EntryWithPhonetic}

%%%%%%%%%% 忍 %%%%%%%%%%
\subsection*{忍}\addcontentsline{loh}{figure}{忍 \dpy{ren3}}

\begin{EntryWithPhonetic}{忍}{ren3}{7}{⼼}[HSK 5]
  \definition{v.}{suportar; aguentar; tolerar; aturar | ter coragem para; ser insensível o suficiente para; ser capaz de endurecer o coração e fazer coisas que não se devem fazer por uma questão de razão}
\end{EntryWithPhonetic}

\begin{EntryWithPhonetic}{忍不住}{ren3bu5zhu4}{7,4,7}{⼼,⼀,⼈}[HSK 5]
  \definition{v.}{incapaz de suportar; não conseguir evitar fazer algo; não conseguir se controlar}
\end{EntryWithPhonetic}

\begin{EntryWithPhonetic}{忍饥挨饿}{ren3ji1-ai2'e4}{7,5,10,10}{⼼,⾷,⼿,⾷}[HSK 7-9]
  \definition{expr.}{``Morrendo de fome.''; suportar os tormentos da fome; famintos}
\end{EntryWithPhonetic}

\begin{EntryWithPhonetic}{忍耐}{ren3nai4}{7,9}{⼼,⽽}[HSK 7-9]
  \definition{v.}{conter-se; exercer paciência (ou contenção); suprimir um determinado sentimento ou emoção para impedir que ele seja expresso}
\end{EntryWithPhonetic}

\begin{EntryWithPhonetic}{忍受}{ren3shou4}{7,8}{⼼,⼜}[HSK 5]
  \definition{v.}{suportar; sofrer; aguentar; tolerar; suportar com dificuldade o sofrimento, as dificuldades e as adversidades da vida}
\end{EntryWithPhonetic}

\begin{EntryWithPhonetic}{忍心}{ren3/xin1}{7,4}{⼼,⼼}[HSK 7-9]
  \definition{v.+compl.}{ter a coragem de; ser suficientemente insensível para; ser capaz de endurecer o coração (fazer coisas que não se suporta fazer)}
\end{EntryWithPhonetic}

%%%%%%%%%% 认 %%%%%%%%%%
\subsection*{认}\addcontentsline{loh}{figure}{认 \dpy{ren4}}

\begin{EntryWithPhonetic}{认}{ren4}{4}{⾔}[HSK 5]
  \definition{v.}{reconhecer; saber; distinguir; identificar | estabelecer uma determinada relação com; adotar | admitir; reconhecer; assumir | comprometer-se a fazer algo | (frequentemente seguido por 了) aceitar como inevitável; resignar-se}
  \seealsoref{了}{le5}
\end{EntryWithPhonetic}

\begin{EntryWithPhonetic}{认出}{ren4 chu1}{4,5}{⾔,⼐}[HSK 3]
  \definition{v.}{reconhecer; identificar; reconhecer alguém ou algo pela observação ou memória}
\end{EntryWithPhonetic}

\begin{EntryWithPhonetic}{认错}{ren4/cuo4}{4,13}{⾔,⾦}[HSK 7-9]
  \definition{v.+compl.}{admitir uma falha; pedir desculpas; reconhecer um erro}
\end{EntryWithPhonetic}

\begin{EntryWithPhonetic}{认得}{ren4 de5}{4,11}{⾔,⼻}[HSK 3]
  \definition{v.}{saber; reconhecer; capacidade de confirmar a pessoa ou coisa que você vê}
\end{EntryWithPhonetic}

\begin{EntryWithPhonetic}{认定}{ren4ding4}{4,8}{⾔,⼧}[HSK 5]
  \definition{v.}{afirmar; manter; acreditar firmemente; considerar com certeza | decidir-se por algo; confirmar; chegar a uma conclusão afirmativa}
\end{EntryWithPhonetic}

\begin{EntryWithPhonetic}{认可}{ren4ke3}{4,5}{⾔,⼝}[HSK 3]
  \definition{v.}{aceitar; aprovar; confirmar; dar força legal a | permitir; concordar}
\end{EntryWithPhonetic}

\begin{EntryWithPhonetic}{认识}{ren4shi5}{4,7}{⾔,⾔}[HSK 1]
  \definition[份]{s.}{cognição; conhecimento; compreensão; refere"-se à reflexão da mente humana sobre o mundo objetivo}
  \definition{v.}{saber; compreender; reconhecer}
\end{EntryWithPhonetic}

\begin{EntryWithPhonetic}{认同}{ren4tong2}{4,6}{⾔,⼝}[HSK 6]
  \definition{v.}{identificar; pensar que a outra pessoa tem algo em comum com você | aprovar; reconhecer}
\end{EntryWithPhonetic}

\begin{EntryWithPhonetic}{认为}{ren4wei2}{4,4}{⾔,⼂}[HSK 2]
  \definition{v.}{pensar; considerar; manter; julgar; formar uma opinião sobre uma pessoa ou coisa, fazer um julgamento}
\end{EntryWithPhonetic}

\begin{EntryWithPhonetic}{认真}{ren4zhen1}{4,10}{⾔,⼗}[HSK 1]
  \definition{adj.}{sério; sério e meticuloso}
  \definition{adv.}{seriamente}
  \definition{v.}{levar algo a sério; considerar como verdadeiro; levar a sério}
\end{EntryWithPhonetic}

\begin{EntryWithPhonetic}{认证}{ren4zheng4}{4,7}{⾔,⾔}[HSK 7-9]
  \definition{v.}{certificar; autenticar; o cartório emitirá uma certidão após verificar a autenticidade dos documentos apresentados pelas partes}
\end{EntryWithPhonetic}

\begin{EntryWithPhonetic}{认知}{ren4zhi1}{4,8}{⾔,⽮}[HSK 7-9]
  \definition{v.}{ter conhecimento; reconhecer; estar ciente de}
\end{EntryWithPhonetic}

%%%%%%%%%% 任 %%%%%%%%%%
\subsection*{任}\addcontentsline{loh}{figure}{任 \dpy{ren4}}

\begin{EntryWithPhonetic}{任}{ren4}{6}{⼈}[HSK 3]
  \definition{clas.}{usado para o número de mandatos cumpridos em um cargo oficial}
  \definition{conj.}{não importa (como, o que, etc.); orações de conexão, ou usadas antes de pronomes interrogativos, para expressar incondicionalidade, equivalente a 不管 ou 无论}
  \definition{s.}{escritório; posto oficial; cargo | dever; fardo; responsabilidade}
  \definition{v.}{nomear; designar alguém para um cargo | assumir um emprego; assumir um posto; assumir uma posição | deixar; permitir; dar rédea solta a | suportar; empreender | ceder; permitir sem restrições; deixar (alguém) fazer o que quiser}
  \seeref{ren2}
  \seealsoref{不管}{bu4guan3}
  \seealsoref{无论}{wu2lun4}
  \antonymref{免}{mian3}
\end{EntryWithPhonetic}

\begin{EntryWithPhonetic}{任何}{ren4he2}{6,7}{⼈,⼈}[HSK 3]
  \definition{pron.}{qualquer; qualquer que seja; o que for; não importa o que}
  \synonymref{所以}{suo3yi3}
  \synonymref{无论}{wu2lun4}
  \synonymref{一切}{yi2qie4}
  \antonymref{唯一}{wei2yi1}
\end{EntryWithPhonetic}

\begin{EntryWithPhonetic}{任命}{ren4ming4}{6,8}{⼈,⼝}[HSK 7-9]
  \definition{v.}{nomear; designar; incumbir; comissionar}
  \synonymref{任职}{ren4/zhi2}
  \antonymref{免职}{mian3/zhi2}
  \antonymref{免除}{mian3chu2}
\end{EntryWithPhonetic}

\begin{EntryWithPhonetic}{任凭}{ren4 ping2}{6,8}{⼈,⼏}
  \definition{conj.}{não importa (como, o quê, etc.) | mesmo que; embora}
  \definition{v.}{permitir; deixar (algo como: fazer o que lhe agrada); conforme a conveniência de alguém}
  \synonymref{听凭}{ting1ping2}
  \antonymref{束缚}{shu4fu4}
\end{EntryWithPhonetic}

\begin{EntryWithPhonetic}{任期}{ren4qi1}{6,12}{⼈,⽉}[HSK 7-9]
  \definition[个,届]{s.}{mandato; duração do mandato; mandato legal}
\end{EntryWithPhonetic}

\begin{EntryWithPhonetic}{任人宰割}{ren4ren2-zai3ge1}{6,2,10,12}{⼈,⼈,⼧,⼑}[HSK 7-9]
  \definition{expr.}{(não pode deixar de) permitir que lhe pisoteiem; ser explorado; ser pisoteado}
\end{EntryWithPhonetic}

\begin{EntryWithPhonetic}{任务}{ren4wu5}{6,5}{⼈,⼒}[HSK 3]
  \definition[项,个,种,些]{s.}{tarefa; dever; missão; designação; trabalho designado; responsabilidades designadas}
  \synonymref{工作}{gong1zuo4}
  \synonymref{使命}{shi3ming4}
  \synonymref{团队}{tuan2dui4}
  \synonymref{义务}{yi4wu4}
  \synonymref{职业}{zhi2ye4}
  \synonymref{职责}{zhi2ze2}
\end{EntryWithPhonetic}

\begin{EntryWithPhonetic}{任意}{ren4yi4}{6,13}{⼈,⼼}[HSK 7-9]
  \definition{adj.}{sem reservas; sem quaisquer condições}
  \definition{adv.}{arbitrariamente; sem restrições, sem limitações, faça o que quiser}
  \synonymref{大肆}{da4si4}
  \synonymref{放肆}{fang4si4}
  \synonymref{随便}{sui2/bian4}
  \synonymref{随意}{sui2/yi4}
  \antonymref{拘束}{ju1shu4}
\end{EntryWithPhonetic}

\begin{EntryWithPhonetic}{任职}{ren4/zhi2}{6,11}{⼈,⽿}[HSK 7-9]
  \definition{v.+compl.}{ocupar um cargo; estar em um cargo de chefia}
  \synonymref{服务}{fu2wu4}
  \synonymref{任命}{ren4ming4}
  \antonymref{离职}{li2/zhi2}
  \antonymref{免职}{mian3/zhi2}
\end{EntryWithPhonetic}

%%%%%%%%%% 韧 %%%%%%%%%%
\subsection*{韧}\addcontentsline{loh}{figure}{韧 \dpy{ren4}}

\begin{EntryWithPhonetic}{韧}{ren4}{7}{⾱}
  \definition{adj.}{flexível, mas forte; tenaz; resistente | resistente; macio e forte, não quebra facilmente}
  \antonymref{脆}{cui4}
\end{EntryWithPhonetic}

\begin{EntryWithPhonetic}{韧性}{ren4xing4}{7,8}{⾱,⼼}[HSK 7-9]
  \definition{s.}{ductilidade; tenacidade; resistência; propriedades de um objeto: macio, porém resistente, e não quebra facilmente | tenacidade; refere"-se a um espírito de perseverança e tenacidade}
\end{EntryWithPhonetic}

%%%%%%%%%% 扔 %%%%%%%%%%
\subsection*{扔}\addcontentsline{loh}{figure}{扔 \dpy{reng1}}

\begin{EntryWithPhonetic}{扔}{reng1}{5}{⼿}[HSK 5]
  \definition{v.}{arremessar; lançar; atirar; jogar | esquecer; jogar fora; descartar | colocar casualmente; deixar as pessoas ou as coisas de lado, não se importar}
\end{EntryWithPhonetic}

\begin{EntryWithPhonetic}{扔掉}{reng1diao4}{5,11}{⼿,⼿}
  \definition{v.}{jogar fora}
\end{EntryWithPhonetic}

\begin{EntryWithPhonetic}{扔弃}{reng1qi4}{5,7}{⼿,⼶}
  \definition{v.}{abandonar | descartar | jogar fora}
\end{EntryWithPhonetic}

\begin{EntryWithPhonetic}{扔下}{reng1xia4}{5,3}{⼿,⼀}
  \definition{v.}{lançar (uma bomba) | derrubar}
\end{EntryWithPhonetic}

%%%%%%%%%% 仍 %%%%%%%%%%
\subsection*{仍}\addcontentsline{loh}{figure}{仍 \dpy{reng2}}

\begin{EntryWithPhonetic}{仍}{reng2}{4}{⼈}[HSK 3]
  \definition{adv.}{ainda; repetidamente; frequentemente; continuamente}
  \definition{v.}{permanecer}
  \synonymref{仍旧}{reng2jiu4}
  \synonymref{仍然}{reng2ran2}
\end{EntryWithPhonetic}

\begin{EntryWithPhonetic}{仍旧}{reng2jiu4}{4,5}{⼈,⽇}[HSK 5]
  \definition{adv.}{ainda; ainda assim; contudo}
  \definition{v.}{permanecer igual; continuar sendo}
  \synonymref{保持}{bao3chi2}
  \synonymref{宝石}{bao3shi2}
  \synonymref{还是}{hai2shi5}
  \synonymref{仍然}{reng2ran2}
  \synonymref{依旧}{yi1jiu4}
  \synonymref{依然}{yi1ran2}
  \synonymref{已经}{yi3jing1}
  \synonymref{照样}{zhao4yang4}
  \antonymref{不再}{bu2zai4}
  \antonymref{改变}{gai3bian4}
\end{EntryWithPhonetic}

\begin{EntryWithPhonetic}{仍然}{reng2ran2}{4,12}{⼈,⽕}[HSK 3]
  \definition{adv.}{ainda; contudo; como antes; indica que a situação continua inalterada ou retorna ao seu estado original}
  \synonymref{还是}{hai2shi5}
  \synonymref{依然}{yi1ran2}
  \synonymref{照样}{zhao4yang4}
  \antonymref{不再}{bu2zai4}
  \antonymref{尚未}{shang4wei4}
\end{EntryWithPhonetic}

%%%%%%%%%% 日 %%%%%%%%%%
\subsection*{日}\addcontentsline{loh}{figure}{日 \dpy{ri4}}

\begin{EntryWithPhonetic}{日}{ri4}{4}{⽇}[HSK 1][Kangxi 72]
  \definition*{s.}{Japão, abreviação de 日本}
  \definition{clas.}{usado para contar o número de dias}
  \definition{s.}{sol | dia; período diurno | diariamente; todos os dias; a cada dia que passa | um dia específico; dia especial | tempo; refere"-se a um período de tempo | dia; uma rotação da Terra}
  \seealsoref{日本}{ri4ben3}
  \antonymref{夜}{ye4}
\end{EntryWithPhonetic}

\begin{EntryWithPhonetic}{日报}{ri4bao4}{4,7}{⽇,⼿}[HSK 2]
  \definition[份,种]{s.}{diário; jornais diários; jornal publicado todas as manhãs}
\end{EntryWithPhonetic}

\begin{EntryWithPhonetic}{日本}{ri4ben3}{4,5}{⽇,⽊}
  \definition*{s.}{Japão}
\end{EntryWithPhonetic}

\begin{EntryWithPhonetic}{日本人}{ri4ben3ren2}{4,5,2}{⽇,⽊,⼈}
  \definition{s.}{japonês | pessoa ou povo do Japão}
\end{EntryWithPhonetic}

\begin{EntryWithPhonetic}{日常}{ri4chang2}{4,11}{⽇,⼱}[HSK 3]
  \definition{adj.}{usual; diário; cotidiano; dia a dia; pertencem ao habitual}
\end{EntryWithPhonetic}

\begin{EntryWithPhonetic}{日程}{ri4cheng2}{4,12}{⽇,⽲}[HSK 7-9]
  \definition[个]{s.}{cronograma; programa; agenda do dia; cronograma diário de tarefas ou atividades}
\end{EntryWithPhonetic}

\begin{EntryWithPhonetic}{日出}{ri4chu1}{4,5}{⽇,⼐}
  \definition{s.}{nascer do sol}
  \seealsoref{夕阳}{xi1yang2}
\end{EntryWithPhonetic}

\begin{EntryWithPhonetic}{日复一日}{ri4fu4yi2ri4}{4,9,1,4}{⽇,⼢,⼀,⽇}[HSK 7-9]
  \definition{expr.}{``Dia a dia.''; dia após dia; realizar uma determinada atividade por vários dias consecutivos}
\end{EntryWithPhonetic}

\begin{EntryWithPhonetic}{日光灯}{ri4guang1deng1}{4,6,6}{⽇,⼉,⽕}
  \definition{s.}{lâmpada fluorescente}
\end{EntryWithPhonetic}

\begin{EntryWithPhonetic}{日后}{ri4hou4}{4,6}{⽇,⼝}[HSK 7-9]
  \definition{s.}{no futuro; nos próximos dias}
\end{EntryWithPhonetic}

\begin{EntryWithPhonetic}{日记}{ri4ji4}{4,5}{⽇,⾔}[HSK 4]
  \definition[本,篇,册]{s.}{diário; artigo que registra eventos e pensamentos diários}
\end{EntryWithPhonetic}

\begin{EntryWithPhonetic}{日历}{ri4li4}{4,4}{⽇,⼚}[HSK 4]
  \definition[张,本]{s.}{caledário; livro com o ano, mês, dia, semana, termo solar, aniversário, etc. registrados, um livro por ano, uma página por dia, aberto diariamente}
\end{EntryWithPhonetic}

\begin{EntryWithPhonetic}{日期}{ri4qi1}{4,12}{⽇,⽉}[HSK 1]
  \definition[个,段]{s.}{data; a data ou período específico em que algo aconteceu}
\end{EntryWithPhonetic}

\begin{EntryWithPhonetic}{日前}{ri4qian2}{4,9}{⽇,⼑}[HSK 7-9]
  \definition{s.}{recentemente; há alguns dias}
\end{EntryWithPhonetic}

\begin{EntryWithPhonetic}{日趋}{ri4qu1}{4,12}{⽇,⾛}[HSK 7-9]
  \definition{adv.}{gradualmente; dia após dia; a cada dia que passa; isso indica que as coisas estão mudando em uma determinada direção a cada dia}[环境污染问题日趋严重。===A poluição ambiental está se tornando cada vez mais grave.]
\end{EntryWithPhonetic}

\begin{EntryWithPhonetic}{日心说}{ri4 xin1 shuo1}{4,4,9}{⽇,⼼,⾔}
  \definition{s.}{teoria heliocêntrica | a teoria de que o sol está no centro do universo}
\end{EntryWithPhonetic}

\begin{EntryWithPhonetic}{日新月异}{ri4xin1-yue4yi4}{4,13,4,6}{⽇,⽄,⽉,⼶}[HSK 7-9]
  \definition{expr.}{``Mudanças rápidas.''; mudar rapidamente; alterar"-se de um dia para o outro; provocar novas mudanças; melhorar a cada dia e a cada mês; prosperar (mudar) a cada dia que passa; mudanças incessantes em; mudanças e melhorias sem fim; mudanças constantes, tanto diárias quanto mensais, o que demonstra um progresso e desenvolvimento rápidos}
\end{EntryWithPhonetic}

\begin{EntryWithPhonetic}{日夜}{ri4ye4}{4,8}{⽇,⼣}[HSK 6]
  \definition{s.}{dia e noite; noite e dia; 24 horas por dia}
\end{EntryWithPhonetic}

\begin{EntryWithPhonetic}{日益}{ri4yi4}{4,10}{⽇,⽫}[HSK 7-9]
  \definition{adv.}{cada vez mais; de modo crescente; dia após dia}
\end{EntryWithPhonetic}

\begin{EntryWithPhonetic}{日语}{ri4yu3}{4,9}{⽇,⾔}[HSK 6]
  \definition{s.}{japonês; língua japonesa}
\end{EntryWithPhonetic}

\begin{EntryWithPhonetic}{日子}{ri4zi5}{4,3}{⽇,⼦}[HSK 2]
  \definition[个,段,些,番]{s.}{dia; data; referência a uma data específica | dias; tempo; referência ao número de dias e horas | vida; subsistência; refere"-se à vida ou ao sustento}
\end{EntryWithPhonetic}

%%%%%%%%%% 荣 %%%%%%%%%%
\subsection*{荣}\addcontentsline{loh}{figure}{荣 \dpy{rong2}}

\begin{EntryWithPhonetic}{荣}{rong2}{9}{⾋}
  \definition*{s.}{Sobrenome: Rong}
  \definition{adj.}{próspero; florescente | exuberante | glorioso}
  \definition{s.}{honra; glória | guarda-sol chinês | flor; flor de planta herbácea | beirais virados para cima}
  \definition{v.}{glorificar; luxuriar; crescer abundantemente; florescer | florescer | lançar}
  \antonymref{辱}{ru3}
\end{EntryWithPhonetic}

\begin{EntryWithPhonetic}{荣获}{rong2huo4}{9,10}{⾋,⾋}[HSK 7-9]
  \definition{v.}{ganhar; ser homenageado com; receber honras}
\end{EntryWithPhonetic}

\begin{EntryWithPhonetic}{荣幸}{rong2xing4}{9,8}{⾋,⼲}[HSK 7-9]
  \definition{adj.}{ser honrado; honroso; glorioso e afortunado}
  \definition[种]{s.}{honra}
\end{EntryWithPhonetic}

\begin{EntryWithPhonetic}{荣誉}{rong2yu4}{9,13}{⾋,⾔}[HSK 7-9]
  \definition[种,份]{s.}{honra; glória; crédito; reputação honrosa; reputação gloriosa}
\end{EntryWithPhonetic}

%%%%%%%%%% 容 %%%%%%%%%%
\subsection*{容}\addcontentsline{loh}{figure}{容 \dpy{rong2}}

\begin{EntryWithPhonetic}{容}{rong2}{10}{⼧}
  \definition*{s.}{Sobrenome: Rong}
  \definition{adv.}{talvez; provavelmente; possivelmente}
  \definition{s.}{expressão facial e tez | aparência; o estado ou condição das coisas}
  \definition{v.}{permitir; quando os outros querem fazer algo, deixe-os fazer | tolerar; ser capaz de aceitar pessoas ou coisas com as quais você não está satisfeito | conter (número de pessoas ou coisas que podem ser colocadas em um determinado espaço)}
\end{EntryWithPhonetic}

\begin{EntryWithPhonetic}{容光焕发}{rong2guang1-huan4fa1}{10,6,11,5}{⼧,⼉,⽕,⼜}[HSK 7-9]
  \definition{expr.}{o rosto radiante de saúde; de bom humor; alegre; só sorrisos; aparência luminosa}
\end{EntryWithPhonetic}

\begin{EntryWithPhonetic}{容量}{rong2liang4}{10,12}{⼧,⾥}[HSK 7-9]
  \definition{s.}{capacidade; a dimensão de um volume é chamada de capacidade; a principal unidade de capacidade no sistema métrico é o litro}
\end{EntryWithPhonetic}

\begin{EntryWithPhonetic}{容貌}{rong2mao4}{10,14}{⼧,⾘}
  \definition{s.}{aparência | aspecto | características}
\end{EntryWithPhonetic}

\begin{EntryWithPhonetic}{容纳}{rong2na4}{10,7}{⼧,⽷}[HSK 7-9]
  \definition{v.}{acomodar; conter; possuir | aceitar (opiniões, etc.); tolerar}
\end{EntryWithPhonetic}

\begin{EntryWithPhonetic}{容忍}{rong2ren3}{10,7}{⼧,⼼}[HSK 7-9]
  \definition{v.}{tolerar; aceitar; suportar}
\end{EntryWithPhonetic}

\begin{EntryWithPhonetic}{容许}{rong2xu3}{10,6}{⼧,⾔}[HSK 7-9]
  \definition{adv.}{Literário: talvez; possivelmente}
  \definition{v.}{tolerar; permitir; autorizar | Literário: admitir; tolerar}
\end{EntryWithPhonetic}

\begin{EntryWithPhonetic}{容颜}{rong2yan2}{10,15}{⼧,⾴}[HSK 7-9]
  \definition{s.}{Literário: aparência; visual; tez}
\end{EntryWithPhonetic}

\begin{EntryWithPhonetic}{容易}{rong2yi4}{10,8}{⼧,⽇}[HSK 3]
  \definition{adj.}{fácil; simples; sem complicações | provável; passível; inclinado; indica uma alta probabilidade de algo acontecer}
\end{EntryWithPhonetic}

%%%%%%%%%% 溶 %%%%%%%%%%
\subsection*{溶}\addcontentsline{loh}{figure}{溶 \dpy{rong2}}

\begin{EntryWithPhonetic}{溶}{rong2}{13}{⽔}
  \definition{v.}{dissolver; solver; solucionar}
\end{EntryWithPhonetic}

\begin{EntryWithPhonetic}{溶解}{rong2jie3}{13,13}{⽔,⾓}[HSK 7-9]
  \definition{v.}{dissolver}[慢慢加热直到糖溶解为止。===Aqueça lentamente até o açúcar dissolver.]
\end{EntryWithPhonetic}

%%%%%%%%%% 融 %%%%%%%%%%
\subsection*{融}\addcontentsline{loh}{figure}{融 \dpy{rong2}}

\begin{EntryWithPhonetic}{融}{rong2}{16}{⿀}[HSK 7-9]
  \definition*{s.}{Sobrenome: Rong}
  \definition{adj.}{permanente; longo prazo; duradouro | muito brilhante | circulante; corrente}
  \definition{s.}{fogo | plena luz do dia}
  \definition{v.}{derreter; descongelar | misturar; fundir; estar em harmonia | circular (dinheiro, etc.)}
\end{EntryWithPhonetic}

\begin{EntryWithPhonetic}{融合}{rong2he2}{16,6}{⿀,⼝}[HSK 6]
  \definition{v.}{fundir; mesclar; misturar; combinar várias coisas diferentes em uma}
\end{EntryWithPhonetic}

\begin{EntryWithPhonetic}{融化}{rong2hua4}{16,4}{⿀,⼔}[HSK 7-9]
  \definition{v.}{derreter; descongelar}
\end{EntryWithPhonetic}

\begin{EntryWithPhonetic}{融洽}{rong2qia4}{16,9}{⿀,⽔}[HSK 7-9]
  \definition{adj.}{harmonioso; em termos amigáveis}
\end{EntryWithPhonetic}

\begin{EntryWithPhonetic}{融入}{rong2ru4}{16,2}{⿀,⼊}[HSK 6]
  \definition{v.}{integrar em; juntar-se, integrar-se ao grupo | misturar-se; enfatizar a mistura e a combinação com o ambiente circundante para se tornar harmonioso e consistente | encher com (um certo sentimento); imbuir com (uma certa qualidade); preparar (chá, ervas, etc.); imergir; infundir (drogas, etc.)}
\end{EntryWithPhonetic}

%%%%%%%%%% 冗 %%%%%%%%%%
\subsection*{冗}\addcontentsline{loh}{figure}{冗 \dpy{rong3}}

\begin{EntryWithPhonetic}{冗}{rong3}{4}{⼍}
  \definition{adj.}{supérfluo; redundante | cheio de detalhes triviais | incapaz}
  \definition{s.}{ocupação; negócio}
\end{EntryWithPhonetic}

\begin{EntryWithPhonetic}{冗长}{rong3chang2}{4,4}{⼍,⾧}[HSK 7-9]
  \definition{adj.}{tediosamente longo; extenso; prolixo; extenso | incômodo}
\end{EntryWithPhonetic}

%%%%%%%%%% 柔 %%%%%%%%%%
\subsection*{柔}\addcontentsline{loh}{figure}{柔 \dpy{rou2}}

\begin{EntryWithPhonetic}{柔}{rou2}{9}{⽊}
  \definition*{s.}{Sobrenome: Rou}
  \definition{adj.}{macio; flexível; maleável | gentil; flexível; brando}
  \definition{v.}{tornar macio; amolecer | apaziguar}
\end{EntryWithPhonetic}

\begin{EntryWithPhonetic}{柔和}{rou2he2}{9,8}{⽊,⼝}[HSK 7-9]
  \definition{adj.}{suave; delicado; ameno; macio}
\end{EntryWithPhonetic}

\begin{EntryWithPhonetic}{柔软}{rou2ruan3}{9,8}{⽊,⾞}[HSK 7-9]
  \definition{adj.}{macio; flexível; maleável}
\end{EntryWithPhonetic}

%%%%%%%%%% 揉 %%%%%%%%%%
\subsection*{揉}\addcontentsline{loh}{figure}{揉 \dpy{rou2}}

\begin{EntryWithPhonetic}{揉}{rou2}{12}{⼿}[HSK 7-9]
  \definition{v.}{esfregar; esfregar ou friccionar com as mãos | amassar; enrolar | Literário: dobrar; torcer}
\end{EntryWithPhonetic}

\begin{EntryWithPhonetic}{揉碎}{rou2sui4}{12,13}{⼿,⽯}
  \definition{v.}{desfazer"-se em pedaços | esmagar}
\end{EntryWithPhonetic}

%%%%%%%%%% 肉 %%%%%%%%%%
\subsection*{肉}\addcontentsline{loh}{figure}{肉 \dpy{rou4}}

\begin{EntryWithPhonetic}{肉}{rou4}{6}{⾁}[HSK 1][Kangxi 130]
  \definition{adj.}{não crocante; mole | lento (em movimento); preguiçoso | carnal; erótico}
  \definition[块]{s.}{carne (especialmente carne de porco) | carne | polpa (da fruta)}
\end{EntryWithPhonetic}

\begin{EntryWithPhonetic}{肉桂}{rou4gui4}{6,10}{⾁,⽊}
  \definition{s.}{canela (árvore) | casca seca desta árvore; canela (uma especiaria aromática) | canela chinesa; cássia}
  \seealsoref{官桂}{guan1gui4}
\end{EntryWithPhonetic}

%%%%%%%%%% 如 %%%%%%%%%%
\subsection*{如}\addcontentsline{loh}{figure}{如 \dpy{ru2}}

\begin{EntryWithPhonetic}{如}{ru2}{6}{⼥}[HSK 6]
  \definition{adv.}{por exemplo; tal como; como}
  \definition{conj.}{se; no caso (de); no caso de; como se; como}
  \definition{prep.}{em conformidade com; de acordo com}
  \definition{v.}{estar em conformidade (ou de acordo) com | (geralmente no negativo) pode ser comparado com; ser comparável a; ser tão bom quanto | superar; exceder | (literário) ir para}
\end{EntryWithPhonetic}

\begin{EntryWithPhonetic}{如此}{ru2ci3}{6,6}{⼥,⽌}[HSK 5]
  \definition{adv.}{assim; tal; dessa forma; dessa maneira; refere"-se a uma situação mencionada anteriormente, equivalente a 这样}
  \seealsoref{这样}{zhe4yang4}
\end{EntryWithPhonetic}

\begin{EntryWithPhonetic}{如果}{ru2guo3}{6,8}{⼥,⽊}[HSK 2]
  \definition{conj.}{se; no caso de; na eventualidade de; supondo que; para expressar suposições, pode"-se usar 要是 na linguagem falada.}
  \seealsoref{要是}{yao4shi5}
\end{EntryWithPhonetic}

\begin{EntryWithPhonetic}{如果说}{ru2guo3 shuo1}{6,8,9}{⼥,⽊,⾔}[HSK 7-9]
  \definition{conj.}{se}[如果说今天没空,就明天见。===Se você estiver ocupado hoje, nos vemos amanhã.]
\end{EntryWithPhonetic}

\begin{EntryWithPhonetic}{如何}{ru2he2}{6,7}{⼥,⼈}[HSK 3]
  \definition{pron.}{como?; o que?; usado para perguntar como resolver um problema | como?; o que?; usado para perguntar sobre a situação ou obter a opinião de outras pessoas}
\end{EntryWithPhonetic}

\begin{EntryWithPhonetic}{如画}{ru2hua4}{6,8}{⼥,⽥}
  \definition{adj.}{pitoresco}
\end{EntryWithPhonetic}

\begin{EntryWithPhonetic}{如今}{ru2jin1}{6,4}{⼥,⼈}[HSK 4]
  \definition{s.}{agora; hoje em dia; atualmente; no presente}
\end{EntryWithPhonetic}

\begin{EntryWithPhonetic}{如实}{ru2shi2}{6,8}{⼥,⼧}[HSK 7-9]
  \definition{adv.}{factualmente; veridicamente; de acordo com a situação real}
\end{EntryWithPhonetic}

\begin{EntryWithPhonetic}{如同}{ru2tong2}{6,6}{⼥,⼝}[HSK 5]
  \definition{v.}{parecer que; usado principalmente em metáforas}
\end{EntryWithPhonetic}

\begin{EntryWithPhonetic}{如下}{ru2xia4}{6,3}{⼥,⼀}[HSK 5]
  \definition{adv.}{como descrito ou listado abaixo; conforme segue; conforme abaixo}
\end{EntryWithPhonetic}

\begin{EntryWithPhonetic}{如一}{ru2yi1}{6,1}{⼥,⼀}[HSK 6]
  \definition{adj.}{consistente; coerente}
\end{EntryWithPhonetic}

\begin{EntryWithPhonetic}{如意}{ru2/yi4}{6,13}{⼥,⼼}[HSK 7-9]
  \definition{adj.}{satisfeito; contente; descreve algo como a realização do desejo do coração}
  \definition{v.+compl.}{estar satisfeito; atender aos desejos de alguém; combinar com o meu gosto}
\end{EntryWithPhonetic}

\begin{EntryWithPhonetic}{如愿以偿}{ru2yuan4yi3chang2}{6,14,4,11}{⼥,⽕,⼈,⼈}[HSK 7-9]
  \definition{expr.}{``Como eu desejava.''; ter um desejo realizado; alcançar (ou obter) o que se deseja; o desejo foi atendido conforme o esperado, ou seja, a aspiração foi realizada}
  \definition{s.}{favorabilidade}
\end{EntryWithPhonetic}

\begin{EntryWithPhonetic}{如醉如痴}{ru2zui4-ru2chi1}{6,15,6,13}{⼥,⾣,⼥,⽧}[HSK 7-9]
  \definition{expr.}{embriagado; como se estivesse embriagado e atordoado; intoxicado por alguma coisa; louco por alguma coisa; obcecado por}
\end{EntryWithPhonetic}

%%%%%%%%%% 儒 %%%%%%%%%%
\subsection*{儒}\addcontentsline{loh}{figure}{儒 \dpy{ru2}}

\begin{EntryWithPhonetic}{儒}{ru2}{16}{⼈}
  \definition*{s.}{Confucionismo; Confucionista | Sobrenome: Ru}
  \definition{s.}{Obsoleto: erudito; homem culto | confucionismo}
\end{EntryWithPhonetic}

\begin{EntryWithPhonetic}{儒家}{ru2jia1}{16,10}{⼈,⼧}[HSK 7-9]
  \definition*{s.}{Confucionismo; Escola Confucionista; Confucionistas; uma corrente de pensamento do período pré"-Qin, representada por Confúcio, que defendia o governo por meio de ritos e enfatizava as relações éticas tradicionais}
\end{EntryWithPhonetic}

\begin{EntryWithPhonetic}{儒教}{ru2jiao4}{16,11}{⼈,⽁}
  \definition{s.}{confucionismo; o confucionismo, a partir das Dinastias do Norte e do Sul, era chamado de confucionismo enquanto religião e era mencionado juntamente com o budismo e o taoísmo}
  \seealsoref{儒家}{ru2jia1}
\end{EntryWithPhonetic}

\begin{EntryWithPhonetic}{儒学}{ru2xue2}{16,8}{⼈,⼦}[HSK 7-9]
  \definition{s.}{confucionismo; ensinamentos confucionistas | Obsoleto: escola administrada pelo governo em nível provincial, de prefeitura ou de condado durante as dinastias Yuan, Ming e Qing}
\end{EntryWithPhonetic}

%%%%%%%%%% 乳 %%%%%%%%%%
\subsection*{乳}\addcontentsline{loh}{figure}{乳 \dpy{ru3}}

\begin{EntryWithPhonetic}{乳}{ru3}{8}{⼄}
  \definition{adj.}{recém-nascido (animal); lactente}
  \definition{s.}{mama; peito | leite (em geral) | qualquer líquido semelhante ao leite}
  \definition{v.}{dar à luz}
\end{EntryWithPhonetic}

\begin{EntryWithPhonetic}{乳房}{ru3fang2}{8,8}{⼄,⼾}
  \definition{s.}{seio | mama | úbere}
\end{EntryWithPhonetic}

\begin{EntryWithPhonetic}{乳制品}{ru3zhi4pin3}{8,8,9}{⼄,⼑,⼝}[HSK 6]
  \definition{s.}{produtos lácteos}
\end{EntryWithPhonetic}

%%%%%%%%%% 辱 %%%%%%%%%%
\subsection*{辱}\addcontentsline{loh}{figure}{辱 \dpy{ru3}}

\begin{EntryWithPhonetic}{辱}{ru3}{10}{⾠}
  \definition*{s.}{Sobrenome: Ru}
  \definition{s.}{desgraça; desonra}
  \definition{v.}{trazer desgraça (ou humilhação) para | trazer desgraça; ser uma desgraça para | estar em dívida (com alguém por uma gentileza) | humilhar; insultar}
  \antonymref{荣}{rong2}
\end{EntryWithPhonetic}

\begin{EntryWithPhonetic}{辱骂}{ru3ma4}{10,9}{⾠,⾺}
  \definition{v.}{insultar | abusar}
\end{EntryWithPhonetic}

%%%%%%%%%% 入 %%%%%%%%%%
\subsection*{入}\addcontentsline{loh}{figure}{入 \dpy{ru4}}

\begin{EntryWithPhonetic}{入}{ru4}{2}{⼊}[HSK 6][Kangxi 11]
  \definition{s.}{renda | tom de entrada}
  \definition{v.}{entrar; entrar | juntar"-se; ser admitido em; tornar"-se membro de | conformar"-se com; concordar com | alcançar; atingir; entrar em (um certo nível ou estado) | fazer entrar; fazer algo entrar; fazer entrada}
  \antonymref{出}{chu1}
\end{EntryWithPhonetic}

\begin{EntryWithPhonetic}{入场}{ru4/chang3}{2,6}{⼊,⼟}[HSK 7-9]
  \definition{v.+compl.}{entrar; ser admitido; entrar no local}
\end{EntryWithPhonetic}

\begin{EntryWithPhonetic}{入场券}{ru4chang3quan4}{2,6,8}{⼊,⼟,⼑}[HSK 7-9]
  \definition{s.}{bilhete (de entrada) | pré-requisito para atingir um objetivo; qualificação para entrar em uma partida | ingresso; admissões}
\end{EntryWithPhonetic}

\begin{EntryWithPhonetic}{入党}{ru4dang3}{2,10}{⼊,⼉}
  \definition{v.}{ingressar em um partido político (especialmente o partido comunista)}
\end{EntryWithPhonetic}

\begin{EntryWithPhonetic}{入境}{ru4/jing4}{2,14}{⼊,⼟}[HSK 7-9]
  \definition{v.+compl.}{entrar em um país; imigrar}
\end{EntryWithPhonetic}

\begin{EntryWithPhonetic}{入口}{ru4/kou3}{2,3}{⼊,⼝}[HSK 2]
  \definition[个]{s.}{entrada; entrada em locais, edifícios, estradas, etc., através de portões ou portas}
  \definition{v.+compl.}{entrar na boca | importar; mercadorias estrangeiras importadas, às vezes também se refere a mercadorias de outras regiões importadas para esta região}
\end{EntryWithPhonetic}

\begin{EntryWithPhonetic}{入门}{ru4/men2}{2,3}{⼊,⾨}[HSK 5]
  \definition{s.}{(geralmente em títulos de livros) curso básico; manual introdutório | ABC; guia; refere"-se a leituras básicas; conhecimentos básicos}
  \definition{v.+compl.}{ultrapassar o limiar; aprender os rudimentos de um assunto | aprender o ABC de; ser introduzido a um assunto; aprender o básico}
\end{EntryWithPhonetic}

\begin{EntryWithPhonetic}{入侵}{ru4qin1}{2,9}{⼊,⼈}[HSK 7-9]
  \definition{v.}{invadir; intrometer"-se; fazer uma incursão; abrir caminho}
\end{EntryWithPhonetic}

\begin{EntryWithPhonetic}{入手}{ru4shou3}{2,4}{⼊,⼿}
  \definition{v.}{começar com; proceder a partir de; tomar como ponto de partida | obter; apoderar"-se  | começar; para dar início}
  \antonymref{出手}{chu1/shou3}
\end{EntryWithPhonetic}

\begin{EntryWithPhonetic}{入乡随俗}{ru4xiang1-sui2su2}{2,3,11,9}{⼊,⼄,⾩,⼈}
  \definition{expr.}{``Em roma, faça como os romanos!''}
\end{EntryWithPhonetic}

\begin{EntryWithPhonetic}{入选}{ru4xuan3}{2,9}{⼊,⾡}[HSK 7-9]
  \definition{v.}{ser escolhido; ser selecionado}
\end{EntryWithPhonetic}

\begin{EntryWithPhonetic}{入学}{ru4/xue2}{2,8}{⼊,⼦}[HSK 6]
  \definition{v.+compl.}{(uma criança) começar a escola; começar a escola primária | entrar em uma escola; matricular-se em uma escola}
\end{EntryWithPhonetic}

%%%%%%%%%% 软 %%%%%%%%%%
\subsection*{软}\addcontentsline{loh}{figure}{软 \dpy{ruan3}}

\begin{EntryWithPhonetic}{软}{ruan3}{8}{⾞}[HSK 5]
  \definition*{s.}{Sobrenome: Ruan}
  \definition{adj.}{macio; flexível; maleável; maleável | suave; brando; delicado | fraco; débil | de baixa qualidade, capacidade, etc. | facilmente movido (ou influenciado) | de maneira suave (ou gentil) | indulgente; tolerante | maleável; flexível | fácil de se emocionar ou abalar}
  \antonymref{硬}{ying4}
\end{EntryWithPhonetic}

\begin{EntryWithPhonetic}{软件}{ruan3jian4}{8,6}{⾞,⼈}[HSK 5]
  \definition[款,个]{s.}{\emph{software}; programas de computador, procedimentos, regras e quaisquer arquivos, documentos e dados relacionados à operação do sistema de computador}
\end{EntryWithPhonetic}

\begin{EntryWithPhonetic}{软弱}{ruan3ruo4}{8,10}{⾞,⼸}[HSK 7-9]
  \definition{adj.}{fraco; descreve o eu interior, a personalidade, etc. de alguém como fraco ou sem força | fraco; débil; flácido; descreve a falta de força física}
\end{EntryWithPhonetic}

\begin{EntryWithPhonetic}{软实力}{ruan3shi2li4}{8,8,2}{⾞,⼧,⼒}[HSK 7-9]
  \definition{s.}{\emph{soft power} (nas relações internacionais)}
\end{EntryWithPhonetic}

%%%%%%%%%% 锐 %%%%%%%%%%
\subsection*{锐}\addcontentsline{loh}{figure}{锐 \dpy{rui4}}

\begin{EntryWithPhonetic}{锐}{rui4}{12}{⾦}
  \definition*{s.}{Sobrenome: Rui}
  \definition{adj.}{afiado; aguçado | agudo; perspicaz | rápido; ágil; veloz}
  \definition{adv.}{rapidamente; de repente}
  \definition{s.}{vigor; espírito de luta | armas afiadas}
  \antonymref{钝}{dun4}
\end{EntryWithPhonetic}

%%%%%%%%%% 瑞 %%%%%%%%%%
\subsection*{瑞}\addcontentsline{loh}{figure}{瑞 \dpy{rui4}}

\begin{EntryWithPhonetic}{瑞}{rui4}{13}{⽟}
  \definition*{s.}{Sobrenome: Rui}
  \definition{adj.}{sortudo; auspicioso}
  \definition{s.}{ficha feita de jade; uma placa de jade usada como símbolo de autoridade e boa fé nos tempos antigos | sinal auspicioso; bom presságio; sorte}
\end{EntryWithPhonetic}

\begin{EntryWithPhonetic}{瑞雪}{rui4xue3}{13,11}{⽟,⾬}[HSK 7-9]
  \definition{s.}{neve oportuna (ou auspiciosa); neve boa e oportuna}
\end{EntryWithPhonetic}

%%%%%%%%%% 润 %%%%%%%%%%
\subsection*{润}\addcontentsline{loh}{figure}{润 \dpy{run4}}

\begin{EntryWithPhonetic}{润}{run4}{10}{⽔}[HSK 7-9]
  \definition{adj.}{úmido; molhado | liso; elegante}
  \definition{s.}{lucro; benefício}
  \definition{v.}{umedecer; lubrificar | embelezar; retocar | retocar; fazer com que fique radiante}
\end{EntryWithPhonetic}

%%%%%%%%%% 若 %%%%%%%%%%
\subsection*{若}\addcontentsline{loh}{figure}{若 \dpy{ruo4}}

\begin{EntryWithPhonetic}{若}{ruo4}{8}{⾋}[HSK 6]
  \definition*{s.}{Sobrenome: Ruo}
  \definition{adv.}{como se; como se fosse; usado antes do verbo para indicar que o que foi dito é mais ou menos assim, equivalente a 好像}
  \definition{conj.}{se; usado na primeira parte de uma frase composta, expressa uma relação hipotética, equivalente a 如果}
  \definition{pron.}{você; referir"-se ao interlocutor como 你 ou 你的}
  \definition{v.}{parecer}
  \seealsoref{好像}{hao3xiang4}
  \seealsoref{你}{ni3}
  \seealsoref{你的}{ni3 de5}
  \seealsoref{如果}{ru2guo3}
\end{EntryWithPhonetic}

\begin{EntryWithPhonetic}{若干}{ruo4gan1}{8,3}{⾋,⼲}[HSK 7-9]
  \definition{pron.}{alguns; vários; um certo número ou quantidade; significam 某些, 有些 ou 一些 | quantos?; quanto custa?}
  \seealsoref{某些}{mou3 xie1}
  \seealsoref{一些}{yi4xie1}
  \seealsoref{有些}{you3xie1}
\end{EntryWithPhonetic}

%%%%%%%%%% 弱 %%%%%%%%%%
\subsection*{弱}\addcontentsline{loh}{figure}{弱 \dpy{ruo4}}

\begin{EntryWithPhonetic}{弱}{ruo4}{10}{⼸}[HSK 4]
  \definition{adj.}{fraco; debilitado | jovem | inferior; pior | colocado depois de uma fração ou decimal para indicar que é um pouco menor que esse número}
  \definition{v.}{perder (através da morte)}
  \antonymref{强}{qiang2}
\end{EntryWithPhonetic}

\begin{EntryWithPhonetic}{弱点}{ruo4dian3}{10,9}{⼸,⽕}[HSK 7-9]
  \definition[个,种]{s.}{fraqueza; ponto fraco; áreas inadequadas; áreas frágeis}
\end{EntryWithPhonetic}

\begin{EntryWithPhonetic}{弱势}{ruo4shi4}{10,8}{⼸,⼒}[HSK 7-9]
  \definition{s.}{uma tendência de queda; o ímpeto está diminuindo ou enfraquecendo | fraqueza; forças fracas}
\end{EntryWithPhonetic}

%%%%% EOF %%%%%


 %%%
%%% S
%%%
\section*{S}\addcontentsline{toc}{section}{S}\addcontentsline{loh}{figure}{\#\#\#\#\#\#\#\# S}

%%%%%%%%%% 撒 %%%%%%%%%%
\subsection*{撒}\addcontentsline{loh}{figure}{撒 \dpy{sa1}}

\begin{EntryWithPhonetic}{撒}{sa1}{15}{⼿}[HSK 7-9]
  \definition{v.}{lançar; soltar; deixar escapar; liberar | abandonar todas as restrições; deixar-se levar; tentar usá-lo ou exibi-lo o máximo possível}
  \seeref{sa3}
\end{EntryWithPhonetic}

\begin{EntryWithPhonetic}{撒旦}{sa1dan4}{15,5}{⼿,⽇}
  \definition*{s.}{Satanás, que significa 抵挡, sinônimo do diabo nas histórias bíblicas; um termo cristão para alguém que se opõe especificamente a Deus e é um inimigo Dele | Satã; Diabo}
  \seealsoref{抵挡}{di3dang3}
\end{EntryWithPhonetic}

\begin{EntryWithPhonetic}{撒旦主义}{sa1dan4 zhu3yi4}{15,5,5,3}{⼿,⽇,⼂,⼂}
  \definition*{s.}{Satanismo}
\end{EntryWithPhonetic}

\begin{EntryWithPhonetic}{撒但}{sa1dan4}{15,7}{⼿,⼈}
  \variantof{撒旦}
  \seealsoref{撒旦}{sa1dan4}
\end{EntryWithPhonetic}

\begin{EntryWithPhonetic}{撒谎}{sa1/huang3}{15,11}{⼿,⾔}[HSK 7-9]
  \definition{v.+compl.}{mentir; contar uma mentira}
\end{EntryWithPhonetic}

%%%%%%%%%% 洒 %%%%%%%%%%
\subsection*{洒}\addcontentsline{loh}{figure}{洒 \dpy{sa3}}

\begin{EntryWithPhonetic}{洒}{sa3}{9}{⽔}[HSK 5]
  \definition{adj.}{natural e sem restrições; confortável (sem restrições)}
  \definition{v.}{derramar; espalhar; borrifar; salpicar; fazer com que (água ou outra coisa) caia de forma dispersa | derramar; cair de forma dispersa}
\end{EntryWithPhonetic}

\begin{EntryWithPhonetic}{洒水}{sa3shui3}{9,4}{⽔,⽔}
  \definition{v.}{borrifar}
\end{EntryWithPhonetic}

%%%%%%%%%% 撒 %%%%%%%%%%
\subsection*{撒}\addcontentsline{loh}{figure}{撒 \dpy{sa3}}

\begin{EntryWithPhonetic}{撒}{sa3}{15}{⼿}
  \definition*{s.}{Sobrenome: Sa}
  \definition{v.}{espalhar; polvilhar; difundir; lançar | derramar; deixar cair}
\end{EntryWithPhonetic}

%%%%%%%%%% 飒 %%%%%%%%%%
\subsection*{飒}\addcontentsline{loh}{figure}{飒 \dpy{sa4}}

\begin{EntryWithPhonetic}{飒}{sa4}{9}{⾵}
  \definition{adj.}{(das mulheres) natural e desenfreada; elegante; valente}
  \definition{interj.}{(onomatopéia) farfalhar; sussurrar | (onomatopéia) som do vento}
  \definition{v.}{murchar}
\end{EntryWithPhonetic}

\begin{EntryWithPhonetic}{飒飒}{sa4sa4}{9,9}{⾵,⾵}
  \definition{s.}{o farfalhar | sussurro | murmúrio (do vento nas árvores, o mar, etc.)}
\end{EntryWithPhonetic}

%%%%%%%%%% 塞 %%%%%%%%%%
\subsection*{塞}\addcontentsline{loh}{figure}{塞 \dpy{sai1}}

\begin{EntryWithPhonetic}{塞}{sai1}{13}{⼟}[HSK 6]
  \definition{s.}{rolha; plugue}
  \definition{v.}{encher; conectar; preencher; espremer; bloquear | superar (para comparação)}
\end{EntryWithPhonetic}

%%%%%%%%%% 赛 %%%%%%%%%%
\subsection*{赛}\addcontentsline{loh}{figure}{赛 \dpy{sai4}}

\begin{EntryWithPhonetic}{赛}{sai4}{14}{⾙}[HSK 6]
  \definition*{s.}{Sobrenome: Sai}
  \definition{s.}{jogo; partida; competição | sacrifício; cerimônia de sacrifício; antigamente, sacrifícios eram feitos para agradecer aos deuses por suas dádivas}
  \definition{v.}{ter uma competição (comparando alto e baixo, forte e fraco) | superar; ser comparável a; comparar com}
\end{EntryWithPhonetic}

\begin{EntryWithPhonetic}{赛场}{sai4 chang3}{14,6}{⾙,⼟}[HSK 6]
  \definition{s.}{local de competição; arena; ringue; terreno | campo (para competição de atletismo) | pista de corrida}
\end{EntryWithPhonetic}

\begin{EntryWithPhonetic}{赛车}{sai4che1}{14,4}{⾙,⾞}[HSK 7-9]
  \definition{s.}{veículo de corrida; carro de corrida; bicicletas, motocicletas ou carros de corrida | corridas de carros}
  \definition{v.}{correr; disputar uma corrida}
\end{EntryWithPhonetic}

\begin{EntryWithPhonetic}{赛跑}{sai4pao3}{14,12}{⾙,⾜}[HSK 7-9]
  \definition{v.}{correr; disputar uma corrida; esportes que testam a velocidade de corrida incluem provas de curta, média, longa e ultra-longa distância, além de corridas com barreiras, revezamentos, corridas com obstáculos e corridas de cross-country}
\end{EntryWithPhonetic}

%%%%%%%%%% 三 %%%%%%%%%%
\subsection*{三}\addcontentsline{loh}{figure}{三 \dpy{san1}}

\begin{EntryWithPhonetic}{三}{san1}{3}{⼀}[HSK 1]
  \definition*{s.}{Sobrenome: San}
  \definition{num.}{três; 3 | muitos; vários; mais de dois; referindo-se a muitos ou à maioria | alguns; poucos; menos; não muitos}
\end{EntryWithPhonetic}

\begin{EntryWithPhonetic}{三番五次}{san1fan1-wu3ci4}{3,12,4,6}{⼀,⽥,⼆,⽋}[HSK 7-9]
  \definition{expr.}{repetidamente; de novo e de novo; várias e várias vezes; diversas vezes}
\end{EntryWithPhonetic}

\begin{EntryWithPhonetic}{三角}{san1jiao3}{3,7}{⼀,⾓}[HSK 7-9]
  \definition{adj.}{tripartido; que constitui uma relação tripartite}
  \definition[个,些]{s.}{triângulo; coisas triangulares | trigonometria, abreviação de 三角学}
  \seealsoref{三角学}{san1jiao3 xue2}
\end{EntryWithPhonetic}

\begin{EntryWithPhonetic}{三角恋爱}{san1jiao3lian4'ai4}{3,7,10,10}{⼀,⾓,⼼,⽖}
  \definition[场]{s.}{triângulo amoroso | triângulo eterno}
\end{EntryWithPhonetic}

\begin{EntryWithPhonetic}{三角学}{san1jiao3 xue2}{3,7,8}{⼀,⾓,⼦}
  \definition{s.}{trigonometria; um ramo da matemática que estuda principalmente as funções trigonométricas e suas propriedades, bem como suas aplicações em geometria}[我三角学学得很好。===Sou muito bom em trigonometria.]
\end{EntryWithPhonetic}

\begin{EntryWithPhonetic}{三轮车}{san1lun2che1}{3,8,4}{⼀,⾞,⾞}
  \definition{s.}{triciclo}
\end{EntryWithPhonetic}

\begin{EntryWithPhonetic}{三明治}{san1 ming2 zhi4}{3,8,8}{⼀,⽇,⽔}[HSK 6]
  \definition[个,些,块]{s.}{Empréstimo linguístico: sanduíche, \emph{sandwich}}
\end{EntryWithPhonetic}

\begin{EntryWithPhonetic}{三维}{san1wei2}{3,11}{⼀,⽷}[HSK 7-9]
  \definition{s.}{três dimensões; 3D; tridimensional}[我们生活在三维空间。===Vivemos em um espaço tridimensional.]
\end{EntryWithPhonetic}

%%%%%%%%%% 伞 %%%%%%%%%%
\subsection*{伞}\addcontentsline{loh}{figure}{伞 \dpy{san3}}

\begin{EntryWithPhonetic}{伞}{san3}{6}{⼈}[HSK 4]
  \definition*{s.}{Sobrenome: San}
  \definition[把]{s.}{guarda-chuva; proteção contra chuva ou sol | algo que tem o formato de um guarda-chuva}
\end{EntryWithPhonetic}

%%%%%%%%%% 散 %%%%%%%%%%
\subsection*{散}\addcontentsline{loh}{figure}{散 \dpy{san3}}

\begin{EntryWithPhonetic}{散}{san3}{12}{⽁}[HSK 5]
  \definition{adj.}{disperso; fragmentado; não integrado}
  \definition{s.}{medicamento em forma de pó}
  \definition{v.}{divergir; espalhar-se; separar-se; soltar-se; não se manter unido;  desintegrar}
  \seeref{san4}
\end{EntryWithPhonetic}

\begin{EntryWithPhonetic}{散文}{san3wen2}{12,4}{⽁,⽂}[HSK 5]
  \definition[篇,种]{s.}{ensaio; prosa; gênero literário, na antiguidade, referia-se a textos em prosa, em oposição à poesia e à prosa paralela; atualmente, refere-se a obras literárias que não sejam poesia, teatro ou romance, incluindo ensaios, contos, crônicas, relatos de viagem, etc.}
\end{EntryWithPhonetic}

\begin{EntryWithPhonetic}{散}{san4}{12}{⽁}
  \definition{v.}{quebrar; fragmentar; dispersar | dar; distribuir; disseminar; divulgar | dissipar; deixar sai  | terminar um acordo ou contrato; demitir}
  \seeref{san3}
\end{EntryWithPhonetic}

\begin{EntryWithPhonetic}{散布}{san4bu4}{12,5}{⽁,⼱}[HSK 7-9]
  \definition{v.}{espalhar; distribuir; disseminar | espalhar; propagar}
\end{EntryWithPhonetic}

\begin{EntryWithPhonetic}{散步}{san4/bu4}{12,7}{⽁,⽌}[HSK 3]
  \definition{v.+compl.}{dar uma volta; dar um passeio; dar uma caminhada}
\end{EntryWithPhonetic}

\begin{EntryWithPhonetic}{散发}{san4fa4}{12,5}{⽁,⼜}[HSK 7-9]
  \definition{v.}{emitir; difundir; enviar; divulgar | emitir; distribuir; dar}
\end{EntryWithPhonetic}

\begin{EntryWithPhonetic}{散心}{san4/xin1}{12,4}{⽁,⼼}
  \definition{v.+compl.}{aliviar o tédio | desfrutar de uma diversão | estar despreocupado}
\end{EntryWithPhonetic}

%%%%%%%%%% 丧 %%%%%%%%%%
\subsection*{丧}\addcontentsline{loh}{figure}{丧 \dpy{sang1}}

\begin{EntryWithPhonetic}{丧}{sang1}{8}{⼗}
  \definition{adj.}{decepcionado; deprimido; desanimado}
  \definition{v.}{perder | desanimar; frustrar}
  \seeref{sang4}
\end{EntryWithPhonetic}

\begin{EntryWithPhonetic}{丧钟}{sang1zhong1}{8,9}{⼗,⾦}
  \definition{s.}{sentença de morte}
\end{EntryWithPhonetic}

%%%%%%%%%% 桑 %%%%%%%%%%
\subsection*{桑}\addcontentsline{loh}{figure}{桑 \dpy{sang1}}

\begin{EntryWithPhonetic}{桑}{sang1}{10}{⽊}
  \definition*{s.}{Sobrenome: Sang}
  \definition[棵]{s.}{amoreira}
\end{EntryWithPhonetic}

\begin{EntryWithPhonetic}{桑巴舞}{sang1ba1wu3}{10,4,14}{⽊,⼰,⾇}
  \definition{s.}{samba}
\end{EntryWithPhonetic}

\begin{EntryWithPhonetic}{桑拿}{sang1na2}{10,10}{⽊,⼿}[HSK 7-9]
  \definition{s.}{Empréstimo Linguístico: sauna; banho turco}
\end{EntryWithPhonetic}

\begin{EntryWithPhonetic}{桑树}{sang1shu4}{10,9}{⽊,⽊}
  \definition{s.}{amoreira, suas folhas são utilizadas para alimentar bichos-da-seda}
\end{EntryWithPhonetic}

%%%%%%%%%% 嗓 %%%%%%%%%%
\subsection*{嗓}\addcontentsline{loh}{figure}{嗓 \dpy{sang3}}

\begin{EntryWithPhonetic}{嗓}{sang3}{13}{⼝}
  \definition{s.}{garganta; laringe | voz}
  \seealsoref{嗓儿}{sang3r5}
\end{EntryWithPhonetic}

\begin{EntryWithPhonetic}{嗓儿}{sang3r5}{13,2}{⼝,⼉}
  \definition{s.}{garganta}
\end{EntryWithPhonetic}

\begin{EntryWithPhonetic}{嗓子}{sang3zi5}{13,3}{⼝,⼦}[HSK 7-9]
  \definition[副,个]{s.}{garganta; laringe | voz}
\end{EntryWithPhonetic}

%%%%%%%%%% 丧 %%%%%%%%%%
\subsection*{丧}\addcontentsline{loh}{figure}{丧 \dpy{sang4}}

\begin{EntryWithPhonetic}{丧}{sang4}{8}{⼗}
  \definition{adj.}{decepcionado | desanimado}
  \definition{v.}{estar enlutado (do cônjuge etc.) | morrer}
  \seeref{sang1}
\end{EntryWithPhonetic}

\begin{EntryWithPhonetic}{丧身}{sang4shen1}{8,7}{⼗,⾝}
  \definition{v.}{morrer; perder a vida}
  \seealsoref{丧生}{sang4/sheng1}
\end{EntryWithPhonetic}

\begin{EntryWithPhonetic}{丧生}{sang4/sheng1}{8,5}{⼗,⽣}[HSK 7-9]
  \definition{v.+compl.}{morrer; encontrar a morte; perder a vida; ser morto}
  \seealsoref{丧身}{sang4shen1}
\end{EntryWithPhonetic}

\begin{EntryWithPhonetic}{丧失}{sang4shi1}{8,5}{⼗,⼤}[HSK 6]
  \definition{v.}{perder (algo que se tem)}
\end{EntryWithPhonetic}

%%%%%%%%%% 骚 %%%%%%%%%%
\subsection*{骚}\addcontentsline{loh}{figure}{骚 \dpy{sao1}}

\begin{EntryWithPhonetic}{骚}{sao1}{12}{⾺}
  \definition*{s.}{Abreviação de Li Sao (Encontrando a Tristeza), um poema do poeta e estadista do século IV a.C. Qu Yuan (屈原)}
  \definition{adj.}{coquete; (de uma mulher) lasciva | masculino (de alguns animais domésticos)}
  \definition{s.}{escritos literários; geralmente se refere à poesia | o cheiro de urina; mau cheiro}
  \definition{v.}{perturbar}
  \seealsoref{屈原}{qu1yuan2}
\end{EntryWithPhonetic}

\begin{EntryWithPhonetic}{骚乱}{sao1luan4}{12,7}{⾺,⼄}[HSK 7-9]
  \definition{s.}{rebelião; perturbação; motim; confusão}
  \definition{v.}{criar perturbação; estar em meio a uma turbulência; causar problemas}
\end{EntryWithPhonetic}

\begin{EntryWithPhonetic}{骚扰}{sao1rao3}{12,7}{⾺,⼿}[HSK 7-9]
  \definition{v.}{assediar; molestar; perturbar}
\end{EntryWithPhonetic}

%%%%%%%%%% 扫 %%%%%%%%%%
\subsection*{扫}\addcontentsline{loh}{figure}{扫 \dpy{sao3}}

\begin{EntryWithPhonetic}{扫}{sao3}{6}{⼿}[HSK 4]
  \definition{v.}{varrer; limpar | passar rapidamente ao longo ou sobre; varrer | juntar tudo | Computação: scanear}
  \seeref{sao4}
\end{EntryWithPhonetic}

\begin{EntryWithPhonetic}{扫除}{sao3chu2}{6,9}{⼿,⾩}[HSK 7-9]
  \definition{s.}{limpeza; arrumação}
  \definition{v.}{limpar; arrumar | limpar; remover; eliminar}
\end{EntryWithPhonetic}

\begin{EntryWithPhonetic}{扫描}{sao3miao2}{6,11}{⼿,⼿}[HSK 7-9]
  \definition{v.}{digitalizar | dar uma olhada rápida; percorrer (o olhar, etc.) | utilizar software especializado para inspecionar e pesquisar (dados, vírus, etc. em computadores)}
\end{EntryWithPhonetic}

\begin{EntryWithPhonetic}{扫墓}{sao3/mu4}{6,13}{⼿,⼟}[HSK 7-9]
  \definition{v.+compl.}{limpar sepulturas e prestar homenagens aos mortos; também se refere à realização de atividades comemorativas nos túmulos dos mártires}
\end{EntryWithPhonetic}

\begin{EntryWithPhonetic}{扫兴}{sao3/xing4}{6,6}{⼿,⼋}[HSK 7-9]
  \definition{v.+compl.}{sentir-se desapontado; ter o ânimo abalado; quando você está se sentindo feliz, algo desagradável pode abalar seu ânimo}
\end{EntryWithPhonetic}

%%%%%%%%%% 嫂 %%%%%%%%%%
\subsection*{嫂}\addcontentsline{loh}{figure}{嫂 \dpy{sao3}}

\begin{EntryWithPhonetic}{嫂}{sao3}{12}{⼥}
  \definition[个,位,名,些]{s.}{esposa do irmão mais velho; cunhada | irmã (uma forma de tratamento para uma mulher casada, mais ou menos da mesma idade)}
\end{EntryWithPhonetic}

\begin{EntryWithPhonetic}{嫂子}{sao3zi5}{12,3}{⼥,⼦}[HSK 7-9]
  \definition[个,名,位]{s.}{esposa do irmão mais velho; cunhada}
\end{EntryWithPhonetic}

%%%%%%%%%% 扫 %%%%%%%%%%
\subsection*{扫}\addcontentsline{loh}{figure}{扫 \dpy{sao4}}

\begin{EntryWithPhonetic}{扫}{sao4}{6}{⼿}
  \definition{s.}{elemento formadore de palavra}
  \seeref{sao3}
  \seealsoref{扫帚}{sao4zhou5}
\end{EntryWithPhonetic}

\begin{EntryWithPhonetic}{扫帚}{sao4zhou5}{6,8}{⼿,⼱}
  \definition[把,个]{s.}{vassoura; ferramenta de varredura feita de varas de bambu, etc., maior que uma vassora}
\end{EntryWithPhonetic}

%%%%%%%%%% 色 %%%%%%%%%%
\subsection*{色}\addcontentsline{loh}{figure}{色 \dpy{se4}}

\begin{EntryWithPhonetic}{色}{se4}{6}{⾊}[HSK 4][Kangxi 139]
  \definition*{s.}{Sobrenome: Se}
  \definition[种]{s.}{cor | aparência; semblante; expressão | tipo; gênero; descrição | cena; cenário;  paisagem | qualidade (de metais preciosos, mercadorias, etc.) | aparência feminina; beleza feminina | erotismo; apetite sexual; luxúria; desejo sexual}
  \seeref{shai3}
\end{EntryWithPhonetic}

\begin{EntryWithPhonetic}{色彩}{se4cai3}{6,11}{⾊,⼺}[HSK 4]
  \definition[种,丝]{s.}{cor; matiz; tonalidade | cor; sabor; característica; metáfora para um determinado estado de espírito ou tendência de pensamento}
\end{EntryWithPhonetic}

\begin{EntryWithPhonetic}{色狼}{se4lang2}{6,10}{⾊,⽝}
  \definition*{s.}{Sátiro}
  \definition{adj.}{lascivo; lobo; pervertido; isso se refere a uma pessoa que persegue mulheres com ganância e as agride sexualmente de forma brutal}
\end{EntryWithPhonetic}

%%%%%%%%%% 森 %%%%%%%%%%
\subsection*{森}\addcontentsline{loh}{figure}{森 \dpy{sen1}}

\begin{EntryWithPhonetic}{森}{sen1}{12}{⽊}
  \definition{adj.}{cheio de árvores | multitudinário; em multidões | escuro; sombrio}
\end{EntryWithPhonetic}

\begin{EntryWithPhonetic}{森林}{sen1lin2}{12,8}{⽊,⽊}[HSK 4]
  \definition[片,座,处]{s.}{floresta; bosque; normalmente, refere-se a uma grande área de árvores em crescimento; na silvicultura, refere-se a um grande número de árvores que crescem em uma área razoavelmente grande de terra, juntamente com os animais e outras plantas}
\end{EntryWithPhonetic}

%%%%%%%%%% 僧 %%%%%%%%%%
\subsection*{僧}\addcontentsline{loh}{figure}{僧 \dpy{seng1}}

\begin{EntryWithPhonetic}{僧}{seng1}{14}{⼈}
  \definition*{s.}{Sobrenome: Seng}
  \definition[位,名,个]{s.}{monge Budista, abreviação de 僧伽}
  \seealsoref{僧伽}{seng1qie2}
\end{EntryWithPhonetic}

\begin{EntryWithPhonetic}{僧伽}{seng1qie2}{14,7}{⼈,⼈}
  \definition{s.}{sangha ou sanga (Budismo) | a comunidade monástica | monge}
\end{EntryWithPhonetic}

\begin{EntryWithPhonetic}{僧人}{seng1ren2}{14,2}{⼈,⼈}[HSK 7-9]
  \definition{s.}{monge budista | monge}
\end{EntryWithPhonetic}

%%%%%%%%%% 杀 %%%%%%%%%%
\subsection*{杀}\addcontentsline{loh}{figure}{杀 \dpy{sha1}}

\begin{EntryWithPhonetic}{杀}{sha1}{6}{⽊}[HSK 5]
  \definition{adv.}{em extremo; excessivamente; usado após um verbo, indica grau intenso}
  \definition{v.}{matar; abater; esquartejar | lutar; entrar em batalha | enfraquecer; reduzir; diminuir | decolar; neutralizar}
\end{EntryWithPhonetic}

\begin{EntryWithPhonetic}{杀毒}{sha1 du2}{6,9}{⽊,⽏}[HSK 5]
  \definition{s.}{Computação: antivírus}
  \definition{v.}{esterilizar; desinfetar | Computação: eliminar um vírus}
\end{EntryWithPhonetic}

\begin{EntryWithPhonetic}{杀害}{sha1hai4}{6,10}{⽊,⼧}[HSK 7-9]
  \definition{v.}{assassinar; massacrar; trucidar; chacinar; matar (uma pessoa) por motivos ilegítimos}
\end{EntryWithPhonetic}

\begin{EntryWithPhonetic}{杀气}{sha1qi4}{6,4}{⽊,⽓}
  \definition{s.}{espírito assassino | aura de morte}
  \definition{v.}{desabafar a raiva de alguém}
\end{EntryWithPhonetic}

\begin{EntryWithPhonetic}{杀手}{sha1shou3}{6,4}{⽊,⼿}[HSK 7-9]
  \definition{s.}{assassino; homicida | pessoa magistral (de certo tipo) | Esporte: jogador formidável | assassino de aluguel}
\end{EntryWithPhonetic}

%%%%%%%%%% 沙 %%%%%%%%%%
\subsection*{沙}\addcontentsline{loh}{figure}{沙 \dpy{sha1}}

\begin{EntryWithPhonetic}{沙}{sha1}{7}{⽔}
  \definition*{s.}{Sobrenome: Sha}
  \definition{adj.}{granulado; em pó | rouco}[我今天感冒了,嗓音有点沙哑。===Estou resfriado hoje e minha voz está um pouco rouca.]
  \definition[车,把,袋,吨]{s.}{areia; cascalho; grânulo; pó}
\end{EntryWithPhonetic}

\begin{EntryWithPhonetic}{沙发}{sha1fa1}{7,5}{⽔,⼜}[HSK 3]
  \definition[套,组,个,张]{s.}{sofá; assentos com molas ou espuma plástica espessa, etc., com apoios de braços em ambos os lados}
\end{EntryWithPhonetic}

\begin{EntryWithPhonetic}{沙龙}{sha1long2}{7,5}{⽔,⿓}[HSK 7-9]
  \definition{s.}{Empréstimo linguístico: \emph{salon}; salão; sala de estar}
\end{EntryWithPhonetic}

\begin{EntryWithPhonetic}{沙漠}{sha1mo4}{7,13}{⽔,⽔}[HSK 5]
  \definition[个,片]{s.}{deserto; superfície totalmente coberta por areia, sem água corrente, clima seco e vegetação escassa}
\end{EntryWithPhonetic}

\begin{EntryWithPhonetic}{沙滩}{sha1tan1}{7,13}{⽔,⽔}[HSK 7-9]
  \definition[片,个]{s.}{areia; praia; praia arenosa; terreno formado por depósitos de areia dentro ou perto da água}
\end{EntryWithPhonetic}

\begin{EntryWithPhonetic}{沙糖}{sha1tang2}{7,16}{⽔,⽶}
  \variantof{砂糖}
\end{EntryWithPhonetic}

\begin{EntryWithPhonetic}{沙特}{sha1te4}{7,10}{⽔,⽜}
  \definition*{s.}{Saudita | Arábia Saudita, abreviação de 沙特阿拉伯}
  \seealsoref{沙特阿拉伯}{sha1te4 a1la1bo2}
\end{EntryWithPhonetic}

\begin{EntryWithPhonetic}{沙特阿拉伯}{sha1te4 a1la1bo2}{7,10,7,8,7}{⽔,⽜,⾩,⼿,⼈}
  \definition*{s.}{Arábia Saudita}
\end{EntryWithPhonetic}

\begin{EntryWithPhonetic}{沙鱼}{sha1yu2}{7,8}{⽔,⿂}
  \variantof{鲨鱼}
\end{EntryWithPhonetic}

\begin{EntryWithPhonetic}{沙子}{sha1 zi5}{7,3}{⽔,⼦}[HSK 3]
  \definition[粒,把,堆,袋,车]{s.}{areia; grão; pequenas pedras | \emph{pellets}; grãos pequenos; coisas parecidas com areia}
\end{EntryWithPhonetic}

%%%%%%%%%% 纱 %%%%%%%%%%
\subsection*{纱}\addcontentsline{loh}{figure}{纱 \dpy{sha1}}

\begin{EntryWithPhonetic}{纱}{sha1}{7}{⽷}[HSK 7-9]
  \definition[层,块儿]{s.}{fio; filamentos finos e soltos, fiados a partir de algodão, cânhamo, etc., podem ser torcidos para formar fios ou tecidos para produzir tecido | gaze; tecidos confeccionados com fios de urdidura e trama muito finos | produtos têxteis; produtos como telas de janela | nomes de classes de certos tecidos}
\end{EntryWithPhonetic}

%%%%%%%%%% 刹 %%%%%%%%%%
\subsection*{刹}\addcontentsline{loh}{figure}{刹 \dpy{sha1}}

\begin{EntryWithPhonetic}{刹}{sha1}{8}{⼑}
  \definition{v.}{acionar o(s) freio(s); frear; brecar}
  \seeref{cha4}
\end{EntryWithPhonetic}

\begin{EntryWithPhonetic}{刹车}{sha1/che1}{8,4}{⼑,⾞}[HSK 7-9]
  \definition{s.}{freios; o mecanismo que impede o veículo de se mover}
  \definition{v.+compl.}{frear; pisar nos freios; utilizar os freios ou outros mecanismos para parar o movimento do veículo ou interromper o funcionamento da máquina | desligar uma máquina; parar uma máquina cortando a energia | desligue ou desconectar a fonte de alimentação para interromper o funcionamento da máquina | interromper (um projeto, etc.); uma metáfora para interromper algo imediatamente}
\end{EntryWithPhonetic}

\begin{EntryWithPhonetic}{刹多罗}{sha1duo1luo2}{8,6,8}{⼑,⼣,⽹}
  \definition*{s.}{Kshatara, sânscrito ``ksetra''}
\end{EntryWithPhonetic}

%%%%%%%%%% 砂 %%%%%%%%%%
\subsection*{砂}\addcontentsline{loh}{figure}{砂 \dpy{sha1}}

\begin{EntryWithPhonetic}{砂}{sha1}{9}{⽯}
  \variantof{沙}
\end{EntryWithPhonetic}

\begin{EntryWithPhonetic}{砂糖}{sha1tang2}{9,16}{⽯,⽶}[HSK 7-9]
  \definition{s.}{açúcar granulado}
\end{EntryWithPhonetic}

%%%%%%%%%% 莎 %%%%%%%%%%
\subsection*{莎}\addcontentsline{loh}{figure}{莎 \dpy{sha1}}

\begin{EntryWithPhonetic}{莎}{sha1}{10}{⾋}
  \definition{s.}{em nomes pessoais e de lugares | cigarra | fonético ``sha'' usado na transliteração}
  \seeref{suo1}
\end{EntryWithPhonetic}

\begin{EntryWithPhonetic}{莎莎舞}{sha1sha1wu3}{10,10,14}{⾋,⾋,⾇}
  \definition{s.}{salsa (dança)}
\end{EntryWithPhonetic}

%%%%%%%%%% 鲨 %%%%%%%%%%
\subsection*{鲨}\addcontentsline{loh}{figure}{鲨 \dpy{sha1}}

\begin{EntryWithPhonetic}{鲨}{sha1}{15}{⿂}
  \definition[只,条]{s.}{tubarão}
\end{EntryWithPhonetic}

\begin{EntryWithPhonetic}{鲨鱼}{sha1yu2}{15,8}{⿂,⿂}[HSK 7-9]
  \definition[只,群,条]{s.}{tubarão}
\end{EntryWithPhonetic}

%%%%%%%%%% 啥 %%%%%%%%%%
\subsection*{啥}\addcontentsline{loh}{figure}{啥 \dpy{sha2}}

\begin{EntryWithPhonetic}{啥}{sha2}{11}{⼝}
  \definition{pron.}{Dialeto: O que?; equivalente a 什么}
\end{EntryWithPhonetic}

%%%%%%%%%% 傻 %%%%%%%%%%
\subsection*{傻}\addcontentsline{loh}{figure}{傻 \dpy{sha3}}

\begin{EntryWithPhonetic}{傻}{sha3}{13}{⼈}[HSK 5]
  \definition{adj.}{estúpido; confuso; burro; idiota; inflexível; (ação ou pensamento) mecânico}
\end{EntryWithPhonetic}

\begin{EntryWithPhonetic}{傻瓜}{sha3gua1}{13,5}{⼈,⽠}[HSK 7-9]
  \definition[个,群]{adj.}{tolo; idiota; cabeça-dura; simplório; uma pessoa estúpida ou lenta para reagir; termo frequentemente usado como insulto ou piada}
  \definition{v.}{enganar | iludir | lograr}
\end{EntryWithPhonetic}

\begin{EntryWithPhonetic}{傻眼}{sha3yan3}{13,11}{⼈,⽬}
  \definition{adj.}{estupefato | atordoado}
\end{EntryWithPhonetic}

%%%%%%%%%% 嗄 %%%%%%%%%%
\subsection*{嗄}\addcontentsline{loh}{figure}{嗄 \dpy{sha4}}

\begin{EntryWithPhonetic}{嗄}{sha4}{13}{⼝}
  \definition{adj.}{rouco}
  \seeref{a2}
\end{EntryWithPhonetic}

%%%%%%%%%% 筛 %%%%%%%%%%
\subsection*{筛}\addcontentsline{loh}{figure}{筛 \dpy{shai1}}

\begin{EntryWithPhonetic}{筛}{shai1}{12}{⽵}[HSK 7-9]
  \definition{s.}{peneira; coador; tela}
  \definition{v.}{peneirar; crivar; triar | aquecer o vinho sobre uma fogueira; aquecer uma panela de vinho em fogo baixo | servir vinho, chá, etc. | bater; golpear | Dialeto: bater (o gongo)}
\end{EntryWithPhonetic}

\begin{EntryWithPhonetic}{筛选}{shai1xuan3}{12,9}{⽵,⾡}[HSK 7-9]
  \definition{v.}{selecionar; filtrar; eliminar por seleção; eliminar o que é ruim e selecionar o que é bom}
\end{EntryWithPhonetic}

%%%%%%%%%% 色 %%%%%%%%%%
\subsection*{色}\addcontentsline{loh}{figure}{色 \dpy{shai3}}

\begin{EntryWithPhonetic}{色}{shai3}{6}{⾊}[Kangxi 139]
  \definition[4]{s.}{cor; (~儿) tem o mesmo significado que 色, usado em algumas palavras faladas}
  \seeref{se4}
\end{EntryWithPhonetic}

%%%%%%%%%% 晒 %%%%%%%%%%
\subsection*{晒}\addcontentsline{loh}{figure}{晒 \dpy{shai4}}

\begin{EntryWithPhonetic}{晒}{shai4}{10}{⽇}[HSK 4]
  \definition{v.}{(sol) brilhar sobre | aquecer-se; secar ao sol; colocar algo sob a luz do sol para secar | ignorar (alguém) | mostrar; divulgar o conteúdo de sua vida privada na Internet}
\end{EntryWithPhonetic}

\begin{EntryWithPhonetic}{晒干}{shai4gan1}{10,3}{⽇,⼲}
  \definition{v.}{secar ao sol}
\end{EntryWithPhonetic}

\begin{EntryWithPhonetic}{晒太阳}{shai4 tai4yang2}{10,4,6}{⽇,⼤,⾩}[HSK 7-9]
  \definition{v.}{colocar algo ao sol (por exemplo, para secar); estar ao sol (aquecer-se ou tomar banho de sol, etc.); absorver luz e calor sob a luz solar}
\end{EntryWithPhonetic}

%%%%%%%%%% 山 %%%%%%%%%%
\subsection*{山}\addcontentsline{loh}{figure}{山 \dpy{shan1}}

\begin{EntryWithPhonetic}{山}{shan1}{3}{⼭}[HSK 1][Kangxi 46]
  \definition*{s.}{Sobrenome: Shan}
  \definition[座]{s.}{colina; maciço; montanha | qualquer coisa que se assemelhe a uma montanha | arbustos nos quais os bichos-da-seda tecem seus casulos; referindo-se a casulos de bicho-da-seda | eco; metáfora para um som muito alto}
\end{EntryWithPhonetic}

\begin{EntryWithPhonetic}{山川}{shan1chuan1}{3,3}{⼭,⼮}[HSK 7-9]
  \definition{s.}{montanhas e rios; paisagem}
\end{EntryWithPhonetic}

\begin{EntryWithPhonetic}{山顶}{shan1ding3}{3,8}{⼭,⾴}[HSK 7-9]
  \definition{s.}{topo de uma colina; cume de uma montanha; o pico de uma montanha; o topo da montanha}[他们距山顶还有100米远。===Eles ainda estavam a 100 metros do cume.]
\end{EntryWithPhonetic}

\begin{EntryWithPhonetic}{山东}{shan1dong1}{3,5}{⼭,⼀}
  \definition*{s.}{Província de Shandong (Shantung) no nordeste da China}
\end{EntryWithPhonetic}

\begin{EntryWithPhonetic}{山峰}{shan1 feng1}{3,10}{⼭,⼭}[HSK 6]
  \definition[座,个]{s.}{pico (montanha); topo alto e pontudo da montanha}
\end{EntryWithPhonetic}

\begin{EntryWithPhonetic}{山冈}{shan1gang1}{3,4}{⼭,⼌}[HSK 7-9]
  \definition[座]{s.}{colina baixa; pequeno morro}
\end{EntryWithPhonetic}

\begin{EntryWithPhonetic}{山谷}{shan1 gu3}{3,7}{⼭,⾕}[HSK 6]
  \definition[条,个]{s.}{vale; desfiladeiro; ravina; a área baixa e estreita entre duas montanhas geralmente tem riachos no meio}
\end{EntryWithPhonetic}

\begin{EntryWithPhonetic}{山岭}{shan1ling3}{3,8}{⼭,⼭}[HSK 7-9]
  \definition[座,片,条]{s.}{cadeia de montanhas; cordilheira | crista; montanhas altas contínuas}
\end{EntryWithPhonetic}

\begin{EntryWithPhonetic}{山路}{shan1lu4}{3,13}{⼭,⾜}[HSK 7-9]
  \definition{s.}{trilha de montanha | estrada de montanha}
\end{EntryWithPhonetic}

\begin{EntryWithPhonetic}{山坡}{shan1 po1}{3,8}{⼭,⼟}[HSK 6]
  \definition[个,座,片]{s.}{encosta; encosta da montanha; a inclinação entre o topo da montanha e o terreno plano}
\end{EntryWithPhonetic}

\begin{EntryWithPhonetic}{山区}{shan1 qu1}{3,4}{⼭,⼖}[HSK 5]
  \definition[片]{s.}{área montanhosa; região montanhosa | colina; serra; montanha | distrito montanhoso}
\end{EntryWithPhonetic}

\begin{EntryWithPhonetic}{山体}{shan1ti3}{3,7}{⼭,⼈}
  \definition{s.}{forma de uma montanha}
\end{EntryWithPhonetic}

\begin{EntryWithPhonetic}{山西}{shan1xi1}{3,6}{⼭,⾑}
  \definition*{s.}{Província de Shanxi (Shansi) no norte da China entre Hebei e Shaanxi, abreviada como 晋 (Shansi)}
  \seealsoref{晋}{jin4}
\end{EntryWithPhonetic}

\begin{EntryWithPhonetic}{山羊}{shan1yang2}{3,6}{⼭,⽺}
  \definition{s.}{cabra | (ginástica) cavalo de salto de pequeno porte}
\end{EntryWithPhonetic}

\begin{EntryWithPhonetic}{山阴}{shan1yin1}{3,6}{⼭,⾩}
  \definition*{s.}{Condado de Shanyin em Shuozhou, Shanxi}
  \definition{s.}{lado norte (ou sombreado) de uma montanha}
\end{EntryWithPhonetic}

\begin{EntryWithPhonetic}{山寨}{shan1zhai4}{3,14}{⼭,⼧}[HSK 7-9]
  \definition{s.}{fortaleza na montanha (especialmente de bandidos); aldeia fortificada na montanha; vila fortificada na colina | Humorístico: “bandido”; imitador | Figurativo: imitação (de produtos); falsificação}
\end{EntryWithPhonetic}

%%%%%%%%%% 删 %%%%%%%%%%
\subsection*{删}\addcontentsline{loh}{figure}{删 \dpy{shan1}}

\begin{EntryWithPhonetic}{删}{shan1}{7}{⼑}[HSK 7-9]
  \definition{v.}{excluir; omitir; remover}
\end{EntryWithPhonetic}

\begin{EntryWithPhonetic}{删除}{shan1chu2}{7,9}{⼑,⾩}[HSK 7-9]
  \definition{v.}{apagar; cortar; eliminar; excluir; riscar; descartar}
\end{EntryWithPhonetic}

%%%%%%%%%% 扇 %%%%%%%%%%
\subsection*{扇}\addcontentsline{loh}{figure}{扇 \dpy{shan1}}

\begin{EntryWithPhonetic}{扇}{shan1}{10}{⼾}[HSK 5]
  \definition{s.}{ventilar; agitar um leque para fazer o ar circular | dar um tapa; bater com a palma da mão | bater asas; esvoaçar | incitar; instigar; estimular; agitar}
  \seeref{shan4}
\end{EntryWithPhonetic}

%%%%%%%%%% 煽 %%%%%%%%%%
\subsection*{煽}\addcontentsline{loh}{figure}{煽 \dpy{shan1}}

\begin{EntryWithPhonetic}{煽}{shan1}{14}{⽕}
  \definition{v.}{abanar (fogo); agitar um leque ou outra folha | incitar; instigar; agitar | vangloriar-se de; esbanjar prêmios em}
\end{EntryWithPhonetic}

\begin{EntryWithPhonetic}{煽动}{shan1dong4}{14,6}{⽕,⼒}[HSK 7-9]
  \definition{v.}{instigar; incitar (alguém a fazer coisas ruins); agitar; inflamar}
\end{EntryWithPhonetic}

%%%%%%%%%% 闪 %%%%%%%%%%
\subsection*{闪}\addcontentsline{loh}{figure}{闪 \dpy{shan3}}

\begin{EntryWithPhonetic}{闪}{shan3}{5}{⾨}[HSK 4]
  \definition*{s.}{Sobrenome: Shan}
  \definition{s.}{relâmpago}
  \definition{v.}{esquivar-se; desviar; sair do caminho | torcer; distender | surgir de repente | cintilar; brilhar | deixar para trás; abandonar | (corpo) oscilar dramaticamente}
\end{EntryWithPhonetic}

\begin{EntryWithPhonetic}{闪存盘}{shan3cun2pan2}{5,6,11}{⾨,⼦,⽫}
  \definition{s.}{unidade de memória \emph{USB} | cartão de memória}
  \seealsoref{优盘}{you1pan2}
\end{EntryWithPhonetic}

\begin{EntryWithPhonetic}{闪电}{shan3dian4}{5,5}{⾨,⽥}[HSK 4]
  \definition[道]{s.}{relâmpago; descargas elétricas entre nuvens ou entre nuvens e o solo}
  \seealsoref{雷电}{lei2dian4}
\end{EntryWithPhonetic}

\begin{EntryWithPhonetic}{闪烁}{shan3shuo4}{5,9}{⾨,⽕}[HSK 7-9]
  \definition{adj.}{vago; evasivo; não comprometido}
  \definition{v.}{cintilar; brilhar; reluzir; tremeluzir}
\end{EntryWithPhonetic}

\begin{EntryWithPhonetic}{闪耀}{shan3yao4}{5,20}{⾨,⽻}[HSK 7-9]
  \definition{v.}{brilhar; cintilar; resplandecer; irradiar; a luz oscila, às vezes brilhante, às vezes fraca | brilhar; irradiar luz deslumbrante}
\end{EntryWithPhonetic}

%%%%%%%%%% 掺 %%%%%%%%%%
\subsection*{掺}\addcontentsline{loh}{figure}{掺 \dpy{shan3}}

\begin{EntryWithPhonetic}{掺}{shan3}{11}{⼿}
  \definition{v.}{misturar; mesclar | conter; reter}
  \seeref{can4}
  \seeref{chan1}
\end{EntryWithPhonetic}

%%%%%%%%%% 单 %%%%%%%%%%
\subsection*{单}\addcontentsline{loh}{figure}{单 \dpy{shan4}}

\begin{EntryWithPhonetic}{单}{shan4}{8}{⼗}
  \definition*{s.}{Sobrenome: Shan}
  \definition{s.}{material de tecido de largura simples (dupla) | número singular (plural)}
  \seeref{chan2}
  \seeref{dan1}
\end{EntryWithPhonetic}

%%%%%%%%%% 扇 %%%%%%%%%%
\subsection*{扇}\addcontentsline{loh}{figure}{扇 \dpy{shan4}}

\begin{EntryWithPhonetic}{扇}{shan4}{10}{⼾}[HSK 5]
  \definition{clas.}{usado para portas, janelas, etc.}
  \definition[把]{s.}{leque | folha; algo em forma de placa ou folha}
  \seeref{shan1}
\end{EntryWithPhonetic}

\begin{EntryWithPhonetic}{扇子}{shan4zi5}{10,3}{⼾,⼦}[HSK 5]
  \definition[把,个]{s.}{leque; abano; abanador; utensílios que produzem vento ao serem agitados}
\end{EntryWithPhonetic}

%%%%%%%%%% 善 %%%%%%%%%%
\subsection*{善}\addcontentsline{loh}{figure}{善 \dpy{shan4}}

\begin{EntryWithPhonetic}{善}{shan4}{12}{⼝}[HSK 7-9]
  \definition*{s.}{Sobrenome: Shan}
  \definition{adj.}{bom; bem | bom; satisfatório | gentil; amigável | familiar}
  \definition{adv.}{bom; bem}
  \definition{s.}{boa ação; ato benevolente; coisas boas (em oposição a 恶)}
  \definition{v.}{fazer sucesso; fazer bem; fazer acontecer | ser bom em; ser especialista (versado) em | ser apto a}
  \seealsoref{恶}{e4}
\end{EntryWithPhonetic}

\begin{EntryWithPhonetic}{善良}{shan4liang2}{12,7}{⼝,⾉}[HSK 4]
  \definition{adj.}{de bom coração; bom e honesto; de bom coração e cheio de boa vontade}
\end{EntryWithPhonetic}

\begin{EntryWithPhonetic}{善意}{shan4yi4}{12,13}{⼝,⼼}[HSK 7-9]
  \definition[片]{s.}{boa vontade;  boa intenção;  benevolência; bondade}
\end{EntryWithPhonetic}

\begin{EntryWithPhonetic}{善于}{shan4yu2}{12,3}{⼝,⼆}[HSK 4]
  \definition{adv./v.}{ser bom em; ser hábil em}
\end{EntryWithPhonetic}

%%%%%%%%%% 禅 %%%%%%%%%%
\subsection*{禅}\addcontentsline{loh}{figure}{禅 \dpy{shan4}}

\begin{EntryWithPhonetic}{禅}{shan4}{12}{⽰}
  \definition{v.}{abdicar e entregar a coroa a outra pessoa}
  \seeref{chan2}
\end{EntryWithPhonetic}

%%%%%%%%%% 擅 %%%%%%%%%%
\subsection*{擅}\addcontentsline{loh}{figure}{擅 \dpy{shan4}}

\begin{EntryWithPhonetic}{擅}{shan4}{16}{⼿}
  \definition{adv.}{sem autorização; arbitrariamente | fazer algo por conta própria}
  \definition{v.}{ser bom em; ser especialista em | arrogar-se a si mesmo; fazer algo por conta própria | reivindicar arbitrariamente; ir além do escopo e ajir arbitrariamente}
\end{EntryWithPhonetic}

\begin{EntryWithPhonetic}{擅长}{shan4chang2}{16,4}{⼿,⾧}[HSK 7-9]
  \definition{v.}{ser bom em; ser especialista em; ser habilidoso em; ter um talento especial em determinada área}
\end{EntryWithPhonetic}

\begin{EntryWithPhonetic}{擅自}{shan4zi4}{16,6}{⼿,⾃}[HSK 7-9]
  \definition{adv.}{arbitrariamente; sem permissão ou autorização; agir por iniciativa própria em assuntos que estão fora da sua alçada}
\end{EntryWithPhonetic}

%%%%%%%%%% 膳 %%%%%%%%%%
\subsection*{膳}\addcontentsline{loh}{figure}{膳 \dpy{shan4}}

\begin{EntryWithPhonetic}{膳}{shan4}{16}{⾁}
  \definition{s.}{refeições; comida; alimentação}
\end{EntryWithPhonetic}

\begin{EntryWithPhonetic}{膳食}{shan4shi2}{16,9}{⾁,⾷}[HSK 7-9]
  \definition{s.}{comida; refeições; refeições consumidas todos os dias}
\end{EntryWithPhonetic}

%%%%%%%%%% 伤 %%%%%%%%%%
\subsection*{伤}\addcontentsline{loh}{figure}{伤 \dpy{shang1}}

\begin{EntryWithPhonetic}{伤}{shang1}{6}{⼈}[HSK 3]
  \definition*{s.}{Sobrenome: Shang}
  \definition[处]{s.}{ferida; ferimento}
  \definition{v.}{ferir; machucar | ter os sentimentos feridos | estar angustiado | enjoar de algo; desenvolver aversão a algo | ser prejudicial a; entravar}
\end{EntryWithPhonetic}

\begin{EntryWithPhonetic}{伤残}{shang1can2}{6,9}{⼈,⽍}[HSK 7-9]
  \definition{adj.}{ferido e incapacitado; deficiente; mutilado; desfigurado; aleijado | (um produto, etc.) defeituoso; falho; danificado}
\end{EntryWithPhonetic}

\begin{EntryWithPhonetic}{伤感}{shang1gan3}{6,13}{⼈,⼼}[HSK 7-9]
  \definition{adj.}{doente de coração; sentimental; piegas; triste}
\end{EntryWithPhonetic}

\begin{EntryWithPhonetic}{伤害}{shang1hai4}{6,10}{⼈,⼧}[HSK 4]
  \definition[种]{v.}{ferir; prejudicar; machucar; magoar; causar danos físicos ou mentais}
\end{EntryWithPhonetic}

\begin{EntryWithPhonetic}{伤痕}{shang1hen2}{6,11}{⼈,⽧}[HSK 7-9]
  \definition[道,处]{s.}{cicatriz; ferida; também se refere a uma marca deixada após um objeto ter sido danificado | cicatriz; ferida; metáfora para trauma psicológico}
\end{EntryWithPhonetic}

\begin{EntryWithPhonetic}{伤口}{shang1 kou3}{6,3}{⼈,⼝}[HSK 6]
  \definition[处]{s.}{corte; ferida; onde a pele, os músculos, etc. são feridos, rompidos ou onde são realizadas aberturas cirúrgicas}
\end{EntryWithPhonetic}

\begin{EntryWithPhonetic}{伤脑筋}{shang1 nao3jin1}{6,10,12}{⼈,⾁,⽵}[HSK 7-9]
  \definition{adj.}{complicado; problemático; incômodo; que causa dor de cabeça em alguém; descreve uma situação como difícil e que exige muita reflexão}
\end{EntryWithPhonetic}

\begin{EntryWithPhonetic}{伤势}{shang1shi4}{6,8}{⼈,⼒}[HSK 7-9]
  \definition{s.}{condição de uma lesão (ou ferida) | o estado de uma lesão (ou ferida)}
\end{EntryWithPhonetic}

\begin{EntryWithPhonetic}{伤亡}{shang1 wang2}{6,3}{⼈,⼇}[HSK 6]
  \definition{s.}{ferimentos e mortes; feridos e mortos; pessoas feridas e mortas; baixas}
  \definition{v.}{ser ferido e morto}
\end{EntryWithPhonetic}

\begin{EntryWithPhonetic}{伤心}{shang1/xin1}{6,4}{⼈,⼼}[HSK 3]
  \definition{v.+compl.}{estar triste; lamentar; estar com o coração partido; sentir-se triste por causa de infortúnio ou decepção}
\end{EntryWithPhonetic}

\begin{EntryWithPhonetic}{伤员}{shang1 yuan2}{6,7}{⼈,⼝}[HSK 6]
  \definition[名,位,个]{s.}{Exército: pessoal ferido; os feridos}
\end{EntryWithPhonetic}

%%%%%%%%%% 汤 %%%%%%%%%%
\subsection*{汤}\addcontentsline{loh}{figure}{汤 \dpy{shang1}}

\begin{EntryWithPhonetic}{汤}{shang1}{6}{⽔}
  \definition{s.}{correnteza forte}
  \seeref{tang1}
\end{EntryWithPhonetic}

%%%%%%%%%% 商 %%%%%%%%%%
\subsection*{商}\addcontentsline{loh}{figure}{商 \dpy{shang1}}

\begin{EntryWithPhonetic}{商}{shang1}{11}{⼝}
  \definition*{s.}{Dinastia Shang (1600-1046 a.C.) | Shang, nome da estrela da constelação do coração entre as vinte e oito constelações | Sobrenome: Shang}
  \definition{s.}{comércio; negócio; a atividade econômica de compra e venda de mercadorias | comerciante; negociante; comerciante; empresário; pessoas que compram e vendem mercadorias | (matemática) quociente;  o resultado de uma operação de divisão em aritmética | uma nota da antiga escala chinesa de cinco tons, correspondente a 2 na notação musical numerada}
  \definition{v.}{discutir; consultar; trocar ideias}
\end{EntryWithPhonetic}

\begin{EntryWithPhonetic}{商标}{shang1biao1}{11,9}{⼝,⽊}[HSK 5]
  \definition[个]{s.}{marca; marca registrada; \emph{trademark}; marca ou símbolo (desenho, padrão, texto, etc.) gravado ou impresso na superfície ou embalagem de um produto, para diferenciá-lo de outros produtos semelhantes}
\end{EntryWithPhonetic}

\begin{EntryWithPhonetic}{商场}{shang1 chang3}{11,6}{⼝,⼟}[HSK 1]
  \definition[家]{s.}{mercado; shopping center; loja de departamentos; loja de grande área com uma variedade completa de produtos | o mundo dos negócios; referindo-se ao mundo dos negócios | mercado; mercado composto por várias lojas reunidas em um ou vários edifícios interligados}
\end{EntryWithPhonetic}

\begin{EntryWithPhonetic}{商城}{shang1 cheng2}{11,9}{⼝,⼟}[HSK 6]
  \definition{s.}{um mercado; um centro comercial; um \emph{shopping center}; refere-se a um complexo comercial contíguo com um grande espaço de construção}
\end{EntryWithPhonetic}

\begin{EntryWithPhonetic}{商店}{shang1dian4}{11,8}{⼝,⼴}[HSK 1]
  \definition[间,家,个]{s.}{loja; armazém; local de venda de mercadorias em recinto fechado}
\end{EntryWithPhonetic}

\begin{EntryWithPhonetic}{商贩}{shang1fan4}{11,8}{⼝,⾙}[HSK 7-9]
  \definition{s.}{varejista; vendedor ambulante; comerciante; pequenos comerciantes que vendem ou comercializam produtos}
\end{EntryWithPhonetic}

\begin{EntryWithPhonetic}{商贾}{shang1gu3}{11,10}{⼝,⾙}[HSK 7-9]
  \definition{s.}{Literário: comerciante}
\end{EntryWithPhonetic}

\begin{EntryWithPhonetic}{商量}{shang1liang5}{11,12}{⼝,⾥}[HSK 2]
  \definition{v.}{consultar; discutir; conversar sobre; discutir e trocar opiniões}
\end{EntryWithPhonetic}

\begin{EntryWithPhonetic}{商贸}{shang1mao4}{11,9}{⼝,⾙}
  \definition{s.}{comércio}
\end{EntryWithPhonetic}

\begin{EntryWithPhonetic}{商品}{shang1pin3}{11,9}{⼝,⼝}[HSK 3]
  \definition[种,个,件,批]{s.}{bens; mercadoria; \emph{merchande}; os produtos do trabalho produzidos para troca têm a dupla natureza de valor de uso e valor; as mercadorias incorporam diferentes relações de produção em diferentes sistemas sociais}
\end{EntryWithPhonetic}

\begin{EntryWithPhonetic}{商人}{shang1 ren2}{11,2}{⼝,⼈}[HSK 2]
  \definition[位,名]{s.}{comerciante; mercador; empresário; homem de negócios; pessoas que trabalham com a distribuição de mercadorias}
\end{EntryWithPhonetic}

\begin{EntryWithPhonetic}{商讨}{shang1tao3}{11,5}{⼝,⾔}[HSK 7-9]
  \definition{v.}{discutir; deliberar sobre; trocar ideias e discutir para resolver problemas maiores e mais complexos}
\end{EntryWithPhonetic}

\begin{EntryWithPhonetic}{商务}{shang1wu4}{11,5}{⼝,⼒}[HSK 4]
  \definition[种,类,项]{s.}{negócios; assuntos de negócios; assuntos comerciais}
\end{EntryWithPhonetic}

\begin{EntryWithPhonetic}{商业}{shang1ye4}{11,5}{⼝,⼀}[HSK 3]
  \definition[个,种]{s.}{barganha; negócio; comércio; atividade econômica que circula mercadorias por meio de compra e venda}
\end{EntryWithPhonetic}

%%%%%%%%%% 上 %%%%%%%%%%
\subsection*{上}\addcontentsline{loh}{figure}{上 \dpy{shang3}}

\begin{EntryWithPhonetic}{上}{shang3}{3}{⼀}
  \definition{s.}{tom descendente-ascendente; significa o segundo tom dos quatro tons do mandarim, e também se refere ao terceiro tom do mandarim padrão}
  \seeref{shang4}
\end{EntryWithPhonetic}

\begin{EntryWithPhonetic}{上声}{shang3sheng1}{3,7}{⼀,⼠}
  \definition{s.}{tom descendente e ascendente | terceiro tom no mandarim moderno}
\end{EntryWithPhonetic}

%%%%%%%%%% 赏 %%%%%%%%%%
\subsection*{赏}\addcontentsline{loh}{figure}{赏 \dpy{shang3}}

\begin{EntryWithPhonetic}{赏}{shang3}{12}{⾙}[HSK 4]
  \definition*{s.}{Sobrenome: Shang}
  \definition{s.}{recompensa; prêmio}
  \definition{v.}{conceder (outorgar) uma recompensa; recompensar; premiar | admirar; desfrutar; apreciar; valorizar}
\end{EntryWithPhonetic}

\begin{EntryWithPhonetic}{赏赐}{shang3ci4}{12,12}{⾙,⾙}
  \definition{s.}{recompensa | prêmio}
  \definition{v.}{recompensar | premiar}
\end{EntryWithPhonetic}

\begin{EntryWithPhonetic}{赏心悦目}{shang3xin1yue4mu4}{12,4,10,5}{⾙,⼼,⼼,⽬}
  \definition{expr.}{``Aquece o coração e encanta os olhos.''; achar a paisagem agradável tanto aos olhos quanto à mente}
\end{EntryWithPhonetic}

%%%%%%%%%% 上 %%%%%%%%%%
\subsection*{上}\addcontentsline{loh}{figure}{上 \dpy{shang4}}

\begin{EntryWithPhonetic}{上}{shang4}{3}{⼀}[HSK 1]
  \definition{adj.}{mais recente; último; anterior; tempo ou a sequência anterior | superior; mais alto; melhor; indica uma posição elevada em termos de qualidade, nível, etc. | lugar elevado; posição superior; em oposição a 下}
  \definition{s.}{superior; acima; para cima; um lugar alto ou mais alto do que um determinado local | na superfície de um objeto; usado após um substantivo, indica a superfície de um objeto | indica estar dentro do escopo de algo; usado após um substantivo, indica que algo está dentro do âmbito de determinada coisa | indica um aspecto específico | antigamente, referia-se ao imperador | usado após palavras que indicam idade, equivale a ``\dots 的时候'' | o primeiro nível da escala da música folclórica chinesa, usado como um símbolo de nota na notação musical, equivalente ao ``1'' na notação simplificada.}
  \definition{v.}{subir; montar; embarcar; entrar | ir para; partir para | estar ocupado (com trabalho, estudos, etc.) em um horário fixo; começar a trabalhar ou estudar na hora marcada, etc. | seguir em frente; prosseguir | encher; abastecer; servir; melhorar; aumentar | aparecer no palco; entrar | colocar algo em posição; ajustar; fixar; montar as duas partes de algo | aplicar; pintar; espalhar | ser registrado; ser publicado (em uma publicação) | atingir; ser suficiente (uma determinada quantidade ou grau) | submeter; enviar; apresentar; submeter à aprovação superior | ventilar; apertar; torcer | trazer; servir; colocar comida, pratos, chá e outras coisas na mesa para os convidados | indicar que uma ação tem um resultado | pesquisar na \emph{Internet} | emaranhar-se; ficar emaranhado; enredar-se}
  \definition{v.aux.}{usado após um verbo para indicar início e continuidade}
  \seeref{shang3}
  \seealsoref{的时候}{de5 shi2hou4}
  \seealsoref{下}{xia4}
\end{EntryWithPhonetic}

\begin{EntryWithPhonetic}{上班}{shang4/ban1}{3,10}{⼀,⽟}[HSK 1]
  \definition{v.+compl.}{ir trabalhar; começar a trabalhar; estar de plantão; ir trabalhar no local de trabalho regular no horário especificado}
\end{EntryWithPhonetic}

\begin{EntryWithPhonetic}{上班族}{shang4 ban1 zu2}{3,10,11}{⼀,⽟,⽅}
  \definition[本]{s.}{trabalhadores de escritório (como grupo social)}
\end{EntryWithPhonetic}

\begin{EntryWithPhonetic}{上报}{shang4bao4}{3,7}{⼀,⼿}[HSK 7-9]
  \definition{v.}{aparecer nos jornais; ser publicado | reportar a um órgão superior; reportar à liderança; reportar-se aos superiores}
\end{EntryWithPhonetic}

\begin{EntryWithPhonetic}{上边}{shang4 bian5}{3,5}{⼀,⾡}[HSK 1]
  \definition{s.}{topo; acima; sobre; superior}
\end{EntryWithPhonetic}

\begin{EntryWithPhonetic}{上场}{shang4/chang3}{3,6}{⼀,⼟}[HSK 7-9]
  \definition{v.+compl.}{Esporte: entrar na quadra (ou campo); participar de uma competição | aparecer no palco; subir no palco; entrar em cena}
\end{EntryWithPhonetic}

\begin{EntryWithPhonetic}{上车}{shang4 che1}{3,4}{⼀,⾞}[HSK 1]
  \definition{v.}{entrar; subir (em um ônibus, trem, carro etc.)}
\end{EntryWithPhonetic}

\begin{EntryWithPhonetic}{上次}{shang4 ci4}{3,6}{⼀,⽋}[HSK 1]
  \definition{adv.}{última vez}
\end{EntryWithPhonetic}

\begin{EntryWithPhonetic}{上当}{shang4/dang4}{3,6}{⼀,⼹}[HSK 6]
  \definition{v.+compl.}{ser enganado; ser ludibriado; morder a isca; cair nas mãos de alguém}
\end{EntryWithPhonetic}

\begin{EntryWithPhonetic}{上帝}{shang4 di4}{3,9}{⼀,⼱}[HSK 6]
  \definition*{s.}{Deus; O Deus Supremo no Cristianismo | O Imperador do Céu; um deus na antiga crença chinesa que pode controlar tudo no mundo}
  \definition[个]{s.}{(figurado) cliente; metáfora para consumidores}
\end{EntryWithPhonetic}

\begin{EntryWithPhonetic}{上方}{shang4fang1}{3,4}{⼀,⽅}[HSK 7-9]
  \definition{s.}{acima; sobre; em cima de (oposto de 下方) | superjacente}
  \seealsoref{下方}{xia4fang1}
\end{EntryWithPhonetic}

\begin{EntryWithPhonetic}{上访}{shang4fang3}{3,6}{⼀,⾔}
  \definition{v.}{buscar uma audiência com superiores (especialmente funcionários do governo) para fazer uma petição por algo}
\end{EntryWithPhonetic}

\begin{EntryWithPhonetic}{上岗}{shang4/gang3}{3,7}{⼀,⼭}[HSK 7-9]
  \definition{v.+compl.}{estar em período probatório | começar a trabalhar; assumir um cargo}
\end{EntryWithPhonetic}

\begin{EntryWithPhonetic}{上个月}{shang4 ge4 yue4}{3,3,4}{⼀,⼈,⽉}[HSK 4]
  \definition{s.}{mês passado; refere-se à hora de um mês atrás, ou seja, o mês passado mais próximo da hora atual}
\end{EntryWithPhonetic}

\begin{EntryWithPhonetic}{上古}{shang4gu3}{3,5}{⼀,⼝}
  \definition{s.}{o passado distante | tempos antigos | antiguidade}
\end{EntryWithPhonetic}

\begin{EntryWithPhonetic}{上海}{shang4hai3}{3,10}{⼀,⽔}
  \definition*{s.}{Município de Xangai (Shanghai), centro-leste da China}
\end{EntryWithPhonetic}

\begin{EntryWithPhonetic}{上火}{shang4/huo3}{3,4}{⼀,⽕}[HSK 7-9]
  \definition{v.+compl.}{ter dor de garganta | ter excesso de calor interno; na medicina tradicional chinesa, sintomas como prisão de ventre ou inflamação da mucosa nasal, da mucosa oral ou da conjuntiva são classificados como ``calor interno'' | ficar com raiva}
  \seealsoref{上火儿}{shang4huo3r5}
\end{EntryWithPhonetic}

\begin{EntryWithPhonetic}{上火儿}{shang4huo3r5}{3,4,2}{⼀,⽕,⼉}
  \definition{v.}{Dialeto: ficar com raiva; explodir}
\end{EntryWithPhonetic}

\begin{EntryWithPhonetic}{上级}{shang4ji2}{3,6}{⼀,⽷}[HSK 5]
  \definition[个,位]{s.}{nível superior; organização ou pessoa em nível superior; organizações ou pessoas de nível superior dentro do mesmo sistema organizacional}
\end{EntryWithPhonetic}

\begin{EntryWithPhonetic}{上课}{shang4/ke4}{3,10}{⼀,⾔}[HSK 1]
  \definition{v.+compl.}{frequentar aulas; ir às aulas; dar uma aula}
\end{EntryWithPhonetic}

\begin{EntryWithPhonetic}{上空}{shang4kong1}{3,8}{⼀,⽳}[HSK 7-9]
  \definition{s.}{no céu; acima da cabeça; no alto, no ar}
\end{EntryWithPhonetic}

\begin{EntryWithPhonetic}{上来}{shang4 lai2}{3,7}{⼀,⽊}[HSK 3]
  \definition{v.}{subir (para a minha localização) | estar no começo; começar; iniciar | surgir; de um lugar baixo para um lugar alto (o interlocutor está em um lugar alto) | usado após o verbo, indica que algo foi concluído com sucesso}
\end{EntryWithPhonetic}

\begin{EntryWithPhonetic}{上流}{shang4liu2}{3,10}{⼀,⽔}[HSK 7-9]
  \definition{adj.}{da classe alta; refinado | pertencente aos círculos superiores; anteriormente, referia-se a pessoas de alto status social}
  \definition{s.}{trecho superior (de um rio); a montante}
\end{EntryWithPhonetic}

\begin{EntryWithPhonetic}{上楼}{shang4 lou2}{3,13}{⼀,⽊}[HSK 4]
  \definition{v.}{subir as escadas; ir para o andar de cima}
\end{EntryWithPhonetic}

\begin{EntryWithPhonetic}{上门}{shang4 men2}{3,3}{⼀,⾨}[HSK 4]
  \definition{v.}{chamar; visitar; aparecer; ir ou vir para ver alguém; ir até a porta; ir até a casa de alguém | trancar a porta; fechar a porta durante a noite | casar-se e morar com a família da noiva}
\end{EntryWithPhonetic}

\begin{EntryWithPhonetic}{上面}{shang4 mian4}{3,9}{⼀,⾯}[HSK 3]
  \definition{s.}{uma posição mais alta que algo; uma posição acima/acima de algo | superfície do objeto | aspecto | a parte acima mencionada; a parte que vem primeiro na ordem; a parte de um artigo ou discurso que vem antes da presente | autoridades superiores | os mais velhos; a geração mais velha da família}
\end{EntryWithPhonetic}

\begin{EntryWithPhonetic}{上坡路}{shang4po1lu4}{3,8,13}{⼀,⼟,⾜}
  \definition{s.}{aclive | progresso | (fig.) tendência ascendente}
\end{EntryWithPhonetic}

\begin{EntryWithPhonetic}{上期}{shang4 qi1}{3,12}{⼀,⽉}[HSK 7-9]
  \definition{s.}{período anterior}
\end{EntryWithPhonetic}

\begin{EntryWithPhonetic}{上去}{shang4 qu4}{3,5}{⼀,⼛}[HSK 3]
  \definition{v.}{subir (a partir da minha localização) | ascender a um lugar (ou estado) considerado mais elevado (ou acima); usado depois de um verbo para indicar movimento, de baixo para cima ou de perto para longe}
\end{EntryWithPhonetic}

\begin{EntryWithPhonetic}{上任}{shang4/ren4}{3,6}{⼀,⼈}[HSK 7-9]
  \definition[班]{s.}{predecessor; ex-funcionário}
  \definition{v.+compl.}{assumir o cargo; ocupar um cargo oficial; refere-se à posse de autoridades}
\end{EntryWithPhonetic}

\begin{EntryWithPhonetic}{上升}{shang4 sheng1}{3,4}{⼀,⼗}[HSK 3]
  \definition{v.}{elevar; subir; mover-se para cima; mover de baixo para cima; aumentar em nível, grau, quantidade, etc.}
\end{EntryWithPhonetic}

\begin{EntryWithPhonetic}{上市}{shang4 shi4}{3,5}{⼀,⼱}[HSK 6]
  \definition{v.}{listar; abrir o capital; ser listado (na bolsa de valores) | estar na estação; estar (aparecer) no mercado | ir ao mercado (para fazer compras)}
\end{EntryWithPhonetic}

\begin{EntryWithPhonetic}{上述}{shang4shu4}{3,8}{⼀,⾡}[HSK 7-9]
  \definition{adj.}{mencionado anteriormente; supracitado; acima citado; conforme dito ou narrado acima}
\end{EntryWithPhonetic}

\begin{EntryWithPhonetic}{上司}{shang4si5}{3,5}{⼀,⼝}[HSK 7-9]
  \definition[位,名,个]{s.}{chefe; superior}
\end{EntryWithPhonetic}

\begin{EntryWithPhonetic}{上诉}{shang4su4}{3,7}{⼀,⾔}[HSK 7-9]
  \definition{s.}{apelação (para um tribunal superior)}
  \definition{v.}{apresentar um recurso; instaurar um recurso}
\end{EntryWithPhonetic}

\begin{EntryWithPhonetic}{上台}{shang4 tai2}{3,5}{⼀,⼝}[HSK 6]
  \definition{v.}{aparecer no palco; subir na plataforma; ir para o palco ou pódio | assumir o poder; chegar (subir) ao poder; começar a assumir papéis de liderança ou a ganhar algum tipo de poder}
\end{EntryWithPhonetic}

\begin{EntryWithPhonetic}{上调}{shang4tiao2}{3,10}{⼀,⾔}[HSK 7-9]
  \definition{v.}{transferir (alguém) para um cargo de nível superior (oposto de 下调) | transferir bens, fundos, etc. para uma unidade de nível superior | ajustar para cima | aumentar (os preços)}
  \seealsoref{下调}{xia4tiao2}
\end{EntryWithPhonetic}

\begin{EntryWithPhonetic}{上头}{shang4tou2}{3,5}{⼀,⼤}
  \definition{v.}{(álcool, amor etc.) subir à cabeça; (uma ideia, uma música etc.) entrar na cabeça de alguém; capturar a atenção de alguém | Obsoleto: (uma garota no dia do seu casamento) começar a usar o cabelo preso em um coque (em vez de uma trança)}
  \seeref{shang4tou5}
\end{EntryWithPhonetic}

\begin{EntryWithPhonetic}{上头}{shang4tou5}{3,5}{⼀,⼤}[HSK 7-9]
  \definition{s.}{acima; em cima de; na superfície de; superior}
  \seeref{shang4tou2}
\end{EntryWithPhonetic}

\begin{EntryWithPhonetic}{上网}{shang4wang3}{3,6}{⼀,⽹}[HSK 1]
  \definition{v.}{conectar-se à \emph{Internet}; acessar a \emph{Internet}; entrar na \emph{Internet}; acessar a rede; refere-se especificamente ao computador do usuário conectado à Internet para pesquisar e consultar informações, etc.}
\end{EntryWithPhonetic}

\begin{EntryWithPhonetic}{上午}{shang4wu3}{3,4}{⼀,⼗}[HSK 1]
  \definition[个]{s.}{manhã; \emph{ante meridiem} (a.m.); geralmente refere-se ao período entre a manhã e o meio-dia}
\end{EntryWithPhonetic}

\begin{EntryWithPhonetic}{上下}{shang4 xia4}{3,3}{⼀,⼀}[HSK 5]
  \definition{adv.}{para cima e para baixo}
  \definition[顶]{s.}{alto e baixo | de cima para baixo; para cima e para baixo | superioridade ou inferioridade relativa | (após números redondos) aproximadamente; mais ou menos; por aí | velhos e jovens; hierarquia em termos de cargo e posição social}
  \definition{v.}{subir ou descer | subir e descer; da alta para a baixa ou da baixa para a alta}
\end{EntryWithPhonetic}

\begin{EntryWithPhonetic}{上限}{shang4xian4}{3,8}{⼀,⾩}[HSK 7-9]
  \definition[个]{s.}{teto; limite superior; refere-se ao primeiro ou mais alto limite dentro de um determinado conjunto de limites (em oposição a 下限)}
  \seealsoref{下限}{xia4xian4}
\end{EntryWithPhonetic}

\begin{EntryWithPhonetic}{上学}{shang4 xue2}{3,8}{⼀,⼦}[HSK 1]
  \definition{v.}{ir à escola; frequentar a escola; estar na escola; ir à escola para estudar | começar a escola; começar a estudar no ensino fundamental}
\end{EntryWithPhonetic}

\begin{EntryWithPhonetic}{上旬}{shang4xun2}{3,6}{⼀,⽇}[HSK 7-9]
  \definition{s.}{primeiro terço do mês; os dez dias do dia 1 ao dia 10 de cada mês}
\end{EntryWithPhonetic}

\begin{EntryWithPhonetic}{上询}{shang4 xun2}{3,8}{⼀,⾔}
  \definition{adv.}{primeira dezena do mês}
\end{EntryWithPhonetic}

\begin{EntryWithPhonetic}{上演}{shang4 yan3}{3,14}{⼀,⽔}[HSK 6]
  \definition{s.}{exibição | encenação}
  \definition{v.}{exibir (um filme); encenar (uma peça); atuar; colocar no palco}
\end{EntryWithPhonetic}

\begin{EntryWithPhonetic}{上衣}{shang4 yi1}{3,6}{⼀,⾐}[HSK 3]
  \definition[件]{s.}{jaqueta; roupas para a parte superior do corpo}
\end{EntryWithPhonetic}

\begin{EntryWithPhonetic}{上瘾}{shang4/yin3}{3,16}{⼀,⽧}[HSK 7-9]
  \definition{v.+compl.}{ser viciado (em algo); adquirir o hábito (de fazer algo); gostar muito de algo, a ponto de não conseguir viver sem}
\end{EntryWithPhonetic}

\begin{EntryWithPhonetic}{上映}{shang4ying4}{3,9}{⼀,⽇}[HSK 7-9]
  \definition{v.}{executar; exibir; mostrar (um filme); (novo filme) lançar para exibição}
\end{EntryWithPhonetic}

\begin{EntryWithPhonetic}{上游}{shang4you2}{3,12}{⼀,⽔}[HSK 7-9]
  \definition{s.}{a montante; trecho superior de um rio; o trecho de um rio próximo à sua nascente; também se refere à área por onde esse trecho flui (em oposição a 中游 ou 下游) | posição avançada; metaforicamente, refere-se a um status ou nível avançado}
  \seealsoref{下游}{xia4you2}
  \seealsoref{中游}{zhong1you2}
\end{EntryWithPhonetic}

\begin{EntryWithPhonetic}{上涨}{shang4 zhang3}{3,10}{⼀,⽔}[HSK 5]
  \definition{v.}{subir; ir para cima; ascender}
\end{EntryWithPhonetic}

\begin{EntryWithPhonetic}{上周}{shang4 zhou1}{3,8}{⼀,⼝}[HSK 2]
  \definition{s.}{semana passada}
\end{EntryWithPhonetic}

%%%%%%%%%% 尚 %%%%%%%%%%
\subsection*{尚}\addcontentsline{loh}{figure}{尚 \dpy{shang4}}

\begin{EntryWithPhonetic}{尚}{shang4}{8}{⼩}[HSK 7-9]
  \definition*{s.}{Sobrenome: Shang}
  \definition{adv.}{ainda}
  \definition{s.}{costume predominante; refere-se à tendência predominante na sociedade; coisas que geralmente são admiradas pelas pessoas}
  \definition{v.}{valorizar; estimar; dar grande importância a; respeitar}
\end{EntryWithPhonetic}

\begin{EntryWithPhonetic}{尚且}{shang4 qie3}{8,5}{⼩,⼀}
  \definition{conj.}{nem\dots; muito menos\dots; é usado antes do verbo da primeira oração de uma frase complexa para apresentar alguns exemplos óbvios para comparação, a segunda oração frequentemente usa 何况 ou 更 para ecoar e tirar conclusões inevitáveis ​​sobre exemplos semelhantes com diferentes graus de gravidade}
  \seealsoref{更}{geng4}
  \seealsoref{何况}{he2kuang4}
\end{EntryWithPhonetic}

\begin{EntryWithPhonetic}{尚且…何况…}{shang4qie3 he2kuang4}{8,5,7,7}{⼩,⼀,⼈,⼎}
  \definition{conj.}{ainda que\dots, \dots; além do mais\dots e muito menos\dots}
\end{EntryWithPhonetic}

\begin{EntryWithPhonetic}{尚未}{shang4wei4}{8,5}{⼩,⽊}[HSK 7-9]
  \definition{adv.}{ainda não}[问题尚未解决。===O problema ainda não foi resolvido.]
\end{EntryWithPhonetic}

%%%%%%%%%% 捎 %%%%%%%%%%
\subsection*{捎}\addcontentsline{loh}{figure}{捎 \dpy{shao1}}

\begin{EntryWithPhonetic}{捎}{shao1}{10}{⼿}[HSK 7-9]
  \definition{v.}{trazer para alguém; levar algo para alguém ou em nome de alguém; levar consigo}
  \seeref{shao4}
\end{EntryWithPhonetic}

%%%%%%%%%% 烧 %%%%%%%%%%
\subsection*{烧}\addcontentsline{loh}{figure}{烧 \dpy{shao1}}

\begin{EntryWithPhonetic}{烧}{shao1}{10}{⽕}[HSK 4]
  \definition[次]{s.}{febre; temperatura corporal mais alta do que o normal}
  \definition{v.}{queimar; pegar fogo | cozinhar; aquecer; assar | guisar depois de fritar ou fritar depois de guisar | assar; grelhar os ingredientes dos alimentos diretamente sobre o fogo | ter febre; estar com febre | danificar (matar ou murchar) as plantas pelo uso excessivo (ou inadequado) de fertilizantes | tornar-se arrogante ou presunçoso; metáfora de estar em uma boa posição e se deixar levar}
\end{EntryWithPhonetic}

\begin{EntryWithPhonetic}{烧毁}{shao1hui3}{10,13}{⽕,⽎}[HSK 7-9]
  \definition{v.}{queimar; destruir pelo fogo}
\end{EntryWithPhonetic}

\begin{EntryWithPhonetic}{烧烤}{shao1kao3}{10,10}{⽕,⽕}[HSK 7-9]
  \definition[顿,次,场]{s.}{churrasco; carne assada ou grelhada}
  \definition{v.}{assar; grelhar; fazer churrasco; grelhar ou assar (carnes) sobre carvão}
\end{EntryWithPhonetic}

%%%%%%%%%% 稍 %%%%%%%%%%
\subsection*{稍}\addcontentsline{loh}{figure}{稍 \dpy{shao1}}

\begin{EntryWithPhonetic}{稍}{shao1}{12}{⽲}[HSK 5]
  \definition{adv.}{ligeiramente; um pouco; um pouquinho}
  \seeref{shao4}
  \seealsoref{稍稍}{shao1shao1}
\end{EntryWithPhonetic}

\begin{EntryWithPhonetic}{稍后}{shao1hou4}{12,6}{⽲,⼝}[HSK 7-9]
  \definition{s.}{um pouco mais tarde (no tempo ou no espaço)}
\end{EntryWithPhonetic}

\begin{EntryWithPhonetic}{稍候}{shao1hou4}{12,10}{⽲,⼈}[HSK 7-9]
  \definition{v.}{aguardar um momento}
\end{EntryWithPhonetic}

\begin{EntryWithPhonetic}{稍稍}{shao1shao1}{12,12}{⽲,⽲}[HSK 7-9]
  \definition{adv.}{um pouco; ligeiramente}
\end{EntryWithPhonetic}

\begin{EntryWithPhonetic}{稍微}{shao1wei1}{12,13}{⽲,⼻}[HSK 5]
  \definition{adv.}{um pouco; um pouquinho; uma ninharia; indica que a quantidade é pequena ou o grau é superficial}
\end{EntryWithPhonetic}

%%%%%%%%%% 勺 %%%%%%%%%%
\subsection*{勺}\addcontentsline{loh}{figure}{勺 \dpy{shao2}}

\begin{EntryWithPhonetic}{勺}{shao2}{3}{⼓}[HSK 6]
  \definition{clas.}{shao; uma unidade tradicional de volume, igual a 0,01 市升, e equivalente a 1 centilitro ou 0,018 \emph{pint}}
  \definition{s.}{colher; concha}
  \seealsoref{市升}{shi4sheng1}
\end{EntryWithPhonetic}

%%%%%%%%%% 少 %%%%%%%%%%
\subsection*{少}\addcontentsline{loh}{figure}{少 \dpy{shao3}}

\begin{EntryWithPhonetic}{少}{shao3}{4}{⼩}[HSK 1]
  \definition{adj.}{menos; pouco (oposto a 多); escasso; não atingir a quantidade original ou esperada}
  \definition{adv.}{um momento; um instante; provisoriamente; ligeiramente}
  \definition{v.}{faltar; ser insuficiente | dever | perder; desaparecer; extraviar | parar; desistir}
  \seeref{shao4}
  \seealsoref{多}{duo1}
\end{EntryWithPhonetic}

\begin{EntryWithPhonetic}{少不了}{shao3bu5liao3}{4,4,2}{⼩,⼀,⼅}[HSK 7-9]
  \definition{v.}{não poder prescindir de; não poder dispensar | estar obrigado a; ser inevitável | não poder ser apenas alguns (ou pouco); não poder ser menos}
\end{EntryWithPhonetic}

\begin{EntryWithPhonetic}{少见}{shao3jian4}{4,4}{⼩,⾒}[HSK 7-9]
  \definition{adj.}{raramente visto; infrequente; raro}
  \definition{expr.}{``Não te vejo há muito tempo.'', ``Tenho te visto muito pouco ultimamente.'' ou ``Estou muito feliz em te ver novamente.''}
  \definition{v.}{difícil de ver | não ser familiar (para o falante) | ser raro (algo)}
\end{EntryWithPhonetic}

\begin{EntryWithPhonetic}{少量}{shao3liang4}{4,12}{⼩,⾥}[HSK 7-9]
  \definition{s.}{uma pequena quantidade; um pouco; alguns; picada de pulga; toque; gosto; tapa; estalo; ninharia; gole}
\end{EntryWithPhonetic}

\begin{EntryWithPhonetic}{少数}{shao3 shu4}{4,13}{⼩,⽁}[HSK 2]
  \definition{s.}{número pequeno; poucos; minoria}
\end{EntryWithPhonetic}

\begin{EntryWithPhonetic}{少有}{shao3you3}{4,6}{⼩,⽉}[HSK 7-9]
  \definition{adj.}{raro; excepcional; escasso}
  \definition{adv.}{raramente}
\end{EntryWithPhonetic}

\begin{EntryWithPhonetic}{少}{shao4}{4}{⼩}
  \definition*{s.}{Sobrenome: Shao}
  \definition{s.}{jovem (em oposição a 老)}
  \definition{s.}{jovem mestre; filho de uma família rica}
  \seeref{shao3}
  \seealsoref{老}{lao3}
\end{EntryWithPhonetic}

\begin{EntryWithPhonetic}{少儿}{shao4 er2}{4,2}{⼩,⼉}[HSK 6]
  \definition{s.}{criança}
\end{EntryWithPhonetic}

\begin{EntryWithPhonetic}{少林寺}{shao4lin2 si4}{4,8,6}{⼩,⽊,⼨}[HSK 7-9]
  \definition*{s.}{Templo (ou Mosteiro) Shaolin, ao pé do Monte Song (嵩山) na província de Henan, onde o kung-fu Shaolin foi desenvolvido}
  \seealsoref{嵩山}{song1shan1}
\end{EntryWithPhonetic}

\begin{EntryWithPhonetic}{少年}{shao4 nian2}{4,6}{⼩,⼲}[HSK 2]
  \definition[个,名,位]{s.}{adolescente; juventude; atualmente, a faixa etária geralmente referida é de 10 anos ou mais a 18 anos ou mais | menor; jovem; juvenil; refere-se a menores na faixa etária anterior | jovem; adolescente; rapaz}
\end{EntryWithPhonetic}

\begin{EntryWithPhonetic}{少女}{shao4nv3}{4,3}{⼩,⼥}[HSK 7-9]
  \definition[位,名]{s.}{empregada doméstica; jovem; moça; mulheres jovens solteiras}
\end{EntryWithPhonetic}

%%%%%%%%%% 召 %%%%%%%%%%
\subsection*{召}\addcontentsline{loh}{figure}{召 \dpy{shao4}}

\begin{EntryWithPhonetic}{召}{shao4}{5}{⼝}
  \definition*{s.}{Sobrenome: Shao}
  \definition{s.}{(frequentemente em nomes de lugares mongóis) templo; mosteiro}
  \definition{v.}{convocar; intimar; invocar}
  \seeref{zhao4}
\end{EntryWithPhonetic}

%%%%%%%%%% 绍 %%%%%%%%%%
\subsection*{绍}\addcontentsline{loh}{figure}{绍 \dpy{shao4}}

\begin{EntryWithPhonetic}{绍}{shao4}{8}{⽷}
  \definition*{s.}{Shaoxing, abreviação de 绍兴 | Sobrenome: Shao}
  \definition{v.}{continuar; herdar}
  \seealsoref{绍兴}{shao4xing1}
\end{EntryWithPhonetic}

\begin{EntryWithPhonetic}{绍兴}{shao4xing1}{8,6}{⽷,⼋}
  \definition*{s.}{Shaoxing, anteriormente conhecida como Kuaiji, é uma cidade de nível de prefeitura na província de Zhejiang, na China; é uma grande cidade localizada na parte centro-norte da província de Zhejiang}
\end{EntryWithPhonetic}

%%%%%%%%%% 捎 %%%%%%%%%%
\subsection*{捎}\addcontentsline{loh}{figure}{捎 \dpy{shao4}}

\begin{EntryWithPhonetic}{捎}{shao4}{10}{⼿}
  \definition{adj.}{2. desbotado (cor)}
  \definition{v.}{recuar; puxar para trás (cavalo, burro); mover-se ligeiramente para trás (geralmente referindo-se a mulas, cavalos, etc.) | desbotar (cor)}
  \seeref{shao1}
\end{EntryWithPhonetic}

%%%%%%%%%% 稍 %%%%%%%%%%
\subsection*{稍}\addcontentsline{loh}{figure}{稍 \dpy{shao4}}

\begin{EntryWithPhonetic}{稍}{shao4}{12}{⽲}
  \definition{adv.}{à vontade}
\end{EntryWithPhonetic}

%%%%%%%%%% 奢 %%%%%%%%%%
\subsection*{奢}\addcontentsline{loh}{figure}{奢 \dpy{she1}}

\begin{EntryWithPhonetic}{奢}{she1}{11}{⼤}
  \definition{adj.}{luxuoso; extravagante | excessivo; desmedido; extravagante}
\end{EntryWithPhonetic}

\begin{EntryWithPhonetic}{奢侈}{she1chi3}{11,8}{⼤,⼈}[HSK 7-9]
  \definition{adj.}{luxuoso; de luxo}
\end{EntryWithPhonetic}

\begin{EntryWithPhonetic}{奢望}{she1wang4}{11,11}{⼤,⽉}[HSK 7-9]
  \definition{s.}{desejos desmedidos; esperanças extravagantes; esperança excessiva}
  \definition{v.}{esperar demais de alguém; nutrir expectativas excessivas}
\end{EntryWithPhonetic}

%%%%%%%%%% 舌 %%%%%%%%%%
\subsection*{舌}\addcontentsline{loh}{figure}{舌 \dpy{she2}}

\begin{EntryWithPhonetic}{舌}{she2}{6}{⾆}[Kangxi 135]
  \definition*{s.}{Sobrenome: She}
  \definition[片,条]{s.}{língua (de um ser humano ou animal); glossa | algo em forma de língua | língua de sino; badalo}
\end{EntryWithPhonetic}

\begin{EntryWithPhonetic}{舌头}{she2tou5}{6,5}{⾆,⼤}[HSK 6]
  \definition[个]{s.}{língua; órgão que auxilia no paladar, na mastigação e na pronúncia | espião}
\end{EntryWithPhonetic}

%%%%%%%%%% 折 %%%%%%%%%%
\subsection*{折}\addcontentsline{loh}{figure}{折 \dpy{she2}}

\begin{EntryWithPhonetic}{折}{she2}{7}{⼿}
  \definition{clas.}{um ato de zaju | um parágrafo em um drama da Dinastia Yuan, aproximadamente equivalente a uma cena ou ato em uma ópera moderna}
  \definition[张,个,些]{s.}{abatimento; desconto | os traços dos caracteres chineses têm o formato de 𠃍 e 乚 | pasta; livreto}
  \definition{v.}{estalar; quebrar; fazer quebrar | perder; sofrer a perda de | dobrar; torcer; curvar-se | voltar; mudar de direção; retornar | estar convencido; estar cheio de admiração | equivaler a; converter em}
  \seeref{zhe1}
  \seeref{zhe2}
\end{EntryWithPhonetic}

%%%%%%%%%% 蛇 %%%%%%%%%%
\subsection*{蛇}\addcontentsline{loh}{figure}{蛇 \dpy{she2}}

\begin{EntryWithPhonetic}{蛇}{she2}{11}{⾍}[HSK 5]
  \definition[条]{s.}{cobra; serpente; répteis}
\end{EntryWithPhonetic}

%%%%%%%%%% 舍 %%%%%%%%%%
\subsection*{舍}\addcontentsline{loh}{figure}{舍 \dpy{she3}}

\begin{EntryWithPhonetic}{舍}{she3}{8}{⾆}
  \definition{v.}{abandonar; desistir; descartar; jogar fora | dar esmola; dispensar caridade}
  \seeref{she4}
\end{EntryWithPhonetic}

\begin{EntryWithPhonetic}{舍不得}{she3bu5de5}{8,4,11}{⾆,⼀,⼻}[HSK 5]
  \definition{v.}{não se pode abandonar ou deixar, não se quer usar ou descartar; detestar separar-me ou usar}
\end{EntryWithPhonetic}

\begin{EntryWithPhonetic}{舍得}{she3 de5}{8,11}{⾆,⼻}[HSK 5]
  \definition{v.}{não guardar rancor; estar disposto a abrir mão de algo; estar disposto a gastar dinheiro, tempo, etc.; estar disposto a abrir mão de pessoas, oportunidades, coisas, etc. que são importantes para você}
\end{EntryWithPhonetic}

%%%%%%%%%% 设 %%%%%%%%%%
\subsection*{设}\addcontentsline{loh}{figure}{设 \dpy{she4}}

\begin{EntryWithPhonetic}{设}{she4}{6}{⾔}[HSK 7-9]
  \definition*{s.}{Sobrenome: She}
  \definition{conj.}{se; no caso | Matemática: dado; suponha; se}
  \definition{v.}{configurar; estabelecer; encontrar; colocar em prática}
\end{EntryWithPhonetic}

\begin{EntryWithPhonetic}{设备}{she4bei4}{6,8}{⾔,⼡}[HSK 3]
  \definition[台,套]{s.}{instalação; equipamento; montagem; um conjunto de edifícios ou equipamentos necessários para executar uma determinada tarefa ou suprir uma determinada necessidade}
\end{EntryWithPhonetic}

\begin{EntryWithPhonetic}{设定}{she4ding4}{6,8}{⾔,⼧}[HSK 7-9]
  \definition{v.}{definir; configurar; instalar}
\end{EntryWithPhonetic}

\begin{EntryWithPhonetic}{设法}{she4fa3}{6,8}{⾔,⽔}[HSK 7-9]
  \definition{v.}{encontrar um jeito de; conseguir; fazer esforços para; significa tentar encontrar uma maneira (de fazer algo)}
\end{EntryWithPhonetic}

\begin{EntryWithPhonetic}{设计}{she4ji4}{6,4}{⾔,⾔}[HSK 3]
  \definition[份]{s.}{plano; esquema; refere-se a um plano de design ou a um projeto para um plano, etc.}
  \definition{v.}{planejar; projetar; formular métodos, desenhos, etc. com antecedência, de acordo com determinados requisitos de finalidade, antes de iniciar oficialmente um trabalho | arquitetar; idear; tramar; fazer um plano}
\end{EntryWithPhonetic}

\begin{EntryWithPhonetic}{设计师}{she4 ji4 shi1}{6,4,6}{⾔,⾔,⼱}[HSK 6]
  \definition[个,位,名,些]{s.}{planejador de projeto; designer | arquiteto}
\end{EntryWithPhonetic}

\begin{EntryWithPhonetic}{设立}{she4li4}{6,5}{⾔,⽴}[HSK 3]
  \definition{v.}{fundar; estabelecer; começar}
\end{EntryWithPhonetic}

\begin{EntryWithPhonetic}{设施}{she4shi1}{6,9}{⾔,⽅}[HSK 4]
  \definition{s.}{facilidade; instalação; instituições, sistemas, organizações, edifícios, etc., estabelecidos para realizar um trabalho ou atender a uma necessidade}
\end{EntryWithPhonetic}

\begin{EntryWithPhonetic}{设想}{she4xiang3}{6,13}{⾔,⼼}[HSK 5]
  \definition[个,种]{s.}{plano provisório (ou ideia); (item, tipo) refere-se a algo hipotético ou imaginário}
  \definition{v.}{imaginar; prever; conceber; supor | ter consideração por}
\end{EntryWithPhonetic}

\begin{EntryWithPhonetic}{设置}{she4zhi4}{6,13}{⾔,⽹}[HSK 4]
  \definition{v.}{estabelecer; colocar em prática; estabelecer ou criar instituições, empregos, profissões ou códigos, etc. | encaixar; ajustar; instalar; configurar; colocar}
\end{EntryWithPhonetic}

%%%%%%%%%% 社 %%%%%%%%%%
\subsection*{社}\addcontentsline{loh}{figure}{社 \dpy{she4}}

\begin{EntryWithPhonetic}{社}{she4}{7}{⽰}[HSK 5]
  \definition[个,家]{s.}{agência; sociedade; órgão organizado; organização; comunidade | comuna popular | o deus da terra, sacrifícios a ele ou altares para tais sacrifícios; na antiguidade, o deus da terra, o local onde ele era venerado, o dia da veneração e o ritual eram chamados de 社 | agência de notícias |  imprensa}
\end{EntryWithPhonetic}

\begin{EntryWithPhonetic}{社会}{she4hui4}{7,6}{⽰,⼈}[HSK 3]
  \definition[个,种]{s.}{sociedade; em um determinado estágio do desenvolvimento histórico, a relação geral entre as pessoas nas atividades de produção | comunidade; geralmente se refere a um grupo de pessoas que estão conectadas por atividades comuns}
\end{EntryWithPhonetic}

\begin{EntryWithPhonetic}{社会主义}{she4hui4 zhu3yi4}{7,6,5,3}{⽰,⼈,⼂,⼂}[HSK 7-9]
  \definition*{s.}{Socialismo}
\end{EntryWithPhonetic}

\begin{EntryWithPhonetic}{社交}{she4jiao1}{7,6}{⽰,⼇}[HSK 7-9]
  \definition{s.}{contato social; interação social; refere-se às interações interpessoais na sociedade}
\end{EntryWithPhonetic}

\begin{EntryWithPhonetic}{社论}{she4lun4}{7,6}{⽰,⾔}[HSK 7-9]
  \definition[篇]{s.}{editorial; artigo principal}
\end{EntryWithPhonetic}

\begin{EntryWithPhonetic}{社区}{she4qu1}{7,4}{⽰,⼖}[HSK 5]
  \definition[个]{s.}{bairro; comunidade residencial; bairros da cidade, divididos de acordo com a localização geográfica | distrito; comunidade (para pessoas da mesma classe social, etc.) ; lugar onde pessoas com características comuns, como classe social, vivem juntas}
\end{EntryWithPhonetic}

\begin{EntryWithPhonetic}{社团}{she4tuan2}{7,6}{⽰,⼞}[HSK 7-9]
  \definition[个]{s.}{clube, associação; sociedade; organização social; termo genérico para diversas organizações de massa, como sindicatos, federações de mulheres e grêmios estudantis}
\end{EntryWithPhonetic}

%%%%%%%%%% 舍 %%%%%%%%%%
\subsection*{舍}\addcontentsline{loh}{figure}{舍 \dpy{she4}}

\begin{EntryWithPhonetic}{舍}{she4}{8}{⾆}
  \definition*{s.}{Sobrenome: She}
  \definition{clas.}{uma unidade antiga de distância igual a 30 li, 里}
  \definition{pron.}{meu, uma palavra humilde usada para se referir aos parentes mais jovens ou de geração inferior}
  \definition{s.}{cabana; casa | minha casa; minha humilde morada | chiqueiro; galpão; curral de gado}
  \seeref{she3}
  \seealsoref{里}{li3}
\end{EntryWithPhonetic}

%%%%%%%%%% 射 %%%%%%%%%%
\subsection*{射}\addcontentsline{loh}{figure}{射 \dpy{she4}}

\begin{EntryWithPhonetic}{射}{she4}{10}{⼨}[HSK 5]
  \definition*{s.}{Sobrenome: She}
  \definition{v.}{atirar; disparar | descarregar em jato; jorrar | emitir (luz, calor, etc.) | irradiar | aludir a algo ou alguém; insinuar}
\end{EntryWithPhonetic}

\begin{EntryWithPhonetic}{射击}{she4ji1}{10,5}{⼨,⼐}[HSK 5]
  \definition{s.}{tiro; tiro ao alvo}
  \definition{v.}{disparar; atirar}
\end{EntryWithPhonetic}

%%%%%%%%%% 涉 %%%%%%%%%%
\subsection*{涉}\addcontentsline{loh}{figure}{涉 \dpy{she4}}

\begin{EntryWithPhonetic}{涉}{she4}{10}{⽔}[HSK 6]
  \definition*{s.}{Sobrenome: She}
  \definition{v.}{vadear; atravessar ou passar um rio ou um obstáculo | passar por; experimentar | envolver; implicar}
\end{EntryWithPhonetic}

\begin{EntryWithPhonetic}{涉及}{she4ji2}{10,3}{⽔,⼃}[HSK 6]
  \definition{v.}{envolver; relacionar-se com; referir-se a; tocar em}
\end{EntryWithPhonetic}

\begin{EntryWithPhonetic}{涉嫌}{she4xian2}{10,13}{⽔,⼥}[HSK 7-9]
  \definition{v.}{ser suspeito; ser suspeito de estar envolvido; ser suspeito de envolvimento em determinado assunto}
\end{EntryWithPhonetic}

%%%%%%%%%% 摄 %%%%%%%%%%
\subsection*{摄}\addcontentsline{loh}{figure}{摄 \dpy{she4}}

\begin{EntryWithPhonetic}{摄}{she4}{13}{⼿}
  \definition*{s.}{Sobrenome: She}
  \definition{v.}{absorver; assimilar | tirar uma fotografia de; fotografar | conservar (a saúde) | atuar}
\end{EntryWithPhonetic}

\begin{EntryWithPhonetic}{摄氏}{she4shi4}{13,4}{⼿,⽒}
  \definition{s.}{graus Celsius (°C), centígrado}
\end{EntryWithPhonetic}

\begin{EntryWithPhonetic}{摄氏度}{she4shi4du4}{13,4,9}{⼿,⽒,⼴}[HSK 7-9]
  \definition{s.}{centígrado; grau Celsius}[水在100摄氏度时沸腾。===A água ferve a 100 graus Celsius.]
\end{EntryWithPhonetic}

\begin{EntryWithPhonetic}{摄像}{she4 xiang4}{13,13}{⼿,⼈}[HSK 5]
  \definition{v.}{gravar; filmar; filmar com câmera; fazer uma gravação de vídeo (com uma câmera de vídeo ou TV)}
\end{EntryWithPhonetic}

\begin{EntryWithPhonetic}{摄像机}{she4 xiang4 ji1}{13,13,6}{⼿,⼈,⽊}[HSK 5]
  \definition[个,部,台]{s.}{câmera de vídeo; dispositivo que pode ser usado para converter imagens captadas em sinais de imagem de televisão}
\end{EntryWithPhonetic}

\begin{EntryWithPhonetic}{摄影}{she4ying3}{13,15}{⼿,⼺}[HSK 5]
  \definition{v.}{fotografar; tirar uma foto; tirar fotos ou filmar}
\end{EntryWithPhonetic}

\begin{EntryWithPhonetic}{摄影师}{she4 ying3 shi1}{13,15,6}{⼿,⼺,⼱}[HSK 5]
  \definition[个,名,位]{s.}{fotógrafo; cinegrafista; operador de câmera; técnico de fotografia em estúdio fotográfico}
\end{EntryWithPhonetic}

%%%%%%%%%% 谁 %%%%%%%%%%
\subsection*{谁}\addcontentsline{loh}{figure}{谁 \dpy{shei2}}

\begin{EntryWithPhonetic}{谁}{shei2}{10}{⾔}[HSK 1]
  \definition{pron.}{quem? | (em pergunta retórica) quem?; usado em perguntas retóricas, para indicar que não há ninguém | refere-se a pessoas que não têm certeza, incluindo aquelas que não sabem | alguém; qualquer pessoa; indica qualquer pessoa ou qualquer um | repetido em uma frase para se referir a uma pessoa | (repetido em duas frases) quem quer que seja; fazer com que o sujeito e o objeto se refiram a duas pessoas diferentes}
  \seeref{shui2}
\end{EntryWithPhonetic}

\begin{EntryWithPhonetic}{谁知道}{shei2 zhi1dao4}{10,8,12}{⾔,⽮,⾡}[HSK 7-9]
  \definition{interj.}{``Quem sabe?''; ``Só Deus sabe\dots''; ``Quem diria?''; ``Quem poderia imaginar?''}
\end{EntryWithPhonetic}

%%%%%%%%%% 申 %%%%%%%%%%
\subsection*{申}\addcontentsline{loh}{figure}{申 \dpy{shen1}}

\begin{EntryWithPhonetic}{申}{shen1}{5}{⽥}
  \definition*{s.}{O nono dos doze Ramos Terrestres | Outro nome para Xangai, 上海 | Sobrenome: Shen}
  \definition{v.}{declarar; explicar; expressar}
  \seealsoref{上海}{shang4hai3}
\end{EntryWithPhonetic}

\begin{EntryWithPhonetic}{申办}{shen1ban4}{5,4}{⽥,⼒}[HSK 7-9]
  \definition{v.}{candidatar-se a sediar; candidatar-se a}
\end{EntryWithPhonetic}

\begin{EntryWithPhonetic}{申报}{shen1bao4}{5,7}{⽥,⼿}[HSK 7-9]
  \definition{v.}{reportar a um órgão superior; reportar aos superiores ou departamentos relevantes por escrito (frequentemente usado em documentos legais); reportar aos superiores em documentos oficiais | declarar algo (à Alfândega)}
\end{EntryWithPhonetic}

\begin{EntryWithPhonetic}{申领}{shen1ling3}{5,11}{⽥,⾴}[HSK 7-9]
  \definition{v.}{candidatar-se a; obter mediante candidatura a | solicitar (licença, visto etc.)}
\end{EntryWithPhonetic}

\begin{EntryWithPhonetic}{申请}{shen1qing3}{5,10}{⽥,⾔}[HSK 4]
  \definition[份,批,项]{s.}{a solicitação para; o requerimento para; um pedido para ser visto pelos superiores ou departamentos relevantes}
  \definition{v.}{solicitar; apresentar uma solicitação; apresentar os motivos e fazer o pedido aos superiores ou aos departamentos competentes}
\end{EntryWithPhonetic}

%%%%%%%%%% 伸 %%%%%%%%%%
\subsection*{伸}\addcontentsline{loh}{figure}{伸 \dpy{shen1}}

\begin{EntryWithPhonetic}{伸}{shen1}{7}{⼈}[HSK 5]
  \definition{v.}{alongar; esticar; estender}
\end{EntryWithPhonetic}

%%%%%%%%%% 身 %%%%%%%%%%
\subsection*{身}\addcontentsline{loh}{figure}{身 \dpy{shen1}}

\begin{EntryWithPhonetic}{身}{shen1}{7}{⾝}[Kangxi 158]
  \definition*{s.}{Sobrenome: Shen}
  \definition{adv.}{eu mesmo; a si mesmo; pessoalmente}
  \definition{s.}{corpo humano ou animal | vida | o caráter moral e a conduta de alguém; cultivo moral | corpo; a parte principal de uma estrutura; o corpo principal ou tronco de um objeto |  uma vida inteira; a vida inteira de alguém | \emph{status} social; identidade}
\end{EntryWithPhonetic}

\begin{EntryWithPhonetic}{身边}{shen1 bian1}{7,5}{⾝,⾡}[HSK 2]
  \definition{adv.}{ao redor; ao lado de alguém; perto do corpo | carregar consigo (transportar); à mão}
\end{EntryWithPhonetic}

\begin{EntryWithPhonetic}{身材}{shen1cai2}{7,7}{⾝,⽊}[HSK 4]
  \definition[种,个,具]{s.}{figura; estatura; altura e peso corporal}
\end{EntryWithPhonetic}

\begin{EntryWithPhonetic}{身份}{shen1fen4}{7,6}{⾝,⼈}[HSK 4]
  \definition[种]{s.}{status; capacidade; identidade; refere-se à origem, ao status e às qualificações de uma pessoa | dignidade; posição honrada; referência especial ao status respeitável}
\end{EntryWithPhonetic}

\begin{EntryWithPhonetic}{身份证}{shen1 fen4 zheng4}{7,6,7}{⾝,⼈,⾔}[HSK 3]
  \definition[张]{s.}{ID; bilhete de identidade; carteira de identidade}
\end{EntryWithPhonetic}

\begin{EntryWithPhonetic}{身高}{shen1 gao1}{7,10}{⾝,⾼}[HSK 4]
  \definition[个,种,段]{s.}{estatura; altura (de uma pessoa)}
\end{EntryWithPhonetic}

\begin{EntryWithPhonetic}{身上}{shen1 shang5}{7,3}{⾝,⼀}[HSK 1]
  \definition{s.}{no corpo de alguém | em um;  com um}
\end{EntryWithPhonetic}

\begin{EntryWithPhonetic}{身体}{shen1ti3}{7,7}{⾝,⼈}[HSK 1]
  \definition[具,个]{s.}{corpo | saúde; saúde das pessoas}
\end{EntryWithPhonetic}

\begin{EntryWithPhonetic}{身体能力}{shen1ti3 neng2li4}{7,7,10,2}{⾝,⼈,⾁,⼒}
  \definition{s.}{habilidade física}
\end{EntryWithPhonetic}

\begin{EntryWithPhonetic}{身体乳}{shen1ti3 ru3}{7,7,8}{⾝,⼈,⼄}
  \definition{s.}{loção corporal}
\end{EntryWithPhonetic}

\begin{EntryWithPhonetic}{身亡}{shen1wang2}{7,3}{⾝,⼇}
  \definition{v.}{morrer}
\end{EntryWithPhonetic}

%%%%%%%%%% 深 %%%%%%%%%%
\subsection*{深}\addcontentsline{loh}{figure}{深 \dpy{shen1}}

\begin{EntryWithPhonetic}{深}{shen1}{11}{⽔}[HSK 3]
  \definition*{s.}{Sobrenome: Shen}
  \definition{adj.}{profundo | difícil; intenso; profundo | completo; penetrante; intenso; profundo | próximo; íntimo; afeição profunda; relacionamento próximo | escuro; profundo | tardio}
  \definition{adv.}{muito; grandemente; profundamente}
  \definition{s.}{profundidade}
  \seealsoref{浅}{qian3}
\end{EntryWithPhonetic}

\begin{EntryWithPhonetic}{深处}{shen1 chu4}{11,5}{⽔,⼡}[HSK 5]
  \definition{s.}{profundidades; recantos; recessos | profundezas}
\end{EntryWithPhonetic}

\begin{EntryWithPhonetic}{深度}{shen1 du4}{11,9}{⽔,⼴}[HSK 5]
  \definition{adj.}{(em grau ou extensão) profundo; sério; grave}
  \definition{s.}{profundidade; grau de profundidade; | profundidade; rigor; meticulosidade; grau de contato com a essência das coisas | estágio avançado (ou em deterioração) de desenvolvimento; grau de crescimento e desenvolvimento das coisas}
\end{EntryWithPhonetic}

\begin{EntryWithPhonetic}{深厚}{shen1hou4}{11,9}{⽔,⼚}[HSK 4]
  \definition{adj.}{profundo; sentimentos fortes | sólido; profundamente enraizado; fundação sólida}
\end{EntryWithPhonetic}

\begin{EntryWithPhonetic}{深化}{shen1 hua4}{11,4}{⽔,⼔}[HSK 6]
  \definition{v.}{aprofundar; avançar; intensificar; tornar-se mais profundo; tornar mais profundo}
\end{EntryWithPhonetic}

\begin{EntryWithPhonetic}{深刻}{shen1ke4}{11,8}{⽔,⼑}[HSK 3]
  \definition{adj.}{profundo; instenso; chegar à essência de um assunto ou problema}
\end{EntryWithPhonetic}

\begin{EntryWithPhonetic}{深入}{shen1 ru4}{11,2}{⽔,⼊}[HSK 3]
  \definition{adj.}{profundo; completo}
  \definition{v.}{ir fundo em; penetrar em; penetrar o exterior; alcançar o interior ou o centro de algo}
\end{EntryWithPhonetic}

\begin{EntryWithPhonetic}{深深}{shen1 shen1}{11,11}{⽔,⽔}[HSK 6]
  \definition{adj.}{profundo; intenso}
  \definition{adv.}{profundamente; intensamente; descreve um grau profundo ou forte}
\end{EntryWithPhonetic}

\begin{EntryWithPhonetic}{深夜}{shen1ye4}{11,8}{⽔,⼣}
  \definition{adv.}{tarde da noite}
\end{EntryWithPhonetic}

%%%%%%%%%% 什 %%%%%%%%%%
\subsection*{什}\addcontentsline{loh}{figure}{什 \dpy{shen2}}

\begin{EntryWithPhonetic}{什}{shen2}{4}{⼈}
  \definition{pron.}{o que; qualquer coisa}
  \seeref{shi2}
  \seealsoref{什么}{shen2me5}
\end{EntryWithPhonetic}

\begin{EntryWithPhonetic}{什么}{shen2me5}{4,3}{⼈,⼃}[HSK 1]
  \definition{pron.}{o que?; expressar dúvida, perguntar sobre o mundo, locais, pessoas ou coisas | usado para se referir a algo indefinido; expressar incerteza | qualquer; todos; refere-se a todas as pessoas ou coisas | dois 什么 são usados juntos, indicando que o primeiro determina o segundo | usado para expressar surpresa ou insatisfação | usado para expressar discordância com o que foi dito; expressar negação | usado antes de elementos paralelos para indicar que a lista é infinita}
\end{EntryWithPhonetic}

\begin{EntryWithPhonetic}{什么时候}{shen2me5shi2hou5}{4,3,7,10}{⼈,⼃,⽇,⼈}
  \definition{adv.}{quando? | a que horas?}
\end{EntryWithPhonetic}

\begin{EntryWithPhonetic}{什么样}{shen2 me5 yang4}{4,3,10}{⼈,⼃,⽊}[HSK 2]
  \definition{pron.}{que tipo?; usado para perguntar sobre a natureza, características ou aparência de algo |  o quê?; de que tipo?; usado para perguntar sobre a situação ou o estado de alguém ou algo}
\end{EntryWithPhonetic}

%%%%%%%%%% 神 %%%%%%%%%%
\subsection*{神}\addcontentsline{loh}{figure}{神 \dpy{shen2}}

\begin{EntryWithPhonetic}{神}{shen2}{9}{⽰}[HSK 5]
  \definition*{s.}{Deus | Sobrenome: Shen}
  \definition{adj.}{inteligente; esperto | mágico; sobrenatural}
  \definition[个,位,尊,名]{s.}{divindade; deidade | espírito; mente; refere-se ao espírito, energia ou atenção de uma pessoa | olhar; expressão; expressões que refletem o estado interior}
\end{EntryWithPhonetic}

\begin{EntryWithPhonetic}{神话}{shen2hua4}{9,8}{⽰,⾔}[HSK 4]
  \definition[段,篇]{s.}{mito; mitologia; conto de fadas; refere-se a deuses e deusas lendários e histórias de heróis antigos deificados | lorota; refere-se a alegações ridículas e infundadas}
\end{EntryWithPhonetic}

\begin{EntryWithPhonetic}{神经}{shen2jing1}{9,8}{⽰,⽷}[HSK 5]
  \definition{adj.}{excêntrico; estranho; peculiar; descreve anormalidade neurológica}
  \definition[根,条]{s.}{nervo; um tipo de tecido presente no corpo humano ou animal que conecta o cérebro aos órgãos, transmitindo as sensações ao cérebro e as informações do cérebro aos órgãos}
\end{EntryWithPhonetic}

\begin{EntryWithPhonetic}{神经病的}{shen2jing1bing4 de5}{9,8,10,8}{⽰,⽷,⽧,⽩}
  \definition{adj.}{neuropático; neurótico}
\end{EntryWithPhonetic}

\begin{EntryWithPhonetic}{神经病学}{shen2jing1bing4 xue2}{9,8,10,8}{⽰,⽷,⽧,⼦}
  \definition{s.}{neurologia}
\end{EntryWithPhonetic}

\begin{EntryWithPhonetic}{神秘}{shen2mi4}{9,10}{⽰,⽲}[HSK 4]
  \definition{adj.}{místico; misterioso}
\end{EntryWithPhonetic}

\begin{EntryWithPhonetic}{神明}{shen2ming2}{9,8}{⽰,⽇}
  \definition{s.}{divindades | deuses}
\end{EntryWithPhonetic}

\begin{EntryWithPhonetic}{神奇}{shen2qi2}{9,8}{⽰,⼤}[HSK 5]
  \definition{adj.}{mágico; peculiar; místico; milagroso; faz as pessoas se sentirem muito revigoradas; é completamente inesperado e geralmente traz boas influências}
  \definition{adj.}{mágico; peculiar; místico; milagroso; algo que parece muito novo; algo que ninguém imaginaria, mas que geralmente traz bons resultados}
\end{EntryWithPhonetic}

\begin{EntryWithPhonetic}{神器}{shen2qi4}{9,16}{⽰,⼝}
  \definition{s.}{objeto mágico | objeto simbólico do poder imperial | arma fina | ferramenta muito útil}
\end{EntryWithPhonetic}

\begin{EntryWithPhonetic}{神情}{shen2 qing2}{9,11}{⽰,⼼}[HSK 5]
  \definition{s.}{aparência; expressão; atividades internas reveladas no rosto das pessoas}
\end{EntryWithPhonetic}

\begin{EntryWithPhonetic}{神兽}{shen2shou4}{9,11}{⽰,⼋}
  \definition{s.}{animal mitológico | fera}
\end{EntryWithPhonetic}

%%%%%%%%%% 审 %%%%%%%%%%
\subsection*{审}\addcontentsline{loh}{figure}{审 \dpy{shen3}}

\begin{EntryWithPhonetic}{审}{shen3}{8}{⼧}[HSK 6]
  \definition*{s.}{Sobrenome: Shen}
  \definition{adj.}{cuidadoso; detalhado; completo}
  \definition{adv.}{Literpario: realmente; de ​​fato; como esperado}
  \definition{v.}{examinar; analizar | julgar; interrogar | Literário: saber}
\end{EntryWithPhonetic}

\begin{EntryWithPhonetic}{审查}{shen3cha2}{8,9}{⼧,⽊}[HSK 6]
  \definition{v.}{examinar; investigar; verificar se algo está correto e apropriado (geralmente referindo-se a planos, propostas, escritos, qualificações pessoais, etc.); ler e avaliar (provas ou trabalhos de exame)}
\end{EntryWithPhonetic}

%%%%%%%%%% 甚 %%%%%%%%%%
\subsection*{甚}\addcontentsline{loh}{figure}{甚 \dpy{shen4}}

\begin{EntryWithPhonetic}{甚}{shen4}{9}{⽢}
  \definition{adv.}{muito; extremamente}
  \definition{pron.}{o que}
  \definition{v.}{exceder; superar}
  \seealsoref{什么}{shen2me5}
\end{EntryWithPhonetic}

\begin{EntryWithPhonetic}{甚而}{shen4'er2}{9,6}{⽢,⽽}
  \definition{conj.}{(ir) tão longe quanto | tanto que}
\end{EntryWithPhonetic}

\begin{EntryWithPhonetic}{甚或}{shen4huo4}{9,8}{⽢,⼽}
  \definition{conj.}{(ir) tão longe quanto | tanto que}
\end{EntryWithPhonetic}

\begin{EntryWithPhonetic}{甚至}{shen4zhi4}{9,6}{⽢,⾄}[HSK 4]
  \definition{conj.}{e até mesmo; nem mesmo; para apresentar uma situação típica e especial, para enfatizar a profundidade e a seriedade de uma situação}
\end{EntryWithPhonetic}

%%%%%%%%%% 升 %%%%%%%%%%
\subsection*{升}\addcontentsline{loh}{figure}{升 \dpy{sheng1}}

\begin{EntryWithPhonetic}{升}{sheng1}{4}{⼗}[HSK 3]
  \definition*{s.}{Sobrenome: Sheng}
  \definition{clas.}{litro (l)}
  \definition{s.}{sheng, uma unidade de medida seca para grãos (= 1 litro), um décimo de 斗}
  \definition{v.}{elevar; içar; subir; ascender; subir ou subir mais alto (oposto de 降) | promover; melhorar (nível)}
  \seealsoref{斗}{dou4}
  \seealsoref{降}{jiang4}
\end{EntryWithPhonetic}

\begin{EntryWithPhonetic}{升高}{sheng1 gao1}{4,10}{⼗,⾼}[HSK 5]
  \definition{v.}{subir; ascender | promover; elevar; intensificar; potencializar; melhorar}
\end{EntryWithPhonetic}

\begin{EntryWithPhonetic}{升级}{sheng1/ji2}{4,6}{⼗,⽷}[HSK 6]
  \definition{v.+compl.}{atualizar (software) | (guerra) escalar; (tensão) aprofundar | subir um ou mais níveis; passar de uma série ou classe inferior para uma série ou classe superior}
\end{EntryWithPhonetic}

\begin{EntryWithPhonetic}{升起}{sheng1qi3}{4,10}{⼗,⾛}
  \definition{v.}{levantar | içar | subir}
\end{EntryWithPhonetic}

\begin{EntryWithPhonetic}{升学}{sheng1 xue2}{4,8}{⼗,⼦}[HSK 6]
  \definition{v.}{ir para uma universidade, faculdade; entrar em uma universidade, faculdade}
\end{EntryWithPhonetic}

\begin{EntryWithPhonetic}{升值}{sheng1 zhi2}{4,10}{⼗,⼈}[HSK 6]
  \definition{v.}{Economia: reavaliar; apreciar | Figurativo: aumento de valor | valorização; apreciação; aumentar o valor; aumentar os preços}
\end{EntryWithPhonetic}

%%%%%%%%%% 生 %%%%%%%%%%
\subsection*{生}\addcontentsline{loh}{figure}{生 \dpy{sheng1}}

\begin{EntryWithPhonetic}{生}{sheng1}{5}{⽣}[HSK 2,3][Kangxi 100]
  \definition*{s.}{Sobrenome: Sheng}
  \definition{adj.}{vivo; vital | verde; não maduro | cru; não cozido; mal cozido | bruto; não refinado; não processado | estranho; desconhecido; não familiarizado | rígido; mecânico; forçado}
  \definition{adv.}{muito; usado antes de certas palavras que expressam emoções e sentimentos | verdadeiramente; realmente; forçosamente}
  \definition{s.}{vida | meio de subsistência | aluno; estudante | estudioso; antigamente chamados de eruditos | o tipo de personagem masculino na ópera de Pequim, etc.}
  \definition{suf.}{certos sufixos substantivos que se referem a pessoas (学生) | sufixos de certos advérbios (好生)}
  \definition{v.}{dar à luz; ter um filho | nascer | crescer; cultivar | viver; existir; sobreviver | favorecer; gerar; ocorrer | acender (uma fogueira); fazer o combustível queimar}
  \seealsoref{好生}{hao3sheng1}
  \seealsoref{学生}{xue2sheng5}
\end{EntryWithPhonetic}

\begin{EntryWithPhonetic}{生病}{sheng1bing4}{5,10}{⽣,⽧}[HSK 1]
  \definition{v.}{adoecer; ficar doente; ficar mal; contrair uma doença}
\end{EntryWithPhonetic}

\begin{EntryWithPhonetic}{生菜}{sheng1cai4}{5,11}{⽣,⾋}
  \definition{s.}{alface}
\end{EntryWithPhonetic}

\begin{EntryWithPhonetic}{生产}{sheng1chan3}{5,6}{⽣,⼇}[HSK 3]
  \definition{v.}{produzir; fabricar; utilizar ferramentas para mudar o objeto de trabalho e criar meios de produção e meios de subsistência | dar à luz uma criança; ter filhos}
\end{EntryWithPhonetic}

\begin{EntryWithPhonetic}{生成}{sheng1 cheng2}{5,6}{⽣,⼽}[HSK 5]
  \definition{v.}{formar; gerar; produzir | ter por natureza; nascer com}
\end{EntryWithPhonetic}

\begin{EntryWithPhonetic}{生词}{sheng1 ci2}{5,7}{⽣,⾔}[HSK 2]
  \definition[个,组,堆,条]{s.}{nova palavra; palavras que não aprendi, não conheço ou não entendo}
\end{EntryWithPhonetic}

\begin{EntryWithPhonetic}{生存}{sheng1cun2}{5,6}{⽣,⼦}[HSK 3]
  \definition{v.}{viver; sobreviver; subsistir; manter a vida; estar vivo}
\end{EntryWithPhonetic}

\begin{EntryWithPhonetic}{生的}{sheng1de5}{5,8}{⽣,⽩}
  \definition{conj.}{para evitar isso | para que\dots não\dots}
\end{EntryWithPhonetic}

\begin{EntryWithPhonetic}{生动}{sheng1dong4}{5,6}{⽣,⼒}[HSK 3]
  \definition{adj.}{vívido; animado; descreve a linguagem e as formas de expressão como sendo ativas e em movimento}
\end{EntryWithPhonetic}

\begin{EntryWithPhonetic}{生活}{sheng1huo2}{5,9}{⽣,⽔}[HSK 2]
  \definition[个,段,种]{s.}{vida; subsistência; as diversas atividades realizadas por pessoas ou seres vivos para sobreviver e se desenvolver | estilo de vida; condições de vida; situação em termos de vestuário, alimentação, habitação e transporte | trabalho (principalmente nas áreas industrial, agrícola e artesanal)}
  \definition{v.}{viver; realizar várias atividades | sobreviver}
\end{EntryWithPhonetic}

\begin{EntryWithPhonetic}{生活费}{sheng1 huo2 fei4}{5,9,9}{⽣,⽔,⾙}[HSK 6]
  \definition{s.}{subsídio; despesas de subsistência; despesas necessárias para manter a vida diária}
\end{EntryWithPhonetic}

\begin{EntryWithPhonetic}{生活垃圾}{sheng1huo2la1ji1}{5,9,8,6}{⽣,⽔,⼟,⼟}
  \definition{s.}{lixo doméstico}
\end{EntryWithPhonetic}

\begin{EntryWithPhonetic}{生活型}{sheng1huo2 xing2}{5,9,9}{⽣,⽔,⼟}
  \definition{s.}{forma de vida}
\end{EntryWithPhonetic}

\begin{EntryWithPhonetic}{生理}{sheng1li3}{5,11}{⽣,⽟}
  \definition{adj.}{fisiológico}
  \definition{s.}{fisiologia}
\end{EntryWithPhonetic}

\begin{EntryWithPhonetic}{生命}{sheng1ming4}{5,8}{⽣,⼝}[HSK 3]
  \definition{s.}{vida; não envolve apenas a existência e as atividades dos organismos, mas também inclui experiências de vida humana, valores e elementos-chave da sobrevivência e do desenvolvimento de várias coisas}
\end{EntryWithPhonetic}

\begin{EntryWithPhonetic}{生气}{sheng1/qi4}{5,4}{⽣,⽓}[HSK 1]
  \definition{s.}{vitalidade; vigor; energia da vida}
  \definition{v.+compl.}{ficar com raiva; ficar ofendido; ficar zangado; encontrar algo que não é do seu agrado e sentir-se descontente}
\end{EntryWithPhonetic}

\begin{EntryWithPhonetic}{生日}{sheng1ri4}{5,4}{⽣,⽇}[HSK 1]
  \definition[个,次]{s.}{aniversário; dia de nascimento, também se refere ao dia em que se completa um ano de idade a cada ano}
\end{EntryWithPhonetic}

\begin{EntryWithPhonetic}{生疏}{sheng1shu1}{5,12}{⽣,⽦}
  \definition{adj.}{inexperiência; desconhecimento de assuntos ou situações, falta de experiência suficiente | falta de prática; habilidades e ofícios enferrujam devido à falta de uso ao longo de um longo período de tempo | não tão perto como antes; as relações entre as pessoas tornaram-se menos íntimas e mais distantes do que antes}
\end{EntryWithPhonetic}

\begin{EntryWithPhonetic}{生态}{sheng1tai4}{5,8}{⽣,⼼}
  \definition{adj.}{ecológico}
  \definition{s.}{ecologia}
\end{EntryWithPhonetic}

\begin{EntryWithPhonetic}{生物}{sheng1wu4}{5,8}{⽣,⽜}
  \definition{adj.}{biológico}
  \definition{s.}{biologia (disciplina) | organismo | ser vivo}
\end{EntryWithPhonetic}

\begin{EntryWithPhonetic}{生意}{sheng1yi4}{5,13}{⽣,⼼}
  \definition[笔,种,次]{s.}{tendência a crescer; vitalidade; vigor; energia}
  \seeref{sheng1yi5}
\end{EntryWithPhonetic}

\begin{EntryWithPhonetic}{生意}{sheng1yi5}{5,13}{⽣,⼼}[HSK 3]
  \definition[笔,种,次]{s.}{comércio, compra e venda; negócios; indústria; colegas do mesmo setor}
  \seeref{sheng1yi4}
\end{EntryWithPhonetic}

\begin{EntryWithPhonetic}{生鱼片}{sheng1yu2pian4}{5,8,4}{⽣,⿂,⽚}
  \definition{s.}{fatias de peixe cru, \emph{sashimi}}
\end{EntryWithPhonetic}

\begin{EntryWithPhonetic}{生长}{sheng1zhang3}{5,4}{⽣,⾧}[HSK 3]
  \definition{v.}{cresçer; sob certas condições de vida, o volume e o peso dos organismos aumentam gradualmente | nascer e crescer}
\end{EntryWithPhonetic}

%%%%%%%%%% 声 %%%%%%%%%%
\subsection*{声}\addcontentsline{loh}{figure}{声 \dpy{sheng1}}

\begin{EntryWithPhonetic}{声}{sheng1}{7}{⼠}[HSK 5]
  \definition{clas.}{indica o número de vezes que um som é emitido}
  \definition{s.}{som; voz | reputação | consoante inicial (de uma sílaba chinesa) | tom; tom de voz | informação; notícia}
  \definition{v.}{declarar; anunciar; emitir um som}
\end{EntryWithPhonetic}

\begin{EntryWithPhonetic}{声明}{sheng1ming2}{7,8}{⼠,⽇}[HSK 3]
  \definition[项,份]{s.}{declaração}
  \definition{v.}{declarar; anunciar; expressar publicamente a sua atitude ou dizer a verdade}
\end{EntryWithPhonetic}

\begin{EntryWithPhonetic}{声音}{sheng1yin1}{7,9}{⼠,⾳}[HSK 2]
  \definition[个,种]{s.}{som; voz; a percepção auditiva das ondas sonoras}
\end{EntryWithPhonetic}

%%%%%%%%%% 绳 %%%%%%%%%%
\subsection*{绳}\addcontentsline{loh}{figure}{绳 \dpy{sheng2}}

\begin{EntryWithPhonetic}{绳}{sheng2}{11}{⽷}
  \definition*{s.}{Sobrenome: Sheng}
  \definition[根]{s.}{corda; cordão; barbante | a linha no marcador de tinta de carpinteiro}
  \definition{v.}{restringir; corrigir; sancionar | medir | continuar}
\end{EntryWithPhonetic}

\begin{EntryWithPhonetic}{绳子}{sheng2zi5}{11,3}{⽷,⼦}
  \definition[条]{s.}{corda | cordão}
\end{EntryWithPhonetic}

%%%%%%%%%% 省 %%%%%%%%%%
\subsection*{省}\addcontentsline{loh}{figure}{省 \dpy{sheng3}}

\begin{EntryWithPhonetic}{省}{sheng3}{9}{⽬}[HSK 2]
  \definition*{s.}{Sobrenome: Sheng}
  \definition{s.}{província; unidade administrativa, subordinada diretamente ao governo central | capital provincial; refere-se à capital da província, localização da administração provincial | abreviação (de palavras)}
  \definition{v.}{economizar; poupar; reduzir o consumo (em oposição a 费) | omitir; deixar de fora}
  \seeref{xing3}
  \seealsoref{费}{fei4}
\end{EntryWithPhonetic}

\begin{EntryWithPhonetic}{省城}{sheng3cheng2}{9,9}{⽬,⼟}
  \definition{s.}{capital da província}
\end{EntryWithPhonetic}

\begin{EntryWithPhonetic}{省会}{sheng3hui4}{9,6}{⽬,⼈}
  \definition{s.}{capital da província}
\end{EntryWithPhonetic}

\begin{EntryWithPhonetic}{省俭}{sheng3jian3}{9,9}{⽬,⼈}
  \definition{s.}{econômico | frugal}
  \definition{v.}{economizar}
\end{EntryWithPhonetic}

\begin{EntryWithPhonetic}{省力}{sheng3li4}{9,2}{⽬,⼒}
  \definition{v.}{economizar esforço ou trabalho}
\end{EntryWithPhonetic}

\begin{EntryWithPhonetic}{省钱}{sheng3 qian2}{9,10}{⽬,⾦}[HSK 6]
  \definition{adj.}{barato; não caro}
  \definition{v.}{economizar dinheiro}
\end{EntryWithPhonetic}

\begin{EntryWithPhonetic}{省却}{sheng3que4}{9,7}{⽬,⼙}
  \definition{v.}{livrar-se (para economizar espaço) | salvar}
\end{EntryWithPhonetic}

\begin{EntryWithPhonetic}{省心}{sheng3xin1}{9,4}{⽬,⼼}
  \definition{adj.}{despreocupado}
  \definition{v.}{ser poupado de preocupações | despreocupar"-se}
\end{EntryWithPhonetic}

\begin{EntryWithPhonetic}{省长}{sheng3zhang3}{9,4}{⽬,⾧}
  \definition[位,任]{s.}{governador; governador de uma província}
\end{EntryWithPhonetic}

%%%%%%%%%% 圣 %%%%%%%%%%
\subsection*{圣}\addcontentsline{loh}{figure}{圣 \dpy{sheng4}}

\begin{EntryWithPhonetic}{圣}{sheng4}{5}{⼟}
  \definition*{s.}{Sobrenome: Sheng}
  \definition{adj.}{santo; sagrado | imperial}
  \definition{s.}{santo; sábio | imperador | o maior mestre de uma determinada arte ou habilidade}
\end{EntryWithPhonetic}

\begin{EntryWithPhonetic}{圣诞节}{sheng4 dan4 jie2}{5,8,5}{⼟,⾔,⾋}[HSK 6]
  \definition*{s.}{Natal; Nascimento de Jesus Cristo em 25 de dezembro}
\end{EntryWithPhonetic}

\begin{EntryWithPhonetic}{圣地}{sheng4di4}{5,6}{⼟,⼟}
  \definition{s.}{terra santa (de uma religião) | lugar sagrado | santuário | cidade santa (como Jerusalém, Meca, etc.) | centro de interesse histórico}
\end{EntryWithPhonetic}

%%%%%%%%%% 胜 %%%%%%%%%%
\subsection*{胜}\addcontentsline{loh}{figure}{胜 \dpy{sheng4}}

\begin{EntryWithPhonetic}{胜}{sheng4}{9}{⾁}[HSK 3]
  \definition{adj.}{soberbo; maravilhoso; adorável}
  \definition[场]{s.}{vitória; sucesso | penteado de mulher; joias usadas pelas mulheres na antiguidade}
  \definition{v.}{vencer (oposto de 负, 败) | derrotar | (frequentemente seguido por 于, etc.) superar; ser superior a; levar a melhor sobre | vencer; ter sucesso; derrotar o adversário | ultrapassar; ser superior ao outro | suportar; ser capaz de suportar ou aguentar}
  \seealsoref{败}{bai4}
  \seealsoref{负}{fu4}
  \seealsoref{于}{yu2}
\end{EntryWithPhonetic}

\begin{EntryWithPhonetic}{胜负}{sheng4fu4}{9,6}{⾁,⾙}[HSK 5]
  \definition{s.}{vitória ou derrota; sucesso ou fracasso}
\end{EntryWithPhonetic}

\begin{EntryWithPhonetic}{胜利}{sheng4li4}{9,7}{⾁,⼑}[HSK 3]
  \definition{adv.}{com sucesso; triunfantemente; atingir o objetivo previsto}
  \definition{v.}{ganhar; vencer; triunfar; ter sucesso}
\end{EntryWithPhonetic}

\begin{EntryWithPhonetic}{胜算}{sheng4suan4}{9,14}{⾁,⽵}
  \definition{s.}{probabilidade de sucesso | estratégia que garante o sucesso}
  \definition{v.}{ter certeza do sucesso}
\end{EntryWithPhonetic}

%%%%%%%%%% 乘 %%%%%%%%%%
\subsection*{乘}\addcontentsline{loh}{figure}{乘 \dpy{sheng4}}

\begin{EntryWithPhonetic}{乘}{sheng4}{10}{⽲}
  \definition{clas.}{usado para carruagens de guerra puxada por quatro cavalos}
  \definition{s.}{obras históricas; livros de história geral | antigamente, uma carruagem puxada por quatro cavalos}
  \seeref{cheng2}
\end{EntryWithPhonetic}

%%%%%%%%%% 盛 %%%%%%%%%%
\subsection*{盛}\addcontentsline{loh}{figure}{盛 \dpy{sheng4}}

\begin{EntryWithPhonetic}{盛}{sheng4}{11}{⽫}
  \definition*{s.}{Sobrenome: Sheng}
  \definition{adj.}{florescente; próspero | vigoroso; enérgico | grandioso; magnífico | abundante; profundo | popular; comum; difundido; universal | amplo; generoso; abundante; suficiente | ótimo}
  \definition{adv.}{muito; profundamente}
  \seeref{cheng2}
\end{EntryWithPhonetic}

\begin{EntryWithPhonetic}{盛行}{sheng4xing2}{11,6}{⽫,⾏}[HSK 6]
  \definition{v.}{predominar; estar atual; estar na moda; ser amplamente popular}
\end{EntryWithPhonetic}

\begin{EntryWithPhonetic}{盛宴}{sheng4yan4}{11,10}{⽫,⼧}
  \definition{s.}{celebração}
\end{EntryWithPhonetic}

%%%%%%%%%% 剩 %%%%%%%%%%
\subsection*{剩}\addcontentsline{loh}{figure}{剩 \dpy{sheng4}}

\begin{EntryWithPhonetic}{剩}{sheng4}{12}{⼑}[HSK 5]
  \definition*{s.}{Sobrenome: Sheng}
  \definition{v.}{permanecer; ser deixado (para trás)}
\end{EntryWithPhonetic}

\begin{EntryWithPhonetic}{剩下}{sheng4 xia4}{12,3}{⼑,⼀}[HSK 5]
  \definition{v.}{permanecer; ser deixado (para trás); consumir e utilizar, restando apenas os resíduos}
\end{EntryWithPhonetic}

%%%%%%%%%% 失 %%%%%%%%%%
\subsection*{失}\addcontentsline{loh}{figure}{失 \dpy{shi1}}

\begin{EntryWithPhonetic}{失}{shi1}{5}{⼤}
  \definition{s.}{deslize; erro; defeito; acidente}
  \definition{v.}{perder (oposto de 得) | perder; deixar escapar | não agir de acordo com; negligenciar; violar | perder o controle de | errar; cometer um deslize; apresentar defeito em | não consiguir encontrar | não conseguir atingir o objetivo | desviar-se do normal | quebrar (uma promessa); voltar atrás (na palavra dada) | não conseguir obter | se perder}
  \seealsoref{得}{de2}
\end{EntryWithPhonetic}

\begin{EntryWithPhonetic}{失败}{shi1bai4}{5,8}{⼤,⾒}[HSK 4]
  \definition{adj.}{insatisfatório; a maneira como as coisas aconteceram deixou muito a desejar; o resultado final deixou muito a desejar}
  \definition{v.}{perder; ser derrotado; não vencer em uma guerra ou competição | falhar; fracassar; não dar em nada; falhar em atingir um objetivo ou meta desejada (trabalho, carreira, etc.)}
\end{EntryWithPhonetic}

\begin{EntryWithPhonetic}{失落}{shi1luo4}{5,12}{⼤,⾋}
  \definition{s.}{frustração | decepção | perda}
  \definition{v.}{perder (algo) | cair (algo) | sentir uma sensação de perda}
\end{EntryWithPhonetic}

\begin{EntryWithPhonetic}{失眠}{shi1mian2}{5,10}{⼤,⽬}
  \definition{s.}{insônia}
  \definition{v.}{ter insônia}
\end{EntryWithPhonetic}

\begin{EntryWithPhonetic}{失去}{shi1qu4}{5,5}{⼤,⼛}[HSK 3]
  \definition{v.}{perder}
\end{EntryWithPhonetic}

\begin{EntryWithPhonetic}{失望}{shi1wang4}{5,11}{⼤,⽉}[HSK 4]
  \definition{adj.}{desapontado; decepcionado}
  \definition{v.}{ficar desapontado; ficar decepcionado; estar desapontado; sentir-se sem esperança; perder a confiança}
\end{EntryWithPhonetic}

\begin{EntryWithPhonetic}{失误}{shi1wu4}{5,9}{⼤,⾔}[HSK 5]
  \definition[个]{s.}{erro; engano; equívoco; erros causados por negligência ou medidas inadequadas}
  \definition{v.}{cometer um erro; cometer um equívoco}
\end{EntryWithPhonetic}

\begin{EntryWithPhonetic}{失业}{shi1ye4}{5,5}{⼤,⼀}[HSK 4]
  \definition{v.}{não ter emprego; estar desempregado; estar sem trabalho; refere-se àqueles que estão dentro da idade legal para trabalhar, têm capacidade para trabalhar, estão desempregados e querem encontrar um emprego, mas não conseguem; embora se envolvam em certos trabalhos sociais, sua remuneração é menor do que o padrão mínimo de vida urbano local e são considerados desempregados}
\end{EntryWithPhonetic}

\begin{EntryWithPhonetic}{失意}{shi1yi4}{5,13}{⼤,⼼}
  \definition{adj.}{desapontado | frustrado}
\end{EntryWithPhonetic}

%%%%%%%%%% 师 %%%%%%%%%%
\subsection*{师}\addcontentsline{loh}{figure}{师 \dpy{shi1}}

\begin{EntryWithPhonetic}{师}{shi1}{6}{⼱}
  \definition*{s.}{Sobrenome: Shi}
  \definition[位,名,个]{s.}{professor; tutor; mestre | exemplo; modelo a seguir | título honorífico para um monge budista; (termo de respeito para um monge ou freira) mestre; mãe | do seu mestre ou professor | divisão; tropas; exército}
  \definition{suf.}{pessoa qualificada em determinada profissão}
  \definition{v.}{Literário: imitar; aprender}
\end{EntryWithPhonetic}

\begin{EntryWithPhonetic}{师父}{shi1 fu5}{6,4}{⼱,⽗}[HSK 6]
  \definition[个,位,名,些]{s.}{mestre; mestre trabalhador; um título respeitoso dado por um aprendiz ao seu mestre | um título respeitoso para monges, freiras e sacerdotes taoístas}
\end{EntryWithPhonetic}

\begin{EntryWithPhonetic}{师傅}{shi1fu5}{6,12}{⼱,⼈}[HSK 5]
  \definition[个,位,名]{s.}{mestre; um trabalhador qualificado; título honorífico para pessoas habilidosas | mestre; professor (em certos ofícios); pessoas que ensinam técnicas em áreas como engenharia, comércio e teatro}
\end{EntryWithPhonetic}

\begin{EntryWithPhonetic}{师生}{shi1 sheng1}{6,5}{⼱,⽣}[HSK 6]
  \definition{s.}{mestre e discípulo; professores e alunos; um nome combinado para professores e alunos}
\end{EntryWithPhonetic}

%%%%%%%%%% 诗 %%%%%%%%%%
\subsection*{诗}\addcontentsline{loh}{figure}{诗 \dpy{shi1}}

\begin{EntryWithPhonetic}{诗}{shi1}{8}{⾔}[HSK 4]
  \definition[首,句,行]{s.}{poesia; verso; poema; um gênero literário que reflete a vida e expressa emoções por meio de uma linguagem rítmica e rimada}
  \seealsoref{诗经}{shi1jing1}
\end{EntryWithPhonetic}

\begin{EntryWithPhonetic}{诗词}{shi1ci2}{8,7}{⾔,⾔}
  \definition{s.}{verso}
\end{EntryWithPhonetic}

\begin{EntryWithPhonetic}{诗歌}{shi1 ge1}{8,14}{⾔,⽋}[HSK 5]
  \definition[本,首,段]{s.}{poesia; poemas e canções; refere-se a todos os tipos de poesia}
\end{EntryWithPhonetic}

\begin{EntryWithPhonetic}{诗经}{shi1jing1}{8,8}{⾔,⽷}
  \definition*{s.}{Shijing, o Livro das Canções, antiga coleção de poemas chineses e um dos Cinco Clássicos do Confucionismo}
\end{EntryWithPhonetic}

\begin{EntryWithPhonetic}{诗句}{shi1ju4}{8,5}{⾔,⼝}
  \definition[行]{s.}{verso | versículo}
\end{EntryWithPhonetic}

\begin{EntryWithPhonetic}{诗人}{shi1 ren2}{8,2}{⾔,⼈}[HSK 4]
  \definition[个,位,名,些]{s.}{poeta; escritor de poesia}
\end{EntryWithPhonetic}

\begin{EntryWithPhonetic}{诗意}{shi1yi4}{8,13}{⾔,⼼}
  \definition{adj.}{poético}
  \definition{s.}{poesia}
\end{EntryWithPhonetic}

%%%%%%%%%% 湿 %%%%%%%%%%
\subsection*{湿}\addcontentsline{loh}{figure}{湿 \dpy{shi1}}

\begin{EntryWithPhonetic}{湿}{shi1}{12}{⽔}[HSK 4]
  \definition{adj.}{molhado; úmido; algo com água ou com muita água dentro}
\end{EntryWithPhonetic}

%%%%%%%%%% 十 %%%%%%%%%%
\subsection*{十}\addcontentsline{loh}{figure}{十 \dpy{shi2}}

\begin{EntryWithPhonetic}{十}{shi2}{2}{⼗}[HSK 1][Kangxi 24]
  \definition*{s.}{Sobrenome: Shi}
  \definition{num.}{dez; 10 | dezena | completo; no topo; máximo; referindo-se a algo que atingiu o ápice da perfeição ou plenitude | um monte de; indica que há muitos}
\end{EntryWithPhonetic}

\begin{EntryWithPhonetic}{十分}{shi2fen1}{2,4}{⼗,⼑}[HSK 2]
  \definition{adv.}{muito; totalmente; completamente; extremamente; indica um nível muito alto}
\end{EntryWithPhonetic}

\begin{EntryWithPhonetic}{十足}{shi2zu2}{2,7}{⼗,⾜}[HSK 5]
  \definition{adj.}{puro e simples; apenas este componente ou esta característica é muito evidente | 100\%; completo; total; muito satisfatório; muito adequado}
\end{EntryWithPhonetic}

%%%%%%%%%% 什 %%%%%%%%%%
\subsection*{什}\addcontentsline{loh}{figure}{什 \dpy{shi2}}

\begin{EntryWithPhonetic}{什}{shi2}{4}{⼈}
  \definition*{s.}{Sobrenome: Shi}
  \definition{adj.}{variado; sortido; diverso; vários; misturados}
  \definition{num.}{(em frações ou múltiplos) dez}
  \definition{s.}{várias coisas; artigos diversos}
  \seeref{shen2}
\end{EntryWithPhonetic}

%%%%%%%%%% 石 %%%%%%%%%%
\subsection*{石}\addcontentsline{loh}{figure}{石 \dpy{shi2}}

\begin{EntryWithPhonetic}{石}{shi2}{5}{⽯}[Kangxi 112]
  \definition*{s.}{Sobrenome: Shi}
  \definition{s.}{pedra; rocha; o material duro que constitui a crosta terrestre é composto por uma coleção de minerais | inscrição em pedra; esculturas em pedra}
  \seeref{dan4}
\end{EntryWithPhonetic}

\begin{EntryWithPhonetic}{石头}{shi2tou5}{5,5}{⽯,⼤}[HSK 3]
  \definition[块,堆,些]{s.}{rocha; pedra; uma substância muito dura que é o principal material da superfície da Terra}
\end{EntryWithPhonetic}

\begin{EntryWithPhonetic}{石油}{shi2you2}{5,8}{⽯,⽔}[HSK 3]
  \definition[桶,吨,升]{s.}{óleo; óleo fóssil; petróleo; um líquido inflamável extraído do solo, geralmente marrom escuro, preto ou verde escuro, do qual gasolina e outras substâncias podem ser obtidas}
\end{EntryWithPhonetic}

%%%%%%%%%% 时 %%%%%%%%%%
\subsection*{时}\addcontentsline{loh}{figure}{时 \dpy{shi2}}

\begin{EntryWithPhonetic}{时}{shi2}{7}{⽇}[HSK 3]
  \definition*{s.}{Sobrenome: Shi}
  \definition{adj.}{atual; presente | temporário; oportuno}
  \definition{adv.}{de vez em quando; ocasionalmente; de ​​tempos em tempos; equivalente a 常常 ou 经常 | às vezes\dots às vezes\dots; dois caracteres 时 usados juntos são equivalentes a ``有时…有时…'' e ``一会儿…一会儿…''}
  \definition{clas.}{hora, cada uma das 24 partes iguais de um dia e uma noite; também usada como unidade legal de tempo}
  \definition{s.}{dias; tempos; longo período de tempo; refere-se a um período de tempo | tempo; tempo fixo; refere-se ao tempo especificado | hora; hora do dia | temporada | chance; oportunidade; momento oportuno | atual; presente | tempo verbal; uma categoria gramatical que utiliza certas formas gramaticais para indicar o momento em que uma ação ocorre; geralmente é dividida em presente, pretérito e futuro}
  \seealsoref{常常}{chang2 chang2}
  \seealsoref{经常}{jing1chang2}
  \seealsoref{一会儿…一会儿…}{yi1hui4r5 yi1hui4r5}
  \seealsoref{有时…有时…}{you3shi2 you3shi2}
\end{EntryWithPhonetic}

\begin{EntryWithPhonetic}{时差}{shi2cha1}{7,9}{⽇,⼯}
  \definition{s.}{diferença de tempo | \emph{jet lag}}
\end{EntryWithPhonetic}

\begin{EntryWithPhonetic}{时常}{shi2chang2}{7,11}{⽇,⼱}[HSK 5]
  \definition{adv.}{frequentemente; com frequência}
\end{EntryWithPhonetic}

\begin{EntryWithPhonetic}{时代}{shi2dai4}{7,5}{⽇,⼈}[HSK 3]
  \definition[个]{s.}{idade; era; tempos; época; períodos e fases históricas divididas de acordo com condições econômicas, políticas, culturais e outras | um período na vida de alguém; uma fase na vida de uma pessoa}
\end{EntryWithPhonetic}

\begin{EntryWithPhonetic}{时而}{shi2'er2}{7,6}{⽇,⽽}[HSK 6]
  \definition{adv.}{às vezes; de tempos em tempos; indica que algo acontece repetidamente em intervalos irregulares}
\end{EntryWithPhonetic}

\begin{EntryWithPhonetic}{时而…,时而…}{shi2'er2 shi2'er2}{7,6,7,6}{⽇,⽽,⽇,⽽}[HSK 6]
  \definition{adv.}{agora\dots, agora\dots; às vezes\dots, às vezes\dots; usado antes e depois; indica que diferentes fenômenos ou coisas ocorrem alternadamente ou mudam continuamente dentro de um determinado período de tempo}[\underline{时而}下雨,\underline{时而}晴天。===Às vezes chove, às vezes faz sol. | 这个地方\underline{时而}热,\underline{时而}冷。===Este lugar às vezes é quente e às vezes frio.]
\end{EntryWithPhonetic}

\begin{EntryWithPhonetic}{时光}{shi2guang1}{7,6}{⽇,⼉}[HSK 5]
  \definition[台]{s.}{tempo; passagem do tempo | dias; horas; anos; épocas; períodos}
\end{EntryWithPhonetic}

\begin{EntryWithPhonetic}{时候}{shi2hou5}{7,10}{⽇,⼈}[HSK 1]
  \definition[个]{s.}{(um ponto no) tempo; momento; um determinado momento no tempo | (a duração do) tempo; um período de tempo com início e fim}
\end{EntryWithPhonetic}

\begin{EntryWithPhonetic}{时机}{shi2ji1}{7,6}{⽇,⽊}[HSK 5]
  \definition[个]{s.}{oportunidade; momento oportuno}
\end{EntryWithPhonetic}

\begin{EntryWithPhonetic}{时间}{shi2jian1}{7,7}{⽇,⾨}[HSK 1]
  \definition[段]{s.}{tempo; refere-se à forma de existência do movimento da matéria, um sistema contínuo composto pelo passado, presente e futuro | tempo; período (duração); um período de tempo com início e fim | tempo (um ponto); em algum momento do tempo}
\end{EntryWithPhonetic}

\begin{EntryWithPhonetic}{时节}{shi2 jie2}{7,5}{⽇,⾋}[HSK 6]
  \definition{s.}{temporada; um período de tempo em um ano com certas características, geralmente relacionadas à estação ou ao termo solar | época; tempo}
\end{EntryWithPhonetic}

\begin{EntryWithPhonetic}{时刻}{shi2ke4}{7,8}{⽇,⼑}[HSK 3]
  \definition{adv.}{constantemente; sempre; a cada momento; frequentemente}
  \definition[个,段]{s.}{tempo; hora; momento; conjuntura; um ponto no tempo}
\end{EntryWithPhonetic}

\begin{EntryWithPhonetic}{时期}{shi2qi1}{7,12}{⽇,⽉}[HSK 6]
  \definition[个,段]{s.}{um período específico; um período de tempo com uma certa característica}
\end{EntryWithPhonetic}

\begin{EntryWithPhonetic}{时时}{shi2 shi2}{7,7}{⽇,⽇}[HSK 6]
  \definition{adv.}{frequentemente; sempre; constantemente; indica que algo acontece várias vezes dentro de um determinado período de tempo}
\end{EntryWithPhonetic}

\begin{EntryWithPhonetic}{时事}{shi2shi4}{7,8}{⽇,⼅}[HSK 5]
  \definition{s.}{acontecimentos atuais; assuntos atuais; eventos atuais | tendências atuais | como as coisas estão indo | a situação atual}
\end{EntryWithPhonetic}

\begin{EntryWithPhonetic}{时装}{shi2 zhuang1}{7,12}{⽇,⾐}[HSK 6]
  \definition{s.}{vestido da moda; a última moda; os últimos estilos de roupas | roupas contemporâneas (em oposição ao 古装)}
  \seealsoref{古装}{gu3 zhuang1}
\end{EntryWithPhonetic}

%%%%%%%%%% 识 %%%%%%%%%%
\subsection*{识}\addcontentsline{loh}{figure}{识 \dpy{shi2}}

\begin{EntryWithPhonetic}{识}{shi2}{7}{⾔}[HSK 6]
  \definition{s.}{percepção; conhecimento}
  \definition{v.}{saber; reconhecer | saber; entender}
  \seeref{zhi4}
\end{EntryWithPhonetic}

\begin{EntryWithPhonetic}{识字}{shi2 zi4}{7,6}{⾔,⼦}[HSK 6]
  \definition{v.}{aprender a ler; tornar-se alfabetizado; reconhecer caracteres}
\end{EntryWithPhonetic}

%%%%%%%%%% 实 %%%%%%%%%%
\subsection*{实}\addcontentsline{loh}{figure}{实 \dpy{shi2}}

\begin{EntryWithPhonetic}{实}{shi2}{8}{⼧}
  \definition{adj.}{sólido; cheio por dentro; sem espaços vazios (oposto de 虚) | verdadeiro; real; atual; sincero | forte; eficaz; concreto; real}
  \definition{adv.}{verdadeiramente; realmente; de fato; originalmente}
  \definition{s.}{fato; realidade | semente; fruto}
  \definition{v.}{preencher}
  \seealsoref{虚}{xu1}
\end{EntryWithPhonetic}

\begin{EntryWithPhonetic}{实惠}{shi2hui4}{8,12}{⼧,⼼}[HSK 5]
  \definition{adj.}{sólido; substancial; benefícios práticos}
  \definition{s.}{benefício material; benefícios tangíveis; benefícios reais}
\end{EntryWithPhonetic}

\begin{EntryWithPhonetic}{实际}{shi2ji4}{8,7}{⼧,⾩}[HSK 2]
  \definition{adj.}{real; efetivo; concreto; prático | factual; prático; realista; de acordo com os fatos}
  \definition{s.}{realidade; prática; coisas e situações que existem objetivamente}
\end{EntryWithPhonetic}

\begin{EntryWithPhonetic}{实际上}{shi2 ji4 shang4}{8,7,3}{⼧,⾩,⼀}[HSK 3]
  \definition{adv.}{de fato; na verdade}
\end{EntryWithPhonetic}

\begin{EntryWithPhonetic}{实践}{shi2jian4}{8,12}{⼧,⾜}[HSK 6]
  \definition{s.}{prática; filosoficamente, refere-se às ações conscientes das pessoas para transformar a natureza e a sociedade; as atividades de produção são as atividades práticas mais básicas e também incluem atividades políticas, experimentos científicos, educação cultural, etc.}
  \definition{v.}{praticar; realizar; implementar planos e intenções em ações específicas}
\end{EntryWithPhonetic}

\begin{EntryWithPhonetic}{实力}{shi2li4}{8,2}{⼧,⼒}[HSK 3]
  \definition{s.}{força real; geralmente se refere à força militar e econômica de um país, grupo ou indivíduo, e também se refere à capacidade de um indivíduo ou grupo em uma competição}
\end{EntryWithPhonetic}

\begin{EntryWithPhonetic}{实施}{shi2shi1}{8,9}{⼧,⽅}[HSK 4]
  \definition{v.}{colocar em vigor; implementar (leis, políticas, etc.); executar; trazer (colocar) algo em vigor; fazer cumprir; colocar algo em (prática)}
\end{EntryWithPhonetic}

\begin{EntryWithPhonetic}{实习}{shi2xi2}{8,3}{⼧,⼄}[HSK 2]
  \definition{s.}{estagiário; prática; estágio}
  \definition{v.}{aplicar e testar os conhecimentos teóricos aprendidos no trabalho prático, a fim de exercitar a capacidade profissional}
\end{EntryWithPhonetic}

\begin{EntryWithPhonetic}{实现}{shi2xian4}{8,8}{⼧,⾒}[HSK 2]
  \definition{v.}{alcançar; atingir; realizar; concretizar; tornar (ideais, planos, etc.) realidade}
\end{EntryWithPhonetic}

\begin{EntryWithPhonetic}{实行}{shi2xing2}{8,6}{⼧,⾏}[HSK 3]
  \definition{v.}{praticar; implementar; executar; colocar em prática; realizar (programa, política, plano, etc.) por meio de ação}
\end{EntryWithPhonetic}

\begin{EntryWithPhonetic}{实验}{shi2yan4}{8,10}{⼧,⾺}[HSK 3]
  \definition[个,次]{s.}{teste; experimento; trabalho de laboratório}
  \definition{v.}{testar; experimentar; realizar uma operação ou se envolver em uma atividade para testar uma teoria ou hipótese científica}
\end{EntryWithPhonetic}

\begin{EntryWithPhonetic}{实验室}{shi2 yan4 shi4}{8,10,9}{⼧,⾺,⼧}[HSK 3]
  \definition[个,间]{s.}{laboratório; salas especiais para experimentos científicos}
\end{EntryWithPhonetic}

\begin{EntryWithPhonetic}{实用}{shi2yong4}{8,5}{⼧,⽤}[HSK 4]
  \definition{adj.}{prático; pragmático; funcional; atende aos requisitos reais da aplicação}
  \definition{v.}{colocar em uso prático}
\end{EntryWithPhonetic}

\begin{EntryWithPhonetic}{实在}{shi2zai4}{8,6}{⼧,⼟}[HSK 2]
  \definition{adj.}{honesto; sincero | verdadeiro; honesto; realista; não é falso, não é enganador}
  \definition{adv.}{verdadeiramente; de fato; na verdade; usado para reforçar o tom afirmativo, enfatizando que a situação é realmente assim}
\end{EntryWithPhonetic}

%%%%%%%%%% 拾 %%%%%%%%%%
\subsection*{拾}\addcontentsline{loh}{figure}{拾 \dpy{shi2}}

\begin{EntryWithPhonetic}{拾}{shi2}{9}{⼿}[HSK 5]
  \definition{num.}{dez (usado no lugar do numeral 十 em cheques, notas bancárias, etc., para evitar erros ou alterações)}
  \definition{v.}{pegar (do chão); recolher}
\end{EntryWithPhonetic}

%%%%%%%%%% 食 %%%%%%%%%%
\subsection*{食}\addcontentsline{loh}{figure}{食 \dpy{shi2}}

\begin{EntryWithPhonetic}{食}{shi2}{9}{⾷}[Kangxi 184]
  \definition{adj.}{para cozinhar; comestível}
  \definition{s.}{refeição; comida; o que as pessoas e os animais comem | alimentação; alimento para animais; ração | eclipse solar; eclipse lunar}
  \definition{v.}{comer}
  \seeref{si4}
\end{EntryWithPhonetic}

\begin{EntryWithPhonetic}{食品}{shi2 pin3}{9,9}{⾷,⼝}[HSK 3]
  \definition[种]{s.}{comida; gêneros alimentícios; provisões; alimentos vendidos em lojas que passaram por algum processamento}
\end{EntryWithPhonetic}

\begin{EntryWithPhonetic}{食堂}{shi2 tang2}{9,11}{⾷,⼟}[HSK 4]
  \definition[个,间]{s.}{cantina; refeitório}
\end{EntryWithPhonetic}

\begin{EntryWithPhonetic}{食物}{shi2wu4}{9,8}{⾷,⽜}[HSK 2]
  \definition[种]{s.}{comida; alimentos; comestíveis}
\end{EntryWithPhonetic}

\begin{EntryWithPhonetic}{食欲}{shi2 yu4}{9,11}{⾷,⽋}[HSK 6]
  \definition{adj.}{apetitoso}
  \definition{s.}{apetite; desejo humano de comer}
\end{EntryWithPhonetic}

%%%%%%%%%% 使 %%%%%%%%%%
\subsection*{使}\addcontentsline{loh}{figure}{使 \dpy{shi3}}

\begin{EntryWithPhonetic}{使}{shi3}{8}{⼈}[HSK 3]
  \definition{conj.}{se; supondo; usado como a primeira cláusula de uma frase complexa; indica uma relação hipotética; equivalente a 假如}
  \definition{s.}{enviado; mensageiro; pessoas em uma missão}
  \definition{v.}{enviar; despachar; dizer a alguém para fazer algo | usar; empregar; aplicar | deixar; chamar; habilitar}
  \seealsoref{假如}{jia3ru2}
\end{EntryWithPhonetic}

\begin{EntryWithPhonetic}{使得}{shi3 de5}{8,11}{⼈,⼻}[HSK 5]
  \definition{v.}{ser utilizável; poder ser usado | ser viável; ser exequível; ser possível;  poder fazer | fazer; tornar; causar um determinado resultado (intenção, plano, coisa)}
\end{EntryWithPhonetic}

\begin{EntryWithPhonetic}{使劲}{shi3/jin4}{8,7}{⼈,⼒}[HSK 4]
  \definition{v.+compl.}{colocar energia; exercer toda a sua força | esforçar-se para ajudar; colocar energia para ajudar}
\end{EntryWithPhonetic}

\begin{EntryWithPhonetic}{使用}{shi3yong4}{8,5}{⼈,⽤}[HSK 2]
  \definition{v.}{usar; empregar; aplicar; fazer com que pessoas, equipamentos, fundos, etc. sirvam a um determinado propósito}
\end{EntryWithPhonetic}

%%%%%%%%%% 始 %%%%%%%%%%
\subsection*{始}\addcontentsline{loh}{figure}{始 \dpy{shi3}}

\begin{EntryWithPhonetic}{始}{shi3}{8}{⼥}
  \definition*{s.}{Sobrenome: Shi}
  \definition{adv.}{somente então; não\dots até}
  \definition{s.}{começo; início}
  \definition{v.}{começar; iniciar}
\end{EntryWithPhonetic}

\begin{EntryWithPhonetic}{始终}{shi3zhong1}{8,8}{⼥,⽷}[HSK 3]
  \definition{adv.}{sempre; o tempo todo; durante todo; do começo ao fim; indica continuidade do início ao fim}
  \definition{s.}{todo o processo do começo ao fim}
\end{EntryWithPhonetic}

%%%%%%%%%% 屎 %%%%%%%%%%
\subsection*{屎}\addcontentsline{loh}{figure}{屎 \dpy{shi3}}

\begin{EntryWithPhonetic}{屎}{shi3}{9}{⼫}
  \definition{s.}{fezes | excrementos | (forma ligada) secreção (do ouvido, olho, etc.)}
\end{EntryWithPhonetic}

%%%%%%%%%% 士 %%%%%%%%%%
\subsection*{士}\addcontentsline{loh}{figure}{士 \dpy{shi4}}

\begin{EntryWithPhonetic}{士}{shi4}{3}{⼠}[Kangxi 33]
  \definition*{s.}{Sobrenome: Shi}
  \definition[位,名,个]{s.}{soldado; militar | oficial não comissionado; primeira classe de soldados | pessoa treinada em uma determinada área; algum tipo de técnico | pessoa (louvável) | bacharel (na China antiga) | classe social, entre os oficiais, 大夫, e o povo comum, 庶民 | estudioso | guarda-costas, uma das peças do xadrez chinês}
  \seealsoref{大夫}{da4fu1}
  \seealsoref{庶民}{shu4min2}
\end{EntryWithPhonetic}

\begin{EntryWithPhonetic}{士兵}{shi4bing1}{3,7}{⼠,⼋}[HSK 4]
  \definition[个,名,位,批,群]{s.}{soldado; militar; termo coletivo para oficiais não comissionados e soldados; os membros mais jovens do exército}
\end{EntryWithPhonetic}

%%%%%%%%%% 世 %%%%%%%%%%
\subsection*{世}\addcontentsline{loh}{figure}{世 \dpy{shi4}}

\begin{EntryWithPhonetic}{世}{shi4}{5}{⼀}
  \definition*{s.}{Sobrenome: Shi}
  \definition{s.}{vida; tempo de vida; vida humana | geração; geração após geração | idade; era | o mundo; sociedade | Geologia: época, abaixo de período}
\end{EntryWithPhonetic}

\begin{EntryWithPhonetic}{世代}{shi4dai4}{5,5}{⼀,⼈}
  \definition{adv.}{por muitas gerações, eras}
  \definition{s.}{geração | era}
\end{EntryWithPhonetic}

\begin{EntryWithPhonetic}{世纪}{shi4ji4}{5,6}{⼀,⽷}[HSK 3]
  \definition[个,段]{s.}{século; uma unidade para calcular anos, cem anos é um século}
\end{EntryWithPhonetic}

\begin{EntryWithPhonetic}{世界}{shi4jie4}{5,9}{⼀,⽥}[HSK 3]
  \definition[个,片,种]{s.}{mundo; todos os lugares da Terra | a soma da natureza e da sociedade humana; refere-se à soma de toda a existência objetiva na natureza e na sociedade humana | campo; refere-se a uma determinada área ou campo | o universo sem limites; costumava ser um termo budista, mas agora também se refere ao mundo natural ilimitado e à sociedade humana | situação social; a situação ou atmosfera social de um determinado período}
\end{EntryWithPhonetic}

\begin{EntryWithPhonetic}{世界杯}{shi4jie4bei1}{5,9,8}{⼀,⽥,⽊}[HSK 3]
  \definition*{s.}{Copa do Mundo; Troféu da Copa do Mundo}
\end{EntryWithPhonetic}

\begin{EntryWithPhonetic}{世锦赛}{shi4jin3sai4}{5,13,14}{⼀,⾦,⾙}
  \definition*{s.}{Campeonato Mundial}
\end{EntryWithPhonetic}

%%%%%%%%%% 市 %%%%%%%%%%
\subsection*{市}\addcontentsline{loh}{figure}{市 \dpy{shi4}}

\begin{EntryWithPhonetic}{市}{shi4}{5}{⼱}[HSK 2]
  \definition{s.}{mercado; lugar onde se concentra o comércio | cidade; município; áreas densamente povoadas, com indústrias, comércio e cultura desenvolvidos | relativo ao sistema tradicional chinês de pesos e medidas; unidades administrativas, incluindo cidades sob jurisdição direta e cidades sob jurisdição provincial (ou autônoma) | unidade padrão de mercado; pertencente ao sistema municipal (unidades de medida) | preço de transação no mercado}
  \definition{v.}{comprar ou vender; fazer transações}
\end{EntryWithPhonetic}

\begin{EntryWithPhonetic}{市场}{shi4chang3}{5,6}{⼱,⼟}[HSK 3]
  \definition[家]{s.}{mercado (também no abstrato); um lugar fixo onde as pessoas compram e vendem coisas juntas | área de \emph{marketing}; região onde o produto é vendido | âmbito de influência (figurado); uma metáfora para o escopo e o grau em que uma determinada ideia ou comportamento é aceito por outros}
\end{EntryWithPhonetic}

\begin{EntryWithPhonetic}{市尺}{shi4 chi3}{5,4}{⼱,⼫}
  \definition{clas.}{chi, uma unidade tradicional de comprimento, equivalente a 0,333 metros ou 1,094 pés}
\end{EntryWithPhonetic}

\begin{EntryWithPhonetic}{市斤}{shi4jin1}{5,4}{⼱,⽄}
  \definition{clas.}{jin, uma unidade tradicional de peso, cada uma contendo 10 liang (市两)  e equivalente a 0,5 quilogramas ou 1,102 libras}
  \seealsoref{市两}{shi4liang3}
\end{EntryWithPhonetic}

\begin{EntryWithPhonetic}{市两}{shi4liang3}{5,7}{⼱,⼀}
  \definition{clas.}{liang, uma unidade tradicional de peso, igual a 0,1 jin (市斤), e equivalente a 50 gramas ou 1,763 onças}
  \seealsoref{市斤}{shi4jin1}
\end{EntryWithPhonetic}

\begin{EntryWithPhonetic}{市民}{shi4 min2}{5,5}{⼱,⽒}[HSK 6]
  \definition[位,名]{s.}{habitantes da cidade; residente da cidade; moradores da cidade | cidadão; refere-se especificamente aos artesãos e comerciantes de pequeno e médio porte nas cidades da sociedade feudal tardia}
\end{EntryWithPhonetic}

\begin{EntryWithPhonetic}{市亩}{shi4mu3}{5,7}{⼱,⼇}
  \definition{clas.}{mu, uma unidade tradicional de área, igual a 60 zhang quadrados (平方市丈) e equivalente a 6,667 ares ou 0,165 acre}
  \seealsoref{平方市丈}{ping2fang1 shi4 zhang4}
\end{EntryWithPhonetic}

\begin{EntryWithPhonetic}{市区}{shi4 qu1}{5,4}{⼱,⼖}[HSK 4]
  \definition[个]{s.}{\emph{downtown}; centro da cidade; distrito urbano; áreas que ficam dentro dos limites da cidade e geralmente têm uma alta concentração de população e estoque de moradias}
\end{EntryWithPhonetic}

\begin{EntryWithPhonetic}{市升}{shi4sheng1}{5,4}{⼱,⼗}
  \definition{clas.}{sheng; uma unidade tradicional de volume, equivalente a 1 litro ou 1,76 \emph{pints} ou 0,22 galão}
\end{EntryWithPhonetic}

\begin{EntryWithPhonetic}{市长}{shi4 zhang3}{5,4}{⼱,⾧}[HSK 2]
  \definition[个,位,名]{s.}{prefeito; chefe administrativo responsável pela administração de uma cidade}
\end{EntryWithPhonetic}

\begin{EntryWithPhonetic}{市中心}{shi4zhong1xin1}{5,4,4}{⼱,⼁,⼼}
  \definition{s.}{centro da cidade}
\end{EntryWithPhonetic}

%%%%%%%%%% 示 %%%%%%%%%%
\subsection*{示}\addcontentsline{loh}{figure}{示 \dpy{shi4}}

\begin{EntryWithPhonetic}{示}{shi4}{5}{⽰}[Kangxi 113]
  \definition*{s.}{Sobrenome: Shi}
  \definition{s.}{(sua) carta  | missiva; instruções; palavras ou escritos para subordinados ou gerações mais jovens}
  \definition{v.}{mostrar; notificar; instruir | indicar; significar; mostrar ou apontar, fazer conhecido}
\end{EntryWithPhonetic}

\begin{EntryWithPhonetic}{示范}{shi4fan4}{5,9}{⽰,⾋}[HSK 5]
  \definition{v.}{demonstrar; dar o exemplo; criar um modelo que todos possam aprender}
\end{EntryWithPhonetic}

%%%%%%%%%% 似 %%%%%%%%%%
\subsection*{似}\addcontentsline{loh}{figure}{似 \dpy{shi4}}

\begin{EntryWithPhonetic}{似}{shi4}{6}{⼈}
  \definition{v.}{ver; parecer}
  \seeref{si4}
\end{EntryWithPhonetic}

\begin{EntryWithPhonetic}{似的}{shi4de5}{6,8}{⼈,⽩}[HSK 4]
  \definition{part.}{como; como\dots como; como se (embora); usada após uma palavra ou frase para indicar uma semelhança com algo ou uma situação | usada para indicar alto grau}
\end{EntryWithPhonetic}

%%%%%%%%%% 式 %%%%%%%%%%
\subsection*{式}\addcontentsline{loh}{figure}{式 \dpy{shi4}}

\begin{EntryWithPhonetic}{式}{shi4}{6}{⼷}[HSK 5]
  \definition*{s.}{Sobrenome: Shi}
  \definition{s.}{tipo; estilo | forma; padrão | ritual; cerimônia | fórmula; conjunto de símbolos que expressam uma lei natural na ciência natural | humor; modo; categoria gramatical que expressa a atitude subjetiva do falante em relação ao que está sendo dito, como narrativa, imperativa e condicional}
\end{EntryWithPhonetic}

%%%%%%%%%% 事 %%%%%%%%%%
\subsection*{事}\addcontentsline{loh}{figure}{事 \dpy{shi4}}

\begin{EntryWithPhonetic}{事}{shi4}{8}{⼅}[HSK 1]
  \definition[件,桩,回]{s.}{assunto; questão; coisa; negócio | problema; acidente | emprego; trabalho | responsabilidade; envolvimento | caso, coisa; o que aconteceu}
  \definition{v.}{servir; atender | estar envolvido em; dedicar-se a}
\end{EntryWithPhonetic}

\begin{EntryWithPhonetic}{事故}{shi4gu4}{8,9}{⼅,⽁}[HSK 3]
  \definition[起,桩,次,场]{s.}{acidente; perdas ou desastres repentinos, muitas vezes relacionados ao transporte, produção, trabalho e segurança pessoal}
\end{EntryWithPhonetic}

\begin{EntryWithPhonetic}{事后}{shi4 hou4}{8,6}{⼅,⼝}[HSK 6]
  \definition{s.}{depois; depois do evento; após o incidente ocorrer ou o problema ser resolvido}
\end{EntryWithPhonetic}

\begin{EntryWithPhonetic}{事件}{shi4jian4}{8,6}{⼅,⼈}[HSK 3]
  \definition[个,件,次]{s.}{evento; incidente; grandes eventos na história ou na sociedade}
\end{EntryWithPhonetic}

\begin{EntryWithPhonetic}{事情}{shi4qing5}{8,11}{⼅,⼼}[HSK 2]
  \definition[件,个,些,种]{s.}{assunto; questão; coisa; negócio | erro; acidente; infortúnio | (coloquial) emprego; trabalho}
\end{EntryWithPhonetic}

\begin{EntryWithPhonetic}{事儿}{shi4r5}{8,2}{⼅,⼉}
  \definition[件,桩]{s.}{o emprego | negócio | afazeres | assunto que precisa ser resolvido | matéria}
\end{EntryWithPhonetic}

\begin{EntryWithPhonetic}{事实}{shi4shi2}{8,8}{⼅,⼧}[HSK 3]
  \definition[个,件]{s.}{mito; lenda; uma narrativa sobre alguém ou algo que foi transmitida oralmente}
  \definition{v.}{dizer; contar; ser dito; contar a história}
\end{EntryWithPhonetic}

\begin{EntryWithPhonetic}{事实上}{shi4 shi2 shang4}{8,8,3}{⼅,⼧,⼀}[HSK 3]
  \definition{adv.}{realmente; de fato; na realidade; na verdade; de fato}
\end{EntryWithPhonetic}

\begin{EntryWithPhonetic}{事物}{shi4wu4}{8,8}{⼅,⽜}[HSK 4]
  \definition[种,类,个]{s.}{coisa; objeto; todos os objetos e fenômenos que existem objetivamente}
\end{EntryWithPhonetic}

\begin{EntryWithPhonetic}{事先}{shi4xian1}{8,6}{⼅,⼉}[HSK 4]
  \definition{adv.}{antes; de antemão; com antecedência; antecipadamente}
\end{EntryWithPhonetic}

\begin{EntryWithPhonetic}{事业}{shi4ye4}{8,5}{⼅,⼀}[HSK 3]
  \definition[个]{s.}{causa; carreira; empreendimento; atividades regulares realizadas por pessoas com um determinado objetivo, escala e sistema que têm impacto no desenvolvimento social | instituição; instalações; unidade de trabalho apoiada financeiramente pelo governo; refere-se especificamente a empresas que não têm rendimentos de produção, são financiadas pelo Estado e não realizam contabilidade económica}
\end{EntryWithPhonetic}

%%%%%%%%%% 势 %%%%%%%%%%
\subsection*{势}\addcontentsline{loh}{figure}{势 \dpy{shi4}}

\begin{EntryWithPhonetic}{势}{shi4}{8}{⼒}
  \definition{s.}{poder; força; influência | momentum; tendência | aparência externa de um objeto natural; fenômenos ou situações naturais | situação; estado de coisas; circunstâncias | sinal; gesto | genitais masculinos}
\end{EntryWithPhonetic}

\begin{EntryWithPhonetic}{势力}{shi4li4}{8,2}{⼒,⼒}[HSK 5]
  \definition[股]{s.}{força; poder; influência; forças políticas, econômicas, militares, etc.}
\end{EntryWithPhonetic}

%%%%%%%%%% 视 %%%%%%%%%%
\subsection*{视}\addcontentsline{loh}{figure}{视 \dpy{shi4}}

\begin{EntryWithPhonetic}{视}{shi4}{8}{⾒}
  \definition{v.}{olhar para | considerar; olhar para | inspecionar; observar}
\end{EntryWithPhonetic}

\begin{EntryWithPhonetic}{视角}{shi4jiao3}{8,7}{⾒,⾓}
  \definition{s.}{ângulo do qual se observa um objeto | (figurativo) perspectiva, ponto de vista, quadro de referência | (cinematografia) ângulo da câmera | (percepção visual) ângulo visual (o ângulo que um objeto visto subtende no olho) | (fotografia) ângulo de visão}
\end{EntryWithPhonetic}

\begin{EntryWithPhonetic}{视频}{shi4pin2}{8,13}{⾒,⾴}[HSK 5]
  \definition[个,段,条]{s.}{vídeo; videoclipe}
\end{EntryWithPhonetic}

\begin{EntryWithPhonetic}{视为}{shi4 wei2}{8,4}{⾒,⼂}[HSK 5]
  \definition{v.}{considerar; ver como; considerar como; considerar ser; achar que é}
\end{EntryWithPhonetic}

%%%%%%%%%% 试 %%%%%%%%%%
\subsection*{试}\addcontentsline{loh}{figure}{试 \dpy{shi4}}

\begin{EntryWithPhonetic}{试}{shi4}{8}{⾔}[HSK 1]
  \definition{s.}{teste; exame; avaliação de conhecimentos ou habilidades através de métodos específicos}
  \definition{v.}{tentar; investigar resultados ou verificar a natureza, não se envolver formalmente (em determinada atividade)}
\end{EntryWithPhonetic}

\begin{EntryWithPhonetic}{试点}{shi4 dian3}{8,9}{⾔,⽕}[HSK 6]
  \definition[个]{s.}{local onde um experimento é conduzido; unidade experimental; local de teste; um lugar para pequenos experimentos}
  \definition{v.}{experimentar; fazer experimentos; realizar testes em pontos selecionados; lançar um projeto piloto}
\end{EntryWithPhonetic}

\begin{EntryWithPhonetic}{试卷}{shi4juan4}{8,8}{⾔,⼙}[HSK 4]
  \definition[分,张]{s.}{folha de teste; folha de exame; papel usado para escrever as respostas nos exames}
\end{EntryWithPhonetic}

\begin{EntryWithPhonetic}{试题}{shi4 ti2}{8,15}{⾔,⾴}[HSK 3]
  \definition[道]{s.}{questões de um exame}
\end{EntryWithPhonetic}

\begin{EntryWithPhonetic}{试图}{shi4tu2}{8,8}{⾔,⼞}[HSK 5]
  \definition{v.}{tentar; pretender, fazer o possível para realizar algo}
\end{EntryWithPhonetic}

\begin{EntryWithPhonetic}{试验}{shi4yan4}{8,10}{⾔,⾺}[HSK 3]
  \definition{v.}{testar; fazer um teste; fazer um experimento; para examinar o efeito ou desempenho de algo, primeiro experimente em um laboratório ou em uma escala menor}
\end{EntryWithPhonetic}

%%%%%%%%%% 室 %%%%%%%%%%
\subsection*{室}\addcontentsline{loh}{figure}{室 \dpy{shi4}}

\begin{EntryWithPhonetic}{室}{shi4}{9}{⼧}[HSK 3]
  \definition*{s.}{Shi, a décima terceira das vinte e oito constelações da esfera celeste, composta por duas estrelas em linha reta na constelação de Pégaso | Sobrenome: Shi}
  \definition{s.}{sala; quarto; casa | departamento; sala como unidade administrativa ou de trabalho; órgãos públicos, fábricas, escolas e outras unidades de trabalho internas | esposa; familiares ou esposa | família; clã | cavidade; órgão com forma semelhante a uma câmara}
\end{EntryWithPhonetic}

%%%%%%%%%% 是 %%%%%%%%%%
\subsection*{是}\addcontentsline{loh}{figure}{是 \dpy{shi4}}

\begin{EntryWithPhonetic}{是}{shi4}{9}{⽇}[HSK 1]
  \definition*{s.}{Sobrenome: Shi}
  \definition{adj.}{correto; certo | verdadeiro}
  \definition{adv.}{(expressar afirmação firme) de fato; realmente}
  \definition{pron.}{isso; isto |  todos; qualquer um; usado antes de substantivos, tem o significado de 凡是}
  \definition{s.}{assuntos (importantes); grandes planos}
  \definition{v.}{usado como ``ser'' antes de substantivos ou pronomes para identificar, descrever ou ampliar o sujeito; indica que duas coisas são iguais, ou que a segunda explica a primeira | usado entre duas palavras idênticas; relacionar duas palavras semelhantes |  (usado antes de substantivos) ser exatamente; ser corretamente; usado antes de substantivos, tem o significado de 适合 | elogiar; justificar | expressar afirmação ou concordância (frequentemente usado sozinho) | usado para escolher perguntas, perguntas sim/não ou perguntas retóricas | (usado no início de uma frase) enfatizar uma determinada parte de uma frase | usado em perguntas sim-não}
  \seealsoref{凡是}{fan2shi4}
  \seealsoref{适合}{shi4he2}
\end{EntryWithPhonetic}

\begin{EntryWithPhonetic}{是不是}{shi4 bu2 shi4}{9,4,9}{⽇,⼀,⽇}[HSK 1]
  \definition{expr.}{sim ou não; é ou não é; se ou não; questões levantadas sobre a confirmação e a negação dos fatos}
\end{EntryWithPhonetic}

\begin{EntryWithPhonetic}{是的}{shi4de5}{9,8}{⽇,⽩}
  \definition{adv.}{sim | está certo}
\end{EntryWithPhonetic}

\begin{EntryWithPhonetic}{是否}{shi4fou3}{9,7}{⽇,⼝}[HSK 4]
  \definition{adv.}{se; se ou não; sim ou não}
\end{EntryWithPhonetic}

%%%%%%%%%% 适 %%%%%%%%%%
\subsection*{适}\addcontentsline{loh}{figure}{适 \dpy{shi4}}

\begin{EntryWithPhonetic}{适}{shi4}{9}{⾡}
  \definition*{s.}{Sobrenome: Shi}
  \definition{adj.}{confortável; bem | adequado; apropriado | certo; oportuno}
  \definition{v.}{ser apto; ser adequado; ser apropriado | ir; seguir; perseguir | (de uma mulher) casar}
\end{EntryWithPhonetic}

\begin{EntryWithPhonetic}{适当}{shi4 dang4}{9,6}{⾡,⼹}[HSK 6]
  \definition{s.}{adequado; apropriado}
\end{EntryWithPhonetic}

\begin{EntryWithPhonetic}{适合}{shi4he2}{9,6}{⾡,⼝}[HSK 3]
  \definition{v.}{servir; caber; se adequar; atender às necessidades de uma determinada situação ou pessoa}
\end{EntryWithPhonetic}

\begin{EntryWithPhonetic}{适应}{shi4ying4}{9,7}{⾡,⼴}[HSK 3]
  \definition{v.}{ajustar-se; adequar-se; adaptar-se; fazer as alterações correspondentes para se adequar à medida que as condições mudam}
\end{EntryWithPhonetic}

\begin{EntryWithPhonetic}{适用}{shi4 yong4}{9,5}{⾡,⽤}[HSK 3]
  \definition{adj.}{adequado; aplicável}
\end{EntryWithPhonetic}

%%%%%%%%%% 收 %%%%%%%%%%
\subsection*{收}\addcontentsline{loh}{figure}{收 \dpy{shou1}}

\begin{EntryWithPhonetic}{收}{shou1}{6}{⽁}[HSK 2]
  \definition{expr.}{aos cuidados de (usado na linha de endereço após o nome)}
  \definition{v.}{recolocar; juntar; reunir e juntar coisas espalhadas ou dispersas | recolher; cobrar | ganhar; obter (benefícios econômicos) | colher; recolher; colher ou cortar frutas, legumes, cereais maduros, etc. | aceitar; receber; acolher | controlar; restringir; restringir, controlar os sentimentos ou ações, para voltar ao estado normal | finalizar; parar; concluir; encerrar | prender; deter; colocar sob custódia}
\end{EntryWithPhonetic}

\begin{EntryWithPhonetic}{收藏}{shou1cang2}{6,17}{⽁,⾋}[HSK 6]
  \definition{v.}{coletar; armazenar; consagrar}
\end{EntryWithPhonetic}

\begin{EntryWithPhonetic}{收到}{shou1 dao4}{6,8}{⽁,⼑}[HSK 2]
  \definition{v.}{conseguir; obter; receber; alcançar}
\end{EntryWithPhonetic}

\begin{EntryWithPhonetic}{收费}{shou1 fei4}{6,9}{⽁,⾙}[HSK 3]
  \definition{v.}{cobrar; cobrar taxas}
\end{EntryWithPhonetic}

\begin{EntryWithPhonetic}{收购}{shou1 gou4}{6,8}{⽁,⾙}[HSK 5]
  \definition{v.}{comprar; adquirir; comprar muito em vários lugares | adquirir uma empresa; obter o controle efetivo de uma empresa por meio de dinheiro, transações de ações, etc.}
\end{EntryWithPhonetic}

\begin{EntryWithPhonetic}{收回}{shou1 hui2}{6,6}{⽁,⼞}[HSK 4]
  \definition{v.}{retomar; recuperar; relembrar; recordar; receber de volta o que foi enviado ou emprestado, ou o dinheiro que foi emprestado ou usado | sacar; retirar; recolher; rescindir; cancelar (uma opinião, ordem, etc.)}
\end{EntryWithPhonetic}

\begin{EntryWithPhonetic}{收获}{shou1huo4}{6,10}{⽁,⾋}[HSK 4]
  \definition[次,番,份]{s.}{resultados; ganhos; metaforicamente falando, conhecimento, experiência, etc. obtidos em estudo ou trabalho; os resultados obtidos por meio de trabalho árduo | colheita; colheita de safras}
  \definition{v.}{colher; juntar as colheitas}
\end{EntryWithPhonetic}

\begin{EntryWithPhonetic}{收集}{shou1 ji2}{6,12}{⽁,⾫}[HSK 5]
  \definition{v.}{coletar; reunir; recolher}
\end{EntryWithPhonetic}

\begin{EntryWithPhonetic}{收据}{shou1ju4}{6,11}{⽁,⼿}
  \definition[张]{s.}{recibo | \emph{voucher}}
\end{EntryWithPhonetic}

\begin{EntryWithPhonetic}{收看}{shou1 kan4}{6,9}{⽁,⽬}[HSK 3]
  \definition{v.}{assistir (a um programa de TV)}
\end{EntryWithPhonetic}

\begin{EntryWithPhonetic}{收敛}{shou1lian3}{6,11}{⽁,⽁}
  \definition{v.}{diminuir | desaparecer | fazer desaparecer | exercer restrição | conter (alegria, arrogância, etc.) | constringir | (matemática) convergir}
\end{EntryWithPhonetic}

\begin{EntryWithPhonetic}{收买}{shou1mai3}{6,6}{⽁,⼄}
  \definition{v.}{subornar | comprar}
\end{EntryWithPhonetic}

\begin{EntryWithPhonetic}{收取}{shou1 qu3}{6,8}{⽁,⼜}[HSK 6]
  \definition{v.}{obter; coletar; receber; aceitar o dinheiro pago pela outra parte}
\end{EntryWithPhonetic}

\begin{EntryWithPhonetic}{收入}{shou1ru4}{6,2}{⽁,⼊}[HSK 2]
  \definition[笔,个]{s.}{renda; salário; dinheiro recebido}
  \definition{v.}{receber dinheiro | coletar; receber}
\end{EntryWithPhonetic}

\begin{EntryWithPhonetic}{收拾}{shou1shi5}{6,9}{⽁,⼿}[HSK 5]
  \definition{v.}{arrumar; empacotar; limpar; organizar, policiar, restaurar a normalidade em situações adversas | consertar; reparar; restaurar algo que está danificado ao seu estado ou função original |  punir; punir alguém, geralmente com medidas mais severas | matar}
\end{EntryWithPhonetic}

\begin{EntryWithPhonetic}{收听}{shou1 ting1}{6,7}{⽁,⼝}[HSK 3]
  \definition{v.}{ouvir (rádio)}
\end{EntryWithPhonetic}

\begin{EntryWithPhonetic}{收养}{shou1 yang3}{6,9}{⽁,⼋}[HSK 6]
  \definition{v.}{acolher e criar; adotar; acolher os filhos dos outros e criá-los como se fossem da sua própria família}
\end{EntryWithPhonetic}

\begin{EntryWithPhonetic}{收益}{shou1yi4}{6,10}{⽁,⽫}[HSK 4]
  \definition{s.}{lucro; renda; benefício; ganhos; vantagens ou benefícios obtidos}
\end{EntryWithPhonetic}

\begin{EntryWithPhonetic}{收音机}{shou1yin1ji1}{6,9,6}{⽁,⾳,⽊}[HSK 3]
  \definition[部,台]{s.}{rádio; sem fio; um termo geral para receptores de rádio}
\end{EntryWithPhonetic}

%%%%%%%%%% 手 %%%%%%%%%%
\subsection*{手}\addcontentsline{loh}{figure}{手 \dpy{shou3}}

\begin{EntryWithPhonetic}{手}{shou3}{4}{⼿}[HSK 1][Kangxi 64]
  \definition{adj.}{prático; conveniente}
  \definition{adv.}{pessoalmente | para habilidade ou destreza}
  \definition{clas.}{usado para habilidades e competências | usado para indicar o número de vezes em que algo foi feito}
  \definition[双,只]{s.}{mão | pessoa proficiente em determinada atividade | habilidade; meios; referência a habilidades, técnicas ou meios | uma pessoa que faz ou é boa em determinado trabalho}
  \definition{v.}{ter na mão; segurar}
\end{EntryWithPhonetic}

\begin{EntryWithPhonetic}{手臂}{shou3bi4}{4,17}{⼿,⾁}
  \definition{s.}{braço}
\end{EntryWithPhonetic}

\begin{EntryWithPhonetic}{手边}{shou3bian1}{4,5}{⼿,⾡}
  \definition{adv.}{à mão | na mão}
\end{EntryWithPhonetic}

\begin{EntryWithPhonetic}{手表}{shou3biao3}{4,8}{⼿,⾐}[HSK 2]
  \definition[块,只,个]{s.}{relógio de pulso}
\end{EntryWithPhonetic}

\begin{EntryWithPhonetic}{手段}{shou3 duan4}{4,9}{⼿,⽎}[HSK 5]
  \definition[种,个]{s.}{meios; meio; medida; método; métodos e técnicas utilizados para atingir um determinado objetivo | truque; artifício; métodos inadequados de lidar com as pessoas | habilidade; capacidade; delicadeza; sutileza; técnica}
\end{EntryWithPhonetic}

\begin{EntryWithPhonetic}{手法}{shou3fa3}{4,8}{⼿,⽔}[HSK 5]
  \definition[种,个]{s.}{habilidade; técnica; técnicas de criação (de obras literárias e artísticas) | truque; artifício; artimanha; refere-se a métodos inadequados usados para lidar com as pessoas}
\end{EntryWithPhonetic}

\begin{EntryWithPhonetic}{手工}{shou3gong1}{4,3}{⼿,⼯}[HSK 4]
  \definition{s.}{trabalho manual; trabalho feito à mão | método de operação manual; método manual, sem máquina | remuneração por trabalho manual, braçal; custo de mão de obra braçal}
\end{EntryWithPhonetic}

\begin{EntryWithPhonetic}{手工艺人}{shou3gong1 yi4ren2}{4,3,4,2}{⼿,⼯,⾋,⼈}
  \definition{s.}{artesão}
\end{EntryWithPhonetic}

\begin{EntryWithPhonetic}{手机}{shou3ji1}{4,6}{⼿,⽊}[HSK 1]
  \definition[部,台,个]{s.}{celular; telefone celular; telefone móvel}
\end{EntryWithPhonetic}

\begin{EntryWithPhonetic}{手里}{shou3 li3}{4,7}{⼿,⾥}[HSK 4]
  \definition[个]{s.}{(uma situação está) nas mãos de alguém | em mãos}
\end{EntryWithPhonetic}

\begin{EntryWithPhonetic}{手刹}{shou3sha1}{4,8}{⼿,⼑}
  \definition{s.}{freio de mão}
\end{EntryWithPhonetic}

\begin{EntryWithPhonetic}{手术}{shou3shu4}{4,5}{⼿,⽊}[HSK 4]
  \definition[个,次]{s.}{cirurgia; operação (cirúrgica); método de tratamento no qual o médico usa uma faca, tesoura etc. para fazer uma incisão em uma parte do corpo do paciente}
  \definition{v.}{realizar uma cirurgia}
\end{EntryWithPhonetic}

\begin{EntryWithPhonetic}{手套}{shou3tao4}{4,10}{⼿,⼤}[HSK 4]
  \definition[副,套,双,种]{s.}{luvas; itens usados ​​nas mãos, feitos de algodão, lã, couro, etc., para proteger as mãos ou manter o frio longe}
\end{EntryWithPhonetic}

\begin{EntryWithPhonetic}{手续}{shou3xu4}{4,11}{⼿,⽷}[HSK 3]
  \definition[项]{s.}{processo; formalidade; procedimento; procedimentos realizados de acordo com os regulamentos}
\end{EntryWithPhonetic}

\begin{EntryWithPhonetic}{手续费}{shou3 xu4 fei4}{4,11,9}{⼿,⽷,⾙}[HSK 6]
  \definition{s.}{comissão; corretagem; taxa de serviço; taxas a pagar pelos procedimentos de manuseio}
\end{EntryWithPhonetic}

\begin{EntryWithPhonetic}{手指}{shou3zhi3}{4,9}{⼿,⼿}[HSK 3]
  \definition[个,根,只]{s.}{dedo da mão}
\end{EntryWithPhonetic}

%%%%%%%%%% 守 %%%%%%%%%%
\subsection*{守}\addcontentsline{loh}{figure}{守 \dpy{shou3}}

\begin{EntryWithPhonetic}{守}{shou3}{6}{⼧}[HSK 4]
  \definition*{s.}{Sobrenome: Shou}
  \definition{adv.}{próximo; perto de; perto de algum lugar em posição, perto de algum lugar}
  \definition{v.}{guardar; defender; estar presente para cuidar; não ir embora | manter vigilância; defender do ataque do oponente em uma luta ou confronto | observar; cumprir; respeitar; fazer as coisas como elas devem ser feitas | manter, observar a integridade; honrar a palavra de alguém; manter a palavra de alguém}
\end{EntryWithPhonetic}

\begin{EntryWithPhonetic}{守门员}{shou3men2yuan2}{6,3,7}{⼧,⾨,⼝}
  \definition{s.}{goleiro}
\end{EntryWithPhonetic}

%%%%%%%%%% 首 %%%%%%%%%%
\subsection*{首}\addcontentsline{loh}{figure}{首 \dpy{shou3}}

\begin{EntryWithPhonetic}{首}{shou3}{9}{⾸}[HSK 4,6][Kangxi 185]
  \definition*{s.}{Sobrenome: Shou}
  \definition{adj.}{primeiro}
  \definition{adv.}{inicialmente; como o primeiro; em primeiro lugar}
  \definition{clas.}{usado para canções e poemas}
  \definition{s.}{cabeça | cabeça; chefe; líder | capital (cidade)}
  \definition{v.}{apresentar acusações contra alguém}
\end{EntryWithPhonetic}

\begin{EntryWithPhonetic}{首次}{shou3 ci4}{9,6}{⾸,⽋}[HSK 6]
  \definition{s.}{o primeiro; pela primeira vez}
\end{EntryWithPhonetic}

\begin{EntryWithPhonetic}{首都}{shou3du1}{9,10}{⾸,⾢}[HSK 3]
  \definition[个,座]{s.}{capital (cidade); a sede do mais alto poder político do país e o centro político do país}
\end{EntryWithPhonetic}

\begin{EntryWithPhonetic}{首脑}{shou3 nao3}{9,10}{⾸,⾁}[HSK 6]
  \definition[位]{s.}{cabeça; líder; chefe}
\end{EntryWithPhonetic}

\begin{EntryWithPhonetic}{首席}{shou3 xi2}{9,10}{⾸,⼱}[HSK 6]
  \definition{adj.}{chefe; a primeira; a posição mais alta}
  \definition{s.}{assento de honra; o assento mais honroso}
\end{EntryWithPhonetic}

\begin{EntryWithPhonetic}{首席执行官}{shou3xi2 zhi2xing2 guan1}{9,10,6,6,8}{⾸,⼱,⼿,⾏,⼧}
  \definition{s.}{\emph{chief executive officer}, CEO}
\end{EntryWithPhonetic}

\begin{EntryWithPhonetic}{首先}{shou3xian1}{9,6}{⾸,⼉}[HSK 3]
  \definition{adv.}{primeiramente; antes de todos os outros}
  \definition{conj.}{acima de tudo; primeiramente; em primeiro lugar}
\end{EntryWithPhonetic}

\begin{EntryWithPhonetic}{首相}{shou3 xiang4}{9,9}{⾸,⽬}[HSK 6]
  \definition*[个,名,位]{s.}{Primeiro-Ministro (Japão, UK, etc.); o mais alto cargo oficial no gabinete de uma monarquia; o chefe do governo central de alguns países não monárquicos às vezes usa esse nome}
\end{EntryWithPhonetic}

%%%%%%%%%% 掱 %%%%%%%%%%
\subsection*{掱}\addcontentsline{loh}{figure}{掱 \dpy{shou3}}

\begin{EntryWithPhonetic}{掱}{shou3}{12}{⼿}
  \variantof{手}
\end{EntryWithPhonetic}

%%%%%%%%%% 寿 %%%%%%%%%%
\subsection*{寿}\addcontentsline{loh}{figure}{寿 \dpy{shou4}}

\begin{EntryWithPhonetic}{寿}{shou4}{7}{⼨}
  \definition[个,份]{s.}{vida longa; velhice | vida; idade | aniversário | (eufenismo) funerário; preparado antes da morte | longevidade}
\end{EntryWithPhonetic}

\begin{EntryWithPhonetic}{寿司}{shou4 si1}{7,5}{⼨,⼝}[HSK 5]
  \definition[份]{s.}{\emph{sushi}; iguaria tradicional japonesa}
\end{EntryWithPhonetic}

%%%%%%%%%% 受 %%%%%%%%%%
\subsection*{受}\addcontentsline{loh}{figure}{受 \dpy{shou4}}

\begin{EntryWithPhonetic}{受}{shou4}{8}{⼜}[HSK 3]
  \definition{v.}{receber; aceitar | sofrer; ser submetido a | aguentar; suportar; tolerar | ser agradável}
\end{EntryWithPhonetic}

\begin{EntryWithPhonetic}{受不了}{shou4bu5liao3}{8,4,2}{⼜,⼀,⼅}[HSK 4]
  \definition{v.}{ser insuportável; não poder suportar algo; não suportar algo}
\end{EntryWithPhonetic}

\begin{EntryWithPhonetic}{受到}{shou4dao4}{8,8}{⼜,⼑}[HSK 2]
  \definition{v.}{receber; receber itens, mensagens, instruções, etc. fornecidos por outras pessoas}
\end{EntryWithPhonetic}

\begin{EntryWithPhonetic}{受得了}{shou4de5liao3}{8,11,2}{⼜,⼻,⼅}
  \definition{v.}{suportar | aguentar}
\end{EntryWithPhonetic}

\begin{EntryWithPhonetic}{受伤}{shou4shang1}{8,6}{⼜,⼈}[HSK 3]
  \definition{v.}{ser ferido; sofrer uma lesão}
\end{EntryWithPhonetic}

\begin{EntryWithPhonetic}{受限}{shou4xian4}{8,8}{⼜,⾩}
  \definition{v.}{ser limitado | ser restrito | ser constrangido}
\end{EntryWithPhonetic}

\begin{EntryWithPhonetic}{受灾}{shou4 zai1}{8,7}{⼜,⽕}[HSK 5]
  \definition{v.}{ser atingido por um desastre natural (ou calamidade) | ser atingido por uma adversidade natural}
\end{EntryWithPhonetic}

%%%%%%%%%% 兽 %%%%%%%%%%
\subsection*{兽}\addcontentsline{loh}{figure}{兽 \dpy{shou4}}

\begin{EntryWithPhonetic}{兽}{shou4}{11}{⼋}
  \definition{adj.}{bestial; brutal}
  \definition{s.}{besta; animal}
\end{EntryWithPhonetic}

\begin{EntryWithPhonetic}{兽力车}{shou4 li4 che1}{11,2,4}{⼋,⼒,⾞}
  \definition{s.}{veículo puxado por animais  (oposto a 人力车) | carruagem; carroça}
  \seealsoref{人力车}{ren2 li4 che1}
\end{EntryWithPhonetic}

\begin{EntryWithPhonetic}{兽行}{shou4xing2}{11,6}{⼋,⾏}
  \definition{s.}{ato brutal; brutalidade | bestialidade}
\end{EntryWithPhonetic}

%%%%%%%%%% 售 %%%%%%%%%%
\subsection*{售}\addcontentsline{loh}{figure}{售 \dpy{shou4}}

\begin{EntryWithPhonetic}{售}{shou4}{11}{⼝}
  \definition{v.}{vender | fazer (o plano, truque, etc.) funcionar; continuar (as intrigas) | realizar (intrigas)}
\end{EntryWithPhonetic}

\begin{EntryWithPhonetic}{售货员}{shou4huo4yuan2}{11,8,7}{⼝,⾙,⼝}[HSK 4]
  \definition[名,位]{s.}{vendedor; balconista; assistente de loja; equipe que vende produtos em lojas}
\end{EntryWithPhonetic}

%%%%%%%%%% 瘦 %%%%%%%%%%
\subsection*{瘦}\addcontentsline{loh}{figure}{瘦 \dpy{shou4}}

\begin{EntryWithPhonetic}{瘦}{shou4}{14}{⽧}[HSK 5]
  \definition{adj.}{magro; esquelético (oposto de 胖, 肥) | magro (oposto de 肥) | apertado (oposto de 肥) | infértil; pobre | esquelético; pouca gordura; pouca carne (em oposição a 或 ou 肥) | (roupas, sapatos, meias, etc.) apertado (em oposição a 肥) |magra; (carne comestível) com baixo teor de gordura (em oposição a 肥)}
  \definition{v.}{perder peso}
  \seealsoref{肥}{fei2}
  \seealsoref{或}{huo4}
  \seealsoref{胖}{pang4}
\end{EntryWithPhonetic}

%%%%%%%%%% 书 %%%%%%%%%%
\subsection*{书}\addcontentsline{loh}{figure}{书 \dpy{shu1}}

\begin{EntryWithPhonetic}{书}{shu1}{4}{⼄}[HSK 1]
  \definition*{s.}{Sobrenome: Shu}
  \definition[本,册,部,套,卷]{s.}{livro; obras encadernadas | carta; carta especial | documento | estilo de caligrafia; escrita}
  \definition{v.}{escrever; registrar}
\end{EntryWithPhonetic}

\begin{EntryWithPhonetic}{书包}{shu1 bao1}{4,5}{⼄,⼓}[HSK 1]
  \definition[个,款]{s.}{mochila para guardar livros e materiais escolares}
\end{EntryWithPhonetic}

\begin{EntryWithPhonetic}{书店}{shu1 dian4}{4,8}{⼄,⼴}[HSK 1]
  \definition[个,家]{s.}{livraria; lojas que vendem livros}
\end{EntryWithPhonetic}

\begin{EntryWithPhonetic}{书法}{shu1fa3}{4,8}{⼄,⽔}[HSK 5]
  \definition[幅,卷,种,派]{s.}{caligrafia; arte de escrever caracteres, especialmente arte de escrever caracteres chineses com um pincel}
\end{EntryWithPhonetic}

\begin{EntryWithPhonetic}{书房}{shu1 fang2}{4,8}{⼄,⼾}[HSK 6]
  \definition[间]{s.}{uma biblioteca (em uma residência privada); espaço para leitura e escrita}
\end{EntryWithPhonetic}

\begin{EntryWithPhonetic}{书柜}{shu1 gui4}{4,8}{⼄,⽊}[HSK 5]
  \definition{s.}{estante; armário de livros}
\end{EntryWithPhonetic}

\begin{EntryWithPhonetic}{书记}{shu1ji5}{4,5}{⼄,⾔}
  \definition{s.}{secretário (chefe de um ramo de um partido socialista ou comunista) | atendente | balconista | escriturário}
\end{EntryWithPhonetic}

\begin{EntryWithPhonetic}{书架}{shu1jia4}{4,9}{⼄,⽊}[HSK 3]
  \definition[个,种,套]{s.}{estante de livros}
\end{EntryWithPhonetic}

\begin{EntryWithPhonetic}{书桌}{shu1 zhuo1}{4,10}{⼄,⽊}[HSK 5]
  \definition[个,张]{s.}{escrivaninha; mesa para ler e escrever}
\end{EntryWithPhonetic}

%%%%%%%%%% 叔 %%%%%%%%%%
\subsection*{叔}\addcontentsline{loh}{figure}{叔 \dpy{shu1}}

\begin{EntryWithPhonetic}{叔}{shu1}{8}{⼜}
  \definition*{s.}{Sobrenome: Shu}
  \definition{s.}{irmão mais novo do pai; tio (por parte de pai)| irmão mais novo do marido | terceiro entre irmãos | tio | uma forma de tratamento para um homem um pouco mais jovem que o pai; tio | terceiro tio (de quatro irmãos) | primo mais novo da mãe}
\end{EntryWithPhonetic}

\begin{EntryWithPhonetic}{叔叔}{shu1shu5}{8,8}{⼜,⼜}
  \definition[个,位,名]{s.}{tio; irmão mais novo do pai | tio, dirigindo-se a um homem da mesma geração que o pai e mais jovem em idade}
\end{EntryWithPhonetic}

%%%%%%%%%% 疏 %%%%%%%%%%
\subsection*{疏}\addcontentsline{loh}{figure}{疏 \dpy{shu1}}

\begin{EntryWithPhonetic}{疏}{shu1}{12}{⽦}
  \definition*{s.}{Sobrenome: Shu}
  \definition{adj.}{fino; esparso; disperso (oposto a 密) | espalhado; disperso; difuso; a distância entre as coisas é grande; as lacunas entre as partes das coisas são grandes | distante; relacionamento distante; não próximo (de relações familiares ou sociais) | não familiarizado com; desconhecido | escasso; vazio}
  \definition{s.}{memorial; memorial ao trono; um texto em que um ministro na era feudal apresentava seus assuntos ao monarca em detalhes | comentário; anotações mais detalhadas de livros antigos do que 注}
  \definition{v.}{dragar (um rio, etc.) | negligenciar | dispersar; espalhar}
  \seealsoref{密}{mi4}
  \seealsoref{注}{zhu4}
\end{EntryWithPhonetic}

%%%%%%%%%% 舒 %%%%%%%%%%
\subsection*{舒}\addcontentsline{loh}{figure}{舒 \dpy{shu1}}

\begin{EntryWithPhonetic}{舒}{shu1}{12}{⾆}
  \definition*{s.}{Sobrenome: Shu}
  \definition{adj.}{lento; vagaroso; sem pressa | confortável; relaxado e feliz}
  \definition{v.}{esticar; desdobrar | alongar; relaxar}
\end{EntryWithPhonetic}

\begin{EntryWithPhonetic}{舒服}{shu1fu5}{12,8}{⾆,⽉}[HSK 2]
  \definition{adj.}{confortável; sentir-se relaxado e feliz, tanto física quanto mentalmente}
\end{EntryWithPhonetic}

\begin{EntryWithPhonetic}{舒适}{shu1shi4}{12,9}{⾆,⾡}[HSK 4]
  \definition{adj.}{aconchegante; confortável; acolhedor; cômodo}
\end{EntryWithPhonetic}

%%%%%%%%%% 输 %%%%%%%%%%
\subsection*{输}\addcontentsline{loh}{figure}{输 \dpy{shu1}}

\begin{EntryWithPhonetic}{输}{shu1}{13}{⾞}[HSK 3]
  \definition{v.}{transportar; entregar | contribuir com dinheiro; doar | perder; falhar; ser batido; ser derrotado}
\end{EntryWithPhonetic}

\begin{EntryWithPhonetic}{输出}{shu1 chu1}{13,5}{⾞,⼐}[HSK 5]
  \definition{v.}{exportar (de dentro para fora); transportar (de dentro) para fora | exportar; vender ou distribuir no exterior ou fora do país | emitir informações, programas, dados, sinais, etc. a partir de uma máquina; enviar por uma determinada instituição ou dispositivo (energia, sinal, etc.)}
\end{EntryWithPhonetic}

\begin{EntryWithPhonetic}{输入}{shu1ru4}{13,2}{⾞,⼊}[HSK 3]
  \definition{v.}{introduzir; importar; comprar bens, introduzir tecnologia, contratar mão de obra, introduzir capital, etc. | inserir informações, programas, dados, sinais, etc. em uma máquina}
\end{EntryWithPhonetic}

%%%%%%%%%% 蔬 %%%%%%%%%%
\subsection*{蔬}\addcontentsline{loh}{figure}{蔬 \dpy{shu1}}

\begin{EntryWithPhonetic}{蔬}{shu1}{15}{⾋}
  \definition{s.}{vegetais}
\end{EntryWithPhonetic}

\begin{EntryWithPhonetic}{蔬菜}{shu1cai4}{15,11}{⾋,⾋}[HSK 5]
  \definition[样,种]{s.}{verduras; legumes; vegetais; ervas que podem ser usadas na culinária}
\end{EntryWithPhonetic}

%%%%%%%%%% 熟 %%%%%%%%%%
\subsection*{熟}\addcontentsline{loh}{figure}{熟 \dpy{shu2}}

\begin{EntryWithPhonetic}{熟}{shu2}{15}{⽕}[HSK 2]
  \definition{adj.}{maduro (frutos) | pronto; cozido | processado, fabricado ou exercitado | familiar, bem conhecido; conhecido por ser comum ou frequentemente utilizado | habilidoso;  (trabalho, tecnologia) experiente; não é novato | profundo; sólido}
\end{EntryWithPhonetic}

\begin{EntryWithPhonetic}{熟练}{shu2lian4}{15,8}{⽕,⽷}[HSK 4]
  \definition{adj.}{especializado; proficiente; qualificado; habilidoso}
\end{EntryWithPhonetic}

\begin{EntryWithPhonetic}{熟人}{shu2 ren2}{15,2}{⽕,⼈}[HSK 3]
  \definition[位,名,个,些]{s.}{amigo; conhecido; pessoas que se conhecem há muito tempo; pessoas que são muito familiares}
\end{EntryWithPhonetic}

\begin{EntryWithPhonetic}{熟悉}{shu2xi1}{15,11}{⽕,⼼}[HSK 5]
  \definition{adj.}{familiarizado com; não ser estranho}
  \definition{v.}{estar familiarizado com; saber claramente que | conhecer bem algo ou alguém; compreender e dominar (a situação) através da observação ou da experiência}
\end{EntryWithPhonetic}

%%%%%%%%%% 属 %%%%%%%%%%
\subsection*{属}\addcontentsline{loh}{figure}{属 \dpy{shu3}}

\begin{EntryWithPhonetic}{属}{shu3}{12}{⼫}[HSK 3]
  \definition{s.}{categoria | gênero | membros da família; dependentes; familiares; parentes}
  \definition{v.}{estar sob; subordinado a | pertencer a | nascer no ano de (um dos doze animais do zodíaco)}
  \seeref{zhu3}
\end{EntryWithPhonetic}

\begin{EntryWithPhonetic}{属于}{shu3yu2}{12,3}{⼫,⼆}[HSK 3]
  \definition{v.}{pertencer a; fazer parte de; pertencer ou ser propriedade de uma determinada parte}
\end{EntryWithPhonetic}

%%%%%%%%%% 暑 %%%%%%%%%%
\subsection*{暑}\addcontentsline{loh}{figure}{暑 \dpy{shu3}}

\begin{EntryWithPhonetic}{暑}{shu3}{12}{⽇}
  \definition{adj.}{calor; clima quente; quente (em oposição a 寒)}
  \definition{s.}{verão}
  \seealsoref{寒}{han2}
\end{EntryWithPhonetic}

\begin{EntryWithPhonetic}{暑假}{shu3 jia4}{12,11}{⽇,⼈}[HSK 4]
  \definition[个]{s.}{férias de verão; feriado de verão; férias escolares de verão, na China, durante o sétimo e o oitavo meses do calendário gregoriano}
\end{EntryWithPhonetic}

%%%%%%%%%% 黍 %%%%%%%%%%
\subsection*{黍}\addcontentsline{loh}{figure}{黍 \dpy{shu3}}

\begin{EntryWithPhonetic}{黍}{shu3}{12}{⿉}[Kangxi 202]
  \definition{s.}{painço}
\end{EntryWithPhonetic}

%%%%%%%%%% 数 %%%%%%%%%%
\subsection*{数}\addcontentsline{loh}{figure}{数 \dpy{shu3}}

\begin{EntryWithPhonetic}{数}{shu3}{13}{⽁}[HSK 2]
  \definition{v.}{contar (número); contar (número) um a um | ser considerado excepcionalmente (bom, ruim, etc.) | enumerar; listar}
  \seeref{shu4}
  \seeref{shuo4}
\end{EntryWithPhonetic}

%%%%%%%%%% 鼠 %%%%%%%%%%
\subsection*{鼠}\addcontentsline{loh}{figure}{鼠 \dpy{shu3}}

\begin{EntryWithPhonetic}{鼠}{shu3}{13}{⿏}[HSK 5][Kangxi 208]
  \definition[只]{s.}{rato; camundongo}
\end{EntryWithPhonetic}

\begin{EntryWithPhonetic}{鼠标}{shu3biao1}{13,9}{⿏,⽊}[HSK 5]
  \definition[个,只]{s.}{\emph{mouse} (de computador); dispositivo de entrada externo para computadores, usado para controlar o movimento do cursor na tela do computador, selecionar objetos de operação, executar vários comandos, etc.}
\end{EntryWithPhonetic}

%%%%%%%%%% 薯 %%%%%%%%%%
\subsection*{薯}\addcontentsline{loh}{figure}{薯 \dpy{shu3}}

\begin{EntryWithPhonetic}{薯}{shu3}{16}{⾋}
  \definition{s.}{batata | inhame}
\end{EntryWithPhonetic}

\begin{EntryWithPhonetic}{薯片}{shu3 pian4}{16,4}{⾋,⽚}[HSK 6]
  \definition{s.}{batatas fritas (\emph{chips}); batatas fritas crocantes ; flocos finos feitos de batatas}
\end{EntryWithPhonetic}

\begin{EntryWithPhonetic}{薯条}{shu3 tiao2}{16,7}{⾋,⽊}[HSK 6]
  \definition{s.}{batatas fritas (palito)}
\end{EntryWithPhonetic}

%%%%%%%%%% 术 %%%%%%%%%%
\subsection*{术}\addcontentsline{loh}{figure}{术 \dpy{shu4}}

\begin{EntryWithPhonetic}{术}{shu4}{5}{⽊}
  \definition*{s.}{Sobrenome: Shu}
  \definition{s.}{arte; habilidade; técnica; tecnologia; acadêmico | método; tática; estratégia}
  \seeref{zhu2}
\end{EntryWithPhonetic}

\begin{EntryWithPhonetic}{术科}{shu4ke1}{5,9}{⽊,⽲}
  \definition{s.}{cursos técnicos oferecidos em treinamento militar ou físico (oposto a 学科)}
  \seealsoref{学科}{xue2 ke1}
\end{EntryWithPhonetic}

%%%%%%%%%% 束 %%%%%%%%%%
\subsection*{束}\addcontentsline{loh}{figure}{束 \dpy{shu4}}

\begin{EntryWithPhonetic}{束}{shu4}{7}{⽊}[HSK 3]
  \definition*{s.}{Sobrenome: Shu}
  \definition{clas.}{usado para cachos, molhos, feixes, feixes de luz, etc.}
  \definition{s.}{monte; pacote; maço; feixe; cacho; coisas agrupadas ou reunidas em tiras}
  \definition{v.}{atar; amarrar; vincular | controlar; restringir}
\end{EntryWithPhonetic}

\begin{EntryWithPhonetic}{束腰}{shu4yao1}{7,13}{⽊,⾁}
  \definition{s.}{cinto | cinta | cinturão}
\end{EntryWithPhonetic}

%%%%%%%%%% 树 %%%%%%%%%%
\subsection*{树}\addcontentsline{loh}{figure}{树 \dpy{shu4}}

\begin{EntryWithPhonetic}{树}{shu4}{9}{⽊}[HSK 1]
  \definition*{s.}{Sobrenome: Shu}
  \definition[棵,株]{s.}{árvore; nome comum das plantas lenhosas}
  \definition{v.}{plantar; cultivar | configurar; manter; estabelecer}
\end{EntryWithPhonetic}

\begin{EntryWithPhonetic}{树林}{shu4 lin2}{9,8}{⽊,⽊}[HSK 4]
  \definition[片,座]{s.}{bosque; muitas árvores que crescem em fragmentos, menores que as florestas}
\end{EntryWithPhonetic}

\begin{EntryWithPhonetic}{树莓}{shu4mei2}{9,10}{⽊,⾋}
  \definition{s.}{framboesa}
\end{EntryWithPhonetic}

\begin{EntryWithPhonetic}{树木}{shu4mu4}{9,4}{⽊,⽊}
  \definition{s.}{árvore}
\end{EntryWithPhonetic}

\begin{EntryWithPhonetic}{树叶}{shu4ye4}{9,5}{⽊,⼝}[HSK 4]
  \definition[片,枚,堆]{s.}{folha; folhagem}
\end{EntryWithPhonetic}

%%%%%%%%%% 竖 %%%%%%%%%%
\subsection*{竖}\addcontentsline{loh}{figure}{竖 \dpy{shu4}}

\begin{EntryWithPhonetic}{竖}{shu4}{9}{⽴}
  \definition*{s.}{Sobrenome: Shu}
  \definition{adj.}{vertical; ereto; perpendicular ao solo}
  \definition{s.}{traço vertical (em caracteres chineses) | empregados domésticos; jovens criados}
  \definition{v.}{colocar em pé; erguer; ficar de pé; colocar o objeto perpendicular ao solo}
\end{EntryWithPhonetic}

\begin{EntryWithPhonetic}{竖向}{shu4xiang4}{9,6}{⽴,⼝}
  \definition{adj.}{vertical}
\end{EntryWithPhonetic}

%%%%%%%%%% 庶 %%%%%%%%%%
\subsection*{庶}\addcontentsline{loh}{figure}{庶 \dpy{shu4}}

\begin{EntryWithPhonetic}{庶}{shu4}{11}{⼴}
  \definition*{s.}{Sobrenome: Shu}
  \definition{adj.}{multitudinário; numeroso}
  \definition{conj.}{para que; de ​​modo a}
  \definition{s.}{da ou pela concubina (diferentemente da esposa legal); no sistema patriarcal, refere-se ao ramo lateral da família}
\end{EntryWithPhonetic}

\begin{EntryWithPhonetic}{庶民}{shu4min2}{11,5}{⼴,⽒}
  \definition{s.}{(antigo) pessoas comuns | (antigo) plebeu; plebeus | (antigo) a multidão de pessoas comuns (na literatura erudita)}
\end{EntryWithPhonetic}

%%%%%%%%%% 数 %%%%%%%%%%
\subsection*{数}\addcontentsline{loh}{figure}{数 \dpy{shu4}}

\begin{EntryWithPhonetic}{数}{shu4}{13}{⽁}
  \definition{num.}{vários; alguns}
  \definition{s.}{número; cifra; figura | número (conceito matemático básico que representa a quantidade de coisas) | número; indica a quantidade de coisas a que se referem os substantivos ou pronomes | destino; sorte}
  \seeref{shu3}
  \seeref{shuo4}
\end{EntryWithPhonetic}

\begin{EntryWithPhonetic}{数据}{shu4ju4}{13,11}{⽁,⼿}[HSK 4]
  \definition[组,个,条]{s.}{dados; valores com base nos quais são realizadas estatísticas, cálculos, pesquisas científicas ou projetos técnicos}
\end{EntryWithPhonetic}

\begin{EntryWithPhonetic}{数量}{shu4liang4}{13,12}{⽁,⾥}[HSK 3]
  \definition[个,种]{s.}{quantidade; quantum; quantia; magnitude; número}
\end{EntryWithPhonetic}

\begin{EntryWithPhonetic}{数码}{shu4ma3}{13,8}{⽁,⽯}[HSK 4]
  \definition{s.}{dígito; numeral; algarismo | número; quantidade (usado principalmente na linguagem falada)}
  \definition{v.}{digitalizar}
\end{EntryWithPhonetic}

\begin{EntryWithPhonetic}{数目}{shu4 mu4}{13,5}{⽁,⽬}[HSK 5]
  \definition{s.}{número; quantidade; quantidade de algo expressa em uma determinada medida padrão (como unidades de medida, etc.)}
\end{EntryWithPhonetic}

\begin{EntryWithPhonetic}{数学}{shu4xue2}{13,8}{⽁,⼦}
  \definition{s.}{matemática; a ciência que estuda as formas espaciais e as relações quantitativas do mundo real, incluindo matemática elementar e matemática superior}
\end{EntryWithPhonetic}

\begin{EntryWithPhonetic}{数字}{shu4zi4}{13,6}{⽁,⼦}[HSK 2]
  \definition{adj.}{digital; usando tecnologia digital}
  \definition[个,串]{s.}{dígito; número; um caractere que representa um número | numeral; símbolos que representam números, como algarismos arábicos, algarismos romanos, etc. | quantidade; montante}
\end{EntryWithPhonetic}

%%%%%%%%%% 刷 %%%%%%%%%%
\subsection*{刷}\addcontentsline{loh}{figure}{刷 \dpy{shua1}}

\begin{EntryWithPhonetic}{刷}{shua1}{8}{⼑}[HSK 4]
  \definition{s.}{escova; pincel | (onomatopéia) farfalhar; descreve o som de uma passagem rápida}
  \definition{v.}{escovar; esfregar; remover com uma escova | borrar; colar; aplicar com um pincel | eliminar; remover; limpar | rolar; navegar; visualizar grandes quantidades de informações muito rapidamente em um curto período de tempo online ou em dispositivos móveis | deslizar (passar o cartão magnético)}
  \seeref{shua4}
\end{EntryWithPhonetic}

\begin{EntryWithPhonetic}{刷牙}{shua1ya2}{8,4}{⼑,⽛}[HSK 4]
  \definition{s.}{escovar os dentes}
\end{EntryWithPhonetic}

\begin{EntryWithPhonetic}{刷子}{shua1zi5}{8,3}{⼑,⼦}[HSK 4]
  \definition[把,个]{s.}{escova; escovão; utensílio feito de lã, fio de plástico, fio de metal, etc., para remover sujeira ou aplicar óleo de unção, etc., geralmente longo ou oval, alguns com alças}
\end{EntryWithPhonetic}

%%%%%%%%%% 耍 %%%%%%%%%%
\subsection*{耍}\addcontentsline{loh}{figure}{耍 \dpy{shua3}}

\begin{EntryWithPhonetic}{耍}{shua3}{9}{⽽}
  \definition{v.}{brincar com | empunhar | agir (legal, calmo, tranquilo, descolado, etc.) | exibir (uma habilidade, o temperamento de alguém, etc.)}
\end{EntryWithPhonetic}

\begin{EntryWithPhonetic}{耍赖}{shua3lai4}{9,13}{⽽,⾙}
  \definition{v.}{agir descaradamente | recusar -se a reconhecer que alguém perdeu o jogo ou fez uma promessa, etc. | agir como um idiota | agir como se algo nunca tivesse acontecido}
\end{EntryWithPhonetic}

%%%%%%%%%% 刷 %%%%%%%%%%
\subsection*{刷}\addcontentsline{loh}{figure}{刷 \dpy{shua4}}

\begin{EntryWithPhonetic}{刷}{shua4}{8}{⼑}
  \definition{adj.}{pálido ou branco-azulado}
  \definition{adv.}{bastante; completamente; extremamente; descreve movimentos ágeis}
  \seeref{shua1}
\end{EntryWithPhonetic}

%%%%%%%%%% 摔 %%%%%%%%%%
\subsection*{摔}\addcontentsline{loh}{figure}{摔 \dpy{shuai1}}

\begin{EntryWithPhonetic}{摔}{shuai1}{14}{⼿}[HSK 5]
  \definition{v.}{cair; tropeçar; perder o equilíbrio | mergulhar; precipitar-se; cair de uma altura elevada | quebrar; fazer cair e quebrar | lançar; atirar; arremessar; joguar coisas com força e para baixo | bater; golpear; bater com força para que o que está grudado cair}
\end{EntryWithPhonetic}

\begin{EntryWithPhonetic}{摔倒}{shuai1dao3}{14,10}{⼿,⼈}[HSK 5]
  \definition{v.}{cair; tropeçar; perder o equilíbrio e cair}
\end{EntryWithPhonetic}

%%%%%%%%%% 帅 %%%%%%%%%%
\subsection*{帅}\addcontentsline{loh}{figure}{帅 \dpy{shuai4}}

\begin{EntryWithPhonetic}{帅}{shuai4}{5}{⼱}[HSK 4]
  \definition*{s.}{Sobrenome: Shuai}
  \definition{adj.}{bonito; arrojado; elegante; inteligente}
  \definition{interj.}{Legal!}
  \definition[位,名,个,些]{s.}{comandante em chefe; o mais alto comandante do exército | comandante em chefe, a peça principal no xadrez chinês}
\end{EntryWithPhonetic}

\begin{EntryWithPhonetic}{帅哥}{shuai4 ge1}{5,10}{⼱,⼝}[HSK 4]
  \definition[个,位,名,些]{s.}{rapaz bonito; um garoto que é bonito e atraente na aparência}
\end{EntryWithPhonetic}

%%%%%%%%%% 率 %%%%%%%%%%
\subsection*{率}\addcontentsline{loh}{figure}{率 \dpy{shuai4}}

\begin{EntryWithPhonetic}{率}{shuai4}{11}{⽞}
  \definition*{s.}{Sobrenome: Shuai}
  \definition{adj.}{precipitado; não cuidadoso; não cauteloso | franco; direto | elegante; bonito; o mesmo que 帅}
  \definition{adv.}{geralmente; expressa uma estimativa incerta, equivalente a 大约 e 大抵}
  \definition{s.}{modelo; exemplo}
  \definition{v.}{liderar; comandar | obedecer; seguir}
  \seeref{lv4}
  \seealsoref{大抵}{da4di3}
  \seealsoref{大约}{da4yue1}
  \seealsoref{帅}{shuai4}
\end{EntryWithPhonetic}

\begin{EntryWithPhonetic}{率领}{shuai4ling3}{11,11}{⽞,⾴}[HSK 5]
  \definition{v.}{liderar (equipe ou grupo); chefiar; comandar}
\end{EntryWithPhonetic}

\begin{EntryWithPhonetic}{率先}{shuai4 xian1}{11,6}{⽞,⼉}[HSK 4]
  \definition{v.}{tomar a iniciativa de fazer algo; ser o primeiro a fazer algo; assumir a liderança}
\end{EntryWithPhonetic}

%%%%%%%%%% 双 %%%%%%%%%%
\subsection*{双}\addcontentsline{loh}{figure}{双 \dpy{shuang1}}

\begin{EntryWithPhonetic}{双}{shuang1}{4}{⼜}[HSK 3]
  \definition*{s.}{Sobrenome: Shuang}
  \definition{adj.}{dois; gêmeo; par; dual; em oposição a 单 | números pares | duplo; dobro}
  \definition{clas.}{usado para certos membros, órgãos ou coisas pareadas que são bilateralmente simétricas, por exemplo, sapatos, meias, pauzinhos, etc.}
  \seealsoref{单}{dan1}
\end{EntryWithPhonetic}

\begin{EntryWithPhonetic}{双层床}{shuang1ceng2chuang2}{4,7,7}{⼜,⼫,⼴}
  \definition{s.}{beliche}
\end{EntryWithPhonetic}

\begin{EntryWithPhonetic}{双打}{shuang1 da3}{4,5}{⼜,⼿}[HSK 6]
  \definition[场,局,次]{s.}{duplas (em esportes)}
\end{EntryWithPhonetic}

\begin{EntryWithPhonetic}{双方}{shuang1fang1}{4,4}{⼜,⽅}[HSK 3]
  \definition{s.}{ambos os lados; as duas partes; duas pessoas ou dois grupos frente a frente em um determinado relacionamento ou situação}
\end{EntryWithPhonetic}

\begin{EntryWithPhonetic}{双方同意}{shuang1fang1tong2yi4}{4,4,6,13}{⼜,⽅,⼝,⼼}
  \definition{s.}{acordo bilateral}
\end{EntryWithPhonetic}

\begin{EntryWithPhonetic}{双手}{shuang1 shou3}{4,4}{⼜,⼿}[HSK 5]
  \definition{s.}{com as duas mãos; ambas as mãos; par de mãos}
\end{EntryWithPhonetic}

%%%%%%%%%% 霜 %%%%%%%%%%
\subsection*{霜}\addcontentsline{loh}{figure}{霜 \dpy{shuang1}}

\begin{EntryWithPhonetic}{霜}{shuang1}{17}{⾬}
  \definition{s.}{geada | pó branco ou creme espalhado por uma superfície | glacê | creme de pele}
\end{EntryWithPhonetic}

%%%%%%%%%% 爽 %%%%%%%%%%
\subsection*{爽}\addcontentsline{loh}{figure}{爽 \dpy{shuang3}}

\begin{EntryWithPhonetic}{爽}{shuang3}{11}{⽘}[HSK 6]
  \definition{adj.}{claro; nítido; brilhante | franco; de coração aberto; direto | relaxado; confortável}
  \definition{v.}{desviar; afastar | tornar confortável; ficar confortável}
\end{EntryWithPhonetic}

%%%%%%%%%% 谁 %%%%%%%%%%
\subsection*{谁}\addcontentsline{loh}{figure}{谁 \dpy{shui2}}

\begin{EntryWithPhonetic}{谁}{shui2}{10}{⾔}[HSK 1]
  \seeref{shei2}
\end{EntryWithPhonetic}

%%%%%%%%%% 水 %%%%%%%%%%
\subsection*{水}\addcontentsline{loh}{figure}{水 \dpy{shui3}}

\begin{EntryWithPhonetic}{水}{shui3}{4}{⽔}[HSK 1][Kangxi 85]
  \definition*{s.}{Etnia Shui, que vive principalmente em Guizhou | Sobrenome: Shui}
  \definition{adj.}{de má qualidade; mal feito; de baixa qualidade e conteúdo}
  \definition{clas.}{usado para número de lavagens}
  \definition[条,杯]{s.}{água | rio | termo geral para rios, lagos, mares, etc.; água | corrente; fluxo de água | um líquido; suco ralo | teor de prata nas moedas | encargos adicionais ou receitas | água, um dos cinco elementos}
\end{EntryWithPhonetic}

\begin{EntryWithPhonetic}{水边}{shui3bian1}{4,5}{⽔,⾡}
  \definition{s.}{beira d'água | beira-mar | costa (de mar, lago ou rio)}
\end{EntryWithPhonetic}

\begin{EntryWithPhonetic}{水波}{shui3bo1}{4,8}{⽔,⽔}
  \definition{s.}{ondulação (na água) | onda}
\end{EntryWithPhonetic}

\begin{EntryWithPhonetic}{水槽}{shui3cao2}{4,15}{⽔,⽊}
  \definition{s.}{pia (de cozinha)}
\end{EntryWithPhonetic}

\begin{EntryWithPhonetic}{水产品}{shui3 chan3 pin3}{4,6,9}{⽔,⼇,⼝}[HSK 5]
  \definition{s.}{produto aquático (peixes, camarões, etc.)}
\end{EntryWithPhonetic}

\begin{EntryWithPhonetic}{水分}{shui3 fen4}{4,4}{⽔,⼑}[HSK 5]
  \definition{s.}{teor de umidade; água contida em um objeto | exagero; metáfora de algo falso}
\end{EntryWithPhonetic}

\begin{EntryWithPhonetic}{水果}{shui3guo3}{4,8}{⽔,⽊}[HSK 1]
  \definition[个]{s.}{fruta; um nome genérico para frutas com alto teor de água que podem ser consumidas, como peras, pêssegos, maçãs, etc.}
\end{EntryWithPhonetic}

\begin{EntryWithPhonetic}{水饺}{shui3jiao3}{4,9}{⽔,⾷}
  \definition{s.}{\emph{dumplings} | pastéis chineses cozidos}
\end{EntryWithPhonetic}

\begin{EntryWithPhonetic}{水库}{shui3 ku4}{4,7}{⽔,⼴}[HSK 5]
  \definition[座]{s.}{reservatório; lago artificial construído pelo homem, que utiliza barragens e outras estruturas para represar a água e regular o fluxo, podendo ser utilizado para armazenamento de água, geração de energia e piscicultura, entre outros fins}
\end{EntryWithPhonetic}

\begin{EntryWithPhonetic}{水灵}{shui3ling2}{4,7}{⽔,⽕}
  \definition{adj.}{cheio de vida (sobre uma pessoa, etc.) | úmido e brilhante (sobre os olhos) | fresco (sobre frutas, etc.) | brilhante | aparência saudável}
\end{EntryWithPhonetic}

\begin{EntryWithPhonetic}{水路}{shui3lu4}{4,13}{⽔,⾜}
  \definition{s.}{hidrovia}
\end{EntryWithPhonetic}

\begin{EntryWithPhonetic}{水泥}{shui3ni2}{4,8}{⽔,⽔}[HSK 6]
  \definition[袋,层]{s.}{cimento; um tipo de material mineral em pó que pode endurecer gradualmente no ar e na água após a mistura com água}
\end{EntryWithPhonetic}

\begin{EntryWithPhonetic}{水培}{shui3pei2}{4,11}{⽔,⼟}
  \definition{v.}{cultivar plantas hidroponicamente}
\end{EntryWithPhonetic}

\begin{EntryWithPhonetic}{水平}{shui3ping2}{4,5}{⽔,⼲}[HSK 2]
  \definition{adj.}{horizontal; nivelado; paralelo à superfície da água}
  \definition{s.}{padrão; nível; o nível alcançado em determinado aspecto}
\end{EntryWithPhonetic}

\begin{EntryWithPhonetic}{水平尺}{shui3ping2chi3}{4,5,4}{⽔,⼲,⼫}
  \definition{s.}{nível espiritual}
\end{EntryWithPhonetic}

\begin{EntryWithPhonetic}{水平度}{shui3ping2 du4}{4,5,9}{⽔,⼲,⼴}
  \definition{s.}{nivelamento}
\end{EntryWithPhonetic}

\begin{EntryWithPhonetic}{水平面}{shui3ping2mian4}{4,5,9}{⽔,⼲,⾯}
  \definition{s.}{superfície nivelada; nível | plano horizontal; nível da água; superfície horizontal | nível da água}
\end{EntryWithPhonetic}

\begin{EntryWithPhonetic}{水平视差}{shui3ping2 shi4cha1}{4,5,8,9}{⽔,⼲,⾒,⼯}
  \definition{s.}{paralaxe horizontal}
\end{EntryWithPhonetic}

\begin{EntryWithPhonetic}{水平仪}{shui3ping2yi2}{4,5,5}{⽔,⼲,⼈}
  \definition{s.}{nível (dispositivo para determinar horizontal) | nível espiritual | nível de topógrafo}
\end{EntryWithPhonetic}

\begin{EntryWithPhonetic}{水平以下}{shui3ping2 yi3xia4}{4,5,4,3}{⽔,⼲,⼈,⼀}
  \definition{s.}{sub-nível}
\end{EntryWithPhonetic}

\begin{EntryWithPhonetic}{水平轴}{shui3ping2zhou2}{4,5,9}{⽔,⼲,⾞}
  \definition{s.}{eixo horizontal}
\end{EntryWithPhonetic}

\begin{EntryWithPhonetic}{水瓶}{shui3 ping2}{4,10}{⽔,⽡}
  \definition{s.}{garrada de água}
\end{EntryWithPhonetic}

\begin{EntryWithPhonetic}{水豚}{shui3tun2}{4,11}{⽔,⾗}
  \definition{s.}{capivara}
\end{EntryWithPhonetic}

\begin{EntryWithPhonetic}{水污染}{shui3wu1ran3}{4,6,9}{⽔,⽔,⽊}
  \definition{s.}{poluição da água}
\end{EntryWithPhonetic}

\begin{EntryWithPhonetic}{水灾}{shui3zai1}{4,7}{⽔,⽕}[HSK 5]
  \definition[场,次]{s.}{inundação; desastres causados por excesso de chuvas, entre outros motivos}
\end{EntryWithPhonetic}

%%%%%%%%%% 说 %%%%%%%%%%
\subsection*{说}\addcontentsline{loh}{figure}{说 \dpy{shui4}}

\begin{EntryWithPhonetic}{说}{shui4}{9}{⾔}
  \definition{v.}{persuadir}
  \seeref{shuo1}
\end{EntryWithPhonetic}

%%%%%%%%%% 税 %%%%%%%%%%
\subsection*{税}\addcontentsline{loh}{figure}{税 \dpy{shui4}}

\begin{EntryWithPhonetic}{税}{shui4}{12}{⽲}[HSK 6]
  \definition*{s.}{Sobrenome: Shui}
  \definition{s.}{imposto; taxa; tarifa}
\end{EntryWithPhonetic}

%%%%%%%%%% 睡 %%%%%%%%%%
\subsection*{睡}\addcontentsline{loh}{figure}{睡 \dpy{shui4}}

\begin{EntryWithPhonetic}{睡}{shui4}{13}{⽬}[HSK 1]
  \definition{v.}{dormir | deitar-se}
\end{EntryWithPhonetic}

\begin{EntryWithPhonetic}{睡觉}{shui4/jiao4}{13,9}{⽬,⾒}[HSK 1]
  \definition{v.+compl.}{dormir; ir para a cama; entrar em estado de sono}
\end{EntryWithPhonetic}

\begin{EntryWithPhonetic}{睡懒觉}{shui4lan3jiao4}{13,16,9}{⽬,⼼,⾒}
  \definition{v.}{levantar-se tarde | passar o tempo a dormir}
\end{EntryWithPhonetic}

\begin{EntryWithPhonetic}{睡眠}{shui4 mian2}{13,10}{⽬,⽬}[HSK 5]
  \definition{s.}{sono; \emph{somnus}; sonolência}
\end{EntryWithPhonetic}

\begin{EntryWithPhonetic}{睡衣}{shui4yi1}{13,6}{⽬,⾐}
  \definition{s.}{pijamas | roupas de dormir}
\end{EntryWithPhonetic}

\begin{EntryWithPhonetic}{睡着}{shui4 zhao2}{13,11}{⽬,⽬}[HSK 4]
  \definition{v.}{dormir; adormecer; cair no sono}
\end{EntryWithPhonetic}

%%%%%%%%%% 顺 %%%%%%%%%%
\subsection*{顺}\addcontentsline{loh}{figure}{顺 \dpy{shun4}}

\begin{EntryWithPhonetic}{顺}{shun4}{9}{⾴}[HSK 6]
  \definition{adj.}{(de escritos) legível; claro e bem escrito; organizado | favorável; harmonioso | favorável; bem-sucedido}
  \definition{prep.}{conforme a conveniência de alguém | ao longo; a introdução da rota, situação ou oportunidade que a ação segue pode ser seguida por 着 | com a corrente; na mesma direção |  com; na mesma direção que}
  \definition{v.}{organizar; colocar em ordem; tornar as coisas organizadas ou ordenadas | obedecer; ceder a; agir em submissão a | ser adequado; ser agradável}
  \seealsoref{着}{zhe5}
\end{EntryWithPhonetic}

\begin{EntryWithPhonetic}{顺便}{shun4bian4}{9,9}{⾴,⼈}
  \definition{adv.}{convenientemente | de passagem | sem muito esforço extra}
\end{EntryWithPhonetic}

\begin{EntryWithPhonetic}{顺畅}{shun4chang4}{9,8}{⾴,⽥}
  \definition{adj.}{liso e sem obstáculos | fluente}
\end{EntryWithPhonetic}

\begin{EntryWithPhonetic}{顺从}{shun4cong2}{9,4}{⾴,⼈}
  \definition{v.}{obedecer | submeter-se}
\end{EntryWithPhonetic}

\begin{EntryWithPhonetic}{顺当}{shun4dang5}{9,6}{⾴,⼹}
  \definition{adv.}{suavemente}
\end{EntryWithPhonetic}

\begin{EntryWithPhonetic}{顺耳}{shun4'er3}{9,6}{⾴,⽿}
  \definition{adj.}{agradável ao ouvido}
\end{EntryWithPhonetic}

\begin{EntryWithPhonetic}{顺境}{shun4jing4}{9,14}{⾴,⼟}
  \definition{s.}{circunstâncias fáceis (ou favoráveis) (oposto de 逆境)}
  \seealsoref{逆境}{ni4jing4}
\end{EntryWithPhonetic}

\begin{EntryWithPhonetic}{顺利}{shun4li4}{9,7}{⾴,⼑}[HSK 2]
  \definition{adj.}{sem problemas; com sucesso; sem dificuldades; sem contratempos; sem obstáculos; sem obstáculos ou dificuldades significativas no desempenho das tarefas}
\end{EntryWithPhonetic}

\begin{EntryWithPhonetic}{顺水}{shun4shui3}{9,4}{⾴,⽔}
  \definition{v.}{ir com o fluxo}
\end{EntryWithPhonetic}

\begin{EntryWithPhonetic}{顺心}{shun4xin1}{9,4}{⾴,⼼}
  \definition{adj.}{satisfatório | satisfeito}
\end{EntryWithPhonetic}

\begin{EntryWithPhonetic}{顺序}{shun4xu4}{9,7}{⾴,⼴}[HSK 4]
  \definition{adv.}{por sua vez; na ordem correta; na devida ordem; na ordem adequada; na ordem apropriada}
  \definition[个]{s.}{ordem; sequência; sucessão; subsequência; sequência simples; ordem de prioridade}
\end{EntryWithPhonetic}

\begin{EntryWithPhonetic}{顺叙}{shun4xu4}{9,9}{⾴,⼜}
  \definition{s.}{narrativa cronológica}
\end{EntryWithPhonetic}

\begin{EntryWithPhonetic}{顺延}{shun4yan2}{9,6}{⾴,⼵}
  \definition{v.}{adiar | procrastinar}
\end{EntryWithPhonetic}

\begin{EntryWithPhonetic}{顺眼}{shun4yan3}{9,11}{⾴,⽬}
  \definition{adj.}{agradável aos olhos}
\end{EntryWithPhonetic}

\begin{EntryWithPhonetic}{顺嘴}{shun4zui3}{9,16}{⾴,⼝}
  \definition{v.}{deixar escapar (sem pensar) | ler suavemente (texto) | adequar-se  ao gosto (comida)}
\end{EntryWithPhonetic}

%%%%%%%%%% 舜 %%%%%%%%%%
\subsection*{舜}\addcontentsline{loh}{figure}{舜 \dpy{shun4}}

\begin{EntryWithPhonetic}{舜}{shun4}{12}{⾇}
  \definition*{s.}{Shun, o nome de um monarca lendário da China antiga | Sobrenome: Shun}
\end{EntryWithPhonetic}

%%%%%%%%%% 说 %%%%%%%%%%
\subsection*{说}\addcontentsline{loh}{figure}{说 \dpy{shuo1}}

\begin{EntryWithPhonetic}{说}{shuo1}{9}{⾔}[HSK 1]
  \definition{s.}{uma teoria (normalmente o último caractere, como em 日心说, teoria heliocêntrica); ensinamentos; doutrina}
  \definition{v.}{falar; conversar; dizer | explicar | repreender | atuar como casamenteiro | referir-se a; indicar | criticar; aconselhar | fazer uma combinação; conciliar; mediar | discutir; falar sobre; conversar sobre | uma forma de expressão linguística da arte cênica}
  \seeref{shui4}
  \seealsoref{日心说}{ri4 xin1 shuo1}
\end{EntryWithPhonetic}

\begin{EntryWithPhonetic}{说不定}{shuo1bu5ding4}{9,4,8}{⾔,⼀,⼧}[HSK 4]
  \definition{adv.}{talvez; indica uma estimativa, possivelmente, provavelmente}
  \definition{v.}{não ter certeza; não estar certo; ser impreciso}
\end{EntryWithPhonetic}

\begin{EntryWithPhonetic}{说法}{shuo1 fa3}{9,8}{⾔,⽔}[HSK 5]
  \definition[种,个]{s.}{formulação; maneira de dizer uma coisa; formas de expressar opiniões | versão; argumento; declaração; opinião | explicação; acordo; palavras justas; razões ou fundamentos legítimos}
\end{EntryWithPhonetic}

\begin{EntryWithPhonetic}{说服}{shuo1fu2}{9,8}{⾔,⽉}[HSK 4]
  \definition{v.}{persuadir; convencer; convencer a outra parte com palavras bem fundamentadas}
\end{EntryWithPhonetic}

\begin{EntryWithPhonetic}{说好}{shuo1hao3}{9,6}{⾔,⼥}
  \definition{v.}{chegar a um acordo | concluir negociações}
\end{EntryWithPhonetic}

\begin{EntryWithPhonetic}{说话}{shuo1hua4}{9,8}{⾔,⾔}[HSK 1]
  \definition{adv.}{imediatamente; em um minuto; refere-se ao tempo que leva para falar, indicando um período muito curto}
  \definition{v.}{falar; conversar; dizer; expressar o significado através da linguagem | conversar (conversa fiada); bater papo | fofocar; conversar; criticar; censurar}
\end{EntryWithPhonetic}

\begin{EntryWithPhonetic}{说谎}{shuo1/huang3}{9,11}{⾔,⾔}
  \definition{v.+compl.}{mentir | contar uma mentira}
\end{EntryWithPhonetic}

\begin{EntryWithPhonetic}{说理}{shuo1li3}{9,11}{⾔,⽟}
  \definition{v.}{racionalizar | discutir logicamente}
\end{EntryWithPhonetic}

\begin{EntryWithPhonetic}{说明}{shuo1ming2}{9,8}{⾔,⽇}[HSK 2]
  \definition[本,个]{s.}{legenda; instrução; explicação}
  \definition{v.}{mostrar; explicar; ilustrar | indicar; mostrar; provar; demonstrar; usar materiais confiáveis para demonstrar ou determinar a autenticidade de pessoas ou coisas}
\end{EntryWithPhonetic}

\begin{EntryWithPhonetic}{说明书}{shuo1 ming2 shu1}{9,8,4}{⾔,⽇,⼄}[HSK 6]
  \definition[本]{s.}{manual; livro de instruções; descrições textuais da finalidade, especificações, desempenho e uso de itens, bem como enredos de peças e filmes, etc.}
\end{EntryWithPhonetic}

\begin{EntryWithPhonetic}{说实话}{shuo1 shi2 hua4}{9,8,8}{⾔,⼧,⾔}[HSK 6]
  \definition{v.}{falar a verdade; dizer a verdade sobre (os próprios erros ou crimes)}
\end{EntryWithPhonetic}

\begin{EntryWithPhonetic}{说完}{shuo1-wan2}{9,7}{⾔,⼧}
  \definition{expr.}{acabar/terminar palavras}
\end{EntryWithPhonetic}

%%%%%%%%%% 硕 %%%%%%%%%%
\subsection*{硕}\addcontentsline{loh}{figure}{硕 \dpy{shuo4}}

\begin{EntryWithPhonetic}{硕}{shuo4}{11}{⽯}
  \definition*{s.}{Sobrenome: Shuo}
  \definition{adj.}{grande; enorme}
  \definition{s.}{mestrado (MBA)}
\end{EntryWithPhonetic}

\begin{EntryWithPhonetic}{硕士}{shuo4shi4}{11,3}{⽯,⼠}[HSK 5]
  \definition[个,位,名]{s.}{mestrado; um diploma concedido por uma universidade ou faculdade a um aluno após um ou dois anos de estudo adicional após o bacharelado}
\end{EntryWithPhonetic}

%%%%%%%%%% 数 %%%%%%%%%%
\subsection*{数}\addcontentsline{loh}{figure}{数 \dpy{shuo4}}

\begin{EntryWithPhonetic}{数}{shuo4}{13}{⽁}
  \definition{adv.}{com frequência; repetidamente; indica uma ação frequente, equivalente a 屡次}
  \seeref{shu3}
  \seeref{shu4}
  \seealsoref{屡次}{lv3ci4}
\end{EntryWithPhonetic}

%%%%%%%%%% 丝 %%%%%%%%%%
\subsection*{丝}\addcontentsline{loh}{figure}{丝 \dpy{si1}}

\begin{EntryWithPhonetic}{丝}{si1}{5}{⼀}
  \definition{clas.}{si, uma unidade de peso (=0,0005 gramas) | usado para expressar a aparência ou expressão de uma pessoa | um décimo de milésimo de certas unidades de medida (medida de comprimento) | usado para representar coisas abstratas}
  \definition[些,种,类,跟,缕]{s.}{seda | uma coisa semelhante a um fio; itens semelhantes à seda | cordas; instrumentos de corda}
\end{EntryWithPhonetic}

%%%%%%%%%% 司 %%%%%%%%%%
\subsection*{司}\addcontentsline{loh}{figure}{司 \dpy{si1}}

\begin{EntryWithPhonetic}{司}{si1}{5}{⼝}
  \definition*{s.}{Sobrenome: Si}
  \definition{s.}{departamento (sob um ministério); um departamento dentro de uma agência de nível ministerial}
  \definition{v.}{assumir o comando de; atender; administrar; operar; gerenciar}
\end{EntryWithPhonetic}

\begin{EntryWithPhonetic}{司机}{si1ji1}{5,6}{⼝,⽊}[HSK 2]
  \definition[个,名,位]{s.}{motorista; motorista particular; chofer; motoristas de veículos de transporte público, como trens, ônibus e bondes}
\end{EntryWithPhonetic}

\begin{EntryWithPhonetic}{司长}{si1 zhang3}{5,4}{⼝,⾧}[HSK 6]
  \definition[位,名]{s.}{diretor-geral | chefe de gabinete}
\end{EntryWithPhonetic}

%%%%%%%%%% 私 %%%%%%%%%%
\subsection*{私}\addcontentsline{loh}{figure}{私 \dpy{si1}}

\begin{EntryWithPhonetic}{私}{si1}{7}{⽲}
  \definition*{s.}{Sobrenome: Si}
  \definition{adj.}{pessoal; privado (oposição a 公) | egoísta (oposto a 公) | secreto; privado | ilícito; ilegal}
  \definition{s.}{interesse privado (ou egoísta); motivo (ou ideia) egoísta (oposição a 公) | contrabando; mercadorias contrabandeadas | propriedade privada | interesses privados; ganho pessoal}
  \seealsoref{公}{gong1}
\end{EntryWithPhonetic}

\begin{EntryWithPhonetic}{私函}{si1han2}{7,8}{⽲,⼐}
  \definition{s.}{carta privada}
\end{EntryWithPhonetic}

\begin{EntryWithPhonetic}{私立}{si1li4}{7,5}{⽲,⽴}
  \definition{s.}{privado; estabelecido privadamente}[这是一所私立学校。===Esta é uma escola particular.]
  \definition{v.}{estabelecer-se ilegalmente}
\end{EntryWithPhonetic}

\begin{EntryWithPhonetic}{私人}{si1ren2}{7,2}{⽲,⼈}[HSK 5]
  \definition{adj.}{privado; pertencente a um indivíduo ou exercido a título individual; não público | pessoal; entre indivíduos}
  \definition[个]{s.}{algo privado; pessoas que se aproximam de você por motivos pessoais ou interesses próprios}
\end{EntryWithPhonetic}

\begin{EntryWithPhonetic}{私人信件}{si1ren2 xin4jian4}{7,2,9,6}{⽲,⼈,⼈,⼈}
  \definition{s.}{carta pessoal}
\end{EntryWithPhonetic}

\begin{EntryWithPhonetic}{私人钥匙}{si1ren2yao4shi5}{7,2,9,11}{⽲,⼈,⾦,⼔}
  \definition{s.}{(criptografia) chave privada}
\end{EntryWithPhonetic}

\begin{EntryWithPhonetic}{私人诊所}{si1ren2 zhen3suo3}{7,2,7,8}{⽲,⼈,⾔,⼾}
  \definition[些]{s.}{clínica privada}
\end{EntryWithPhonetic}

\begin{EntryWithPhonetic}{私生活}{si1sheng1huo2}{7,5,9}{⽲,⽣,⽔}
  \definition{s.}{vida privada}
\end{EntryWithPhonetic}

\begin{EntryWithPhonetic}{私事}{si1shi4}{7,8}{⽲,⼅}
  \definition[件,桩]{s.}{privacidade; assuntos privados; assuntos pessoais (oposto a 公事)}
  \seealsoref{公事}{gong1shi4}
\end{EntryWithPhonetic}

\begin{EntryWithPhonetic}{私自}{si1zi4}{7,6}{⽲,⾃}
  \definition{adj.}{privado | pessoal}
  \definition{adv.}{secretamente | sem aprovação explícita}
\end{EntryWithPhonetic}

%%%%%%%%%% 思 %%%%%%%%%%
\subsection*{思}\addcontentsline{loh}{figure}{思 \dpy{si1}}

\begin{EntryWithPhonetic}{思}{si1}{9}{⼼}
  \definition*{s.}{Sobrenome: Si}
  \definition{s.}{pensamento; ideias | pensamentos; emoções; humor}
  \definition{v.}{pensar; considerar; deliberar | pensar em; ansiar por}
\end{EntryWithPhonetic}

\begin{EntryWithPhonetic}{思考}{si1kao3}{9,6}{⼼,⽼}[HSK 4]
  \definition{v.}{pensar; ponderar; considerar; deliberar; envolver-se em atividades de pensamento, como análise, síntese, julgamento, raciocínio e generalização}
\end{EntryWithPhonetic}

\begin{EntryWithPhonetic}{思维}{si1wei2}{9,11}{⼼,⽷}[HSK 5]
  \definition[种]{s.}{pensamento; reflexão; organizar e transformar os materiais obtidos através do conhecimento sensorial para formar conceitos, julgamentos e raciocínios}
  \definition{v.}{pensar}
\end{EntryWithPhonetic}

\begin{EntryWithPhonetic}{思想}{si1xiang3}{9,13}{⼼,⼼}[HSK 3]
  \definition[个,种]{s.}{reflexão; pensamento; ideologia; a existência objetiva é refletida na consciência das pessoas por meio de atividades de pensamento, que pertencem à cognição racional | ideia; pensamento}
\end{EntryWithPhonetic}

%%%%%%%%%% 斯 %%%%%%%%%%
\subsection*{斯}\addcontentsline{loh}{figure}{斯 \dpy{si1}}

\begin{EntryWithPhonetic}{斯}{si1}{12}{⽄}
  \definition*{s.}{Sobrenome: Si}
  \definition{adv.}{então; assim}
  \definition{pron.}{isto; aqui}
\end{EntryWithPhonetic}

\begin{EntryWithPhonetic}{斯巴达}{si1ba1da2}{12,4,6}{⽄,⼰,⾡}
  \definition*{s.}{Esparta}
\end{EntryWithPhonetic}

%%%%%%%%%% 死 %%%%%%%%%%
\subsection*{死}\addcontentsline{loh}{figure}{死 \dpy{si3}}

\begin{EntryWithPhonetic}{死}{si3}{6}{⽍}[HSK 3]
  \definition{adj.}{até a morte | implacável; mortal | fixo; rígido; inflexível | intransitável; fechado | (expressando raiva, reclamação, etc., às vezes jocosamente) maldito}
  \definition{adv.}{(frequentemente no negativo) teimosamente; inflexivelmente}
  \definition{v.}{morrer; estar morto (oposto a 生 e 活)}
  \seealsoref{活}{huo2}
  \seealsoref{生}{sheng1}
\end{EntryWithPhonetic}

\begin{EntryWithPhonetic}{死亡}{si3wang2}{6,3}{⽍,⼇}[HSK 6]
  \definition{s.}{morte; condenação; dar o último suspiro; refere-se ao estado de vida desaparecendo}
  \definition{v.}{morrer; estar morto; perder a vida (em oposição à 生存)}
  \seealsoref{生存}{sheng1cun2}
\end{EntryWithPhonetic}

%%%%%%%%%% 四 %%%%%%%%%%
\subsection*{四}\addcontentsline{loh}{figure}{四 \dpy{si4}}

\begin{EntryWithPhonetic}{四}{si4}{5}{⼞}[HSK 1]
  \definition*{s.}{Sobrenome: Si}
  \definition{num.}{quatro; 4}
  \definition{s.}{uma nota da escala em Gongchepu (工尺谱), correspondente ao 6 na notação musical numerada}
  \seealsoref{工尺谱}{gong1 che3 pu3}
\end{EntryWithPhonetic}

\begin{EntryWithPhonetic}{四处}{si4 chu4}{5,5}{⼞,⼡}[HSK 6]
  \definition{adv.}{em volta; ao redor; em todos os lugares; em todas as direções}
\end{EntryWithPhonetic}

\begin{EntryWithPhonetic}{四川}{si4chuan1}{5,3}{⼞,⼮}
  \definition*{s.}{Província de Sichuan}
\end{EntryWithPhonetic}

\begin{EntryWithPhonetic}{四季分明}{si4ji4-fen1ming2}{5,8,4,8}{⼞,⼦,⼑,⽇}
  \definition{expr.}{as quatro estações são muito distintas}
\end{EntryWithPhonetic}

\begin{EntryWithPhonetic}{四季如春}{si4ji4-ru2chun1}{5,8,6,9}{⼞,⼦,⼥,⽇}
  \definition{expr.}{é primavera todo o ano | clima favorável durante todo o ano | quatro estações como a primavera}
\end{EntryWithPhonetic}

\begin{EntryWithPhonetic}{四周}{si4 zhou1}{5,8}{⼞,⼝}[HSK 5]
  \definition{s.}{ao redor; por todos os lados; a parte que circunda o centro}
\end{EntryWithPhonetic}

%%%%%%%%%% 似 %%%%%%%%%%
\subsection*{似}\addcontentsline{loh}{figure}{似 \dpy{si4}}

\begin{EntryWithPhonetic}{似}{si4}{6}{⼈}
  \definition*{s.}{Sobrenome: Si}
  \definition{adv.}{parece; como se}
  \definition{v.}{ser semelhante; parecer-se com | parecer; aparecer | exceder}
  \seeref{shi4}
\end{EntryWithPhonetic}

\begin{EntryWithPhonetic}{似曾相识}{si4ceng2xiang1shi2}{6,12,9,7}{⼈,⽈,⽬,⾔}
  \definition{s.}{\emph{déjà vu} (a experiência de ver exatamente a mesma situação pela segunda vez) | situação aparentemente familiar}
\end{EntryWithPhonetic}

\begin{EntryWithPhonetic}{似乎}{si4hu1}{6,5}{⼈,⼃}[HSK 4]
  \definition{adv.}{como se; aparentemente; se parece como}
\end{EntryWithPhonetic}

%%%%%%%%%% 寺 %%%%%%%%%%
\subsection*{寺}\addcontentsline{loh}{figure}{寺 \dpy{si4}}

\begin{EntryWithPhonetic}{寺}{si4}{6}{⼨}[HSK 6]
  \definition*{s.}{Sobrenome: Si}
  \definition[座]{s.}{templo | (Islã) mesquita | Obsoleto: ministério; agência governamental na China antiga}
\end{EntryWithPhonetic}

\begin{EntryWithPhonetic}{寺庙}{si4miao4}{6,8}{⼨,⼴}
  \definition[座]{s.}{mosteiro; casa de Deus; templo; templos budistas}
\end{EntryWithPhonetic}

%%%%%%%%%% 伺 %%%%%%%%%%
\subsection*{伺}\addcontentsline{loh}{figure}{伺 \dpy{si4}}

\begin{EntryWithPhonetic}{伺}{si4}{7}{⼈}
  \definition{v.}{aguardar; observar; esperar por}
  \seeref{ci4}
\end{EntryWithPhonetic}

%%%%%%%%%% 食 %%%%%%%%%%
\subsection*{食}\addcontentsline{loh}{figure}{食 \dpy{si4}}

\begin{EntryWithPhonetic}{食}{si4}{9}{⾷}[Kangxi 184]
  \definition{v.}{alimentar; dar comida a}
  \seeref{shi2}
\end{EntryWithPhonetic}

%%%%%%%%%% 肆 %%%%%%%%%%
\subsection*{肆}\addcontentsline{loh}{figure}{肆 \dpy{si4}}

\begin{EntryWithPhonetic}{肆}{si4}{13}{⾀}
  \definition*{s.}{Sobrenome: Si}
  \definition{adj.}{desenfreado; sem limites; descuidado; imprudente}
  \definition{num.}{quatro (usado para o numeral 四 em cheques, etc., para evitar erros ou alterações)}
  \definition{s.}{Literário: loja; armazém}
  \seealsoref{四}{si4}
\end{EntryWithPhonetic}

%%%%%%%%%% 厕 %%%%%%%%%%
\subsection*{厕}\addcontentsline{loh}{figure}{厕 \dpy{si5}}

\begin{EntryWithPhonetic}{厕}{si5}{8}{⼚}
  \definition{s.}{componente formador de palavras | latrina; fossa sanitária}
  \seeref{ce4}
  \seealsoref{茅厕}{mao2ce4}
\end{EntryWithPhonetic}

%%%%%%%%%% 松 %%%%%%%%%%
\subsection*{松}\addcontentsline{loh}{figure}{松 \dpy{song1}}

\begin{EntryWithPhonetic}{松}{song1}{8}{⽊}[HSK 4]
  \definition*{s.}{Sobrenome: Song}
  \definition{adj.}{solto; frouxo; folgado | leve e crocante; macio | relaxado; confortável}
  \definition[棵]{s.}{pinheiro | fio de carne seca; carne moída seca; alimentos macios ou quebradiços}
  \definition{v.}{afrouxar; relaxar; abrandar | desamarrar; desatar; liberar}
\end{EntryWithPhonetic}

\begin{EntryWithPhonetic}{松木}{song1mu4}{8,4}{⽊,⽊}
  \definition{s.}{pinheiro}
\end{EntryWithPhonetic}

\begin{EntryWithPhonetic}{松树}{song1 shu4}{8,9}{⽊,⽊}[HSK 4]
  \definition[棵]{s.}{pinheiro; conífera comum, geralmente com folhas longas e pontiagudas e cones lenhosos}
\end{EntryWithPhonetic}

%%%%%%%%%% 嵩 %%%%%%%%%%
\subsection*{嵩}\addcontentsline{loh}{figure}{嵩 \dpy{song1}}

\begin{EntryWithPhonetic}{嵩}{song1}{13}{⼭}
  \definition*{s.}{Sobrenome: Song}
  \definition{adj.}{Arcaico: (montanhas) alto; elevado}
\end{EntryWithPhonetic}

\begin{EntryWithPhonetic}{嵩山}{song1shan1}{13,3}{⼭,⼭}
  \definition*{s.}{Monte Song em Henan, montanha central das Cinco Montanhas Sagradas (五岳)}
  \seealsoref{五岳}{wu3yue4}
\end{EntryWithPhonetic}

%%%%%%%%%% 宋 %%%%%%%%%%
\subsection*{宋}\addcontentsline{loh}{figure}{宋 \dpy{song4}}

\begin{EntryWithPhonetic}{宋}{song4}{7}{⼧}
  \definition*{s.}{Dinastia Song (960-1279) | Song das dinastias do sul (420-479) | Sobrenome: Song}
  \definition{clas.}{sone; unidade de intensidade sonora}
\end{EntryWithPhonetic}

%%%%%%%%%% 送 %%%%%%%%%%
\subsection*{送}\addcontentsline{loh}{figure}{送 \dpy{song4}}

\begin{EntryWithPhonetic}{送}{song4}{9}{⾡}[HSK 1]
  \definition*{s.}{Sobrenome: Song}
  \definition{v.}{transportar; entregar | dar; dar como presente; presentear | acompanhar; despedir-se de alguém (ao sair); acompanhar a pessoa que está partindo até o destino ou caminhar um trecho com ela | escoltar}
\end{EntryWithPhonetic}

\begin{EntryWithPhonetic}{送到}{song4 dao4}{9,8}{⾡,⼑}[HSK 2]
  \definition{v.}{enviar para (lugar)}
\end{EntryWithPhonetic}

\begin{EntryWithPhonetic}{送给}{song4 gei3}{9,9}{⾡,⽷}[HSK 2]
  \definition{v.}{dar a (alguém ou organização); dar como algo gratuito; dar como presente}
\end{EntryWithPhonetic}

\begin{EntryWithPhonetic}{送礼}{song4 li3}{9,5}{⾡,⽰}[HSK 6]
  \definition{v.}{dar um presente a alguém; presentear alguém com um presente | enviar presentes (para obter favores) | dar um presente; enviar um presente}
\end{EntryWithPhonetic}

\begin{EntryWithPhonetic}{送行}{song4 xing2}{9,6}{⾡,⾏}[HSK 6]
  \definition{v.}{ver alguém partir; ir até o local onde o viajante iniciou sua jornada, despedir-se dele e observar ele partir | dar uma festa de despedida; realizar uma festa de despedida | despedir-se do falecido}
\end{EntryWithPhonetic}

%%%%%%%%%% 㮸 %%%%%%%%%%
\subsection*{㮸}\addcontentsline{loh}{figure}{㮸 \dpy{song4}}

\begin{EntryWithPhonetic}{㮸}{song4}{14}{⽊}
  \variantof{送}
\end{EntryWithPhonetic}

%%%%%%%%%% 搜 %%%%%%%%%%
\subsection*{搜}\addcontentsline{loh}{figure}{搜 \dpy{sou1}}

\begin{EntryWithPhonetic}{搜}{sou1}{12}{⼿}[HSK 5]
  \definition{v.}{procurar | pesquisar | coletar; reunir | procurar ou revistar um lugar de forma completa e desordenada}
\end{EntryWithPhonetic}

\begin{EntryWithPhonetic}{搜索}{sou1suo3}{12,10}{⼿,⽷}[HSK 5]
  \definition{v.}{procurar; caçar; explorar; pesquisar cuidadosamente; refere-se especificamente à busca militar para identificar situações suspeitas em determinada região, área marítima ou aérea}
\end{EntryWithPhonetic}

%%%%%%%%%% 苏 %%%%%%%%%%
\subsection*{苏}\addcontentsline{loh}{figure}{苏 \dpy{su1}}

\begin{EntryWithPhonetic}{苏}{su1}{7}{⾋}
  \definition*{s.}{Suzhou, abreviação de 苏州 | Província de Jiangsu, abreviação de 江苏 | União Soviética, abreviação de 苏联 | Sobrenome: Su}
  \definition{s.}{perilla planta da família das mentas}
  \definition{v.}{reviver; vir a; acordar}
  \seealsoref{江苏}{jiang1su1}
  \seealsoref{苏联}{su1lian2}
  \seealsoref{苏州}{su1zhou1}
\end{EntryWithPhonetic}

\begin{EntryWithPhonetic}{苏格兰}{su1ge2lan2}{7,10,5}{⾋,⽊,⼋}
  \definition*{s.}{Escócia}
\end{EntryWithPhonetic}

\begin{EntryWithPhonetic}{苏联}{su1lian2}{7,12}{⾋,⽿}
  \definition*{s.}{União das Repúblicas Socialistas Soviéticas (1922-1991)}
\end{EntryWithPhonetic}

\begin{EntryWithPhonetic}{苏州}{su1zhou1}{7,6}{⾋,⼮}
  \definition*{s.}{Suzhou, cidade na Província de Jiangsu}
\end{EntryWithPhonetic}

%%%%%%%%%% 素 %%%%%%%%%%
\subsection*{素}\addcontentsline{loh}{figure}{素 \dpy{su4}}

\begin{EntryWithPhonetic}{素}{su4}{10}{⽷}
  \definition{adj.}{branco; de cor natural | simples; natural; singelo; de cor simples | nativo; original | normal; usual; geral}
  \definition{adv.}{geralmente; sempre; habitualmente}
  \definition{s.}{vegetais, frutas e outros alimentos (em oposição à 荤) | matéria-prima; matéria-prima básico; tecidos de seda naturais e não processados | elemento; os componentes básicos de algo}
  \seealsoref{荤}{hun1}
\end{EntryWithPhonetic}

\begin{EntryWithPhonetic}{素质}{su4zhi4}{10,8}{⽷,⾙}[HSK 6]
  \definition[个,种]{s.}{qualidade; características; caráter; o nível físico, moral, mental, intelectual e cultural de uma pessoa}
\end{EntryWithPhonetic}

%%%%%%%%%% 速 %%%%%%%%%%
\subsection*{速}\addcontentsline{loh}{figure}{速 \dpy{su4}}

\begin{EntryWithPhonetic}{速}{su4}{10}{⾡}
  \definition{adj.}{rápido; veloz}
  \definition{s.}{velocidade}
  \definition{v.aux.}{convidar}
\end{EntryWithPhonetic}

\begin{EntryWithPhonetic}{速度}{su4du4}{10,9}{⾡,⼴}[HSK 3]
  \definition[个,种]{s.}{velocidade; taxa; ritmo; andamento; uma quantidade física que indica a velocidade e a direção do movimento de um objeto, ou seja, a distância que um objeto percorre em uma direção por unidade de tempo | velocidade; rapidez; geralmente se refere ao grau de velocidade}
\end{EntryWithPhonetic}

%%%%%%%%%% 宿 %%%%%%%%%%
\subsection*{宿}\addcontentsline{loh}{figure}{宿 \dpy{su4}}

\begin{EntryWithPhonetic}{宿}{su4}{11}{⼧}
  \definition*{s.}{Sobrenome: Su}
  \definition{adj.}{de longa data; antigo; velho | veterano; velho; experiente}
  \definition{v.}{hospedar-se para passar a noite; passar a noite}
  \seeref{xiu3}
  \seeref{xiu4}
\end{EntryWithPhonetic}

\begin{EntryWithPhonetic}{宿舍}{su4she4}{11,8}{⼧,⾆}[HSK 5]
  \definition[间,幢]{s.}{alojamento; dormitório; república; albergue; casas onde escolas, empresas, etc. acomodam seus alunos ou funcionários}
\end{EntryWithPhonetic}

%%%%%%%%%% 塑 %%%%%%%%%%
\subsection*{塑}\addcontentsline{loh}{figure}{塑 \dpy{su4}}

\begin{EntryWithPhonetic}{塑}{su4}{13}{⼟}
  \definition{s.}{plástico; material plástico}
  \definition{v.}{modelo; molde; forma}
\end{EntryWithPhonetic}

\begin{EntryWithPhonetic}{塑料}{su4 liao4}{13,10}{⼟,⽃}[HSK 4]
  \definition[块,种]{s.}{plástico; compostos de polímeros feitos de resinas naturais ou sintéticas como componente principal}
\end{EntryWithPhonetic}

\begin{EntryWithPhonetic}{塑料袋}{su4liao4dai4}{13,10,11}{⼟,⽃,⾐}[HSK 4]
  \definition[个,只]{s.}{saco plástico; sacola de plástico}
\end{EntryWithPhonetic}

%%%%%%%%%% 痠 %%%%%%%%%%
\subsection*{痠}\addcontentsline{loh}{figure}{痠 \dpy{suan1}}

\begin{EntryWithPhonetic}{痠}{suan1}{12}{⽧}
  \definition{v.}{doer | estar dolorido}
  \variantof{酸}
\end{EntryWithPhonetic}

%%%%%%%%%% 酸 %%%%%%%%%%
\subsection*{酸}\addcontentsline{loh}{figure}{酸 \dpy{suan1}}

\begin{EntryWithPhonetic}{酸}{suan1}{14}{⾣}[HSK 4]
  \definition{adj.}{azedo; ácido | aflito; angustiado; doente do coração | pedante; descreve uma pessoa que finge ser culta e também descreve uma pessoa que é muito inflexível com suas próprias ideias e não está disposta a mudá-las para atender às exigências da época, é usado principalmente para satirizar intelectuais que fingem ser capazes de escrever poemas e artigos | ciumento; invejoso; sentimentos desconfortáveis porque outra pessoa é melhor do que você e, em geral, também apresenta comportamento hostil}
  \definition{s.}{ácido; produto químico que tem um sabor ácido quando misturado com água}
  \definition{v.}{estar dolorido (devido à fadiga ou doença); descreve a sensação de não ter força muscular e um pouco de dor por estar doente ou muito cansado}
\end{EntryWithPhonetic}

\begin{EntryWithPhonetic}{酸辣汤}{suan1la4tang1}{14,14,6}{⾣,⾟,⽔}
  \definition{s.}{sopa avinagrada e picante (prato)}
\end{EntryWithPhonetic}

\begin{EntryWithPhonetic}{酸奶}{suan1 nai3}{14,5}{⾣,⼥}[HSK 4]
  \definition[瓶,杯,盒,袋]{s.}{iogurte; produto lácteo fermentado por bactérias de ácido láctico}
\end{EntryWithPhonetic}

\begin{EntryWithPhonetic}{酸甜苦辣}{suan1 tian2 ku3 la4}{14,11,8,14}{⾣,⽢,⾋,⾟}[HSK 5]
  \definition{expr.}{os altos e baixos da vida; as experiências agridoces da vida; os aspectos doces, azedos, amargos e picantes da vida; refere-se a todos os tipos de sabores, como metáfora para experiências diversas, como felicidade, sofrimento, etc.; azedo, doce, amargo, picante (alegrias e tristezas da vida)}
\end{EntryWithPhonetic}

%%%%%%%%%% 算 %%%%%%%%%%
\subsection*{算}\addcontentsline{loh}{figure}{算 \dpy{suan4}}

\begin{EntryWithPhonetic}{算}{suan4}{14}{⽵}[HSK 2]
  \definition{adv.}{finalmente; por fim; no final; significa que, após um longo período de tempo ou muitas dificuldades, finalmente se alcançou o objetivo, equivalente a 总算}
  \definition{v.}{calcular; estimar; computar | contar; incluir | planejar; calcular; projetar | pensar; supor; especular | considerar; considerar como; contar como; reconhecer como | (aritmética) contar; ter peso | deixe estar; deixe passar; seguido por 了: desistir, não se importar mais}
  \seealsoref{了}{le5}
  \seealsoref{总算}{zong3suan4}
\end{EntryWithPhonetic}

\begin{EntryWithPhonetic}{算了}{suan4 le5}{14,2}{⽵,⼅}[HSK 6]
  \definition{part.}{deixe estar; deixe passar; usado no final de uma frase para expressar imperativo, término, etc.}
  \definition{v.}{deixar; deixe estar; deixe passar; esquecer isso; não querer continuar; é usado para persuadir os outros ou para expressar que posso aceitar a situação atual, para encerrar o assunto ou assunto atual, ou para dizer ``esqueça''}
\end{EntryWithPhonetic}

\begin{EntryWithPhonetic}{算命}{suan4ming4}{14,8}{⽵,⼝}
  \definition{s.}{cartomante}
  \definition{v.}{ler a sorte | fazer advinhações}
\end{EntryWithPhonetic}

\begin{EntryWithPhonetic}{算是}{suan4 shi4}{14,9}{⽵,⽇}[HSK 6]
  \definition{adv.}{finalmente; por fim; depois de muito tempo, o objetivo foi finalmente alcançado}
  \definition{v.}{contar como; pensar que; ser considerado}
\end{EntryWithPhonetic}

%%%%%%%%%% 尿 %%%%%%%%%%
\subsection*{尿}\addcontentsline{loh}{figure}{尿 \dpy{sui1}}

\begin{EntryWithPhonetic}{尿}{sui1}{7}{⼫}
  \definition[泡]{s.}{urina}
  \definition{v.}{urinar}
  \seeref{niao4}
\end{EntryWithPhonetic}

%%%%%%%%%% 虽 %%%%%%%%%%
\subsection*{虽}\addcontentsline{loh}{figure}{虽 \dpy{sui1}}

\begin{EntryWithPhonetic}{虽}{sui1}{9}{⾍}[HSK 6]
  \definition{conj.}{no entanto; embora | mesmo se}
\end{EntryWithPhonetic}

\begin{EntryWithPhonetic}{虽然}{sui1 ran2}{9,12}{⾍,⽕}[HSK 2]
  \definition{conj.}{apesar de; embora (frequentemente usado correlativamente com 可是, 但是, etc); geralmente é usado no início de uma frase para indicar que o fato anterior foi reconhecido, mas não mudará o que acontecerá em seguida}
  \seealsoref{但是}{dan4 shi4}
  \seealsoref{可是}{ke3shi4}
\end{EntryWithPhonetic}

%%%%%%%%%% 随 %%%%%%%%%%
\subsection*{随}\addcontentsline{loh}{figure}{随 \dpy{sui2}}

\begin{EntryWithPhonetic}{随}{sui2}{11}{⾩}[HSK 3]
  \definition*{s.}{Sobrenome: Sui}
  \definition{adv.}{fazer algo imediatamente assim que ocorre, sem demora ou hesitação; usado antes de dois verbos ou frases verbais para indicar que a última ação segue a anterior}
  \definition{prep.}{junto com (alguma outra ação) | apresentando as condições das quais a ação depende}
  \definition{v.}{seguir; vir (ou ir) junto com | concordar com; adaptar-se a | deixar (alguém fazer o que quiser) | (dialeto) parecer-se com; assemelhar-se a | seguir ou agir de acordo com a condição ou circunstância da qual a ação depende}
\end{EntryWithPhonetic}

\begin{EntryWithPhonetic}{随便}{sui2bian4}{11,9}{⾩,⼈}[HSK 2]
  \definition{adj.}{relaxado; descontraído; sem restrições; sem limitações | aleatório; casual; descuidado; indiferente; distraído, não pensa bem antes de falar ou agir | casual; informal; não dá importância aos detalhes}
  \definition{conj.}{qualquer; qualquer que seja; não importa}
  \definition{v.}{deixar alguém à vontade}
\end{EntryWithPhonetic}

\begin{EntryWithPhonetic}{随处}{sui2chu4}{11,5}{⾩,⼡}
  \definition{adv.}{em qualquer lugar}
\end{EntryWithPhonetic}

\begin{EntryWithPhonetic}{随地}{sui2di4}{11,6}{⾩,⼟}
  \definition{adv.}{qualquer lugar | todo lugar}
\end{EntryWithPhonetic}

\begin{EntryWithPhonetic}{随后}{sui2 hou4}{11,6}{⾩,⼝}[HSK 5]
  \definition{adv.}{logo em seguida; logo depois; indica que segue imediatamente após a ação ou situação anterior (geralmente usado em conjunto com 就)}
  \seealsoref{就}{jiu4}
\end{EntryWithPhonetic}

\begin{EntryWithPhonetic}{随机存取存储器}{sui2ji1cun2qu3cun2chu3qi4}{11,6,6,8,6,12,16}{⾩,⽊,⼦,⼜,⼦,⼈,⼝}
  \definition{s.}{RAM (\emph{random access memory})}
  \seealsoref{内存}{nei4cun2}
  \seealsoref{随机存取记忆体}{sui2ji1cun2qu3ji4yi4ti3}
\end{EntryWithPhonetic}

\begin{EntryWithPhonetic}{随机存取记忆体}{sui2ji1cun2qu3ji4yi4ti3}{11,6,6,8,5,4,7}{⾩,⽊,⼦,⼜,⾔,⼼,⼈}
  \definition{s.}{RAM (\emph{random access memory})}
  \seealsoref{内存}{nei4cun2}
  \seealsoref{随机存取存储器}{sui2ji1cun2qu3cun2chu3qi4}
\end{EntryWithPhonetic}

\begin{EntryWithPhonetic}{随时}{sui2shi2}{11,7}{⾩,⽇}[HSK 2]
  \definition{adv.}{a qualquer momento; em todos os momentos}
\end{EntryWithPhonetic}

\begin{EntryWithPhonetic}{随手}{sui2shou3}{11,4}{⾩,⼿}[HSK 4]
  \definition{adv.}{convenientemente; sem problemas adicionais; casualmente}
\end{EntryWithPhonetic}

\begin{EntryWithPhonetic}{随意}{sui2yi4}{11,13}{⾩,⼼}[HSK 5]
  \definition{adj.}{aleatório; casual; à vontade; como se deseja}
\end{EntryWithPhonetic}

\begin{EntryWithPhonetic}{随着}{sui2zhe5}{11,11}{⾩,⽬}[HSK 5]
  \definition{prep.}{junto com; na esteira de; em sintonia com; usado no início da frase ou antes do verbo, indica as condições necessárias para que uma ação, comportamento ou evento ocorra}
\end{EntryWithPhonetic}

%%%%%%%%%% 岁 %%%%%%%%%%
\subsection*{岁}\addcontentsline{loh}{figure}{岁 \dpy{sui4}}

\begin{EntryWithPhonetic}{岁}{sui4}{6}{⼭}[HSK 1]
  \definition{clas.}{usado para anos (de idade)}
  \definition{s.}{ano (literário) | colheita do ano (literário) | idade | tempo (literário) | ano (de idade) | ano (para as colheitas)}
\end{EntryWithPhonetic}

\begin{EntryWithPhonetic}{岁数}{sui4 shu4}{6,13}{⼭,⽁}[HSK 6]
  \definition{s.}{idade; anos; a idade de uma pessoa}
\end{EntryWithPhonetic}

\begin{EntryWithPhonetic}{岁月}{sui4yue4}{6,4}{⼭,⽉}[HSK 5]
  \definition[段,番]{s.}{anos; ano e mês; refere-se a tempo em geral}
\end{EntryWithPhonetic}

%%%%%%%%%% 碎 %%%%%%%%%%
\subsection*{碎}\addcontentsline{loh}{figure}{碎 \dpy{sui4}}

\begin{EntryWithPhonetic}{碎}{sui4}{13}{⽯}[HSK 5]
  \definition*{s.}{Sobrenome: Sui}
  \definition{adj.}{quebrado; fragmentado | tagarela; falante}
  \definition{v.}{quebrar em pedaços; esmagar}
\end{EntryWithPhonetic}

%%%%%%%%%% 隧 %%%%%%%%%%
\subsection*{隧}\addcontentsline{loh}{figure}{隧 \dpy{sui4}}

\begin{EntryWithPhonetic}{隧}{sui4}{14}{⾩}
  \definition{s.}{túnel; passagem subterrânea | estrada | subúrbios; áreas suburbanas}
  \definition{v.}{virar}
\end{EntryWithPhonetic}

\begin{EntryWithPhonetic}{隧道}{sui4dao4}{14,12}{⾩,⾡}
  \definition{s.}{túnel}
\end{EntryWithPhonetic}

%%%%%%%%%% 孙 %%%%%%%%%%
\subsection*{孙}\addcontentsline{loh}{figure}{孙 \dpy{sun1}}

\begin{EntryWithPhonetic}{孙}{sun1}{6}{⼦}
  \definition*{s.}{Sobrenome: Sun}
  \definition{s.}{neto; neta | gerações abaixo da do neto | parentes pertencentes à geração do neto | segundo crescimento das plantas}
\end{EntryWithPhonetic}

\begin{EntryWithPhonetic}{孙女}{sun1nv3}{6,3}{⼦,⼥}[HSK 4]
  \definition[个]{s.}{filha do filho; neta}
\end{EntryWithPhonetic}

\begin{EntryWithPhonetic}{孙武}{sun1wu3}{6,8}{⼦,⽌}
  \definition*{s.}{Sun Wu, também conhecido por Sun Tzu (孙子) general, estrategista e filósofo autor do ``Arte da Guerra''《孙子兵法》}
  \seealsoref{孙子}{sun1zi3}
  \seealsoref{孙子兵法}{sun1zi3 bing1fa3}
\end{EntryWithPhonetic}

\begin{EntryWithPhonetic}{孙子}{sun1zi3}{6,3}{⼦,⼦}
  \definition*{s.}{Sun Tzu, também conhecido por Sun Wu (孙武), general, estrategista e filósofo autor do ``Arte da Guerra''《孙子兵法》}
  \seeref{sun1zi5}
  \seealsoref{孙武}{sun1wu3}
  \seealsoref{孙子兵法}{sun1zi3 bing1fa3}
\end{EntryWithPhonetic}

\begin{EntryWithPhonetic}{孙子兵法}{sun1zi3 bing1fa3}{6,3,7,8}{⼦,⼦,⼋,⽔}
  \definition*{s.}{``Arte da Guerra'', o antigo clássico chinês sobre estratégia militar, escrito por Sun Tzu (孫子)}
  \seealsoref{孙武}{sun1wu3}
  \seealsoref{孙子}{sun1zi3}
\end{EntryWithPhonetic}

\begin{EntryWithPhonetic}{孙子}{sun1zi5}{6,3}{⼦,⼦}[HSK 4]
  \definition[个]{s.}{filho do filho; neto}
  \seeref{sun1zi3}
\end{EntryWithPhonetic}

%%%%%%%%%% 损 %%%%%%%%%%
\subsection*{损}\addcontentsline{loh}{figure}{损 \dpy{sun3}}

\begin{EntryWithPhonetic}{损}{sun3}{10}{⼿}
  \definition{adj.}{sarcástico; cortante; de ​​língua afiada; maldoso; mau; cruel}
  \definition{v.}{diminuir; perder; reduzir | prejudicar; danificar; degradar; destruir; arruinar; destruir o estado original ou fazê-lo perder sua eficácia original | ser sarcástico; ser cáustico; ser cortante; ferir; insultar; usar palavras duras para zombar de alguém}
\end{EntryWithPhonetic}

\begin{EntryWithPhonetic}{损害}{sun3 hai4}{10,10}{⼿,⼧}[HSK 5]
  \definition{v.}{prejudicar; danificar; ferir; causar danos; causar perdas}
\end{EntryWithPhonetic}

\begin{EntryWithPhonetic}{损失}{sun3shi1}{10,5}{⼿,⼤}[HSK 5]
  \definition{s.}{perda; desperdício; algo que se consome ou se perde sem custo algum}
  \definition{v.}{perder; consumir ou perder}
\end{EntryWithPhonetic}

%%%%%%%%%% 笋 %%%%%%%%%%
\subsection*{笋}\addcontentsline{loh}{figure}{笋 \dpy{sun3}}

\begin{EntryWithPhonetic}{笋}{sun3}{10}{⽵}
  \definition{s.}{broto de bambu}
\end{EntryWithPhonetic}

%%%%%%%%%% 莎 %%%%%%%%%%
\subsection*{莎}\addcontentsline{loh}{figure}{莎 \dpy{suo1}}

\begin{EntryWithPhonetic}{莎}{suo1}{10}{⾋}
  \seeref{sha1}
\end{EntryWithPhonetic}

%%%%%%%%%% 缩 %%%%%%%%%%
\subsection*{缩}\addcontentsline{loh}{figure}{缩 \dpy{suo1}}

\begin{EntryWithPhonetic}{缩}{suo1}{14}{⽷}
  \definition*{s.}{Sobrenome: Suo}
  \definition{v.}{contrair; encolher | recuar; retirar-se | economizar}
\end{EntryWithPhonetic}

\begin{EntryWithPhonetic}{缩短}{suo1duan3}{14,12}{⽷,⽮}[HSK 4]
  \definition{v.}{encurtar; reduzir; diminuir}
\end{EntryWithPhonetic}

\begin{EntryWithPhonetic}{缩小}{suo1 xiao3}{14,3}{⽷,⼩}[HSK 4]
  \definition{v.}{reduzir, estreitar, encolher;  tornar menor (em oposição a 放大)}
  \seealsoref{放大}{fang4da4}
\end{EntryWithPhonetic}

\begin{EntryWithPhonetic}{缩影卡片}{suo1ying3 ka3pian4}{14,15,5,4}{⽷,⼺,⼘,⽚}
  \definition{s.}{cartão em miniatura; microcartão}
\end{EntryWithPhonetic}

%%%%%%%%%% 所 %%%%%%%%%%
\subsection*{所}\addcontentsline{loh}{figure}{所 \dpy{suo3}}

\begin{EntryWithPhonetic}{所}{suo3}{8}{⼾}[HSK 3,6]
  \definition*{s.}{Sobrenome: Suo}
  \definition{clas.}{usado para casas, etc.}
  \definition{part.}{usado com 为 ou 被 para indicar voz passiva | usado antes do verbo para formar um substantivo ou para qualificar um substantivo | usado antes do verbo na estrutura sujeito-predicado usada como complemento, indica que o termo central é o objeto}
  \definition{s.}{lugar | usado como nome de órgãos governamentais ou outros locais de trabalho}
  \seealsoref{被}{bei4}
  \seealsoref{为}{wei4}
\end{EntryWithPhonetic}

\begin{EntryWithPhonetic}{所长}{suo3 chang2}{8,4}{⼾,⾧}
  \definition{s.}{aquilo em que alguém é bom; o ponto forte de alguém; o forte de alguém}
  \seeref{suo3 zhang3}
\end{EntryWithPhonetic}

\begin{EntryWithPhonetic}{所以}{suo3 yi3}{8,4}{⼾,⼈}[HSK 2]
  \definition{conj.}{assim; portanto; como resultado; conecta frases, expressa resultados e costuma corresponder a expressões como 因为 e 由于}
  \definition[个]{s.}{motivo real; causa real; comportamento adequado}
  \seealsoref{因为}{yin1wei4}
  \seealsoref{由于}{you2yu2}
\end{EntryWithPhonetic}

\begin{EntryWithPhonetic}{所有}{suo3you3}{8,6}{⼾,⽉}[HSK 2]
  \definition{adj.}{todo | tudo}
  \definition{adj.}{tudo}
  \definition{s.}{bens; posses;}
  \definition{v.}{possuir; ter}
\end{EntryWithPhonetic}

\begin{EntryWithPhonetic}{所在}{suo3 zai4}{8,6}{⼾,⼟}[HSK 5]
  \definition[个]{s.}{lugar; local; localização | o lugar onde alguém ou algo está}
\end{EntryWithPhonetic}

\begin{EntryWithPhonetic}{所长}{suo3 zhang3}{8,4}{⼾,⾧}[HSK 3]
  \definition{s.}{chefe de um instituto, etc. | superintendente}
  \seeref{suo3 chang2}
\end{EntryWithPhonetic}

%%%%%%%%%% 索 %%%%%%%%%%
\subsection*{索}\addcontentsline{loh}{figure}{索 \dpy{suo3}}

\begin{EntryWithPhonetic}{索}{suo3}{10}{⽷}
  \definition*{s.}{Sobrenome: Suo}
  \definition{adj.}{completamente sozinho; sozinho | maçante; insípido; sem significado}
  \definition[根]{s.}{corda; cabo; cordão; corrente | uma corda grande}
  \definition{v.}{(literário) pesquisar | exigir; pedir}
\end{EntryWithPhonetic}

\begin{EntryWithPhonetic}{索性}{suo3xing4}{10,8}{⽷,⼼}
  \definition{adv.}{poderia muito bem | simplesmente | apenas}
\end{EntryWithPhonetic}

%%%%%%%%%% 锁 %%%%%%%%%%
\subsection*{锁}\addcontentsline{loh}{figure}{锁 \dpy{suo3}}

\begin{EntryWithPhonetic}{锁}{suo3}{12}{⾦}[HSK 5]
  \definition[把]{s.}{fechadura; dispositivo que impede que as pessoas abram facilmente a parte que se abre e fecha | correntes; cadeado e correntes | qualquer coisa com a forma de um cadeado antigo}
  \definition{v.}{trancar; trancar com chave | costurar com ponto fixo | tricotar}
\end{EntryWithPhonetic}

%%%%% EOF %%%%%


 %%%
%%% T
%%%
\section*{T}\addcontentsline{toc}{section}{T}\addcontentsline{loh}{figure}{\#\#\#\#\#\#\#\# T}

\begin{EntryWithPhonetic}{T-恤}{t5 xu4}{∅,9}{∅,⼼}
  \definition{s.}{camiseta | pulôver | suéter}
\end{EntryWithPhonetic}

%%%%%%%%%% 他 %%%%%%%%%%
\subsection*{他}\addcontentsline{loh}{figure}{他 \dpy{ta1}}

\begin{EntryWithPhonetic}{他}{ta1}{5}{⼈}[HSK 1]
  \definition{pron.}{ele | outro; referindo"-se a outro; diferente | usado após o verbo, indica referência vaga | alguém; todos; usado em conjunto com 你, significa qualquer pessoa ou muitas pessoas | em outro lugar; outro lugar}
  \seealsoref{你}{ni3}
  \seealsoref{怹}{tan1}
  \antonymref{她}{ta1}
\end{EntryWithPhonetic}

\begin{EntryWithPhonetic}{他的}{ta1 de5}{5,8}{⼈,⽩}
  \definition{pron.}{dele}
\end{EntryWithPhonetic}

\begin{EntryWithPhonetic}{他妈的}{ta1ma1de5}{5,6,8}{⼈,⼥,⽩}
  \definition{interj.}{Palavrão tabu: ``Dane-se!''; ``Droga!''; ``Que se dane!''}
\end{EntryWithPhonetic}

\begin{EntryWithPhonetic}{他们}{ta1men5}{5,5}{⼈,⼈}[HSK 1]
  \definition{pron.}{eles}
  \synonymref{常常}{chang2chang2}
  \synonymref{你们}{ni3men5}
  \synonymref{她们}{ta1men5}
  \synonymref{我们}{wo3men5}
  \synonymref{相信}{xiang1xin4}
\end{EntryWithPhonetic}

\begin{EntryWithPhonetic}{他们的}{ta1men5 de5}{5,5,8}{⼈,⼈,⽩}
  \definition{pron.}{deles}
\end{EntryWithPhonetic}

\begin{EntryWithPhonetic}{他人}{ta1ren2}{5,2}{⼈,⼈}[HSK 7-9]
  \definition[任]{pron.}{outros; outras pessoas; outra pessoa}
  \synonymref{别人}{bie2ren5}
  \synonymref{对方}{dui4fang1}
  \antonymref{本人}{ben3ren2}
  \antonymref{自己}{zi4ji3}
  \antonymref{自我}{zi4wo3}
\end{EntryWithPhonetic}

%%%%%%%%%% 它 %%%%%%%%%%
\subsection*{它}\addcontentsline{loh}{figure}{它 \dpy{ta1}}

\begin{EntryWithPhonetic}{它}{ta1}{5}{⼧}[HSK 2]
  \definition*{s.}{Sobrenome: Ta}
  \definition{pron.}{ele; referência a algo além da pessoa (para objetos inanimados) | ele; usado após o verbo, indica referência vaga}
\end{EntryWithPhonetic}

\begin{EntryWithPhonetic}{它们}{ta1men5}{5,5}{⼧,⼈}[HSK 2]
  \definition{pron.}{eles; usado para se referir a mais de uma coisa não humana; geralmente se refere a animais, objetos ou conceitos abstratos}
\end{EntryWithPhonetic}

%%%%%%%%%% 她 %%%%%%%%%%
\subsection*{她}\addcontentsline{loh}{figure}{她 \dpy{ta1}}

\begin{EntryWithPhonetic}{她}{ta1}{6}{⼥}[HSK 1]
  \definition{pron.}{ela | ela; referir-se a coisas que se ama ou aprecia, como a pátria, a bandeira nacional, etc.}
\end{EntryWithPhonetic}

\begin{EntryWithPhonetic}{她的}{ta1 de5}{6,8}{⼥,⽩}
  \definition{pron.}{dela}
\end{EntryWithPhonetic}

\begin{EntryWithPhonetic}{她们}{ta1men5}{6,5}{⼥,⼈}[HSK 1]
  \definition{pron.}{elas; referindo"-se a várias mulheres: em textos escritos, use 她们 quando todas as pessoas forem mulheres e 他们 quando houver homens e mulheres}
  \seealsoref{他们}{ta1men5}
\end{EntryWithPhonetic}

\begin{EntryWithPhonetic}{她们的}{ta1men5 de5}{6,5,8}{⼥,⼈,⽩}
  \definition{pron.}{delas}
\end{EntryWithPhonetic}

%%%%%%%%%% 塌 %%%%%%%%%%
\subsection*{塌}\addcontentsline{loh}{figure}{塌 \dpy{ta1}}

\begin{EntryWithPhonetic}{塌}{ta1}{13}{⼟}[HSK 7-9]
  \definition{v.}{colapsar; desabar; cair; ruir | afundar; tombar; cair ou afundar (aquilo que estava sendo sustentado) | acalmar"-se; tranquilizar"-se; estabilizar | Dialeto: (moral, etc.) desmoronar; afundar | murchar; curvar; cair de cara no chão}
\end{EntryWithPhonetic}

%%%%%%%%%% 踏 %%%%%%%%%%
\subsection*{踏}\addcontentsline{loh}{figure}{踏 \dpy{ta1}}

\begin{EntryWithPhonetic}{踏}{ta1}{15}{⾜}
  \definition{part.}{Caracter formador de palavras}
  \seeref{ta4}
\end{EntryWithPhonetic}

\begin{EntryWithPhonetic}{踏上}{ta1shang5}{15,3}{⾜,⼀}[HSK 7-9]
  \definition{v.}{pôr os pés em; pisar; pisar em | pisar em ou entrar; começar a entrar ou se envolver em uma área}
\end{EntryWithPhonetic}

\begin{EntryWithPhonetic}{踏实}{ta1shi5}{15,8}{⾜,⼧}[HSK 6]
  \definition{adj.}{confiável; sério; estável e seguro; descreve uma atitude séria em relação ao trabalho ou estudo | à vontade; livre de ansiedade; descreve uma mente ou sentimento estável, sem qualquer preocupação ou ansiedade}
\end{EntryWithPhonetic}

%%%%%%%%%% 塔 %%%%%%%%%%
\subsection*{塔}\addcontentsline{loh}{figure}{塔 \dpy{ta3}}

\begin{EntryWithPhonetic}{塔}{ta3}{12}{⼟}[HSK 6]
  \definition*{s.}{Sobrenome: Ta}
  \definition[个,座]{s.}{pagode budista; pagode | torre | (química) coluna; torre}[蒸馏塔===torre de destilação]
\end{EntryWithPhonetic}

%%%%%%%%%% 拓 %%%%%%%%%%
\subsection*{拓}\addcontentsline{loh}{figure}{拓 \dpy{ta4}}

\begin{EntryWithPhonetic}{拓}{ta4}{8}{⼿}
  \definition{v.}{fazer decalques de formas, textos e gráficos em inscrições em pedra, artefatos de bronze, etc.}
  \seeref{tuo4}
\end{EntryWithPhonetic}

%%%%%%%%%% 踏 %%%%%%%%%%
\subsection*{踏}\addcontentsline{loh}{figure}{踏 \dpy{ta4}}

\begin{EntryWithPhonetic}{踏}{ta4}{15}{⾜}[HSK 6]
  \definition{v.}{por os pés em; pisar em; esmagar com o pé | fazer uma investigação ou levantamento no local}
  \seeref{ta1}
\end{EntryWithPhonetic}

\begin{EntryWithPhonetic}{踏板}{ta4ban3}{15,8}{⾜,⽊}
  \definition{s.}{pedal (em um carro, em um piano, etc.) |  apoio para os pés | estribo}
\end{EntryWithPhonetic}

%%%%%%%%%% 胎 %%%%%%%%%%
\subsection*{胎}\addcontentsline{loh}{figure}{胎 \dpy{tai1}}

\begin{EntryWithPhonetic}{胎}{tai1}{9}{⾁}[HSK 7-9]
  \definition{s.}{feto; embrião | nascimento | enchimento; acolchoamento; estofamento | reboco (na fabricação de porcelana, cloisonné, etc.) | pneu | base; reboco; os espaços em branco de certos objetos}
  \seealsoref{胎儿}{tai1'er2}
\end{EntryWithPhonetic}

\begin{EntryWithPhonetic}{胎儿}{tai1'er2}{9,2}{⾁,⼉}[HSK 7-9]
  \definition{s.}{embrião; feto; criança não nascida}
\end{EntryWithPhonetic}

%%%%%%%%%% 台 %%%%%%%%%%
\subsection*{台}\addcontentsline{loh}{figure}{台 \dpy{tai2}}

\begin{EntryWithPhonetic}{台}{tai2}{5}{⼝}[HSK 3]
  \definition*{s.}{Sobrenome: Tai}
  \definition{clas.}{usado para certas máquinas, aparelhos, instrumentos, etc | usado para uma performance completa, como drama, música e dança}
  \definition{s.}{torre | plataforma; palco | suporte; pedestal | qualquer coisa em forma de plataforma ou palco | mesa; escrivaninha | estação de transmissão; refere"-se a estações de rádio | um serviço telefônico especial; refere"-se à estação telefônica | ``seu'', um termo respeitoso usado antigamente para se dirigir a alguém | tufão}
\end{EntryWithPhonetic}

\begin{EntryWithPhonetic}{台灯}{tai2deng1}{5,6}{⼝,⽕}[HSK 6]
  \definition[个,盏]{s.}{luminária de mesa; luminária de leitura; uma luminária com base para uso sobre uma mesa}
\end{EntryWithPhonetic}

\begin{EntryWithPhonetic}{台风}{tai2feng1}{5,4}{⼝,⾵}[HSK 5]
  \definition[场,阵,级]{s.}{tufão; classificação de um ciclone tropical ocorrido no oeste do Pacífico Norte | postura; presença de palco; comportamento ou estilo que os atores demonstram no palco}
\end{EntryWithPhonetic}

\begin{EntryWithPhonetic}{台阶}{tai2jie1}{5,6}{⼝,⾩}[HSK 4]
  \definition[个,级]{s.}{escada; escadaria | passos; metáfora para uma maneira ou oportunidade de evitar constrangimentos causados por um impasse | nova fase; novo nível; novo patamar; metáfora para novas conquistas ou novos patamares alcançados no estudo ou no trabalho}
\end{EntryWithPhonetic}

\begin{EntryWithPhonetic}{台球}{tai2qiu2}{5,11}{⼝,⽟}[HSK 7-9]
  \definition[场,个,颗]{s.}{bilhar; um esporte com bola praticado em uma mesa especialmente projetada, utilizando bastões de madeira dura para golpear uma bola | bola de bilhar; a bola sólida usada no bilhar é feita de materiais resistentes, como plástico, e tem cerca de sete centímetros de diâmetro}
\end{EntryWithPhonetic}

\begin{EntryWithPhonetic}{台上}{tai2shang4}{5,3}{⼝,⼀}[HSK 4]
  \definition{s.}{no palco}
\end{EntryWithPhonetic}

\begin{EntryWithPhonetic}{台下}{tai2xia4}{5,3}{⼝,⼀}
  \definition{s.}{platéia | fora do palco}
\end{EntryWithPhonetic}

%%%%%%%%%% 抬 %%%%%%%%%%
\subsection*{抬}\addcontentsline{loh}{figure}{抬 \dpy{tai2}}

\begin{EntryWithPhonetic}{抬}{tai2}{8}{⼿}[HSK 5]
  \definition{clas.}{usado para objetos que precisam ser carregados por pessoas quando transportados (por exemplo, móveis)}
  \definition{v.}{levantar; elevar; puxar para cima | (por duas ou mais pessoas) carregar; transportar; duas ou mais pessoas carregando algo com as mãos ou nos ombros | discutir, debater (geralmente sem sentido ou sem importância)}
\end{EntryWithPhonetic}

\begin{EntryWithPhonetic}{抬杠}{tai2/gang4}{8,7}{⼿,⽊}
  \definition{v.+compl.}{brigar; discutir; discutir por discutir; discutir sobre o certo e o errado (geralmente sem princípios) | Arcaico: carregar um caixão em barras resistentes}
\end{EntryWithPhonetic}

\begin{EntryWithPhonetic}{抬头}{tai2 tou2}{8,5}{⼿,⼤}[HSK 5]
  \definition{s.}{(em recibos, contas, etc.) nome do comprador ou beneficiário, o local no documento onde o nome do beneficiário ou destinatário é escrito}
  \definition{v.}{levantar a cabeça}
\end{EntryWithPhonetic}

%%%%%%%%%% 太 %%%%%%%%%%
\subsection*{太}\addcontentsline{loh}{figure}{太 \dpy{tai4}}

\begin{EntryWithPhonetic}{太}{tai4}{4}{⼤}[HSK 1]
  \definition*{s.}{Sobrenome: Tai}
  \definition{adj.}{mais alto; maior; mais distante | maior; extremo | bisavô; mais velho ou mais antigo; o de maior posição social ou hierarquia}
  \definition{adv.}{demais; expressa um grau excessivo (usado principalmente para coisas indesejáveis) | muito; extremamente; excessivamente; indica um grau extremamente elevado | muito; usado após o advérbio negativo 不, enfraquece o grau de negação e contém um tom diplomático}
\end{EntryWithPhonetic}

\begin{EntryWithPhonetic}{太极}{tai4ji2}{4,7}{⼤,⽊}[HSK 7-9]
  \definition*{s.}{Filosofia: Tai Chi; Supremo Último, o Absoluto na antiga cosmologia chinesa, apresentado como a fonte primária de todas as coisas criadas (万物)}
  \seealsoref{万物}{wan4wu4}
\end{EntryWithPhonetic}

\begin{EntryWithPhonetic}{太极拳}{tai4ji2quan2}{4,7,10}{⼤,⽊,⼿}[HSK 7-9]
  \definition*[套]{s.}{Tai Chi Chuan, Taiji, T'aichi ou T'aichichuan; uma arte marcial chinesa originária do início da Dinastia Qing; seus movimentos são suaves, lentos e fluidos; acredita"-se que ela aprimore o condicionamento físico e promova a saúde}
\end{EntryWithPhonetic}

\begin{EntryWithPhonetic}{太空}{tai4kong1}{4,8}{⼤,⽳}[HSK 5]
  \definition[把]{s.}{firmamento; espaço sideral; espaço além da atmosfera terrestre; o céu vasto e infinito}
\end{EntryWithPhonetic}

\begin{EntryWithPhonetic}{太平}{tai4ping2}{4,5}{⼤,⼲}[HSK 7-9]
  \definition*{s.}{Província de Thai Binh, nome de um local no norte do Vietnã, uma das províncias do Vietnã}
  \definition{adj.}{seguro e protegido; pacífico e tranquilo; boa ordem social sem guerra; estabilidade (social, nacional); paz}
  \synonymref{安定}{an1ding4}
\end{EntryWithPhonetic}

\begin{EntryWithPhonetic}{太平洋}{tai4ping2 yang2}{4,5,9}{⼤,⼲,⽔}
  \definition*{s.}{Oceano Pacífico}
\end{EntryWithPhonetic}

\begin{EntryWithPhonetic}{太太}{tai4tai5}{4,4}{⼤,⼤}[HSK 2]
  \definition[位,名,个,些]{s.}{senhora; madame; títulos para mulheres casadas | esposa; senhora; madame; referir-se à própria esposa ou à esposa de outra pessoa}
\end{EntryWithPhonetic}

\begin{EntryWithPhonetic}{太阳}{tai4yang2}{4,6}{⼤,⾩}[HSK 2]
  \definition[个,轮,枚,颗,盏]{s.}{o Sol | luz do sol; luz solar}
\end{EntryWithPhonetic}

\begin{EntryWithPhonetic}{太阳窗}{tai4yang2chuang1}{4,6,12}{⼤,⾩,⽳}
  \definition{s.}{teto solar (de veículos)}
\end{EntryWithPhonetic}

\begin{EntryWithPhonetic}{太阳灯}{tai4yang2deng1}{4,6,6}{⼤,⾩,⽕}
  \definition{s.}{lâmpada solar (com células fotovoltaicas)}
\end{EntryWithPhonetic}

\begin{EntryWithPhonetic}{太阳风}{tai4yang2feng1}{4,6,4}{⼤,⾩,⾵}
  \definition{s.}{vento solar}
\end{EntryWithPhonetic}

\begin{EntryWithPhonetic}{太阳镜}{tai4yang2jing4}{4,6,16}{⼤,⾩,⾦}
  \definition{s.}{óculos de sol}
\end{EntryWithPhonetic}

\begin{EntryWithPhonetic}{太阳能}{tai4yang2neng2}{4,6,10}{⼤,⾩,⾁}[HSK 6]
  \definition{s.}{energia solar; a energia de radiação emitida pelo Sol é a energia solar produzida pela reação de fusão dos núcleos de hidrogênio no Sol, é a fonte de luz e calor na Terra}
\end{EntryWithPhonetic}

\begin{EntryWithPhonetic}{太阳日}{tai4yang2ri4}{4,6,4}{⼤,⾩,⽇}
  \definition{s.}{dia solar}
\end{EntryWithPhonetic}

\begin{EntryWithPhonetic}{太阳穴}{tai4yang2xue2}{4,6,5}{⼤,⾩,⽳}
  \definition{s.}{têmpora (nas laterais da cabeça humana)}
\end{EntryWithPhonetic}

\begin{EntryWithPhonetic}{太阳翼}{tai4yang2yi4}{4,6,17}{⼤,⾩,⽻}
  \definition{s.}{painel solar}
\end{EntryWithPhonetic}

\begin{EntryWithPhonetic}{太阳雨}{tai4yang2yu3}{4,6,8}{⼤,⾩,⾬}
  \definition{s.}{banho de sol}
\end{EntryWithPhonetic}

%%%%%%%%%% 态 %%%%%%%%%%
\subsection*{态}\addcontentsline{loh}{figure}{态 \dpy{tai4}}

\begin{EntryWithPhonetic}{态}{tai4}{8}{⼼}
  \definition{s.}{forma; aparência; condição | (física) estado | (linguística) voz}[气态===estado gasoso | 被动态===voz passiva]
\end{EntryWithPhonetic}

\begin{EntryWithPhonetic}{态度}{tai4du5}{8,9}{⼼,⼴}[HSK 2]
  \definition[种,个]{s.}{maneira; comportamento; atitude; comportamento e expressão facial das pessoas | atitude; abordagem; opinião sobre o assunto e medidas tomadas}
\end{EntryWithPhonetic}

%%%%%%%%%% 泰 %%%%%%%%%%
\subsection*{泰}\addcontentsline{loh}{figure}{泰 \dpy{tai4}}

\begin{EntryWithPhonetic}{泰}{tai4}{10}{⽔}
  \definition*{s.}{Tai, um dos Sessenta Diagramas | Tailândia, abreviação de 泰国 | Sobrenome: Tai}
  \definition{adj.}{seguro; pacífico | Literário: extremo; demasiado | calmo | arrogante}
  \definition{adv.}{extremamente; o mais; demais}
  \seealsoref{泰国}{tai4guo2}
\end{EntryWithPhonetic}

\begin{EntryWithPhonetic}{泰斗}{tai4dou3}{10,4}{⽔,⽃}[HSK 7-9]
  \definition[位]{s.}{grão-mestre; uma autoridade líder; eminente estudioso, músico, artista, etc.}
\end{EntryWithPhonetic}

\begin{EntryWithPhonetic}{泰国}{tai4guo2}{10,8}{⽔,⼞}
  \definition*{s.}{Tailândia}
\end{EntryWithPhonetic}

\begin{EntryWithPhonetic}{泰山}{tai4shan1}{10,3}{⽔,⼭}
  \definition*{s.}{Monte Tai na província de Shandong, montanha oriental das Cinco Montanhas Sagradas (五岳); na antiguidade, era considerado um representante das altas montanhas e frequentemente usado para simbolizar pessoas veneradas e coisas de grande importância e valor}
  \definition{s.}{filha da esposa; sogro}
  \seealsoref{五岳}{wu3yue4}
\end{EntryWithPhonetic}

%%%%%%%%%% 贪 %%%%%%%%%%
\subsection*{贪}\addcontentsline{loh}{figure}{贪 \dpy{tan1}}

\begin{EntryWithPhonetic}{贪}{tan1}{8}{⾙}[HSK 7-9]
  \definition{v.}{apropriar-se indevidamente; desviar fundos; praticar corrupção; ser corrupto; originalmente, referia-se ao amor ao dinheiro; posteriormente, passou a ser associado à corrupção | ter um desejo insaciável por; ter um desejo voraz por | cobiçar; ansiar por; ser ganancioso por}
\end{EntryWithPhonetic}

\begin{EntryWithPhonetic}{贪婪}{tan1lan2}{8,11}{⾙,⼥}[HSK 7-9]
  \definition{adj.}{ganancioso; avarento; descreve uma pessoa ou animal como alguém que nunca está satisfeito | ganancioso; ávido; descreve alguém que nunca está satisfeito com coisas boas como conhecimento e ar puro}
  \definition{s.}{ganância}
  \synonymref{海量}{hai3liang4}
  \antonymref{满足}{man3zu2}
\end{EntryWithPhonetic}

\begin{EntryWithPhonetic}{贪玩儿}{tan1wan2r5}{8,8,2}{⾙,⽟,⼉}[HSK 7-9]
  \definition{s.}{brincalhão}
  \definition{v.}{ser brincalhão}
\end{EntryWithPhonetic}

\begin{EntryWithPhonetic}{贪污}{tan1wu1}{8,6}{⾙,⽔}[HSK 7-9]
  \definition{v.}{desviar fundos; obter ilegalmente terras, dinheiro ou propriedade estatal, coletiva ou unitária, abusando do poder ou da posição}
  \synonymref{腐败}{fu3bai4}
  \antonymref{廉洁}{lian2jie2}
\end{EntryWithPhonetic}

%%%%%%%%%% 怹 %%%%%%%%%%
\subsection*{怹}\addcontentsline{loh}{figure}{怹 \dpy{tan1}}

\begin{EntryWithPhonetic}{怹}{tan1}{9}{⼼}
  \definition{pron.}{ele, ela (cortês)}
  \synonymref{他}{ta1}
\end{EntryWithPhonetic}

%%%%%%%%%% 摊 %%%%%%%%%%
\subsection*{摊}\addcontentsline{loh}{figure}{摊 \dpy{tan1}}

\begin{EntryWithPhonetic}{摊}{tan1}{13}{⼿}[HSK 7-9]
  \definition{clas.}{utilizado para pastas ou líquidos espalhados}[路边是一摊泥。===Havia uma poça de lama à beira da estrada.]
  \definition{s.}{banca de vendedor; barraca; quiosque; pontos de venda instalados ao longo das estradas ou em praças}
  \definition{v.}{espalhar; estender | cozinhar em uma camada fina e uniforme | partilhar; dividir; repartir; compartilhar | (algo desagradável) acontecer; ocorrer a; encontrar-se com; deparar-se com}
\end{EntryWithPhonetic}

%%%%%%%%%% 瘫 %%%%%%%%%%
\subsection*{瘫}\addcontentsline{loh}{figure}{瘫 \dpy{tan1}}

\begin{EntryWithPhonetic}{瘫}{tan1}{15}{⽧}[HSK 7-9]
  \definition{adj.}{paralisado}
  \definition{s.}{paralisia}
  \definition{v.}{ficar fisicamente paralisado}
\end{EntryWithPhonetic}

\begin{EntryWithPhonetic}{瘫痪}{tan1huan4}{15,12}{⽧,⽧}[HSK 7-9]
  \definition{v.}{ficar paralisado; devido a disfunções neurológicas, uma parte do corpo pode perder total ou parcialmente a capacidade de se mover | entrar em colapso; parar completamente; essa metáfora descreve uma organização desorganizada e incapaz de funcionar adequadamente}
  \synonymref{残疾}{can2ji5}
  \antonymref{健康}{jian4kang1}
\end{EntryWithPhonetic}

%%%%%%%%%% 坛 %%%%%%%%%%
\subsection*{坛}\addcontentsline{loh}{figure}{坛 \dpy{tan2}}

\begin{EntryWithPhonetic}{坛}{tan2}{7}{⼟}[HSK 7-9]
  \definition{s.}{altar | terreno elevado para plantio de flores, etc. | plataforma; fórum | círculos; mundo | Obsoleto: organização criada por uma sociedade secreta para adorar deuses em uma manifestação | jarro de barro; cântaro; tonel | garrafão; projetado especialmente para armazenar líquidos corrosivos, como o ácido sulfúrico}
  \seealsoref{坛儿}{tan2r5}
\end{EntryWithPhonetic}

\begin{EntryWithPhonetic}{坛儿}{tan2r5}{7,2}{⼟,⼉}
  \definition{s.}{jarro de barro; cântaro; tonel}
\end{EntryWithPhonetic}

%%%%%%%%%% 谈 %%%%%%%%%%
\subsection*{谈}\addcontentsline{loh}{figure}{谈 \dpy{tan2}}

\begin{EntryWithPhonetic}{谈}{tan2}{10}{⾔}[HSK 3]
  \definition*{s.}{Sobrenome: Tan}
  \definition{s.}{o que é dito ou falado; discurso}
  \definition{v.}{falar; bater papo; discutir}
\end{EntryWithPhonetic}

\begin{EntryWithPhonetic}{谈不上}{tan2 bu5 shang4}{10,4,3}{⾔,⼀,⼀}[HSK 7-9]
  \definition{expr.}{nem pensar; não conta, longe de atingir um certo nível; não merecedor de; não digno de ser chamado}
\end{EntryWithPhonetic}

\begin{EntryWithPhonetic}{谈到}{tan2dao4}{10,8}{⾔,⼑}[HSK 7-9]
  \definition{v.}{falar de (sobre); comentar sobre (algo relacionado a); referir"-se a}
\end{EntryWithPhonetic}

\begin{EntryWithPhonetic}{谈话}{tan2/hua4}{10,8}{⾔,⾔}[HSK 3]
  \definition[次]{s.}{declaração; opiniões (principalmente políticas) expressas na forma de conversas}
  \definition{v.+compl.}{conversar; discutir | falar; refere"-se especificamente ao uso da conversa para entender a situação, fazer trabalho ideológico, etc. (usado principalmente por superiores para subordinados)}
\end{EntryWithPhonetic}

\begin{EntryWithPhonetic}{谈恋爱}{tan2lian4'ai4}{10,10,10}{⾔,⼼,⽖}
  \definition{v.}{namorar | apaixonar-se}
\end{EntryWithPhonetic}

\begin{EntryWithPhonetic}{谈论}{tan2lun4}{10,6}{⾔,⾔}[HSK 7-9]
  \definition{v.}{debater; discutir; falar sobre; conversar sobre}
  \synonymref{辩论}{bian4lun4}
  \synonymref{带来}{dai4 lai2}
  \synonymref{评论}{ping2lun4}
  \synonymref{讨论}{tao3lun4}
  \synonymref{议论}{yi4lun4}
  \antonymref{沉默}{chen2mo4}
\end{EntryWithPhonetic}

\begin{EntryWithPhonetic}{谈判}{tan2pan4}{10,7}{⾔,⼑}[HSK 3]
  \definition{v.}{negociar; manter conversações; para resolver um grande problema, as partes relevantes trocaram opiniões entre si, na esperança de encontrar uma solução com a qual todos pudessem concordar}
\end{EntryWithPhonetic}

\begin{EntryWithPhonetic}{谈起}{tan2qi3}{10,10}{⾔,⾛}[HSK 7-9]
  \definition{v.}{mencionar; falar de}
\end{EntryWithPhonetic}

%%%%%%%%%% 弹 %%%%%%%%%%
\subsection*{弹}\addcontentsline{loh}{figure}{弹 \dpy{tan2}}

\begin{EntryWithPhonetic}{弹}{tan2}{11}{⼸}[HSK 5]
  \definition{v.}{enviar; atirar (como com uma catapulta, etc.); usar a elasticidade de um objeto para lançar outro objeto | afofar; preparar fibras; usar um dispositivo elástico para amolecer as fibras | virar; sacudir | dedilhar; tocar (um instrumento musical de cordas) | acusar; atacar; criticar; relatar | saltar; quicar}
  \seeref{dan4}
\end{EntryWithPhonetic}

\begin{EntryWithPhonetic}{弹性}{tan2xing4}{11,8}{⼸,⼼}[HSK 7-9]
  \definition{adj.}{elástico; flexível; essa metáfora descreve a natureza das coisas, que são ajustáveis ​​e adaptáveis ​​de acordo com as necessidades reais}
  \definition{s.}{elasticidade; resiliência; a propriedade de um objeto se deformar sob a ação de uma força externa e retornar à sua forma original após a remoção dessa força}
  \synonymref{韧性}{ren4xing4}
\end{EntryWithPhonetic}

%%%%%%%%%% 痰 %%%%%%%%%%
\subsection*{痰}\addcontentsline{loh}{figure}{痰 \dpy{tan2}}

\begin{EntryWithPhonetic}{痰}{tan2}{13}{⽧}[HSK 7-9]
  \definition[口]{s.}{catarro; escarro; muco secretado pelos alvéolos e pela traqueia}
\end{EntryWithPhonetic}

%%%%%%%%%% 忐 %%%%%%%%%%
\subsection*{忐}\addcontentsline{loh}{figure}{忐 \dpy{tan3}}

\begin{EntryWithPhonetic}{忐}{tan3}{7}{⼼}
  \definition{adj.}{nervoso; inquieto; agitado}
\end{EntryWithPhonetic}

%%%%%%%%%% 坦 %%%%%%%%%%
\subsection*{坦}\addcontentsline{loh}{figure}{坦 \dpy{tan3}}

\begin{EntryWithPhonetic}{坦}{tan3}{8}{⼟}
  \definition*{s.}{Sobrenome: Tan}
  \definition{adj.}{nivelado; suave; plano | calmo; composto | aberto; sincero; franco}
\end{EntryWithPhonetic}

\begin{EntryWithPhonetic}{坦白}{tan3bai2}{8,5}{⼟,⽩}[HSK 7-9]
  \definition{adj.}{honesto; franco; sincero; puros de coração}
  \definition{v.}{ser franco; confessar; dizer a verdade sobre erros ou crimes}
  \synonymref{坦率}{tan3shuai4}
  \antonymref{抗拒}{kang4ju4}
  \antonymref{通知}{tong1zhi1}
\end{EntryWithPhonetic}

\begin{EntryWithPhonetic}{坦诚}{tan3cheng2}{8,8}{⼟,⾔}[HSK 7-9]
  \definition{adj.}{franco e sincero; franco e aberto}
  \synonymref{诚恳}{cheng2ken3}
  \synonymref{坦率}{tan3shuai4}
  \synonymref{真诚}{zhen1cheng2}
  \antonymref{撒谎}{sa1/huang3}
\end{EntryWithPhonetic}

\begin{EntryWithPhonetic}{坦克}{tan3ke4}{8,7}{⼟,⼗}[HSK 7-9]
  \definition[辆]{s.}{Empréstimo linguístico: tanque (veículo militar); veículos de combate blindados sobre esteiras, equipados com canhões, metralhadoras e torres giratórias}
\end{EntryWithPhonetic}

\begin{EntryWithPhonetic}{坦然}{tan3ran2}{8,12}{⼟,⽕}[HSK 7-9]
  \definition{adj.}{calmo; sereno; imperturbável; sem receios; descreve um estado de espírito como calmo e sem preocupações}
  \synonymref{安心}{an1xin1}
  \synonymref{平静}{ping2jing4}
  \antonymref{狼狈}{lang2bei4}
  \antonymref{受惊}{shou4/jing1}
\end{EntryWithPhonetic}

\begin{EntryWithPhonetic}{坦率}{tan3shuai4}{8,11}{⼟,⽞}[HSK 7-9]
  \definition{adj.}{1. sincero; franco; direto}
  \synonymref{爽快}{shuang3kuai5}
  \synonymref{坦白}{tan3bai2}
  \synonymref{坦诚}{tan3cheng2}
  \antonymref{通知}{tong1zhi1}
  \antonymref{委婉}{wei3wan3}
\end{EntryWithPhonetic}

%%%%%%%%%% 毯 %%%%%%%%%%
\subsection*{毯}\addcontentsline{loh}{figure}{毯 \dpy{tan3}}

\begin{EntryWithPhonetic}{毯}{tan3}{12}{⽑}
  \definition[条]{s.}{cobertor; tapete; carpete}
\end{EntryWithPhonetic}

\begin{EntryWithPhonetic}{毯子}{tan3zi5}{12,3}{⽑,⼦}[HSK 7-9]
  \definition[条,张,床,面]{s.}{cobertor; tecidos grossos e macios podem ser usados ​​como cortinas, capas ou decorações}
  \synonymref{被子}{bei4zi5}
  \synonymref{垫子}{dian4zi5}
\end{EntryWithPhonetic}

%%%%%%%%%% 叹 %%%%%%%%%%
\subsection*{叹}\addcontentsline{loh}{figure}{叹 \dpy{tan4}}

\begin{EntryWithPhonetic}{叹}{tan4}{5}{⼝}
  \definition{v.}{suspirar | exclamar com admiração; aclamar; louvar |recitar com cadência; entoar cântico; entoar}
\end{EntryWithPhonetic}

\begin{EntryWithPhonetic}{叹气}{tan4/qi4}{5,4}{⼝,⽓}[HSK 6]
  \definition{v.+compl.}{suspirar; soltar um suspiro; soltar um longo suspiro e fazer um som devido à insatisfação ou desamparo}
\end{EntryWithPhonetic}

%%%%%%%%%% 炭 %%%%%%%%%%
\subsection*{炭}\addcontentsline{loh}{figure}{炭 \dpy{tan4}}

\begin{EntryWithPhonetic}{炭}{tan4}{9}{⽕}[HSK 7-9]
  \definition*{s.}{Sobrenome: Tan}
  \definition[块,吨]{s.}{carvão; algo semelhante a carvão}
  \antonymref{冰}{bing1}
\end{EntryWithPhonetic}

%%%%%%%%%% 探 %%%%%%%%%%
\subsection*{探}\addcontentsline{loh}{figure}{探 \dpy{tan4}}

\begin{EntryWithPhonetic}{探}{tan4}{11}{⼿}[HSK 7-9]
  \definition[个,位,名]{s.}{batedor; espião; detetive}
  \definition{v.}{tentar descobrir; explorar; soar | explorar; espionar | visitar; fazer uma visita em | se destacar | preocupar-se com; envolver-se em | ver; invocar}
\end{EntryWithPhonetic}

\begin{EntryWithPhonetic}{探测}{tan4ce4}{11,9}{⼿,⽔}[HSK 7-9]
  \definition{v.}{sondar; examinar; explorar; utilizar ferramentas para observar ou medir coisas que não podem ser observadas ou medidas diretamente}
\end{EntryWithPhonetic}

\begin{EntryWithPhonetic}{探亲}{tan4/qin1}{11,9}{⼿,⼇}[HSK 7-9]
  \definition{v.+compl.}{ir para casa para visitar a família; visitar parentes em outra cidade (geralmente referindo-se à visita aos pais ou cônjuge)}
\end{EntryWithPhonetic}

\begin{EntryWithPhonetic}{探求}{tan4qiu2}{11,7}{⼿,⽔}[HSK 7-9]
  \definition{v.}{procurar; perseguir; buscar; explorar e buscar}
  \synonymref{考虑}{kao3lv4}
  \synonymref{商量}{shang1liang5}
  \synonymref{搜索}{sou1suo3}
  \synonymref{探索}{tan4suo3}
  \synonymref{推测}{tui1ce4}
  \synonymref{寻求}{xun2qiu2}
  \synonymref{寻找}{xun2zhao3}
  \synonymref{研究}{yan2jiu1}
  \synonymref{追求}{zhui1qiu2}
\end{EntryWithPhonetic}

\begin{EntryWithPhonetic}{探索}{tan4suo3}{11,10}{⼿,⽷}[HSK 6]
  \definition{v.}{sondar; explorar; procurar respostas de várias fontes para resolver dúvidas}
\end{EntryWithPhonetic}

\begin{EntryWithPhonetic}{探讨}{tan4tao3}{11,5}{⼿,⾔}[HSK 6]
  \definition{v.}{examinar; indagar; investigar; discutir}
\end{EntryWithPhonetic}

\begin{EntryWithPhonetic}{探望}{tan4wang4}{11,11}{⼿,⽉}[HSK 7-9]
  \definition{v.}{olhar ao redor; observar (tentar descobrir o que está acontecendo) | ver; visitar; fazer uma visita a alguém (geralmente de longe)}
  \synonymref{拜访}{bai4fang3}
  \synonymref{访问}{fang3wen4}
  \synonymref{看望}{kan4wang5}
  \antonymref{回避}{hui2bi4}
\end{EntryWithPhonetic}

\begin{EntryWithPhonetic}{探险}{tan4/xian3}{11,9}{⼿,⾩}[HSK 7-9]
  \definition{v.+compl.}{explorar; fazer explorações; aventurar"-se no desconhecido; investigar lugares onde ninguém jamais esteve ou onde pouquíssimas pessoas estiveram (o mundo natural)}
  \synonymref{冒险}{mao4/xian3}
  \synonymref{探求}{tan4qiu2}
\end{EntryWithPhonetic}

%%%%%%%%%% 碳 %%%%%%%%%%
\subsection*{碳}\addcontentsline{loh}{figure}{碳 \dpy{tan4}}

\begin{EntryWithPhonetic}{碳}{tan4}{14}{⽯}[HSK 7-9]
  \definition[块]{s.}{C; carbono (elemento químico)}
\end{EntryWithPhonetic}

\begin{EntryWithPhonetic}{碳足迹}{tan4 zu2ji4}{14,7,9}{⽯,⾜,⾡}
  \definition{s.}{pegada de carbono}
\end{EntryWithPhonetic}

%%%%%%%%%% 汤 %%%%%%%%%%
\subsection*{汤}\addcontentsline{loh}{figure}{汤 \dpy{tang1}}

\begin{EntryWithPhonetic}{汤}{tang1}{6}{⽔}[HSK 3]
  \definition*{s.}{Sobrenome: Tang}
  \definition[勺,碗,杯,锅]{s.}{água quente; água fervente | fontes termais | água utilizada para ferver algo| sopa; caldo | uma preparação líquida de ervas medicinais; decocção}
  \seeref{shang1}
\end{EntryWithPhonetic}

\begin{EntryWithPhonetic}{汤圆}{tang1yuan2}{6,10}{⽔,⼞}[HSK 7-9]
  \definition{s.}{bolinho doce; (geralmente bolinhos recheados) feitos de farinha de arroz glutinoso servidos em sopa}
  \synonymref{元宵}{yuan2xiao1}
\end{EntryWithPhonetic}

%%%%%%%%%% 趟 %%%%%%%%%%
\subsection*{趟}\addcontentsline{loh}{figure}{趟 \dpy{tang1}}

\begin{EntryWithPhonetic}{趟}{tang1}{15}{⾛}
  \definition{v.}{atravessar; andar na grama ou onde não haja caminho | usar arados, capinadores, etc. para virar o solo e remover ervas daninhas | vadear; atravessar a vau; caminhar por águas rasas}[我们趟水去那小岛。===Nós vadeamos até a ilha.]
  \seeref{tang4}
\end{EntryWithPhonetic}

%%%%%%%%%% 唐 %%%%%%%%%%
\subsection*{唐}\addcontentsline{loh}{figure}{唐 \dpy{tang2}}

\begin{EntryWithPhonetic}{唐}{tang2}{10}{⼝}
  \definition*{s.}{Dinastia estabelecida pelo Imperador Yao, 尧, no período lendário da história chinesa | Dinastia Tang (618-907) | Dinastia Tang posterior (923-936), uma das cinco dinastias | Sobrenome: Tang}
  \definition{adj.}{exagerado; bombástico; orgulhoso | em vão; por nada}
  \seealsoref{尧}{yao2}
\end{EntryWithPhonetic}

\begin{EntryWithPhonetic}{唐初四大家}{tang2 chu1 si4 da4jia1}{10,7,5,3,10}{⼝,⾐,⼞,⼤,⼧}
  \definition*{s.}{Quatro grandes calígrafos do início da dinastia Tang; refere"-se a Yu Shi'nan (虞世南), Ouyang Xun (欧阳询), Chu Suiliang (褚遂良) e Xue Ji (薛稷)}
  \seealsoref{褚遂良}{chu3 sui4liang2}
  \seealsoref{欧阳询}{ou1yang2 xun2}
  \seealsoref{薛稷}{xue1 ji4}
  \seealsoref{虞世南}{yu2 shi4nan2}
\end{EntryWithPhonetic}

\begin{EntryWithPhonetic}{唐人街}{tang2ren2 jie1}{10,2,12}{⼝,⼈,⾏}
  \definition*[条,座]{s.}{Bairro Chinês; Chinatown; refere"-se ao mercado de rua onde os chineses do exterior vivem e abrem muitas lojas com características chinesas}
  \seealsoref{中国城}{zhong1guo2cheng2}
\end{EntryWithPhonetic}

%%%%%%%%%% 堂 %%%%%%%%%%
\subsection*{堂}\addcontentsline{loh}{figure}{堂 \dpy{tang2}}

\begin{EntryWithPhonetic}{堂}{tang2}{11}{⼟}[HSK 7-9]
  \definition*{s.}{Sobrenome: Tang}
  \definition{clas.}{classe; uma turma é dividida em seções; uma seção é chamada de classe | caso; um julgamento de cada vez é chamado de sessão judicial | conjunto; conjuntos de móveis | utilizado para cenas; murais, etc.}
  \definition[节,门]{s.}{sala; cômodos principais; hall como símbolo das casas principais no sistema familiar | tribunal; antigamente, era um local onde se realizavam cerimônias em repartições públicas; um local para audiências judiciais | placa da loja; o nome de uma loja; utilizado para a identidade visual da loja | mãe; pais; salão interno; metaforicamente referindo-se à mãe | do mesmo clã; parentesco entre primos, etc., do mesmo avô paterno ou bisavô | salão; casas projetadas especificamente para uma determinada atividade}[我参观了三槐堂。===Visitei o Salão Sanhuaitang.]
\end{EntryWithPhonetic}

%%%%%%%%%% 糖 %%%%%%%%%%
\subsection*{糖}\addcontentsline{loh}{figure}{糖 \dpy{tang2}}

\begin{EntryWithPhonetic}{糖}{tang2}{16}{⽶}[HSK 3]
  \definition[包,斤,勺,袋,块]{s.}{açúcar; um tipo de açúcar; um tipo de composto orgânico, que pode ser dividido em três tipos: monossacarídeos, dissacarídeos e polissacarídeos; é a principal substância que produz energia térmica no corpo humano, como glicose, sacarose, lactose, amido, etc. | açúcar; açúcar comestível; termo geral para açúcar | doces; balas | carboidrato; algo doce e calórico}
\end{EntryWithPhonetic}

\begin{EntryWithPhonetic}{糖醋鱼}{tang2cu4yu2}{16,15,8}{⽶,⾣,⿂}
  \definition{s.}{peixe guisado em molho agridoce (prato)}
\end{EntryWithPhonetic}

\begin{EntryWithPhonetic}{糖果}{tang2guo3}{16,8}{⽶,⽊}[HSK 7-9]
  \definition[个,颗,包,袋]{s.}{doce; alimentos açucarados, que geralmente contêm suco de frutas, especiarias, leite ou café, etc.}
\end{EntryWithPhonetic}

\begin{EntryWithPhonetic}{糖尿病}{tang2niao4bing4}{16,7,10}{⽶,⼫,⽧}[HSK 7-9]
  \definition{s.}{diabetes; diabetes mellitus; doença crônica causada pela secreção insuficiente de insulina, levando a distúrbios no metabolismo da glicose e níveis elevados de açúcar no sangue}[糖尿病会带来严重的问题。===O diabetes pode causar problemas graves.]
\end{EntryWithPhonetic}

%%%%%%%%%% 倘 %%%%%%%%%%
\subsection*{倘}\addcontentsline{loh}{figure}{倘 \dpy{tang3}}

\begin{EntryWithPhonetic}{倘}{tang3}{10}{⼈}
  \definition{conj.}{se; supondo; no caso}
  \seeref{chang2}
\end{EntryWithPhonetic}

\begin{EntryWithPhonetic}{倘或}{tang3huo4}{10,8}{⼈,⼽}
  \definition{conj.}{se | supondo que | no caso}
\end{EntryWithPhonetic}

\begin{EntryWithPhonetic}{倘若}{tang3ruo4}{10,8}{⼈,⾋}[HSK 7-9]
  \definition{conj.}{se; caso; supondo que; indica uma hipótese e geralmente corresponde a palavras como 那, 那么 e 就}[倘若她不同意呢?===E se ela discordar?]
  \seealsoref{就}{jiu4}
  \seealsoref{那}{na4}
  \seealsoref{那么}{na4me5}
  \synonymref{假使}{jia3shi3}
\end{EntryWithPhonetic}

\begin{EntryWithPhonetic}{倘使}{tang3shi3}{10,8}{⼈,⼈}
  \definition{conj.}{se | supondo que | no caso}
\end{EntryWithPhonetic}

%%%%%%%%%% 淌 %%%%%%%%%%
\subsection*{淌}\addcontentsline{loh}{figure}{淌 \dpy{tang3}}

\begin{EntryWithPhonetic}{淌}{tang3}{11}{⽔}[HSK 7-9]
  \definition{v.}{gotejar; escorrer; pingar | fluir para baixo}
  \synonymref{流}{liu2}
\end{EntryWithPhonetic}

%%%%%%%%%% 躺 %%%%%%%%%%
\subsection*{躺}\addcontentsline{loh}{figure}{躺 \dpy{tang3}}

\begin{EntryWithPhonetic}{躺}{tang3}{15}{⾝}[HSK 4]
  \definition{v.}{deitar; reclinar; cair no chão ou sobre um objeto}
\end{EntryWithPhonetic}

%%%%%%%%%% 烫 %%%%%%%%%%
\subsection*{烫}\addcontentsline{loh}{figure}{烫 \dpy{tang4}}

\begin{EntryWithPhonetic}{烫}{tang4}{10}{⽕}[HSK 7-9]
  \definition{adj.}{escaldante; muito quente; alta temperatura do objeto}
  \definition{v.}{queimar; escaldar; objetos em alta temperatura causam dor ao entrarem em contato com a pele | passar a ferro; prensar; aquecer; aquecer em água quente; utilizar um objeto mais quente para aumentar a temperatura de outro objeto ou provocar outras alterações | fazer permanente (cabelo)}
\end{EntryWithPhonetic}

%%%%%%%%%% 趟 %%%%%%%%%%
\subsection*{趟}\addcontentsline{loh}{figure}{趟 \dpy{tang4}}

\begin{EntryWithPhonetic}{趟}{tang4}{15}{⾛}[HSK 6]
  \definition{clas.}{usado para o número de vezes de viagens de ida e volta |  usado para coisas dispostas em fileiras ou tiras | usado para a programação de veículos, navios, etc. que circulam em uma determinada ordem | usado em conjuntos de movimentos de artes marciais}
  \definition{s.}{marcha; procissão; jornada; viagem}
  \seeref{tang1}
\end{EntryWithPhonetic}

%%%%%%%%%% 掏 %%%%%%%%%%
\subsection*{掏}\addcontentsline{loh}{figure}{掏 \dpy{tao1}}

\begin{EntryWithPhonetic}{掏}{tao1}{11}{⼿}[HSK 6]
  \definition{v.}{extrair; retirar; pescar | cavar (um buraco, etc.); escavar; retirar | (coloquial) roubar do bolso de alguém | tirar}
\end{EntryWithPhonetic}

\begin{EntryWithPhonetic}{掏钱}{tao1 qian2}{11,10}{⼿,⾦}[HSK 7-9]
  \definition{v.}{pagar; pagar dinheiro; gastar dinheiro}
\end{EntryWithPhonetic}

%%%%%%%%%% 滔 %%%%%%%%%%
\subsection*{滔}\addcontentsline{loh}{figure}{滔 \dpy{tao1}}

\begin{EntryWithPhonetic}{滔}{tao1}{13}{⽔}
  \definition{adj.}{(de água) transbordando | arrogante | turbulento | largo e longo; grande}
  \definition{v.}{inundar; alagar}
\end{EntryWithPhonetic}

\begin{EntryWithPhonetic}{滔滔不绝}{tao1tao1-bu4jue2}{13,13,4,9}{⽔,⽔,⼀,⽷}[HSK 7-9]
  \definition{expr.}{tagarelando sem parar; falando sem parar; como um fluxo contínuo de água, continua sem parar.; frequentemente usado para descrever alguém que fala muito e nunca para}
  \synonymref{夸夸其谈}{kua1kua1-qi2tan2}
\end{EntryWithPhonetic}

\begin{EntryWithPhonetic}{滔天}{tao1tian1}{13,4}{⽔,⼤}
  \definition{adj.}{(ondas, raiva, desastres, crimes, etc.) imponente, avassalador, imenso}
\end{EntryWithPhonetic}

%%%%%%%%%% 逃 %%%%%%%%%%
\subsection*{逃}\addcontentsline{loh}{figure}{逃 \dpy{tao2}}

\begin{EntryWithPhonetic}{逃}{tao2}{9}{⾡}[HSK 5]
  \definition{v.}{fugir; escapar; correr; dar no pé | evadir; esquivar-se; escapar}
\end{EntryWithPhonetic}

\begin{EntryWithPhonetic}{逃避}{tao2bi4}{9,16}{⾡,⾌}[HSK 7-9]
  \definition{v.}{esquivar-se; evadir-se; escapar; evitar coisas que você não quer ou não se atreve a tocar}
  \synonymref{躲避}{duo3bi4}
  \synonymref{躲藏}{duo3cang2}
  \synonymref{回避}{hui2bi4}
  \synonymref{隐藏}{yin3cang2}
  \antonymref{尝试}{chang2shi4}
  \antonymref{交锋}{jiao1/feng1}
  \antonymref{经受}{jing1shou4}
  \antonymref{面对}{mian4dui4}
\end{EntryWithPhonetic}

\begin{EntryWithPhonetic}{逃跑}{tao2pao3}{9,12}{⾡,⾜}[HSK 5]
  \definition{v.}{fugir; escapar; correr; partir para fugir de um ambiente ou de coisas que não lhe são favoráveis}
\end{EntryWithPhonetic}

\begin{EntryWithPhonetic}{逃生}{tao2sheng1}{9,5}{⾡,⽣}[HSK 7-9]
  \definition{v.}{escapar; fugir para salvar a vida; escapar com vida; escapar de um ambiente perigoso para sobreviver}
  \synonymref{脱离}{tuo1li2}
\end{EntryWithPhonetic}

\begin{EntryWithPhonetic}{逃亡}{tao2wang2}{9,3}{⾡,⼇}[HSK 7-9]
  \definition{v.}{fugir de casa; ir para o exílio; tornar"-se um fugitivo}
  \synonymref{避难}{bi4/nan4}
  \synonymref{流浪}{liu2lang4}
\end{EntryWithPhonetic}

\begin{EntryWithPhonetic}{逃走}{tao2 zou3}{9,7}{⾡,⾛}[HSK 5]
  \definition{v.}{escapar; afastar-se de pessoas, coisas ou lugares que não são bons para você ou que você não gosta}
\end{EntryWithPhonetic}

%%%%%%%%%% 桃 %%%%%%%%%%
\subsection*{桃}\addcontentsline{loh}{figure}{桃 \dpy{tao2}}

\begin{EntryWithPhonetic}{桃}{tao2}{10}{⽊}[HSK 5]
  \definition*{s.}{Sobrenome: Tao}
  \definition[个,箱,袋,斤,棵,种]{s.}{pêssego | em forma de pêssego | pessegueiro}
\end{EntryWithPhonetic}

\begin{EntryWithPhonetic}{桃花}{tao2hua1}{10,7}{⽊,⾋}[HSK 5]
  \definition[朵,枝,株]{s.}{Figurativo: caso amoroso | flor de pessegueiro}
\end{EntryWithPhonetic}

\begin{EntryWithPhonetic}{桃树}{tao2shu4}{10,9}{⽊,⽊}[HSK 5]
  \definition[棵,株]{s.}{pêssego (árvore) | pessegueiro; pêssegos}
\end{EntryWithPhonetic}

%%%%%%%%%% 陶 %%%%%%%%%%
\subsection*{陶}\addcontentsline{loh}{figure}{陶 \dpy{tao2}}

\begin{EntryWithPhonetic}{陶}{tao2}{10}{⾩}
  \definition*{s.}{Sobrenome: Tao}
  \definition{adj.}{satisfeito; feliz | de barro; feito de argila}
  \definition{s.}{cerâmica; louça}
  \definition{v.}{fazer cerâmica (louça de barro) | cultivar; moldar; educar}
  \seeref{yao2}
\end{EntryWithPhonetic}

\begin{EntryWithPhonetic}{陶瓷}{tao2ci2}{10,10}{⾩,⽡}[HSK 7-9]
  \definition[件,套,个]{s.}{cerâmica; louça e porcelana; termo geral para cerâmica e porcelana}
\end{EntryWithPhonetic}

\begin{EntryWithPhonetic}{陶冶}{tao2ye3}{10,7}{⾩,⼎}[HSK 7-9]
  \definition{v.}{fazer cerâmica e fundir metal | exercer uma influência favorável (no caráter de uma pessoa, etc.); moldar}
  \synonymref{锻炼}{duan4lian4}
  \synonymref{训练}{xun4lian4}
\end{EntryWithPhonetic}

\begin{EntryWithPhonetic}{陶醉}{tao2zui4}{10,15}{⾩,⾣}[HSK 7-9]
  \definition{v.}{deleitar"-se com; embriagar"-se (de felicidade, etc.); estar imerso em um determinado estado ou sentimento}
  \synonymref{沉迷}{chen2mi2}
  \synonymref{迷恋}{mi2lian4}
\end{EntryWithPhonetic}

%%%%%%%%%% 淘 %%%%%%%%%%
\subsection*{淘}\addcontentsline{loh}{figure}{淘 \dpy{tao2}}

\begin{EntryWithPhonetic}{淘}{tao2}{11}{⽔}[HSK 7-9]
  \definition{adj.}{travesso}
  \definition{v.}{lavar em uma bacia ou cesto; enxaguar | limpar; dragar; retirar com concha; recolher; retirar esgoto, etc. | causar problemas; ser um fardo para a mente; perturbar o cérebro com algo}
  \seealsoref{掏}{tao1}
\end{EntryWithPhonetic}

\begin{EntryWithPhonetic}{淘气}{tao2/qi4}{11,4}{⽔,⽓}[HSK 7-9]
  \definition*{v.+compl.}{Dialeto: ficar com raiva; perder a paciência}
  \definition{adj.}{travesso; malicioso; descreve uma criança como particularmente brincalhona e travessa}
  \synonymref{调皮}{tiao2pi2}
  \synonymref{顽皮}{wan2pi2}
  \antonymref{老实}{lao3shi5}
  \antonymref{听话}{ting1/hua4}
\end{EntryWithPhonetic}

\begin{EntryWithPhonetic}{淘汰}{tao2tai4}{11,7}{⽔,⽔}[HSK 7-9]
  \definition{v.}{1. Eliminar por meio de seleção ou competição
Ao selecionar, você pode eliminar os itens ruins ou inadequados.}
  \synonymref{减少}{jian3shao3}
  \synonymref{剔除}{ti1chu2}
  \antonymref{达标}{da2biao1}
  \antonymref{合格}{he2ge2}
  \antonymref{挑选}{tiao1xuan3}
  \antonymref{通过}{tong1guo4}
  \antonymref{选择}{xuan3ze2}
  \antonymref{引进}{yin3jin4}
\end{EntryWithPhonetic}

%%%%%%%%%% 讨 %%%%%%%%%%
\subsection*{讨}\addcontentsline{loh}{figure}{讨 \dpy{tao3}}

\begin{EntryWithPhonetic}{讨}{tao3}{5}{⾔}[HSK 7-9]
  \definition{v.}{enviar forças armadas para suprimir; enviar uma expedição punitiva contra; enviar exército ou despachar tropas para suprimir ou atacar | denunciar; condenar; censurar | exigir; pedir; implorar por | casar (com uma mulher) | incorrer; convidar | discutir; estudar | provocar; cortejar}
\end{EntryWithPhonetic}

\begin{EntryWithPhonetic}{讨好}{tao3/hao3}{5,6}{⾔,⼥}[HSK 7-9]
  \definition{v.+compl.}{bajular; tentar agradar; bajular; tentar ganhar a simpatia de; (agradar aos outros) lisonjear os outros e obter seu favor ou elogio | obter um bom resultado; ser recompensado com um resultado frutífero; (bajulação) produzir bons resultados (frequentemente usada em contextos negativos)}
  \synonymref{取悦}{qu3yue4}
\end{EntryWithPhonetic}

\begin{EntryWithPhonetic}{讨价还价}{tao3jia4-huan2jia4}{5,6,7,6}{⾔,⼈,⾡,⼈}[HSK 7-9]
  \definition{expr.}{regatear o preço; barganhar; essa metáfora descreve o ato de fazer várias exigências e pechinchar sobre cada pequeno detalhe ao aceitar uma tarefa ou negociar; também pode ser descrita como barganha}
\end{EntryWithPhonetic}

\begin{EntryWithPhonetic}{讨论}{tao3lun4}{5,6}{⾔,⾔}[HSK 2]
  \definition{v.}{discutir; conversar sobre; trocar opiniões ou debater as questões levantadas}
\end{EntryWithPhonetic}

\begin{EntryWithPhonetic}{讨人喜欢}{tao3 ren2 xi3huan5}{5,2,12,6}{⾔,⼈,⼝,⽋}[HSK 7-9]
  \definition{adj.}{encantador}
  \definition{v.}{atrair o afeto das pessoas}
\end{EntryWithPhonetic}

\begin{EntryWithPhonetic}{讨生活}{tao3sheng1huo2}{5,5,9}{⾔,⽣,⽔}
  \definition{v.}{ganhar a vida}
\end{EntryWithPhonetic}

\begin{EntryWithPhonetic}{讨厌}{tao3/yan4}{5,6}{⾔,⼚}[HSK 5]
  \definition{adj.}{desagradável; repugnante; repulsivo; irritante; incômodo}
  \definition{v.+compl.}{odiar; não gostar; sentir repulsa por}
\end{EntryWithPhonetic}

%%%%%%%%%% 套 %%%%%%%%%%
\subsection*{套}\addcontentsline{loh}{figure}{套 \dpy{tao4}}

\begin{EntryWithPhonetic}{套}{tao4}{10}{⼤}[HSK 2]
  \definition{clas.}{usado para coisas agrupadas: conjuntos, coleções, séries, etc.}
  \definition{s.}{estojo; capa; bainha | local onde o rio ou a cordilheira faz uma curva (usado principalmente em nomes de lugares) | enchimento de algodão em roupas e edredons | arreios; corda para amarrar animais | nó; laço; um objeto circular feito com corda ou algo semelhante | cortersia; convenção; conversa fiada; métodos repetitivos | armadilha; truque; conspiração}
  \definition{v.}{sobrepor; interligar | deslizar sobre; cobrir por fora | atrelar; engatar; usar um cinto de segurança | copiar; imitar; seguir o modelo de | extrair; induzir a falar; persuadir alguém a revelar um segredo; induzir; provocar | tentar vencer; aproximar-se de; aproximar-se intencionalmente de outras pessoas para algum propósito | fazer a rosca de um parafuso; usar um macho de rosca ou uma chave de rosca para fazer roscas}
\end{EntryWithPhonetic}

\begin{EntryWithPhonetic}{套餐}{tao4can1}{10,16}{⼤,⾷}[HSK 4]
  \definition{s.}{combo; pacote de produtos; pacote de serviços; metaforicamente, bens ou projetos que são combinados e levados ao mercado | refeição preparada; pacotes de refeições completos}
\end{EntryWithPhonetic}

\begin{EntryWithPhonetic}{套问}{tao4wen4}{10,6}{⼤,⾨}
  \definition{s.}{retórica}
  \definition{v.}{descobrir por meio de questionamento indireto diplomático}
\end{EntryWithPhonetic}

%%%%%%%%%% 特 %%%%%%%%%%
\subsection*{特}\addcontentsline{loh}{figure}{特 \dpy{te4}}

\begin{EntryWithPhonetic}{特}{te4}{10}{⽜}[HSK 6]
  \definition{adj.}{especial; incomum; particular; excepcional; diferente do geral | especial; solteiro; solitário}
  \definition{adv.}{muito; extremamente | especialmente; para um propósito especial |mas; somente}
  \definition{clas.}{TEX; abreviação para unidades de medida como TEX; a unidade de medida TEX indica a espessura de um fio têxtil através do seu peso}
  \definition{s.}{espião; agente secreto}
\end{EntryWithPhonetic}

\begin{EntryWithPhonetic}{特别}{te4bie2}{10,7}{⽜,⼑}[HSK 2]
  \definition{adj.}{especial; particular; fora do comum; diferente dos outros, com características próprias}
  \definition{adv.}{especialmente; particularmente | ainda mais; em particular; frequentemente usado com 是 | especialmente; deliberadamente; para um propósito específico}
  \seealsoref{是}{shi4}
\end{EntryWithPhonetic}

\begin{EntryWithPhonetic}{特别快车}{te4bie2 kuai4che1}{10,7,7,4}{⽜,⼑,⼼,⾞}
  \definition{s.}{trem expresso; expresso; expresso especial; refere"-se a trens de passageiros que param em menos estações e têm menor tempo de viagem do que trens expressos diretos}
\end{EntryWithPhonetic}

\begin{EntryWithPhonetic}{特产}{te4chan3}{10,6}{⽜,⼇}[HSK 7-9]
  \definition{s.}{produto nativo; especialidade; produto local especial; produtos que são exclusivos de um determinado local ou país, ou que são particularmente famosos}
\end{EntryWithPhonetic}

\begin{EntryWithPhonetic}{特长}{te4chang2}{10,4}{⽜,⾧}[HSK 7-9]
  \definition[项,个]{s.}{talento; ponto forte; especialidade; aquilo em que alguém é habilidoso; habilidades especiais}
  \synonymref{爱好}{ai4hao4}
  \synonymref{拿手}{na2shou3}
  \synonymref{擅长}{shan4chang2}
  \synonymref{善于}{shan4yu2}
  \synonymref{特技}{te4ji4}
  \antonymref{缺陷}{que1xian4}
\end{EntryWithPhonetic}

\begin{EntryWithPhonetic}{特大}{te4da4}{10,3}{⽜,⼤}[HSK 6]
  \definition{adj.}{especialmente (excepcionalmente) grande; o mais}
\end{EntryWithPhonetic}

\begin{EntryWithPhonetic}{特地}{te4di4}{10,6}{⽜,⼟}[HSK 6]
  \definition{adv.}{especialmente; propositalmente; para um propósito especial}
\end{EntryWithPhonetic}

\begin{EntryWithPhonetic}{特点}{te4dian3}{10,9}{⽜,⽕}[HSK 2]
  \definition[个,大]{s.}{característica; peculiaridade; traço distintivo; a singularidade de uma pessoa ou coisa}
\end{EntryWithPhonetic}

\begin{EntryWithPhonetic}{特定}{te4ding4}{10,8}{⽜,⼧}[HSK 5]
  \definition{adj.}{particular; específico; especialmente designado | dado; especificado; específico; uma pessoa específica, um determinado momento, lugar, ambiente, etc.}
\end{EntryWithPhonetic}

\begin{EntryWithPhonetic}{特技}{te4ji4}{10,7}{⽜,⼿}
  \definition{s.}{efeito especial | dublê}
\end{EntryWithPhonetic}

\begin{EntryWithPhonetic}{特价}{te4jia4}{10,6}{⽜,⼈}[HSK 4]
  \definition{s.}{oferta especial; preço de barganha; preço especial reduzido}
\end{EntryWithPhonetic}

\begin{EntryWithPhonetic}{特快}{te4kuai4}{10,7}{⽜,⼼}[HSK 6]
  \definition{adj.}{expresso (trem, entrega etc.)}
  \definition{s.}{trem expresso; abreviação de 特别快车}
  \seealsoref{特别快车}{te4bie2 kuai4che1}
  \antonymref{普快}{pu3 kuai4}
\end{EntryWithPhonetic}

\begin{EntryWithPhonetic}{特例}{te4li4}{10,8}{⽜,⼈}[HSK 7-9]
  \definition{s.}{exemplo isolado | caso especial}
  \synonymref{特制}{te4zhi4}
  \antonymref{惯例}{guan4li4}
\end{EntryWithPhonetic}

\begin{EntryWithPhonetic}{特权}{te4quan2}{10,6}{⽜,⽊}[HSK 7-9]
  \definition[种]{s.}{privilégio; prerrogativa | direito e privilégio especiais}
\end{EntryWithPhonetic}

\begin{EntryWithPhonetic}{特色}{te4se4}{10,6}{⽜,⾊}[HSK 3]
  \definition{s.}{característica; característica distintiva; a cor única, estilo, etc. de um objeto}
\end{EntryWithPhonetic}

\begin{EntryWithPhonetic}{特殊}{te4shu1}{10,10}{⽜,⽍}[HSK 4]
  \definition{adj.}{especial; particular; peculiar; excepcional; incomum}
\end{EntryWithPhonetic}

\begin{EntryWithPhonetic}{特性}{te4xing4}{10,8}{⽜,⼼}[HSK 5]
  \definition[种,个]{s.}{propriedade específica (ou característica) | característica; sabores | propriedade}
\end{EntryWithPhonetic}

\begin{EntryWithPhonetic}{特邀}{te4yao1}{10,16}{⽜,⾡}[HSK 7-9]
  \definition{v.}{convidar especialmente (um convidado para um evento)}
  \synonymref{敬请}{jing4qing3}
\end{EntryWithPhonetic}

\begin{EntryWithPhonetic}{特意}{te4yi4}{10,13}{⽜,⼼}[HSK 6]
  \definition{adv.}{especialmente; para um propósito especial}
\end{EntryWithPhonetic}

\begin{EntryWithPhonetic}{特有}{te4you3}{10,6}{⽜,⽉}[HSK 5]
  \definition{adj.}{específico; peculiar; característico; único; exclusivo; especial}
\end{EntryWithPhonetic}

\begin{EntryWithPhonetic}{特征}{te4zheng1}{10,8}{⽜,⼻}[HSK 4]
  \definition[个,种]{s.}{característica; aparência ou o fenômeno característico de uma pessoa ou coisa que pode ser visto de fora}
\end{EntryWithPhonetic}

\begin{EntryWithPhonetic}{特制}{te4zhi4}{10,8}{⽜,⼑}[HSK 7-9]
  \definition{v.}{fabricar especialmente; fazer sob encomenda}
  \synonymref{特别}{te4bie2}
  \synonymref{特例}{te4li4}
  \synonymref{特殊}{te4shu1}
  \antonymref{通用}{tong1yong4}
\end{EntryWithPhonetic}

\begin{EntryWithPhonetic}{特质}{te4zhi4}{10,8}{⽜,⾙}[HSK 7-9]
  \definition[种]{s.}{qualidade especial; característica; qualidades ou propriedades únicas}
  \synonymref{属性}{shu3xing4}
  \synonymref{特点}{te4dian3}
  \synonymref{特色}{te4se4}
  \synonymref{特性}{te4xing4}
  \synonymref{特征}{te4zheng1}
  \antonymref{共性}{gong4xing4}
\end{EntryWithPhonetic}

%%%%%%%%%% 疼 %%%%%%%%%%
\subsection*{疼}\addcontentsline{loh}{figure}{疼 \dpy{teng2}}

\begin{EntryWithPhonetic}{疼}{teng2}{10}{⽧}[HSK 2]
  \definition{adj.}{dolorido; doído; sensação de extremo desconforto causada por ferimentos, doenças, etc.}
  \definition{v.}{ferir; machucar | adorar; amar profundamente; gostar muito; cuidar}
\end{EntryWithPhonetic}

\begin{EntryWithPhonetic}{疼痛}{teng2tong4}{10,12}{⽧,⽧}[HSK 6]
  \definition[阵,种]{s.}{dor; sofrimento; ferimento; descreve a sensação de dor causada por lesão ou doença}
\end{EntryWithPhonetic}

%%%%%%%%%% 腾 %%%%%%%%%%
\subsection*{腾}\addcontentsline{loh}{figure}{腾 \dpy{teng2}}

\begin{EntryWithPhonetic}{腾}{teng2}{13}{⾁}[HSK 7-9]
  \definition*{s.}{Sobrenome: Teng}
  \definition{v.}{galopar; saltar; trotar | subir; planar | abrir espaço; desocupar; esvaziar | ascender; subir aos céus | excitar; agitar; mexer}
  \definition{v.aux.}{usado após certos verbos para indicar repetição}
\end{EntryWithPhonetic}

%%%%%%%%%% 藤 %%%%%%%%%%
\subsection*{藤}\addcontentsline{loh}{figure}{藤 \dpy{teng2}}

\begin{EntryWithPhonetic}{藤}{teng2}{18}{⾋}
  \definition*{s.}{Teng}
  \definition[根,条]{s.}{cana; vime | videira}
\end{EntryWithPhonetic}

\begin{EntryWithPhonetic}{藤椅}{teng2yi3}{18,12}{⾋,⽊}[HSK 7-9]
  \definition{s.}{cadeira de vime}
\end{EntryWithPhonetic}

%%%%%%%%%% 剔 %%%%%%%%%%
\subsection*{剔}\addcontentsline{loh}{figure}{剔 \dpy{ti1}}

\begin{EntryWithPhonetic}{剔}{ti1}{10}{⼑}
  \definition{s.}{traço ascendente (em caracteres chineses)}
  \definition{v.}{limpar com um instrumento pontiagudo; cutucar | selecionar e descartar; rejeitar; eliminar | raspar a carne do osso | escolher; selecionar de dentro para fora}
\end{EntryWithPhonetic}

\begin{EntryWithPhonetic}{剔除}{ti1chu2}{10,9}{⼑,⾩}[HSK 7-9]
  \definition{v.}{rejeitar; excluir; remover; livrar"-se de; significa remover, excluir ou limpar}
  \synonymref{淘汰}{tao2tai4}
  \antonymref{吸收}{xi1shou1}
\end{EntryWithPhonetic}

%%%%%%%%%% 梯 %%%%%%%%%%
\subsection*{梯}\addcontentsline{loh}{figure}{梯 \dpy{ti1}}

\begin{EntryWithPhonetic}{梯}{ti1}{11}{⽊}
  \definition*{s.}{Sobrenome: Ti}
  \definition{adj.}{em forma de escada; em socalcos}
  \definition[个]{s.}{escada; degrau; socalco (são plataformas niveladas, semelhantes a degraus, cortadas em encostas de morros para permitir o cultivo agrícola e evitar a erosão do solo)}
\end{EntryWithPhonetic}

\begin{EntryWithPhonetic}{梯恩梯}{ti1'en1ti1}{11,10,11}{⽊,⼼,⽊}
  \definition{s.}{Empréstimo linguístico: TNT, trinitrotolueno}
\end{EntryWithPhonetic}

\begin{EntryWithPhonetic}{梯子}{ti1zi5}{11,3}{⽊,⼦}[HSK 7-9]
  \definition[个,把]{s.}{escada; escada de mão; ferramentas para facilitar o acesso das pessoas}
\end{EntryWithPhonetic}

%%%%%%%%%% 踢 %%%%%%%%%%
\subsection*{踢}\addcontentsline{loh}{figure}{踢 \dpy{ti1}}

\begin{EntryWithPhonetic}{踢}{ti1}{15}{⾜}[HSK 6]
  \definition{v.}{chutar | jogar (por exemplo, futebol)}
\end{EntryWithPhonetic}

\begin{EntryWithPhonetic}{踢爆}{ti1bao4}{15,19}{⾜,⽕}
  \definition{v.}{expor | revelar}
\end{EntryWithPhonetic}

\begin{EntryWithPhonetic}{踢蹋舞}{ti1ta4wu3}{15,17,14}{⾜,⾜,⾇}
  \definition{s.}{sapateado | passo de dança}
\end{EntryWithPhonetic}

%%%%%%%%%% 提 %%%%%%%%%%
\subsection*{提}\addcontentsline{loh}{figure}{提 \dpy{ti2}}

\begin{EntryWithPhonetic}{提}{ti2}{12}{⼿}[HSK 2]
  \definition*{s.}{Sobrenome: Ti}
  \definition{s.}{concha; utensílio para servir óleo ou vinho | traço ascendente (em caracteres chineses)}
  \definition{v.}{carregar (na mão, com o braço para baixo) ; segurar com as mãos para baixo | elevar; levantar; promover | avançar; antecipar uma data; mudar para uma data anterior; adiar o prazo previsto | levantar; apresentar; indicar ou citar | extrair; retirar (tirar) | (prisioneiros) trazer; entregar | mencionar; referir-se a; abordar}
\end{EntryWithPhonetic}

\begin{EntryWithPhonetic}{提拔}{ti2ba2}{12,8}{⼿,⼿}[HSK 7-9]
  \definition{v.}{1. promover; favorecer
Selecionar pessoal para assumir cargos mais importantes.}
  \synonymref{教育}{jiao4yu4}
  \synonymref{培育}{pei2yu4}
  \synonymref{提升}{ti2sheng1}
  \synonymref{选拔}{xuan3ba2}
\end{EntryWithPhonetic}

\begin{EntryWithPhonetic}{提倡}{ti2chang4}{12,10}{⼿,⼈}[HSK 5]
  \definition{v.}{promover; incentivar; recomendar; apresentar as vantagens de algo para incentivar as pessoas a usá-lo ou implementá-lo}
\end{EntryWithPhonetic}

\begin{EntryWithPhonetic}{提出}{ti2 chu1}{12,5}{⼿,⼐}[HSK 2]
  \definition{v.}{levantar; propor; apresentar; expressar seus desejos, ideias, sugestões, etc. por meio de palavras ou textos}
\end{EntryWithPhonetic}

\begin{EntryWithPhonetic}{提到}{ti2dao4}{12,8}{⼿,⼑}[HSK 2]
  \definition{v.}{mencionar; referir-se a; levantar (assunto)}
\end{EntryWithPhonetic}

\begin{EntryWithPhonetic}{提高}{ti2/gao1}{12,10}{⼿,⾼}[HSK 2]
  \definition{v.+compl.}{elevar; aprimorar; aumentar; melhorar a posição, o nível, a quantidade, a qualidade e outros aspectos em relação ao estado original}
\end{EntryWithPhonetic}

\begin{EntryWithPhonetic}{提供}{ti2gong1}{12,8}{⼿,⼈}[HSK 4]
  \definition{v.}{oferecer; fornecer; suprir; prover; proporcionar}
\end{EntryWithPhonetic}

\begin{EntryWithPhonetic}{提及}{ti2ji2}{12,3}{⼿,⼃}
  \definition{v.}{mencionar | levantar (um assunto) | chamar a atenção de alguém}
\end{EntryWithPhonetic}

\begin{EntryWithPhonetic}{提交}{ti2jiao1}{12,6}{⼿,⼇}[HSK 6]
  \definition{v.}{referir-se a; submeter (um problema, etc.) a; enviar questões que precisam ser discutidas, decididas ou tratadas para agências ou reuniões relevantes}
\end{EntryWithPhonetic}

\begin{EntryWithPhonetic}{提炼}{ti2lian4}{12,9}{⼿,⽕}[HSK 7-9]
  \definition{v.}{extrair e purificar; abstrair; refinar | extrair}
  \antonymref{融合}{rong2he2}
\end{EntryWithPhonetic}

\begin{EntryWithPhonetic}{提名}{ti2/ming2}{12,6}{⼿,⼝}[HSK 7-9]
  \definition{v.+compl.}{indicar; nomear; propor pessoas ou coisas que provavelmente serão eleitas em uma seleção ou eleição}
\end{EntryWithPhonetic}

\begin{EntryWithPhonetic}{提起}{ti2qi3}{12,10}{⼿,⾛}[HSK 5]
  \definition{v.}{mencionar; falar sobre; abordar | levantar; despertar; estimular; revigorar; alegrar/animar | iniciar; instituir; propor | levantar; pegar}
\end{EntryWithPhonetic}

\begin{EntryWithPhonetic}{提前}{ti2qian2}{12,9}{⼿,⼑}[HSK 3]
  \definition{adv.}{antecipadamente; faça uma coisa antes de fazer outra}
  \definition{v.}{avançar; adiantar; mudar para uma data anterior; trazer para frente}
\end{EntryWithPhonetic}

\begin{EntryWithPhonetic}{提升}{ti2sheng1}{12,4}{⼿,⼗}[HSK 6]
  \definition{v.}{promover; avançar; melhorar (posição, grau, qualidade, etc.) | içar; elevar; transportar (minerais, materiais, etc.) para um local mais alto usando um guincho, etc.}
\end{EntryWithPhonetic}

\begin{EntryWithPhonetic}{提示}{ti2shi4}{12,5}{⼿,⽰}[HSK 5]
  \definition[个]{s.}{dica; lembrete; pistas ou informações fornecidas para chamar a atenção, fazer com que a outra pessoa pense ou compreenda}
  \definition{v.}{solicitar; lembrar; indicar; alertar; levantar questões que o outro não tenha pensado ou não tenha imaginado, para chamar a atenção dele}
\end{EntryWithPhonetic}

\begin{EntryWithPhonetic}{提速}{ti2/su4}{12,10}{⼿,⾡}[HSK 7-9]
  \definition{v.+compl.}{acelerar; aumentar a velocidade | para aumentar a velocidade de cruzeiro especificada | ganhar velocidade}
  \synonymref{加速}{jia1su4}
\end{EntryWithPhonetic}

\begin{EntryWithPhonetic}{提问}{ti2wen4}{12,6}{⼿,⾨}[HSK 3]
  \definition{v.}{\emph{quiz}; fazer uma pergunta; colocar questões para}
\end{EntryWithPhonetic}

\begin{EntryWithPhonetic}{提心吊胆}{ti2xin1-diao4dan3}{12,4,6,9}{⼿,⼼,⼝,⾁}[HSK 7-9]
  \definition{expr.}{nervoso; com o coração na boca; em constante medo; extremamente preocupado; em estado de ansiedade; estar em suspense; descreve um estado de grande preocupação ou medo}
  \seealsoref{悬心吊胆}{xuan2xin1-diao4dan3}
\end{EntryWithPhonetic}

\begin{EntryWithPhonetic}{提醒}{ti2/xing3}{12,16}{⼿,⾣}[HSK 4]
  \definition{v.+compl.}{alertar; avisar; advertir; lembrar; apontar para ou chamar a atenção para}
\end{EntryWithPhonetic}

\begin{EntryWithPhonetic}{提议}{ti2yi4}{12,5}{⼿,⾔}[HSK 7-9]
  \definition[项,个]{s.}{proposta; moção}
  \definition{v.}{propor; sugerir; ao discutir assuntos, propor ideias para que todos possam debatê-las}
  \synonymref{倡导}{chang4dao3}
  \synonymref{倡议}{chang4yi4}
  \synonymref{发起}{fa1qi3}
  \synonymref{建议}{jian4yi4}
  \synonymref{提出}{ti2 chu1}
  \synonymref{提倡}{ti2chang4}
\end{EntryWithPhonetic}

\begin{EntryWithPhonetic}{提早}{ti2zao3}{12,6}{⼿,⽇}[HSK 7-9]
  \definition{v.}{antecipar"-se ao horário previsto; chegar mais cedo do que o planejado ou esperado}
  \synonymref{赶早}{gan3zao3}
  \synonymref{尽快}{jin3kuai4}
  \synonymref{尽早}{jin3zao3}
  \synonymref{提前}{ti2qian2}
  \antonymref{顺延}{shun4yan2}
  \antonymref{推迟}{tui1chi2}
\end{EntryWithPhonetic}

%%%%%%%%%% 题 %%%%%%%%%%
\subsection*{题}\addcontentsline{loh}{figure}{题 \dpy{ti2}}

\begin{EntryWithPhonetic}{题}{ti2}{15}{⾴}[HSK 2]
  \definition*{s.}{Sobrenome: Ti}
  \definition[个,道]{s.}{tópico; título; assunto; problema; frases que indicam o conteúdo de poemas ou discursos | questão; questões que devem ser respondidas durante os exercícios ou exames | antigamente, referia"-se à testa}
  \definition{v.}{inscrever; escrever; assinar}
\end{EntryWithPhonetic}

\begin{EntryWithPhonetic}{题材}{ti2cai2}{15,7}{⾴,⽊}[HSK 5]
  \definition{s.}{tema; assunto; material que compõe as obras literárias e artísticas, ou seja, os eventos ou fenômenos da vida descritos concretamente nas obras}
\end{EntryWithPhonetic}

\begin{EntryWithPhonetic}{题目}{ti2mu4}{15,5}{⾴,⽬}[HSK 3]
  \definition[个,道]{s.}{título; assunto; tópico; o título de um poema ou discurso | quebra-cabeça; problema de exercício; questões a serem respondidas em exercícios ou provas}
\end{EntryWithPhonetic}

%%%%%%%%%% 体 %%%%%%%%%%
\subsection*{体}\addcontentsline{loh}{figure}{体 \dpy{ti3}}

\begin{EntryWithPhonetic}{体}{ti3}{7}{⼈}
  \definition{s.}{corpo; parte do corpo | substância; objeto; estado de uma substância | estilo; forma | sistema | estilo de caligrafia | tipo de letra; fonte | Linguística: aspecto (de um verbo) | estrutura; a forma escrita do texto; o gênero da obra}
  \definition{v.}{fazer ou vivenciar algo pessoalmente | colocar"-se na posição de outro; colocar"-se mentalmente na posição do outro; colocar"-se no lugar do outro}
\end{EntryWithPhonetic}

\begin{EntryWithPhonetic}{体操}{ti3cao1}{7,16}{⼈,⼿}[HSK 4]
  \definition{s.}{ginástica; esportes, exercícios ou performances de vários movimentos, sem armas ou com o auxílio de determinados equipamentos}
\end{EntryWithPhonetic}

\begin{EntryWithPhonetic}{体会}{ti3hui4}{7,6}{⼈,⼈}[HSK 3]
  \definition[个,些,种]{s.}{conhecimento; compreensão; experiência pessoal}
  \definition{v.}{perceber; saber (ou aprender) com a experiência}
\end{EntryWithPhonetic}

\begin{EntryWithPhonetic}{体积}{ti3ji1}{7,10}{⼈,⽲}[HSK 5]
  \definition[个]{s.}{volume; quantidade; o tamanho do espaço ocupado pelo objeto}
\end{EntryWithPhonetic}

\begin{EntryWithPhonetic}{体检}{ti3jian3}{7,11}{⼈,⽊}[HSK 4]
  \definition{v.}{fazer um exame médico}
\end{EntryWithPhonetic}

\begin{EntryWithPhonetic}{体力}{ti3li4}{7,2}{⼈,⼒}[HSK 5]
  \definition{s.}{força física; vigor físico (ou corporal); a força do corpo humano para sustentar suas próprias atividades}
\end{EntryWithPhonetic}

\begin{EntryWithPhonetic}{体谅}{ti3liang4}{7,10}{⼈,⾔}[HSK 7-9]
  \definition{v.}{levar em consideração; demonstrar compreensão e simpatia por; colocar-se no lugar deles e ser compreensivo}
  \synonymref{宽容}{kuan1rong2}
  \synonymref{谅解}{liang4jie3}
  \synonymref{体贴}{ti3tie1}
  \synonymref{原谅}{yuan2liang4}
\end{EntryWithPhonetic}

\begin{EntryWithPhonetic}{体面}{ti3mian4}{7,9}{⼈,⾯}[HSK 7-9]
  \definition{adj.}{honroso; digno de crédito; respeitável | bonito; atraente}
  \definition{s.}{face; dignidade; decoro; conduta adequada; \emph{status}}
  \synonymref{场合}{chang3he2}
  \synonymref{场面}{chang3mian4}
  \synonymref{得体}{de2ti3}
  \synonymref{好看}{hao3kan4}
  \synonymref{合适}{he2shi4}
  \synonymref{局面}{ju2mian4}
  \synonymref{美观}{mei3guan1}
  \synonymref{面子}{mian4zi5}
  \antonymref{难堪}{nan2kan1}
  \antonymref{难看}{nan2kan4}
  \antonymref{难听}{nan2ting1}
\end{EntryWithPhonetic}

\begin{EntryWithPhonetic}{体内}{ti3nei4}{7,4}{⼈,⼌}
  \definition{adj.}{dentro do corpo | \emph{in vivo} (versus \emph{in vitro} | interno a}
\end{EntryWithPhonetic}

\begin{EntryWithPhonetic}{体能}{ti3neng2}{7,10}{⼈,⾁}[HSK 7-9]
  \definition{s.}{resistência (física) | capacidade física}
\end{EntryWithPhonetic}

\begin{EntryWithPhonetic}{体贴}{ti3tie1}{7,9}{⼈,⾙}[HSK 7-9]
  \definition{adj.}{atencioso; considere com atenção os sentimentos e as circunstâncias das outras pessoas e ofereça"-lhes cuidado e consideração}
  \definition{v.}{cuidar de; demonstrar consideração por; dedicar toda a atenção a}
  \synonymref{爱护}{ai4hu4}
  \synonymref{关爱}{guan1'ai4}
  \synonymref{关怀}{guan1huai2}
  \synonymref{关心}{guan1xin1}
  \synonymref{关注}{guan1zhu4}
  \synonymref{谅解}{liang4jie3}
  \synonymref{体谅}{ti3liang4}
  \synonymref{温柔}{wen1rou2}
  \synonymref{照顾}{zhao4gu5}
  \antonymref{冷淡}{leng3dan4}
  \antonymref{虐待}{nve4dai4}
\end{EntryWithPhonetic}

\begin{EntryWithPhonetic}{体温}{ti3wen1}{7,12}{⼈,⽔}[HSK 7-9]
  \definition{s.}{temperatura corporal}
  \synonymref{温度}{wen1du4}
\end{EntryWithPhonetic}

\begin{EntryWithPhonetic}{体系}{ti3xi4}{7,7}{⼈,⽷}[HSK 7-9]
  \definition[个,套]{s.}{configuração; sistema; um todo formado pela interconexão de muitas coisas ou ideias relacionadas}
  \synonymref{编制}{bian1zhi4}
  \synonymref{机制}{ji1zhi4}
  \synonymref{体制}{ti3zhi4}
  \synonymref{系列}{xi4lie4}
  \synonymref{系统}{xi4tong3}
\end{EntryWithPhonetic}

\begin{EntryWithPhonetic}{体现}{ti3xian4}{7,8}{⼈,⾒}[HSK 3]
  \definition{v.}{refletir; incorporar; encarnar; uma certa qualidade ou fenômeno se manifesta especificamente em uma determinada coisa}
\end{EntryWithPhonetic}

\begin{EntryWithPhonetic}{体验}{ti3yan4}{7,10}{⼈,⾺}[HSK 3]
  \definition[种]{s.}{experiência; a sensação adquirida pela experiência pessoal}
  \definition{v.}{aprender através da prática; aprender através da experiência pessoal; entender as coisas através da experiência pessoal}
\end{EntryWithPhonetic}

\begin{EntryWithPhonetic}{体育}{ti3yu4}{7,8}{⼈,⾁}[HSK 2]
  \definition{s.}{cultura física; treinamento físico; educação cuja principal tarefa é desenvolver a capacidade física e fortalecer a constituição física, alcançada através da participação em várias atividades esportivas | esportes; atividades esportivas; refere"-se a esportes}
\end{EntryWithPhonetic}

\begin{EntryWithPhonetic}{体育场}{ti3yu4chang3}{7,8,6}{⼈,⾁,⼟}[HSK 2]
  \definition[个,座]{s.}{estádio; campo esportivo; espaço ao ar livre para a prática de exercícios físicos ou competições esportivas}
\end{EntryWithPhonetic}

\begin{EntryWithPhonetic}{体育馆}{ti3yu4guan3}{7,8,11}{⼈,⾁,⾷}[HSK 2]
  \definition[个,座,家]{s.}{ginásio; locais esportivos ou competições em ambientes fechados geralmente têm arquibancadas fixas}
\end{EntryWithPhonetic}

\begin{EntryWithPhonetic}{体制}{ti3zhi4}{7,8}{⼈,⼑}[HSK 7-9]
  \definition{s.}{estrutura; sistema (de organização); sistemas organizacionais de agências governamentais, empresas, instituições públicas, etc. | forma; estilo (de escrita literária); o gênero e a estrutura das obras de arte}
  \synonymref{机制}{ji1zhi4}
  \synonymref{体系}{ti3xi4}
\end{EntryWithPhonetic}

\begin{EntryWithPhonetic}{体质}{ti3zhi4}{7,8}{⼈,⾙}[HSK 7-9]
  \definition{s.}{compleição física; constituição; nível de saúde humana e adaptabilidade ao ambiente externo}
  \synonymref{身体}{shen1ti3}
  \synonymref{提高}{ti2/gao1}
  \synonymref{增强}{zeng1qiang2}
\end{EntryWithPhonetic}

\begin{EntryWithPhonetic}{体重}{ti3zhong4}{7,9}{⼈,⾥}[HSK 4]
  \definition{s.}{peso corporal}
\end{EntryWithPhonetic}

%%%%%%%%%% 剃 %%%%%%%%%%
\subsection*{剃}\addcontentsline{loh}{figure}{剃 \dpy{ti4}}

\begin{EntryWithPhonetic}{剃}{ti4}{9}{⼑}[HSK 7-9]
  \definition{v.}{depilar; raspar; cortar; usar uma lâmina especial para raspar (cabelo, barba, etc.)}
\end{EntryWithPhonetic}

%%%%%%%%%% 替 %%%%%%%%%%
\subsection*{替}\addcontentsline{loh}{figure}{替 \dpy{ti4}}

\begin{EntryWithPhonetic}{替}{ti4}{12}{⽈}[HSK 4]
  \definition{prep.}{para; em nome de}
  \definition{s.}{decadência; declínio; enfraquecimento}
  \definition{v.}{substituir; substituir por; tomar o lugar de}
\end{EntryWithPhonetic}

\begin{EntryWithPhonetic}{替代}{ti4dai4}{12,5}{⽈,⼈}[HSK 4]
  \definition{v.}{substituir; suplantar}
\end{EntryWithPhonetic}

\begin{EntryWithPhonetic}{替换}{ti4huan4}{12,10}{⽈,⼿}[HSK 7-9]
  \definition{v.}{substituir; substituir por; deslocar; tomar o lugar de; alternar}
  \synonymref{撤换}{che4huan4}
  \synonymref{更换}{geng1huan4}
  \synonymref{交换}{jiao1huan4}
  \synonymref{替代}{ti4dai4}
  \antonymref{代替}{dai4ti4}
\end{EntryWithPhonetic}

\begin{EntryWithPhonetic}{替身}{ti4shen1}{12,7}{⽈,⾝}[HSK 7-9]
  \definition{s.}{bode expiatório | dublê de corpo | dublê}
  \definition{v.}{substituir; substituir alguém; ocupar o lugar de}
\end{EntryWithPhonetic}

%%%%%%%%%% 天 %%%%%%%%%%
\subsection*{天}\addcontentsline{loh}{figure}{天 \dpy{tian1}}

\begin{EntryWithPhonetic}{天}{tian1}{4}{⼤}[HSK 1]
  \definition*{s.}{Sobrenome: Tian}
  \definition{adj.}{localizado no topo; suspenso no ar | inato; natural}
  \definition{clas.}{usado para contar dias}
  \definition{s.}{céu; paraíso; espaço onde se encontram o sol, a lua e as estrelas | dia; as 24 horas do dia, às vezes referindo"-se especificamente ao período diurno | um período de tempo em um dia; em algum momento do dia | temporada; estação do ano | clima | natureza | Deus; céu; o criador | paraíso; refere"-se ao local onde residem os deuses, budas e imortais}
\end{EntryWithPhonetic}

\begin{EntryWithPhonetic}{天才}{tian1cai2}{4,3}{⼤,⼿}[HSK 5]
  \definition{adj.}{talentoso | superdotado | genial}
  \definition[个,位,名]{s.}{dom; genialidade; talento natural; inteligência e sabedoria acima da média}
\end{EntryWithPhonetic}

\begin{EntryWithPhonetic}{天长地久}{tian1chang2-di4jiu3}{4,4,6,3}{⼤,⾧,⼟,⼃}[HSK 7-9]
  \definition{expr.}{eterno; perpétuo; que dura tanto quanto o céu e a terra; enquanto houver céu e terra, descrevendo algo que é eterno e imutável (frequentemente referindo-se ao amor)}
\end{EntryWithPhonetic}

\begin{EntryWithPhonetic}{天秤座}{tian1cheng4zuo4}{4,10,10}{⼤,⽲,⼴}
  \definition*{s.}{Libra (signo do zodíaco) | Astronomia: Constelação de Libra}
\end{EntryWithPhonetic}

\begin{EntryWithPhonetic}{天地}{tian1di4}{4,6}{⼤,⼟}[HSK 7-9]
  \definition{s.}{mundo; universo; céu e terra; Céu e Terra, o mundo natural e a sociedade | campo de atividade; âmbito de atuação; âmbito das atividades | situação difícil; situação ruim | 4. título de um periódico; palavra; coluna (em jornais, etc.); nomes usados ​​em jornais, revistas ou colunas}
\end{EntryWithPhonetic}

\begin{EntryWithPhonetic}{天鹅}{tian1'e2}{4,12}{⼤,⿃}[HSK 7-9]
  \definition{s.}{cisne}
\end{EntryWithPhonetic}

\begin{EntryWithPhonetic}{天分}{tian1fen4}{4,4}{⼤,⼑}[HSK 7-9]
  \definition{s.}{talento; dom natural; dons especiais}
  \synonymref{本性}{ben3xing4}
  \synonymref{天才}{tian1cai2}
  \synonymref{天赋}{tian1fu4}
  \synonymref{天生}{tian1sheng1}
  \synonymref{天性}{tian1xing4}
  \synonymref{先天}{xian1tian1}
  \synonymref{性格}{xing4ge2}
\end{EntryWithPhonetic}

\begin{EntryWithPhonetic}{天赋}{tian1fu4}{4,12}{⼤,⾙}[HSK 7-9]
  \definition[项,种]{adj.}{talento; habilidade inata; dom natural; dons; dotado naturalmente}
  \synonymref{天才}{tian1cai2}
  \synonymref{天分}{tian1fen4}
  \synonymref{天生}{tian1sheng1}
  \synonymref{天性}{tian1xing4}
  \synonymref{先天}{xian1tian1}
\end{EntryWithPhonetic}

\begin{EntryWithPhonetic}{天公}{tian1gong1}{4,4}{⼤,⼋}
  \definition{s.}{céu, paraíso | senhor do céu}
\end{EntryWithPhonetic}

\begin{EntryWithPhonetic}{天花板}{tian1hua1ban3}{4,7,8}{⼤,⾋,⽊}
  \definition{s.}{teto}
\end{EntryWithPhonetic}

\begin{EntryWithPhonetic}{天津}{tian1jin1}{4,9}{⼤,⽔}
  \definition*{s.}{Tianjin, um município no nordeste da China}
\end{EntryWithPhonetic}

\begin{EntryWithPhonetic}{天经地义}{tian1jing1-di4yi4}{4,8,6,3}{⼤,⽷,⼟,⼂}[HSK 7-9]
  \definition{expr.}{(em conformidade com) os princípios do céu e da terra; correto e apropriado; perfeitamente justificado; algo natural; lei do céu e princípio da terra; Figurativo: correto e próprio}
  \synonymref{理所当然}{li3suo3dang1ran2}
\end{EntryWithPhonetic}

\begin{EntryWithPhonetic}{天空}{tian1kong1}{4,8}{⼤,⽳}[HSK 3]
  \definition{s.}{o céu; o firmamento}
\end{EntryWithPhonetic}

\begin{EntryWithPhonetic}{天平}{tian1ping2}{4,5}{⼤,⼲}[HSK 7-9]
  \definition{s.}{balança; balanço analítico}
  \definition{s.}{Signo Libra (天秤座)}
  \seealsoref{天秤座}{tian1cheng4zuo4}
\end{EntryWithPhonetic}

\begin{EntryWithPhonetic}{天气}{tian1qi4}{4,4}{⼤,⽓}[HSK 1]
  \definition{s.}{clima, tempo; mudanças meteorológicas que ocorrem na atmosfera em uma determinada área e durante um determinado período de tempo, tais como temperatura, umidade, pressão atmosférica, precipitação, vento, nuvens, etc.}
\end{EntryWithPhonetic}

\begin{EntryWithPhonetic}{天桥}{tian1qiao2}{4,10}{⼤,⽊}[HSK 7-9]
  \definition{s.}{ponte elevada; ponte plataforma; passarela suspensa; faixa de pedestres elevada; pontes erguidas sobre ferrovias, rodovias, ruas, etc., para facilitar a circulação de pedestres | ponte (um instrumento semelhante a uma escada); um equipamento esportivo em formato de ponte de madeira, é alto e comprido, com escadas em ambas as extremidades}
\end{EntryWithPhonetic}

\begin{EntryWithPhonetic}{天然}{tian1ran2}{4,12}{⼤,⽕}[HSK 6]
  \definition{adj.}{natural; produzido ou ocorrido narturalmente}
\end{EntryWithPhonetic}

\begin{EntryWithPhonetic}{天然气}{tian1ran2qi4}{4,12,4}{⼤,⽕,⽓}[HSK 5]
  \definition{s.}{gás; gás natural; gás combustível produzido em campos petrolíferos, carboníferos e pântanos}
\end{EntryWithPhonetic}

\begin{EntryWithPhonetic}{天上}{tian1shang4}{4,3}{⼤,⼀}[HSK 2]
  \definition[期]{s.}{o céu; o paraíso}
\end{EntryWithPhonetic}

\begin{EntryWithPhonetic}{天生}{tian1sheng1}{4,5}{⼤,⽣}[HSK 7-9]
  \definition{adj.}{inato; inerente; congênito; nascido; descreve algo presente desde o nascimento; que ocorre naturalmente}
  \synonymref{生成}{sheng1cheng2}
  \synonymref{天才}{tian1cai2}
  \synonymref{天分}{tian1fen4}
  \synonymref{天赋}{tian1fu4}
  \synonymref{先天}{xian1tian1}
  \antonymref{后天}{hou4tian1}
  \antonymref{遗传}{yi2chuan2}
\end{EntryWithPhonetic}

\begin{EntryWithPhonetic}{天使}{tian1shi3}{4,8}{⼤,⼈}[HSK 7-9]
  \definition[个,位,名]{s.}{anjo; em religiões como o judaísmo, o cristianismo e o islamismo, os anjos são frequentemente representados como meninas ou crianças aladas, e hoje em dia o termo é comumente usado para descrever pessoas inocentes e adoráveis ​​(geralmente referindo-se a mulheres ou crianças) | enviado imperial; mensageiro do imperador}
\end{EntryWithPhonetic}

\begin{EntryWithPhonetic}{天堂}{tian1tang2}{4,11}{⼤,⼟}[HSK 6]
  \definition[间]{s.}{paraíso, céu; em algumas religiões, refere"-se ao paraíso para onde as almas das pessoas boas retornam após a morte (diferente do 地狱) | lugar perfeito; ambiente de vida extremamente feliz e bonito; uma metáfora para um ambiente de vida feliz e bonito}
  \seealsoref{地狱}{di4yu4}
\end{EntryWithPhonetic}

\begin{EntryWithPhonetic}{天天}{tian1tian1}{4,4}{⼤,⼤}
  \definition{adv.}{todo dia}
\end{EntryWithPhonetic}

\begin{EntryWithPhonetic}{天文}{tian1wen2}{4,4}{⼤,⽂}[HSK 5]
  \definition[对]{s.}{astronomia; a distribuição e o movimento dos corpos celestes, como o sol, a lua e as estrelas, no universo}
\end{EntryWithPhonetic}

\begin{EntryWithPhonetic}{天下}{tian1xia4}{4,3}{⼤,⼀}[HSK 6]
  \definition[期]{s.}{China ou o mundo; refere"-se à China ou ao mundo | dominação; o poder dominante de um país | situação; um determinado campo; uma metáfora para uma determinada área ou situação}
\end{EntryWithPhonetic}

\begin{EntryWithPhonetic}{天线}{tian1xian4}{4,8}{⼤,⽷}[HSK 7-9]
  \definition{s.}{antena | conexão com altos funcionários | mastro}
\end{EntryWithPhonetic}

\begin{EntryWithPhonetic}{天性}{tian1xing4}{4,8}{⼤,⼼}[HSK 7-9]
  \definition{s.}{natureza; instintos naturais; refere"-se às qualidades inatas ou ao temperamento de uma pessoa |  inteligência; qualidades humanas}
  \synonymref{本性}{ben3xing4}
  \synonymref{个性}{ge4xing4}
  \synonymref{天才}{tian1cai2}
  \synonymref{天分}{tian1fen4}
  \synonymref{天赋}{tian1fu4}
  \synonymref{性格}{xing4ge2}
  \antonymref{后天}{hou4tian1}
\end{EntryWithPhonetic}

\begin{EntryWithPhonetic}{天择}{tian1ze2}{4,8}{⼤,⼿}
  \definition{s.}{seleção natural}
\end{EntryWithPhonetic}

\begin{EntryWithPhonetic}{天真}{tian1zhen1}{4,10}{⼤,⼗}[HSK 4]
  \definition{adj.}{ingênuo; inocente; ignorante; simples de coração, direto por natureza, livre de fingimento e hipocrisia}
\end{EntryWithPhonetic}

\begin{EntryWithPhonetic}{天主教}{tian1zhu3jiao4}{4,5,11}{⼤,⼂,⽁}[HSK 7-9]
  \definition*{s.}{Catolicismo; uma das antigas igrejas do cristianismo, após a queda do Império Romano do Ocidente em 476 d.C., viu uma divisão entre os ramos oriental e ocidental do cristianismo; suas características incluem a unidade, a santidade e o catolicismo; a adoração a Deus e a Jeová, e a veneração de Maria como a Virgem Maria; A Igreja Católica é um sistema hierárquico, que enfatiza a obediência dos fiéis à autoridade eclesiástica; também é conhecida como ``Igreja Católica Romana'' ou ``Igreja Romana''}
\end{EntryWithPhonetic}

\begin{EntryWithPhonetic}{天柱}{tian1zhu4}{4,9}{⼤,⽊}
  \definition{s.}{pilar celestial, que sustenta o céu}
\end{EntryWithPhonetic}

%%%%%%%%%% 兲 %%%%%%%%%%
\subsection*{兲}\addcontentsline{loh}{figure}{兲 \dpy{tian1}}

\begin{EntryWithPhonetic}{兲}{tian1}{6}{⼋}
  \variantof{天}
\end{EntryWithPhonetic}

%%%%%%%%%% 添 %%%%%%%%%%
\subsection*{添}\addcontentsline{loh}{figure}{添 \dpy{tian1}}

\begin{EntryWithPhonetic}{添}{tian1}{11}{⽔}[HSK 6]
  \definition{v.}{adicionar; aumentar | dar à luz}
\end{EntryWithPhonetic}

\begin{EntryWithPhonetic}{添加}{tian1jia1}{11,5}{⽔,⼒}[HSK 7-9]
  \definition{v.}{adicionar à; aumentar; adicionar}
  \synonymref{补充}{bu3chong1}
  \synonymref{加上}{jia1shang5}
  \synonymref{增加}{zeng1jia1}
  \antonymref{扣除}{kou4chu2}
  \antonymref{去除}{qu4chu2}
  \antonymref{删除}{shan1chu2}
\end{EntryWithPhonetic}

%%%%%%%%%% 田 %%%%%%%%%%
\subsection*{田}\addcontentsline{loh}{figure}{田 \dpy{tian2}}

\begin{EntryWithPhonetic}{田}{tian2}{5}{⽥}[HSK 6][Kangxi 102]
  \definition*{s.}{Sobrenome: Tian}
  \definition[亩,块,片]{s.}{campo; terra; terra de cultivo | área aberta rica em algum produto natural; campo}
  \definition{v.}{(arcaico) caçar}
\end{EntryWithPhonetic}

\begin{EntryWithPhonetic}{田径}{tian2jing4}{5,8}{⽥,⼻}[HSK 6]
  \definition{s.}{Esporte: atletismo}[他参加了这次的田径赛。===Ele participou da competição de atletismo.]
\end{EntryWithPhonetic}

\begin{EntryWithPhonetic}{田园}{tian2yuan2}{5,7}{⽥,⼞}
  \definition{adj.}{bucólico}
  \definition{s.}{campo | interior | rural}
\end{EntryWithPhonetic}

%%%%%%%%%% 钿 %%%%%%%%%%
\subsection*{钿}\addcontentsline{loh}{figure}{钿 \dpy{tian2}}

\begin{EntryWithPhonetic}{钿}{tian2}{10}{⾦}
  \definition{s.}{(dialeto) moeda | dinheiro; moeda | uma quantia de dinheiro}
  \seeref{dian4}
\end{EntryWithPhonetic}

%%%%%%%%%% 甜 %%%%%%%%%%
\subsection*{甜}\addcontentsline{loh}{figure}{甜 \dpy{tian2}}

\begin{EntryWithPhonetic}{甜}{tian2}{11}{⽢}[HSK 3]
  \definition{adj.}{doce; melado | agradável; confortável; fazer as pessoas se sentirem confortáveis e felizes | (sono) profundo | feliz; descreve o sentimento de felicidade}
\end{EntryWithPhonetic}

\begin{EntryWithPhonetic}{甜酒}{tian2jiu3}{11,10}{⽢,⾣}
  \definition{s.}{licor doce}
\end{EntryWithPhonetic}

\begin{EntryWithPhonetic}{甜菊}{tian2ju2}{11,11}{⽢,⾋}
  \definition{s.}{estévia, arbusto cujas folhas produzem um substituto para o açúcar}
\end{EntryWithPhonetic}

\begin{EntryWithPhonetic}{甜美}{tian2mei3}{11,9}{⽢,⽺}[HSK 7-9]
  \definition{adj.}{doce; melado; adocicado | agradável; refrescante; descreve uma sensação de prazer, conforto ou beleza}
  \synonymref{甜蜜}{tian2mi4}
  \antonymref{苦恼}{ku3nao3}
\end{EntryWithPhonetic}

\begin{EntryWithPhonetic}{甜蜜}{tian2mi4}{11,14}{⽢,⾍}[HSK 7-9]
  \definition{adj.}{doce; feliz; descrevendo a sensação de felicidade, alegria e conforto}
  \seealsoref{甜}{tian2}
  \antonymref{痛苦}{tong4ku3}
\end{EntryWithPhonetic}

\begin{EntryWithPhonetic}{甜品}{tian2pin3}{11,9}{⽢,⼝}
  \definition{s.}{sobremesa}
\end{EntryWithPhonetic}

\begin{EntryWithPhonetic}{甜食}{tian2shi2}{11,9}{⽢,⾷}
  \definition{s.}{doces | sobremesa}
\end{EntryWithPhonetic}

\begin{EntryWithPhonetic}{甜酸}{tian2suan1}{11,14}{⽢,⾣}
  \definition{adj.}{agridoce}
\end{EntryWithPhonetic}

\begin{EntryWithPhonetic}{甜甜圈}{tian2tian2quan1}{11,11,11}{⽢,⽢,⼞}
  \definition{s.}{rosquinha | \emph{doughnut}}
\end{EntryWithPhonetic}

\begin{EntryWithPhonetic}{甜筒}{tian2tong3}{11,12}{⽢,⽵}
  \definition{s.}{sorvete de casquinha}
\end{EntryWithPhonetic}

\begin{EntryWithPhonetic}{甜头}{tian2tou5}{11,5}{⽢,⼤}[HSK 7-9]
  \definition{s.}{sabor doce; sabor agradável; (doçura) Um sabor levemente adocicado, geralmente referindo-se a um sabor delicioso | bom; benefício (como incentivo); (doçura) benefícios; vantagens (frequentemente referindo-se a algo tentador)}
  \synonymref{便宜}{bian4yi2}
  \synonymref{长处}{chang2chu4}
  \synonymref{好处}{hao3chu5}
  \synonymref{利益}{li4yi4}
  \synonymref{优点}{you1dian3}
\end{EntryWithPhonetic}

\begin{EntryWithPhonetic}{甜心}{tian2xin1}{11,4}{⽢,⼼}
  \definition{s.}{querido}
\end{EntryWithPhonetic}

\begin{EntryWithPhonetic}{甜言}{tian2yan2}{11,7}{⽢,⾔}
  \definition{s.}{boa conversa | palavras amáveis}
\end{EntryWithPhonetic}

\begin{EntryWithPhonetic}{甜玉米}{tian2 yu4mi3}{11,5,6}{⽢,⽟,⽶}
  \definition{s.}{milho doce}
\end{EntryWithPhonetic}

\begin{EntryWithPhonetic}{甜稚}{tian2zhi4}{11,13}{⽢,⽲}
  \definition{s.}{doce e inocente}
\end{EntryWithPhonetic}

%%%%%%%%%% 填 %%%%%%%%%%
\subsection*{填}\addcontentsline{loh}{figure}{填 \dpy{tian2}}

\begin{EntryWithPhonetic}{填}{tian2}{13}{⼟}[HSK 4]
  \definition{v.}{encher; rechear | reabastecer; suplementar; complementar | preencher; escrever dados em uma caixa (em um questionário ou formulário da \emph{Web})}
\end{EntryWithPhonetic}

\begin{EntryWithPhonetic}{填补}{tian2bu3}{13,7}{⼟,⾐}[HSK 7-9]
  \definition{v.}{preencher | preencher (uma vaga, lacuna, etc.); assumir a responsabilidade}
  \synonymref{补充}{bu3chong1}
  \synonymref{弥补}{mi2bu3}
  \synonymref{填充}{tian2chong1}
  \synonymref{增加}{zeng1jia1}
\end{EntryWithPhonetic}

\begin{EntryWithPhonetic}{填充}{tian2chong1}{13,6}{⼟,⼉}[HSK 7-9]
  \definition{v.}{encher; rechear; preencher | preencher os espaços em branco (em uma prova) | preencher; empacotar; estofar; acolchoar}
  \synonymref{补充}{bu3chong1}
  \synonymref{弥补}{mi2bu3}
  \synonymref{填补}{tian2bu3}
  \synonymref{增加}{zeng1jia1}
\end{EntryWithPhonetic}

\begin{EntryWithPhonetic}{填空}{tian2/kong4}{13,8}{⼟,⽳}[HSK 4]
  \definition{v.+compl.}{preencher o espaço em branco (por exemplo, em um teste)}
\end{EntryWithPhonetic}

\begin{EntryWithPhonetic}{填写}{tian2xie3}{13,5}{⼟,⼍}[HSK 7-9]
  \definition{v.}{escrever; preencher; completar; escrever (texto ou números) nos espaços em branco de formulários, documentos, etc., conforme necessário}
  \synonymref{填}{tian2}
\end{EntryWithPhonetic}

%%%%%%%%%% 舔 %%%%%%%%%%
\subsection*{舔}\addcontentsline{loh}{figure}{舔 \dpy{tian3}}

\begin{EntryWithPhonetic}{舔}{tian3}{14}{⾆}[HSK 7-9]
  \definition{v.}{lamber; dar uma lambida; tocar com a língua}
\end{EntryWithPhonetic}

%%%%%%%%%% 挑 %%%%%%%%%%
\subsection*{挑}\addcontentsline{loh}{figure}{挑 \dpy{tiao1}}

\begin{EntryWithPhonetic}{挑}{tiao1}{9}{⼿}[HSK 4]
  \definition{clas.}{usado para coisas que são escolhidas ou selecionadas | usado para coisas que podem ser usadas como palhetas}
  \definition{s.}{vara comprida com algo pendurado nas pontas; haste de transporte}
  \definition{v.}{escolher; selecionar | fazer picuinhas; ser hipercrítico; ser meticuloso; ser excessivamente rigoroso nos detalhes | carregar com uma haste de transporte; carregar no ombro; pendurar coisas nas pontas de varas longas e carregá-las em seus ombros}
  \seeref{tiao3}
\end{EntryWithPhonetic}

\begin{EntryWithPhonetic}{挑剔}{tiao1ti5}{9,10}{⼿,⼑}[HSK 7-9]
  \definition{v.}{ser meticuloso; ser excessivamente crítico; ser exigente; ser excessivamente crítico em relação aos detalhes}
  \synonymref{批判}{pi1pan4}
  \synonymref{评论}{ping2lun4}
  \synonymref{指责}{zhi3ze2}
\end{EntryWithPhonetic}

\begin{EntryWithPhonetic}{挑选}{tiao1xuan3}{9,9}{⼿,⾡}[HSK 4]
  \definition{v.}{escolher; optar; selecionar; escolher a pessoa ou coisa certa para o trabalho}
\end{EntryWithPhonetic}

%%%%%%%%%% 条 %%%%%%%%%%
\subsection*{条}\addcontentsline{loh}{figure}{条 \dpy{tiao2}}

\begin{EntryWithPhonetic}{条}{tiao2}{7}{⽊}[HSK 2]
  \definition*{s.}{Sobrenome: Tiao}
  \definition{clas.}{usado para objetos longos e finos; usado para sintetizar certas coisas longas e retangulares em quantidades fixas | usado para itemização | aplicado ao corpo humano}
  \definition{s.}{galho; galhos finos e longos | tira; faixa | item; artigo | ordem; método | nota; anotação em papel}
\end{EntryWithPhonetic}

\begin{EntryWithPhonetic}{条幅}{tiao2fu2}{7,12}{⽊,⼱}
  \definition{s.}{faixa | banner | pergaminho de parede (para pintura ou caligrafia)}
\end{EntryWithPhonetic}

\begin{EntryWithPhonetic}{条贯}{tiao2guan4}{7,8}{⽊,⾙}
  \definition{s.}{ordem | procedimentos | sequência | sistema}
\end{EntryWithPhonetic}

\begin{EntryWithPhonetic}{条件}{tiao2jian4}{7,6}{⽊,⼈}[HSK 2]
  \definition[个,项,些]{s.}{condição; termo; fator; fatores que restringem a ocorrência, existência ou desenvolvimento das coisas | requisito; pré-requisito; qualificação; requisitos ou padrões estabelecidos para determinadas coisas | situação; estado; condição}
\end{EntryWithPhonetic}

\begin{EntryWithPhonetic}{条款}{tiao2kuan3}{7,12}{⽊,⽋}[HSK 7-9]
  \definition[项,个]{s.}{artigo; disposição; cláusula (em um documento formal); itens em leis, tratados, estatutos sociais e outros documentos e contratos}
  \synonymref{条件}{tiao2jian4}
  \synonymref{条目}{tiao2mu4}
\end{EntryWithPhonetic}

\begin{EntryWithPhonetic}{条例}{tiao2li4}{7,8}{⽊,⼈}[HSK 7-9]
  \definition[项]{s.}{regras; decretos; imperativo; regulamentos; regras e regulamentos formulados por departamentos e organizações administrativas nacionais}
  \synonymref{规则}{gui1ze2}
\end{EntryWithPhonetic}

\begin{EntryWithPhonetic}{条目}{tiao2mu4}{7,5}{⽊,⽬}
  \definition{s.}{cláusulas e subcláusulas (em documento formal) | verbete (em um dicionário, enciclopédia, etc.)}
\end{EntryWithPhonetic}

\begin{EntryWithPhonetic}{条约}{tiao2yue1}{7,6}{⽊,⽷}[HSK 7-9]
  \definition[个,项,些]{s.}{tratado; pacto; instrumentos assinados entre Estados referentes a direitos e obrigações em matéria política, militar, econômica ou cultural}
  \synonymref{公约}{gong1yue1}
  \synonymref{合同}{he2tong5}
  \synonymref{契约}{qi4yue1}
  \synonymref{协议}{xie2yi4}
\end{EntryWithPhonetic}

%%%%%%%%%% 调 %%%%%%%%%%
\subsection*{调}\addcontentsline{loh}{figure}{调 \dpy{tiao2}}

\begin{EntryWithPhonetic}{调}{tiao2}{10}{⾔}[HSK 3]
  \definition{adj.}{harmonioso; boa coordenação}
  \definition{v.}{misturar; ajustar; fazer o ajuste uniforme e apropriado | provocar; importunar; fazer pouco de | incitar; instigar; provocar; semear discórdia | mediar; trazer harmonia}
  \seeref{diao4}
\end{EntryWithPhonetic}

\begin{EntryWithPhonetic}{调节}{tiao2jie2}{10,5}{⾔,⾋}[HSK 5]
  \definition{v.}{regular; ajustar; ajustar e controlar de várias maneiras para atender aos requisitos}
\end{EntryWithPhonetic}

\begin{EntryWithPhonetic}{调解}{tiao2jie3}{10,13}{⾔,⾓}[HSK 5]
  \definition{v.}{mediar; fazer as pazes; resolver conflitos através da persuasão}
\end{EntryWithPhonetic}

\begin{EntryWithPhonetic}{调侃}{tiao2kan3}{10,8}{⾔,⼈}[HSK 7-9]
  \definition{v.}{ridicularizar; zombar; caçoar; brincar; provocar ou zombar com humor}
  \synonymref{嘲弄}{chao2nong4}
  \synonymref{嘲笑}{chao2xiao4}
  \synonymref{讥笑}{ji1xiao4}
  \synonymref{戏弄}{xi4nong4}
\end{EntryWithPhonetic}

\begin{EntryWithPhonetic}{调控}{tiao2kong4}{10,11}{⾔,⼿}[HSK 7-9]
  \definition{v.}{regular e controlar}
  \synonymref{操作}{cao1zuo4}
  \synonymref{调节}{tiao2jie2}
  \synonymref{调整}{tiao2zheng3}
\end{EntryWithPhonetic}

\begin{EntryWithPhonetic}{调料}{tiao2liao4}{10,10}{⾔,⽃}[HSK 7-9]
  \definition[种]{s.}{tempero; condimento; aromatizante; no preparo dos pratos, os ingredientes usados ​​para temperar os alimentos incluem óleo, sal, molho de soja, vinagre, cebolinha, gengibre e alho}
\end{EntryWithPhonetic}

\begin{EntryWithPhonetic}{调律}{tiao2lv4}{10,9}{⾔,⼻}
  \definition{v.}{afinar (por exemplo, um piano)}
\end{EntryWithPhonetic}

\begin{EntryWithPhonetic}{调皮}{tiao2pi2}{10,5}{⾔,⽪}[HSK 4]
  \definition{adj.}{travesso; malicioso; malandro | indisciplinado; desordeiro; indomável; astuto | inteligente e desonesto}
\end{EntryWithPhonetic}

\begin{EntryWithPhonetic}{调试}{tiao2shi4}{10,8}{⾔,⾔}[HSK 7-9]
  \definition{v.}{depurar; testar estabilidade; testar e ajustar (máquinas, instrumentos, etc.)}
\end{EntryWithPhonetic}

\begin{EntryWithPhonetic}{调整}{tiao2zheng3}{10,16}{⾔,⽁}[HSK 3]
  \definition{v.}{ajustar; revisar; regularizar; fazer as alterações apropriadas no estado original para se adaptar à nova situação}
\end{EntryWithPhonetic}

%%%%%%%%%% 挑 %%%%%%%%%%
\subsection*{挑}\addcontentsline{loh}{figure}{挑 \dpy{tiao3}}

\begin{EntryWithPhonetic}{挑}{tiao3}{9}{⼿}[HSK 4]
  \definition{s.}{um dos traços dos caracteres chineses; inclinado para cima da esquerda para a direita}
  \definition{v.}{levantar; elevar; erguer | levantar ou apoiar com uma extremidade de uma vara ou objeto semelhante; segurar ou apoiar com a ponta de uma vara etc. | causar conflitos deliberadamente; provocar deliberadamente um conflito | (método de bordado) usar uma agulha para levantar os fios de urdidura ou trama, com a agulha e a linha passando por baixo para formar padrões e desenhos}
  \seeref{tiao1}
\end{EntryWithPhonetic}

\begin{EntryWithPhonetic}{挑起}{tiao3qi3}{9,10}{⼿,⾛}[HSK 7-9]
  \definition{v.}{provocar; incitar; instigar}
\end{EntryWithPhonetic}

\begin{EntryWithPhonetic}{挑衅}{tiao3xin4}{9,11}{⼿,⾎}[HSK 7-9]
  \definition{s.}{provocação}
  \definition{v.}{provocar; causar problemas; tentar causar conflito ou guerra}
  \synonymref{搬弄}{ban1nong4}
  \synonymref{挑战}{tiao3/zhan4}
  \antonymref{顺从}{shun4cong2}
  \antonymref{妥协}{tuo3xie2}
  \antonymref{友好}{you3hao3}
\end{EntryWithPhonetic}

\begin{EntryWithPhonetic}{挑战}{tiao3/zhan4}{9,9}{⼿,⼽}[HSK 4]
  \definition{v.+compl.}{desafiar; deixar um oponente deliberadamente irritado e sair para lutar ou lutar consigo mesmo; estimular um oponente a lutar consigo mesmo}
\end{EntryWithPhonetic}

%%%%%%%%%% 跳 %%%%%%%%%%
\subsection*{跳}\addcontentsline{loh}{figure}{跳 \dpy{tiao4}}

\begin{EntryWithPhonetic}{跳}{tiao4}{13}{⾜}[HSK 3]
  \definition{v.}{pular; saltar | mover para cima e para baixo | pular (por cima); fazer omissões | quicar; a força elástica faz com que o objeto se mova repentinamente para cima | pulsar; palpitar; contrair-se | pular sobre;  saltar sobre; cruzar}
\end{EntryWithPhonetic}

\begin{EntryWithPhonetic}{跳槽}{tiao4/cao2}{13,15}{⾜,⽊}[HSK 7-9]
  \definition{v.+compl.}{mudar de emprego; abandonar uma ocupação em favor de outra; trocar de emprego com frequência; mudar de emprego constantemente; essa metáfora descreve alguém que deixa seu emprego ou local de trabalho original para trabalhar em outra organização ou mudar de profissão | pular o cocho; o gado saiu do cocho que lhe fora designado para comer em outros cochos}
  \synonymref{离职}{li2/zhi2}
  \antonymref{坚守}{jian1shou3}
\end{EntryWithPhonetic}

\begin{EntryWithPhonetic}{跳挡}{tiao4dang3}{13,9}{⾜,⼿}
  \definition{v.}{pular marcha (de um carro) | perder a marcha}
\end{EntryWithPhonetic}

\begin{EntryWithPhonetic}{跳电}{tiao4dian4}{13,5}{⾜,⽥}
  \definition{v.}{desarmar (um disjuntor ou interruptor)}
\end{EntryWithPhonetic}

\begin{EntryWithPhonetic}{跳动}{tiao4dong4}{13,6}{⾜,⼒}[HSK 7-9]
  \definition{v.}{bater; piscar; pular}
  \synonymref{跳跃}{tiao4yue4}
  \antonymref{静止}{jing4zhi3}
\end{EntryWithPhonetic}

\begin{EntryWithPhonetic}{跳高}{tiao4gao1}{13,10}{⾜,⾼}[HSK 3]
  \definition{s.}{salto em altura (atletismo)}
  \definition{v.}{saltar em altura}
\end{EntryWithPhonetic}

\begin{EntryWithPhonetic}{跳频}{tiao4pin2}{13,13}{⾜,⾴}
  \definition{s.}{FHSS, \emph{Frequency-Hopping Spread Spectrum}, método de transmissão de sinais de rádio}
\end{EntryWithPhonetic}

\begin{EntryWithPhonetic}{跳伞}{tiao4/san3}{13,6}{⾜,⼈}[HSK 7-9]
  \definition{s.}{salto de paraquedas}
  \definition{v.+compl.}{saltar de paraquedas; ejetar"-se}
\end{EntryWithPhonetic}

\begin{EntryWithPhonetic}{跳绳}{tiao4sheng2}{13,11}{⾜,⽷}
  \definition{v.}{pular corda}
\end{EntryWithPhonetic}

\begin{EntryWithPhonetic}{跳水}{tiao4shui3}{13,4}{⾜,⽔}[HSK 6]
  \definition{s.}{Esporte: mergulho}
  \definition{v.}{mergulhar | Figurativo: (preços, lucros, etc.) cair drasticamente; cair repentinamente; mergulhar; despencar | cometer suicídio pulando na água | mergulhar (na água)}
\end{EntryWithPhonetic}

\begin{EntryWithPhonetic}{跳跳糖}{tiao4tiao4tang2}{13,13,16}{⾜,⾜,⽶}
  \definition{s.}{\emph{Pop Rocks}, \emph{popping candy}}
\end{EntryWithPhonetic}

\begin{EntryWithPhonetic}{跳舞}{tiao4/wu3}{13,14}{⾜,⾇}[HSK 3]
  \definition{v.+compl.}{dançar (como performance); executar dança, especialmente dança de salão}
\end{EntryWithPhonetic}

\begin{EntryWithPhonetic}{跳远}{tiao4yuan3}{13,7}{⾜,⾡}[HSK 3]
  \definition{s.}{salto em distância (atletismo)}
\end{EntryWithPhonetic}

\begin{EntryWithPhonetic}{跳跃}{tiao4yue4}{13,11}{⾜,⾜}[HSK 7-9]
  \definition{v.}{pular; saltar; dar um pulo}
  \synonymref{跨越}{kua4yue4}
  \synonymref{跳动}{tiao4dong4}
\end{EntryWithPhonetic}

\begin{EntryWithPhonetic}{跳蚤}{tiao4zao5}{13,9}{⾜,⾍}
  \definition{s.}{pulga}
\end{EntryWithPhonetic}

%%%%%%%%%% 帖 %%%%%%%%%%
\subsection*{帖}\addcontentsline{loh}{figure}{帖 \dpy{tie1}}

\begin{EntryWithPhonetic}{帖}{tie1}{8}{⼱}
  \definition{adj.}{apropriado; adequado; seguro}
  \definition{v.}{obedecer; cumprir; seguir}
  \seeref{tie3}
  \seeref{tie4}
\end{EntryWithPhonetic}

%%%%%%%%%% 贴 %%%%%%%%%%
\subsection*{贴}\addcontentsline{loh}{figure}{贴 \dpy{tie1}}

\begin{EntryWithPhonetic}{贴}{tie1}{9}{⾙}[HSK 4]
  \definition{adj.}{submisso; obediente | apropriado}
  \definition{clas.}{usado em gessos, emplastros}
  \definition{s.}{subsídio; subvenção}
  \definition{v.}{grudar; colar | aninhar-se a; aconchegar-se a; aconchegar-se em | subsidiar; ajudar financeiramente}
\end{EntryWithPhonetic}

\begin{EntryWithPhonetic}{贴近}{tie1jin4}{9,7}{⾙,⾡}[HSK 7-9]
  \definition{adj.}{próximo; íntimo}
  \definition{v.}{apertar"-se contra; aconchegar"-se contra; encostar"-se a; aproximar}
  \synonymref{逼近}{bi1jin4}
  \synonymref{接近}{jie1jin4}
  \synonymref{靠近}{kao4jin4}
  \synonymref{靠拢}{kao4long3}
  \synonymref{亲切}{qin1qie4}
  \synonymref{贴切}{tie1qie4}
  \antonymref{远离}{yuan3li2}
\end{EntryWithPhonetic}

\begin{EntryWithPhonetic}{贴切}{tie1qie4}{9,4}{⾙,⼑}[HSK 7-9]
  \definition{adj.}{(palavras) apto; adequado; apropriado; conveniente}
  \synonymref{恰当}{qia4dang4}
  \synonymref{贴近}{tie1jin4}
\end{EntryWithPhonetic}

%%%%%%%%%% 帖 %%%%%%%%%%
\subsection*{帖}\addcontentsline{loh}{figure}{帖 \dpy{tie3}}

\begin{EntryWithPhonetic}{帖}{tie3}{8}{⼱}
  \definition{clas.}{prescrição (uma combinação de vários ingredientes medicinais); fórmula (usada para se referir a vários ingredientes em uma decocção); Dialeto: para fitoterapia}
  \definition{s.}{convite | nota; cartão | \emph{Internet}: \emph{post}; postagem; publicação; tópico}
  \seeref{tie1}
  \seeref{tie4}
  \seealsoref{帖儿}{tie3r5}
  \seealsoref{帖子}{tie3zi5}
\end{EntryWithPhonetic}

\begin{EntryWithPhonetic}{帖儿}{tie3r5}{8,2}{⼱,⼉}
  \definition{s.}{\emph{Internet}: \emph{post}; postagem; publicação}
\end{EntryWithPhonetic}

\begin{EntryWithPhonetic}{帖子}{tie3zi5}{8,3}{⼱,⼦}[HSK 7-9]
  \definition[个,张]{s.}{convite; notificação de convite para convidado | cartão; nota; pequenos pedaços de papel com escrita neles | postagens; tópicos; isso se refere a textos, imagens, etc., publicados na \emph{Internet} sobre um tópico específico}
\end{EntryWithPhonetic}

%%%%%%%%%% 铁 %%%%%%%%%%
\subsection*{铁}\addcontentsline{loh}{figure}{铁 \dpy{tie3}}

\begin{EntryWithPhonetic}{铁}{tie3}{10}{⾦}[HSK 3]
  \definition*{s.}{Sobrenome: Tie}
  \definition{adj.}{duro; forte; sólido como ferro; metáfora para natureza dura; vontade forte | violento | inabalável; inalterável; determinado; metáfora para violência ou crueldade}
  \definition{s.}{Fe; ferro | arma; armamento; refere"-se a facas, armas de fogo, etc.}
  \definition{v.}{resolver; determinar}
\end{EntryWithPhonetic}

\begin{EntryWithPhonetic}{铁轨}{tie3gui3}{10,6}{⾦,⾞}
  \definition[根]{s.}{trilho | trilho ferroviário}
\end{EntryWithPhonetic}

\begin{EntryWithPhonetic}{铁路}{tie3lu4}{10,13}{⾦,⾜}[HSK 3]
  \definition[条,公里]{s.}{ferrovia; estrada de ferro; uma estrada com trilhos de aço dispostos no leito da estrada para a circulação de trens}
\end{EntryWithPhonetic}

%%%%%%%%%% 帖 %%%%%%%%%%
\subsection*{帖}\addcontentsline{loh}{figure}{帖 \dpy{tie4}}

\begin{EntryWithPhonetic}{帖}{tie4}{8}{⼱}
  \definition{s.}{um livro contendo modelos de caligrafia ou pintura para os alunos copiarem; exemplos para copiar}
  \seeref{tie1}
  \seeref{tie3}
\end{EntryWithPhonetic}

%%%%%%%%%% 厅 %%%%%%%%%%
\subsection*{厅}\addcontentsline{loh}{figure}{厅 \dpy{ting1}}

\begin{EntryWithPhonetic}{厅}{ting1}{4}{⼚}[HSK 5]
  \definition{s.}{salão; sala grande para reuniões ou receber convidados | escritório; nome de um departamento administrativo de uma grande organização | departamento governamental a nível provincial; nomes de alguns órgãos estaduais}
\end{EntryWithPhonetic}

%%%%%%%%%% 听 %%%%%%%%%%
\subsection*{听}\addcontentsline{loh}{figure}{听 \dpy{ting1}}

\begin{EntryWithPhonetic}{听}{ting1}{7}{⼝}[HSK 1]
  \definition{clas.}{latas; usado para bebidas e alimentos para levar consigo}
  \definition{s.}{lata; embalagem metálica; recipiente cilíndrico utilizado para armazenar bebidas, alimentos, etc.}
  \definition{v.}{ouvir; escutar | obedecer; dar ouvidos; aceitar | supervisionar; administrar; tratar (assuntos políticos); julgar (casos) | permitir; deixar ser; deixar fazer}
  \seeref{yin3}
\end{EntryWithPhonetic}

\begin{EntryWithPhonetic}{听从}{ting1cong2}{7,4}{⼝,⼈}[HSK 7-9]
  \definition{v.}{obedecer; acatar; cumprir com; fazer as coisas de acordo com os desejos dos outros}
  \synonymref{服从}{fu2cong2}
  \synonymref{顺从}{shun4cong2}
  \synonymref{听命}{ting1ming4}
  \synonymref{听取}{ting1qu3}
  \synonymref{遵守}{zun1shou3}
  \antonymref{使唤}{shi3huan5}
  \antonymref{违反}{wei2fan3}
\end{EntryWithPhonetic}

\begin{EntryWithPhonetic}{听到}{ting1dao4}{7,8}{⼝,⼑}[HSK 1]
  \definition{v.}{ouvir, escutar; ouvir atentamente, escutar atentamente}
\end{EntryWithPhonetic}

\begin{EntryWithPhonetic}{听断}{ting1duan4}{7,11}{⼝,⽄}
  \definition{v.}{ouvir e decidir | julgar (ou seja, ouvir e julgar em um tribunal)}
\end{EntryWithPhonetic}

\begin{EntryWithPhonetic}{听骨}{ting1gu3}{7,9}{⼝,⾻}
  \definition{s.}{ossículos (do ouvido médio)}
  \seealsoref{听小骨}{ting1xiao3gu3}
\end{EntryWithPhonetic}

\begin{EntryWithPhonetic}{听话}{ting1/hua4}{7,8}{⼝,⾔}[HSK 7-9]
  \definition{v.+compl.}{ser obediente; escutar as palavras dos mais velhos ou dos líderes | ouvir; escutar; ouvir as pessoas falarem | aguardar a resposta}
  \synonymref{懂事}{dong3shi4}
  \synonymref{乖乖}{guai1guai1}
  \synonymref{乖巧}{guai1qiao3}
  \antonymref{淘气}{tao2/qi4}
  \antonymref{调皮}{tiao2pi2}
\end{EntryWithPhonetic}

\begin{EntryWithPhonetic}{听会}{ting1hui4}{7,6}{⼝,⼈}
  \definition{v.}{participar de uma reunião (e ouvir o que é discutido)}
\end{EntryWithPhonetic}

\begin{EntryWithPhonetic}{听见}{ting1/jian5}{7,4}{⼝,⾒}[HSK 1]
  \definition{v.+compl.}{ouvir; conseguir ouvir}
\end{EntryWithPhonetic}

\begin{EntryWithPhonetic}{听讲}{ting1/jiang3}{7,6}{⼝,⾔}[HSK 2]
  \definition{v.+compl.}{assistir a uma palestra; ouvir palestras ou discursos}
\end{EntryWithPhonetic}

\begin{EntryWithPhonetic}{听来}{ting1lai2}{7,7}{⼝,⽊}
  \definition{v.}{ouvir de algum lugar | soar (antigo, estrangeiro, excitante, certo, etc.) | soar como se (ou seja, dar uma impressão ao ouvinte)}
\end{EntryWithPhonetic}

\begin{EntryWithPhonetic}{听力}{ting1li4}{7,2}{⼝,⼒}[HSK 3]
  \definition{s.}{audição; capacidade auditiva | compreensão auditiva (na aprendizagem de línguas)}
\end{EntryWithPhonetic}

\begin{EntryWithPhonetic}{听力理解}{ting1li4li3jie3}{7,2,11,13}{⼝,⼒,⽟,⾓}
  \definition{s.}{compreensão auditiva}
\end{EntryWithPhonetic}

\begin{EntryWithPhonetic}{听命}{ting1ming4}{7,8}{⼝,⼝}
  \definition{v.}{obedecer ordens | receber ordens}
\end{EntryWithPhonetic}

\begin{EntryWithPhonetic}{听凭}{ting1ping2}{7,8}{⼝,⼏}
  \definition{v.}{permitir (alguém a fazer o que desejar)}
\end{EntryWithPhonetic}

\begin{EntryWithPhonetic}{听取}{ting1qu3}{7,8}{⼝,⼜}[HSK 6]
  \definition{v.}{ouvir (opiniões, reflexões, relatórios, etc.)}
\end{EntryWithPhonetic}

\begin{EntryWithPhonetic}{听说}{ting1shuo1}{7,9}{⼝,⾔}[HSK 2]
  \definition{v.}{ser informado; ouvir falar de; ouvir dizer | ouvir e falar}
\end{EntryWithPhonetic}

\begin{EntryWithPhonetic}{听随}{ting1sui2}{7,11}{⼝,⾩}
  \definition{v.}{permitir | obedecer}
\end{EntryWithPhonetic}

\begin{EntryWithPhonetic}{听戏}{ting1xi4}{7,6}{⼝,⼽}
  \definition{v.}{assistir a uma ópera | ver uma ópera}
\end{EntryWithPhonetic}

\begin{EntryWithPhonetic}{听小骨}{ting1xiao3gu3}{7,3,9}{⼝,⼩,⾻}
  \definition{s.}{ossículos (do ouvido médio)}
  \seealsoref{听骨}{ting1gu3}
\end{EntryWithPhonetic}

\begin{EntryWithPhonetic}{听写}{ting1xie3}{7,5}{⼝,⼍}[HSK 1]
  \definition{s.}{ditado}
  \definition{v.}{ouvir e escrever}
\end{EntryWithPhonetic}

\begin{EntryWithPhonetic}{听众}{ting1zhong4}{7,6}{⼝,⼈}[HSK 3]
  \definition[位,名,个]{s.}{audiência; ouvintes; pessoas que ouvem palestras, música ou transmissões}
\end{EntryWithPhonetic}

%%%%%%%%%% 聼 %%%%%%%%%%
\subsection*{聼}\addcontentsline{loh}{figure}{聼 \dpy{ting1}}

\begin{EntryWithPhonetic}{聼}{ting1}{19}{⼼}
  \variantof{听}
\end{EntryWithPhonetic}

%%%%%%%%%% 亭 %%%%%%%%%%
\subsection*{亭}\addcontentsline{loh}{figure}{亭 \dpy{ting2}}

\begin{EntryWithPhonetic}{亭}{ting2}{9}{⼇}
  \definition{s.}{pavilhão | cabine | quiosque}
\end{EntryWithPhonetic}

%%%%%%%%%% 停 %%%%%%%%%%
\subsection*{停}\addcontentsline{loh}{figure}{停 \dpy{ting2}}

\begin{EntryWithPhonetic}{停}{ting2}{11}{⼈}[HSK 2]
  \definition{adj.}{pronto; resolvido; bem organizado}
  \definition{clas.}{usado para partes (de um total); porções}
  \definition{v.}{parar; interromper; cessar; fazer uma pausa | permanecer; ficar; fazer uma parada (para descansar) | estacionar; ancorar; atracar}
\end{EntryWithPhonetic}

\begin{EntryWithPhonetic}{停办}{ting2ban4}{11,4}{⼈,⼒}
  \definition{v.}{cancelar | sair do negócio | desligar | terminar}
\end{EntryWithPhonetic}

\begin{EntryWithPhonetic}{停泊}{ting2bo2}{11,8}{⼈,⽔}[HSK 7-9]
  \definition{v.}{(navios) atracar; fundear; ancorar}
\end{EntryWithPhonetic}

\begin{EntryWithPhonetic}{停车}{ting2 che1}{11,4}{⼈,⾞}[HSK 2]
  \definition{v.}{(veículo) parar; frear | estacionar o veículo | parar; deixar de funcionar}
\end{EntryWithPhonetic}

\begin{EntryWithPhonetic}{停车场}{ting2che1chang3}{11,4,6}{⼈,⾞,⼟}[HSK 2]
  \definition[个]{s.}{estacionamento; área de estacionamento; local para estacionamento de veículos}
\end{EntryWithPhonetic}

\begin{EntryWithPhonetic}{停车位}{ting2che1wei4}{11,4,7}{⼈,⾞,⼈}[HSK 7-9]
  \definition[个]{s.}{vagas de estacionamento; um espaço onde carros ou outros veículos podem ser estacionados}
\end{EntryWithPhonetic}

\begin{EntryWithPhonetic}{停当}{ting2dang5}{11,6}{⼈,⼹}
  \definition{adj.}{realizado | preparado | assentado}
\end{EntryWithPhonetic}

\begin{EntryWithPhonetic}{停电}{ting2dian4}{11,5}{⼈,⽥}[HSK 7-9]
  \definition{v.}{cortar o fornecimento de energia; ter uma falha de energia}
  \antonymref{来电}{lai2dian4}
\end{EntryWithPhonetic}

\begin{EntryWithPhonetic}{停顿}{ting2dun4}{11,10}{⼈,⾴}[HSK 7-9]
  \definition{v.}{parar; interromper; pausar; o assunto foi suspenso ou interrompido | dar pausas (na fala)}
  \synonymref{堵塞}{du3se4}
  \synonymref{平息}{ping2xi1}
  \synonymref{停留}{ting2liu2}
  \synonymref{停止}{ting2zhi3}
  \synonymref{休息}{xiu1xi5}
  \synonymref{中断}{zhong1duan4}
  \antonymref{畅通}{chang4tong1}
  \antonymref{持续}{chi2xu4}
\end{EntryWithPhonetic}

\begin{EntryWithPhonetic}{停放}{ting2fang4}{11,8}{⼈,⽅}[HSK 7-9]
  \definition{v.}{estacionar (um veículo) | colocar (um caixão) | deixar algo (em um lugar) | atracar (um barco, etc.)}
\end{EntryWithPhonetic}

\begin{EntryWithPhonetic}{停工}{ting2gong1}{11,3}{⼈,⼯}
  \definition{v.}{parar de trabalhar | parar a produção}
\end{EntryWithPhonetic}

\begin{EntryWithPhonetic}{停火}{ting2/huo3}{11,4}{⼈,⽕}
  \definition{s.}{cessar-fogo}
  \definition{v.+compl.}{cessar fogo}
\end{EntryWithPhonetic}

\begin{EntryWithPhonetic}{停课}{ting2ke4}{11,10}{⼈,⾔}
  \definition{v.}{fechar (escola) | parar as aulas}
\end{EntryWithPhonetic}

\begin{EntryWithPhonetic}{停留}{ting2liu2}{11,10}{⼈,⽥}[HSK 5]
  \definition{v.}{permanecer; ficar por muito tempo; parar temporariamente em algum lugar, sem continuar avançando | permanecer; parar por um longo tempo; parar em um determinado estágio ou nível, sem evoluir}
\end{EntryWithPhonetic}

\begin{EntryWithPhonetic}{停息}{ting2xi1}{11,10}{⼈,⼼}
  \definition{v.}{cessar | parar}
\end{EntryWithPhonetic}

\begin{EntryWithPhonetic}{停下}{ting2xia4}{11,3}{⼈,⼀}[HSK 4]
  \definition{v.}{encerrar; desligar; parar}
\end{EntryWithPhonetic}

\begin{EntryWithPhonetic}{停歇}{ting2xie1}{11,13}{⼈,⽋}
  \definition{v.}{parar para descansar}
\end{EntryWithPhonetic}

\begin{EntryWithPhonetic}{停业}{ting2/ye4}{11,5}{⼈,⼀}[HSK 7-9]
  \definition{v.+compl.}{cessar as atividades comerciais; encerramento das atividades comerciais; finalizar os negócios; suspensão das atividades comerciais; fechar as portas}[修理内部,停业5天。===O estabelecimento ficará fechado por 5 dias para reparos internos.]
  \synonymref{倒闭}{dao3bi4}
  \synonymref{破产}{po4/chan3}
  \antonymref{开业}{kai1 ye4}
  \antonymref{开张}{kai1/zhang1}
  \antonymref{营业}{ying2ye4}
\end{EntryWithPhonetic}

\begin{EntryWithPhonetic}{停用}{ting2yong4}{11,5}{⼈,⽤}
  \definition{v.}{desabilitar | descontinuar | parar de usar | suspender}
\end{EntryWithPhonetic}

\begin{EntryWithPhonetic}{停止}{ting2zhi3}{11,4}{⼈,⽌}[HSK 3]
  \definition{v.}{parar; suspender; cessar; cancelar}
\end{EntryWithPhonetic}

%%%%%%%%%% 挺 %%%%%%%%%%
\subsection*{挺}\addcontentsline{loh}{figure}{挺 \dpy{ting3}}

\begin{EntryWithPhonetic}{挺}{ting3}{9}{⼿}[HSK 2,4]
  \definition{adj.}{rígido; ereto; vertical; reto | notável; destacado; distinto}
  \definition{adv.}{muito; bastante}
  \definition{clas.}{usado para metralhadoras}
  \definition{v.}{sobressair; endireitar-se; protrudir (protuberância ou saliência) | suportar; aguentar; resistir; perseverar}
\end{EntryWithPhonetic}

\begin{EntryWithPhonetic}{挺拔}{ting3ba2}{9,8}{⼿,⼿}
  \definition{adj.}{alto e reto}
\end{EntryWithPhonetic}

\begin{EntryWithPhonetic}{挺杆}{ting3gan3}{9,7}{⼿,⽊}
  \definition{s.}{tucho (peça de máquina)}
\end{EntryWithPhonetic}

\begin{EntryWithPhonetic}{挺过}{ting3guo4}{9,6}{⼿,⾡}
  \definition{s.}{sobreviver}
\end{EntryWithPhonetic}

\begin{EntryWithPhonetic}{挺好}{ting3hao3}{9,6}{⼿,⼥}[HSK 2]
  \definition{adj.}{nada mal; surpreendentemente bom}
\end{EntryWithPhonetic}

\begin{EntryWithPhonetic}{挺进}{ting3jin4}{9,7}{⼿,⾡}
  \definition{s.}{progresso | avanço}
  \definition{v.}{progredir | avançar}
\end{EntryWithPhonetic}

\begin{EntryWithPhonetic}{挺立}{ting3li4}{9,5}{⼿,⽴}
  \definition{v.}{ficar ereto | ficar de pé}
\end{EntryWithPhonetic}

\begin{EntryWithPhonetic}{挺身}{ting3shen1}{9,7}{⼿,⾝}
  \definition{v.}{endireitar as costas}
\end{EntryWithPhonetic}

\begin{EntryWithPhonetic}{挺尸}{ting3shi1}{9,3}{⼿,⼫}
  \definition{v.}{(coloquial) dormir | (literalmente) ficar deitado duro como um cadáver}
\end{EntryWithPhonetic}

\begin{EntryWithPhonetic}{挺腰}{ting3yao1}{9,13}{⼿,⾁}
  \definition{v.}{arquear as costas | endireitar as costas}
\end{EntryWithPhonetic}

\begin{EntryWithPhonetic}{挺住}{ting3zhu4}{9,7}{⼿,⼈}
  \definition{v.}{permanecer firme | manter-se firme (diante da adversidade ou da dor)}
\end{EntryWithPhonetic}

%%%%%%%%%% 通 %%%%%%%%%%
\subsection*{通}\addcontentsline{loh}{figure}{通 \dpy{tong1}}

\begin{EntryWithPhonetic}{通}{tong1}{10}{⾡}[HSK 2]
  \definition*{s.}{Sobrenome: Tong}
  \definition{adj.}{lógico; coerente | geral; comum | tudo; inteiro | aberto; através de | total}
  \definition{clas.}{(antigo) usado para cartas, telegramas, documentos oficiais, etc.}
  \definition{s.}{autoridade; especialista}
  \definition{suf.}{especialista}
  \definition{v.}{abrir; atravessar | abrir ou limpar cutucando ou espetando | levar a; ir a | conectar; comunicar | notificar; informar | compreender; saber | cutucar; dar uma pancada | transmitir; conectar; interagir | dominar; compreender; entender}
  \seeref{tong4}
\end{EntryWithPhonetic}

\begin{EntryWithPhonetic}{通报}{tong1bao4}{10,7}{⾡,⼿}[HSK 6]
  \definition[份]{s.}{circular | boletim; jornal; publicação | sumário; notificação para informações gerais}
  \definition{v.}{circular um aviso (aviso por escrito) | notificar; dar informações com; compartilhar informações com}
\end{EntryWithPhonetic}

\begin{EntryWithPhonetic}{通常}{tong1chang2}{10,11}{⾡,⼱}[HSK 3]
  \definition{adj.}{usual; normal; geral}
  \definition{adv.}{habitualmente; usualmente; geralmente; ordinariamente}
\end{EntryWithPhonetic}

\begin{EntryWithPhonetic}{通畅}{tong1chang4}{10,8}{⾡,⽥}[HSK 7-9]
  \definition{adj.}{claro; livre; aberto; desimpedido; desobstruído | fácil e suave; (pensamentos e escrita) suave}
  \synonymref{畅通}{chang4tong1}
  \synonymref{灵通}{ling2tong1}
  \synonymref{流畅}{liu2chang4}
  \synonymref{流利}{liu2li4}
  \synonymref{流通}{liu2tong1}
  \synonymref{通顺}{tong1shun4}
  \synonymref{通行}{tong1xing2}
  \antonymref{堵塞}{du3se4}
  \antonymref{障碍}{zhang4'ai4}
\end{EntryWithPhonetic}

\begin{EntryWithPhonetic}{通车}{tong1/che1}{10,4}{⾡,⾞}[HSK 7-9]
  \definition{v.+compl.}{(uma ferrovia, rodovia, ponte, etc.) abrir para o tráfego | prestar serviço de transporte | comutar | (uma localidade) possuir serviço de transporte}
\end{EntryWithPhonetic}

\begin{EntryWithPhonetic}{通道}{tong1dao4}{10,12}{⾡,⾡}[HSK 6]
  \definition[条,个]{s.}{acesso; corredor; passagem; caminhos que levam ao exterior de teatros, minas, etc. | passagem; via pública}
\end{EntryWithPhonetic}

\begin{EntryWithPhonetic}{通牒}{tong1die2}{10,13}{⾡,⽚}
  \definition{s.}{nota diplomática}
\end{EntryWithPhonetic}

\begin{EntryWithPhonetic}{通风}{tong1/feng1}{10,4}{⾡,⾵}[HSK 7-9]
  \definition{v.}{arejar; ventilar; permitir a circulação de ar | ser bem ventilado; ter circulação de ar; ser respirável | divulgar informações; vazar de informações}
  \synonymref{透气}{tou4/qi4}
  \antonymref{封闭}{feng1bi4}
\end{EntryWithPhonetic}

\begin{EntryWithPhonetic}{通告}{tong1gao4}{10,7}{⾡,⼝}[HSK 7-9]
  \definition[张,个,则]{s.}{aviso público; anúncio; circular; proclamação}
  \definition{v.}{dar publicidade; anunciar; proclamar | informar}
  \seealsoref{通报}{tong1bao4}
  \synonymref{告诉}{gao4su4}
  \synonymref{告示}{gao4shi5}
  \synonymref{公布}{gong1bu4}
  \synonymref{公告}{gong1gao4}
  \synonymref{通知}{tong1zhi1}
  \synonymref{宣布}{xuan1bu4}
\end{EntryWithPhonetic}

\begin{EntryWithPhonetic}{通观}{tong1guan1}{10,6}{⾡,⾒}
  \definition{v.}{ter uma visão geral de algo}
\end{EntryWithPhonetic}

\begin{EntryWithPhonetic}{通过}{tong1guo4}{10,6}{⾡,⾡}[HSK 2]
  \definition{prep.}{por; através de; por meio de; por meio de; meios, métodos, etc. para introduzir ações}
  \definition{v.}{atravessar; passar por; transitar | aprovar; adotar | solicitar o consentimento ou aprovação de}
\end{EntryWithPhonetic}

\begin{EntryWithPhonetic}{通红}{tong1hong2}{10,6}{⾡,⽷}[HSK 6]
  \definition{adj.}{muito vermelho; vermelho por completo}
\end{EntryWithPhonetic}

\begin{EntryWithPhonetic}{通话}{tong1hua4}{10,8}{⾡,⾔}[HSK 6]
  \definition{v.}{comunicar por telefone | conversar; comunicar; falar em uma língua que ambos possam entender}
\end{EntryWithPhonetic}

\begin{EntryWithPhonetic}{通缉}{tong1ji1}{10,12}{⾡,⽷}[HSK 7-9]
  \definition{v.}{incluir na lista de procurados; ordenar a prisão de um criminoso foragido; as autoridades de segurança pública ou judiciais emitem ordens em suas jurisdições para prender criminosos foragidos}
\end{EntryWithPhonetic}

\begin{EntryWithPhonetic}{通识}{tong1shi2}{10,7}{⾡,⾔}
  \definition{s.}{conhecimento comum | erudição | conhecimento geral | amplamente conhecido}
\end{EntryWithPhonetic}

\begin{EntryWithPhonetic}{通顺}{tong1shun4}{10,9}{⾡,⾴}[HSK 7-9]
  \definition{adj.}{suave; claro e coerente (na escrita); o artigo não apresenta erros lógicos ou gramaticais; a linguagem é fluida e natural}
  \synonymref{畅通}{chang4tong1}
  \synonymref{顺心}{shun4/xin1}
  \synonymref{顺畅}{shun4chang4}
  \synonymref{通畅}{tong1chang4}
  \antonymref{别扭}{bie4niu5}
\end{EntryWithPhonetic}

\begin{EntryWithPhonetic}{通俗}{tong1su2}{10,9}{⾡,⼈}[HSK 7-9]
  \definition{adj.}{popular; comum; simples e fácil de entender; adequado ao nível e às necessidades da pessoa média}
  \synonymref{广泛}{guang3fan4}
  \synonymref{平常}{ping2chang2}
  \synonymref{平凡}{ping2fan2}
  \synonymref{平方}{ping2fang1}
  \synonymref{普通}{pu3tong1}
  \synonymref{通常}{tong1chang2}
  \antonymref{高贵}{gao1gui4}
  \antonymref{深奥}{shen1'ao4}
  \antonymref{诗意}{shi1yi4}
\end{EntryWithPhonetic}

\begin{EntryWithPhonetic}{通通}{tong1tong1}{10,10}{⾡,⾡}[HSK 7-9]
  \definition{adv.}{todo; tudo; inteiramente; completamente; indica sumarização}
  \synonymref{统统}{tong3tong3}
\end{EntryWithPhonetic}

\begin{EntryWithPhonetic}{通往}{tong1wang3}{10,8}{⾡,⼻}[HSK 7-9]
  \definition{v.}{levar a (um determinado lugar)}
\end{EntryWithPhonetic}

\begin{EntryWithPhonetic}{通宵}{tong1xiao1}{10,10}{⾡,⼧}[HSK 7-9]
  \definition{s.}{a noite toda; a noite inteira; durante toda a noite}
  \synonymref{熬夜}{ao2/ye4}
\end{EntryWithPhonetic}

\begin{EntryWithPhonetic}{通信}{tong1/xin4}{10,9}{⾡,⼈}[HSK 3]
  \definition{v.+compl.}{corresponder; comunicar por carta; comunicar situações e informações escrevendo cartas | transmitir (ou transportar) mensagem; passar (ou transmitir) informação; usar ondas de rádio e outros sinais para transmitir texto, imagens, etc.}
\end{EntryWithPhonetic}

\begin{EntryWithPhonetic}{通行}{tong1xing2}{10,6}{⾡,⾏}[HSK 6]
  \definition{adj.}{atual; geral}
  \definition{v.}{passar (ou ir) através; passar por; atravessar | prevalecer; predominar; ser corrente | (pedestres, veículos, etc.) passar na linha de trânsito}
\end{EntryWithPhonetic}

\begin{EntryWithPhonetic}{通行证}{tong1xing2zheng4}{10,6,7}{⾡,⾏,⾔}[HSK 7-9]
  \definition{v.}{passar; permitir; conceder salvo-conduto; dar carta branca; permissões que autorizam a entrada e a saída de áreas designadas}
\end{EntryWithPhonetic}

\begin{EntryWithPhonetic}{通讯}{tong1xun4}{10,5}{⾡,⾔}[HSK 6]
  \definition[个,种]{s.}{relatório; comunicação; boletim informativo; correspondência; reportagem; despacho de notícias; artigos que relatam fatos objetivos ou números típicos de forma detalhada e vívida}
  \definition{v.}{usar equipamentos de telecomunicações para transmitir mensagens}
\end{EntryWithPhonetic}

\begin{EntryWithPhonetic}{通用}{tong1yong4}{10,5}{⾡,⽤}[HSK 5]
  \definition[家]{adj.}{de uso comum; universal; (em um determinado âmbito) de uso generalizado | intercambiável; alguns caracteres chineses com grafia diferente, mas pronúncia igual, podem ser usados indistintamente (alguns limitados a um determinado significado)}
\end{EntryWithPhonetic}

\begin{EntryWithPhonetic}{通知}{tong1zhi1}{10,8}{⾡,⽮}[HSK 2]
  \definition[份,个,张]{s.}{aviso; circular; notificação por escrito ou verbal}
  \definition{v.}{aconselhar; notificar; informar; dar aviso prévio}
\end{EntryWithPhonetic}

\begin{EntryWithPhonetic}{通知书}{tong1zhi1shu1}{10,8,4}{⾡,⽮,⼄}[HSK 4]
  \definition[份]{s.}{aviso; observação; notificação}
\end{EntryWithPhonetic}

%%%%%%%%%% 同 %%%%%%%%%%
\subsection*{同}\addcontentsline{loh}{figure}{同 \dpy{tong2}}

\begin{EntryWithPhonetic}{同}{tong2}{6}{⼝}[HSK 6]
  \definition{adj.}{como; igual; parecido; similar; o mesmo; sem diferença}
  \definition{adv.}{juntos; em comum; indica que diferentes atores realizam uma determinada ação juntos ou estão na mesma situação, o que equivale a 一同 ou 一起}
  \definition{v.}{ser o mesmo que}
  \seeref{tong4}
  \seealsoref{一起}{yi4qi3}
  \seealsoref{一同}{yi4tong2}
\end{EntryWithPhonetic}

\begin{EntryWithPhonetic}{同伴}{tong2ban4}{6,7}{⼝,⼈}[HSK 7-9]
  \definition[名,个]{s.}{companheiro; uma pessoa com quem você trabalha, mora ou realiza alguma atividade específica}
  \synonymref{伴侣}{ban4lv3}
  \synonymref{差错}{cha1cuo4}
  \synonymref{错误}{cuo4wu4}
  \synonymref{搭档}{da1dang4}
  \synonymref{过错}{guo4cuo4}
  \synonymref{伙伴}{huo3ban4}
  \synonymref{朋友}{peng2you5}
  \synonymref{同伙}{tong2huo3}
\end{EntryWithPhonetic}

\begin{EntryWithPhonetic}{同胞}{tong2bao1}{6,9}{⼝,⾁}[HSK 6]
  \definition{s.}{nascidos dos mesmos pais | compatriota; conterrâneo; pessoas do mesmo país ou etnia}
\end{EntryWithPhonetic}

\begin{EntryWithPhonetic}{同步}{tong2bu4}{6,7}{⼝,⽌}[HSK 7-9]
  \definition{s.}{sincronizar; sincronizar com; coordenar em tempo de progresso; geralmente se refere ao ritmo de ação coordenado e consistente de coisas interconectadas}
  \definition{s.}{sincronismo; sincronização; em ciência e tecnologia, refere-se a duas ou mais grandezas que se alteram ao longo do tempo, mantendo uma determinada relação relativa durante o processo de mudança}
  \synonymref{协调}{xie2tiao2}
  \synonymref{一起}{yi4qi3}
\end{EntryWithPhonetic}

\begin{EntryWithPhonetic}{同等}{tong2deng3}{6,12}{⼝,⽵}[HSK 7-9]
  \definition{adj.}{igual; que pertence à mesma classe social ou posição social; pessoas da mesma posição ou \emph{status}}
  \synonymref{平等}{ping2deng3}
  \synonymref{一律}{yi2lv4}
\end{EntryWithPhonetic}

\begin{EntryWithPhonetic}{同感}{tong2gan3}{6,13}{⼝,⼼}[HSK 7-9]
  \definition{s.}{consenso; simpatia; os mesmos pensamentos ou sentimentos}
\end{EntryWithPhonetic}

\begin{EntryWithPhonetic}{同行}{tong2hang2}{6,6}{⼝,⾏}[HSK 6]
  \definition{s.}{do mesmo ofício ou ocupação; pessoas no mesmo setor}
  \definition{v.}{ser do mesmo ofício ou ocupação; trabalhar no mesmo setor}
\end{EntryWithPhonetic}

\begin{EntryWithPhonetic}{同伙}{tong2huo3}{6,6}{⼝,⼈}[HSK 7-9]
  \definition[个]{s.}{parceiro; cúmplice; aliado}
  \definition{v.}{trabalhar em parceria; conspirar (para praticar o mal) | conspirar com; participar conjuntamente em uma organização ou envolver-se em determinada atividade | associar"-se}
  \synonymref{伴侣}{ban4lv3}
  \synonymref{伙伴}{huo3ban4}
  \synonymref{朋友}{peng2you5}
  \synonymref{同伴}{tong2ban4}
  \synonymref{同盟}{tong2meng2}
  \antonymref{对手}{dui4shou3}
\end{EntryWithPhonetic}

\begin{EntryWithPhonetic}{同类}{tong2lei4}{6,9}{⼝,⽶}[HSK 7-9]
  \definition{adj.}{similar; semelhante; do mesmo tipo; pessoas ou coisas do mesmo tipo}
  \synonymref{同样}{tong2yang4}
\end{EntryWithPhonetic}

\begin{EntryWithPhonetic}{同流合污}{tong2liu2he2wu1}{6,10,6,6}{⼝,⽔,⼝,⽔}
  \definition{expr.}{chafurdar na lama com alguém | seguir o mau exemplo dos outros}
\end{EntryWithPhonetic}

\begin{EntryWithPhonetic}{同盟}{tong2meng2}{6,13}{⼝,⽫}[HSK 7-9]
  \definition[个]{s.}{aliança; liga}
  \synonymref{合作}{he2zuo4}
  \synonymref{联盟}{lian2meng2}
  \synonymref{同伙}{tong2huo3}
  \synonymref{战友}{zhan4you3}
\end{EntryWithPhonetic}

\begin{EntryWithPhonetic}{同年}{tong2nian2}{6,6}{⼝,⼲}[HSK 7-9]
  \definition{adj.}{Dialeto: da mesma idade; no mesmo ano}
  \definition{s.}{mesmo ano | Obsoleto: candidatos que passaram nos exames imperiais no mesmo ano}
\end{EntryWithPhonetic}

\begin{EntryWithPhonetic}{同期}{tong2qi1}{6,12}{⼝,⽉}[HSK 6]
  \definition{s.}{o período correspondente; o mesmo período; no mesmo tempo}
\end{EntryWithPhonetic}

\begin{EntryWithPhonetic}{同情}{tong2qing2}{6,11}{⼝,⼼}[HSK 4]
  \definition{s.}{simpatia}
  \definition{v.}{simpatizar com; solidarizar-se; compadecer-se; ter empatia emocional pelo que os outros estão passando}
\end{EntryWithPhonetic}

\begin{EntryWithPhonetic}{同人}{tong2ren2}{6,2}{⼝,⼈}[HSK 7-9]
  \definition{s.}{colega de trabalho; pessoas do mesmo local de trabalho ou profissão | colega | entusiastas da cultura pop que criam fanfics etc.}
\end{EntryWithPhonetic}

\begin{EntryWithPhonetic}{同时}{tong2shi2}{6,7}{⼝,⽇}[HSK 2]
  \definition{conj.}{além disso; além do mais; ainda mais; indica uma relação de equivalência, geralmente com um significado mais profundo}
  \definition{s.}{enquanto isso; ao mesmo tempo}
\end{EntryWithPhonetic}

\begin{EntryWithPhonetic}{同事}{tong2shi4}{6,8}{⼝,⼅}[HSK 2]
  \definition[个,位,名]{s.}{companheiro; colega; colega de trabalho; pessoas que trabalham juntas}
  \definition{v.}{trabalhar no mesmo lugar; trabalhar juntos; trabalhar na mesma unidade}
\end{EntryWithPhonetic}

\begin{EntryWithPhonetic}{同屋}{tong2wu1}{6,9}{⼝,⼫}
  \definition[个]{s.}{companheiro de quarto | colega de quarto}
\end{EntryWithPhonetic}

\begin{EntryWithPhonetic}{同性恋}{tong2xing4lian4}{6,8,10}{⼝,⼼,⼼}
  \definition{s.}{homossexualidade | pessoa gay | amor gay}
\end{EntryWithPhonetic}

\begin{EntryWithPhonetic}{同学}{tong2xue2}{6,8}{⼝,⼦}[HSK 1]
  \definition[位,个,些]{s.}{colega de escola; colega de turma; colega de estudos; pessoas que estudam na mesma escola}
\end{EntryWithPhonetic}

\begin{EntryWithPhonetic}{同砚}{tong2yan4}{6,9}{⼝,⽯}
  \definition[位,个]{s.}{colega de classe | colega estudante}
\end{EntryWithPhonetic}

\begin{EntryWithPhonetic}{同样}{tong2yang4}{6,10}{⼝,⽊}[HSK 2]
  \definition{adj.}{igual; semelhante; similar; idêntico; sem diferença}
\end{EntryWithPhonetic}

\begin{EntryWithPhonetic}{同一}{tong2yi1}{6,1}{⼝,⼀}[HSK 6]
  \definition{adj.}{mesmo; idêntico}
  \definition[讲]{s.}{identidade; unidade}
\end{EntryWithPhonetic}

\begin{EntryWithPhonetic}{同意}{tong2yi4}{6,13}{⼝,⼼}[HSK 3]
  \definition{v.}{concordar; consentir; aprovar; concordar com; dizer sim}
\end{EntryWithPhonetic}

\begin{EntryWithPhonetic}{同志}{tong2zhi4}{6,7}{⼝,⼼}[HSK 7-9]
  \definition[个,位,名,些]{s.}{camarada; pessoas que compartilham objetivos comuns; especificamente, membros do mesmo partido político | título habitual usado em ocasiões formais; a forma como as pessoas se tratam atualmente é geralmente usada em contextos formais e é menos comum na linguagem falada}
  \antonymref{敌人}{di2ren2}
\end{EntryWithPhonetic}

\begin{EntryWithPhonetic}{同舟共济}{tong2zhou1-gong4ji4}{6,6,6,9}{⼝,⾈,⼋,⽔}[HSK 7-9]
  \definition{expr.}{atravessar um rio no mesmo barco; remar juntos em tempos difíceis; unir-se em momentos de adversidade; pessoas no mesmo barco se ajudam mutuamente; unem forças para superar dificuldades}
  \antonymref{各奔前程}{ge4ben4qian2cheng2}
\end{EntryWithPhonetic}

%%%%%%%%%% 铜 %%%%%%%%%%
\subsection*{铜}\addcontentsline{loh}{figure}{铜 \dpy{tong2}}

\begin{EntryWithPhonetic}{铜}{tong2}{11}{⾦}[HSK 7-9]
  \definition[块]{s.}{Cu; cobre}
\end{EntryWithPhonetic}

\begin{EntryWithPhonetic}{铜牌}{tong2pai2}{11,12}{⾦,⽚}[HSK 6]
  \definition[枚]{s.}{medalha de bronze; o bronze | placa de bronze com nome ou logotipo comercial, etc.}
\end{EntryWithPhonetic}

%%%%%%%%%% 童 %%%%%%%%%%
\subsection*{童}\addcontentsline{loh}{figure}{童 \dpy{tong2}}

\begin{EntryWithPhonetic}{童}{tong2}{12}{⽴}
  \definition*{s.}{Sobrenome: Tong}
  \definition{adj.}{virgem; solteira | nu; careca | árido; estéril}
  \definition{s.}{criança | jovem servo; antigamente, referia"-se a um servo menor de idade}
\end{EntryWithPhonetic}

\begin{EntryWithPhonetic}{童话}{tong2hua4}{12,8}{⽴,⾔}[HSK 4]
  \definition[个,部]{s.}{conto de fadas; gênero de literatura infantil no qual as histórias adequadas para a diversão das crianças são escritas com muita imaginação, fantasia e exagero}
\end{EntryWithPhonetic}

\begin{EntryWithPhonetic}{童年}{tong2nian2}{12,6}{⽴,⼲}[HSK 4]
  \definition[对]{s.}{infância; primeiros anos de vida}
\end{EntryWithPhonetic}

%%%%%%%%%% 僮 %%%%%%%%%%
\subsection*{僮}\addcontentsline{loh}{figure}{僮 \dpy{tong2}}

\begin{EntryWithPhonetic}{僮}{tong2}{14}{⼈}
  \definition*{s.}{Sobrenome: Tong}
  \seeref{zhuang4}
\end{EntryWithPhonetic}

%%%%%%%%%% 獞 %%%%%%%%%%
\subsection*{獞}\addcontentsline{loh}{figure}{獞 \dpy{tong2}}

\begin{EntryWithPhonetic}{獞}{tong2}{15}{⽝}
  \definition{s.}{nome de uma variedade de cão | tribos selvagens no sul da China}
  \seeref{zhuang4}
\end{EntryWithPhonetic}

%%%%%%%%%% 统 %%%%%%%%%%
\subsection*{统}\addcontentsline{loh}{figure}{统 \dpy{tong3}}

\begin{EntryWithPhonetic}{统}{tong3}{9}{⽷}
  \definition{adv.}{todos; juntos; de forma unificada | inteiramente; totalmente}
  \definition{s.}{interligado; inter-relacionado | sistema interconectado | qualquer parte em forma de tubo de uma peça de roupa, etc.; o mesmo que 筒}
  \definition{v.}{reunir em um; unir | unir; liderar; comandar}
  \seealsoref{筒}{tong3}
\end{EntryWithPhonetic}

\begin{EntryWithPhonetic}{统筹}{tong3chou2}{9,13}{⽷,⽵}[HSK 7-9]
  \definition{v.}{planejar como um todo; elaborar planos gerais}
  \synonymref{兼顾}{jian1gu4}
\end{EntryWithPhonetic}

\begin{EntryWithPhonetic}{统计}{tong3ji4}{9,4}{⽷,⾔}[HSK 4]
  \definition{v.}{compilar estatísticas; refere"-se à realização de trabalho estatístico, ou seja, coletar, reunir, analisar e extrapolar dados sobre um fenômeno | somar; adicionar; contar}
\end{EntryWithPhonetic}

\begin{EntryWithPhonetic}{统统}{tong3tong3}{9,9}{⽷,⽷}[HSK 7-9]
  \definition{adv.}{todos; completamente; inteiramente}
  \seealsoref{一律}{yi2lv4}
  \synonymref{绝对}{jue2dui4}
  \synonymref{全部}{quan2bu4}
  \synonymref{十足}{shi2zu2}
  \synonymref{所有}{suo3you3}
  \synonymref{完全}{wan2quan2}
  \synonymref{完整}{wan2zheng3}
  \synonymref{一共}{yi2gong4}
  \synonymref{一切}{yi2qie4}
  \synonymref{整个}{zheng3ge4}
  \synonymref{总共}{zong3gong4}
  \antonymref{仅仅}{jin3jin3}
  \antonymref{少数}{shao3shu4}
\end{EntryWithPhonetic}

\begin{EntryWithPhonetic}{统一}{tong3yi1}{9,1}{⽷,⼀}[HSK 4]
  \definition{adj.}{unificado; unitário; centralizado; consistente}
  \definition{v.}{unificar; unir; integrar; padronizar}
\end{EntryWithPhonetic}

\begin{EntryWithPhonetic}{统治}{tong3zhi4}{9,8}{⽷,⽔}[HSK 7-9]
  \definition{v.}{governar; dominar; controlar e governar um país ou região através do poder político}
  \synonymref{办理}{ban4li3}
  \synonymref{处理}{chu3li3}
  \synonymref{管理}{guan3li3}
  \synonymref{管辖}{guan3xia2}
  \synonymref{争霸}{zheng1ba4}
  \synonymref{治理}{zhi4li3}
\end{EntryWithPhonetic}

%%%%%%%%%% 捅 %%%%%%%%%%
\subsection*{捅}\addcontentsline{loh}{figure}{捅 \dpy{tong3}}

\begin{EntryWithPhonetic}{捅}{tong3}{10}{⼿}[HSK 7-9]
  \definition{v.}{esfaquear; dar uma estocada; cutucar | cutucar; esbarrar; bater | expor; revelar}
\end{EntryWithPhonetic}

%%%%%%%%%% 桶 %%%%%%%%%%
\subsection*{桶}\addcontentsline{loh}{figure}{桶 \dpy{tong3}}

\begin{EntryWithPhonetic}{桶}{tong3}{11}{⽊}[HSK 7-9]
  \definition{clas.}{barril; uma unidade de capacidade}
  \definition[只,个]{s.}{barril; tina; tonel; balde; vaso sanitário}
\end{EntryWithPhonetic}

%%%%%%%%%% 筒 %%%%%%%%%%
\subsection*{筒}\addcontentsline{loh}{figure}{筒 \dpy{tong3}}

\begin{EntryWithPhonetic}{筒}{tong3}{12}{⽵}[HSK 7-9]
  \definition[个]{s.}{seção de bambu grosso; tubo grosso de bambu | objeto em forma de tubo largo | a parte em forma de tubo das roupas etc.}
  \definition{v.}{colocar dentro de (um objeto cilíndrico)}
\end{EntryWithPhonetic}

%%%%%%%%%% 同 %%%%%%%%%%
\subsection*{同}\addcontentsline{loh}{figure}{同 \dpy{tong4}}

\begin{EntryWithPhonetic}{同}{tong4}{6}{⼝}
  \definition[条,处]{s.}{beco; rua estreita}
  \seeref{tong2}
  \seealsoref{胡同}{hu2tong5}
\end{EntryWithPhonetic}

%%%%%%%%%% 通 %%%%%%%%%%
\subsection*{通}\addcontentsline{loh}{figure}{通 \dpy{tong4}}

\begin{EntryWithPhonetic}{通}{tong4}{10}{⾡}
  \definition{clas.}{usado para uma atividade, tomada em sua totalidade (discurso de abuso, período de reprodução de música, bebedeira, etc.)}
  \seeref{tong1}
\end{EntryWithPhonetic}

%%%%%%%%%% 痛 %%%%%%%%%%
\subsection*{痛}\addcontentsline{loh}{figure}{痛 \dpy{tong4}}

\begin{EntryWithPhonetic}{痛}{tong4}{12}{⽧}[HSK 3,7-9]
  \definition{adv.}{extremamente; profundamente; amargamente}
  \definition{s.}{dor; sofrimento | tristeza; pesar}
  \definition{v.}{sentir dor; ter dor | sentir tristeza; sentir aflição}
\end{EntryWithPhonetic}

\begin{EntryWithPhonetic}{痛苦}{tong4ku3}{12,8}{⽧,⾋}[HSK 3]
  \definition{adj.}{doloroso; angustiado; sentindo"-se muito desconfortável física ou mentalmente}
  \definition[降,种]{s.}{dor; agonia; sofrimento; refere"-se a um estado ou sentimento de extremo desconforto físico ou mental}
\end{EntryWithPhonetic}

\begin{EntryWithPhonetic}{痛快}{tong4kuai5}{12,7}{⽧,⼼}[HSK 4]
  \definition{adj.}{encantado; alegre; muito feliz; confortável | franco; direto; simples e direto}
\end{EntryWithPhonetic}

\begin{EntryWithPhonetic}{痛骂}{tong4ma4}{12,9}{⽧,⾺}
  \definition{v.}{repreender severamente}
\end{EntryWithPhonetic}

\begin{EntryWithPhonetic}{痛心}{tong4xin1}{12,4}{⽧,⼼}[HSK 7-9]
  \definition{adj.}{de partir o coração; dolorido; angustiado; aflito; enlutado; extrema tristeza}
\end{EntryWithPhonetic}

%%%%%%%%%% 偷 %%%%%%%%%%
\subsection*{偷}\addcontentsline{loh}{figure}{偷 \dpy{tou1}}

\begin{EntryWithPhonetic}{偷}{tou1}{11}{⼈}[HSK 5]
  \definition{adv.}{furtivamente; secretamente; às escondidas}
  \definition{s.}{ladrão; furtador}
  \definition{v.}{roubar; furtar; levar sem pagar; roubar os bens alheios às escondidas | encontrar (tempo) | deixar-se levar; viver apenas para o presente, sem se preocupar com o futuro}
\end{EntryWithPhonetic}

\begin{EntryWithPhonetic}{偷安}{tou1'an1}{11,6}{⼈,⼧}
  \definition{v.}{buscar facilidade temporária}
\end{EntryWithPhonetic}

\begin{EntryWithPhonetic}{偷渡}{tou1du4}{11,12}{⼈,⽔}
  \definition{s.}{contrabando | imigração ilegal | clandestino (em um navio)}
  \definition{v.}{executar um bloqueio | roubar através da fronteira internacional}
\end{EntryWithPhonetic}

\begin{EntryWithPhonetic}{偷看}{tou1kan4}{11,9}{⼈,⽬}[HSK 7-9]
  \definition{v.}{dar uma olhada rápida; espiar; dar uma olhadinha | espiar; espreitar}
\end{EntryWithPhonetic}

\begin{EntryWithPhonetic}{偷窥}{tou1kui1}{11,13}{⼈,⽳}[HSK 7-9]
  \definition{v.}{espiar; espionar; bisbilhotar; observar secretamente (especialmente para prazer sexual)}
\end{EntryWithPhonetic}

\begin{EntryWithPhonetic}{偷懒}{tou1/lan3}{11,16}{⼈,⼼}[HSK 7-9]
  \definition{v.+compl.}{ser preguiçoso; vadiar no trabalho; buscar conforto e conveniência, esquivar"-se das responsabilidades}
  \synonymref{怠慢}{dai4man4}
  \synonymref{懒惰}{lan3duo4}
  \synonymref{懒散}{lan3san3}
  \antonymref{练习}{lian4xi2}
\end{EntryWithPhonetic}

\begin{EntryWithPhonetic}{偷窃}{tou1qie4}{11,9}{⼈,⽳}
  \definition{v.}{furtar | roubar}
\end{EntryWithPhonetic}

\begin{EntryWithPhonetic}{偷情}{tou1qing2}{11,11}{⼈,⼼}
  \definition{v.}{manter um caso amoroso clandestino; anteriormente usado para se referir a ter um relacionamento romântico secreto, agora geralmente se refere a ter um relacionamento impróprio entre um homem e uma mulher}
\end{EntryWithPhonetic}

\begin{EntryWithPhonetic}{偷税}{tou1shui4}{11,12}{⼈,⽲}
  \definition{s.}{evasão fiscal}
\end{EntryWithPhonetic}

\begin{EntryWithPhonetic}{偷听}{tou1ting1}{11,7}{⼈,⼝}
  \definition{v.}{bisbilhotar; monitorar (secretamente)}
\end{EntryWithPhonetic}

\begin{EntryWithPhonetic}{偷偷}{tou1tou1}{11,11}{⼈,⼈}[HSK 5]
  \definition{adv.}{secretamente; dissimuladamente; furtivamente; às escondidas; descreve uma ação que não é notada pelos outros; em segredo ou em privado, não revelada}
\end{EntryWithPhonetic}

\begin{EntryWithPhonetic}{偷袭}{tou1xi2}{11,11}{⼈,⾐}
  \definition{s.}{ataque surpresa}
  \definition{v.}{montar um ataque furtivo | invadir}
\end{EntryWithPhonetic}

%%%%%%%%%% 偸 %%%%%%%%%%
\subsection*{偸}\addcontentsline{loh}{figure}{偸 \dpy{tou1}}

\begin{EntryWithPhonetic}{偸}{tou1}{11}{⼈}
  \variantof{偷}
\end{EntryWithPhonetic}

%%%%%%%%%% 头 %%%%%%%%%%
\subsection*{头}\addcontentsline{loh}{figure}{头 \dpy{tou2}}

\begin{EntryWithPhonetic}{头}{tou2}{5}{⼤}[HSK 2,3]
  \definition{adj.}{(antes de um numeral) primeiro | (antes de 年 ou 天) último; anterior}
  \definition{clas.}{usado para suínos ou gado (animais de criação) | usado para cabeças de alho ou coisas com formato de cabeça}
  \definition{num.}{primeiro}
  \definition{prep.}{antes de; perto de; introduz o tempo de uma ação, equivalente a  在……之前 ou 临近 | (entre dois algarismos, indicando um número aproximado) cerca de}
  \definition[个,颗]{s.}{cabeça; a parte do corpo humano ou animal que possui órgãos como boca, nariz, olhos e ouvidos | cabelo ou penteado | topo; fim; a parte superior ou final de um objeto | começo ou fim; o ponto inicial ou final de algo | fim; remanescente; os restos de algo | cabeça; chefe; líder | lado; aspecto}
  \seeref{tou5}
  \seealsoref{临近}{lin2jin4}
  \seealsoref{年}{nian2}
  \seealsoref{天}{tian1}
  \seealsoref{在}{zai4}
  \seealsoref{之前}{zhi1qian2}
\end{EntryWithPhonetic}

\begin{EntryWithPhonetic}{头部}{tou2bu4}{5,10}{⼤,⾢}[HSK 7-9]
  \definition{s.}{cabeça; cefalossoma}
\end{EntryWithPhonetic}

\begin{EntryWithPhonetic}{头顶}{tou2ding3}{5,8}{⼤,⾴}[HSK 7-9]
  \definition{s.}{topo da cabeça; cocuruto}
\end{EntryWithPhonetic}

\begin{EntryWithPhonetic}{头发}{tou2fa5}{5,5}{⼤,⼜}[HSK 2]
  \definition[根,缕,头]{s.}{cabelo}
\end{EntryWithPhonetic}

\begin{EntryWithPhonetic}{头号}{tou2hao4}{5,5}{⼤,⼝}[HSK 7-9]
  \definition{adj.}{\emph{top rank}; número um; tamanho um | de primeira classe; de ​​qualidade superior; o melhor}
\end{EntryWithPhonetic}

\begin{EntryWithPhonetic}{头脑}{tou2nao3}{5,10}{⼤,⾁}[HSK 3]
  \definition{s.}{inteligência; mente | pista; tópicos principais | chefe; líder; capitão}
\end{EntryWithPhonetic}

\begin{EntryWithPhonetic}{头脑风暴}{tou2nao3feng1bao4}{5,10,4,15}{⼤,⾁,⾵,⽇}
  \definition{s.}{\emph{brainstorm}}
\end{EntryWithPhonetic}

\begin{EntryWithPhonetic}{头疼}{tou2teng2}{5,10}{⼤,⽧}[HSK 6]
  \definition{s.}{dor de cabeça}
  \definition{v.}{estar preocupado ou incomodado por alguém ou algo}
\end{EntryWithPhonetic}

\begin{EntryWithPhonetic}{头条}{tou2tiao2}{5,7}{⼤,⽊}[HSK 7-9]
  \definition{s.}{manchete; notícia principal; notícia importante}
\end{EntryWithPhonetic}

\begin{EntryWithPhonetic}{头头}{tou2tou2}{5,5}{⼤,⼤}
  \definition{s.}{chefe | o cabeça}
\end{EntryWithPhonetic}

\begin{EntryWithPhonetic}{头头是道}{tou2tou2shi4dao4}{5,5,9,12}{⼤,⼤,⽇,⾡}[HSK 7-9]
  \definition{expr.}{bem fundamentado e argumentado; parecer impressionante; claro e lógico; fazer algo para a perfeita satisfação de alguém; (soar) plausível; (parecer) bem fundamentado e racional; com razão; sistemático e ordenado; coerente e convincente}
  \antonymref{乱七八糟}{luan4qi1ba1zao1}
\end{EntryWithPhonetic}

\begin{EntryWithPhonetic}{头衔}{tou2xian2}{5,11}{⼤,⾏}[HSK 7-9]
  \definition{s.}{título (um sinal de posição, profissão, etc.); refere"-se a títulos como títulos oficiais, títulos acadêmicos e títulos profissionais}
  \synonymref{称号}{cheng1hao4}
\end{EntryWithPhonetic}

\begin{EntryWithPhonetic}{头像}{tou2xiang4}{5,13}{⼤,⼈}
  \definition{s.}{retrato | busto | avatar | imagem de perfil (computação)}
\end{EntryWithPhonetic}

\begin{EntryWithPhonetic}{头晕}{tou2yun1}{5,10}{⼤,⽇}[HSK 7-9]
  \definition{s.}{fraco; tonto; vertiginoso (em grandes alturas); sentindo tontura}
  \synonymref{头疼}{tou2teng2}
  \antonymref{清醒}{qing1xing3}
\end{EntryWithPhonetic}

%%%%%%%%%% 投 %%%%%%%%%%
\subsection*{投}\addcontentsline{loh}{figure}{投 \dpy{tou2}}

\begin{EntryWithPhonetic}{投}{tou2}{7}{⼿}[HSK 4]
  \definition*{s.}{Sobrenome: Tou}
  \definition{pron.}{para; indica tempo, equivalente a 到, 临 | para; em direção a; indica orientação, direção, equivalente a 朝 ou 向}
  \definition{s.}{um jogo durante uma festa em que o vencedor era decidido pelo número de flechas lançadas em um pote distante | jogo de dados}
  \definition{v.}{lançar; arremessar; atirar | deixar cair; colocar em; lançar | mergulhar em; lançar-se em; pular dentro | lançar; projetar; sombrear | entregar; postar; enviar | ir até; ir para; buscar; juntar-se | sentir-se atraído por; adaptar-se a; concordar com; atender a}
  \seealsoref{朝}{chao2}
  \seealsoref{到}{dao4}
  \seealsoref{临}{lin2}
  \seealsoref{向}{xiang4}
\end{EntryWithPhonetic}

\begin{EntryWithPhonetic}{投奔}{tou2ben4}{7,8}{⼿,⼤}[HSK 7-9]
  \definition{v.}{ir para (um amigo ou um lugar) em busca de abrigo; recorrer a (alguém ou organização) e confiar em}
\end{EntryWithPhonetic}

\begin{EntryWithPhonetic}{投递}{tou2di4}{7,10}{⼿,⾡}
  \definition{v.}{despachar | enviar}
\end{EntryWithPhonetic}

\begin{EntryWithPhonetic}{投稿}{tou2/gao3}{7,15}{⼿,⽲}[HSK 7-9]
  \definition{v.+compl.}{submeter um texto para publicação; contribuir (para um jornal, revista, etc.); envar o manuscrito para redações de jornais, editoras, etc., solicitando a publicação}
\end{EntryWithPhonetic}

\begin{EntryWithPhonetic}{投机}{tou2ji1}{7,6}{⼿,⽊}[HSK 7-9]
  \definition{adj.}{afável; agradável; opiniões compartilhadas}
  \definition{v.}{especular; ser oportunista; aproveitar uma oportunidade para obter ganho pessoal}
  \synonymref{炒股}{chao3/gu3}
  \antonymref{本分}{ben3fen4}
\end{EntryWithPhonetic}

\begin{EntryWithPhonetic}{投票}{tou2/piao4}{7,11}{⼿,⽰}[HSK 6]
  \definition{v.+compl.}{votar; dar um voto; um método de eleição no qual os eleitores escrevem o nome da pessoa que querem eleger na cédula, ou marcam a cédula com o nome do candidato impresso e depois a colocam na urna para votar na resolução}
\end{EntryWithPhonetic}

\begin{EntryWithPhonetic}{投入}{tou2ru4}{7,2}{⼿,⼊}[HSK 4]
  \definition{adj.}{sisudo; dedicado; devotado; absorto}
  \definition{s.}{investimento; insumo; refere"-se à aplicação de recursos}
  \definition{v.}{lançar em; colocar em; jogar em; por em | entrar em uma situação; participar de | aplicar; investir; colocar fundos em}
\end{EntryWithPhonetic}

\begin{EntryWithPhonetic}{投射}{tou2she4}{7,10}{⼿,⼨}[HSK 7-9]
  \definition{v.}{lançar; arremessar; atirar | lançar; projetar (um raio de luz, etc.)}
\end{EntryWithPhonetic}

\begin{EntryWithPhonetic}{投身}{tou2shen1}{7,7}{⼿,⾝}[HSK 7-9]
  \definition{v.}{dedicar"-se; entregar"-se de corpo e alma a algo}
  \synonymref{奔赴}{ben1fu4}
  \synonymref{投入}{tou2ru4}
\end{EntryWithPhonetic}

\begin{EntryWithPhonetic}{投诉}{tou2su4}{7,7}{⼿,⾔}[HSK 4]
  \definition{v.}{reclamar; queixar-se; reclamar às autoridades ou pessoas envolvidas}
\end{EntryWithPhonetic}

\begin{EntryWithPhonetic}{投降}{tou2xiang2}{7,8}{⼿,⾩}[HSK 7-9]
  \definition{v.}{render"-se; capitular; cessar a resistência; render"-se ao outro lado}
  \synonymref{背叛}{bei4pan4}
  \synonymref{俘虏}{fu2lu3}
  \synonymref{屈服}{qu1fu2}
  \synonymref{顺从}{shun4cong2}
  \synonymref{征服}{zheng1fu2}
  \antonymref{抵抗}{di3kang4}
  \antonymref{抵御}{di3yu4}
  \antonymref{对抗}{dui4kang4}
  \antonymref{反抗}{fan3kang4}
  \antonymref{抗拒}{kang4ju4}
  \antonymref{侵略}{qin1lve4}
\end{EntryWithPhonetic}

\begin{EntryWithPhonetic}{投资}{tou2zi1}{7,10}{⼿,⾙}[HSK 4]
  \definition[笔]{s.}{investimento}
  \definition{v.}{investir; aplicar dinheiro; investir dinheiro em negócios}
\end{EntryWithPhonetic}

\begin{EntryWithPhonetic}{投资风险}{tou2zi1 feng1xian3}{7,10,4,9}{⼿,⾙,⾵,⾩}
  \definition*{s.}{risco de investimento}
\end{EntryWithPhonetic}

\begin{EntryWithPhonetic}{投资回报率}{tou2zi1 hui2bao4 lv4}{7,10,6,7,11}{⼿,⾙,⼞,⼿,⽞}
  \definition{s.}{retorno sobre o investimento (ROI)}
\end{EntryWithPhonetic}

\begin{EntryWithPhonetic}{投资家}{tou2zi1jia1}{7,10,10}{⼿,⾙,⼧}
  \definition{s.}{investidor}
  \seealsoref{投资人}{tou2zi1ren2}
  \seealsoref{投资者}{tou2zi1zhe3}
\end{EntryWithPhonetic}

\begin{EntryWithPhonetic}{投资人}{tou2zi1ren2}{7,10,2}{⼿,⾙,⼈}
  \definition{s.}{investidor}
  \seealsoref{投资家}{tou2zi1jia1}
  \seealsoref{投资者}{tou2zi1zhe3}
\end{EntryWithPhonetic}

\begin{EntryWithPhonetic}{投资者}{tou2zi1zhe3}{7,10,8}{⼿,⾙,⽼}
  \definition{s.}{investidor}
  \seealsoref{投资家}{tou2zi1jia1}
  \seealsoref{投资人}{tou2zi1ren2}
\end{EntryWithPhonetic}

%%%%%%%%%% 透 %%%%%%%%%%
\subsection*{透}\addcontentsline{loh}{figure}{透 \dpy{tou4}}

\begin{EntryWithPhonetic}{透}{tou4}{10}{⾡}[HSK 4]
  \definition{adv.}{totalmente; completamente; minuciosamente | profundamente; extremamente}
  \definition{v.}{penetrar; passar através de; infiltrar-se através de | revelar; deixar transparecer; contar secretamente |mostrar; aparecer}
\end{EntryWithPhonetic}

\begin{EntryWithPhonetic}{透彻}{tou4che4}{10,7}{⾡,⼻}[HSK 7-9]
  \definition{adj.}{penetrante; minucioso; incisivo; (compreensão da situação e análise das razões) detalhado e aprofundado}
  \synonymref{彻底}{che4di3}
  \synonymref{透顶}{tou4ding3}
  \synonymref{透辟}{tou4pi4}
  \antonymref{朦胧}{meng2long2}
\end{EntryWithPhonetic}

\begin{EntryWithPhonetic}{透澈}{tou4che4}{10,15}{⾡,⽔}
  \variantof{透彻}
\end{EntryWithPhonetic}

\begin{EntryWithPhonetic}{透顶}{tou4ding3}{10,8}{⾡,⾴}
  \definition{adv.}{completamente}
\end{EntryWithPhonetic}

\begin{EntryWithPhonetic}{透过}{tou4guo4}{10,6}{⾡,⾡}[HSK 7-9]
  \definition{v.}{passar por (algo ou espaço) | penetrar; infiltrar}
  \synonymref{穿过}{chuan1guo4}
\end{EntryWithPhonetic}

\begin{EntryWithPhonetic}{透亮}{tou4liang4}{10,9}{⾡,⼇}
  \definition{adj.}{brilhante | claro como cristal}
\end{EntryWithPhonetic}

\begin{EntryWithPhonetic}{透露}{tou4lu4}{10,21}{⾡,⾬}[HSK 6]
  \definition{v.}{vazar; revelar; expor; divulgar; contar deliberadamente um segredo a alguém; revelar um certo significado}
\end{EntryWithPhonetic}

\begin{EntryWithPhonetic}{透明}{tou4ming2}{10,8}{⾡,⽇}[HSK 4]
  \definition{adj.}{transparente; diáfano; capaz de transmitir luz | evidente; transparente; situação ou assunto que seja aberto e não oculto | transparente; diáfano; indica pureza, ausência de impurezas}
\end{EntryWithPhonetic}

\begin{EntryWithPhonetic}{透辟}{tou4pi4}{10,13}{⾡,⾟}
  \definition{adj.}{incisivo | penetrante}
\end{EntryWithPhonetic}

\begin{EntryWithPhonetic}{透气}{tou4/qi4}{10,4}{⾡,⽓}[HSK 7-9]
  \definition{v.+compl.}{ventilar; poder passar o ar | respirar livremente; respirar ar puro | vazar (ou revelar) informações; dar uma dica; alertar}
  \synonymref{通风}{tong1/feng1}
\end{EntryWithPhonetic}

\begin{EntryWithPhonetic}{透水}{tou4shui3}{10,4}{⾡,⽔}
  \definition{adj.}{permeável}
  \definition{s.}{vazamento de água}
\end{EntryWithPhonetic}

\begin{EntryWithPhonetic}{透支}{tou4zhi1}{10,4}{⾡,⽀}[HSK 7-9]
  \definition{v.}{fazer um descoberto bancário | exceder as despesas sobre as receitas; gastar em excesso | receber o salário antecipadamente}
  \synonymref{过度}{guo4du4}
  \antonymref{充沛}{chong1pei4}
\end{EntryWithPhonetic}

%%%%%%%%%% 头 %%%%%%%%%%
\subsection*{头}\addcontentsline{loh}{figure}{头 \dpy{tou5}}

\begin{EntryWithPhonetic}{头}{tou5}{5}{⼤}
  \definition{suf.}{adicionado após componentes nominais comuns | adicionado após o componente verbal, forma um substantivo abstrato, geralmente indicando que vale a pena realizar essa ação | adicionado após um componente adjetival, forma um substantivo, geralmente indicando algo abstrato | adicionado após o componente substantivo que indica a direção}
  \seeref{tou2}
\end{EntryWithPhonetic}

%%%%%%%%%% 凸 %%%%%%%%%%
\subsection*{凸}\addcontentsline{loh}{figure}{凸 \dpy{tu1}}

\begin{EntryWithPhonetic}{凸}{tu1}{5}{⼐}[HSK 7-9]
  \definition{adj.}{saliente; elevado | levantado; mais alto que o entorno}
  \definition{s.}{protuberância; saliência}
  \antonymref{凹}{ao1}
  \antonymref{平}{ping2}
\end{EntryWithPhonetic}

\begin{EntryWithPhonetic}{凸显}{tu1xian3}{5,9}{⼐,⽇}[HSK 7-9]
  \definition{v.}{apresentar com clareza; dar destaque a; enfatizar; revelar com clareza}
  \synonymref{显出}{xian3 chu1}
  \antonymref{隐藏}{yin3cang2}
\end{EntryWithPhonetic}

%%%%%%%%%% 秃 %%%%%%%%%%
\subsection*{秃}\addcontentsline{loh}{figure}{秃 \dpy{tu1}}

\begin{EntryWithPhonetic}{秃}{tu1}{7}{⽲}[HSK 7-9]
  \definition{adj.}{careca | rombudo; sem ponta | Coloquial: incompleto; insatisfatório | estéril; sem galhos ou folhas; sem árvores}
  \antonymref{尖}{jian1}
\end{EntryWithPhonetic}

%%%%%%%%%% 突 %%%%%%%%%%
\subsection*{突}\addcontentsline{loh}{figure}{突 \dpy{tu1}}

\begin{EntryWithPhonetic}{突}{tu1}{9}{⽳}
  \definition{adv.}{de repente; abruptamente; inesperadamente}
  \definition{s.}{chaminé}
  \definition{v.}{avançar rapidamente; atacar | projetar; destacar-se | romper | projetar-se; inchar; fazer bojo}
\end{EntryWithPhonetic}

\begin{EntryWithPhonetic}{突出}{tu1/chu1}{9,5}{⽳,⼐}[HSK 3]
  \definition{adj.}{proeminente; excelente; mais que a média}
  \definition{v.+compl.}{romper | enfatizar; destacar; dar destaque a | sobressair; projetar-se; destacar-se}
\end{EntryWithPhonetic}

\begin{EntryWithPhonetic}{突发}{tu1fa1}{9,5}{⽳,⼜}[HSK 7-9]
  \definition{v.}{surgir de repente; aparecer inesperadamente | explodir repentinamente}
\end{EntryWithPhonetic}

\begin{EntryWithPhonetic}{突击}{tu1ji1}{9,5}{⽳,⼐}[HSK 7-9]
  \definition[方]{s.}{ataque repentino e violento; agressão | Figurativo: trabalho apressado; esforço concentrado para terminar um trabalho rapidamente}
  \definition{v.}{fazer um ataque repentino e violento; agredir | fazer um esforço concentrado para terminar um trabalho rapidamente; fazer um trabalho às pressas}
\end{EntryWithPhonetic}

\begin{EntryWithPhonetic}{突破}{tu1/po4}{9,10}{⽳,⽯}[HSK 5]
  \definition{v.+compl.}{romper; fazer uma descoberta revolucionária; concentrar esforços em um único ponto de ataque, reunir o sucesso | quebrar (limite); superar (dificuldade); superar dificuldades; ultrapassar os números ou limites anteriores, superar recordes anteriores, etc.; romper com as limitações e restrições anteriores}
\end{EntryWithPhonetic}

\begin{EntryWithPhonetic}{突破口}{tu1po4kou3}{9,10,3}{⽳,⽯,⼝}[HSK 7-9]
  \definition{s.}{ponto de virada; avanço; solução inovadora | brecha; lacuna}
\end{EntryWithPhonetic}

\begin{EntryWithPhonetic}{突然}{tu1ran2}{9,12}{⽳,⽕}[HSK 3]
  \definition{adj.}{repentino; abrupto; inesperado}
  \definition{adv.}{de repente; abruptamente; inesperadamente; subitamente}
\end{EntryWithPhonetic}

\begin{EntryWithPhonetic}{突如其来}{tu1ru2-qi2lai2}{9,6,8,7}{⽳,⼥,⼋,⽊}[HSK 7-9]
  \definition{expr.}{surgir repentinamente; aparecer de repente; surgir do nada; aconteceu de forma inesperada e repentina}
\end{EntryWithPhonetic}

%%%%%%%%%% 图 %%%%%%%%%%
\subsection*{图}\addcontentsline{loh}{figure}{图 \dpy{tu2}}

\begin{EntryWithPhonetic}{图}{tu2}{8}{⼞}[HSK 3]
  \definition*{s.}{Sobrenome: Tu}
  \definition[张]{s.}{mapa; gráfico; imagem; desenho | plano; esquema; tentativa}
  \definition{v.}{procurar; perseguir; esperar obter| desenhar; retratar; pintar | imaginar; planejar; pensar; maquinar}
\end{EntryWithPhonetic}

\begin{EntryWithPhonetic}{图案}{tu2'an4}{8,10}{⼞,⽊}[HSK 4]
  \definition[种,个]{s.}{padrão; desenho; padrões e gráficos usados para decoração de edifícios, tecidos, artes e artesanato, etc.}
\end{EntryWithPhonetic}

\begin{EntryWithPhonetic}{图标}{tu2biao1}{8,9}{⼞,⽊}
  \definition{s.}{ícone (informática)}[这个图标代表设置。===Este ícone representa as configurações.]
\end{EntryWithPhonetic}

\begin{EntryWithPhonetic}{图表}{tu2biao3}{8,8}{⼞,⾐}[HSK 7-9]
  \definition[张,个]{s.}{carta; diagrama; gráfico | figura; gráfico; diagrama; pictograma; pictografia; cronograma; folha; tabela; termo geral para diagramas e tabelas que representam diversas situações e indicam vários números, como diagramas esquemáticos e tabelas estatísticas}
\end{EntryWithPhonetic}

\begin{EntryWithPhonetic}{图画}{tu2hua4}{8,8}{⼞,⽥}[HSK 3]
  \definition[幅,张,套]{s.}{desenho; imagem; pintura}
\end{EntryWithPhonetic}

\begin{EntryWithPhonetic}{图片}{tu2pian4}{8,4}{⼞,⽚}[HSK 2]
  \definition[张,幅]{s.}{imagem; fotografia; um termo geral para imagens, fotografias, decalques, etc. usados para ilustrar algo}
\end{EntryWithPhonetic}

\begin{EntryWithPhonetic}{图书}{tu2shu1}{8,4}{⼞,⼄}[HSK 6]
  \definition{s.}{livros; um termo geral para publicações como livros e álbuns de imagens}[这些图书都可以借阅。===Esses livros estão disponíveis para empréstimo.]
\end{EntryWithPhonetic}

\begin{EntryWithPhonetic}{图书馆}{tu2shu1guan3}{8,4,11}{⼞,⼄,⾷}[HSK 1]
  \definition[个,家]{s.}{biblioteca; instituição que coleta, organiza e armazena livros e materiais para leitura e consulta}
\end{EntryWithPhonetic}

\begin{EntryWithPhonetic}{图像}{tu2xiang4}{8,13}{⼞,⼈}[HSK 7-9]
  \definition{s.}{imagem; figura; imagens desenhadas, fotografadas ou impressas}
\end{EntryWithPhonetic}

\begin{EntryWithPhonetic}{图形}{tu2xing2}{8,7}{⼞,⼺}[HSK 7-9]
  \definition[种,个]{s.}{figura; gráfico | sinal; carta; desenho; diagrama}
  \synonymref{图案}{tu2'an4}
\end{EntryWithPhonetic}

\begin{EntryWithPhonetic}{图纸}{tu2zhi3}{8,7}{⼞,⽷}[HSK 7-9]
  \definition[张]{s.}{desenho; planta; folha de desenho; \emph{blueprint}; um documento técnico que utiliza desenhos e texto para descrever a estrutura, forma, dimensões e outros requisitos de estruturas de engenharia, máquinas, equipamentos, etc.}
\end{EntryWithPhonetic}

%%%%%%%%%% 徒 %%%%%%%%%%
\subsection*{徒}\addcontentsline{loh}{figure}{徒 \dpy{tu2}}

\begin{EntryWithPhonetic}{徒}{tu2}{10}{⼻}
  \definition{adj.}{vazio; nu}
  \definition{adv.}{somente; meramente; apenas | a pé | em vão; sem sucesso; sem sucesso}
  \definition{s.}{aprendiz; aluno | seguidor; crente | (pejorativo) pessoas da mesma facção | (pejorativo) pessoa; companheiro | (prisão) pena; prisão; sentença | estudante}
  \definition{v.}{estar a pé | andar}
\end{EntryWithPhonetic}

\begin{EntryWithPhonetic}{徒步}{tu2bu4}{10,7}{⼻,⽌}[HSK 7-9]
  \definition{adv.}{a pé}
  \synonymref{步行}{bu4xing2}
  \synonymref{散步}{san4/bu4}
\end{EntryWithPhonetic}

\begin{EntryWithPhonetic}{徒弟}{tu2di5}{10,7}{⼻,⼸}[HSK 6]
  \definition[位,名,个]{s.}{discípulo; aprendiz; uma pessoa que aprende com um mestre; geralmente se refere a uma pessoa que aprende com um especialista}[他是我的徒弟。===Ele é meu aprendiz.]
\end{EntryWithPhonetic}

\begin{EntryWithPhonetic}{徒手}{tu2shou3}{10,4}{⼻,⼿}
  \definition{adj.}{com as mãos vazias | desarmado | mão livre (desenho) | lutando mão-a-mão}
\end{EntryWithPhonetic}

%%%%%%%%%% 涂 %%%%%%%%%%
\subsection*{涂}\addcontentsline{loh}{figure}{涂 \dpy{tu2}}

\begin{EntryWithPhonetic}{涂}{tu2}{10}{⽔}[HSK 7-9]
  \definition*{s.}{Sobrenome: Tu}
  \definition{s.}{Literário: lama; pântano | praia; praia marítima; planícies de maré | estrada; caminho}
  \definition{v.}{aplicar; espalhar; aplicar sobre; permitir que tinta, corantes, cosméticos, medicamentos, etc., adiram a um objeto | rabiscar; escrever à mão; rabiscar ou desenhar aleatoriamente; escrever ou desenhar o que se quiser | riscar; anular; apagar}
  \synonymref{擦}{ca1}
  \synonymref{抹}{mo3}
\end{EntryWithPhonetic}

%%%%%%%%%% 途 %%%%%%%%%%
\subsection*{途}\addcontentsline{loh}{figure}{途 \dpy{tu2}}

\begin{EntryWithPhonetic}{途}{tu2}{10}{⾡}
  \definition[条]{s.}{caminho; estrada; rota | jornada; caminho}
\end{EntryWithPhonetic}

\begin{EntryWithPhonetic}{途径}{tu2jing4}{10,8}{⾡,⼻}[HSK 6]
  \definition[种,条,个]{s.}{caminho; canal; metaforicamente falando, uma maneira ou método de resolver um problema ou fazer algo}
\end{EntryWithPhonetic}

\begin{EntryWithPhonetic}{途中}{tu2zhong1}{10,4}{⾡,⼁}[HSK 4]
  \definition[家]{adv.}{no caminho; ao longo do caminho}
\end{EntryWithPhonetic}

%%%%%%%%%% 屠 %%%%%%%%%%
\subsection*{屠}\addcontentsline{loh}{figure}{屠 \dpy{tu2}}

\begin{EntryWithPhonetic}{屠}{tu2}{11}{⼫}
  \definition*{s.}{Sobrenome: Tu}
  \definition[个,位,名,些]{v.}{abater (animais para alimentação) | massacre; carnificina}
\end{EntryWithPhonetic}

\begin{EntryWithPhonetic}{屠杀}{tu2sha1}{11,6}{⼫,⽊}[HSK 7-9]
  \definition{s.}{massacre; carnificina; matança; assassinatos em massa}
\end{EntryWithPhonetic}

%%%%%%%%%% 土 %%%%%%%%%%
\subsection*{土}\addcontentsline{loh}{figure}{土 \dpy{tu3}}

\begin{EntryWithPhonetic}{土}{tu3}{3}{⼟}[HSK 3,6][Kangxi 32]
  \definition*{s.}{Sobrenome: Tu}
  \definition{adj.}{local; nativo; local com características regionais| caseiro; indígena; o que é tradicional no país; popular | não refinado; não esclarecido; não está na moda; não é popular}
  \definition[堆,捧,层]{s.}{solo; terra | terra; território | ópio bruto | cidade natal; terra natal; pátria}
\end{EntryWithPhonetic}

\begin{EntryWithPhonetic}{土地}{tu3di4}{3,6}{⼟,⼟}[HSK 4]
  \definition[片,块,顷]{s.}{terra; solo; chão; superfície terrestre da Terra usada para cultivar, construir edifícios e viver | território; território de um país}
  \seeref{tu3di5}
\end{EntryWithPhonetic}

\begin{EntryWithPhonetic}{土地}{tu3di5}{3,6}{⼟,⼟}
  \definition[片,块,顷]{s.}{deus da audeia; deus local; \emph{genius loci} deidade protetora de um local; Superstição: refere"-se ao deus da terra que governa uma pequena área}
  \seeref{tu3di4}
\end{EntryWithPhonetic}

\begin{EntryWithPhonetic}{土豆}{tu3dou4}{3,7}{⼟,⾖}[HSK 5]
  \definition[颗,斤,个,棵]{s.}{batata; denominação comum da batata}
\end{EntryWithPhonetic}

\begin{EntryWithPhonetic}{土豆泥}{tu3dou4ni2}{3,7,8}{⼟,⾖,⽔}
  \definition{s.}{purê de batata}[她的土豆泥确实不错。===O purê de batatas dela estava realmente muito bom.]
\end{EntryWithPhonetic}

\begin{EntryWithPhonetic}{土匪}{tu3fei3}{3,10}{⼟,⼕}[HSK 7-9]
  \definition{s.}{bandido; salteador}
  \seealsoref{胡匪}{hu2fei3}
  \synonymref{强盗}{qiang2dao4}
\end{EntryWithPhonetic}

\begin{EntryWithPhonetic}{土鸡}{tu3ji1}{3,7}{⼟,⿃}
  \definition{s.}{galinha caipira}
\end{EntryWithPhonetic}

\begin{EntryWithPhonetic}{土壤}{tu3rang3}{3,20}{⼟,⼟}[HSK 7-9]
  \definition{s.}{solo; uma camada solta de material na superfície terrestre, composta por diversos minerais granulares, matéria orgânica, água, ar, microrganismos, etc., que permite o crescimento de plantas}
  \synonymref{泥土}{ni2tu3}
\end{EntryWithPhonetic}

\begin{EntryWithPhonetic}{土生土长}{tu3sheng1-tu3zhang3}{3,5,3,4}{⼟,⽣,⼟,⾧}[HSK 7-9]
  \definition{expr.}{1. nativo; nascido e criado
Cultivado localmente}
\end{EntryWithPhonetic}

%%%%%%%%%% 吐 %%%%%%%%%%
\subsection*{吐}\addcontentsline{loh}{figure}{吐 \dpy{tu3}}

\begin{EntryWithPhonetic}{吐}{tu3}{6}{⼝}[HSK 5]
  \definition{v.}{cuspir; sair pela boca | surgir ou aparecer pela boca ou por uma fenda | dizer; contar; falar abertamente}
  \seeref{tu4}
\end{EntryWithPhonetic}

\begin{EntryWithPhonetic}{吐}{tu4}{6}{⼝}[HSK 5]
  \definition{v.}{vomitar; sair pela boca | vomitar; expelir; metáfora para ser forçado a devolver bens usurpados}
  \seeref{tu3}
\end{EntryWithPhonetic}

%%%%%%%%%% 兔 %%%%%%%%%%
\subsection*{兔}\addcontentsline{loh}{figure}{兔 \dpy{tu4}}

\begin{EntryWithPhonetic}{兔}{tu4}{8}{⼉}[HSK 5]
  \definition[只]{s.}{lebre; coelho}
\end{EntryWithPhonetic}

\begin{EntryWithPhonetic}{兔子}{tu4zi5}{8,3}{⼉,⼦}
  \definition[只]{s.}{coelho | lebre}
\end{EntryWithPhonetic}

%%%%%%%%%% 团 %%%%%%%%%%
\subsection*{团}\addcontentsline{loh}{figure}{团 \dpy{tuan2}}

\begin{EntryWithPhonetic}{团}{tuan2}{6}{⼞}[HSK 3]
  \definition*{s.}{Liga da Juventude Comunista da China; Liga}
  \definition{adj.}{redondo; circular}
  \definition{clas.}{usado para algo em forma de bola}
  \definition[个]{s.}{bolinho de massa; comida em forma de bola feita de arroz ou farinha | algo em forma de bola | grupo; corpo; sociedade; organização; um grupo envolvido em um determinado trabalho ou atividade | regimento; unidade organizacional militar, geralmente abaixo do nível de divisão e acima do nível de batalhão}
  \definition{v.}{enrolar algo para formar uma bola; rolar | reunir; unir; conglomerar}
\end{EntryWithPhonetic}

\begin{EntryWithPhonetic}{团队}{tuan2dui4}{6,4}{⼞,⾩}[HSK 6]
  \definition[个,支,种]{s.}{equipe; time; grupo; um grupo de alguma natureza}
\end{EntryWithPhonetic}

\begin{EntryWithPhonetic}{团伙}{tuan2huo3}{6,6}{⼞,⼈}[HSK 7-9]
  \definition[个]{s.}{gangue; bando; panelinha | gangue (criminosa) | cúmplice; comparsa; membro de gangue}
\end{EntryWithPhonetic}

\begin{EntryWithPhonetic}{团结}{tuan2jie2}{6,9}{⼞,⽷}[HSK 3]
  \definition{adj.}{unido; amigável; harmonioso; relação harmoniosa e coexistência harmoniosa}
  \definition{v.}{unir; reunir}
\end{EntryWithPhonetic}

\begin{EntryWithPhonetic}{团聚}{tuan2ju4}{6,14}{⼞,⽿}[HSK 7-9]
  \definition{v.}{reunir; reencontrar familiares ou amigos muito próximos após a separação | unir; reunir; congregar}
  \synonymref{重逢}{chong2feng2}
  \synonymref{欢聚}{huan1ju4}
  \synonymref{聚会}{ju4hui4}
  \synonymref{团圆}{tuan2yuan2}
  \synonymref{团员}{tuan2yuan2}
  \antonymref{分开}{fen1/kai1}
  \antonymref{分别}{fen1bie2}
  \antonymref{分离}{fen1li2}
  \antonymref{分散}{fen1san4}
\end{EntryWithPhonetic}

\begin{EntryWithPhonetic}{团体}{tuan2ti3}{6,7}{⼞,⼈}[HSK 3]
  \definition[种,个]{s.}{equipe; grupo; organização; um grupo de pessoas com objetivos e interesses comuns}
\end{EntryWithPhonetic}

\begin{EntryWithPhonetic}{团员}{tuan2yuan2}{6,7}{⼞,⼝}[HSK 7-9]
  \definition[名,位,个]{s.}{membro; membro de um grupo específico | membro da Liga; membro da Liga da Juventude Comunista da China; refere"-se especificamente aos membros da Liga da Juventude Comunista da China}
  \synonymref{会员}{hui4yuan2}
  \synonymref{团聚}{tuan2ju4}
\end{EntryWithPhonetic}

\begin{EntryWithPhonetic}{团圆}{tuan2yuan2}{6,10}{⼞,⼞}[HSK 7-9]
  \definition{adj.}{redondo; indica que a forma é circular}
  \definition{v.}{reunir; reencontrar membros da família após um período de separação}
  \synonymref{团聚}{tuan2ju4}
  \antonymref{分离}{fen1li2}
\end{EntryWithPhonetic}

\begin{EntryWithPhonetic}{团长}{tuan2zhang3}{6,4}{⼞,⾧}[HSK 5]
  \definition[位,名]{s.}{comandante do regimento | chefe (ou presidente) de uma delegação, trupe, etc. | líder de uma delegação}
\end{EntryWithPhonetic}

%%%%%%%%%% 推 %%%%%%%%%%
\subsection*{推}\addcontentsline{loh}{figure}{推 \dpy{tui1}}

\begin{EntryWithPhonetic}{推}{tui1}{11}{⼿}[HSK 2]
  \definition{v.}{empurrar; dar um encontrão | girar um moinho ou uma pedra de amolar; moer | cortar; aparar | impulsionar; promover; avançar | inferir; deduzir | afastar; fugir; deslocar | adiar | eleger; escolher | ter em alta estima; elogiar muito | declinar | selecionar | elogiar muito}
\end{EntryWithPhonetic}

\begin{EntryWithPhonetic}{推测}{tui1ce4}{11,9}{⼿,⽔}[HSK 7-9]
  \definition{v.}{inferir; supor; conjecturar; especular; estimar ou imaginar o desconhecido com base no conhecido}
  \synonymref{猜测}{cai1ce4}
  \synonymref{猜想}{cai1xiang3}
  \synonymref{揣测}{chuai3ce4}
  \synonymref{揣摩}{chuai3mo2}
  \synonymref{估计}{gu1ji4}
  \synonymref{探求}{tan4qiu2}
  \synonymref{推断}{tui1duan4}
  \antonymref{断定}{duan4ding4}
\end{EntryWithPhonetic}

\begin{EntryWithPhonetic}{推迟}{tui1chi2}{11,7}{⼿,⾡}[HSK 4]
  \definition{v.}{adiar; postergar; tardar; deixar para mais tarde}
\end{EntryWithPhonetic}

\begin{EntryWithPhonetic}{推出}{tui1chu1}{11,5}{⼿,⼐}[HSK 6]
  \definition{v.}{lançar; apresentar; fazer com que apareça diante do público | deduzir; tirar conclusões da análise}
\end{EntryWithPhonetic}

\begin{EntryWithPhonetic}{推辞}{tui1ci2}{11,13}{⼿,⾟}[HSK 7-9]
  \definition{s.}{indicar recusa (de compromissos, convites, presentes, etc.); recusar (pedidos, opiniões ou presentes)}
  \synonymref{不要}{bu2yao4}
  \synonymref{拒绝}{ju4jue2}
  \synonymref{推卸}{tui1xie4}
  \synonymref{退却}{tui4que4}
  \antonymref{承诺}{cheng2nuo4}
  \antonymref{答应}{da1ying5}
  \antonymref{接纳}{jie1na4}
  \antonymref{接收}{jie1shou1}
  \antonymref{接受}{jie1shou4}
  \antonymref{提出}{ti2 chu1}
\end{EntryWithPhonetic}

\begin{EntryWithPhonetic}{推动}{tui1 dong4}{11,6}{⼿,⼒}[HSK 3]
  \definition{v.}{promover; atuar; impulsionar; empurrar para a frente; dar ímpeto a; começar ou avançar algo (com alguma força); começar a trabalhar}
\end{EntryWithPhonetic}

\begin{EntryWithPhonetic}{推断}{tui1duan4}{11,11}{⼿,⽄}[HSK 7-9]
  \definition{v.}{inferir; deduzir; especular e concluir}
  \synonymref{猜测}{cai1ce4}
  \synonymref{猜想}{cai1xiang3}
  \synonymref{揣测}{chuai3ce4}
  \synonymref{估计}{gu1ji4}
  \synonymref{判断}{pan4duan4}
  \synonymref{推测}{tui1ce4}
  \synonymref{推理}{tui1li3}
  \antonymref{断定}{duan4ding4}
\end{EntryWithPhonetic}

\begin{EntryWithPhonetic}{推翻}{tui1/fan1}{11,18}{⼿,⽻}[HSK 7-9]
  \definition{v.+compl.}{tombar; virar; derrubar; derrubar o regime original ou mudar o sistema social | cancelar; reverter; repudiar; negar completamente as declarações, conclusões, decisões, etc., existentes}
  \synonymref{撤销}{che4xiao1}
  \synonymref{摧毁}{cui1hui3}
  \synonymref{打倒}{da3/dao3}
  \synonymref{颠覆}{dian1fu4}
  \antonymref{创建}{chuang4jian4}
  \antonymref{创立}{chuang4li4}
  \antonymref{建立}{jian4li4}
  \antonymref{证明}{zheng4ming2}
\end{EntryWithPhonetic}

\begin{EntryWithPhonetic}{推广}{tui1guang3}{11,3}{⼿,⼴}[HSK 3]
  \definition{v.}{espalhar; estender; promover; popularizar; expandir o escopo de uso ou função de algo}
\end{EntryWithPhonetic}

\begin{EntryWithPhonetic}{推荐}{tui1jian4}{11,9}{⼿,⾋}[HSK 7-9]
  \definition[份]{s.}{recomendação}
  \definition{v.}{recomendar; apresentar pessoas ou coisas boas a pessoas ou organizações, na esperança de empregá-las ou aceitá-las}
  \seealsoref{介绍}{jie4shao4}
  \synonymref{推出}{tui1chu1}
  \synonymref{推选}{tui1xuan3}
\end{EntryWithPhonetic}

\begin{EntryWithPhonetic}{推介}{tui1jie4}{11,4}{⼿,⼈}
  \definition{s.}{promoção}
  \definition{v.}{promover | introduzir e recomendar}
\end{EntryWithPhonetic}

\begin{EntryWithPhonetic}{推进}{tui1jin4}{11,7}{⼿,⾡}[HSK 3]
  \definition{v.}{avançar; empurrar; levar adiante; dar ímpeto a; promover o trabalho e fazê-lo avançar | empurrar; dirigir; avançar; seguir em frente; seguir em frente}
\end{EntryWithPhonetic}

\begin{EntryWithPhonetic}{推开}{tui1kai1}{11,4}{⼿,⼶}[HSK 3]
  \definition{v.}{declinar; rejeitar | empurrar para longe; aplicar força em uma determinada direção para mover uma pessoa ou objeto para longe de seu lugar original | empurrar para abrir (um portão, etc.); empurrar para fora para abrir algo que está fechado | estender; popularizar; promover para um alcance mais amplo e realizar em uma escala mais ampla}
\end{EntryWithPhonetic}

\begin{EntryWithPhonetic}{推理}{tui1li3}{11,11}{⼿,⽟}[HSK 7-9]
  \definition{s.}{inferência; raciocínio; o processo de tirar novas conclusões com base em informações existentes}
  \synonymref{推断}{tui1duan4}
\end{EntryWithPhonetic}

\begin{EntryWithPhonetic}{推敲}{tui1qiao1}{11,14}{⼿,⽁}[HSK 7-9]
  \definition{v.}{pesar; deliberar; pensar repetidamente ao escrever ou realizar tarefas}
  \synonymref{揣摩}{chuai3mo2}
  \synonymref{考虑}{kao3lv4}
  \synonymref{商量}{shang1liang5}
  \synonymref{思考}{si1kao3}
  \synonymref{思索}{si1suo3}
  \synonymref{研究}{yan2jiu1}
\end{EntryWithPhonetic}

\begin{EntryWithPhonetic}{推算}{tui1suan4}{11,14}{⼿,⽵}[HSK 7-9]
  \definition{v.}{calcular; estimar; calcular os valores relevantes com base nos dados existentes}
  \seealsoref{推测}{tui1ce4}
  \synonymref{计算}{ji4suan4}
  \synonymref{算计}{suan4ji4}
  \synonymref{阴谋}{yin1mou2}
  \antonymref{断定}{duan4ding4}
\end{EntryWithPhonetic}

\begin{EntryWithPhonetic}{推销}{tui1xiao1}{11,12}{⼿,⾦}[HSK 4]
  \definition{v.}{vender; comercializar; promover vendas; promover a comercialização de mercadorias}
\end{EntryWithPhonetic}

\begin{EntryWithPhonetic}{推卸}{tui1xie4}{11,9}{⼿,⼙}[HSK 7-9]
  \definition{v.}{esquivar"-se (da responsabilidade); recusar"-se a assumir (responsabilidades e obrigações, etc.)}
  \synonymref{推辞}{tui1ci2}
  \synonymref{退却}{tui4que4}
  \antonymref{承担}{cheng2dan1}
  \antonymref{承受}{cheng2shou4}
  \antonymref{担负}{dan1fu4}
\end{EntryWithPhonetic}

\begin{EntryWithPhonetic}{推行}{tui1xing2}{11,6}{⼿,⾏}[HSK 5]
  \definition{v.}{realizar; prosseguir; praticar | implementar; praticar; implementação generalizada; divulgar (experiências, métodos, etc.)}
\end{EntryWithPhonetic}

\begin{EntryWithPhonetic}{推选}{tui1xuan3}{11,9}{⼿,⾡}[HSK 7-9]
  \definition{v.}{eleger; escolher; nomear}
  \seealsoref{选}{xuan3}
  \synonymref{竞选}{jing4xuan3}
  \synonymref{推荐}{tui1jian4}
  \synonymref{选举}{xuan3ju3}
  \antonymref{指定}{zhi3ding4}
\end{EntryWithPhonetic}

\begin{EntryWithPhonetic}{推移}{tui1yi2}{11,11}{⼿,⽲}[HSK 7-9]
  \definition{v.}{passar; decorrer; movimento, mudança ou desenvolvimento}
  \synonymref{发展}{fa1zhan3}
  \synonymref{推动}{tui1 dong4}
  \synonymref{推进}{tui1jin4}
  \synonymref{移动}{yi2dong4}
\end{EntryWithPhonetic}

%%%%%%%%%% 颓 %%%%%%%%%%
\subsection*{颓}\addcontentsline{loh}{figure}{颓 \dpy{tui2}}

\begin{EntryWithPhonetic}{颓}{tui2}{13}{⾴}
  \definition{adj.}{arruinado; dilapidado | em declínio; decadente | abatido; desanimado; apático}
  \definition{s.}{Literário: vento de tempestade}
  \definition{v.}{colapsar | deteriorar; declinar | desanimar; deprimir | fluir para baixo}
\end{EntryWithPhonetic}

\begin{EntryWithPhonetic}{颓废}{tui2fei4}{13,8}{⾴,⼴}[HSK 7-9]
  \definition{adj.}{decadente; desanimado e apático}
  \definition{s.}{decadência}
  \synonymref{悲哀}{bei1'ai1}
  \synonymref{悲观}{bei1guan1}
  \synonymref{悲伤}{bei1shang1}
  \synonymref{灰心}{hui1/xin1}
  \synonymref{沮丧}{ju3sang4}
  \synonymref{失恋}{shi1/lian4}
  \synonymref{失望}{shi1wang4}
  \synonymref{消极}{xiao1ji2}
  \antonymref{鼓舞}{gu3wu3}
  \antonymref{积极}{ji1ji2}
  \antonymref{激情}{ji1qing2}
  \antonymref{励志}{li4zhi4}
  \antonymref{拼搏}{pin1bo2}
\end{EntryWithPhonetic}

%%%%%%%%%% 腿 %%%%%%%%%%
\subsection*{腿}\addcontentsline{loh}{figure}{腿 \dpy{tui3}}

\begin{EntryWithPhonetic}{腿}{tui3}{13}{⾁}[HSK 2]
  \definition[条,双]{s.}{perna; as partes dos humanos e dos animais que sustentam o corpo e permitem caminhar | um suporte em forma de perna; a parte inferior de um objeto que atua como uma perna e serve de suporte | presunto}
\end{EntryWithPhonetic}

\begin{EntryWithPhonetic}{腿号}{tui3hao4}{13,5}{⾁,⼝}
  \definition{s.}{anilha numerada (por exemplo, usada para identificar pássaros)}
  \seealsoref{腿号箍}{tui3hao4gu1}
\end{EntryWithPhonetic}

\begin{EntryWithPhonetic}{腿号箍}{tui3hao4gu1}{13,5,14}{⾁,⼝,⽵}
  \definition{s.}{anilha numerada (por exemplo, usada para identificar pássaros)}
  \seealsoref{腿号}{tui3hao4}
\end{EntryWithPhonetic}

%%%%%%%%%% 退 %%%%%%%%%%
\subsection*{退}\addcontentsline{loh}{figure}{退 \dpy{tui4}}

\begin{EntryWithPhonetic}{退}{tui4}{9}{⾡}[HSK 3]
  \definition{v.}{recuar; mover"-se para trás | remover; retirar; fazer recuar; mover para trás | desistir; retirar"-se de | refluir; declinar; retroceder | aposentar"-se; deixar o emprego por atingir a idade estipulada ou por problemas de saúde | retornar; reembolsar; devolver | romper; cancelar o que foi decidido}
  \antonymref{进}{jin4}
\end{EntryWithPhonetic}

\begin{EntryWithPhonetic}{退场}{tui4chang3}{9,6}{⾡,⼟}
  \definition{v.}{(atletas) retirar-se da arena (como após a cerimônia de abertura); sair da arena em marcha; atletas deixando o campo após a competição | (uma plateia) sair do teatro (como quando uma peça termina)}
\end{EntryWithPhonetic}

\begin{EntryWithPhonetic}{退出}{tui4 chu1}{9,5}{⾡,⼐}[HSK 3]
  \definition{v.}{desistir; retirar-se; separar-se; retirar-se de; abandonar o local ou outro lugar e parar de participar; abandonaar o grupo ou organização}
\end{EntryWithPhonetic}

\begin{EntryWithPhonetic}{退回}{tui4hui2}{9,6}{⾡,⼞}[HSK 7-9]
  \definition{v.}{devolver; retornar; enviar de volta | voltar (ou retornar); retornar ao lugar original}
  \synonymref{归还}{gui1huan2}
  \antonymref{索取}{suo3qu3}
\end{EntryWithPhonetic}

\begin{EntryWithPhonetic}{退票}{tui4piao4}{9,11}{⾡,⽰}[HSK 6]
  \definition{s.}{bilhete devolvido (ou não utilizado) | reembolso do bilhete}
  \definition{v.}{devolver um bilhete; obter um reembolso por um bilhete | devolver (um cheque)}
\end{EntryWithPhonetic}

\begin{EntryWithPhonetic}{退却}{tui4que4}{9,7}{⾡,⼙}[HSK 7-9]
  \definition{v.}{recuar; retirar"-se | recuar; encolher"-se; estremecer | voltar}
  \synonymref{后退}{hou4tui4}
  \synonymref{推辞}{tui1ci2}
  \synonymref{推卸}{tui1xie4}
  \synonymref{退缩}{tui4suo1}
  \synonymref{畏惧}{wei4ju4}
  \synonymref{畏缩}{wei4suo1}
  \antonymref{冲锋}{chong1feng1}
  \antonymref{坚守}{jian1shou3}
  \antonymref{进攻}{jin4gong1}
  \antonymref{前进}{qian2jin4}
  \antonymref{挺身}{ting3shen1}
\end{EntryWithPhonetic}

\begin{EntryWithPhonetic}{退让}{tui4rang4}{9,5}{⾡,⾔}[HSK 7-9]
  \definition{v.}{ceder; fazer uma concessão; aceitar | recuar; afastar"-e}
  \synonymref{让步}{rang4/bu4}
  \synonymref{退却}{tui4que4}
  \synonymref{妥协}{tuo3xie2}
\end{EntryWithPhonetic}

\begin{EntryWithPhonetic}{退缩}{tui4suo1}{9,14}{⾡,⽷}[HSK 7-9]
  \definition{v.}{encolher"-se; recuar; acovardar"-se}
  \synonymref{退却}{tui4que4}
  \synonymref{畏缩}{wei4suo1}
  \antonymref{进展}{jin4zhan3}
  \antonymref{扩张}{kuo4zhang1}
  \antonymref{前进}{qian2jin4}
\end{EntryWithPhonetic}

\begin{EntryWithPhonetic}{退休}{tui4/xiu1}{9,6}{⾡,⼈}[HSK 3]
  \definition{v.+compl.}{aposentar-se; os trabalhadores que deixarem o emprego por velhice ou invalidez causada pelo trabalho receberão as despesas de subsistência conforme o cronograma}
\end{EntryWithPhonetic}

\begin{EntryWithPhonetic}{退休金}{tui4xiu1jin1}{9,6,8}{⾡,⼈,⾦}[HSK 7-9]
  \definition{s.}{aposentadoria; pensão}
\end{EntryWithPhonetic}

\begin{EntryWithPhonetic}{退学}{tui4/xue2}{9,8}{⾡,⼦}[HSK 7-9]
  \definition{v.+compl.}{abandonar a escola; interromper os estudos; a matrícula de um aluno pode ser cancelada caso ele não consiga continuar seus estudos por qualquer motivo, ou caso seja desqualificado de prosseguir com os estudos devido a graves infrações disciplinares}
  \antonymref{入学}{ru4/xue2}
\end{EntryWithPhonetic}

\begin{EntryWithPhonetic}{退役}{tui4/yi4}{9,7}{⾡,⼻}[HSK 7-9]
  \definition{v.+compl.}{aposentar-se das forças armadas; os soldados deixam as forças armadas após servirem por um determinado número de anos | aposentar-se da carreira profissional; os atletas já não encaram as competições desportivas como a sua profissão | ser retirado de serviço; equipamentos militares, instalações de transporte, etc., são desativados após um certo número de anos de uso}
\end{EntryWithPhonetic}

%%%%%%%%%% 吞 %%%%%%%%%%
\subsection*{吞}\addcontentsline{loh}{figure}{吞 \dpy{tun1}}

\begin{EntryWithPhonetic}{吞}{tun1}{7}{⼝}[HSK 6]
  \definition*{s.}{Sobrenome: Tun}
  \definition{v.}{engolir; engolir em seco | tomar posse de; anexar | engolir; tragar; devorar; engolir inteiro ou em pedaços | absorver; engolir; engolfar}
\end{EntryWithPhonetic}

%%%%%%%%%% 屯 %%%%%%%%%%
\subsection*{屯}\addcontentsline{loh}{figure}{屯 \dpy{tun2}}

\begin{EntryWithPhonetic}{屯}{tun2}{4}{⼬}[HSK 7-9]
  \definition*{s.}{Sobrenome: Tun}
  \definition{s.}{vila (geralmente usado em nomes de vilas); vilarejos; aldeias; povoados}
  \definition{v.}{coletar; estocar; armazenar; acumular | estacionar (tropas); aquartelar}
  \seeref{zhun1}
\end{EntryWithPhonetic}

%%%%%%%%%% 托 %%%%%%%%%%
\subsection*{托}\addcontentsline{loh}{figure}{托 \dpy{tuo1}}

\begin{EntryWithPhonetic}{托}{tuo1}{6}{⼿}[HSK 6]
  \definition{clas.}{torr, uma unidade de pressão, 1 torr é igual à pressão de 1 mmHg, ou 133,322 Pa}
  \definition{s.}{algo servindo como suporte | fantoche; cúmplice; pessoas que ajudam golpistas a enganar outras pessoas}
  \definition{v.}{segurar na palma; apoiar com a mão ou palma; suportar (um objeto) com um objeto ou com a palma da mão | destacar; servir como contraste | pedir; confiar | implorar; dar como pretexto | dever a; confiar em}
\end{EntryWithPhonetic}

\begin{EntryWithPhonetic}{托付}{tuo1fu4}{6,5}{⼿,⼈}[HSK 7-9]
  \definition{v.}{confiar; entregar algo aos cuidados de alguém; pedir a alguém que cuide de você ou que faça algo por você}
  \synonymref{拜托}{bai4tuo1}
  \synonymref{吩咐}{fen1fu4}
  \synonymref{寄托}{ji4tuo1}
  \synonymref{交付}{jiao1fu4}
  \synonymref{委托}{wei3tuo1}
  \synonymref{嘱托}{zhu3tuo1}
\end{EntryWithPhonetic}

%%%%%%%%%% 拖 %%%%%%%%%%
\subsection*{拖}\addcontentsline{loh}{figure}{拖 \dpy{tuo1}}

\begin{EntryWithPhonetic}{拖}{tuo1}{8}{⼿}[HSK 6]
  \definition{v.}{puxar; arrastar; transportar; puxar um objeto para movê-lo contra o solo ou outra superfície | esfregar; limpar o chão com uma ferramenta especial para esfregar | atrasar; prolongar; procrastinar; arrastar; coisas que deveriam ser feitas nunca são iniciadas ou concluídas; uma certa nota é prolongada por um longo tempo | atrasar; conter; segurar; restringir}
\end{EntryWithPhonetic}

\begin{EntryWithPhonetic}{拖拉机}{tuo1la1ji1}{8,8,6}{⼿,⼿,⽊}
  \definition[台]{s.}{trator}
\end{EntryWithPhonetic}

\begin{EntryWithPhonetic}{拖累}{tuo1lei3}{8,11}{⼿,⽷}[HSK 7-9]
  \definition{v.}{onerar; ser um fardo para; sobrecarregar outras pessoas ou coisas, impedindo seu desenvolvimento harmonioso}
  \antonymref{轻松}{qing1song1}
\end{EntryWithPhonetic}

\begin{EntryWithPhonetic}{拖欠}{tuo1qian4}{8,4}{⼿,⽋}[HSK 7-9]
  \definition{v.}{estar em atraso; estar com pagamentos atrasados; atrasar a devolução ou reter o pagamento}
  \synonymref{拖延}{tuo1yan2}
  \antonymref{偿还}{chang2huan2}
\end{EntryWithPhonetic}

\begin{EntryWithPhonetic}{拖鞋}{tuo1xie2}{8,15}{⼿,⾰}[HSK 6]
  \definition[双,只]{s.}{chinelos; sandálias; babouche; sapatos sem cabedal geralmente são usados em ambientes fechados}
\end{EntryWithPhonetic}

\begin{EntryWithPhonetic}{拖延}{tuo1yan2}{8,6}{⼿,⼵}[HSK 7-9]
  \definition{v.}{adiar; postergar; procrastinar; pendurar; prolongar o tempo, não processar rapidamente}
  \synonymref{迟到}{chi2/dao4}
  \synonymref{耽搁}{dan1ge5}
  \synonymref{耽误}{dan1wu5}
  \synonymref{缓慢}{huan3man4}
  \synonymref{拖欠}{tuo1qian4}
\end{EntryWithPhonetic}

%%%%%%%%%% 脱 %%%%%%%%%%
\subsection*{脱}\addcontentsline{loh}{figure}{脱 \dpy{tuo1}}

\begin{EntryWithPhonetic}{脱}{tuo1}{11}{⾁}[HSK 4]
  \definition{conj.}{se; no caso}
  \definition{v.}{(cabelo, pele) soltar-se; desprender-se; cair | retirar peça de roupa do corpo | sair de; escapar de | perder (palavras) | livrar-se de algo}
\end{EntryWithPhonetic}

\begin{EntryWithPhonetic}{脱节}{tuo1/jie2}{11,5}{⾁,⾋}[HSK 7-9]
  \definition{v.+compl.}{desfazer-se; estar desalinhado; estar em desacordo com | desengatar | estar fora de articulação}
  \synonymref{摆脱}{bai3/tuo1}
  \synonymref{离开}{li2/kai1}
  \synonymref{脱离}{tuo1li2}
  \antonymref{连接}{lian2jie1}
  \antonymref{联系}{lian2xi4}
\end{EntryWithPhonetic}

\begin{EntryWithPhonetic}{脱口而出}{tuo1kou3'er2chu1}{11,3,6,5}{⾁,⼝,⽽,⼐}[HSK 7-9]
  \definition{expr.}{deixar escapar; dizer algo sem querer; sem pensar, simplesmente deixando escapar}
  \synonymref{不假思索}{bu4jia3-si1suo3}
\end{EntryWithPhonetic}

\begin{EntryWithPhonetic}{脱离}{tuo1li2}{11,10}{⾁,⼇}[HSK 5]
  \definition{v.}{separar-se; divorciar-se; afastar-se; sair (de um determinado ambiente ou situação); romper (uma determinada relação)}
\end{EntryWithPhonetic}

\begin{EntryWithPhonetic}{脱落}{tuo1luo4}{11,12}{⾁,⾋}[HSK 7-9]
  \definition{v.}{cair; despencar; desprender; descascar | omitir (um caractere ao escrever); omitir texto}
\end{EntryWithPhonetic}

\begin{EntryWithPhonetic}{脱毛}{tuo1mao2}{11,4}{⾁,⽑}
  \definition{s.}{depilação}
  \definition{v.}{perder cabelo ou penas | depilar | fazer a barba}
\end{EntryWithPhonetic}

\begin{EntryWithPhonetic}{脱身}{tuo1/shen1}{11,7}{⾁,⾝}[HSK 7-9]
  \definition{v.+compl.}{escapar; libertar"-se; livrar"-se; sair de um determinado lugar; livrar"-se de algo}
\end{EntryWithPhonetic}

\begin{EntryWithPhonetic}{脱险}{tuo1xian3}{11,9}{⾁,⾩}
  \definition{v.}{sair do perigo}
\end{EntryWithPhonetic}

\begin{EntryWithPhonetic}{脱颖而出}{tuo1ying3'er2chu1}{11,13,6,5}{⾁,⾴,⽽,⼐}[HSK 7-9]
  \definition{expr.}{``A ponta inteira da sovela é visível através do saco de pano.''; isso significa, metaforicamente, que os talentos de uma pessoa talentosa acabarão sendo revelados; destacar"-se; sobressair"-se da multidão; sobressair"-se; emergir de forma proeminente; distinguir"-se}
\end{EntryWithPhonetic}

%%%%%%%%%% 驮 %%%%%%%%%%
\subsection*{驮}\addcontentsline{loh}{figure}{驮 \dpy{tuo2}}

\begin{EntryWithPhonetic}{驮}{tuo2}{6}{⾺}[HSK 7-9]
  \definition{v.}{carregar nas costas; apoiar objetos com as costas; suportar nas costas}
  \seeref{duo4}
\end{EntryWithPhonetic}

%%%%%%%%%% 柁 %%%%%%%%%%
\subsection*{柁}\addcontentsline{loh}{figure}{柁 \dpy{tuo2}}

\begin{EntryWithPhonetic}{柁}{tuo2}{9}{⽊}
  \definition{s.}{leme; leme | viga; uma grande viga horizontal em uma treliça de telhado de madeira}
  \seealsoref{舵}{duo4}
\end{EntryWithPhonetic}

%%%%%%%%%% 鸵 %%%%%%%%%%
\subsection*{鸵}\addcontentsline{loh}{figure}{鸵 \dpy{tuo2}}

\begin{EntryWithPhonetic}{鸵}{tuo2}{10}{⿃}
  \definition[只]{s.}{avestruz}
\end{EntryWithPhonetic}

\begin{EntryWithPhonetic}{鸵鸟}{tuo2niao3}{10,5}{⿃,⿃}
  \definition{s.}{avestruz}
\end{EntryWithPhonetic}

%%%%%%%%%% 妥 %%%%%%%%%%
\subsection*{妥}\addcontentsline{loh}{figure}{妥 \dpy{tuo3}}

\begin{EntryWithPhonetic}{妥}{tuo3}{7}{⼥}[HSK 7-9]
  \definition*{s.}{Sobrenome: Tuo}
  \definition{adj.}{apropriado; adequado | (geralmente após um verbo) pronto; resolvido; terminado}
\end{EntryWithPhonetic}

\begin{EntryWithPhonetic}{妥当}{tuo3dang4}{7,6}{⼥,⼹}[HSK 7-9]
  \definition{adj.}{adequado; apropriado; conveniente; prudente e apropriado}
  \synonymref{得当}{de2dang4}
  \synonymref{恰当}{qia4dang4}
  \synonymref{适当}{shi4dang4}
  \synonymref{事宜}{shi4yi2}
  \synonymref{停当}{ting2dang5}
  \synonymref{妥善}{tuo3shan4}
\end{EntryWithPhonetic}

\begin{EntryWithPhonetic}{妥善}{tuo3shan4}{7,12}{⼥,⼝}[HSK 7-9]
  \definition{adj.}{adequado; apropriado; bem organizado; adequado e completo}
  \synonymref{得当}{de2dang4}
  \synonymref{恰当}{qia4dang4}
  \synonymref{适宜}{shi4yi2}
  \synonymref{事宜}{shi4yi2}
  \synonymref{停当}{ting2dang5}
  \synonymref{妥当}{tuo3dang4}
\end{EntryWithPhonetic}

\begin{EntryWithPhonetic}{妥协}{tuo3xie2}{7,6}{⼥,⼗}[HSK 7-9]
  \definition{v.}{fazer concessões; chegar a um acordo; evitar conflitos ou disputas fazendo concessões}
  \synonymref{和解}{he2jie3}
  \synonymref{和睦}{he2mu4}
  \synonymref{迁就}{qian1jiu4}
  \synonymref{让步}{rang4/bu4}
  \synonymref{退让}{tui4rang4}
  \synonymref{协调}{xie2tiao2}
  \antonymref{斗争}{dou4zheng1}
  \antonymref{对立}{dui4li4}
  \antonymref{对峙}{dui4zhi4}
  \antonymref{挑衅}{tiao3xin4}
  \antonymref{争议}{zheng1yi4}
\end{EntryWithPhonetic}

%%%%%%%%%% 拓 %%%%%%%%%%
\subsection*{拓}\addcontentsline{loh}{figure}{拓 \dpy{tuo4}}

\begin{EntryWithPhonetic}{拓}{tuo4}{8}{⼿}
  \definition*{s.}{Sobrenome: Tuo}
  \definition{v.}{abrir; desenvolver | expandir}
  \seeref{ta4}
\end{EntryWithPhonetic}

\begin{EntryWithPhonetic}{拓宽}{tuo4kuan1}{8,10}{⼿,⼧}[HSK 7-9]
  \definition{v.}{ampliar; expandir}
  \synonymref{开拓}{kai1tuo4}
  \synonymref{拓展}{tuo4zhan3}
  \antonymref{收缩}{shou1suo1}
\end{EntryWithPhonetic}

\begin{EntryWithPhonetic}{拓展}{tuo4zhan3}{8,10}{⼿,⼫}[HSK 7-9]
  \definition{v.}{expandir; desenvolver; disseminar; ampliar o escopo}
  \synonymref{开阔}{kai1kuo4}
  \synonymref{开拓}{kai1tuo4}
  \synonymref{扩展}{kuo4zhan3}
  \synonymref{拓宽}{tuo4kuan1}
  \synonymref{延伸}{yan2shen1}
  \antonymref{缩小}{suo1/xiao3}
\end{EntryWithPhonetic}

%%%%%%%%%% 唾 %%%%%%%%%%
\subsection*{唾}\addcontentsline{loh}{figure}{唾 \dpy{tuo4}}

\begin{EntryWithPhonetic}{唾}{tuo4}{11}{⼝}
  \definition[口]{s.}{saliva; cuspe}
  \definition{v.}{cuspir (mostrar desprezo) | rejeitar}
\end{EntryWithPhonetic}

\begin{EntryWithPhonetic}{唾骂}{tuo4ma4}{11,9}{⼝,⾺}
  \definition{v.}{insultar | amaldiçoar}
\end{EntryWithPhonetic}

\begin{EntryWithPhonetic}{唾液}{tuo4ye4}{11,11}{⼝,⽔}[HSK 7-9]
  \definition[口]{s.}{saliva; cuspe}
  \synonymref{口水}{kou3shui3}
\end{EntryWithPhonetic}

%%%%%%%%%% 魄 %%%%%%%%%%
\subsection*{魄}\addcontentsline{loh}{figure}{魄 \dpy{tuo4}}

\begin{EntryWithPhonetic}{魄}{tuo4}{14}{⿁}
  \definition{adj.}{desanimado; sem ânimo; mentalmente abatido; outra pronúncia de 魄 em 落魄}
  \seealsoref{落魄}{luo4po4}
\end{EntryWithPhonetic}

%%%%% EOF %%%%%


%%%%%%%%%%%%%%%%%%%%%%%%%%%%%%%% Não existem palavras com pinyin iniciado em "U"
%%%%%%%%%%%%%%%%%%%%%%%%%%%%%%%% Não existem palavras com pinyin iniciado em "V"
 %%%
%%% W
%%%
\section*{W}\addcontentsline{toc}{section}{W}\addcontentsline{loh}{figure}{\#\#\#\#\#\#\#\# W}

%%%%%%%%%% 凹 %%%%%%%%%%
\subsection*{凹}\addcontentsline{loh}{figure}{凹 \dpy{wa1}}

\begin{EntryWithPhonetic}{凹}{wa1}{5}{⼐}
  \variantof{洼}
  \seeref{ao1}
\end{EntryWithPhonetic}

%%%%%%%%%% 哇 %%%%%%%%%%
\subsection*{哇}\addcontentsline{loh}{figure}{哇 \dpy{wa1}}

\begin{EntryWithPhonetic}{哇}{wa1}{9}{⼝}
  \definition{interj.}{Onomatopéia: som de choro ou vômito | ``Uau!''; expressa surpresa}
  \seeref{wa5}
\end{EntryWithPhonetic}

\begin{EntryWithPhonetic}{哇塞}{wa1sai1}{9,13}{⼝,⼟}
  \definition{interj.}{``Uau!'; exclamação de espanto, admiração, etc.}
\end{EntryWithPhonetic}

\begin{EntryWithPhonetic}{哇噻}{wa1sai1}{9,16}{⼝,⼝}
  \variantof{哇塞}
\end{EntryWithPhonetic}

%%%%%%%%%% 挖 %%%%%%%%%%
\subsection*{挖}\addcontentsline{loh}{figure}{挖 \dpy{wa1}}

\begin{EntryWithPhonetic}{挖}{wa1}{9}{⼿}[HSK 6]
  \definition{v.}{cavar; escavar; arrancar | explorar; sondar | (dialeto) arranhar | escavar a superfície de um objeto com ferramentas ou mãos}
\end{EntryWithPhonetic}

\begin{EntryWithPhonetic}{挖掘}{wa1jue2}{9,11}{⼿,⼿}[HSK 7-9]
  \definition{v.}{desenterrar; escavar; aprofundar"-se | sondar; conectar"-se a; metaforicamente, significa desenvolver em profundidade e buscar}
  \synonymref{发掘}{fa1jue2}
  \synonymref{发现}{fa1xian4}
  \synonymref{开采}{kai1cai3}
  \antonymref{埋藏}{mai2cang2}
  \antonymref{埋没}{mai2mo4}
\end{EntryWithPhonetic}

\begin{EntryWithPhonetic}{挖掘机}{wa1jue2ji1}{9,11,6}{⼿,⼿,⽊}
  \definition{s.}{escavadeira; máquina de escavar; pá mecânica}
  \seealsoref{挖土机}{wa1tu3ji1}
\end{EntryWithPhonetic}

\begin{EntryWithPhonetic}{挖苦}{wa1ku5}{9,8}{⼿,⾋}[HSK 7-9]
  \definition{v.}{falar sarcasticamente; ridicularizar alguém com palavras ácidas e sarcásticas}
  \synonymref{嘲弄}{chao2nong4}
  \synonymref{嘲笑}{chao2xiao4}
  \synonymref{讽刺}{feng3ci4}
  \synonymref{讥笑}{ji1xiao4}
  \antonymref{称赞}{cheng1zan4}
  \antonymref{恭维}{gong1wei2}
  \antonymref{赞赏}{zan4shang3}
  \antonymref{赞扬}{zan4yang2}
\end{EntryWithPhonetic}

\begin{EntryWithPhonetic}{挖土机}{wa1tu3ji1}{9,3,6}{⼿,⼟,⽊}
  \definition{s.}{escavadeira; máquina de escavar | retroescavadeira}
\end{EntryWithPhonetic}

%%%%%%%%%% 洼 %%%%%%%%%%
\subsection*{洼}\addcontentsline{loh}{figure}{洼 \dpy{wa1}}

\begin{EntryWithPhonetic}{洼}{wa1}{9}{⽔}
  \definition{adj.}{oco; baixo}
  \definition{s.}{área baixa; depressão; oco}
\end{EntryWithPhonetic}

%%%%%%%%%% 娃 %%%%%%%%%%
\subsection*{娃}\addcontentsline{loh}{figure}{娃 \dpy{wa2}}

\begin{EntryWithPhonetic}{娃}{wa2}{9}{⼥}
  \definition[个,名,位,只]{s.}{bebê; criança | filho ou filha; criança | Dialeto: animal recém-nascido | Literário: menina; jovem mulher | Literário: menina bonita}
\end{EntryWithPhonetic}

\begin{EntryWithPhonetic}{娃娃}{wa2wa5}{9,9}{⼥,⼥}[HSK 6]
  \definition[个,名,位]{s.}{bebê; criança; criança pequena | boneca; brinquedos em forma de crianças}
\end{EntryWithPhonetic}

%%%%%%%%%% 瓦 %%%%%%%%%%
\subsection*{瓦}\addcontentsline{loh}{figure}{瓦 \dpy{wa3}}

\begin{EntryWithPhonetic}{瓦}{wa3}{4}{⽡}[HSK 7-9][Kangxi 98]
  \definition{clas.}{W, watt; medida de potência elétrica}
  \definition[片,块]{s.}{telha; os materiais para telhados são geralmente feitos de barro cozido, mas também podem ser feitos de cimento ou outros materiai; eles vêm em vários formatos, incluindo arqueados, planos ou semicilíndricos | cerâmica; algo feito de argila ou barro cozido}
  \seeref{wa4}
  \seealsoref{瓦特}{wa3te4}
\end{EntryWithPhonetic}

\begin{EntryWithPhonetic}{瓦努阿图}{wa3nu3'a1tu2}{4,7,7,8}{⽡,⼒,⾩,⼞}
  \definition*{s.}{Vanuatu, país do sudoeste do Oceano Pacífico}
\end{EntryWithPhonetic}

\begin{EntryWithPhonetic}{瓦特}{wa3te4}{4,10}{⽡,⽜}
  \definition{s.}{Empréstimo Linguístico: watt | medida de potência}
\end{EntryWithPhonetic}

\begin{EntryWithPhonetic}{瓦}{wa4}{4}{⽡}[Kangxi 98]
  \definition{v.}{colocar telhas (cobertura de telhados)}
  \seeref{wa3}
\end{EntryWithPhonetic}

%%%%%%%%%% 袜 %%%%%%%%%%
\subsection*{袜}\addcontentsline{loh}{figure}{袜 \dpy{wa4}}

\begin{EntryWithPhonetic}{袜}{wa4}{10}{⾐}
  \definition[只,双,打]{s.}{meias; meias-calças}
\end{EntryWithPhonetic}

\begin{EntryWithPhonetic}{袜子}{wa4zi5}{10,3}{⾐,⼦}[HSK 4]
  \definition[双,只,对]{s.}{meias; peúgas; meias-calças}
\end{EntryWithPhonetic}

%%%%%%%%%% 哇 %%%%%%%%%%
\subsection*{哇}\addcontentsline{loh}{figure}{哇 \dpy{wa5}}

\begin{EntryWithPhonetic}{哇}{wa5}{9}{⼝}[HSK 6]
  \definition{part.}{a mudança do som de 啊 devido à influência do som final da palavra anterior, ``u'' ou ``ao''}
  \seeref{wa1}
  \seealsoref{啊}{a5}
\end{EntryWithPhonetic}

%%%%%%%%%% 歪 %%%%%%%%%%
\subsection*{歪}\addcontentsline{loh}{figure}{歪 \dpy{wai1}}

\begin{EntryWithPhonetic}{歪}{wai1}{9}{⽌}[HSK 7-9]
  \definition{adj.}{torto; desalinhado; oblíquo; inclinado; não reto; enviesado | tortuoso; ardiloso; indecente; impróprio; desonesto}
  \definition{v.}{estar inclinado | reclinar-se para descansar}
  \antonymref{偏}{pian1}
  \antonymref{斜}{xie2}
  \antonymref{正}{zheng4}
\end{EntryWithPhonetic}

\begin{EntryWithPhonetic}{歪果仁}{wai1 guo3 ren2}{9,8,4}{⽌,⽊,⼈}
  \definition{s.}{gíria na \emph{Internet} para estrangeiro (外国人)}
  \seealsoref{外国人}{wai4 guo2 ren2}
\end{EntryWithPhonetic}

\begin{EntryWithPhonetic}{歪曲}{wai1qu1}{9,6}{⽌,⽈}[HSK 7-9]
  \definition{v.}{distorcer as coisas intencionalmente (geralmente se referindo a retratar coisas boas como ruins)}
  \synonymref{扭曲}{niu3qu1}
  \synonymref{误解}{wu4jie3}
\end{EntryWithPhonetic}

%%%%%%%%%% 外 %%%%%%%%%%
\subsection*{外}\addcontentsline{loh}{figure}{外 \dpy{wai4}}

\begin{EntryWithPhonetic}{外}{wai4}{5}{⼣}[HSK 1]
  \definition{adj.}{outro (que não o próprio) | não íntimo; não intimamente relacionado | não oficial | exterior; externo; do lado de fora | outros; referindo"-se a um local fora da sua localização atual | do lado da mãe, da irmã ou da filha; referir"-se a parentes do lado materno, irmãs ou filhas | informal; não oficial}
  \definition{adv.}{adicionalmente; além disso | para fora; para o exterior; fora | extra; além disso}
  \definition{s.}{fora; externo; exterior | outro local; outro lugar | estrangeiro; país estrangeiro | lado externo | parentes de sua mãe, irmãs ou filhas}
  \antonymref{里}{li3}
  \antonymref{内}{nei4}
\end{EntryWithPhonetic}

\begin{EntryWithPhonetic}{外币}{wai4bi4}{5,4}{⼣,⼱}[HSK 6]
  \definition[种]{s.}{moeda estrangeira}
\end{EntryWithPhonetic}

\begin{EntryWithPhonetic}{外边}{wai4bian5}{5,5}{⼣,⾡}[HSK 1]
  \definition{s.}{fora; exterior; externo; além de um determinado limite | local diferente de onde se vive ou trabalha; referindo"-se a lugares distantes | exterior; externo; superfície}
\end{EntryWithPhonetic}

\begin{EntryWithPhonetic}{外表}{wai4biao3}{5,8}{⼣,⾐}[HSK 7-9]
  \definition[个]{s.}{exterior; superfície; aparência}
  \synonymref{表面}{biao3mian4}
  \synonymref{概况}{gai4kuang4}
  \synonymref{轮廓}{lun2kuo4}
  \synonymref{外观}{wai4guan1}
  \synonymref{外貌}{wai4mao4}
  \synonymref{外形}{wai4xing2}
  \synonymref{外面}{wai4mian5}
  \antonymref{内涵}{nei4han2}
  \antonymref{内心}{nei4xin1}
  \antonymref{内在}{nei4zai4}
\end{EntryWithPhonetic}

\begin{EntryWithPhonetic}{外部}{wai4bu4}{5,10}{⼣,⾢}[HSK 6]
  \definition{s.}{fora; externo; fora de um certo intervalo | exterior; superfície}
\end{EntryWithPhonetic}

\begin{EntryWithPhonetic}{外插}{wai4cha1}{5,12}{⼣,⼿}
  \definition{v.}{extrapolar | (computação) conectar (um dispositivo periférico, etc.)}
\end{EntryWithPhonetic}

\begin{EntryWithPhonetic}{外出}{wai4chu1}{5,5}{⼣,⼐}[HSK 6]
  \definition{v.}{sair, especialmente para ir a outro lugar a negócios}
\end{EntryWithPhonetic}

\begin{EntryWithPhonetic}{外地}{wai4di4}{5,6}{⼣,⼟}[HSK 2]
  \definition{s.}{não local; outros lugares; locais fora da área local}
\end{EntryWithPhonetic}

\begin{EntryWithPhonetic}{外公}{wai4gong1}{5,4}{⼣,⼋}[HSK 7-9]
  \definition[位,名,个]{s.}{avô materno; pai da mãe}
  \antonymref{孙子}{sun1zi5}
\end{EntryWithPhonetic}

\begin{EntryWithPhonetic}{外观}{wai4guan1}{5,6}{⼣,⾒}[HSK 6]
  \definition{s.}{aspecto; semblante; aparência; aparência exterior; a aparência de um objeto}
\end{EntryWithPhonetic}

\begin{EntryWithPhonetic}{外国}{wai4guo2}{5,8}{⼣,⼞}[HSK 1]
  \definition[个]{s.}{país estrangeiro}
\end{EntryWithPhonetic}

\begin{EntryWithPhonetic}{外国人}{wai4 guo2 ren2}{5,8,2}{⼣,⼞,⼈}
  \definition[个]{s.}{estrangeiro | alienígena}
\end{EntryWithPhonetic}

\begin{EntryWithPhonetic}{外海}{wai4hai3}{5,10}{⼣,⽔}
  \definition{s.}{mar aberto}
\end{EntryWithPhonetic}

\begin{EntryWithPhonetic}{外行}{wai4hang2}{5,6}{⼣,⾏}[HSK 7-9]
  \definition{adj.}{sem habilidade, inexperiente, não especialista; descreve alguém que não possui absolutamente nenhum conhecimento ou experiência em determinada profissão ou tecnologia}
  \definition[个]{s.}{leigo; forasteiro; pessoas que não possuem conhecimento ou experiência em determinada profissão ou tecnologia}
  \antonymref{行家}{hang2jia5}
  \antonymref{内行}{nei4hang2}
  \antonymref{在行}{zai4hang2}
  \antonymref{专家}{zhuan1jia1}
\end{EntryWithPhonetic}

\begin{EntryWithPhonetic}{外号}{wai4hao4}{5,5}{⼣,⼝}[HSK 7-9]
  \definition[个]{s.}{apelido; nomes dados por outras pessoas, diferentes do nome real, muitas vezes têm conotações de carinho, brincadeira, elogio ou ódio}[她的外号是“胖胖”。===Seu apelido é "Gordinha".]
  \synonymref{绰号}{chuo4hao4}
\end{EntryWithPhonetic}

\begin{EntryWithPhonetic}{外汇}{wai4hui4}{5,5}{⼣,⽔}[HSK 4]
  \definition{s.}{câmbio estrangeiro; moeda estrangeira; moedas estrangeiras e títulos, como cheques, letras de câmbio, notas promissórias, etc., conversíveis em moedas estrangeiras, usados na compensação do comércio internacional}
\end{EntryWithPhonetic}

\begin{EntryWithPhonetic}{外积}{wai4ji1}{5,10}{⼣,⽲}
  \definition{s.}{produto exterior | (matemática) o produto vetorial de dois vetores}
\end{EntryWithPhonetic}

\begin{EntryWithPhonetic}{外籍}{wai4ji2}{5,20}{⼣,⽵}[HSK 7-9]
  \definition[个]{s.}{nacionalidade estrangeira | registro de residência permanente não local; \emph{status} de visitante}
  \synonymref{国外}{guo2wai4}
\end{EntryWithPhonetic}

\begin{EntryWithPhonetic}{外交}{wai4jiao1}{5,6}{⼣,⼇}[HSK 3]
  \definition[个]{s.}{diplomacia; relações exteriores; atividades de um país nas relações internacionais, como participar de organizações e conferências internacionais, trocar enviados com outros países, conduzir negociações, assinar tratados e acordos, etc.}
\end{EntryWithPhonetic}

\begin{EntryWithPhonetic}{外交官}{wai4jiao1guan1}{5,6,8}{⼣,⼇,⼧}[HSK 4]
  \definition[位,名]{s.}{diplomata}
\end{EntryWithPhonetic}

\begin{EntryWithPhonetic}{外交家}{wai4jiao1jia1}{5,6,10}{⼣,⼇,⼧}
  \definition{s.}{diplomata}[老资格的外交家===diplomata veterano]
\end{EntryWithPhonetic}

\begin{EntryWithPhonetic}{外交学}{wai4jiao1 xue2}{5,6,8}{⼣,⼇,⼦}
  \definition{s.}{diplomacia}
\end{EntryWithPhonetic}

\begin{EntryWithPhonetic}{外界}{wai4jie4}{5,9}{⼣,⽥}[HSK 5]
  \definition{s.}{o exterior; o mundo externo; área fora de um determinado âmbito; sociedade externa}
\end{EntryWithPhonetic}

\begin{EntryWithPhonetic}{外科}{wai4ke1}{5,9}{⼣,⽲}[HSK 6]
  \definition[名]{s.}{cirurgia; departamento cirúrgico; um departamento em uma instituição médica que usa principalmente cirurgia para tratar doenças internas e externas}
\end{EntryWithPhonetic}

\begin{EntryWithPhonetic}{外来}{wai4lai2}{5,7}{⼣,⽊}[HSK 6]
  \definition{adj.}{de fora; externo; estrangeiro}
\end{EntryWithPhonetic}

\begin{EntryWithPhonetic}{外卖}{wai4mai4}{5,8}{⼣,⼗}[HSK 2]
  \definition[份,单,盒]{s.}{comida para viagem; levar para viagem}
  \definition{v.}{entregar; oferecer; refere"-se à ação do comerciante entregar alimentos no local especificado pelo cliente}
\end{EntryWithPhonetic}

\begin{EntryWithPhonetic}{外贸}{wai4mao4}{5,9}{⼣,⾙}[HSK 7-9]
  \definition{s.}{comércio exterior, abreviação de 对外贸易}
  \seealsoref{对外贸易}{dui4wai4 mao4yi4}
  \synonymref{概念}{gai4nian4}
  \synonymref{金钱}{jin1qian2}
  \synonymref{贸易}{mao4yi4}
\end{EntryWithPhonetic}

\begin{EntryWithPhonetic}{外贸协会}{wai4mao4xie2hui4}{5,9,6,6}{⼣,⾙,⼗,⼈}
  \definition*{s.}{Associação de Comércio Exterior}
\end{EntryWithPhonetic}

\begin{EntryWithPhonetic}{外貌}{wai4mao4}{5,14}{⼣,⾘}[HSK 7-9]
  \definition{s.}{aparência; exterior; aspecto; a aparência de uma pessoa ou coisa}
  \synonymref{模样}{mu2yang4}
  \synonymref{容貌}{rong2mao4}
  \synonymref{容颜}{rong2yan2}
  \synonymref{外表}{wai4biao3}
  \antonymref{内涵}{nei4han2}
  \antonymref{内心}{nei4xin1}
  \antonymref{内在}{nei4zai4}
  \antonymref{心灵}{xin1ling2}
\end{EntryWithPhonetic}

\begin{EntryWithPhonetic}{外貌协会}{wai4mao4xie2hui4}{5,14,6,6}{⼣,⾘,⼗,⼈}
  \definition{s.}{o ``clube da boa aparência'': pessoas que atribuem grande importância à aparência de uma pessoa (trocadilho com a Associação de Comércio Exterior, 外贸协会)}
  \seealsoref{外贸协会}{wai4mao4xie2hui4}
  \seealsoref{外协}{wai4xie2}
\end{EntryWithPhonetic}

\begin{EntryWithPhonetic}{外面}{wai4mian5}{5,9}{⼣,⾯}[HSK 3]
  \definition{s.}{o lado de fora; fora de um certo intervalo | exterior; aparência externa; a superfície de um objeto}
\end{EntryWithPhonetic}

\begin{EntryWithPhonetic}{外婆}{wai4po2}{5,11}{⼣,⼥}[HSK 7-9]
  \definition{s.}{avó materna; a mãe da minha mãe; no Sul, ela geralmente é chamada de 外婆, enquanto no Norte, ela geralmente é chamada de 姥姥}
  \synonymref{姥姥}{lao3lao5}
\end{EntryWithPhonetic}

\begin{EntryWithPhonetic}{外企}{wai4qi3}{5,6}{⼣,⼈}[HSK 7-9]
  \definition[家]{s.}{empresa estrangeira; empresas com investimento estrangeiro}
\end{EntryWithPhonetic}

\begin{EntryWithPhonetic}{外事}{wai4shi4}{5,8}{⼣,⼅}
  \definition{s.}{assuntos ou relações exteriores}
\end{EntryWithPhonetic}

\begin{EntryWithPhonetic}{外水}{wai4shui3}{5,4}{⼣,⽔}
  \definition{s.}{renda extra}
\end{EntryWithPhonetic}

\begin{EntryWithPhonetic}{外孙}{wai4sun1}{5,6}{⼣,⼦}
  \definition{s.}{filho da filha}
\end{EntryWithPhonetic}

\begin{EntryWithPhonetic}{外孙女}{wai4sun1nv3}{5,6,3}{⼣,⼦,⼥}
  \definition{s.}{filha da filha}
\end{EntryWithPhonetic}

\begin{EntryWithPhonetic}{外套}{wai4tao4}{5,10}{⼣,⼤}[HSK 4]
  \definition[件,套,个]{s.}{casaco; jaqueta; paletó; sobretudo}
\end{EntryWithPhonetic}

\begin{EntryWithPhonetic}{外头}{wai4tou5}{5,5}{⼣,⼤}[HSK 6]
  \definition{s.}{Coloquial: fora; ao ar livre}
  \antonymref{里头}{li3tou5}
\end{EntryWithPhonetic}

\begin{EntryWithPhonetic}{外围}{wai4wei2}{5,7}{⼣,⼞}
  \definition{adv.}{arredores}
\end{EntryWithPhonetic}

\begin{EntryWithPhonetic}{外文}{wai4wen2}{5,4}{⼣,⽂}[HSK 3]
  \definition[种,门]{s.}{língua ou escrita estrangeira}
\end{EntryWithPhonetic}

\begin{EntryWithPhonetic}{外协}{wai4xie2}{5,6}{⼣,⼗}
  \definition{s.}{terceirização | pessoas que julgam os outros pela aparência}
  \seealsoref{外貌协会}{wai4mao4xie2hui4}
\end{EntryWithPhonetic}

\begin{EntryWithPhonetic}{外星人}{wai4xing1ren2}{5,9,2}{⼣,⽇,⼈}[HSK 7-9]
  \definition{s.}{extraterrestre; alienígena | alienígena espacial}[我们真的遇到过外星人吗?===Alguma vez já nos deparamos realmente com extraterrestres?]
\end{EntryWithPhonetic}

\begin{EntryWithPhonetic}{外形}{wai4xing2}{5,7}{⼣,⼺}[HSK 7-9]
  \definition{s.}{forma; exterior; aparência; formato externo}
  \synonymref{轮廓}{lun2kuo4}
  \synonymref{外表}{wai4biao3}
  \synonymref{外观}{wai4guan1}
  \synonymref{外貌}{wai4mao4}
\end{EntryWithPhonetic}

\begin{EntryWithPhonetic}{外需}{wai4xu1}{5,14}{⼣,⾬}
  \definition{s.}{Economia: demanda externa}
  \seealsoref{内需}{nei4xu1}
\end{EntryWithPhonetic}

\begin{EntryWithPhonetic}{外衣}{wai4yi1}{5,6}{⼣,⾐}[HSK 6]
  \definition[件]{s.}{casaco; jaqueta; colete; sobreveste; envoltório; roupa externa (ou vestimenta); capa externa; vestido externo | semblante; aparência; feição}
\end{EntryWithPhonetic}

\begin{EntryWithPhonetic}{外语}{wai4yu3}{5,9}{⼣,⾔}[HSK 1]
  \definition[种,门]{s.}{língua estrangeira}
\end{EntryWithPhonetic}

\begin{EntryWithPhonetic}{外援}{wai4yuan2}{5,12}{⼣,⼿}[HSK 7-9]
  \definition[个,名]{s.}{ajuda externa; auxílio externo; assistência externa | Esporte: jogador estrangeiro}
\end{EntryWithPhonetic}

\begin{EntryWithPhonetic}{外资}{wai4zi1}{5,10}{⼣,⾙}[HSK 6]
  \definition{s.}{capital estrangeiro; investimento estrangeiro; fundos estrangeiros; capital investido por países estrangeiros}
  \antonymref{内资}{nei4 zi1}
\end{EntryWithPhonetic}

%%%%%%%%%% 弯 %%%%%%%%%%
\subsection*{弯}\addcontentsline{loh}{figure}{弯 \dpy{wan1}}

\begin{EntryWithPhonetic}{弯}{wan1}{9}{⼸}[HSK 4]
  \definition{adj.}{curvo; tortuoso; torto | para algo curvo, como a lua, etc. | dobrado; flexível}
  \definition[个,道]{s.}{curva; dobra; volta}
  \definition{v.}{dobrar; flexionar; curvar | Literário: desenhar}
\end{EntryWithPhonetic}

\begin{EntryWithPhonetic}{弯曲}{wan1qu1}{9,6}{⼸,⽈}[HSK 6]
  \definition{s.}{torto; curvo; sinuoso; tortuoso; não reto}
  \definition{v.}{dobrar; curvar; flexionar}
\end{EntryWithPhonetic}

%%%%%%%%%% 豌 %%%%%%%%%%
\subsection*{豌}\addcontentsline{loh}{figure}{豌 \dpy{wan1}}

\begin{EntryWithPhonetic}{豌}{wan1}{15}{⾖}
  \definition[粒]{s.}{ervilhas}
\end{EntryWithPhonetic}

\begin{EntryWithPhonetic}{豌豆}{wan1dou4}{15,7}{⾖,⾖}
  \definition{s.}{ervilha}
\end{EntryWithPhonetic}

%%%%%%%%%% 丸 %%%%%%%%%%
\subsection*{丸}\addcontentsline{loh}{figure}{丸 \dpy{wan2}}

\begin{EntryWithPhonetic}{丸}{wan2}{3}{⼂}[HSK 7-9]
  \definition{clas.}{utilizado para medicamentos em comprimido}
  \definition[个]{s.}{bola; grânulo | comprimido; bolo (como em bolo alimentar); pílula}
  \seealsoref{丸儿}{wan2r5}
\end{EntryWithPhonetic}

\begin{EntryWithPhonetic}{丸儿}{wan2r5}{3,2}{⼂,⼉}
  \definition{s.}{bola; grânulo}
\end{EntryWithPhonetic}

%%%%%%%%%% 完 %%%%%%%%%%
\subsection*{完}\addcontentsline{loh}{figure}{完 \dpy{wan2}}

\begin{EntryWithPhonetic}{完}{wan2}{7}{⼧}[HSK 2]
  \definition*{s.}{Sobrenome: Wan}
  \definition{adj.}{inteiro; intacto; completo}
  \definition{v.}{acabar; terminar; completar | pagar | estar terminado; estar pronto para | esgotar; ser usado}
\end{EntryWithPhonetic}

\begin{EntryWithPhonetic}{完备}{wan2bei4}{7,8}{⼧,⼡}[HSK 7-9]
  \definition{adj.}{perfeito; completo; impecável; descreve ter tudo o que se deve ter, sem lhe faltar nada}
  \definition{v.}{não deixar nada a desejar}
  \synonymref{具备}{ju4bei4}
  \synonymref{齐全}{qi2quan2}
  \synonymref{完好}{wan2hao3}
  \synonymref{完满}{wan2man3}
  \synonymref{完美}{wan2mei3}
  \synonymref{完全}{wan2quan2}
  \synonymref{完善}{wan2shan4}
  \synonymref{完整}{wan2zheng3}
  \synonymref{圆满}{yuan2man3}
  \antonymref{简陋}{jian3lou4}
  \antonymref{欠缺}{qian4que1}
\end{EntryWithPhonetic}

\begin{EntryWithPhonetic}{完毕}{wan2bi4}{7,6}{⼧,⽐}[HSK 7-9]
  \definition{v.}{terminar; concluir; finalizar; estar concluído}
  \seealsoref{结束}{jie2shu4}
  \synonymref{了结}{liao3jie2}
  \synonymref{完成}{wan2/cheng2}
  \synonymref{完了}{wan2le5}
  \synonymref{完了}{wan2liao3}
  \antonymref{进行}{jin4xing2}
  \antonymref{开始}{kai1shi3}
\end{EntryWithPhonetic}

\begin{EntryWithPhonetic}{完成}{wan2/cheng2}{7,6}{⼧,⼽}[HSK 2]
  \definition{v.+compl.}{realizar; completar; terminar; cumprir; levar ao sucesso}
\end{EntryWithPhonetic}

\begin{EntryWithPhonetic}{完蛋}{wan2/dan4}{7,11}{⼧,⾍}[HSK 7-9]
  \definition{v.+compl.}{estar terminado; estar concluído para; estar destruído ou colapsado}
  \antonymref{成功}{cheng2gong1}
\end{EntryWithPhonetic}

\begin{EntryWithPhonetic}{完好}{wan2hao3}{7,6}{⼧,⼥}[HSK 7-9]
  \definition{adj.}{intacto; inteiro; em bom estado}
  \synonymref{齐全}{qi2quan2}
  \synonymref{完备}{wan2bei4}
  \synonymref{完满}{wan2man3}
  \synonymref{完美}{wan2mei3}
  \synonymref{完全}{wan2quan2}
  \synonymref{完善}{wan2shan4}
  \synonymref{完整}{wan2zheng3}
  \synonymref{圆满}{yuan2man3}
  \antonymref{残缺}{can2que1}
  \antonymref{粉碎}{fen3sui4}
  \antonymref{漏洞}{lou4dong4}
  \antonymref{破碎}{po4sui4}
\end{EntryWithPhonetic}

\begin{EntryWithPhonetic}{完了}{wan2le5}{7,2}{⼧,⼅}[HSK 5]
  \definition{v.}{acabar; terminar; concluir; chegar ao fim}
  \seeref{wan2liao3}
\end{EntryWithPhonetic}

\begin{EntryWithPhonetic}{完了}{wan2liao3}{7,2}{⼧,⼅}
  \definition{v.}{estar terminado; estar concluído (uma tarefa)}
\end{EntryWithPhonetic}

\begin{EntryWithPhonetic}{完满}{wan2man3}{7,13}{⼧,⽔}
  \definition{adj.}{satisfatório; bem"-sucedido; perfeito}
  \synonymref{美满}{mei3man3}
  \synonymref{齐全}{qi2quan2}
  \synonymref{完备}{wan2bei4}
  \synonymref{完好}{wan2hao3}
  \synonymref{完美}{wan2mei3}
  \synonymref{完善}{wan2shan4}
  \synonymref{圆满}{yuan2man3}
\end{EntryWithPhonetic}

\begin{EntryWithPhonetic}{完美}{wan2mei3}{7,9}{⼧,⽺}[HSK 3]
  \definition{adj.}{perfeito; impecável; consumado}
\end{EntryWithPhonetic}

\begin{EntryWithPhonetic}{完全}{wan2quan2}{7,6}{⼧,⼊}[HSK 2]
  \definition{adj.}{inteiro; completo; não falta nada, está tudo completo}
  \definition{adv.}{completamente; representa tudo}
\end{EntryWithPhonetic}

\begin{EntryWithPhonetic}{完人}{wan2ren2}{7,2}{⼧,⼈}
  \definition{s.}{pessoa perfeita}
\end{EntryWithPhonetic}

\begin{EntryWithPhonetic}{完善}{wan2shan4}{7,12}{⼧,⼝}[HSK 3]
  \definition{adj.}{perfeito; consumado}
  \definition{v.}{refinar; melhorar; tornar perfeito}
\end{EntryWithPhonetic}

\begin{EntryWithPhonetic}{完税}{wan2shui4}{7,12}{⼧,⽲}
  \definition{v.}{pagar imposto}
\end{EntryWithPhonetic}

\begin{EntryWithPhonetic}{完完全全}{wan2wan2quan2quan2}{7,7,6,6}{⼧,⼧,⼊,⼊}
  \definition{adv.}{completamente}
\end{EntryWithPhonetic}

\begin{EntryWithPhonetic}{完整}{wan2zheng3}{7,16}{⼧,⽁}[HSK 3]
  \definition{adj.}{intacto; inteiro; completo; integrado; nenhum dano ou mutilação}
\end{EntryWithPhonetic}

%%%%%%%%%% 玩 %%%%%%%%%%
\subsection*{玩}\addcontentsline{loh}{figure}{玩 \dpy{wan2}}

\begin{EntryWithPhonetic}{玩}{wan2}{8}{⽟}
  \definition*{s.}{Sobrenome: Wan}
  \definition{s.}{objeto de apreciação; coisas para assistir}
  \definition{v.}{(~儿) divertir"-se; entreter"-se; fazer atividades que te deixem feliz | jogar; praticar algum tipo de atividade cultural, de entretenimento ou esportiva | recorrer a; usar métodos e meios impróprios para atingir o objetivo | provocar; subestimar; tratar com uma atitude frívola; desprezar | desfrutar; apreciar; observar | (~儿) envolver"-se em; tomar parte em; perseguir ou expressar deliberadamente um certo sentimento | ponderar; pensar cuidadosamente; apreciar}
\end{EntryWithPhonetic}

\begin{EntryWithPhonetic}{玩伴}{wan2ban4}{8,7}{⽟,⼈}
  \definition{s.}{parceiro de brincadeira}
\end{EntryWithPhonetic}

\begin{EntryWithPhonetic}{玩遍}{wan2bian4}{8,12}{⽟,⾡}
  \definition{v.}{passear (todo o país, toda a cidade, etc.) | visitar (um grande número de lugares)}
\end{EntryWithPhonetic}

\begin{EntryWithPhonetic}{玩家}{wan2jia1}{8,10}{⽟,⼧}
  \definition{s.}{entusiasta (áudio, modelos de aviões, etc.) | jogador (de um jogo)}
\end{EntryWithPhonetic}

\begin{EntryWithPhonetic}{玩具}{wan2ju4}{8,8}{⽟,⼋}[HSK 3]
  \definition[个,件,套]{s.}{brinquedo; coisas para brincar}
\end{EntryWithPhonetic}

\begin{EntryWithPhonetic}{玩具厂}{wan2ju4chang3}{8,8,2}{⽟,⼋,⼚}
  \definition{s.}{fábrica de brinquedos}
\end{EntryWithPhonetic}

\begin{EntryWithPhonetic}{玩具车}{wan2ju4 che1}{8,8,4}{⽟,⼋,⾞}
  \definition{s.}{carrinho de brinquedo}
\end{EntryWithPhonetic}

\begin{EntryWithPhonetic}{玩偶}{wan2'ou3}{8,11}{⽟,⼈}
  \definition{s.}{estatueta de brinquedo | boneco de ação | bicho de pelúcia | boneca}
\end{EntryWithPhonetic}

\begin{EntryWithPhonetic}{玩儿}{wan2r5}{8,2}{⽟,⼉}[HSK 1]
  \definition{v.}{divertir"-se; (entretenimento) relaxar ou experimentar alguma atividade}
\end{EntryWithPhonetic}

\begin{EntryWithPhonetic}{玩耍}{wan2shua3}{8,9}{⽟,⽽}[HSK 7-9]
  \definition{v.}{brincar; divertir"-se; entreter"-se; envolver"-se em atividades que lhe tragam alegria; jogar jogos}
  \synonymref{游玩}{you2wan2}
  \synonymref{游戏}{you2xi4}
  \antonymref{休息}{xiu1xi5}
\end{EntryWithPhonetic}

\begin{EntryWithPhonetic}{玩味}{wan2wei4}{8,8}{⽟,⼝}
  \definition{v.}{ponderar sutilezas | ruminar (pensamentos)}
\end{EntryWithPhonetic}

\begin{EntryWithPhonetic}{玩艺}{wan2yi4}{8,4}{⽟,⾋}
  \variantof{玩意}
\end{EntryWithPhonetic}

\begin{EntryWithPhonetic}{玩意}{wan2yi4}{8,13}{⽟,⼼}
  \definition{s.}{ato | brinquedo | coisa | truque (em uma performance, show de palco, acrobacias, etc.)}
\end{EntryWithPhonetic}

\begin{EntryWithPhonetic}{玩意儿}{wan2yi4r5}{8,13,2}{⽟,⼼,⼉}[HSK 7-9]
  \definition[个]{s.}{brinquedo; objeto de diversão | mágica, acrobacias, diálogos cruzados, canto de baladas, etc.; refere"-se às artes folclóricas, acrobacias, artes marciais, etc. | coisa; objeto | usado de forma desdenhosa; termos pejorativos para pessoas ou coisas}[他就一个没用的玩意儿。===Ele não passa de um lixo inútil.]
\end{EntryWithPhonetic}

\begin{EntryWithPhonetic}{玩者}{wan2zhe3}{8,8}{⽟,⽼}
  \definition{s.}{jogador}
\end{EntryWithPhonetic}

%%%%%%%%%% 顽 %%%%%%%%%%
\subsection*{顽}\addcontentsline{loh}{figure}{顽 \dpy{wan2}}

\begin{EntryWithPhonetic}{顽}{wan2}{10}{⾴}
  \definition*{s.}{Sobrenome: Wan}
  \definition{adj.}{estúpido; denso; insensível | teimoso; obstinado; não é facilmente persuadido ou subjugado | travesso; pernicioso | cabeça dura; estúpido e ignorante}
  \definition{v.}{brincar; divertir"-se | empregar; recorrer a | envolver"-se em; tomar parte em}
\end{EntryWithPhonetic}

\begin{EntryWithPhonetic}{顽固}{wan2gu4}{10,8}{⾴,⼞}[HSK 7-9]
  \definition{adj.}{obstinado; teimoso; cabeçudo; pensamento conservador, sem vontade de mudar | intransigente; veementemente contrário à mudança; postura reacionária | crônico; profundamente fixado; refere"-se a um estado difícil de mudar}
  \synonymref{保守}{bao3shou3}
  \synonymref{固执}{gu4zhi5}
  \synonymref{坚定}{jian1ding4}
  \synonymref{坚决}{jian1jue2}
  \synonymref{坚强}{jian1qiang2}
  \synonymref{鉴定}{jian4ding4}
  \synonymref{倔强}{jue2jiang4}
  \synonymref{顽强}{wan2qiang2}
  \antonymref{开通}{kai1tong5}
\end{EntryWithPhonetic}

\begin{EntryWithPhonetic}{顽皮}{wan2pi2}{10,5}{⾴,⽪}[HSK 6]
  \definition{adj.}{atrevido; travesso; arteiro; levado; (crianças, adolescentes, etc.) adoram brincar e causar problemas e não dão ouvidos a conselhos}
\end{EntryWithPhonetic}

\begin{EntryWithPhonetic}{顽强}{wan2qiang2}{10,12}{⾴,⼸}[HSK 6]
  \definition{adj.}{firme; tenaz; indomável; forte; resistente}
\end{EntryWithPhonetic}

%%%%%%%%%% 挽 %%%%%%%%%%
\subsection*{挽}\addcontentsline{loh}{figure}{挽 \dpy{wan3}}

\begin{EntryWithPhonetic}{挽}{wan3}{10}{⼿}[HSK 7-9]
  \definition{s.}{canção fúnebre; canção memorial; elegia}
  \definition{v.}{puxar | enrolar (as roupas) | rebocar (veículos) | enganchar o braço | lamentar a morte de alguém}
\end{EntryWithPhonetic}

\begin{EntryWithPhonetic}{挽回}{wan3hui2}{10,6}{⼿,⼞}[HSK 7-9]
  \definition{v.}{recuperar; resgatar; reaver; reconquistar; recuperar o que foi perdido | transformar (o mal em bem); reverter a situação desfavorável estabelecida}
  \synonymref{补救}{bu3jiu4}
  \synonymref{回旋}{hui2xuan2}
  \synonymref{解救}{jie3jiu4}
  \synonymref{扭转}{niu3zhuan3}
  \synonymref{抢救}{qiang3jiu4}
  \synonymref{挽救}{wan3jiu4}
  \synonymref{旋转}{xuan2zhuan3}
  \antonymref{剥夺}{bo1duo2}
\end{EntryWithPhonetic}

\begin{EntryWithPhonetic}{挽救}{wan3jiu4}{10,11}{⼿,⽁}[HSK 7-9]
  \definition{v.}{salvar; resgatar; remediar; resgatar do perigo}
  \synonymref{补救}{bu3jiu4}
  \synonymref{急救}{ji2jiu4}
  \synonymref{救济}{jiu4ji4}
  \synonymref{名言}{ming2yan2}
  \synonymref{扭转}{niu3zhuan3}
  \synonymref{抢救}{qiang3jiu4}
  \synonymref{调解}{tiao2jie3}
  \synonymref{挽回}{wan3hui2}
  \synonymref{旋转}{xuan2zhuan3}
\end{EntryWithPhonetic}

%%%%%%%%%% 埦 %%%%%%%%%%
\subsection*{埦}\addcontentsline{loh}{figure}{埦 \dpy{wan3}}

\begin{EntryWithPhonetic}{埦}{wan3}{11}{⼟}
  \variantof{碗}
\end{EntryWithPhonetic}

%%%%%%%%%% 惋 %%%%%%%%%%
\subsection*{惋}\addcontentsline{loh}{figure}{惋 \dpy{wan3}}

\begin{EntryWithPhonetic}{惋}{wan3}{11}{⼼}
  \definition{s.}{Literário: suspiro}
  \definition{v.}{Literário: suspirar}
\end{EntryWithPhonetic}

\begin{EntryWithPhonetic}{惋惜}{wan3xi1}{11,11}{⼼,⼼}[HSK 7-9]
  \definition{v.}{lamentar; sentir que é uma pena; sentir pena de alguém; estar arrependido; expressar simpatia e pena pelos infortúnios das pessoas ou por mudanças insatisfatórias nas coisas}
  \synonymref{可惜}{ke3xi1}
  \synonymref{怜惜}{lian2xi1}
  \synonymref{无奈}{wu2nai4}
  \synonymref{遗憾}{yi2han4}
  \antonymref{庆幸}{qing4xing4}
\end{EntryWithPhonetic}

%%%%%%%%%% 晚 %%%%%%%%%%
\subsection*{晚}\addcontentsline{loh}{figure}{晚 \dpy{wan3}}

\begin{EntryWithPhonetic}{晚}{wan3}{11}{⽇}[HSK 1]
  \definition*{s.}{Sobrenome: Wan}
  \definition{adj.}{tarde; tardio; passado o prazo acordado | júnior; mais jovem | mais tarde no tempo}
  \definition{s.}{noite; à noite; após o pôr do sol | últimos anos; última vida; um período posterior; refere"-se especificamente à velhice de uma pessoa | pôr do sol; ao pôr do sol}
\end{EntryWithPhonetic}

\begin{EntryWithPhonetic}{晚安}{wan3'an1}{11,6}{⽇,⼧}[HSK 2]
  \definition{expr.}{Tenha uma boa noite; uma frase educada usada para se despedir ou cumprimentar as pessoas à noite}
\end{EntryWithPhonetic}

\begin{EntryWithPhonetic}{晚报}{wan3bao4}{11,7}{⽇,⼿}[HSK 2]
  \definition[份,张]{s.}{jornal vespertino; um jornal publicado todas as tardes}
\end{EntryWithPhonetic}

\begin{EntryWithPhonetic}{晚餐}{wan3can1}{11,16}{⽇,⾷}[HSK 2]
  \definition[份,顿,次]{s.}{ceia; jantar}
\end{EntryWithPhonetic}

\begin{EntryWithPhonetic}{晚点}{wan3 dian3}{11,9}{⽇,⽕}[HSK 4]
  \definition{v.}{atrasar; adiar; (veículo, navio ou avião) partir, operar ou chegar mais tarde do que o horário especificado}
\end{EntryWithPhonetic}

\begin{EntryWithPhonetic}{晚饭}{wan3fan4}{11,7}{⽇,⾷}[HSK 1]
  \definition[顿]{s.}{jantar}
\end{EntryWithPhonetic}

\begin{EntryWithPhonetic}{晚会}{wan3hui4}{11,6}{⽇,⼈}[HSK 2]
  \definition[场,个,次]{s.}{festa noturna; entretenimento noturno}
\end{EntryWithPhonetic}

\begin{EntryWithPhonetic}{晚间}{wan3jian1}{11,7}{⽇,⾨}[HSK 7-9]
  \definition{s.}{(à) noite; (ao) anoitecer}
  \synonymref{晚上}{wan3shang5}
  \antonymref{早晨}{zao3chen5}
\end{EntryWithPhonetic}

\begin{EntryWithPhonetic}{晚近}{wan3jin4}{11,7}{⽇,⾡}
  \definition{s.}{nos últimos anos; durante os últimos anos | tarde | mais recente no passado | recentemente}
\end{EntryWithPhonetic}

\begin{EntryWithPhonetic}{晚景}{wan3jing3}{11,12}{⽇,⽇}
  \definition{s.}{circunstâncias dos anos de declínio de alguém | cena noturna}
\end{EntryWithPhonetic}

\begin{EntryWithPhonetic}{晚年}{wan3nian2}{11,6}{⽇,⼲}[HSK 7-9]
  \definition{s.}{velhice; ocaso; os últimos anos (restantes) da vida; crepúsculo; os estágios finais da vida}
\end{EntryWithPhonetic}

\begin{EntryWithPhonetic}{晚期}{wan3qi1}{11,12}{⽇,⽉}[HSK 7-9]
  \definition{s.}{estágio tardio; (da doença) estágio terminal | período posterior | fase final | terminal}
  \antonymref{早期}{zao3qi1}
\end{EntryWithPhonetic}

\begin{EntryWithPhonetic}{晚上}{wan3shang5}{11,3}{⽇,⼀}[HSK 1]
  \definition[个]{s.}{noite; o período entre o pôr do sol e a madrugada}
\end{EntryWithPhonetic}

\begin{EntryWithPhonetic}{晚育}{wan3yu4}{11,8}{⽇,⾁}
  \definition{s.}{parto tardio}
  \definition{v.}{ter um filho mais tarde}
\end{EntryWithPhonetic}

%%%%%%%%%% 碗 %%%%%%%%%%
\subsection*{碗}\addcontentsline{loh}{figure}{碗 \dpy{wan3}}

\begin{EntryWithPhonetic}{碗}{wan3}{13}{⽯}[HSK 2]
  \definition*{s.}{Sobrenome: Wan}
  \definition{clas.}{usado para medição de alimentos e bebidas}
  \definition[只,个]{s.}{tigela | objeto em forma de tigela}
\end{EntryWithPhonetic}

\begin{EntryWithPhonetic}{碗柜}{wan3gui4}{13,8}{⽯,⽊}
  \definition{s.}{armário}
\end{EntryWithPhonetic}

\begin{EntryWithPhonetic}{碗子}{wan3zi5}{13,3}{⽯,⼦}
  \definition{s.}{tigela}
\end{EntryWithPhonetic}

%%%%%%%%%% 万 %%%%%%%%%%
\subsection*{万}\addcontentsline{loh}{figure}{万 \dpy{wan4}}

\begin{EntryWithPhonetic}{万}{wan4}{3}{⼀}[HSK 2]
  \definition*{s.}{Sobrenome: Wan}
  \definition{adv.}{absolutamente; indica um grau extremamente alto, equivalente a 完全, 绝对 e 极}
  \definition{num.}{dez mil; 10.000; 1.0000 | miríade; um número muito grande}
  \seealsoref{极}{ji2}
  \seealsoref{绝对}{jue2dui4}
  \seealsoref{完全}{wan2quan2}
\end{EntryWithPhonetic}

\begin{EntryWithPhonetic}{万分}{wan4fen1}{3,4}{⼀,⼑}[HSK 7-9]
  \definition{adv.}{extremamente; muito}
  \synonymref{非常}{fei1chang2}
  \synonymref{极度}{ji2du4}
  \synonymref{极端}{ji2duan1}
  \synonymref{十分}{shi2fen1}
  \synonymref{特别}{te4bie2}
  \synonymref{异常}{yi4chang2}
\end{EntryWithPhonetic}

\begin{EntryWithPhonetic}{万福}{wan4fu2}{3,13}{⼀,⽰}
  \definition{s.}{(antigo) reverência feminina; reverência}
\end{EntryWithPhonetic}

\begin{EntryWithPhonetic}{万古长青}{wan4gu3-chang2qing1}{3,5,4,8}{⼀,⼝,⾧,⾭}[HSK 7-9]
  \definition{expr.}{seja perene; seja sempre verde; perene e eterno; sempre vivo; florescer para sempre; durar para sempre; permanecer fresco para sempre; sempre será verde como pinheiros e ciprestes por milhares de gerações; uma metáfora para um espírito nobre ou uma amizade profunda que nunca desaparecerá}
\end{EntryWithPhonetic}

\begin{EntryWithPhonetic}{万能}{wan4neng2}{3,10}{⼀,⾁}[HSK 7-9]
  \definition{adj.}{onipotente; todo"-poderoso | universal; multifacetado; de múltiplos usos}
  \synonymref{全能}{quan2neng2}
\end{EntryWithPhonetic}

\begin{EntryWithPhonetic}{万圣节}{wan4 sheng4 jie2}{3,5,5}{⼀,⼟,⾋}
  \definition*{s.}{Dia de Todos os Santos}
  \seealsoref{万圣节前夕}{wan4sheng4 jie2 qian2xi1}
\end{EntryWithPhonetic}

\begin{EntryWithPhonetic}{万圣节前夕}{wan4sheng4 jie2 qian2xi1}{3,5,5,9,3}{⼀,⼟,⾋,⼑,⼣}
  \definition*{s.}{Véspera do Dia de Todos os Santos | Halloween}
  \seealsoref{万圣节}{wan4 sheng4 jie2}
\end{EntryWithPhonetic}

\begin{EntryWithPhonetic}{万万}{wan4wan4}{3,3}{⼀,⼀}[HSK 7-9]
  \definition{adv.}{totalmente; absolutamente; em qualquer caso; não importa o que aconteça}
  \definition{num.}{cem milhões; 100.000.000; 1.0000.0000}
  \synonymref{绝对}{jue2dui4}
  \synonymref{千万}{qian1wan4}
  \synonymref{完全}{wan2quan2}
\end{EntryWithPhonetic}

\begin{EntryWithPhonetic}{万无一失}{wan4wu2-yi1shi1}{3,4,1,5}{⼀,⽆,⼀,⼤}[HSK 7-9]
  \definition{expr.}{não há perigo de algo dar errado; esteja do lado seguro; \dots não pode falhar em nenhuma circunstância; garantir sucesso total; nenhum risco; não há chance de erro; (faça com que seja mais do que provável que) nada dê errado; perfeitamente seguro; infalível; as chances são de mil para uma, não falharemos}
\end{EntryWithPhonetic}

\begin{EntryWithPhonetic}{万物}{wan4wu4}{3,8}{⼀,⽜}
  \definition{s.}{toda a criação; todas as coisas na Terra; todos os seres vivos; tudo no universo}
  \synonymref{生物}{sheng1wu4}
\end{EntryWithPhonetic}

\begin{EntryWithPhonetic}{万一}{wan4yi1}{3,1}{⼀,⼀}[HSK 4]
  \definition{conj.}{por via das dúvidas; se por acaso; só por precaução; expressa uma suposição muito improvável (usado para coisas desagradáveis)}
  \definition{num.}{um décimo milionésimo; uma porcentagem muito pequena}
  \definition{s.}{contingência; eventualidade; contingências muito improváveis}
  \synonymref{如果}{ru2guo3}
  \synonymref{要是}{yao4shi5}
  \synonymref{一旦}{yi2dan4}
\end{EntryWithPhonetic}

%%%%%%%%%% 蔓 %%%%%%%%%%
\subsection*{蔓}\addcontentsline{loh}{figure}{蔓 \dpy{wan4}}

\begin{EntryWithPhonetic}{蔓}{wan4}{14}{⾋}
  \definition*{s.}{Sobrenome: Wan}
  \definition{s.}{uma videira com gavinhas; caule fino que não consegue ficar em pé}
  \seeref{man2}
  \seeref{man4}
\end{EntryWithPhonetic}

%%%%%%%%%% 汪 %%%%%%%%%%
\subsection*{汪}\addcontentsline{loh}{figure}{汪 \dpy{wang1}}

\begin{EntryWithPhonetic}{汪}{wang1}{7}{⽔}
  \definition*{s.}{Sobrenome: Wang}
  \definition{adj.}{(um corpo de água) profundo e vasto}
  \definition{clas.}{utilizado para líquidos: piscina, poça}
  \definition{interj.}{Onomatopéia: latido}
  \definition{s.}{Dialeto: lagoa; piscina}
  \definition{v.}{(líquido) coletar; acumular | (líquido) exsudar; escorrer}
\end{EntryWithPhonetic}

\begin{EntryWithPhonetic}{汪洋}{wang1yang2}{7,9}{⽔,⽔}[HSK 7-9]
  \definition{adj.}{(um corpo de água) vasto; ilimitado; descreve a aparência de uma vastidão de água}
  \synonymref{海洋}{hai3yang2}
\end{EntryWithPhonetic}

%%%%%%%%%% 亡 %%%%%%%%%%
\subsection*{亡}\addcontentsline{loh}{figure}{亡 \dpy{wang2}}

\begin{EntryWithPhonetic}{亡}{wang2}{3}{⼇}
  \definition{adj.}{falecido}
  \definition{v.}{fugir; escapar | perder; ir embora; jogar fora | morrer; perecer; falecer | conquistar; subjugar | ser destruído; morrer}
  \synonymref{灭}{mie4}
  \synonymref{死}{si3}
  \synonymref{卒}{zu2}
  \antonymref{存}{cun2}
  \antonymref{兴}{xing1}
\end{EntryWithPhonetic}

\begin{EntryWithPhonetic}{亡羊补牢}{wang2yang2-bu3lao2}{3,6,7,7}{⼇,⽺,⾐,⼧}[HSK 7-9]
  \definition{expr.}{consertar a situação antes que seja tarde demais; agir tardiamente após a ocorrência de um acidente; reparar o curral depois que uma ovelha se perde é uma metáfora para encontrar maneiras de fazer as pazes depois de sofrer uma perda, de modo a evitar sofrer perdas novamente no futuro}
\end{EntryWithPhonetic}

%%%%%%%%%% 王 %%%%%%%%%%
\subsection*{王}\addcontentsline{loh}{figure}{王 \dpy{wang2}}

\begin{EntryWithPhonetic}{王}{wang2}{4}{⽟}[HSK 4][Kangxi 96]
  \definition*{s.}{Sobrenome: Wang}
  \definition{adj.}{grande; ótimo; honoríficos antigos para avós}
  \definition{s.}{rei; monarca; imperador; governante supremo de uma monarquia | cabeça; chefe; líder | o primeiro, maior ou mais forte de seu tipo | duque; príncipe; o título mais alto da sociedade feudal após a dinastia Han}
  \seeref{wang4}
\end{EntryWithPhonetic}

\begin{EntryWithPhonetic}{王八蛋}{wang2 ba1 dan4}{4,2,11}{⽟,⼋,⾍}
  \definition{s.}{bastardo; filho da puta; miserável}
  \synonymref{混蛋}{hun4dan4}
\end{EntryWithPhonetic}

\begin{EntryWithPhonetic}{王朝}{wang2chao2}{4,12}{⽟,⽉}
  \definition{s.}{dinastia}
\end{EntryWithPhonetic}

\begin{EntryWithPhonetic}{王国}{wang2guo2}{4,8}{⽟,⼞}[HSK 7-9]
  \definition[个]{s.}{reino; nação | reino; domínio; campo}
\end{EntryWithPhonetic}

\begin{EntryWithPhonetic}{王后}{wang2hou4}{4,6}{⽟,⼝}[HSK 6]
  \definition[个,位,名,些]{s.}{rainha consorte; rainha}
\end{EntryWithPhonetic}

\begin{EntryWithPhonetic}{王牌}{wang2pai2}{4,12}{⽟,⽚}[HSK 7-9]
  \definition[张,大]{s.}{trunfo; ás}
  \synonymref{绝招}{jue2zhao1}
  \synonymref{首席}{shou3xi2}
\end{EntryWithPhonetic}

\begin{EntryWithPhonetic}{王五}{wang2wu3}{4,4}{⽟,⼆}
  \definition{s.}{Wang Wu | Zé Ninguém | nome para uma pessoa não especificada, 3 de 3}
  \seealsoref{李四}{li3si4}
  \seealsoref{张三}{zhang1san1}
\end{EntryWithPhonetic}

\begin{EntryWithPhonetic}{王子}{wang2zi3}{4,3}{⽟,⼦}[HSK 6]
  \definition[位]{s.}{príncipe; filho do rei}
\end{EntryWithPhonetic}

%%%%%%%%%% 网 %%%%%%%%%%
\subsection*{网}\addcontentsline{loh}{figure}{网 \dpy{wang3}}

\begin{EntryWithPhonetic}{网}{wang3}{6}{⽹}[HSK 2][Kangxi 122]
  \definition[张]{s.}{rede; um dispositivo feito de corda ou barbante para capturar peixes e pássaros | algo que parece uma rede | rede; uma rede de organizações; um sistema}
  \definition{v.}{pegar com uma rede | cobrir como com uma rede}
\end{EntryWithPhonetic}

\begin{EntryWithPhonetic}{网吧}{wang3ba1}{6,7}{⽹,⼝}[HSK 6]
  \definition[家,间]{s.}{cybercafé; \emph{Internet} café; refere"-se a um local comercial aberto ao público que utiliza redes de computadores para fornecer serviços de navegação, consulta e outras informações}
\end{EntryWithPhonetic}

\begin{EntryWithPhonetic}{网点}{wang3dian3}{6,9}{⽹,⽕}[HSK 7-9]
  \definition{s.}{pontos de venda; rede (de estabelecimento comercial)}
\end{EntryWithPhonetic}

\begin{EntryWithPhonetic}{网罟}{wang3gu3}{6,10}{⽹,⽹}
  \definition{s.}{(fig.) a rede da justiça | rede usada para capturar peixes (ou outros animais, como pássaros)}
\end{EntryWithPhonetic}

\begin{EntryWithPhonetic}{网际网路}{wang3 ji4 wang3 lu4}{6,7,6,13}{⽹,⾩,⽹,⾜}
  \definition*{s.}{Internet}
  \seealsoref{互联网}{hu4lian2wang3}
  \seealsoref{网际网络}{wang3 ji4 wang3 luo4}
  \seealsoref{网路}{wang3 lu4}
\end{EntryWithPhonetic}

\begin{EntryWithPhonetic}{网际网络}{wang3 ji4 wang3 luo4}{6,7,6,9}{⽹,⾩,⽹,⽷}
  \definition*{s.}{Internet}
  \seealsoref{互联网}{hu4lian2wang3}
  \seealsoref{网际网路}{wang3 ji4 wang3 lu4}
  \seealsoref{网路}{wang3 lu4}
\end{EntryWithPhonetic}

\begin{EntryWithPhonetic}{网路}{wang3 lu4}{6,13}{⽹,⾜}
  \definition*{s.}{Internet}
  \seealsoref{互联网}{hu4lian2wang3}
  \seealsoref{网际网路}{wang3 ji4 wang3 lu4}
  \seealsoref{网际网络}{wang3 ji4 wang3 luo4}
\end{EntryWithPhonetic}

\begin{EntryWithPhonetic}{网络}{wang3luo4}{6,9}{⽹,⽷}[HSK 4]
  \definition{s.}{rede; um sistema que consiste em ramificações interconectadas; em um sistema elétrico, um circuito ou parte de um circuito que consiste em vários elementos que permitem a transmissão de sinais elétricos de acordo com determinados requisitos | rede; rede de computadores}
\end{EntryWithPhonetic}

\begin{EntryWithPhonetic}{网民}{wang3min2}{6,5}{⽹,⽒}[HSK 7-9]
  \definition[位,名]{s.}{internauta; geralmente se refere a um usuário de rede de computadores}
\end{EntryWithPhonetic}

\begin{EntryWithPhonetic}{网球}{wang3qiu2}{6,11}{⽹,⽟}[HSK 2]
  \definition[个,颗,些]{s.}{tênis (esporte) | bola de tênis}
\end{EntryWithPhonetic}

\begin{EntryWithPhonetic}{网上}{wang3shang4}{6,3}{⽹,⼀}[HSK 1]
  \definition{s.}{\emph{online}; refere"-se a acessar a \emph{Internet} através de um computador ou celular para pesquisar e consultar informações na rede}
\end{EntryWithPhonetic}

\begin{EntryWithPhonetic}{网上银行}{wang3shang4yin2hang2}{6,3,11,6}{⽹,⼀,⾦,⾏}
  \definition[个]{s.}{banco \emph{online} | acesso a operações bancárias via \emph{Internet}}
  \seealsoref{网银}{wang3yin2}
\end{EntryWithPhonetic}

\begin{EntryWithPhonetic}{网页}{wang3ye4}{6,6}{⽹,⾴}[HSK 6]
  \definition[个]{s.}{site; página da web; \emph{website}; \emph{web page}}
\end{EntryWithPhonetic}

\begin{EntryWithPhonetic}{网银}{wang3yin2}{6,11}{⽹,⾦}
  \definition{s.}{banco \emph{online} | acesso a operações bancárias via \emph{Internet}}
  \seealsoref{网上银行}{wang3shang4yin2hang2}
\end{EntryWithPhonetic}

\begin{EntryWithPhonetic}{网友}{wang3you3}{6,4}{⽹,⼜}[HSK 1]
  \definition{s.}{internauta; usuário da \emph{Internet}; amigos que se conhecem pela Internet; também usado como forma de tratamento entre internautas}
\end{EntryWithPhonetic}

\begin{EntryWithPhonetic}{网站}{wang3zhan4}{6,10}{⽹,⽴}[HSK 2]
  \definition[个,家]{s.}{\emph{web}; \emph{website}; um site virtual na Internet para uma organização ou indivíduo, geralmente consistindo em uma página inicial e muitas páginas da web}
\end{EntryWithPhonetic}

\begin{EntryWithPhonetic}{网址}{wang3zhi3}{6,7}{⽹,⼟}[HSK 4]
  \definition[个]{s.}{\emph{website}; endereço da \emph{web}; endereço de um \emph{site} na \emph{Internet}, que os usuários podem acessar, consultar e obter recursos de informações nesse \emph{site} clicando nele}
\end{EntryWithPhonetic}

%%%%%%%%%% 往 %%%%%%%%%%
\subsection*{往}\addcontentsline{loh}{figure}{往 \dpy{wang3}}

\begin{EntryWithPhonetic}{往}{wang3}{8}{⼻}[HSK 2]
  \definition{adj.}{passado; anterior}
  \definition{prep.}{para; em direção a; na direção de}
  \definition{v.}{ir}
\end{EntryWithPhonetic}

\begin{EntryWithPhonetic}{往常}{wang3chang2}{8,11}{⼻,⼱}[HSK 7-9]
  \definition{s.}{habitual; normal; como sempre}
  \synonymref{平常}{ping2chang2}
  \synonymref{平时}{ping2shi2}
  \synonymref{以前}{yi3qian2}
  \antonymref{如今}{ru2jin1}
\end{EntryWithPhonetic}

\begin{EntryWithPhonetic}{往程}{wang3cheng2}{8,12}{⼻,⽲}
  \definition{s.}{saída (de uma viagem de ônibus ou trem, etc.)}
\end{EntryWithPhonetic}

\begin{EntryWithPhonetic}{往返}{wang3fan3}{8,7}{⼻,⾡}[HSK 7-9]
  \definition{v.}{transportar; ir e voltar; viajar de e para}
  \synonymref{来回}{lai2hui2}
  \synonymref{往复}{wang3fu4}
  \synonymref{往来}{wang3lai2}
\end{EntryWithPhonetic}

\begin{EntryWithPhonetic}{往复}{wang3fu4}{8,9}{⼻,⼢}
  \definition{s.}{para trás e para frente (por exemplo, da ação do pistão ou da bomba)}
  \definition{v.}{ir e voltar | fazer uma viagem de volta}
\end{EntryWithPhonetic}

\begin{EntryWithPhonetic}{往后}{wang3hou4}{8,6}{⼻,⼝}[HSK 6]
  \definition{s.}{de agora em diante; mais tarde; no futuro | na parte traseira; na parte de trás | para trás; depois; à ré}
\end{EntryWithPhonetic}

\begin{EntryWithPhonetic}{往迹}{wang3ji4}{8,9}{⼻,⾡}
  \definition{s.}{Literário: eventos passados; coisa do passado; tempos antigos}
\end{EntryWithPhonetic}

\begin{EntryWithPhonetic}{往来}{wang3lai2}{8,7}{⼻,⽊}[HSK 6]
  \definition{s.}{contatos comerciais; relações comerciais; relações diplomáticas | negociações; visitas mútuas; comunicação}
  \definition{v.}{ir e vir | contatar; ter relações}
\end{EntryWithPhonetic}

\begin{EntryWithPhonetic}{往例}{wang3li4}{8,8}{⼻,⼈}
  \definition{s.}{prática (habitual) do passado | precedente}
\end{EntryWithPhonetic}

\begin{EntryWithPhonetic}{往年}{wang3nian2}{8,6}{⼻,⼲}[HSK 6]
  \definition{s.}{(em) anos anteriores}
\end{EntryWithPhonetic}

\begin{EntryWithPhonetic}{往日}{wang3ri4}{8,4}{⼻,⽇}[HSK 7-9]
  \definition{adv.}{(nos) dias anteriores; (nos) últimos dias; (em) dias passados}
  \definition{s.}{o passado}
  \synonymref{昔日}{xi1ri4}
\end{EntryWithPhonetic}

\begin{EntryWithPhonetic}{往生}{wang3sheng1}{8,5}{⼻,⽣}
  \definition{v.}{renascer | morrer | (Budismo) viver no paraíso}
\end{EntryWithPhonetic}

\begin{EntryWithPhonetic}{往事}{wang3shi4}{8,8}{⼻,⼅}[HSK 7-9]
  \definition[件,段]{s.}{eventos passados; o passado; coisas passadas}
\end{EntryWithPhonetic}

\begin{EntryWithPhonetic}{往往}{wang3wang3}{8,8}{⼻,⼻}[HSK 3]
  \definition{adv.}{frequentemente; muitas vezes; mais frequentemente do que não; indica que uma situação existe ou ocorre com frequência}
\end{EntryWithPhonetic}

\begin{EntryWithPhonetic}{往昔}{wang3xi1}{8,8}{⼻,⽇}
  \definition{s.}{o passado}
\end{EntryWithPhonetic}

%%%%%%%%%% 罔 %%%%%%%%%%
\subsection*{罔}\addcontentsline{loh}{figure}{罔 \dpy{wang3}}

\begin{EntryWithPhonetic}{罔}{wang3}{8}{⼌}
  \definition{v.}{enganar}
\end{EntryWithPhonetic}

%%%%%%%%%% 王 %%%%%%%%%%
\subsection*{王}\addcontentsline{loh}{figure}{王 \dpy{wang4}}

\begin{EntryWithPhonetic}{王}{wang4}{4}{⽟}[Kangxi 96]
  \definition{v.}{reger; governar; reinar; dominar}
  \seeref{wang2}
\end{EntryWithPhonetic}

%%%%%%%%%% 妄 %%%%%%%%%%
\subsection*{妄}\addcontentsline{loh}{figure}{妄 \dpy{wang4}}

\begin{EntryWithPhonetic}{妄}{wang4}{6}{⼥}
  \definition{adj.}{absurdo; absurdo | ultrajante; ridículo e irracional | precipitado; irresponsável; presunçoso; irracional; fora da rotina; aleatório}
\end{EntryWithPhonetic}

\begin{EntryWithPhonetic}{妄想}{wang4xiang3}{6,13}{⼥,⼼}[HSK 7-9]
  \definition{s.}{ilusão; vã esperança; idéias falsas e irrealizáveis}
  \definition{v.}{fazer uma tentativa vã de; esperar em vão fazer algo; planos que não podem ser realizados}
  \synonymref{梦想}{meng4xiang3}
\end{EntryWithPhonetic}

%%%%%%%%%% 忘 %%%%%%%%%%
\subsection*{忘}\addcontentsline{loh}{figure}{忘 \dpy{wang4}}

\begin{EntryWithPhonetic}{忘}{wang4}{7}{⼼}[HSK 1]
  \definition{v.}{esquecer | ignorar; negligenciar}
\end{EntryWithPhonetic}

\begin{EntryWithPhonetic}{忘本}{wang4ben3}{7,5}{⼼,⽊}
  \definition{v.}{esquecer as próprias raízes}
\end{EntryWithPhonetic}

\begin{EntryWithPhonetic}{忘不了}{wang4 bu5 liao3}{7,4,2}{⼼,⼀,⼅}[HSK 7-9]
  \definition{v.}{não poder esquecer}
\end{EntryWithPhonetic}

\begin{EntryWithPhonetic}{忘餐}{wang4can1}{7,16}{⼼,⾷}
  \definition{v.}{esquecer as refeições}
\end{EntryWithPhonetic}

\begin{EntryWithPhonetic}{忘掉}{wang4/diao4}{7,11}{⼼,⼿}[HSK 7-9]
  \definition{v.+compl.}{esquecer; deixar escapar da mente; não se lembrar}
  \synonymref{忘怀}{wang4huai2}
  \synonymref{忘记}{wang4ji4}
  \synonymref{忘却}{wang4que4}
  \antonymref{记住}{ji4 zhu5}
\end{EntryWithPhonetic}

\begin{EntryWithPhonetic}{忘恩}{wang4'en1}{7,10}{⼼,⼼}
  \definition{v.}{ser ingrato}
\end{EntryWithPhonetic}

\begin{EntryWithPhonetic}{忘怀}{wang4huai2}{7,7}{⼼,⼼}
  \definition{v.}{esquecer}
\end{EntryWithPhonetic}

\begin{EntryWithPhonetic}{忘记}{wang4ji4}{7,5}{⼼,⾔}[HSK 1]
  \definition{v.}{esquecer | ignorar; negligenciar | sair da memória de alguém; não ser lembrado | descartar da mente; ignorar}
\end{EntryWithPhonetic}

\begin{EntryWithPhonetic}{忘却}{wang4que4}{7,7}{⼼,⼙}
  \definition{v.}{esquecer}
\end{EntryWithPhonetic}

%%%%%%%%%% 旺 %%%%%%%%%%
\subsection*{旺}\addcontentsline{loh}{figure}{旺 \dpy{wang4}}

\begin{EntryWithPhonetic}{旺}{wang4}{8}{⽇}[HSK 7-9]
  \definition{adj.}{próspero; florescente; vigoroso | abundante; numeroso}
\end{EntryWithPhonetic}

\begin{EntryWithPhonetic}{旺季}{wang4ji4}{8,8}{⽇,⼦}[HSK 7-9]
  \definition{s.}{alta temporada; período de pico; temporada movimentada; a estação em que um determinado produto é produzido em grandes quantidades ou quando os negócios estão crescendo}
  \seealsoref{淡季}{dan4ji4}
  \antonymref{淡季}{dan4ji4}
\end{EntryWithPhonetic}

\begin{EntryWithPhonetic}{旺盛}{wang4sheng4}{8,11}{⽇,⽫}[HSK 7-9]
  \definition{adj.}{vigoroso; exuberante; de forte vitalidade; de bom humor}
  \synonymref{繁华}{fan2hua2}
  \synonymref{繁荣}{fan2rong2}
  \synonymref{红火}{hong2huo5}
  \synonymref{焕发}{huan4fa1}
  \synonymref{茂盛}{mao4sheng4}
  \synonymref{蓬勃}{peng2bo2}
  \synonymref{兴旺}{xing1wang4}
  \antonymref{衰退}{shuai1tui4}
\end{EntryWithPhonetic}

%%%%%%%%%% 望 %%%%%%%%%%
\subsection*{望}\addcontentsline{loh}{figure}{望 \dpy{wang4}}

\begin{EntryWithPhonetic}{望}{wang4}{11}{⽉}[HSK 7-9]
  \definition*{s.}{Sobrenome: Wang}
  \definition{prep.}{para; em direção a; em ``olhando para frente (望前看)'', ``caminhando para o leste (望东走)'', etc.; 望 é frequentemente escrito como 往}
  \definition{s.}{prestígio; reputação; fama | lua cheia | o 15º dia de um mês lunar}
  \definition{v.}{olhar por cima; olhar para a distância; olhar para longe na distância | visitar; ligar para | ter esperança; esperar | odiar; ressentir"-se | pensar em atingir um determinado objetivo ou uma determinada situação em mente}
  \seealsoref{往}{wang3}
  \synonymref{看}{kan4}
  \synonymref{瞧}{qiao2}
  \synonymref{视}{shi4}
\end{EntryWithPhonetic}

\begin{EntryWithPhonetic}{望见}{wang4jian4}{11,4}{⽉,⾒}[HSK 6]
  \definition{v.}{espiar; ver; pôr os olhos em | detectar}
\end{EntryWithPhonetic}

\begin{EntryWithPhonetic}{望远镜}{wang4yuan3jing4}{11,7,16}{⽉,⾡,⾦}[HSK 7-9]
  \definition[架,个,付,副,部]{s.}{telescópio; o telescópio refrator mais simples consiste em dois conjuntos de lentes | binóculos; um instrumento óptico para observar objetos distantes}
\end{EntryWithPhonetic}

%%%%%%%%%% 危 %%%%%%%%%%
\subsection*{危}\addcontentsline{loh}{figure}{危 \dpy{wei1}}

\begin{EntryWithPhonetic}{危}{wei1}{6}{⼙}
  \definition*{s.}{Wei, a décima segunda das vinte e oito constelações em que a esfera celeste foi dividida, consistindo de três estrelas em forma de triângulo obtuso, uma em Aquário e duas em Pégaso | Wei, uma das mansões lunares | Sobrenome: Wei}
  \definition{adj.}{arriscado; inseguro; perigoso | estar gravemente doente; estar morrendo | alto; íngreme}
  \definition{s.}{perigo | cumeeira (de um telhado)}
  \definition{v.}{pôr em perigo; colocar em perigo; comprometer}
  \antonymref{安}{an1}
\end{EntryWithPhonetic}

\begin{EntryWithPhonetic}{危害}{wei1hai4}{6,10}{⼙,⼧}[HSK 3]
  \definition[个,种]{s.}{prejuízo; perigo; dano}
  \definition{v.}{destruir; prejudicar; pôr em perigo; pôr em risco}
\end{EntryWithPhonetic}

\begin{EntryWithPhonetic}{危机}{wei1ji1}{6,6}{⼙,⽊}[HSK 6]
  \definition[个,次]{s.}{crise}
\end{EntryWithPhonetic}

\begin{EntryWithPhonetic}{危及}{wei1ji2}{6,3}{⼙,⼃}[HSK 7-9]
  \definition{v.}{prejudicar; colocar em perigo; comprometer; ameaçar}
\end{EntryWithPhonetic}

\begin{EntryWithPhonetic}{危急}{wei1ji2}{6,9}{⼙,⼼}[HSK 7-9]
  \definition{adj.}{crítico; em perigo iminente; em uma situação desesperadora; perigoso e urgente}
  \synonymref{风险}{feng1xian3}
  \synonymref{紧急}{jin3ji2}
  \synonymref{紧迫}{jin3po4}
  \synonymref{紧张}{jin3zhang1}
  \synonymref{迫切}{po4qie4}
  \synonymref{危害}{wei1hai4}
  \synonymref{危机}{wei1ji1}
  \synonymref{危险}{wei1xian3}
  \synonymref{严重}{yan2zhong4}
  \antonymref{安全}{an1quan2}
  \antonymref{安稳}{an1wen3}
\end{EntryWithPhonetic}

\begin{EntryWithPhonetic}{危难}{wei1nan4}{6,10}{⼙,⾫}
  \definition{s.}{calamidade}
\end{EntryWithPhonetic}

\begin{EntryWithPhonetic}{危险}{wei1xian3}{6,9}{⼙,⾩}[HSK 3]
  \definition{adj.}{arriscado; perigoso}
\end{EntryWithPhonetic}

%%%%%%%%%% 委 %%%%%%%%%%
\subsection*{委}\addcontentsline{loh}{figure}{委 \dpy{wei1}}

\begin{EntryWithPhonetic}{委}{wei1}{8}{⼥}
  \definition{adj./adv.}{o mesmo que 逶 em 逶迤 sinuoso, curvo}
  \seeref{wei3}
  \seealsoref{逶}{wei1}
  \seealsoref{逶迤}{wei1yi2}
\end{EntryWithPhonetic}

%%%%%%%%%% 威 %%%%%%%%%%
\subsection*{威}\addcontentsline{loh}{figure}{威 \dpy{wei1}}

\begin{EntryWithPhonetic}{威}{wei1}{9}{⼥}
  \definition*{s.}{Sobrenome: Wei}
  \definition{adj.}{forte; poderoso}
  \definition{s.}{força impressionante; poder; força}
  \definition{v.}{ameaçar pela força; intimidar com força}
\end{EntryWithPhonetic}

\begin{EntryWithPhonetic}{威风}{wei1feng1}{9,4}{⼥,⾵}[HSK 7-9]
  \definition{adj.}{imponente; impressionante; inspirador | de influência dominadora; de alto prestígio apoiado em poder; de momento ou estilo inspirador | de aparência majestosa; majestoso}
  \definition{s.}{poder; prestígio}
  \synonymref{威信}{wei1xin4}
\end{EntryWithPhonetic}

\begin{EntryWithPhonetic}{威力}{wei1li4}{9,2}{⼥,⼒}[HSK 7-9]
  \definition{s.}{poder; vigor; força}
  \synonymref{魄力}{po4li4}
  \synonymref{气魄}{qi4po4}
  \synonymref{气势}{qi4shi4}
\end{EntryWithPhonetic}

\begin{EntryWithPhonetic}{威慑}{wei1she4}{9,13}{⼥,⼼}[HSK 7-9]
  \definition{v.}{aterrorizar com força; dissuadir | acovardar"-se pela força militar; deter}
  \synonymref{恐吓}{kong3he4}
  \synonymref{威胁}{wei1xie2}
\end{EntryWithPhonetic}

\begin{EntryWithPhonetic}{威胁}{wei1xie2}{9,8}{⼥,⾁}[HSK 6]
  \definition{v.}{pôr em perigo; ameaçar; intimidar}
\end{EntryWithPhonetic}

\begin{EntryWithPhonetic}{威信}{wei1xin4}{9,9}{⼥,⼈}[HSK 7-9]
  \definition{s.}{prestígio; confiança popular; prestígio e credibilidade}
  \synonymref{威风}{wei1feng1}
\end{EntryWithPhonetic}

%%%%%%%%%% 萎 %%%%%%%%%%
\subsection*{萎}\addcontentsline{loh}{figure}{萎 \dpy{wei1}}

\begin{EntryWithPhonetic}{萎}{wei1}{11}{⾋}
  \definition{v.}{murchar; cair}
  \seeref{wei3}
  \synonymref{缩}{suo1}
  \synonymref{谢}{xie4}
\end{EntryWithPhonetic}

%%%%%%%%%% 逶 %%%%%%%%%%
\subsection*{逶}\addcontentsline{loh}{figure}{逶 \dpy{wei1}}

\begin{EntryWithPhonetic}{逶}{wei1}{11}{⾡}
  \definition{adj.}{sinuoso; tortuoso}
\end{EntryWithPhonetic}

\begin{EntryWithPhonetic}{逶迤}{wei1yi2}{11,8}{⾡,⾡}
  \definition{adj.}{sinuoso; tortuoso; descreve a aparência sinuosa e contínua de estradas, montanhas, rios, etc.}
\end{EntryWithPhonetic}

%%%%%%%%%% 微 %%%%%%%%%%
\subsection*{微}\addcontentsline{loh}{figure}{微 \dpy{wei1}}

\begin{EntryWithPhonetic}{微}{wei1}{13}{⼻}
  \definition{adj.}{minúsculo; leve | profundo; abstruso | humilde; tendo pouca influência; baixo \emph{status}}
  \definition{adv.}{pouco; ligeiramente; indica um grau menor, equivalente a 稍 ou 略}
  \definition{num.}{um milionésimo de uma determinada unidade de medida}
  \definition{suf.}{micro-}
  \seealsoref{略}{lve4}
  \seealsoref{稍}{shao1}
\end{EntryWithPhonetic}

\begin{EntryWithPhonetic}{微波炉}{wei1bo1lu2}{13,8,8}{⼻,⽔,⽕}[HSK 6]
  \definition[台,个]{s.}{forno de micro-ondas}
\end{EntryWithPhonetic}

\begin{EntryWithPhonetic}{微博}{wei1bo2}{13,12}{⼻,⼗}[HSK 5]
  \definition*{s.}{Weibo (um aplicativo de mídia social chinês)}
  \definition[条]{s.}{\emph{microblog}; abreviação de 微型博客}
  \seealsoref{微型博客}{wei1xing2 bo2ke4}
\end{EntryWithPhonetic}

\begin{EntryWithPhonetic}{微不足道}{wei1bu4zu2dao4}{13,4,7,12}{⼻,⼀,⾜,⾡}[HSK 7-9]
  \definition{expr.}{muito trivial ou insignificante para ser mencionado; insignificante; muito trivial, não vale a pena comentar}
\end{EntryWithPhonetic}

\begin{EntryWithPhonetic}{微风}{wei1feng1}{13,4}{⼻,⾵}
  \definition{s.}{brisa | vento leve}
\end{EntryWithPhonetic}

\begin{EntryWithPhonetic}{微观}{wei1guan1}{13,6}{⼻,⾒}[HSK 7-9]
  \definition{adj.}{microscópico; microcósmico}
  \antonymref{宏观}{hong2guan1}
\end{EntryWithPhonetic}

\begin{EntryWithPhonetic}{微妙}{wei1miao4}{13,7}{⼻,⼥}[HSK 7-9]
  \definition{adj.}{delicado; sutil; esotérico; indescritível}
  \synonymref{奥秘}{ao4mi4}
  \synonymref{奇妙}{qi2miao4}
  \antonymref{豁达}{huo4da2}
  \antonymref{突出}{tu1/chu1}
  \antonymref{显著}{xian3zhu4}
\end{EntryWithPhonetic}

\begin{EntryWithPhonetic}{微软}{wei1ruan3}{13,8}{⼻,⾞}
  \definition*{s.}{Microsoft Corporation}
\end{EntryWithPhonetic}

\begin{EntryWithPhonetic}{微弱}{wei1ruo4}{13,10}{⼻,⼸}[HSK 7-9]
  \definition{adj.}{fraco}
  \synonymref{薄弱}{bo2ruo4}
  \synonymref{单薄}{dan1bo2}
  \synonymref{轻微}{qing1wei1}
  \synonymref{衰弱}{shuai1ruo4}
  \synonymref{微小}{wei1xiao3}
  \antonymref{猛烈}{meng3lie4}
  \antonymref{强大}{qiang2da4}
  \antonymref{强劲}{qiang2jing4}
  \antonymref{强烈}{qiang2lie4}
\end{EntryWithPhonetic}

\begin{EntryWithPhonetic}{微小}{wei1xiao3}{13,3}{⼻,⼩}
  \definition{adj.}{pequeno; minúsculo; diminuto}
  \definition{s.}{vírus de RNA}
\end{EntryWithPhonetic}

\begin{EntryWithPhonetic}{微笑}{wei1xiao4}{13,10}{⼻,⽵}[HSK 4]
  \definition[个,丝]{s.}{sorriso; sorriso sutil}
  \definition{v.}{sorrir; rir baixinho e sutilmente}
\end{EntryWithPhonetic}

\begin{EntryWithPhonetic}{微信}{wei1xin4}{13,9}{⼻,⼈}[HSK 4]
  \definition*[个,条]{s.}{WeChat, aplicativo gratuito lançado pela Tencent em 21 de janeiro de 2011 para fornecer serviços de mensagens instantâneas para terminais inteligentes}
\end{EntryWithPhonetic}

\begin{EntryWithPhonetic}{微型}{wei1xing2}{13,9}{⼻,⼟}[HSK 7-9]
  \definition{adj.}{minúsculo}
  \definition{pref.}{micro-; mini-}
  \definition{s.}{miniatura; microescala}
  \synonymref{小型}{xiao3xing2}
  \antonymref{大型}{da4xing2}
  \antonymref{巨型}{ju4xing2}
\end{EntryWithPhonetic}

\begin{EntryWithPhonetic}{微型博客}{wei1xing2 bo2ke4}{13,9,12,9}{⼻,⼟,⼗,⼧}
  \definition{s.}{\emph{microblog}}
\end{EntryWithPhonetic}

%%%%%%%%%% 为 %%%%%%%%%%
\subsection*{为}\addcontentsline{loh}{figure}{为 \dpy{wei2}}

\begin{EntryWithPhonetic}{为}{wei2}{4}{⼂}[HSK 3]
  \definition*{s.}{Sobrenome: Wei}
  \definition{part.}{frequentemente usado com 何 em uma pergunta retórica}
  \definition{prep.}{por; usado em frases passivas para introduzir o agente da ação, equivalente a 被 (frequentemente usado com 所)}
  \definition{suf.}{é anexado a alguns adjetivos ou advérbios monossilábicos para formar advérbios dissilábicos que expressam grau ou amplitude, geralmente modificando adjetivos ou verbos dissilábicos}
  \definition{v.}{fazer; agir | tornar"-se; transformar"-se em | ser; significar | servir como; agir como; desempenhar o papel de | fazer; trabalhar; indica certas ações e comportamentos, incluindo os significados de governança, engajamento, cenário e pesquisa}
  \seeref{wei4}
  \seealsoref{被}{bei4}
  \seealsoref{何}{he2}
  \seealsoref{所}{suo3}
  \synonymref{替}{ti4}
\end{EntryWithPhonetic}

\begin{EntryWithPhonetic}{为难}{wei2nan2}{4,10}{⼂,⾫}[HSK 5]
  \definition{adj.}{envergonhado; sentir"-se constrangido; sentir"-se sobrecarregado; sentir"-se incapaz de lidar com algo}
  \definition{v.}{dificultar as coisas para; dificultar; contrariar}
  \synonymref{刁难}{diao1nan4}
  \synonymref{尴尬}{gan1ga4}
  \antonymref{乐意}{le4yi4}
  \antonymref{愿意}{yuan4yi5}
\end{EntryWithPhonetic}

\begin{EntryWithPhonetic}{为期}{wei2qi1}{4,12}{⼂,⽉}[HSK 5]
  \definition{s.}{Literário: tempo restante}
  \definition{v.}{Literário: a ser concluído (até uma data definida, por um determinado período de tempo)}
\end{EntryWithPhonetic}

\begin{EntryWithPhonetic}{为人}{wei2ren2}{4,2}{⼂,⼈}[HSK 7-9]
  \definition[方]{s.}{comportamento; conduta; atitude em relação aos outros e às coisas}
  \definition{v.}{comportar"-se; conduzir"-se}
  \synonymref{君子}{jun1zi3}
  \synonymref{人品}{ren2pin3}
\end{EntryWithPhonetic}

\begin{EntryWithPhonetic}{为止}{wei2zhi3}{4,4}{⼂,⽌}[HSK 5]
  \definition{adv.}{até; até um determinado momento}
  \synonymref{截止}{jie2zhi3}
  \synonymref{截至}{jie2zhi4}
\end{EntryWithPhonetic}

\begin{EntryWithPhonetic}{为主}{wei2zhu3}{4,5}{⼂,⼂}[HSK 5]
  \definition{v.}{dar prioridade a; dar preferência a; dar importância a}
\end{EntryWithPhonetic}

%%%%%%%%%% 围 %%%%%%%%%%
\subsection*{围}\addcontentsline{loh}{figure}{围 \dpy{wei2}}

\begin{EntryWithPhonetic}{围}{wei2}{7}{⼞}[HSK 3]
  \definition*{s.}{Sobrenome: Wei}
  \definition{clas.}{o comprimento das duas mãos com os polegares e os dedos indicadores juntos ou dos dois braços juntos}
  \definition{s.}{em volta de tudo; ao redor}
  \definition{v.}{cercar; rodear; circundar; encurralar; cercar tudo, impedindo a passagem entre o interior e o exterior | envolver; contornar}
\end{EntryWithPhonetic}

\begin{EntryWithPhonetic}{围巾}{wei2jin1}{7,3}{⼞,⼱}[HSK 4]
  \definition[条]{s.}{lenço; cachecol; echarpe; gravata; tiras longas de malha ou tecido usadas ao redor do pescoço para aquecimento, proteção do colarinho ou decoração}
\end{EntryWithPhonetic}

\begin{EntryWithPhonetic}{围墙}{wei2qiang2}{7,14}{⼞,⼟}[HSK 7-9]
  \definition[堵,道,面]{s.}{recinto; parede envolvente; construído em torno de uma casa, jardim, pátio, etc.; uma parede que serve de barreira}
\end{EntryWithPhonetic}

\begin{EntryWithPhonetic}{围绕}{wei2rao4}{7,9}{⼞,⽷}[HSK 5]
  \definition{v.}{girar; circundar; dar voltas; girar em torno de algo; cercar | concentrar-se em; centrar-se em; centrar-se em uma questão ou evento (para realizar atividades)}
\end{EntryWithPhonetic}

%%%%%%%%%% 违 %%%%%%%%%%
\subsection*{违}\addcontentsline{loh}{figure}{违 \dpy{wei2}}

\begin{EntryWithPhonetic}{违}{wei2}{7}{⾡}
  \definition{v.}{desobedecer; violar | ser separado; separar"-se de | desafiar; não cumprir; não obedecer}
\end{EntryWithPhonetic}

\begin{EntryWithPhonetic}{违背}{wei2bei4}{7,9}{⾡,⾁}[HSK 7-9]
  \definition{v.}{violar; infringir; ir contra; correr contra; desviar}
  \synonymref{违反}{wei2fan3}
  \antonymref{按照}{an4zhao4}
  \antonymref{服从}{fu2cong2}
  \antonymref{符合}{fu2he2}
  \antonymref{顺从}{shun4cong2}
  \antonymref{遵守}{zun1shou3}
\end{EntryWithPhonetic}

\begin{EntryWithPhonetic}{违法}{wei2 fa3}{7,8}{⾡,⽔}[HSK 5]
  \definition{v.}{ser ilegal; infringir a lei; violar a lei ou os regulamentos}
\end{EntryWithPhonetic}

\begin{EntryWithPhonetic}{违反}{wei2fan3}{7,4}{⾡,⼜}[HSK 5]
  \definition{v.}{violar; transgredir; contrariar; não estar em conformidade (com as regras, regulamentos, etc.)}
\end{EntryWithPhonetic}

\begin{EntryWithPhonetic}{违规}{wei2 gui1}{7,8}{⾡,⾒}[HSK 5]
  \definition{v.}{violar (regras); infringir as regras e regulamentos}
\end{EntryWithPhonetic}

\begin{EntryWithPhonetic}{违宪}{wei2xian4}{7,9}{⾡,⼧}
  \definition{adj.}{inconstitucional}
\end{EntryWithPhonetic}

\begin{EntryWithPhonetic}{违约}{wei2/yue1}{7,6}{⾡,⽷}[HSK 7-9]
  \definition{v.+compl.}{quebrar uma promessa; violar um acordo; ter inadimplência (em um empréstimo ou contrato); violar ou deixar de cumprir um acordo mútuo}
  \antonymref{契约}{qi4yue1}
\end{EntryWithPhonetic}

\begin{EntryWithPhonetic}{违章}{wei2/zhang1}{7,11}{⾡,⾳}[HSK 7-9]
  \definition{v.+compl.}{quebrar regras e regulamentos}
\end{EntryWithPhonetic}

%%%%%%%%%% 唯 %%%%%%%%%%
\subsection*{唯}\addcontentsline{loh}{figure}{唯 \dpy{wei2}}

\begin{EntryWithPhonetic}{唯}{wei2}{11}{⼝}[HSK 7-9]
  \definition{adv.}{somente; sozinho | ainda; somente; exceto que}
  \seeref{wei3}
\end{EntryWithPhonetic}

\begin{EntryWithPhonetic}{唯独}{wei2du2}{11,9}{⼝,⽝}[HSK 7-9]
  \definition{adv.}{apenas; sozinho; pode ser usado antes de frases verbais, significando 只 ou 仅仅; também pode ser usado antes de uma frase sujeito"-predicado, geralmente no início de uma frase, significando 只有; às vezes pode ser colocado diretamente antes do substantivo ou pronome, mas precisa ser seguido por um verbo e uma cláusula sexual}
  \seealsoref{仅仅}{jin3jin3}
  \seealsoref{只}{zhi3}
  \synonymref{唯一}{wei2yi1}
  \synonymref{只有}{zhi3you3}
  \antonymref{众多}{zhong4duo1}
\end{EntryWithPhonetic}

\begin{EntryWithPhonetic}{唯一}{wei2yi1}{11,1}{⼝,⼀}[HSK 5]
  \definition{adj.}{único; exclusivo; singular; apenas um; nenhum outro}
\end{EntryWithPhonetic}

%%%%%%%%%% 维 %%%%%%%%%%
\subsection*{维}\addcontentsline{loh}{figure}{维 \dpy{wei2}}

\begin{EntryWithPhonetic}{维}{wei2}{11}{⽷}
  \definition*{s.}{Sobrenome: Wei}
  \definition{s.}{pensamento | dimensão; conceitos básicos de geometria e teoria do espaço}
  \definition{v.}{ligar; amarrar; manter unido; conectar | manter; manter; salvaguardar; preservar}
\end{EntryWithPhonetic}

\begin{EntryWithPhonetic}{维持}{wei2chi2}{11,9}{⽷,⼿}[HSK 4]
  \definition{v.}{manter; conservar; guardar; manter vivo}
\end{EntryWithPhonetic}

\begin{EntryWithPhonetic}{维护}{wei2hu4}{11,7}{⽷,⼿}[HSK 4]
  \definition{v.}{defender; proteger; manter; preservar}
\end{EntryWithPhonetic}

\begin{EntryWithPhonetic}{维生素}{wei2sheng1su4}{11,5,10}{⽷,⽣,⽷}[HSK 6]
  \definition[点]{s.}{vitamina}[西瓜中含丰富的维生素。===A melancia é rica em vitaminas.]
\end{EntryWithPhonetic}

\begin{EntryWithPhonetic}{维吾尔}{wei2wu2'er3}{11,7,5}{⽷,⼝,⼩}
  \definition*{s.}{Etnia Uigur de Xinjiang}
\end{EntryWithPhonetic}

\begin{EntryWithPhonetic}{维修}{wei2xiu1}{11,9}{⽷,⼈}[HSK 4]
  \definition{v.}{prestar serviços; manter; reparar; manter em (bom) estado de conservação}
\end{EntryWithPhonetic}

%%%%%%%%%% 伟 %%%%%%%%%%
\subsection*{伟}\addcontentsline{loh}{figure}{伟 \dpy{wei3}}

\begin{EntryWithPhonetic}{伟}{wei3}{6}{⼈}
  \definition*{s.}{Sobrenome: Wei}
  \definition{adj.}{grande; ótimo; poderoso | Literário: grande}
\end{EntryWithPhonetic}

\begin{EntryWithPhonetic}{伟大}{wei3da4}{6,3}{⼈,⼤}[HSK 3]
  \definition{adj.}{ótimo; excelente; extrovertido; descreve uma pessoa com moral e qualidades excelentes, habilidades e realizações excepcionais, que inspira grande respeito | ótimo; magnífico; descreve algo de grande importância, com impacto significativo, acima do normal, algo notável}
  \synonymref{崇高}{chong2gao1}
  \synonymref{广大}{guang3da4}
  \synonymref{宏大}{hong2da4}
  \synonymref{宏伟}{hong2wei3}
  \synonymref{巨大}{ju4da4}
  \synonymref{庞大}{pang2da4}
  \synonymref{雄伟}{xiong2wei3}
  \antonymref{渺小}{miao3xiao3}
  \antonymref{平凡}{ping2fan2}
\end{EntryWithPhonetic}

%%%%%%%%%% 伪 %%%%%%%%%%
\subsection*{伪}\addcontentsline{loh}{figure}{伪 \dpy{wei3}}

\begin{EntryWithPhonetic}{伪}{wei3}{6}{⼈}
  \definition{adj.}{falso; falsificado | fantoche; colaboracionista; ilegal | forjado; falso}
  \definition{pref.}{pseudo-; quasi-; quase-}
  \synonymref{假}{jia3}
  \antonymref{真}{zhen1}
\end{EntryWithPhonetic}

\begin{EntryWithPhonetic}{伪造}{wei3zao4}{6,10}{⼈,⾡}[HSK 7-9]
  \definition{v.}{forjar; falsificar; fabricar; fingir}
\end{EntryWithPhonetic}

\begin{EntryWithPhonetic}{伪装}{wei3zhuang1}{6,12}{⼈,⾐}[HSK 7-9]
  \definition[种,个]{s.}{máscara; disfarce; camuflagem; aparência fingida; algo usado como disfarce}
  \definition{v.}{ser falso; fingir; camuflar; usar certos meios secretos em assuntos militares para enganar e confundir o inimigo}
  \synonymref{假装}{jia3zhuang1}
  \antonymref{真诚}{zhen1cheng2}
\end{EntryWithPhonetic}

%%%%%%%%%% 尾 %%%%%%%%%%
\subsection*{尾}\addcontentsline{loh}{figure}{尾 \dpy{wei3}}

\begin{EntryWithPhonetic}{尾}{wei3}{7}{⼫}
  \definition*{s.}{Wei, sexta das vinte e oito constelações nas quais a esfera celeste foi dividida, consistindo de nove estrelas em forma de gancho em Escorpião| Wei, uma das mansões lunares | Sobrenome: Wei}
  \definition{clas.}{usado para peixes}
  \definition{s.}{cauda; traseira | parte semelhante a uma cauda | fim | parte restante (ou inacabada); remanescente; a parte fora da parte principal; negócio inacabado}
  \seeref{yi3}
\end{EntryWithPhonetic}

\begin{EntryWithPhonetic}{尾巴}{wei3ba5}{7,4}{⼫,⼰}[HSK 4]
  \definition[条,根]{s.}{cauda; projeções na extremidade do corpo de certos animais | parte semelhante a uma cauda; refere"-se, em geral, ao final de algo | apêndice; anexo; adepto servil; pessoa que segue ou concorda com outra pessoa | Figura de linguagem: alguém que faz sombra a outro | fim; remanescente; parte restante (ou inacabada)}
\end{EntryWithPhonetic}

\begin{EntryWithPhonetic}{尾气}{wei3qi4}{7,4}{⼫,⽓}[HSK 7-9]
  \definition{s.}{gás residual; escape; emissões; gases inúteis emitidos quando carros ou máquinas estão funcionando}
\end{EntryWithPhonetic}

\begin{EntryWithPhonetic}{尾声}{wei3sheng1}{7,7}{⼫,⼠}[HSK 7-9]
  \definition[个]{s.}{fim; coda; epílogo; a última parte de uma atividade, evento, obra literária e artística, etc.}
  \synonymref{结束}{jie2shu4}
  \synonymref{结尾}{jie2wei3}
\end{EntryWithPhonetic}

%%%%%%%%%% 纬 %%%%%%%%%%
\subsection*{纬}\addcontentsline{loh}{figure}{纬 \dpy{wei3}}

\begin{EntryWithPhonetic}{纬}{wei3}{7}{⽷}
  \definition*{s.}{Sobrenome: Wei}
  \definition[本]{s.}{trama; o fio ou linha horizontal no tecido | latitude}
  \antonymref{经}{jing1}
\end{EntryWithPhonetic}

\begin{EntryWithPhonetic}{纬度}{wei3du4}{7,9}{⽷,⼴}[HSK 7-9]
  \definition{s.}{latitude}
\end{EntryWithPhonetic}

\begin{EntryWithPhonetic}{纬线}{wei3xian4}{7,8}{⽷,⽷}
  \definition{s.}{trama (têxtil); tecido | paralelo; linha de latitude}
\end{EntryWithPhonetic}

%%%%%%%%%% 委 %%%%%%%%%%
\subsection*{委}\addcontentsline{loh}{figure}{委 \dpy{wei3}}

\begin{EntryWithPhonetic}{委}{wei3}{8}{⼥}
  \definition*{s.}{Sobrenome: Wei}
  \definition{adj.}{indireto; desviado | apático; abatido | sinuoso; tortuoso | desanimado; apático; sem inspiração}
  \definition{adv.}{realmente; certamente; na verdade}
  \definition{s.}{membro do comitê | comitê; comissão; conselho}
  \definition{v.}{confiar; nomear |  jogar fora; deixar de lado | culpar os outros | confiar | descartar; abandonar | mudar; empurrar | acumular}
  \seeref{wei1}
\end{EntryWithPhonetic}

\begin{EntryWithPhonetic}{委内瑞拉}{wei3nei4rui4la1}{8,4,13,8}{⼥,⼌,⽟,⼿}
  \definition*{s.}{Venezuela}
\end{EntryWithPhonetic}

\begin{EntryWithPhonetic}{委屈}{wei3qu5}{8,8}{⼥,⼫}[HSK 7-9]
  \definition{adj.}{injustiçado; ofendido}
  \definition[点,次]{s.}{reclamação; tratamento injusto, injustiça ou maus tratos, enfatizando uma experiência ou estado negativo, triste e injusto}
  \definition{v.}{sentir"-se injustiçado; cuidar de uma queixa; sofrer com a injustiça; fazer alguém se sentir injustiçado; tratar alguém imerecidamente}
  \synonymref{无辜}{wu2gu1}
\end{EntryWithPhonetic}

\begin{EntryWithPhonetic}{委托}{wei3tuo1}{8,6}{⼥,⼿}[HSK 5]
  \definition{v.}{confiar; confiar uma tarefa a outra pessoa ou instituição (para que seja realizada)}
\end{EntryWithPhonetic}

\begin{EntryWithPhonetic}{委婉}{wei3wan3}{8,11}{⼥,⼥}[HSK 7-9]
  \definition{adj.}{diplomático; indireto (de palavras); rítmico (de voz); a linguagem usada para descrevê"-lo não é muito direta ou o som é alto ou baixo, muito bonito}
  \synonymref{含蓄}{han2xu4}
  \antonymref{坦白}{tan3bai2}
\end{EntryWithPhonetic}

\begin{EntryWithPhonetic}{委员}{wei3yuan2}{8,7}{⼥,⼝}[HSK 7-9]
  \definition[个,位,名]{s.}{membro; membro do comitê; membro de uma comissão; membros de organizações de liderança coletiva, tais como instituições, grupos ou escolas; membros de organizações especializadas criadas para realizar determinadas tarefas | enviado com comissão especial; uma pessoa com responsabilidade atribuída a uma tarefa específica}
\end{EntryWithPhonetic}

\begin{EntryWithPhonetic}{委员会}{wei3yuan2hui4}{8,7,6}{⼥,⼝,⼈}[HSK 7-9]
  \definition[个]{s.}{conselho; comitê; comissão; organizações de liderança coletiva em partidos políticos, grupos, agências e escolas | comitê; nome do departamento ou agência governamental | comitê; organizações especializadas estabelecidas por agências, grupos, escolas, etc. para completar certas tarefas}
  \synonymref{部门}{bu4men2}
  \synonymref{机构}{ji1gou4}
\end{EntryWithPhonetic}

%%%%%%%%%% 唯 %%%%%%%%%%
\subsection*{唯}\addcontentsline{loh}{figure}{唯 \dpy{wei3}}

\begin{EntryWithPhonetic}{唯}{wei3}{11}{⼝}
  \definition{interj.}{Onomatopéia: ``Sim!''; ``Yea!''; significa uma palavra que indica acordo}
  \seeref{wei2}
\end{EntryWithPhonetic}

%%%%%%%%%% 萎 %%%%%%%%%%
\subsection*{萎}\addcontentsline{loh}{figure}{萎 \dpy{wei3}}

\begin{EntryWithPhonetic}{萎}{wei3}{11}{⾋}
  \definition{adj.}{em declínio; decadente | sem ânimo; abatido}
  \definition{v.}{murchar; definhar; tombar; deixar cair}
  \seeref{wei1}
  \synonymref{缩}{suo1}
  \synonymref{谢}{xie4}
\end{EntryWithPhonetic}

\begin{EntryWithPhonetic}{萎缩}{wei3suo1}{11,14}{⾋,⽷}[HSK 7-9]
  \definition{v.}{murchar; (corpo, vegetação, etc.) secar | ceder; encolher; contrair}
  \synonymref{收缩}{shou1suo1}
  \antonymref{发达}{fa1da2}
  \antonymref{蔓延}{man4yan2}
  \antonymref{蓬勃}{peng2bo2}
  \antonymref{膨胀}{peng2zhang4}
\end{EntryWithPhonetic}

%%%%%%%%%% 卫 %%%%%%%%%%
\subsection*{卫}\addcontentsline{loh}{figure}{卫 \dpy{wei4}}

\begin{EntryWithPhonetic}{卫}{wei4}{3}{⼙}
  \definition*{s.}{Wei, um estado da Dinastia Zhou | Sobrenome: Wei}
  \definition{s.}{uma palavra usada no nome do lugar | outro nome para um burro}
  \definition{v.}{defender; guardar; proteger}
\end{EntryWithPhonetic}

\begin{EntryWithPhonetic}{卫生}{wei4sheng1}{3,5}{⼙,⽣}[HSK 3]
  \definition{adj.}{bom para a saúde; higiênico; limpo; capaz de prevenir doenças e benéfico para a saúde}
  \definition{s.}{higiene; saneamento; situação limpa}
\end{EntryWithPhonetic}

\begin{EntryWithPhonetic}{卫生部}{wei4sheng1bu4}{3,5,10}{⼙,⽣,⾢}
  \definition*{s.}{Ministério da Saúde}
\end{EntryWithPhonetic}

\begin{EntryWithPhonetic}{卫生防疫}{wei4sheng1 fang2yi4}{3,5,6,9}{⼙,⽣,⾩,⽧}
  \definition{s.}{prevenção contra a epidemia}
\end{EntryWithPhonetic}

\begin{EntryWithPhonetic}{卫生间}{wei4sheng1jian1}{3,5,7}{⼙,⽣,⾨}[HSK 3]
  \definition[间,个]{s.}{banheiro; sanitário; \emph{toilette}; quartos com instalações sanitárias em hotéis ou residências}
\end{EntryWithPhonetic}

\begin{EntryWithPhonetic}{卫生巾}{wei4sheng1jin1}{3,5,3}{⼙,⽣,⼱}
  \definition{s.}{absorvente higiênico}
\end{EntryWithPhonetic}

\begin{EntryWithPhonetic}{卫生局}{wei4sheng1ju2}{3,5,7}{⼙,⽣,⼫}
  \definition*{s.}{Departamento de Saúde | Escritório de Saúde}
\end{EntryWithPhonetic}

\begin{EntryWithPhonetic}{卫生棉}{wei4sheng1mian2}{3,5,12}{⼙,⽣,⽊}
  \definition{s.}{absorvente | algodão absorvente esterilizado (usado para curativos ou limpeza de feridas) | absorvente tampão}
\end{EntryWithPhonetic}

\begin{EntryWithPhonetic}{卫生球}{wei4sheng1qiu2}{3,5,11}{⼙,⽣,⽟}
  \definition{s.}{naftalina}
\end{EntryWithPhonetic}

\begin{EntryWithPhonetic}{卫生署}{wei4sheng1shu3}{3,5,13}{⼙,⽣,⽹}
  \definition*{s.}{Agência de Saúde (ou Escritório, ou Departamento)}
\end{EntryWithPhonetic}

\begin{EntryWithPhonetic}{卫生套}{wei4sheng1tao4}{3,5,10}{⼙,⽣,⼤}
  \definition[只]{s.}{preservativo | camisinha}
\end{EntryWithPhonetic}

\begin{EntryWithPhonetic}{卫生厅}{wei4 sheng1 ting1}{3,5,4}{⼙,⽣,⼚}
  \definition*{s.}{Departamento de Saúde (da Província)}
\end{EntryWithPhonetic}

\begin{EntryWithPhonetic}{卫生纸}{wei4sheng1zhi3}{3,5,7}{⼙,⽣,⽷}
  \definition{s.}{papel higiênico}
\end{EntryWithPhonetic}

\begin{EntryWithPhonetic}{卫视}{wei4shi4}{3,8}{⼙,⾒}[HSK 7-9]
  \definition[大,家]{s.}{televisão por satélite; abreviação de 卫星电视}
  \seealsoref{卫星电视}{wei4xing1 dian4shi4}
\end{EntryWithPhonetic}

\begin{EntryWithPhonetic}{卫星}{wei4xing1}{3,9}{⼙,⽇}[HSK 5]
  \definition[个,颗]{s.}{satélite; lua; corpos celestes orbitando planetas | satélite artificial | algo que gira em torno de um centro}
\end{EntryWithPhonetic}

\begin{EntryWithPhonetic}{卫星电视}{wei4xing1 dian4shi4}{3,9,5,8}{⼙,⽇,⽥,⾒}
  \definition{s.}{TV por satélite; televisão por satélite}
\end{EntryWithPhonetic}

%%%%%%%%%% 为 %%%%%%%%%%
\subsection*{为}\addcontentsline{loh}{figure}{为 \dpy{wei4}}

\begin{EntryWithPhonetic}{为}{wei4}{4}{⼂}[HSK 2]
  \definition*{s.}{Sobrenome: Wei}
  \definition{part.}{com 何 em uma pergunta retórica para expressar dúvida}
  \definition{prep.}{por; usado em frases passivas para introduzir o agente da ação, equivalente a 被 (frequentemente usado com 所)}
  \definition{suf.}{é anexado a alguns adjetivos ou advérbios monossilábicos para formar advérbios dissilábicos que expressam grau ou amplitude, geralmente modificando adjetivos ou verbos dissilábicos}
  \definition{v.}{fazer; agir | tornar"-se; transformar-se em | ser;  significar | servir como; agir como; desempenhar o papel de | fazer; trabalhar; indica certas ações e comportamentos, incluindo os significados de governança, engajamento, cenário e pesquisa}
  \seeref{wei2}
  \seealsoref{被}{bei4}
  \seealsoref{何}{he2}
  \seealsoref{所}{suo3}
  \synonymref{替}{ti4}
\end{EntryWithPhonetic}

\begin{EntryWithPhonetic}{为此}{wei4ci3}{4,6}{⼂,⽌}[HSK 6]
  \definition{conj.}{portanto; para este fim; por esta razão; para este propósito; nesta conexão; contexto de conexão, indicando que o comportamento descrito é devido aos motivos mencionados anteriormente}
  \synonymref{所以}{suo3yi3}
  \synonymref{因而}{yin1'er2}
  \synonymref{因此}{yin1ci3}
\end{EntryWithPhonetic}

\begin{EntryWithPhonetic}{为何}{wei4he2}{4,7}{⼂,⼈}[HSK 6]
  \definition{adv.}{por que?; por qual razão?}
  \seealsoref{为什么}{wei4shen2me5}
  \synonymref{何故}{he2gu4}
\end{EntryWithPhonetic}

\begin{EntryWithPhonetic}{为了}{wei4le5}{4,2}{⼂,⼅}[HSK 3]
  \definition{prep.}{para; por causa de; a fim de; o objetivo da introdução de ações comportamentais}
  \synonymref{使得}{shi3de5}
  \synonymref{为}{wei4}
  \synonymref{因为}{yin1wei5}
  \synonymref{着想}{zhuo2xiang3}
\end{EntryWithPhonetic}

\begin{EntryWithPhonetic}{为什么}{wei4shen2me5}{4,4,3}{⼂,⼈,⼃}[HSK 2]
  \definition{adv.}{por que?; por que é que?; como é que?;  nota: 为什么不 geralmente tem o significado de conselho, o mesmo que 何不}
  \seealsoref{何不}{he2bu4}
\end{EntryWithPhonetic}

%%%%%%%%%% 未 %%%%%%%%%%
\subsection*{未}\addcontentsline{loh}{figure}{未 \dpy{wei4}}

\begin{EntryWithPhonetic}{未}{wei4}{5}{⽊}[HSK 7-9]
  \definition*{s.}{Sobrenome: Wei}
  \definition{adv.}{Literário: não tem; não fez | Literário: não}
  \definition{part.}{ou não; no final das perguntas, indicando dúvida}[今可以言未?===Posso falar agora?]
  \definition{s.}{wei (oitavo dos doze Ramos Terrestres)}
  \antonymref{已}{yi3}
\end{EntryWithPhonetic}

\begin{EntryWithPhonetic}{未必}{wei4bi4}{5,5}{⽊,⼼}[HSK 4]
  \definition{adv.}{não tenho certeza; talvez não; não necessariamente}
\end{EntryWithPhonetic}

\begin{EntryWithPhonetic}{未成年人}{wei4cheng2nian2ren2}{5,6,6,2}{⽊,⼽,⼲,⼈}[HSK 7-9]
  \definition[名,个]{s.}{menores; juvenis; uma pessoa que não atingiu a maioridade e é legalmente incapaz de exercer direitos privados, e que necessita da administração de terceiros}
\end{EntryWithPhonetic}

\begin{EntryWithPhonetic}{未经}{wei4jing1}{5,8}{⽊,⽷}[HSK 7-9]
  \definition{v.}{não ter sido submetido a; sem ter passado por um determinado processo; não passar por (um determinado processo)}
\end{EntryWithPhonetic}

\begin{EntryWithPhonetic}{未来}{wei4lai2}{5,7}{⽊,⽊}[HSK 4]
  \definition{adj.}{próximo (refere"-se ao tempo)}
  \definition[个,段,种]{s.}{futuro; o amanhã}
\end{EntryWithPhonetic}

\begin{EntryWithPhonetic}{未免}{wei4mian3}{5,7}{⽊,⼉}[HSK 7-9]
  \definition{adv.}{bastante; verdadeiramente; um pouco demais (implicando discordância); isso é absolutamente verdade; chegamos ao ponto em que precisamos dizer isso; isso indica que o falante sente que não deveria ser feito dessa maneira e está rejeitando uma determinada abordagem | naturalmente; inevitavelmente; é inevitável que certas situações ocorram}
  \synonymref{不免}{bu4mian3}
  \synonymref{难免}{nan2mian3}
\end{EntryWithPhonetic}

\begin{EntryWithPhonetic}{未知数}{wei4zhi1shu4}{5,8,13}{⽊,⽮,⽁}[HSK 7-9]
  \definition[个]{s.}{Matemática: número desconhecido | algo desconhecido; incerteza | desconhecido; incerto}
\end{EntryWithPhonetic}

%%%%%%%%%% 位 %%%%%%%%%%
\subsection*{位}\addcontentsline{loh}{figure}{位 \dpy{wei4}}

\begin{EntryWithPhonetic}{位}{wei4}{7}{⼈}[HSK 2]
  \definition*{s.}{Sobrenome: Wei}
  \definition{clas.}{usado para pessoas (com cortesia, respeito) | usado para bits binários}[十六位===16 bits]
  \definition{s.}{lugar; localização; o lugar onde ou onde alguém está localizado | posto; \emph{status}; posição; a posição de uma pessoa em uma determinada área da vida social | trono; refere"-se especificamente ao status do imperador | lugar; dígito; a posição de cada dígito em um número}
\end{EntryWithPhonetic}

\begin{EntryWithPhonetic}{位居}{wei4ju1}{7,8}{⼈,⼫}
  \definition{v.}{estar localizado em}
\end{EntryWithPhonetic}

\begin{EntryWithPhonetic}{位于}{wei4yu2}{7,3}{⼈,⼆}[HSK 4]
  \definition{v.}{estar localizado; estar situado}
  \synonymref{处于}{chu3yu2}
\end{EntryWithPhonetic}

\begin{EntryWithPhonetic}{位置}{wei4zhi5}{7,13}{⼈,⽹}[HSK 4]
  \definition[个,处]{s.}{assento; lugar; localização | lugar; posição; \emph{status} | posição (por exemplo: cargo no escritório)}
  \synonymref{场所}{chang3suo3}
  \synonymref{地点}{di4dian3}
  \synonymref{地位}{di4wei4}
  \synonymref{地址}{di4zhi3}
  \synonymref{地方}{di4fang5}
  \synonymref{对称}{dui4chen4}
  \synonymref{方向}{fang1xiang5}
  \synonymref{位子}{wei4zi5}
\end{EntryWithPhonetic}

\begin{EntryWithPhonetic}{位子}{wei4zi5}{7,3}{⼈,⼦}[HSK 7-9]
  \definition[个]{s.}{assento; lugar; posição; o espaço ocupado por pessoas}
  \synonymref{地位}{di4wei4}
  \synonymref{位置}{wei4zhi5}
\end{EntryWithPhonetic}

%%%%%%%%%% 味 %%%%%%%%%%
\subsection*{味}\addcontentsline{loh}{figure}{味 \dpy{wei4}}

\begin{EntryWithPhonetic}{味}{wei4}{8}{⼝}
  \definition{clas.}{usado para ingredientes (de uma receita de medicina chinesa)}
  \definition{s.}{gosto; sabor | cheiro; odor | interesse; deleite | acepipe; \emph{delicacy} | significância; significado}
  \definition{v.}{distinguir (provar) o sabor de; saborear}
\end{EntryWithPhonetic}

\begin{EntryWithPhonetic}{味道}{wei4dao5}{8,12}{⼝,⾡}[HSK 2]
  \definition[个,股,种]{s.}{gosto; sabor | sensação; gosto; experiência | interesse; deleite | cheiro; odor}
\end{EntryWithPhonetic}

\begin{EntryWithPhonetic}{味精}{wei4jing1}{8,14}{⼝,⽶}[HSK 7-9]
  \definition[勺,袋,克]{s.}{glutamato monossódico (MSG); pó gourmet; temperos, em pó branco ou cristais granulados, adicionados a sopas e pratos para realçar o sabor umami}
\end{EntryWithPhonetic}

\begin{EntryWithPhonetic}{味儿}{wei4r5}{8,2}{⼝,⼉}[HSK 4]
  \definition{s.}{gosto; sabor; propriedade de uma substância que dá à língua uma determinada sensação de sabor | cheiro; odor; propriedade de uma substância que dá ao nariz um determinado sentido de cheiro | interesse; significado; deleite}
\end{EntryWithPhonetic}

%%%%%%%%%% 畏 %%%%%%%%%%
\subsection*{畏}\addcontentsline{loh}{figure}{畏 \dpy{wei4}}

\begin{EntryWithPhonetic}{畏}{wei4}{9}{⽥}
  \definition{s.}{medo; temor}
  \definition{v.}{ter medo; temer | respeitar; admirar}
\end{EntryWithPhonetic}

\begin{EntryWithPhonetic}{畏惧}{wei4ju4}{9,11}{⽥,⼼}[HSK 7-9]
  \definition{v.}{temer; recear; estar muito assustado}
  \synonymref{胆怯}{dan3qie4}
  \synonymref{害怕}{hai4pa4}
  \synonymref{恐惧}{kong3ju4}
  \synonymref{恐怕}{kong3pa4}
  \synonymref{畏缩}{wei4suo1}
  \antonymref{勇敢}{yong3gan3}
\end{EntryWithPhonetic}

\begin{EntryWithPhonetic}{畏缩}{wei4suo1}{9,14}{⽥,⽷}[HSK 7-9]
  \definition{v.}{recuar; encolher; sobressaltar; o medo impede o avanço}
  \synonymref{胆怯}{dan3qie4}
  \synonymref{害怕}{hai4pa4}
  \synonymref{后退}{hou4tui4}
  \synonymref{恐惧}{kong3ju4}
  \synonymref{退却}{tui4que4}
  \synonymref{退缩}{tui4suo1}
  \synonymref{畏惧}{wei4ju4}
  \antonymref{挺身}{ting3shen1}
\end{EntryWithPhonetic}

%%%%%%%%%% 胃 %%%%%%%%%%
\subsection*{胃}\addcontentsline{loh}{figure}{胃 \dpy{wei4}}

\begin{EntryWithPhonetic}{胃}{wei4}{9}{⾁}[HSK 5]
  \definition*{s.}{Wei, uma das mansões lunares | Wei, uma das vinte e oito constelações}
  \definition{s.}{estômago; parte do aparelho digestivo}
\end{EntryWithPhonetic}

\begin{EntryWithPhonetic}{胃口}{wei4kou3}{9,3}{⾁,⼝}[HSK 7-9]
  \definition{s.}{apetite; a sensação de querer comer | gostar; ter interesse em algo ou em atividades; metaforicamente, refere"-se ao interesse por coisas ou atividades | ambição; apetite}
\end{EntryWithPhonetic}

%%%%%%%%%% 喂 %%%%%%%%%%
\subsection*{喂}\addcontentsline{loh}{figure}{喂 \dpy{wei4}}

\begin{EntryWithPhonetic}{喂}{wei4}{12}{⼝}[HSK 2,4]
  \definition{interj.}{``Ei!'', ``Olá!'', para chamar atenção | ``Alô?'' (quando respondendo uma chamada telefônica, pronuncia"-se como \dpy{wei2})}
  \definition{v.}{criar; alimentar (animais); dar comida a um animal | alimentar (pessoas); colocar alimentos, medicamentos, etc. na boca de alguém}
\end{EntryWithPhonetic}

\begin{EntryWithPhonetic}{喂哺}{wei4bu3}{12,10}{⼝,⼝}
  \definition{v.}{alimentar (um bebê)}
\end{EntryWithPhonetic}

\begin{EntryWithPhonetic}{喂料}{wei4liao4}{12,10}{⼝,⽃}
  \definition{v.}{alimentar (também no sentido figurativo)}
\end{EntryWithPhonetic}

\begin{EntryWithPhonetic}{喂母乳}{wei4mu3ru3}{12,5,8}{⼝,⽏,⼄}
  \definition{s.}{amamentação}
\end{EntryWithPhonetic}

\begin{EntryWithPhonetic}{喂奶}{wei4nai3}{12,5}{⼝,⼥}
  \definition{v.}{amamentar}
\end{EntryWithPhonetic}

\begin{EntryWithPhonetic}{喂食}{wei4shi2}{12,9}{⼝,⾷}
  \definition{v.}{alimentar}
\end{EntryWithPhonetic}

\begin{EntryWithPhonetic}{喂养}{wei4yang3}{12,9}{⼝,⼋}[HSK 7-9]
  \definition{v.}{alimentar (uma criança, animal doméstico, etc.); manter; criar (um animal); alimentar as crianças pequenas ou os animais e cuidar deles para ajudá-los a crescer}
  \synonymref{放养}{fang4yang3}
  \synonymref{饲养}{si4yang3}
  \synonymref{喂食}{wei4shi2}
\end{EntryWithPhonetic}

%%%%%%%%%% 慰 %%%%%%%%%%
\subsection*{慰}\addcontentsline{loh}{figure}{慰 \dpy{wei4}}

\begin{EntryWithPhonetic}{慰}{wei4}{15}{⼼}
  \definition{adj.}{aliviado; em paz; confortável}
  \definition{v.}{consolar; confortar | ser (ficar) aliviado}
\end{EntryWithPhonetic}

\begin{EntryWithPhonetic}{慰劳}{wei4lao2}{15,7}{⼼,⼒}[HSK 7-9]
  \definition{v.}{levar presentes ou enviar votos de felicidades em reconhecimento aos serviços prestados | confortar | demonstrar apreço (através de palavras gentis, pequenos presentes, etc.)}
  \synonymref{安慰}{an1wei4}
  \synonymref{慰问}{wei4wen4}
  \synonymref{问候}{wen4hou4}
  \antonymref{打击}{da3ji1}
\end{EntryWithPhonetic}

\begin{EntryWithPhonetic}{慰问}{wei4wen4}{15,6}{⼼,⾨}[HSK 5]
  \definition{v.}{visitar; consolar; expressar simpatia por; confortar e cumprimentar com palavras e presentes;  enfatizar o conforto e o cumprimento, frequentemente usado por superiores para subordinados}
\end{EntryWithPhonetic}

%%%%%%%%%% 温 %%%%%%%%%%
\subsection*{温}\addcontentsline{loh}{figure}{温 \dpy{wen1}}

\begin{EntryWithPhonetic}{温}{wen1}{12}{⽔}
  \definition{adj.}{morno; quente; suave}
  \definition{s.}{temperatura | doenças transmissíveis agudas; praga}
  \definition{v.}{aquecer; reaquecer; aquecer ligeiramente | revisar; repassar}
\end{EntryWithPhonetic}

\begin{EntryWithPhonetic}{温度}{wen1du4}{12,9}{⽔,⼴}[HSK 2]
  \definition[度,级,档,个]{s.}{temperatura}
\end{EntryWithPhonetic}

\begin{EntryWithPhonetic}{温度表}{wen1du4biao3}{12,9,8}{⽔,⼴,⾐}
  \definition{s.}{termômetro}
\end{EntryWithPhonetic}

\begin{EntryWithPhonetic}{温度计}{wen1du4ji4}{12,9,4}{⽔,⼴,⾔}[HSK 7-9]
  \definition[支,个]{s.}{termômetro; termógrafo}
  \synonymref{温度表}{wen1du4biao3}
\end{EntryWithPhonetic}

\begin{EntryWithPhonetic}{温度梯度}{wen1du4ti1du4}{12,9,11,9}{⽔,⼴,⽊,⼴}
  \definition{s.}{gradiente de temperatura}
\end{EntryWithPhonetic}

\begin{EntryWithPhonetic}{温和}{wen1he2}{12,8}{⽔,⼝}[HSK 5]
  \definition{adj.}{gentil; suave; moderado}
\end{EntryWithPhonetic}

\begin{EntryWithPhonetic}{温暖}{wen1nuan3}{12,13}{⽔,⽇}[HSK 3]
  \definition{adj.}{caloroso; gentil; amigável | caloroso; quente}
  \definition{v.}{aquecer; fazer com que se sinta calor}
\end{EntryWithPhonetic}

\begin{EntryWithPhonetic}{温泉}{wen1quan2}{12,9}{⽔,⽔}[HSK 7-9]
  \definition[处,座,眼]{s.}{fonte termal; água de nascente com temperatura superior à temperatura média anual local}
\end{EntryWithPhonetic}

\begin{EntryWithPhonetic}{温柔}{wen1rou2}{12,9}{⽔,⽊}[HSK 7-9]
  \definition{adj.}{manso; gentil; agradavelmente afetuoso; delicado e suave; gentil e submissa (geralmente usado para descrever mulheres)}
  \seealsoref{温和}{wen1he2}
  \synonymref{温顺}{wen1shun4}
  \antonymref{粗暴}{cu1bao4}
\end{EntryWithPhonetic}

\begin{EntryWithPhonetic}{温室}{wen1shi4}{12,9}{⽔,⼧}[HSK 7-9]
  \definition[个,座,间]{s.}{estufa; jardim de inverno; originalmente referindo"-se a um ambiente interno aquecido, o termo é agora usado principalmente no contexto do aquecimento global | um ambiente criado propositadamente para que as pessoas cresçam sem desvantagens; uma metáfora para um ambiente de vida confortável}
\end{EntryWithPhonetic}

\begin{EntryWithPhonetic}{温顺}{wen1shun4}{12,9}{⽔,⾴}
  \definition{adj.}{dócil; manso; gentil e obediente}
  \synonymref{和气}{he2qi5}
  \synonymref{暖和}{nuan3huo5}
  \synonymref{平和}{ping2he2}
  \synonymref{温和}{wen1he2}
  \synonymref{温暖}{wen1nuan3}
  \synonymref{温柔}{wen1rou2}
  \antonymref{暴躁}{bao4zao4}
  \antonymref{乖张}{guai1zhang1}
\end{EntryWithPhonetic}

\begin{EntryWithPhonetic}{温习}{wen1xi2}{12,3}{⽔,⼄}[HSK 7-9]
  \definition{v.}{revisar; analisar; reaprender o que você já aprendeu ajuda a consolidar seu entendimento}
  \synonymref{复习}{fu4xi2}
\end{EntryWithPhonetic}

\begin{EntryWithPhonetic}{温馨}{wen1xin1}{12,20}{⽔,⾹}[HSK 7-9]
  \definition{adj.}{quente; macio; doce; descreve um ambiente ou atmosfera que faz as pessoas se sentirem acolhidas e confortáveis}
  \synonymref{和睦}{he2mu4}
  \synonymref{融洽}{rong2qia4}
  \antonymref{冷漠}{leng3mo4}
\end{EntryWithPhonetic}

%%%%%%%%%% 瘟 %%%%%%%%%%
\subsection*{瘟}\addcontentsline{loh}{figure}{瘟 \dpy{wen1}}

\begin{EntryWithPhonetic}{瘟}{wen1}{14}{⽧}
  \definition{adj.}{(ópera tradicional) enfadonho e insípido; (ópera tradicional) deprimente e entediante; (ópera tradicional) monótona e entediante}
  \definition[场]{s.}{doenças transmissíveis agudas}
\end{EntryWithPhonetic}

\begin{EntryWithPhonetic}{瘟疫}{wen1yi4}{14,9}{⽧,⽧}[HSK 7-9]
  \definition{s.}{praga; pestilência; refere"-se a doenças infecciosas agudas epidêmicas}
\end{EntryWithPhonetic}

%%%%%%%%%% 文 %%%%%%%%%%
\subsection*{文}\addcontentsline{loh}{figure}{文 \dpy{wen2}}

\begin{EntryWithPhonetic}{文}{wen2}{4}{⽂}[HSK 7-9][Kangxi 67]
  \definition*{s.}{Sobrenome: Wen}
  \definition{adj.}{civil; não militar | suave; refinado | refinado; literário; descreve que o conteúdo de um artigo ou discurso é difícil de entender}
  \definition{clas.}{usado para moedas de cobre antigas (moedas de cobre com palavras gravadas em um lado)}
  \definition{s.}{roteiro; escrita; personagem | escrita; composição literária; artigo | linguagem literária; redação | cultura; refere"-se ao estado manifestado quando a sociedade atinge um estágio superior de desenvolvimento | ritual; refere"-se ao antigo sistema ritual e musical | certos fenômenos naturais; refere"-se a certos fenômenos na natureza ou na sociedade humana | literatos; coisas não militares | arte liberal; refere"-se às ciências humanas e sociais | documento; refere"-se a documentos oficiais | padrão; textura}
  \definition{v.}{tatuar padrões ou palavras no corpo ou no rosto | cobrir; pintar por cima}
  \antonymref{武}{wu3}
\end{EntryWithPhonetic}

\begin{EntryWithPhonetic}{文化}{wen2hua4}{4,4}{⽂,⼔}[HSK 3]
  \definition[个,种]{s.}{cultura; civilização; tudo o que os seres humanos criaram em termos materiais e espirituais ao longo da história social | cultura; alfabetização; escolaridade; educação; o nível de conhecimento das pessoas e a capacidade de usar a linguagem escrita}
\end{EntryWithPhonetic}

\begin{EntryWithPhonetic}{文化层}{wen2hua4ceng2}{4,4,7}{⽂,⼔,⼫}
  \definition{s.}{nível de cultura (em sítio arqueológico)}
\end{EntryWithPhonetic}

\begin{EntryWithPhonetic}{文化宫}{wen2hua4gong1}{4,4,9}{⽂,⼔,⼧}
  \definition{s.}{palácio cultural}
\end{EntryWithPhonetic}

\begin{EntryWithPhonetic}{文化圈}{wen2hua4quan1}{4,4,11}{⽂,⼔,⼞}
  \definition{s.}{esfera de influência cultural}
\end{EntryWithPhonetic}

\begin{EntryWithPhonetic}{文化热}{wen2hua4re4}{4,4,10}{⽂,⼔,⽕}
  \definition{s.}{mania cultural | febre cultural}
\end{EntryWithPhonetic}

\begin{EntryWithPhonetic}{文化史}{wen2hua4shi3}{4,4,5}{⽂,⼔,⼝}
  \definition*{s.}{História Cultural}
\end{EntryWithPhonetic}

\begin{EntryWithPhonetic}{文化水平}{wen2hua4 shui3ping2}{4,4,4,5}{⽂,⼔,⽔,⼲}
  \definition{s.}{nível educacional}
\end{EntryWithPhonetic}

\begin{EntryWithPhonetic}{文化障碍}{wen2hua4 zhang4'ai4}{4,4,13,13}{⽂,⼔,⾩,⽯}
  \definition{s.}{barreira cultural}
\end{EntryWithPhonetic}

\begin{EntryWithPhonetic}{文件}{wen2jian4}{4,6}{⽂,⼈}[HSK 3]
  \definition[份,堆,叠]{s.}{documentos oficiais; papéis; instrumentos; termo genérico para documentos oficiais, cartas, etc. | os arquivos no computador; informações registradas no celular ou computador | artigos ou trabalhos sobre teorias políticas, atualidades, pesquisas acadêmicas, etc.; textos ou artigos sobre teoria política, políticas, etc.}
\end{EntryWithPhonetic}

\begin{EntryWithPhonetic}{文具}{wen2ju4}{4,8}{⽂,⼋}[HSK 7-9]
  \definition[件,套]{s.}{artigos de papelaria; materiais de escrita; termo genérico para instrumentos de escrita, como pincéis, tinta, papel e tinteiros}
\end{EntryWithPhonetic}

\begin{EntryWithPhonetic}{文科}{wen2ke1}{4,9}{⽂,⽲}[HSK 7-9]
  \definition{s.}{artes liberais; uma disciplina que engloba matérias como literatura, línguas, história, economia, filosofia e direito, em oposição a matérias científicas como matemática, física e química}
  \antonymref{理科}{li3ke1}
\end{EntryWithPhonetic}

\begin{EntryWithPhonetic}{文盲}{wen2mang2}{4,8}{⽂,⽬}[HSK 7-9]
  \definition[个,批]{s.}{analfabetismo; pessoa analfabeta; adultos analfabetos ou cujas habilidades de alfabetização não atendem aos padrões nacionais e que não possuem habilidades básicas de leitura e escrita}
  \antonymref{文明}{wen2ming2}
\end{EntryWithPhonetic}

\begin{EntryWithPhonetic}{文明}{wen2ming2}{4,8}{⽂,⽇}[HSK 3]
  \definition{adj.}{civilizado; sociedade desenvolvida e com alto nível cultural}
  \definition[个,种]{s.}{cultura; civilização}
\end{EntryWithPhonetic}

\begin{EntryWithPhonetic}{文凭}{wen2ping2}{4,8}{⽂,⼏}[HSK 7-9]
  \definition[张]{s.}{diploma; documento oficial usado como comprovante da escolaridade recebida; anteriormente referente a documentos oficiais usados como comprovante, agora se refere especificamente a certificados de conclusão de curso}
  \synonymref{证书}{zheng4shu1}
\end{EntryWithPhonetic}

\begin{EntryWithPhonetic}{文人}{wen2ren2}{4,2}{⽂,⼈}[HSK 7-9]
  \definition[位,个]{s.}{homem de letras; erudito; literato; um acadêmico que escreve bem}
\end{EntryWithPhonetic}

\begin{EntryWithPhonetic}{文物}{wen2wu4}{4,8}{⽂,⽜}[HSK 7-9]
  \definition[件,个,批]{s.}{relíquia cultural; relíquia histórica; objetos valiosos deixados pela história}
\end{EntryWithPhonetic}

\begin{EntryWithPhonetic}{文献}{wen2xian4}{4,13}{⽂,⽝}[HSK 7-9]
  \definition[篇,种,份]{s.}{documento; literatura; livros e materiais com valor histórico ou de referência}
  \synonymref{文件}{wen2jian4}
\end{EntryWithPhonetic}

\begin{EntryWithPhonetic}{文学}{wen2xue2}{4,8}{⽂,⼦}[HSK 3]
  \definition[个,种]{s.}{literatura; a arte de moldar imagens e refletir a vida social através da linguagem e da escrita, incluindo romances, poesia, prosa, teatro, etc.}
\end{EntryWithPhonetic}

\begin{EntryWithPhonetic}{文学系}{wen2xue2 xi4}{4,8,7}{⽂,⼦,⽷}
  \definition*{s.}{Departamento de Literatura}
\end{EntryWithPhonetic}

\begin{EntryWithPhonetic}{文雅}{wen2ya3}{4,12}{⽂,⾫}[HSK 7-9]
  \definition{adj.}{refinado; elegante (na fala e nos modos); descreve alguém como alguém que fala e se comporta de maneira gentil, educada e sem vulgaridades}
  \synonymref{大方}{da4fang5}
  \synonymref{高雅}{gao1ya3}
  \synonymref{美丽}{mei3li4}
  \synonymref{时髦}{shi2mao2}
  \synonymref{文化}{wen2hua4}
  \synonymref{文明}{wen2ming2}
  \antonymref{粗鲁}{cu1lu3}
  \antonymref{难听}{nan2ting1}
\end{EntryWithPhonetic}

\begin{EntryWithPhonetic}{文艺}{wen2yi4}{4,4}{⽂,⾋}[HSK 5]
  \definition{s.}{termo genérico para literatura e arte | performance (arte); refere"-se especificamente às artes performativas, como música e dança}
\end{EntryWithPhonetic}

\begin{EntryWithPhonetic}{文艺界}{wen2 yi4 jie4}{4,4,9}{⽂,⾋,⽥}
  \definition{s.}{círculos literários e artísticos; o mundo da literatura e da arte}
\end{EntryWithPhonetic}

\begin{EntryWithPhonetic}{文娱}{wen2yu2}{4,10}{⽂,⼥}[HSK 6]
  \definition{s.}{recreação cultural; entretenimento}
\end{EntryWithPhonetic}

\begin{EntryWithPhonetic}{文章}{wen2zhang1}{4,11}{⽂,⾳}[HSK 3]
  \definition[篇,段,页,系列]{s.}{ensaio; artigo; texto independente; também se refere a obras literárias em geral | significado oculto; significado implícito | trabalho; coisas que podem ser feitas}
\end{EntryWithPhonetic}

\begin{EntryWithPhonetic}{文字}{wen2zi4}{4,6}{⽂,⼦}[HSK 3]
  \definition[种,类,段,行,篇]{s.}{caracteres; caligrafia; escrita; símbolos escritos para registrar a linguagem | linguagem escrita; a forma escrita da língua}
\end{EntryWithPhonetic}

%%%%%%%%%% 纹 %%%%%%%%%%
\subsection*{纹}\addcontentsline{loh}{figure}{纹 \dpy{wen2}}

\begin{EntryWithPhonetic}{纹}{wen2}{7}{⽷}
  \definition[个]{s.}{linhas; veios; grãos; rugas na pele | padrão; desenho em tecido de seda; listras ou padrões em tecidos de seda; geralmente se refere a padrões lineares na superfície de um objeto}
\end{EntryWithPhonetic}

\begin{EntryWithPhonetic}{纹路}{wen2lu4}{7,13}{⽷,⾜}
  \definition{s.}{padrão de linhas | rugas | veias | veias (em mármore ou impressão digital) | grãos (em madeira, etc.)}
\end{EntryWithPhonetic}

%%%%%%%%%% 闻 %%%%%%%%%%
\subsection*{闻}\addcontentsline{loh}{figure}{闻 \dpy{wen2}}

\begin{EntryWithPhonetic}{闻}{wen2}{9}{⾨}[HSK 2]
  \definition*{s.}{Sobrenome: Wen}
  \definition{adj.}{bem conhecido; famoso}
  \definition{s.}{notícia; história | reputação | boato; rumor}
  \definition{v.}{cheirar | ouvir}
\end{EntryWithPhonetic}

\begin{EntryWithPhonetic}{闻名}{wen2ming2}{9,6}{⾨,⼝}[HSK 7-9]
  \definition{adj.}{famoso; renomado; conhecido}
  \definition{v.}{conhecer alguém por reputação; estar familiarizado com o nome de alguém; ouvir falar da reputação}
  \synonymref{驰名}{chi2ming2}
  \synonymref{出名}{chu1 ming2}
  \synonymref{有名}{you3 ming2}
  \synonymref{知名}{zhi1ming2}
  \synonymref{著名}{zhu4ming2}
  \antonymref{普通}{pu3tong1}
\end{EntryWithPhonetic}

%%%%%%%%%% 蚊 %%%%%%%%%%
\subsection*{蚊}\addcontentsline{loh}{figure}{蚊 \dpy{wen2}}

\begin{EntryWithPhonetic}{蚊}{wen2}{10}{⾍}
  \definition{s.}{mosquito; pernilongo}
\end{EntryWithPhonetic}

\begin{EntryWithPhonetic}{蚊香}{wen2xiang1}{10,9}{⾍,⾹}
  \definition{s.}{incenso ou espiral repelente de mosquitos}
\end{EntryWithPhonetic}

\begin{EntryWithPhonetic}{蚊帐}{wen2zhang4}{10,7}{⾍,⼱}[HSK 7-9]
  \definition[顶]{s.}{mosquiteiro; as redes mosquiteiras, penduradas acima e ao redor da cama para bloquear os mosquitos, vêm em formatos de guarda"-chuva e retangulares}
\end{EntryWithPhonetic}

\begin{EntryWithPhonetic}{蚊子}{wen2zi5}{10,3}{⾍,⼦}[HSK 7-9]
  \definition[只]{s.}{pernilongo; mosquito}
\end{EntryWithPhonetic}

%%%%%%%%%% 吻 %%%%%%%%%%
\subsection*{吻}\addcontentsline{loh}{figure}{吻 \dpy{wen3}}

\begin{EntryWithPhonetic}{吻}{wen3}{7}{⼝}[HSK 7-9]
  \definition{s.}{tom; voz | lábios | bocas de animais}
  \definition{v.}{beijar; tocar uma pessoa ou objeto com os lábios | tocar; estar perto de}[他轻轻地吻了她一下。===Ele a beijou delicadamente. | 几乎吻着地面。===Quase tocou o chão.]
\end{EntryWithPhonetic}

\begin{EntryWithPhonetic}{吻合}{wen3he2}{7,6}{⼝,⼝}[HSK 7-9]
  \definition{s.}{correspondente; consistente; idêntico; totalmente compatível}
  \definition{v.}{conectar por anastomose; Medicina: refere"-se à união de duas superfícies fraturadas de um órgão}
  \synonymref{符合}{fu2he2}
  \synonymref{适合}{shi4he2}
  \antonymref{出入}{chu1ru4}
  \antonymref{抵触}{di3chu4}
\end{EntryWithPhonetic}

%%%%%%%%%% 紊 %%%%%%%%%%
\subsection*{紊}\addcontentsline{loh}{figure}{紊 \dpy{wen3}}

\begin{EntryWithPhonetic}{紊}{wen3}{10}{⽷}
  \definition{adj.}{desordenado; confuso}
\end{EntryWithPhonetic}

\begin{EntryWithPhonetic}{紊乱}{wen3luan4}{10,7}{⽷,⼄}[HSK 7-9]
  \definition{adj.}{caótico; desordenado; confuso; sem ordem ou sequência}
  \synonymref{混乱}{hun4luan4}
  \antonymref{整齐}{zheng3qi2}
\end{EntryWithPhonetic}

%%%%%%%%%% 稳 %%%%%%%%%%
\subsection*{稳}\addcontentsline{loh}{figure}{稳 \dpy{wen3}}

\begin{EntryWithPhonetic}{稳}{wen3}{14}{⽲}[HSK 4]
  \definition{adj.}{constante; estável; firme | estável; estático; sedado | seguro; confiável; certo}
  \definition{adv.}{certamente; com certeza; seguramente; sem dúvida}
  \definition{v.}{estabilizar, manter estável; acalmar}
\end{EntryWithPhonetic}

\begin{EntryWithPhonetic}{稳定}{wen3ding4}{14,8}{⽲,⼧}[HSK 4]
  \definition{adj.}{estável; firme; descreve uma natureza, um estado, etc. relativamente fixo; não muda significativamente}
  \definition{v.}{manter estável; estabilizar}
\end{EntryWithPhonetic}

\begin{EntryWithPhonetic}{稳固}{wen3gu4}{14,8}{⽲,⼞}[HSK 7-9]
  \definition{adj.}{firme; constante; estável; estável e sólido}
  \definition{v.}{estabilizar; o estado de manter as coisas inalteradas}
  \synonymref{安稳}{an1wen3}
  \synonymref{巩固}{gong3gu4}
  \synonymref{坚固}{jian1gu4}
  \synonymref{坚韧}{jian1ren4}
  \synonymref{坚实}{jian1shi2}
  \synonymref{结识}{jie2shi2}
  \synonymref{牢固}{lao2gu4}
  \synonymref{平稳}{ping2wen3}
  \synonymref{稳定}{wen3ding4}
  \antonymref{动摇}{dong4yao2}
  \antonymref{摇晃}{yao2huang4}
\end{EntryWithPhonetic}

\begin{EntryWithPhonetic}{稳健}{wen3jian4}{14,10}{⽲,⼈}[HSK 7-9]
  \definition{adj.}{firme; constante; estável; descreve um modo de andar ou o desenvolvimento de algo como suave e poderoso | seguro; constante; confiável; descreve alguém que age com maturidade e ponderação; que permanece calmo e lida bem com as situações}
  \synonymref{保守}{bao3shou3}
  \synonymref{妥当}{tuo3dang4}
  \synonymref{稳妥}{wen3tuo3}
  \synonymref{稳重}{wen3zhong4}
\end{EntryWithPhonetic}

\begin{EntryWithPhonetic}{稳妥}{wen3tuo3}{14,7}{⽲,⼥}[HSK 7-9]
  \definition{adj.}{seguro; confiável; prudente; estável; constante e adequado}
  \synonymref{恰当}{qia4dang4}
  \synonymref{停当}{ting2dang5}
  \synonymref{妥当}{tuo3dang4}
  \synonymref{妥善}{tuo3shan4}
  \synonymref{稳健}{wen3jian4}
  \antonymref{冒险}{mao4/xian3}
\end{EntryWithPhonetic}

\begin{EntryWithPhonetic}{稳重}{wen3zhong4}{14,9}{⽲,⾥}[HSK 7-9]
  \definition{adj.}{calmo; estável; sereno (na fala ou no comportamento); descreve alguém que fala e age com maturidade; não é descuidado; permanece calmo sob pressão e lida bem com as situações, transmitindo tranquilidade aos outros}
  \synonymref{安宁}{an1ning2}
  \synonymref{沉着}{chen2zhuo2}
  \synonymref{从容}{cong2rong2}
  \synonymref{谨慎}{jin3shen4}
  \synonymref{耐心}{nai4xin1}
  \synonymref{慎重}{shen4zhong4}
  \synonymref{严肃}{yan2su4}
  \synonymref{自在}{zi4zai4}
  \antonymref{暴躁}{bao4zao4}
  \antonymref{浮躁}{fu2zao4}
  \antonymref{慌乱}{huang1luan4}
  \antonymref{急忙}{ji2mang2}
  \antonymref{鲁莽}{lu3mang3}
\end{EntryWithPhonetic}

%%%%%%%%%% 问 %%%%%%%%%%
\subsection*{问}\addcontentsline{loh}{figure}{问 \dpy{wen4}}

\begin{EntryWithPhonetic}{问}{wen4}{6}{⾨}[HSK 1]
  \definition*{s.}{Sobrenome: Wen}
  \definition{prep.}{de; introduzir o objeto da ação, equivalente a 向 e 跟}
  \definition{v.}{perguntar; indagar; fazer com que as pessoas respondam ou esclareçam coisas que não sabem ou não têm certeza | perguntar (ou indagar) sobre | examinar; interrogar | intervir; responsabilizar; investigar | cuidar; preocupar"-se; gerenciar; interferir}
  \seealsoref{跟}{gen1}
  \seealsoref{向}{xiang4}
\end{EntryWithPhonetic}

\begin{EntryWithPhonetic}{问安}{wen4'an1}{6,6}{⾨,⼧}
  \definition{s.}{saudações}
  \definition{v.}{dar cumprimentos a | prestar homenagem}
\end{EntryWithPhonetic}

\begin{EntryWithPhonetic}{问鼎}{wen4ding3}{6,12}{⾨,⿍}
  \definition{v.}{visar (o primeiro lugar, etc.) | aspirar ao trono}
\end{EntryWithPhonetic}

\begin{EntryWithPhonetic}{问候}{wen4hou4}{6,10}{⾨,⼈}[HSK 4]
  \definition{v.}{prestar homenagem; enviar uma saudação;  dar os respeitos (cumprimentos) a alguém}
\end{EntryWithPhonetic}

\begin{EntryWithPhonetic}{问卷}{wen4juan4}{6,8}{⾨,⼙}[HSK 7-9]
  \definition[份]{s.}{questionário; um questionário utilizado para realizar pesquisas ou solicitar opiniões, que lista diversas perguntas para as pessoas responderem}
\end{EntryWithPhonetic}

\begin{EntryWithPhonetic}{问路}{wen4 lu4}{6,13}{⾨,⾜}[HSK 2]
  \definition{v.}{perguntar o caminho; pedir direções}
\end{EntryWithPhonetic}

\begin{EntryWithPhonetic}{问世}{wen4shi4}{6,5}{⾨,⼀}[HSK 7-9]
  \definition{v.}{sair; ser publicado; (obras, invenções, novos produtos, etc.) ser apresentados ao mundo}
  \synonymref{诞生}{dan4sheng1}
  \synonymref{上市}{shang4 shi4}
  \synonymref{研制}{yan2zhi4}
  \antonymref{下线}{xia4xian4}
\end{EntryWithPhonetic}

\begin{EntryWithPhonetic}{问市}{wen4shi4}{6,5}{⾨,⼱}
  \definition{v.}{chegar ao mercado | bater o mercado | atingir o mercado}
\end{EntryWithPhonetic}

\begin{EntryWithPhonetic}{问题}{wen4ti2}{6,15}{⾨,⾴}[HSK 2]
  \definition{adj.}{desqualificado; indesejável; anormal, não atende aos requisitos}
  \definition[个,种,类,串]{s.}{pergunta; problema; perguntas a serem respondidas | problema; questão; contradições que precisam ser estudadas e resolvidas | problema; acidente; incidente | chave; ponto crucial; pontos importantes}
\end{EntryWithPhonetic}

%%%%%%%%%% 嗡 %%%%%%%%%%
\subsection*{嗡}\addcontentsline{loh}{figure}{嗡 \dpy{weng1}}

\begin{EntryWithPhonetic}{嗡}{weng1}{13}{⼝}
  \definition[出]{part.}{(onomatopéia) zumbido; zunido; zunzum; descreve o som do bater de asas de um inseto}
\end{EntryWithPhonetic}

\begin{EntryWithPhonetic}{嗡嗡}{weng1weng1}{13,13}{⼝,⼝}
  \definition{s.}{zumbido}
  \definition{v.}{zumbir}
\end{EntryWithPhonetic}

%%%%%%%%%% 蕹 %%%%%%%%%%
\subsection*{蕹}\addcontentsline{loh}{figure}{蕹 \dpy{weng4}}

\begin{EntryWithPhonetic}{蕹}{weng4}{16}{⾋}
  \definition{s.}{espinafre-d’água ou \emph{ong choy}, usado como vegetal no sul da China e no sudeste da Ásia}
\end{EntryWithPhonetic}

\begin{EntryWithPhonetic}{蕹菜}{weng4cai4}{16,11}{⾋,⾋}
  \definition{s.}{espinafre aquático | \emph{ong choy} | repolho do pântano | convolvulus aquático | glória-da-manhã aquática}
  \seealsoref{空心菜}{kong1xin1cai4}
\end{EntryWithPhonetic}

%%%%%%%%%% 窝 %%%%%%%%%%
\subsection*{窝}\addcontentsline{loh}{figure}{窝 \dpy{wo1}}

\begin{EntryWithPhonetic}{窝}{wo1}{12}{⽳}[HSK 7-9]
  \definition{clas.}{ninhada; cria; usado para animais}
  \definition[个,只,块]{s.}{ninho; habitat de aves, animais e insetos | toca; covil; ninho; uma metáfora para um lugar onde pessoas más se reúnem | lugar; metaforicamente, a posição ocupada por um corpo humano ou objeto | poço; cavidade; área rebaixada}
  \definition{v.}{abrigar; dar abrigo | reprimir; suprimir; reter (voltar); emoções reprimidas não podem ser expressas ou manifestadas | dobrar; flexionar; torcer}
\end{EntryWithPhonetic}

%%%%%%%%%% 我 %%%%%%%%%%
\subsection*{我}\addcontentsline{loh}{figure}{我 \dpy{wo3}}

\begin{EntryWithPhonetic}{我}{wo3}{7}{⼽}[HSK 1]
  \definition{pron.}{eu; mim | um; qualquer um; usado para contrastar 他 e 我; refere"-se a muitas pessoas em geral}
  \seealsoref{他}{ta1}
\end{EntryWithPhonetic}

\begin{EntryWithPhonetic}{我的}{wo3 de5}{7,8}{⼽,⽩}
  \definition{pron.}{meu, meus}
\end{EntryWithPhonetic}

\begin{EntryWithPhonetic}{我们}{wo3men5}{7,5}{⼽,⼈}[HSK 1]
  \definition{pron.}{nós; nos}
\end{EntryWithPhonetic}

\begin{EntryWithPhonetic}{我们的}{wo3men5 de5}{7,5,8}{⼽,⼈,⽩}
  \definition{pron.}{nosso, nossos}
\end{EntryWithPhonetic}

\begin{EntryWithPhonetic}{我去}{wo3qu4}{7,5}{⼽,⼛}
  \definition{interj.}{Gíria: ``O que\dots!!'' | ``Oh meu Deus!'' | ``Isso é insano!''}
\end{EntryWithPhonetic}

%%%%%%%%%% 卧 %%%%%%%%%%
\subsection*{卧}\addcontentsline{loh}{figure}{卧 \dpy{wo4}}

\begin{EntryWithPhonetic}{卧}{wo4}{8}{⾂}[HSK 7-9]
  \definition{adj.}{para dormir}
  \definition{s.}{vagão-leito (ou carruagem); leito | beliche | quarto | Dialeto: pochê (ovos)}
  \definition{v.}{deitar | Dialeto: deitar um bebê | (animais ou pássaros) agachar"-se; sentar"-se; empoleirar"-se | Figurativo: viver em reclusão}
  \antonymref{坐}{zuo4}
\end{EntryWithPhonetic}

\begin{EntryWithPhonetic}{卧病}{wo4bing4}{8,10}{⾂,⽧}
  \definition{s.}{acamado | doente na cama}
\end{EntryWithPhonetic}

\begin{EntryWithPhonetic}{卧舱}{wo4cang1}{8,10}{⾂,⾈}
  \definition{s.}{cabine de dormir em um barco ou trem}
\end{EntryWithPhonetic}

\begin{EntryWithPhonetic}{卧车}{wo4che1}{8,4}{⾂,⾞}
  \definition{s.}{um carro-leito | vagão-leito}
\end{EntryWithPhonetic}

\begin{EntryWithPhonetic}{卧床}{wo4chuang2}{8,7}{⾂,⼴}
  \definition{adj.}{acamado}
  \definition{s.}{cama}
  \definition{v.}{deitar na cama}
\end{EntryWithPhonetic}

\begin{EntryWithPhonetic}{卧倒}{wo4dao3}{8,10}{⾂,⼈}
  \definition{v.}{cair no chão | deitar-se}
\end{EntryWithPhonetic}

\begin{EntryWithPhonetic}{卧铺}{wo4pu4}{8,12}{⾂,⾦}[HSK 6]
  \definition[个,排]{s.}{beliche para dormir; um beliche em um trem ou ônibus de longa distância}
\end{EntryWithPhonetic}

\begin{EntryWithPhonetic}{卧式}{wo4shi4}{8,6}{⾂,⼷}
  \definition{adj.}{horizontal}
\end{EntryWithPhonetic}

\begin{EntryWithPhonetic}{卧室}{wo4shi4}{8,9}{⾂,⼧}[HSK 5]
  \definition[间,个]{s.}{quarto de dormir; quarto de uma casa usado para dormir}
\end{EntryWithPhonetic}

\begin{EntryWithPhonetic}{卧榻}{wo4ta4}{8,14}{⾂,⽊}
  \definition{s.}{um sofá | uma cama estreita}
\end{EntryWithPhonetic}

\begin{EntryWithPhonetic}{卧推}{wo4tui1}{8,11}{⾂,⼿}
  \definition{s.}{supino}
\end{EntryWithPhonetic}

%%%%%%%%%% 握 %%%%%%%%%%
\subsection*{握}\addcontentsline{loh}{figure}{握 \dpy{wo4}}

\begin{EntryWithPhonetic}{握}{wo4}{12}{⼿}[HSK 5]
  \definition{v.}{segurar; agarrar | agarrar; segurar; empunhar; controlar | pegar pela mão}
\end{EntryWithPhonetic}

\begin{EntryWithPhonetic}{握手}{wo4/shou3}{12,4}{⼿,⼿}[HSK 3]
  \definition{v.+compl.}{apertar as mãos; dar um aperto de mão; estender a mão e apertar a mão do outro é uma forma de saudação ao se encontrar ou se despedir, e também é usado para expressar felicitações ou condolências}
\end{EntryWithPhonetic}

%%%%%%%%%% 斡 %%%%%%%%%%
\subsection*{斡}\addcontentsline{loh}{figure}{斡 \dpy{wo4}}

\begin{EntryWithPhonetic}{斡}{wo4}{14}{⽃}
  \definition{v.}{virar"-se}
\end{EntryWithPhonetic}

\begin{EntryWithPhonetic}{斡旋}{wo4xuan2}{14,11}{⽃,⽅}
  \definition{v.}{mediar (um conflito, etc.)}
\end{EntryWithPhonetic}

%%%%%%%%%% 乌 %%%%%%%%%%
\subsection*{乌}\addcontentsline{loh}{figure}{乌 \dpy{wu1}}

\begin{EntryWithPhonetic}{乌}{wu1}{4}{⼃}
  \definition*{s.}{Sobrenome: Wu}
  \definition{adj.}{preto; escuro}
  \definition{pron.}{como; o que é}
  \definition{s.}{corvo; gralha}
  \seeref{wu4}
  \synonymref{黑}{hei1}
\end{EntryWithPhonetic}

\begin{EntryWithPhonetic}{乌龟}{wu1gui1}{4,7}{⼃,⿔}
  \definition[头,只]{s.}{tartaruga}
\end{EntryWithPhonetic}

\begin{EntryWithPhonetic}{乌克兰}{wu1ke4lan2}{4,7,5}{⼃,⼗,⼋}
  \definition*{s.}{Ucrânia}
\end{EntryWithPhonetic}

\begin{EntryWithPhonetic}{乌云}{wu1yun2}{4,4}{⼃,⼆}[HSK 6]
  \definition[片]{s.}{nuvens negras; nuvens escuras | cabelo preto (de mulher); uma metáfora para o cabelo preto brilhante de uma mulher}
  \antonymref{彩虹}{cai3hong2}
\end{EntryWithPhonetic}

%%%%%%%%%% 污 %%%%%%%%%%
\subsection*{污}\addcontentsline{loh}{figure}{污 \dpy{wu1}}

\begin{EntryWithPhonetic}{污}{wu1}{6}{⽔}
  \definition{adj.}{sujo; imundo; imundo | corrupto}
  \definition{s.}{sujeira; imundície | esgoto; água suja; coisas sujas}
  \definition{v.}{contaminar; sujar | manchar}
\end{EntryWithPhonetic}

\begin{EntryWithPhonetic}{污秽}{wu1hui4}{6,11}{⽔,⽲}[HSK 7-9]
  \definition{adj.}{Literário: imundo; repugnante}
  \definition{s.}{sujeira; imundície}
  \antonymref{干净}{gan1jing4}
  \antonymref{洁净}{jie2jing4}
  \antonymref{清洁}{qing1jie2}
\end{EntryWithPhonetic}

\begin{EntryWithPhonetic}{污染}{wu1ran3}{6,9}{⽔,⽊}[HSK 5]
  \definition{v.}{poluir; contaminar com substâncias nocivas e prejudiciais; refere"-se especificamente à destruição do ambiente natural causada por substâncias nocivas, tais como gases, líquidos e resíduos emitidos por indústrias, minas, veículos, etc. | contaminar; metáfora de que pensamentos prejudiciais causam efeitos negativos nas pessoas}
\end{EntryWithPhonetic}

\begin{EntryWithPhonetic}{污染区}{wu1ran3qu1}{6,9,4}{⽔,⽊,⼖}
  \definition{s.}{área contaminada}
\end{EntryWithPhonetic}

\begin{EntryWithPhonetic}{污染物}{wu1ran3wu4}{6,9,8}{⽔,⽊,⽜}
  \definition{s.}{poluente}
  \seealsoref{污染物质}{wu1ran3 wu4zhi4}
\end{EntryWithPhonetic}

\begin{EntryWithPhonetic}{污染物质}{wu1ran3 wu4zhi4}{6,9,8,8}{⽔,⽊,⽜,⾙}
  \definition{s.}{poluente}
  \seealsoref{污染物}{wu1ran3wu4}
\end{EntryWithPhonetic}

\begin{EntryWithPhonetic}{污水}{wu1shui3}{6,4}{⽔,⽔}[HSK 5]
  \definition[桶,滩]{s.}{água suja (ou poluída, residual); esgoto; lodo | efluente; drenagem; água suja; água poluída; água residual}
\end{EntryWithPhonetic}

%%%%%%%%%% 呜 %%%%%%%%%%
\subsection*{呜}\addcontentsline{loh}{figure}{呜 \dpy{wu1}}

\begin{EntryWithPhonetic}{呜}{wu1}{7}{⼝}
  \definition{s.}{Onomatopéia: buzina; pio; ``zoom''}
\end{EntryWithPhonetic}

\begin{EntryWithPhonetic}{呜咽}{wu1ye4}{7,9}{⼝,⼝}[HSK 7-9]
  \definition{v.}{soluçar; choramingar | (água, vento, instrumentos de corda, etc.) chorar; lamentar; prantear |  emitir sons melancólicos}
  \synonymref{哭泣}{ku1qi4}
\end{EntryWithPhonetic}

%%%%%%%%%% 巫 %%%%%%%%%%
\subsection*{巫}\addcontentsline{loh}{figure}{巫 \dpy{wu1}}

\begin{EntryWithPhonetic}{巫}{wu1}{7}{⼯}
  \definition*{s.}{Sobrenome: Wu}
  \definition[位,名,个,些]{s.}{xamã; bruxa; mago | bruxas, feiticeiros}
\end{EntryWithPhonetic}

\begin{EntryWithPhonetic}{巫婆}{wu1po2}{7,11}{⼯,⼥}[HSK 7-9]
  \definition[个,位,名]{s.}{bruxa; feiticeira; xamã | xamã feminina}
\end{EntryWithPhonetic}

%%%%%%%%%% 屋 %%%%%%%%%%
\subsection*{屋}\addcontentsline{loh}{figure}{屋 \dpy{wu1}}

\begin{EntryWithPhonetic}{屋}{wu1}{9}{⼫}[HSK 5]
  \definition[间,座]{s.}{casa | quarto}
\end{EntryWithPhonetic}

\begin{EntryWithPhonetic}{屋顶}{wu1ding3}{9,8}{⼫,⾴}[HSK 7-9]
  \definition[个]{s.}{teto; telhado; cobertura; telhado de uma casa; o topo da casa}
  \antonymref{地板}{di4ban3}
\end{EntryWithPhonetic}

\begin{EntryWithPhonetic}{屋子}{wu1zi5}{9,3}{⼫,⼦}[HSK 3]
  \definition[间,座,栋]{s.}{quarto; sala}
\end{EntryWithPhonetic}

%%%%%%%%%% 恶 %%%%%%%%%%
\subsection*{恶}\addcontentsline{loh}{figure}{恶 \dpy{wu1}}

\begin{EntryWithPhonetic}{恶}{wu1}{10}{⼼}
  \definition{interj.}{``Droga!''; ``Ah não!''; expressa surpresa}
  \definition{pron.}{como?; por que?; refere"-se a um lugar ou coisa; expressa uma pergunta retórica; equivalente a 何 ou 怎么}
  \seeref{e3}
  \seeref{e4}
  \seeref{wu4}
  \seealsoref{何}{he2}
  \seealsoref{怎么}{zen3me5}
\end{EntryWithPhonetic}

%%%%%%%%%% 无 %%%%%%%%%%
\subsection*{无}\addcontentsline{loh}{figure}{无 \dpy{wu2}}

\begin{EntryWithPhonetic}{无}{wu2}{4}{⽆}[HSK 4][Kangxi 71]
  \definition{adv.}{não (em posição a 有); não ter algo; não há\dots}
  \definition{conj.}{independentemente de; não importa se, o que, etc.}
  \definition{v.}{não ter; estar sem; não existir}
  \seealsoref{有}{you3}
\end{EntryWithPhonetic}

\begin{EntryWithPhonetic}{无比}{wu2bi3}{4,4}{⽆,⽐}[HSK 7-9]
  \definition{adj.}{inigualável; incomparável; sem paralelo; nada se compara (frequentemente usado por razões positivas)}
  \synonymref{非常}{fei1chang2}
  \synonymref{格外}{ge2wai4}
  \synonymref{极其}{ji2qi2}
  \synonymref{十分}{shi2fen1}
  \antonymref{一般}{yi4ban1}
\end{EntryWithPhonetic}

\begin{EntryWithPhonetic}{无边}{wu2bian1}{4,5}{⽆,⾡}[HSK 6]
  \definition{adj.}{ilimitado; vasto; sem limites; sem abas; sem bordas}
\end{EntryWithPhonetic}

\begin{EntryWithPhonetic}{无不}{wu2bu4}{4,4}{⽆,⼀}[HSK 7-9]
  \definition{adv.}{invariavelmente; todos sem exceção; nenhum (nem tanto);  significa que não há exceções}
  \synonymref{全都}{quan2dou1}
\end{EntryWithPhonetic}

\begin{EntryWithPhonetic}{无偿}{wu2chang2}{4,11}{⽆,⼈}[HSK 7-9]
  \definition{adj.}{grátis; gratuito; sem custo; não pago; que não exige pagamento da outra parte}
  \synonymref{免费}{mian3/fei4}
  \synonymref{义务}{yi4wu4}
\end{EntryWithPhonetic}

\begin{EntryWithPhonetic}{无敌}{wu2di2}{4,10}{⽆,⾆}[HSK 7-9]
  \definition{adj.}{inigualável; invencível; imbatível; sem rivais}
\end{EntryWithPhonetic}

\begin{EntryWithPhonetic}{无恶不作}{wu2'e4-bu2zuo4}{4,10,4,7}{⽆,⼼,⼀,⼈}[HSK 7-9]
  \definition{expr.}{``Não se furtar a nenhum crime.'' (expressão idiomática); cometer qualquer delito imaginável; essa expressão descreve alguém extremamente perverso e capaz de praticar qualquer tipo de maldade; não hesitar em praticar o mal; não se deter diante de nenhum mal; cometer todo tipo de crime}
\end{EntryWithPhonetic}

\begin{EntryWithPhonetic}{无法}{wu2fa3}{4,8}{⽆,⽔}[HSK 4]
  \definition{adj.}{incapaz; incapacitado; não tem jeito}
\end{EntryWithPhonetic}

\begin{EntryWithPhonetic}{无非}{wu2fei1}{4,8}{⽆,⾮}[HSK 7-9]
  \definition{adv.}{somente; simplesmente; nada além de; não mais que; nada mais do que; significando tudo dentro de um certo intervalo}
  \synonymref{并非}{bing4fei1}
\end{EntryWithPhonetic}

\begin{EntryWithPhonetic}{无辜}{wu2gu1}{4,12}{⽆,⾟}[HSK 7-9]
  \definition{adj.}{não culpado; inocente}
  \definition[个,名,位]{s.}{pessoa inocente}
  \synonymref{可怜}{ke3lian2}
  \synonymref{委屈}{wei3qu5}
  \antonymref{罪恶}{zui4'e4}
\end{EntryWithPhonetic}

\begin{EntryWithPhonetic}{无骨}{wu2 gu3}{4,9}{⽆,⾻}
  \definition{adj.}{desossado}
\end{EntryWithPhonetic}

\begin{EntryWithPhonetic}{无故}{wu2gu4}{4,9}{⽆,⽁}[HSK 7-9]
  \definition{adv.}{sem motivo; por nenhuma razão}
  \synonymref{无理}{wu2li3}
\end{EntryWithPhonetic}

\begin{EntryWithPhonetic}{无关}{wu2guan1}{4,6}{⽆,⼋}[HSK 6]
  \definition{v.}{não ter nada a ver com; nada a fazer | não envolver; ser irrelevante; não ter efeito sobre}
\end{EntryWithPhonetic}

\begin{EntryWithPhonetic}{无关紧要}{wu2guan1-jin3yao4}{4,6,10,9}{⽆,⼋,⽷,⾑}[HSK 7-9]
  \definition{expr.}{sem importância; irrelevante; indiferente; fragilidade; insignificante}
  \synonymref{无足轻重}{wu2zu2-qing1zhong4}
\end{EntryWithPhonetic}

\begin{EntryWithPhonetic}{无话可说}{wu2hua4-ke3shuo1}{4,8,5,9}{⽆,⾔,⼝,⾔}[HSK 7-9]
  \definition{expr.}{``Não ter nada a dizer.''; não havia nada a dizer, nenhuma opinião ou razão a apresentar}
  \synonymref{无可奉告}{wu2ke3feng4gao4}
\end{EntryWithPhonetic}

\begin{EntryWithPhonetic}{无济于事}{wu2ji4yu2shi4}{4,9,3,8}{⽆,⽔,⼆,⼅}[HSK 7-9]
  \definition{expr.}{não adiantou nada; não ajuda em nada; inútil; sem efeito; sem utilidade; sem sucesso}
  \antonymref{潜移默化}{qian2yi2-mo4hua4}
\end{EntryWithPhonetic}

\begin{EntryWithPhonetic}{无家可归}{wu2jia1-ke3gui1}{4,10,5,5}{⽆,⼧,⼝,⼹}[HSK 7-9]
  \definition{expr.}{sem"-teto; morador de rua; significa não ter um lar para onde voltar, referindo"-se a estar deslocado e sem"-teto; vagar sem rumo}
\end{EntryWithPhonetic}

\begin{EntryWithPhonetic}{无精打采}{wu2jing1-da3cai3}{4,14,5,8}{⽆,⽶,⼿,⾤}[HSK 7-9]
  \definition{expr.}{``Desanimado e abatido.''; apático; deprimido; descreve alguém que está apático, sem energia, infeliz ou desanimado; indisposto; desleixado}
  \synonymref{垂头丧气}{chui2tou2-sang4qi4}
  \antonymref{容光焕发}{rong2guang1-huan4fa1}
\end{EntryWithPhonetic}

\begin{EntryWithPhonetic}{无可奉告}{wu2ke3feng4gao4}{4,5,8,7}{⽆,⼝,⼤,⼝}[HSK 7-9]
  \definition{expr.}{``Sem comentários.''}
  \synonymref{无话可说}{wu2hua4-ke3shuo1}
\end{EntryWithPhonetic}

\begin{EntryWithPhonetic}{无可厚非}{wu2ke3hou4fei1}{4,5,9,8}{⽆,⼝,⼚,⾮}[HSK 7-9]
  \definition{expr.}{desculpável; compreensível; inegavelmente; é importante não ser excessivamente crítico; embora possam existir falhas, elas são perdoáveis e compreensíveis}
\end{EntryWithPhonetic}

\begin{EntryWithPhonetic}{无可奈何}{wu2ke3nai4he2}{4,5,8,7}{⽆,⼝,⼤,⼈}[HSK 7-9]
  \definition{expr.}{desamparado; não ter saída; estar completamente impotente; não ter alternativa; não há a mínima possibilidade; não há a menor ideia; não há absolutamente nenhuma possibilidade}
\end{EntryWithPhonetic}

\begin{EntryWithPhonetic}{无理}{wu2li3}{4,11}{⽆,⽟}[HSK 7-9]
  \definition{adj.}{irracional; injustificável}
  \definition{s.}{Matemática: irracional (número)}
  \synonymref{荒谬}{huang1miu4}
  \synonymref{畸形}{ji1xing2}
  \synonymref{无故}{wu2gu4}
  \antonymref{合理}{he2li3}
\end{EntryWithPhonetic}

\begin{EntryWithPhonetic}{无力}{wu2li4}{4,2}{⽆,⼒}[HSK 7-9]
  \definition{v.}{sentir"-se fraco; não ter força | ser incapaz; estar impossibilitado; não ter poder para}
  \seealsoref{无法}{wu2fa3}
  \synonymref{没劲}{mei2jin4}
  \synonymref{疲惫}{pi2bei4}
  \synonymref{疲劳}{pi2lao2}
  \antonymref{强劲}{qiang2jing4}
  \antonymref{有力}{you3li4}
\end{EntryWithPhonetic}

\begin{EntryWithPhonetic}{无聊}{wu2liao2}{4,11}{⽆,⽿}[HSK 4]
  \definition{adj.}{entediado; aborrecido; sentir"-se desinteressado porque não há nada para fazer | tolo; bobo; sem sentido; descreve palavras ou coisas ditas ou feitas como sem sentido e irritantes; descreve pessoas ou coisas como sem sentido e pouco atraentes}
\end{EntryWithPhonetic}

\begin{EntryWithPhonetic}{无论}{wu2lun4}{4,6}{⽆,⾔}[HSK 4]
  \definition{conj.}{não importa o quê; não importa como; independentemente de; indica que as condições são diferentes, mas resultado é o mesmo}
  \seealsoref{无论……也……}{wu2lun4 ye3}
\end{EntryWithPhonetic}

\begin{EntryWithPhonetic}{无论如何}{wu2lun4-ru2he2}{4,6,6,7}{⽆,⾔,⼥,⼈}[HSK 7-9]
  \definition{adv.}{em qualquer caso; de qualquer forma; como puder; em todo caso; pelo menos; por todos os meios possíveis; independentemente das circunstâncias, o resultado será o mesmo}
  \seealsoref{不管}{bu4guan3}
  \seealsoref{无论}{wu2lun4}
  \synonymref{不论}{bu2lun4}
\end{EntryWithPhonetic}

\begin{EntryWithPhonetic}{无论……也……}{wu2lun4 ye3}{4,6,3}{⽆,⾔,⼄}
  \definition{conj.}{não apenas\dots, (o que, quem, como, etc.), \dots}
\end{EntryWithPhonetic}

\begin{EntryWithPhonetic}{无奈}{wu2nai4}{4,8}{⽆,⼤}[HSK 5]
  \definition{conj.}{mas (infelizmente); no entanto}
  \definition{v.}{não poder evitar; não ter alternativa; não ter escolha; não haver nada a fazer}
\end{EntryWithPhonetic}

\begin{EntryWithPhonetic}{无能}{wu2neng2}{4,10}{⽆,⾁}[HSK 7-9]
  \definition{adj.}{incompetente; incapaz; incapaz de fazer qualquer coisa}
  \antonymref{才能}{cai2neng2}
  \antonymref{技能}{ji4neng2}
  \antonymref{能干}{neng2gan4}
  \antonymref{作用}{zuo4yong4}
\end{EntryWithPhonetic}

\begin{EntryWithPhonetic}{无能为力}{wu2neng2wei2li4}{4,10,4,2}{⽆,⾁,⼂,⼒}[HSK 7-9]
  \definition{expr.}{impotente; indefeso; incapaz de agir; incapaz de usar a força; com falta de força ou com força fraca}
  \synonymref{力不从心}{li4bu4cong2xin1}
  \synonymref{无可奈何}{wu2ke3nai4he2}
  \antonymref{力所能及}{li4suo3neng2ji2}
\end{EntryWithPhonetic}

\begin{EntryWithPhonetic}{无情}{wu2qing2}{4,11}{⽆,⼼}[HSK 7-9]
  \definition{adj.}{insensível; desalmado; sem emoção | impiedoso; intransigente; descreve uma atitude resoluta que não se preocupa em demonstrar respeito pelos outros | inexorável; descreve como as leis e a realidade não podem mudar de acordo com os desejos individuais}
  \synonymref{冷酷}{leng3ku4}
  \antonymref{深情}{shen1qing2}
\end{EntryWithPhonetic}

\begin{EntryWithPhonetic}{无情无义}{wu2qing2-wu2yi4}{4,11,4,3}{⽆,⼼,⽆,⼂}[HSK 7-9]
  \definition{expr.}{``Completamente desprovido de qualquer sentimento ou senso de justiça.''; frio e implacável; sem coração e ingrato; sem coração e sem fé}
  \synonymref{冷酷无情}{leng3ku4-wu2qing2}
\end{EntryWithPhonetic}

\begin{EntryWithPhonetic}{无穷}{wu2qiong2}{4,7}{⽆,⽳}[HSK 7-9]
  \definition{adj.}{infinito; sem fim; ilimitado; inesgotável; sem limites}
  \seealsoref{无数}{wu2shu4}
  \synonymref{无限}{wu2xian4}
  \antonymref{有限}{you3xian4}
\end{EntryWithPhonetic}

\begin{EntryWithPhonetic}{无人}{wu2ren2}{4,2}{⽆,⼈}
  \definition{adj.}{não tripulado | desabitado}
\end{EntryWithPhonetic}

\begin{EntryWithPhonetic}{无人机}{wu2ren2ji1}{4,2,6}{⽆,⼈,⽊}
  \definition{s.}{\emph{drone} | veículo aéreo não tripulado}
\end{EntryWithPhonetic}

\begin{EntryWithPhonetic}{无日}{wu2ri4}{4,4}{⽆,⽇}
  \definition{adv.}{Literário: o tempo todo; nenhum dia sequer | Literário: em breve; antes de muito tempo}
\end{EntryWithPhonetic}

\begin{EntryWithPhonetic}{无视}{wu2shi4}{4,8}{⽆,⾒}
  \definition{v.}{ignorar | desconsiderar}
\end{EntryWithPhonetic}

\begin{EntryWithPhonetic}{无数}{wu2shu4}{4,13}{⽆,⽁}[HSK 4]
  \definition{adj.}{incontável; inumerável | inseguro; incerto; não conhecer a história ou os detalhes internos; não ter certeza}
\end{EntryWithPhonetic}

\begin{EntryWithPhonetic}{无私}{wu2si1}{4,7}{⽆,⽲}[HSK 7-9]
  \definition{adj.}{altruísta; generoso; desinteresseiro; sem egoísmo}
\end{EntryWithPhonetic}

\begin{EntryWithPhonetic}{无所事事}{wu2suo3shi4shi4}{4,8,8,8}{⽆,⼾,⼅,⼅}[HSK 7-9]
  \definition{expr.}{não fazer nada; passar o tempo ocioso}
  \synonymref{无所作为}{wu2suo3zuo4wei2}
  \antonymref{废寝忘食}{fei4qin3-wang4shi2}
\end{EntryWithPhonetic}

\begin{EntryWithPhonetic}{无所谓}{wu2suo3wei4}{4,8,11}{⽆,⼾,⾔}[HSK 4]
  \definition{v.}{não pode ser designado como; não merece o nome de; ser incapaz de dizer ou contar | não ter importância; ser indiferente}
\end{EntryWithPhonetic}

\begin{EntryWithPhonetic}{无所作为}{wu2suo3zuo4wei2}{4,8,7,4}{⽆,⼾,⼈,⼂}[HSK 7-9]
  \definition{expr.}{``Sem tentar nada e sem conseguir nada.''; não tentar nada e não realizar nada; passivo; sem qualquer iniciativa ou motivação; irresponsável}
  \synonymref{无所事事}{wu2suo3shi4shi4}
  \antonymref{大有可为}{da4you3-ke3wei2}
  \antonymref{发愤图强}{fa1fen4-tu2qiang2}
  \antonymref{发奋图强}{fa1fen4-tu2qiang2}
\end{EntryWithPhonetic}

\begin{EntryWithPhonetic}{无条件}{wu2tiao2jian4}{4,7,6}{⽆,⽊,⼈}[HSK 7-9]
  \definition{adj.}{incondicional; sem pré-condições; irrestrito; nenhuma condição foi estabelecida; nenhuma condição foi proposta}
\end{EntryWithPhonetic}

\begin{EntryWithPhonetic}{无微不至}{wu2wei1-bu2zhi4}{4,13,4,6}{⽆,⼻,⼀,⾄}[HSK 7-9]
  \definition{expr.}{``De todas as maneiras possíveis.''; meticuloso; em todos os sentidos possíveis; com grande cuidado; significa ser muito atencioso e cuidadoso ao lidar com as pessoas}
\end{EntryWithPhonetic}

\begin{EntryWithPhonetic}{无线}{wu2xian4}{4,8}{⽆,⽷}[HSK 7-9]
  \definition{adj.}{sem fio; sem cabos}
\end{EntryWithPhonetic}

\begin{EntryWithPhonetic}{无线电}{wu2xian4dian4}{4,8,5}{⽆,⽷,⽥}[HSK 7-9]
  \definition[台]{s.}{rádio | sem fio; comunicação sem fio; comunicação por rádio; um tipo de comunicação onde o sinal não é transmitido por fios, mas sim pelo ar na forma de ondas eletromagnéticas}
  \synonymref{收音机}{shou1yin1ji1}
\end{EntryWithPhonetic}

\begin{EntryWithPhonetic}{无限}{wu2xian4}{4,8}{⽆,⾩}[HSK 4]
  \definition{adj.}{infinito; ilimitado; sem limites; sem fim à vista}
\end{EntryWithPhonetic}

\begin{EntryWithPhonetic}{无效}{wu2xiao4}{4,10}{⽆,⽁}[HSK 6]
  \definition{adj.}{sem efeito; de (ou sem) utilidade; em vão; (uma certa abordagem) é inútil e não pode resolver o problema ou atingir o objetivo | inválido; nulo e sem efeito; (um determinado comportamento) não é reconhecido por lei e os direitos relevantes não serão protegidos}
\end{EntryWithPhonetic}

\begin{EntryWithPhonetic}{无形}{wu2xing2}{4,7}{⽆,⼺}[HSK 7-9]
  \definition{adj.}{invisível; intangível; aquilo que tem função semelhante, mas não possui a forma ou o nome de determinada coisa; aquilo que não pode ser percebido pelos sentidos}
  \definition{adv.}{sutilmente; inconscientemente; imperceptivelmente}
\end{EntryWithPhonetic}

\begin{EntryWithPhonetic}{无形中}{wu2xing2zhong1}{4,7,4}{⽆,⼺,⼁}[HSK 7-9]
  \definition{adv.}{invisivelmente; imperceptivelmente; virtualmente; inconscientemente; numa situação em que algo tem substância, mas não tem nome, também se pode dizer que é intangível}
\end{EntryWithPhonetic}

\begin{EntryWithPhonetic}{无须}{wu2xu1}{4,9}{⽆,⾴}[HSK 7-9]
  \definition{adv.}{desnecessariamente; sem necessidade}
  \synonymref{不必}{bu2bi4}
  \synonymref{不用}{bu2yong4}
  \antonymref{必须}{bi4xu1}
\end{EntryWithPhonetic}

\begin{EntryWithPhonetic}{无氧}{wu2yang3}{4,10}{⽆,⽓}
  \definition{adj.}{anaeróbico}
\end{EntryWithPhonetic}

\begin{EntryWithPhonetic}{无疑}{wu2yi2}{4,14}{⽆,⽦}[HSK 5]
  \definition{adv.}{indubitavelmente; sem dúvida; sem sombra de dúvida}
\end{EntryWithPhonetic}

\begin{EntryWithPhonetic}{无意}{wu2yi4}{4,13}{⽆,⼼}[HSK 7-9]
  \definition{adv.}{involuntariamente; por acaso; conscientemente; não intencionalmente}
  \definition{v.}{não ter inclinação para; não ter intenção (de fazer algo)}
  \antonymref{存心}{cun2xin1}
  \antonymref{故意}{gu4yi4}
\end{EntryWithPhonetic}

\begin{EntryWithPhonetic}{无忧无虑}{wu2you1-wu2lv4}{4,7,4,10}{⽆,⼼,⽆,⾌}[HSK 7-9]
  \definition{expr.}{despreocupado; sem ansiedade; sem preocupações, a pessoa vive em estado de contentamento}
\end{EntryWithPhonetic}

\begin{EntryWithPhonetic}{无缘}{wu2yuan2}{4,12}{⽆,⽷}[HSK 7-9]
  \definition{v.}{não ter tido sorte (para fazer algo); sem ter sorte ou tribulação que atraem as pessoas}
\end{EntryWithPhonetic}

\begin{EntryWithPhonetic}{无知}{wu2zhi1}{4,8}{⽆,⽮}[HSK 7-9]
  \definition{adj.}{ignorante; falta de conhecimento; ignorância da razão}
  \antonymref{经验}{jing1yan4}
  \antonymref{知识}{zhi1shi5}
\end{EntryWithPhonetic}

\begin{EntryWithPhonetic}{无足轻重}{wu2zu2-qing1zhong4}{4,7,9,9}{⽆,⾜,⾞,⾥}[HSK 7-9]
  \definition{expr.}{insignificante; até mesmo coisas sem importância são descritas como insignificantes}
  \synonymref{无关紧要}{wu2guan1-jin3yao4}
\end{EntryWithPhonetic}

%%%%%%%%%% 吾 %%%%%%%%%%
\subsection*{吾}\addcontentsline{loh}{figure}{吾 \dpy{wu2}}

\begin{EntryWithPhonetic}{吾}{wu2}{7}{⼝}
  \definition*{s.}{Sobrenome: Wu}
  \definition{pron.}{eu; nós}
\end{EntryWithPhonetic}

%%%%%%%%%% 捂 %%%%%%%%%%
\subsection*{捂}\addcontentsline{loh}{figure}{捂 \dpy{wu2}}

\begin{EntryWithPhonetic}{捂}{wu2}{10}{⼿}
  \definition{v.}{encobrir; esconder; evitar; falar de forma vaga ou evasiva: ignorar}
  \seeref{wu3}
  \antonymref{揭}{jie1}
\end{EntryWithPhonetic}

%%%%%%%%%% 五 %%%%%%%%%%
\subsection*{五}\addcontentsline{loh}{figure}{五 \dpy{wu3}}

\begin{EntryWithPhonetic}{五}{wu3}{4}{⼆}[HSK 1]
  \definition*{s.}{Sobrenome: Wu}
  \definition{num.}{cinco; 5}
  \definition{s.}{uma nota da escala em Gongchepu (工尺谱), correspondente a 6 na notação musical numerada}
  \seealsoref{工尺谱}{gong1 che3 pu3}
\end{EntryWithPhonetic}

\begin{EntryWithPhonetic}{五花八门}{wu3hua1-ba1men2}{4,7,2,3}{⼆,⾋,⼋,⾨}[HSK 7-9]
  \definition{expr.}{de todos os tipos; uma metáfora para uma grande variedade de estilos ou infinitas variações; multifacetado; de uma grande (ou rica) variedade}
  \synonymref{丰富多彩}{feng1fu4-duo1cai3}
  \synonymref{各式各样}{ge4shi4-ge4yang4}
  \synonymref{千变万化}{qian1bian4-wan4hua4}
  \synonymref{五颜六色}{wu3yan2liu4se4}
\end{EntryWithPhonetic}

\begin{EntryWithPhonetic}{五体投地}{wu3ti3tou2di4}{4,7,7,6}{⼆,⼈,⼿,⼟}
  \definition{expr.}{prostrar-se em admiração | adular alguém}
  \antonymref{不以为然}{bu4yi3wei2ran2}
\end{EntryWithPhonetic}

\begin{EntryWithPhonetic}{五五}{wu3wu3}{4,4}{⼆,⼆}
  \definition{num.}{50--50}
  \definition{s.}{igual (partilha, parceria, etc.)}
\end{EntryWithPhonetic}

\begin{EntryWithPhonetic}{五星级}{wu3xing1ji2}{4,9,6}{⼆,⽇,⽷}[HSK 7-9]
  \definition{adj.}{cinco estrelas}
  \definition{s.}{cinco estrelas (hotel, restautante)}
\end{EntryWithPhonetic}

\begin{EntryWithPhonetic}{五颜六色}{wu3yan2liu4se4}{4,15,4,6}{⼆,⾴,⼋,⾊}[HSK 4]
  \definition{adj.}{de várias (ou todas) cores; multicolorido; colorido}
  \synonymref{五花八门}{wu3hua1-ba1men2}
\end{EntryWithPhonetic}

\begin{EntryWithPhonetic}{五岳}{wu3yue4}{4,8}{⼆,⼭}
  \definition*{s.}{Cinco Montanhas Sagradas dos taoístas, a saber: Monte Tai (泰山) na província de Shandong (Pico Oriental), Monte Hua (华山) na província de Shaanxi (Pico Ocidental), Monte Heng (衡山) na província de Hunan (Pico Meridional), Monte Heng (恒山) na província de Shanxi (Pico Setentrional) e Monte Song (嵩山) na província de Henan (Pico Central) são as cinco montanhas mais famosas da história da China}
  \seealsoref{衡山}{heng2shan1}
  \seealsoref{恒山}{heng2shan1}
  \seealsoref{华山}{hua4shan1}
  \seealsoref{嵩山}{song1shan1}
  \seealsoref{泰山}{tai4shan1}
\end{EntryWithPhonetic}

%%%%%%%%%% 午 %%%%%%%%%%
\subsection*{午}\addcontentsline{loh}{figure}{午 \dpy{wu3}}

\begin{EntryWithPhonetic}{午}{wu3}{4}{⼗}
  \definition{s.}{meio"-dia; período entre 11h00 e 13h00 | wu (sétimo dos doze Ramos Terrestres)}
\end{EntryWithPhonetic}

\begin{EntryWithPhonetic}{午餐}{wu3can1}{4,16}{⼗,⾷}[HSK 2]
  \definition[份,顿,次]{s.}{almoço}
  \seealsoref{午饭}{wu3fan4}
\end{EntryWithPhonetic}

\begin{EntryWithPhonetic}{午饭}{wu3fan4}{4,7}{⼗,⾷}[HSK 1]
  \definition[顿]{s.}{almoço}
  \seealsoref{午餐}{wu3can1}
\end{EntryWithPhonetic}

\begin{EntryWithPhonetic}{午后}{wu3hou4}{4,6}{⼗,⼝}
  \definition{s.}{tarde | período da tarde}
\end{EntryWithPhonetic}

\begin{EntryWithPhonetic}{午前}{wu3qian2}{4,9}{⼗,⼑}
  \definition{s.}{\emph{A.M.} | manhã | período da manhã}
\end{EntryWithPhonetic}

\begin{EntryWithPhonetic}{午睡}{wu3shui4}{4,13}{⼗,⽬}[HSK 2]
  \definition{s.}{\emph{siesta}; cochilo da tarde; soneca do meio"-dia}
  \definition{v.}{tirar uma soneca depois do almoço}
\end{EntryWithPhonetic}

\begin{EntryWithPhonetic}{午休}{wu3xiu1}{4,6}{⼗,⼈}
  \definition{s.}{pausa para almoço | cochilo na hora do almoço | intervalo do meio"-dia}
\end{EntryWithPhonetic}

\begin{EntryWithPhonetic}{午宴}{wu3yan4}{4,10}{⼗,⼧}
  \definition{s.}{banquete de almoço}
\end{EntryWithPhonetic}

\begin{EntryWithPhonetic}{午夜}{wu3ye4}{4,8}{⼗,⼣}
  \definition{s.}{meia-noite}
\end{EntryWithPhonetic}

%%%%%%%%%% 武 %%%%%%%%%%
\subsection*{武}\addcontentsline{loh}{figure}{武 \dpy{wu3}}

\begin{EntryWithPhonetic}{武}{wu3}{8}{⽌}
  \definition*{s.}{Sobrenome: Wu}
  \definition{adj.}{valente; corajoso}
  \definition{s.}{militar; atividades e comportamentos relacionados a habilidades militares e de combate | arte marcial | passo; meio passo; pegadas}
  \antonymref{文}{wen2}
\end{EntryWithPhonetic}

\begin{EntryWithPhonetic}{武大戏}{wu3 da4xi4}{8,3,6}{⽌,⼤,⼽}
  \definition*{s.}{Drama de Luta Acrobática | Drama Wu}
\end{EntryWithPhonetic}

\begin{EntryWithPhonetic}{武断}{wu3duan4}{8,11}{⽌,⽄}
  \definition{adj.}{arbitrário | dogmático | subjetivo}
\end{EntryWithPhonetic}

\begin{EntryWithPhonetic}{武官}{wu3guan1}{8,8}{⽌,⼧}
  \definition{s.}{oficial militar | adido militar}
\end{EntryWithPhonetic}

\begin{EntryWithPhonetic}{武力}{wu3li4}{8,2}{⽌,⼒}[HSK 7-9]
  \definition{s.}{força | força militar; poderio armado; poderio bélico; força das armas | força armada (poder)}[霸道政策依靠武力。===Políticas autoritárias dependem da força.]
  \synonymref{暴力}{bao4li4}
\end{EntryWithPhonetic}

\begin{EntryWithPhonetic}{武器}{wu3qi4}{8,16}{⽌,⼝}[HSK 3]
  \definition[批,个,件,种]{s.}{arma; equipamentos e dispositivos utilizados diretamente para matar inimigos ou destruir suas instalações defensivas e ofensivas | armas; armamento; metáfora usada como ferramenta de luta}
\end{EntryWithPhonetic}

\begin{EntryWithPhonetic}{武士}{wu3shi4}{8,3}{⽌,⼠}
  \definition{s.}{samurai | guerreiro}
\end{EntryWithPhonetic}

\begin{EntryWithPhonetic}{武术}{wu3shu4}{8,5}{⽌,⽊}[HSK 3]
  \definition[种,套,门]{s.}{arte marcial; autodefesa; \emph{wushu}; um esporte tradicional chinês que utiliza técnicas com os punhos, pernas, pés ou armas como facas e espadas}
\end{EntryWithPhonetic}

\begin{EntryWithPhonetic}{武艺}{wu3yi4}{8,4}{⽌,⾋}
  \definition{s.}{arte marcial | habilidade militar}
\end{EntryWithPhonetic}

\begin{EntryWithPhonetic}{武装}{wu3zhuang1}{8,12}{⽌,⾐}[HSK 7-9]
  \definition{s.}{armas; equipamento militar; uniforme de combate | forças armadas}
  \definition{v.}{armar; equipar com armas; fornecer armas}
\end{EntryWithPhonetic}

%%%%%%%%%% 侮 %%%%%%%%%%
\subsection*{侮}\addcontentsline{loh}{figure}{侮 \dpy{wu3}}

\begin{EntryWithPhonetic}{侮}{wu3}{9}{⼈}
  \definition{v.}{insultar; intimidar | humilhar; caluniar; assediar (assédio moral); menosprezar}
\end{EntryWithPhonetic}

\begin{EntryWithPhonetic}{侮辱}{wu3ru3}{9,10}{⼈,⾠}[HSK 7-9]
  \definition{v.}{insultar; humilhar; submeter alguém a indignidades; a dignidade e a reputação de uma pessoa são prejudicadas por palavras ou ações | molestar; assediar (uma mulher)}
  \synonymref{耻辱}{chi3ru3}
  \synonymref{欺负}{qi1fu5}
  \antonymref{敬重}{jing4zhong4}
  \antonymref{尊敬}{zun1jing4}
  \antonymref{尊重}{zun1zhong4}
\end{EntryWithPhonetic}

%%%%%%%%%% 捂 %%%%%%%%%%
\subsection*{捂}\addcontentsline{loh}{figure}{捂 \dpy{wu3}}

\begin{EntryWithPhonetic}{捂}{wu3}{10}{⼿}[HSK 7-9]
  \definition{v.}{selar; cobrir; abafar; embrulhar; fechar}
  \seeref{wu2}
  \antonymref{揭}{jie1}
\end{EntryWithPhonetic}

%%%%%%%%%% 舞 %%%%%%%%%%
\subsection*{舞}\addcontentsline{loh}{figure}{舞 \dpy{wu3}}

\begin{EntryWithPhonetic}{舞}{wu3}{14}{⾇}[HSK 5]
  \definition[支,段,个]{s.}{dança | palco; metáfora do domínio das atividades sociais}
  \definition{v.}{mover"-se como numa dança | dançar com algo nas mãos; brincar com | florescer; empunhar; brandir | esvoaçar | fazer malabarismos; brincar com}
\end{EntryWithPhonetic}

\begin{EntryWithPhonetic}{舞抃}{wu3bian4}{14,7}{⾇,⼿}
  \definition{s.}{dançar por prazer}
\end{EntryWithPhonetic}

\begin{EntryWithPhonetic}{舞蹈}{wu3dao3}{14,17}{⾇,⾜}[HSK 6]
  \definition[段,支,场,个]{s.}{dança; uma forma de arte que usa movimentos rítmicos como principal meio de expressão, podendo expressar a vida, os pensamentos e os sentimentos das pessoas, geralmente acompanhada de música}
  \definition{v.}{dançar}
\end{EntryWithPhonetic}

\begin{EntryWithPhonetic}{舞会}{wu3hui4}{14,6}{⾇,⼈}
  \definition{s.}{baile}
\end{EntryWithPhonetic}

\begin{EntryWithPhonetic}{舞会舞}{wu3hui4wu3}{14,6,14}{⾇,⼈,⾇}
  \definition{s.}{baile}
\end{EntryWithPhonetic}

\begin{EntryWithPhonetic}{舞台}{wu3tai2}{14,5}{⾇,⼝}[HSK 3]
  \definition[个]{s.}{palco; plataforma elevada usada exclusivamente para apresentações artísticas, geralmente localizada na parte frontal de teatros e auditórios | palco; metáfora do campo das atividades sociais}
\end{EntryWithPhonetic}

\begin{EntryWithPhonetic}{舞厅}{wu3ting1}{14,4}{⾇,⼚}[HSK 7-9]
  \definition[家,间]{s.}{salão de dança; salão de baile}
\end{EntryWithPhonetic}

\begin{EntryWithPhonetic}{舞厅舞}{wu3ting1wu3}{14,4,14}{⾇,⼚,⾇}
  \definition{s.}{dança de salão}
\end{EntryWithPhonetic}

%%%%%%%%%% 乌 %%%%%%%%%%
\subsection*{乌}\addcontentsline{loh}{figure}{乌 \dpy{wu4}}

\begin{EntryWithPhonetic}{乌}{wu4}{4}{⼃}
  \definition{s.}{sapatos u-la (sapatos com forro de grama para aquecimento) | um tipo de grama chamada u-la}
  \seeref{wu1}
\end{EntryWithPhonetic}

%%%%%%%%%% 勿 %%%%%%%%%%
\subsection*{勿}\addcontentsline{loh}{figure}{勿 \dpy{wu4}}

\begin{EntryWithPhonetic}{勿}{wu4}{4}{⼓}[HSK 7-9]
  \definition{adv.}{não; indica proibição ou dissuasão, como 不要}
  \seealsoref{不要}{bu2yao4}
\end{EntryWithPhonetic}

%%%%%%%%%% 务 %%%%%%%%%%
\subsection*{务}\addcontentsline{loh}{figure}{务 \dpy{wu4}}

\begin{EntryWithPhonetic}{务}{wu4}{5}{⼒}
  \definition*{s.}{Sobrenome: Wu}
  \definition{s.}{caso; negócio | usado em nomes de lugares}
  \definition{v.}{engajar"-se em; dedicar seus esforços a | procurar; perseguir; ir atrás | estar envolvido em; dedicar"-se a; envolver"-se em; comprometer"-se com | deve; deveria; ter certeza de}
\end{EntryWithPhonetic}

\begin{EntryWithPhonetic}{务必}{wu4bi4}{5,5}{⼒,⼼}[HSK 7-9]
  \definition{adv.}{deve; ter certeza de; necessariamente; usado principalmente em frases afirmativas}
  \seealsoref{一定}{yi2ding4}
  \synonymref{必须}{bi4xu1}
  \synonymref{必需}{bi4xu1}
\end{EntryWithPhonetic}

\begin{EntryWithPhonetic}{务实}{wu4shi2}{5,8}{⼒,⼧}[HSK 7-9]
  \definition{adj.}{pragmático; prático; pé-no-chão; focado em resultados práticos, evitando superficialidades}
  \definition{v.}{lidar com assuntos concretos relacionados ao trabalho; discutir e estudar problemas específicos; executar tarefas específicas}
\end{EntryWithPhonetic}

%%%%%%%%%% 物 %%%%%%%%%%
\subsection*{物}\addcontentsline{loh}{figure}{物 \dpy{wu4}}

\begin{EntryWithPhonetic}{物}{wu4}{8}{⽜}
  \definition{s.}{coisa; matéria; objeto | mundo exterior distinto de si mesmo; outras pessoas; refere"-se a outras pessoas além de si mesmo ou ao ambiente em relação a si mesmo | essência; conteúdo; substância | criatura; criação}
\end{EntryWithPhonetic}

\begin{EntryWithPhonetic}{物价}{wu4jia4}{8,6}{⽜,⼈}[HSK 5]
  \definition[个]{s.}{preços das commodities; preços das matérias-primas; preço das mercadorias}
\end{EntryWithPhonetic}

\begin{EntryWithPhonetic}{物理}{wu4li3}{8,11}{⽜,⽟}
  \definition{s.}{física (disciplina)}
\end{EntryWithPhonetic}

\begin{EntryWithPhonetic}{物流}{wu4liu2}{8,10}{⽜,⽔}[HSK 7-9]
  \definition{s.}{logística; a movimentação de mercadorias de um lugar para outro, incluindo embalagem, armazenamento e transporte}
  \synonymref{货运}{huo4yun4}
\end{EntryWithPhonetic}

\begin{EntryWithPhonetic}{物品}{wu4pin3}{8,9}{⽜,⼝}[HSK 6]
  \definition[件,个]{s.}{artigos; itens; bens}
\end{EntryWithPhonetic}

\begin{EntryWithPhonetic}{物体}{wu4ti3}{8,7}{⽜,⼈}[HSK 7-9]
  \definition[个]{s.}{corpo; substância; objeto; coisa}
  \synonymref{东西}{dong1xi5}
\end{EntryWithPhonetic}

\begin{EntryWithPhonetic}{物业}{wu4ye4}{8,5}{⽜,⼀}[HSK 5]
  \definition[处]{s.}{propriedade; gestão de propriedades; gestão patrimonial; administração de imóveis | empresa de administração de imóveis; empresa de gestão imobiliária; empresa de administração de bens imóveis}
\end{EntryWithPhonetic}

\begin{EntryWithPhonetic}{物证}{wu4zheng4}{8,7}{⽜,⾔}[HSK 7-9]
  \definition[件]{s.}{provas materiais | evidências físicas | prova}
\end{EntryWithPhonetic}

\begin{EntryWithPhonetic}{物质}{wu4zhi4}{8,8}{⽜,⾙}[HSK 5]
  \definition[种,类,个]{s.}{matéria; substância; algo que existe além do espírito, que pode ser visto, tocado, cheirado ou detectado por instrumentos científicos | material; meios de subsistência; coisas que permitem às pessoas viver ou viver melhor, como comida, roupas, casas, dinheiro, etc.}
\end{EntryWithPhonetic}

\begin{EntryWithPhonetic}{物资}{wu4zi1}{8,10}{⽜,⾙}[HSK 7-9]
  \definition[份]{s.}{suprimentos; bens e materiais; recursos materiais necessários para a produção e a vida diária}
  \synonymref{物质}{wu4zhi4}
  \synonymref{资源}{zi1yuan2}
\end{EntryWithPhonetic}

%%%%%%%%%% 误 %%%%%%%%%%
\subsection*{误}\addcontentsline{loh}{figure}{误 \dpy{wu4}}

\begin{EntryWithPhonetic}{误}{wu4}{9}{⾔}[HSK 6]
  \definition{adj.}{errado; falso; impreciso | acidental}
  \definition{adv.}{por engano; por acidente; não intencional}
  \definition{s.}{engano; erro}
  \definition{v.}{perder | dificultar; impedir; prejudicar | confundir; entender mal; cometer um erro | causar desvantagem a. causar dano}
\end{EntryWithPhonetic}

\begin{EntryWithPhonetic}{误差}{wu4cha1}{9,9}{⾔,⼯}[HSK 7-9]
  \definition{s.}{erro; a diferença entre o valor medido ou outra aproximação e o valor verdadeiro é chamada de erro | erro; inconsistência}
  \synonymref{差错}{cha1cuo4}
  \synonymref{过错}{guo4cuo4}
  \synonymref{过失}{guo4shi1}
  \synonymref{偏差}{pian1cha1}
  \synonymref{缺点}{que1dian3}
\end{EntryWithPhonetic}

\begin{EntryWithPhonetic}{误导}{wu4dao3}{9,6}{⾔,⼨}[HSK 7-9]
  \definition{v.}{enganar; orientar incorretamente}
  \antonymref{提醒}{ti2/xing3}
\end{EntryWithPhonetic}

\begin{EntryWithPhonetic}{误点}{wu4/dian3}{9,9}{⾔,⽕}
  \definition{v.+compl.}{atrasar | chegar tarde}
\end{EntryWithPhonetic}

\begin{EntryWithPhonetic}{误会}{wu4hui4}{9,6}{⾔,⼈}[HSK 4]
  \definition[场]{s.}{mal"-entendido; desentendimentos ou conflitos decorrentes de mal"-entendidos}
  \definition{v.}{entender mal; entender errado; interpretar mal; não entender; não entender corretamente o significado}
\end{EntryWithPhonetic}

\begin{EntryWithPhonetic}{误解}{wu4jie3}{9,13}{⾔,⾓}[HSK 5]
  \definition[个,种]{s.}{equívoco; mal"-entendido; desentendimento}
  \definition{v.}{interpretar mal; interpretar erroneamente; não compreender corretamente}
\end{EntryWithPhonetic}

\begin{EntryWithPhonetic}{误区}{wu4qu1}{9,4}{⾔,⼖}[HSK 7-9]
  \definition[个]{s.}{equívoco; a área de erro; zona errônea; ideia equivocada persistente; refere"-se a equívocos ou práticas errôneas de longa data}
  \synonymref{埋伏}{mai2fu2}
  \synonymref{圈套}{quan1tao4}
  \synonymref{阴谋}{yin1mou2}
\end{EntryWithPhonetic}

%%%%%%%%%% 恶 %%%%%%%%%%
\subsection*{恶}\addcontentsline{loh}{figure}{恶 \dpy{wu4}}

\begin{EntryWithPhonetic}{恶}{wu4}{10}{⼼}
  \definition{v.}{não gostar; odiar; detestar; repugnar}
  \seeref{e3}
  \seeref{e4}
  \seeref{wu1}
\end{EntryWithPhonetic}

%%%%%%%%%% 雾 %%%%%%%%%%
\subsection*{雾}\addcontentsline{loh}{figure}{雾 \dpy{wu4}}

\begin{EntryWithPhonetic}{雾}{wu4}{13}{⾬}[HSK 7-9]
  \definition[层,场,阵]{s.}{neblina; pequenas gotas de água condensadas do vapor de água | pulverização fina; como muitas pequenas gotas de água na neblina}
\end{EntryWithPhonetic}

\begin{EntryWithPhonetic}{雾气}{wu4qi4}{13,4}{⾬,⽓}
  \definition{s.}{nevoeiro; névoa; neblina; vapor}
\end{EntryWithPhonetic}

%%%%% EOF %%%%%


 %%%
%%% X
%%%
\section*{X}\addcontentsline{toc}{section}{X}\addcontentsline{loh}{figure}{\#\#\#\#\#\#\#\# X}

%%%%%%%%%% 夕 %%%%%%%%%%
\subsection*{夕}\addcontentsline{loh}{figure}{夕 \dpy{xi1}}

\begin{EntryWithPhonetic}{夕}{xi1}{3}{⼣}[Kangxi 36]
  \definition*{s.}{Sobrenome: Xi}
  \definition{s.}{pôr do sol; crepúsculo | tarde; noite}
\end{EntryWithPhonetic}

\begin{EntryWithPhonetic}{夕阳}{xi1yang2}{3,6}{⼣,⾩}
  \definition{s.}{pôr do sol}
  \seealsoref{日出}{ri4chu1}
\end{EntryWithPhonetic}

%%%%%%%%%% 吸 %%%%%%%%%%
\subsection*{吸}\addcontentsline{loh}{figure}{吸 \dpy{xi1}}

\begin{EntryWithPhonetic}{吸}{xi1}{6}{⼝}[HSK 4]
  \definition{v.}{inalar; inspirar; aspirar (oposto a 呼) | sugar (líquidos) | absorver; sugar | atrair; atrair para si mesmo | aspirar; introdução de líquidos, gases, etc. no corpo}
  \seealsoref{呼}{hu1}
\end{EntryWithPhonetic}

\begin{EntryWithPhonetic}{吸毒}{xi1 du2}{6,9}{⼝,⽏}[HSK 6]
  \definition{s.}{droga}
  \definition{v.}{usar drogas viciantes; ser viciado em um narcótico; consumir drogas}
\end{EntryWithPhonetic}

\begin{EntryWithPhonetic}{吸管}{xi1 guan3}{6,14}{⼝,⽵}[HSK 4]
  \definition[根,个,支]{s.}{tubo de sucção; sugador; canudo (para beber); refere-se ao tubo fino usado para sugar bebidas | conta-gotas; pipeta; cateter para bombeamento de líquidos usando pressão de ar}
\end{EntryWithPhonetic}

\begin{EntryWithPhonetic}{吸收}{xi1shou1}{6,6}{⼝,⽁}[HSK 4]
  \definition{v.}{imbuir; absorver; assimilar; sugar;  chupar; (animais, plantas, etc.) extrair material de fora dos tecidos para o interior dos tecidos | absorver; chupar;  sugar alguma substância de fora para dentro | recrutar; alistar; inscrever-se; matricular-se; admitir; (organizações ou coletivos) aceitar novos membros | absorver; aproveitar e usar a experiência, o conhecimento, o dinheiro e outras coisas valiosas de outras pessoas | absorver; diminuir, atenuar ou eliminar determinados efeitos ou fenômenos}
\end{EntryWithPhonetic}

\begin{EntryWithPhonetic}{吸铁石}{xi1tie3shi2}{6,10,5}{⼝,⾦,⽯}
  \definition{s.}{imã | magneto}
  \seealsoref{磁铁}{ci2tie3}
\end{EntryWithPhonetic}

\begin{EntryWithPhonetic}{吸烟}{xi1/yan1}{6,10}{⼝,⽕}[HSK 4]
  \definition{v.+compl.}{fumar}
\end{EntryWithPhonetic}

\begin{EntryWithPhonetic}{吸引}{xi1yin3}{6,4}{⼝,⼸}[HSK 4]
  \definition{v.}{atrair; apelar para; chamar a atenção de outros objetos, forças ou pessoas para si mesmo}
\end{EntryWithPhonetic}

%%%%%%%%%% 西 %%%%%%%%%%
\subsection*{西}\addcontentsline{loh}{figure}{西 \dpy{xi1}}

\begin{EntryWithPhonetic}{西}{xi1}{6}{⾑}[HSK 1][Kangxi 146]
  \definition*{s.}{Espanha, abreviatura de 西班牙 | Paraíso Ocidental | Sobrenome: Xi}
  \definition{s.}{oeste; uma das quatro direções básicas, o lado onde o sol se põe (oposto ao 东) | ocidental; refere-se ao Ocidente (principalmente aos países europeus e americanos) | aqui e ali; em contraposição a 东, significa 到处 ou 零散, 没有次序}
  \seealsoref{到处}{dao4chu4}
  \seealsoref{东}{dong1}
  \seealsoref{零散}{ling2san3}
  \seealsoref{没有次序}{mei2you3 ci4xu4}
  \seealsoref{西班牙}{xi1ban1ya2}
\end{EntryWithPhonetic}

\begin{EntryWithPhonetic}{西安}{xi1'an1}{6,6}{⾑,⼧}
  \definition*{s.}{Xi'an, Capital da Província de Shaanxi}
\end{EntryWithPhonetic}

\begin{EntryWithPhonetic}{西班牙}{xi1ban1ya2}{6,10,4}{⾑,⽟,⽛}
  \definition*{s.}{Espanha}
\end{EntryWithPhonetic}

\begin{EntryWithPhonetic}{西班牙文}{xi1ban1ya2wen2}{6,10,4,4}{⾑,⽟,⽛,⽂}
  \definition{s.}{espanhol, língua espanhola}
  \seealsoref{西文}{xi1wen2}
\end{EntryWithPhonetic}

\begin{EntryWithPhonetic}{西班牙语}{xi1 ban1 ya2 yu3}{6,10,4,9}{⾑,⽟,⽛,⾔}[HSK 6]
  \definition[句]{s.}{espanhol | língua espanhola}
  \seealsoref{西语}{xi1yu3}
\end{EntryWithPhonetic}

\begin{EntryWithPhonetic}{西半球}{xi1ban4qiu2}{6,5,11}{⾑,⼗,⽟}
  \definition{s.}{hemisfério oeste}
\end{EntryWithPhonetic}

\begin{EntryWithPhonetic}{西北}{xi1 bei3}{6,5}{⾑,⼔}[HSK 2]
  \definition{s.}{noroeste | noroeste da China; o Noroeste}
\end{EntryWithPhonetic}

\begin{EntryWithPhonetic}{西边}{xi1bian1}{6,5}{⾑,⾡}[HSK 1]
  \definition{s.}{lado oeste; (oeste) Uma das quatro direções principais; uma das direções cardeais, oposta ao 东方}
  \seealsoref{东方}{dong1 fang1}
\end{EntryWithPhonetic}

\begin{EntryWithPhonetic}{西部}{xi1 bu4}{6,10}{⾑,⾢}[HSK 3]
  \definition{s.}{(EUA) filme de faroeste; filme de \emph{cowboys}; um faroeste | filme da região ocidental (China) | parte ocidental; região oeste da China}
\end{EntryWithPhonetic}

\begin{EntryWithPhonetic}{西餐}{xi1 can1}{6,16}{⾑,⾷}[HSK 2]
  \definition[份,顿,桌]{s.}{comida ocidental; comida de estilo ocidental, comida com garfo e faca (diferente da 中餐)}
  \seealsoref{中餐}{zhong1 can1}
\end{EntryWithPhonetic}

\begin{EntryWithPhonetic}{西方}{xi1 fang1}{6,4}{⾑,⽅}[HSK 2]
  \definition{s.}{oeste | o Ocidente; o Oeste; países europeus e americanos | Paraíso Ocidental, termo budista}
\end{EntryWithPhonetic}

\begin{EntryWithPhonetic}{西瓜}{xi1gua1}{6,5}{⾑,⽠}[HSK 4]
  \definition[个,颗,粒]{s.}{melancia; fruto que é uma baga de formato grande, globular ou oval, com muita polpa aguada e doce}
\end{EntryWithPhonetic}

\begin{EntryWithPhonetic}{西红柿}{xi1hong2shi4}{6,6,9}{⾑,⽷,⽊}[HSK 5]
  \definition[种,只,株]{s.}{tomate}
\end{EntryWithPhonetic}

\begin{EntryWithPhonetic}{西兰花}{xi1lan2hua1}{6,5,7}{⾑,⼋,⾋}
  \definition{s.}{brócolis}
\end{EntryWithPhonetic}

\begin{EntryWithPhonetic}{西蓝花}{xi1lan2hua1}{6,13,7}{⾑,⾋,⾋}
  \variantof{西兰花}
\end{EntryWithPhonetic}

\begin{EntryWithPhonetic}{西面}{xi1mian4}{6,9}{⾑,⾯}
  \definition{s.}{oeste | lado oeste}
\end{EntryWithPhonetic}

\begin{EntryWithPhonetic}{西南}{xi1 nan2}{6,9}{⾑,⼗}[HSK 2]
  \definition{s.}{sudoeste | o Sudoeste; Sudoeste da China}
\end{EntryWithPhonetic}

\begin{EntryWithPhonetic}{西文}{xi1wen2}{6,4}{⾑,⽂}
  \definition{s.}{espanhol | língua espanhola}
  \seealsoref{西班牙文}{xi1ban1ya2wen2}
\end{EntryWithPhonetic}

\begin{EntryWithPhonetic}{西西}{xi1xi1}{6,6}{⾑,⾑}
  \definition{num.}{centímetro cúbico}
\end{EntryWithPhonetic}

\begin{EntryWithPhonetic}{西药}{xi1 yao4}{6,9}{⾑,⾋}
  \definition[片,粒]{s.}{medicina ocidental; refere-se aos medicamentos usados ​​na medicina ocidental, geralmente feitos por métodos sintéticos ou extraídos de produtos naturais, como comprimidos anti-inflamatórios, aspirina, tintura de iodo, penicilina, etc.}
\end{EntryWithPhonetic}

\begin{EntryWithPhonetic}{西医}{xi1 yi1}{6,7}{⾑,⼖}[HSK 2]
  \definition[名,位]{s.}{medicina ocidental; medicina introduzida na China a partir da Europa e da América | um médico treinado em medicina ocidental}
\end{EntryWithPhonetic}

\begin{EntryWithPhonetic}{西语}{xi1yu3}{6,9}{⾑,⾔}
  \definition{s.}{línguas ocidentais | espanhol | língua espanhola}
  \seealsoref{西班牙语}{xi1 ban1 ya2 yu3}
\end{EntryWithPhonetic}

\begin{EntryWithPhonetic}{西藏}{xi1zang4}{6,17}{⾑,⾋}
  \definition*{s.}{Xizang; Região Autônoma do Tibete, 西藏自治区}
  \seealsoref{西藏自治区}{xi1zang4 zi4zhi4qu1}
\end{EntryWithPhonetic}

\begin{EntryWithPhonetic}{西藏自治区}{xi1zang4 zi4zhi4qu1}{6,17,6,8,4}{⾑,⾋,⾃,⽔,⼖}
  \definition*{s.}{Região Autônoma do Tibete}
\end{EntryWithPhonetic}

\begin{EntryWithPhonetic}{西装}{xi1 zhuang1}{6,12}{⾑,⾐}[HSK 5]
  \definition[件,套,个]{s.}{terno; roupas de estilo ocidental; roupas ocidentais, divididas em masculinas e femininas}
\end{EntryWithPhonetic}

%%%%%%%%%% 希 %%%%%%%%%%
\subsection*{希}\addcontentsline{loh}{figure}{希 \dpy{xi1}}

\begin{EntryWithPhonetic}{希}{xi1}{7}{⼱}
  \definition*{s.}{Sobrenome: Xi}
  \definition{v.}{ter esperança}
\end{EntryWithPhonetic}

\begin{EntryWithPhonetic}{希望}{xi1wang4}{7,11}{⼱,⽉}[HSK 3]
  \definition[个,丝,点]{s.}{esperança; desejo; expectativa; a possibilidade de alcançar um determinado objetivo ou de ocorrer uma determinada situação ideal no futuro | aquilo em que a esperança é depositada; o objeto da esperança}
  \definition{v.}{ter esperança; desejar; esperar; pensar em alcançar algum objetivo ou que alguma situação ocorra}
\end{EntryWithPhonetic}

%%%%%%%%%% 昔 %%%%%%%%%%
\subsection*{昔}\addcontentsline{loh}{figure}{昔 \dpy{xi1}}

\begin{EntryWithPhonetic}{昔}{xi1}{8}{⽇}
  \definition{s.}{tempos antigos; o passado; era uma vez}
\end{EntryWithPhonetic}

\begin{EntryWithPhonetic}{昔日}{xi1ri4}{8,4}{⽇,⽇}
  \definition{adj.}{passado}
\end{EntryWithPhonetic}

%%%%%%%%%% 牺 %%%%%%%%%%
\subsection*{牺}\addcontentsline{loh}{figure}{牺 \dpy{xi1}}

\begin{EntryWithPhonetic}{牺}{xi1}{10}{⽜}
  \definition{s.}{um animal de cor uniforme para sacrifício; sacrifício; gado com pelagem pura usado para sacrifício}
\end{EntryWithPhonetic}

\begin{EntryWithPhonetic}{牺牲}{xi1sheng1}{10,9}{⽜,⽜}[HSK 6]
  \definition[份]{s.}{sacrifício; um animal abatido para sacrifício; refere-se ao sacrifício da própria vida ou dos próprios interesses por um propósito justo, ou refere-se ao preço pago por um determinado propósito}
  \definition{v.}{sacrificar-se; morrer como mártir; dar a própria vida; sacrificar sua vida pela justiça | sacrificar; desistir; fazer algo às custas de; geralmente se refere a pagar um preço ou sofrer danos por alguém ou algo}
\end{EntryWithPhonetic}

%%%%%%%%%% 悉 %%%%%%%%%%
\subsection*{悉}\addcontentsline{loh}{figure}{悉 \dpy{xi1}}

\begin{EntryWithPhonetic}{悉}{xi1}{11}{⼼}
  \definition*{s.}{Sobrenome: Xi}
  \definition{adj.}{tudo; inteiro; total | detalhado}
  \definition{v.}{saber; aprender; ser informado de}
\end{EntryWithPhonetic}

\begin{EntryWithPhonetic}{悉尼}{xi1ni2}{11,5}{⼼,⼫}
  \definition*{s.}{Sidney}
\end{EntryWithPhonetic}

\begin{EntryWithPhonetic}{悉数}{xi1shu3}{11,13}{⼼,⽁}
  \definition{adv.}{enumerar em detalhes | explicar claramente}
  \seeref{xi1shu4}
\end{EntryWithPhonetic}

\begin{EntryWithPhonetic}{悉数}{xi1shu4}{11,13}{⼼,⽁}
  \definition{adv.}{todos | cada um | toda a soma}
  \seeref{xi1shu3}
\end{EntryWithPhonetic}

\begin{EntryWithPhonetic}{悉心}{xi1xin1}{11,4}{⼼,⼼}
  \definition{adv.}{colocar o coração (e a alma) em algo | com muito cuidado}
\end{EntryWithPhonetic}

%%%%%%%%%% 腊 %%%%%%%%%%
\subsection*{腊}\addcontentsline{loh}{figure}{腊 \dpy{xi1}}

\begin{EntryWithPhonetic}{腊}{xi1}{12}{⾁}
  \definition{s.}{carne seca}
  \seeref{la4}
\end{EntryWithPhonetic}

%%%%%%%%%% 蜥 %%%%%%%%%%
\subsection*{蜥}\addcontentsline{loh}{figure}{蜥 \dpy{xi1}}

\begin{EntryWithPhonetic}{蜥}{xi1}{14}{⾍}
  \definition{s.}{lagarto}
\end{EntryWithPhonetic}

\begin{EntryWithPhonetic}{蜥易}{xi1yi4}{14,8}{⾍,⽇}
  \variantof{蜥蜴}
\end{EntryWithPhonetic}

\begin{EntryWithPhonetic}{蜥蜴}{xi1yi4}{14,14}{⾍,⾍}
  \definition{s.}{lagarto}
\end{EntryWithPhonetic}

%%%%%%%%%% 习 %%%%%%%%%%
\subsection*{习}\addcontentsline{loh}{figure}{习 \dpy{xi2}}

\begin{EntryWithPhonetic}{习}{xi2}{3}{⼄}
  \definition*{s.}{Sobrenome: Xi}
  \definition{s.}{hábito; costume; prática usual; um comportamento que se desenvolve inconscientemente por meio de ações repetidas ao longo de um longo período de tempo}
  \definition{v.}{revisar; praticar; exercitar | acostumado a; familiarizado com; familiarizado com algo por meio de contato frequente | estudar; aprender (pássaro)}
\end{EntryWithPhonetic}

\begin{EntryWithPhonetic}{习惯}{xi2guan4}{3,11}{⼄,⼼}[HSK 2]
  \definition[个,种]{s.}{hábito; costume; prática usual; comportamentos, tendências ou tendências sociais que se desenvolvem gradualmente ao longo de um longo período de tempo e são difíceis de mudar}
  \definition{v.}{estar acostumado a; ter o hábito de}
\end{EntryWithPhonetic}

%%%%%%%%%% 席 %%%%%%%%%%
\subsection*{席}\addcontentsline{loh}{figure}{席 \dpy{xi2}}

\begin{EntryWithPhonetic}{席}{xi2}{10}{⼱}
  \definition*{s.}{Sobrenome: Xi}
  \definition[卷,张]{s.}{esteira | assento; lugar; caixa | assento (em uma assembleia legislativa) | festim; banquete; jantar}
\end{EntryWithPhonetic}

\begin{EntryWithPhonetic}{席卷}{xi2juan3}{10,8}{⼱,⼙}
  \definition{v.}{engolfar | varrer | levar tudo para fora}
\end{EntryWithPhonetic}

%%%%%%%%%% 袭 %%%%%%%%%%
\subsection*{袭}\addcontentsline{loh}{figure}{袭 \dpy{xi2}}

\begin{EntryWithPhonetic}{袭}{xi2}{11}{⾐}
  \definition*{s.}{Sobrenome: Xi}
  \definition{clas.}{usado para conjuntos completos de roupas}
  \definition{v.}{fazer um ataque surpresa a; invadir | seguir o padrão de; continuar como antes; fazer o mesmo}
\end{EntryWithPhonetic}

\begin{EntryWithPhonetic}{袭击}{xi2ji1}{11,5}{⾐,⼐}
  \definition{s.}{ataque (especialmente um ataque surpresa) | invasão}
  \definition{v.}{atacar}
\end{EntryWithPhonetic}

%%%%%%%%%% 洗 %%%%%%%%%%
\subsection*{洗}\addcontentsline{loh}{figure}{洗 \dpy{xi3}}

\begin{EntryWithPhonetic}{洗}{xi3}{9}{⽔}[HSK 1]
  \definition[个]{s.}{pequeno recipiente contendo água para enxaguar os pincéis de escrever | batismo}
  \definition{v.}{lavar; tomar banho; remover a sujeira do objeto com água ou outro solvente | batizar | eliminar; corrigir; reparar | saquear; matar e pilhar; matar ou roubar tudo, como se tivesse sido lavado | revelar filmes; imprimir fotos | apagar; limpar (uma gravação, etc.) | embaralhar (cartas, etc.)}
\end{EntryWithPhonetic}

\begin{EntryWithPhonetic}{洗涤}{xi3di2}{9,10}{⽔,⽔}
  \definition{s.}{enxágue | lava}
  \definition{v.}{enxaguar | lavar}
\end{EntryWithPhonetic}

\begin{EntryWithPhonetic}{洗涤间}{xi3di2jian1}{9,10,7}{⽔,⽔,⾨}
  \definition{s.}{lavanderia}
\end{EntryWithPhonetic}

\begin{EntryWithPhonetic}{洗劫}{xi3jie2}{9,7}{⽔,⼒}
  \definition{v.}{saquear; pilhar; roubar}
\end{EntryWithPhonetic}

\begin{EntryWithPhonetic}{洗净}{xi3jing4}{9,8}{⽔,⼎}
  \definition{v.}{lavar (limpeza)}
\end{EntryWithPhonetic}

\begin{EntryWithPhonetic}{洗礼}{xi3li3}{9,5}{⽔,⽰}
  \definition{s.}{batismo}
  \definition{v.}{batizar}
\end{EntryWithPhonetic}

\begin{EntryWithPhonetic}{洗手}{xi3shou3}{9,4}{⽔,⼿}
  \definition{v.}{ir ao banheiro | lavar as mãos}
\end{EntryWithPhonetic}

\begin{EntryWithPhonetic}{洗手不干}{xi3shou3bu2gan4}{9,4,4,3}{⽔,⼿,⼀,⼲}
  \definition{v.}{parar totalmente de fazer algo}
\end{EntryWithPhonetic}

\begin{EntryWithPhonetic}{洗手池}{xi3shou3chi2}{9,4,6}{⽔,⼿,⽔}
  \definition{s.}{pia de banheiro | lavatório}
  \seealsoref{洗手盆}{xi3shou3pen2}
\end{EntryWithPhonetic}

\begin{EntryWithPhonetic}{洗手间}{xi3shou3jian1}{9,4,7}{⽔,⼿,⾨}[HSK 1]
  \definition[个]{s.}{banheiro; lavatório; lavabo}
\end{EntryWithPhonetic}

\begin{EntryWithPhonetic}{洗手盆}{xi3shou3pen2}{9,4,9}{⽔,⼿,⽫}
  \definition{s.}{pia de banheiro | lavatório}
  \seealsoref{洗手池}{xi3shou3chi2}
\end{EntryWithPhonetic}

\begin{EntryWithPhonetic}{洗手乳}{xi3shou3ru3}{9,4,8}{⽔,⼿,⼄}
  \definition{s.}{sabonete líquido para lavar as mãos}
  \seealsoref{洗手液}{xi3shou3ye4}
\end{EntryWithPhonetic}

\begin{EntryWithPhonetic}{洗手液}{xi3shou3ye4}{9,4,11}{⽔,⼿,⽔}
  \definition{s.}{sabonete líquido para lavar as mãos}
  \seealsoref{洗手乳}{xi3shou3ru3}
\end{EntryWithPhonetic}

\begin{EntryWithPhonetic}{洗脱}{xi3tuo1}{9,11}{⽔,⾁}
  \definition{v.}{limpar | purgar | lavar}
\end{EntryWithPhonetic}

\begin{EntryWithPhonetic}{洗碗}{xi3wan3}{9,13}{⽔,⽯}
  \definition{v.}{lavar pratos}
\end{EntryWithPhonetic}

\begin{EntryWithPhonetic}{洗胃}{xi3wei4}{9,9}{⽔,⾁}
  \definition{s.}{(medicina) lavagem gástrica}
  \definition{v.}{ter o estômago lavado}
\end{EntryWithPhonetic}

\begin{EntryWithPhonetic}{洗衣粉}{xi3 yi1 fen3}{9,6,10}{⽔,⾐,⽶}[HSK 6]
  \definition[袋,包,勺]{s.}{sabão em pó; detergente para roupa (em pó); detergente em pó sintetizado quimicamente, específico para uso em lavanderia}
\end{EntryWithPhonetic}

\begin{EntryWithPhonetic}{洗衣机}{xi3 yi1 ji1}{9,6,6}{⽔,⾐,⽊}[HSK 2]
  \definition[台]{s.}{máquina de lavar roupa; eletrodomésticos para lavagem automática ou semiautomática de roupas}
\end{EntryWithPhonetic}

\begin{EntryWithPhonetic}{洗澡}{xi3/zao3}{9,16}{⽔,⽔}[HSK 2]
  \definition{v.+compl.}{tomar banho; tomar banho de chuveiro; lavar-se}
\end{EntryWithPhonetic}

\begin{EntryWithPhonetic}{洗澡间}{xi3zao3jian1}{9,16,7}{⽔,⽔,⾨}
  \definition[间]{s.}{banheiro}
\end{EntryWithPhonetic}

%%%%%%%%%% 喜 %%%%%%%%%%
\subsection*{喜}\addcontentsline{loh}{figure}{喜 \dpy{xi3}}

\begin{EntryWithPhonetic}{喜}{xi3}{12}{⼝}
  \definition{adj.}{feliz; satisfeito; encantado}
  \definition[桩,件]{s.}{evento feliz (especialmente casamento); ocasião para celebração; algo para comemorar | gravidez | casamento ou coisas relacionadas a ele}
  \definition{v.}{gostar; fonte de; ter inclinação para | precisa; requer; combina melhor com; (um certo organismo) precisa ou é adequado para (um certo ambiente ou algo)}
\end{EntryWithPhonetic}

\begin{EntryWithPhonetic}{喜爱}{xi3 ai4}{12,10}{⼝,⽖}[HSK 4]
  \definition{v.}{gostar; amar; ter afeição por; estar interessado em; ter uma queda ou sentir interesse por pessoas ou coisas}
\end{EntryWithPhonetic}

\begin{EntryWithPhonetic}{喜欢}{xi3huan5}{12,6}{⼝,⽋}[HSK 1]
  \definition{adj.}{feliz; encantado; exultante; cheio de alegria}
  \definition{v.}{gostar; amar; ter afeição por; estar interessado em; ter uma boa impressão ou interesse por alguém ou algo}
\end{EntryWithPhonetic}

\begin{EntryWithPhonetic}{喜剧}{xi3 ju4}{12,10}{⼝,⼑}[HSK 5]
  \definition[部,出]{s.}{comédia (oposto de 悲剧) | comédia; uma das principais categorias do teatro; usa o exagero para satirizar e ridicularizar o feio; fenômenos retrógrados; destaca as contradições inerentes a esses fenômenos e seu conflito com coisas saudáveis; costuma provocar risadas; o final geralmente é feliz}
  \seealsoref{悲剧}{bei1 ju4}
\end{EntryWithPhonetic}

%%%%%%%%%% 戏 %%%%%%%%%%
\subsection*{戏}\addcontentsline{loh}{figure}{戏 \dpy{xi4}}

\begin{EntryWithPhonetic}{戏}{xi4}{6}{⼽}[HSK 5]
  \definition*{s.}{Sobrenome: Xi}
  \definition[场,部,出,台]{s.}{drama; peça; espetáculo; \emph{show}}
  \definition{v.}{brincar; praticar esportes; jogar | zombar; brincar; provocar}
\end{EntryWithPhonetic}

\begin{EntryWithPhonetic}{戏法}{xi4fa3}{6,8}{⼽,⽔}
  \definition{s.}{truque de mágica | prestidigitação}
\end{EntryWithPhonetic}

\begin{EntryWithPhonetic}{戏剧}{xi4ju4}{6,10}{⼽,⼑}[HSK 5]
  \definition[出,部]{s.}{drama; peça; teatro | roteiro; peça; cenário}
\end{EntryWithPhonetic}

\begin{EntryWithPhonetic}{戏剧般}{xi4ju4ban1}{6,10,10}{⼽,⼑,⾈}
  \definition{adj.}{melodramático}
\end{EntryWithPhonetic}

\begin{EntryWithPhonetic}{戏剧编剧}{xi4ju4bian1ju4}{6,10,12,10}{⼽,⼑,⽷,⼑}
  \definition{s.}{dramaturgo}
\end{EntryWithPhonetic}

\begin{EntryWithPhonetic}{戏剧化地}{xi4ju4hua4di4}{6,10,4,6}{⼽,⼑,⼔,⼟}
  \definition{adv.}{dramaticamente | teatralmente}
\end{EntryWithPhonetic}

\begin{EntryWithPhonetic}{戏剧家}{xi4ju4jia1}{6,10,10}{⼽,⼑,⼧}
  \definition{s.}{dramaturgo}
\end{EntryWithPhonetic}

\begin{EntryWithPhonetic}{戏剧效果}{xi4ju4xiao4guo3}{6,10,10,8}{⼽,⼑,⽁,⽊}
  \definition{s.}{efeito dramático}
\end{EntryWithPhonetic}

\begin{EntryWithPhonetic}{戏剧性}{xi4ju4xing4}{6,10,8}{⼽,⼑,⼼}
  \definition{adj.}{dramático}
\end{EntryWithPhonetic}

\begin{EntryWithPhonetic}{戏剧演出}{xi4ju4yan3chu1}{6,10,14,5}{⼽,⼑,⽔,⼐}
  \definition{s.}{performance dramática}
\end{EntryWithPhonetic}

\begin{EntryWithPhonetic}{戏弄}{xi4nong4}{6,7}{⼽,⼶}
  \definition{v.}{zombar de | pregar peças | provocar}
\end{EntryWithPhonetic}

\begin{EntryWithPhonetic}{戏曲}{xi4 qu3}{6,6}{⼽,⽈}[HSK 6]
  \definition{s.}{drama; ópera chinesa; ópera tradicional; forma teatral tradicional | partes cantadas em 传奇 e zaju 杂剧}
  \seealsoref{传奇}{chuan2qi2}
  \seealsoref{杂剧}{za2ju4}
\end{EntryWithPhonetic}

\begin{EntryWithPhonetic}{戏耍}{xi4shua3}{6,9}{⼽,⽽}
  \definition{v.}{divertir-me | brincar com | provocar}
\end{EntryWithPhonetic}

\begin{EntryWithPhonetic}{戏谑}{xi4xue4}{6,11}{⼽,⾔}
  \definition{v.}{brincar | fazer piadas | ridicularizar}
\end{EntryWithPhonetic}

\begin{EntryWithPhonetic}{戏院}{xi4yuan4}{6,9}{⼽,⾩}
  \definition{s.}{teatro}
\end{EntryWithPhonetic}

%%%%%%%%%% 系 %%%%%%%%%%
\subsection*{系}\addcontentsline{loh}{figure}{系 \dpy{xi4}}

\begin{EntryWithPhonetic}{系}{xi4}{7}{⽷}[HSK 3,4]
  \definition*{s.}{Sobrenome: Xi}
  \definition{s.}{sistema; série | departamento; faculdade; unidades administrativas de ensino divididas por disciplina nas instituições de ensino superior}
  \definition{v.}{relacionar-se com; suportar; depender de | sentir-se ansioso; estar preocupado | amarrar; prender | ser; expressa julgamento, equivalente a 是}
  \seeref{ji4}
  \seealsoref{是}{shi4}
\end{EntryWithPhonetic}

\begin{EntryWithPhonetic}{系列}{xi4lie4}{7,6}{⽷,⼑}[HSK 4]
  \definition{s.}{série; conjunto; conjunto de coisas relacionadas (matemática)}
\end{EntryWithPhonetic}

\begin{EntryWithPhonetic}{系囚}{xi4qiu2}{7,5}{⽷,⼞}
  \definition{s.}{prisioneiro}
\end{EntryWithPhonetic}

\begin{EntryWithPhonetic}{系统}{xi4tong3}{7,9}{⽷,⽷}[HSK 4]
  \definition{adj.}{sistemático; organizado}
  \definition[套,个]{s.}{sistema; relação de tipos semelhantes (ou seja, grupo de coisas semelhantes)}
\end{EntryWithPhonetic}

%%%%%%%%%% 细 %%%%%%%%%%
\subsection*{细}\addcontentsline{loh}{figure}{细 \dpy{xi4}}

\begin{EntryWithPhonetic}{细}{xi4}{8}{⽷}[HSK 4]
  \definition{adj.}{fino; delgado; esguio; esbelto; em oposição a 粗 | fino; em partículas pequenas; grãos pequenos | fino e macio;  um sussuro | fino; requintado; delicado | cuidadoso; detalhado; meticuloso | ínfimo; minúsculo; insignificante; diminuto | jovem; pequeno}
  \seealsoref{粗}{cu1}
\end{EntryWithPhonetic}

\begin{EntryWithPhonetic}{细胞}{xi4bao1}{8,9}{⽷,⾁}[HSK 6]
  \definition[个]{s.}{célula; unidade estrutural e funcional básica de um organismo, com uma variedade de formas, composta principalmente pelo núcleo, citoplasma e membrana celular; as plantas também possuem paredes celulares fora da membrana celular}
\end{EntryWithPhonetic}

\begin{EntryWithPhonetic}{细节}{xi4jie2}{8,5}{⽷,⾋}[HSK 4]
  \definition[处]{s.}{detalhe; particularidade; aspectos secundários ou partes sutis de um enredo ou episódios secundários usados em uma obra literária para expressar o caráter de uma pessoa ou as características essenciais de uma coisa}
\end{EntryWithPhonetic}

\begin{EntryWithPhonetic}{细菌}{xi4jun1}{8,11}{⽷,⾋}[HSK 6]
  \definition[个]{s.}{germe; bactéria; um organismo muito pequeno, invisível aos olhos humanos}
\end{EntryWithPhonetic}

\begin{EntryWithPhonetic}{细菌战}{xi4jun1zhan4}{8,11,9}{⽷,⾋,⼽}
  \definition{s.}{guerra biológica}
\end{EntryWithPhonetic}

\begin{EntryWithPhonetic}{细致}{xi4zhi4}{8,10}{⽷,⾄}[HSK 4]
  \definition{adj.}{meticuloso; cuidadoso; minucioso | intrincado; delicado}
\end{EntryWithPhonetic}

%%%%%%%%%% 虾 %%%%%%%%%%
\subsection*{虾}\addcontentsline{loh}{figure}{虾 \dpy{xia1}}

\begin{EntryWithPhonetic}{虾}{xia1}{9}{⾍}
  \definition{s.}{camarão}
\end{EntryWithPhonetic}

%%%%%%%%%% 狭 %%%%%%%%%%
\subsection*{狭}\addcontentsline{loh}{figure}{狭 \dpy{xia2}}

\begin{EntryWithPhonetic}{狭}{xia2}{9}{⽝}
  \definition{adj.}{estreito (oposto a 广)}
  \seealsoref{广}{guang3}
\end{EntryWithPhonetic}

%%%%%%%%%% 下 %%%%%%%%%%
\subsection*{下}\addcontentsline{loh}{figure}{下 \dpy{xia4}}

\begin{EntryWithPhonetic}{下}{xia4}{3}{⼀}[HSK 1,2]
  \definition{clas.}{número de vezes usado para a ação | volume de um contêiner; quantidade de objetos que cabem em um utensílio | usado depois de 两 e 几 para expressar habilidade, capacidade, destreza}
  \definition{s.}{abaixo | próximo; último; segundo; referindo-se ao que está por vir ou ao que vem em seguida | mais baixo; inferior; de baixo nível ou grau | próximo; último; segundo; em ordem ou em ordem cronológica | indica pertencer a uma determinada faixa, situação, condição, etc. | indica uma determinada época ou estação | usado após um número para indicar posição ou direção | para baixo (após uma preposição) | sob (depois de um substantivo) | para baixo (antes de um verbo)}
  \definition{v.}{desembarcar; descer; sair | cair (chuva, neve, etc.) | enviar; emitir; entregar | ir para | sair; partir; retirar-se | lançar; colocar | descarregar; desmontar; tirar (fora) | formar (uma opinião, ideia, etc.); tomar decisões, fazer julgamentos, etc. | usar; aplicar | dar à luz (animais) | tomar; capturar; conquistar | ceder | terminar; deixar de lado; terminar o trabalho ou os estudos diários na hora prevista | para negação; ser inferior a; ser menor que}
  \seealsoref{几}{ji3}
  \seealsoref{两}{liang3}
\end{EntryWithPhonetic}

\begin{EntryWithPhonetic}{下巴}{xia4ba5}{3,4}{⼀,⼰}
  \definition[个]{s.}{queixo}
\end{EntryWithPhonetic}

\begin{EntryWithPhonetic}{下班}{xia4/ban1}{3,10}{⼀,⽟}[HSK 1]
  \definition{v.+compl.}{sair do trabalho; bater ponto; terminar o trabalho na hora prevista e sair do local de trabalho}
\end{EntryWithPhonetic}

\begin{EntryWithPhonetic}{下边}{xia4 bian5}{3,5}{⼀,⾡}[HSK 1]
  \definition{s.}{abaixo; sob; por baixo | próximo em ordem; seguinte | nível inferior; subordinado | a parte inferior}
\end{EntryWithPhonetic}

\begin{EntryWithPhonetic}{下车}{xia4 che1}{3,4}{⼀,⾞}[HSK 1]
  \definition{v.}{descer ou sair de (um ônibus, trem, carro etc.)}
\end{EntryWithPhonetic}

\begin{EntryWithPhonetic}{下次}{xia4 ci4}{3,6}{⼀,⽋}[HSK 1]
  \definition{s.}{na próxima vez; na próxima oportunidade ou no próximo evento}
\end{EntryWithPhonetic}

\begin{EntryWithPhonetic}{下蛋}{xia4dan4}{3,11}{⼀,⾍}
  \definition{v.}{botar ovos}
\end{EntryWithPhonetic}

\begin{EntryWithPhonetic}{下调}{xia4diao4}{3,10}{⼀,⾔}
  \definition{v.}{rebaixar; diminuir a regulamentação | passar para uma unidade inferior}
  \seeref{xia4tiao2}
\end{EntryWithPhonetic}

\begin{EntryWithPhonetic}{下方}{xia4fang1}{3,4}{⼀,⽅}
  \definition{s.}{parte inferior (oposto a 上方) | abaixo | embaixo | mundo dos mortais}
  \definition{v.}{descer ao mundo dos mortais (deuses)}
  \seealsoref{上方}{shang4fang1}
\end{EntryWithPhonetic}

\begin{EntryWithPhonetic}{下个月}{xia4 ge4 yue4}{3,3,4}{⼀,⼈,⽉}[HSK 4]
  \definition{s.}{próximo mês; mês que vem; refere-se ao próximo mês do mês atual}
\end{EntryWithPhonetic}

\begin{EntryWithPhonetic}{下海}{xia4/hai3}{3,10}{⼀,⽔}
  \definition{v.+compl.}{ir para o mar; (barco) deixar o porto e iniciar uma jornada | ir pescar no mar | tornar-se ator profissional}
\end{EntryWithPhonetic}

\begin{EntryWithPhonetic}{下降}{xia4 jiang4}{3,8}{⼀,⾩}[HSK 4]
  \definition{v.}{cair; despencar; declinar; descer; diminuir; ir para baixo}
\end{EntryWithPhonetic}

\begin{EntryWithPhonetic}{下课}{xia4/ke4}{3,10}{⼀,⾔}[HSK 1]
  \definition{v.+compl.}{terminar a aula; sair da aula}
\end{EntryWithPhonetic}

\begin{EntryWithPhonetic}{下来}{xia4 lai5}{3,7}{⼀,⽊}[HSK 3]
  \definition{part.}{usado após o verbo, indica que a ação ou o comportamento se dirige para a posição do falante ou que a ação é contínua ou concluída | usado após um adjetivo, indica que uma determinada situação começou a ocorrer e continuará a se desenvolver}
  \definition{v.}{descer (para a minha localização) | (colheitas/frutas/vegetais, etc.) ser colhido; estar maduro o suficiente para ser colhido | (período de tempo) acabar; passar; chegar ao fim; indicar o fim de um período de tempo}
\end{EntryWithPhonetic}

\begin{EntryWithPhonetic}{下楼}{xia4 lou2}{3,13}{⼀,⽊}[HSK 4]
  \definition{v.}{descer as escadas}
\end{EntryWithPhonetic}

\begin{EntryWithPhonetic}{下面}{xia4 mian4}{3,9}{⼀,⾯}[HSK 3]
  \definition{s.}{em baixo; abaixo; parte de baixo | próximo; seguinte; a parte posterior; a parte posterior de um artigo ou discurso em relação ao que está sendo narrado no momento | subordinado; o nível inferior; homens nos níveis inferiores | por baixo}
\end{EntryWithPhonetic}

\begin{EntryWithPhonetic}{下去}{xia4 qu4}{3,5}{⼀,⼛}[HSK 3]
  \definition{part.}{usado depois de verbos para indicar de alto a baixo | usado depois de um verbo para indicar continuação}
  \definition{v.}{descer; baixar (a partir da minha localização) | (após um verbo) continuar (fazendo algo); prosseguir | usado após o verbo, indica uma descida de um ponto alto para um ponto baixo | usado após o verbo, indica continuidade | usado após um adjetivo, indica que o grau continua aumentando}
\end{EntryWithPhonetic}

\begin{EntryWithPhonetic}{下水道}{xia4shui3dao4}{3,4,12}{⼀,⽔,⾡}
  \definition{s.}{esgoto}
\end{EntryWithPhonetic}

\begin{EntryWithPhonetic}{下调}{xia4tiao2}{3,10}{⼀,⾔}
  \definition{v.}{regular para baixo; ajustar para baixo}
  \seeref{xia4diao4}
\end{EntryWithPhonetic}

\begin{EntryWithPhonetic}{下午}{xia4wu3}{3,4}{⼀,⼗}[HSK 1]
  \definition[个]{s.}{tarde; \emph{post meridiem} (p.m.); refere-se ao período entre o meio-dia e o pôr do sol}
\end{EntryWithPhonetic}

\begin{EntryWithPhonetic}{下午茶}{xia4wu3cha2}{3,4,9}{⼀,⼗,⾋}
  \definition{s.}{chá da tarde (normalmente chás com doces)}
\end{EntryWithPhonetic}

\begin{EntryWithPhonetic}{下线}{xia4xian4}{3,8}{⼀,⽷}
  \definition{v.}{ficar \emph{offline} | (um produto) sair da linha de montagem | pessoa abaixo de si em um esquema de pirâmide}
\end{EntryWithPhonetic}

\begin{EntryWithPhonetic}{下限}{xia4xian4}{3,8}{⼀,⾩}
  \definition{s.}{limite mínimo ou mais recente permitido; limite inferior; limiar; mínimo prescrito; piso (nível) (oposto de 上限)}
  \seealsoref{上限}{shang4xian4}
\end{EntryWithPhonetic}

\begin{EntryWithPhonetic}{下雪}{xia4/xue3}{3,11}{⼀,⾬}[HSK 2]
  \definition{v.+compl.}{nevar}
\end{EntryWithPhonetic}

\begin{EntryWithPhonetic}{下旬}{xia4xun2}{3,6}{⼀,⽇}
  \definition{adv.}{última dezena do mês}
\end{EntryWithPhonetic}

\begin{EntryWithPhonetic}{下游}{xia4you2}{3,12}{⼀,⽔}
  \definition{s.}{a jusante; rio abaixo; trechos inferiores; o trecho do rio próximo à sua foz | para trás; a posição inferior; referindo-se metaforicamente a uma posição invertida}
\end{EntryWithPhonetic}

\begin{EntryWithPhonetic}{下雨}{xia4/yu3}{3,8}{⼀,⾬}[HSK 1]
  \definition{v.+compl.}{chover}
\end{EntryWithPhonetic}

\begin{EntryWithPhonetic}{下载}{xia4zai3}{3,10}{⼀,⾞}[HSK 4]
  \definition{v.}{\emph{download}; baixar; salvar informações da \emph{Web} em um dispositivo, como um computador}
\end{EntryWithPhonetic}

\begin{EntryWithPhonetic}{下崽}{xia4zai3}{3,12}{⼀,⼭}
  \definition{v.}{dar à luz (animais) | parir}
\end{EntryWithPhonetic}

\begin{EntryWithPhonetic}{下周}{xia4 zhou1}{3,8}{⼀,⼝}[HSK 2]
  \definition{s.}{próxima semana}
\end{EntryWithPhonetic}

%%%%%%%%%% 吓 %%%%%%%%%%
\subsection*{吓}\addcontentsline{loh}{figure}{吓 \dpy{xia4}}

\begin{EntryWithPhonetic}{吓}{xia4}{6}{⼝}[HSK 5]
  \definition{interj.}{interjeição que demonstra espanto; Interjeição que expressa insatisfação}
  \definition{v.}{ameaçar; intimidar; usar ameaças ou meios coercitivos para intimidar ou assustar}
\end{EntryWithPhonetic}

\begin{EntryWithPhonetic}{吓人}{xia4/ren2}{6,2}{⼝,⼈}
  \definition{adj.}{apavorante | assustador}
  \definition{v.+compl.}{assustar-se | tomar um susto}
\end{EntryWithPhonetic}

%%%%%%%%%% 夏 %%%%%%%%%%
\subsection*{夏}\addcontentsline{loh}{figure}{夏 \dpy{xia4}}

\begin{EntryWithPhonetic}{夏}{xia4}{10}{⼢}
  \definition*{s.}{Dinastia Xia (2070-1600 a.C.) | China; refere-se à China | Sobrenome: Xia}
  \definition{s.}{verão}
\end{EntryWithPhonetic}

\begin{EntryWithPhonetic}{夏季}{xia4 ji4}{10,8}{⼢,⼦}[HSK 4]
  \definition[个]{s.}{verão; segundo trimestre do ano, habitualmente chamado na China de período de três meses, do início do verão ao início do outono, também chamado de ``quarto, quinto e sexto'' meses do calendário lunar}
\end{EntryWithPhonetic}

\begin{EntryWithPhonetic}{夏日}{xia4ri4}{10,4}{⼢,⽇}
  \definition{s.}{horário de verão}
\end{EntryWithPhonetic}

\begin{EntryWithPhonetic}{夏天}{xia4 tian1}{10,4}{⼢,⼤}[HSK 2]
  \definition[个]{s.}{verão}
\end{EntryWithPhonetic}

%%%%%%%%%% 仙 %%%%%%%%%%
\subsection*{仙}\addcontentsline{loh}{figure}{仙 \dpy{xian1}}

\begin{EntryWithPhonetic}{仙}{xian1}{5}{⼈}
  \definition{s.}{imortal}
\end{EntryWithPhonetic}

%%%%%%%%%% 先 %%%%%%%%%%
\subsection*{先}\addcontentsline{loh}{figure}{先 \dpy{xian1}}

\begin{EntryWithPhonetic}{先}{xian1}{6}{⼉}[HSK 1]
  \definition*{s.}{Sobrenome: Xian}
  \definition{adv.}{primeiro; antes; mais cedo; com antecedência | no momento; por enquanto; em um curto espaço de tempo; temporariamente}
  \definition{s.}{início; começo; em ordem cronológica ou de precedência | ancestral; geração mais velha; antepassado | tardio; falecido; morto (honrar os mortos)}
\end{EntryWithPhonetic}

\begin{EntryWithPhonetic}{先不先}{xian1bu4xian1}{6,4,6}{⼉,⼀,⼉}
  \definition{adv.}{(dialeto) antes de tudo | em primeiro lugar}
\end{EntryWithPhonetic}

\begin{EntryWithPhonetic}{先到先得}{xian1dao4xian1de2}{6,8,6,11}{⼉,⼑,⼉,⼻}
  \definition{expr.}{primeiro a chegar | primeiro a ser servido}
\end{EntryWithPhonetic}

\begin{EntryWithPhonetic}{先锋}{xian1 feng1}{6,12}{⼉,⾦}[HSK 6]
  \definition{s.}{pioneiro; vanguarda; a vanguarda de uma batalha ou marcha; geralmente se refere a uma pessoa ou grupo que desempenha um papel de vanguarda}
\end{EntryWithPhonetic}

\begin{EntryWithPhonetic}{先后}{xian1 hou4}{6,6}{⼉,⼝}[HSK 5]
  \definition{adv.}{sucessivamente; um após o outro}
  \definition{s.}{prioridade; ordem; cedo ou tarde; primeiro e último}
\end{EntryWithPhonetic}

\begin{EntryWithPhonetic}{先进}{xian1jin4}{6,7}{⼉,⾡}[HSK 3]
  \definition{adj.}{avançado; progressos rápidos e nível elevado, podendo servir de exemplo a seguir}
  \definition{s.}{indivíduos ou grupos avançados}
\end{EntryWithPhonetic}

\begin{EntryWithPhonetic}{先烈}{xian1lie4}{6,10}{⼉,⽕}
  \definition{s.}{mártir}
\end{EntryWithPhonetic}

\begin{EntryWithPhonetic}{先期}{xian1qi1}{6,12}{⼉,⽉}
  \definition{adv.}{antecipadamente}
  \definition{s.}{prematuro | \emph{front-end}}
\end{EntryWithPhonetic}

\begin{EntryWithPhonetic}{先前}{xian1qian2}{6,9}{⼉,⼑}[HSK 5]
  \definition[出]{s.}{antes; anteriormente; refere-se ao passado ou a um certo tempo anterior}
\end{EntryWithPhonetic}

\begin{EntryWithPhonetic}{先生}{xian1sheng5}{6,5}{⼉,⽣}[HSK 1]
  \definition[个,位]{s.}{professor; títulos honoríficos para professores, médicos, etc. | marido; antigamente, referia-se ao marido de outra pessoa ou ao próprio marido (ambos com pronomes pessoais como determinantes) | médico; títulos usados para se referir aos médicos no passado | refere-se a pessoas cuja profissão envolve contar histórias, adivinhação, etc.; antigamente, era chamado de contador | senhor; \emph{sir}; título dado aos intelectuais}
\end{EntryWithPhonetic}

\begin{EntryWithPhonetic}{先天}{xian1tian1}{6,4}{⼉,⼤}
  \definition{adj.}{congênito | inato | natural}
  \definition{s.}{período embrionário}
\end{EntryWithPhonetic}

\begin{EntryWithPhonetic}{先验}{xian1yan4}{6,10}{⼉,⾺}
  \definition{adj.}{(filosofia) a priori}
\end{EntryWithPhonetic}

\begin{EntryWithPhonetic}{先有}{xian1you3}{6,6}{⼉,⽉}
  \definition{adj.}{preexistente | anterior}
\end{EntryWithPhonetic}

%%%%%%%%%% 鲜 %%%%%%%%%%
\subsection*{鲜}\addcontentsline{loh}{figure}{鲜 \dpy{xian1}}

\begin{EntryWithPhonetic}{鲜}{xian1}{14}{⿂}[HSK 4]
  \definition*{s.}{Sobrenome: Xian}
  \definition{adj.}{fresco; novo; fresco (experiência, comida etc.) |brilhante; de cores vivas | saboroso; delicioso | exuberante; luxuriante}
  \definition{s.}{aves e animais recém-abatidos; vegetais recém-colhidos; frutas, etc. | alimentos aquáticos; geralmente, peixes vivos, camarões, etc., para alimentação}
  \seeref{xian3}
\end{EntryWithPhonetic}

\begin{EntryWithPhonetic}{鲜花}{xian1 hua1}{14,7}{⿂,⾋}[HSK 4]
  \definition[朵,束,支]{s.}{flor; flores frescas; flores bonitas e frescas}
\end{EntryWithPhonetic}

\begin{EntryWithPhonetic}{鲜明}{xian1ming2}{14,8}{⿂,⽇}[HSK 4]
  \definition{adj.}{brilhante (cor) | distinto; bem definido; nítido; claro; característico}
\end{EntryWithPhonetic}

\begin{EntryWithPhonetic}{鲜艳}{xian1yan4}{14,10}{⿂,⾊}[HSK 5]
  \definition{adj.}{de cores alegres; de cores brilhantes}
\end{EntryWithPhonetic}

%%%%%%%%%% 闲 %%%%%%%%%%
\subsection*{闲}\addcontentsline{loh}{figure}{闲 \dpy{xian2}}

\begin{EntryWithPhonetic}{闲}{xian2}{7}{⾨}[HSK 5]
  \definition{adj.}{ocioso; não ocupado; desocupado; sem coisas para fazer; sem atividades; tempo livre | desocupado; (casa, objeto, etc.) não em uso; ocioso | não oficial; não sério; não relacionado ao negócio}
  \definition{s.}{lazer; tempo livre}
\end{EntryWithPhonetic}

%%%%%%%%%% 咸 %%%%%%%%%%
\subsection*{咸}\addcontentsline{loh}{figure}{咸 \dpy{xian2}}

\begin{EntryWithPhonetic}{咸}{xian2}{9}{⼝}[HSK 4]
  \definition*{s.}{Sobrenome: Xian}
  \definition{adj.}{salgado; em conserva; sabor salgado}
  \definition{adv.}{todos; indica a totalidade de um intervalo, equivalente a 全 e 都}
  \seealsoref{都}{dou1}
  \seealsoref{全}{quan2}
\end{EntryWithPhonetic}

\begin{EntryWithPhonetic}{咸菜}{xian2cai4}{9,11}{⼝,⾋}
  \definition{s.}{legumes salgados | \emph{pickles}}
\end{EntryWithPhonetic}

\begin{EntryWithPhonetic}{咸淡}{xian2dan4}{9,11}{⼝,⽔}
  \definition{s.}{água salobra | grau de salinidade | salgado e sem sal (sabores)}
\end{EntryWithPhonetic}

\begin{EntryWithPhonetic}{咸肉}{xian2rou4}{9,6}{⼝,⾁}
  \definition{s.}{\emph{bacon} | carne curada com sal}
\end{EntryWithPhonetic}

\begin{EntryWithPhonetic}{咸涩}{xian2se4}{9,10}{⼝,⽔}
  \definition{s.}{ácido | salgado e amargo}
\end{EntryWithPhonetic}

\begin{EntryWithPhonetic}{咸水}{xian2shui3}{9,4}{⼝,⽔}
  \definition{s.}{salmora | água salgada}
\end{EntryWithPhonetic}

\begin{EntryWithPhonetic}{咸盐}{xian2yan2}{9,10}{⼝,⽫}
  \definition{s.}{(coloquial) sal | sal de mesa}
\end{EntryWithPhonetic}

\begin{EntryWithPhonetic}{咸鱼}{xian2yu2}{9,8}{⼝,⿂}
  \definition{s.}{peixe salgado}
\end{EntryWithPhonetic}

%%%%%%%%%% 嫌 %%%%%%%%%%
\subsection*{嫌}\addcontentsline{loh}{figure}{嫌 \dpy{xian2}}

\begin{EntryWithPhonetic}{嫌}{xian2}{13}{⼥}[HSK 6]
  \definition{s.}{suspeita; suspeição; cisma | inimizade; rancor; má vontade; ressentimento}
  \definition{v.}{importar-se com; não gostar e evitar; reclamar de}
\end{EntryWithPhonetic}

%%%%%%%%%% 显 %%%%%%%%%%
\subsection*{显}\addcontentsline{loh}{figure}{显 \dpy{xian3}}

\begin{EntryWithPhonetic}{显}{xian3}{9}{⽇}[HSK 5]
  \definition*{s.}{Sobrenome: Xian}
  \definition{adj.}{aparente; óbvio; perceptível | ilustre e influente | evidente; óbvio}
  \definition{v.}{mostrar; exibir; manifestar | aparecer; mostrar; revelar}
\end{EntryWithPhonetic}

\begin{EntryWithPhonetic}{显出}{xian3 chu1}{9,5}{⽇,⼐}[HSK 6]
  \definition{v.}{mostrar; revelar | dar provas; expressar; exibir}
\end{EntryWithPhonetic}

\begin{EntryWithPhonetic}{显得}{xian3de5}{9,11}{⽇,⼻}[HSK 3]
  \definition{v.}{parecer; aparecer; manifestar (alguma situação)}
\end{EntryWithPhonetic}

\begin{EntryWithPhonetic}{显然}{xian3ran2}{9,12}{⽇,⽕}[HSK 3]
  \definition{adj.}{claro; evidente; óbvio; fatos, verdades e outras coisas que são fáceis de descobrir, perceber ou sentir claramente}
\end{EntryWithPhonetic}

\begin{EntryWithPhonetic}{显示}{xian3shi4}{9,5}{⽇,⽰}[HSK 3]
  \definition{v.}{mostrar; manifestar-se claramente | exibir; ostentar}
\end{EntryWithPhonetic}

\begin{EntryWithPhonetic}{显著}{xian3zhu4}{9,11}{⽇,⽬}[HSK 4]
  \definition{adj.}{notável; significativo; notável; extraordinário; muito óbvio; muito claramente demonstrado; muito facilmente visto ou sentido}
\end{EntryWithPhonetic}

%%%%%%%%%% 险 %%%%%%%%%%
\subsection*{险}\addcontentsline{loh}{figure}{险 \dpy{xian3}}

\begin{EntryWithPhonetic}{险}{xian3}{9}{⾩}[HSK 6]
  \definition{adj.}{perigoso; arriscado | sinistro; cruel; venenoso}
  \definition{adv.}{por um fio de cabelo; por centímetros; quase}
  \definition{s.}{lugar de difícil acesso; lugar perigoso e difícil de atravessar; passagem estreita; desfiladeiro | abreviação de seguro, 保险 | perigo; risco}
  \seealsoref{保险}{bao3xian3}
\end{EntryWithPhonetic}

%%%%%%%%%% 猃 %%%%%%%%%%
\subsection*{猃}\addcontentsline{loh}{figure}{猃 \dpy{xian3}}

\begin{EntryWithPhonetic}{猃}{xian3}{10}{⽝}
  \definition{s.}{(arcaico) um tipo de cão com focinho longo}
\end{EntryWithPhonetic}

\begin{EntryWithPhonetic}{猃狁}{xian3yun3}{10,7}{⽝,⽝}
  \definition*{s.}{Termo da dinastia Zhou para uma tribo nômade do norte mais tarde chamou o Xiongnu (匈奴) nas dinastias Qin e Han}
  \seealsoref{匈奴}{xiong1nu2}
\end{EntryWithPhonetic}

%%%%%%%%%% 鲜 %%%%%%%%%%
\subsection*{鲜}\addcontentsline{loh}{figure}{鲜 \dpy{xian3}}

\begin{EntryWithPhonetic}{鲜}{xian3}{14}{⿂}
  \definition{adj.}{raro; pouco; pequeno}
  \definition{adv.}{raramente}
  \seeref{xian1}
\end{EntryWithPhonetic}

%%%%%%%%%% 见 %%%%%%%%%%
\subsection*{见}\addcontentsline{loh}{figure}{见 \dpy{xian4}}

\begin{EntryWithPhonetic}{见}{xian4}{4}{⾒}[Kangxi 147]
  \definition{v.}{aparecer; também escrito como 现}
  \seeref{jian4}
  \seealsoref{现}{xian4}
\end{EntryWithPhonetic}

%%%%%%%%%% 县 %%%%%%%%%%
\subsection*{县}\addcontentsline{loh}{figure}{县 \dpy{xian4}}

\begin{EntryWithPhonetic}{县}{xian4}{7}{⼛}[HSK 4]
  \definition[个]{s.}{condado; unidade de divisão administrativa}
\end{EntryWithPhonetic}

%%%%%%%%%% 现 %%%%%%%%%%
\subsection*{现}\addcontentsline{loh}{figure}{现 \dpy{xian4}}

\begin{EntryWithPhonetic}{现}{xian4}{8}{⾒}
  \definition{adj.}{(dinheiro, etc.) em mãos}
  \definition{adv.}{recente; de improviso; naquela época; temporariamente}
  \definition{s.}{presente; atual; existente | dinheiro; dinheiro de pronto}
  \definition{v.}{mostrar; revelar; aparecer; tornar-se visível}
  \seealsoref{见}{xian4}
\end{EntryWithPhonetic}

\begin{EntryWithPhonetic}{现场}{xian4chang3}{8,6}{⾒,⼟}[HSK 3]
  \definition[个,处]{s.}{local onde ocorreu o acidente, incidente ou desastre | local; ponto; local onde se realizam diretamente atividades como produção, apresentações e competições}
\end{EntryWithPhonetic}

\begin{EntryWithPhonetic}{现代}{xian4dai4}{8,5}{⾒,⼈}[HSK 3]
  \definition*{s.}{Hyundai, empresa sul-coreana}
  \definition{adj.}{moderno; contemporâneo; com características, estilo e conceitos modernos, refletindo a vanguarda, a moda e a inovação da atualidade}
  \definition{s.}{tempos modernos; era contemporânea; atualmente, na divisão cronológica da história da China, refere-se principalmente ao período desde o Movimento 4 de Maio até os dias atuais}
\end{EntryWithPhonetic}

\begin{EntryWithPhonetic}{现货}{xian4huo4}{8,8}{⾒,⾙}
  \definition{s.}{produtos à vista}
\end{EntryWithPhonetic}

\begin{EntryWithPhonetic}{现货的}{xian4huo4 de5}{8,8,8}{⾒,⾙,⽩}
  \definition{s.}{produtos em estoque}
\end{EntryWithPhonetic}

\begin{EntryWithPhonetic}{现金}{xian4jin1}{8,8}{⾒,⾦}[HSK 3]
  \definition[笔]{s.}{dinheiro; dinheiro vivo; moeda que pode ser usada diretamente | reserva de dinheiro em um banco; o dinheiro guardado no cofre do banco}
\end{EntryWithPhonetic}

\begin{EntryWithPhonetic}{现实}{xian4shi2}{8,8}{⾒,⼧}[HSK 3]
  \definition{adj.}{real; efetivo; verdadeiro; de acordo com circunstâncias objetivas}
  \definition[个]{s.}{realidade; factualidade; coisas que existem objetivamente}
\end{EntryWithPhonetic}

\begin{EntryWithPhonetic}{现象}{xian4xiang4}{8,11}{⾒,⾗}[HSK 3]
  \definition[个,种]{s.}{aparência (das coisas); fenômeno; a forma externa e as relações manifestadas pelas coisas em seu desenvolvimento e mudança}
\end{EntryWithPhonetic}

\begin{EntryWithPhonetic}{现有}{xian4 you3}{8,6}{⾒,⽉}[HSK 5]
  \definition{adj.}{agora disponível; existente}
  \definition{v.}{estar disponível agora; existir | Literário: ter em mãos; ter em posse}
\end{EntryWithPhonetic}

\begin{EntryWithPhonetic}{现在}{xian4zai4}{8,6}{⾒,⼟}[HSK 1]
  \definition{adv.}{agora; no momento; atualmente; neste momento, quando se fala, às vezes inclui um período de tempo mais ou menos longo antes ou depois da fala (diferente de 过去 ou 将来)}
  \seealsoref{过去}{guo4 qu4}
  \seealsoref{将来}{jiang1lai2}
\end{EntryWithPhonetic}

\begin{EntryWithPhonetic}{现抓}{xian4zhua1}{8,7}{⾒,⼿}
  \definition{v.}{improvisar}
\end{EntryWithPhonetic}

\begin{EntryWithPhonetic}{现状}{xian4zhuang4}{8,7}{⾒,⽝}[HSK 5]
  \definition{s.}{situação atual}
\end{EntryWithPhonetic}

\begin{EntryWithPhonetic}{现做}{xian4zuo4}{8,11}{⾒,⼈}
  \definition{adj.}{fresco}
  \definition{v.}{fazer (comida) no local}
\end{EntryWithPhonetic}

%%%%%%%%%% 线 %%%%%%%%%%
\subsection*{线}\addcontentsline{loh}{figure}{线 \dpy{xian4}}

\begin{EntryWithPhonetic}{线}{xian4}{8}{⽷}[HSK 3]
  \definition{clas.}{usado para coisas abstratas, o número é limitado a ``一'' (1)}
  \definition[根,个]{s.}{fio; corda; arame; objetos finos e longos feitos de seda, algodão, metal, etc. | linha; figura formada pelo movimento arbitrário de um ponto| feito de fio de algodão | algo em forma de linha, fio, etc. | rota de transporte; linha | linha de demarcação; limite; zona de fronteira; zona de transição | beira; borda | linha ideológica e política | pista; fio}
\end{EntryWithPhonetic}

\begin{EntryWithPhonetic}{线路}{xian4 lu4}{8,13}{⽷,⾜}[HSK 6]
  \definition[条]{s.}{linha; rota; as rotas que os veículos de transporte percorrem, etc., que as pessoas podem usar para chegar aos seus destinos | Eletricidade: linha; circuito; a rota da corrente elétrica}
\end{EntryWithPhonetic}

\begin{EntryWithPhonetic}{线索}{xian4suo3}{8,10}{⽷,⽷}[HSK 5]
  \definition[条,个]{s.}{pista; fio; metáfora para o desenvolvimento das coisas ou a maneira de explorar um problema | fio; linha; refere-se ao contexto de desenvolvimento do enredo em obras literárias}
\end{EntryWithPhonetic}

\begin{EntryWithPhonetic}{线香}{xian4xiang1}{8,9}{⽷,⾹}
  \definition{s.}{bastão ou vareta de incenso}
\end{EntryWithPhonetic}

%%%%%%%%%% 限 %%%%%%%%%%
\subsection*{限}\addcontentsline{loh}{figure}{限 \dpy{xian4}}

\begin{EntryWithPhonetic}{限}{xian4}{8}{⾩}
  \definition{s.}{limite | limiar}
  \definition{v.}{definir um limite; limitar; restringir}
\end{EntryWithPhonetic}

\begin{EntryWithPhonetic}{限制}{xian4zhi4}{8,8}{⾩,⼑}[HSK 4]
  \definition[些]{s.}{limite; restrição; confinamento}
  \definition{v.}{limitar; adstringir; restringir; confinar; fechar em (sobre)}
\end{EntryWithPhonetic}

%%%%%%%%%% 宪 %%%%%%%%%%
\subsection*{宪}\addcontentsline{loh}{figure}{宪 \dpy{xian4}}

\begin{EntryWithPhonetic}{宪}{xian4}{9}{⼧}
  \definition*{s.}{Sobrenome: Xian}
  \definition{s.}{estatuto; decreto | constituição}
\end{EntryWithPhonetic}

\begin{EntryWithPhonetic}{宪法法院}{xian4fa3fa3yuan4}{9,8,8,9}{⼧,⽔,⽔,⾩}
  \definition{s.}{tribunal constitucional}
\end{EntryWithPhonetic}

\begin{EntryWithPhonetic}{宪政}{xian4zheng4}{9,9}{⼧,⽁}
  \definition{s.}{governo constitucional}
\end{EntryWithPhonetic}

\begin{EntryWithPhonetic}{宪制}{xian4zhi4}{9,8}{⼧,⼑}
  \definition{adj.}{constitucional}
  \definition{s.}{sistema de governo constitucional}
\end{EntryWithPhonetic}

%%%%%%%%%% 陷 %%%%%%%%%%
\subsection*{陷}\addcontentsline{loh}{figure}{陷 \dpy{xian4}}

\begin{EntryWithPhonetic}{陷}{xian4}{10}{⾩}
  \definition[个]{s.}{armadilha; cilada | defeito | deficiência; desvantagem}
  \definition{v.}{ficar preso (ou atolado); enredar | afundar; desabar | acusar falsamente; incriminar; armar | (de uma cidade, etc.) ser capturado; cair | ser enquadrado; ser capturado}
\end{EntryWithPhonetic}

\begin{EntryWithPhonetic}{陷入}{xian4ru4}{10,2}{⾩,⼊}[HSK 6]
  \definition{v.}{afundar em; cair em; cair em uma situação desfavorável | estar perdido em; estar profundamente em; estar imerso em; metaforicamente, estar profundamente imerso em (uma situação ou pensamento) | estar atolado (lama fofa, areia, etc.)}
\end{EntryWithPhonetic}

%%%%%%%%%% 羡 %%%%%%%%%%
\subsection*{羡}\addcontentsline{loh}{figure}{羡 \dpy{xian4}}

\begin{EntryWithPhonetic}{羡}{xian4}{12}{⽺}
  \definition{v.}{admirar; invejar}
\end{EntryWithPhonetic}

\begin{EntryWithPhonetic}{羡慕}{xian4mu4}{12,14}{⽺,⼼}
  \definition{v.}{invejar; admirar; ver os outros terem certos pontos fortes ou vantagens e desejar tê-los também}
\end{EntryWithPhonetic}

%%%%%%%%%% 献 %%%%%%%%%%
\subsection*{献}\addcontentsline{loh}{figure}{献 \dpy{xian4}}

\begin{EntryWithPhonetic}{献}{xian4}{13}{⽝}[HSK 5]
  \definition{v.}{oferecer; apresentar; dedicar; doar | mostrar; apresentar; exibir | exibir-se; mostrar-se para que os outros vejam}
\end{EntryWithPhonetic}

%%%%%%%%%% 乡 %%%%%%%%%%
\subsection*{乡}\addcontentsline{loh}{figure}{乡 \dpy{xiang1}}

\begin{EntryWithPhonetic}{乡}{xiang1}{3}{⼄}[HSK 5]
  \definition[个,座,片]{s.}{país; campo; vilarejo; área rural | local de origem; vila ou cidade natal | município (uma unidade administrativa rural subordinada ao condado) | vila natal; cidade natal | terra ou local famoso por produzir algo}
\end{EntryWithPhonetic}

\begin{EntryWithPhonetic}{乡巴佬}{xiang1ba1lao3}{3,4,8}{⼄,⼰,⼈}
  \definition{s.}{aldeão | caipira}
\end{EntryWithPhonetic}

\begin{EntryWithPhonetic}{乡村}{xiang1 cun1}{3,7}{⼄,⽊}[HSK 5]
  \definition{adj.}{rural | rústico}
  \definition[个]{s.}{vila; campo; área rural; principalmente envolvido na agricultura; áreas com distribuição populacional mais dispersa em relação às cidades}
\end{EntryWithPhonetic}

%%%%%%%%%% 相 %%%%%%%%%%
\subsection*{相}\addcontentsline{loh}{figure}{相 \dpy{xiang1}}

\begin{EntryWithPhonetic}{相}{xiang1}{9}{⽬}
  \definition*{s.}{Sobrenome: Xiang}
  \definition{adv.}{uns aos outros; mutuamente | (para uma ação realizada por uma pessoa em relação a outra) | indica a ação de uma parte em relação à outra parte}
  \definition{s.}{qualidade; substância}
  \definition{v.}{ver por si mesmo (se algo ou algo é do seu agrado)}
  \seeref{xiang4}
\end{EntryWithPhonetic}

\begin{EntryWithPhonetic}{相比}{xiang1 bi3}{9,4}{⽬,⽐}[HSK 3]
  \definition{v.}{combinar; comparar com | comparar mutuamente, usar uma coisa como padrão, perceber as características de outra coisa ou obter uma opinião}
\end{EntryWithPhonetic}

\begin{EntryWithPhonetic}{相处}{xiang1chu3}{9,5}{⽬,⼡}[HSK 4]
  \definition{v.}{dar-se bem; viver juntos; dar-se bem (uns com os outros); viver uns com os outros; entrar em contato uns com os outros, tratar uns aos outros}
\end{EntryWithPhonetic}

\begin{EntryWithPhonetic}{相当}{xiang1dang1}{9,6}{⽬,⼹}[HSK 3]
  \definition{adj.}{adequado; apropriado}
  \definition{adv.}{bastante; razoavelmente; consideravelmente; indica um grau relativamente alto e profundo}
  \definition{v.}{combinar; equilibrar; corresponder a; ser aproximadamente igual a; ser proporcional a}
\end{EntryWithPhonetic}

\begin{EntryWithPhonetic}{相等}{xiang1deng3}{9,12}{⽬,⽵}[HSK 5]
  \definition{v.}{ser igual a; possuir a mesma quantidade, peso, tamanho e grau}
\end{EntryWithPhonetic}

\begin{EntryWithPhonetic}{相反}{xiang1fan3}{9,4}{⽬,⼜}[HSK 4]
  \definition{adj.}{oposto; contrário; dois aspectos das coisas são contraditórios e mutuamente exclusivos}
  \definition{conj.}{pelo contrário; usado no início ou no meio de uma frase para indicar uma contradição de significado com o que foi dito anteriormente.}
\end{EntryWithPhonetic}

\begin{EntryWithPhonetic}{相关}{xiang1guan1}{9,6}{⽬,⼋}[HSK 3]
  \definition{v.}{estar mutuamente relacionado; estar intimamente relacionado; estar inter-relacionado}
\end{EntryWithPhonetic}

\begin{EntryWithPhonetic}{相互}{xiang1 hu4}{9,4}{⽬,⼆}[HSK 3]
  \definition{adj.}{mútuo; recíproco; entre duas pessoas ou coisas}
  \definition{adv.}{mutuamente; um ao outro; tratamento recíproco}
\end{EntryWithPhonetic}

\begin{EntryWithPhonetic}{相聚}{xiang1ju4}{9,14}{⽬,⽿}
  \definition{v.}{reunir-se | montar}
\end{EntryWithPhonetic}

\begin{EntryWithPhonetic}{相亲}{xiang1qin1}{9,9}{⽬,⼇}
  \definition{s.}{encontro às cegas | entrevista arranjada para avaliar a proposta de um parceiro de casamento | apegar-se profundamente um ao outro}
\end{EntryWithPhonetic}

\begin{EntryWithPhonetic}{相思病}{xiang1si1bing4}{9,9,10}{⽬,⼼,⽧}
  \definition{s.}{saudade de amor}
\end{EntryWithPhonetic}

\begin{EntryWithPhonetic}{相似}{xiang1si4}{9,6}{⽬,⼈}[HSK 3]
  \definition{v.}{assemelhar-se; ser semelhante; ser parecido}
\end{EntryWithPhonetic}

\begin{EntryWithPhonetic}{相同}{xiang1tong2}{9,6}{⽬,⼝}[HSK 2]
  \definition{adj.}{semelhante; similar; igual; idêntico; o mesmo; consistentes entre si, sem diferença}
\end{EntryWithPhonetic}

\begin{EntryWithPhonetic}{相信}{xiang1xin4}{9,9}{⽬,⼈}[HSK 2]
  \definition{v.}{acreditar em; estar convencido de; ter fé em; acreditar que algo é certo ou verdadeiro sem dúvida}
\end{EntryWithPhonetic}

\begin{EntryWithPhonetic}{相宜}{xiang1yi2}{9,8}{⽬,⼧}
  \definition{adj.}{adequado | apropriado}
  \definition{v.}{ser adequado ou apropriado}
\end{EntryWithPhonetic}

\begin{EntryWithPhonetic}{相应}{xiang1ying4}{9,7}{⽬,⼴}[HSK 5]
  \definition{adj.}{Dialeto: barato}
  \definition{v.}{corresponder}
\end{EntryWithPhonetic}

\begin{EntryWithPhonetic}{相遇}{xiang1yu4}{9,12}{⽬,⾡}
  \definition{v.}{encontrar (reunião, encontro, etc.)}
\end{EntryWithPhonetic}

%%%%%%%%%% 香 %%%%%%%%%%
\subsection*{香}\addcontentsline{loh}{figure}{香 \dpy{xiang1}}

\begin{EntryWithPhonetic}{香}{xiang1}{9}{⾹}[HSK 3][Kangxi 186]
  \definition*{s.}{Sobrenome: Xiang}
  \definition{adj.}{aromático; perfumado; fragrante; cheiroso; oposto a 臭 | saboroso; saboroso; delicioso; apetitoso | com gosto; com bom apetite | (sono) profundo; dormir confortavelmente e tranquilamente | popular; valorizado; apreciado}
  \definition[根,炷]{s.}{especiaria; perfume; fragrância; aromatizante; substância com aroma intenso | incenso; bastão de incenso; tiras finas feitas de serragem e especiarias, queimadas em rituais para honrar os antepassados ou deuses e budas, e também usadas para afastar odores desagradáveis ou mosquitos| antigamente, referia-se a coisas relacionadas com mulheres ou mulheres}
  \seealsoref{臭}{chou4}
\end{EntryWithPhonetic}

\begin{EntryWithPhonetic}{香槟酒}{xiang1bin1jiu3}{9,14,10}{⾹,⽊,⾣}
  \definition[杯]{s.}{(empréstimo linguístico) \emph{champagne}}
\end{EntryWithPhonetic}

\begin{EntryWithPhonetic}{香波}{xiang1bo1}{9,8}{⾹,⽔}
  \definition{s.}{xampu}
\end{EntryWithPhonetic}

\begin{EntryWithPhonetic}{香肠}{xiang1chang2}{9,7}{⾹,⾁}[HSK 5]
  \definition[根]{s.}{salsicha; linguiça; alimento feito com intestino de porco, recheado com carne picada e temperos}
\end{EntryWithPhonetic}

\begin{EntryWithPhonetic}{香港}{xiang1gang3}{9,12}{⾹,⽔}
  \definition*{s.}{Hong Kong}
  \seealsoref{香港岛}{xiang1gang3 dao3}
\end{EntryWithPhonetic}

\begin{EntryWithPhonetic}{香港岛}{xiang1gang3 dao3}{9,12,7}{⾹,⽔,⼭}
  \definition*{s.}{Ilha de Hong Kong}
  \seealsoref{香港}{xiang1gang3}
\end{EntryWithPhonetic}

\begin{EntryWithPhonetic}{香蕉}{xiang1jiao1}{9,15}{⾹,⾋}[HSK 3]
  \definition[枝,根,个,把,串,束,弓]{s.}{banana}
\end{EntryWithPhonetic}

\begin{EntryWithPhonetic}{香炉}{xiang1lu2}{9,8}{⾹,⽕}
  \definition{s.}{incensário (para queimar incenso) | queimador de incenso | insensório, turíbulo}
\end{EntryWithPhonetic}

\begin{EntryWithPhonetic}{香气}{xiang1qi4}{9,4}{⾹,⽓}
  \definition{s.}{fragrância | aroma | incenso}
\end{EntryWithPhonetic}

\begin{EntryWithPhonetic}{香味}{xiang1wei4}{9,8}{⾹,⼝}
  \definition[股]{s.}{fragrância | cheiro doce}
\end{EntryWithPhonetic}

\begin{EntryWithPhonetic}{香蕈}{xiang1xun4}{9,15}{⾹,⾋}
  \definition{s.}{\emph{shiitake}, cogumelo comestível}
\end{EntryWithPhonetic}

\begin{EntryWithPhonetic}{香烟}{xiang1yan1}{9,10}{⾹,⽕}
  \definition[支,条]{s.}{cigarro | fumaça de incenso queimado}
\end{EntryWithPhonetic}

\begin{EntryWithPhonetic}{香艳}{xiang1yan4}{9,10}{⾹,⾊}
  \definition{adj.}{atraente | erótico | romântico}
\end{EntryWithPhonetic}

\begin{EntryWithPhonetic}{香皂}{xiang1zao4}{9,7}{⾹,⽩}
  \definition{s.}{sabonete | sabonete perfumado}
\end{EntryWithPhonetic}

%%%%%%%%%% 箱 %%%%%%%%%%
\subsection*{箱}\addcontentsline{loh}{figure}{箱 \dpy{xiang1}}

\begin{EntryWithPhonetic}{箱}{xiang1}{15}{⾋}[HSK 4]
  \definition{s.}{caixa; estojo; baú | qualquer coisa no formato de caixa}
\end{EntryWithPhonetic}

\begin{EntryWithPhonetic}{箱子}{xiang1 zi5}{15,3}{⾋,⼦}[HSK 4]
  \definition[个,只]{s.}{baú; caixa; estojo; maleta; pasta executiva}
\end{EntryWithPhonetic}

%%%%%%%%%% 镶 %%%%%%%%%%
\subsection*{镶}\addcontentsline{loh}{figure}{镶 \dpy{xiang1}}

\begin{EntryWithPhonetic}{镶}{xiang1}{22}{⾦}
  \definition{v.}{para incrustar; cravar; montar | marginar; orlar; rodear; decorar; colocar uma moldura | inserir; integrar}
\end{EntryWithPhonetic}

%%%%%%%%%% 详 %%%%%%%%%%
\subsection*{详}\addcontentsline{loh}{figure}{详 \dpy{xiang2}}

\begin{EntryWithPhonetic}{详}{xiang2}{8}{⾔}
  \definition{adj.}{conhecido; reconhecido; saber claramente | detalhado; minucioso; pormenorizado (oposto a 略)}
  \definition{s.}{detalhes; particularidades}
  \definition{v.}{contar; explicar; elaborar | saber claramente}
  \seealsoref{略}{lve4}
\end{EntryWithPhonetic}

\begin{EntryWithPhonetic}{详细}{xiang2xi4}{8,8}{⾔,⽷}[HSK 5]
  \definition{adj.}{explícito; detalhado; minucioso; circunstancial; meticuloso}
\end{EntryWithPhonetic}

%%%%%%%%%% 享 %%%%%%%%%%
\subsection*{享}\addcontentsline{loh}{figure}{享 \dpy{xiang3}}

\begin{EntryWithPhonetic}{享}{xiang3}{8}{⼇}
  \definition{v.}{aproveitar}
\end{EntryWithPhonetic}

\begin{EntryWithPhonetic}{享受}{xiang3shou4}{8,8}{⼇,⼜}[HSK 5]
  \definition{v.}{aproveitar; desfrutar; estar satisfeito material ou espiritualmente}
\end{EntryWithPhonetic}

%%%%%%%%%% 响 %%%%%%%%%%
\subsection*{响}\addcontentsline{loh}{figure}{响 \dpy{xiang3}}

\begin{EntryWithPhonetic}{响}{xiang3}{9}{⼝}[HSK 2]
  \definition{adj.}{barulhento; ressonante}
  \definition[声,阵]{s.}{som; ruído; barulho | eco}
  \definition{v.}{tocar; soar; ressoar; fazer um som | soar; fazer algo emitir um som}
\end{EntryWithPhonetic}

\begin{EntryWithPhonetic}{响亮}{xiang3liang4}{9,9}{⼝,⼇}
  \definition{adj.}{vibrante; ressonante; sonoro; ressonante; alto e claro}
\end{EntryWithPhonetic}

\begin{EntryWithPhonetic}{响声}{xiang3 sheng1}{9,7}{⼝,⼠}[HSK 6]
  \definition{s.}{som; ruído}
\end{EntryWithPhonetic}

%%%%%%%%%% 想 %%%%%%%%%%
\subsection*{想}\addcontentsline{loh}{figure}{想 \dpy{xiang3}}

\begin{EntryWithPhonetic}{想}{xiang3}{13}{⼼}[HSK 1]
  \definition{v.}{pensar; ponderar; refletir | supor; contar; considerar; pensar; estimar | querer; gostaria de; sentir vontade (de fazer algo) | lembrar com saudade; sentir falta}
\end{EntryWithPhonetic}

\begin{EntryWithPhonetic}{想不到}{xiang3 bu2 dao4}{13,4,8}{⼼,⼀,⼑}[HSK 6]
  \definition{adj.}{inesperado; imprevisto}
\end{EntryWithPhonetic}

\begin{EntryWithPhonetic}{想到}{xiang3 dao4}{13,8}{⼼,⼑}[HSK 2]
  \definition{v.}{pensar em; trazer à mente; ter no coração; ter uma ideia (na mente); ter uma ideia (no coração)}
\end{EntryWithPhonetic}

\begin{EntryWithPhonetic}{想法}{xiang3 fa3}{13,8}{⼼,⽔}[HSK 2]
  \definition[种]{s.}{ideia; opinião; pensamento; noção; o que alguém tem em mente; visões e opiniões sobre alguém ou algo obtidas através do pensamento}
  \definition{s.}{maneira de pensar | opinião | noção}
  \definition{v.}{tentar; pensar em uma maneira (de fazer algo); fazer o que puder; encontrar um jeito}
\end{EntryWithPhonetic}

\begin{EntryWithPhonetic}{想念}{xiang3nian4}{13,8}{⼼,⼼}[HSK 4]
  \definition{v.}{sentir falta; pensar em; lembrar com carinho; ficar doente por; desejar ver novamente; lembrar com saudade}
\end{EntryWithPhonetic}

\begin{EntryWithPhonetic}{想起}{xiang3 qi3}{13,10}{⼼,⾛}[HSK 2]
  \definition{v.}{recordar; lembrar; pensar em; trazer à mente; cruzar pelos pensamentos de alguém; passar pelo pensamento de alguém}
\end{EntryWithPhonetic}

\begin{EntryWithPhonetic}{想想看}{xiang3xiang3kan4}{13,13,9}{⼼,⼼,⽬}
  \definition{v.}{pensar sobre isso}
\end{EntryWithPhonetic}

\begin{EntryWithPhonetic}{想象}{xiang3xiang4}{13,11}{⼼,⾗}[HSK 4]
  \definition[个,种,面]{s.}{imaginação; refere-se ao processo mental de processamento e transformação de representações armazenadas na mente para formar novas imagens}
  \definition{v.}{imaginar; vislumbrar; visualizar; refere-se a ter uma imagem concreta de algo que não está na frente dos olhos}
\end{EntryWithPhonetic}

%%%%%%%%%% 向 %%%%%%%%%%
\subsection*{向}\addcontentsline{loh}{figure}{向 \dpy{xiang4}}

\begin{EntryWithPhonetic}{向}{xiang4}{6}{⼝}[HSK 2]
  \definition*{s.}{Sobrenome: Xiang}
  \definition{adv.}{sempre; o tempo todo}
  \definition{prep.}{em direção a; para}
  \definition{s.}{direção | a janela voltada para o norte}
  \definition{v.}{encarar; virar-se para | estar do lado de; ser parcial com; tomar o partido de alguém}
\end{EntryWithPhonetic}

\begin{EntryWithPhonetic}{向导}{xiang4dao3}{6,6}{⼝,⼨}[HSK 5]
  \definition[位]{s.}{guia; a pessoa que lidera todos e lhes indica a direção ao caminhar}
  \definition{v.}{agir como um guia; mostrar a alguém o caminho; levar alguém a algum lugar}
\end{EntryWithPhonetic}

\begin{EntryWithPhonetic}{向前}{xiang4 qian2}{6,9}{⼝,⼑}[HSK 5]
  \definition{adv.}{para frente; adiante}
  \definition{v.}{avançar; ir em direção à frente; mover-se para frente; avançar um pouco mais}
\end{EntryWithPhonetic}

\begin{EntryWithPhonetic}{向上}{xiang4 shang4}{6,3}{⼝,⼀}[HSK 5]
  \definition{adv.}{o superior; acima}
  \definition{v.}{mover-se; subir; ir para um lugar mais alto; ir para um lugar mais alto em relação a um determinado ponto; ir para um desenvolvimento mais alto que o atual | avançar; continuar se aperfeiçoar; subir na vida; desenvolver-se em direção ao progresso}
\end{EntryWithPhonetic}

\begin{EntryWithPhonetic}{向汪}{xiang4wang1}{6,7}{⼝,⽔}
  \definition{v.}{esperar que}
\end{EntryWithPhonetic}

\begin{EntryWithPhonetic}{向往}{xiang4wang3}{6,8}{⼝,⼻}
  \definition{v.}{ansiar por | esperar ansiosamente por}
\end{EntryWithPhonetic}

%%%%%%%%%% 相 %%%%%%%%%%
\subsection*{相}\addcontentsline{loh}{figure}{相 \dpy{xiang4}}

\begin{EntryWithPhonetic}{相}{xiang4}{9}{⽬}
  \definition*{s.}{Sobrenome: Xiang}
  \definition{s.}{aparência | postura; porte; postura sentada, em pé, etc. | (física) fase; refere-se a uma parte homogênea de uma substância com a mesma composição e as mesmas propriedades físicas e químicas | fotografia | primeiro-ministro (na China antiga) | ministro; títulos oficiais de certos países | fácies marinha (carvão) | elefante, uma das peças do xadrez chinês | recepcionista (pessoa que ajuda o anfitrião a receber o hóspede); antigamente, referia-se a alguém que ajudava o anfitrião a receber convidados}
  \definition{v.}{olhar e avaliar; observe a aparência das coisas; julgar sua qualidade | assistir; ajudar; auxiliar}
  \seeref{xiang1}
\end{EntryWithPhonetic}

\begin{EntryWithPhonetic}{相机}{xiang4 ji1}{9,6}{⽬,⽊}[HSK 2]
  \definition[台,部,架,个]{s.}{câmera; máquina fotográfica}
  \definition{v.}{ficar atento a uma oportunidade; procurar oportunidades}
\end{EntryWithPhonetic}

\begin{EntryWithPhonetic}{相片}{xiang4 pian4}{9,4}{⽬,⽚}[HSK 4]
  \definition[张]{s.}{foto; fotografia; uma imagem de uma pessoa ou objeto feita pela exposição de papel fotográfico a um negativo fotográfico e, em seguida, revelando e fixando a imagem.}
\end{EntryWithPhonetic}

\begin{EntryWithPhonetic}{相声}{xiang4sheng5}{9,7}{⽬,⼠}[HSK 5]
  \definition[个,段]{s.}{conversa cruzada; diálogo cômico; forma de performance humorística, em que os atores usam piadas, canções e imitações para satirizar e elogiar}
\end{EntryWithPhonetic}

%%%%%%%%%% 项 %%%%%%%%%%
\subsection*{项}\addcontentsline{loh}{figure}{项 \dpy{xiang4}}

\begin{EntryWithPhonetic}{项}{xiang4}{9}{⾴}[HSK 4]
  \definition*{s.}{Sobrenome: Xiang}
  \definition{clas.}{usado para itens discriminados; taxonomia}
  \definition{s.}{nuca (do pescoço); a parte de trás do pescoço | soma (de dinheiro); fundos para fins especiais | termo; em álgebra, significa uma única fórmula que não é unida por um sinal de mais ou de menos | item}
\end{EntryWithPhonetic}

\begin{EntryWithPhonetic}{项目}{xiang4mu4}{9,5}{⾴,⽬}[HSK 4]
  \definition{s.}{evento; categorias em que as coisas são divididas | item; projeto; trabalhos de engenharia, acadêmicos, etc., de conteúdo específico}
\end{EntryWithPhonetic}

%%%%%%%%%% 象 %%%%%%%%%%
\subsection*{象}\addcontentsline{loh}{figure}{象 \dpy{xiang4}}

\begin{EntryWithPhonetic}{象}{xiang4}{11}{⾗}
  \definition*{s.}{Sobrenome: Xiang}
  \definition[头,群,个]{s.}{elefante | elefante, uma das peças do xadrez chinês | aparência; forma; imagem}
  \definition{v.}{imitar | latir}
\end{EntryWithPhonetic}

\begin{EntryWithPhonetic}{象棋}{xiang4qi2}{11,12}{⾗,⽊}
  \definition[副]{s.}{xadrez chinês; um tipo de jogo de xadrez em que dois jogadores têm dezesseis peças cada: um general, dois soldados, dois elefantes, duas carruagens, dois cavalos, dois canhões e cinco soldados ; cada jogador joga de acordo com as regras e o vencedor é aquele que der o xeque no general do adversário}
\end{EntryWithPhonetic}

\begin{EntryWithPhonetic}{象征}{xiang4zheng1}{11,8}{⾗,⼻}[HSK 5]
  \definition[种]{s.}{símbolo; emblema; insígnia; \emph{token}; objeto concreto que simboliza um significado especial}
  \definition{v.}{simbolizar; significar; representar; expressar um significado especial através de algo concreto}
\end{EntryWithPhonetic}

%%%%%%%%%% 像 %%%%%%%%%%
\subsection*{像}\addcontentsline{loh}{figure}{像 \dpy{xiang4}}

\begin{EntryWithPhonetic}{像}{xiang4}{13}{⼈}[HSK 2]
  \definition{adv.}{parecer; parecer como se}
  \definition{s.}{imagem; retrato; semelhança a alguém | imagem}
  \definition{v.}{assemelhar-se; ser como; parecer-se com | ser como; ser tal como}
\end{EntryWithPhonetic}

%%%%%%%%%% 消 %%%%%%%%%%
\subsection*{消}\addcontentsline{loh}{figure}{消 \dpy{xiao1}}

\begin{EntryWithPhonetic}{消}{xiao1}{10}{⽔}
  \definition{v.}{desaparecer | dissipar; remover; eliminar; fazer desaparecer | passar o tempo de forma descontraída (recreação) | precisar; tomar (necessidade, geralmente precedido por 不, 几, 何)}
  \seealsoref{不}{bu4}
  \seealsoref{何}{he2}
  \seealsoref{几}{ji3}
\end{EntryWithPhonetic}

\begin{EntryWithPhonetic}{消除}{xiao1chu2}{10,9}{⽔,⾩}[HSK 5]
  \definition{v.}{dissipar; eliminar; limpar; tornar algo inexistente; remover (algo desfavorável)}
\end{EntryWithPhonetic}

\begin{EntryWithPhonetic}{消毒}{xiao1du2}{10,9}{⽔,⽏}[HSK 5]
  \definition{v.}{desinfetar; esterilizar; matar os microrganismos causadores de doenças por meios físicos ou químicos}
\end{EntryWithPhonetic}

\begin{EntryWithPhonetic}{消防}{xiao1fang2}{10,6}{⽔,⾩}[HSK 5]
  \definition{s.}{combate a incêncios; controle de incêndios}
\end{EntryWithPhonetic}

\begin{EntryWithPhonetic}{消防员}{xiao1fang2yuan2}{10,6,7}{⽔,⾩,⼝}
  \definition{s.}{bombeiro}
\end{EntryWithPhonetic}

\begin{EntryWithPhonetic}{消费}{xiao1fei4}{10,9}{⽔,⾙}[HSK 3]
  \definition{v.}{gastar; consumir; consumir materiais para satisfazer as necessidades de produção ou de vida (geralmente refere-se ao consumo doméstico) | consumir (recursos naturais)}
\end{EntryWithPhonetic}

\begin{EntryWithPhonetic}{消费者}{xiao1 fei4 zhe3}{10,9,8}{⽔,⾙,⽼}[HSK 5]
  \definition{s.}{consumidor; cliente; consumo; indivíduos membros da sociedade que compram e utilizam bens e serviços para consumo pessoal}
\end{EntryWithPhonetic}

\begin{EntryWithPhonetic}{消耗}{xiao1hao4}{10,10}{⽔,⽾}[HSK 6]
  \definition{v.}{gastar; esgotar; consumir; usar; (espírito, força, coisas, etc.) diminuir gradualmente devido ao uso ou perda}
\end{EntryWithPhonetic}

\begin{EntryWithPhonetic}{消化}{xiao1hua4}{10,4}{⽔,⼔}[HSK 4]
  \definition{v.}{digerir (alimentos) | digerir (conhecimento); pensar e absorver; uma metáfora para a compreensão total de novos conhecimentos ou informações e a capacidade de transformá-los em algo que possa ser usado}
\end{EntryWithPhonetic}

\begin{EntryWithPhonetic}{消极}{xiao1ji2}{10,7}{⽔,⽊}[HSK 5]
  \definition{adj.}{negativo; oposto; adverso | passivo; inativo; sem ambição; sem iniciativa; desanimado; apático}
\end{EntryWithPhonetic}

\begin{EntryWithPhonetic}{消灭}{xiao1mie4}{10,5}{⽔,⽕}[HSK 6]
  \definition{v.}{perecer; morrer; falecer; desaparecer | abolir; erradicar; eliminar; aniquilar; exterminar; acabar com; fazer com que não exista}
\end{EntryWithPhonetic}

\begin{EntryWithPhonetic}{消失}{xiao1shi1}{10,5}{⽔,⼤}[HSK 3]
  \definition{v.}{desaparecer; desvanecer; dissolver; dissipar; evaporar; sumir}
\end{EntryWithPhonetic}

\begin{EntryWithPhonetic}{消息}{xiao1xi5}{10,10}{⽔,⼼}[HSK 3]
  \definition[个,条,篇,些]{s.}{notícias; informação; reportagem sobre pessoas ou situações | notícias; novidades;}
\end{EntryWithPhonetic}

%%%%%%%%%% 销 %%%%%%%%%%
\subsection*{销}\addcontentsline{loh}{figure}{销 \dpy{xiao1}}

\begin{EntryWithPhonetic}{销}{xiao1}{12}{⾦}
  \definition*{s.}{Sobrenome: Xiao}
  \definition{s.}{gasto; despesa | pino}
  \definition{v.}{derreter (metal) | cancelar; anular | vender; comercializar | aferrolhar; fixar; prender; pregar | fixar com um parafuso; parafusar | gastar (consumo) | inserir um pino}
\end{EntryWithPhonetic}

\begin{EntryWithPhonetic}{销售}{xiao1shou4}{12,11}{⾦,⼝}[HSK 4]
  \definition{v.}{vender; comercializar}
\end{EntryWithPhonetic}

%%%%%%%%%% 嚣 %%%%%%%%%%
\subsection*{嚣}\addcontentsline{loh}{figure}{嚣 \dpy{xiao1}}

\begin{EntryWithPhonetic}{嚣}{xiao1}{18}{⼝}
  \definition*{s.}{Sobrenome: Xiao}
  \definition{adj.}{lazer}
  \definition{v.}{clamar; fazer barulho}
\end{EntryWithPhonetic}

\begin{EntryWithPhonetic}{嚣张}{xiao1zhang1}{18,7}{⼝,⼸}
  \definition{adj.}{desenfreado | arrogante | agressivo}
\end{EntryWithPhonetic}

%%%%%%%%%% 小 %%%%%%%%%%
\subsection*{小}\addcontentsline{loh}{figure}{小 \dpy{xiao3}}

\begin{EntryWithPhonetic}{小}{xiao3}{3}{⼩}[HSK 1,2][Kangxi 42]
  \definition*{s.}{Sobrenome: Xiao}
  \definition{adj.}{menor; pequeno; insignificante; pouco; volume, área, quantidade, intensidade, etc. não são grandes | jovem | expressões humildes, referindo-se a si mesmo ou a pessoas ou coisas relacionadas a si mesmo | por um tempo; por um curto período; por um curto período de tempo | o mais novo; o último na ordem de antiguidade; em último lugar na classificação}
  \definition{pref.}{usado antes do sobrenome, nome, posição na família, etc.}
  \definition{s.}{os jovens; pessoas mais jovens | concubina}
\end{EntryWithPhonetic}

\begin{EntryWithPhonetic}{小白菜}{xiao3bai2cai4}{3,5,11}{⼩,⽩,⾋}
  \definition[棵]{s.}{\emph{bok choy} | couve chinesa}
\end{EntryWithPhonetic}

\begin{EntryWithPhonetic}{小吃}{xiao3chi1}{3,6}{⼩,⼝}[HSK 4]
  \definition[家]{s.}{lanche; petiscos; comida com especialidades locais, não muito para uma porção | prato frio; prato feito; cortes de frios na culinária ocidental | pratos pequenos e baratos; pratos simples em restaurantes com porções pequenas e preços baixos}
\end{EntryWithPhonetic}

\begin{EntryWithPhonetic}{小费}{xiao3 fei4}{3,9}{⼩,⾙}[HSK 6]
  \definition[笔]{s.}{gorjeta; gratificação; dinheiro extra pago por clientes e viajantes a funcionários de serviços em setores de serviços, como hotéis e pousadas}
\end{EntryWithPhonetic}

\begin{EntryWithPhonetic}{小狗}{xiao3 gou3}{3,8}{⼩,⽝}
  \definition{s.}{filhote de cachorro}
\end{EntryWithPhonetic}

\begin{EntryWithPhonetic}{小孩儿}{xiao3hai2r5}{3,9,2}{⼩,⼦,⼉}[HSK 1]
  \definition[个]{s.}{criança; bebê}
\end{EntryWithPhonetic}

\begin{EntryWithPhonetic}{小伙子}{xiao3huo3zi5}{3,6,3}{⼩,⼈,⼦}[HSK 4]
  \definition[位]{s.}{rapaz jovem; jovem colega}
\end{EntryWithPhonetic}

\begin{EntryWithPhonetic}{小姐}{xiao3jie5}{3,8}{⼩,⼥}[HSK 1]
  \definition[个,位]{s.}{jovem senhora; anteriormente, era assim que se referiam às filhas de famílias ricas. | senhorita; título honorífico para mulheres jovens | (gíria) prostituta}
\end{EntryWithPhonetic}

\begin{EntryWithPhonetic}{小麦}{xiao3mai4}{3,7}{⼩,⿆}[HSK 6]
  \definition[粒,公斤,吨,棵]{s.}{trigo}
\end{EntryWithPhonetic}

\begin{EntryWithPhonetic}{小朋友}{xiao3 peng2 you3}{3,8,4}{⼩,⽉,⼜}[HSK 1]
  \definition[个]{s.}{criança; crianças; refere-se a crianças e adolescentes | (termo usado por um adulto para se dirigir a uma criança) amiguinho; menino (ou menina); termo carinhoso para se referir a crianças e adolescentes}
\end{EntryWithPhonetic}

\begin{EntryWithPhonetic}{小气鬼}{xiao3qi4gui3}{3,4,9}{⼩,⽓,⿁}
  \definition{adj.}{avarento | mão-de-vaca | miserável | pão-duro}
\end{EntryWithPhonetic}

\begin{EntryWithPhonetic}{小区}{xiao3qu1}{3,4}{⼩,⼖}
  \definition{s.}{conjunto habitacional, comunidade, bairro | célula (telecomunicações)}
\end{EntryWithPhonetic}

\begin{EntryWithPhonetic}{小声}{xiao3 sheng1}{3,7}{⼩,⼠}[HSK 2]
  \definition{v.}{falar em voz baixa; falar baixinho; sussurar}
\end{EntryWithPhonetic}

\begin{EntryWithPhonetic}{小时}{xiao3shi2}{3,7}{⼩,⽇}[HSK 1]
  \definition{clas.}{hora; unidade de medida legal do tempo, 1 hora equivale a 60 minutos, é 1/24 de um dia}
  \definition[个]{s.}{hora; refere-se a um período de uma hora}
\end{EntryWithPhonetic}

\begin{EntryWithPhonetic}{小时候}{xiao3 shi2 hou5}{3,7,10}{⼩,⽇,⼈}[HSK 2]
  \definition{s.}{na infância; quando alguém era jovem; refere-se à infância}
\end{EntryWithPhonetic}

\begin{EntryWithPhonetic}{小树}{xiao3shu4}{3,9}{⼩,⽊}
  \definition[棵]{s.}{muda | arbusto | árvore pequena}
\end{EntryWithPhonetic}

\begin{EntryWithPhonetic}{小说}{xiao3shuo1}{3,9}{⼩,⾔}[HSK 2]
  \definition[本,部,篇,章]{s.}{história; romance; ficção; uma forma literária que reflete a vida social por meio da descrição de personagens, ambiente e enredo}
\end{EntryWithPhonetic}

\begin{EntryWithPhonetic}{小偷儿}{xiao3 tou1er5}{3,11,2}{⼩,⼈,⼉}[HSK 5]
  \definition{s.}{ladrão insignificante (ou furtivo); ladrãozinho | ladrão}
\end{EntryWithPhonetic}

\begin{EntryWithPhonetic}{小腿}{xiao3tui3}{3,13}{⼩,⾁}
  \definition{s.}{perna (do joelho ao calcanhar) | haste}
\end{EntryWithPhonetic}

\begin{EntryWithPhonetic}{小屋}{xiao3wu1}{3,9}{⼩,⼫}
  \definition{s.}{cabana | chalé | cabine}
\end{EntryWithPhonetic}

\begin{EntryWithPhonetic}{小小}{xiao3xiao3}{3,3}{⼩,⼩}
  \definition{adj.}{muito pequeno}
\end{EntryWithPhonetic}

\begin{EntryWithPhonetic}{小心}{xiao3xin1}{3,4}{⼩,⼼}[HSK 2]
  \definition{adj.}{cuidadoso; atento; com cautela}
  \definition{v.}{ter cuidado; ser cauteloso; estar atento; tomar cuidado; prestar atenção}
\end{EntryWithPhonetic}

\begin{EntryWithPhonetic}{小型}{xiao3 xing2}{3,9}{⼩,⼟}[HSK 4]
  \definition{adj.}{de tamanho pequeno; em pequena escala; miniatura; tipo pequeno; tamanho de bolso; tipo compacto}
  \definition{s.}{Mediterrâneo: escunas, pequenos veleiros de pesca ou turismo | pequeno \emph{rover} lunar (duas pessoas)}
\end{EntryWithPhonetic}

\begin{EntryWithPhonetic}{小学}{xiao3 xue2}{3,8}{⼩,⼦}[HSK 1]
  \definition[个]{s.}{escola primária (ou fundamental); escolas que oferecem ensino fundamental básico | estudos filológicos; antigamente, referia-se ao estudo da escrita, da fonética e da exegese}
\end{EntryWithPhonetic}

\begin{EntryWithPhonetic}{小学生}{xiao3 xue2 sheng1}{3,8,5}{⼩,⼦,⽣}[HSK 1]
  \definition{s.}{aluno; estudante; estudante do sexo masculino (男); estudante do sexo feminino (女) | um aluno mais novo (do que os outros da sua turma) | (dialeto) um menino pequeno}
  \seealsoref{男}{nan2}
  \seealsoref{女}{nv3}
\end{EntryWithPhonetic}

\begin{EntryWithPhonetic}{小洋白菜}{xiao3 yang2bai2cai4}{3,9,5,11}{⼩,⽔,⽩,⾋}
  \definition{s.}{couve de bruxelas}
\end{EntryWithPhonetic}

\begin{EntryWithPhonetic}{小于}{xiao3 yu2}{3,3}{⼩,⼆}[HSK 6]
  \definition{prep.}{menor que; menos que; indica que um número ou quantidade é menor que outro}
\end{EntryWithPhonetic}

\begin{EntryWithPhonetic}{小众}{xiao3zhong4}{3,6}{⼩,⼈}
  \definition{s.}{minoria da população | nicho (mercado, etc.)}
\end{EntryWithPhonetic}

\begin{EntryWithPhonetic}{小组}{xiao3 zu3}{3,8}{⼩,⽷}[HSK 2]
  \definition[个,名,位]{s.}{grupo; um pequeno grupo de pessoas}
\end{EntryWithPhonetic}

%%%%%%%%%% 晓 %%%%%%%%%%
\subsection*{晓}\addcontentsline{loh}{figure}{晓 \dpy{xiao3}}

\begin{EntryWithPhonetic}{晓}{xiao3}{10}{⽇}
  \definition{s.}{amanhecer; alvorada}
  \definition{v.}{(um dia) amanhecer; romper | saber; deixar alguém saber; dizer}
\end{EntryWithPhonetic}

\begin{EntryWithPhonetic}{晓得}{xiao3 de2}{10,11}{⽇,⼻}[HSK 6]
  \definition{v.}{saber; entender}[我不晓得他在哪里。===Não sei onde ele está.]
\end{EntryWithPhonetic}

%%%%%%%%%% 哮 %%%%%%%%%%
\subsection*{哮}\addcontentsline{loh}{figure}{哮 \dpy{xiao4}}

\begin{EntryWithPhonetic}{哮}{xiao4}{10}{⼝}
  \definition{s.}{respiração pesada; chiado}
  \definition{v.}{rugir; uivar}
\end{EntryWithPhonetic}

\begin{EntryWithPhonetic}{哮喘}{xiao4chuan3}{10,12}{⼝,⼝}
  \definition{s.}{asma; sintomas de dispneia: os pacientes sentem que a respiração está muito difícil; pneumonia, insuficiência cardíaca, bronquite crônica e outras doenças causadas por espasmo da musculatura lisa respiratória frequentemente apresentam esse sintoma}
  \definition{v.}{sofrer de asma}
\end{EntryWithPhonetic}

%%%%%%%%%% 效 %%%%%%%%%%
\subsection*{效}\addcontentsline{loh}{figure}{效 \dpy{xiao4}}

\begin{EntryWithPhonetic}{效}{xiao4}{10}{⽁}
  \definition{s.}{efeito; função | eficiência; resultado}
  \definition{v.}{imitar; seguir o exemplo de | dedicar (a energia ou a vida de alguém) a; prestar (um serviço)}
\end{EntryWithPhonetic}

\begin{EntryWithPhonetic}{效果}{xiao4guo3}{10,8}{⽁,⽊}[HSK 3]
  \definition[种,个]{s.}{efeito; resultado | efeitos sonoros; vários sons ou fenômenos naturais criados para combinar com o enredo em dramas e filmes, como vento e chuva, tiros, fogo, neve, etc.}
\end{EntryWithPhonetic}

\begin{EntryWithPhonetic}{效率}{xiao4lv4}{10,11}{⽁,⽞}[HSK 4]
  \definition[种]{s.}{eficiência; produtividade; a quantidade de trabalho concluído por unidade de tempo}
\end{EntryWithPhonetic}

%%%%%%%%%% 校 %%%%%%%%%%
\subsection*{校}\addcontentsline{loh}{figure}{校 \dpy{xiao4}}

\begin{EntryWithPhonetic}{校}{xiao4}{10}{⽊}
  \definition[所]{s.}{oficial militar | escola}
  \seeref{jiao4}
\end{EntryWithPhonetic}

\begin{EntryWithPhonetic}{校服}{xiao4fu2}{10,8}{⽊,⽉}
  \definition{s.}{uniforme escolar}
\end{EntryWithPhonetic}

\begin{EntryWithPhonetic}{校规}{xiao4gui1}{10,8}{⽊,⾒}
  \definition{s.}{regras e regulamentos escolares}
\end{EntryWithPhonetic}

\begin{EntryWithPhonetic}{校监}{xiao4jian1}{10,10}{⽊,⽫}
  \definition{s.}{diretor | supervisor (de escola)}
\end{EntryWithPhonetic}

\begin{EntryWithPhonetic}{校园}{xiao4 yuan2}{10,7}{⽊,⼞}[HSK 2]
  \definition[个]{s.}{campus; pátio da escola; refere-se a todos os terrenos e edifícios dentro da área escolar}
\end{EntryWithPhonetic}

\begin{EntryWithPhonetic}{校长}{xiao4zhang3}{10,4}{⽊,⾧}[HSK 2]
  \definition[个,位,名]{s.}{diretor; presidente; reitor; o mais alto líder administrativo e empresarial de uma escola}
\end{EntryWithPhonetic}

%%%%%%%%%% 笑 %%%%%%%%%%
\subsection*{笑}\addcontentsline{loh}{figure}{笑 \dpy{xiao4}}

\begin{EntryWithPhonetic}{笑}{xiao4}{10}{⽵}[HSK 1]
  \definition{adj.}{ridículo; engraçado; risível; hilário}
  \definition{v.}{sorrir; rir; mostrar expressão de alegria; emitir sons de alegria | ridicularizar; rir de; zombar}
\end{EntryWithPhonetic}

\begin{EntryWithPhonetic}{笑话儿}{xiao4 hua4r5}{10,8,2}{⽵,⾔,⼉}[HSK 2]
  \definition{s.}{piada; brincadeira; gracejo}
\end{EntryWithPhonetic}

\begin{EntryWithPhonetic}{笑话}{xiao4hua5}{10,8}{⽵,⾔}[HSK 2]
  \definition[个]{s.}{piada; brincadeira; uma conversa ou história que faz as pessoas rirem; algo que as pessoas usam como piada}
  \definition{v.}{ridicularizar; zombar; rir de;}
\end{EntryWithPhonetic}

\begin{EntryWithPhonetic}{笑脸}{xiao4 lian3}{10,11}{⽵,⾁}[HSK 6]
  \definition{s.}{\emph{smiley}; rosto sorridente (emoji)}
\end{EntryWithPhonetic}

\begin{EntryWithPhonetic}{笑容}{xiao4 rong2}{10,10}{⽵,⼧}[HSK 6]
  \definition[丝,抹,个]{s.}{sorriso; expressão sorridente; o olhar no rosto de alguém ao sorrir}
\end{EntryWithPhonetic}

\begin{EntryWithPhonetic}{笑声}{xiao4 sheng1}{10,7}{⽵,⼠}[HSK 6]
  \definition{s.}{riso; risada}
\end{EntryWithPhonetic}

%%%%%%%%%% 些 %%%%%%%%%%
\subsection*{些}\addcontentsline{loh}{figure}{些 \dpy{xie1}}

\begin{EntryWithPhonetic}{些}{xie1}{8}{⼆}[HSK 4]
  \definition{adv.}{um pouco; um pouco mais; usado após um adjetivo ou parte de um verbo para indicar uma pequena quantidade, equivalente a 一点儿}
  \definition{clas.}{alguns; um pouco; denota uma quantidade indefinida}
  \seealsoref{一点儿}{yi4dian3r5}
\end{EntryWithPhonetic}

\begin{EntryWithPhonetic}{些许}{xie1xu3}{8,6}{⼆,⾔}
  \definition{num.}{um pouco}
\end{EntryWithPhonetic}

%%%%%%%%%% 楔 %%%%%%%%%%
\subsection*{楔}\addcontentsline{loh}{figure}{楔 \dpy{xie1}}

\begin{EntryWithPhonetic}{楔}{xie1}{13}{⽊}
  \definition[个]{s.}{cunha | pino; pregos de madeira; pregos de bambu}
  \definition{v.}{cunhar}
\end{EntryWithPhonetic}

\begin{EntryWithPhonetic}{楔子}{xie1zi5}{13,3}{⽊,⼦}
  \definition{s.}{cunha | pino | prólogo ou interlúdio no drama da Dinastia Yuan | prólogo em alguns romances modernos; introduções a óperas e romances | calço; chuteira; lascas de madeira inseridas nas juntas de encaixe e espiga, etc. | estaca de madeira; estaca de bambu; pregos de madeira; pregos de bambu}
\end{EntryWithPhonetic}

%%%%%%%%%% 歇 %%%%%%%%%%
\subsection*{歇}\addcontentsline{loh}{figure}{歇 \dpy{xie1}}

\begin{EntryWithPhonetic}{歇}{xie1}{13}{⽋}[HSK 5]
  \definition*{s.}{Sobrenome: Xie}
  \definition{s.}{um pouco de tempo}
  \definition{v.}{descansar; fazer uma pausa | parar (o trabalho); encerrar o expediente | dormir; ir para a cama}
\end{EntryWithPhonetic}

%%%%%%%%%% 协 %%%%%%%%%%
\subsection*{协}\addcontentsline{loh}{figure}{协 \dpy{xie2}}

\begin{EntryWithPhonetic}{协}{xie2}{6}{⼗}
  \definition*{s.}{Sobrenome: Xie}
  \definition{adv.}{conjuntamente; coordenadamente; juntos}
  \definition{s.}{harmonioso}
  \definition{v.}{auxiliar; assistir; ajudar}
\end{EntryWithPhonetic}

\begin{EntryWithPhonetic}{协会}{xie2hui4}{6,6}{⼗,⼈}[HSK 6]
  \definition[个]{s.}{sociedade; instituto; associação; uma organização de massa formada para promover uma causa comum}
\end{EntryWithPhonetic}

\begin{EntryWithPhonetic}{协商}{xie2shang1}{6,11}{⼗,⼝}[HSK 6]
  \definition{v.}{discutir; consultar; negociar; várias partes discutiram e decidiram em conjunto para chegar à mesma visão}
\end{EntryWithPhonetic}

\begin{EntryWithPhonetic}{协调}{xie2tiao2}{6,10}{⼗,⾔}[HSK 6]
  \definition{adj.}{coordenado; harmonioso; em sintonia}
  \definition{v.}{coordenar; concertar; integrar; harmonizar; fazer a harmonia apropriada}
\end{EntryWithPhonetic}

\begin{EntryWithPhonetic}{协议}{xie2yi4}{6,5}{⼗,⾔}[HSK 5]
  \definition[份,项]{s.}{acordo; tratado; decisão conjunta alcançada através de negociação e consulta}
  \definition{v.}{concordar em}
\end{EntryWithPhonetic}

\begin{EntryWithPhonetic}{协议书}{xie2 yi4 shu1}{6,5,4}{⼗,⾔,⼄}[HSK 5]
  \definition{s.}{contrato | protocolo}
\end{EntryWithPhonetic}

\begin{EntryWithPhonetic}{协助}{xie2zhu4}{6,7}{⼗,⼒}[HSK 6]
  \definition{v.}{ajudar; auxiliar; dar assistência; fornecer ajuda}
\end{EntryWithPhonetic}

%%%%%%%%%% 斜 %%%%%%%%%%
\subsection*{斜}\addcontentsline{loh}{figure}{斜 \dpy{xie2}}

\begin{EntryWithPhonetic}{斜}{xie2}{11}{⽃}[HSK 5]
  \definition{adj.}{oblíquo; inclinado | enviesado; chanfrado; diagonal; torto; nem paralelo nem perpendicular a um plano ou linha}
  \definition{v.}{virar de lado; inclinar}
\end{EntryWithPhonetic}

\begin{EntryWithPhonetic}{斜阳}{xie2yang2}{11,6}{⽃,⾩}
  \definition{s.}{sol poente}
\end{EntryWithPhonetic}

%%%%%%%%%% 谐 %%%%%%%%%%
\subsection*{谐}\addcontentsline{loh}{figure}{谐 \dpy{xie2}}

\begin{EntryWithPhonetic}{谐}{xie2}{11}{⾔}
  \definition{adj.}{harmonioso | humorístico}
\end{EntryWithPhonetic}

%%%%%%%%%% 鞋 %%%%%%%%%%
\subsection*{鞋}\addcontentsline{loh}{figure}{鞋 \dpy{xie2}}

\begin{EntryWithPhonetic}{鞋}{xie2}{15}{⾰}[HSK 2]
  \definition[双,只]{s.}{sapatos; usado nos pés; algo que toca o chão ao caminhar; sem cano alto}
\end{EntryWithPhonetic}

%%%%%%%%%% 写 %%%%%%%%%%
\subsection*{写}\addcontentsline{loh}{figure}{写 \dpy{xie3}}

\begin{EntryWithPhonetic}{写}{xie3}{5}{⼍}[HSK 1]
  \definition{v.}{escrever | compor; escrever (como autor, repórter, etc.) | descrever; retratar | pintar; desenhar | expressar a imagem das coisas através da linguagem e da escrita | desenhar (pintura)}
\end{EntryWithPhonetic}

\begin{EntryWithPhonetic}{写意}{xie3yi4}{5,13}{⼍,⼼}
  \definition{s.}{estilo de pintura chinesa à mão livre, caracterizado por traços ousados em vez de detalhes precisos}
  \definition{v.}{sugerir (em vez de descrever em detalhes)}
  \seeref{xie4yi4}
\end{EntryWithPhonetic}

\begin{EntryWithPhonetic}{写照}{xie3zhao4}{5,13}{⼍,⽕}
  \definition{s.}{retrato}
\end{EntryWithPhonetic}

\begin{EntryWithPhonetic}{写真}{xie3zhen1}{5,10}{⼍,⼗}
  \definition{s.}{retrato}
  \definition{v.}{descrever algo com precisão}
\end{EntryWithPhonetic}

\begin{EntryWithPhonetic}{写字}{xie3zi4}{5,6}{⼍,⼦}
  \definition{v.}{escrever (à mão) | praticar caligrafia}
\end{EntryWithPhonetic}

\begin{EntryWithPhonetic}{写字匠}{xie3zi4 jiang4}{5,6,6}{⼍,⼦,⼕}
  \definition{s.}{calígrafo}
\end{EntryWithPhonetic}

\begin{EntryWithPhonetic}{写字楼}{xie3 zi4 lou2}{5,6,13}{⼍,⼦,⽊}[HSK 6]
  \definition{s.}{prédio de escritórios}
\end{EntryWithPhonetic}

\begin{EntryWithPhonetic}{写字台}{xie3 zi4 tai2}{5,6,5}{⼍,⼦,⼝}[HSK 6]
  \definition[个,张]{s.}{escrivaninha; secretária; escrivaninha de escrever; uma mesa retangular usada para escrever e trabalhar, com gavetas e algumas com pequenos armários}
\end{EntryWithPhonetic}

\begin{EntryWithPhonetic}{写作}{xie3zuo4}{5,7}{⼍,⼈}[HSK 3]
  \definition{v.}{escrever artigos; escrever livros, etc.; também se refere especificamente à criação de obras literárias}
\end{EntryWithPhonetic}

%%%%%%%%%% 血 %%%%%%%%%%
\subsection*{血}\addcontentsline{loh}{figure}{血 \dpy{xie3}}

\begin{EntryWithPhonetic}{血}{xie3}{6}{⾎}[Kangxi 143]
  \seeref{xue4}
\end{EntryWithPhonetic}

%%%%%%%%%% 写 %%%%%%%%%%
\subsection*{写}\addcontentsline{loh}{figure}{写 \dpy{xie4}}

\begin{EntryWithPhonetic}{写意}{xie4yi4}{5,13}{⼍,⼼}
  \definition{adj.}{confortável | agradável | relaxado}
  \seeref{xie3yi4}
\end{EntryWithPhonetic}

%%%%%%%%%% 泄 %%%%%%%%%%
\subsection*{泄}\addcontentsline{loh}{figure}{泄 \dpy{xie4}}

\begin{EntryWithPhonetic}{泄}{xie4}{8}{⽔}
  \definition*{s.}{Sobrenome: Xie}
  \definition{v.}{deixar sair (um fluido ou gás); descarregar; liberar | revelar (um segredo); vazar (notícias, segredos, etc.) | dar vazão a; desabafar}
\end{EntryWithPhonetic}

\begin{EntryWithPhonetic}{泄底}{xie4di3}{8,8}{⽔,⼴}
  \definition{v.}{revelar ou expor o que está no fundo de algo | divulgar a história interna; vazar segredos}
\end{EntryWithPhonetic}

\begin{EntryWithPhonetic}{泄愤}{xie4fen4}{8,12}{⽔,⼼}
  \definition{v.}{dar vazão à raiva}
\end{EntryWithPhonetic}

\begin{EntryWithPhonetic}{泄洪}{xie4hong2}{8,9}{⽔,⽔}
  \definition{v.}{liberar água da enchente (descarga de inundação)}
\end{EntryWithPhonetic}

\begin{EntryWithPhonetic}{泄露}{xie4lou4}{8,21}{⽔,⾬}
  \definition{v.}{vazar; deixar escapar; divulgar; revelar (um segredo, etc.) | vazar; escapar; descarregar (um fluido ou gás)}
\end{EntryWithPhonetic}

\begin{EntryWithPhonetic}{泄气}{xie4/qi4}{8,4}{⽔,⽓}
  \definition{adj.}{decepcionante | frustrante | patético}
  \definition{v.+compl.}{perder o coração | sentir-se desencorajado | ficar desanimado}
\end{EntryWithPhonetic}

%%%%%%%%%% 契 %%%%%%%%%%
\subsection*{契}\addcontentsline{loh}{figure}{契 \dpy{xie4}}

\begin{EntryWithPhonetic}{契}{xie4}{9}{⼤}
  \definition*{s.}{Xie (ancestral da dinastia Yin, considerado ministro do Imperador Shun)}
  \seeref{qi4}
\end{EntryWithPhonetic}

%%%%%%%%%% 谢 %%%%%%%%%%
\subsection*{谢}\addcontentsline{loh}{figure}{谢 \dpy{xie4}}

\begin{EntryWithPhonetic}{谢}{xie4}{12}{⾔}
  \definition*{s.}{Sobrenome: Xie}
  \definition{v.}{agradecer | desculpar-se; pedir desculpas; admitir a própria culpa | recusar; declinar; renunciar | murchar; perder de flores ou folhas}
\end{EntryWithPhonetic}

\begin{EntryWithPhonetic}{谢病}{xie4bing4}{12,10}{⾔,⽧}
  \definition{v.}{desculpar-se por causa de doença}
\end{EntryWithPhonetic}

\begin{EntryWithPhonetic}{谢恩}{xie4'en1}{12,10}{⾔,⼼}
  \definition{v.}{agradecer a alguém pelo favor (especialmente imperador ou oficial superior)}
\end{EntryWithPhonetic}

\begin{EntryWithPhonetic}{谢媒}{xie4mei2}{12,12}{⾔,⼥}
  \definition{v.}{agradecer ao casamenteiro}
\end{EntryWithPhonetic}

\begin{EntryWithPhonetic}{谢世}{xie4shi4}{12,5}{⾔,⼀}
  \definition{v.}{morrer | falecer}
\end{EntryWithPhonetic}

\begin{EntryWithPhonetic}{谢天谢地}{xie4tian1xie4di4}{12,4,12,6}{⾔,⼤,⾔,⼟}
  \definition{expr.}{agradecer a Deus | agradecer aos céus}
\end{EntryWithPhonetic}

\begin{EntryWithPhonetic}{谢谢}{xie4xie5}{12,12}{⾔,⾔}[HSK 1]
  \definition{interj.}{Obrigado!}
  \definition{v.}{agradecer; agradecer a gentileza dos outros}
\end{EntryWithPhonetic}

\begin{EntryWithPhonetic}{谢意}{xie4yi4}{12,13}{⾔,⼼}
  \definition{s.}{gratidão}
\end{EntryWithPhonetic}

%%%%%%%%%% 心 %%%%%%%%%%
\subsection*{心}\addcontentsline{loh}{figure}{心 \dpy{xin1}}

\begin{EntryWithPhonetic}{心}{xin1}{4}{⼼}[HSK 3][Kangxi 61]
  \definition*{s.}{Xin, uma das mansões lunares; uma das vinte e oito constelações}
  \definition[颗,个]{s.}{o coraçã; órgão que impulsiona a circulação sanguínea no corpo humano e nos vertebrados| coração; mente; sentimento; intenção; refere-se aos órgãos do pensamento e ao pensamento, sentimentos, etc. | centro; núcleo; parte central}
\end{EntryWithPhonetic}

\begin{EntryWithPhonetic}{心机}{xin1ji1}{4,6}{⼼,⽊}
  \definition{s.}{pensamento | esquema}
\end{EntryWithPhonetic}

\begin{EntryWithPhonetic}{心里}{xin1 li3}{4,7}{⼼,⾥}[HSK 2]
  \definition[个]{s.}{no coração; no coração de alguém | no coração; na mente; na cabeça e no peito}
\end{EntryWithPhonetic}

\begin{EntryWithPhonetic}{心理}{xin1li3}{4,11}{⼼,⽟}[HSK 4]
  \definition[个]{s.}{mentalidade; refere-se à reflexão da mente humana sobre coisas objetivas, incluindo sensação, percepção, memória, pensamento e emoções | psicologia}
\end{EntryWithPhonetic}

\begin{EntryWithPhonetic}{心灵}{xin1ling2}{4,7}{⼼,⽕}[HSK 6]
  \definition[个,颗]{s.}{alma; coração; espírito; refere-se ao coração, espírito, pensamentos, etc.}
\end{EntryWithPhonetic}

\begin{EntryWithPhonetic}{心情}{xin1qing2}{4,11}{⼼,⼼}[HSK 2]
  \definition{s.}{humor; tom de sentimento; estado de espírito; estado emocional interior}
\end{EntryWithPhonetic}

\begin{EntryWithPhonetic}{心声}{xin1sheng1}{4,7}{⼼,⼠}
  \definition{s.}{desejo sincero | voz interior | aspiração}
\end{EntryWithPhonetic}

\begin{EntryWithPhonetic}{心态}{xin1tai4}{4,8}{⼼,⼼}[HSK 5]
  \definition[种,个]{s.}{mentalidade; psicologia; estado mental}
\end{EntryWithPhonetic}

\begin{EntryWithPhonetic}{心疼}{xin1teng2}{4,10}{⼼,⽧}[HSK 5]
  \definition{v.}{amar profundamente; sentir pena porque coisas valiosas foram destruídas ou perdidas; não querer se separar delas | sentir pena; ficar angustiado; preocupar-se e sofrer pelo sofrimento dos outros; estar disposto a cuidar mais por causa da preocupação}
\end{EntryWithPhonetic}

\begin{EntryWithPhonetic}{心愿}{xin1 yuan4}{4,14}{⼼,⽕}[HSK 6]
  \definition[桩]{s.}{desejo acalentado; aspiração; desejo; sonho | o desejo do coração}
\end{EntryWithPhonetic}

\begin{EntryWithPhonetic}{心脏}{xin1zang4}{4,10}{⼼,⾁}[HSK 6]
  \definition[颗,个]{s.}{coração; um órgão importante no corpo de humanos ou animais superiores que faz o sangue circular | coração; o centro ou a parte mais importante de uma metáfora}
\end{EntryWithPhonetic}

\begin{EntryWithPhonetic}{心脏病}{xin1 zang4 bing4}{4,10,10}{⼼,⾁,⽧}[HSK 6]
  \definition{s.}{doença cardíaca; cardiopatia um termo geral para anormalidades ou doenças na estrutura e função do coração humano}
\end{EntryWithPhonetic}

\begin{EntryWithPhonetic}{心中}{xin1zhong1}{4,4}{⼼,⼁}[HSK 2]
  \definition{s.}{no coração; na mente}
\end{EntryWithPhonetic}

%%%%%%%%%% 芯 %%%%%%%%%%
\subsection*{芯}\addcontentsline{loh}{figure}{芯 \dpy{xin1}}

\begin{EntryWithPhonetic}{芯}{xin1}{7}{⾋}
  \definition{s.}{medula de junco | pavio}
  \seeref{xin4}
\end{EntryWithPhonetic}

\begin{EntryWithPhonetic}{芯片}{xin1pian4}{7,4}{⾋,⽚}
  \definition{s.}{\emph{chip} de computador; \emph{microchip}; um substrato (geralmente uma pastilha de silício) que contém um circuito integrado completo}
\end{EntryWithPhonetic}

%%%%%%%%%% 辛 %%%%%%%%%%
\subsection*{辛}\addcontentsline{loh}{figure}{辛 \dpy{xin1}}

\begin{EntryWithPhonetic}{辛}{xin1}{7}{⾟}[Kangxi 160]
  \definition*{s.}{Sobrenome: Xin}
  \definition{adj.}{quente (no sabor, etc.); pungente | difícil; trabalhoso | ponto da bússola chinesa antiga: 285° | oitavo na ordem}
  \definition{pref.}{octa-}
  \definition{s.}{sofrimento}
  \definition{s.}{oitavo dos dez caules celestiais}
\end{EntryWithPhonetic}

\begin{EntryWithPhonetic}{辛苦}{xin1ku3}{7,8}{⾟,⾋}[HSK 5]
  \definition{adj.}{difícil; trabalhoso; árduo; descreve muito trabalho, alta intensidade e pouco descanso}
  \definition{s.}{dificuldades}
  \definition{v.}{trabalhar duro; passar por grandes dificuldades; passar por dificuldades}
\end{EntryWithPhonetic}

%%%%%%%%%% 欣 %%%%%%%%%%
\subsection*{欣}\addcontentsline{loh}{figure}{欣 \dpy{xin1}}

\begin{EntryWithPhonetic}{欣}{xin1}{8}{⽋}
  \definition*{s.}{Sobrenome: Xin}
  \definition{adj.}{alegre; feliz; contente}
\end{EntryWithPhonetic}

\begin{EntryWithPhonetic}{欣赏}{xin1shang3}{8,12}{⽋,⾙}[HSK 5]
  \definition{v.}{apreciar; admirar; valorizar; apreciar as coisas boas e descubrir o prazer que elas proporcionam | apreciar; gostar; considerar bom}
\end{EntryWithPhonetic}

%%%%%%%%%% 新 %%%%%%%%%%
\subsection*{新}\addcontentsline{loh}{figure}{新 \dpy{xin1}}

\begin{EntryWithPhonetic}{新}{xin1}{13}{⽄}[HSK 1]
  \definition*{s.}{Xinjiang, abreviação de 新疆 | Singapura, abreviação de 新加坡 | Sobrenome: Xin}
  \definition{adj.}{novo; fresco; inovador; atualizado; aparecer ou ser experimentado pela primeira vez | nunca utilizado; novo; não foi usado ou foi usado por pouco tempo | recém-casado}
  \definition{adv.}{recém; recentemente; há pouco tempo}
  \definition{pref.}{Química: meso-}
  \definition{v.}{atualizar; renovar}
  \seealsoref{新加坡}{xin1jia1po1}
  \seealsoref{新疆}{xin1jiang1}
\end{EntryWithPhonetic}

\begin{EntryWithPhonetic}{新加坡}{xin1jia1po1}{13,5,8}{⽄,⼒,⼟}
  \definition*{s.}{Singapura}
\end{EntryWithPhonetic}

\begin{EntryWithPhonetic}{新疆}{xin1jiang1}{13,19}{⽄,⼸}
  \definition*{s.}{Região Autônoma Uigur de Xinjiang}
\end{EntryWithPhonetic}

\begin{EntryWithPhonetic}{新疆维吾尔自治区}{xin1jiang1 wei2wu2'er3 zi4zhi4qu1}{13,19,11,7,5,6,8,4}{⽄,⼸,⽷,⼝,⼩,⾃,⽔,⼖}
  \definition*{s.}{Região Autônoma Uigur de Xinjiang}
\end{EntryWithPhonetic}

\begin{EntryWithPhonetic}{新郎}{xin1lang2}{13,8}{⽄,⾢}[HSK 4]
  \definition[位,名,个,些]{s.}{noivo; homens no momento do casamento}
\end{EntryWithPhonetic}

\begin{EntryWithPhonetic}{新年}{xin1 nian2}{13,6}{⽄,⼲}[HSK 1]
  \definition*[个]{s.}{Ano Novo}
\end{EntryWithPhonetic}

\begin{EntryWithPhonetic}{新娘}{xin1niang2}{13,10}{⽄,⼥}[HSK 4]
  \definition[位,个]{s.}{noiva; a mulher no momento do casamento}
  \seealsoref{新娘子}{xin1niang2zi5}
\end{EntryWithPhonetic}

\begin{EntryWithPhonetic}{新娘服装}{xin1niang2 fu2zhuang1}{13,10,8,12}{⽄,⼥,⽉,⾐}
  \definition{s.}{vestido de noiva}
\end{EntryWithPhonetic}

\begin{EntryWithPhonetic}{新娘子}{xin1niang2zi5}{13,10,3}{⽄,⼥,⼦}
  \definition{s.}{noiva}
  \seealsoref{新娘}{xin1niang2}
\end{EntryWithPhonetic}

\begin{EntryWithPhonetic}{新人}{xin1 ren2}{13,2}{⽄,⼈}[HSK 6]
  \definition[位]{s.}{pessoas de um novo tipo; nova pessoa;  pessoa que virou uma nova página | nova personalidade; novo talento | recém-chegado; novo membro | noiva ou noivo; recém-casado | \emph{neoanthropus}; \emph{homo sapiens}}
\end{EntryWithPhonetic}

\begin{EntryWithPhonetic}{新闻}{xin1wen2}{13,9}{⽄,⾨}[HSK 2]
  \definition[个,条,则,版]{s.}{notícias; notícias nacionais e internacionais reportadas em jornais, estações de rádio, etc. | notícias; refere-se a coisas importantes ou novas que aconteceram recentemente na sociedade}
\end{EntryWithPhonetic}

\begin{EntryWithPhonetic}{新鲜}{xin1xian1}{13,14}{⽄,⿂}
  \definition{adj.}{fresco (experiência, alimento, etc.)}
  \definition{s.}{frescor}
\end{EntryWithPhonetic}

\begin{EntryWithPhonetic}{新兴}{xin1 xing1}{13,6}{⽄,⼋}[HSK 6]
  \definition[个]{adj.}{recém-desenvolvido; crescente; florescente; emergente; descreve algo que está apenas começando a se tornar popular ou se desenvolver}
\end{EntryWithPhonetic}

\begin{EntryWithPhonetic}{新型}{xin1 xing2}{13,9}{⽄,⼟}[HSK 4]
  \definition[种]{s.}{ultimo modelo; novo tipo; novo padrão; novo estilo}
\end{EntryWithPhonetic}

%%%%%%%%%% 薪 %%%%%%%%%%
\subsection*{薪}\addcontentsline{loh}{figure}{薪 \dpy{xin1}}

\begin{EntryWithPhonetic}{薪}{xin1}{16}{⾋}
  \definition{s.}{lenha; combustível | salário; ordenado; pagamento}
\end{EntryWithPhonetic}

\begin{EntryWithPhonetic}{薪水}{xin1shui3}{16,4}{⾋,⽔}[HSK 6]
  \definition[份,笔]{s.}{pagamento; salário; ordenados; dinheiro ou bens pagos regularmente aos trabalhadores como compensação pelo seu trabalho}
\end{EntryWithPhonetic}

%%%%%%%%%% 芯 %%%%%%%%%%
\subsection*{芯}\addcontentsline{loh}{figure}{芯 \dpy{xin4}}

\begin{EntryWithPhonetic}{芯}{xin4}{7}{⾋}
  \definition{s.}{núcleo; a parte central de um objeto | língua de cobra}
  \seeref{xin1}
\end{EntryWithPhonetic}

%%%%%%%%%% 信 %%%%%%%%%%
\subsection*{信}\addcontentsline{loh}{figure}{信 \dpy{xin4}}

\begin{EntryWithPhonetic}{信}{xin4}{9}{⼈}[HSK 2,3]
  \definition*{s.}{Sobrenome: Xin}
  \definition{adj.}{verdade}
  \definition[封,个,张]{s.}{carta; correio | mensagem; notícia; informação | sinal; evidência | confiança; fé; crédito | detonador (de bombas, etc.) | arsênico}
  \definition{v.}{acreditar; fazer um balanço; dar crédito | deixar à vontade; deixar à mercê; deixar ao acaso | professar fé em; acreditar em}
\end{EntryWithPhonetic}

\begin{EntryWithPhonetic}{信访}{xin4fang3}{9,6}{⼈,⾔}
  \definition{s.}{carta de reclamação | carta de petição}
  \seealsoref{上访}{shang4fang3}
\end{EntryWithPhonetic}

\begin{EntryWithPhonetic}{信封}{xin4feng1}{9,9}{⼈,⼨}[HSK 3]
  \definition[个,封]{s.}{envelope para cartas}
\end{EntryWithPhonetic}

\begin{EntryWithPhonetic}{信号}{xin4hao4}{9,5}{⼈,⼝}[HSK 2]
  \definition[个,道]{s.}{sinal; luz, ondas de rádio, som, movimento, etc. usados para transmitir mensagens ou comandos | ponte de sinalização; marcação para chamar a atenção, ajudar na identificação e na memória}
\end{EntryWithPhonetic}

\begin{EntryWithPhonetic}{信号灯}{xin4hao4deng1}{9,5,6}{⼈,⼝,⽕}
  \definition[盏]{s.}{lâmpada de sinalização; semáforo | luz de sinalização}
\end{EntryWithPhonetic}

\begin{EntryWithPhonetic}{信经}{xin4jing1}{9,8}{⼈,⽷}
  \definition[个]{s.}{crença | credo (seção da missa católica)}
\end{EntryWithPhonetic}

\begin{EntryWithPhonetic}{信念}{xin4nian4}{9,8}{⼈,⼼}[HSK 5]
  \definition[个,种]{s.}{fé; crença; convicção; concepções consideradas corretas e acreditadas com convicção}
\end{EntryWithPhonetic}

\begin{EntryWithPhonetic}{信任}{xin4ren4}{9,6}{⼈,⼈}[HSK 3]
  \definition{s.}{confiança; um estado mental positivo e conexão emocional}
  \definition{v.}{confiar; ter confiança em; acreditar e ousar confiar}
\end{EntryWithPhonetic}

\begin{EntryWithPhonetic}{信息}{xin4xi1}{9,10}{⼈,⼼}[HSK 2]
  \definition[个,条,段,些]{s.}{notícias; informações; as últimas notícias sobre alguém ou alguma coisa | mensagem; informação; na teoria da informação, uma mensagem transmitida usando símbolos, cujo conteúdo é desconhecido pelo receptor}
\end{EntryWithPhonetic}

\begin{EntryWithPhonetic}{信箱}{xin4 xiang1}{9,15}{⼈,⾋}[HSK 5]
  \definition{s.}{caixa de correio; caixa postal instalada pelos correios para que as pessoas possam depositar cartas | caixa postal; caixas com números, localizadas nos correios, que podem ser alugadas para receber correspondência; chamadas de caixas postais exclusivas}
\end{EntryWithPhonetic}

\begin{EntryWithPhonetic}{信心}{xin4xin1}{9,4}{⼈,⼼}[HSK 2]
  \definition[个]{s.}{confiança; fé (em alguém ou algo); a crença de que os desejos se tornarão realidade}
\end{EntryWithPhonetic}

\begin{EntryWithPhonetic}{信仰}{xin4yang3}{9,6}{⼈,⼈}[HSK 6]
  \definition[种]{s.}{crença; religião; refere-se à ideia de acreditar, adorar e tomar algo como padrão e guia para palavras e ações}
  \definition{v.}{acreditar; crer em; acreditar e adorar uma determinada religião ou doutrina e tomá-la como guia para palavras e ações}
\end{EntryWithPhonetic}

\begin{EntryWithPhonetic}{信用}{xin4 yong4}{9,5}{⼈,⽤}[HSK 6]
  \definition{adj.}{crédito; não é necessária nenhuma garantia material e o dinheiro pode ser reembolsado no prazo}
  \definition[些]{s.}{crédito; confiabilidade; a confiança que você ganha ao fazer o que prometeu | crédito; uma relação de empréstimo-pagamento ou situação em que o empréstimo é condicionado ao pagamento; uma situação em que um banco empresta dinheiro temporariamente a um cliente e este posteriormente devolve o dinheiro ao banco}
\end{EntryWithPhonetic}

\begin{EntryWithPhonetic}{信用卡}{xin4yong4ka3}{9,5,5}{⼈,⽤,⼘}[HSK 2]
  \definition[张]{s.}{cartão de crédito; moeda eletrônica emitida por um banco ou outra instituição especializada para consumidores; os titulares do cartão podem usá-lo para sacar dinheiro ou fazer compras de acordo com os regulamentos}
\end{EntryWithPhonetic}

%%%%%%%%%% 兴 %%%%%%%%%%
\subsection*{兴}\addcontentsline{loh}{figure}{兴 \dpy{xing1}}

\begin{EntryWithPhonetic}{兴}{xing1}{6}{⼋}
  \definition*{s.}{Sobrenome: Xing}
  \definition{adj.}{próspero; florescente}
  \definition{adv.}{Dialeto: talvez}
  \definition{v.}{ascender; prosperar; prevalecer; tornar-se popular | promover; encorajar; fazer prevalecer | começar; iniciar; lançar; mobilizar | erguer-se; levantar-se | (usualmente no negativo) permitir; deixar}
  \seeref{xing4}
\end{EntryWithPhonetic}

\begin{EntryWithPhonetic}{兴奋}{xing1fen4}{6,8}{⼋,⼤}[HSK 4]
  \definition{adj.}{animado; excitante; empolgante;}
  \definition{s.}{excitação; empolgação}
  \definition{v.}{excitar; intoxicar}
\end{EntryWithPhonetic}

\begin{EntryWithPhonetic}{兴旺}{xing1wang4}{6,8}{⼋,⽇}[HSK 6]
  \definition{adj.}{próspero; propício; favorável; auspicioso}
\end{EntryWithPhonetic}

%%%%%%%%%% 星 %%%%%%%%%%
\subsection*{星}\addcontentsline{loh}{figure}{星 \dpy{xing1}}

\begin{EntryWithPhonetic}{星}{xing1}{9}{⽇}
  \definition*{s.}{Xing, a vigésima quinta das vinte e oito constelações em que a esfera celeste era dividida na antiga astronomia chinesa, consistindo em sete estrelas em Hydra}
  \definition[颗]{s.}{estrela | (astronomia) corpo celeste | partícula | pequenas marcas no braço de uma balança romana indicando jin e suas frações | artista famoso (estrela de cinema, estrela de jogos de bola, etc.) | satélite (artificial) | pequena quantidade}
\end{EntryWithPhonetic}

\begin{EntryWithPhonetic}{星表}{xing1biao3}{9,8}{⽇,⾐}
  \definition{s.}{catálogo de estrelas}
\end{EntryWithPhonetic}

\begin{EntryWithPhonetic}{星辰}{xing1chen2}{9,7}{⽇,⾠}
  \definition{s.}{estrelas}
\end{EntryWithPhonetic}

\begin{EntryWithPhonetic}{星火}{xing1huo3}{9,4}{⽇,⽕}
  \definition{s.}{trilha de meteoro (usada principalmente em expressões como 急如星火) | faísca}
\end{EntryWithPhonetic}

\begin{EntryWithPhonetic}{星期}{xing1qi1}{9,12}{⽇,⽉}[HSK 1]
  \definition[个]{s.}{semana | dias da semana; usado em conjunto com 日, 一, 二, 三, 四, 五, 六, 天, indica um determinado dia da semana | abreviação de domingo}
  \seealsoref{星期二}{xing1 qi1 er4}
  \seealsoref{星期六}{xing1 qi1 liu4}
  \seealsoref{星期日}{xing1 qi1 ri4}
  \seealsoref{星期三}{xing1 qi1 san1}
  \seealsoref{星期四}{xing1 qi1 si4}
  \seealsoref{星期天}{xing1 qi1 tian1}
  \seealsoref{星期五}{xing1 qi1 wu3}
  \seealsoref{星期一}{xing1 qi1 yi1}
\end{EntryWithPhonetic}

\begin{EntryWithPhonetic}{星期二}{xing1 qi1 er4}{9,12,2}{⽇,⽉,⼆}[HSK 1]
  \definition{s.}{terça-feira}
\end{EntryWithPhonetic}

\begin{EntryWithPhonetic}{星期六}{xing1 qi1 liu4}{9,12,4}{⽇,⽉,⼋}[HSK 1]
  \definition{s.}{sábado}
\end{EntryWithPhonetic}

\begin{EntryWithPhonetic}{星期日}{xing1 qi1 ri4}{9,12,4}{⽇,⽉,⽇}[HSK 1]
  \definition{s.}{domingo}
  \seealsoref{星期天}{xing1 qi1 tian1}
\end{EntryWithPhonetic}

\begin{EntryWithPhonetic}{星期三}{xing1 qi1 san1}{9,12,3}{⽇,⽉,⼀}[HSK 1]
  \definition{s.}{quarta-feira}
\end{EntryWithPhonetic}

\begin{EntryWithPhonetic}{星期四}{xing1 qi1 si4}{9,12,5}{⽇,⽉,⼞}[HSK 1]
  \definition{s.}{quinta-feira}
\end{EntryWithPhonetic}

\begin{EntryWithPhonetic}{星期天}{xing1 qi1 tian1}{9,12,4}{⽇,⽉,⼤}[HSK 1]
  \definition{s.}{domingo}
  \seealsoref{星期日}{xing1 qi1 ri4}
\end{EntryWithPhonetic}

\begin{EntryWithPhonetic}{星期五}{xing1 qi1 wu3}{9,12,4}{⽇,⽉,⼆}[HSK 1]
  \definition{s.}{sexta-feira}
\end{EntryWithPhonetic}

\begin{EntryWithPhonetic}{星期一}{xing1 qi1 yi1}{9,12,1}{⽇,⽉,⼀}[HSK 1]
  \definition{s.}{segunda-feira}
\end{EntryWithPhonetic}

\begin{EntryWithPhonetic}{星星}{xing1 xing5}{9,9}{⽇,⽇}[HSK 2]
  \definition[颗,群,片]{s.}{estrela; em astronomia, refere-se aos corpos celestes luminosos no universo, como as estrelas que brilham no céu noturno | estrela; uma metáfora para alguém ou algo que se destaca em um determinado campo e atrai atenção | objetos em forma de estrela}
\end{EntryWithPhonetic}

\begin{EntryWithPhonetic}{星座}{xing1zuo4}{9,10}{⽇,⼴}
  \definition[张]{s.}{Astronomia: constelação, cada uma das várias divisões (regiões) do céu noturno}
\end{EntryWithPhonetic}

%%%%%%%%%% 猩 %%%%%%%%%%
\subsection*{猩}\addcontentsline{loh}{figure}{猩 \dpy{xing1}}

\begin{EntryWithPhonetic}{猩}{xing1}{12}{⽝}
  \definition[只]{s.}{orangotango}
\end{EntryWithPhonetic}

\begin{EntryWithPhonetic}{猩猩}{xing1xing5}{12,12}{⽝,⽝}
  \definition{s.}{orangotango}
\end{EntryWithPhonetic}

%%%%%%%%%% 行 %%%%%%%%%%
\subsection*{行}\addcontentsline{loh}{figure}{行 \dpy{xing2}}

\begin{EntryWithPhonetic}{行}{xing2}{6}{⾏}[HSK 1][Kangxi 144]
  \definition*{s.}{Sobrenome: Xing}
  \definition{adj.}{de viajar; relacionado a viagens | temporário; improvisado; provisório | capaz; competente}
  \definition{adv.}{em breve}
  \definition{s.}{comportamento; conduta | caligrafia cursiva (na caligrafia chinesa); escrita cursiva}
  \definition{v.}{ir | fazer uma viagem | estar em voga; prevalecer; circular | fazer; executar; realizar; envolver-se em | estar tudo bem; O.K. | indica a realização de uma determinada atividade (usado principalmente antes de verbos dissilábicos) | (em medicina) fazer efeito}
  \seeref{hang2}
  \seeref{heng2}
\end{EntryWithPhonetic}

\begin{EntryWithPhonetic}{行程}{xing2 cheng2}{6,12}{⾏,⽲}[HSK 6]
  \definition{s.}{rota ou distância de viagem; distância; jornada | curso; progresso; processo | curso; deslocamento; viagem}[活塞行程有点不对劲。===Há algo errado com o curso do pistão.]
\end{EntryWithPhonetic}

\begin{EntryWithPhonetic}{行动}{xing2dong4}{6,6}{⾏,⼒}[HSK 2]
  \definition[次,场,项]{s.}{ação; operação; comportamento;}
  \definition{v.}{circular; mover-se; andar | agir; tomar medidas; atividades para atingir um determinado propósito}
\end{EntryWithPhonetic}

\begin{EntryWithPhonetic}{行进}{xing2jin4}{6,7}{⾏,⾡}
  \definition{s.}{avançar | movimentar-se para frente}
\end{EntryWithPhonetic}

\begin{EntryWithPhonetic}{行礼}{xing2li3}{6,5}{⾏,⽰}
  \definition{v.}{saudar | fazer saudação}
\end{EntryWithPhonetic}

\begin{EntryWithPhonetic}{行李}{xing2li5}{6,7}{⾏,⽊}[HSK 3]
  \definition[点,个]{s.}{bagagem, malas, cestas de vime, etc. que você leva quando sai de casa}
\end{EntryWithPhonetic}

\begin{EntryWithPhonetic}{行人}{xing2ren2}{6,2}{⾏,⼈}[HSK 2]
  \definition[个]{s.}{pedestre; transeunte; viajante à pé; pessoas caminhando na estrada}
\end{EntryWithPhonetic}

\begin{EntryWithPhonetic}{行驶}{xing2 shi3}{6,8}{⾏,⾺}[HSK 5]
  \definition{v.}{ir; navegar; viajar (utilizando um veículo, navio, etc.)}
\end{EntryWithPhonetic}

\begin{EntryWithPhonetic}{行为}{xing2wei2}{6,4}{⾏,⼂}[HSK 2]
  \definition[个,种,类]{s.}{ação; comportamento; conduta; atividades que são controladas por pensamentos e manifestadas externamente}
\end{EntryWithPhonetic}

\begin{EntryWithPhonetic}{行星}{xing2xing1}{6,9}{⾏,⽇}
  \definition[颗]{s.}{planeta}
  \seealsoref{惑星}{huo4xing1}
\end{EntryWithPhonetic}

\begin{EntryWithPhonetic}{行凶}{xing2/xiong1}{6,4}{⾏,⼐}
  \definition{v.+compl.}{cometer agressão física ou assassinato | fazer algo violento}
\end{EntryWithPhonetic}

%%%%%%%%%% 形 %%%%%%%%%%
\subsection*{形}\addcontentsline{loh}{figure}{形 \dpy{xing2}}

\begin{EntryWithPhonetic}{形}{xing2}{7}{⼺}[HSK 6]
  \definition{s.}{forma; formato | corpo; entidade}
  \definition{v.}{aparecer; revelar; mostrar | comparar; contrastar}
\end{EntryWithPhonetic}

\begin{EntryWithPhonetic}{形成}{xing2cheng2}{7,6}{⼺,⼽}[HSK 3]
  \definition{v.}{moldar; formar; tomar forma; tornar-se algo ou surgir uma situação após mudanças e desenvolvimentos}
\end{EntryWithPhonetic}

\begin{EntryWithPhonetic}{形而上学}{xing2'er2shang4xue2}{7,6,3,8}{⼺,⽽,⼀,⼦}
  \definition{s.}{metafísica}
\end{EntryWithPhonetic}

\begin{EntryWithPhonetic}{形容}{xing2rong2}{7,10}{⼺,⼧}[HSK 4]
  \definition{s.}{aparência; semblante}
  \definition{v.}{descrever}
\end{EntryWithPhonetic}

\begin{EntryWithPhonetic}{形式}{xing2shi4}{7,6}{⼺,⼷}[HSK 3]
  \definition[种,个]{s.}{forma; formato; modalidade; a aparência, estrutura ou estado das coisas, etc.}
\end{EntryWithPhonetic}

\begin{EntryWithPhonetic}{形势}{xing2shi4}{7,8}{⼺,⼒}[HSK 4]
  \definition[个,种]{s.}{terreno; características topográficas; situação geográfica, principalmente de uma perspectiva militar | situação; circunstâncias; a situação geral, a tendência de como as coisas estão se desenvolvendo e mudando | geralmente não é usado em situações pessoais}
\end{EntryWithPhonetic}

\begin{EntryWithPhonetic}{形态}{xing2tai4}{7,8}{⼺,⼼}[HSK 5]
  \definition[种]{s.}{forma; forma como as coisas se apresentam | forma; padrão; postura | morfologia; forma; Gramática: refere-se às formas internas de mudança das palavras, incluindo a formação de palavras e as mudanças morfológicas}
\end{EntryWithPhonetic}

\begin{EntryWithPhonetic}{形象}{xing2xiang4}{7,11}{⼺,⾗}[HSK 3]
  \definition{adj.}{vívido; expressão concreta e vívida}
  \definition[个,种]{s.}{imagem; forma; figura; formas ou posturas específicas que podem despertar pensamentos ou emoções nas pessoas | imagem literária; imagem artística; pessoas ou coisas com características diferentes criadas na literatura, no cinema e em outras artes}
\end{EntryWithPhonetic}

\begin{EntryWithPhonetic}{形状}{xing2zhuang4}{7,7}{⼺,⽝}[HSK 3]
  \definition[个,种]{s.}{forma; aparência ; aspecto; a aparência de um objeto ou figura, representada pela combinação de superfícies ou linhas externas}
\end{EntryWithPhonetic}

%%%%%%%%%% 型 %%%%%%%%%%
\subsection*{型}\addcontentsline{loh}{figure}{型 \dpy{xing2}}

\begin{EntryWithPhonetic}{型}{xing2}{9}{⼟}[HSK 4]
  \definition{s.}{molde; modelo | modelo; tipo; padrão}
\end{EntryWithPhonetic}

\begin{EntryWithPhonetic}{型号}{xing2 hao4}{9,5}{⼟,⼝}[HSK 4]
  \definition[个,种]{s.}{modelo; tipo; refere-se ao desempenho, às especificações e ao tamanho de aeronaves, máquinas, implementos agrícolas, etc.}
\end{EntryWithPhonetic}

%%%%%%%%%% 省 %%%%%%%%%%
\subsection*{省}\addcontentsline{loh}{figure}{省 \dpy{xing3}}

\begin{EntryWithPhonetic}{省}{xing3}{9}{⽬}
  \definition{v.}{examinar-se criticamente; verificar (os próprios pensamentos, palavras e ações) | visitar (especialmente os pais ou pessoas mais velhas) | estar ciente; tornar-se consciente; compreender; tomar consciência | examinar minuciosamente; inspecionar; escrutinar}
  \seeref{sheng3}
\end{EntryWithPhonetic}

\begin{EntryWithPhonetic}{省悟}{xing3wu4}{9,10}{⽬,⼼}
  \definition{v.}{voltar a si | constatar | ver a verdade | acordar para a realidade}
\end{EntryWithPhonetic}

%%%%%%%%%% 醒 %%%%%%%%%%
\subsection*{醒}\addcontentsline{loh}{figure}{醒 \dpy{xing3}}

\begin{EntryWithPhonetic}{醒}{xing3}{16}{⾣}[HSK 4]
  \definition{adj.}{impressionante; notável; admirável; atraente; chamativo}
  \definition{v.}{ficar sóbrio; voltar a si; recuperar a consciência; retornar à normalidade após intoxicação, anestesia ou coma | despertar; estar acordado | ter a mente clara; mover a consciência da confusão para a compreensão | vir a entender; tornar-se ciente de; tomar consciência de}
\end{EntryWithPhonetic}

%%%%%%%%%% 兴 %%%%%%%%%%
\subsection*{兴}\addcontentsline{loh}{figure}{兴 \dpy{xing4}}

\begin{EntryWithPhonetic}{兴}{xing4}{6}{⼋}
  \definition{s.}{sentimento ou desejo de fazer algo | interesse em algo | excitação}
  \seeref{xing1}
\end{EntryWithPhonetic}

\begin{EntryWithPhonetic}{兴趣}{xing4 qu4}{6,15}{⼋,⾛}[HSK 4]
  \definition[个,种,点,股,份]{s.}{interesse (desejo de conhecer sobre alguma coisa ou coisa no qual está interessado) | \emph{hobby}}
\end{EntryWithPhonetic}

%%%%%%%%%% 姓 %%%%%%%%%%
\subsection*{姓}\addcontentsline{loh}{figure}{姓 \dpy{xing4}}

\begin{EntryWithPhonetic}{姓}{xing4}{8}{⼥}[HSK 2]
  \definition[个]{s.}{sobrenome; nome de família; um caractere que representa um sistema familiar, os chineses colocam o sobrenome em primeiro lugar e o nome em segundo}
  \definition{v.}{ter como sobrenome; tratar um ou mais caracteres como sobrenome}
\end{EntryWithPhonetic}

\begin{EntryWithPhonetic}{姓名}{xing4ming2}{8,6}{⼥,⼝}[HSK 2]
  \definition{s.}{nome; nome completo; sobrenome e nome próprio}
\end{EntryWithPhonetic}

\begin{EntryWithPhonetic}{姓氏}{xing4shi4}{8,4}{⼥,⽒}
  \definition{s.}{sobrenome}
\end{EntryWithPhonetic}

%%%%%%%%%% 幸 %%%%%%%%%%
\subsection*{幸}\addcontentsline{loh}{figure}{幸 \dpy{xing4}}

\begin{EntryWithPhonetic}{幸}{xing4}{8}{⼲}
  \definition*{s.}{Sobrenome: Xing}
  \definition{adj.}{feliz}
  \definition{adv.}{afortunadamente; felizmente}
  \definition{s.}{felicidade}
  \definition{v.}{alegrar-se; sentir-se feliz e contente | favorecer; patrocinar | vir; chegar; antigamente, referia-se à chegada de um monarca a um determinado lugar}
\end{EntryWithPhonetic}

\begin{EntryWithPhonetic}{幸福}{xing4fu2}{8,13}{⼲,⽰}[HSK 3]
  \definition{adj.}{feliz; a vida, a família e outras circunstâncias deixam as pessoas satisfeitas e felizes}
  \definition{s.}{felicidade; bem estar; sensação ou experiência satisfatória e feliz, etc.}
\end{EntryWithPhonetic}

\begin{EntryWithPhonetic}{幸亏}{xing4kui1}{8,3}{⼲,⼆}
  \definition{adv.}{felizmente}
\end{EntryWithPhonetic}

\begin{EntryWithPhonetic}{幸运}{xing4yun4}{8,7}{⼲,⾡}[HSK 3]
  \definition{adj.}{sortudo; feliz; afortunado}
  \definition[个,点,丝]{s.}{boa sorte; boa fortuna}
\end{EntryWithPhonetic}

\begin{EntryWithPhonetic}{幸运抽奖}{xing4yun4chou1jiang3}{8,7,8,9}{⼲,⾡,⼿,⼤}
  \definition{s.}{loteria | sorteio}
\end{EntryWithPhonetic}

\begin{EntryWithPhonetic}{幸运儿}{xing4yun4'er2}{8,7,2}{⼲,⾡,⼉}
  \definition{s.}{pessoa de sorte}
\end{EntryWithPhonetic}

%%%%%%%%%% 性 %%%%%%%%%%
\subsection*{性}\addcontentsline{loh}{figure}{性 \dpy{xing4}}

\begin{EntryWithPhonetic}{性}{xing4}{8}{⼼}[HSK 3]
  \definition[个]{s.}{natureza; caráter; personalidade | propriedade; qualidade; natureza e características das coisas | sexo; gênero | sexualidade; relacionado com a reprodução e a sexualidade | caráter; temperamento}
  \definition{suf.}{indica uma determinada propriedade ou característica de algo; segue um substantivo, verbo ou adjetivo, formando um substantivo abstrato ou um adjetivo que expressa uma propriedade}
\end{EntryWithPhonetic}

\begin{EntryWithPhonetic}{性别}{xing4bie2}{8,7}{⼼,⼑}[HSK 3]
  \definition[种]{s.}{sexo; gênero}
\end{EntryWithPhonetic}

\begin{EntryWithPhonetic}{性格}{xing4ge2}{8,10}{⼼,⽊}[HSK 3]
  \definition[种,个]{s.}{caráter; temperamento; as características psicológicas manifestadas na atitude e no comportamento em relação às pessoas e às coisas}
\end{EntryWithPhonetic}

\begin{EntryWithPhonetic}{性能}{xing4neng2}{8,10}{⼼,⾁}[HSK 5]
  \definition{s.}{natureza; propriedade; desempenho; função (de uma máquina, etc.); grau de conformidade dos produtos mecânicos ou outros produtos industriais com os requisitos de projeto}
\end{EntryWithPhonetic}

\begin{EntryWithPhonetic}{性侵}{xing4qin1}{8,9}{⼼,⼈}
  \definition{s.}{agressão sexual}
  \definition{v.}{agredir sexualmente}
\end{EntryWithPhonetic}

\begin{EntryWithPhonetic}{性生活}{xing4sheng1huo2}{8,5,9}{⼼,⽣,⽔}
  \definition{s.}{vida sexual}
\end{EntryWithPhonetic}

\begin{EntryWithPhonetic}{性质}{xing4zhi4}{8,8}{⼼,⾙}[HSK 4]
  \definition[个,种,类]{s.}{natureza; qualidade; caráter; propriedade; propriedade fundamental que distingue uma coisa de outra}
\end{EntryWithPhonetic}

%%%%%%%%%% 凶 %%%%%%%%%%
\subsection*{凶}\addcontentsline{loh}{figure}{凶 \dpy{xiong1}}

\begin{EntryWithPhonetic}{凶}{xiong1}{4}{⼐}[HSK 6]
  \definition{adj.}{sinistro; desfavorável; azarado (oposto de 吉) | ruim para as colheitas; improdutivo; ameaçado pela fome | feroz; vicioso; cruel | medroso; terrível}
  \definition[个]{s.}{mal; assassinato; ato de violência; atos de matar ou ferir pessoas | assassino; malfeitor; criminoso; pessoa má; pessoa violenta}
  \definition{v.}{ser feroz; tratar cruelmente}
  \seealsoref{吉}{ji2}
\end{EntryWithPhonetic}

\begin{EntryWithPhonetic}{凶手}{xiong1shou3}{4,4}{⼐,⼿}[HSK 6]
  \definition[名]{s.}{assassino; homicida | agressor (que causou ferimentos a alguém)}
\end{EntryWithPhonetic}

%%%%%%%%%% 兄 %%%%%%%%%%
\subsection*{兄}\addcontentsline{loh}{figure}{兄 \dpy{xiong1}}

\begin{EntryWithPhonetic}{兄}{xiong1}{5}{⼉}
  \definition{s.}{irmão mais velho | parente mais velho do sexo masculino da mesma geração | uma forma cortês de tratamento entre amigos homens; um título respeitoso para amigos homens}
\end{EntryWithPhonetic}

\begin{EntryWithPhonetic}{兄弟}{xiong1di4}{5,7}{⼉,⼸}[HSK 4]
  \definition{adj.}{fraternal}
  \definition{pron.}{eu, me (termo de uso humilde por homens em discurso público)}
  \definition[个,位]{s.}{irmãos; irmão}
\end{EntryWithPhonetic}

%%%%%%%%%% 匈 %%%%%%%%%%
\subsection*{匈}\addcontentsline{loh}{figure}{匈 \dpy{xiong1}}

\begin{EntryWithPhonetic}{匈}{xiong1}{6}{⼓}
  \definition*{s.}{Hungria, abreviação de 匈牙利}
  \definition{s.}{peito; seio; tórax}
  \seealsoref{匈牙利}{xiong1ya2li4}
\end{EntryWithPhonetic}

\begin{EntryWithPhonetic}{匈奴}{xiong1nu2}{6,5}{⼓,⼥}
  \definition*{s.}{Xiongnu, um povo da estepe oriental que criou um império que floresceu na época das dinastias Qin e Han}
\end{EntryWithPhonetic}

\begin{EntryWithPhonetic}{匈牙利}{xiong1ya2li4}{6,4,7}{⼓,⽛,⼑}
  \definition*{s.}{Hungria}
\end{EntryWithPhonetic}

%%%%%%%%%% 汹 %%%%%%%%%%
\subsection*{汹}\addcontentsline{loh}{figure}{汹 \dpy{xiong1}}

\begin{EntryWithPhonetic}{汹}{xiong1}{7}{⽔}
  \definition{adj.}{turbulento; tempestuoso | rugindo; estrondoso | tumultuado}
\end{EntryWithPhonetic}

\begin{EntryWithPhonetic}{汹涌}{xiong1yong3}{7,10}{⽔,⽔}
  \definition{adj.}{turbulento}
  \definition{v.}{aumentar ou emergir violentamente (oceano, rio, lago, etc.)}
\end{EntryWithPhonetic}

%%%%%%%%%% 胸 %%%%%%%%%%
\subsection*{胸}\addcontentsline{loh}{figure}{胸 \dpy{xiong1}}

\begin{EntryWithPhonetic}{胸}{xiong1}{10}{⾁}
  \definition{s.}{peito | tórax}
\end{EntryWithPhonetic}

\begin{EntryWithPhonetic}{胸部}{xiong1 bu4}{10,10}{⾁,⾢}[HSK 4]
  \definition{s.}{peito; tórax; seios}
\end{EntryWithPhonetic}

%%%%%%%%%% 雄 %%%%%%%%%%
\subsection*{雄}\addcontentsline{loh}{figure}{雄 \dpy{xiong2}}

\begin{EntryWithPhonetic}{雄}{xiong2}{12}{⾫}
  \definition*{s.}{Sobrenome: Xiong}
  \definition{adj.}{masculino | grandioso; imponente; audacioso | poderoso}
  \definition{s.}{uma pessoa ou país com grande poder e influência}
\end{EntryWithPhonetic}

\begin{EntryWithPhonetic}{雄伟}{xiong2wei3}{12,6}{⾫,⼈}[HSK 5]
  \definition{adj.}{magnífico; magnificente | imponente; magnífico}
\end{EntryWithPhonetic}

%%%%%%%%%% 熊 %%%%%%%%%%
\subsection*{熊}\addcontentsline{loh}{figure}{熊 \dpy{xiong2}}

\begin{EntryWithPhonetic}{熊}{xiong2}{14}{⽕}[HSK 5]
  \definition*{s.}{Sobrenome: Xiong}
  \definition[头,只]{s.}{urso}
  \definition{v.}{repreender; censurar}
\end{EntryWithPhonetic}

\begin{EntryWithPhonetic}{熊猫}{xiong2mao1}{14,11}{⽕,⽝}
  \definition[把,只]{s.}{panda gigante}
  \seealsoref{猫熊}{mao1xiong2}
\end{EntryWithPhonetic}

%%%%%%%%%% 休 %%%%%%%%%%
\subsection*{休}\addcontentsline{loh}{figure}{休 \dpy{xiu1}}

\begin{EntryWithPhonetic}{休}{xiu1}{6}{⼈}
  \definition{adj.}{feliz; alegre; festivo}
  \definition{adv.}{não; indica proibição ou dissuasão, equivalente a 别 ou 不要}
  \definition{s.}{fortuna e infortúnio; bom e mau}
  \definition{v.}{parar; cessar | descansar | abandonar a esposa e mandá-la para casa; antigamente, o marido mandava a esposa de volta para a casa dos pais e rompia o relacionamento conjugal}
  \seealsoref{别}{bie2}
  \seealsoref{不要}{bu2 yao4}
\end{EntryWithPhonetic}

\begin{EntryWithPhonetic}{休兵}{xiu1bing1}{6,7}{⼈,⼋}
  \definition{s.}{armistício; cessar fogo}
  \definition{v.}{cessar fogo}
\end{EntryWithPhonetic}

\begin{EntryWithPhonetic}{休假}{xiu1/jia4}{6,11}{⼈,⼈}[HSK 2]
  \definition{v.+compl.}{ter um feriado; tirar férias; sair de férias}
\end{EntryWithPhonetic}

\begin{EntryWithPhonetic}{休憩}{xiu1qi4}{6,16}{⼈,⼼}
  \definition{v.}{relaxar | descansar | dar um tempo}
\end{EntryWithPhonetic}

\begin{EntryWithPhonetic}{休息室}{xiu1xi1shi4}{6,10,9}{⼈,⼼,⼧}
  \definition{s.}{saguão | salão}
\end{EntryWithPhonetic}

\begin{EntryWithPhonetic}{休息}{xiu1xi5}{6,10}{⼈,⼼}[HSK 1]
  \definition{s.}{descanço}
  \definition{v.}{descansar; descansar um pouco; fazer uma pausa; interromper o trabalho, os estudos ou as atividades para recuperar as energias | dormir}
\end{EntryWithPhonetic}

\begin{EntryWithPhonetic}{休闲}{xiu1xian2}{6,7}{⼈,⾨}[HSK 5]
  \definition{s.}{ócio; lazer; tempo livre}
  \definition{v.}{desfrutar do lazer; sair de férias; aproveitar o tempo livre; parar de trabalhar ou estudar, estar em um estado de lazer e descontração | ficar ocioso}
\end{EntryWithPhonetic}

\begin{EntryWithPhonetic}{休整}{xiu1zheng3}{6,16}{⼈,⽁}
  \definition{v.}{(militar) descansar e reorganizar}
\end{EntryWithPhonetic}

%%%%%%%%%% 修 %%%%%%%%%%
\subsection*{修}\addcontentsline{loh}{figure}{修 \dpy{xiu1}}

\begin{EntryWithPhonetic}{修}{xiu1}{9}{⼈}[HSK 3]
  \definition*{s.}{Sobrenome: Xiu}
  \definition{adj.}{comprido; alto e esbelto}
  \definition{s.}{revisionismo}
  \definition{v.}{embelezar; decorar | consertar; reparar; reformar | escrever; redigir; compilar | estudar; cultivar; aprender e praticar para aperfeiçoar ou melhorar (o caráter e o conhecimento) | construir; edificar | cortar ou aparar, para deixar bonito e arrumado | dedicar-se à prática da religião}
\end{EntryWithPhonetic}

\begin{EntryWithPhonetic}{修车}{xiu1 che1}{9,4}{⼈,⾞}[HSK 6]
  \definition{v.}{consertar uma bicicleta (carro etc.)}[我打算明天去修车。===Pretendo consertar meu carro amanhã.]
\end{EntryWithPhonetic}

\begin{EntryWithPhonetic}{修复}{xiu1fu4}{9,9}{⼈,⼢}[HSK 5]
  \definition{v.}{reparar; restaurar; renovar | reparar; melhorar e restaurar (o relacionamento)}
\end{EntryWithPhonetic}

\begin{EntryWithPhonetic}{修改}{xiu1gai3}{9,7}{⼈,⽁}[HSK 3]
  \definition{v.}{revisar; retocar; corrigir erros e falhas em artigos, planos, etc.}
\end{EntryWithPhonetic}

\begin{EntryWithPhonetic}{修规}{xiu1gui1}{9,8}{⼈,⾒}
  \definition{s.}{plano de construção}
\end{EntryWithPhonetic}

\begin{EntryWithPhonetic}{修建}{xiu1jian4}{9,8}{⼈,⼵}[HSK 5]
  \definition{v.}{construir; erguer; animar; edificar; construir com tijolos, telhas, madeira, cimento, areia, etc.}
\end{EntryWithPhonetic}

\begin{EntryWithPhonetic}{修理}{xiu1li3}{9,11}{⼈,⽟}[HSK 4]
  \definition{v.}{consertar; reparar; restaurar algo danificado à sua forma ou função original | aparar; podar; cortar com tesouras e outras ferramentas para deixar árvores, flores, cabelos, etc. arrumados | culpar; punir; criticar ou punir uma pessoa para mostrar que ela está errada}
\end{EntryWithPhonetic}

\begin{EntryWithPhonetic}{修养}{xiu1yang3}{9,9}{⼈,⼋}[HSK 5]
  \definition[种]{s.}{treinamento; domínio; realização; refere-se a um determinado nível em termos de teoria, conhecimento, arte, pensamento, etc. | auto-cultivo; refere-se à atitude e ao comportamento cultivados ao longo do tempo, em conformidade com as exigências sociais}
\end{EntryWithPhonetic}

%%%%%%%%%% 宿 %%%%%%%%%%
\subsection*{宿}\addcontentsline{loh}{figure}{宿 \dpy{xiu3}}

\begin{EntryWithPhonetic}{宿}{xiu3}{11}{⼧}
  \definition{s.}{usado para calcular a noite}[谈了半宿。===Conversamos por metade da noite.]
  \seeref{su4}
  \seeref{xiu4}
\end{EntryWithPhonetic}

%%%%%%%%%% 绣 %%%%%%%%%%
\subsection*{绣}\addcontentsline{loh}{figure}{绣 \dpy{xiu4}}

\begin{EntryWithPhonetic}{绣}{xiu4}{10}{⽷}
  \definition{s.}{bordado}
  \definition{v.}{bordar}
\end{EntryWithPhonetic}

%%%%%%%%%% 臭 %%%%%%%%%%
\subsection*{臭}\addcontentsline{loh}{figure}{臭 \dpy{xiu4}}

\begin{EntryWithPhonetic}{臭}{xiu4}{10}{⾃}
  \definition{s.}{odor; cheiro}
  \definition{v.}{cheirar; farejar; o mesmo que 嗅}
  \seeref{chou4}
  \seealsoref{嗅}{xiu4}
\end{EntryWithPhonetic}

%%%%%%%%%% 袖 %%%%%%%%%%
\subsection*{袖}\addcontentsline{loh}{figure}{袖 \dpy{xiu4}}

\begin{EntryWithPhonetic}{袖}{xiu4}{10}{⾐}
  \definition{s.}{manga (de camisa, de camiseta, etc.)}
\end{EntryWithPhonetic}

\begin{EntryWithPhonetic}{袖珍}{xiu4 zhen1}{10,9}{⾐,⽟}[HSK 6]
  \definition{adj.}{do tamanho do bolso; de bolso (livro, agenda, etc.)}
\end{EntryWithPhonetic}

%%%%%%%%%% 宿 %%%%%%%%%%
\subsection*{宿}\addcontentsline{loh}{figure}{宿 \dpy{xiu4}}

\begin{EntryWithPhonetic}{宿}{xiu4}{11}{⼧}
  \definition{s.}{(astronomia) um termo antigo para constelação}
  \seeref{su4}
  \seeref{xiu3}
\end{EntryWithPhonetic}

%%%%%%%%%% 嗅 %%%%%%%%%%
\subsection*{嗅}\addcontentsline{loh}{figure}{嗅 \dpy{xiu4}}

\begin{EntryWithPhonetic}{嗅}{xiu4}{13}{⼝}
  \definition{v.}{cheirar; farejar; identificar odores pelo nariz}
\end{EntryWithPhonetic}

%%%%%%%%%% 虚 %%%%%%%%%%
\subsection*{虚}\addcontentsline{loh}{figure}{虚 \dpy{xu1}}

\begin{EntryWithPhonetic}{虚}{xu1}{11}{⾌}
  \definition*{s.}{Xu, a décima primeira das vinte e oito constelações em que a esfera celeste foi dividida, consistindo de duas estrelas em linha reta, uma em Aquário e a outra em Equuleus | Xu, uma das mansões lunares | Sobrenome: Xu}
  \definition{adj.}{vazio; oco; desocupado | desconfiado; tímido | falso; nominal (oposto a 实) | humilde; modesto | fraco; com saúde debilitada | (física) virtual}
  \definition{adv.}{em vão}
  \definition{s.}{vazio; nulidade; anulação | resumo; teoria; princípios orientadores; ideologia política e outros aspectos}
  \definition{v.}{reservar espaço}
  \seealsoref{实}{shi2}
\end{EntryWithPhonetic}

\begin{EntryWithPhonetic}{虚伪}{xu1wei3}{11,6}{⾌,⼈}
  \definition{adj.}{falso | hipócrita | artificial}
\end{EntryWithPhonetic}

\begin{EntryWithPhonetic}{虚心}{xu1xin1}{11,4}{⾌,⼼}[HSK 5]
  \definition{adj.}{modesto; humilde; de mente aberta; não ser presunçoso, ser capaz de aceitar as opiniões dos outros}
\end{EntryWithPhonetic}

%%%%%%%%%% 需 %%%%%%%%%%
\subsection*{需}\addcontentsline{loh}{figure}{需 \dpy{xu1}}

\begin{EntryWithPhonetic}{需}{xu1}{14}{⾬}
  \definition*{s.}{Sobrenome: Xu}
  \definition{s.}{necessidades; bens de primeira necessidade}
  \definition{v.}{precisar; querer; exigir}
\end{EntryWithPhonetic}

\begin{EntryWithPhonetic}{需求}{xu1qiu2}{14,7}{⾬,⽔}[HSK 3]
  \definition[种]{s.}{necessidades; demanda; exigência; solicitações decorrentes de necessidades}
\end{EntryWithPhonetic}

\begin{EntryWithPhonetic}{需要}{xu1yao4}{14,9}{⾬,⾑}[HSK 3]
  \definition[种]{s.}{necessidade; desejo ou exigência em relação a algo}
  \definition{v.}{precisar; querer; exigir; demandar; solicitar}
\end{EntryWithPhonetic}

%%%%%%%%%% 徐 %%%%%%%%%%
\subsection*{徐}\addcontentsline{loh}{figure}{徐 \dpy{xu2}}

\begin{EntryWithPhonetic}{徐}{xu2}{10}{⼻}
  \definition*{s.}{Sobrenome: Xu}
  \definition{adv.}{lentamente; suavemente}
\end{EntryWithPhonetic}

\begin{EntryWithPhonetic}{徐徐}{xu2xu2}{10,10}{⼻,⼻}
  \definition{adv.}{lentamente; suavemente}
\end{EntryWithPhonetic}

%%%%%%%%%% 许 %%%%%%%%%%
\subsection*{许}\addcontentsline{loh}{figure}{许 \dpy{xu3}}

\begin{EntryWithPhonetic}{许}{xu3}{6}{⾔}
  \definition*{s.}{Xu, um estado da Dinastia Zhou | Sobrenome: Xu}
  \definition{adv.}{um pouco;  talvez; expressa especulação ou estimativa, equivalente a 或者 ou 可能}
  \definition{part.}{cerca de; aproximadamente; usado depois de certos numerais, frases de quantidade ou 些 ou 少 para indicar um número próximo a um certo número}
  \definition{pron.}{muitos; um monte de}
  \definition{v.}{elogiar; aprovar | prometer; prometer dar antecipadamente; dedicar | permitir; concordar; aprovar | (uma menina) estar prometida a; refere-se especificamente ao noivado}
  \seealsoref{或者}{huo4zhe3}
  \seealsoref{可能}{ke3neng2}
  \seealsoref{少}{shao3}
  \seealsoref{些}{xie1}
\end{EntryWithPhonetic}

\begin{EntryWithPhonetic}{许多}{xu3duo1}{6,6}{⾔,⼣}[HSK 2]
  \definition{num.}{muitos; muito; numerosos; uma grande quantidade de}
\end{EntryWithPhonetic}

\begin{EntryWithPhonetic}{许可}{xu3ke3}{6,5}{⾔,⼝}[HSK 5]
  \definition{v.}{permitir; autorizar}
\end{EntryWithPhonetic}

%%%%%%%%%% 恤 %%%%%%%%%%
\subsection*{恤}\addcontentsline{loh}{figure}{恤 \dpy{xu4}}

\begin{EntryWithPhonetic}{恤}{xu4}{9}{⼼}
  \definition{v.}{ter pena; simpatizar | dar alívio; compensar}
\end{EntryWithPhonetic}

%%%%%%%%%% 畜 %%%%%%%%%%
\subsection*{畜}\addcontentsline{loh}{figure}{畜 \dpy{xu4}}

\begin{EntryWithPhonetic}{畜}{xu4}{10}{⽥}
  \definition{v.}{criar (animais domésticos)}
  \seeref{chu4}
\end{EntryWithPhonetic}

%%%%%%%%%% 宣 %%%%%%%%%%
\subsection*{宣}\addcontentsline{loh}{figure}{宣 \dpy{xuan1}}

\begin{EntryWithPhonetic}{宣}{xuan1}{9}{⼧}
  \definition*{s.}{Sobrenome: Xuan}
  \definition{v.}{declarar; proclamar; anunciar; falar publicamente | drenar (líquidos)}
\end{EntryWithPhonetic}

\begin{EntryWithPhonetic}{宣布}{xuan1bu4}{9,5}{⼧,⼱}[HSK 3]
  \definition{v.}{declarar; proclamar; pronunciar; anunciar; informar oficialmente a todos sobre as últimas decisões e situações}
\end{EntryWithPhonetic}

\begin{EntryWithPhonetic}{宣传}{xuan1chuan2}{9,6}{⼧,⼈}[HSK 3]
  \definition[个]{v.}{propagar; divulgar; fazer propaganda; explicar e esclarecer às pessoas, para que elas acreditem e sigam as ações}
\end{EntryWithPhonetic}

\begin{EntryWithPhonetic}{宣扬}{xuan1yang2}{9,6}{⼧,⼿}
  \definition{v.}{divulgar | anunciar | espalhar por toda parte}
\end{EntryWithPhonetic}

%%%%%%%%%% 玄 %%%%%%%%%%
\subsection*{玄}\addcontentsline{loh}{figure}{玄 \dpy{xuan2}}

\begin{EntryWithPhonetic}{玄}{xuan2}{5}{⽞}[Kangxi 95]
  \definition*{s.}{Sobrenome: Xuan}
  \definition{adj.}{preto; escuro | profundo; abstruso; escondido | não confiável; irrealista; não confiável}
\end{EntryWithPhonetic}

\begin{EntryWithPhonetic}{玄学}{xuan2xue2}{5,8}{⽞,⼦}
  \definition{s.}{Escola Philosófica Wei e Jin amalgamando os ideais daoísta e confucionistas | tradução da metafísica (形而上学) | Obsoleto: metafísica}
  \seealsoref{形而上学}{xing2'er2shang4xue2}
\end{EntryWithPhonetic}

%%%%%%%%%% 悬 %%%%%%%%%%
\subsection*{悬}\addcontentsline{loh}{figure}{悬 \dpy{xuan2}}

\begin{EntryWithPhonetic}{悬}{xuan2}{11}{⼼}[HSK 6]
  \definition{adj.}{pendente; não resolvido; sem nenhum resultado | distante; a distância é grande; a diferença é grande | (dialeto) perigoso}
  \definition{v.}{pendurar; suspender | levantar; elevar | sentir-se ansioso; ser solícito | imaginar}
\end{EntryWithPhonetic}

\begin{EntryWithPhonetic}{悬挂}{xuan2gua4}{11,9}{⼼,⼿}
  \definition{v.}{pendurar; pender; suspender; prender um objeto em um ou mais pontos em algum lugar com a ajuda de uma corda, gancho, prego, etc.}
\end{EntryWithPhonetic}

\begin{EntryWithPhonetic}{悬崖}{xuan2ya2}{11,11}{⼼,⼭}
  \definition{s.}{precipício | penhasco}
\end{EntryWithPhonetic}

%%%%%%%%%% 旋 %%%%%%%%%%
\subsection*{旋}\addcontentsline{loh}{figure}{旋 \dpy{xuan2}}

\begin{EntryWithPhonetic}{旋}{xuan2}{11}{⽅}
  \definition*{s.}{Sobrenome: Xuan}
  \definition{adv.}{em breve; rapidamente}
  \definition{s.}{redemoinho; turbilhão; vórtice}
  \definition{v.}{girar; circular; rodar | retornar; voltar}
\end{EntryWithPhonetic}

\begin{EntryWithPhonetic}{旋转}{xuan2zhuan3}{11,8}{⽅,⾞}[HSK 6]
  \definition{v.}{girar; rodar; revolver; rodopiar; o movimento circular de um objeto em torno de um ponto ou eixo}
\end{EntryWithPhonetic}

%%%%%%%%%% 选 %%%%%%%%%%
\subsection*{选}\addcontentsline{loh}{figure}{选 \dpy{xuan3}}

\begin{EntryWithPhonetic}{选}{xuan3}{9}{⾡}[HSK 2]
  \definition{s.}{pessoa ou coisa selecionada | seleções; antologia; trabalhos selecionados e compilados}
  \definition{v.}{selecionar; escolher | eleger}
\end{EntryWithPhonetic}

\begin{EntryWithPhonetic}{选拔}{xuan3ba2}{9,8}{⾡,⼿}[HSK 6]
  \definition{v.}{selecionar; escolher}
\end{EntryWithPhonetic}

\begin{EntryWithPhonetic}{选举}{xuan3ju3}{9,9}{⾡,⼂}[HSK 6]
  \definition[次,个]{s.}{eleição; as eleições são o processo pelo qual os cidadãos escolhem os seus representantes ou líderes através do voto}
  \definition{v.}{votar; eleger; eleger representantes ou responsáveis ​​votando ou levantando as mãos}
\end{EntryWithPhonetic}

\begin{EntryWithPhonetic}{选手}{xuan3shou3}{9,4}{⾡,⼿}[HSK 3]
  \definition[位,名,个,些]{s.}{jogador; (selecionado) competidor; atleta selecionado para uma competição esportiva; participantes selecionados entre um grande número de candidatos}
\end{EntryWithPhonetic}

\begin{EntryWithPhonetic}{选修}{xuan3 xiu1}{9,9}{⾡,⼈}[HSK 5]
  \definition{v.}{fazer como disciplina eletiva; escolher entre uma seleção de cursos disponíveis}
\end{EntryWithPhonetic}

\begin{EntryWithPhonetic}{选择}{xuan3ze2}{9,8}{⾡,⼿}[HSK 4]
  \definition[个,种,次]{s.}{escolha; opção; resultado da escolha; possibilidade de escolha}
  \definition{v.}{selecionar; escolher}
\end{EntryWithPhonetic}

%%%%%%%%%% 薛 %%%%%%%%%%
\subsection*{薛}\addcontentsline{loh}{figure}{薛 \dpy{xue1}}

\begin{EntryWithPhonetic}{薛}{xue1}{16}{⾋}
  \definition*{s.}{Estado vassalo durante a Dinastia Zhou (1046-256 a.C.) | Sobrenome: Xue}
  \definition{s.}{erva semelhante ao absinto (clássico)}
\end{EntryWithPhonetic}

\begin{EntryWithPhonetic}{薛稷}{xue1 ji4}{16,15}{⾋,⽲}
  \definition*{s.}{Xue Ji (649-713), um dos quatro grandes calígrafos do início da dinastia Tang, 唐初四大家}
  \seealsoref{唐初四大家}{tang2 chu1 si4 da4jia1}
\end{EntryWithPhonetic}

%%%%%%%%%% 学 %%%%%%%%%%
\subsection*{学}\addcontentsline{loh}{figure}{学 \dpy{xue2}}

\begin{EntryWithPhonetic}{学}{xue2}{8}{⼦}[HSK 1]
  \definition[所]{s.}{aprendizagem; conhecimento; sabedoria; erudição | objeto de estudo; ramo do conhecimento | escola; faculdade | teoria; doutrina}
  \definition{v.}{estudar; aprender | imitar; copiar}
\end{EntryWithPhonetic}

\begin{EntryWithPhonetic}{学费}{xue2 fei4}{8,9}{⼦,⾙}[HSK 3]
  \definition[笔]{s.}{mensalidade (taxa); prêmio; taxas que os alunos devem pagar para estudar na escola, conforme estabelecido pela escola | preço pelo que se aprendeu ao custo do próprio bolso; a metáfora do preço a pagar para obter uma determinada experiência | custo; preço; todas as despesas necessárias durante o período de estudos do aluno}
\end{EntryWithPhonetic}

\begin{EntryWithPhonetic}{学分}{xue2fen1}{8,4}{⼦,⼑}[HSK 4]
  \definition{s.}{créditos de um curso; uma unidade de medida do peso e do tempo do curso no ensino superior; cada curso vale um crédito para uma aula por semana durante um semestre; alunos devem concluir o número necessário de créditos para se formar}
\end{EntryWithPhonetic}

\begin{EntryWithPhonetic}{学好}{xue2hao3}{8,6}{⼦,⼥}
  \definition{v.}{seguir bons exemplos | aprender bem}
\end{EntryWithPhonetic}

\begin{EntryWithPhonetic}{学会}{xue2 hui4}{8,6}{⼦,⼈}[HSK 6]
  \definition[个]{s.}{sociedade; instituto; sociedade científica; um grupo acadêmico composto por pessoas que estudam um determinado assunto, como a Sociedade de Física, a Sociedade de Biologia, etc.}
  \definition{v.}{aprender; dominar; aprender e aplicar}
\end{EntryWithPhonetic}

\begin{EntryWithPhonetic}{学科}{xue2 ke1}{8,9}{⼦,⽲}[HSK 5]
  \definition[门,级]{s.}{ramo do aprendizado; disciplina | disciplina escolar; curso de estudo | cursos teóricos oferecidos em treinamento militar ou físico (oposto a 术科)  | disciplina acadêmica | curso | assunto; tema}
  \seealsoref{术科}{shu4ke1}
\end{EntryWithPhonetic}

\begin{EntryWithPhonetic}{学年}{xue2 nian2}{8,6}{⼦,⼲}[HSK 4]
  \definition{s.}{ano letivo; ano acadêmico}
\end{EntryWithPhonetic}

\begin{EntryWithPhonetic}{学期}{xue2qi1}{8,12}{⼦,⽉}[HSK 2]
  \definition[个,段]{s.}{semestre; período escolar; um ano acadêmico é dividido em dois semestres, um semestre do início do outono até as férias de inverno e um semestre do início da primavera até as férias de verão}
\end{EntryWithPhonetic}

\begin{EntryWithPhonetic}{学生}{xue2sheng5}{8,5}{⼦,⽣}[HSK 1]
  \definition{s.}{aluno; estudante; pupilo}
\end{EntryWithPhonetic}

\begin{EntryWithPhonetic}{学生证}{xue2sheng5zheng4}{8,5,7}{⼦,⽣,⾔}
  \definition{s.}{cartão de identidade de estudante}
\end{EntryWithPhonetic}

\begin{EntryWithPhonetic}{学时}{xue2 shi2}{8,7}{⼦,⽇}[HSK 4]
  \definition{s.}{hora-aula; hora de aula; período}
\end{EntryWithPhonetic}

\begin{EntryWithPhonetic}{学术}{xue2shu4}{8,5}{⼦,⽊}[HSK 4]
  \definition[种]{s.}{aprendizagem; aprendizado; ciências; aprendizado sistemático e especializado}
\end{EntryWithPhonetic}

\begin{EntryWithPhonetic}{学位}{xue2wei4}{8,7}{⼦,⼈}[HSK 5]
  \definition[个]{s.}{grau; grau acadêmico; título concedido com base no nível acadêmico profissional, como doutorado, mestrado, etc.}
\end{EntryWithPhonetic}

\begin{EntryWithPhonetic}{学问}{xue2wen4}{8,6}{⼦,⾨}[HSK 4]
  \definition[门,种,个,项]{s.}{aprendizado, conhecimento, erudição; a compreensão correta do mundo objetivo que alguém tem | conhecimento; aprendizado sistemático; conhecimento sistemático sobre algo ou uma ciência que pode ser aprendido em um livro ou em uma experiência prática}
\end{EntryWithPhonetic}

\begin{EntryWithPhonetic}{学习}{xue2xi2}{8,3}{⼦,⼄}[HSK 1]
  \definition{s.}{estudo}
  \definition{v.}{estudar; aprender; adquirir conhecimentos ou habilidades através da leitura, da audição, da pesquisa e da prática}
\end{EntryWithPhonetic}

\begin{EntryWithPhonetic}{学校}{xue2xiao4}{8,10}{⼦,⽊}[HSK 1]
  \definition[所,个]{s.}{escola; instituição de ensino}
\end{EntryWithPhonetic}

\begin{EntryWithPhonetic}{学员}{xue2 yuan2}{8,7}{⼦,⼝}[HSK 6]
  \definition[位,名,批,个]{s.}{estudante; estagiário; geralmente se refere a pessoas que estudam em escolas ou cursos de treinamento diferentes de faculdades, escolas de ensino médio e escolas primárias}
\end{EntryWithPhonetic}

\begin{EntryWithPhonetic}{学院}{xue2yuan4}{8,9}{⼦,⾩}[HSK 1]
  \definition[个,所]{s.}{academia; instituto; um tipo de instituição de ensino superior que se concentra em uma determinada área de especialização, como faculdades de engenharia, faculdades de música, faculdades de educação, etc.}
\end{EntryWithPhonetic}

\begin{EntryWithPhonetic}{学者}{xue2 zhe3}{8,8}{⼦,⽼}[HSK 5]
  \definition[位]{s.}{erudito; homem culto; pessoas que fazem pesquisas acadêmicas geralmente se referem àquelas que alcançaram certo sucesso acadêmico}
\end{EntryWithPhonetic}

%%%%%%%%%% 雪 %%%%%%%%%%
\subsection*{雪}\addcontentsline{loh}{figure}{雪 \dpy{xue3}}

\begin{EntryWithPhonetic}{雪}{xue3}{11}{⾬}[HSK 2]
  \definition*{s.}{Sobrenome: Xue}
  \definition[场,层]{s.}{neve | algo parecido com neve}
  \definition{v.}{limpar; enxugar; remover}
\end{EntryWithPhonetic}

\begin{EntryWithPhonetic}{雪板}{xue3ban3}{11,8}{⾬,⽊}
  \definition{s.}{prancha de \emph{snowboard}}
  \definition{v.}{praticar \textit{snowboard}}
\end{EntryWithPhonetic}

\begin{EntryWithPhonetic}{雪糕}{xue3gao1}{11,16}{⾬,⽶}
  \definition{s.}{picolé}
\end{EntryWithPhonetic}

\begin{EntryWithPhonetic}{雪花}{xue3hua1}{11,7}{⾬,⾋}
  \definition{s.}{floco de neve}
\end{EntryWithPhonetic}

\begin{EntryWithPhonetic}{雪葩}{xue3pa1}{11,12}{⾬,⾋}
  \definition{s.}{sorvete}
\end{EntryWithPhonetic}

\begin{EntryWithPhonetic}{雪人}{xue3ren2}{11,2}{⾬,⼈}
  \definition{s.}{boneco de neve | \emph{Yeti}}
\end{EntryWithPhonetic}

\begin{EntryWithPhonetic}{雪山}{xue3shan1}{11,3}{⾬,⼭}
  \definition{s.}{montanha coberta de neve}
\end{EntryWithPhonetic}

\begin{EntryWithPhonetic}{雪鞋}{xue3xie2}{11,15}{⾬,⾰}
  \definition[双]{s.}{sapatos de neve}
\end{EntryWithPhonetic}

%%%%%%%%%% 血 %%%%%%%%%%
\subsection*{血}\addcontentsline{loh}{figure}{血 \dpy{xue4}}

\begin{EntryWithPhonetic}{血}{xue4}{6}{⾎}[HSK 3][Kangxi 143]
  \definition[滴,袋,口,毫升]{s.}{sangue | parente consanguíneo; com laços de parentesco | pessoa ativa e animada; metáfora para uma personalidade ou espírito forte e sincero | medicina tradicional chinesa refere-se à menstruação}
  \seeref{xie3}
\end{EntryWithPhonetic}

\begin{EntryWithPhonetic}{血管}{xue4 guan3}{6,14}{⾎,⽵}[HSK 6]
  \definition[根,条,种]{s.}{vaso; vaso sanguíneo; os canais tubulares pelos quais o sangue circula são divididos em três tipos: artérias, veias e capilares}
\end{EntryWithPhonetic}

\begin{EntryWithPhonetic}{血汗}{xue4han4}{6,6}{⾎,⽔}
  \definition{s.}{(fig.) suor e labuta, trabalho duro}
\end{EntryWithPhonetic}

\begin{EntryWithPhonetic}{血液}{xue4 ye4}{6,11}{⾎,⽔}[HSK 6]
  \definition[毫升]{s.}{sangue | linha de vida; sangue vital; uma metáfora para o importante componente ou força que mantém a vitalidade coletiva}
\end{EntryWithPhonetic}

%%%%%%%%%% 熏 %%%%%%%%%%
\subsection*{熏}\addcontentsline{loh}{figure}{熏 \dpy{xun1}}

\begin{EntryWithPhonetic}{熏}{xun1}{14}{⽕}
  \definition{v.}{expor à fumaça ou vapores; fumigar | tratar (carne, peixe, etc.) com fumaça; defumar | tornar perfumado com incenso, etc. | sufocar (asfixia e envenenamento por gás)}
\end{EntryWithPhonetic}

\begin{EntryWithPhonetic}{熏香}{xun1xiang1}{14,9}{⽕,⾹}
  \definition{s.}{incenso}
\end{EntryWithPhonetic}

%%%%%%%%%% 寻 %%%%%%%%%%
\subsection*{寻}\addcontentsline{loh}{figure}{寻 \dpy{xun2}}

\begin{EntryWithPhonetic}{寻}{xun2}{6}{⼨}
  \definition*{s.}{Sobrenome: Xun}
  \definition{clas.}{uma unidade antiga de comprimento, igual a 8 尺}
  \definition{v.}{procurar; pesquisar; buscar}
  \seealsoref{尺}{chi3}
\end{EntryWithPhonetic}

\begin{EntryWithPhonetic}{寻求}{xun2 qiu2}{6,7}{⼨,⽔}[HSK 5]
  \definition{v.}{procurar; perseguir; explorar; ir em busca de}
\end{EntryWithPhonetic}

\begin{EntryWithPhonetic}{寻找}{xun2zhao3}{6,7}{⼨,⼿}[HSK 4]
  \definition{v.}{buscar; procurar; pesquisar; encontrar, que pode ser usado tanto para coisas concretas quanto para coisas abstratas}
\end{EntryWithPhonetic}

%%%%%%%%%% 巡 %%%%%%%%%%
\subsection*{巡}\addcontentsline{loh}{figure}{巡 \dpy{xun2}}

\begin{EntryWithPhonetic}{巡}{xun2}{6}{⾡}
  \definition{clas.}{rodada de bebidas | usado para servir vinho a todos}
  \definition{v.}{patrulhar; fazer rondas; fazer uma excursão de inspeção}
\end{EntryWithPhonetic}

\begin{EntryWithPhonetic}{巡逻}{xun2luo2}{6,11}{⾡,⾡}
  \definition{s.}{patrulha}
  \definition{v.}{patrulhar (polícia, exército ou marinha)}
\end{EntryWithPhonetic}

%%%%%%%%%% 询 %%%%%%%%%%
\subsection*{询}\addcontentsline{loh}{figure}{询 \dpy{xun2}}

\begin{EntryWithPhonetic}{询}{xun2}{8}{⾔}
  \definition{v.}{perguntar; indagar; reunir informações | consultar; buscar conselho}
\end{EntryWithPhonetic}

\begin{EntryWithPhonetic}{询问}{xun2wen4}{8,6}{⾔,⾨}[HSK 5]
  \definition{v.}{indagar; perguntar sobre; pedir conselho}
\end{EntryWithPhonetic}

%%%%%%%%%% 循 %%%%%%%%%%
\subsection*{循}\addcontentsline{loh}{figure}{循 \dpy{xun2}}

\begin{EntryWithPhonetic}{循}{xun2}{12}{⼻}
  \definition{v.}{seguir; cumprir; cumprir com}
\end{EntryWithPhonetic}

\begin{EntryWithPhonetic}{循环}{xun2huan2}{12,8}{⼻,⽟}[HSK 6]
  \definition{s.}{ciclo; circulação}
  \definition{v.}{circular; as coisas se movem ou mudam em um ciclo}
\end{EntryWithPhonetic}

%%%%%%%%%% 训 %%%%%%%%%%
\subsection*{训}\addcontentsline{loh}{figure}{训 \dpy{xun4}}

\begin{EntryWithPhonetic}{训}{xun4}{5}{⾔}
  \definition{s.}{instrução; ensinamento; ensino | padrão; modelo; exemplo; regra; diretriz | explicação ou interpretação crítica de um texto | treinamento; exercício}
  \definition{v.}{instruir; admoestar; dar uma palestra a alguém; ensinar | explicar; instruir; explicação do significado da palavra | treinar}
\end{EntryWithPhonetic}

\begin{EntryWithPhonetic}{训诂}{xun4gu3}{5,7}{⾔,⾔}
  \definition{s.}{estudos exegéticos (de textos antigos); exegese}
  \definition{v.}{explicação de palavras e frases em livros antigos | interpretar e elaborar glossários e comentários sobre textos clássicos}
\end{EntryWithPhonetic}

\begin{EntryWithPhonetic}{训练}{xun4lian4}{5,8}{⾔,⽷}[HSK 3]
  \definition{v.}{treinar; exercitar; planejar e executar de forma sistemática o desenvolvimento de habilidades ou competências específicas}
\end{EntryWithPhonetic}

%%%%%%%%%% 迅 %%%%%%%%%%
\subsection*{迅}\addcontentsline{loh}{figure}{迅 \dpy{xun4}}

\begin{EntryWithPhonetic}{迅}{xun4}{6}{⾡}
  \definition{adj.}{rápido; veloz}
  \definition{adv.}{rapidamente; velozmente}
\end{EntryWithPhonetic}

\begin{EntryWithPhonetic}{迅速}{xun4su4}{6,10}{⾡,⾡}[HSK 4]
  \definition{adv.}{rapidamente; velozmente; prontamente}
\end{EntryWithPhonetic}

%%%%% EOF %%%%%


 %%%
%%% Y
%%%
\section*{Y}\addcontentsline{toc}{section}{Y}\addcontentsline{loh}{figure}{\#\#\#\#\#\#\#\# Y}

%%%%%%%%%% 压 %%%%%%%%%%
\subsection*{压}\addcontentsline{loh}{figure}{压 \dpy{ya1}}

\begin{EntryWithPhonetic}{压}{ya1}{6}{⼚}[HSK 3]
  \definition{v.}{pressionar; empurrar para baixo; segurar; pesar | acalmar emoções agitadas ou situações ruins; tranquilizar | intimidar; reprimir; exercer pressão sobre; usar poder, posição ou padrões morais para coagir ou restringir as pessoas, impedindo-as de se expressar, decidir ou se desenvolver livremente | aproximar-se; estar chegando perto | arquivar; deixar de lado | pressionar; metáfora para uma grande carga emocional e psicológica | superar; ultrapassar; voz, capacidade e presença mais fortes do que os outros | apostar em um determinado resultado ao jogar | pressionar; força na superfície de contato do objeto}
  \seeref{ya4}
\end{EntryWithPhonetic}

\begin{EntryWithPhonetic}{压力}{ya1li4}{6,2}{⼚,⼒}[HSK 3]
  \definition[份,个]{s.}{pressão; força atuando perpendicularmente à superfície de um objeto | pressão; força esmagadora; metáfora para a força que coage e intimida as pessoas (principalmente nos aspectos espirituais e psicológicos) | tensão; fardo; os encargos econômicos, psicológicos e espirituais impostos pelo mundo exterior}
\end{EntryWithPhonetic}

\begin{EntryWithPhonetic}{压迫}{ya1po4}{6,8}{⼚,⾡}[HSK 6]
  \definition{v.}{oprimir; reprimir; confiar no poder para suprimir e forçar | contrair; uma força externa comprime uma parte de um organismo}
\end{EntryWithPhonetic}

\begin{EntryWithPhonetic}{压岁钱}{ya1sui4qian2}{6,6,10}{⼚,⼭,⾦}
  \definition{s.}{dinheiro da sorte | dinheiro dado às crianças como presente no Ano Novo Chinês}
\end{EntryWithPhonetic}

\begin{EntryWithPhonetic}{压碎}{ya1sui4}{6,13}{⼚,⽯}
  \definition{v.}{esmagar em pedaços}
\end{EntryWithPhonetic}

\begin{EntryWithPhonetic}{压韵}{ya1yun4}{6,13}{⼚,⾳}
  \variantof{押韵}
\end{EntryWithPhonetic}

%%%%%%%%%% 押 %%%%%%%%%%
\subsection*{押}\addcontentsline{loh}{figure}{押 \dpy{ya1}}

\begin{EntryWithPhonetic}{押}{ya1}{8}{⼿}
  \definition*{s.}{Sobrenome: Ya}
  \definition{s.}{assinatura; marca em vez de assinatura; nome assinado ou símbolo desenhado}
  \definition{v.}{dar como garantia; hipotecar; penhorar | deter; levar sob custódia | escoltar | assinar (um documento, contrato, etc.); colocar sua assinatura (ou marcar no lugar da assinatura)}
\end{EntryWithPhonetic}

\begin{EntryWithPhonetic}{押后}{ya1hou4}{8,6}{⼿,⼝}
  \definition{v.}{encerrar | adiar}
\end{EntryWithPhonetic}

\begin{EntryWithPhonetic}{押金}{ya1jin1}{8,8}{⼿,⾦}[HSK 5]
  \definition[笔,份,些]{s.}{caução; sinal; depósito; dinheiro como garantia}
\end{EntryWithPhonetic}

\begin{EntryWithPhonetic}{押送}{ya1song4}{8,9}{⼿,⾡}
  \definition{v.}{enviar sob escolta | transportar um detido}
\end{EntryWithPhonetic}

\begin{EntryWithPhonetic}{押运}{ya1yun4}{8,7}{⼿,⾡}
  \definition{v.}{escoltar sob guarda | escoltar (bens ou fundos)}
\end{EntryWithPhonetic}

\begin{EntryWithPhonetic}{押韵}{ya1yun4}{8,13}{⼿,⾳}
  \definition{v.}{rimar}
\end{EntryWithPhonetic}

\begin{EntryWithPhonetic}{押注}{ya1zhu4}{8,8}{⼿,⽔}
  \definition{v.}{apostar}
\end{EntryWithPhonetic}

\begin{EntryWithPhonetic}{押租}{ya1zu1}{8,10}{⼿,⽲}
  \definition{s.}{depósito de aluguel}
\end{EntryWithPhonetic}

%%%%%%%%%% 鸭 %%%%%%%%%%
\subsection*{鸭}\addcontentsline{loh}{figure}{鸭 \dpy{ya1}}

\begin{EntryWithPhonetic}{鸭}{ya1}{10}{⿃}
  \definition[只]{s.}{pato | (gíria) prostituto}
\end{EntryWithPhonetic}

\begin{EntryWithPhonetic}{鸭子}{ya1 zi5}{10,3}{⿃,⼦}[HSK 5]
  \definition[只,群]{s.}{pato | Gíria: prostituto}
\end{EntryWithPhonetic}

%%%%%%%%%% 牙 %%%%%%%%%%
\subsection*{牙}\addcontentsline{loh}{figure}{牙 \dpy{ya2}}

\begin{EntryWithPhonetic}{牙}{ya2}{4}{⽛}[HSK 4][Kangxi 92]
  \definition*{s.}{Sobrenome: Ya}
  \definition[颗,排]{s.}{dente | marfim | algo semelhante a um dente}
\end{EntryWithPhonetic}

\begin{EntryWithPhonetic}{牙齿}{ya2chi3}{4,8}{⽛,⿒}
  \definition{adv.}{dental}
  \definition[颗]{s.}{dente}
\end{EntryWithPhonetic}

\begin{EntryWithPhonetic}{牙膏}{ya2gao1}{4,14}{⽛,⾁}
  \definition[管]{s.}{pasta de dente}
\end{EntryWithPhonetic}

\begin{EntryWithPhonetic}{牙行}{ya2hang2}{4,6}{⽛,⾏}
  \definition{s.}{corretor | \emph{broker}}
\end{EntryWithPhonetic}

\begin{EntryWithPhonetic}{牙刷}{ya2 shua1}{4,8}{⽛,⼑}[HSK 4]
  \definition[把,个,支]{s.}{escova de dentes}
\end{EntryWithPhonetic}

\begin{EntryWithPhonetic}{牙线}{ya2xian4}{4,8}{⽛,⽷}
  \definition[条]{s.}{fio dental}
\end{EntryWithPhonetic}

\begin{EntryWithPhonetic}{牙医}{ya2yi1}{4,7}{⽛,⼖}
  \definition{s.}{dentista}
\end{EntryWithPhonetic}

%%%%%%%%%% 崖 %%%%%%%%%%
\subsection*{崖}\addcontentsline{loh}{figure}{崖 \dpy{ya2}}

\begin{EntryWithPhonetic}{崖}{ya2}{11}{⼭}
  \definition{s.}{precipício | penhasco}
\end{EntryWithPhonetic}

%%%%%%%%%% 亚 %%%%%%%%%%
\subsection*{亚}\addcontentsline{loh}{figure}{亚 \dpy{ya4}}

\begin{EntryWithPhonetic}{亚}{ya4}{6}{⼆}
  \definition*{s.}{Ásia, abreviação de 亚洲 | Sobrenome: Ya}
  \definition{adj.}{inferior | abaixo do padrão | (química) de menor valência atômica}
  \definition{pref.}{sub-}
  \seealsoref{亚洲}{ya4zhou1}
\end{EntryWithPhonetic}

\begin{EntryWithPhonetic}{亚军}{ya4jun1}{6,6}{⼆,⼍}[HSK 5]
  \definition[个]{s.}{segundo lugar; vice-campeão; medalhista de prata}
\end{EntryWithPhonetic}

\begin{EntryWithPhonetic}{亚热带}{ya4re4dai4}{6,10,9}{⼆,⽕,⼱}
  \definition{s.}{zona ou clima subtropical; subtropical; semitropical}
\end{EntryWithPhonetic}

\begin{EntryWithPhonetic}{亚细亚洲}{ya4xi4ya4zhou1}{6,8,6,9}{⼆,⽷,⼆,⽔}
  \definition*{s.}{Ásia}
\end{EntryWithPhonetic}

\begin{EntryWithPhonetic}{亚运会}{ya4 yun4 hui4}{6,7,6}{⼆,⾡,⼈}[HSK 4]
  \definition*{s.}{Jogos Asiáticos}
\end{EntryWithPhonetic}

\begin{EntryWithPhonetic}{亚洲}{ya4zhou1}{6,9}{⼆,⽔}
  \definition*{s.}{Ásia, abreviação de 亚细亚洲}
  \seealsoref{亚细亚洲}{ya4xi4ya4zhou1}
\end{EntryWithPhonetic}

\begin{EntryWithPhonetic}{亚洲人}{ya4zhou1ren2}{6,9,2}{⼆,⽔,⼈}
  \definition{s.}{asiático | pessoa ou povo da Ásia}
\end{EntryWithPhonetic}

%%%%%%%%%% 压 %%%%%%%%%%
\subsection*{压}\addcontentsline{loh}{figure}{压 \dpy{ya4}}

\begin{EntryWithPhonetic}{压}{ya4}{6}{⼚}
  \definition{adv.}{fundamentalmente; nunca (usado principalmente em frases negativas)}
  \seeref{ya1}
  \seealsoref{压根儿}{ya4gen1r5}
\end{EntryWithPhonetic}

\begin{EntryWithPhonetic}{压根儿}{ya4gen1r5}{6,10,2}{⼚,⽊,⼉}
  \definition{adv.}{(geralmente no negativo) nunca; fundamentalmente}
\end{EntryWithPhonetic}

%%%%%%%%%% 呀 %%%%%%%%%%
\subsection*{呀}\addcontentsline{loh}{figure}{呀 \dpy{ya5}}

\begin{EntryWithPhonetic}{呀}{ya5}{7}{⼝}[HSK 4]
  \definition{part.}{usado no lugar de 啊 quando a palavra anterior termina com o som a, e, i, o ou ü}
  \seealsoref{啊}{a5}
\end{EntryWithPhonetic}

%%%%%%%%%% 烟 %%%%%%%%%%
\subsection*{烟}\addcontentsline{loh}{figure}{烟 \dpy{yan1}}

\begin{EntryWithPhonetic}{烟}{yan1}{10}{⽕}[HSK 3]
  \definition[股,支,根,盒,包]{s.}{fumaça; gás produzido pela combustão de materiais, misturado com pequenas partículas não completamente queimadas | névoa; neblina | tabaco; planta de tabaco | fumo; cigarro; termo geral para cigarros, charutos, etc. | ópio | fuligem; fumaça de carvão}
  \definition{v.}{ficar irritado com a fumaça (os olhos lacrimejam ou não conseguem abrir)}
\end{EntryWithPhonetic}

\begin{EntryWithPhonetic}{烟草}{yan1cao3}{10,9}{⽕,⾋}
  \definition{s.}{tabaco}
\end{EntryWithPhonetic}

\begin{EntryWithPhonetic}{烟囱}{yan1cong1}{10,7}{⽕,⼞}
  \definition{s.}{chaminé}
\end{EntryWithPhonetic}

\begin{EntryWithPhonetic}{烟花}{yan1 hua1}{10,7}{⽕,⾋}[HSK 6]
  \definition[场,朵]{s.}{fogos de artifício; uma coisa que emite faíscas de várias cores quando exposta à observação | prostituta; antigamente, referia-se a algo relacionado à prostituição}
\end{EntryWithPhonetic}

\begin{EntryWithPhonetic}{烟火}{yan1huo3}{10,4}{⽕,⽕}
  \definition{s.}{fogo de artifício}
\end{EntryWithPhonetic}

\begin{EntryWithPhonetic}{烟头}{yan1tou2}{10,5}{⽕,⼤}
  \definition[根]{s.}{bituca de cigarro}
\end{EntryWithPhonetic}

\begin{EntryWithPhonetic}{烟叶}{yan1ye4}{10,5}{⽕,⼝}
  \definition{s.}{folha de tabaco}
\end{EntryWithPhonetic}

\begin{EntryWithPhonetic}{烟雨}{yan1yu3}{10,8}{⽕,⾬}
  \definition{s.}{chuvisco | garoa}
\end{EntryWithPhonetic}

%%%%%%%%%% 延 %%%%%%%%%%
\subsection*{延}\addcontentsline{loh}{figure}{延 \dpy{yan2}}

\begin{EntryWithPhonetic}{延}{yan2}{6}{⼵}
  \definition*{s.}{Sobrenome: Yan}
  \definition{v.}{prolongar; estender; alongar | adiar; atrasar | envolver (um professor, conselheiro, etc.); enviar para; convidar}
\end{EntryWithPhonetic}

\begin{EntryWithPhonetic}{延长}{yan2chang2}{6,4}{⼵,⾧}[HSK 4]
  \definition{v.}{estender; prolongar; alongar; aumentar o tempo, a distância ou a duração de algo específico}
\end{EntryWithPhonetic}

\begin{EntryWithPhonetic}{延期}{yan2/qi1}{6,12}{⼵,⽉}[HSK 4]
  \definition{v.+compl.}{atrasar; adiar; postergar}
\end{EntryWithPhonetic}

\begin{EntryWithPhonetic}{延伸}{yan2shen1}{6,7}{⼵,⼈}[HSK 5]
  \definition{v.}{estender; esticar; alongar; estender-se}
\end{EntryWithPhonetic}

\begin{EntryWithPhonetic}{延续}{yan2xu4}{6,11}{⼵,⽷}[HSK 4]
  \definition{v.}{durar; continuar; prosseguir; continuar como antes; prolongar}
\end{EntryWithPhonetic}

%%%%%%%%%% 严 %%%%%%%%%%
\subsection*{严}\addcontentsline{loh}{figure}{严 \dpy{yan2}}

\begin{EntryWithPhonetic}{严}{yan2}{7}{⼀}[HSK 4]
  \definition*{s.}{Sobrenome: Yan}
  \definition{adj.}{apertado; próximo | rigoroso; severo; duro; áspero; rigoroso; austero | severo; extremo; difícil}
  \definition{s.}{pai; refere-se ao pai}
\end{EntryWithPhonetic}

\begin{EntryWithPhonetic}{严格}{yan2ge2}{7,10}{⼀,⽊}[HSK 4]
  \definition{adj.}{rígido; estrito; rigoroso; muito consciente e meticuloso na implementação de sistemas e no domínio de padrões}
  \definition{v.}{tornar (sistemas, provisões, etc.) rigorosos}
\end{EntryWithPhonetic}

\begin{EntryWithPhonetic}{严厉}{yan2li4}{7,5}{⼀,⼚}[HSK 5]
  \definition{adj.}{severo; rigoroso; as palavras e atitudes de crítica ou punição são muito sérias e severas}
\end{EntryWithPhonetic}

\begin{EntryWithPhonetic}{严肃}{yan2su4}{7,8}{⼀,⾀}[HSK 5]
  \definition{adj.}{sério; solene; sincero; (expressão, atmosfera, etc.) faz as pessoas se sentirem admiradas e desconfortáveis | sóbrio; grave; sério; sincero}
  \definition{v.}{aplicar rigorosamente; fazer algo sério}
\end{EntryWithPhonetic}

\begin{EntryWithPhonetic}{严重}{yan2zhong4}{7,9}{⼀,⾥}[HSK 4]
  \definition{adj.}{sério; grave; crítico; severo}
\end{EntryWithPhonetic}

\begin{EntryWithPhonetic}{严重打伤}{yan2zhong4 da3 shang1}{7,9,5,6}{⼀,⾥,⼿,⼈}
  \definition{s.}{gravemente ferido}
\end{EntryWithPhonetic}

\begin{EntryWithPhonetic}{严重地}{yan2zhong4 di4}{7,9,6}{⼀,⾥,⼟}
  \definition{adv.}{seriamente | gravemente}
\end{EntryWithPhonetic}

\begin{EntryWithPhonetic}{严重关切}{yan2zhong4guan1qie4}{7,9,6,4}{⼀,⾥,⼋,⼑}
  \definition{s.}{preocupação séria}
\end{EntryWithPhonetic}

\begin{EntryWithPhonetic}{严重后果}{yan2zhong4hou4guo3}{7,9,6,8}{⼀,⾥,⼝,⽊}
  \definition{s.}{consequências sérias | repercursões graves}
\end{EntryWithPhonetic}

\begin{EntryWithPhonetic}{严重破坏}{yan2zhong4 po4huai4}{7,9,10,7}{⼀,⾥,⽯,⼟}
  \definition{s.}{destruição grave}
\end{EntryWithPhonetic}

\begin{EntryWithPhonetic}{严重伤害}{yan2zhong4 shang1hai4}{7,9,6,10}{⼀,⾥,⼈,⼧}
  \definition{s.}{ferimento grave; lesão grave}
\end{EntryWithPhonetic}

\begin{EntryWithPhonetic}{严重危害}{yan2zhong4 wei1hai4}{7,9,6,10}{⼀,⾥,⼙,⼧}
  \definition{s.}{perigo crítico | dano grave}
\end{EntryWithPhonetic}

\begin{EntryWithPhonetic}{严重问题}{yan2zhong4 wen4ti2}{7,9,6,15}{⼀,⾥,⾨,⾴}
  \definition{s.}{problema sério}
\end{EntryWithPhonetic}

\begin{EntryWithPhonetic}{严重性}{yan2zhong4xing4}{7,9,8}{⼀,⾥,⼼}
  \definition{s.}{seriedade | gravidade}
\end{EntryWithPhonetic}

%%%%%%%%%% 言 %%%%%%%%%%
\subsection*{言}\addcontentsline{loh}{figure}{言 \dpy{yan2}}

\begin{EntryWithPhonetic}{言}{yan2}{7}{⾔}[Kangxi 149]
  \definition*{s.}{Sobrenome: Yan}
  \definition{s.}{palavra; discurso; o que foi dito | palavra; caracter; uma frase ou palavra chinesa}
  \definition{v.}{dizer; falar}
\end{EntryWithPhonetic}

\begin{EntryWithPhonetic}{言论}{yan2lun4}{7,6}{⾔,⾔}
  \definition{s.}{expressão de opinião |  visualizações | comentários | argumentos}
\end{EntryWithPhonetic}

\begin{EntryWithPhonetic}{言语}{yan2 yu3}{7,9}{⾔,⾔}[HSK 5]
  \definition{s.}{verbal; fala; linguagem falada; conversa; palavras}
\end{EntryWithPhonetic}

%%%%%%%%%% 沿 %%%%%%%%%%
\subsection*{沿}\addcontentsline{loh}{figure}{沿 \dpy{yan2}}

\begin{EntryWithPhonetic}{沿}{yan2}{8}{⽔}[HSK 6]
  \definition{prep.}{ao longo}
  \definition{s.}{beira; borda; acabamento}
  \definition{v.}{seguir (uma tradição, padrão, etc.) | enfeitar (com fita, faixa, etc.)}
\end{EntryWithPhonetic}

\begin{EntryWithPhonetic}{沿海}{yan2hai3}{8,10}{⽔,⽔}[HSK 6]
  \definition{s.}{costa; litoral; área ou região ao longo da costa}
\end{EntryWithPhonetic}

\begin{EntryWithPhonetic}{沿着}{yan2 zhe5}{8,11}{⽔,⽬}[HSK 6]
  \definition{prep.}{ao longo (de uma determinada rota)}
\end{EntryWithPhonetic}

%%%%%%%%%% 炎 %%%%%%%%%%
\subsection*{炎}\addcontentsline{loh}{figure}{炎 \dpy{yan2}}

\begin{EntryWithPhonetic}{炎}{yan2}{8}{⽕}
  \definition{adj.}{escaldante; ardente}
  \definition{s.}{inflamação | poder; influência}
\end{EntryWithPhonetic}

\begin{EntryWithPhonetic}{炎热}{yan2re4}{8,10}{⽕,⽕}
  \definition{adj.}{extremamente quente | escaldante (clima)}
\end{EntryWithPhonetic}

%%%%%%%%%% 研 %%%%%%%%%%
\subsection*{研}\addcontentsline{loh}{figure}{研 \dpy{yan2}}

\begin{EntryWithPhonetic}{研}{yan2}{9}{⽯}
  \definition{s.}{(abreviação)  pesquisador adjunto, 副研}
  \definition{v.}{moer; esmerilhar; triturar; pulverizar | estudar; pesquisar}
  \seealsoref{副研}{fu4yan2}
\end{EntryWithPhonetic}

\begin{EntryWithPhonetic}{研发}{yan2 fa1}{9,5}{⽯,⼜}[HSK 6]
  \definition{s.}{pesquisa e desenvolvimento; P\&D}
  \definition{v.}{pesquisar e/ou desenvolver}
\end{EntryWithPhonetic}

\begin{EntryWithPhonetic}{研究}{yan2jiu1}{9,7}{⽯,⽳}[HSK 4]
  \definition{v.}{estudar; pesquisar | discutir; considerar}
\end{EntryWithPhonetic}

\begin{EntryWithPhonetic}{研究生}{yan2 jiu1 sheng1}{9,7,5}{⽯,⽳,⽣}[HSK 4]
  \definition[位,名,个,些]{s.}{pós-graduado; estudante de pós-graduação}
\end{EntryWithPhonetic}

\begin{EntryWithPhonetic}{研究所}{yan2 jiu1 suo3}{9,7,8}{⽯,⽳,⼾}[HSK 5]
  \definition[家,个]{s.}{instituto de pesquisa; instituição de pesquisa científica envolvida em pesquisas em um determinado campo}
\end{EntryWithPhonetic}

\begin{EntryWithPhonetic}{研制}{yan2 zhi4}{9,8}{⽯,⼑}[HSK 4]
  \definition{v.}{desenvolver; fabricar; produzir | triturar; (medicina chinesa) moer}
\end{EntryWithPhonetic}

%%%%%%%%%% 盐 %%%%%%%%%%
\subsection*{盐}\addcontentsline{loh}{figure}{盐 \dpy{yan2}}

\begin{EntryWithPhonetic}{盐}{yan2}{10}{⽫}[HSK 4]
  \definition[袋,勺,把,包,粒]{s.}{sal (de cozinha) | Química: sal (produto formado pela neutralização de um ácido por uma base)}
\end{EntryWithPhonetic}

%%%%%%%%%% 颜 %%%%%%%%%%
\subsection*{颜}\addcontentsline{loh}{figure}{颜 \dpy{yan2}}

\begin{EntryWithPhonetic}{颜}{yan2}{15}{⾴}
  \definition*{s.}{Sobrenome: Yan}
  \definition{s.}{rosto; semblante; expressão facial | rosto; prestígio; dignidade | cor}
\end{EntryWithPhonetic}

\begin{EntryWithPhonetic}{颜色}{yan2 se4}{15,6}{⾴,⾊}[HSK 2]
  \definition[个,种]{s.}{cor; a sensação visual de um objeto é uma impressão diferente produzida pelas diferentes quantidades de luz absorvidas e refletidas pelo objeto | tez; semblante; aparência; geralmente se refere à aparência de uma garota | olhar severo no rosto como um aviso; um olhar ou ação que faz os outros parecerem particularmente ferozes | a expressão mostrada no rosto}
\end{EntryWithPhonetic}

%%%%%%%%%% 广 %%%%%%%%%%
\subsection*{广}\addcontentsline{loh}{figure}{广 \dpy{yan3}}

\begin{EntryWithPhonetic}{广}{yan3}{3}{⼴}[Kangxi 53]
  \definition[家]{s.}{casa ou edifício construído contra ou ao longo da encosta de uma montanha ou penhasco}
  \seeref{an1}
  \seeref{guang3}
\end{EntryWithPhonetic}

%%%%%%%%%% 眼 %%%%%%%%%%
\subsection*{眼}\addcontentsline{loh}{figure}{眼 \dpy{yan3}}

\begin{EntryWithPhonetic}{眼}{yan3}{11}{⽬}[HSK 2]
  \definition{clas.}{usado para grandes coisas ocas: poços, fogões, panelas, etc.}
  \definition[双,只]{s.}{olho; o órgão visual dos humanos ou animais | abertura; pequeno furo; pequeno buraco | ponto-chave; refere-se ao ponto-chave das coisas | armadilha; um termo do jogo Go que se refere a um espaço vazio cercado pelas peças de um jogador, onde o outro jogador não pode colocar uma peça, a menos que haja circunstâncias especiais | uma batida sem acento na música tradicional chinesa}
\end{EntryWithPhonetic}

\begin{EntryWithPhonetic}{眼柄}{yan3bing3}{11,9}{⽬,⽊}
  \definition{s.}{pedúnculo ocular (de crustáceo, etc.)}
\end{EntryWithPhonetic}

\begin{EntryWithPhonetic}{眼袋}{yan3dai4}{11,11}{⽬,⾐}
  \definition{s.}{inchaço sob os olhos}
\end{EntryWithPhonetic}

\begin{EntryWithPhonetic}{眼光}{yan3guang1}{11,6}{⽬,⼉}[HSK 5]
  \definition{s.}{olho; visão | visão; percepção; previsão; capacidade de observar e identificar coisas | vista; ponto de vista}
\end{EntryWithPhonetic}

\begin{EntryWithPhonetic}{眼花缭乱}{yan3hua1liao2luan4}{11,7,15,7}{⽬,⾋,⽷,⼄}
  \definition{v.}{ficar deslumbrado | deslumbrar}
\end{EntryWithPhonetic}

\begin{EntryWithPhonetic}{眼镜}{yan3jing4}{11,16}{⽬,⾦}[HSK 4]
  \definition[副]{s.}{óculos; óculos de grau; lentes usadas nos olhos para melhorar a visão ou proteger os olhos, feitas de vidro ou cristal incolor ou colorido}
\end{EntryWithPhonetic}

\begin{EntryWithPhonetic}{眼睛}{yan3jing5}{11,13}{⽬,⽬}[HSK 2]
  \definition[双,只]{s.}{olho(s)}
\end{EntryWithPhonetic}

\begin{EntryWithPhonetic}{眼看}{yan3 kan4}{11,9}{⽬,⽬}[HSK 6]
  \definition{adv.}{em breve; em um momento; imediatamente}
  \definition{v.}{observar impotentemente; olhar passivamente; observar (o que está acontecendo)}
\end{EntryWithPhonetic}

\begin{EntryWithPhonetic}{眼泪}{yan3 lei4}{11,8}{⽬,⽔}[HSK 4]
  \definition[滴,行]{s.}{lágrimas; termo genérico para lágrimas; fluido incolor e transparente secretado pelas glândulas lacrimais no olho, que serve para proteger o olho}
\end{EntryWithPhonetic}

\begin{EntryWithPhonetic}{眼里}{yan3 li3}{11,7}{⽬,⾥}[HSK 4]
  \definition{s.}{aos olhos de alguém; na opinião (ou visão) de alguém}
\end{EntryWithPhonetic}

\begin{EntryWithPhonetic}{眼前}{yan3 qian2}{11,9}{⽬,⼑}[HSK 3]
  \definition{adv.}{agora; (no) momento}
  \definition{s.}{diante dos olhos; diante de | agora; (no) momento}
\end{EntryWithPhonetic}

\begin{EntryWithPhonetic}{眼证}{yan3zheng4}{11,7}{⽬,⾔}
  \definition{s.}{testemunha ocular}
\end{EntryWithPhonetic}

%%%%%%%%%% 演 %%%%%%%%%%
\subsection*{演}\addcontentsline{loh}{figure}{演 \dpy{yan3}}

\begin{EntryWithPhonetic}{演}{yan3}{14}{⽔}[HSK 3]
  \definition{v.}{desenvolver; evoluir | deduzir; elaborar | exercitar; praticar | representar; atuar; encenar | desempenhar}
\end{EntryWithPhonetic}

\begin{EntryWithPhonetic}{演唱}{yan3 chang4}{14,11}{⽔,⼝}[HSK 3]
  \definition{v.}{cantar em uma performance; apresentar canções, óperas, peças teatrais, etc.}
\end{EntryWithPhonetic}

\begin{EntryWithPhonetic}{演唱会}{yan3 chang4 hui4}{14,11,6}{⽔,⼝,⼈}[HSK 3]
  \definition[个,场]{s.}{recital vocal; concerto vocal; uma forma de apresentação centrada no canto, acompanhada por movimentos de dança simples}
\end{EntryWithPhonetic}

\begin{EntryWithPhonetic}{演出}{yan3chu1}{14,5}{⽔,⼐}[HSK 3]
  \definition[场,次]{s.}{show; concerto; performance}
  \definition{v.}{apresentar; representar; fazer um show; apresentar peças teatrais, danças, artes cênicas, acrobacias, etc. para o público apreciar}
\end{EntryWithPhonetic}

\begin{EntryWithPhonetic}{演讲}{yan3jiang3}{14,6}{⽔,⾔}[HSK 4]
  \definition[场,次]{s.}{palestra; discurso; ato ou a atividade de apresentar ou expressar ideias, opiniões ou informações oralmente em público ou diante de um público}
  \definition{v.}{dar uma palestra; fazer um discurso; informar o público sobre uma determinada área de conhecimento ou opinião sobre um determinado assunto}
\end{EntryWithPhonetic}

\begin{EntryWithPhonetic}{演员}{yan3yuan2}{14,7}{⽔,⼝}[HSK 3]
  \definition[个,位,名]{s.}{ator; artista; pessoas que participam de apresentações teatrais, cinematográficas, de dança, de artes cênicas, de acrobacias, etc.}
\end{EntryWithPhonetic}

\begin{EntryWithPhonetic}{演奏}{yan3zou4}{14,9}{⽔,⼤}[HSK 6]
  \definition{v.}{tocar um instrumento musical; fazer uma apresentação instrumental}
\end{EntryWithPhonetic}

%%%%%%%%%% 宴 %%%%%%%%%%
\subsection*{宴}\addcontentsline{loh}{figure}{宴 \dpy{yan4}}

\begin{EntryWithPhonetic}{宴}{yan4}{10}{⼧}
  \definition{adj.}{tranquilo e confortável}
  \definition[个,场]{s.}{festa; banquete | facilidade e conforto; felicidade; lazer}
  \definition{v.}{entreter em um banquete; festejar}
\end{EntryWithPhonetic}

\begin{EntryWithPhonetic}{宴会}{yan4hui4}{10,6}{⼧,⼈}[HSK 6]
  \definition[个,场,次]{s.}{festa; banquete; jantar; uma reunião onde convidados e anfitriões bebem e comem juntos (referindo-se a uma ocasião mais solene)}
\end{EntryWithPhonetic}

%%%%%%%%%% 扬 %%%%%%%%%%
\subsection*{扬}\addcontentsline{loh}{figure}{扬 \dpy{yang2}}

\begin{EntryWithPhonetic}{扬}{yang2}{6}{⼿}
  \definition*{s.}{Yangzhou, abreviação de 扬州 | Sobrenome: Yang}
  \definition{v.}{levantar | separar e espalhar; peneirar | espalhar; fazer conhecido}
  \seealsoref{扬州}{yang2zhou1}
\end{EntryWithPhonetic}

\begin{EntryWithPhonetic}{扬雄}{yang2xiong2}{6,12}{⼿,⾫}
  \definition*{s.}{Yang Xiong (53 AC-18 DC), estudioso, poeta e lexicógrafo, autor do primeiro dicionário de dialeto chinês 方言}
  \seealsoref{方言}{fang1yan2}
\end{EntryWithPhonetic}

\begin{EntryWithPhonetic}{扬州}{yang2zhou1}{6,6}{⼿,⼮}
  \definition*{s.}{Yangzhou, uma cidade na província de Jiangsu}
\end{EntryWithPhonetic}

%%%%%%%%%% 羊 %%%%%%%%%%
\subsection*{羊}\addcontentsline{loh}{figure}{羊 \dpy{yang2}}

\begin{EntryWithPhonetic}{羊}{yang2}{6}{⽺}[HSK 3][Kangxi 123]
  \definition*{s.}{Sobrenome: Yang}
  \definition[只,头,群]{s.}{carneiro; ovelha; bode; cabra; antílope}
\end{EntryWithPhonetic}

%%%%%%%%%% 阳 %%%%%%%%%%
\subsection*{阳}\addcontentsline{loh}{figure}{阳 \dpy{yang2}}

\begin{EntryWithPhonetic}{阳}{yang2}{6}{⾩}
  \definition*{s.}{Sobrenome: Yang}
  \definition{adj.}{em relevo | aberto; evidente; revelado | pertencente a este mundo; preocupado com os seres vivos; superstição se refere a coisas que pertencem aos vivos e ao mundo | positivo; carregado positivamente}
  \definition{s.}{o Sol; luz solar | ao sul de uma colina ou ao norte de um rio | Yang (o princípio masculino ou positivo da natureza); na filosofia chinesa antiga, refere-se a um dos dois opostos que permeiam todas as coisas no universo (o outro lado é Yin) | masculino; refere-se aos órgãos genitais masculinos ou à função reprodutiva | varanda; refere-se ao lugar onde o sol brilha (oposto a 阴)}
  \seealsoref{阴}{yin1}
  \seealsoref{阴阳}{yin1yang2}
\end{EntryWithPhonetic}

\begin{EntryWithPhonetic}{阳光}{yang2guang1}{6,6}{⾩,⼉}[HSK 3]
  \definition{adj.}{alegre; otimista; personalidade positiva e alegre; cheio de vitalidade juvenil | aberto; transparente; público; conduzido sob supervisão pública}
  \definition[缕,束,道]{s.}{luz do sol; raio de sol}
\end{EntryWithPhonetic}

\begin{EntryWithPhonetic}{阳台}{yang2tai2}{6,5}{⾩,⼝}[HSK 4]
  \definition[个,块,处]{s.}{varanda; terraço; sacada; pequeno terraço do edifício com grades para se refrescar, tomar sol ou olhar o horizonte}
\end{EntryWithPhonetic}

%%%%%%%%%% 洋 %%%%%%%%%%
\subsection*{洋}\addcontentsline{loh}{figure}{洋 \dpy{yang2}}

\begin{EntryWithPhonetic}{洋}{yang2}{9}{⽔}[HSK 6]
  \definition*{s.}{Sobrenome: Yang}
  \definition{adj.}{vasto; rico; transbordante | estrangeiro (especialmente ocidental) | moderno (oposto a 土)}
  \definition[个,片]{s.}{oceano | moeda de prata}
  \seealsoref{土}{tu3}
\end{EntryWithPhonetic}

\begin{EntryWithPhonetic}{洋葱}{yang2cong1}{9,12}{⽔,⾋}
  \definition{s.}{cebola}
\end{EntryWithPhonetic}

%%%%%%%%%% 仰 %%%%%%%%%%
\subsection*{仰}\addcontentsline{loh}{figure}{仰 \dpy{yang3}}

\begin{EntryWithPhonetic}{仰}{yang3}{6}{⼈}[HSK 6]
  \definition*{s.}{Sobrenome: Yang}
  \definition{v.}{levantar (oposto a 俯) | virar para cima | admirar; respeitar | confiar em; depender de}
  \seealsoref{俯}{fu3}
\end{EntryWithPhonetic}

%%%%%%%%%% 养 %%%%%%%%%%
\subsection*{养}\addcontentsline{loh}{figure}{养 \dpy{yang3}}

\begin{EntryWithPhonetic}{养}{yang3}{9}{⼋}[HSK 2]
  \definition*{s.}{Sobrenome: Yang}
  \definition{adj.}{adotivo; órfão; adotado; não biológico}
  \definition{s.}{qualidade; (caráter moral, desempenho acadêmico, etc.) boas qualidades}
  \definition{v.}{apoiar; prover; fornecer dinheiro e materiais necessários para viver | aumentar; manter; crescer; alimentar os animais e cuidar de suas vidas para que possam crescer | dar à luz | formar; adquirir; cultivar | descansar; curar; convalescer; recuperar a saúde | manter; manter em bom estado | deixar (o cabelo) crescer | ajudar; apoiar | cultivar (plantações ou flores)}
\end{EntryWithPhonetic}

\begin{EntryWithPhonetic}{养成}{yang3cheng2}{9,6}{⼋,⼽}[HSK 4]
  \definition{v.}{cultivar; desenvolver; cultivar para formar; nutrir para crescer}
\end{EntryWithPhonetic}

\begin{EntryWithPhonetic}{养分}{yang3fen4}{9,4}{⼋,⼑}
  \definition{s.}{nutriente}
\end{EntryWithPhonetic}

\begin{EntryWithPhonetic}{养老}{yang3 lao3}{9,6}{⼋,⽼}[HSK 6]
  \definition{v.}{prover assistência aos idosos (geralmente os pais) | viver a vida na aposentadoria; refere-se ao idoso que descansa em casa}
\end{EntryWithPhonetic}

\begin{EntryWithPhonetic}{养料}{yang3liao4}{9,10}{⼋,⽃}
  \definition{s.}{nutrição}
\end{EntryWithPhonetic}

%%%%%%%%%% 氧 %%%%%%%%%%
\subsection*{氧}\addcontentsline{loh}{figure}{氧 \dpy{yang3}}

\begin{EntryWithPhonetic}{氧}{yang3}{10}{⽓}
  \definition{s.}{oxigênio}
\end{EntryWithPhonetic}

\begin{EntryWithPhonetic}{氧气}{yang3qi4}{10,4}{⽓,⽓}[HSK 6]
  \definition{s.}{oxigênio (O); gás oxigênio}
\end{EntryWithPhonetic}

%%%%%%%%%% 样 %%%%%%%%%%
\subsection*{样}\addcontentsline{loh}{figure}{样 \dpy{yang4}}

\begin{EntryWithPhonetic}{样}{yang4}{10}{⽊}[HSK 6]
  \definition{clas.}{usado para tipos de coisas}[这里有四样东西。===Há quatro coisas aqui.]
  \definition[个]{s.}{aparência; aspecto;  forma; aparência; a forma do objeto | modelo; amostra; padrão; coisas usadas como padrões | ar; maneira; aparência; a aparência ou expressão de uma pessoa | tendência; probabilidade; a situação ou tendência das coisas}
\end{EntryWithPhonetic}

\begin{EntryWithPhonetic}{样品}{yang4pin3}{10,9}{⽊,⼝}
  \definition{s.}{amostra | espécime}
\end{EntryWithPhonetic}

\begin{EntryWithPhonetic}{样儿}{yang4r5}{10,2}{⽊,⼉}
  \definition{s.}{aparência | forma | modelo}
  \seealsoref{样子}{yang4zi5}
\end{EntryWithPhonetic}

\begin{EntryWithPhonetic}{样样}{yang4yang4}{10,10}{⽊,⽊}
  \definition{adv.}{todos os tipos}
\end{EntryWithPhonetic}

\begin{EntryWithPhonetic}{样章}{yang4zhang1}{10,11}{⽊,⾳}
  \definition{s.}{capítulo de amostra}
\end{EntryWithPhonetic}

\begin{EntryWithPhonetic}{样子}{yang4zi5}{10,3}{⽊,⼦}[HSK 2]
  \definition[个,种,副]{s.}{forma; aparência; estilo | ar; maneira; modalidade; estado | tendência; probabilidade; usado com 看 e 照 para expressar uma estimativa de uma tendência | modelo; amostra; padrão; uma pessoa ou coisa que pode ser usada como um padrão para as pessoas verificarem, seguirem ou aprenderem com ela}
  \seealsoref{看}{kan4}
  \seealsoref{样儿}{yang4r5}
  \seealsoref{照}{zhao4}
\end{EntryWithPhonetic}

%%%%%%%%%% 约 %%%%%%%%%%
\subsection*{约}\addcontentsline{loh}{figure}{约 \dpy{yao1}}

\begin{EntryWithPhonetic}{约}{yao1}{6}{⽷}
  \definition{adj.}{econômico; frugal | simples; breve | indistinto}
  \definition{adv.}{cerca de; ao redor; aproximadamente}
  \definition{s.}{pacto; acordo; nomeação; coisa prometida}
  \definition{v.}{marcar uma consulta; organizar | perguntar ou convidar com antecedência | restringir; conter | reduzir (fração aproximada)}
  \seeref{yue1}
\end{EntryWithPhonetic}

%%%%%%%%%% 妖 %%%%%%%%%%
\subsection*{妖}\addcontentsline{loh}{figure}{妖 \dpy{yao1}}

\begin{EntryWithPhonetic}{妖}{yao1}{7}{⼥}
  \definition{adj.}{maligno e fraudulento | sedutor; encantador | paquerador}
  \definition[个,只]{s.}{\emph{goblin}; demônio; espírito maligno}
\end{EntryWithPhonetic}

%%%%%%%%%% 祅 %%%%%%%%%%
\subsection*{祅}\addcontentsline{loh}{figure}{祅 \dpy{yao1}}

\begin{EntryWithPhonetic}{祅}{yao1}{8}{⽰}
  \definition{s.}{espírito maligno | \emph{goblin} | bruxaria}
  \variantof{妖}
\end{EntryWithPhonetic}

%%%%%%%%%% 要 %%%%%%%%%%
\subsection*{要}\addcontentsline{loh}{figure}{要 \dpy{yao1}}

\begin{EntryWithPhonetic}{要}{yao1}{9}{⾑}[HSK 1]
  \definition*{s.}{Sobrenome: Yao}
  \definition{v.}{exigir; pedir; requerer; solicitar; buscar; insistir com base em algo em que se apoia | forçar; coagir; ameaçar}
  \seeref{yao4}
\end{EntryWithPhonetic}

\begin{EntryWithPhonetic}{要求}{yao1qiu2}{9,7}{⾑,⽔}[HSK 2]
  \definition[个,点]{s.}{exigência; demanda; reivindicação; desejos ou condições específicas propostas}
  \definition{v.}{pedir; exigir; exigir; reivindicar; apresentar desejos ou condições específicas, esperando que sejam satisfeitos ou realizados}
\end{EntryWithPhonetic}

\begin{EntryWithPhonetic}{要挟}{yao1xie2}{9,9}{⾑,⼿}
  \definition{v.}{chantagear | ameaçar}
\end{EntryWithPhonetic}

%%%%%%%%%% 腰 %%%%%%%%%%
\subsection*{腰}\addcontentsline{loh}{figure}{腰 \dpy{yao1}}

\begin{EntryWithPhonetic}{腰}{yao1}{13}{⾁}[HSK 4]
  \definition*{s.}{Sobrenome: Yao}
  \definition[个,尺]{s.}{cintura; região lombar | cós | bolso | parte do meio das coisas | lombo}
\end{EntryWithPhonetic}

\begin{EntryWithPhonetic}{腰包}{yao1bao1}{13,5}{⾁,⼓}
  \definition{s.}{pochete | bolso}
\end{EntryWithPhonetic}

\begin{EntryWithPhonetic}{腰椎}{yao1zhui1}{13,12}{⾁,⽊}
  \definition{s.}{vértebra lombar (espinha dorsal inferior)}
\end{EntryWithPhonetic}

%%%%%%%%%% 邀 %%%%%%%%%%
\subsection*{邀}\addcontentsline{loh}{figure}{邀 \dpy{yao1}}

\begin{EntryWithPhonetic}{邀}{yao1}{16}{⾡}
  \definition{v.}{convidar; requerer | (literário)  buscar aprovação; pedir permissão | interceptar}
\end{EntryWithPhonetic}

\begin{EntryWithPhonetic}{邀请}{yao1qing3}{16,10}{⾡,⾔}[HSK 5]
  \definition[份,个]{s.}{convite}
  \definition{v.}{convidar; solicitar; convidar pessoas para irem à sua casa ou a um local combinado}
\end{EntryWithPhonetic}

%%%%%%%%%% 尧 %%%%%%%%%%
\subsection*{尧}\addcontentsline{loh}{figure}{尧 \dpy{yao2}}

\begin{EntryWithPhonetic}{尧}{yao2}{6}{⼪}
  \definition*{s.}{Yao, um monarca lendário da China antiga | Sobrenome: Yao}
\end{EntryWithPhonetic}

%%%%%%%%%% 摇 %%%%%%%%%%
\subsection*{摇}\addcontentsline{loh}{figure}{摇 \dpy{yao2}}

\begin{EntryWithPhonetic}{摇}{yao2}{13}{⼿}[HSK 4]
  \definition{v.}{chacoalhar; ondular; balançar; fazer com que um objeto se mova para frente e para trás | agitar algo | sacudir; chacoalhar; agitar algo para que se mova}
\end{EntryWithPhonetic}

\begin{EntryWithPhonetic}{摇晃}{yao2huang4}{13,10}{⼿,⽇}
  \definition{v.}{sacudir | agitar | balançar | chacoalhar}
\end{EntryWithPhonetic}

\begin{EntryWithPhonetic}{摇头}{yao2/tou2}{13,5}{⼿,⼤}[HSK 5]
  \definition{v.+compl.}{sacudir; balançar a cabeça; balançar a cabeça para a esquerda e para a direita, indicando negação, desacordo ou impedimento}
\end{EntryWithPhonetic}

%%%%%%%%%% 遥 %%%%%%%%%%
\subsection*{遥}\addcontentsline{loh}{figure}{遥 \dpy{yao2}}

\begin{EntryWithPhonetic}{遥}{yao2}{13}{⾡}
  \definition{adj.}{distante; remoto; longe}
\end{EntryWithPhonetic}

\begin{EntryWithPhonetic}{遥控}{yao2kong4}{13,11}{⾡,⼿}
  \definition{s.}{controle remoto}
  \definition{v.}{dirigir operações de um local remoto | controlar remotamente}
\end{EntryWithPhonetic}

%%%%%%%%%% 咬 %%%%%%%%%%
\subsection*{咬}\addcontentsline{loh}{figure}{咬 \dpy{yao3}}

\begin{EntryWithPhonetic}{咬}{yao3}{9}{⼝}[HSK 5]
  \definition{v.}{morder; estalar; pressionar os dentes superiores e inferiores com força | latir | agarrar; morder | incriminar outra pessoa (geralmente inocente) quando culpada ou interrogada | pronunciar; articular; pronunciar corretamente | corroer (metais); irritar (a pele) | ser minucioso (com relação ao uso de palavras) | aproximar-se de; pressionar em direção a; avançar sobre}
\end{EntryWithPhonetic}

%%%%%%%%%% 药 %%%%%%%%%%
\subsection*{药}\addcontentsline{loh}{figure}{药 \dpy{yao4}}

\begin{EntryWithPhonetic}{药}{yao4}{9}{⾋}[HSK 2]
  \definition*{s.}{Sobrenome: Yao}
  \definition[片,粒,颗,瓶,服]{s.}{droga; loção; remédio; medicamento; substâncias que podem prevenir e tratar doenças, pragas ou melhorar funções corporais | certos produtos químicos com efeitos específicos}
  \definition{v.}{curar com remédios; tomar remédios para tratar doenças | matar com veneno; envenenar}
\end{EntryWithPhonetic}

\begin{EntryWithPhonetic}{药补}{yao4bu3}{9,7}{⾋,⾐}
  \definition{s.}{suplemento dietético medicinal que ajuda a melhorar a saúde}
\end{EntryWithPhonetic}

\begin{EntryWithPhonetic}{药典}{yao4dian3}{9,8}{⾋,⼋}
  \definition{s.}{farmacopéia}
\end{EntryWithPhonetic}

\begin{EntryWithPhonetic}{药店}{yao4 dian4}{9,8}{⾋,⼴}[HSK 2]
  \definition[家]{s.}{farmácia; drogaria; lojas que vendem medicamentos}
\end{EntryWithPhonetic}

\begin{EntryWithPhonetic}{药房}{yao4fang2}{9,8}{⾋,⼾}
  \definition{s.}{farmácia | drogaria}
\end{EntryWithPhonetic}

\begin{EntryWithPhonetic}{药罐}{yao4guan4}{9,23}{⾋,⽸}
  \definition{s.}{frasco de remédio; pote de remédio}
\end{EntryWithPhonetic}

\begin{EntryWithPhonetic}{药片}{yao4 pian4}{9,4}{⾋,⽚}[HSK 2]
  \definition[颗,片]{s.}{pílula; comprimido; preparações em comprimidos}
\end{EntryWithPhonetic}

\begin{EntryWithPhonetic}{药品}{yao4pin3}{9,9}{⾋,⼝}[HSK 6]
  \definition[个,些,种,类,批]{s.}{medicamentos e reagentes químicos; um termo geral para vários medicamentos e reagentes químicos}
\end{EntryWithPhonetic}

\begin{EntryWithPhonetic}{药签}{yao4qian1}{9,13}{⾋,⽵}
  \definition{s.}{cotonete médico}
\end{EntryWithPhonetic}

\begin{EntryWithPhonetic}{药膳}{yao4shan4}{9,16}{⾋,⾁}
  \definition{s.}{alimentos medicamentosos; alimentos cozidos com ervas medicinais | cozinha medicinal}
\end{EntryWithPhonetic}

\begin{EntryWithPhonetic}{药水}{yao4 shui3}{9,4}{⾋,⽔}[HSK 2]
  \definition*{s.}{Yaksu na Coreia do Norte, perto da fronteira com Liaoning e a província de Jilin}
  \definition{s.}{medicamento líquido; líquido medicinal | loção | remédio engarrafado | medicamento em forma líquida}
\end{EntryWithPhonetic}

\begin{EntryWithPhonetic}{药丸}{yao4wan2}{9,3}{⾋,⼂}
  \definition[粒]{s.}{pílula}
\end{EntryWithPhonetic}

\begin{EntryWithPhonetic}{药物}{yao4 wu4}{9,8}{⾋,⽜}[HSK 4]
  \definition[种]{s.}{droga; medicamento; remédio; substâncias que controlam doenças, pragas, etc.}
\end{EntryWithPhonetic}

%%%%%%%%%% 要 %%%%%%%%%%
\subsection*{要}\addcontentsline{loh}{figure}{要 \dpy{yao4}}

\begin{EntryWithPhonetic}{要}{yao4}{9}{⾑}[HSK 4]
  \definition{adj.}{importante; essencial}
  \definition{conj.}{suponha; no caso; se, indicando um relacionamento hipotético | ou; ou\dots ou\dots}
  \definition{s.}{ponto principal; manchete; conteúdo importante}
  \definition{v.}{querer; desejar; pensar | querer; pedir; deseja; querer obter; querer manter | recuperar algo; dizer a alguém para guardar algo para você ou devolver | pedir (ou querer) que alguém faça algo; pedir a alguém para fazer algo, quando usado para conseguir que alguém faça algo, tem um tom de comando e pode ser indelicado | precisar; tomar; pegar | deve; deveria; é necessário (imperativo, essencial) que\dots | estar indo para | querer; ter um desejo por; expressar determinação ou desejo de fazer algo | poder; dever;  indica uma estimativa, usada para comparação}
  \seeref{yao1}
  \seealsoref{要是}{yao4shi5}
\end{EntryWithPhonetic}

\begin{EntryWithPhonetic}{要不}{yao4 bu4}{9,4}{⾑,⼀}
  \definition{conj.}{ou então; caso contrário; se você não fizer isso (haverá um resultado ruim) | usado para propor educadamente; usado para fazer uma sugestão educadamente | ou; se você não fizer isso, faça aquilo}
\end{EntryWithPhonetic}

\begin{EntryWithPhonetic}{要不然}{yao4 bu4 ran2}{9,4,12}{⾑,⼀,⽕}[HSK 6]
  \definition{conj.}{caso contrário; ou então; se você não fizer isso (haverá um resultado ruim) | ou então; usado entre duas frases em um relacionamento de escolha; significa escolher uma entre as duas; equivalente a 要不}
  \seealsoref{要不}{yao4 bu4}
\end{EntryWithPhonetic}

\begin{EntryWithPhonetic}{要点}{yao4dian3}{9,9}{⾑,⽕}
  \definition{s.}{pontos principais | essencial}
\end{EntryWithPhonetic}

\begin{EntryWithPhonetic}{要好}{yao4 hao3}{9,6}{⾑,⼥}[HSK 6]
  \definition{adj.}{estar em bons termos; ser amigos próximos; relacionamento harmonioso | estar ansioso para melhorar a si mesmo; esforçar-se para progredir | ansioso para melhorar a si mesmo; esforçar-se para progredir}
\end{EntryWithPhonetic}

\begin{EntryWithPhonetic}{要谎}{yao4huang3}{9,11}{⾑,⾔}
  \definition{v.}{pedir um preço enorme (como primeiro passo de negociação)}
\end{EntryWithPhonetic}

\begin{EntryWithPhonetic}{要么}{yao4 me5}{9,3}{⾑,⼃}[HSK 6]
  \definition{conj.}{ou; ou\dots ou\dots; indica uma escolha entre duas situações ou dois desejos}
  \seealsoref{要么…要么…}{yao4 me5 yao4 me5}
\end{EntryWithPhonetic}

\begin{EntryWithPhonetic}{要么…要么…}{yao4 me5 yao4 me5}{9,3,9,3}{⾑,⼃,⾑,⼃}[HSK 6]
  \definition{conj.}{ou\dots ou\dots}
  \seealsoref{要么}{yao4 me5}
\end{EntryWithPhonetic}

\begin{EntryWithPhonetic}{要强}{yao4qiang2}{9,12}{⾑,⼸}
  \definition{adj.}{ansioso para se destacar | ansioso para progredir na vida | obstinado}
\end{EntryWithPhonetic}

\begin{EntryWithPhonetic}{要是}{yao4shi5}{9,9}{⾑,⽇}[HSK 3]
  \definition{conj.}{se; suponha; no caso de; conecta frases, expressa uma relação hipotética, equivalente a 如果, e pode ser usado em conjunto com 的话}
  \seealsoref{的话}{de5 hua4}
  \seealsoref{如果}{ru2guo3}
\end{EntryWithPhonetic}

\begin{EntryWithPhonetic}{要是…的话}{yao4shi5 de5hua4}{9,9,8,8}{⾑,⽇,⽩,⾔}
  \definition{conj.}{se for assim\dots}
\end{EntryWithPhonetic}

\begin{EntryWithPhonetic}{要死}{yao4si3}{9,6}{⾑,⽍}
  \definition{adv.}{extremamente | muito}
\end{EntryWithPhonetic}

\begin{EntryWithPhonetic}{要素}{yao4su4}{9,10}{⾑,⽷}[HSK 6]
  \definition[个]{s.}{fator essencial; elemento-chave; os elementos essenciais que compõem as coisas}
\end{EntryWithPhonetic}

\begin{EntryWithPhonetic}{要义}{yao4yi4}{9,3}{⾑,⼂}
  \definition{s.}{resumo | o essencial}
\end{EntryWithPhonetic}

%%%%%%%%%% 钥 %%%%%%%%%%
\subsection*{钥}\addcontentsline{loh}{figure}{钥 \dpy{yao4}}

\begin{EntryWithPhonetic}{钥}{yao4}{9}{⾦}
  \definition{s.}{chave}
\end{EntryWithPhonetic}

\begin{EntryWithPhonetic}{钥匙}{yao4shi5}{9,11}{⾦,⼔}
  \definition[把]{s.}{chave}
\end{EntryWithPhonetic}

\begin{EntryWithPhonetic}{钥匙洞孔}{yao4shi5dong4kong3}{9,11,9,4}{⾦,⼔,⽔,⼦}
  \definition{s.}{buraco da fechadura}
\end{EntryWithPhonetic}

\begin{EntryWithPhonetic}{钥匙卡}{yao4shi5ka3}{9,11,5}{⾦,⼔,⼘}
  \definition{s.}{cartão de acesso | cartão-chave; cartão magnético}
\end{EntryWithPhonetic}

\begin{EntryWithPhonetic}{钥匙孔}{yao4shi5kong3}{9,11,4}{⾦,⼔,⼦}
  \definition{s.}{buraco da fechadura}
\end{EntryWithPhonetic}

\begin{EntryWithPhonetic}{钥匙圈}{yao4shi5quan1}{9,11,11}{⾦,⼔,⼞}
  \definition{s.}{chaveiro}
\end{EntryWithPhonetic}

%%%%%%%%%% 椰 %%%%%%%%%%
\subsection*{椰}\addcontentsline{loh}{figure}{椰 \dpy{ye1}}

\begin{EntryWithPhonetic}{椰}{ye1}{12}{⽊}
  \definition[只,棵]{s.}{coqueiro; coco}
\end{EntryWithPhonetic}

\begin{EntryWithPhonetic}{椰汁}{ye1zhi1}{12,5}{⽊,⽔}
  \definition{s.}{água de coco}
\end{EntryWithPhonetic}

%%%%%%%%%% 噎 %%%%%%%%%%
\subsection*{噎}\addcontentsline{loh}{figure}{噎 \dpy{ye1}}

\begin{EntryWithPhonetic}{噎}{ye1}{15}{⼝}
  \definition{v.}{engasgar | sufocar}
\end{EntryWithPhonetic}

%%%%%%%%%% 爷 %%%%%%%%%%
\subsection*{爷}\addcontentsline{loh}{figure}{爷 \dpy{ye2}}

\begin{EntryWithPhonetic}{爷}{ye2}{6}{⽗}
  \definition[个,位,名,些]{s.}{(dialeto) pai | (dialeto) avô | (uma forma respeitosa de se dirigir a um homem idoso) tio | (uma forma de se dirigir a um oficial ou homem rico) senhor; mestre; lorde; o antigo nome para burocratas, pessoas ricas, etc. | deus; forma de tratamento de um adorador para um deus}
\end{EntryWithPhonetic}

\begin{EntryWithPhonetic}{爷爷}{ye2ye5}{6,6}{⽗,⽗}[HSK 1]
  \definition[个,位]{s.}{avô (paterno)}
\end{EntryWithPhonetic}

%%%%%%%%%% 也 %%%%%%%%%%
\subsection*{也}\addcontentsline{loh}{figure}{也 \dpy{ye3}}

\begin{EntryWithPhonetic}{也}{ye3}{3}{⼄}[HSK 1]
  \definition*{s.}{Sobrenome: Ye}
  \definition{adv.}{também; igualmente; assim como; da mesma forma; usado em frases simples, implica que é igual a outra coisa | assim como (expressar ênfase) | (expressar que as consequências são as mesmas) | também (expressar ufemismo; expressar um tom diplomático) | usado em frases compostas paralelas, indica que duas ou mais coisas têm algo em comum (pode ser usado em todas as frases ou apenas na última frase)}
  \definition{part.}{usado no meio de uma frase, destacando um elemento da frase sobre o qual deve ser feita uma afirmação | usado no final de uma frase, indicando uma explicação ou um julgamento; usado no final da frase, indica tom afirmativo e também pode reforçar o tom interrogativo, exclamativo ou imperativo}
\end{EntryWithPhonetic}

\begin{EntryWithPhonetic}{也好}{ye3 hao3}{3,6}{⼄,⼥}[HSK 5]
  \definition{part.}{pode não ser uma má ideia; também pode ser | (reduplicado) se\dots ou\dots; não importa se | pode não ser uma má ideia | se\dots ou\dots; usado em conjunto, significa que não está condicionado a uma determinada situação}
\end{EntryWithPhonetic}

\begin{EntryWithPhonetic}{也就是}{ye3jiu4shi4}{3,12,9}{⼄,⼪,⽇}
  \definition{adv.}{i.e., isso é | ou seja}
\end{EntryWithPhonetic}

\begin{EntryWithPhonetic}{也就是说}{ye3jiu4shi4shuo1}{3,12,9,9}{⼄,⼪,⽇,⾔}
  \definition{adv.}{em outras palavras | então | isto é | por isso}
\end{EntryWithPhonetic}

\begin{EntryWithPhonetic}{也许}{ye3xu3}{3,6}{⼄,⾔}[HSK 2]
  \definition{adv.}{talvez; provavelmente; estou com medo; para expressar incerteza; para expressar uma alta probabilidade}
\end{EntryWithPhonetic}

\begin{EntryWithPhonetic}{也有今天}{ye3you3jin1tian1}{3,6,4,4}{⼄,⽉,⼈,⼤}
  \definition{expr.}{obter apenas o que merece | todo cachorro tem seu dia | obter a sua parte (coisas boas ou ruins) | servir alguém bem}
\end{EntryWithPhonetic}

%%%%%%%%%% 野 %%%%%%%%%%
\subsection*{野}\addcontentsline{loh}{figure}{野 \dpy{ye3}}

\begin{EntryWithPhonetic}{野}{ye3}{11}{⾥}[HSK 6]
  \definition*{s.}{Sobrenome: Ye}
  \definition{adj.}{(de plantas ou animais) selvagem; incultivado; não domesticado; indomável (opp. 家) | rude; áspero | desenfreado; abandonado; indisciplinado | ilícito; sem licença}
  \definition{s.}{espaço aberto; o aberto | limite; fronteira | não está no poder; fora do cargo}
  \seealsoref{家}{jia1}
\end{EntryWithPhonetic}

\begin{EntryWithPhonetic}{野生}{ye3 sheng1}{11,5}{⾥,⽣}[HSK 6]
  \definition{adj.}{selvagem; não cultivado; não domesticado}
\end{EntryWithPhonetic}

%%%%%%%%%% 业 %%%%%%%%%%
\subsection*{业}\addcontentsline{loh}{figure}{业 \dpy{ye4}}

\begin{EntryWithPhonetic}{业}{ye4}{5}{⼀}
  \definition*{s.}{Sobrenome: Ye}
  \definition{adv.}{já; indica que a ação foi concluída, equivalente a 已经}
  \definition{s.}{comércio; indústria; ramo de negócios | emprego; ocupação; profissão | curso de estudo | causa; empreendimento | propriedade | carma; o budismo se refere a todas as ações, palavras e pensamentos humanos como carma, que são chamados de carma corporal, carma da fala e carma mental; o carma inclui aspectos bons e ruins, geralmente referindo-se ao destino ou ao pecado}
  \definition{v.}{envolver-se em; exercer uma determinada profissão}
  \seealsoref{已经}{yi3jing1}
\end{EntryWithPhonetic}

\begin{EntryWithPhonetic}{业务}{ye4wu4}{5,5}{⼀,⼒}[HSK 5]
  \definition[项,笔,个,类,种]{s.}{negócios; trabalho vocacional; trabalho profissional}
\end{EntryWithPhonetic}

\begin{EntryWithPhonetic}{业余}{ye4yu2}{5,7}{⼀,⼈}[HSK 4]
  \definition{adj.}{tempo livre; depois do expediente; fora do horário de trabalho | amador; não profissional}
\end{EntryWithPhonetic}

%%%%%%%%%% 叶 %%%%%%%%%%
\subsection*{叶}\addcontentsline{loh}{figure}{叶 \dpy{ye4}}

\begin{EntryWithPhonetic}{叶}{ye4}{5}{⼝}
  \definition*{s.}{Sobrenome: Ye}
  \definition[枝]{s.}{folha; folhagem | coisa parecida com uma folha | página; folha | parte de um período histórico; segmentos de período mais longos | lóbulo; lóbulos do cérebro, pulmões e fígado}
\end{EntryWithPhonetic}

\begin{EntryWithPhonetic}{叶子}{ye4zi5}{5,3}{⼝,⼦}[HSK 4]
  \definition[片]{s.}{folha; termo genérico para as folhas de uma planta}
\end{EntryWithPhonetic}

%%%%%%%%%% 页 %%%%%%%%%%
\subsection*{页}\addcontentsline{loh}{figure}{页 \dpy{ye4}}

\begin{EntryWithPhonetic}{页}{ye4}{6}{⾴}[HSK 1][Kangxi 181]
  \definition{clas.}{página; folha de papel; lâmina; antigamente, referia-se a uma folha de um livro encadernado; atualmente, refere-se a uma das faces de um livro impresso em ambos os lados}
  \definition{s.}{página; folha de papel; folhas soltas de um livro}
\end{EntryWithPhonetic}

%%%%%%%%%% 夜 %%%%%%%%%%
\subsection*{夜}\addcontentsline{loh}{figure}{夜 \dpy{ye4}}

\begin{EntryWithPhonetic}{夜}{ye4}{8}{⼣}[HSK 2]
  \definition{s.}{noite; tarde; noturno; o período do anoitecer ao amanhecer (em oposição a 日 ou 昼); em meteorologia, refere-se especificamente ao período das 20h do dia atual às 8h do dia seguinte}
  \seealsoref{日}{ri4}
  \seealsoref{昼}{zhou4}
\end{EntryWithPhonetic}

\begin{EntryWithPhonetic}{夜场}{ye4chang3}{8,6}{⼣,⼟}
  \definition{s.}{show noturno (em um teatro, etc.) | local de entretenimento noturno (bar, boate, discoteca, etc.)}
\end{EntryWithPhonetic}

\begin{EntryWithPhonetic}{夜店}{ye4dian4}{8,8}{⼣,⼴}
  \definition{s.}{boate | \emph{nightclub}}
\end{EntryWithPhonetic}

\begin{EntryWithPhonetic}{夜间}{ye4 jian1}{8,7}{⼣,⾨}[HSK 5]
  \definition{s.}{noite; à noite; noturno; durante a noite}
\end{EntryWithPhonetic}

\begin{EntryWithPhonetic}{夜里}{ye4li5}{8,7}{⼣,⾥}[HSK 2]
  \definition{s.}{noturno; à noite; o período do anoitecer ao amanhecer}
\end{EntryWithPhonetic}

\begin{EntryWithPhonetic}{夜幕}{ye4mu4}{8,13}{⼣,⼱}
  \definition{s.}{escuridão crescente; cortina da noite; o véu da noite}
\end{EntryWithPhonetic}

\begin{EntryWithPhonetic}{夜鸟}{ye4niao3}{8,5}{⼣,⿃}
  \definition{s.}{ave noturna}
\end{EntryWithPhonetic}

\begin{EntryWithPhonetic}{夜深人静}{ye4shen1ren2jing4}{8,11,2,14}{⼣,⽔,⼈,⾭}
  \definition{expr.}{``Na calada da noite.''; ``No silêncio da noite.''}
\end{EntryWithPhonetic}

\begin{EntryWithPhonetic}{夜生活}{ye4sheng1huo2}{8,5,9}{⼣,⽣,⽔}
  \definition{s.}{vida noturna}
\end{EntryWithPhonetic}

\begin{EntryWithPhonetic}{夜晚}{ye4wan3}{8,11}{⼣,⽇}
  \definition[个]{s.}{noite}
\end{EntryWithPhonetic}

\begin{EntryWithPhonetic}{夜夜}{ye4ye4}{8,8}{⼣,⼣}
  \definition{adv.}{toda noite}
\end{EntryWithPhonetic}

%%%%%%%%%% 液 %%%%%%%%%%
\subsection*{液}\addcontentsline{loh}{figure}{液 \dpy{ye4}}

\begin{EntryWithPhonetic}{液}{ye4}{11}{⽔}
  \definition{s.}{líquido; fluido; suco}
\end{EntryWithPhonetic}

\begin{EntryWithPhonetic}{液体}{ye4ti3}{11,7}{⽔,⼈}
  \definition{adj./s.}{líquido}
\end{EntryWithPhonetic}

%%%%%%%%%% 一 %%%%%%%%%%
\subsection*{一}\addcontentsline{loh}{figure}{一 \dpy{yi1}}

\begin{EntryWithPhonetic}{一}{yi1}[(quando usado sozinho)]{1}{⼀}[HSK 1][Kangxi 1]
  \definition{adv.}{uma vez; assim que; indica que duas ações ocorreram em um intervalo de tempo muito curto, uma terminando e a outra começando imediatamente em seguida | indica que primeiro se realiza uma ação e, em seguida, o resultado dessa ação  | indica uma ação única, indicando que a ação é muito curta ou apenas uma tentativa}
  \definition{num.}{um; 1 | pronunciado como \dpy{yao1} quando dito número a número | igual; refere-se ao mesmo ou igual | inteiro; todo; por toda parte | exclusivo ou único | refere-se a algo específico | também; caso contrário; referindo-se a outro ou mais um}
  \definition{part.}{antes de certas palavras para dar ênfase}
  \definition{prep.}{cada; por; toda vez}
  \definition{s.}{uma nota da escala em Gongchepu (工尺谱), correspondente ao 17 na notação musical numerada}
  \seeref{yi2}
  \seeref{yi4}
  \seealsoref{工尺谱}{gong1 che3 pu3}
\end{EntryWithPhonetic}

\begin{EntryWithPhonetic}{一会儿…一会儿…}{yi1hui4r5 yi1hui4r5}{1,6,2,1,6,2}{⼀,⼈,⼉,⼀,⼈,⼉}
  \definition{adv.}{um tempo\dots um tempo\dots}
\end{EntryWithPhonetic}

\begin{EntryWithPhonetic}{一…就…}{yi1 jiu4}{1,12}{⼀,⼪}
  \definition{expr.}{logo que |  uma vez que}
\end{EntryWithPhonetic}

\begin{EntryWithPhonetic}{一行}{yi1 xing2}{1,6}{⼀,⾏}[HSK 6]
  \definition{s.}{delegação; um grupo viajando junto; festa}
\end{EntryWithPhonetic}

%%%%%%%%%% 伊 %%%%%%%%%%
\subsection*{伊}\addcontentsline{loh}{figure}{伊 \dpy{yi1}}

\begin{EntryWithPhonetic}{伊}{yi1}{6}{⼈}
  \definition*{s.}{Iraque, abreviação de 伊拉克 | Irã,abreviação de  伊朗 | Sobrenome: Yi}
  \definition{part.}{(chinês clássico) partícula introdutória sem significado específico}
  \definition{pron.}{Literário: pronome de terceira pessoa do singular (ele ou ela) | pronome de segunda pessoa do singular (você) | disso (precedendo um substantivo)}
  \seealsoref{伊拉克}{yi1la1ke4}
  \seealsoref{伊朗}{yi1lang3}
\end{EntryWithPhonetic}

\begin{EntryWithPhonetic}{伊拉克}{yi1la1ke4}{6,8,7}{⼈,⼿,⼗}
  \definition*{s.}{Iraque}
\end{EntryWithPhonetic}

\begin{EntryWithPhonetic}{伊朗}{yi1lang3}{6,10}{⼈,⽉}
  \definition*{s.}{Irã}
\end{EntryWithPhonetic}

\begin{EntryWithPhonetic}{伊马姆}{yi1ma3mu3}{6,3,8}{⼈,⾺,⼥}
  \definition*{s.}{Islã}
  \seealsoref{伊玛目}{yi1ma3mu4}
  \seealsoref{伊曼}{yi1man4}
  \seealsoref{伊斯兰}{yi1si1lan2}
\end{EntryWithPhonetic}

\begin{EntryWithPhonetic}{伊玛目}{yi1ma3mu4}{6,7,5}{⼈,⽟,⽬}
  \definition*{s.}{Islã}
  \seealsoref{伊马姆}{yi1ma3mu3}
  \seealsoref{伊曼}{yi1man4}
  \seealsoref{伊斯兰}{yi1si1lan2}
\end{EntryWithPhonetic}

\begin{EntryWithPhonetic}{伊曼}{yi1man4}{6,11}{⼈,⽈}
  \definition*{s.}{Islã}
  \seealsoref{伊马姆}{yi1ma3mu3}
  \seealsoref{伊玛目}{yi1ma3mu4}
  \seealsoref{伊斯兰}{yi1si1lan2}
\end{EntryWithPhonetic}

\begin{EntryWithPhonetic}{伊斯兰}{yi1si1lan2}{6,12,5}{⼈,⽄,⼋}
  \definition*{s.}{Islã}
  \seealsoref{伊马姆}{yi1ma3mu3}
  \seealsoref{伊玛目}{yi1ma3mu4}
  \seealsoref{伊曼}{yi1man4}
\end{EntryWithPhonetic}

%%%%%%%%%% 衣 %%%%%%%%%%
\subsection*{衣}\addcontentsline{loh}{figure}{衣 \dpy{yi1}}

\begin{EntryWithPhonetic}{衣}{yi1}{6}{⾐}
  \definition[件]{s.}{roupa}
  \seeref{yi4}
\end{EntryWithPhonetic}

\begin{EntryWithPhonetic}{衣服}{yi1fu5}{6,8}{⾐,⽉}[HSK 1]
  \definition[套,件]{s.}{roupas; vestuário; algo que se veste para cobrir o corpo e se proteger do frio}
\end{EntryWithPhonetic}

\begin{EntryWithPhonetic}{衣柜}{yi1gui4}{6,8}{⾐,⽊}
  \definition[个]{s.}{armário | guarda-roupa}
\end{EntryWithPhonetic}

\begin{EntryWithPhonetic}{衣甲}{yi1jia3}{6,5}{⾐,⽥}
  \definition{s.}{armadura}
\end{EntryWithPhonetic}

\begin{EntryWithPhonetic}{衣架}{yi1 jia4}{6,9}{⾐,⽊}[HSK 3]
  \definition[个,副,组]{s.}{cabideiro; móvel para pendurar roupas | estatura; figura; refere-se ao tipo físico de uma pessoa; estrutura corporal}
\end{EntryWithPhonetic}

%%%%%%%%%% 医 %%%%%%%%%%
\subsection*{医}\addcontentsline{loh}{figure}{医 \dpy{yi1}}

\begin{EntryWithPhonetic}{医}{yi1}{7}{⼖}
  \definition*{s.}{Sobrenome: Yi}
  \definition{s.}{médico | medicina; ciência médica}
  \definition{v.}{curar; tratar}
\end{EntryWithPhonetic}

\begin{EntryWithPhonetic}{医疗}{yi1 liao2}{7,7}{⼖,⽧}[HSK 4]
  \definition{s.}{tratamento médico; tratamento de doenças}
\end{EntryWithPhonetic}

\begin{EntryWithPhonetic}{医生}{yi1sheng1}{7,5}{⼖,⽣}[HSK 1]
  \definition[位,个,名]{s.}{médico; clínico; pessoa que possui conhecimentos médicos e cuja profissão é tratar doenças}
\end{EntryWithPhonetic}

\begin{EntryWithPhonetic}{医学}{yi1 xue2}{7,8}{⼖,⼦}[HSK 4]
  \definition{s.}{medicina; iatrologia; ciência médica; ciência da prevenção e do tratamento de doenças e da proteção e promoção da saúde humana}
\end{EntryWithPhonetic}

\begin{EntryWithPhonetic}{医药}{yi1 yao4}{7,9}{⼖,⾋}[HSK 6]
  \definition{s.}{medicina | médico | cuidados médicos e medicamentos | medicamento (droga) | farmacêutica}
\end{EntryWithPhonetic}

\begin{EntryWithPhonetic}{医院}{yi1yuan4}{7,9}{⼖,⾩}[HSK 1]
  \definition[家,所,个]{s.}{hospital; instituições que tratam e cuidam de pacientes, e também realizam exames de saúde, prevenção de doenças, etc.}
\end{EntryWithPhonetic}

%%%%%%%%%% 依 %%%%%%%%%%
\subsection*{依}\addcontentsline{loh}{figure}{依 \dpy{yi1}}

\begin{EntryWithPhonetic}{依}{yi1}{8}{⼈}
  \definition*{s.}{Sobrenome: Yi}
  \definition{prep.}{de acordo com; à luz de; julgando por}
  \definition{v.}{depender de; ser dependente de; confiar em | cumprir; ouvir; ceder a | inclinar-se; descansar sobre (ou contra)}
\end{EntryWithPhonetic}

\begin{EntryWithPhonetic}{依次}{yi1 ci4}{8,6}{⼈,⽋}[HSK 6]
  \definition{adv.}{sucessivamente; na ordem correta; em ordem}
\end{EntryWithPhonetic}

\begin{EntryWithPhonetic}{依法}{yi1 fa3}{8,8}{⼈,⽔}[HSK 5]
  \definition{adv.}{e acordo com regras (ou métodos) fixas | de acordo com a lei; por força da lei; em conformidade com as disposições legais}
\end{EntryWithPhonetic}

\begin{EntryWithPhonetic}{依旧}{yi1jiu4}{8,5}{⼈,⽇}[HSK 5]
  \definition{adv.}{ainda; como antes; como sempre}
\end{EntryWithPhonetic}

\begin{EntryWithPhonetic}{依据}{yi1ju4}{8,11}{⼈,⼿}[HSK 5]
  \definition{prep.}{julgando por; de acordo com; à luz de; com base em; de acordo com; introduzir algo que possa servir como premissa ou base}
  \definition[个]{s.}{base; evidência; fundamento; base para tomar uma decisão ou realizar uma ação}
  \definition{v.}{basear-se em; confiar em; depdender de; usar algo como premissa ou base}
\end{EntryWithPhonetic}

\begin{EntryWithPhonetic}{依靠}{yi1kao4}{8,15}{⼈,⾮}[HSK 4]
  \definition{s.}{apoio; suporte; algo em que se apoiar; alguém ou algo em quem você pode confiar}
  \definition{v.}{depender de; confiar em (alguém ou alguma coisa para atingir um determinado objetivo)}
\end{EntryWithPhonetic}

\begin{EntryWithPhonetic}{依赖}{yi1lai4}{8,13}{⼈,⾙}[HSK 6]
  \definition{v.}{confiar em; ser dependente de; ser completamente dependente e inseparável | depender de; ser mutuamente dependentes e inseparáveis}
\end{EntryWithPhonetic}

\begin{EntryWithPhonetic}{依然}{yi1ran2}{8,12}{⼈,⽕}[HSK 4]
  \definition{adv.}{ainda; como antes}
  \definition{v.}{estar quieto; estar como antes; estar como o original, sem alterações}
\end{EntryWithPhonetic}

\begin{EntryWithPhonetic}{依偎}{yi1wei1}{8,11}{⼈,⼈}
  \definition{v.}{aninhar-se | aconchegar-se}
\end{EntryWithPhonetic}

\begin{EntryWithPhonetic}{依照}{yi1 zhao4}{8,13}{⼈,⽕}[HSK 5]
  \definition{prep.}{de acordo com; à luz de; introduzir certos padrões para os eventos, o que equivale a 按照}
  \definition{v.}{seguir (com base em algo)}
  \seealsoref{按照}{an4zhao4}
\end{EntryWithPhonetic}

%%%%%%%%%% 毉 %%%%%%%%%%
\subsection*{毉}\addcontentsline{loh}{figure}{毉 \dpy{yi1}}

\begin{EntryWithPhonetic}{毉}{yi1}{18}{⼖}
  \variantof{医}
\end{EntryWithPhonetic}

%%%%%%%%%% 一 %%%%%%%%%%
\subsection*{一}\addcontentsline{loh}{figure}{一 \dpy{yi2}}

\begin{EntryWithPhonetic}{一}{yi2}[(antes de quarto tom)]{1}{⼀}[HSK 1][Kangxi 1]
  \definition{num.}{um; 1 | um (artigo)}
  \seeref{yi1}
  \seeref{yi4}
\end{EntryWithPhonetic}

\begin{EntryWithPhonetic}{一半}{yi2ban4}{1,5}{⼀,⼗}[HSK 1]
  \definition{num.}{metade; em parte; uma metade}
\end{EntryWithPhonetic}

\begin{EntryWithPhonetic}{一辈子}{yi2bei4zi5}{1,12,3}{⼀,⾞,⼦}[HSK 5]
  \definition{s.}{uma vida inteira; vida inteira; toda a vida; durante toda a vida; enquanto se vive; todo o tempo entre o nascimento e a morte}
\end{EntryWithPhonetic}

\begin{EntryWithPhonetic}{一部分}{yi2 bu4 fen4}{1,10,4}{⼀,⾢,⼑}[HSK 2]
  \definition{adj.}{parcial}
  \definition{adv.}{parcialmente}
  \definition{num.}{parte; porção; seção; fração}
\end{EntryWithPhonetic}

\begin{EntryWithPhonetic}{一次性}{yi2 ci4 xing4}{1,6,8}{⼀,⽋,⼼}[HSK 6]
  \definition{adj.}{único; uso único; descartável (produtos); apenas uma vez, sem necessidade ou necessidade de fazer novamente}
\end{EntryWithPhonetic}

\begin{EntryWithPhonetic}{一代}{yi2 dai4}{1,5}{⼀,⼈}[HSK 6]
  \definition{s.}{uma dinastia | era; época atual | vida; geração; toda a vida de uma pessoa}
\end{EntryWithPhonetic}

\begin{EntryWithPhonetic}{一带}{yi2 dai4}{1,9}{⼀,⼱}[HSK 5]
  \definition{s.}{a área em torno de um determinado local; refere-se a um determinado local e suas proximidades}
\end{EntryWithPhonetic}

\begin{EntryWithPhonetic}{一旦}{yi2dan4}{1,5}{⼀,⽇}[HSK 5]
  \definition{adv.}{uma vez; no caso; agora que | de repente; uma vez}
  \definition{s.}{em um único dia; em um tempo muito curto;}
\end{EntryWithPhonetic}

\begin{EntryWithPhonetic}{一道}{yi2 dao4}{1,12}{⼀,⾡}[HSK 6]
  \definition{adv.}{juntos; lado a lado; junto com}
\end{EntryWithPhonetic}

\begin{EntryWithPhonetic}{一定}{yi2ding4}{1,8}{⼀,⼧}[HSK 2]
  \definition{adj.}{certo; particular; tendo um certo nível de especificidade; (objeto, situação) determinado em um ou mais | devido; certo; sempre foi assim, não vai mudar | fixo; especificado; há requisitos claros quanto à maneira, método, quantidade, etc.}
  \definition{adv.}{certamente; necessariamente; expressando determinação ou certeza | certamente; indica especulação ou avaliação de que um evento ou situação definitivamente acontecerá ou realmente existirá}
\end{EntryWithPhonetic}

\begin{EntryWithPhonetic}{一个样}{yi2ge5yang4}{1,3,10}{⼀,⼈,⽊}
  \definition{s.}{o mesmo}
  \seealsoref{一样}{yi2yang4}
\end{EntryWithPhonetic}

\begin{EntryWithPhonetic}{一共}{yi2gong4}{1,6}{⼀,⼋}[HSK 2]
  \definition{adv.}{completamente; em tudo; no todo}
\end{EntryWithPhonetic}

\begin{EntryWithPhonetic}{一贯}{yi2guan4}{1,8}{⼀,⾙}[HSK 6]
  \definition{adj./adv.}{do começo ao fim; inabalável; consistente; persistente; o tempo todo}
\end{EntryWithPhonetic}

\begin{EntryWithPhonetic}{一会儿}{yi2 hui4r5}{1,6,2}{⼀,⼈,⼉}[HSK 1,2]
  \definition{adv.}{agora\dots agora\dots; um momento\dots o próximo\dots; usado antes de dois antônimos, indica a alternância de situações}
  \definition{s.}{um pouquinho de tempo; muito pouco tempo}
\end{EntryWithPhonetic}

\begin{EntryWithPhonetic}{一句话}{yi2 ju4 hua4}{1,5,8}{⼀,⼝,⾔}[HSK 5]
  \definition{s.}{em resumo; em uma palavra; expressar um conteúdo complexo de forma sucinta | trabalho fácil; fácil de fazer; descrever uma tarefa ou trabalho como muito simples e fácil de realizar}
\end{EntryWithPhonetic}

\begin{EntryWithPhonetic}{一块}{yi2kuai4}{1,7}{⼀,⼟}
  \definition{adv.}{(principalmente mandarim) juntos}
\end{EntryWithPhonetic}

\begin{EntryWithPhonetic}{一块儿}{yi2 kuai4r5}{1,7,2}{⼀,⼟,⼉}[HSK 1]
  \definition{adv.}{juntos; em conjunto}
  \definition{s.}{no mesmo lugar; no mesmo local}
\end{EntryWithPhonetic}

\begin{EntryWithPhonetic}{一路}{yi2 lu4}{1,13}{⼀,⾜}[HSK 5]
  \definition{adv.}{o tempo todo; persistentemente; continuamente | juntos; sem parar; continuamente}
  \definition{s.}{o mesmo caminho; a mesma rota; ao longo de toda a viagem, ao longo do caminho | do mesmo tipo; da mesma categoria}
\end{EntryWithPhonetic}

\begin{EntryWithPhonetic}{一路平安}{yi2 lu4 ping2 an1}{1,13,5,6}{⼀,⾜,⼲,⼧}[HSK 2]
  \definition{expr.}{Boa viagem!; Tenha uma boa viagem!}
  \definition{v.}{ter uma viagem agradável}
\end{EntryWithPhonetic}

\begin{EntryWithPhonetic}{一路上}{yi2 lu4 shang4}{1,13,3}{⼀,⾜,⼀}[HSK 6]
  \definition{s.}{ao longo do caminho; todo o caminho}
\end{EntryWithPhonetic}

\begin{EntryWithPhonetic}{一路顺风}{yi2 lu4 shun4 feng1}{1,13,9,4}{⼀,⾜,⾴,⾵}[HSK 2]
  \definition{expr.}{ter uma viagem agradável; toda a viagem foi segura e tranquila; é uma metáfora para cada etapa do processo de lidar com algo que ocorre sem problemas | Tenha uma boa viagem!; Boa viagem!}
\end{EntryWithPhonetic}

\begin{EntryWithPhonetic}{一律}{yi2lv4}{1,9}{⼀,⼻}[HSK 4]
  \definition{adj.}{igual; semelhante; uniforme; parecido; idêntico}
  \definition{adv.}{todos; tudo; sem exceção; enfatiza que todos devem ser assim, sem exceção, e é usado principalmente em regulamentos ou requisitos}
\end{EntryWithPhonetic}

\begin{EntryWithPhonetic}{一切}{yi2qie4}{1,4}{⼀,⼑}[HSK 3]
  \definition{pron.}{tudo; todo; todas as coisas}
\end{EntryWithPhonetic}

\begin{EntryWithPhonetic}{一下}{yi2xia4}{1,3}{⼀,⼀}
  \definition{adv.}{em um curto tempo | rapidamente}
\end{EntryWithPhonetic}

\begin{EntryWithPhonetic}{一下儿}{yi2 xia4r5}{1,3,2}{⼀,⼀,⼉}[HSK 1,5]
  \definition{s.}{um tempo; um momento}
\end{EntryWithPhonetic}

\begin{EntryWithPhonetic}{一下子}{yi2 xia4 zi5}{1,3,3}{⼀,⼀,⼦}[HSK 5]
  \definition{adv.}{tudo de uma vez; de repente; em pouco tempo; em um curto espaço de tempo}
\end{EntryWithPhonetic}

\begin{EntryWithPhonetic}{一向}{yi2xiang4}{1,6}{⼀,⼝}[HSK 5]
  \definition{adv.}{desde o início; indica do passado até o presente}
\end{EntryWithPhonetic}

\begin{EntryWithPhonetic}{一样}{yi2yang4}{1,10}{⼀,⽊}[HSK 1]
  \definition{adj.}{o mesmo; igualmente; semelhante; tão\dots quanto\dots}
  \definition{part.}{na mesma medida; anexado a verbos ou palavras nominais, indica uma comparação ou semelhança, equivalente a 似的}
  \seealsoref{似的}{shi4de5}
\end{EntryWithPhonetic}

\begin{EntryWithPhonetic}{一再}{yi2zai4}{1,6}{⼀,⼌}[HSK 4]
  \definition{adv.}{repetidamente; de novo e de novo; repetidas vezes; uma e outra vez}
\end{EntryWithPhonetic}

\begin{EntryWithPhonetic}{一战}{yi2zhan4}{1,9}{⼀,⼽}
  \definition*{s.}{Primeira Guerra Mundial}
\end{EntryWithPhonetic}

\begin{EntryWithPhonetic}{一致}{yi2zhi4}{1,10}{⼀,⾄}[HSK 4]
  \definition{adj.}{equado; idêntico; uniforme; unânime; nenhuma diferença (de opinião ou ação)}
  \definition{adv.}{juntos; em conjunto}
\end{EntryWithPhonetic}

%%%%%%%%%% 仪 %%%%%%%%%%
\subsection*{仪}\addcontentsline{loh}{figure}{仪 \dpy{yi2}}

\begin{EntryWithPhonetic}{仪}{yi2}{5}{⼈}
  \definition*{s.}{Sobrenome: Yi}
  \definition{s.}{aparência; porte | cerimônia; rito | presente; dádiva | aparelho; instrumento}
  \definition{v.}{(literário) admirar; ansiar por}
\end{EntryWithPhonetic}

\begin{EntryWithPhonetic}{仪器}{yi2qi4}{5,16}{⼈,⼝}[HSK 6]
  \definition[台]{s.}{aparelho; instrumento; ferramentas ou equipamentos utilizados para observação, medição, inspeção, etc. em pesquisas ou experimentos científicos; geralmente, são relativamente precisos e padronizados}
\end{EntryWithPhonetic}

\begin{EntryWithPhonetic}{仪式}{yi2shi4}{5,6}{⼈,⼷}[HSK 6]
  \definition{s.}{rito; cerimônia; procedimento e formato da cerimônia}
\end{EntryWithPhonetic}

%%%%%%%%%% 移 %%%%%%%%%%
\subsection*{移}\addcontentsline{loh}{figure}{移 \dpy{yi2}}

\begin{EntryWithPhonetic}{移}{yi2}{11}{⽲}[HSK 4]
  \definition*{s.}{Sobrenome: Yi}
  \definition{v.}{mover; remover; deslocar; mudar | mudar; alterar}
\end{EntryWithPhonetic}

\begin{EntryWithPhonetic}{移动}{yi2dong4}{11,6}{⽲,⼒}[HSK 4]
  \definition{v.}{deslocar; mover; mudar}
\end{EntryWithPhonetic}

\begin{EntryWithPhonetic}{移民}{yi2min2}{11,5}{⽲,⽒}[HSK 4]
  \definition[个,批]{s.}{emigrante; migrantes; aqueles que se mudam para um país ou estado estrangeiro para se estabelecer}
  \definition{v.}{migrar; imigrar}
\end{EntryWithPhonetic}

%%%%%%%%%% 遗 %%%%%%%%%%
\subsection*{遗}\addcontentsline{loh}{figure}{遗 \dpy{yi2}}

\begin{EntryWithPhonetic}{遗}{yi2}{12}{⾡}
  \definition*{s.}{Sobrenome: Yi}
  \definition{s.}{descarga involuntária de urina, etc. | algo perdido}
  \definition{v.}{perder | omitir | deixar para trás; guardar; não dar | deixar para trás após a morte; legar; transmitir}
\end{EntryWithPhonetic}

\begin{EntryWithPhonetic}{遗案}{yi2'an4}{12,10}{⾡,⽊}
  \definition{s.}{(lei) caso não resolvido}
\end{EntryWithPhonetic}

\begin{EntryWithPhonetic}{遗产}{yi2chan3}{12,6}{⾡,⼇}[HSK 4]
  \definition[笔,份]{s.}{legado; herança; patrimônio; propriedade deixada pelo falecido | patrimônio; riqueza cultural ou riqueza material transmitida pela história}
\end{EntryWithPhonetic}

\begin{EntryWithPhonetic}{遗传}{yi2chuan2}{12,6}{⾡,⼈}[HSK 4]
  \definition{v.}{herdar, descender, transmitir, passar adiante}
\end{EntryWithPhonetic}

\begin{EntryWithPhonetic}{遗骸}{yi2hai2}{12,15}{⾡,⾻}
  \definition{v.}{restos mortais}
\end{EntryWithPhonetic}

\begin{EntryWithPhonetic}{遗憾}{yi2han4}{12,16}{⾡,⼼}[HSK 6]
  \definition{adj.}{triste; arrependido; contrito; sentir pena de situações que estão fora de controle ou são insatisfatórias}
  \definition{s.}{pena; arrependimento; sentindo pena que os desejos não se realizaram}
\end{EntryWithPhonetic}

\begin{EntryWithPhonetic}{遗迹}{yi2ji4}{12,9}{⾡,⾡}
  \definition{s.}{vestígio histórico; sítio; vestígio; traço; ruína; vestígios deixados por tempos antigos ou eras passadas}
\end{EntryWithPhonetic}

\begin{EntryWithPhonetic}{遗落}{yi2luo4}{12,12}{⾡,⾋}
  \definition{v.}{esquecer | deixar para trás (inadvertidamente) | deixar de fora | omitir}
\end{EntryWithPhonetic}

\begin{EntryWithPhonetic}{遗男}{yi2nan2}{12,7}{⾡,⽥}
  \definition{s.}{órfão | filho póstumo}
\end{EntryWithPhonetic}

\begin{EntryWithPhonetic}{遗嘱}{yi2zhu3}{12,15}{⾡,⼝}
  \definition{s.}{testamento}
\end{EntryWithPhonetic}

%%%%%%%%%% 颐 %%%%%%%%%%
\subsection*{颐}\addcontentsline{loh}{figure}{颐 \dpy{yi2}}

\begin{EntryWithPhonetic}{颐}{yi2}{13}{⾴}
  \definition{s.}{bochecha}
  \definition{v.}{manter-se em forma; cuidar de si mesmo}
\end{EntryWithPhonetic}

\begin{EntryWithPhonetic}{颐和园}{yi2he2yuan2}{13,8,7}{⾴,⼝,⼞}
  \definition*{s.}{Palácio de Verão}
\end{EntryWithPhonetic}

%%%%%%%%%% 疑 %%%%%%%%%%
\subsection*{疑}\addcontentsline{loh}{figure}{疑 \dpy{yi2}}

\begin{EntryWithPhonetic}{疑}{yi2}{14}{⽦}
  \definition{adj.}{duvidoso; incerto}
  \definition{v.}{duvidar; desacreditar; suspeitar}
\end{EntryWithPhonetic}

\begin{EntryWithPhonetic}{疑问}{yi2wen4}{14,6}{⽦,⾨}[HSK 4]
  \definition[个,些]{s.}{dúvida; consulta; pergunta; questionamento; coisas que não podem ser determinadas ou explicadas}
\end{EntryWithPhonetic}

%%%%%%%%%% 乙 %%%%%%%%%%
\subsection*{乙}\addcontentsline{loh}{figure}{乙 \dpy{yi3}}

\begin{EntryWithPhonetic}{乙}{yi3}{1}{⼄}[HSK 5][Kangxi 5]
  \definition*{s.}{Sobrenome: Yi}
  \definition{num.}{segundo}
  \definition{s.}{o segundo lugar do Tian Gan | uma nota da escala em Gongchepu (工尺谱); nível superior na música tradicional chinesa}
  \seealsoref{工尺谱}{gong1 che3 pu3}
\end{EntryWithPhonetic}

%%%%%%%%%% 已 %%%%%%%%%%
\subsection*{已}\addcontentsline{loh}{figure}{已 \dpy{yi3}}

\begin{EntryWithPhonetic}{已}{yi3}{3}{⼰}[HSK 3]
  \definition{adv.}{já | posteriormente; mais tarde; depois de algum tempo | demasiadamente; excessivamente}
  \definition{v.}{terminar; parar; cessar}
\end{EntryWithPhonetic}

\begin{EntryWithPhonetic}{已故}{yi3gu4}{3,9}{⼰,⽁}
  \definition{adj.}{falecido}
\end{EntryWithPhonetic}

\begin{EntryWithPhonetic}{已婚}{yi3hun1}{3,11}{⼰,⼥}
  \definition{adj.}{casado}
\end{EntryWithPhonetic}

\begin{EntryWithPhonetic}{已经}{yi3jing1}{3,8}{⼰,⽷}[HSK 2]
  \definition{adv.}{já; indica que uma ação ou mudança foi concluída ou atingiu um determinado nível}
\end{EntryWithPhonetic}

\begin{EntryWithPhonetic}{已久}{yi3jiu3}{3,3}{⼰,⼃}
  \definition{adv.}{já faz muito tempo}
\end{EntryWithPhonetic}

\begin{EntryWithPhonetic}{已灭}{yi3mie4}{3,5}{⼰,⽕}
  \definition{adj.}{extinto}
\end{EntryWithPhonetic}

\begin{EntryWithPhonetic}{已然}{yi3ran2}{3,12}{⼰,⽕}
  \definition{adv.}{já | já ser assim}
\end{EntryWithPhonetic}

\begin{EntryWithPhonetic}{已知}{yi3zhi1}{3,8}{⼰,⽮}
  \definition{v.}{conhecido (ter ciência)}
\end{EntryWithPhonetic}

%%%%%%%%%% 以 %%%%%%%%%%
\subsection*{以}\addcontentsline{loh}{figure}{以 \dpy{yi3}}

\begin{EntryWithPhonetic}{以}{yi3}{4}{⼈}
  \definition*{s.}{Sobrenome: Yi}
  \definition{conj.}{e; bem como; o mesmo que 而 | de modo a; a fim de; usado no início da frase seguinte para expressar o propósito de atingir um determinado objetivo}
  \definition{prep.}{por; com | de acordo com | por causa de; porque | em (uma data fixa); em (um certo momento) | colocado antes de palavras posicionais simples, indica os limites de tempo, posição e quantidade}
  \seealsoref{而}{er2}
\end{EntryWithPhonetic}

\begin{EntryWithPhonetic}{以便}{yi3bian4}{4,9}{⼈,⼈}[HSK 5]
  \definition{conj.}{para que; de modo que; a fim de; com o objetivo de; para o propósito de; usado no início da segunda parte da frase, indica que o objetivo mencionado na segunda parte será facilmente alcançado}
\end{EntryWithPhonetic}

\begin{EntryWithPhonetic}{以此}{yi3ci3}{4,6}{⼈,⽌}
  \definition{adv.}{devido a esta | deste modo | por isso | com isso}
\end{EntryWithPhonetic}

\begin{EntryWithPhonetic}{以后}{yi3 hou4}{4,6}{⼈,⼝}[HSK 2]
  \definition{s.}{depois; mais tarde; após; daqui em diante}
\end{EntryWithPhonetic}

\begin{EntryWithPhonetic}{以及}{yi3ji2}{4,3}{⼈,⼃}[HSK 4]
  \definition{conj.}{assim como; juntamente como; bem como; também}
\end{EntryWithPhonetic}

\begin{EntryWithPhonetic}{以来}{yi3lai2}{4,7}{⼈,⽊}[HSK 3]
  \definition{s.}{desde (em termos de tempo); indica um período de tempo desde um determinado momento no passado até o presente}
\end{EntryWithPhonetic}

\begin{EntryWithPhonetic}{以免}{yi3mian3}{4,7}{⼈,⼉}
  \definition{conj.}{para evitar isso}
\end{EntryWithPhonetic}

\begin{EntryWithPhonetic}{以内}{yi3 nei4}{4,4}{⼈,⼌}[HSK 4]
  \definition{adv.}{dentro de; menos que; não mais que; dentro de certos limites de tempo, premissas, quantidade e escopo}
\end{EntryWithPhonetic}

\begin{EntryWithPhonetic}{以期}{yi3qi1}{4,12}{⼈,⽉}
  \definition{v.}{tentando | esperando | esperando por}
\end{EntryWithPhonetic}

\begin{EntryWithPhonetic}{以前}{yi3qian2}{4,9}{⼈,⼑}[HSK 2]
  \definition{s.}{antes; antigamente; anteriormente (no tempo); agora ou o período anterior ao tempo indicado}
\end{EntryWithPhonetic}

\begin{EntryWithPhonetic}{以求}{yi3qiu2}{4,7}{⼈,⽔}
  \definition{conj.}{a fim de}
\end{EntryWithPhonetic}

\begin{EntryWithPhonetic}{以色列}{yi3se4lie4}{4,6,6}{⼈,⾊,⼑}
  \definition*{s.}{Israel}
\end{EntryWithPhonetic}

\begin{EntryWithPhonetic}{以上}{yi3 shang4}{4,3}{⼈,⼀}[HSK 2]
  \definition[本]{s.}{mais do que; sobre; acima; indica posição, ordem ou número acima de um determinado ponto | o acima; o precedente; o acima mencionado; refere-se às palavras anteriores}
\end{EntryWithPhonetic}

\begin{EntryWithPhonetic}{以外}{yi3 wai4}{4,5}{⼈,⼣}[HSK 2]
  \definition{s.}{além; exceto; fora; diferente de; fora dos limites de um determinado tempo, quantidade ou lugar}
\end{EntryWithPhonetic}

\begin{EntryWithPhonetic}{以往}{yi3wang3}{4,8}{⼈,⼻}[HSK 5]
  \definition{s.}{antes; anterior; no passado}
\end{EntryWithPhonetic}

\begin{EntryWithPhonetic}{以为}{yi3wei2}{4,4}{⼈,⼂}[HSK 2]
  \definition{v.}{pensar; acreditar; considerar (geralmente erroneamente); expressa opiniões e atitudes em relação às coisas, geralmente erradas}
\end{EntryWithPhonetic}

\begin{EntryWithPhonetic}{以下}{yi3 xia4}{4,3}{⼈,⼀}[HSK 2]
  \definition[所]{s.}{abaixo; sob; indica posição, ordem ou número abaixo de um certo ponto | seguinte; refere-se às seguintes palavras}
\end{EntryWithPhonetic}

\begin{EntryWithPhonetic}{以至}{yi3zhi4}{4,6}{⼈,⾄}
  \definition{adv.}{até}
  \definition{conj.}{a tal ponto que\dots}
  \seealsoref{以至于}{yi3zhi4yu2}
\end{EntryWithPhonetic}

\begin{EntryWithPhonetic}{以至于}{yi3zhi4yu2}{4,6,3}{⼈,⾄,⼆}
  \definition{adv.}{até}
  \definition{conj.}{na medida em que\dots}
  \seealsoref{以至}{yi3zhi4}
\end{EntryWithPhonetic}

%%%%%%%%%% 尾 %%%%%%%%%%
\subsection*{尾}\addcontentsline{loh}{figure}{尾 \dpy{yi3}}

\begin{EntryWithPhonetic}{尾}{yi3}{7}{⼫}
  \definition{s.}{rabo do cavalo | parte posterior pontiaguda de um gafanhoto etc.}
  \seeref{wei3}
\end{EntryWithPhonetic}

%%%%%%%%%% 椅 %%%%%%%%%%
\subsection*{椅}\addcontentsline{loh}{figure}{椅 \dpy{yi3}}

\begin{EntryWithPhonetic}{椅}{yi3}{12}{⽊}
  \definition*{s.}{Sobrenome: Yi}
  \definition{s.}{cadeira}
\end{EntryWithPhonetic}

\begin{EntryWithPhonetic}{椅子}{yi3zi5}{12,3}{⽊,⼦}[HSK 2]
  \definition[把,套,排]{s.}{cadeira; assentos com encosto, feitos principalmente de madeira, bambu, rattan, etc.; móveis com pernas, mas sem encosto para as pessoas se sentarem}
\end{EntryWithPhonetic}

%%%%%%%%%% 一 %%%%%%%%%%
\subsection*{一}\addcontentsline{loh}{figure}{一 \dpy{yi4}}

\begin{EntryWithPhonetic}{一}{yi4}{1}{⼀}[HSK 1][Kangxi 1]
  \definition{adv.}{uma vez | assim que | ao}
  \definition{num.}{um; 1 | um (artigo)}
  \seeref{yi1}
  \seeref{yi2}
\end{EntryWithPhonetic}

\begin{EntryWithPhonetic}{一般}{yi4ban1}{1,10}{⼀,⾈}[HSK 2]
  \definition{adj.}{o mesmo que; exatamente como | geral; ordinário; comum | médio; medíocre; o grau ou nível não é muito alto}
  \definition{adv.}{frequentemente; geralmente}
\end{EntryWithPhonetic}

\begin{EntryWithPhonetic}{一般来说}{yi4 ban1 lai2 shuo1}{1,10,7,9}{⼀,⾈,⽊,⾔}[HSK 4]
  \definition{expr.}{de modo geral; na média; no caso usual; a declaração usual}
\end{EntryWithPhonetic}

\begin{EntryWithPhonetic}{一边}{yi4bian1}{1,5}{⼀,⾡}[HSK 1]
  \definition{adj.}{igual; idêntico; da mesma forma}
  \definition{adv.}{enquanto; ao mesmo tempo; simultaneamente; indica que uma ação ocorre simultaneamente a outra ação}
  \definition{s.}{lado; um lado; um aspecto | ambos os lados; ao lado de}
\end{EntryWithPhonetic}

\begin{EntryWithPhonetic}{一点点}{yi4 dian3 dian3}{1,9,9}{⼀,⽕,⽕}[HSK 2]
  \definition{adj.}{um pouco; muito pouco ou um pouquinho}
\end{EntryWithPhonetic}

\begin{EntryWithPhonetic}{一点儿}{yi4dian3r5}{1,9,2}{⼀,⽕,⼉}[HSK 1]
  \definition{adv.}{um pouco; uma pitada; uma gota; uma amostra; uma pequena quantidade; ({adj.} + (一)点儿, 一点儿 + {s.} ou 有 + (一)点儿 + {s.})}
\end{EntryWithPhonetic}

\begin{EntryWithPhonetic}{一番}{yi4 fan1}{1,12}{⼀,⽥}[HSK 6]
  \definition{adv.}{uma demonstração de, uma dose de, um pedaço de (conversa, investigação, pensamento)}
\end{EntryWithPhonetic}

\begin{EntryWithPhonetic}{一方面}{yi4 fang1 mian4}{1,4,9}{⼀,⽅,⾯}[HSK 3]
  \definition{s.}{um lado; um dos dois aspectos opostos ou um lado de algo que está relacionado a outro}
  \seealsoref{一方面…,一方面…}{yi4 fang1 mian4 yi4 fang1 mian4}
\end{EntryWithPhonetic}

\begin{EntryWithPhonetic}{一方面…,一方面…}{yi4 fang1 mian4 yi4 fang1 mian4}{1,4,9,1,4,9}{⼀,⽅,⾯,⼀,⽅,⾯}[HSK 3]
  \definition{conj.}{por um lado\dots, por outro lado\dots; conecta duas orações paralelas (devem ser usadas juntas)}[\underline{一方面}觉得兴奋,\underline{一方面}又害怕。===Por um lado, sinto-me entusiasmado, mas, por outro, também sinto medo.]
\end{EntryWithPhonetic}

\begin{EntryWithPhonetic}{一口气}{yi4 kou3 qi4}{1,3,4}{⼀,⼝,⽓}[HSK 5]
  \definition{adv.}{em um só fôlego; sem pausa; fazer algo continuamente}
\end{EntryWithPhonetic}

\begin{EntryWithPhonetic}{一流}{yi4liu2}{1,10}{⼀,⽔}[HSK 5]
  \definition{adj.}{clássico; de primeira linha; de primeira classe; o melhor}
  \definition[些]{s.}{tipo; mesmo tipo; da mesma classe; da mesma categoria; uma categoria}
\end{EntryWithPhonetic}

\begin{EntryWithPhonetic}{一模一样}{yi4 mu2 yi2 yang4}{1,14,1,10}{⼀,⽊,⼀,⽊}[HSK 6]
  \definition{expr.}{tão parecidos quanto duas ervilhas; ser exatamente iguais; muito parecido, a mesma aparência}
\end{EntryWithPhonetic}

\begin{EntryWithPhonetic}{一齐}{yi4 qi2}{1,6}{⼀,⿑}[HSK 6]
  \definition{adv.}{juntos; em uníssono; simultaneamente; ao mesmo tempo; indica que diferentes sujeitos emitem simultaneamente o mesmo comportamento ou o mesmo sujeito emite vários comportamentos diferentes ao mesmo tempo}
\end{EntryWithPhonetic}

\begin{EntryWithPhonetic}{一起}{yi4qi3}{1,10}{⼀,⾛}[HSK 1]
  \definition{adv.}{juntos; em companhia; indica o mesmo local, ao mesmo tempo que se faz algo | no total; em todos; no conjunto}
  \definition{s.}{no mesmo lugar}
\end{EntryWithPhonetic}

\begin{EntryWithPhonetic}{一身}{yi4 shen1}{1,7}{⼀,⾝}[HSK 5]
  \definition{s.}{o corpo inteiro; em todo o corpo | um terno; (um conjunto completo de) roupas | sozinho; uma única pessoa; relativo a uma única pessoa}
\end{EntryWithPhonetic}

\begin{EntryWithPhonetic}{一生}{yi4 sheng1}{1,5}{⼀,⽣}[HSK 2]
  \definition{s.}{vida inteira; toda a vida; ao longo da vida; todo o tempo desde o nascimento até a morte; às vezes exagerado para indicar um longo período de tempo no curso da vida}
\end{EntryWithPhonetic}

\begin{EntryWithPhonetic}{一时}{yi4 shi2}{1,7}{⼀,⽇}[HSK 6]
  \definition{adv.}{por um curto período; temporário | (usado em pares) agora\dots, agora\dots; este momento\dots, e o próximo\dots; o mesmo que 时而}
  \definition{s.}{um período de tempo | um momento; um breve momento; um tempo muito curto}
  \seealsoref{时而}{shi2'er2}
  \seealsoref{一时…,一时…}{yi4 shi2 yi4 shi2}
\end{EntryWithPhonetic}

\begin{EntryWithPhonetic}{一时…,一时…}{yi4 shi2 yi4 shi2}{1,7,1,7}{⼀,⽇,⼀,⽇}[HSK 6]
  \definition{adv.}{por um tempo\dots, por um tempo\dots}
  \seealsoref{一时}{yi4 shi2}
\end{EntryWithPhonetic}

\begin{EntryWithPhonetic}{一同}{yi4tong2}{1,6}{⼀,⼝}[HSK 6]
  \definition{adv.}{juntos; ao mesmo tempo e lugar}
\end{EntryWithPhonetic}

\begin{EntryWithPhonetic}{一些}{yi4 xie1}{1,8}{⼀,⼆}[HSK 1]
  \definition{clas.}{alguns; um número de; quantidade indeterminada | um pouco; uma pequena quantidade | mais de um; mais de uma vez; indica mais de um ou mais de uma vez, etc. | uma ligeira mudança no grau, intensidade; usado após certos verbos, adjetivos, etc., para indicar uma quantidade muito pequena}
  \definition{pron.}{uns; alguns}
\end{EntryWithPhonetic}

\begin{EntryWithPhonetic}{一直}{yi4zhi2}{1,8}{⼀,⽬}[HSK 2]
  \definition{adv.}{direto; indica que permanece inalterado em uma direção | sempre; continuamente; o tempo todo; o tempo todo; indica que a ação é sempre ininterrupta ou o estado é sempre inalterado | de um ponto a outro sem enfatizar nenhuma exceção}
\end{EntryWithPhonetic}

%%%%%%%%%% 义 %%%%%%%%%%
\subsection*{义}\addcontentsline{loh}{figure}{义 \dpy{yi4}}

\begin{EntryWithPhonetic}{义}{yi4}{3}{⼂}
  \definition*{s.}{Sobrenome: Yi}
  \definition{adj.}{justo; equitativo | adotado; adotivo | juramentado | artificial; falso}
  \definition[个]{s.}{justiça; retidão | laços humanos; relacionamento | significado; importância}
\end{EntryWithPhonetic}

\begin{EntryWithPhonetic}{义务}{yi4wu4}{3,5}{⼂,⼒}[HSK 4]
  \definition{adj.}{voluntário; fornecer serviços ou ajuda a outros gratuitamente}
  \definition[项]{s.}{dever; obrigação; responsabilidades perante a lei, em oposição a 权利 | obrigação moral; responsabilidade moral}
  \seealsoref{权利}{quan2li4}
\end{EntryWithPhonetic}

%%%%%%%%%% 亿 %%%%%%%%%%
\subsection*{亿}\addcontentsline{loh}{figure}{亿 \dpy{yi4}}

\begin{EntryWithPhonetic}{亿}{yi4}{3}{⼈}[HSK 2]
  \definition*{s.}{Sobrenome: Yi}
  \definition{num.}{cem milhões; 100.000.000; 1.0000.0000}
\end{EntryWithPhonetic}

%%%%%%%%%% 艺 %%%%%%%%%%
\subsection*{艺}\addcontentsline{loh}{figure}{艺 \dpy{yi4}}

\begin{EntryWithPhonetic}{艺}{yi4}{4}{⾋}
  \definition*{s.}{Sobrenome: Yi}
  \definition[个,种]{s.}{habilidade | arte | regra; norma | padrão; critério; diretrizes | limite}
  \definition{v.}{plantar; crescer}
\end{EntryWithPhonetic}

\begin{EntryWithPhonetic}{艺人}{yi4 ren2}{4,2}{⾋,⼈}[HSK 6]
  \definition[位]{s.}{artista performático; ator profissional; ator ou artista (em teatro local, narradores, acrobacia ou outro show business); atores de ópera, arte popular, acrobacia, cinema e televisão, etc. | artesão; artífice}
\end{EntryWithPhonetic}

\begin{EntryWithPhonetic}{艺术}{yi4shu4}{4,5}{⾋,⽊}[HSK 3]
  \definition{adj.}{artístico,único; elegante}
  \definition[个,种,门,项,类]{s.}{arte; literatura e arte | habilidade; arte; ofício; métodos criativos}
\end{EntryWithPhonetic}

%%%%%%%%%% 艾 %%%%%%%%%%
\subsection*{艾}\addcontentsline{loh}{figure}{艾 \dpy{yi4}}

\begin{EntryWithPhonetic}{艾}{yi4}{5}{⾋}
  \definition{adj.}{estável}
  \definition{v.}{ser corrigido; estar corrigido}
  \seeref{ai4}
\end{EntryWithPhonetic}

%%%%%%%%%% 议 %%%%%%%%%%
\subsection*{议}\addcontentsline{loh}{figure}{议 \dpy{yi4}}

\begin{EntryWithPhonetic}{议}{yi4}{5}{⾔}
  \definition[个,则,条]{s.}{opinião; visão}
  \definition{v.}{discutir; trocar pontos de vista sobre; conversar sobre | comentar; observar | fofocar; comentar}
\end{EntryWithPhonetic}

\begin{EntryWithPhonetic}{议论}{yi4lun4}{5,6}{⾔,⾔}[HSK 4]
  \definition[个]{s.}{comentário; discussão; opiniões ou pontos de vista sobre o que é bom ou ruim, certo ou errado em relação a pessoas ou coisas}
  \definition{v.}{discutir; comentar; falar sobre; expressar opiniões e trocar pontos de vista sobre o bom, o ruim, o certo e o errado de pessoas ou coisas}
\end{EntryWithPhonetic}

\begin{EntryWithPhonetic}{议题}{yi4 ti2}{5,15}{⾔,⾴}[HSK 6]
  \definition[项,个]{s.}{assunto; assunto em discussão; tópico para discussão}
\end{EntryWithPhonetic}

%%%%%%%%%% 亦 %%%%%%%%%%
\subsection*{亦}\addcontentsline{loh}{figure}{亦 \dpy{yi4}}

\begin{EntryWithPhonetic}{亦}{yi4}{6}{⼇}
  \definition*{s.}{Sobrenome: Yi}
  \definition{adv.}{também; também (que significa o mesmo)}
\end{EntryWithPhonetic}

%%%%%%%%%% 异 %%%%%%%%%%
\subsection*{异}\addcontentsline{loh}{figure}{异 \dpy{yi4}}

\begin{EntryWithPhonetic}{异}{yi4}{6}{⼶}
  \definition{adj.}{diferente | estranho; incomum; extraordinário; especial | outro}
  \definition{v.}{surpreender | separar; divorciar-se}
\end{EntryWithPhonetic}

\begin{EntryWithPhonetic}{异常}{yi4chang2}{6,11}{⼶,⼱}[HSK 6]
  \definition{adj.}{incomum; anormal; descreve uma situação diferente do normal}
  \definition{adv.}{extremamente; particularmente; excepcionalmente; descreve uma situação que atingiu um nível extremamente alto}
\end{EntryWithPhonetic}

%%%%%%%%%% 衣 %%%%%%%%%%
\subsection*{衣}\addcontentsline{loh}{figure}{衣 \dpy{yi4}}

\begin{EntryWithPhonetic}{衣}{yi4}{6}{⾐}
  \definition{v.}{vestir-se; vestir alguém}
  \seeref{yi1}
\end{EntryWithPhonetic}

%%%%%%%%%% 易 %%%%%%%%%%
\subsection*{易}\addcontentsline{loh}{figure}{易 \dpy{yi4}}

\begin{EntryWithPhonetic}{易}{yi4}{8}{⽇}
  \definition*{s.}{Sobrenome: Yi}
  \definition{adj.}{fácil | amigável; pacífico}
  \definition{v.}{modificar; transformar | trocar | subestimar; desprezar}
\end{EntryWithPhonetic}

%%%%%%%%%% 益 %%%%%%%%%%
\subsection*{益}\addcontentsline{loh}{figure}{益 \dpy{yi4}}

\begin{EntryWithPhonetic}{益}{yi4}{10}{⽫}
  \definition*{s.}{Sobrenome: Yi}
  \definition{adj.}{benéfico}
  \definition{adv.}{Literário: mais; cada vez mais}[空气污染问题日益严重。===O problema da poluição do ar está se tornando cada vez mais sério.]
  \definition{s.}{benefício; lucro; vantagem}
  \definition{v.}{aumentar}
\end{EntryWithPhonetic}

\begin{EntryWithPhonetic}{益虫}{yi4chong2}{10,6}{⽫,⾍}
  \definition{s.}{inseto benéfico (oposto a 害虫)}
  \seealsoref{害虫}{hai4chong2}
\end{EntryWithPhonetic}

%%%%%%%%%% 意 %%%%%%%%%%
\subsection*{意}\addcontentsline{loh}{figure}{意 \dpy{yi4}}

\begin{EntryWithPhonetic}{意}{yi4}{13}{⼼}
  \definition*{s.}{Itália, abreviação de 意大利}
  \definition{s.}{ideia; significado; pensamento | desejo; vontade; intenção | significância}
  \seealsoref{意大利}{yi4da4li4}
\end{EntryWithPhonetic}

\begin{EntryWithPhonetic}{意大利}{yi4da4li4}{13,3,7}{⼼,⼤,⼑}
  \definition*{s.}{Itália}
\end{EntryWithPhonetic}

\begin{EntryWithPhonetic}{意见}{yi4jian4}{13,4}{⼼,⾒}[HSK 2]
  \definition[种,点,条]{s.}{ideia; visão; opinião; sugestão; uma certa visão ou ideia sobre algo | objeção; reclamação; opinião divergente; (em relação a uma pessoa ou coisa) o sentimento de estar insatisfeito com algo porque está errado}
\end{EntryWithPhonetic}

\begin{EntryWithPhonetic}{意识}{yi4shi2}{13,7}{⼼,⾔}[HSK 5]
  \definition{s.}{consciência; percepção; grau de reconhecimento e importância atribuído a uma determinada questão}
  \definition{s.}{consciência; percepção; o reflexo da mente humana no mundo material objetivo é a soma de vários processos psicológicos, como sensação e pensamento | consciência; conscientização; o grau de conscientização e atenção dada a um problema}
  \definition{v.}{perceber; despertar para; estar ciente de; sentir, descobrir o que antes não se sentia ou não se descobria; geralmente é usado junto com 到}
  \seealsoref{到}{dao4}
\end{EntryWithPhonetic}

\begin{EntryWithPhonetic}{意思}{yi4si5}{13,9}{⼼,⼼}[HSK 2]
  \definition[个]{s.}{ideia; significado; o significado da linguagem e das palavras; conteúdo ideológico | desejo; vontade; opiniões | um símbolo de afeto, apreciação, gratidão, etc. | dica; traço; sugestão; refere-se principalmente ao afeto entre homens e mulheres | diversão; interesse}
  \definition{v.}{dar uma dica; demonstrar sua gratidão com presentes ou outros meios}
\end{EntryWithPhonetic}

\begin{EntryWithPhonetic}{意外}{yi4wai4}{13,5}{⼼,⼣}[HSK 3]
  \definition{adj.}{inesperado; imprevisto}
  \definition{adv.}{acidentalmente}
  \definition[个,种]{s.}{acidente; infortúnio; um infortúnio inesperado}
\end{EntryWithPhonetic}

\begin{EntryWithPhonetic}{意味着}{yi4wei4zhe5}{13,8,11}{⼼,⼝,⽬}[HSK 5]
  \definition{v.}{significar; subentender; implicar; entender que tem vários significados}
\end{EntryWithPhonetic}

\begin{EntryWithPhonetic}{意想不到}{yi4 xiang3 bu2 dao4}{13,13,4,8}{⼼,⼼,⼀,⼑}[HSK 6]
  \definition{expr.}{anteriormente inimaginável | inesperado}
\end{EntryWithPhonetic}

\begin{EntryWithPhonetic}{意义}{yi4yi4}{13,3}{⼼,⼂}[HSK 3]
  \definition[个,种,层,重,点]{s.}{sentido; significado; o significado expresso por meio de linguagem escrita ou outros sinais; o significado identificado por meio de ações ou obtenção | valor; efeito; significado; influência; impacto}
\end{EntryWithPhonetic}

\begin{EntryWithPhonetic}{意译}{yi4yi4}{13,7}{⼼,⾔}
  \definition{s.}{tradução livre | significado (de expressão estrangeira) | paráfrase | tradução do significado (em oposição à tradução literal)}
  \seealsoref{直译}{zhi2yi4}
\end{EntryWithPhonetic}

\begin{EntryWithPhonetic}{意愿}{yi4 yuan4}{13,14}{⼼,⽕}[HSK 6]
  \definition{s.}{desejo; aspiração; vontade}
\end{EntryWithPhonetic}

\begin{EntryWithPhonetic}{意指}{yi4zhi3}{13,9}{⼼,⼿}
  \definition{v.}{implicar | significar}
\end{EntryWithPhonetic}

\begin{EntryWithPhonetic}{意志}{yi4zhi4}{13,7}{⼼,⼼}[HSK 5]
  \definition[个,股]{s.}{vontade; determinação; desejo; força de vontade; o estado psicológico produzido pela determinação de atingir um determinado objetivo, muitas vezes expresso por meio de linguagem e ações}
\end{EntryWithPhonetic}

%%%%%%%%%% 因 %%%%%%%%%%
\subsection*{因}\addcontentsline{loh}{figure}{因 \dpy{yin1}}

\begin{EntryWithPhonetic}{因}{yin1}{6}{⼞}[HSK 6]
  \definition*{s.}{Sobrenome: Yin}
  \definition{conj.}{porque; orações de conexão, indicando relações de causa e efeito}
  \definition{prep.}{com base em; à luz de; de acordo com; a introdução da ação comportamental equivale a 按照 ou 根据}
  \definition{s.}{causa; motivo; condições em que algo ocorre ou causa um determinado resultado (em oposição a 果)}
  \definition{v.}{seguir; continuar; fazer como sempre fez | estar em conformidade com; estar de acordo com; depender; contar com}
  \seealsoref{按照}{an4zhao4}
  \seealsoref{根据}{gen1ju4}
  \seealsoref{果}{guo3}
\end{EntryWithPhonetic}

\begin{EntryWithPhonetic}{因此}{yin1ci3}{6,6}{⼞,⽌}[HSK 3]
  \definition{conj.}{assim; portanto; consequentemente}
\end{EntryWithPhonetic}

\begin{EntryWithPhonetic}{因此就}{yin1ci3 jiu4}{6,6,12}{⼞,⽌,⼪}
  \definition{conj.}{portanto}
\end{EntryWithPhonetic}

\begin{EntryWithPhonetic}{因而}{yin1'er2}{6,6}{⼞,⽽}[HSK 5]
  \definition{conj.}{assim; como resultado; com o resultado que; conecta frases, indicando relação de causa e efeito}
\end{EntryWithPhonetic}

\begin{EntryWithPhonetic}{因素}{yin1su4}{6,10}{⼞,⽷}[HSK 6]
  \definition[个,种]{s.}{fator; elemento; os componentes que constituem a essência das coisas | fator; as razões ou condições que determinam o sucesso ou o fracasso de algo}
\end{EntryWithPhonetic}

\begin{EntryWithPhonetic}{因为}{yin1wei4}{6,4}{⼞,⼂}[HSK 2]
  \definition{conj.}{porque; indica o motivo e a frase seguinte indica o resultado}
  \definition{prep.}{por causa de; por conta de; indica razão ou justificativa}
\end{EntryWithPhonetic}

\begin{EntryWithPhonetic}{因为…所以…}{yin1wei4 suo3yi3}{6,4,8,4}{⼞,⼂,⼾,⼈}[HSK 2]
  \definition{conj.}{porque\dots portanto\dots}
\end{EntryWithPhonetic}

%%%%%%%%%% 阴 %%%%%%%%%%
\subsection*{阴}\addcontentsline{loh}{figure}{阴 \dpy{yin1}}

\begin{EntryWithPhonetic}{阴}{yin1}{6}{⾩}[HSK 2]
  \definition*{s.}{Yin, o princípio negativo de Yin e Yang | A Lua; refere-se a Taiyin | Sobrenome: Yin}
  \definition{adj.}{nublado; opaco; sombrio | escondido; secreto; não exposto | sinistro | do mundo inferior; dos fantasmas | Física: negativo; cátodo | nublado; mais de 80\% do céu estão cobertos por nuvens | em talhe-doce; rebaixado | (matéria) carregada negativamente}
  \definition[片]{s.}{sombra; lugar sombrio | partes íntimas (especialmente da mulher) | ao norte de uma colina ou ao sul de um rio | verso | entalhe}
  \seealsoref{阳}{yang2}
  \seealsoref{阴阳}{yin1yang2}
\end{EntryWithPhonetic}

\begin{EntryWithPhonetic}{阴谋}{yin1mou2}{6,11}{⾩,⾔}[HSK 6]
  \definition[个,场,起]{s.}{trama; conspiração; um esquema para fazer o mal em segredo}
  \definition{v.}{tramar; conspirar secretamente (fazer algo ruim)}
\end{EntryWithPhonetic}

\begin{EntryWithPhonetic}{阴天}{yin1 tian1}{6,4}{⾩,⼤}[HSK 2]
  \definition[个]{s.}{nublado; céu nublado; dia nublado; uma condição climática em que 80\% do céu está coberto por nuvens e apenas um pouco de sol pode ser visto}
\end{EntryWithPhonetic}

\begin{EntryWithPhonetic}{阴阳}{yin1yang2}{6,6}{⾩,⾩}
  \definition*{s.}{Yin e Yang}
  \seealsoref{阳}{yang2}
  \seealsoref{阴}{yin1}
\end{EntryWithPhonetic}

\begin{EntryWithPhonetic}{阴影}{yin1 ying3}{6,15}{⾩,⼺}[HSK 6]
  \definition{s.}{sombra; sombra escura | uma analogia de elementos negativos em negócios, relacionamentos, estado mental, etc.}
\end{EntryWithPhonetic}

%%%%%%%%%% 音 %%%%%%%%%%
\subsection*{音}\addcontentsline{loh}{figure}{音 \dpy{yin1}}

\begin{EntryWithPhonetic}{音}{yin1}{9}{⾳}[Kangxi 180]
  \definition[个,种]{s.}{som; som musical | notícias; novidades; informação | tom; refere-se especificamente a uma sílaba ou fonética | sílaba; refere-se a sílabas (um caractere chinês é uma sílaba)}
  \definition{v.}{vocalizar}
\end{EntryWithPhonetic}

\begin{EntryWithPhonetic}{音节}{yin1 jie2}{9,5}{⾳,⾋}[HSK 2]
  \definition{s.}{sílaba}
\end{EntryWithPhonetic}

\begin{EntryWithPhonetic}{音量}{yin1 liang4}{9,12}{⾳,⾥}[HSK 6]
  \definition[把]{s.}{volume; volume do som; a força de um som}
\end{EntryWithPhonetic}

\begin{EntryWithPhonetic}{音像}{yin1 xiang4}{9,13}{⾳,⼈}[HSK 6]
  \definition{s.}{audiovisual; produtos audiovisuais; o nome coletivo para gravações de áudio e vídeo}
\end{EntryWithPhonetic}

\begin{EntryWithPhonetic}{音乐}{yin1yue4}{9,5}{⾳,⼃}[HSK 2]
  \definition[种,段,张,曲]{s.}{música; ramo da arte que cria imagens artísticas, expressa pensamentos e sentimentos e reflete a vida real por meio da melodia e do ritmo da música; geralmente é dividido em duas categorias: música vocal e música instrumental}
\end{EntryWithPhonetic}

\begin{EntryWithPhonetic}{音乐光碟}{yin1yue4 guang1die2}{9,5,6,14}{⾳,⼃,⼉,⽯}
  \definition{s.}{CD de música}
\end{EntryWithPhonetic}

\begin{EntryWithPhonetic}{音乐会}{yin1 yue4 hui4}{9,5,6}{⾳,⼃,⼈}[HSK 2]
  \definition[场]{s.}{concerto; atividades de execução de obras musicais}
\end{EntryWithPhonetic}

\begin{EntryWithPhonetic}{音乐家}{yin1yue4jia1}{9,5,10}{⾳,⼃,⼧}
  \definition{s.}{músico}
\end{EntryWithPhonetic}

\begin{EntryWithPhonetic}{音乐节}{yin1yue4jie2}{9,5,5}{⾳,⼃,⾋}
  \definition{s.}{festival de música}
\end{EntryWithPhonetic}

\begin{EntryWithPhonetic}{音乐厅}{yin1yue4ting1}{9,5,4}{⾳,⼃,⼚}
  \definition{s.}{auditório | teatro | \emph{concert hall}}
\end{EntryWithPhonetic}

\begin{EntryWithPhonetic}{音乐学}{yin1yue4xue2}{9,5,8}{⾳,⼃,⼦}
  \definition{s.}{musicologia}
\end{EntryWithPhonetic}

\begin{EntryWithPhonetic}{音乐学院}{yin1yue4xue2yuan4}{9,5,8,9}{⾳,⼃,⼦,⾩}
  \definition{s.}{conservatório | academia de música}
\end{EntryWithPhonetic}

\begin{EntryWithPhonetic}{音乐院}{yin1yue4yuan4}{9,5,9}{⾳,⼃,⾩}
  \definition{s.}{conservatório | instituto de música}
\end{EntryWithPhonetic}

%%%%%%%%%% 吟 %%%%%%%%%%
\subsection*{吟}\addcontentsline{loh}{figure}{吟 \dpy{yin2}}

\begin{EntryWithPhonetic}{吟}{yin2}{7}{⼝}
  \definition*{s.}{Sobrenome: Yin}
  \definition{s.}{canção (como um tipo de poesia clássica) | grito de certos animais ou insetos}
  \definition{v.}{cantar; recitar | gemer; lamentar}
\end{EntryWithPhonetic}

\begin{EntryWithPhonetic}{吟诗}{yin2shi1}{7,8}{⼝,⾔}
  \definition{v.}{recitar poesia}
\end{EntryWithPhonetic}

%%%%%%%%%% 银 %%%%%%%%%%
\subsection*{银}\addcontentsline{loh}{figure}{银 \dpy{yin2}}

\begin{EntryWithPhonetic}{银}{yin2}{11}{⾦}[HSK 3]
  \definition*{s.}{Sobrenome: Yin}
  \definition{adj.}{prateado; como a cor da prata}
  \definition[锭]{s.}{Ag, prata | refere-se a moeda ou a coisas relacionadas com moeda}
\end{EntryWithPhonetic}

\begin{EntryWithPhonetic}{银行}{yin2hang2}{11,6}{⾦,⾏}[HSK 2]
  \definition[个,家,所]{s.}{banco; instituições financeiras que operam depósitos, empréstimos, câmbio, poupança e outros negócios}
\end{EntryWithPhonetic}

\begin{EntryWithPhonetic}{银行卡}{yin2 hang2 ka3}{11,6,5}{⾦,⾏,⼘}[HSK 2]
  \definition{s.}{cartão bancário; cartão ATM}
\end{EntryWithPhonetic}

\begin{EntryWithPhonetic}{银河}{yin2he2}{11,8}{⾦,⽔}
  \definition*{s.}{Via Láctea}
  \seealsoref{银河系}{yin2he2xi4}
\end{EntryWithPhonetic}

\begin{EntryWithPhonetic}{银河系}{yin2he2xi4}{11,8,7}{⾦,⽔,⽷}
  \definition*{s.}{Galáxia Via Láctea}
  \seealsoref{银河}{yin2he2}
\end{EntryWithPhonetic}

\begin{EntryWithPhonetic}{银牌}{yin2 pai2}{11,12}{⾦,⽚}[HSK 3]
  \definition[枚]{s.}{medalha de prata; um tipo de medalha, concedida ao segundo colocado}
\end{EntryWithPhonetic}

\begin{EntryWithPhonetic}{银色}{yin2 se4}{11,6}{⾦,⾊}
  \definition{s.}{cor prata; prateado}
\end{EntryWithPhonetic}

%%%%%%%%%% 引 %%%%%%%%%%
\subsection*{引}\addcontentsline{loh}{figure}{引 \dpy{yin3}}

\begin{EntryWithPhonetic}{引}{yin3}{4}{⼸}[HSK 4]
  \definition*{s.}{Sobrenome: Yin}
  \definition{clas.}{uma unidade de comprimento (=33,333\dots metros)}
  \definition{v.}{puxar; esticar | liderar; conduzir; guiar | sair; deixar | sobressair | atrair; provocar; trazer à existência | causar; provocar | citar; ser usado como evidência ou justificativa}
\end{EntryWithPhonetic}

\begin{EntryWithPhonetic}{引导}{yin3dao3}{4,6}{⼸,⼨}[HSK 4]
  \definition{v.}{conduzir; guiar; liderar; andar na frente e deixar que os outros sigam atrás para ver ou andar; usar imagens ou sinais para mostrar às pessoas para onde ir | esclarecer; fornecer orientação em termos de ideias, métodos, conceitos, etc.}
\end{EntryWithPhonetic}

\begin{EntryWithPhonetic}{引进}{yin3 jin4}{4,7}{⼸,⾡}[HSK 4]
  \definition{v.}{importar; trazer de fora | recomendar; dar uma indicação}
\end{EntryWithPhonetic}

\begin{EntryWithPhonetic}{引起}{yin3qi3}{4,10}{⼸,⾛}[HSK 4]
  \definition{v.}{causar; despertar; levar a; desencadear; dar origem a}
\end{EntryWithPhonetic}

\begin{EntryWithPhonetic}{引擎}{yin3qing2}{4,16}{⼸,⼿}
  \definition[台]{s.}{motor | (empréstimo linguístico) \emph{engine}}
\end{EntryWithPhonetic}

%%%%%%%%%% 听 %%%%%%%%%%
\subsection*{听}\addcontentsline{loh}{figure}{听 \dpy{yin3}}

\begin{EntryWithPhonetic}{听}{yin3}{7}{⼝}
  \definition[个]{s.}{lata; embalagem metálica}
  \seeref{ting1}
\end{EntryWithPhonetic}

%%%%%%%%%% 饮 %%%%%%%%%%
\subsection*{饮}\addcontentsline{loh}{figure}{饮 \dpy{yin3}}

\begin{EntryWithPhonetic}{饮}{yin3}{7}{⾷}
  \definition{s.}{bebidas; \emph{drinks}; algo para beber | uma decocção da medicina chinesa para ser tomada fria | fluido retido}
  \definition{v.}{beber | cuidar; engolir a pílula amarga}
  \seeref{yin4}
\end{EntryWithPhonetic}

\begin{EntryWithPhonetic}{饮料}{yin3liao4}{7,10}{⾷,⽃}[HSK 5]
  \definition[杯,瓶,种]{s.}{bebida; drinque; líquidos processados e fabricados para consumo, como vinho, chá, refrigerantes, suco de laranja, etc.}
\end{EntryWithPhonetic}

\begin{EntryWithPhonetic}{饮食}{yin3shi2}{7,9}{⾷,⾷}[HSK 5]
  \definition{s.}{dieta; alimentos e bebidas}
  \definition{v.}{comer; beber}
\end{EntryWithPhonetic}

%%%%%%%%%% 隐 %%%%%%%%%%
\subsection*{隐}\addcontentsline{loh}{figure}{隐 \dpy{yin3}}

\begin{EntryWithPhonetic}{隐}{yin3}{11}{⾩}[HSK 6]
  \definition*{s.}{Sobrenome: Yin}
  \definition{adj.}{escondido; escondido profundamente | latente; adormecido; à espreita}
  \definition{pref.}{cripto-}
  \definition{s.}{segredo; assuntos ocultos}
  \definition{v.}{esconder; esconder da vista; ocultar}
\end{EntryWithPhonetic}

\begin{EntryWithPhonetic}{隐藏}{yin3 cang2}{11,17}{⾩,⾋}[HSK 6]
  \definition{v.}{esconder; ocultar}
\end{EntryWithPhonetic}

\begin{EntryWithPhonetic}{隐私}{yin3si1}{11,7}{⾩,⽲}[HSK 6]
  \definition[点,些]{s.}{privacidade; segredos de alguém; assuntos pessoais que você não quer contar ou tornar públicos}
\end{EntryWithPhonetic}

%%%%%%%%%% 印 %%%%%%%%%%
\subsection*{印}\addcontentsline{loh}{figure}{印 \dpy{yin4}}

\begin{EntryWithPhonetic}{印}{yin4}{5}{⼙}[HSK 6]
  \definition*{s.}{Sobrenome: Yin}
  \definition[个,枚,道,条]{s.}{selo; lacre; carimbo; estampilha | marca; estampa; impressão}
  \definition{v.}{imprimir; gravar | corresponder; conformar; estar em conformidade com}
\end{EntryWithPhonetic}

\begin{EntryWithPhonetic}{印刷}{yin4shua1}{5,8}{⼙,⼑}[HSK 5]
  \definition{v.}{imprimir; imprimir textos, imagens, etc. em papel}
\end{EntryWithPhonetic}

\begin{EntryWithPhonetic}{印象}{yin4xiang4}{5,11}{⼙,⾗}[HSK 3]
  \definition[种]{s.}{impressão; marca; ideia; os vestígios deixados por coisas objetivas na mente das pessoas}
\end{EntryWithPhonetic}

%%%%%%%%%% 饮 %%%%%%%%%%
\subsection*{饮}\addcontentsline{loh}{figure}{饮 \dpy{yin4}}

\begin{EntryWithPhonetic}{饮}{yin4}{7}{⾷}
  \definition{v.}{dar (aos animais) água para beber}
  \seeref{yin3}
\end{EntryWithPhonetic}

%%%%%%%%%% 应 %%%%%%%%%%
\subsection*{应}\addcontentsline{loh}{figure}{应 \dpy{ying1}}

\begin{EntryWithPhonetic}{应}{ying1}{7}{⼴}[HSK 4,5]
  \definition{v.}{ecoar; responder; responder a; responder às chamadas, saudações, perguntas, etc. de outras pessoas | conceder; cumprir | adequar; adaptar; responder a | lidar com; enfrentar; abordar | tornar-se realidade; ser cumprido}
\end{EntryWithPhonetic}

\begin{EntryWithPhonetic}{应当}{ying1 dang1}{7,6}{⼴,⼹}[HSK 3]
  \definition{v.}{dever}[学生们应当努力学习。===Os alunos devem se esforçar nos estudos.]
\end{EntryWithPhonetic}

\begin{EntryWithPhonetic}{应该}{ying1gai1}{7,8}{⼴,⾔}[HSK 2]
  \definition{v.}{deveria; deve ser assim | deveria; acho que deve ser esse o caso}
\end{EntryWithPhonetic}

%%%%%%%%%% 英 %%%%%%%%%%
\subsection*{英}\addcontentsline{loh}{figure}{英 \dpy{ying1}}

\begin{EntryWithPhonetic}{英}{ying1}{8}{⾋}
  \definition*{s.}{Reino Unido, abreviação de 英国 | Sobrenome: Ying}
  \definition{s.}{flor | herói; pessoa excepcional | uma pessoa de talento ou sabedoria extraordinários}
  \seealsoref{英国}{ying1guo2}
\end{EntryWithPhonetic}

\begin{EntryWithPhonetic}{英国}{ying1guo2}{8,8}{⾋,⼞}
  \definition*{s.}{Reino Unido; Grã-Bretanha; Inglaterra}
\end{EntryWithPhonetic}

\begin{EntryWithPhonetic}{英国人}{ying1guo2ren2}{8,8,2}{⾋,⼞,⼈}
  \definition{s.}{inglês | pessoa ou povo do Reino Unido}
\end{EntryWithPhonetic}

\begin{EntryWithPhonetic}{英明}{ying1ming2}{8,8}{⾋,⽇}
  \definition{adj.}{sábio; brilhante; excelente e sábio}
\end{EntryWithPhonetic}

\begin{EntryWithPhonetic}{英文}{ying1 wen2}{8,4}{⾋,⽂}[HSK 2]
  \definition{s.}{inglês, língua inglesa; a forma escrita do inglês}
\end{EntryWithPhonetic}

\begin{EntryWithPhonetic}{英雄}{ying1xiong2}{8,12}{⾋,⾫}[HSK 6]
  \definition{adj.}{heróico}
  \definition[名,个,位]{s.}{herói; uma pessoa cujas habilidades e coragem superam as das pessoas comuns | herói; aqueles que não têm medo das dificuldades, dos perigos ou da morte e que lutam bravamente pelos interesses do povo, mesmo ao custo das suas próprias vidas}
\end{EntryWithPhonetic}

\begin{EntryWithPhonetic}{英勇}{ying1yong3}{8,9}{⾋,⼒}[HSK 4]
  \definition{adj.}{heroico; valente; bravo; corajoso; extraordinariamente corajoso}
\end{EntryWithPhonetic}

\begin{EntryWithPhonetic}{英语}{ying1 yu3}{8,9}{⾋,⾔}[HSK 2]
  \definition{s.}{inglês, língua inglesa}
\end{EntryWithPhonetic}

%%%%%%%%%% 樱 %%%%%%%%%%
\subsection*{樱}\addcontentsline{loh}{figure}{樱 \dpy{ying1}}

\begin{EntryWithPhonetic}{樱}{ying1}{15}{⽊}
  \definition[个,棵,朵]{s.}{cereja | cerejeira oriental; flores de cerejeira}
\end{EntryWithPhonetic}

\begin{EntryWithPhonetic}{樱桃}{ying1tao2}{15,10}{⽊,⽊}
  \definition{s.}{cereja}
\end{EntryWithPhonetic}

%%%%%%%%%% 鹦 %%%%%%%%%%
\subsection*{鹦}\addcontentsline{loh}{figure}{鹦 \dpy{ying1}}

\begin{EntryWithPhonetic}{鹦}{ying1}{16}{⿃}
  \definition[只]{s.}{papagaio}
\end{EntryWithPhonetic}

\begin{EntryWithPhonetic}{鹦鹉}{ying1wu3}{16,13}{⿃,⿃}
  \definition{s.}{papagaio (ave)}
\end{EntryWithPhonetic}

%%%%%%%%%% 迎 %%%%%%%%%%
\subsection*{迎}\addcontentsline{loh}{figure}{迎 \dpy{ying2}}

\begin{EntryWithPhonetic}{迎}{ying2}{7}{⾡}
  \definition{v.}{ir ao encontro; cumprimentar; acolher; receber | mover-se em direção a; encontrar-se cara a cara}
\end{EntryWithPhonetic}

\begin{EntryWithPhonetic}{迎接}{ying2jie1}{7,11}{⾡,⼿}[HSK 3]
  \definition{v.}{conhecer; cumprimentar; felicitar; dar as boas-vindas | cumprimentar; felicitar; dar as boas-vindas; preparar-se; aguardar a chegada de um determinado momento ou evento}
\end{EntryWithPhonetic}

\begin{EntryWithPhonetic}{迎来}{ying2 lai2}{7,7}{⾡,⽊}[HSK 6]
  \definition{v.}{dar boas-vindas; cumprimentar | introduzir}
\end{EntryWithPhonetic}

%%%%%%%%%% 营 %%%%%%%%%%
\subsection*{营}\addcontentsline{loh}{figure}{营 \dpy{ying2}}

\begin{EntryWithPhonetic}{营}{ying2}{11}{⾋}
  \definition*{s.}{Sobrenome: Ying}
  \definition{s.}{acampamento; quartel; onde o exército está estacionado | batalhão; unidades militares}
  \definition{v.}{procurar | operar; executar; gerenciar}
\end{EntryWithPhonetic}

\begin{EntryWithPhonetic}{营养}{ying2yang3}{11,9}{⾋,⼋}[HSK 3]
  \definition[种]{s.}{nutrição; alimentação; a função do organismo de absorver as substâncias necessárias do meio externo para manter atividades vitais, como crescimento e desenvolvimento | nutrição; alimentação; ato ou processo de fornecer nutrição}
\end{EntryWithPhonetic}

\begin{EntryWithPhonetic}{营业}{ying2ye4}{11,5}{⾋,⼀}[HSK 4]
  \definition{v.}{fazer negócios; estar aberto para negócios}
\end{EntryWithPhonetic}

%%%%%%%%%% 赢 %%%%%%%%%%
\subsection*{赢}\addcontentsline{loh}{figure}{赢 \dpy{ying2}}

\begin{EntryWithPhonetic}{赢}{ying2}{17}{⾙}[HSK 3]
  \definition{v.}{vencer; derrotar | ganhar; lucrar}
\end{EntryWithPhonetic}

\begin{EntryWithPhonetic}{赢得}{ying2 de2}{17,11}{⾙,⼻}[HSK 4]
  \definition{v.}{ganhar; obter; conquistar; assegurar; garantir}
\end{EntryWithPhonetic}

%%%%%%%%%% 影 %%%%%%%%%%
\subsection*{影}\addcontentsline{loh}{figure}{影 \dpy{ying3}}

\begin{EntryWithPhonetic}{影}{ying3}{15}{⼺}
  \definition*{s.}{Sobrenome: Ying}
  \definition{s.}{sombra | reflexão; imagem | traço; sinal; impressão vaga | fotografia; imagem | filme | jogo de sombras; pantomima de sombra}
  \definition{v.}{(dialeto) esconder; ocultar | copiar; rastrear | fotocopiar}
\end{EntryWithPhonetic}

\begin{EntryWithPhonetic}{影迷}{ying3 mi2}{15,9}{⼺,⾡}[HSK 6]
  \definition[个,名,位]{s.}{fã de cinema; entusiasta de cinema; pessoas viciadas em assistir filmes}
\end{EntryWithPhonetic}

\begin{EntryWithPhonetic}{影片}{ying3 pian4}{15,4}{⼺,⽚}[HSK 2]
  \definition[部,盘,盒,卷]{s.}{filme; imagem | filme; película usada para reproduzir filmes}
\end{EntryWithPhonetic}

\begin{EntryWithPhonetic}{影视}{ying3 shi4}{15,8}{⼺,⾒}[HSK 3]
  \definition{s.}{cinema e televisão combinados; denominação conjunta para cinema e TV}
\end{EntryWithPhonetic}

\begin{EntryWithPhonetic}{影响}{ying3xiang3}{15,9}{⼺,⼝}[HSK 2]
  \definition{s.}{efeito; influência; efeitos sobre pessoas ou coisas}
  \definition{v.}{afetar; influenciar; influência sobre os pensamentos ou ações dos outros}
\end{EntryWithPhonetic}

\begin{EntryWithPhonetic}{影响力}{ying3 xiang3 li4}{15,9,2}{⼺,⼝,⼒}[HSK 6]
  \definition{s.}{impacto | influência}
\end{EntryWithPhonetic}

\begin{EntryWithPhonetic}{影像}{ying3xiang4}{15,13}{⼺,⼈}
  \definition{s.}{imagem}
\end{EntryWithPhonetic}

\begin{EntryWithPhonetic}{影星}{ying3 xing1}{15,9}{⼺,⽇}[HSK 6]
  \definition{s.}{estrela de cinema}
\end{EntryWithPhonetic}

\begin{EntryWithPhonetic}{影子}{ying3zi5}{15,3}{⼺,⼦}[HSK 4]
  \definition[个,片]{s.}{sombra; imagem projetada por um objeto, etc., que bloqueia a luz | reflexão; reflexo; imagem de um objeto, etc., conforme aparece em um refletor, como um espelho, uma superfície de água, etc. | sinal; vestígio; vaga impressão}
\end{EntryWithPhonetic}

%%%%%%%%%% 应 %%%%%%%%%%
\subsection*{应}\addcontentsline{loh}{figure}{应 \dpy{ying4}}

\begin{EntryWithPhonetic}{应对}{ying4 dui4}{7,5}{⼴,⼨}[HSK 6]
  \definition{v.}{reagir; responder; lidar com; dar uma resposta; tomar medidas e contramedidas para lidar com a situação}
\end{EntryWithPhonetic}

\begin{EntryWithPhonetic}{应急}{ying4 ji2}{7,9}{⼴,⼼}[HSK 6]
  \definition{v.}{atender a uma necessidade urgente (emergência, contingência, etc.)}
\end{EntryWithPhonetic}

\begin{EntryWithPhonetic}{应用}{ying4yong4}{7,5}{⼴,⽤}[HSK 3]
  \definition{adj.}{aplicado (na vida ou na produção); usado diretamente na vida ou na produção}
  \definition{v.}{usar; aplicar}
\end{EntryWithPhonetic}

\begin{EntryWithPhonetic}{应用程序}{ying4yong4 cheng2xu4}{7,5,12,7}{⼴,⽤,⽲,⼴}
  \definition{s.}{programa aplicativo; principais categorias de \emph{software}}
\end{EntryWithPhonetic}

\begin{EntryWithPhonetic*}{应用程序编程接口}{ying4yong4 cheng2xu4 bian1cheng2 jie1kou3}{7,5,12,7,12,12,11,3}{⼴,⽤,⽲,⼴,⽷,⽲,⼿,⼝}
  \definition{s.}{API (\emph{application programming interface})}
  \seealsoref{应用程序接口}{ying4yong4 cheng2xu4 jie1kou3}
\end{EntryWithPhonetic*}

\begin{EntryWithPhonetic}{应用程序接口}{ying4yong4 cheng2xu4 jie1kou3}{7,5,12,7,11,3}{⼴,⽤,⽲,⼴,⼿,⼝}
  \definition{s.}{API (\emph{application programming interface})}
  \seealsoref{应用程序编程接口}{ying4yong4 cheng2xu4 bian1cheng2 jie1kou3}
\end{EntryWithPhonetic}

%%%%%%%%%% 硬 %%%%%%%%%%
\subsection*{硬}\addcontentsline{loh}{figure}{硬 \dpy{ying4}}

\begin{EntryWithPhonetic}{硬}{ying4}{12}{⽯}[HSK 4,5]
  \definition{adj.}{duro; rígido; resistente;  objeto resistente e não se deforma facilmente quando submetido a forças externas (em oposição a 软) | firme; forte; resistente; obstinado; (vontade, atitude, etc.) inabalável, forte e poderoso | capaz (pessoa); boa (qualidade) | rígido; severo; sem flexibilidade | duro; rígido; rigoroso; imutável}
  \definition{adv.}{conseguir fazer algo com dificuldade; indica fazer algo à força, independentemente das circunstâncias}
  \seealsoref{软}{ruan3}
\end{EntryWithPhonetic}

\begin{EntryWithPhonetic}{硬件}{ying4jian4}{12,6}{⽯,⼈}[HSK 5]
  \definition[种]{s.}{\emph{hardware}; nome genérico dado aos vários elementos, componentes e dispositivos que constituem um computador | máquina, materiais; equipamento; referência a máquinas, equipamentos, materiais físicos, etc., utilizados nos processos de produção, pesquisa científica, gestão, etc.}
\end{EntryWithPhonetic}

%%%%%%%%%% 拥 %%%%%%%%%%
\subsection*{拥}\addcontentsline{loh}{figure}{拥 \dpy{yong1}}

\begin{EntryWithPhonetic}{拥}{yong1}{8}{⼿}
  \definition{v.}{segurar nos braços; abraçar | reunir em volta; envolver em volta | aglomerar-se; enxamear | para apoiar | (literário) ter; possuir}
\end{EntryWithPhonetic}

\begin{EntryWithPhonetic}{拥抱}{yong1bao4}{8,8}{⼿,⼿}[HSK 5]
  \definition[个,次]{s.}{abraço}
  \definition{v.}{abraçar; segurar em seus braços; abraçar para demonstrar afeto}
\end{EntryWithPhonetic}

\begin{EntryWithPhonetic}{拥有}{yong1you3}{8,6}{⼿,⽉}[HSK 5]
  \definition{v.}{possuir; deter; ter (grande quantidade de terras, população, bens, etc.)}
\end{EntryWithPhonetic}

%%%%%%%%%% 永 %%%%%%%%%%
\subsection*{永}\addcontentsline{loh}{figure}{永 \dpy{yong3}}

\begin{EntryWithPhonetic}{永}{yong3}{5}{⽔}
  \definition*{s.}{Sobrenome: Yong}
  \definition{adj.}{sempre; para sempre; perpetuamente}
  \definition{adv.}{para sempre; significa um tempo muito longo sem fim, o que equivale a 永远}
  \seealsoref{永远}{yong3yuan3}
\end{EntryWithPhonetic}

\begin{EntryWithPhonetic}{永不}{yong3bu4}{5,4}{⽔,⼀}
  \definition{adv.}{nunca}
\end{EntryWithPhonetic}

\begin{EntryWithPhonetic}{永远}{yong3yuan3}{5,7}{⽔,⾡}[HSK 2]
  \definition{adv.}{sempre; para sempre; Indica um longo período de tempo sem fim}
  \definition{s.}{eternidade; um futuro que nunca acaba}
\end{EntryWithPhonetic}

%%%%%%%%%% 泳 %%%%%%%%%%
\subsection*{泳}\addcontentsline{loh}{figure}{泳 \dpy{yong3}}

\begin{EntryWithPhonetic}{泳}{yong3}{8}{⽔}
  \definition{v.}{nadar}
\end{EntryWithPhonetic}

\begin{EntryWithPhonetic}{泳池}{yong3chi2}{8,6}{⽔,⽔}
  \definition{s.}{piscina}
  \seealsoref{游泳池}{you2 yong3 chi2}
  \seealsoref{游泳馆}{you2yong3guan3}
\end{EntryWithPhonetic}

\begin{EntryWithPhonetic}{泳衣}{yong3yi1}{8,6}{⽔,⾐}
  \definition{s.}{roupa de banho | maiô}
  \seealsoref{游泳衣}{you2yong3yi1}
\end{EntryWithPhonetic}

%%%%%%%%%% 勇 %%%%%%%%%%
\subsection*{勇}\addcontentsline{loh}{figure}{勇 \dpy{yong3}}

\begin{EntryWithPhonetic}{勇}{yong3}{9}{⼒}
  \definition*{s.}{Sobrenome: Yong}
  \definition{adj.}{bravo; valente; corajoso}
  \definition{s.}{recrutas temporários em tempos de guerra na Dinastia Qing}
\end{EntryWithPhonetic}

\begin{EntryWithPhonetic}{勇敢}{yong3gan3}{9,11}{⼒,⽁}[HSK 4]
  \definition{adj.}{bravo; valente; galante; corajoso}
\end{EntryWithPhonetic}

\begin{EntryWithPhonetic}{勇气}{yong3qi4}{9,4}{⼒,⽓}[HSK 4]
  \definition[种,股]{s.}{coragem; arrojo; nervos; coragem para agir sem medo}
\end{EntryWithPhonetic}

\begin{EntryWithPhonetic}{勇士}{yong3shi4}{9,3}{⼒,⼠}
  \definition{s.}{um guerreiro | uma pessoa corajosa}
\end{EntryWithPhonetic}

%%%%%%%%%% 用 %%%%%%%%%%
\subsection*{用}\addcontentsline{loh}{figure}{用 \dpy{yong4}}

\begin{EntryWithPhonetic}{用}{yong4}{5}{⽤}[HSK 1][Kangxi 101]
  \definition*{s.}{Sobrenome: Yong}
  \definition{conj.}{portanto; por isso; assim sendo; razões para a introdução, equivalentes a 因}
  \definition{prep.}{com; ação de introduzir ferramentas, meios, etc. utilizados ou empregados}
  \definition{s.}{despesas; gastos; custos | uso; utilidade; eficácia}
  \definition{v.}{usar; aplicar; empregar | necessitar (normalmente na forma negativa) | respeitosamente: comer; beber}
  \seealsoref{因}{yin1}
\end{EntryWithPhonetic}

\begin{EntryWithPhonetic}{用不着}{yong4 bu4 zhao2}{5,4,11}{⽤,⼀,⽬}[HSK 5]
  \definition{v.}{não precisar; não ter utilidade para; não haver necessidade de}
\end{EntryWithPhonetic}

\begin{EntryWithPhonetic}{用处}{yong4 chu3}{5,5}{⽤,⼡}[HSK 6]
  \definition[个]{s.}{uso; usabilidade; utilidade}
\end{EntryWithPhonetic}

\begin{EntryWithPhonetic}{用得着}{yong4 de5 zhao2}{5,11,11}{⽤,⼻,⽬}[HSK 6]
  \definition{adj.}{útil; necessário}
  \definition{v.}{precisar; achar algo útil | ter necessidade de;  ser necessário; valer a pena}
\end{EntryWithPhonetic}

\begin{EntryWithPhonetic}{用法}{yong4 fa3}{5,8}{⽤,⽔}[HSK 6]
  \definition[种,个]{s.}{uso; emprego; a maneira de usar}
\end{EntryWithPhonetic}

\begin{EntryWithPhonetic}{用户}{yong4hu4}{5,4}{⽤,⼾}[HSK 5]
  \definition[个,位,名]{s.}{usuário; consumidor; entidades e indivíduos que utilizam determinados equipamentos públicos ou bens de consumo}
\end{EntryWithPhonetic}

\begin{EntryWithPhonetic}{用来}{yong4 lai2}{5,7}{⽤,⽊}[HSK 5]
  \definition{v.}{ser usado para; depender (dele) ou usar (ele) para atingir algum objetivo}
\end{EntryWithPhonetic}

\begin{EntryWithPhonetic}{用料}{yong4liao4}{5,10}{⽤,⽃}
  \definition{s.}{ingredientes | materiais}
\end{EntryWithPhonetic}

\begin{EntryWithPhonetic}{用品}{yong4 pin3}{5,9}{⽤,⼝}[HSK 6]
  \definition[批,件,种]{s.}{suprimentos; artigos para uso; itens para usar}
\end{EntryWithPhonetic}

\begin{EntryWithPhonetic}{用途}{yong4tu2}{5,10}{⽤,⾡}[HSK 4]
  \definition[个,种]{s.}{uso; aplicação; aspectos ou escopo da aplicação}
\end{EntryWithPhonetic}

\begin{EntryWithPhonetic}{用心}{yong4 xin1}{5,4}{⽤,⼼}[HSK 6]
  \definition{adj.}{diligente; atento; com atenção concentrada}
  \definition{s.}{motivo; intenção; o verdadeiro propósito ou razão para fazer algo}
\end{EntryWithPhonetic}

\begin{EntryWithPhonetic}{用于}{yong4 yu2}{5,3}{⽤,⼆}[HSK 5]
  \definition{v.}{usar para; ser usado para; usar em}
\end{EntryWithPhonetic}

%%%%%%%%%% 优 %%%%%%%%%%
\subsection*{优}\addcontentsline{loh}{figure}{优 \dpy{you1}}

\begin{EntryWithPhonetic}{优}{you1}{6}{⼈}
  \definition{adj.}{excelente; bom; excepcional | amplo; abundante}
  \definition{s.}{Arcaico: ator ou atriz}
  \definition{s.}{Sobrenome: You}
  \definition{v.}{dar tratamento preferencial}
\end{EntryWithPhonetic}

\begin{EntryWithPhonetic}{优等}{you1deng3}{6,12}{⼈,⽵}
  \definition{adj.}{excelente | de primeira linha | alta classe | da mais alta ordem, superior}
\end{EntryWithPhonetic}

\begin{EntryWithPhonetic}{优点}{you1dian3}{6,9}{⼈,⽕}[HSK 3]
  \definition[个,项,种,些]{s.}{mérito; virtude; ponto forte; vantagem (em oposição a 缺点)}
  \seealsoref{缺点}{que1dian3}
\end{EntryWithPhonetic}

\begin{EntryWithPhonetic}{优格}{you1ge2}{6,10}{⼈,⽊}
  \definition{s.}{iogurte}
\end{EntryWithPhonetic}

\begin{EntryWithPhonetic}{优厚}{you1hou4}{6,9}{⼈,⼚}
  \definition{adj.}{generoso}
\end{EntryWithPhonetic}

\begin{EntryWithPhonetic}{优惠}{you1hui4}{6,12}{⼈,⼼}[HSK 5]
  \definition{adj.}{especial; pechincha; reduzido; com desconto | favorável; preferencial; melhores condições ou tratamento do que o normal, permitindo que as pessoas obtenham mais benefícios}
\end{EntryWithPhonetic}

\begin{EntryWithPhonetic}{优良}{you1 liang2}{6,7}{⼈,⾉}[HSK 4]
  \definition{adj.}{ótimo; bom; excelente; (variedade, qualidade, desempenho, estilo, etc.) muito bom}
\end{EntryWithPhonetic}

\begin{EntryWithPhonetic}{优伶}{you1ling2}{6,7}{⼈,⼈}
  \definition{s.}{ator}
\end{EntryWithPhonetic}

\begin{EntryWithPhonetic}{优美}{you1mei3}{6,9}{⼈,⽺}[HSK 4]
  \definition{adj.}{fino; elegante; gracioso; bonito}
\end{EntryWithPhonetic}

\begin{EntryWithPhonetic}{优盘}{you1pan2}{6,11}{⼈,⽫}
  \definition{s.}{unidade de memória USB}
  \seealsoref{闪存盘}{shan3cun2pan2}
\end{EntryWithPhonetic}

\begin{EntryWithPhonetic}{优势}{you1shi4}{6,8}{⼈,⼒}[HSK 3]
  \definition[种,个]{s.}{vantagem; superioridade; preponderância; posição dominante; uma situação favorável que permite superar o adversário}
\end{EntryWithPhonetic}

\begin{EntryWithPhonetic}{优先}{you1xian1}{6,6}{⼈,⼉}[HSK 5]
  \definition{adj.}{anterior; sênior; subjacente}
  \definition{v.}{ter prioridade; ter precedência; colocar-se à frente de outras pessoas ou assuntos}
\end{EntryWithPhonetic}

\begin{EntryWithPhonetic}{优秀}{you1xiu4}{6,7}{⼈,⽲}[HSK 4]
  \definition{adj.}{esplêndido; excelente; extraordinário; excepcional; notável; descreve moral, qualidades, realizações, aprendizado, etc. muito bons.}
\end{EntryWithPhonetic}

\begin{EntryWithPhonetic}{优选}{you1xuan3}{6,9}{⼈,⾡}
  \definition{v.}{otimizar}
\end{EntryWithPhonetic}

\begin{EntryWithPhonetic}{优于}{you1yu2}{6,3}{⼈,⼆}
  \definition{v.}{superar}
\end{EntryWithPhonetic}

\begin{EntryWithPhonetic}{优裕}{you1yu4}{6,12}{⼈,⾐}
  \definition{adj.}{abundante | bastante}
  \definition{s.}{abundância}
\end{EntryWithPhonetic}

\begin{EntryWithPhonetic}{优质}{you1 zhi4}{6,8}{⼈,⾙}[HSK 6]
  \definition{adj.}{excelente qualidade; alta qualidade; qualidade superior; alto grau}
\end{EntryWithPhonetic}

%%%%%%%%%% 忧 %%%%%%%%%%
\subsection*{忧}\addcontentsline{loh}{figure}{忧 \dpy{you1}}

\begin{EntryWithPhonetic}{忧}{you1}{7}{⼼}
  \definition{s.}{tristeza; ansiedade; preocupação; cuidado; coisas que causam tristeza}
  \definition{v.}{preocupar-se; estar preocupado; estar ansioso; estar triste}
\end{EntryWithPhonetic}

\begin{EntryWithPhonetic}{忧郁}{you1yu4}{7,8}{⼼,⾢}
  \definition{adj.}{deprimido | melancólico | desanimado}
  \definition{s.}{depressão | melancolia}
\end{EntryWithPhonetic}

%%%%%%%%%% 幽 %%%%%%%%%%
\subsection*{幽}\addcontentsline{loh}{figure}{幽 \dpy{you1}}

\begin{EntryWithPhonetic}{幽}{you1}{9}{⼳}
  \definition*{s.}{Sobrenome: You}
  \definition{adj.}{profundo e remoto; isolado; escuro | secreto; escondido; oculto; não público | quieto; tranquilo; sereno | do mundo inferior}
  \definition{s.}{mundo inferior}
\end{EntryWithPhonetic}

\begin{EntryWithPhonetic}{幽默}{you1mo4}{9,16}{⼳,⿊}[HSK 5]
  \definition{adj.}{humorístico; interessante ou engraçado, mas com um significado profundo}
  \definition{s.}{humor; lado engraçado; graça; características, temperamento, palavras ou comportamentos interessantes, engraçados ou significativos}
\end{EntryWithPhonetic}

%%%%%%%%%% 尤 %%%%%%%%%%
\subsection*{尤}\addcontentsline{loh}{figure}{尤 \dpy{you2}}

\begin{EntryWithPhonetic}{尤}{you2}{4}{⼪}
  \definition*{s.}{Sobrenome: You}
  \definition{adj.}{excelente; peculiar; notável}
  \definition{adv.}{particularmente; especialmente}
  \definition{s.}{falha; erro | irregularidade}
  \definition{v.}{ter rancor contra; culpar}
\end{EntryWithPhonetic}

\begin{EntryWithPhonetic}{尤其}{you2qi2}{4,8}{⼪,⼋}[HSK 5]
  \definition{adv.}{especialmente; particularmente; indica um grau mais avançado, equivalente a 更加}
  \seealsoref{更加}{geng4 jia1}
\end{EntryWithPhonetic}

%%%%%%%%%% 由 %%%%%%%%%%
\subsection*{由}\addcontentsline{loh}{figure}{由 \dpy{you2}}

\begin{EntryWithPhonetic}{由}{you2}{5}{⽥}[HSK 3]
  \definition*{s.}{Sobrenome: You}
  \definition{prep.}{por causa de; devido a | por; indica que algo deve ser feito por alguém | indica confiança em; indica dependência em | de; indica o ponto de partida | por; através de}
  \definition[个]{s.}{causa; razão; motivo}
  \definition{v.}{atravessar; passar por; seguir o caminho de | obedecer; seguir}
\end{EntryWithPhonetic}

\begin{EntryWithPhonetic}{由此}{you2 ci3}{5,6}{⽥,⽌}[HSK 5]
  \definition{adv.}{assim; por meio disto; disto; daí; por causa disto; portanto; daqui; de agora em diante}
\end{EntryWithPhonetic}

\begin{EntryWithPhonetic}{由于}{you2yu2}{5,3}{⽥,⼆}[HSK 3]
  \definition{conj.}{porque; uma vez que; visto que;  usado no início da frase anterior, indica a razão, e a frase seguinte indica o resultado}
  \definition{prep.}{devido a; graças a; por causa de; em virtude de; como resultado de; introduzir a causa da ocorrência de eventos, ações, etc.}
\end{EntryWithPhonetic}

%%%%%%%%%% 犹 %%%%%%%%%%
\subsection*{犹}\addcontentsline{loh}{figure}{犹 \dpy{you2}}

\begin{EntryWithPhonetic}{犹}{you2}{7}{⽝}
  \definition*{s.}{Sobrenome: You}
  \definition{adv.}{ainda | assim como; exatamente como; como se}
  \definition{v.}{ser exatamente como; ser como}
\end{EntryWithPhonetic}

\begin{EntryWithPhonetic}{犹豫}{you2yu4}{7,15}{⽝,⾗}[HSK 5]
  \definition{adj.}{hesitante; indeciso, incapaz de decidir ou agir}
  \definition{v.}{hesitar; ser indeciso}
\end{EntryWithPhonetic}

%%%%%%%%%% 邮 %%%%%%%%%%
\subsection*{邮}\addcontentsline{loh}{figure}{邮 \dpy{you2}}

\begin{EntryWithPhonetic}{邮}{you2}{7}{⾢}
  \definition*{s.}{Sobrenome: You}
  \definition{s.}{postal; correio; refere-se a serviços postais | agência dos correios}
  \definition{v.}{postar; enviar pelo correio}
\end{EntryWithPhonetic}

\begin{EntryWithPhonetic}{邮包}{you2bao1}{7,5}{⾢,⼓}
  \definition{s.}{encomenda postal}
\end{EntryWithPhonetic}

\begin{EntryWithPhonetic}{邮递}{you2di4}{7,10}{⾢,⾡}
  \definition{v.}{enviar por correio}
\end{EntryWithPhonetic}

\begin{EntryWithPhonetic}{邮电}{you2dian4}{7,5}{⾢,⽥}
  \definition*{s.}{Correios e Telecomunicações}
\end{EntryWithPhonetic}

\begin{EntryWithPhonetic}{邮费}{you2fei4}{7,9}{⾢,⾙}
  \definition{s.}{postagem}
  \definition{v.}{postar}
\end{EntryWithPhonetic}

\begin{EntryWithPhonetic}{邮件}{you2 jian4}{7,6}{⾢,⼈}[HSK 3]
  \definition[封,份,个,条]{s.}{correspondência; correio; assunto postal; termo que se refere a cartas, encomendas, etc., recebidos, transportados e entregues pelos correios | \emph{e-mail}; refere-se a e-mails, informações recebidas e enviadas através de caixas de correio eletrônico na \emph{Internet}, etc.}
\end{EntryWithPhonetic}

\begin{EntryWithPhonetic}{邮局}{you2ju2}{7,7}{⾢,⼫}[HSK 4]
  \definition[家,个]{s.}{correio; agência dos correios; organizações que lidam com serviços postais}
\end{EntryWithPhonetic}

\begin{EntryWithPhonetic}{邮迷}{you2mi2}{7,9}{⾢,⾡}
  \definition{s.}{filatelista | colecionador de selos}
\end{EntryWithPhonetic}

\begin{EntryWithPhonetic}{邮票}{you2 piao4}{7,11}{⾢,⽰}[HSK 3]
  \definition[枚,张,套,版]{s.}{selo; selo postal; comprovante vendido pelos correios, usado para colar nas correspondências para indicar que o porte foi pago}
\end{EntryWithPhonetic}

\begin{EntryWithPhonetic}{邮市}{you2shi4}{7,5}{⾢,⼱}
  \definition{s.}{mercado postal}
\end{EntryWithPhonetic}

\begin{EntryWithPhonetic}{邮箱}{you2 xiang1}{7,15}{⾢,⾋}[HSK 3]
  \definition{s.}{caixa de correio | \emph{mailbox}; refere-se ao endereço de \emph{e-mail}}
\end{EntryWithPhonetic}

\begin{EntryWithPhonetic}{邮资}{you2zi1}{7,10}{⾢,⾙}
  \definition{s.}{postagem}
\end{EntryWithPhonetic}

%%%%%%%%%% 油 %%%%%%%%%%
\subsection*{油}\addcontentsline{loh}{figure}{油 \dpy{you2}}

\begin{EntryWithPhonetic}{油}{you2}{8}{⽔}[HSK 2]
  \definition*{s.}{Sobrenome: You}
  \definition{adj.}{oleoso; gorduroso}
  \definition[瓶,滴,层]{s.}{óleo; gordura; graxa; petróleo}
  \definition{v.}{aplicar óleo de tungue, verniz ou tinta | estar manchado ou sujo com óleo ou graxa | aplicar óleo de tungue ou tinta}
\end{EntryWithPhonetic}

%%%%%%%%%% 游 %%%%%%%%%%
\subsection*{游}\addcontentsline{loh}{figure}{游 \dpy{you2}}

\begin{EntryWithPhonetic}{游}{you2}{12}{⽔}[HSK 3]
  \definition*{s.}{Sobrenome: You}
  \definition{adj.}{itinerante; não fixo; que se move frequentemente}
  \definition{s.}{parte de um rio; uma seção do rio; trecho; bacia; curso}
  \definition{v.}{nadar | vagar por aí; caminhar; viajar; fazer turismo | associar com (comunicação) | vagar; passear; andar tranquilamente por todos os lugares}
\end{EntryWithPhonetic}

\begin{EntryWithPhonetic}{游客}{you2 ke4}{12,9}{⽔,⼧}[HSK 2]
  \definition[个,位,名,群]{s.}{visitante; turista | (jogo online) jogador convidado}
\end{EntryWithPhonetic}

\begin{EntryWithPhonetic}{游人}{you2 ren2}{12,2}{⽔,⼈}[HSK 6]
  \definition[个,名,位,批]{s.}{visitante (de um parque, etc.); turista}
\end{EntryWithPhonetic}

\begin{EntryWithPhonetic}{游艇}{you2ting3}{12,12}{⽔,⾈}
  \definition[只]{s.}{barcaça | iate}
\end{EntryWithPhonetic}

\begin{EntryWithPhonetic}{游玩}{you2 wan2}{12,8}{⽔,⽟}[HSK 6]
  \definition{v.}{brincar; jogar; divertir-se | passear; vagar; fazer turismo}
\end{EntryWithPhonetic}

\begin{EntryWithPhonetic}{游戏}{you2xi4}{12,6}{⽔,⼽}[HSK 3]
  \definition[场]{s.}{jogo; recreação; atividades recreativas, como esconde-esconde, adivinhar charadas, etc.; certas atividades esportivas não competitivas; jogos recreativos}
  \definition{v.}{jogar; fazer atividades divertidas e agradáveis, sozinho ou com outras pessoas}
\end{EntryWithPhonetic}

\begin{EntryWithPhonetic}{游戏机}{you2 xi4 ji1}{12,6,6}{⽔,⼽,⽊}[HSK 6]
  \definition[台]{s.}{jogador de videogame | console | videogame}
\end{EntryWithPhonetic}

\begin{EntryWithPhonetic}{游行}{you2 xing2}{12,6}{⽔,⾏}[HSK 6]
  \definition{s.}{desfilar; marchar; manifestar-se; marchar em grupos nas ruas para celebrar, comemorar, manifestar-se, etc.}
\end{EntryWithPhonetic}

\begin{EntryWithPhonetic}{游泳}{you2/yong3}{12,8}{⽔,⽔}[HSK 3]
  \definition[次]{s.}{natação; refere-se ao esporte ou atividade de natação}
  \definition{v.+compl.}{nadar; pessoas ou animais nadando na água}
\end{EntryWithPhonetic}

\begin{EntryWithPhonetic}{游泳池}{you2 yong3 chi2}{12,8,6}{⽔,⽔,⽔}[HSK 5]
  \definition[场,个]{s.}{piscina; piscinas artificiais para natação, divididas em duas categorias: internas e externas}
  \seealsoref{泳池}{yong3chi2}
  \seealsoref{游泳馆}{you2yong3guan3}
\end{EntryWithPhonetic}

\begin{EntryWithPhonetic}{游泳馆}{you2yong3guan3}{12,8,11}{⽔,⽔,⾷}
  \definition{s.}{natatório; piscina coberta; edifícios esportivos usados ​​principalmente para esportes aquáticos, como natação, mergulho e polo aquático}
  \seealsoref{泳池}{yong3chi2}
  \seealsoref{游泳池}{you2 yong3 chi2}
\end{EntryWithPhonetic}

\begin{EntryWithPhonetic}{游泳镜}{you2yong3jing4}{12,8,16}{⽔,⽔,⾦}
  \definition{s.}{óculos de natação}
\end{EntryWithPhonetic}

\begin{EntryWithPhonetic}{游泳衣}{you2yong3yi1}{12,8,6}{⽔,⽔,⾐}
  \definition{s.}{roupa de banho}
  \seealsoref{泳衣}{yong3yi1}
\end{EntryWithPhonetic}

%%%%%%%%%% 友 %%%%%%%%%%
\subsection*{友}\addcontentsline{loh}{figure}{友 \dpy{you3}}

\begin{EntryWithPhonetic}{友}{you3}{4}{⼜}
  \definition{adj.}{amigável; bom relacionamento; próximo | de relações amigáveis}
  \definition{s.}{amigo; pessoas intimamente relacionadas}
\end{EntryWithPhonetic}

\begin{EntryWithPhonetic}{友好}{you3hao3}{4,6}{⼜,⼥}[HSK 2]
  \definition{adj.}{amigável; amistoso; muito próximo, relacionamento muito bom; como bons amigos}
  \definition{s.}{amigo próximo, íntimo; em ocasiões formais referem-se a bons amigos}
\end{EntryWithPhonetic}

\begin{EntryWithPhonetic}{友谊}{you3yi4}{4,10}{⼜,⾔}[HSK 5]
  \definition[段,份]{s.}{amizade; amizade entre amigos}
\end{EntryWithPhonetic}

%%%%%%%%%% 有 %%%%%%%%%%
\subsection*{有}\addcontentsline{loh}{figure}{有 \dpy{you3}}

\begin{EntryWithPhonetic}{有}{you3}{6}{⽉}[HSK 1]
  \definition*{s.}{Sobrenome: You}
  \definition{pref.}{usado antes do nome de certas dinastias ou etnias}
  \definition{v.}{ter; possuir; indica posse ou propriedade | existe; há; indica que certas coisas existem em certos lugares | fazer uma estimativa ou uma comparação; expressar estimativa ou comparação | indicar ação; indica que algo aconteceu ou ocorreu | usado antes de substantivos abstratos, indica quantidade ou grandeza | em termos gerais, semelhante a 某; refere-se de maneira geral a algo semelhante | usado antes de pessoa, hora e lugar, indica a existência parcial | usado antes de certos verbos para formar uma expressão idiomática, indicando cortesia, polidez}
  \seealsoref{某}{mou3}
\end{EntryWithPhonetic}

\begin{EntryWithPhonetic}{有道理}{you3dao4li5}{6,12,11}{⽉,⾡,⽟}
  \definition{v.}{fazer sentido; ser bem fundamentado; haver verdade em}
\end{EntryWithPhonetic}

\begin{EntryWithPhonetic}{有的}{you3 de5}{6,8}{⽉,⽩}[HSK 1]
  \definition{pron.}{algum, alguns}
\end{EntryWithPhonetic}

\begin{EntryWithPhonetic}{有的时候}{you3 de5 shi2 hou4}{6,8,7,10}{⽉,⽩,⽇,⼈}
  \definition{adv.}{às vezes; ocasionalmente}
  \seealsoref{有时}{you3 shi2}
  \seealsoref{有时候}{you3 shi2 hou5}
\end{EntryWithPhonetic}

\begin{EntryWithPhonetic}{有的是}{you3 de5 shi4}{6,8,9}{⽉,⽩,⽇}[HSK 3]
  \definition{expr.}{ter em abundância; não faltar; enfatizar que há muitos}
\end{EntryWithPhonetic}

\begin{EntryWithPhonetic}{有点儿}{you3 dian3r5}{6,9,2}{⽉,⽕,⼉}
  \definition{adv.}{um pouco; indica um grau inferior, equivalente a 稍微 (usado principalmente para coisas que são insatisfatórias)}
  \definition{v.}{há um pouco; tem (ou ser de) algum; existem alguns}
  \seealsoref{稍微}{shao1wei1}
  \seealsoref{有(一)点儿}{you3 yi4 dian3r5}
\end{EntryWithPhonetic}

\begin{EntryWithPhonetic}{有毒}{you3 du2}{6,9}{⽉,⽏}[HSK 5]
  \definition{adj.}{venenoso; tóxico; nocivo; geralmente é usada para descrever as propriedades nocivas à saúde de produtos químicos, plantas ou animais.}
\end{EntryWithPhonetic}

\begin{EntryWithPhonetic}{有关}{you3 guan1}{6,6}{⽉,⼋}[HSK 6]
  \definition{prep.}{no caminho de; sobre}
  \definition{v.}{preocupar-se com; relacionar-se com; ter algo a ver com; existir algum tipo de relacionamento}
\end{EntryWithPhonetic}

\begin{EntryWithPhonetic}{有害}{you3 hai4}{6,10}{⽉,⼧}[HSK 5]
  \definition{adj.}{prejudicial; nocivo; danoso; que pode causar danos ou prejuízos a algo}
\end{EntryWithPhonetic}

\begin{EntryWithPhonetic}{有劲儿}{you3 jin4er5}{6,7,2}{⽉,⼒,⼉}[HSK 4]
  \definition{adj.}{interessante; divertido; estimulante | energético}
  \definition{v.}{ter força}
\end{EntryWithPhonetic}

\begin{EntryWithPhonetic}{有空儿}{you3 kong4r5}{6,8,2}{⽉,⽳,⼉}[HSK 2]
  \definition{v.}{estar livre; ter tempo livre}
\end{EntryWithPhonetic}

\begin{EntryWithPhonetic}{有劳}{you3lao2}{6,7}{⽉,⼒}
  \definition{v.}{posso incomodá-lo; desculpe incomodá-lo | (educado) obrigado pelo seu trabalho (usado ao pedir um favor ou após ter recebido um)}
\end{EntryWithPhonetic}

\begin{EntryWithPhonetic}{有力}{you3 li4}{6,2}{⽉,⼒}[HSK 5]
  \definition{adj.}{forte; vigoroso; poderoso; energético}
\end{EntryWithPhonetic}

\begin{EntryWithPhonetic}{有利}{you3li4}{6,7}{⽉,⼑}[HSK 3]
  \definition{adj.}{benéfico; favorável; vantajoso}
\end{EntryWithPhonetic}

\begin{EntryWithPhonetic}{有利于}{you3 li4 yu2}{6,7,3}{⽉,⼑,⼆}[HSK 5]
  \definition{prep.}{disponível; é benéfico para alguém ou alguma coisa e pode ajudar e promovê-lo}
\end{EntryWithPhonetic}

\begin{EntryWithPhonetic}{有没有}{you3 mei2 you3}{6,7,6}{⽉,⽔,⽉}[HSK 6]
  \definition{adv.}{Você tem\dots?; Você já\dots? ; Existe algum\dots?}
\end{EntryWithPhonetic}

\begin{EntryWithPhonetic}{有名}{you3ming2}{6,6}{⽉,⼝}[HSK 1]
  \definition{adj.}{conhecido; famoso; célebre; nome conhecido por todos}
\end{EntryWithPhonetic}

\begin{EntryWithPhonetic}{有名无实}{you3ming2wu2shi2}{6,6,4,8}{⽉,⼝,⽆,⼧}
  \definition{v.}{(literal) tem um nome, mas não tem realidade | existe apenas no nome}
\end{EntryWithPhonetic}

\begin{EntryWithPhonetic}{有趣}{you3qu4}{6,15}{⽉,⾛}[HSK 4]
  \definition{adj.}{interessante; fascinante; divertido; excitante; estimulante}
\end{EntryWithPhonetic}

\begin{EntryWithPhonetic}{有人}{you3 ren2}{6,2}{⽉,⼈}[HSK 2]
  \definition{adj.}{ocupado (como no banheiro)}
  \definition{pron.}{qualquer um; alguém}
  \definition[所]{s.}{pessoas}
  \definition{v.}{ter alguém ali}
\end{EntryWithPhonetic}

\begin{EntryWithPhonetic}{有时}{you3 shi2}{6,7}{⽉,⽇}[HSK 1]
  \definition{expr.}{às vezes; ocasionalmente; de vez em quando}
  \seealsoref{有的时候}{you3 de5 shi2 hou4}
  \seealsoref{有时候}{you3 shi2 hou5}
\end{EntryWithPhonetic}

\begin{EntryWithPhonetic}{有时候}{you3 shi2 hou5}{6,7,10}{⽉,⽇,⼈}[HSK 1]
  \definition{adv.}{às vezes; indica um momento incerto, mas não frequente}
  \seealsoref{有的时候}{you3 de5 shi2 hou4}
  \seealsoref{有时}{you3 shi2}
\end{EntryWithPhonetic}

\begin{EntryWithPhonetic}{有时…有时…}{you3shi2 you3shi2}{6,7,6,7}{⽉,⽇,⽉,⽇}
  \definition{adv.}{às vezes\dots às vezes\dots}
\end{EntryWithPhonetic}

\begin{EntryWithPhonetic}{有事}{you3 shi4}{6,8}{⽉,⼅}[HSK 6]
  \definition{v.}{estar ocupado; estar envolvido | ter algo acontecendo; sofrer um acidente; se meter em encrenca | (com 心里) ter algo em mente; estar ansioso; preocupar-se | ter um emprego; estar empregado}[看他这几天愁眉苦脸的, 心里一定有事。===Vendo como ele parece triste ultimamente, deve haver algo em sua mente.]
  \seealsoref{心里}{xin1 li3}
\end{EntryWithPhonetic}

\begin{EntryWithPhonetic}{有限}{you3 xian4}{6,8}{⽉,⾩}[HSK 4]
  \definition{adj.}{finito; limitado; restrito | indica baixo grau; indica pouco número; número baixo; nível baixo}
\end{EntryWithPhonetic}

\begin{EntryWithPhonetic}{有限公司}{you3xian4gong1si1}{6,8,4,5}{⽉,⾩,⼋,⼝}
  \definition{s.}{companhia limitada | corporação}
\end{EntryWithPhonetic}

\begin{EntryWithPhonetic}{有效}{you3 xiao4}{6,10}{⽉,⽁}[HSK 3]
  \definition{adj.}{válido; eficiente; eficaz; capaz de alcançar os objetivos esperados}
\end{EntryWithPhonetic}

\begin{EntryWithPhonetic}{有些}{you3 xie1}{6,8}{⽉,⼆}[HSK 1]
  \definition{adv.}{um pouco; bastante; ligeiramente}
  \definition{pron.}{uma parte; alguns}
  \definition{v.}{usado para indicar que há alguns, mas não muitos;}
  \seealsoref{有(一)些}{you3 (yi4) xie1}
\end{EntryWithPhonetic}

\begin{EntryWithPhonetic}{有(一)点儿}{you3 yi4 dian3r5}{6,1,9,2}{⽉,⼀,⽕,⼉}[HSK 2]
  \definition{adv.}{um pouco (有点儿 + {s.} ou {v. mental})}
  \seealsoref{有点儿}{you3 dian3r5}
\end{EntryWithPhonetic}

\begin{EntryWithPhonetic}{有(一)些}{you3 (yi4) xie1}{6,1,8}{⽉,⼀,⼆}[HSK 1]
  \definition{adv.}{em vez disso; em vez de; de certa forma}
  \definition{pron.}{de certa forma}
  \seealsoref{有些}{you3 xie1}
\end{EntryWithPhonetic}

\begin{EntryWithPhonetic}{有意思}{you3 yi4 si5}{6,13,9}{⽉,⼼,⼼}[HSK 2]
  \definition{adj.}{significativo; significativo e intrigante | interessante; agradável}
  \definition{v.}{ter interesse por; ser atraído sexualmente}
\end{EntryWithPhonetic}

\begin{EntryWithPhonetic}{有用}{you3yong4}{6,5}{⽉,⽤}[HSK 1]
  \definition{adj.}{útil; prático; funcional}
\end{EntryWithPhonetic}

\begin{EntryWithPhonetic}{有着}{you3 zhe5}{6,11}{⽉,⽬}[HSK 5]
  \definition{v.}{ter; possuir; haver; existir}
\end{EntryWithPhonetic}

%%%%%%%%%% 又 %%%%%%%%%%
\subsection*{又}\addcontentsline{loh}{figure}{又 \dpy{you4}}

\begin{EntryWithPhonetic}{又}{you4}{2}{⼜}[HSK 2][Kangxi 29]
  \definition{adv.}{indica repetição ou continuação | indica que várias situações ou propriedades existem simultaneamente | indica um nível mais profundo de significado | indica adicionar zero a números inteiros | indica duas coisas contraditórias | indica um ponto de virada, significando 可是 | usado em frases negativas ou perguntas retóricas para fortalecer o tom | além disso; indica informações adicionais ou suplementares}
  \seealsoref{可是}{ke3shi4}
\end{EntryWithPhonetic}

\begin{EntryWithPhonetic}{又称}{you4cheng1}{2,10}{⼜,⽲}
  \definition{s.}{também conhecido como}
\end{EntryWithPhonetic}

\begin{EntryWithPhonetic}{又及}{you4ji2}{2,3}{⼜,⼃}
  \definition{s.}{P.S., \emph{postscript}}
\end{EntryWithPhonetic}

\begin{EntryWithPhonetic}{又名}{you4ming2}{2,6}{⼜,⼝}
  \definition{s.}{também conhecido como | nome alternativo}
\end{EntryWithPhonetic}

\begin{EntryWithPhonetic}{又一次}{you4yi2ci4}{2,1,6}{⼜,⼀,⽋}
  \definition{adv.}{outra vez | mais uma vez | de novo}
\end{EntryWithPhonetic}

\begin{EntryWithPhonetic}{又…又…}{you4 you4}{2,2}{⼜,⼜}
  \definition{conj.}{\dots e\dots; tanto\dots como\dots; as duas palavras usadas depois de 又 não devem ter nenhuma conotação de contraste, ambas devem ser positivas ou negativas}[这件毛衣挺不错的,\underline{又}便宜\underline{又}漂亮。===Este suéter é muito bom, barato e bonito.]
\end{EntryWithPhonetic}

%%%%%%%%%% 右 %%%%%%%%%%
\subsection*{右}\addcontentsline{loh}{figure}{右 \dpy{you4}}

\begin{EntryWithPhonetic}{右}{you4}{5}{⼝}[HSK 1]
  \definition*{s.}{Sobrenome: You}
  \definition{adj.}{conservador; reacionário}
  \definition{s.}{a direita; o lado direito | oeste; na antiguidade, referia-se especificamente à direção oeste (com base na orientação para o sul) | o lado direito como o lado de precedência; posição ou nível mais elevado (os antigos costumavam considerar a direita como mais respeitável)}
  \definition{v.}{favorecer; apoiar; reverenciar}
\end{EntryWithPhonetic}

\begin{EntryWithPhonetic}{右边}{you4bian5}{5,5}{⼝,⾡}[HSK 1]
  \definition{s.}{a direita; o lado direito; do lado direito}
\end{EntryWithPhonetic}

\begin{EntryWithPhonetic}{右侧}{you4ce4}{5,8}{⼝,⼈}
  \definition{s.}{lateral direita | lado direito}
\end{EntryWithPhonetic}

\begin{EntryWithPhonetic}{右面}{you4mian4}{5,9}{⼝,⾯}
  \definition{s.}{lado direito}
\end{EntryWithPhonetic}

\begin{EntryWithPhonetic}{右倾}{you4qing1}{5,10}{⼝,⼈}
  \definition{adj.}{conservador | reacionário}
\end{EntryWithPhonetic}

\begin{EntryWithPhonetic}{右手}{you4shou3}{5,4}{⼝,⼿}
  \definition{s.}{mão direita | lado direito}
\end{EntryWithPhonetic}

\begin{EntryWithPhonetic}{右袒}{you4tan3}{5,10}{⼝,⾐}
  \definition{v.}{ser tendencioso | ser parcial | favorecer um lado | tomar partido}
\end{EntryWithPhonetic}

\begin{EntryWithPhonetic}{右转}{you4zhuan3}{5,8}{⼝,⾞}
  \definition{v.}{virar à direita}
\end{EntryWithPhonetic}

%%%%%%%%%% 幼 %%%%%%%%%%
\subsection*{幼}\addcontentsline{loh}{figure}{幼 \dpy{you4}}

\begin{EntryWithPhonetic}{幼}{you4}{5}{⼳}
  \definition{adj.}{jovem; menor de idade (oposto a 老)}
  \definition{s.}{crianças; os jovens}
  \seealsoref{老}{lao3}
\end{EntryWithPhonetic}

\begin{EntryWithPhonetic}{幼儿园}{you4'er2yuan2}{5,2,7}{⼳,⼉,⼞}[HSK 4]
  \definition[家,所]{s.}{jardim de infância; escola maternal; escola infantil; instituição para a educação de crianças pequenas}
\end{EntryWithPhonetic}

%%%%%%%%%% 诱 %%%%%%%%%%
\subsection*{诱}\addcontentsline{loh}{figure}{诱 \dpy{you4}}

\begin{EntryWithPhonetic}{诱}{you4}{9}{⾔}
  \definition{v.}{guiar; liderar; dirigir | atrair; seduzir; aliciar | induzir; causar; resultar de; levar a}
\end{EntryWithPhonetic}

\begin{EntryWithPhonetic}{诱人}{you4ren2}{9,2}{⾔,⼈}
  \definition{adj.}{atraente | cativante}
\end{EntryWithPhonetic}

%%%%%%%%%% 淤 %%%%%%%%%%
\subsection*{淤}\addcontentsline{loh}{figure}{淤 \dpy{yu1}}

\begin{EntryWithPhonetic}{淤}{yu1}{11}{⽔}
  \definition{adj.}{assoreado}
  \definition{s.}{lodo}
  \definition[出]{s.}{(medicina chinesa) estase de sangue}
  \definition{v.}{ficar assoreado; ficar sufocado com lodo | derramar; transbordar}
\end{EntryWithPhonetic}

\begin{EntryWithPhonetic}{淤泥}{yu1ni2}{11,8}{⽔,⽔}
  \definition[出]{s.}{gosma | limo | lodo}
\end{EntryWithPhonetic}

%%%%%%%%%% 于 %%%%%%%%%%
\subsection*{于}\addcontentsline{loh}{figure}{于 \dpy{yu2}}

\begin{EntryWithPhonetic}{于}{yu2}{3}{⼆}[HSK 6]
  \definition*{s.}{Sobrenome: Yu}
  \definition{prep.}{indica hora, lugar, alcance, etc. | indica a direção da ação | usado depois de um verbo para indicar dar, entregar, etc. | apresentar a relação do objeto ou entidade introduzida | indica o ponto de início ou de partida | indica comparação | indica passividade}
\end{EntryWithPhonetic}

\begin{EntryWithPhonetic}{于是}{yu2shi4}{3,9}{⼆,⽇}[HSK 4]
  \definition{conj.}{então; portanto; consequentemente; como resultado; indica que o último segue o primeiro e que o último é frequentemente causado pelo primeiro}
\end{EntryWithPhonetic}

%%%%%%%%%% 鱼 %%%%%%%%%%
\subsection*{鱼}\addcontentsline{loh}{figure}{鱼 \dpy{yu2}}

\begin{EntryWithPhonetic}{鱼}{yu2}{8}{⿂}[HSK 2][Kangxi 195]
  \definition*{s.}{Sobrenome: Yu}
  \definition[条,种,尾]{s.}{peixe; um vertebrado que vive na água; geralmente possui um corpo achatado lateralmente, fusiforme e com muitas escamas; nada com as nadadeiras e respira com as brânquias; sua temperatura corporal varia de acordo com a temperatura externa; existem muitas espécies, a maioria das quais comestíveis | carne de peixe; peixe (como alimento)}
\end{EntryWithPhonetic}

\begin{EntryWithPhonetic}{鱼船}{yu2chuan2}{8,11}{⿂,⾈}
  \definition{s.}{barco de pesca}
  \seealsoref{渔船}{yu2chuan2}
\end{EntryWithPhonetic}

\begin{EntryWithPhonetic}{鱼具}{yu2ju4}{8,8}{⿂,⼋}
  \variantof{渔具}
\end{EntryWithPhonetic}

\begin{EntryWithPhonetic}{鱼片}{yu2pian4}{8,4}{⿂,⽚}
  \definition{s.}{fatia de peixe | filé de peixe}
\end{EntryWithPhonetic}

\begin{EntryWithPhonetic}{鱼网}{yu2wang3}{8,6}{⿂,⽹}
  \variantof{渔网}
\end{EntryWithPhonetic}

\begin{EntryWithPhonetic}{鱼香}{yu2xiang1}{8,9}{⿂,⾹}
  \definition{s.}{um tempero da culinária chinesa que normalmente contém alho, cebolinha, gengibre, açúcar, sal, pimenta, etc.; embora 鱼香 signifique literalmente ``fragrância de peixe'', não contém frutos do mar}
\end{EntryWithPhonetic}

\begin{EntryWithPhonetic}{鱼香肉丝}{yu2xiang1rou4si1}{8,9,6,5}{⿂,⾹,⾁,⼀}
  \definition{s.}{Prato: tiras de carne de porco salteadas com molho picante}
  \seealsoref{鱼香}{yu2xiang1}
\end{EntryWithPhonetic}

\begin{EntryWithPhonetic}{鱼汛}{yu2xun4}{8,6}{⿂,⽔}
  \variantof{渔汛}
\end{EntryWithPhonetic}

%%%%%%%%%% 舁 %%%%%%%%%%
\subsection*{舁}\addcontentsline{loh}{figure}{舁 \dpy{yu2}}

\begin{EntryWithPhonetic}{舁}{yu2}{9}{⾅}
  \definition{v.}{levantar; elevar | aumentar}
\end{EntryWithPhonetic}

%%%%%%%%%% 娱 %%%%%%%%%%
\subsection*{娱}\addcontentsline{loh}{figure}{娱 \dpy{yu2}}

\begin{EntryWithPhonetic}{娱}{yu2}{10}{⼥}
  \definition{s.}{alegria; prazer; diversão; felicidade}
  \definition{v.}{dar prazer a; divertir; fazer feliz}
\end{EntryWithPhonetic}

\begin{EntryWithPhonetic}{娱乐}{yu2le4}{10,5}{⼥,⼃}[HSK 6]
  \definition[项]{s.}{entretenimento; diversão; recreação; passa-tempo; atividades recreativas, prazerosas e divertidas}
  \definition{v.}{recrear; divertir; distrair; entreter; passar o tempo}
\end{EntryWithPhonetic}

%%%%%%%%%% 渔 %%%%%%%%%%
\subsection*{渔}\addcontentsline{loh}{figure}{渔 \dpy{yu2}}

\begin{EntryWithPhonetic}{渔}{yu2}{11}{⽔}
  \definition[条]{s.}{pescador}
  \definition{v.}{pescar}
\end{EntryWithPhonetic}

\begin{EntryWithPhonetic}{渔场}{yu2chang3}{11,6}{⽔,⼟}
  \definition{s.}{área de pesca}
\end{EntryWithPhonetic}

\begin{EntryWithPhonetic}{渔船}{yu2chuan2}{11,11}{⽔,⾈}
  \definition[条]{s.}{barco de pesca}
  \seealsoref{鱼船}{yu2chuan2}
\end{EntryWithPhonetic}

\begin{EntryWithPhonetic}{渔船队}{yu2chuan2 dui4}{11,11,4}{⽔,⾈,⾩}
  \definition{s.}{frota pesqueira}
\end{EntryWithPhonetic}

\begin{EntryWithPhonetic}{渔夫}{yu2fu1}{11,4}{⽔,⼤}
  \definition{s.}{pescador}
\end{EntryWithPhonetic}

\begin{EntryWithPhonetic}{渔具}{yu2ju4}{11,8}{⽔,⼋}
  \definition{s.}{equipamento de pesca}
\end{EntryWithPhonetic}

\begin{EntryWithPhonetic}{渔捞}{yu2lao1}{11,10}{⽔,⼿}
  \definition{s.}{pesca (como atividade comercial); pesca em massa; operações de pesca em grande escala}
\end{EntryWithPhonetic}

\begin{EntryWithPhonetic}{渔猎}{yu2lie4}{11,11}{⽔,⽝}
  \definition{s.}{pesca e caça}
  \definition{v.}{saquear | pilhar}
\end{EntryWithPhonetic}

\begin{EntryWithPhonetic}{渔笼}{yu2long2}{11,11}{⽔,⽵}
  \definition{s.}{gaiola de pesca | armadilha de pesca}
\end{EntryWithPhonetic}

\begin{EntryWithPhonetic}{渔轮}{yu2lun2}{11,8}{⽔,⾞}
  \definition{s.}{navio de pesca}
\end{EntryWithPhonetic}

\begin{EntryWithPhonetic}{渔民}{yu2min2}{11,5}{⽔,⽒}
  \definition{s.}{pescadores | povo pescador}
\end{EntryWithPhonetic}

\begin{EntryWithPhonetic}{渔网}{yu2wang3}{11,6}{⽔,⽹}
  \definition{s.}{rede de pesca | tresmalho}
\end{EntryWithPhonetic}

\begin{EntryWithPhonetic}{渔汛}{yu2xun4}{11,6}{⽔,⽔}
  \definition{s.}{temporada de pesca}
\end{EntryWithPhonetic}

%%%%%%%%%% 愉 %%%%%%%%%%
\subsection*{愉}\addcontentsline{loh}{figure}{愉 \dpy{yu2}}

\begin{EntryWithPhonetic}{愉}{yu2}{12}{⼼}
  \definition{adj.}{satisfeito; feliz; alegre}
\end{EntryWithPhonetic}

\begin{EntryWithPhonetic}{愉快}{yu2kuai4}{12,7}{⼼,⼼}[HSK 6]
  \definition{adj.}{feliz; alegre; de bom humor, muito feliz}
\end{EntryWithPhonetic}

%%%%%%%%%% 瑜 %%%%%%%%%%
\subsection*{瑜}\addcontentsline{loh}{figure}{瑜 \dpy{yu2}}

\begin{EntryWithPhonetic}{瑜}{yu2}{13}{⽟}
  \definition{s.}{(arcaico) jade fino; gema | (literário) brilho das gemas — virtudes; pontos positivos | excelência}
\end{EntryWithPhonetic}

\begin{EntryWithPhonetic}{瑜伽}{yu2jia1}{13,7}{⽟,⼈}
  \definition*{s.}{Ioga}
\end{EntryWithPhonetic}

\begin{EntryWithPhonetic}{瑜珈}{yu2jia1}{13,9}{⽟,⽟}
  \variantof{瑜伽}
\end{EntryWithPhonetic}

%%%%%%%%%% 虞 %%%%%%%%%%
\subsection*{虞}\addcontentsline{loh}{figure}{虞 \dpy{yu2}}

\begin{EntryWithPhonetic}{虞}{yu2}{13}{⾌}
  \definition*{s.}{Reino Yu, uma dinastia lendária fundada por Shun 舜 |Yu (um estado da Dinastia Zhou 周) | Sobrenome: Yu}
  \definition{s.}{Literário: suposição; previsão | Literário: ansiedade; preocupação}
  \definition{v.}{Literário: enganar; trapacear; fazer de bobo}
  \seealsoref{舜}{shun4}
  \seealsoref{周}{zhou1}
\end{EntryWithPhonetic}

\begin{EntryWithPhonetic}{虞世南}{yu2 shi4'nan2}{13,5,9}{⾌,⼀,⼗}
  \definition*{s.}{Yu Shi'nan (558-638), político dos períodos Sui e Tang inicial, poeta e calígrafo, um dos Quatro Grandes Calígrafos do início da Dinastia Tang, 唐初四大家}
  \seealsoref{唐初四大家}{tang2 chu1 si4 da4jia1}
\end{EntryWithPhonetic}

%%%%%%%%%% 与 %%%%%%%%%%
\subsection*{与}\addcontentsline{loh}{figure}{与 \dpy{yu3}}

\begin{EntryWithPhonetic}{与}{yu3}{3}{⼀}[HSK 6]
  \definition*{s.}{Sobrenome: Yu}
  \definition{conj.}{e; junto com}
  \definition{prep.}{com}
  \definition{v.}{dar; oferecer; conceder | conviver com; estar em bons termos com; socializar; ser amigável | ajudar; apoiar; patrocinar | Literário: esperar}
  \seeref{yu4}
\end{EntryWithPhonetic}

\begin{EntryWithPhonetic}{与其}{yu3qi2}{3,8}{⼀,⼋}
  \definition{conj.}{mais do que}
\end{EntryWithPhonetic}

\begin{EntryWithPhonetic}{与其…不如…}{yu3qi2 bu4ru2}{3,8,4,6}{⼀,⼋,⼀,⼥}
  \definition{conj.}{ao invés de\dots melhor que\dots}
\end{EntryWithPhonetic}

\begin{EntryWithPhonetic}{与其…宁可…}{yu3qi2 ning4ke3}{3,8,5,5}{⼀,⼋,⼧,⼝}
  \definition{conj.}{ao invés de\dots melhor que\dots}
\end{EntryWithPhonetic}

%%%%%%%%%% 宇 %%%%%%%%%%
\subsection*{宇}\addcontentsline{loh}{figure}{宇 \dpy{yu3}}

\begin{EntryWithPhonetic}{宇}{yu3}{6}{⼧}
  \definition*{s.}{Sobrenome: Yu}
  \definition[座,栋]{s.}{beirais; calha; casa | espaço; universo; mundo | postura; porte}
\end{EntryWithPhonetic}

\begin{EntryWithPhonetic}{宇航员}{yu3 hang2 yuan2}{6,10,7}{⼧,⾈,⼝}[HSK 6]
  \definition[位,名,个,些]{s.}{astronauta; cosmonauta;}
\end{EntryWithPhonetic}

\begin{EntryWithPhonetic}{宇宙}{yu3zhou4}{6,8}{⼧,⼧}
  \definition{s.}{universo | cosmos}
\end{EntryWithPhonetic}

%%%%%%%%%% 羽 %%%%%%%%%%
\subsection*{羽}\addcontentsline{loh}{figure}{羽 \dpy{yu3}}

\begin{EntryWithPhonetic}{羽}{yu3}{6}{⽻}[Kangxi 124]
  \definition*{s.}{Sobrenome: Yu}
  \definition{s.}{pena; pluma | asas (de pássaros ou insetos) | uma nota da antiga escala chinesa de cinco tons, correspondente a 6 na notação musical numerada}
\end{EntryWithPhonetic}

\begin{EntryWithPhonetic}{羽冠}{yu3guan1}{6,9}{⽻,⼍}
  \definition{s.}{crista emplumada (de pássaro)}
\end{EntryWithPhonetic}

\begin{EntryWithPhonetic}{羽林}{yu3lin2}{6,8}{⽻,⽊}
  \definition{s.}{escolta armada}
\end{EntryWithPhonetic}

\begin{EntryWithPhonetic}{羽流}{yu3liu2}{6,10}{⽻,⽔}
  \definition{s.}{pluma}
\end{EntryWithPhonetic}

\begin{EntryWithPhonetic}{羽毛}{yu3mao2}{6,4}{⽻,⽑}
  \definition{s.}{pena | plumagem | pluma}
\end{EntryWithPhonetic}

\begin{EntryWithPhonetic}{羽毛笔}{yu3mao2bi3}{6,4,10}{⽻,⽑,⽵}
  \definition{s.}{caneta de pena}
\end{EntryWithPhonetic}

\begin{EntryWithPhonetic}{羽毛球}{yu3mao2qiu2}{6,4,11}{⽻,⽑,⽟}[HSK 5]
  \definition[只,个]{s.}{\emph{badminton}; esporte com bola, as regras e equipamentos são bastante semelhantes ao tênis | peteca}
\end{EntryWithPhonetic}

\begin{EntryWithPhonetic}{羽绒服}{yu3rong2fu2}{6,9,8}{⽻,⽷,⽉}[HSK 5]
  \definition[件,个]{s.}{jaqueta de plumas; peça de vestuário com enchimento de plumas; casaco cujo interior é preenchido com penas de pato ou ganso}
\end{EntryWithPhonetic}

%%%%%%%%%% 雨 %%%%%%%%%%
\subsection*{雨}\addcontentsline{loh}{figure}{雨 \dpy{yu3}}

\begin{EntryWithPhonetic}{雨}{yu3}{8}{⾬}[HSK 1][Kangxi 173]
  \definition*{s.}{Sobrenome: Yu}
  \definition[场,阵,滴]{s.}{chuva; água que cai das nuvens para o solo}
  \seeref{yu4}
\end{EntryWithPhonetic}

\begin{EntryWithPhonetic}{雨伞}{yu3san3}{8,6}{⾬,⼈}
  \definition[把]{s.}{guarda-chuva}
\end{EntryWithPhonetic}

\begin{EntryWithPhonetic}{雨蚀}{yu3shi2}{8,9}{⾬,⾷}
  \definition{s.}{erosão da chuva}
\end{EntryWithPhonetic}

\begin{EntryWithPhonetic}{雨水}{yu3 shui3}{8,4}{⾬,⽔}[HSK 5]
  \definition{s.}{água da chuva; precipitação; chuva; água proveniente da chuva}
\end{EntryWithPhonetic}

\begin{EntryWithPhonetic}{雨靴}{yu3xue1}{8,13}{⾬,⾰}
  \definition[双]{s.}{botas de chuva}
\end{EntryWithPhonetic}

\begin{EntryWithPhonetic}{雨衣}{yu3 yi1}{8,6}{⾬,⾐}[HSK 6]
  \definition[件,个]{s.}{capa de chuva; jaqueta impermeável; roupas impermeáveis}
\end{EntryWithPhonetic}

%%%%%%%%%% 语 %%%%%%%%%%
\subsection*{语}\addcontentsline{loh}{figure}{语 \dpy{yu3}}

\begin{EntryWithPhonetic}{语}{yu3}{9}{⾔}
  \definition{s.}{língua; linguagem | dito; provérbio; refere-se especialmente a coloquialismos, provérbios, expressões idiomáticas ou palavras de livros antigos | sinal; meio não linguístico de comunicar ideias ; ações ou sinais que substituem palavras para expressar significado | palavras; expressão; refere-se a uma palavra, frase ou sentença}
  \definition{v.}{dizer; falar | (pássaros, insetos, etc.) gorjear; pipilar}
  \seeref{yu4}
\end{EntryWithPhonetic}

\begin{EntryWithPhonetic}{语调}{yu3diao4}{9,10}{⾔,⾔}
  \definition[个]{s.}{entonação}
\end{EntryWithPhonetic}

\begin{EntryWithPhonetic}{语法}{yu3fa3}{9,8}{⾔,⽔}[HSK 4]
  \definition[个]{s.}{gramática; maneira como o idioma é estruturado, incluindo a formação e as variações de palavras, a organização de frases e sentenças | estudo da gramática; estudo das regras de estrutura linguística}
\end{EntryWithPhonetic}

\begin{EntryWithPhonetic}{语法术语}{yu3fa3 shu4yu3}{9,8,5,9}{⾔,⽔,⽊,⾔}
  \definition{s.}{termo gramatical}
\end{EntryWithPhonetic}

\begin{EntryWithPhonetic}{语气}{yu3qi4}{9,4}{⾔,⽓}
  \definition[个]{s.}{maneira de falar | tom}
\end{EntryWithPhonetic}

\begin{EntryWithPhonetic}{语言}{yu3yan2}{9,7}{⾔,⾔}[HSK 2]
  \definition[种,门]{s.}{linguagem; é uma ferramenta exclusiva dos humanos para expressar ideias e comunicar pensamentos; é um fenômeno social especial e consiste em um sistema específico de pronúncia, vocabulário e gramática | linguagem falada}
\end{EntryWithPhonetic}

\begin{EntryWithPhonetic}{语言实验室}{yu3yan2shi2yan4shi4}{9,7,8,10,9}{⾔,⾔,⼧,⾺,⼧}
  \definition{s.}{laboratório de línguas}
\end{EntryWithPhonetic}

\begin{EntryWithPhonetic}{语音}{yu3 yin1}{9,9}{⾔,⾳}[HSK 4]
  \definition{s.}{voz; pronúncia; sons da fala; som de alguém falando | pronúncia; som do idioma}
\end{EntryWithPhonetic}

%%%%%%%%%% 与 %%%%%%%%%%
\subsection*{与}\addcontentsline{loh}{figure}{与 \dpy{yu4}}

\begin{EntryWithPhonetic}{与}{yu4}{3}{⼀}
  \definition{v.}{participar de; tomar parte em}
  \seeref{yu3}
\end{EntryWithPhonetic}

%%%%%%%%%% 玉 %%%%%%%%%%
\subsection*{玉}\addcontentsline{loh}{figure}{玉 \dpy{yu4}}

\begin{EntryWithPhonetic}{玉}{yu4}{5}{⽟}[HSK 4][Kangxi 96]
  \definition*{s.}{Sobrenome: Yu}
  \definition{adj.}{(pessoa, especialmente uma mulher) pura; justa; bonita; bela | cristalino, branco e belo como o jade | (vida) rica; luxuosa}
  \definition{pron.}{seu; um termo de respeito, usado para honrar o corpo, as ações ou as coisas associadas à outra pessoa}
  \definition[块,种]{s.}{jade}
\end{EntryWithPhonetic}

\begin{EntryWithPhonetic}{玉帛}{yu4bo2}{5,8}{⽟,⼱}
  \definition{s.}{objetos de jade e tecidos de seda, usados ​​como presentes de estado (oposto a 干戈); paz e harmonia}[化干戈为玉帛。===Transforme guerra em paz.]
  \seealsoref{干戈}{gan1ge1}
\end{EntryWithPhonetic}

\begin{EntryWithPhonetic}{玉米}{yu4mi3}{5,6}{⽟,⽶}[HSK 4]
  \definition[根,粒,棵,片]{s.}{milho}
\end{EntryWithPhonetic}

\begin{EntryWithPhonetic}{玉米饼}{yu4mi3bing3}{5,6,9}{⽟,⽶,⾷}
  \definition{s.}{tortilha mexicana | bolo de milho}
\end{EntryWithPhonetic}

\begin{EntryWithPhonetic}{玉米粉}{yu4mi3fen3}{5,6,10}{⽟,⽶,⽶}
  \definition{s.}{amido de milho | farinha de milho}
\end{EntryWithPhonetic}

\begin{EntryWithPhonetic}{玉米糕}{yu4mi3gao1}{5,6,16}{⽟,⽶,⽶}
  \definition{s.}{bolo de milho | polenta}
\end{EntryWithPhonetic}

\begin{EntryWithPhonetic}{玉米花}{yu4mi3hua1}{5,6,7}{⽟,⽶,⾋}
  \definition{s.}{pipoca}
\end{EntryWithPhonetic}

\begin{EntryWithPhonetic}{玉米面}{yu4mi3mian4}{5,6,9}{⽟,⽶,⾯}
  \definition{s.}{fubá | farinha de milho}
\end{EntryWithPhonetic}

\begin{EntryWithPhonetic}{玉米片}{yu4mi3pian4}{5,6,4}{⽟,⽶,⽚}
  \definition{s.}{flocos de milho | chips de tortilha}
\end{EntryWithPhonetic}

\begin{EntryWithPhonetic}{玉米糁}{yu4mi3 san3}{5,6,14}{⽟,⽶,⽶}
  \definition{s.}{grãos de milho}
\end{EntryWithPhonetic}

\begin{EntryWithPhonetic}{玉米笋}{yu4mi3 sun3}{5,6,10}{⽟,⽶,⽵}
  \definition{s.}{broto de milho}
\end{EntryWithPhonetic}

%%%%%%%%%% 芋 %%%%%%%%%%
\subsection*{芋}\addcontentsline{loh}{figure}{芋 \dpy{yu4}}

\begin{EntryWithPhonetic}{芋}{yu4}{6}{⾋}
  \definition*{s.}{Sobrenome: Yu}
  \definition{s.}{taro; erva perene | tubérculos; geralmente se refere a batatas, etc.}
\end{EntryWithPhonetic}

\begin{EntryWithPhonetic}{芋头}{yu4tou5}{6,5}{⾋,⼤}
  \definition{s.}{taro, similar ao inhame e batata doce}
\end{EntryWithPhonetic}

\begin{EntryWithPhonetic}{芋头色}{yu4tou5se4}{6,5,6}{⾋,⼤,⾊}
  \definition{s.}{cor lilás}
\end{EntryWithPhonetic}

%%%%%%%%%% 郁 %%%%%%%%%%
\subsection*{郁}\addcontentsline{loh}{figure}{郁 \dpy{yu4}}

\begin{EntryWithPhonetic}{郁}{yu4}{8}{⾢}
  \definition*{s.}{Sobrenome: Yu}
  \definition{adj.}{fortemente perfumado | luxuriante; exuberante | sombrio; deprimido}
\end{EntryWithPhonetic}

\begin{EntryWithPhonetic}{郁郁葱葱}{yu4yu4cong1cong1}{8,8,12,12}{⾢,⾢,⾋,⾋}
  \definition{adj.}{exuberante e verde}
  \definition{expr.}{verdejante e exuberante; uma profusão selvagem de vegetação; luxuriantemente verde; ela cresce mais verde e mais fresca}
\end{EntryWithPhonetic}

%%%%%%%%%% 雨 %%%%%%%%%%
\subsection*{雨}\addcontentsline{loh}{figure}{雨 \dpy{yu4}}

\begin{EntryWithPhonetic}{雨}{yu4}{8}{⾬}[Kangxi 173]
  \definition{v.}{cair (chuva, neve, etc.) | precipitar | chover | molhar}
  \seeref{yu3}
\end{EntryWithPhonetic}

%%%%%%%%%% 语 %%%%%%%%%%
\subsection*{语}\addcontentsline{loh}{figure}{语 \dpy{yu4}}

\begin{EntryWithPhonetic}{语}{yu4}{9}{⾔}
  \definition{v.}{contar; informar}
  \seeref{yu3}
\end{EntryWithPhonetic}

%%%%%%%%%% 预 %%%%%%%%%%
\subsection*{预}\addcontentsline{loh}{figure}{预 \dpy{yu4}}

\begin{EntryWithPhonetic}{预}{yu4}{10}{⾴}
  \definition{adv.}{antecipadamente}
  \definition{v.}{avançar | preparar}
\end{EntryWithPhonetic}

\begin{EntryWithPhonetic}{预报}{yu4bao4}{10,7}{⾴,⼿}[HSK 3]
  \definition[个,项]{s.}{boletim meteorológico; previsões meteorológicas antecipadas}
  \definition{v.}{prever (o tempo); relatar antes que algo aconteça, usado principalmente em relação ao clima, astronomia, desastres naturais, etc.}
\end{EntryWithPhonetic}

\begin{EntryWithPhonetic}{预备}{yu4 bei4}{10,8}{⾴,⼡}[HSK 5]
  \definition{v.}{preparar-se; ficar pronto}
\end{EntryWithPhonetic}

\begin{EntryWithPhonetic}{预测}{yu4 ce4}{10,9}{⾴,⽔}[HSK 4]
  \definition{v.}{prever; prognosticar; predizer}
\end{EntryWithPhonetic}

\begin{EntryWithPhonetic}{预订}{yu4ding4}{10,4}{⾴,⾔}[HSK 4]
  \definition{v.}{reservar; fazer uma reserva}
\end{EntryWithPhonetic}

\begin{EntryWithPhonetic}{预定}{yu4ding4}{10,8}{⾴,⼧}
  \definition{v.}{agendar com antecedência}
\end{EntryWithPhonetic}

\begin{EntryWithPhonetic}{预防}{yu4fang2}{10,6}{⾴,⾩}[HSK 3]
  \definition{v.}{prevenir; proteger-se contra; tomar precauções contra; preparar-se com antecedência para evitar que algo ruim aconteça}
\end{EntryWithPhonetic}

\begin{EntryWithPhonetic}{预付}{yu4fu4}{10,5}{⾴,⼈}
  \definition{s.}{pré-pago}
  \definition{v.}{pagar antecipadamente}
\end{EntryWithPhonetic}

\begin{EntryWithPhonetic}{预感}{yu4gan3}{10,13}{⾴,⼼}
  \definition{s.}{premonição}
  \definition{v.}{ter uma premonição}
\end{EntryWithPhonetic}

\begin{EntryWithPhonetic}{预购}{yu4gou4}{10,8}{⾴,⾙}
  \definition{s.}{compra antecipada}
  \definition{v.}{comprar antecipadamente}
\end{EntryWithPhonetic}

\begin{EntryWithPhonetic}{预计}{yu4 ji4}{10,4}{⾴,⾔}[HSK 3]
  \definition{v.}{estimar; calcular com antecedência}
\end{EntryWithPhonetic}

\begin{EntryWithPhonetic}{预见}{yu4jian4}{10,4}{⾴,⾒}
  \definition{s.}{previsão; intuição; vislumbre}
  \definition{v.}{prever}
\end{EntryWithPhonetic}

\begin{EntryWithPhonetic}{预警}{yu4jing3}{10,19}{⾴,⾔}
  \definition{s.}{aviso | aviso antecipado}
\end{EntryWithPhonetic}

\begin{EntryWithPhonetic}{预览}{yu4lan3}{10,9}{⾴,⾒}
  \definition{s.}{visualização}
  \definition{v.}{visualizar}
\end{EntryWithPhonetic}

\begin{EntryWithPhonetic}{预留}{yu4liu2}{10,10}{⾴,⽥}
  \definition{v.}{separar | reservar}
\end{EntryWithPhonetic}

\begin{EntryWithPhonetic}{预谋}{yu4mou2}{10,11}{⾴,⾔}
  \definition{adj.}{premeditado}
  \definition{v.}{planejar algo com antecedência (especialmente um crime)}
\end{EntryWithPhonetic}

\begin{EntryWithPhonetic}{预判}{yu4pan4}{10,7}{⾴,⼑}
  \definition{v.}{prever | antecipar}
\end{EntryWithPhonetic}

\begin{EntryWithPhonetic}{预配}{yu4pei4}{10,10}{⾴,⾣}
  \definition{s.}{pré-alocado | pré-cabeado}
  \definition{v.}{pré-alocar | pré-cabear}
\end{EntryWithPhonetic}

\begin{EntryWithPhonetic}{预期}{yu4qi1}{10,12}{⾴,⽉}[HSK 5]
  \definition{v.}{esperar; antecipar; imaginar; antecipar com expectativa}
\end{EntryWithPhonetic}

\begin{EntryWithPhonetic}{预提}{yu4ti2}{10,12}{⾴,⼿}
  \definition{s.}{retenção}
  \definition{v.}{reter (imposto)}
\end{EntryWithPhonetic}

\begin{EntryWithPhonetic}{预习}{yu4xi2}{10,3}{⾴,⼄}[HSK 3]
  \definition{v.}{pré-visualizar; preparar uma lição; estudar antecipadamente as matérias que serão abordadas nas aulas}
\end{EntryWithPhonetic}

\begin{EntryWithPhonetic}{预约}{yu4 yue1}{10,6}{⾴,⽷}[HSK 6]
  \definition[个]{s.}{reserva}
  \definition{v.}{reservar; agendar; marcar compromisso; marcar uma consulta}
\end{EntryWithPhonetic}

\begin{EntryWithPhonetic}{预祝}{yu4zhu4}{10,9}{⾴,⽰}
  \definition{v.}{parabenizar de antemão | oferecer os melhores votos para}
\end{EntryWithPhonetic}

%%%%%%%%%% 欲 %%%%%%%%%%
\subsection*{欲}\addcontentsline{loh}{figure}{欲 \dpy{yu4}}

\begin{EntryWithPhonetic}{欲}{yu4}{11}{⽋}
  \definition{adj.}{desejo | apetite | paixão | luxúria | ganância}
  \definition{v.}{desejar}
\end{EntryWithPhonetic}

%%%%%%%%%% 喻 %%%%%%%%%%
\subsection*{喻}\addcontentsline{loh}{figure}{喻 \dpy{yu4}}

\begin{EntryWithPhonetic}{喻}{yu4}{12}{⼝}
  \definition{s.}{analogia | símile | metáfora | alegoria}
  \definition{v.}{descrever algo como}
\end{EntryWithPhonetic}

%%%%%%%%%% 寓 %%%%%%%%%%
\subsection*{寓}\addcontentsline{loh}{figure}{寓 \dpy{yu4}}

\begin{EntryWithPhonetic}{寓}{yu4}{12}{⼧}
  \definition[座,间,栋]{s.}{residência; morada}
  \definition{v.}{(literário) residir; viver | implicar; conter}
\end{EntryWithPhonetic}

\begin{EntryWithPhonetic}{寓意}{yu4yi4}{12,13}{⼧,⼼}
  \definition{s.}{moral (de uma história),  lição a ser aprendida, implicação, mensagem, significado metafórico}
\end{EntryWithPhonetic}

%%%%%%%%%% 粥 %%%%%%%%%%
\subsection*{粥}\addcontentsline{loh}{figure}{粥 \dpy{yu4}}

\begin{EntryWithPhonetic}{粥}{yu4}{12}{⽶}
  \definition{v.}{dar a luz; ter filhos}
  \seeref{zhou1}
\end{EntryWithPhonetic}

%%%%%%%%%% 遇 %%%%%%%%%%
\subsection*{遇}\addcontentsline{loh}{figure}{遇 \dpy{yu4}}

\begin{EntryWithPhonetic}{遇}{yu4}{12}{⾡}[HSK 4]
  \definition*{s.}{Sobrenome: Yu}
  \definition{s.}{chance; oportunidade}
  \definition{v.}{encontrar; deparar-se com; encontrar-se | tratar; receber}
\end{EntryWithPhonetic}

\begin{EntryWithPhonetic}{遇到}{yu4dao4}{12,8}{⾡,⼑}[HSK 4]
  \definition{v.}{esbarrar em; encontrar; deparar-se com; conhecer alguém ou algo (inesperado)}
\end{EntryWithPhonetic}

\begin{EntryWithPhonetic}{遇见}{yu4 jian4}{12,4}{⾡,⾒}[HSK 4]
  \definition{v.}{encontrar; deparar-se com}
\end{EntryWithPhonetic}

%%%%%%%%%% 愈 %%%%%%%%%%
\subsection*{愈}\addcontentsline{loh}{figure}{愈 \dpy{yu4}}

\begin{EntryWithPhonetic}{愈}{yu4}{13}{⼼}
  \definition{adv.}{mais e mais | ainda mais}
  \definition{v.}{recuperar | curar}
\end{EntryWithPhonetic}

%%%%%%%%%% 豫 %%%%%%%%%%
\subsection*{豫}\addcontentsline{loh}{figure}{豫 \dpy{yu4}}

\begin{EntryWithPhonetic}{豫}{yu4}{15}{⾗}
  \definition*{s.}{Província de Henan, abreviatura de 河南}
  \definition{adj.}{satisfeito; encantado | anterior; preliminar; preparatório}
  \definition{adv.}{com antecedência; antecipadamente}
  \definition{v.}{viver com facilidade e conforto | participar de}
  \seealsoref{河南}{he2nan2}
  \seealsoref{预}{yu4}
\end{EntryWithPhonetic}

%%%%%%%%%% 元 %%%%%%%%%%
\subsection*{元}\addcontentsline{loh}{figure}{元 \dpy{yuan2}}

\begin{EntryWithPhonetic}{元}{yuan2}{4}{⼉}[HSK 1]
  \definition*{s.}{Dinastia Yuan (1271-1368) | Sobrenome: Yuan}
  \definition{adj.}{primeiro; principal; primário | chefe; diretor; líder | básico; fundamental; principal | unidade; componente; formando um todo}
  \definition{clas.}{yuan, a unidade monetária da China}
  \definition{s.}{moeda com valor e peso fixos | origem; elemento}
\end{EntryWithPhonetic}

\begin{EntryWithPhonetic}{元旦}{yuan2dan4}{4,5}{⼉,⽇}[HSK 5]
  \definition*{s.}{Dia de Ano Novo (1 de janeiro)}
\end{EntryWithPhonetic}

\begin{EntryWithPhonetic}{元来}{yuan2lai2}{4,7}{⼉,⽊}
  \variantof{原来}
\end{EntryWithPhonetic}

\begin{EntryWithPhonetic}{元气}{yuan2qi4}{4,4}{⼉,⽓}
  \definition{s.}{força | vigor | vitalidade | energial vital}
\end{EntryWithPhonetic}

\begin{EntryWithPhonetic}{元素}{yuan2su4}{4,10}{⼉,⽷}[HSK 6]
  \definition{s.}{elemento; fator essencial | elemento químico | Geometria: as partes que compõem uma figura, como os lados e ângulos de um triângulo | Algebra: os elementos algébricos incluem números e símbolos}
\end{EntryWithPhonetic}

\begin{EntryWithPhonetic}{元宵}{yuan2xiao1}{4,10}{⼉,⼧}
  \definition*{s.}{Festival das Lanternas}
  \seealsoref{元宵节}{yuan2xiao1jie2}
  \seealsoref{元夜}{yuan2ye4}
\end{EntryWithPhonetic}

\begin{EntryWithPhonetic}{元宵节}{yuan2xiao1jie2}{4,10,5}{⼉,⼧,⾋}
  \definition*{s.}{Festival das Lanternas (15º~dia do primeiro mês lunar)}
  \seealsoref{元宵}{yuan2xiao1}
  \seealsoref{元夜}{yuan2ye4}
\end{EntryWithPhonetic}

\begin{EntryWithPhonetic}{元夜}{yuan2ye4}{4,8}{⼉,⼣}
  \definition*{s.}{Festival das Lanternas}
  \seealsoref{元宵}{yuan2xiao1}
  \seealsoref{元宵节}{yuan2xiao1jie2}
\end{EntryWithPhonetic}

%%%%%%%%%% 员 %%%%%%%%%%
\subsection*{员}\addcontentsline{loh}{figure}{员 \dpy{yuan2}}

\begin{EntryWithPhonetic}{员}{yuan2}{7}{⼝}[HSK 3]
  \definition{clas.}{para comandantes militares}
  \definition{s.}{uma pessoa envolvida em algum campo de atividade; refere-se a pessoas que trabalham ou estudam | membro; refere-se aos membros de um grupo ou organização | vizinhança}
\end{EntryWithPhonetic}

\begin{EntryWithPhonetic}{员工}{yuan2gong1}{7,3}{⼝,⼯}[HSK 3]
  \definition[位,名,个]{s.}{equipe; funcionário; trabalhador; pessoal}
\end{EntryWithPhonetic}

%%%%%%%%%% 园 %%%%%%%%%%
\subsection*{园}\addcontentsline{loh}{figure}{园 \dpy{yuan2}}

\begin{EntryWithPhonetic}{园}{yuan2}{7}{⼞}[HSK 6]
  \definition*{s.}{Sobrenome: Yuan}
  \definition[个]{s.}{jardim; terreno; plantação; terra para cultivar plantas | local para recreação pública; locais para passeios turísticos e entretenimento | área para fins especiais | uma área de terra para o cultivo de plantas; um lugar onde vegetais, flores, frutas e árvores são cultivados}
\end{EntryWithPhonetic}

\begin{EntryWithPhonetic}{园地}{yuan2 di4}{7,6}{⼞,⼟}[HSK 6]
  \definition{s.}{jardim; campo | Figurativo: campo; escopo}
\end{EntryWithPhonetic}

\begin{EntryWithPhonetic}{园林}{yuan2lin2}{7,8}{⼞,⽊}[HSK 5]
  \definition[处,座,个]{s.}{parque; jardim; área paisagística com plantas e árvores para as pessoas apreciarem e descansarem.}
\end{EntryWithPhonetic}

%%%%%%%%%% 原 %%%%%%%%%%
\subsection*{原}\addcontentsline{loh}{figure}{原 \dpy{yuan2}}

\begin{EntryWithPhonetic}{原}{yuan2}{10}{⼚}[HSK 6]
  \definition*{s.}{Sobrenome: Yuan}
  \definition{adj.}{inicial; básico; primitivo | cru; bruto; não processado | virgem; primário; original; antigo; inalterado}
  \definition{adv.}{originalmente}
  \definition[项,条,片]{s.}{planície; país aberto; terreno plano e amplo | início; fonte; origem; aparência original | origem; a raiz ou o começo das coisas}
  \definition{v.}{desculpar; perdoar; tolerar; compreender | rastrear; sondar; investigar (a origem das coisas)}
\end{EntryWithPhonetic}

\begin{EntryWithPhonetic}{原告}{yuan2gao4}{10,7}{⼚,⼝}[HSK 6]
  \definition{s.}{(em casos civis) autor; solicitante | (em casos criminais) promotor; acusador; reclamante (oposto a 被告)}
  \seealsoref{被告}{bei4gao4}
\end{EntryWithPhonetic}

\begin{EntryWithPhonetic}{原来}{yuan2lai2}{10,7}{⼚,⽊}[HSK 2]
  \definition{adj.}{original; anterior; em primeiro lugar; inicialmente; inalterado}
  \definition{adv.}{na verdade; de fato; como se vê; expressar compreensão repentina}
  \definition{s.}{a princípio; no passado; antigamente}
\end{EntryWithPhonetic}

\begin{EntryWithPhonetic}{原理}{yuan2li3}{10,11}{⼚,⽟}[HSK 5]
  \definition[个,条]{s.}{princípio; axioma; teoria; teoria básica ou princípio científico de significado universal}
\end{EntryWithPhonetic}

\begin{EntryWithPhonetic}{原谅}{yuan2liang4}{10,10}{⼚,⾔}[HSK 6]
  \definition{v.}{perdoar; perdoar a negligência, os erros ou as falhas das pessoas sem culpá-las ou puni-las}
\end{EntryWithPhonetic}

\begin{EntryWithPhonetic}{原料}{yuan2liao4}{10,10}{⼚,⽃}[HSK 4]
  \definition[种,个]{s.}{matéria-prima; refere-se a materiais que não foram processados e fabricados, como minérios para metalurgia e algodão para têxteis}
\end{EntryWithPhonetic}

\begin{EntryWithPhonetic}{原木}{yuan2mu4}{10,4}{⼚,⽊}
  \definition{s.}{registro | \emph{logs}}
\end{EntryWithPhonetic}

\begin{EntryWithPhonetic}{原色}{yuan2 se4}{10,6}{⼚,⾊}
  \definition{s.}{cor primária}
\end{EntryWithPhonetic}

\begin{EntryWithPhonetic}{原始}{yuan2shi3}{10,8}{⼚,⼥}[HSK 5]
  \definition{s.}{original; de primeira mão | primitivo; mais antigo; não desenvolvido; não civilizado}
\end{EntryWithPhonetic}

\begin{EntryWithPhonetic}{原先}{yuan2xian1}{10,6}{⼚,⼉}[HSK 5]
  \definition{adj.}{antigo; original}
  \definition{s.}{antigamente; no início; no passado; no começo}
\end{EntryWithPhonetic}

\begin{EntryWithPhonetic}{原因}{yuan2yin1}{10,6}{⼚,⼞}[HSK 2]
  \definition[个,条,种,些]{s.}{causa; razão; motivo; as condições que fazem com que algo aconteça ou produzam um certo resultado}
\end{EntryWithPhonetic}

\begin{EntryWithPhonetic}{原有}{yuan2 you3}{10,6}{⼚,⽉}[HSK 5]
  \definition{v.}{já estar pronto, não é necessário fazer ou procurar nada; ser o original}
\end{EntryWithPhonetic}

\begin{EntryWithPhonetic}{原则}{yuan2ze2}{10,6}{⼚,⼑}[HSK 4]
  \definition{adv.}{em geral; em princípio; refere-se a um aspecto geral; geralmente}
  \definition[个,条,项,点]{s.}{princípios; leis ou padrões pelos quais alguém fala ou age}
\end{EntryWithPhonetic}

%%%%%%%%%% 圆 %%%%%%%%%%
\subsection*{圆}\addcontentsline{loh}{figure}{圆 \dpy{yuan2}}

\begin{EntryWithPhonetic}{圆}{yuan2}{10}{⼞}[HSK 4]
  \definition*{s.}{Sobrenome: Yuan}
  \definition{adj.}{redondo; circular; esférico; arredondado | diplomático; satisfatório}
  \definition[个,轮]{s.}{círculo; circunferência | uma moeda de valor e peso fixos}
  \definition{v.}{tornar plausível; justificar; tornar completo; completar}
\end{EntryWithPhonetic}

\begin{EntryWithPhonetic}{圆满}{yuan2man3}{10,13}{⼞,⽔}[HSK 4]
  \definition{adj.}{perfeito; satisfatório; sem defeitos}
\end{EntryWithPhonetic}

\begin{EntryWithPhonetic}{圆珠笔}{yuan2 zhu1 bi3}{10,10,10}{⼞,⽟,⽵}[HSK 6]
  \definition[支,枝]{s.}{caneta esferográfica}
\end{EntryWithPhonetic}

%%%%%%%%%% 援 %%%%%%%%%%
\subsection*{援}\addcontentsline{loh}{figure}{援 \dpy{yuan2}}

\begin{EntryWithPhonetic}{援}{yuan2}{12}{⼿}
  \definition*{s.}{Sobrenome: Yuan}
  \definition{v.}{puxar com a mão; segurar | citar; referenciar | ajudar; auxiliar; resgatar}
\end{EntryWithPhonetic}

\begin{EntryWithPhonetic}{援助}{yuan2 zhu4}{12,7}{⼿,⼒}[HSK 6]
  \definition{s.}{ajuda; assistência; auxílio}
  \definition{v.}{ajudar; apoiar; auxiliar}
\end{EntryWithPhonetic}

%%%%%%%%%% 缘 %%%%%%%%%%
\subsection*{缘}\addcontentsline{loh}{figure}{缘 \dpy{yuan2}}

\begin{EntryWithPhonetic}{缘}{yuan2}{12}{⽷}
  \definition{s.}{causa | razão | karma | destino | predestinação}
\end{EntryWithPhonetic}

\begin{EntryWithPhonetic}{缘分}{yuan2fen4}{12,4}{⽷,⼑}
  \definition{s.}{destino ou acaso que une as pessoas | afinidade ou relacionamento predestinado | destino (Budismo)}
\end{EntryWithPhonetic}

\begin{EntryWithPhonetic}{缘故}{yuan2gu4}{12,9}{⽷,⽁}[HSK 6]
  \definition{s.}{causa; razão}
\end{EntryWithPhonetic}

%%%%%%%%%% 源 %%%%%%%%%%
\subsection*{源}\addcontentsline{loh}{figure}{源 \dpy{yuan2}}

\begin{EntryWithPhonetic}{源}{yuan2}{13}{⽔}
  \definition*{s.}{Sobrenome: Yuan}
  \definition{s.}{nascente (de um rio); fonte | fonte; origem; causa}
  \definition{v.}{originar-se; provir de}
\end{EntryWithPhonetic}

%%%%%%%%%% 远 %%%%%%%%%%
\subsection*{远}\addcontentsline{loh}{figure}{远 \dpy{yuan3}}

\begin{EntryWithPhonetic}{远}{yuan3}{7}{⾡}[HSK 1]
  \definition*{s.}{Sobrenome: Yuan}
  \definition{adj.}{distante (no tempo ou no espaço); longe; remoto; Longa distância espacial ou temporal (em oposição a 近) | (relações de parentesco) distante | com grande diferença}
  \definition{v.}{manter-se afastado de; não se aproximar}
  \seealsoref{近}{jin4}
\end{EntryWithPhonetic}

\begin{EntryWithPhonetic}{远处}{yuan3 chu4}{7,5}{⾡,⼡}[HSK 5]
  \definition{s.}{distância; lugar distante}
\end{EntryWithPhonetic}

\begin{EntryWithPhonetic}{远方}{yuan3 fang1}{7,4}{⾡,⽅}[HSK 6]
  \definition{s.}{distância; de longe; lugar distante}
\end{EntryWithPhonetic}

\begin{EntryWithPhonetic}{远离}{yuan3 li2}{7,10}{⾡,⼇}[HSK 6]
  \definition{adj.}{afastado; distante}
  \definition{v.}{partir para; ficar longe}
\end{EntryWithPhonetic}

\begin{EntryWithPhonetic}{远天}{yuan3tian1}{7,4}{⾡,⼤}
  \definition{s.}{paraíso | o céu distante}
\end{EntryWithPhonetic}

\begin{EntryWithPhonetic}{远远}{yuan3 yuan3}{7,7}{⾡,⾡}[HSK 6]
  \definition{adv.}{de longe; em grande medida; para descrever um alto grau ou uma grande quantidade}
\end{EntryWithPhonetic}

\begin{EntryWithPhonetic}{远征}{yuan3zheng1}{7,8}{⾡,⼻}
  \definition{s.}{uma expedição militar | marcha para regiões remotas}
\end{EntryWithPhonetic}

%%%%%%%%%% 怨 %%%%%%%%%%
\subsection*{怨}\addcontentsline{loh}{figure}{怨 \dpy{yuan4}}

\begin{EntryWithPhonetic}{怨}{yuan4}{9}{⼼}[HSK 5]
  \definition{s.}{ressentimento; inimizade; rancor}
  \definition{v.}{culpar; reclamar}
\end{EntryWithPhonetic}

%%%%%%%%%% 院 %%%%%%%%%%
\subsection*{院}\addcontentsline{loh}{figure}{院 \dpy{yuan4}}

\begin{EntryWithPhonetic}{院}{yuan4}{9}{⾩}[HSK 2]
  \definition*{s.}{Sobrenome: Yuan}
  \definition[个]{s.}{pátio; quintal; complexo | designação para certos escritórios governamentais e locais públicos | faculdade; academia; instituto de ensino superior | hospital}
\end{EntryWithPhonetic}

\begin{EntryWithPhonetic}{院长}{yuan4zhang3}{9,4}{⾩,⾧}[HSK 2]
  \definition[个,位,名]{s.}{reitor; diretor; o mais alto funcionário de qualquer instituição ou escola pública ou privada}
\end{EntryWithPhonetic}

\begin{EntryWithPhonetic}{院子}{yuan4zi5}{9,3}{⾩,⼦}[HSK 2]
  \definition[个,座,处]{s.}{quintal; pátio; o espaço aberto na frente ou atrás de uma casa cercado por muros ou cercas}
\end{EntryWithPhonetic}

%%%%%%%%%% 愿 %%%%%%%%%%
\subsection*{愿}\addcontentsline{loh}{figure}{愿 \dpy{yuan4}}

\begin{EntryWithPhonetic}{愿}{yuan4}{14}{⽕}[HSK 5]
  \definition{adj.}{honesto e prudente}
  \definition{s.}{esperança; desejo; vontade; a ideia de alcançar algum objetivo no futuro | voto (feito perante o Buda ou um deus); o desejo de retribuição feito ao rezar para os deuses e Buda}
  \definition{v.}{estar disposto; estar pronto; de bom grado, concordar porque está de acordo com seus desejos | ter esperança; desejar; qerer alcançar algum desejo}
\end{EntryWithPhonetic}

\begin{EntryWithPhonetic}{愿望}{yuan4wang4}{14,11}{⽕,⽉}[HSK 3]
  \definition[个,种]{s.}{desejo; aspiração; a ideia de alcançar algum objetivo no futuro.}
\end{EntryWithPhonetic}

\begin{EntryWithPhonetic}{愿意}{yuan4yi4}{14,13}{⽕,⼼}[HSK 2]
  \definition{v.}{estar disposto; estar pronto | desejar; ter esperança}
\end{EntryWithPhonetic}

%%%%%%%%%% 约 %%%%%%%%%%
\subsection*{约}\addcontentsline{loh}{figure}{约 \dpy{yue1}}

\begin{EntryWithPhonetic}{约}{yue1}{6}{⽷}[HSK 3]
  \definition*{s.}{Sobrenome: Yue}
  \definition{adj.}{econômico; frugal | simples; breve; resumido | indistinto; confuso}
  \definition{adv.}{cerca de; ao redor; aproximadamente}
  \definition{s.}{pacto; acordo; nomeação; o que foi combinado}
  \definition{v.}{combinar; propor ou discutir antecipadamente (o que deve ser respeitado por todos) | convidar com antecedência | restringir; conter | reduzir (fração aproximada)}
  \seeref{yao1}
\end{EntryWithPhonetic}

\begin{EntryWithPhonetic}{约定}{yue1 ding4}{6,8}{⽷,⼧}[HSK 6]
  \definition{s.}{acordo; compromisso}
  \definition{v.}{determinar; chegar a um acordo; concordar com}
\end{EntryWithPhonetic}

\begin{EntryWithPhonetic}{约会}{yue1hui4}{6,6}{⽷,⼈}[HSK 4]
  \definition[个,次]{s.}{data; compromisso; engajamento; reunião pré-agendada}
  \definition{v.}{marcar uma reunião; marcar um encontro}
\end{EntryWithPhonetic}

\begin{EntryWithPhonetic}{约束}{yue1shu4}{6,7}{⽷,⽊}[HSK 5]
  \definition{adj.}{amarrado}
  \definition{s.}{restrição; constrangimento; engajamento}
  \definition{v.}{amarrar; prender; reprimir; restringir; manter dentro de si}
\end{EntryWithPhonetic}

%%%%%%%%%% 月 %%%%%%%%%%
\subsection*{月}\addcontentsline{loh}{figure}{月 \dpy{yue4}}

\begin{EntryWithPhonetic}{月}{yue4}{4}{⽉}[HSK 1][Kangxi 74]
  \definition*{s.}{Sobrenome: Yue}
  \definition[个]{s.}{mês | por mês | lua | redondo; em forma de lua cheia}
\end{EntryWithPhonetic}

\begin{EntryWithPhonetic}{月饼}{yue4 bing3}{4,9}{⽉,⾷}[HSK 5]
  \definition[个,块,盒,筒]{s.}{bolinho da lua; comida típica do Festival do Meio Outono; redonda e recheada; simboliza a reunião familiar}
\end{EntryWithPhonetic}

\begin{EntryWithPhonetic}{月底}{yue4 di3}{4,8}{⽉,⼴}[HSK 4]
  \definition[个]{s.}{final do mês; últimos dias do mês}
\end{EntryWithPhonetic}

\begin{EntryWithPhonetic}{月份}{yue4 fen4}{4,6}{⽉,⼈}[HSK 2]
  \definition[个]{s.}{mês; refere-se a um determinado mês}
\end{EntryWithPhonetic}

\begin{EntryWithPhonetic}{月径}{yue4jing4}{4,8}{⽉,⼻}
  \definition{s.}{diâmetro da lua | diâmetro da órbita da lua | caminho iluminado pela lua}
\end{EntryWithPhonetic}

\begin{EntryWithPhonetic}{月亮}{yue4liang5}{4,9}{⽉,⼇}[HSK 2]
  \definition[个,轮,挂,快,页]{s.}{Lua; Lua é o nome comum do satélite da Terra}
\end{EntryWithPhonetic}

\begin{EntryWithPhonetic}{月球}{yue4 qiu2}{4,11}{⽉,⽟}[HSK 5]
  \definition*[颗,个]{s.}{Lua}
  \definition{pref.}{seleno-; seleni-}
\end{EntryWithPhonetic}

\begin{EntryWithPhonetic}{月壤}{yue4rang3}{4,20}{⽉,⼟}
  \definition{s.}{solo lunar}
\end{EntryWithPhonetic}

\begin{EntryWithPhonetic}{月相}{yue4xiang4}{4,9}{⽉,⽬}
  \definition{s.}{fases da lua, a saber: lua nova 朔, lua crescente 上弦, lua cheia 望 e lua minguante 下弦}
\end{EntryWithPhonetic}

\begin{EntryWithPhonetic}{月月}{yue4yue4}{4,4}{⽉,⽉}
  \definition{adv.}{todo mês}
\end{EntryWithPhonetic}

%%%%%%%%%% 乐 %%%%%%%%%%
\subsection*{乐}\addcontentsline{loh}{figure}{乐 \dpy{yue4}}

\begin{EntryWithPhonetic}{乐}{yue4}{5}{⼃}
  \definition*{s.}{Sobrenome: Yue}
  \definition{s.}{música}
  \seeref{le4}
\end{EntryWithPhonetic}

\begin{EntryWithPhonetic}{乐队}{yue4 dui4}{5,4}{⼃,⾩}[HSK 3]
  \definition[支,个]{s.}{orquestra; banda; um grupo composto por muitas pessoas que tocam diferentes instrumentos musicais}
\end{EntryWithPhonetic}

\begin{EntryWithPhonetic}{乐曲}{yue4 qu3}{5,6}{⼃,⽈}[HSK 6]
  \definition[支,首,段]{s.}{música; composição musical}
\end{EntryWithPhonetic}

%%%%%%%%%% 阅 %%%%%%%%%%
\subsection*{阅}\addcontentsline{loh}{figure}{阅 \dpy{yue4}}

\begin{EntryWithPhonetic}{阅}{yue4}{10}{⾨}
  \definition{v.}{ler; repassar; examinar | revisar; inspecionar | experimentar; passar por}
\end{EntryWithPhonetic}

\begin{EntryWithPhonetic}{阅兵式}{yue4bing1shi4}{10,7,6}{⾨,⼋,⼷}
  \definition{s.}{parada militar; desfile militar}
\end{EntryWithPhonetic}

\begin{EntryWithPhonetic}{阅读}{yue4du2}{10,10}{⾨,⾔}[HSK 4]
  \definition{v.}{ler; examinar; olhar (livros, jornais, etc.) e entender seu conteúdo}
\end{EntryWithPhonetic}

\begin{EntryWithPhonetic}{阅读广度}{yue4du2guang3du4}{10,10,3,9}{⾨,⾔,⼴,⼴}
  \definition{s.}{intervalo de leitura}
\end{EntryWithPhonetic}

\begin{EntryWithPhonetic}{阅读理解}{yue4du2li3jie3}{10,10,11,13}{⾨,⾔,⽟,⾓}
  \definition{s.}{compreensão de leitura}
\end{EntryWithPhonetic}

\begin{EntryWithPhonetic}{阅读器}{yue4du2qi4}{10,10,16}{⾨,⾔,⼝}
  \definition{s.}{leitor (\emph{software})}
\end{EntryWithPhonetic}

\begin{EntryWithPhonetic}{阅读时间}{yue4 du2 shi2 jian1}{10,10,7,7}{⾨,⾔,⽇,⾨}
  \definition{s.}{tempo de leitura}
\end{EntryWithPhonetic}

\begin{EntryWithPhonetic}{阅读障碍}{yue4du2zhang4ai4}{10,10,13,13}{⾨,⾔,⾩,⽯}
  \definition{s.}{dislexia}
\end{EntryWithPhonetic}

\begin{EntryWithPhonetic}{阅读装置}{yue4du2zhuang1zhi4}{10,10,12,13}{⾨,⾔,⾐,⽹}
  \definition{s.}{dispositivo de leitura (por exemplo, para códigos de barras, etiquetas RFID, etc.)}
\end{EntryWithPhonetic}

\begin{EntryWithPhonetic}{阅览室}{yue4 lan3 shi4}{10,9,9}{⾨,⾒,⼧}[HSK 5]
  \definition[个,间]{s.}{sala de leitura; a biblioteca dispõe de salas para leitura e pesquisa, equipadas com mesas e cadeiras adequadas, livros, jornais, revistas, etc.}
\end{EntryWithPhonetic}

%%%%%%%%%% 粤 %%%%%%%%%%
\subsection*{粤}\addcontentsline{loh}{figure}{粤 \dpy{yue4}}

\begin{EntryWithPhonetic}{粤}{yue4}{12}{⾔}
  \definition*{s.}{Outro nome para a Província de Guangdong, 广东}
  \seealsoref{广东}{guang3dong1}
\end{EntryWithPhonetic}

\begin{EntryWithPhonetic}{粤语}{yue4yu3}{12,9}{⾔,⾔}
  \definition{s.}{cantonês | língua cantonesa}
\end{EntryWithPhonetic}

%%%%%%%%%% 越 %%%%%%%%%%
\subsection*{越}\addcontentsline{loh}{figure}{越 \dpy{yue4}}

\begin{EntryWithPhonetic}{越}{yue4}{12}{⾛}[HSK 2]
  \definition{adj.}{superior; excede ou ultrapassa o ordinário}
  \definition{adv.}{quanto mais\dots mais; usados juntos, eles formam o formato de ``越…越…'' para indicar que o grau de uma situação se torna mais sério à medida que se desenvolve; ``成年…'' para indicar que o grau de uma situação se torna mais sério à medida que o tempo passa}
  \definition{v.}{passar por cima; pular; cruzar | exceder; ultrapassar | estar em um tom alto; estar animado | saquear; pilhar; expoliar; apreender; roubar | passar; passar através; atravessar}
  \seealsoref{越来越…}{yue4 lai2 yue4}
  \seealsoref{越…越…}{yue4 yue4}
\end{EntryWithPhonetic}

\begin{EntryWithPhonetic}{越境}{yue4jing4}{12,14}{⾛,⼟}
  \definition{v.}{cruzar a fronteira ilegalmente; entrar ou sair clandestinamente de um país}
\end{EntryWithPhonetic}

\begin{EntryWithPhonetic}{越来越…}{yue4 lai2 yue4}{12,7,12}{⾛,⽊,⾛}[HSK 2]
  \definition{adv.}{cada vez mais\dots; isso significa que o grau de algo se aprofunda à medida que o tempo passa}
\end{EntryWithPhonetic}

\begin{EntryWithPhonetic}{越…越…}{yue4 yue4}{12,12}{⾛,⾛}[HSK 2]
  \definition{expr.}{quanto mais\dots tanto mais\dots}
\end{EntryWithPhonetic}

\begin{EntryWithPhonetic}{越障}{yue4zhang4}{12,13}{⾛,⾩}
  \definition{s.}{curso com obstáculos para treinamento de tropas}
  \definition{v.}{superar obstáculos}
\end{EntryWithPhonetic}

%%%%%%%%%% 龠 %%%%%%%%%%
\subsection*{龠}\addcontentsline{loh}{figure}{龠 \dpy{yue4}}

\begin{EntryWithPhonetic}{龠}{yue4}{17}{⿕}[Kangxi 214]
  \definition{clas.}{yue, uma unidade de medida seca para grãos (= 0,5 decilitro);}
  \definition{s.}{uma flauta curta antiga}
\end{EntryWithPhonetic}

%%%%%%%%%% 晕 %%%%%%%%%%
\subsection*{晕}\addcontentsline{loh}{figure}{晕 \dpy{yun1}}

\begin{EntryWithPhonetic}{晕}{yun1}{10}{⽇}[HSK 6]
  \definition{adj.}{tonto; vertiginoso; confuso; sensação de que as coisas estão girando ao seu redor e, às vezes, sensação de que você vai cair}
  \definition{v.}{desmaiar; desfalecer}
  \seeref{yun4}
\end{EntryWithPhonetic}

%%%%%%%%%% 云 %%%%%%%%%%
\subsection*{云}\addcontentsline{loh}{figure}{云 \dpy{yun2}}

\begin{EntryWithPhonetic}{云}{yun2}{4}{⼆}[HSK 2]
  \definition*{s.}{Província de Yunnan, abreviação de 云南 | Sobrenome: Yun}
  \definition[片,朵]{s.}{nuvem}
  \definition{v.}{dizer}
  \seealsoref{云南}{yun2nan2}
\end{EntryWithPhonetic}

\begin{EntryWithPhonetic}{云端}{yun2duan1}{4,14}{⼆,⽴}
  \definition{s.}{alto nas nuvens | (computação) a nuvem}
\end{EntryWithPhonetic}

\begin{EntryWithPhonetic}{云南}{yun2nan2}{4,9}{⼆,⼗}
  \definition*{s.}{Província de Yunnan}
\end{EntryWithPhonetic}

\begin{EntryWithPhonetic}{云云}{yun2yun2}{4,4}{⼆,⼆}
  \definition{adv.}{e assim por diante | assim e assim}
\end{EntryWithPhonetic}

%%%%%%%%%% 允 %%%%%%%%%%
\subsection*{允}\addcontentsline{loh}{figure}{允 \dpy{yun3}}

\begin{EntryWithPhonetic}{允}{yun3}{4}{⼉}
  \definition*{s.}{Sobrenome: Yun}
  \definition{adj.}{justo; imparcial}
  \definition{v.}{permitir; deixar; consentir}
\end{EntryWithPhonetic}

\begin{EntryWithPhonetic}{允许}{yun3xu3}{4,6}{⼉,⾔}[HSK 6]
  \definition{s.}{permitido; permissão}
  \definition{v.}{permitir; deixar; concordar com alguém para fazer algo}
\end{EntryWithPhonetic}

%%%%%%%%%% 运 %%%%%%%%%%
\subsection*{运}\addcontentsline{loh}{figure}{运 \dpy{yun4}}

\begin{EntryWithPhonetic}{运}{yun4}{7}{⾡}[HSK 5]
  \definition*{s.}{Sobrenome: Yun}
  \definition{s.}{sorte; destino; fortuna}
  \definition{v.}{mover; deslocar | transportar; levar | usar; empunhar; utilizar}
\end{EntryWithPhonetic}

\begin{EntryWithPhonetic}{运动}{yun4dong4}{7,6}{⾡,⼒}[HSK 2]
  \definition[项,种,场,次]{s.}{esportes; atletismo; exercício; atividades esportivas | movimento; campanha (política); atividades de massa organizadas, intencionais e de alto nível na política, cultura, produção, etc. | movimento; refere-se a todas as mudanças}
  \definition{v.}{exercitar; fazer atividade física | mover-se; refere-se à mudança na posição de um objeto}
\end{EntryWithPhonetic}

\begin{EntryWithPhonetic}{运动病}{yun4dong4bing4}{7,6,10}{⾡,⼒,⽧}
  \definition{s.}{enjôo (movimento, carro, etc.)}
\end{EntryWithPhonetic}

\begin{EntryWithPhonetic}{运动场}{yun4dong4chang3}{7,6,6}{⾡,⼒,⼟}
  \definition{s.}{campo desportivo | campo de jogos}
\end{EntryWithPhonetic}

\begin{EntryWithPhonetic}{运动服}{yun4dong4fu2}{7,6,8}{⾡,⼒,⽉}
  \definition{s.}{roupa para prática de esporte}
\end{EntryWithPhonetic}

\begin{EntryWithPhonetic}{运动会}{yun4 dong4 hui4}{7,6,6}{⾡,⼒,⼈}[HSK 4]
  \definition[届,场,次,个]{s.}{jogos; encontro esportivo; dia de esportes; encontro atlético; competição esportiva abrangente}
\end{EntryWithPhonetic}

\begin{EntryWithPhonetic}{运动家}{yun4dong4jia1}{7,6,10}{⾡,⼒,⼧}
  \definition{s.}{ativista | atleta | esportista}
\end{EntryWithPhonetic}

\begin{EntryWithPhonetic}{运动衫}{yun4dong4shan1}{7,6,8}{⾡,⼒,⾐}
  \definition[件]{s.}{moletom | camisa esportiva}
\end{EntryWithPhonetic}

\begin{EntryWithPhonetic}{运动鞋}{yun4dong4xie2}{7,6,15}{⾡,⼒,⾰}
  \definition{s.}{tênis | sapatos esportivos}
\end{EntryWithPhonetic}

\begin{EntryWithPhonetic}{运动学}{yun4dong4xue2}{7,6,8}{⾡,⼒,⼦}
  \definition{s.}{cinemática; um ramo da ciência do esporte que usa a anatomia e a mecânica humanas para explicar várias atividades esportivas}
\end{EntryWithPhonetic}

\begin{EntryWithPhonetic}{运动员}{yun4 dong4 yuan2}{7,6,7}{⾡,⼒,⼝}[HSK 4]
  \definition[名,个,班]{s.}{jogador; atleta; esportista; pessoas que participam de competições esportivas}
\end{EntryWithPhonetic}

\begin{EntryWithPhonetic}{运河}{yun4he2}{7,8}{⾡,⽔}
  \definition{s.}{canal (em um rio)}
\end{EntryWithPhonetic}

\begin{EntryWithPhonetic}{运气}{yun4/qi4}{7,4}{⾡,⽓}
  \definition{v.+compl.}{tentar a sorte | concentrar a energia em uma parte do corpo}[他们地运一口气。===Eles respiraram fundo.]
  \seeref{yun4qi5}
\end{EntryWithPhonetic}

\begin{EntryWithPhonetic}{运气}{yun4qi5}{7,4}{⾡,⽓}[HSK 4]
  \definition{adj.}{sortudo; afortunado}
  \definition{s.}{sorte; fortuna}
  \seeref{yun4/qi4}
\end{EntryWithPhonetic}

\begin{EntryWithPhonetic}{运输}{yun4shu1}{7,13}{⾡,⾞}[HSK 3]
  \definition{v.}{enviar; transportar; transportar pessoas ou coisas de um lugar para outro usando carros, barcos, aviões, etc.}
\end{EntryWithPhonetic}

\begin{EntryWithPhonetic}{运行}{yun4xing2}{7,6}{⾡,⾏}[HSK 5]
  \definition{v.}{correr; mover; trabalhar; estar em movimento; (veículo, nave, planeta, etc.) mover-se em um ciclo repetitivo; avançar de maneira regular e direcional}
\end{EntryWithPhonetic}

\begin{EntryWithPhonetic}{运用}{yun4yong4}{7,5}{⾡,⽤}[HSK 4]
  \definition{v.}{usar; utilizar; manejar; aplicar; explorar as coisas de acordo com suas características}
\end{EntryWithPhonetic}

\begin{EntryWithPhonetic}{运作}{yun4 zuo4}{7,7}{⾡,⼈}[HSK 6]
  \definition{v.}{trabalhar; operar; (uma instituição, organização, etc.) realizar trabalho; realizar atividades}
\end{EntryWithPhonetic}

%%%%%%%%%% 晕 %%%%%%%%%%
\subsection*{晕}\addcontentsline{loh}{figure}{晕 \dpy{yun4}}

\begin{EntryWithPhonetic}{晕}{yun4}{10}{⽇}
  \definition{s.}{auréola; o círculo de luz formado pela refração da luz solar ou do luar através dos cristais de gelo nas nuvens | halo em torno de alguma cor ou luz; áreas desfocadas em torno de luz, sombra e cor}
  \definition{v.}{ficar tonto; desmaiar; desfalecer; sensação de tontura, como se os objetos ao seu redor estivessem girando e como se você estivesse prestes a cair}
  \seeref{yun1}
\end{EntryWithPhonetic}

\begin{EntryWithPhonetic}{晕车}{yun4 che1}{10,4}{⽇,⾞}[HSK 6]
  \definition{v.}{ter enjoo no carro; ter tontura e vômito ao andar de carro}
\end{EntryWithPhonetic}

%%%%% EOF %%%%%


 %%%
%%% Z
%%%
\section*{Z}\addcontentsline{toc}{section}{Z}\addcontentsline{loh}{figure}{\#\#\#\#\#\#\#\# Z}

%%%%%%%%%% 扎 %%%%%%%%%%
\subsection*{扎}\addcontentsline{loh}{figure}{扎 \dpy{za1}}

\begin{EntryWithPhonetic}{扎}{za1}{4}{⼿}
  \definition{clas.}{usado para pacotes, feixes, maços, etc.}
  \definition{v.}{atar; amarrar; embrulhar}
  \seeref{zha1}
  \seeref{zha2}
\end{EntryWithPhonetic}

%%%%%%%%%% 杂 %%%%%%%%%%
\subsection*{杂}\addcontentsline{loh}{figure}{杂 \dpy{za2}}

\begin{EntryWithPhonetic}{杂}{za2}{6}{⽊}[HSK 6]
  \definition{adj.}{diversos; misto; misturados | extra; irregular | variado}
  \definition{v.}{misturar}
\end{EntryWithPhonetic}

\begin{EntryWithPhonetic}{杂技}{za2ji4}{6,7}{⽊,⼿}
  \definition[场,个]{s.}{acrobacia; um termo geral para várias performances (como habilidades com carros, ventriloquia, equilíbrio de tigelas, andar na corda bamba, dança do leão, mágica, etc.)}
\end{EntryWithPhonetic}

\begin{EntryWithPhonetic}{杂剧}{za2ju4}{6,10}{⽊,⼑}
  \definition{s.}{(na Dinastia Song) peça de variedades que consiste em um prelúdio, a peça principal em uma ou duas cenas e um epílogo musical (nenhuma das peças de variedades Song existe até hoje) |  (na Dinastia Yuan) drama poético que consiste em quatro atos ou sequências de canções (折), ocasionalmente incluindo uma ``cunha'' (楔子) na forma de um prólogo (colocado antes do primeiro ato) ou um interlúdio (colocado entre os atos), todas as partes cantadas nos quatro atos atribuídas ao protagonista, seja homem ou mulher | uma forma de comédia musical da dinastia Yuan; uma forma de performance caracterizada pelo humor e pela brincadeira na Dinastia Song, desenvolveu-se como uma forma de ópera na Dinastia Yuan, cada obra consiste em quatro atos, às vezes com um prólogo no início ou entre os atos, cada ato é composto por um conjunto de melodias nórdicas na mesma melodia e rima palaciana, além de versos de convidados}
  \seealsoref{楔子}{xie1zi5}
  \seealsoref{折}{zhe2}
\end{EntryWithPhonetic}

\begin{EntryWithPhonetic}{杂志}{za2zhi4}{6,7}{⽊,⼼}[HSK 3]
  \definition[本,期,份,种]{s.}{jornal; revista; publicação}
\end{EntryWithPhonetic}

\begin{EntryWithPhonetic}{杂志社}{za2zhi4 she4}{6,7,7}{⽊,⼼,⽰}
  \definition{s.}{editora de revista; a organização responsável pela edição, publicação e distribuição de revistas}
\end{EntryWithPhonetic}

%%%%%%%%%% 咱 %%%%%%%%%%
\subsection*{咱}\addcontentsline{loh}{figure}{咱 \dpy{za2}}

\begin{EntryWithPhonetic}{咱}{za2}{9}{⼝}
  \seeref{zan2}
  \seeref{zan5}
\end{EntryWithPhonetic}

\begin{EntryWithPhonetic}{咱家}{za2jia1}{9,10}{⼝,⼧}
  \definition{pron.}{eu (frequentemente usado na literatura vernácula antiga) | me | mim | comigo}
\end{EntryWithPhonetic}

%%%%%%%%%% 砸 %%%%%%%%%%
\subsection*{砸}\addcontentsline{loh}{figure}{砸 \dpy{za2}}

\begin{EntryWithPhonetic}{砸}{za2}{10}{⽯}
  \definition{v.}{esmagar | bater | falhar | estragar}
\end{EntryWithPhonetic}

%%%%%%%%%% 灾 %%%%%%%%%%
\subsection*{灾}\addcontentsline{loh}{figure}{灾 \dpy{zai1}}

\begin{EntryWithPhonetic}{灾}{zai1}{7}{⽕}[HSK 5]
  \definition[个,场]{s.}{calamidade; desastre | infortúnio pessoal; adversidade | azar}
\end{EntryWithPhonetic}

\begin{EntryWithPhonetic}{灾害}{zai1hai4}{7,10}{⽕,⼧}[HSK 5]
  \definition[场,次,个]{s.}{desastre; calamidade; danos causados pela seca, inundações, pragas, granizo, guerras, etc.}
\end{EntryWithPhonetic}

\begin{EntryWithPhonetic}{灾难}{zai1nan4}{7,10}{⽕,⾫}[HSK 5]
  \definition[场,次,个,种]{s.}{desastre; sofrimento; calamidade; catástrofe; danos e sofrimentos causados por desastres naturais ou guerras}
\end{EntryWithPhonetic}

\begin{EntryWithPhonetic}{灾区}{zai1qu1}{7,4}{⽕,⼖}[HSK 5]
  \definition{s.}{área de desastre; área afetada por catástrofes}
\end{EntryWithPhonetic}

%%%%%%%%%% 栽 %%%%%%%%%%
\subsection*{栽}\addcontentsline{loh}{figure}{栽 \dpy{zai1}}

\begin{EntryWithPhonetic}{栽}{zai1}{10}{⽊}
  \definition{v.}{cultivar | plantar}
\end{EntryWithPhonetic}

\begin{EntryWithPhonetic}{栽倒}{zai1dao3}{10,10}{⽊,⼈}
  \definition{v.}{cair | sofrer uma queda}
\end{EntryWithPhonetic}

\begin{EntryWithPhonetic}{栽培}{zai1pei2}{10,11}{⽊,⼟}
  \definition{v.}{cultivar | educar | patrocinar | treinar}
\end{EntryWithPhonetic}

\begin{EntryWithPhonetic}{栽培种}{zai1pei2 zhong3}{10,11,9}{⽊,⼟,⽲}
  \definition{s.}{espécies cultivadas}
\end{EntryWithPhonetic}

\begin{EntryWithPhonetic}{栽赃}{zai1zang1}{10,10}{⽊,⾙}
  \definition{v.}{enquadrar alguém (plantar provas nele)}
\end{EntryWithPhonetic}

\begin{EntryWithPhonetic}{栽植}{zai1zhi2}{10,12}{⽊,⽊}
  \definition{v.}{plantar | transplantar}
\end{EntryWithPhonetic}

\begin{EntryWithPhonetic}{栽种}{zai1zhong4}{10,9}{⽊,⽲}
  \definition{v.}{plantar}
\end{EntryWithPhonetic}

%%%%%%%%%% 再 %%%%%%%%%%
\subsection*{再}\addcontentsline{loh}{figure}{再 \dpy{zai4}}

\begin{EntryWithPhonetic}{再}{zai4}{6}{⼌}[HSK 1]
  \definition{adv.}{mais uma vez; além disso; ainda mais; indica a repetição ou continuação de uma mesma ação ou comportamento; refere"-se principalmente a ações ou comportamentos não realizados ou contínuos | usado antes do adjetivo, indica intensificação, equivalente a 更 ou 更加 | (para uma ação adiada, precedida por uma expressão de tempo ou condição) então; somente então; depois de algo; indica que a ação ocorrerá após a conclusão de outra ação | além disso; indica um complemento, equivalente a 另外 ou 又 | próxima vez; indica que a ação ocorrerá após um determinado período de tempo | novamente; de novo}
  \seealsoref{更}{geng4}
  \seealsoref{更加}{geng4jia1}
  \seealsoref{另外}{ling4wai4}
  \seealsoref{又}{you4}
\end{EntryWithPhonetic}

\begin{EntryWithPhonetic}{再不}{zai4bu4}{6,4}{⼌,⼀}
  \definition{adv.}{nunca mais}
\end{EntryWithPhonetic}

\begin{EntryWithPhonetic}{再次}{zai4ci4}{6,6}{⼌,⽋}[HSK 5]
  \definition{adv.}{mais uma vez; uma segunda vez; outra vez}
\end{EntryWithPhonetic}

\begin{EntryWithPhonetic}{再读}{zai4du2}{6,10}{⼌,⾔}
  \definition{v.}{ler novamente | rever (uma lição, etc.)}
\end{EntryWithPhonetic}

\begin{EntryWithPhonetic}{再度}{zai4du4}{6,9}{⼌,⼴}
  \definition{adv.}{outra vez | mais uma vez}
\end{EntryWithPhonetic}

\begin{EntryWithPhonetic}{再发}{zai4 fa1}{6,5}{⼌,⼜}
  \definition{v.}{reenviar; reeditar}
\end{EntryWithPhonetic}

\begin{EntryWithPhonetic}{再见}{zai4jian4}{6,4}{⼌,⾒}[HSK 1]
  \definition{v.}{adeus; tchau; até logo; até mais; até mais tarde}
\end{EntryWithPhonetic}

\begin{EntryWithPhonetic}{再临}{zai4lin2}{6,9}{⼌,⼁}
  \definition{v.}{vir de novo; voltar}
\end{EntryWithPhonetic}

\begin{EntryWithPhonetic}{再三}{zai4san1}{6,3}{⼌,⼀}[HSK 4]
  \definition{adv.}{repetidamente; repetidas vezes; de novo e de novo}
\end{EntryWithPhonetic}

\begin{EntryWithPhonetic}{再审}{zai4shen3}{6,8}{⼌,⼧}
  \definition{s.}{novo julgamento | revisão}
  \definition{v.}{ouvir um caso novamente}
\end{EntryWithPhonetic}

\begin{EntryWithPhonetic}{再生}{zai4sheng1}{6,5}{⼌,⽣}[HSK 6]
  \definition{v.}{reviver; ressuscitar; ressuscitar dos mortos | reproduzir; regenerar | reprocessar; reciclar; regenerar; processar um determinado produto residual para restaurar seu desempenho original e transformá-lo em um novo produto}
\end{EntryWithPhonetic}

\begin{EntryWithPhonetic}{再说}{zai4shuo1}{6,9}{⼌,⾔}[HSK 6]
  \definition{conj.}{além disso; o que é mais; indicando uma razão adicional, equivalente a 况且}
  \definition{v.}{adiar para mais tarde; deixar para processamento ou consideração posterior}
  \seealsoref{况且}{kuang4qie3}
\end{EntryWithPhonetic}

\begin{EntryWithPhonetic}{再也}{zai4ye3}{6,3}{⼌,⼄}[HSK 5]
  \definition{adv.}{não mais; nunca mais; uma determinada situação ou ação nunca mais ocorrerá}
\end{EntryWithPhonetic}

\begin{EntryWithPhonetic}{再育}{zai4yu4}{6,8}{⼌,⾁}
  \definition{v.}{aumentar | multiplicar | proliferar}
\end{EntryWithPhonetic}

\begin{EntryWithPhonetic}{再者}{zai4zhe3}{6,8}{⼌,⽼}
  \definition{conj.}{além do mais | além disso}
\end{EntryWithPhonetic}

%%%%%%%%%% 在 %%%%%%%%%%
\subsection*{在}\addcontentsline{loh}{figure}{在 \dpy{zai4}}

\begin{EntryWithPhonetic}{在}{zai4}{6}{⼟}[HSK 1]
  \definition{adv.}{em processo de; em curso de}
  \definition{prep.}{em; no (um lugar ou momento); indica tempo, local, âmbito, etc.}
  \definition{v.}{existir; estar vivo | estar em; estar no; estar em (um lugar); indica a localização de pessoas ou coisas | permanecer; ficar | depender de; residir em; repousar com | ingressar ou pertencer a uma organização; ser membro de uma organização}
\end{EntryWithPhonetic}

\begin{EntryWithPhonetic}{在场}{zai4chang3}{6,6}{⼟,⼟}[HSK 5]
  \definition{v.}{estar presente; estar no local; estar em cena; estar presente onde as coisas estão acontecendo}
\end{EntryWithPhonetic}

\begin{EntryWithPhonetic}{在此}{zai4 ci3}{6,6}{⼟,⽌}
  \definition{s.}{aqui}
\end{EntryWithPhonetic}

\begin{EntryWithPhonetic}{在地}{zai4di4}{6,6}{⼟,⼟}
  \definition{s.}{local}
\end{EntryWithPhonetic}

\begin{EntryWithPhonetic}{在行}{zai4hang2}{6,6}{⼟,⾏}
  \definition{v.}{ser adepto de algo | ser um especialista em um comércio ou profissão}
\end{EntryWithPhonetic}

\begin{EntryWithPhonetic}{在乎}{zai4hu5}{6,5}{⼟,⼃}[HSK 4]
  \definition{v.}{preocupar-se; preocupar-se com; levar a sério | ser responsável por; caber ao; ser da competência de}
\end{EntryWithPhonetic}

\begin{EntryWithPhonetic}{在家}{zai4jia1}{6,10}{⼟,⼧}[HSK 1]
  \definition{v.}{estar em; estar em casa; estar no local de trabalho ou alojamento; sem sair de casa | continuar sendo um leigo; permanecer leigo; para monges, freiras, taoístas e outros que 出家, as pessoas comuns são consideradas leigas}
  \seealsoref{出家}{chu1 jia1}
\end{EntryWithPhonetic}

\begin{EntryWithPhonetic}{在教}{zai4 jiao4}{6,11}{⼟,⽁}
  \definition{v.}{ser um crente (em uma religião)}
\end{EntryWithPhonetic}

\begin{EntryWithPhonetic}{在内}{zai4nei4}{6,4}{⼟,⼌}[HSK 5]
  \definition{adj.}{incluido}
  \definition{adv.}{dentro; internamente; entre eles}
  \definition{v.}{ser incluído}
\end{EntryWithPhonetic}

\begin{EntryWithPhonetic}{在下}{zai4xia4}{6,3}{⼟,⼀}
  \definition{pron.}{eu mesmo (humildemente)}
\end{EntryWithPhonetic}

\begin{EntryWithPhonetic}{在线}{zai4xian4}{6,8}{⼟,⽷}
  \definition{s.}{\emph{online}}
\end{EntryWithPhonetic}

\begin{EntryWithPhonetic}{在意}{zai4/yi4}{6,13}{⼟,⼼}
  \definition{v.+compl.}{preocupar-se | importar-se | levar a sério}
\end{EntryWithPhonetic}

\begin{EntryWithPhonetic}{在于}{zai4yu2}{6,3}{⼟,⼆}[HSK 4]
  \definition{v.}{ser responsável por; caber a;  ser da competência de;  apontar a essência das coisas, ou do que elas se tratam | depender de; ser determinado por;  ser devido a (um determinado atributo)/(de um assunto a ser determinado)}
\end{EntryWithPhonetic}

%%%%%%%%%% 咱 %%%%%%%%%%
\subsection*{咱}\addcontentsline{loh}{figure}{咱 \dpy{zan2}}

\begin{EntryWithPhonetic}{咱}{zan2}{9}{⼝}[HSK 2]
  \definition{pron.}{nós; nos (incluindo tanto o falante quanto a pessoa ou pessoas às quais se dirige) | eu; mim}
  \seeref{za2}
  \seeref{zan5}
\end{EntryWithPhonetic}

\begin{EntryWithPhonetic}{咱俩}{zan2lia3}{9,9}{⼝,⼈}
  \definition{pron.}{nós dois}
\end{EntryWithPhonetic}

\begin{EntryWithPhonetic}{咱们}{zan2men5}{9,5}{⼝,⼈}[HSK 2]
  \definition{pron.}{dirige"-se tanto ao falante (eu, nós) quanto ao ouvinte (você, vocês) | eu; mim; refere"-se ao próprio orador, eu}
\end{EntryWithPhonetic}

%%%%%%%%%% 暂 %%%%%%%%%%
\subsection*{暂}\addcontentsline{loh}{figure}{暂 \dpy{zan4}}

\begin{EntryWithPhonetic}{暂}{zan4}{12}{⽇}
  \definition{adj.}{de curta duração | curto; momentâneo; pouco tempo}
  \definition{adv.}{temporariamente; por enquanto}
  \antonymref{久}{jiu3}
\end{EntryWithPhonetic}

\begin{EntryWithPhonetic}{暂时}{zan4shi2}{12,7}{⽇,⽇}[HSK 5]
  \definition{adj.}{transitório; temporário}
  \definition{adv.}{por enquanto; em pouco tempo}
\end{EntryWithPhonetic}

\begin{EntryWithPhonetic}{暂停}{zan4ting2}{12,11}{⽇,⼈}[HSK 5]
  \definition{s.}{suspensão temporária; refere"-se especificamente à suspensão temporária de certas competições desportivas de acordo com as regras}
  \definition{v.}{pausar; suspender; esgotar o tempo}
\end{EntryWithPhonetic}

%%%%%%%%%% 赞 %%%%%%%%%%
\subsection*{赞}\addcontentsline{loh}{figure}{赞 \dpy{zan4}}

\begin{EntryWithPhonetic}{赞}{zan4}{16}{⾙}
  \definition{v.}{patrocinar | apoiar | elogiar | (gíria na \emph{Internet}) para curtir (uma postagem \emph{on-line})}
\end{EntryWithPhonetic}

\begin{EntryWithPhonetic}{赞成}{zan4cheng2}{16,6}{⾙,⼽}[HSK 4]
  \definition{v.}{endossar; favorecer; aprovar; concordar com; concordar ou apoiar as ideias, os planos, as propostas ou o comportamento de outra pessoa}
\end{EntryWithPhonetic}

\begin{EntryWithPhonetic}{赞赏}{zan4shang3}{16,12}{⾙,⾙}[HSK 4]
  \definition{v.}{admirar; apreciar; valorizar}
\end{EntryWithPhonetic}

\begin{EntryWithPhonetic}{赞扬}{zan4yang2}{16,6}{⾙,⼿}
  \definition{v.}{elogiar | aprovar | demonstrar aprovação}
\end{EntryWithPhonetic}

\begin{EntryWithPhonetic}{赞助}{zan4zhu4}{16,7}{⾙,⼒}[HSK 4]
  \definition{v.}{apoiar; patrocinar; concordar e ajudar (refere"-se principalmente a oferecer dinheiro para ajudar)}
\end{EntryWithPhonetic}

%%%%%%%%%% 咱 %%%%%%%%%%
\subsection*{咱}\addcontentsline{loh}{figure}{咱 \dpy{zan5}}

\begin{EntryWithPhonetic}{咱}{zan5}{9}{⼝}
  \definition{adv.}{quando; agora; então; naquele momento; usado em 这咱, 那咱, 多咱, uma combinação das duas palavras 早晚}
  \seeref{za2}
  \seeref{zan2}
  \seealsoref{多咱}{duo1 zan5}
  \seealsoref{那咱}{na4 zan5}
  \seealsoref{早晚}{zao3wan3}
  \seealsoref{这咱}{zhe4 zan5}
\end{EntryWithPhonetic}

%%%%%%%%%% 脏 %%%%%%%%%%
\subsection*{脏}\addcontentsline{loh}{figure}{脏 \dpy{zang1}}

\begin{EntryWithPhonetic}{脏}{zang1}{10}{⾁}
  \definition{adj.}{sujo; imundo | imundo; metáfora para vulgaridade e obscenidade}
  \definition{v.}{tornar algo sujo ou impuro}
  \seeref{zang4}
\end{EntryWithPhonetic}

\begin{EntryWithPhonetic}{脏辫}{zang1bian4}{10,17}{⾁,⾟}
  \definition{s.}{\emph{dreadlocks}}
\end{EntryWithPhonetic}

\begin{EntryWithPhonetic}{脏病}{zang1bing4}{10,10}{⾁,⽧}
  \definition{s.}{doença venérea}
\end{EntryWithPhonetic}

\begin{EntryWithPhonetic}{脏煤}{zang1mei2}{10,13}{⾁,⽕}
  \definition{s.}{carvão sujo | sujeira (de uma mina de carvão)}
\end{EntryWithPhonetic}

\begin{EntryWithPhonetic}{脏土}{zang1tu3}{10,3}{⾁,⼟}
  \definition{s.}{solo sujo | lama | lixo}
\end{EntryWithPhonetic}

\begin{EntryWithPhonetic}{脏脏}{zang1zang1}{10,10}{⾁,⾁}
  \definition{adj.}{sujo}
\end{EntryWithPhonetic}

\begin{EntryWithPhonetic}{脏字}{zang1zi4}{10,6}{⾁,⼦}
  \definition{s.}{obscenidade}
\end{EntryWithPhonetic}

\begin{EntryWithPhonetic}{脏}{zang4}{10}{⾁}[HSK 2]
  \definition[处]{s.}{vísceras; órgãos internos do corpo, geralmente o coração, o fígado, o baço, os pulmões e os rins; um termo geral para órgãos nas cavidades torácica e abdominal de humanos ou animais | (anatomia) órgão; a medicina tradicional chinesa chama o coração, o fígado, o baço, os pulmões e os rins de órgãos internos}
  \seeref{zang1}
\end{EntryWithPhonetic}

\begin{EntryWithPhonetic}{脏器}{zang4qi4}{10,16}{⾁,⼝}
  \definition{s.}{órgãos internos}
\end{EntryWithPhonetic}

%%%%%%%%%% 葬 %%%%%%%%%%
\subsection*{葬}\addcontentsline{loh}{figure}{葬 \dpy{zang4}}

\begin{EntryWithPhonetic}{葬}{zang4}{12}{⾋}
  \definition{v.}{enterrar (os mortos) | sepultar}
\end{EntryWithPhonetic}

%%%%%%%%%% 藏 %%%%%%%%%%
\subsection*{藏}\addcontentsline{loh}{figure}{藏 \dpy{zang4}}

\begin{EntryWithPhonetic}{藏}{zang4}{17}{⾋}
  \definition*{s.}{Escrituras budistas ou taoístas; um termo geral para clássicos budistas ou taoístas | Região Autônoma do Tibete, 西藏}
  \definition{s.}{depósito; local de armazenamento; armazém; local onde grandes quantidades de coisas são armazenadas | os tibetanos, 藏族; grupo étnico Zang (ou tibetano)}
  \seeref{cang2}
  \seealsoref{西藏}{xi1zang4}
  \seealsoref{藏族}{zang4zu2}
\end{EntryWithPhonetic}

\begin{EntryWithPhonetic}{藏族}{zang4zu2}{17,11}{⾋,⽅}
  \definition*{s.}{Etnia Zang (ou tibetana); Os Zangs (ou tibetanos) , distribuídos pela Região Autônoma do Tibete e pelas províncias de Qinghai, Sichuan, Gansu e Yunnan}
\end{EntryWithPhonetic}

%%%%%%%%%% 遭 %%%%%%%%%%
\subsection*{遭}\addcontentsline{loh}{figure}{遭 \dpy{zao1}}

\begin{EntryWithPhonetic}{遭}{zao1}{14}{⾡}
  \definition{clas.}{tempo; vez; ocasião | rodadas}
  \definition{v.}{encontrar-se com (desastre, infortúnio, etc.); sofrer}
\end{EntryWithPhonetic}

\begin{EntryWithPhonetic}{遭到}{zao1dao4}{14,8}{⾡,⼑}[HSK 6]
  \definition{v.}{sofrer; ser rejeitado; receber crítica; significa sofrer infortúnio ou dano}[我们遭到意外事故。===Nós sofremos um acidente.]
\end{EntryWithPhonetic}

\begin{EntryWithPhonetic}{遭受}{zao1shou4}{14,8}{⾡,⼜}[HSK 6]
  \definition{v.}{sofrer; aguentar; ser submetido a; encontrar ou vivenciar coisas dolorosas que você não quer que aconteçam}
\end{EntryWithPhonetic}

\begin{EntryWithPhonetic}{遭遇}{zao1yu4}{14,12}{⾡,⾡}[HSK 6]
  \definition[场,次,种,段]{s.}{sorte (difícil); experiência (amarga); encontrando coisas ruins}
  \definition{v.}{encontrar; encontrar-se com; esbarrar em; encontros inesperados com pessoas ou coisas que não são boas para você}
\end{EntryWithPhonetic}

%%%%%%%%%% 糟 %%%%%%%%%%
\subsection*{糟}\addcontentsline{loh}{figure}{糟 \dpy{zao1}}

\begin{EntryWithPhonetic}{糟}{zao1}{17}{⽶}[HSK 5]
  \definition{adj.}{pobre; apodrecido; deteriorado | estragado; em uma bagunça; em um estado miserável (terrível) | (situação ou circunstância) ruim; desfavorável}
  \definition{s.}{resíduos de destilação de bebidas alcoólicas; resíduos do processo de fermentação do vinho}
  \definition{v.}{marinar alimentos em vinho ou mosto | desperdiçar; estragar; destruir}
\end{EntryWithPhonetic}

\begin{EntryWithPhonetic}{糟糕}{zao1gao1}{17,16}{⽶,⽶}[HSK 5]
  \definition{adj.}{(corpo, situação, etc.) muito ruim, péssimo}
  \definition{interj.}{``Que terrível!''; ``Que má sorte!''; ``Muito ruim!''}
\end{EntryWithPhonetic}

%%%%%%%%%% 早 %%%%%%%%%%
\subsection*{早}\addcontentsline{loh}{figure}{早 \dpy{zao3}}

\begin{EntryWithPhonetic}{早}{zao3}{6}{⽇}[HSK 1]
  \definition{adj.}{precoce; antes do previsto ou planejado; antes do tempo; antes de um determinado momento}
  \definition{adv.}{há muito tempo; desde cedo; por muito tempo; há muito tempo atrás}
  \definition{interj.}{``Bom dia!''; ``Saudações!''; usadas para cumprimentar uns aos outros ao se encontrarem pela manhã}
  \definition[个]{s.}{manhã}
\end{EntryWithPhonetic}

\begin{EntryWithPhonetic}{早安}{zao3'an1}{6,6}{⽇,⼧}
  \definition{interj.}{``Bom dia!''}
\end{EntryWithPhonetic}

\begin{EntryWithPhonetic}{早餐}{zao3can1}{6,16}{⽇,⾷}[HSK 2]
  \definition[份,桌,顿]{s.}{café da manhã; desejum}
\end{EntryWithPhonetic}

\begin{EntryWithPhonetic}{早车}{zao3che1}{6,4}{⽇,⾞}
  \definition{s.}{trem matutino | ônibus matutino}
\end{EntryWithPhonetic}

\begin{EntryWithPhonetic}{早晨}{zao3chen5}{6,11}{⽇,⽇}[HSK 2]
  \definition[个,段,番]{s.}{manhã cedo; manhãzinha; o período do amanhecer às oito ou nove horas; às vezes, o período da meia-noite ao meio-dia}
\end{EntryWithPhonetic}

\begin{EntryWithPhonetic}{早饭}{zao3fan4}{6,7}{⽇,⾷}[HSK 1]
  \definition[份,顿]{s.}{o café da manhã}
\end{EntryWithPhonetic}

\begin{EntryWithPhonetic}{早就}{zao3jiu4}{6,12}{⽇,⼪}[HSK 2]
  \definition{adv.}{já; há muito tempo; há muito tempo atrás}
\end{EntryWithPhonetic}

\begin{EntryWithPhonetic}{早期}{zao3qi1}{6,12}{⽇,⽉}[HSK 5]
  \definition{s.}{prófase; estágio inicial; fase inicial; a fase inicial de uma determinada época, processo ou vida de uma pessoa}
\end{EntryWithPhonetic}

\begin{EntryWithPhonetic}{早前}{zao3qian2}{6,9}{⽇,⼑}
  \definition{adv.}{previamente}
\end{EntryWithPhonetic}

\begin{EntryWithPhonetic}{早上}{zao3shang5}{6,3}{⽇,⼀}[HSK 1]
  \definition[个]{s.}{de manhã cedo; madrugada; o período antes e depois do nascer do sol; geralmente, desde o amanhecer até às 8h ou 9h da manhã; às vezes também se refere ao período entre o amanhecer e o meio-dia}
\end{EntryWithPhonetic}

\begin{EntryWithPhonetic}{早晚}{zao3wan3}{6,11}{⽇,⽇}[HSK 6]
  \definition{adv./s.}{manhã e noite | mais cedo ou mais tarde; cedo ou tarde | algum tempo no futuro; algum dia; em algum momento no futuro}
\end{EntryWithPhonetic}

\begin{EntryWithPhonetic}{早亡}{zao3wang2}{6,3}{⽇,⼇}
  \definition[个]{s.}{morte prematura}
  \definition{v.}{morrer prematuramente}
\end{EntryWithPhonetic}

\begin{EntryWithPhonetic}{早已}{zao3yi3}{6,3}{⽇,⼰}[HSK 3]
  \definition{adv.}{há muito tempo; por muito tempo | (dialeto) no passado}
\end{EntryWithPhonetic}

\begin{EntryWithPhonetic}{早早儿}{zao3zao3r5}{6,6,2}{⽇,⽇,⼉}
  \definition{adv.}{o mais cedo possível | o mais breve possível}
\end{EntryWithPhonetic}

\begin{EntryWithPhonetic}{早知}{zao3zhi1}{6,8}{⽇,⽮}
  \definition{v.}{prever | se alguém soubesse antes, \dots}
\end{EntryWithPhonetic}

%%%%%%%%%% 灶 %%%%%%%%%%
\subsection*{灶}\addcontentsline{loh}{figure}{灶 \dpy{zao4}}

\begin{EntryWithPhonetic}{灶}{zao4}{7}{⽕}
  \definition[口,个]{s.}{fogão de cozinha; fogão de cozinha | cozinha; bagunça; cantina}
\end{EntryWithPhonetic}

\begin{EntryWithPhonetic}{灶台}{zao4tai2}{7,5}{⽕,⼝}
  \definition{s.}{fogão}
\end{EntryWithPhonetic}

%%%%%%%%%% 造 %%%%%%%%%%
\subsection*{造}\addcontentsline{loh}{figure}{造 \dpy{zao4}}

\begin{EntryWithPhonetic}{造}{zao4}{10}{⾡}[HSK 3]
  \definition*{s.}{Sobrenome: Zao}
  \definition{clas.}{para colheitas ou número de colheitas de safras}
  \definition{s.}{uma das duas partes em um acordo legal ou um processo judicial | (dialeto) colheita; safra | realizações; conquistas}
  \definition{v.}{fazer; construir; criar; produzir | forjar; inventar | correr solto; bagunçar as coisas | expor sem restrições |  treinar; educar | fabricar | alcançar; atingir}
\end{EntryWithPhonetic}

\begin{EntryWithPhonetic}{造成}{zao4cheng2}{10,6}{⾡,⼽}[HSK 3]
  \definition{v.}{criar; dar origem a; provocar; causar (geralmente se refere a resultados negativos)}
\end{EntryWithPhonetic}

\begin{EntryWithPhonetic}{造型}{zao4xing2}{10,9}{⾡,⼟}[HSK 4]
  \definition[个,种]{s.}{molde; modelo; formato; forma; moldagem}
  \definition{v.}{modelar; moldar}
\end{EntryWithPhonetic}

%%%%%%%%%% 艁 %%%%%%%%%%
\subsection*{艁}\addcontentsline{loh}{figure}{艁 \dpy{zao4}}

\begin{EntryWithPhonetic}{艁}{zao4}{13}{⾈}
  \variantof{造}
\end{EntryWithPhonetic}

%%%%%%%%%% 责 %%%%%%%%%%
\subsection*{责}\addcontentsline{loh}{figure}{责 \dpy{ze2}}

\begin{EntryWithPhonetic}{责}{ze2}{8}{⾙}
  \definition{s.}{dever; responsabilidade}
  \definition{v.}{exigir; requerer; exigir que algo seja feito ou que atenda a certos padrões | questionar atentamente; chamar alguém para prestar contas; interrogar| reprovar; culpar | punir}
\end{EntryWithPhonetic}

\begin{EntryWithPhonetic}{责怪}{ze2guai4}{8,8}{⾙,⼼}
  \definition{v.}{repreender | censurar}
\end{EntryWithPhonetic}

\begin{EntryWithPhonetic}{责任}{ze2ren4}{8,6}{⾙,⼈}[HSK 3]
  \definition[个,种,份]{s.}{dever; responsabilidade; de acordo com a profissão, cargo, identidade, etc., as coisas que você deve fazer ou as tarefas que deve assumir | culpa; responsabilidade por uma falha ou erro; não ter feito o que era sua obrigação e, portanto, ser responsável pela falha}
\end{EntryWithPhonetic}

%%%%%%%%%% 怎 %%%%%%%%%%
\subsection*{怎}\addcontentsline{loh}{figure}{怎 \dpy{zen3}}

\begin{EntryWithPhonetic}{怎}{zen3}{9}{⼼}
  \definition{adv.}{como}
\end{EntryWithPhonetic}

\begin{EntryWithPhonetic}{怎么}{zen3me5}{9,3}{⼼,⼃}[HSK 1]
  \definition{pron.}{como?; o quê?; perguntas sobre natureza, situação, método, motivo, etc. | de qualquer maneira; não importa como; de uma certa maneira; referência geral à natureza, condição ou modo | que? (usado sozinho no início de uma frase para expressar surpresa) | usado após 不 e 没, indica um grau baixo e é uma forma mais educada de se expressar | usado em perguntas retóricas}
  \seealsoref{不}{bu4}
  \seealsoref{没}{mei2}
\end{EntryWithPhonetic}

\begin{EntryWithPhonetic}{怎么办}{zen3me5ban4}{9,3,4}{⼼,⼃,⼒}[HSK 2]
  \definition{adv.}{o que fazer?; o que deve ser feito?}
\end{EntryWithPhonetic}

\begin{EntryWithPhonetic}{怎么得了}{zen3me5de2liao3}{9,3,11,2}{⼼,⼃,⼻,⼅}
  \definition{expr.}{Como isso pode ser? | Que bagunça horrível! | O que deve ser feito?}
\end{EntryWithPhonetic}

\begin{EntryWithPhonetic}{怎么搞的}{zen3me5gao3de5}{9,3,13,8}{⼼,⼃,⼿,⽩}
  \definition{expr.}{Como isso aconteceu? | O que deu errado? | E aí? | O que está errado?}
\end{EntryWithPhonetic}

\begin{EntryWithPhonetic}{怎么回事}{zen3me5hui2shi4}{9,3,6,8}{⼼,⼃,⼞,⼅}
  \definition{expr.}{O que aconteceu? | O que se passou?}
\end{EntryWithPhonetic}

\begin{EntryWithPhonetic}{怎么了}{zen3me5le5}{9,3,2}{⼼,⼃,⼅}
  \definition{expr.}{O que aconteceu? | O que está acontecendo? | E aí?}
\end{EntryWithPhonetic}

\begin{EntryWithPhonetic}{怎么样}{zen3me5yang4}{9,3,10}{⼼,⼃,⽊}[HSK 2]
  \definition{adv.}{como?; o que?; como é?; como estão as coisas?; o que você acha?; pergunte sobre o método, natureza, situação, opinião, etc. | substitui uma ação ou situação não dita (usado apenas na forma negativa, mais eufemístico do que uma declaração direta); indaga sobre a natureza, condição, método, razão, etc.}
\end{EntryWithPhonetic}

\begin{EntryWithPhonetic}{怎样}{zen3yang4}{9,10}{⼼,⽊}[HSK 2]
  \definition{pron.}{como?; o que?; indagar sobre a natureza, condição ou método, etc. | como?; indica uma referência virtual | de uma certa maneira; de qualquer maneira; não importa como; indica qualquer | como?; usado como predicado, objeto ou complemento para indagar sobre uma situação}
\end{EntryWithPhonetic}

%%%%%%%%%% 曾 %%%%%%%%%%
\subsection*{曾}\addcontentsline{loh}{figure}{曾 \dpy{zeng1}}

\begin{EntryWithPhonetic}{曾}{zeng1}{12}{⽈}
  \definition*{s.}{Sobrenome: Zeng}
  \definition{s.}{relacionamento entre bisnetos e bisavós; (parentesco) duas gerações de diferença}
  \seeref{ceng2}
\end{EntryWithPhonetic}

%%%%%%%%%% 增 %%%%%%%%%%
\subsection*{增}\addcontentsline{loh}{figure}{增 \dpy{zeng1}}

\begin{EntryWithPhonetic}{增}{zeng1}{15}{⼟}[HSK 5]
  \definition*{s.}{Sobrenome: Zeng}
  \definition{v.}{aumentar; ganhar; adicionar}
\end{EntryWithPhonetic}

\begin{EntryWithPhonetic}{增产}{zeng1/chan3}{15,6}{⼟,⼇}[HSK 5]
  \definition{v.+compl.}{aumentar a produção}
\end{EntryWithPhonetic}

\begin{EntryWithPhonetic}{增大}{zeng1da4}{15,3}{⼟,⼤}[HSK 5]
  \definition{v.}{ampliar; expandir; estender | amplificar}
\end{EntryWithPhonetic}

\begin{EntryWithPhonetic}{增多}{zeng1duo1}{15,6}{⼟,⼣}[HSK 5]
  \definition{v.}{aumentar; crescer em número ou quantidade}
\end{EntryWithPhonetic}

\begin{EntryWithPhonetic}{增加}{zeng1jia1}{15,5}{⼟,⼒}[HSK 3]
  \definition{v.}{adicionar; aumentar; incrementar; adicionar mais ao que já existe}
\end{EntryWithPhonetic}

\begin{EntryWithPhonetic}{增进}{zeng1jin4}{15,7}{⼟,⾡}[HSK 6]
  \definition{v.}{melhorar; promover; aprofundar}
\end{EntryWithPhonetic}

\begin{EntryWithPhonetic}{增强}{zeng1qiang2}{15,12}{⼟,⼸}[HSK 5]
  \definition{v.}{impulsionar; aprimorar; aumentar; fortalecer; tornar mais forte ou mais poderoso}
\end{EntryWithPhonetic}

\begin{EntryWithPhonetic}{增速}{zeng1su4}{15,10}{⼟,⾡}
  \definition{s.}{Economia: taxa de crescimento}
  \definition{v.}{acelerar; aumentar a velocidade}
\end{EntryWithPhonetic}

\begin{EntryWithPhonetic}{增长}{zeng1zhang3}{15,4}{⼟,⾧}[HSK 3]
  \definition{v.}{subir; crescer; aumentar; melhorar a partir da base existente}
\end{EntryWithPhonetic}

\begin{EntryWithPhonetic}{增值}{zeng1zhi2}{15,10}{⼟,⼈}[HSK 6]
  \definition{s.}{aumento de valor; apreciação; incremento | valor agregado}
\end{EntryWithPhonetic}

%%%%%%%%%% 综 %%%%%%%%%%
\subsection*{综}\addcontentsline{loh}{figure}{综 \dpy{zeng4}}

\begin{EntryWithPhonetic}{综}{zeng4}{11}{⽷}
  \definition{s.}{liço; fuso; um dispositivo em um tear que separa os fios da urdidura em um padrão alternado para permitir a passagem da lançadeira}
  \seeref{zong1}
\end{EntryWithPhonetic}

%%%%%%%%%% 赠 %%%%%%%%%%
\subsection*{赠}\addcontentsline{loh}{figure}{赠 \dpy{zeng4}}

\begin{EntryWithPhonetic}{赠}{zeng4}{16}{⾙}[HSK 5]
  \definition{s.}{um presente (de despedida), uma lembrança; honrarias póstumas; uma patente de título}
  \definition{v.}{dar um presente; presentear com um brinde}
\end{EntryWithPhonetic}

\begin{EntryWithPhonetic}{赠送}{zeng4song4}{16,9}{⾙,⾡}[HSK 5]
  \definition{v.}{dar; dar de presente; dar algo de graça a alguém}
\end{EntryWithPhonetic}

%%%%%%%%%% 扎 %%%%%%%%%%
\subsection*{扎}\addcontentsline{loh}{figure}{扎 \dpy{zha1}}

\begin{EntryWithPhonetic}{扎}{zha1}{4}{⼿}[HSK 6]
  \definition{s.}{chope; cerveja de pressão | caneca para chope}
  \definition{v.}{furar; esfaquear; enfiar (uma agulha, etc.) em | estacionar; aquartelar | entrar em; mergulhar em | trapacear; defraudar}
  \seeref{za1}
  \seeref{zha2}
\end{EntryWithPhonetic}

\begin{EntryWithPhonetic}{扎实}{zha1shi5}{4,8}{⼿,⼧}[HSK 6]
  \definition{adj.}{robusto; forte; sólido; confiável | sólido; pé no chão; prático}
\end{EntryWithPhonetic}

%%%%%%%%%% 查 %%%%%%%%%%
\subsection*{查}\addcontentsline{loh}{figure}{查 \dpy{zha1}}

\begin{EntryWithPhonetic}{查}{zha1}{9}{⽊}
  \definition*{s.}{Sobrenome: Zha}
  \definition{s.}{espinheiro-chinês}
  \seeref{cha2}
\end{EntryWithPhonetic}

%%%%%%%%%% 扎 %%%%%%%%%%
\subsection*{扎}\addcontentsline{loh}{figure}{扎 \dpy{zha2}}

\begin{EntryWithPhonetic}{扎}{zha2}{4}{⼿}
  \definition{v.}{lutar}
  \seeref{za1}
  \seeref{zha1}
\end{EntryWithPhonetic}

%%%%%%%%%% 闸 %%%%%%%%%%
\subsection*{闸}\addcontentsline{loh}{figure}{闸 \dpy{zha2}}

\begin{EntryWithPhonetic}{闸}{zha2}{8}{⾨}
  \definition[个,道]{s.}{comporta; comporta | freio | (coloquial) interruptor}
  \definition{v.}{represar um córrego, rio, etc. | represar a água; parar a água}
\end{EntryWithPhonetic}

\begin{EntryWithPhonetic}{闸门}{zha2men2}{8,3}{⾨,⾨}
  \definition{s.}{eclusa | comporta}
\end{EntryWithPhonetic}

%%%%%%%%%% 炸 %%%%%%%%%%
\subsection*{炸}\addcontentsline{loh}{figure}{炸 \dpy{zha2}}

\begin{EntryWithPhonetic}{炸}{zha2}{9}{⽕}
  \definition{v.}{explodir; estourar; romper | dinamitar; bombardear; explodir; detonar com explosivos | encolerizar-se; explodir em fúria | correr; fugir em pânico}
  \seeref{zha4}
\end{EntryWithPhonetic}

\begin{EntryWithPhonetic}{炸}{zha4}{9}{⽕}[HSK 6]
  \definition{v.}{fritar em gordura ou óleo | escaldar (como forma de cozinhar)}
  \seeref{zha2}
\end{EntryWithPhonetic}

\begin{EntryWithPhonetic}{炸弹}{zha4dan4}{9,11}{⽕,⼸}[HSK 6]
  \definition{s.}{bomba; uma arma com invólucro de ferro e explosivos dentro que explodem quando um fusível é acionado, geralmente lançada de um avião}
\end{EntryWithPhonetic}

\begin{EntryWithPhonetic}{炸药}{zha4yao4}{9,9}{⽕,⾋}[HSK 6]
  \definition[包,种]{s.}{explosivo; cargas explosivas; dinamite; substâncias que explodem quando aquecidas ou impactadas, produzindo grandes quantidades de energia e gases de alta temperatura, como dinamite e pólvora negra}
\end{EntryWithPhonetic}

%%%%%%%%%% 蜡 %%%%%%%%%%
\subsection*{蜡}\addcontentsline{loh}{figure}{蜡 \dpy{zha4}}

\begin{EntryWithPhonetic}{蜡}{zha4}{14}{⾍}
  \definition{s.}{uma antiga cerimônia de sacrifício de fim de ano}
  \seeref{la4}
\end{EntryWithPhonetic}

%%%%%%%%%% 侧 %%%%%%%%%%
\subsection*{侧}\addcontentsline{loh}{figure}{侧 \dpy{zhai1}}

\begin{EntryWithPhonetic}{侧}{zhai1}{8}{⼈}
  \definition{adj.}{inclinado; torto}
  \seeref{ce4}
\end{EntryWithPhonetic}

%%%%%%%%%% 摘 %%%%%%%%%%
\subsection*{摘}\addcontentsline{loh}{figure}{摘 \dpy{zhai1}}

\begin{EntryWithPhonetic}{摘}{zhai1}{14}{⼿}[HSK 5]
  \definition{v.}{pegar; arrancar; tirar; colher (flores, frutos, folhas de plantas); retirar (coisas que estão sendo usadas ou penduradas) | selecionar; fazer extrações de | pedir dinheiro emprestado em caso de necessidade urgente | vencer; ganhar; alcançar; obter}
\end{EntryWithPhonetic}

%%%%%%%%%% 窄 %%%%%%%%%%
\subsection*{窄}\addcontentsline{loh}{figure}{窄 \dpy{zhai3}}

\begin{EntryWithPhonetic}{窄}{zhai3}{10}{⽳}
  \definition{adj.}{estreito; pequena distância horizontal | mesquinho; estreito; (mente) não alegre; (capacidade) pequena | difícil; mal; falta de; (vida) não bem de vida}
\end{EntryWithPhonetic}

%%%%%%%%%% 债 %%%%%%%%%%
\subsection*{债}\addcontentsline{loh}{figure}{债 \dpy{zhai4}}

\begin{EntryWithPhonetic}{债}{zhai4}{10}{⼈}[HSK 6]
  \definition[笔]{s.}{dívida | empréstimo}
\end{EntryWithPhonetic}

%%%%%%%%%% 祭 %%%%%%%%%%
\subsection*{祭}\addcontentsline{loh}{figure}{祭 \dpy{zhai4}}

\begin{EntryWithPhonetic}{祭}{zhai4}{11}{⽰}
  \definition*{s.}{Sobrenome: Zhai}
  \seeref{ji4}
\end{EntryWithPhonetic}

%%%%%%%%%% 寨 %%%%%%%%%%
\subsection*{寨}\addcontentsline{loh}{figure}{寨 \dpy{zhai4}}

\begin{EntryWithPhonetic}{寨}{zhai4}{14}{⼧}
  \definition{s.}{fortaleza | paliçada | acampamento | vila (paliçada)}
\end{EntryWithPhonetic}

%%%%%%%%%% 占 %%%%%%%%%%
\subsection*{占}\addcontentsline{loh}{figure}{占 \dpy{zhan1}}

\begin{EntryWithPhonetic}{占}{zhan1}{5}{⼘}
  \definition*{s.}{Sobrenome: Zhan}
  \definition{v.}{praticar adivinhação; antigamente, as pessoas usavam cascos de tartaruga e mil-folhas para prever boa ou má sorte; mais tarde, a palavra passou a se referir à previsão de boa ou má sorte por vários meios}
  \seeref{zhan4}
\end{EntryWithPhonetic}

%%%%%%%%%% 斩 %%%%%%%%%%
\subsection*{斩}\addcontentsline{loh}{figure}{斩 \dpy{zhan3}}

\begin{EntryWithPhonetic}{斩}{zhan3}{8}{⽄}
  \definition*{s.}{Sobrenome: Zhan}
  \definition{v.}{matar; cortar; picar | (dialeto) tosquiar; chantagear | decapitar}
\end{EntryWithPhonetic}

\begin{EntryWithPhonetic}{斩获}{zhan3huo4}{8,10}{⽄,⾋}
  \definition{v.}{matar ou capturar (em batalha) | (figurativo) (esportes) marcar (um gol), ganhar (uma medalha) | (figurativo) colher recompensas, obter ganhos}
\end{EntryWithPhonetic}

%%%%%%%%%% 展 %%%%%%%%%%
\subsection*{展}\addcontentsline{loh}{figure}{展 \dpy{zhan3}}

\begin{EntryWithPhonetic}{展}{zhan3}{10}{⼫}
  \definition*{s.}{Sobrenome: Zhan}
  \definition{s.}{exposição}
  \definition{v.}{abrir; espalhar; desdobrar | fazer bom uso; dar liberdade para | adiar; estender; prolongar | expandir | abrir; deixar ir | exibir; mostrar}
\end{EntryWithPhonetic}

\begin{EntryWithPhonetic}{展开}{zhan3kai1}{10,4}{⼫,⼶}[HSK 3]
  \definition{s.}{desenvolvimento; expansão; explosão; evolução}
  \definition{v.}{espalhar; desdobrar; abrir | lançar; desdobrar; desenvolver; realizar em grande escala | espalhar; desenrolar; amplificar; desenvolver; expandir; explodir; evoluir; alongar}
\end{EntryWithPhonetic}

\begin{EntryWithPhonetic}{展览}{zhan3lan3}{10,9}{⼫,⾒}[HSK 5]
  \definition[个,次,场]{s.}{exposição; exibição; atividades expostas; itens expostos}
  \definition{v.}{mostrar; exibir; expor; expor algo para que as pessoas vejam}
\end{EntryWithPhonetic}

\begin{EntryWithPhonetic}{展示}{zhan3shi4}{10,5}{⼫,⽰}[HSK 5]
  \definition{v.}{mostrar; revelar; pôr a nu; abrir diante de alguém; expor claramente; manifestar de forma evidente}
\end{EntryWithPhonetic}

\begin{EntryWithPhonetic}{展现}{zhan3xian4}{10,8}{⼫,⾒}[HSK 5]
  \definition{v.}{mostrar; surgir; manifestar}
\end{EntryWithPhonetic}

%%%%%%%%%% 盏 %%%%%%%%%%
\subsection*{盏}\addcontentsline{loh}{figure}{盏 \dpy{zhan3}}

\begin{EntryWithPhonetic}{盏}{zhan3}{10}{⽫}
  \definition{clas.}{usado para lâmpadas, iluminação}[一盏煤油灯。===Uma lamparina de querosene.]
  \definition{s.}{copo pequeno}
\end{EntryWithPhonetic}

%%%%%%%%%% 辗 %%%%%%%%%%
\subsection*{辗}\addcontentsline{loh}{figure}{辗 \dpy{zhan3}}

\begin{EntryWithPhonetic}{辗}{zhan3}{14}{⾞}
  \definition{v.}{(arcaico) virar | (arcaico) rolar para o lado | (arcaico) virar a metade}
\end{EntryWithPhonetic}

%%%%%%%%%% 占 %%%%%%%%%%
\subsection*{占}\addcontentsline{loh}{figure}{占 \dpy{zhan4}}

\begin{EntryWithPhonetic}{占}{zhan4}{5}{⼘}[HSK 2]
  \definition{v.}{tomar; apreender; ocupar; obter e possuir (terra, lugar, etc.) pela força ou outros meios impróprios | manter; inventar; constituir; explicar; estar em (uma certa posição); pertencer a (uma certa situação) | usar; ocupar; tomar; possuir}
  \seeref{zhan1}
\end{EntryWithPhonetic}

\begin{EntryWithPhonetic}{占据}{zhan4ju4}{5,11}{⼘,⼿}[HSK 6]
  \definition{v.}{segurar; ocupar; assumir; tomar ou ocupar à força (uma região, lugar, etc.)}
\end{EntryWithPhonetic}

\begin{EntryWithPhonetic}{占领}{zhan4ling3}{5,11}{⼘,⾴}[HSK 5]
  \definition{v.}{manter; tomar; ocupar; capturar; conquistar (posições ou territórios) com forças armadas | ocupar; capturar; possuir}
\end{EntryWithPhonetic}

\begin{EntryWithPhonetic}{占有}{zhan4you3}{5,6}{⼘,⽉}[HSK 5]
  \definition{v.}{possuir; ter; ocupar e possuir | manter; ocupar; estar em (uma determinada posição) | possuir; deter; ter; dominar}
\end{EntryWithPhonetic}

%%%%%%%%%% 战 %%%%%%%%%%
\subsection*{战}\addcontentsline{loh}{figure}{战 \dpy{zhan4}}

\begin{EntryWithPhonetic}{战}{zhan4}{9}{⼽}
  \definition{s.}{luta | guerra | batalha}
  \definition{v.}{lutar}
\end{EntryWithPhonetic}

\begin{EntryWithPhonetic}{战场}{zhan4chang3}{9,6}{⼽,⼟}[HSK 6]
  \definition[个,片,处]{s.}{campo de batalha; frente de batalha}
\end{EntryWithPhonetic}

\begin{EntryWithPhonetic}{战斗}{zhan4dou4}{9,4}{⼽,⽃}[HSK 4]
  \definition[场,次]{s.}{luta; batalha; combate; ação; conflito armado entre as partes oponentes}
  \definition{v.}{lutar; combater | lutar; metáfora para trabalho duro ou labor}
\end{EntryWithPhonetic}

\begin{EntryWithPhonetic}{战略}{zhan4lve4}{9,11}{⼽,⽥}[HSK 6]
  \definition[个,条]{s.}{estratégia; uma estratégia que orienta todo o processo de guerra (diferente de 战术) | estratégia; refere"-se a uma diretriz geral}
  \seealsoref{战术}{zhan4shu4}
\end{EntryWithPhonetic}

\begin{EntryWithPhonetic}{战胜}{zhan4sheng4}{9,9}{⼽,⾁}[HSK 4]
  \definition{v.}{derrotar; vencer; superar; triunfar sobre; metáfora para superar dificuldades e alcançar o sucesso}
\end{EntryWithPhonetic}

\begin{EntryWithPhonetic}{战士}{zhan4shi4}{9,3}{⼽,⼠}[HSK 4]
  \definition[个,些,名,位]{s.}{soldado; membros mais jovens do exército | campeão; guerreiro; lutador; geralmente, uma pessoa que se engaja em alguma causa justa ou participa de alguma luta justa}
\end{EntryWithPhonetic}

\begin{EntryWithPhonetic}{战术}{zhan4shu4}{9,5}{⼽,⽊}[HSK 6]
  \definition[种,套]{s.}{táticas para resolver problemas; geralmente se refere ao método de resolução de problemas locais | táticas (militares); princípios e métodos de condução de combate}
\end{EntryWithPhonetic}

\begin{EntryWithPhonetic}{战友}{zhan4you3}{9,4}{⼽,⼜}[HSK 6]
  \definition{s.}{camarada de armas; companheiro de guerra | companheiro de batalha; pessoas lutando juntas}
\end{EntryWithPhonetic}

\begin{EntryWithPhonetic}{战争}{zhan4zheng1}{9,6}{⼽,⼑}[HSK 4]
  \definition[场,次]{s.}{guerra; conflito; luta armada entre povos, entre nações, entre classes ou entre grupos políticos}
\end{EntryWithPhonetic}

%%%%%%%%%% 站 %%%%%%%%%%
\subsection*{站}\addcontentsline{loh}{figure}{站 \dpy{zhan4}}

\begin{EntryWithPhonetic}{站}{zhan4}{10}{⽴}[HSK 1,2]
  \definition*{s.}{Sobrenome: Zhan}
  \definition{s.}{parada; estação; ponto de parada | central; estação; instituição criada para um determinado tipo de atividade | filial de uma empresa ou organização; local de trabalho criado para realizar uma determinada tarefa | \emph{website}; na rede de computadores, refere"-se a um \emph{site}}
  \definition{v.}{ficar em pé; estar em pé | parar; interromper; fazer uma pausa}
\end{EntryWithPhonetic}

\begin{EntryWithPhonetic}{站点}{zhan4dian3}{10,9}{⽴,⽕}
  \definition{s.}{\emph{website}}
\end{EntryWithPhonetic}

\begin{EntryWithPhonetic}{站台}{zhan4tai2}{10,5}{⽴,⼝}[HSK 6]
  \definition{s.}{plataforma (em uma estação ferroviária)}
\end{EntryWithPhonetic}

\begin{EntryWithPhonetic}{站长}{zhan4zhang3}{10,4}{⽴,⾧}
  \definition{s.}{pessoa responsável pela estação de trem | chefe da estação | \emph{webmaster} | gerente de centro de voluntariado}
\end{EntryWithPhonetic}

\begin{EntryWithPhonetic}{站住}{zhan4zhu4}{10,7}{⽴,⼈}[HSK 2]
  \definition{v.}{parar; deter; parar enquanto se move | ficar firme nos pés; manter os pés; permanecer firme | manter-se firme; consolidar a posição de alguém; estabelecer-se em uma determinada unidade ou lugar | sustentar a opinião}
\end{EntryWithPhonetic}

\begin{EntryWithPhonetic}{站姿}{zhan4zi1}{10,9}{⽴,⼥}
  \definition{s.}{postura}
\end{EntryWithPhonetic}

%%%%%%%%%% 张 %%%%%%%%%%
\subsection*{张}\addcontentsline{loh}{figure}{张 \dpy{zhang1}}

\begin{EntryWithPhonetic}{张}{zhang1}{7}{⼸}[HSK 3]
  \definition*{s.}{Zhang, uma das vinte e oito constelações | Zhang, uma das mansões lunares | Sobrenome: Zhang}
  \definition{adj.}{indulgente; desenfreado; devasso; libertino}
  \definition{clas.}{usado para papel, couro, etc. | usado para cama, mesa, etc. | usado para o rosto, boca, etc. | usado para arco}
  \definition{v.}{abrir; espalhar; esticar | expor; exibir | (de uma loja) iniciar atividades comerciais; abrir | olhar; contemplar | expandir; estender; ampliar; exagerar | fixar (uma corda de arco); encordoar (um instrumento musical); puxar a corda do arco}
\end{EntryWithPhonetic}

\begin{EntryWithPhonetic}{张狂}{zhang1kuang2}{7,7}{⼸,⽝}
  \definition{adj.}{impetuoso | frenético | insolente}
\end{EntryWithPhonetic}

\begin{EntryWithPhonetic}{张南庄}{zhang1 nan2zhuang1}{7,9,6}{⼸,⼗,⼴}
  \definition*{s.}{Zhang Nanzhuang (anos de nascimento e morte desconhecidos), que viveu durante os períodos Qianlong e Jiaqing da Dinastia Qing, era conhecido como ``Passante'' (过路人); ele era bom em caligrafia no estilo de Ouyang Xun e imitava Fan Chengda e Lu You na escrita de poesia; ele gastava milhares de moedas de ouro todos os anos para colecionar livros raros, e sua coleção era a maior de Shanghai (上海) naquela época}
  \seealsoref{范成大}{fan4 cheng2da4}
  \seealsoref{过路人}{guo4lu4 ren2}
  \seealsoref{陆游}{lu4 you2}
  \seealsoref{欧阳询}{ou1yang2 xun2}
  \seealsoref{上海}{shang4hai3}
\end{EntryWithPhonetic}

\begin{EntryWithPhonetic}{张三}{zhang1san1}{7,3}{⼸,⼀}
  \definition*{s.}{Zhang San | Zé Ninguém | Nome para uma pessoa não especificada, 1 de 3}
  \seealsoref{李四}{li3si4}
  \seealsoref{王五}{wang2wu3}
\end{EntryWithPhonetic}

%%%%%%%%%% 章 %%%%%%%%%%
\subsection*{章}\addcontentsline{loh}{figure}{章 \dpy{zhang1}}

\begin{EntryWithPhonetic}{章}{zhang1}{11}{⾳}[HSK 6]
  \definition*{s.}{Sobrenome: Zhang}
  \definition[枚,个,方]{s.}{(para um livro, carta, etc.) capítulo; seção | ordem | regras; regulamentos; constituição | item; cláusula | (arcaico) memorial ao imperador; memorial ao trono | (arcaico) figura; padrão decorativo | selo; carimbo | distintivo; insígnia; medalha | verso; trecho do poema | escrita literária}
\end{EntryWithPhonetic}

\begin{EntryWithPhonetic}{章鱼}{zhang1yu2}{11,8}{⾳,⿂}
  \definition{s.}{polvo | octópode}
\end{EntryWithPhonetic}

%%%%%%%%%% 长 %%%%%%%%%%
\subsection*{长}\addcontentsline{loh}{figure}{长 \dpy{zhang3}}

\begin{EntryWithPhonetic}{长}{zhang3}{4}{⾧}[HSK 2,6][Kangxi 168]
  \definition{adj.}{mais velho; ancião; sênior; mais antigo; o primeiro da fila}
  \definition{s.}{ancião; pessoas mais velhas ou de posição social mais elevada | chefe; líder; dirigente; responsável}
  \definition{v.}{crescer; desenvolver"-se | surgir; começar a crescer; formar"-se; nascer em pessoas, animais, plantas ou objetos (algo) | adquirir; melhorar; aumentar; aumentar conhecimento, capacidade, etc.; tornar"-se cada vez mais ou cada vez melhor}
  \seeref{chang2}
\end{EntryWithPhonetic}

\begin{EntryWithPhonetic}{长大}{zhang3 da4}{4,3}{⾧,⼤}[HSK 2]
  \definition{v.}{crescer; ser criado; um estado de maturidade física ou mental completa}
\end{EntryWithPhonetic}

%%%%%%%%%% 涨 %%%%%%%%%%
\subsection*{涨}\addcontentsline{loh}{figure}{涨 \dpy{zhang3}}

\begin{EntryWithPhonetic}{涨}{zhang3}{10}{⽔}[HSK 5]
  \definition{v.}{subir; inchar; aumentar; elevar; melhorar}
  \seeref{zhang4}
\end{EntryWithPhonetic}

\begin{EntryWithPhonetic}{涨价}{zhang3/jia4}{10,6}{⽔,⼈}[HSK 5]
  \definition{s.}{aumento de preços}
  \definition{v.+compl.}{(preços) subir; aumentar o preço}
\end{EntryWithPhonetic}

%%%%%%%%%% 掌 %%%%%%%%%%
\subsection*{掌}\addcontentsline{loh}{figure}{掌 \dpy{zhang3}}

\begin{EntryWithPhonetic}{掌}{zhang3}{12}{⼿}
  \definition{s.}{palma da mão | sola do pé | pata | ferradura}
  \definition{v.}{dar um tapa | segurar na mão | empunhar}
\end{EntryWithPhonetic}

\begin{EntryWithPhonetic}{掌声}{zhang3sheng1}{12,7}{⼿,⼠}[HSK 6]
  \definition[阵]{s.}{aplausos; palmas; o som dos aplausos}
\end{EntryWithPhonetic}

\begin{EntryWithPhonetic}{掌握}{zhang3wo4}{12,12}{⼿,⼿}[HSK 5]
  \definition{v.}{compreender; dominar; conhecer bem; compreender as coisas; ser capaz de dominar ou utilizar plenamente | segurar; controlar; ter em mãos; tomar nas mãos}
\end{EntryWithPhonetic}

%%%%%%%%%% 丈 %%%%%%%%%%
\subsection*{丈}\addcontentsline{loh}{figure}{丈 \dpy{zhang4}}

\begin{EntryWithPhonetic}{丈}{zhang4}{3}{⼀}
  \definition{clas.}{zhang, uma unidade tradicional de comprimento, igual a 10 市尺 e equivalente a 3,333 metros ou 3,65 jardas}
  \definition{s.}{zhang, uma unidade de comprimento (= 3,333\dots metros)}
  \definition{s.}{sênior; ancião | marido (em certos termos de parentesco) | tratamento respeitoso ao idoso na China antiga; um título respeitoso para homens idosos nos tempos antigos | uma forma de tratamento para certos parentes do sexo masculino por casamento}
  \seealsoref{市尺}{shi4 chi3}
\end{EntryWithPhonetic}

\begin{EntryWithPhonetic}{丈夫}{zhang4fu5}{3,4}{⼀,⼤}[HSK 4]
  \definition[个,位,名]{s.}{marido; esposo}
\end{EntryWithPhonetic}

%%%%%%%%%% 账 %%%%%%%%%%
\subsection*{账}\addcontentsline{loh}{figure}{账 \dpy{zhang4}}

\begin{EntryWithPhonetic}{账}{zhang4}{8}{⾙}[HSK 6]
  \definition[笔,本]{s.}{conta | livro de contas | dívida; conta | crédito (de dívidas)}
\end{EntryWithPhonetic}

\begin{EntryWithPhonetic}{账户}{zhang4hu4}{8,4}{⾙,⼾}[HSK 6]
  \definition[个]{s.}{conta; refere"-se à classificação de vários usos de fundos, fontes e processos de rotatividade no livro contábil}
\end{EntryWithPhonetic}

%%%%%%%%%% 涨 %%%%%%%%%%
\subsection*{涨}\addcontentsline{loh}{figure}{涨 \dpy{zhang4}}

\begin{EntryWithPhonetic}{涨}{zhang4}{10}{⽔}[HSK 6]
  \definition{v.}{inchar; ter o volume aumentado | ser inundado por uma torrente de sangue; ter uma dor de cabeça; ficar com o rosto vermelho de raiva | ser mais, maior, etc. do que o esperado}
  \seeref{zhang3}
\end{EntryWithPhonetic}

%%%%%%%%%% 障 %%%%%%%%%%
\subsection*{障}\addcontentsline{loh}{figure}{障 \dpy{zhang4}}

\begin{EntryWithPhonetic}{障}{zhang4}{13}{⾩}
  \definition{s.}{barreira; bloco; obstáculo; obstruções}
  \definition{v.}{atrapalhar; obstruir; bloquear; cobrir}
\end{EntryWithPhonetic}

\begin{EntryWithPhonetic}{障碍}{zhang4'ai4}{13,13}{⾩,⽯}[HSK 6]
  \definition[个,种]{s.}{barreira; obstáculo; bloqueio; obstrução; impedimento}
\end{EntryWithPhonetic}

%%%%%%%%%% 招 %%%%%%%%%%
\subsection*{招}\addcontentsline{loh}{figure}{招 \dpy{zhao1}}

\begin{EntryWithPhonetic}{招}{zhao1}{8}{⼿}[HSK 6]
  \definition*{s.}{Sobrenome: Zhao}
  \definition{s.}{\emph{banner}; Faixas e outros itens costumavam ser pendurados nas entradas de hotéis, restaurantes ou lojas para atrair clientes | movimento; estratagema; artifício; meios ou táticas | movimentos de artes marciais}
  \definition{v.}{acenar; gestuar para alguém ver | alistar; inscrever; recrutar | incorrer; cortejar; atrair; provocar (um certo resultado ou reação) | provocar; tocar ou provocar a outra pessoa com palavras ou ações | confessar (culpa); assumir (culpa) | infectar; ser contagioso}
\end{EntryWithPhonetic}

\begin{EntryWithPhonetic}{招呼}{zhao1hu5}{8,8}{⼿,⼝}[HSK 4]
  \definition{v.}{chamar; chamar a atenção com palavras ou gestos | cumprimentar; saudar; cumprimentar ou despedir-se das pessoas com palavras ou gestos | pedir a alguém para fazer algo; fazer solicitações, pedir ajuda ou fazer coisas | receber e dar boas-vindas aos convidados}
\end{EntryWithPhonetic}

\begin{EntryWithPhonetic}{招聘}{zhao1pin4}{8,13}{⼿,⽿}[HSK 6]
  \definition{v.}{contratar; procurar; recrutar; convidar candidatos para um emprego}
\end{EntryWithPhonetic}

\begin{EntryWithPhonetic}{招生}{zhao1/sheng1}{8,5}{⼿,⽣}[HSK 5]
  \definition{v.+compl.}{conseguir alunos; matricular novos alunos; recrutar novos alunos}
\end{EntryWithPhonetic}

\begin{EntryWithPhonetic}{招手}{zhao1/shou3}{8,4}{⼿,⼿}[HSK 5]
  \definition{v.+compl.}{acenar; chamar a atenção; levantar a mão e acenar com a palma, para indicar que a outra pessoa se aproxime ou para cumprimentá-la}
\end{EntryWithPhonetic}

\begin{EntryWithPhonetic}{招数}{zhao1shu4}{8,13}{⼿,⽁}
  \definition{s.}{estratégia | movimento (no xadrez, no palco, nas artes marciais) | esquema | truque}
\end{EntryWithPhonetic}

%%%%%%%%%% 着 %%%%%%%%%%
\subsection*{着}\addcontentsline{loh}{figure}{着 \dpy{zhao1}}

\begin{EntryWithPhonetic}{着}{zhao1}{11}{⽬}
  \definition{interj.}{``Tudo bem!''; ``Tudo certo!''; ``O.K.!''}
  \definition{s.}{uma jogada no xadrez | truque; meio; artifício; manobra; estratégia}
  \definition{v.}{colocar dentro; guardar}
  \seeref{zhao2}
  \seeref{zhe5}
  \seeref{zhuo2}
\end{EntryWithPhonetic}

\begin{EntryWithPhonetic}{着数}{zhao1shu4}{11,13}{⽬,⽁}
  \definition{s.}{estratégia | movimento (no xadrez, no palco, nas artes marciais) | esquema | truque}
\end{EntryWithPhonetic}

%%%%%%%%%% 朝 %%%%%%%%%%
\subsection*{朝}\addcontentsline{loh}{figure}{朝 \dpy{zhao1}}

\begin{EntryWithPhonetic}{朝}{zhao1}{12}{⽉}
  \definition{s.}{manhã cedo; manhã | dia}
  \seeref{chao2}
\end{EntryWithPhonetic}

%%%%%%%%%% 嘲 %%%%%%%%%%
\subsection*{嘲}\addcontentsline{loh}{figure}{嘲 \dpy{zhao1}}

\begin{EntryWithPhonetic}{嘲}{zhao1}{15}{⼝}
  \definition{s.}{Onomatopéia: barulho clamoroso feito por várias pessoas falando ou cantando, ou por instrumentos musicais, ou pássaros cantando; descreve um som caótico e fragmentado}
  \seeref{chao2}
\end{EntryWithPhonetic}

%%%%%%%%%% 着 %%%%%%%%%%
\subsection*{着}\addcontentsline{loh}{figure}{着 \dpy{zhao2}}

\begin{EntryWithPhonetic}{着}{zhao2}{11}{⽬}[HSK 4]
  \definition{v.}{tocar (contato físico) | sentir; ser afetado por | queimar; acender | adormecer; cair no sono | acertar em cheio; ter sucesso em; usado após o verbo, indica que o objetivo foi alcançado ou que houve um resultado}
  \seeref{zhao1}
  \seeref{zhe5}
  \seeref{zhuo2}
\end{EntryWithPhonetic}

\begin{EntryWithPhonetic}{着地}{zhao2di4}{11,6}{⽬,⼟}
  \definition{v.}{pousar | tocar o chão}
\end{EntryWithPhonetic}

\begin{EntryWithPhonetic}{着花}{zhao2hua1}{11,7}{⽬,⾋}
  \definition{v.}{florescer}
  \seeref{zhuo2hua1}
\end{EntryWithPhonetic}

\begin{EntryWithPhonetic}{着火}{zhao2/huo3}{11,4}{⽬,⽕}[HSK 4]
  \definition{v.+compl.}{pegar fogo; estar em chamas}
\end{EntryWithPhonetic}

\begin{EntryWithPhonetic}{着急}{zhao2/ji2}{11,9}{⽬,⼼}[HSK 4]
  \definition{adj.}{ansioso; preocupado}
  \definition{s.}{preocupação; ansiedade}
  \definition{v.+compl.}{preocupar-se; sentir-se ansioso | sentir urgência; estar com pressa}
\end{EntryWithPhonetic}

\begin{EntryWithPhonetic}{着凉}{zhao2liang2}{11,10}{⽬,⼎}
  \definition{v.}{pegar um resfriado}
\end{EntryWithPhonetic}

%%%%%%%%%% 找 %%%%%%%%%%
\subsection*{找}\addcontentsline{loh}{figure}{找 \dpy{zhao3}}

\begin{EntryWithPhonetic}{找}{zhao3}{7}{⼿}[HSK 1]
  \definition{v.}{procurar; tentar encontrar; buscar | querer ver; visitar; abordar; solicitar | dar troco | descobrir; esforçar-se para ver ou obter a pessoa ou coisa desejada | examinar; investigar; completar as partes que faltam | causar intencionalmente (um resultado indesejável, negativo)}
\end{EntryWithPhonetic}

\begin{EntryWithPhonetic}{找遍}{zhao3bian4}{7,12}{⼿,⾡}
  \definition{v.}{pentear | pesquisar em todos os lugares}
\end{EntryWithPhonetic}

\begin{EntryWithPhonetic}{找出}{zhao3chu1}{7,5}{⼿,⼐}[HSK 2]
  \definition{v.}{encontrar | procurar}
\end{EntryWithPhonetic}

\begin{EntryWithPhonetic}{找到}{zhao3dao4}{7,8}{⼿,⼑}[HSK 1]
  \definition{v.}{encontrar; procurar; achar; encontar através de pesquisa, exploração, etc.;  ver ou encontrar coisas ou padrões que os antepassados não viram}
\end{EntryWithPhonetic}

\begin{EntryWithPhonetic}{找回}{zhao3hui2}{7,6}{⼿,⼞}
  \definition{v.}{recuperar algo}
\end{EntryWithPhonetic}

\begin{EntryWithPhonetic}{找见}{zhao3jian4}{7,4}{⼿,⾒}
  \definition{v.}{encontrar (algo que está procurando)}
\end{EntryWithPhonetic}

\begin{EntryWithPhonetic}{找零}{zhao3ling2}{7,13}{⼿,⾬}
  \definition{v.}{trocar dinheiro | dar troco}
\end{EntryWithPhonetic}

\begin{EntryWithPhonetic}{找钱}{zhao3qian2}{7,10}{⼿,⾦}
  \definition{v.}{dar troco}
\end{EntryWithPhonetic}

\begin{EntryWithPhonetic}{找事}{zhao3shi4}{7,8}{⼿,⼅}
  \definition{v.}{procurar emprego | começar uma briga}
\end{EntryWithPhonetic}

\begin{EntryWithPhonetic}{找寻}{zhao3xun2}{7,6}{⼿,⼨}
  \definition{v.}{encontrar falhas | procurar | buscar}
\end{EntryWithPhonetic}

\begin{EntryWithPhonetic}{找着}{zhao3zhao2}{7,11}{⼿,⽬}
  \definition{v.}{encontrar}
\end{EntryWithPhonetic}

\begin{EntryWithPhonetic}{找辙}{zhao3zhe2}{7,16}{⼿,⾞}
  \definition{v.}{procurar um pretexto}
\end{EntryWithPhonetic}

%%%%%%%%%% 召 %%%%%%%%%%
\subsection*{召}\addcontentsline{loh}{figure}{召 \dpy{zhao4}}

\begin{EntryWithPhonetic}{召}{zhao4}{5}{⼝}
  \definition*{s.}{Sobrenome: Zhao}
  \definition{s.}{templo}
  \definition{v.}{chamar; intimar; convocar; invocar}
  \seeref{shao4}
\end{EntryWithPhonetic}

\begin{EntryWithPhonetic}{召开}{zhao4kai1}{5,4}{⼝,⼶}[HSK 4]
  \definition{v.}{convocar; chamar pessoas para uma reunião; realizar (uma reunião)}
\end{EntryWithPhonetic}

%%%%%%%%%% 兆 %%%%%%%%%%
\subsection*{兆}\addcontentsline{loh}{figure}{兆 \dpy{zhao4}}

\begin{EntryWithPhonetic}{兆}{zhao4}{6}{⼉}
  \definition{num.}{trilhão}
\end{EntryWithPhonetic}

%%%%%%%%%% 照 %%%%%%%%%%
\subsection*{照}\addcontentsline{loh}{figure}{照 \dpy{zhao4}}

\begin{EntryWithPhonetic}{照}{zhao4}{13}{⽕}[HSK 3]
  \definition{adv.}{de acordo com; significa agir de acordo com o original ou com determinados padrões}
  \definition{prep.}{em direção a; na direção de | de acordo com; em conformidade com}
  \definition{s.}{imagem; fotografia | permissão; licença; autorização | brilho; iluminação}
  \definition{v.}{brilhar; iluminar | refletir; espelhar; olhar para sua própria imagem em um espelho, etc. | filmar; fotografar; tirar uma foto (fotografia) | cuidar de; tomar conta de; zelar por | notificar; informar | contrastar; comparar; verificar | entender; compreender}
\end{EntryWithPhonetic}

\begin{EntryWithPhonetic}{照顾}{zhao4gu5}{13,10}{⽕,⾴}[HSK 2]
  \definition{v.}{cuidar; cuidar de; atender | oferecer tratamento preferencial; prestar atenção especial e dar tratamento preferencial | (de um cliente) patrocinar; comprar em; cuidar de clientes que vêm comprar coisas ou solicitar serviços em lojas ou indústrias de serviços | dar consideração a; mostrar consideração por; levar em conta; fazer concessões a}
\end{EntryWithPhonetic}

\begin{EntryWithPhonetic}{照亮}{zhao4liang4}{13,9}{⽕,⼇}
  \definition{s.}{iluminação}
  \definition{v.}{iluminar}
\end{EntryWithPhonetic}

\begin{EntryWithPhonetic}{照片}{zhao4pian4}{13,4}{⽕,⽚}[HSK 2]
  \definition[张,套,幅]{s.}{fotografia, foto, imagem}
\end{EntryWithPhonetic}

\begin{EntryWithPhonetic}{照片底版}{zhao4pian4 di3ban3}{13,4,8,8}{⽕,⽚,⼴,⽚}
  \definition{s.}{placa fotográfica; negativo fotográfico}
\end{EntryWithPhonetic}

\begin{EntryWithPhonetic}{照片子}{zhao4pian4zi5}{13,4,3}{⽕,⽚,⼦}
  \definition{v.}{tirar um raio X}
\end{EntryWithPhonetic}

\begin{EntryWithPhonetic}{照骗}{zhao4pian4}{13,12}{⽕,⾺}
  \definition{s.}{Gíria da \emph{Internet}:  foto lisonjeira (trocadilho com 照片) | imagem alterada digitalmente; ``photoshopada''}
  \seealsoref{照片}{zhao4pian4}
\end{EntryWithPhonetic}

\begin{EntryWithPhonetic}{照相}{zhao4/xiang4}{13,9}{⽕,⽬}[HSK 2]
  \definition{v.+compl.}{fotografar; tirar fotos; tirar uma foto; tirar uma fotografia}
\end{EntryWithPhonetic}

\begin{EntryWithPhonetic}{照相机}{zhao4xiang4ji1}{13,9,6}{⽕,⽬,⽊}
  \definition[个,架,部,台,只]{s.}{câmera/máquina fotográfica}
\end{EntryWithPhonetic}

\begin{EntryWithPhonetic}{照像}{zhao4/xiang4}{13,13}{⽕,⼈}
  \variantof{照相}
\end{EntryWithPhonetic}

\begin{EntryWithPhonetic}{照像机}{zhao4xiang4ji1}{13,13,6}{⽕,⼈,⽊}
  \variantof{照相机}
\end{EntryWithPhonetic}

\begin{EntryWithPhonetic}{照样}{zhao4yang4}{13,10}{⽕,⽊}[HSK 6]
  \definition{adv.}{como antes; da mesma maneira; isso significa que, embora as condições externas tenham mudado ou sido afetadas, uma determinada situação ou estado permanece inalterado | após um padrão; de um modelo}
  \definition{v.}{fazer algo de uma certa maneira}
\end{EntryWithPhonetic}

\begin{EntryWithPhonetic}{照耀}{zhao4yao4}{13,20}{⽕,⽻}[HSK 6]
  \definition{v.}{brilhar; iluminar | esclarecer}
\end{EntryWithPhonetic}

\begin{EntryWithPhonetic}{照准}{zhao4zhun3}{13,10}{⽕,⼎}
  \definition{s.}{solicitação concedida (uso formal em documento antigo)}
  \definition{v.}{mirar (arma)}
\end{EntryWithPhonetic}

%%%%%%%%%% 折 %%%%%%%%%%
\subsection*{折}\addcontentsline{loh}{figure}{折 \dpy{zhe1}}

\begin{EntryWithPhonetic}{折}{zhe1}{7}{⼿}
  \definition{v.}{rolar; virar | despejar algo de um recipiente em outro; ficar despejando algo entre dois recipientes}
  \seeref{she2}
  \seeref{zhe2}
\end{EntryWithPhonetic}

\begin{EntryWithPhonetic}{折}{zhe2}{7}{⼿}[HSK 4]
  \definition*{s.}{Sobrenome: Zhe}
  \definition{clas.}{uma passagem em um roteiro de ópera miscelânea de Yuan, aproximadamente equivalente a uma cena ou ato em uma ópera moderna}
  \definition[张,个,些]{s.}{fratura; quebra | abatimento; desconto | traços dos caracteres chineses que têm o formato de ``𠃍'' e ``乚'', etc. | pasta; livreto; \emph{folder}}
  \definition{v.}{estalar; quebrar; fazer quebrar | perder; sofrer a perda de | voltar para trás; mudar de direção; retornar |ser convencido; estar cheio de admiração | equivaler a; converter em | dobrar}
  \seeref{she2}
  \seeref{zhe1}
\end{EntryWithPhonetic}

\begin{EntryWithPhonetic}{折转}{zhe2zhuan3}{7,8}{⼿,⾞}
  \definition{s.}{reflexo (ângulo)}
  \definition{v.}{voltar atrás}
\end{EntryWithPhonetic}

%%%%%%%%%% 哲 %%%%%%%%%%
\subsection*{哲}\addcontentsline{loh}{figure}{哲 \dpy{zhe2}}

\begin{EntryWithPhonetic}{哲}{zhe2}{10}{⼝}
  \definition{adj.}{sábio; sagaz}
  \definition[位,名,个]{s.}{pessoas sábias; sábio | sabedoria}
\end{EntryWithPhonetic}

\begin{EntryWithPhonetic}{哲理}{zhe2li3}{10,11}{⼝,⽟}
  \definition{s.}{filosofia | teoria filosófica}
\end{EntryWithPhonetic}

\begin{EntryWithPhonetic}{哲学}{zhe2xue2}{10,8}{⼝,⼦}[HSK 6]
  \definition{s.}{filosofia; é uma disciplina que explora questões fundamentais e conceitos básicos}
\end{EntryWithPhonetic}

%%%%%%%%%% 者 %%%%%%%%%%
\subsection*{者}\addcontentsline{loh}{figure}{者 \dpy{zhe3}}

\begin{EntryWithPhonetic}{者}{zhe3}{8}{⽼}[HSK 3]
  \definition*{s.}{Sobrenome: Zhe}
  \definition{part.}{significa 是 e é usado após palavras, frases e orações para indicar uma pausa}
  \definition{pron.}{usado para se referir à pessoa, coisa ou assunto que realiza uma ação ou possui um determinado atributo | pessoas; caras (Usado para se referir a alguém envolvido em uma determinada profissão, que acredita em uma determinada ideologia ou que tem uma forte tendência para algo) | usado após certos números ou palavras direcionais para se referir a coisas mencionadas anteriormente | significado semelhante a 这 (mais comum na linguagem coloquial antiga)}
  \seealsoref{是}{shi4}
  \seealsoref{这}{zhe4}
\end{EntryWithPhonetic}

%%%%%%%%%% 这 %%%%%%%%%%
\subsection*{这}\addcontentsline{loh}{figure}{这 \dpy{zhe4}}

\begin{EntryWithPhonetic}{这}{zhe4}{7}{⾡}[HSK 1]
  \definition{pron.}{este, isto; substitui pessoas ou coisas que estão mais próximas | agora; em vez de 这时候, tem o efeito de reforçar a ênfase}
  \seeref{zhei4}
  \seealsoref{这时候}{zhe4 shi2hou4}
\end{EntryWithPhonetic}

\begin{EntryWithPhonetic}{这边}{zhe4bian1}{7,5}{⾡,⾡}[HSK 1]
  \definition{pron.}{aqui; deste lado; refere"-se a um lugar próximo}
\end{EntryWithPhonetic}

\begin{EntryWithPhonetic}{这个}{zhe4ge5}{7,3}{⾡,⼈}
  \definition{pron.}{isto; este | isso; em vez das coisas mencionadas anteriormente | assim; tal; usado antes de verbos e adjetivos, indica um grau muito profundo, com um sentido exagerado | usado junto com 那个 para indicar pessoas ou objetos indefinidos}
  \seealsoref{那个}{na4ge5}
\end{EntryWithPhonetic}

\begin{EntryWithPhonetic}{这会儿}{zhe4 hui4r5}{7,6,2}{⾡,⼈,⼉}
  \definition{adv./pron./s.}{agora; no momento; no presente}
\end{EntryWithPhonetic}

\begin{EntryWithPhonetic}{这就是说}{zhe4jiu4shi4shuo1}{7,12,9,9}{⾡,⼪,⽇,⾔}[HSK 6]
  \definition{expr.}{isto significa que; isto é dizer}
\end{EntryWithPhonetic}

\begin{EntryWithPhonetic}{这里}{zhe4li3}{7,7}{⾡,⾥}[HSK 1]
  \definition{pron.}{aqui; pronomes demonstrativo, indicando locais próximos}
\end{EntryWithPhonetic}

\begin{EntryWithPhonetic}{这么}{zhe4me5}{7,3}{⾡,⼃}[HSK 2]
  \definition{pron.}{tal (usado para mostrar o grau) | então (usado para mostrar exagero e exclamação) | desta forma; assim; formas de expressar ações | tal; indica quantidade}
\end{EntryWithPhonetic}

\begin{EntryWithPhonetic}{这麽}{zhe4me5}{7,14}{⾡,⿇}
  \variantof{这么}
\end{EntryWithPhonetic}

\begin{EntryWithPhonetic}{这儿}{zhe4r5}{7,2}{⾡,⼉}[HSK 1]
  \definition{pron.}{aqui | agora; neste momento (utilizado apenas após 打, 从, 由)}
  \seealsoref{从}{cong2}
  \seealsoref{打}{da3}
  \seealsoref{由}{you2}
\end{EntryWithPhonetic}

\begin{EntryWithPhonetic}{这时}{zhe4 shi2}{7,7}{⾡,⽇}
  \definition{adv.}{neste momento}
\end{EntryWithPhonetic}

\begin{EntryWithPhonetic}{这时候}{zhe4 shi2hou4}{7,7,10}{⾡,⽇,⼈}[HSK 2]
  \definition{adv.}{neste momento}
\end{EntryWithPhonetic}

\begin{EntryWithPhonetic}{这些}{zhe4xie1}{7,8}{⾡,⼆}[HSK 1]
  \definition{pron.}{estes; pronome demonstrativo, que indicam duas ou mais pessoas ou coisas que estão próximas}
\end{EntryWithPhonetic}

\begin{EntryWithPhonetic}{这样}{zhe4yang4}{7,10}{⾡,⽊}[HSK 2]
  \definition{pron.}{assim; tal; assim; deste jeito; pronome demonstrativo, que indica a natureza, estado, maneira, grau, etc.}
\end{EntryWithPhonetic}

\begin{EntryWithPhonetic}{这咱}{zhe4 zan5}{7,9}{⾡,⼝}
  \definition{s.}{agora; no momento; no presente | neste momento}
\end{EntryWithPhonetic}

%%%%%%%%%% 浙 %%%%%%%%%%
\subsection*{浙}\addcontentsline{loh}{figure}{浙 \dpy{zhe4}}

\begin{EntryWithPhonetic}{浙}{zhe4}{10}{⽔}
  \definition{s.}{abreviação de província de Zhejiang,  浙江, no leste da China}
  \seealsoref{浙江}{zhe4jiang1}
\end{EntryWithPhonetic}

\begin{EntryWithPhonetic}{浙江}{zhe4jiang1}{10,6}{⽔,⽔}
  \definition*{s.}{Província de Zhejiang}
\end{EntryWithPhonetic}

%%%%%%%%%% 着 %%%%%%%%%%
\subsection*{着}\addcontentsline{loh}{figure}{着 \dpy{zhe5}}

\begin{EntryWithPhonetic}{着}{zhe5}{11}{⽬}[HSK 1]
  \definition{part.}{adicionar a um verbo ou adjetivo para indicar uma ação ou estado contínuo | em frases que começam com uma palavra que indica um lugar, acrescente ao verbo para indicar um estado resultante | em frases imperativas, usado após verbos ou adjetivos para dar ênfase | adicionado após certos verbos, transforma-se em preposição}
  \definition{s.}{um movimento no xadrez | movimento; estratégia; estratagema}
  \seeref{zhao1}
  \seeref{zhao2}
  \seeref{zhuo2}
\end{EntryWithPhonetic}

%%%%%%%%%% 这 %%%%%%%%%%
\subsection*{这}\addcontentsline{loh}{figure}{这 \dpy{zhei4}}

\begin{EntryWithPhonetic}{这}{zhei4}{7}{⾡}
  \definition{pron.}{(coloquial) este}
  \seeref{zhe4}
\end{EntryWithPhonetic}

%%%%%%%%%% 针 %%%%%%%%%%
\subsection*{针}\addcontentsline{loh}{figure}{针 \dpy{zhen1}}

\begin{EntryWithPhonetic}{针}{zhen1}{7}{⾦}[HSK 4]
  \definition*{s.}{Sobrenome: Zhen}
  \definition[根,枚,套,支]{s.}{agulha; ferramentas para costura de roupas | objetos semelhantes a agulhas; algo longo e fino como uma agulha | injeção | ponto de costura | pontos de acupuntura na medicina chinesa}
\end{EntryWithPhonetic}

\begin{EntryWithPhonetic}{针对}{zhen1dui4}{7,5}{⾦,⼨}[HSK 4]
  \definition{prep.}{em conexão com; de acordo com; à luz de; introdução de objetos de comportamento com uma finalidade clara}
  \definition{v.}{contrariar; apontar para; ter como objetivo; ser direcionado contra; fazer algo especificamente sobre um problema ou uma pessoa}
\end{EntryWithPhonetic}

%%%%%%%%%% 珍 %%%%%%%%%%
\subsection*{珍}\addcontentsline{loh}{figure}{珍 \dpy{zhen1}}

\begin{EntryWithPhonetic}{珍}{zhen1}{9}{⽟}
  \definition{adj.}{precioso; valioso; raro | inestimável}
  \definition{s.}{tesouro | objetos de valor}
  \definition{v.}{valorizar muito; estimar}
\end{EntryWithPhonetic}

\begin{EntryWithPhonetic}{珍贵}{zhen1gui4}{9,9}{⽟,⾙}[HSK 5]
  \definition{adj.}{raro; valioso; precioso; de grande valor; profundo significado}
\end{EntryWithPhonetic}

\begin{EntryWithPhonetic}{珍惜}{zhen1xi1}{9,11}{⽟,⼼}[HSK 5]
  \definition{v.}{valorizar; estimar; valorizar e evitar o desperdício}
\end{EntryWithPhonetic}

\begin{EntryWithPhonetic}{珍珠}{zhen1zhu1}{9,10}{⽟,⽟}[HSK 5]
  \definition[颗,串]{s.}{pérola; grânulos redondos produzidos nas conchas de certos animais aquáticos, de cor branca, rosa, etc., bonitos e brilhantes, frequentemente usados como adornos}
\end{EntryWithPhonetic}

%%%%%%%%%% 眞 %%%%%%%%%%
\subsection*{眞}\addcontentsline{loh}{figure}{眞 \dpy{zhen1}}

\begin{EntryWithPhonetic}{眞}{zhen1}{10}{⽬}
  \variantof{真}
\end{EntryWithPhonetic}

%%%%%%%%%% 真 %%%%%%%%%%
\subsection*{真}\addcontentsline{loh}{figure}{真 \dpy{zhen1}}

\begin{EntryWithPhonetic}{真}{zhen1}{10}{⼗}[HSK 1]
  \definition*{s.}{Sobrenome: Zhen}
  \definition{adj.}{verdadeiro; real; genuíno | claro; inequívoco | genuíno; conforme os fatos objetivos | sincero}
  \definition{adv.}{realmente; verdadeiramente; de fato}
  \definition{s.}{escrita regular | retrato; imagem; cópia exata de algo | instintos naturais (ou caráter, disposição); natureza; qualidade inerente; origem | estado original; refere"-se à forma original das coisas}
  \antonymref{假}{jia4}
  \antonymref{伪}{wei3}
\end{EntryWithPhonetic}

\begin{EntryWithPhonetic}{真诚}{zhen1cheng2}{10,8}{⼗,⾔}[HSK 5]
  \definition{adj.}{verdadeiro; honesto; sério; sincero; genuíno; descreve uma pessoa que fala e age com sinceridade, de coração, fazendo com que os outros acreditem nela}
\end{EntryWithPhonetic}

\begin{EntryWithPhonetic}{真的}{zhen1 de5}{10,8}{⼗,⽩}[HSK 1]
  \definition{adv.}{realmente; salientar que a situação existe realmente | verdadeiramente; realmente; existente na realidade; consistente com os fatos objetivos}
\end{EntryWithPhonetic}

\begin{EntryWithPhonetic}{真理}{zhen1li3}{10,11}{⼗,⽟}[HSK 5]
  \definition[条,个]{s.}{verdade; o reflexo correto das coisas objetivas e suas leis no cérebro humano}
\end{EntryWithPhonetic}

\begin{EntryWithPhonetic}{真牛}{zhen1niu2}{10,4}{⼗,⽜}
  \definition{adj.}{(gíria) muito legal, incrível}
\end{EntryWithPhonetic}

\begin{EntryWithPhonetic}{真切}{zhen1qie4}{10,4}{⼗,⼑}
  \definition{adj.}{claro | distinto | honesto | sincero | vívido}
\end{EntryWithPhonetic}

\begin{EntryWithPhonetic}{真声}{zhen1sheng1}{10,7}{⼗,⼠}
  \definition{s.}{voz modal; voz natural; voz verdadeira}
  \antonymref{假声}{jia3sheng1}
\end{EntryWithPhonetic}

\begin{EntryWithPhonetic}{真实}{zhen1shi2}{10,8}{⼗,⼧}[HSK 3]
  \definition{adj.}{verdadeiro; real; autêntico; de acordo com fatos objetivos}
\end{EntryWithPhonetic}

\begin{EntryWithPhonetic}{真释}{zhen1shi4}{10,12}{⼗,⾤}
  \definition{s.}{razão genuína | explicação verdadeira}
\end{EntryWithPhonetic}

\begin{EntryWithPhonetic}{真相}{zhen1xiang4}{10,9}{⼗,⽬}[HSK 5]
  \definition[个]{s.}{face; verdade; verdade nua e crua; a situação real; o estado real das coisas; a verdadeira situação}
\end{EntryWithPhonetic}

\begin{EntryWithPhonetic}{真心}{zhen1xin1}{10,4}{⼗,⼼}
  \definition{adj.}{sincero}
  \definition[片]{s.}{sinceridade}
\end{EntryWithPhonetic}

\begin{EntryWithPhonetic}{真真}{zhen1zhen1}{10,10}{⼗,⼗}
  \definition{adv.}{genuinamente | realmente | escrupulosamente}
\end{EntryWithPhonetic}

\begin{EntryWithPhonetic}{真正}{zhen1zheng4}{10,5}{⼗,⽌}[HSK 2]
  \definition{adj.}{verdadeiro; real; genuíno}
  \definition{adv.}{realmente; de fato; expressa afirmação de uma ação ou situação, equivalente a 确实}
  \seealsoref{确实}{que4shi2}
\end{EntryWithPhonetic}

\begin{EntryWithPhonetic}{真珠}{zhen1zhu1}{10,10}{⼗,⽟}
  \variantof{珍珠}
\end{EntryWithPhonetic}

%%%%%%%%%% 诊 %%%%%%%%%%
\subsection*{诊}\addcontentsline{loh}{figure}{诊 \dpy{zhen3}}

\begin{EntryWithPhonetic}{诊}{zhen3}{7}{⾔}
  \definition{v.}{examinar (um paciente)}
\end{EntryWithPhonetic}

\begin{EntryWithPhonetic}{诊断}{zhen3duan4}{7,11}{⾔,⽄}[HSK 5]
  \definition{s.}{diagnóstico; diacrisis}
  \definition{v.}{diagnosticar; após examinar os sintomas do paciente, determinar a doença e seu desenvolvimento}
\end{EntryWithPhonetic}

%%%%%%%%%% 枕 %%%%%%%%%%
\subsection*{枕}\addcontentsline{loh}{figure}{枕 \dpy{zhen3}}

\begin{EntryWithPhonetic}{枕}{zhen3}{8}{⽊}
  \definition*{s.}{Sobrenome: Zhen}
  \definition[个]{s.}{travesseiro; almofada | Mecânica: bloco}
  \definition{v.}{descansar a cabeça no travesseiro, almofada}
\end{EntryWithPhonetic}

%%%%%%%%%% 阵 %%%%%%%%%%
\subsection*{阵}\addcontentsline{loh}{figure}{阵 \dpy{zhen4}}

\begin{EntryWithPhonetic}{阵}{zhen4}{6}{⾩}[HSK 4]
  \definition{clas.}{um parágrafo que mostra o processo de um evento ou ação}
  \definition{s.}{matriz de batalha (formação); termo tático antigo para as fileiras ou formações de uma equipe de combate | \emph{front}; frente de batalha; posição | um período de tempo}
\end{EntryWithPhonetic}

\begin{EntryWithPhonetic}{阵地}{zhen4di4}{6,6}{⾩,⼟}
  \definition{s.}{posição (militar) | frente de batalha | \emph{front}}
\end{EntryWithPhonetic}

%%%%%%%%%% 振 %%%%%%%%%%
\subsection*{振}\addcontentsline{loh}{figure}{振 \dpy{zhen4}}

\begin{EntryWithPhonetic}{振}{zhen4}{10}{⼿}
  \definition{v.}{sacudir; acenar; bater as asas; empunhar | vibrar | recompor-se; levantar-se; animar}
\end{EntryWithPhonetic}

\begin{EntryWithPhonetic}{振动}{zhen4dong4}{10,6}{⼿,⼒}[HSK 5]
  \definition{s.}{vibração}
  \definition{v.}{sacudir; balançar; tremer; roncar; tagarelar; vibrar; oscilar; a física se refere ao movimento contínuo de um objeto em torno de um determinado ponto no espaço, como o movimento de um pêndulo, um diapasão ou uma corda de violão}
\end{EntryWithPhonetic}

%%%%%%%%%% 镇 %%%%%%%%%%
\subsection*{镇}\addcontentsline{loh}{figure}{镇 \dpy{zhen4}}

\begin{EntryWithPhonetic}{镇}{zhen4}{15}{⾦}[HSK 6]
  \definition{adj.}{inteiro; indica um período inteiro de tempo}
  \definition{adv.}{frequentemente; muitas vezes}
  \definition{s.}{posto de guarnição | cidade; divisão administrativa | centro comercial}
  \definition{v.}{suprimir; segurar; manter pressionado |  acalmar"-se; recompor"-se; estabilizar | guardar; guarnecer; fortalecer; usar a força para manter a estabilidade | resfriar com gelo; esfriar em água fria | acalmar; suprimir; dissuadir | suprimir pela força; sancionar}
\end{EntryWithPhonetic}

%%%%%%%%%% 震 %%%%%%%%%%
\subsection*{震}\addcontentsline{loh}{figure}{震 \dpy{zhen4}}

\begin{EntryWithPhonetic}{震}{zhen4}{15}{⾬}
  \definition*{s.}{Zhen, um dos Oito Trigramas que representa o trovão | Sobrenome: Zhen}
  \definition{adj.}{Coloquial: muito animado; profundamente surpreso; chocado}
  \definition{s.}{vibração; trepidação; tremor; abalo | terremoto; refere"-se especificamente a terremotos | trovão; relâmpago}
  \definition{v.}{sacudir; chocar; vibrar; estremecer | ficar muito animado; ficar profundamente surpreso; ficar chocado | superar; vencer}
\end{EntryWithPhonetic}

\begin{EntryWithPhonetic}{震撼}{zhen4han4}{15,16}{⾬,⼿}
  \definition{v.}{sacudir | chocar | atordoar}
\end{EntryWithPhonetic}

\begin{EntryWithPhonetic}{震惊}{zhen4jing1}{15,11}{⾬,⼼}[HSK 5]
  \definition{adj.}{chocado; atordoado; espantado; atônito}
  \definition{v.}{chocar; surpreender; espantar}
\end{EntryWithPhonetic}

%%%%%%%%%% 丁 %%%%%%%%%%
\subsection*{丁}\addcontentsline{loh}{figure}{丁 \dpy{zheng1}}

\begin{EntryWithPhonetic}{丁}{zheng1}{2}{⼀}
  \definition{s.}{Onomatopéia: som agudo e metálico (como o de cortar madeira, jogar xadrez ou tocar instrumentos musicais)}
  \seeref{ding1}
\end{EntryWithPhonetic}

%%%%%%%%%% 正 %%%%%%%%%%
\subsection*{正}\addcontentsline{loh}{figure}{正 \dpy{zheng1}}

\begin{EntryWithPhonetic}{正}{zheng1}{5}{⽌}
  \definition{s.}{o primeiro mês do ano lunar; a primeira lua}
  \seeref{zheng4}
\end{EntryWithPhonetic}

%%%%%%%%%% 争 %%%%%%%%%%
\subsection*{争}\addcontentsline{loh}{figure}{争 \dpy{zheng1}}

\begin{EntryWithPhonetic}{争}{zheng1}{6}{⼑}[HSK 3]
  \definition*{s.}{Sobrenome: Zheng}
  \definition{adv.}{como; por que}
  \definition{v.}{competir; disputar; lutar; esforçar-se para obter ou alcançar | discutir; argumentar; contestar; debater | faltar; estar em falta}
\end{EntryWithPhonetic}

\begin{EntryWithPhonetic}{争霸}{zheng1ba4}{6,21}{⼑,⾬}
  \definition{s.}{hegemonia | uma luta pelo poder}
  \definition{v.}{disputar a hegemonia}
\end{EntryWithPhonetic}

\begin{EntryWithPhonetic}{争夺}{zheng1duo2}{6,6}{⼑,⼤}[HSK 6]
  \definition{v.}{disputar; competir}
\end{EntryWithPhonetic}

\begin{EntryWithPhonetic}{争风吃醋}{zheng1feng1chi1cu4}{6,4,6,15}{⼑,⾵,⼝,⾣}
  \definition{v.}{rivalizar com alguém pelo carinho de um homem ou mulher | estar com ciúmes de um rival em um caso de amor}
\end{EntryWithPhonetic}

\begin{EntryWithPhonetic}{争论}{zheng1lun4}{6,6}{⼑,⾔}[HSK 4]
  \definition[番,场,次]{s.}{debate; discussão; argumentação; disputa}
  \definition{v.}{discutir; disputar; debater; argumentar; contestar}
\end{EntryWithPhonetic}

\begin{EntryWithPhonetic}{争取}{zheng1qu3}{6,8}{⼑,⼜}[HSK 3]
  \definition{v.}{lutar por; conquistar; vencer; se esforçar para conseguir}
\end{EntryWithPhonetic}

\begin{EntryWithPhonetic}{争先}{zheng1xian1}{6,6}{⼑,⼉}
  \definition{v.}{competir para ser o primeiro | contestar o primeiro lugar}
\end{EntryWithPhonetic}

\begin{EntryWithPhonetic}{争议}{zheng1yi4}{6,5}{⼑,⾔}[HSK 5]
  \definition{s.}{disputa; controvérsia; situações e questões em que há divergências de opinião}
  \definition{v.}{debater; discutir}
\end{EntryWithPhonetic}

%%%%%%%%%% 征 %%%%%%%%%%
\subsection*{征}\addcontentsline{loh}{figure}{征 \dpy{zheng1}}

\begin{EntryWithPhonetic}{征}{zheng1}{8}{⼻}
  \definition{s.}{prova; evidência | sinal; símbolo; presságio; sinais de manifestação; fenômeno}
  \definition{v.}{viajar; fazer uma jornada; pegar o caminho mais longo | iniciar uma campanha; fazer uma expedição punitiva | convocar; selecionar; recrutar | cobrar; impor; coletar | solicitar; pedir; procurar}
\end{EntryWithPhonetic}

\begin{EntryWithPhonetic}{征服}{zheng1fu2}{8,8}{⼻,⽉}[HSK 4]
  \definition{v.}{conquistar; cativar; usar a força para fazer a outra parte se submeter | subjugar; dominar; convencer as pessoas com poder infeccioso}
\end{EntryWithPhonetic}

\begin{EntryWithPhonetic}{征求}{zheng1qiu2}{8,7}{⼻,⽔}[HSK 4]
  \definition{v.}{procurar; buscar; solicitar; pedir abertamente opiniões, pontos de vista, etc.}
\end{EntryWithPhonetic}

%%%%%%%%%% 挣 %%%%%%%%%%
\subsection*{挣}\addcontentsline{loh}{figure}{挣 \dpy{zheng1}}

\begin{EntryWithPhonetic}{挣}{zheng1}{9}{⼿}
  \definition{v.}{tentar; fazer o possível para apoiar ou perseverar | lutar; lutar e apoiar}
  \seeref{zheng4}
\end{EntryWithPhonetic}

\begin{EntryWithPhonetic}{挣扎}{zheng1zha2}{9,4}{⼿,⼿}
  \definition{v.}{lutar}
\end{EntryWithPhonetic}

%%%%%%%%%% 症 %%%%%%%%%%
\subsection*{症}\addcontentsline{loh}{figure}{症 \dpy{zheng1}}

\begin{EntryWithPhonetic}{症}{zheng1}{10}{⽧}
  \definition{s.}{doença; enfermidade | (figurativo) ponto de atrito | tumor abdominal | obstrução intestinal}
  \seeref{zheng4}
\end{EntryWithPhonetic}

%%%%%%%%%% 整 %%%%%%%%%%
\subsection*{整}\addcontentsline{loh}{figure}{整 \dpy{zheng3}}

\begin{EntryWithPhonetic}{整}{zheng3}{16}{⽁}[HSK 3]
  \definition*{s.}{Sobrenome: Zheng}
  \definition{adj.}{cheio; integral; inteiro; completo; sem defeitos | limpo; arrumado; organizado; em boa ordem | redondo (não é uma fração)}
  \definition{s.}{número inteiro (não fracionário)}
  \definition{v.}{retificar; corrigir; pôr em ordem | consertar; renovar; reparar | corrigir; punir; causar sofrimento;  fazer alguém sofrer | fazer; realizar; trabalhar; em algumas regiões, significa 做, 搞}
  \seealsoref{搞}{gao3}
  \seealsoref{做}{zuo4}
\end{EntryWithPhonetic}

\begin{EntryWithPhonetic}{整顿}{zheng3dun4}{16,10}{⽁,⾴}[HSK 6]
  \definition{v.}{retificar; consolidar; reorganizar; tornar fenômenos, disciplinas e estilos desordenados e irracionais, ordenados e razoáveis}
\end{EntryWithPhonetic}

\begin{EntryWithPhonetic}{整个}{zheng3ge4}{16,3}{⽁,⼈}[HSK 3]
  \definition{adj.}{total; inteiro; completo}
\end{EntryWithPhonetic}

\begin{EntryWithPhonetic}{整理}{zheng3li3}{16,11}{⽁,⽟}[HSK 3]
  \definition{v.}{organizar; reorganizar; classificar; ordenar; colocar em ordem}
\end{EntryWithPhonetic}

\begin{EntryWithPhonetic}{整齐}{zheng3qi2}{16,6}{⽁,⿑}[HSK 3]
  \definition{adj.}{arrumado; organizado; em boa ordem | uniforme; regular; tamanho, comprimento, grau, etc. são relativamente consistentes | usado para descrever que todas as coisas necessárias estão prontas}
  \definition{v.}{estar em boas condições; manter a ordem e a organização}
\end{EntryWithPhonetic}

\begin{EntryWithPhonetic}{整体}{zheng3ti3}{16,7}{⽁,⼈}[HSK 3]
  \definition[个]{s.}{um todo; totalidade}
\end{EntryWithPhonetic}

\begin{EntryWithPhonetic}{整天}{zheng3tian1}{16,4}{⽁,⼤}[HSK 3]
  \definition{s.}{o dia inteiro; o dia todo; durante todo o dia; de manhã à noite}
\end{EntryWithPhonetic}

\begin{EntryWithPhonetic}{整整}{zheng3zheng3}{16,16}{⽁,⽁}[HSK 3]
  \definition{adv.}{inteiramente; completamente; solidamente; continuamente}
\end{EntryWithPhonetic}

\begin{EntryWithPhonetic}{整治}{zheng3zhi4}{16,8}{⽁,⽔}[HSK 6]
  \definition{v.}{renovar; consertar; arrumar; dragar (um rio, etc.) | punir; fazer alguém sofrer}
\end{EntryWithPhonetic}

%%%%%%%%%% 正 %%%%%%%%%%
\subsection*{正}\addcontentsline{loh}{figure}{正 \dpy{zheng4}}

\begin{EntryWithPhonetic}{正}{zheng4}{5}{⽌}[HSK 1,3]
  \definition*{s.}{Sobrenome: Zheng}
  \definition{adj.}{reto; ereto; vertical | principal; posicionado no meio | direito; anverso | honesto; íntegro; justo | puro; sem mistura (de cor ou sabor) | regular; padronizado; de acordo com a lei; correto | chefe; comandante; diretor | regular; as laterais e os ângulos do gráfico têm comprimentos e tamanhos iguais | positivo; Matemática: significa maior que zero; Física: significa perda de elétrons | exato; preciso; usado para indicar tempo, refere"-se ao momento exato ou ao ponto médio de um período}
  \definition{adv.}{apenas; certo; exatamente; precisamente | agora mesmo; neste momento; indica a continuidade de uma ação ou a permanência de um estado}
  \definition{v.}{definir (colocar) corretamente; alinhar; endireitar | ajustar; corrigir; retificar}
  \seeref{zheng1}
  \antonymref{负}{fu4}
\end{EntryWithPhonetic}

\begin{EntryWithPhonetic}{正版}{zheng4ban3}{5,8}{⽌,⽚}[HSK 5]
  \definition{s.}{versão genuína; versão autorizada; versão publicada e distribuída oficialmente por uma editora legal (em contraste com a 盗版)}
  \seealsoref{盗版}{dao4ban3}
\end{EntryWithPhonetic}

\begin{EntryWithPhonetic}{正常}{zheng4chang2}{5,11}{⽌,⼱}[HSK 2]
  \definition{adj.}{normal; regular; conforma-se com regras ou circunstâncias gerais}
\end{EntryWithPhonetic}

\begin{EntryWithPhonetic}{正当}{zheng4dang1}{5,6}{⽌,⼹}
  \definition{adj./adv.}{exatamente quando; exatamente o momento para}
  \seeref{zheng4dang4}
\end{EntryWithPhonetic}

\begin{EntryWithPhonetic}{正当}{zheng4dang4}{5,6}{⽌,⼹}[HSK 6]
  \definition{adv.}{exatamente quando; exatamente o momento para; está em (um certo período ou estágio)}
  \seeref{zheng4dang1}
\end{EntryWithPhonetic}

\begin{EntryWithPhonetic}{正规}{zheng4gui1}{5,8}{⽌,⾒}[HSK 5]
  \definition{adj.}{normal; regular; padrão; está em conformidade com padrões formalmente definidos ou geralmente reconhecidos}
\end{EntryWithPhonetic}

\begin{EntryWithPhonetic}{正好}{zheng4hao3}{5,6}{⽌,⼥}[HSK 2]
  \definition{adj.}{na hora certa; na hora certa; o suficiente}
  \definition{adv.}{acontecer com; chance de; como acontece}
\end{EntryWithPhonetic}

\begin{EntryWithPhonetic}{正面}{zheng4mian4}{5,9}{⽌,⾯}
  \definition{adj.}{bom; positivo | direto; cara a cara}
  \definition{s.}{frente; fachada; a frente do corpo; o lado de um edifício voltado para uma praça ou rua, que é decorado de forma mais elegante; a direção da viagem | lado anverso (ou direito); o lado de uma folha que é usado principalmente ou está em contato com o mundo externo | superfície; o lado diretamente exibido das coisas, problemas, etc.}
  \antonymref{背面}{bei4mian4}
  \antonymref{侧面}{ce4mian4}
  \antonymref{反面}{fan3mian4}
  \antonymref{负面}{fu4mian4}
\end{EntryWithPhonetic}

\begin{EntryWithPhonetic}{正确}{zheng4que4}{5,12}{⽌,⽯}[HSK 2]
  \definition{adj.}{correto; certo; próprio; conforma-se com fatos, razão ou algum padrão geralmente aceito}
\end{EntryWithPhonetic}

\begin{EntryWithPhonetic}{正如}{zheng4ru2}{5,6}{⽌,⼥}[HSK 5]
  \definition{adv.}{exatamente como; assim como}
\end{EntryWithPhonetic}

\begin{EntryWithPhonetic}{正式}{zheng4shi4}{5,6}{⽌,⼷}[HSK 3]
  \definition{adj.}{formal; oficial; descreve uma atmosfera séria, atitudes ou comportamentos que não são fáceis ou descontraídos | formal; oficial; descreve o cumprimento de determinados trâmites e procedimentos}
\end{EntryWithPhonetic}

\begin{EntryWithPhonetic}{正是}{zheng4shi4}{5,9}{⽌,⽇}[HSK 2]
  \definition{v.}{ser precisamente; ser exatamente}
\end{EntryWithPhonetic}

\begin{EntryWithPhonetic}{正义}{zheng4yi4}{5,3}{⽌,⼂}[HSK 5]
  \definition{adj.}{justo; íntegro}
  \definition{s.}{justiça; o que é certo; o que é benéfico para o povo | (frequentemente em títulos de livros) interpretação ortodoxa ou retificada (de textos antigos)}
\end{EntryWithPhonetic}

\begin{EntryWithPhonetic}{正在}{zheng4zai4}{5,6}{⽌,⼟}[HSK 1]
  \definition{adv.}{em processo de; em andamento; indica que uma ação está em andamento ou que uma situação está em curso.}
  \definition{v.}{estar a + {v.inf.} | estar + {v.ger.}}
\end{EntryWithPhonetic}

\begin{EntryWithPhonetic}{正正}{zheng4zheng4}{5,5}{⽌,⽌}
  \definition{adv.}{na hora certa | ordenadamente}
\end{EntryWithPhonetic}

\begin{EntryWithPhonetic}{正宗}{zheng4zong1}{5,8}{⽌,⼧}
  \definition{adj.}{autêntico | genuíno | \emph{old school} | (fig.) tradicional}
\end{EntryWithPhonetic}

%%%%%%%%%% 证 %%%%%%%%%%
\subsection*{证}\addcontentsline{loh}{figure}{证 \dpy{zheng4}}

\begin{EntryWithPhonetic}{证}{zheng4}{7}{⾔}[HSK 3]
  \definition{s.}{evidência; prova; testemunho | certificado; cartão | evidência; testemunha | doença; enfermidade}
  \definition{v.}{provar; demonstrar | verificar}
\end{EntryWithPhonetic}

\begin{EntryWithPhonetic}{证件}{zheng4jian4}{7,6}{⾔,⼈}[HSK 3]
  \definition[个,本,张,份]{s.}{documentos; credenciais; certificado; documentos que comprovem a identidade, experiência, etc., tais como carteira de estudante, carteira de trabalho, diploma de graduação, etc.}
\end{EntryWithPhonetic}

\begin{EntryWithPhonetic}{证据}{zheng4ju4}{7,11}{⾔,⼿}[HSK 3]
  \definition{s.}{prova; evidência; testemunho; fatos ou materiais que comprovam a veracidade de algo}
\end{EntryWithPhonetic}

\begin{EntryWithPhonetic}{证明}{zheng4ming2}{7,8}{⾔,⽇}[HSK 3]
  \definition[个,份]{s.}{certificado; atestado; identificação; certificado ou carta de referência; documentos que comprovem identidade, experiência, etc., tais como carteira de estudante, carteira de trabalho, diploma de graduação, etc.}
  \definition{v.}{provar; testemunhar; sustentar; usar materiais confiáveis para demonstrar ou determinar a autenticidade de pessoas ou coisas}
\end{EntryWithPhonetic}

\begin{EntryWithPhonetic}{证实}{zheng4shi2}{7,8}{⾔,⼧}[HSK 5]
  \definition{v.}{verificar; afirmar; confirmar; corroborar; demonstrar; autenticar; provar que é verdadeiro}
\end{EntryWithPhonetic}

\begin{EntryWithPhonetic}{证书}{zheng4shu1}{7,4}{⾔,⼄}[HSK 5]
  \definition[张,份,些]{s.}{certificado; documentos emitidos por instituições, grupos, etc., que comprovem experiência, nível, honras, poderes, etc.}
\end{EntryWithPhonetic}

%%%%%%%%%% 挣 %%%%%%%%%%
\subsection*{挣}\addcontentsline{loh}{figure}{挣 \dpy{zheng4}}

\begin{EntryWithPhonetic}{挣}{zheng4}{9}{⼿}[HSK 5]
  \definition{v.}{empurrar e puxar; tentar se livrar; lutar para se libertar; esforçar-se para se libertar das amarras | ganhar; fazer; trabalhar para; trocar trabalho por}
  \seeref{zheng1}
\end{EntryWithPhonetic}

\begin{EntryWithPhonetic}{挣得}{zheng4de2}{9,11}{⼿,⼻}
  \definition{v.}{ganhar renda ou dinheiro}
\end{EntryWithPhonetic}

\begin{EntryWithPhonetic}{挣钱}{zheng4/qian2}{9,10}{⼿,⾦}[HSK 5]
  \definition{v.+compl.}{ganhar dinheiro; fazer dinheiro; lucrar; trabalhar para ganhar dinheiro}
\end{EntryWithPhonetic}

%%%%%%%%%% 政 %%%%%%%%%%
\subsection*{政}\addcontentsline{loh}{figure}{政 \dpy{zheng4}}

\begin{EntryWithPhonetic}{政}{zheng4}{9}{⽁}
  \definition*{s.}{Sobrenome: Zheng}
  \definition{s.}{política; assuntos políticos | certos aspectos administrativos do governo | assuntos de uma família ou de uma organização; refere"-se a assuntos familiares ou de grupo}
\end{EntryWithPhonetic}

\begin{EntryWithPhonetic}{政策}{zheng4ce4}{9,12}{⽁,⽵}[HSK 6]
  \definition[项,条,个]{s.}{política; um código de conduta formulado por um país ou partido político para alcançar sua política em um determinado período histórico}
\end{EntryWithPhonetic}

\begin{EntryWithPhonetic}{政党}{zheng4dang3}{9,10}{⽁,⼉}[HSK 6]
  \definition[个,些]{s.}{partido político; uma organização política que representa um determinado estágio, classe ou grupo e luta para concretizar seus interesses}
\end{EntryWithPhonetic}

\begin{EntryWithPhonetic}{政府}{zheng4fu3}{9,8}{⽁,⼴}[HSK 4]
  \definition{s.}{governo;  órgãos executivos do poder do Estado, ou seja, órgãos administrativos do Estado, como o Conselho de Estado (Governo Popular Central) e os governos populares locais em todos os níveis na China}
\end{EntryWithPhonetic}

\begin{EntryWithPhonetic}{政纲}{zheng4gang1}{9,7}{⽁,⽷}
  \definition{s.}{programa ou plataforma política}
\end{EntryWithPhonetic}

\begin{EntryWithPhonetic}{政权}{zheng4quan2}{9,6}{⽁,⽊}[HSK 6]
  \definition{s.}{poder político ou estatal; regime}
\end{EntryWithPhonetic}

\begin{EntryWithPhonetic}{政治}{zheng4zhi4}{9,8}{⽁,⽔}[HSK 4]
  \definition{s.}{política; assuntos políticos; questões políticas; as atividades de governos, partidos políticos, grupos sociais e indivíduos em assuntos internos e relações internacionais}
\end{EntryWithPhonetic}

\begin{EntryWithPhonetic}{政治局}{zheng4zhi4ju2}{9,8,7}{⽁,⽔,⼫}
  \definition{s.}{o principal comitê de políticas de um partido comunista}
\end{EntryWithPhonetic}

%%%%%%%%%% 症 %%%%%%%%%%
\subsection*{症}\addcontentsline{loh}{figure}{症 \dpy{zheng4}}

\begin{EntryWithPhonetic}{症}{zheng4}{10}{⽧}
  \definition{s.}{doença; enfermidade}
  \seeref{zheng1}
\end{EntryWithPhonetic}

\begin{EntryWithPhonetic}{症状}{zheng4zhuang4}{10,7}{⽧,⽝}[HSK 6]
  \definition[种,些]{s.}{sintoma; estado anormal de um organismo devido a uma doença, como tosse, febre, etc.}
\end{EntryWithPhonetic}

%%%%%%%%%% 之 %%%%%%%%%%
\subsection*{之}\addcontentsline{loh}{figure}{之 \dpy{zhi1}}

\begin{EntryWithPhonetic}{之}{zhi1}{3}{⼂}
  \definition*{s.}{Sobrenome: Zhi}
  \definition{part.}{entre um atributivo e a palavra que ele modifica; equivalente a 的 | usado entre o sujeito e o predicado, em estruturas sujeito-predicado, de modo a torná-lo nominalizado}
  \definition{pron.}{substituto de uma pessoa ou coisa, limitado a ser usado como um objeto; substituir a pessoa ou coisa mencionada anteriormente | isto; isso; não substitui uma pessoa ou coisa específica, mas serve apenas para complementar sílabas}
  \definition{v.}{ir; deixar}
  \seealsoref{的}{de5}
\end{EntryWithPhonetic}

\begin{EntryWithPhonetic}{之后}{zhi1hou4}{3,6}{⼂,⼝}[HSK 4]
  \definition{s.}{mais tarde; posteriormente; depois; desde então; para indicar que é depois de um determinado tempo ou de uma determinada coisa, 以后 é usado com frequência na linguagem falada; às vezes, também pode indicar que é depois de um determinado lugar ou local,  后面 é usado com frequência na linguagem falada}
  \seealsoref{后面}{hou4mian5}
  \seealsoref{以后}{yi3hou4}
  \synonymref{后来}{hou4lai2}
  \synonymref{继而}{ji4'er2}
  \synonymref{接着}{jie1zhe5}
  \synonymref{其后}{qi2hou4}
  \synonymref{事后}{shi4hou4}
  \synonymref{以后}{yi3hou4}
  \antonymref{之前}{zhi1qian2}
\end{EntryWithPhonetic}

\begin{EntryWithPhonetic}{之间}{zhi1jian1}{3,7}{⼂,⾨}[HSK 4]
  \definition{s.}{(depois de um substantivo) entre; dentro de duas delimitações de tempo, local ou quantitativas | colocado após certos verbos ou advérbios de duas sílabas para indicar um curto período de tempo}
\end{EntryWithPhonetic}

\begin{EntryWithPhonetic}{之类}{zhi1lei4}{3,9}{⼂,⽶}[HSK 6]
  \definition{s.}{usado para dar exemplos (coisas do tipo, desse tipo, assim); uma categoria de pessoas ou coisas que compartilham as mesmas características das pessoas ou coisas mencionadas anteriormente}[我喜欢香蕉、苹果之类的水果。===Eu gosto de frutas como bananas e maçãs.]
\end{EntryWithPhonetic}

\begin{EntryWithPhonetic}{之内}{zhi1nei4}{3,4}{⼂,⼌}[HSK 5]
  \definition{adv.}{em; dentro de; indica dentro de um determinado intervalo, limite ou período de tempo, etc.}
  \synonymref{以内}{yi3nei4}
\end{EntryWithPhonetic}

\begin{EntryWithPhonetic}{之前}{zhi1qian2}{3,9}{⼂,⼑}[HSK 4]
  \definition{adv.}{(referindo"-se ao tempo) antes, antes de, atrás | (referindo"-se ao local físico) na frente de | (usado independentemente) no passado, antes disso}
  \synonymref{曾经}{ceng2jing1}
  \synonymref{此前}{ci3qian2}
  \synonymref{以前}{yi3qian2}
  \antonymref{正在}{zheng4zai4}
  \antonymref{之后}{zhi1hou4}
\end{EntryWithPhonetic}

\begin{EntryWithPhonetic}{之外}{zhi1wai4}{3,5}{⼂,⼣}[HSK 5]
  \definition{adv.}{lado de fora; exceto; além de; além disso; refere"-se a algo que excede um determinado limite}
  \synonymref{除外}{chu2wai4}
  \synonymref{以外}{yi3wai4}
\end{EntryWithPhonetic}

\begin{EntryWithPhonetic}{之下}{zhi1xia4}{3,3}{⼂,⼀}[HSK 5]
  \definition{s.}{usado para indicar algo abaixo de um determinado intervalo, posição, grau, etc.; indica um aspecto inferior em termos de alcance, posição, status, nível, Chengdu, etc. | usado para indicar as condições sob as quais algo acontece | usado para indicar o humor, estado em que alguém faz algo; expressa um determinado comportamento em um determinado estado de espírito ou situação}
  \synonymref{以下}{yi3xia4}
\end{EntryWithPhonetic}

\begin{EntryWithPhonetic}{之一}{zhi1yi1}{3,1}{⼂,⼀}[HSK 4]
  \definition[分]{s.}{um de (algo); pertence a um ou a todo um grupo de coisas com as mesmas características}
  \synonymref{之中}{zhi1zhong1}
\end{EntryWithPhonetic}

\begin{EntryWithPhonetic}{之中}{zhi1zhong1}{3,4}{⼂,⼁}[HSK 5]
  \definition{prep.}{em; no meio de; entre}
  \synonymref{之一}{zhi1yi1}
\end{EntryWithPhonetic}

%%%%%%%%%% 支 %%%%%%%%%%
\subsection*{支}\addcontentsline{loh}{figure}{支 \dpy{zhi1}}

\begin{EntryWithPhonetic}{支}{zhi1}{4}{⽀}[HSK 3,4][Kangxi 65]
  \definition*{s.}{Sobrenome: Zhi}
  \definition{clas.}{usado para equipes, etc. | usado em canções ou peças musicais | intensidade luminosa utilizada para luzes elétricas | usado para objetos longos e finos}
  \definition{s.}{ramo; ramificação; tribo | os doze ramos terrestres}
  \definition{v.}{segurar; apoiar; sustentar | sobressair; esticar; erguer; estender | ordenar; enviar | receber; pagar; obter pagamento (dinheiro)}
\end{EntryWithPhonetic}

\begin{EntryWithPhonetic}{支承}{zhi1cheng2}{4,8}{⽀,⼿}
  \definition{v.}{suportar o peso de (um edifício) | suportar}
\end{EntryWithPhonetic}

\begin{EntryWithPhonetic}{支撑}{zhi1cheng5}{4,15}{⽀,⼿}[HSK 6]
  \definition{v.}{sustentar; apoiar}[柱子支撑着整个建筑。===Os pilares sustentam todo o edifício. | 一家的生活由他支撑。===Ele sustenta a vida da família.]
\end{EntryWithPhonetic}

\begin{EntryWithPhonetic}{支持}{zhi1chi2}{4,9}{⽀,⼿}[HSK 3]
  \definition{v.}{suportar; aguentar; resistir; manter-se com dificuldade | apoiar; dar incentivo ou patrocínio}
\end{EntryWithPhonetic}

\begin{EntryWithPhonetic}{支出}{zhi1chu1}{4,5}{⽀,⼐}[HSK 5]
  \definition[笔,项]{s.}{despesas; gastos}
  \definition{v.}{pagar; gastar; desembolsar; efetuar o pagamento}
\end{EntryWithPhonetic}

\begin{EntryWithPhonetic}{支付}{zhi1fu4}{4,5}{⽀,⼈}[HSK 3]
  \definition{v.}{pagar (dinheiro); custear; efetuar pagamento}
\end{EntryWithPhonetic}

\begin{EntryWithPhonetic}{支根}{zhi1gen1}{4,10}{⽀,⽊}
  \definition{s.}{raiz ramificada | raízes de apoio | radícula}
\end{EntryWithPhonetic}

\begin{EntryWithPhonetic}{支配}{zhi1pei4}{4,10}{⽀,⾣}[HSK 5]
  \definition{v.}{organizar; alocar; orçar; distribuir | controlar; dominar; governar; exercer influência e controle sobre pessoas ou coisas}
\end{EntryWithPhonetic}

\begin{EntryWithPhonetic}{支票}{zhi1piao4}{4,11}{⽀,⽰}
  \definition[本]{s.}{cheque (banco)}
\end{EntryWithPhonetic}

\begin{EntryWithPhonetic}{支应}{zhi1ying4}{4,7}{⽀,⼴}
  \definition{v.}{lidar com | fornecer}
\end{EntryWithPhonetic}

\begin{EntryWithPhonetic}{支援}{zhi1yuan2}{4,12}{⽀,⼿}[HSK 6]
  \definition{v.}{ajudar; auxiliar; apoiar; utilizar ações humanas, materiais, financeiras ou outras ações práticas para apoiar e auxiliar}
\end{EntryWithPhonetic}

\begin{EntryWithPhonetic}{支支吾吾}{zhi1zhi1wu2wu2}{4,4,7,7}{⽀,⽀,⼝,⼝}
  \definition{v.}{falhar | murmurar | paralisar | gaguejar}
\end{EntryWithPhonetic}

%%%%%%%%%% 只 %%%%%%%%%%
\subsection*{只}\addcontentsline{loh}{figure}{只 \dpy{zhi1}}

\begin{EntryWithPhonetic}{只}{zhi1}{5}{⼝}[HSK 3]
  \definition{adj.}{solteiro; solitário; único; muito raro}
  \definition{clas.}{usado para um de um par | usado para animais pequenos (pássaros, gatos, cães, etc.) | usado para certos utensílios, aparelhos | usado para navios}
  \seeref{zhi3}
\end{EntryWithPhonetic}

\begin{EntryWithPhonetic}{只身}{zhi1shen1}{5,7}{⼝,⾝}
  \definition{adv.}{sozinho | por si só}
\end{EntryWithPhonetic}

%%%%%%%%%% 芝 %%%%%%%%%%
\subsection*{芝}\addcontentsline{loh}{figure}{芝 \dpy{zhi1}}

\begin{EntryWithPhonetic}{芝}{zhi1}{6}{⾋}
  \definition*{s.}{Sobrenome: Zhi}
  \definition{s.}{Arcaico: fungo mágico, ganoderma brilhante | Arcaico: raiz de angélica dahuriana}
\end{EntryWithPhonetic}

\begin{EntryWithPhonetic}{芝麻}{zhi1ma5}{6,11}{⾋,⿇}
  \definition{s.}{semente de gergelim}
\end{EntryWithPhonetic}

%%%%%%%%%% 吱 %%%%%%%%%%
\subsection*{吱}\addcontentsline{loh}{figure}{吱 \dpy{zhi1}}

\begin{EntryWithPhonetic}{吱}{zhi1}{7}{⼝}
  \definition{v.}{Onomatopéia: rangido; estalo}
  \seeref{zi1}
\end{EntryWithPhonetic}

%%%%%%%%%% 枝 %%%%%%%%%%
\subsection*{枝}\addcontentsline{loh}{figure}{枝 \dpy{zhi1}}

\begin{EntryWithPhonetic}{枝}{zhi1}{8}{⽊}[HSK 6]
  \definition*{s.}{Sobrenome: Zhi}
  \definition{clas.}{usado para flores com galhos, ramos | usado para objetos em forma de haste}
  \definition{s.}{ramo; galho}
\end{EntryWithPhonetic}

%%%%%%%%%% 知 %%%%%%%%%%
\subsection*{知}\addcontentsline{loh}{figure}{知 \dpy{zhi1}}

\begin{EntryWithPhonetic}{知}{zhi1}{8}{⽮}
  \definition{s.}{conhecimento | amigo íntimo; refere"-se a um confidente}
  \definition{v.}{saber; entender; perceber; estar ciente de | contar; informar; notificar; tornar conhecido | administrar; estar encarregado de; supervisionar}
\end{EntryWithPhonetic}

\begin{EntryWithPhonetic}{知道了}{zhi1dao4le5}{8,12,2}{⽮,⾡,⼅}
  \definition{interj.}{``Entendi!''; ``O.K.!''}
\end{EntryWithPhonetic}

\begin{EntryWithPhonetic}{知道}{zhi1dao5}{8,12}{⽮,⾡}[HSK 1]
  \definition{v.}{saber; perceber; estar ciente de; ter conhecimento dos fatos ou da razão; ser sensato}
\end{EntryWithPhonetic}

\begin{EntryWithPhonetic}{知了}{zhi1liao3}{8,2}{⽮,⼅}
  \definition[通]{s.}{cigarra}
\end{EntryWithPhonetic}

\begin{EntryWithPhonetic}{知名}{zhi1ming2}{8,6}{⽮,⼝}[HSK 6]
  \definition{adj.}{notável; famoso; celebrado; bem conhecido}
\end{EntryWithPhonetic}

\begin{EntryWithPhonetic}{知识}{zhi1shi5}{8,7}{⽮,⾔}[HSK 1]
  \definition[个,门,种]{s.}{conhecimento; conjunto de conhecimentos e experiências adquiridos pelas pessoas na prática de transformar o mundo | intelectual; refere"-se à cultura acadêmica}
\end{EntryWithPhonetic}

%%%%%%%%%% 织 %%%%%%%%%%
\subsection*{织}\addcontentsline{loh}{figure}{织 \dpy{zhi1}}

\begin{EntryWithPhonetic}{织}{zhi1}{8}{⽷}[HSK 6]
  \definition{v.}{tecer; fazer fios ou linhas cruzarem para fazer seda, tecido, lã, etc. | tricotar; usar agulhas para fazer fios ou linhas entrelaçados para confeccionar suéteres, meias, rendas, redes, etc. | sobrepor-se; entrelaçar-se; cruzar; entrelaçar}
\end{EntryWithPhonetic}

%%%%%%%%%% 脂 %%%%%%%%%%
\subsection*{脂}\addcontentsline{loh}{figure}{脂 \dpy{zhi1}}

\begin{EntryWithPhonetic}{脂}{zhi1}{10}{⾁}
  \definition*{s.}{Sobrenome: Zhi}
  \definition{s.}{gordura; graxa; sebo | (cosméticos) rouge | (cosméticos) baton; protetor labial}
\end{EntryWithPhonetic}

\begin{EntryWithPhonetic}{脂麻}{zhi1ma5}{10,11}{⾁,⿇}
  \variantof{芝麻}
\end{EntryWithPhonetic}

%%%%%%%%%% 蜘 %%%%%%%%%%
\subsection*{蜘}\addcontentsline{loh}{figure}{蜘 \dpy{zhi1}}

\begin{EntryWithPhonetic}{蜘}{zhi1}{14}{⾍}
  \definition[只]{s.}{aranha}
  \seealsoref{蜘蛛}{zhi1zhu1}
\end{EntryWithPhonetic}

\begin{EntryWithPhonetic}{蜘蛛}{zhi1zhu1}{14,12}{⾍,⾍}
  \definition{s.}{aranha}
\end{EntryWithPhonetic}

\begin{EntryWithPhonetic}{蜘蛛网}{zhi1zhu1 wang3}{14,12,6}{⾍,⾍,⽹}
  \definition{s.}{teia de aranha}
\end{EntryWithPhonetic}

%%%%%%%%%% 执 %%%%%%%%%%
\subsection*{执}\addcontentsline{loh}{figure}{执 \dpy{zhi2}}

\begin{EntryWithPhonetic}{执}{zhi2}{6}{⼿}
  \definition*{s.}{Sobrenome: Zhi}
  \definition[期]{s.}{reconhecimento por escrito | (literário) amigo íntimo (ou do peito)}
  \definition{v.}{segurar; agarrar; pegar; capturar | assumir o comando de; dirigir; gerenciar; controlar; administrar; exercer | manter (os próprios pontos de vista, etc.); persistir; persistir em; manter-se em; insistir em | realizar; executar; implementar}
\end{EntryWithPhonetic}

\begin{EntryWithPhonetic}{执行}{zhi2xing2}{6,6}{⼿,⾏}[HSK 5]
  \definition{v.}{executar; implementar; realizar}
\end{EntryWithPhonetic}

\begin{EntryWithPhonetic}{执着}{zhi2zhuo2}{6,11}{⼿,⽬}
  \definition{s.}{(budismo) apego}
  \definition{v.}{estar fortemente apegado a | ser dedicado | apegar-se a}
\end{EntryWithPhonetic}

%%%%%%%%%% 直 %%%%%%%%%%
\subsection*{直}\addcontentsline{loh}{figure}{直 \dpy{zhi2}}

\begin{EntryWithPhonetic}{直}{zhi2}{8}{⽬}[HSK 3]
  \definition*{s.}{Sobrenome: Zhi}
  \definition{adj.}{reto | vertical; perpendicular | justo; íntegro; imparcial | franco; sincero; direto ao ponto | rígido; entorpecido | direto; em linha reta; rígido | ereto; perpendicular ao solo}
  \definition{adv.}{diretamente; sempre; reto | continuamente; constantemente | apenas; simplesmente | de fato}
  \definition[条]{s.}{traço vertical (em caracteres chineses, 竖)}
  \definition{v.}{endireitar; tornar reto | alongar}
  \seealsoref{竖}{shu4}
  \antonymref{横}{heng2}
  \antonymref{曲}{qu1}
  \antonymref{弯}{wan1}
\end{EntryWithPhonetic}

\begin{EntryWithPhonetic}{直播}{zhi2bo1}{8,15}{⽬,⼿}[HSK 3]
  \definition{v.}{transmissão ao vivo; transmitir diretamente, sem gravar}
\end{EntryWithPhonetic}

\begin{EntryWithPhonetic}{直到}{zhi2dao4}{8,8}{⽬,⼑}[HSK 3]
  \definition{adv.}{até (geralmente se refere ao tempo); até que}
\end{EntryWithPhonetic}

\begin{EntryWithPhonetic}{直接}{zhi2jie1}{8,11}{⽬,⼿}[HSK 2]
  \definition{adj.}{direto | imediato}
  \antonymref{间接}{jian4jie1}
\end{EntryWithPhonetic}

\begin{EntryWithPhonetic}{直升机}{zhi2sheng1ji1}{8,4,6}{⽬,⼗,⽊}[HSK 6]
  \definition[架,台,个]{s.}{helicóptero; uma aeronave que pode decolar e pousar verticalmente, com uma hélice montada na parte superior da fuselagem que gira horizontalmente, permitindo que ela permaneça no ar e decole e pouse em uma pequena área}
\end{EntryWithPhonetic}

\begin{EntryWithPhonetic}{直线}{zhi2xian4}{8,8}{⽬,⽷}[HSK 5]
  \definition{adj.}{direto; retilíneo | íngreme; acentuada (subida ou descida)}
  \definition[条]{s.}{linha reta}
\end{EntryWithPhonetic}

\begin{EntryWithPhonetic}{直译}{zhi2yi4}{8,7}{⽬,⾔}
  \definition{s.}{tradução literal}
  \seealsoref{意译}{yi4yi4}
\end{EntryWithPhonetic}

\begin{EntryWithPhonetic}{直译器}{zhi2yi4qi4}{8,7,16}{⽬,⾔,⼝}
  \definition{s.}{(computação) interpretador}
\end{EntryWithPhonetic}

%%%%%%%%%% 值 %%%%%%%%%%
\subsection*{值}\addcontentsline{loh}{figure}{值 \dpy{zhi2}}

\begin{EntryWithPhonetic}{值}{zhi2}{10}{⼈}[HSK 3]
  \definition{adj.}{significativo; valioso; digno de nota}
  \definition{prep.}{quando; introduz o momento em que algo acontece ou existe, equivalente a 当 ou 在}
  \definition{s.}{preço; valor | valor de um número, de uma variável}
  \definition{v.}{valer; custar; a mercadoria é adequada ao preço | ir de encontro; encontrar; cruzar | estar de serviço; ter sua vez em algo; assumir o cargo que lhe cabe | é a vez de (executar uma determinada função pública)}
  \seealsoref{当}{dang1}
  \seealsoref{在}{zai4}
\end{EntryWithPhonetic}

\begin{EntryWithPhonetic}{值班}{zhi2/ban1}{10,10}{⼈,⽟}[HSK 5]
  \definition{v.+compl.}{estar em serviço ou plantão; trabalhar em um turno; (em rodízio) desempenhar funções durante um período de tempo determinado}
\end{EntryWithPhonetic}

\begin{EntryWithPhonetic}{值得}{zhi2/de5}{10,11}{⼈,⼻}[HSK 3]
  \definition{adj.}{que tem valor; (fazer algo) é vantajoso, sem prejuízos}
  \definition{v.+compl.}{merecer; ter valor; significa que fazer isso terá bons resultados; que é valioso e significativo}
\end{EntryWithPhonetic}

%%%%%%%%%% 职 %%%%%%%%%%
\subsection*{职}\addcontentsline{loh}{figure}{职 \dpy{zhi2}}

\begin{EntryWithPhonetic}{职}{zhi2}{11}{⽿}
  \definition*{s.}{Sobrenome: Zhi}
  \definition{prep.}{para; devido a; por causa de}
  \definition{prep.}{Obsoleto: Eu (em relatórios oficiais aos superiores)}
  \definition{s.}{dever; trabalho | cargo; posto; função; responsabilidades; posição}
  \definition{v.}{gerenciar; dirigir | administrar}
\end{EntryWithPhonetic}

\begin{EntryWithPhonetic}{职工}{zhi2gong1}{11,3}{⽿,⼯}[HSK 3]
  \definition[个,位,名,些]{s.}{pessoal; trabalhadores e funcionários administrativos}
\end{EntryWithPhonetic}

\begin{EntryWithPhonetic}{职能}{zhi2neng2}{11,10}{⽿,⾁}[HSK 5]
  \definition[种,项]{s.}{função; funções ou papéis que as organizações, instituições, etc. devem desempenhar}
\end{EntryWithPhonetic}

\begin{EntryWithPhonetic}{职位}{zhi2wei4}{11,7}{⽿,⼈}[HSK 5]
  \definition[个]{s.}{posto; posição; cargo que exerce determinadas funções em órgãos ou entidades}
\end{EntryWithPhonetic}

\begin{EntryWithPhonetic}{职务}{zhi2wu4}{11,5}{⽿,⼒}[HSK 5]
  \definition{s.}{cargo; posto; deveres; função; funções que devem ser desempenhadas de acordo com as especificações do cargo}
\end{EntryWithPhonetic}

\begin{EntryWithPhonetic}{职业}{zhi2ye4}{11,5}{⽿,⼀}[HSK 3]
  \definition{adj.}{profissional; não amador}
  \definition[种,份,个]{s.}{ocupação; profissão; vocação; o trabalho que um indivíduo realiza na sociedade como sua principal fonte de subsistência}
\end{EntryWithPhonetic}

\begin{EntryWithPhonetic}{职员}{zhi2yuan2}{11,7}{⽿,⼝}
  \definition[个,位]{s.}{empregado | trabalhador de escritório | membro da equipe}
\end{EntryWithPhonetic}

\begin{EntryWithPhonetic}{职责}{zhi2ze2}{11,8}{⽿,⾙}[HSK 6]
  \definition[种]{s.}{dever; obrigação; responsabilidade; coisas que você deve fazer por causa de sua profissão ou identidade}
\end{EntryWithPhonetic}

%%%%%%%%%% 植 %%%%%%%%%%
\subsection*{植}\addcontentsline{loh}{figure}{植 \dpy{zhi2}}

\begin{EntryWithPhonetic}{植}{zhi2}{12}{⽊}
  \definition*{s.}{Sobrenome: Zhi}
  \definition{s.}{flora; planta; vegetação}
  \definition{v.}{plantar; crescer; cultivar | configurar; estabelecer}
\end{EntryWithPhonetic}

\begin{EntryWithPhonetic}{植物}{zhi2wu4}{12,8}{⽊,⽜}[HSK 4]
  \definition[种,株,棵,盆]{s.}{planta; vegetação; flora}
\end{EntryWithPhonetic}

%%%%%%%%%% 殖 %%%%%%%%%%
\subsection*{殖}\addcontentsline{loh}{figure}{殖 \dpy{zhi2}}

\begin{EntryWithPhonetic}{殖}{zhi2}{12}{⽍}
  \definition{v.}{crescer | reproduzir}
\end{EntryWithPhonetic}

%%%%%%%%%% 止 %%%%%%%%%%
\subsection*{止}\addcontentsline{loh}{figure}{止 \dpy{zhi3}}

\begin{EntryWithPhonetic}{止}{zhi3}{4}{⽌}[HSK 6][Kangxi 77]
  \definition{adv.}{somente; apenas}
  \definition{v.}{parar; cortar; bloquear | finalizar; fechar (até o prazo final) | não ser permitido}
\end{EntryWithPhonetic}

%%%%%%%%%% 只 %%%%%%%%%%
\subsection*{只}\addcontentsline{loh}{figure}{只 \dpy{zhi3}}

\begin{EntryWithPhonetic}{只}{zhi3}{5}{⼝}[HSK 2]
  \definition{adv.}{somente; apenas; meramente | simplesmente; usado para limitar o escopo, indicando que não há nada além disso, equivalente a 仅仅}
  \seeref{zhi1}
  \seealsoref{仅仅}{jin3jin3}
\end{EntryWithPhonetic}

\begin{EntryWithPhonetic}{只不过}{zhi3bu5guo4}{5,4,6}{⼝,⼀,⾡}[HSK 5]
  \definition{adv.}{somente; apenas; meramente; não mais do que}
\end{EntryWithPhonetic}

\begin{EntryWithPhonetic}{只得}{zhi3de2}{5,11}{⼝,⼻}[HSK 6]
  \definition{v.}{não ter alternativa senão; obrigado a; ter que; ser obrigado a}
\end{EntryWithPhonetic}

\begin{EntryWithPhonetic}{只读}{zhi3du2}{5,10}{⼝,⾔}
  \definition{s.}{somente leitura (computação) | \emph{read-only}}
\end{EntryWithPhonetic}

\begin{EntryWithPhonetic}{只顾}{zhi3gu4}{5,10}{⼝,⾴}[HSK 6]
  \definition{adv.}{meramente; simplesmente; apenas se importa com; indica que a atenção está focada em apenas um aspecto}
  \definition{v.}{considerar apenas uma coisa}
\end{EntryWithPhonetic}

\begin{EntryWithPhonetic}{只管}{zhi3guan3}{5,14}{⼝,⽵}[HSK 6]
  \definition{adv.}{por todos os meios; expressa incentivo para que os outros façam algo com confiança, sem se preocuparem com outras coisas | apenas; simplesmente; significa fazer uma coisa com seriedade, sem se preocupar com outras coisas}
\end{EntryWithPhonetic}

\begin{EntryWithPhonetic}{只好}{zhi3hao3}{5,6}{⼝,⼥}[HSK 3]
  \definition{v.}{ter que; ser forçado a; não ter escolha a não ser; significa que só pode ser assim, não há outra opção}
\end{EntryWithPhonetic}

\begin{EntryWithPhonetic}{只见}{zhi3jian4}{5,4}{⼝,⾒}[HSK 5]
  \definition{v.}{somente ver; ver; só vi, e de repente percebi uma certa situação}
\end{EntryWithPhonetic}

\begin{EntryWithPhonetic}{只能}{zhi3neng2}{5,10}{⼝,⾁}[HSK 2]
  \definition{adv.}{só pode; obrigado a fazer algo; isso significa que devido à limitação da capacidade pessoal ou às condições objetivas, não há outra escolha senão esta}
\end{EntryWithPhonetic}

\begin{EntryWithPhonetic}{只怕}{zhi3pa4}{5,8}{⼝,⼼}
  \definition{adv.}{receio que\dots | talvez | muito provavelmente}
\end{EntryWithPhonetic}

\begin{EntryWithPhonetic}{只是}{zhi3shi4}{5,9}{⼝,⽇}[HSK 3]
  \definition{adv.}{somente; meramente; apenas; expressa ênfase limitada a uma determinada situação ou âmbito}
  \definition{conj.}{somente; mas; exceto que; conecta frases, indicando uma ligeira transição, equivalente a 不过}
  \seealsoref{不过}{bu2guo4}
\end{EntryWithPhonetic}

\begin{EntryWithPhonetic}{只消}{zhi3xiao1}{5,10}{⼝,⽔}
  \definition{conj.}{desde que}
\end{EntryWithPhonetic}

\begin{EntryWithPhonetic}{只要}{zhi3yao4}{5,9}{⼝,⾑}[HSK 2]
  \definition{conj.}{desde que; se apenas; contanto que; indica condições necessárias (就 ou 可 são frequentemente usados depois)}
  \seealsoref{便}{bian4}
  \seealsoref{就}{jiu4}
\end{EntryWithPhonetic}

\begin{EntryWithPhonetic}{只要……就……}{zhi3yao4 jiu4}{5,9,12}{⼝,⾑,⼪}
  \definition{conj.}{contanto que/desde que/se somente\dots, então\dots}
\end{EntryWithPhonetic}

\begin{EntryWithPhonetic}{只有}{zhi3you3}{5,6}{⼝,⽉}[HSK 3]
  \definition{adv.}{somente; tem que; forçado a}
  \definition{conj.}{somente se; conecta frases, expressa condições necessárias, geralmente corresponde a 才 e 方}
  \seealsoref{才}{cai2}
  \seealsoref{方}{fang1}
\end{EntryWithPhonetic}

\begin{EntryWithPhonetic}{只有……才……}{zhi3you3 cai2}{5,6,3}{⼝,⽉,⼿}
  \definition{conj.}{só se\dots então\dots}
\end{EntryWithPhonetic}

%%%%%%%%%% 纸 %%%%%%%%%%
\subsection*{纸}\addcontentsline{loh}{figure}{纸 \dpy{zhi3}}

\begin{EntryWithPhonetic}{纸}{zhi3}{7}{⽷}[HSK 2]
  \definition{clas.}{usado para documentos, cartas, etc.}
  \definition[张,沓]{s.}{papel; uma folha fina de material usada para escrever, pintar, imprimir, embalar, etc., feita principalmente de fibras vegetais | papel joss; papel de incenso; refere"-se especificamente a itens supersticiosos, como papel"-moeda}
\end{EntryWithPhonetic}

\begin{EntryWithPhonetic}{纸币}{zhi3bi4}{7,4}{⽷,⼱}
  \definition[张]{s.}{nota (dinheiro) | cédula}
\end{EntryWithPhonetic}

\begin{EntryWithPhonetic}{纸巾}{zhi3jin1}{7,3}{⽷,⼱}
  \definition[张,包]{s.}{lenço | guardanapo | papel toalha}
\end{EntryWithPhonetic}

\begin{EntryWithPhonetic}{纸尿裤}{zhi3niao4ku4}{7,7,12}{⽷,⼫,⾐}
  \definition{s.}{fralda descartável}
\end{EntryWithPhonetic}

\begin{EntryWithPhonetic}{纸烟}{zhi3yan1}{7,10}{⽷,⽕}
  \definition{s.}{cigarro}
\end{EntryWithPhonetic}

\begin{EntryWithPhonetic}{纸张}{zhi3zhang1}{7,7}{⽷,⼸}
  \definition{s.}{papel}
\end{EntryWithPhonetic}

%%%%%%%%%% 指 %%%%%%%%%%
\subsection*{指}\addcontentsline{loh}{figure}{指 \dpy{zhi3}}

\begin{EntryWithPhonetic}{指}{zhi3}{9}{⼿}[HSK 3]
  \definition*{s.}{Sobrenome: Zhi}
  \definition{clas.}{dígito; largura do dedo; a largura de um dedo é chamada de 一指, que é usado para medir profundidade, largura, etc.}
  \definition{s.}{dedo}
  \definition{v.}{apontar para; indicar; usar o dedo ou a ponta de um objeto para apontar | (pelo) eriçar;  (cabelo) ficar em pé | indicar; mostrar; apontar; demonstrar | referir-se a; dirigir-se a | confiar em; contar com; depender de | criticar; repreender}
\end{EntryWithPhonetic}

\begin{EntryWithPhonetic}{指标}{zhi3biao1}{9,9}{⼿,⽊}[HSK 5]
  \definition[个,种]{s.}{meta; cota; norma; índice; objetivos a serem alcançados | alvo; índice; refletir os requisitos de desenvolvimento em determinados aspectos através de números absolutos ou percentagens de aumento ou diminuição, inclui indicadores quantitativos e qualitativos}
\end{EntryWithPhonetic}

\begin{EntryWithPhonetic}{指出}{zhi3chu1}{9,5}{⼿,⼐}[HSK 3]
  \definition{v.}{apontar; indicar}
\end{EntryWithPhonetic}

\begin{EntryWithPhonetic}{指导}{zhi3dao3}{9,6}{⼿,⼨}[HSK 3]
  \definition[位]{s.}{guia; diretor; pessoa que dá orientações}
  \definition{v.}{orientar; dirigir; instruir}
\end{EntryWithPhonetic}

\begin{EntryWithPhonetic}{指定}{zhi3ding4}{9,8}{⼿,⼧}[HSK 6]
  \definition{adv.}{certamente; com certeza; reforça o tom de palpite e estimativa}
  \definition{v.}{nomear; atribuir; determinar a pessoa, evento, lugar, conteúdo, etc. que faz algo}
\end{EntryWithPhonetic}

\begin{EntryWithPhonetic}{指挥}{zhi3hui1}{9,9}{⼿,⼿}[HSK 4]
  \definition[个,位,名]{s.}{diretor; comandante; despachante; operador | maestro; condutor; pessoa na frente de uma orquestra ou coro que dá instruções sobre como tocar ou cantar}
  \definition{v.}{dirigir; conduzir; comandar; direcionar; emitir ordens de agendamento}
\end{EntryWithPhonetic}

\begin{EntryWithPhonetic}{指甲}{zhi3jia5}{9,5}{⼿,⽥}[HSK 5]
  \definition[个,种]{s.}{unha; unha de agulha; unha de dedo; camada córnea na ponta dos dedos}
\end{EntryWithPhonetic}

\begin{EntryWithPhonetic}{指南针}{zhi3nan2zhen1}{9,9,7}{⼿,⼗,⾦}
  \definition{s.}{bússola}
\end{EntryWithPhonetic}

\begin{EntryWithPhonetic}{指示}{zhi3shi4}{9,5}{⼿,⽰}[HSK 5]
  \definition[点,条,项,个]{s.}{diretriz; instruções; para orientar o trabalho, os superiores emitem opiniões verbais ou escritas aos subordinados}
  \definition{v.}{indicar; apontar; apontar para alguém | instruir; superiores emitem opiniões verbais ou escritas para orientar o trabalho dos subordinados}
\end{EntryWithPhonetic}

\begin{EntryWithPhonetic}{指数}{zhi3shu4}{9,13}{⼿,⽁}[HSK 6]
  \definition{s.}{Matemática: expoente; refere"-se ao número de vezes que um número é multiplicado por si mesmo, o que é registrado no canto superior direito do número | Estatística: índice, refere"-se à razão entre o valor de um fenômeno econômico em um determinado período e o valor de outro período usado como padrão de comparação, geralmente expresso como uma porcentagem, como o índice de preços ao consumidor}
\end{EntryWithPhonetic}

\begin{EntryWithPhonetic}{指头}{zhi3tou5}{9,5}{⼿,⼤}[HSK 6]
  \definition[个,根,只]{s.}{dedo da mão ou do pé}
\end{EntryWithPhonetic}

\begin{EntryWithPhonetic}{指责}{zhi3ze2}{9,8}{⼿,⾙}[HSK 5]
  \definition{v.}{censurar; criticar; encontrar falhas; repreender}
\end{EntryWithPhonetic}

\begin{EntryWithPhonetic}{指着}{zhi3zhe5}{9,11}{⼿,⽬}[HSK 6]
  \definition{v.}{apontar}
\end{EntryWithPhonetic}

%%%%%%%%%% 黹 %%%%%%%%%%
\subsection*{黹}\addcontentsline{loh}{figure}{黹 \dpy{zhi3}}

\begin{EntryWithPhonetic}{黹}{zhi3}{12}{⿋}[Kangxi 204]
  \definition{v.}{costurar; bordar}
\end{EntryWithPhonetic}

%%%%%%%%%% 至 %%%%%%%%%%
\subsection*{至}\addcontentsline{loh}{figure}{至 \dpy{zhi4}}

\begin{EntryWithPhonetic}{至}{zhi4}{6}{⾄}[HSK 5][Kangxi 133]
  \definition{adv.}{a maior parte; extremamente; indica o grau mais alto, equivalente a 极 ou 最}
  \definition{prep.}{para; até; chegar a um determinado ponto}
  \definition{s.}{extremo, máximo}
  \definition{v.}{chegar; alcançar}
  \seealsoref{极}{ji2}
  \seealsoref{最}{zui4}
\end{EntryWithPhonetic}

\begin{EntryWithPhonetic}{至今}{zhi4jin1}{6,4}{⾄,⼈}[HSK 3]
  \definition{adv.}{até agora; até o momento; até hoje}
\end{EntryWithPhonetic}

\begin{EntryWithPhonetic}{至少}{zhi4shao3}{6,4}{⾄,⼩}[HSK 3]
  \definition{adv.}{pelo menos; indica o limite mínimo}
\end{EntryWithPhonetic}

\begin{EntryWithPhonetic}{至于}{zhi4yu2}{6,3}{⾄,⼆}[HSK 6]
  \definition{adv.}{quanto a; na medida em que; indica atingir um certo nível, frequentemente usado em frases negativas e perguntas retóricas}
  \definition{conj.}{a respeito de; quanto a; geralmente é usado entre duas frases ou no início da frase seguinte para introduzir ou mudar um novo tópico}[至于他的计划,我不太了解。===Quanto aos seus planos, não sei muito sobre eles.]
\end{EntryWithPhonetic}

%%%%%%%%%% 志 %%%%%%%%%%
\subsection*{志}\addcontentsline{loh}{figure}{志 \dpy{zhi4}}

\begin{EntryWithPhonetic}{志}{zhi4}{7}{⼼}
  \definition{s.}{vontade; ideal; aspiração; ambição | anais; registros; transcrição | marca; sinal}
  \definition{v.}{ter em mente; lembrar | pesar; medir comprimento e quantidade}
\end{EntryWithPhonetic}

\begin{EntryWithPhonetic}{志愿}{zhi4yuan4}{7,14}{⼼,⽕}[HSK 3]
  \definition{s.}{desejo; ideal; aspiração; os ideais, desejos ou objetivos que se deseja realizar no coração}
  \definition{v.}{ser voluntário; ser proativo e disposto a realizar trabalhos sem remuneração ou com remuneração baixa, mas que possam ajudar outras pessoas}
\end{EntryWithPhonetic}

\begin{EntryWithPhonetic}{志愿书}{zhi4yuan4shu1}{7,14,4}{⼼,⽕,⼄}
  \definition{s.}{formulário de inscrição; formulário de adesão; carta de intenções}
\end{EntryWithPhonetic}

\begin{EntryWithPhonetic}{志愿者}{zhi4yuan4zhe3}{7,14,8}{⼼,⽕,⽼}[HSK 3]
  \definition[名,位,个]{s.}{voluntário; pessoas que se voluntariam para prestar serviços em atividades sociais, grandes eventos esportivos, conferências, etc.}
\end{EntryWithPhonetic}

%%%%%%%%%% 识 %%%%%%%%%%
\subsection*{识}\addcontentsline{loh}{figure}{识 \dpy{zhi4}}

\begin{EntryWithPhonetic}{识}{zhi4}{7}{⾔}
  \definition{s.}{marca; sinal; símbolo}
  \definition{v.}{lembrar; memorizar | anotar; registrar}
  \seeref{shi2}
\end{EntryWithPhonetic}

%%%%%%%%%% 制 %%%%%%%%%%
\subsection*{制}\addcontentsline{loh}{figure}{制 \dpy{zhi4}}

\begin{EntryWithPhonetic}{制}{zhi4}{8}{⼑}
  \definition[套,项]{s.}{sistema | regras; regulamentos}
  \definition{v.}{formular; elaborar | fazer; fabricar | restringir; limitar; controlar; disciplinar}
\end{EntryWithPhonetic}

\begin{EntryWithPhonetic}{制裁}{zhi4cai2}{8,12}{⼑,⾐}
  \definition{s.}{punição | sanção (inclusive econômica)}
  \definition{v.}{punir}
\end{EntryWithPhonetic}

\begin{EntryWithPhonetic}{制成}{zhi4cheng2}{8,6}{⼑,⼽}[HSK 5]
  \definition{v.}{fabricar; ser feito de; produzir}
\end{EntryWithPhonetic}

\begin{EntryWithPhonetic}{制订}{zhi4ding4}{8,4}{⼑,⾔}[HSK 4]
  \definition{v.}{esboçar; formular; elaborar; mapear}
\end{EntryWithPhonetic}

\begin{EntryWithPhonetic}{制定}{zhi4ding4}{8,8}{⼑,⼧}[HSK 3]
  \definition{v.}{rascunhar; formular; elaborar; estabelecer (leis, regulamentos, planos, etc.)}
\end{EntryWithPhonetic}

\begin{EntryWithPhonetic}{制度}{zhi4du4}{8,9}{⼑,⼴}[HSK 3]
  \definition[项,条,套,种]{s.}{regulamentação; regulamento; procedimentos operacionais ou diretrizes de conduta que todos devem seguir | sistema; o sistema político, econômico e cultural formado sob determinadas condições históricas}
\end{EntryWithPhonetic}

\begin{EntryWithPhonetic}{制约}{zhi4yue1}{8,6}{⼑,⽷}[HSK 5]
  \definition{v.}{limitar; verificar; restringir; a existência e a mudança de uma coisa determinam a existência e a mudança de outra coisa}
\end{EntryWithPhonetic}

\begin{EntryWithPhonetic}{制造}{zhi4zao4}{8,10}{⼑,⾡}[HSK 3]
  \definition{v.}{fazer; produzir; manufaturar; transformar matérias-primas em produtos acabados | criar; agitar; criar artificialmente uma situação ou atmosfera desfavorável}
\end{EntryWithPhonetic}

\begin{EntryWithPhonetic}{制作}{zhi4zuo4}{8,7}{⼑,⼈}[HSK 3]
  \definition{v.}{fazer; produzir; itens feitos com matérias-primas, geralmente pequenos e feitos à mão | fazer; produzir; criar gráficos, anúncios, filmes, jogos, etc., utilizando texto, imagens, sons, imagens, etc.}
\end{EntryWithPhonetic}

%%%%%%%%%% 治 %%%%%%%%%%
\subsection*{治}\addcontentsline{loh}{figure}{治 \dpy{zhi4}}

\begin{EntryWithPhonetic}{治}{zhi4}{8}{⽔}[HSK 4]
  \definition*{s.}{Sobrenome: Zhi}
  \definition{adj.}{calmo e pacífico}
  \definition{s.}{sede de um antigo governo local}
  \definition{v.}{reger; administrar; governar; gerenciar; gerir | tratar (uma doença); curar; sarar | eliminar; controlar pragas | controlar (um rio); restaurar um curso d'água por meio de dragagem | punir; castigar | estudar; pesquisar; explorar}
\end{EntryWithPhonetic}

\begin{EntryWithPhonetic}{治安}{zhi4'an1}{8,6}{⽔,⼧}[HSK 5]
  \definition{s.}{ordem pública; segurança pública; ordem social estável}
\end{EntryWithPhonetic}

\begin{EntryWithPhonetic}{治病}{zhi4bing4}{8,10}{⽔,⽧}[HSK 6]
  \definition{v.}{tratar uma doença; eliminar doenças por meio de medicamentos, cirurgias, etc.}
\end{EntryWithPhonetic}

\begin{EntryWithPhonetic}{治理}{zhi4li3}{8,11}{⽔,⽟}[HSK 5]
  \definition{s.}{governo | governança}
  \definition{v.}{dirigir; gerenciar; governar; administrar | tratar; aproveitar; colocar sob controle; colocar em ordem}
\end{EntryWithPhonetic}

\begin{EntryWithPhonetic}{治疗}{zhi4liao2}{8,7}{⽔,⽧}[HSK 4]
  \definition{s.}{diagnóstico; tratamento}
  \definition{v.}{tratar; curar; remediar; eliminar doenças por meio de medicamentos, cirurgia, etc.}
\end{EntryWithPhonetic}

\begin{EntryWithPhonetic}{治愈}{zhi4yu4}{8,13}{⽔,⼼}
  \definition{v.}{curar | restaurar a saúde}
\end{EntryWithPhonetic}

%%%%%%%%%% 质 %%%%%%%%%%
\subsection*{质}\addcontentsline{loh}{figure}{质 \dpy{zhi4}}

\begin{EntryWithPhonetic}{质}{zhi4}{8}{⾙}
  \definition*{s.}{Sobrenome: Zhi}
  \definition{adj.}{simples; claro; sem adornos}
  \definition{s.}{natureza; caráter; essência; substância | qualidade | matéria; substância | segurança; penhor; garantia}
  \definition{v.}{penhorar | hipotecar | questionar; chamar à responsabilidade; acusar}
\end{EntryWithPhonetic}

\begin{EntryWithPhonetic}{质量}{zhi4liang4}{8,12}{⾙,⾥}[HSK 4]
  \definition{s.}{qualidade; o quão bom ou ruim é o produto ou o trabalho | Física: massa}
\end{EntryWithPhonetic}

%%%%%%%%%% 致 %%%%%%%%%%
\subsection*{致}\addcontentsline{loh}{figure}{致 \dpy{zhi4}}

\begin{EntryWithPhonetic}{致}{zhi4}{10}{⾄}
  \definition{adj.}{fino; delicado; meticuloso; preciso}
  \definition{s.}{interesse}
  \definition{v.}{enviar; estender; entregar; dar; mostrar (cortesia, afeto, etc.) à outra parte | concentrar-se; trabalhar para; dedicar (os próprios esforços, etc.); focar em um aspecto | causar; incorrer; convidar; levar a | alcançar}
\end{EntryWithPhonetic}

\begin{EntryWithPhonetic}{致敬}{zhi4jing4}{10,12}{⾄,⽁}
  \definition{v.}{saudar; prestar homenagem a; demonstrar respeito (homenagem) a}
\end{EntryWithPhonetic}

%%%%%%%%%% 智 %%%%%%%%%%
\subsection*{智}\addcontentsline{loh}{figure}{智 \dpy{zhi4}}

\begin{EntryWithPhonetic}{智}{zhi4}{12}{⽇}
  \definition*{s.}{Sobrenome: Zhi}
  \definition{adj.}{engenhoso; sábio; inteligente; astuto}
  \definition{s.}{discernimento; engenhosidade; sagacidade | inteligência; conhecimento; sabedoria; percepção}
\end{EntryWithPhonetic}

\begin{EntryWithPhonetic}{智慧}{zhi4hui4}{12,15}{⽇,⼼}[HSK 6]
  \definition[种]{s.}{sagacidade; sabedoria; inteligência; capacidade de analisar, julgar, inventar, criar e resolver problemas}
\end{EntryWithPhonetic}

\begin{EntryWithPhonetic}{智力}{zhi4li4}{12,2}{⽇,⼒}[HSK 4]
  \definition{s.}{inteligência; refere"-se à capacidade de uma pessoa de conhecer e entender coisas objetivas e aplicar o conhecimento e a experiência para resolver problemas, incluindo memória, observação, imaginação, pensamento e julgamento}
\end{EntryWithPhonetic}

\begin{EntryWithPhonetic}{智能}{zhi4neng2}{12,10}{⽇,⾁}[HSK 4]
  \definition{adj.}{inteligente (telefone, sistema, etc.); descreve máquinas, equipamentos, tecnologia, etc. que foram processados com alta tecnologia e têm a capacidade de falar, pensar, calcular, resolver problemas, etc., como um ser humano}
  \definition{s.}{intelecto; a capacidade de aprender, agir, pensar, inventar, criar, resolver problemas, etc.}
\end{EntryWithPhonetic}

\begin{EntryWithPhonetic}{智商}{zhi4shang1}{12,11}{⽇,⼝}
  \definition{s.}{quociente de inteligência, QI}
\end{EntryWithPhonetic}

\begin{EntryWithPhonetic}{智障}{zhi4zhang4}{12,13}{⽇,⾩}
  \definition{adj./s.}{retardado}
\end{EntryWithPhonetic}

%%%%%%%%%% 置 %%%%%%%%%%
\subsection*{置}\addcontentsline{loh}{figure}{置 \dpy{zhi4}}

\begin{EntryWithPhonetic}{置}{zhi4}{13}{⽹}
  \definition{v.}{colocar | configurar; estabelecer; instalar | comprar | organizar; consertar}
\end{EntryWithPhonetic}

\begin{EntryWithPhonetic}{置疑}{zhi4yi2}{13,14}{⽹,⽦}
  \definition{v.}{duvidar}
\end{EntryWithPhonetic}

%%%%%%%%%% 中 %%%%%%%%%%
\subsection*{中}\addcontentsline{loh}{figure}{中 \dpy{zhong1}}

\begin{EntryWithPhonetic}{中}{zhong1}{4}{⼁}[HSK 1]
  \definition*{s.}{China; referindo"-se à China | Sobrenome: Zhong}
  \definition{adj.}{\emph{O.K.}; tudo bem, ótimo, adequado}
  \definition{s.}{centro; meio; a parte que está à mesma distância de todos os lados, acima e abaixo ou nas duas extremidades | em; entre (usado para indicar coisas dentro de um determinado intervalo) | meio; centro; localizado entre os dois extremos | médio; intermediário; classificação entre os dois extremos | médio; a meio caminho entre dois extremos; imparcial | intermediário | enquanto; durante (use após um verbo para mostrar que a ação está em andamento)}
  \definition{v.}{ser adequado para; ser compatível com}
  \seeref{zhong4}
  \seealsoref{中国}{zhong1guo2}
  \antonymref{外}{wai4}
  \antonymref{西}{xi1}
  \antonymref{洋}{yang2}
\end{EntryWithPhonetic}

\begin{EntryWithPhonetic}{中部}{zhong1bu4}{4,10}{⼁,⾢}[HSK 3]
  \definition{s.}{parte do meio; região central; seção central; região ou parte intermediária | parte do meio; parte central; refere"-se à parte intermediária de uma série de três partes, como romances, obras cinematográficas e televisivas}
\end{EntryWithPhonetic}

\begin{EntryWithPhonetic}{中餐}{zhong1can1}{4,16}{⼁,⾷}[HSK 2]
  \definition[份,顿,桌]{s.}{refeição chinesa; comida chinesa; comida de estilo chinês (diferente de 西餐) | almoço}
  \seealsoref{西餐}{xi1can1}
  \antonymref{西餐}{xi1can1}
\end{EntryWithPhonetic}

\begin{EntryWithPhonetic}{中等}{zhong1deng3}{4,12}{⼁,⽵}[HSK 6]
  \definition{adj.}{médio; a nota está entre superior e inferior ou entre avançado e elementar | de altura média; nem baixo nem alto}
\end{EntryWithPhonetic}

\begin{EntryWithPhonetic}{中东}{zhong1dong1}{4,5}{⼁,⼀}
  \definition*{s.}{Oriente Médio}
\end{EntryWithPhonetic}

\begin{EntryWithPhonetic}{中断}{zhong1duan4}{4,11}{⼁,⽄}[HSK 5]
  \definition{v.}{suspender; romper; descontinuar; interromper; quebrar | dividir; quebrar; ser quebrado}
  \synonymref{打断}{da3 duan4}
  \synonymref{结束}{jie2shu4}
  \synonymref{拒绝}{ju4jue2}
  \synonymref{收缩}{shou1suo1}
  \synonymref{停留}{ting2liu2}
  \synonymref{停止}{ting2zhi3}
  \synonymref{终止}{zhong1zhi3}
  \antonymref{不停}{bu4ting2}
  \antonymref{持续}{chi2xu4}
  \antonymref{继续}{ji4xu4}
  \antonymref{连续}{lian2xu4}
  \antonymref{陆续}{lu4xu4}
  \antonymref{延续}{yan2xu4}
  \antonymref{一直}{yi4zhi2}
\end{EntryWithPhonetic}

\begin{EntryWithPhonetic}{中国}{zhong1guo2}{4,8}{⼁,⼞}[HSK 1]
  \definition*{s.}{China; os povos Huaxia e Han estabeleceram suas capitais principalmente ao sul e ao norte do rio Huang He, e por isso chamaram essa região de 中国, com o mesmo significado de 中土, 中原l, 中州 e 中华}
  \seealsoref{中土}{zhong1 tu3}
  \seealsoref{中华}{zhong1hua2}
  \seealsoref{中原}{zhong1yuan2}
  \seealsoref{中州}{zhong1zhou1}
  \synonymref{华夏}{hua2xia4}
  \synonymref{中华}{zhong1hua2}
\end{EntryWithPhonetic}

\begin{EntryWithPhonetic}{中国城}{zhong1guo2cheng2}{4,8,9}{⼁,⼞,⼟}
  \definition*[座]{s.}{Bairro Chinês, Chinatown}
  \seealsoref{唐人街}{tang2ren2 jie1}
\end{EntryWithPhonetic}

\begin{EntryWithPhonetic}{中国科学院}{zhong1guo2 ke1xue2yuan4}{4,8,9,8,9}{⼁,⼞,⽲,⼦,⾩}
  \definition*{s.}{Academia Chinesa de Ciências}
\end{EntryWithPhonetic}

\begin{EntryWithPhonetic}{中国人}{zhong1guo2ren2}{4,8,2}{⼁,⼞,⼈}
  \definition{s.}{chinês | pessoa ou povo da China}
\end{EntryWithPhonetic}

\begin{EntryWithPhonetic}{中国通}{zhong1guo2tong1}{4,8,10}{⼁,⼞,⾡}
  \definition*{s.}{Sinólogo | Conhecedor da China, especialista em tudo sobre a China; refere"-se a estrangeiros que estão familiarizados com a China}
\end{EntryWithPhonetic}

\begin{EntryWithPhonetic}{中华}{zhong1hua2}{4,6}{⼁,⼗}[HSK 6]
  \definition{s.}{China; na antiguidade, a região do rio Amarelo era chamada de Zhonghua, sendo o local onde a etnia Han surgiu inicialmente. Posteriormente, passou a designar a China}
  \synonymref{华夏}{hua2xia4}
  \synonymref{中国}{zhong1guo2}
\end{EntryWithPhonetic}

\begin{EntryWithPhonetic}{中华民族}{zhong1hua2min2zu2}{4,6,5,11}{⼁,⼗,⽒,⽅}[HSK 3]
  \definition*{s.}{O Povo Chinês; o nome genérico para todas as etnias da China, incluindo 56 etnias, com uma longa história, um rico patrimônio cultural e uma gloriosa tradição revolucionária | A Nação Chinesa}
\end{EntryWithPhonetic}

\begin{EntryWithPhonetic}{中级}{zhong1ji2}{4,6}{⼁,⽷}[HSK 2]
  \definition{adj.}{nível médio; nível intermediário; entre avançado e iniciante}
\end{EntryWithPhonetic}

\begin{EntryWithPhonetic}{中间}{zhong1jian1}{4,7}{⼁,⾨}[HSK 1]
  \definition[本]{s.}{em meio a; entre; dentro de um determinado intervalo | meio; centro; posição entre os dois extremos de uma coisa ou entre duas coisas | posição intermediária; espaço entre duas extremidades; posição entre os dois extremos de uma coisa, dois momentos, duas coisas, etc.}
  \synonymref{当中}{dang1zhong1}
  \synonymref{之间}{zhi1jian1}
  \synonymref{中心}{zhong1xin1}
  \synonymref{中央}{zhong1yang1}
  \antonymref{角落}{jiao3luo4}
  \antonymref{旁边}{pang2bian1}
  \antonymref{四周}{si4zhou1}
  \antonymref{周围}{zhou1wei2}
\end{EntryWithPhonetic}

\begin{EntryWithPhonetic}{中介}{zhong1jie4}{4,4}{⼁,⼈}[HSK 4]
  \definition[个]{s.}{agente; intermediário; o meio de conexão}
\end{EntryWithPhonetic}

\begin{EntryWithPhonetic}{中年}{zhong1nian2}{4,6}{⼁,⼲}[HSK 2]
  \definition{s.}{meia-idade; na casa dos quarenta ou cinquenta anos}
\end{EntryWithPhonetic}

\begin{EntryWithPhonetic}{中期}{zhong1qi1}{4,12}{⼁,⽉}[HSK 6]
  \definition{s.}{período intermediário | médio prazo (plano, previsão etc.); meio-termo | meio (de um período de tempo)}
\end{EntryWithPhonetic}

\begin{EntryWithPhonetic}{中情局}{zhong1qing2ju2}{4,11,7}{⼁,⼼,⼫}
  \definition*{s.}{Agência Central de Inteligência dos EUA, CIA, abreviação de 中央情报局}
  \seealsoref{中央情报局}{zhong1yang1 qing2bao4ju2}
\end{EntryWithPhonetic}

\begin{EntryWithPhonetic}{中秋节}{zhong1qiu1jie2}{4,9,5}{⼁,⽲,⾋}[HSK 5]
  \definition*[个]{s.}{Festival do Meio-Outono | Festival do Bolo Lunar (15º dia do oitavo mês lunar)}
\end{EntryWithPhonetic}

\begin{EntryWithPhonetic}{中土}{zhong1 tu3}{4,3}{⼁,⼟}
  \definition*{s.}{Obsoleto: China (Terra Média) | Sino-turco}
\end{EntryWithPhonetic}

\begin{EntryWithPhonetic}{中外}{zhong1wai4}{4,5}{⼁,⼣}[HSK 6]
  \definition{s.}{China e países estrangeiros}[这里吸引了许多中外游客。===Este lugar atrai muitos turistas chineses e estrangeiros.]
  \synonymref{内外}{nei4wai4}
\end{EntryWithPhonetic}

\begin{EntryWithPhonetic}{中文}{zhong1wen2}{4,4}{⼁,⽂}[HSK 1]
  \definition{s.}{a língua chinesa; chinês; a língua e a escrita da China; especificamente, o chinês e os caracteres chineses}
  \antonymref{外文}{wai4wen2}
\end{EntryWithPhonetic}

\begin{EntryWithPhonetic}{中午}{zhong1wu3}{4,4}{⼁,⼗}[HSK 1]
  \definition[个]{s.}{meio"-dia; refere"-se ao período entre 12 e 13 horas}
  \antonymref{半夜}{ban4ye4}
\end{EntryWithPhonetic}

\begin{EntryWithPhonetic}{中小学}{zhong1xiao3xue2}{4,3,8}{⼁,⼩,⼦}[HSK 2]
  \definition{s.}{escolas primárias e secundárias}
\end{EntryWithPhonetic}

\begin{EntryWithPhonetic}{中心}{zhong1xin1}{4,4}{⼁,⼼}[HSK 2]
  \definition[个]{s.}{núcleo; coração; meio; centro; posições com distâncias iguais de todas as áreas circundantes | chave; coração; a parte principal de algo; a pessoa ou coisa que desempenha um papel importante | centro; uma cidade ou lugar que tem um impacto significativo ou desempenha um papel importante em uma determinada área | centro; uma instituição com equipamentos e tecnologia relativamente completos e avançados em uma determinada área}
  \synonymref{核心}{he2xin1}
  \synonymref{焦点}{jiao1dian3}
  \synonymref{中间}{zhong1jian1}
  \synonymref{中央}{zhong1yang1}
  \synonymref{重点}{zhong4dian3}
  \synonymref{主题}{zhu3ti2}
  \antonymref{边缘}{bian1yuan2}
  \antonymref{周围}{zhou1wei2}
\end{EntryWithPhonetic}

\begin{EntryWithPhonetic}{中性}{zhong1 xing4}{4,8}{⼁,⼼}
  \definition{adj.}{neutro | (penteado, roupa, etc.) unissex}
\end{EntryWithPhonetic}

\begin{EntryWithPhonetic}{中学}{zhong1xue2}{4,8}{⼁,⼦}[HSK 1]
  \definition[个]{s.}{escola ensino médio; escolas onde os jovens recebem educação secundária geralmente incluem o ensino fundamental II e o ensino médio | aprendizagem chinesa (um termo da dinastia Qing tardia para a aprendizagem tradicional chinesa); antigamente, referia"-se à academia tradicional da China, como filosofia, linguística, medicina tradicional chinesa, etc.}
\end{EntryWithPhonetic}

\begin{EntryWithPhonetic}{中学生}{zhong1xue2sheng1}{4,8,5}{⼁,⼦,⽣}[HSK 1]
  \definition{s.}{aluno, estudante do ensino médio; alunos matriculados no ensino médio. Inclui alunos do ensino fundamental II e do ensino médio}
\end{EntryWithPhonetic}

\begin{EntryWithPhonetic}{中旬}{zhong1xun2}{4,6}{⼁,⽇}
  \definition[月]{adv.}{meio do mês; os dez dias centrais de um mês; o período do dia 11 ao dia 20 de cada mês}
  \seealsoref{上旬}{shang4xun2}
  \seealsoref{下旬}{xia4xun2}
\end{EntryWithPhonetic}

\begin{EntryWithPhonetic}{中央}{zhong1yang1}{4,5}{⼁,⼤}[HSK 5]
  \definition{s.}{centro; meio; localização central | autoridades centrais; refere"-se especificamente ao órgão máximo de liderança de um país ou partido político}
  \synonymref{核心}{he2xin1}
  \synonymref{焦点}{jiao1dian3}
  \synonymref{中间}{zhong1jian1}
  \synonymref{中心}{zhong1xin1}
  \synonymref{主题}{zhu3ti2}
  \antonymref{边缘}{bian1yuan2}
  \antonymref{地方}{di4fang1}
  \antonymref{地方}{di4fang5}
  \antonymref{角落}{jiao3luo4}
  \antonymref{四周}{si4zhou1}
  \antonymref{周围}{zhou1wei2}
\end{EntryWithPhonetic}

\begin{EntryWithPhonetic}{中央情报局}{zhong1yang1 qing2bao4ju2}{4,5,11,7,7}{⼁,⼤,⼼,⼿,⼫}
  \definition*{s.}{Agência Central de Inteligência dos EUA, CIA}
\end{EntryWithPhonetic}

\begin{EntryWithPhonetic}{中药}{zhong1yao4}{4,9}{⼁,⾋}[HSK 5]
  \definition[些,种]{s.}{medicina herbal; medicina tradicional chinesa; fitoterapia; medicamentos utilizados na medicina tradicional chinesa; incluem medicamentos naturais e seus produtos processados (em contraste com 西药)}
  \seealsoref{西药}{xi1 yao4}
  \antonymref{西药}{xi1 yao4}
\end{EntryWithPhonetic}

\begin{EntryWithPhonetic}{中医}{zhong1yi1}{4,7}{⼁,⼖}[HSK 2]
  \definition[位,个,名,些,群]{s.}{ciência médica tradicional chinesa; medicina chinesa | médico de medicina tradicional chinesa; praticante de medicina chinesa}
  \antonymref{西医}{xi1yi1}
\end{EntryWithPhonetic}

\begin{EntryWithPhonetic}{中游}{zhong1you2}{4,12}{⼁,⽔}
  \definition{s.}{seção central do rio; trecho médio (de um rio); meio do curso | o estado de ser mediano; estado intermediário}
\end{EntryWithPhonetic}

\begin{EntryWithPhonetic}{中原}{zhong1yuan2}{4,10}{⼁,⼚}
  \definition*{s.}{Planícies Centrais (compreendendo os trechos médio e inferior do rio Huang He) | Planície Central, regiões média e baixa do rio Amarelo, incluindo Henan, oeste de Shandong, sul de Shanxi e Hebei}
\end{EntryWithPhonetic}

\begin{EntryWithPhonetic}{中州}{zhong1zhou1}{4,6}{⼁,⼮}
  \definition*{s.}{Região Central, ou seja, província de Henan, devido à sua localização central no país}
\end{EntryWithPhonetic}

%%%%%%%%%% 忠 %%%%%%%%%%
\subsection*{忠}\addcontentsline{loh}{figure}{忠 \dpy{zhong1}}

\begin{EntryWithPhonetic}{忠}{zhong1}{8}{⼼}
  \definition{adj.}{leal; fiel; devotado | honesto}
\end{EntryWithPhonetic}

\begin{EntryWithPhonetic}{忠心}{zhong1xin1}{8,4}{⼼,⼼}[HSK 6]
  \definition{s.}{lealdade; devoção; fidelidade}
\end{EntryWithPhonetic}

%%%%%%%%%% 终 %%%%%%%%%%
\subsection*{终}\addcontentsline{loh}{figure}{终 \dpy{zhong1}}

\begin{EntryWithPhonetic}{终}{zhong1}{8}{⽷}
  \definition*{s.}{Sobrenome: Zhong}
  \definition{adj.}{tudo; todo; inteiro; o tempo todo do começo ao fim}
  \definition{adv.}{afinal; no final; eventualmente; finalmente}
  \definition{s.}{fim; término | tempo todo; período inteiro; o tempo todo | final | morte; refere"-se à morte}
\end{EntryWithPhonetic}

\begin{EntryWithPhonetic}{终点}{zhong1dian3}{8,9}{⽷,⽕}[HSK 5]
  \definition[个]{s.}{destino; ponto terminal; ponto de chegada; lugar onde uma jornada termina | final; refere"-se especificamente ao local onde a corrida é interrompida}
\end{EntryWithPhonetic}

\begin{EntryWithPhonetic}{终究}{zhong1jiu1}{8,7}{⽷,⽳}
  \definition{adv.}{afinal de contas; enfatiza que, não importa o que aconteça, a natureza das pessoas e das coisas não mudará e que as características básicas devem ser reconhecidas (tem o efeito de fortalecer o tom) |  no final; indica que um determinado resultado ocorrerá ou não, frequentemente usado em especulações, julgamentos etc. | afinal de contas; indica que, apesar do grande esforço ou da grande esperança, o resultado objetivo ainda é insatisfatório, geralmente com o significado de pesar ou pena | afinal de contas; indica que um resultado desejado finalmente aparece}
\end{EntryWithPhonetic}

\begin{EntryWithPhonetic}{终身}{zhong1shen1}{8,7}{⽷,⾝}[HSK 5]
  \definition{s.}{vida inteira; por toda a vida; por toda a vida}
\end{EntryWithPhonetic}

\begin{EntryWithPhonetic}{终于}{zhong1yu2}{8,3}{⽷,⼆}[HSK 3]
  \definition{adv.}{finalmente; por fim; eventualmente; no final; indica uma situação que surge após várias mudanças ou espera}
\end{EntryWithPhonetic}

\begin{EntryWithPhonetic}{终止}{zhong1zhi3}{8,4}{⽷,⽌}[HSK 5]
  \definition{v.}{parar; terminar | anular; encerrar; expirar; revogar}
\end{EntryWithPhonetic}

%%%%%%%%%% 钟 %%%%%%%%%%
\subsection*{钟}\addcontentsline{loh}{figure}{钟 \dpy{zhong1}}

\begin{EntryWithPhonetic}{钟}{zhong1}{9}{⾦}[HSK 3]
  \definition*{s.}{Sobrenome: Zhong}
  \definition[顶,个,口]{s.}{sino; campainha; um instrumento de percussão antigo, oco, feito de cobre ou ferro | relógio; um aparelho para medir o tempo que não se leva consigo | tempo medido em horas e minutos; referindo"-se ao tempo ou momento| um recipiente antigo para guardar vinho, com barriga grande e gargalo pequeno | sino; refere"-se especificamente aos sinos pendurados em templos ou outros locais, cujo som é usado para marcar as horas, alertar ou convocar pessoas}
  \definition{v.}{focar; concentrar (as afeições de alguém, etc.)}
\end{EntryWithPhonetic}

\begin{EntryWithPhonetic}{钟室}{zhong1shi4}{9,9}{⾦,⼧}
  \definition{s.}{campanário | sala do relógio}
\end{EntryWithPhonetic}

\begin{EntryWithPhonetic}{钟头}{zhong1tou2}{9,5}{⾦,⼤}[HSK 6]
  \definition[个]{s.}{hora; 60 minutos}[三四个钟头过去了。===Três ou quatro horas se passaram.]
\end{EntryWithPhonetic}

\begin{EntryWithPhonetic}{钟罩}{zhong1zhao4}{9,13}{⾦,⽹}
  \definition{s.}{redoma | dossel de sino}
\end{EntryWithPhonetic}

%%%%%%%%%% 锺 %%%%%%%%%%
\subsection*{锺}\addcontentsline{loh}{figure}{锺 \dpy{zhong1}}

\begin{EntryWithPhonetic}{锺}{zhong1}{14}{⾦}
  \variantof{钟}
\end{EntryWithPhonetic}

%%%%%%%%%% 肿 %%%%%%%%%%
\subsection*{肿}\addcontentsline{loh}{figure}{肿 \dpy{zhong3}}

\begin{EntryWithPhonetic}{肿}{zhong3}{8}{⾁}[HSK 6]
  \definition{s.}{inchaço; protuberância}
  \definition{v.}{inchar; estar inchado}
\end{EntryWithPhonetic}

%%%%%%%%%% 种 %%%%%%%%%%
\subsection*{种}\addcontentsline{loh}{figure}{种 \dpy{zhong3}}

\begin{EntryWithPhonetic}{种}{zhong3}{9}{⽲}[HSK 3]
  \definition{clas.}{indica tipo, usado para pessoas e qualquer coisa}
  \definition{s.}{espécie | etnia | semente; estirpe; linhagem; material para reprodução biológica em cadeia | coragem; determinação; garra; força de caráter; refere"-se à coragem ou determinação}
  \seeref{zhong4}
\end{EntryWithPhonetic}

\begin{EntryWithPhonetic}{种类}{zhong3lei4}{9,9}{⽲,⽶}[HSK 4]
  \definition[个]{s.}{espécie; classe; tipo; variedade; categoria; classificação de alguma coisa de acordo com sua natureza e características}
\end{EntryWithPhonetic}

\begin{EntryWithPhonetic}{种麻}{zhong3ma2}{9,11}{⽲,⿇}
  \definition{s.}{planta de cânhamo (feminina)}
\end{EntryWithPhonetic}

\begin{EntryWithPhonetic}{种薯}{zhong3shu3}{9,16}{⽲,⾋}
  \definition{s.}{tubérculo semente}
\end{EntryWithPhonetic}

\begin{EntryWithPhonetic}{种种}{zhong3zhong3}{9,9}{⽲,⽲}[HSK 6]
  \definition{adj.}{todos os tipos de; uma variedade de}
\end{EntryWithPhonetic}

\begin{EntryWithPhonetic}{种子}{zhong3zi5}{9,3}{⽲,⼦}[HSK 3]
  \definition[颗,粒,个,号]{s.}{semente; um órgão exclusivo de certas plantas, geralmente composto de três partes: tegumento, embrião e endosperma, as sementes podem germinar e se tornar novas plantas sob certas condições | jogador classificado; durante a competição, nas eliminatórias, os jogadores mais fortes de cada equipe são escalados}
\end{EntryWithPhonetic}

\begin{EntryWithPhonetic}{种族灭绝}{zhong3zu2mie4jue2}{9,11,5,9}{⽲,⽅,⽕,⽷}
  \definition{s.}{genocídio | extinção étnica}
\end{EntryWithPhonetic}

%%%%%%%%%% 中 %%%%%%%%%%
\subsection*{中}\addcontentsline{loh}{figure}{中 \dpy{zhong4}}

\begin{EntryWithPhonetic}{中}{zhong4}{4}{⼁}
  \definition{v.}{acertar; encaixar perfeitamente | sofrer; cair em; ser atingido por; ser afetado por}
  \seeref{zhong1}
  \antonymref{外}{wai4}
  \antonymref{西}{xi1}
  \antonymref{洋}{yang2}
\end{EntryWithPhonetic}

\begin{EntryWithPhonetic}{中毒}{zhong4 du2}{4,9}{⼁,⽏}[HSK 5]
  \definition{v.}{envenenar; intoxicar | ser envenenado; ideia de que o pensamento foi contaminado}
  \synonymref{毒害}{du2hai4}
  \synonymref{腹泻}{fu4xie4}
  \synonymref{呕吐}{ou3tu4}
\end{EntryWithPhonetic}

\begin{EntryWithPhonetic}{中奖}{zhong4 jiang3}{4,9}{⼁,⼤}[HSK 4]
  \definition{v.}{ganhar um prêmio (em uma loteria, etc.)}
\end{EntryWithPhonetic}

\begin{EntryWithPhonetic}{中意}{zhong4yi4}{4,13}{⼁,⼼}
  \definition{s.}{ser do seu agrado | começar a gostar muito de algo ou de alguém}
  \synonymref{满意}{man3yi4}
\end{EntryWithPhonetic}

%%%%%%%%%% 众 %%%%%%%%%%
\subsection*{众}\addcontentsline{loh}{figure}{众 \dpy{zhong4}}

\begin{EntryWithPhonetic}{众}{zhong4}{6}{⼈}
  \definition*{s.}{Câmara dos Deputados, abreviação de 众议院}
  \definition{adj.}{numerosos}
  \definition{s.}{multidão; as massas}
  \seealsoref{众议院}{zhong4yi4yuan4}
\end{EntryWithPhonetic}

\begin{EntryWithPhonetic}{众多}{zhong4duo1}{6,6}{⼈,⼣}[HSK 5]
  \definition{adj.}{muitos; numerosos; multitudinários}
\end{EntryWithPhonetic}

\begin{EntryWithPhonetic}{众议院}{zhong4yi4yuan4}{6,5,9}{⼈,⾔,⾩}
  \definition*{s.}{Casa baixa da Assembléia Bicameral | Câmara dos Deputados}
\end{EntryWithPhonetic}

%%%%%%%%%% 种 %%%%%%%%%%
\subsection*{种}\addcontentsline{loh}{figure}{种 \dpy{zhong4}}

\begin{EntryWithPhonetic}{种}{zhong4}{9}{⽲}[HSK 4]
  \definition{v.}{semear; cultivar; plantar}
  \seeref{zhong3}
\end{EntryWithPhonetic}

\begin{EntryWithPhonetic}{种地}{zhong4di4}{9,6}{⽲,⼟}
  \definition{v.}{cultivar | trabalhar a terra}
\end{EntryWithPhonetic}

\begin{EntryWithPhonetic}{种植}{zhong4zhi2}{9,12}{⽲,⽊}[HSK 4]
  \definition{v.}{plantar; crescer; cultivar; enterrar as sementes de uma planta no solo; plantar as mudas de uma planta no solo}
\end{EntryWithPhonetic}

%%%%%%%%%% 重 %%%%%%%%%%
\subsection*{重}\addcontentsline{loh}{figure}{重 \dpy{zhong4}}

\begin{EntryWithPhonetic}{重}{zhong4}{9}{⾥}[HSK 1]
  \definition{adj.}{pesado; densidade elevada | profundo; sério; grau profundo | importante; significativo | discreto; prudente | considerável em quantidade ou valor}
  \definition[斤,公,斤,吨]{s.}{peso}
  \definition{v.}{colocar (colocar, pôr) ênfase em; dar valor a; atribuir importância a}
  \seeref{chong2}
\end{EntryWithPhonetic}

\begin{EntryWithPhonetic}{重大}{zhong4da4}{9,3}{⾥,⼤}[HSK 3]
  \definition{adj.}{excelente; importante; significativo; de grande importância}
\end{EntryWithPhonetic}

\begin{EntryWithPhonetic}{重点}{zhong4dian3}{9,9}{⾥,⽕}[HSK 2]
  \definition[个]{s.}{nota principal; ponto-chave; ponto}
  \seeref{chong2dian3}
\end{EntryWithPhonetic}

\begin{EntryWithPhonetic}{重量}{zhong4liang4}{9,12}{⾥,⾥}[HSK 4]
  \definition[个]{s.}{peso; a magnitude da força da gravidade em um objeto}
\end{EntryWithPhonetic}

\begin{EntryWithPhonetic}{重视}{zhong4shi4}{9,8}{⾥,⾒}[HSK 2]
  \definition{v.}{valorizar; dar peso a; atribuir importância a; prestar atenção a; considerar a virtude ou o talento de uma pessoa ou o papel de algo como importante e levá-lo a sério}
\end{EntryWithPhonetic}

\begin{EntryWithPhonetic}{重型}{zhong4xing2}{9,9}{⾥,⼟}
  \definition{adj.}{pesado (serviço); pesado}
  \antonymref{轻型}{qing1xing2}
\end{EntryWithPhonetic}

\begin{EntryWithPhonetic}{重要}{zhong4yao4}{9,9}{⾥,⾑}[HSK 1]
  \definition{adj.}{importante; significativo; relevante; de grande importância, função e impacto}
  \synonymref{关键}{guan1jian4}
  \synonymref{首要}{shou3yao4}
  \synonymref{严重}{yan2zhong4}
  \synonymref{重大}{zhong4da4}
  \synonymref{主要}{zhu3yao4}
\end{EntryWithPhonetic}

\begin{EntryWithPhonetic}{重重}{zhong4zhong4}{9,9}{⾥,⾥}
  \definition{adv.}{fortemente | severamente}
  \seeref{chong2chong2}
\end{EntryWithPhonetic}

%%%%%%%%%% 周 %%%%%%%%%%
\subsection*{周}\addcontentsline{loh}{figure}{周 \dpy{zhou1}}

\begin{EntryWithPhonetic}{周}{zhou1}{8}{⼝}[HSK 2]
  \definition*{s.}{Dinastia Zhou (1046-256 BC) | Dinastia Zhou do Norte (557-581), uma das Dinastias do Norte | Dinastia Zhou Posterior (951-960), uma das Cinco Dinastias | Sobrenome: Zhou}
  \definition{adj.}{universal; inteiro; por toda parte | atencioso; pensativo; completo; minucioso}
  \definition{adv.}{semanalmente}
  \definition{clas.}{usado para rodadas, voltas}
  \definition{s.}{periferia; arredores; círculo | semana | ciclo}
  \definition{v.}{fazer um circuito; mover-se em um curso circular | ajudar alguém}
\end{EntryWithPhonetic}

\begin{EntryWithPhonetic}{周末}{zhou1mo4}{8,5}{⼝,⽊}[HSK 2]
  \definition[个]{s.}{final-de-semana}
\end{EntryWithPhonetic}

\begin{EntryWithPhonetic}{周年}{zhou1nian2}{8,6}{⼝,⼲}[HSK 2]
  \definition{s.}{aniversário}
\end{EntryWithPhonetic}

\begin{EntryWithPhonetic}{周期}{zhou1qi1}{8,12}{⼝,⽉}[HSK 5]
  \definition[个]{s.}{período; ciclo; no processo de mudança e movimento das coisas, certas características se repetem várias vezes, com um intervalo de tempo entre cada repetição | período; ciclo; refere"-se a um processo em que certas características se repetem várias vezes, e o tempo decorrido entre duas ocorrências consecutivas | classificação dos elementos na tabela periódica}
\end{EntryWithPhonetic}

\begin{EntryWithPhonetic}{周围}{zhou1wei2}{8,7}{⼝,⼞}[HSK 3]
  \definition{s.}{ao redor; redondeza; vizinhança; a parte ao redor do centro}
\end{EntryWithPhonetic}

%%%%%%%%%% 洲 %%%%%%%%%%
\subsection*{洲}\addcontentsline{loh}{figure}{洲 \dpy{zhou1}}

\begin{EntryWithPhonetic}{洲}{zhou1}{9}{⽔}
  \definition{s.}{continente | ilha em um rio}
\end{EntryWithPhonetic}

%%%%%%%%%% 粥 %%%%%%%%%%
\subsection*{粥}\addcontentsline{loh}{figure}{粥 \dpy{zhou1}}

\begin{EntryWithPhonetic}{粥}{zhou1}{12}{⽶}[HSK 6]
  \definition[碗,锅,口]{s.}{mingau; mingau de aveia; alimentos semilíquidos feitos de grãos ou grãos misturados com outras coisas}
  \seeref{yu4}
\end{EntryWithPhonetic}

%%%%%%%%%% 轴 %%%%%%%%%%
\subsection*{轴}\addcontentsline{loh}{figure}{轴 \dpy{zhou2}}

\begin{EntryWithPhonetic}{轴}{zhou2}{9}{⾞}
  \definition{adj.}{(movimento) inflexível; rígido; desajeitado | direto; franco; decidido | em pergaminho}
  \definition{clas.}{usado para as linhas enroladas ao redor do eixo e as pinturas montadas no eixo}
  \definition{s.}{eixo | carretel; haste | rolo; pergaminho; objeto de enrolamento cilíndrico}
  \seeref{zhou4}
\end{EntryWithPhonetic}

\begin{EntryWithPhonetic}{轴承}{zhou2cheng2}{9,8}{⾞,⼿}
  \definition{s.}{(mecânico) rolamento}
\end{EntryWithPhonetic}

%%%%%%%%%% 咒 %%%%%%%%%%
\subsection*{咒}\addcontentsline{loh}{figure}{咒 \dpy{zhou4}}

\begin{EntryWithPhonetic}{咒}{zhou4}{8}{⼝}
  \definition[个,句]{s.}{encantamento; feitiço}
  \definition{v.}{amaldiçoar; condenar | maltratar; dizer que você espera que as pessoas não tenham sucesso}
\end{EntryWithPhonetic}

\begin{EntryWithPhonetic}{咒骂}{zhou4ma4}{8,9}{⼝,⾺}
  \definition{v.}{xingar | amaldiçoar | execrar}
\end{EntryWithPhonetic}

%%%%%%%%%% 昼 %%%%%%%%%%
\subsection*{昼}\addcontentsline{loh}{figure}{昼 \dpy{zhou4}}

\begin{EntryWithPhonetic}{昼}{zhou4}{9}{⽇}
  \definition*{s.}{Sobrenome: Zhou}
  \definition{s.}{diurno; luz do dia; dia | dia; o período do amanhecer ao anoitecer; diurno}
  \antonymref{夜}{ye4}
\end{EntryWithPhonetic}

%%%%%%%%%% 轴 %%%%%%%%%%
\subsection*{轴}\addcontentsline{loh}{figure}{轴 \dpy{zhou4}}

\begin{EntryWithPhonetic}{轴}{zhou4}{9}{⾞}
  \definition{s.}{a parte final da performance; a última e central peça de uma peça dramática}
  \seeref{zhou2}
\end{EntryWithPhonetic}

%%%%%%%%%% 骤 %%%%%%%%%%
\subsection*{骤}\addcontentsline{loh}{figure}{骤 \dpy{zhou4}}

\begin{EntryWithPhonetic}{骤}{zhou4}{17}{⾺}
  \definition{adj.}{súbito; abrupto | rápido; veloz}
  \definition{adv.}{subitamente; abruptamente}
  \definition{v.}{(um cavalo) trotar; andar rápido}
\end{EntryWithPhonetic}

\begin{EntryWithPhonetic}{骤然}{zhou4ran2}{17,12}{⾺,⽕}
  \definition{adv.}{Literário: de repente; abruptamente; inesperadamente; de uma só vez}
\end{EntryWithPhonetic}

%%%%%%%%%% 珠 %%%%%%%%%%
\subsection*{珠}\addcontentsline{loh}{figure}{珠 \dpy{zhu1}}

\begin{EntryWithPhonetic}{珠}{zhu1}{10}{⽟}
  \definition[粒,颗]{s.}{pérola | conta (de colar, ábaco, etc.) | coisa parecida com uma bola (como um globo ocular)}
\end{EntryWithPhonetic}

\begin{EntryWithPhonetic}{珠宝}{zhu1bao3}{10,8}{⽟,⼧}[HSK 6]
  \definition[串]{s.}{joias; pérolas; um termo geral para pérolas, pedras preciosas e outros ornamentos}
\end{EntryWithPhonetic}

\begin{EntryWithPhonetic}{珠子}{zhu1zi5}{10,3}{⽟,⼦}
  \definition[粒,颗]{s.}{pérola | contas}
\end{EntryWithPhonetic}

%%%%%%%%%% 诸 %%%%%%%%%%
\subsection*{诸}\addcontentsline{loh}{figure}{诸 \dpy{zhu1}}

\begin{EntryWithPhonetic}{诸}{zhu1}{10}{⾔}
  \definition*{s.}{Sobrenome: Zhu}
  \definition{adj.}{todos; cada; vários}
  \definition{prep.}{em; para; de}
\end{EntryWithPhonetic}

\begin{EntryWithPhonetic}{诸位}{zhu1wei4}{10,7}{⾔,⼈}[HSK 6]
  \definition{pron.}{senhores; todos; todos vocês; senhoras e senhores; um termo educado que se refere a várias pessoas}
\end{EntryWithPhonetic}

%%%%%%%%%% 猪 %%%%%%%%%%
\subsection*{猪}\addcontentsline{loh}{figure}{猪 \dpy{zhu1}}

\begin{EntryWithPhonetic}{猪}{zhu1}{11}{⽝}[HSK 3]
  \definition[头,只,口]{s.}{porco; suíno}
\end{EntryWithPhonetic}

\begin{EntryWithPhonetic}{猪窠}{zhu1ke1}{11,13}{⽝,⽳}
  \definition{s.}{chiqueiro}
\end{EntryWithPhonetic}

\begin{EntryWithPhonetic}{猪柳}{zhu1liu3}{11,9}{⽝,⽊}
  \definition{s.}{filé de porco}
\end{EntryWithPhonetic}

\begin{EntryWithPhonetic}{猪笼}{zhu1long2}{11,11}{⽝,⽵}
  \definition{s.}{estrutura cilíndrica de bambu ou arame usada para restringir um porco durante o transporte}
\end{EntryWithPhonetic}

\begin{EntryWithPhonetic}{猪头}{zhu1tou2}{11,5}{⽝,⼤}
  \definition{s.}{tolo | idiota}
\end{EntryWithPhonetic}

%%%%%%%%%% 术 %%%%%%%%%%
\subsection*{术}\addcontentsline{loh}{figure}{术 \dpy{zhu2}}

\begin{EntryWithPhonetic}{术}{zhu2}{5}{⽊}
  \definition{s.}{vários gêneros de flores da família Asteraceae (margaridas e crisântemos)}
  \seeref{shu4}
\end{EntryWithPhonetic}

%%%%%%%%%% 竹 %%%%%%%%%%
\subsection*{竹}\addcontentsline{loh}{figure}{竹 \dpy{zhu2}}

\begin{EntryWithPhonetic}{竹}{zhu2}{6}{⽵}[Kangxi 118]
  \definition[根]{s.}{bambu | instrumento de sopro | tira de bambu}
\end{EntryWithPhonetic}

\begin{EntryWithPhonetic}{竹编}{zhu2bian1}{6,12}{⽵,⽷}
  \definition{s.}{vime | tecelagem de bambu}
\end{EntryWithPhonetic}

\begin{EntryWithPhonetic}{竹马}{zhu2ma3}{6,3}{⽵,⾺}
  \definition{s.}{cavalo de bambu | vara de bambu usada como cavalo de brinquedo}
\end{EntryWithPhonetic}

\begin{EntryWithPhonetic}{竹排}{zhu2pai2}{6,11}{⽵,⼿}
  \definition{s.}{jangada de bambu}
\end{EntryWithPhonetic}

\begin{EntryWithPhonetic}{竹子}{zhu2zi5}{6,3}{⽵,⼦}[HSK 5]
  \definition[根,棵,丛,支]{s.}{bambu; nome genérico para os tipos de bambu}
\end{EntryWithPhonetic}

%%%%%%%%%% 逐 %%%%%%%%%%
\subsection*{逐}\addcontentsline{loh}{figure}{逐 \dpy{zhu2}}

\begin{EntryWithPhonetic}{逐}{zhu2}{10}{⾡}
  \definition{prep.}{um por um; um a um}[逐月===mês a mês]
  \definition{v.}{ir atrás de; perseguir | expulsar; banir | correr atrás; alcançar}
\end{EntryWithPhonetic}

\begin{EntryWithPhonetic}{逐步}{zhu2bu4}{10,7}{⾡,⽌}[HSK 4]
  \definition{adv.}{gradualmente; passo a passo; progressivamente}
\end{EntryWithPhonetic}

\begin{EntryWithPhonetic}{逐渐}{zhu2jian4}{10,11}{⾡,⽔}[HSK 4]
  \definition{adv.}{gradualmente; aos poucos; por etapas; indica mudanças lentas e ordenadas no grau, na quantidade, etc.}
\end{EntryWithPhonetic}

%%%%%%%%%% 主 %%%%%%%%%%
\subsection*{主}\addcontentsline{loh}{figure}{主 \dpy{zhu3}}

\begin{EntryWithPhonetic}{主}{zhu3}{5}{⼂}
  \definition*{s.}{Deus; Senhor; o nome do Deus em que se acredita o cristianismo, o judaísmo, etc.}
  \definition{adj.}{principal; primário; o mais básico; o mais importante | de si mesmo; por vontade própria; próprio; do próprio}
  \definition[位,名,个]{s.}{anfitrião; alguém que convida e recebe convidados | mestre; dono; uma pessoa que possui poder ou propriedade; uma pessoa em posição dominante | pessoa ou parte interessada | decisão; opinião; visão definitiva | placa espiritual (ou memorial)}
  \definition{v.}{dirigir; administrar; assumir o comando de; presidir; assumir a responsabilidade primária | decidir; reivindicar | significar; indicar; prever um certo resultado}
  \antonymref{宾}{bin1}
  \antonymref{次}{ci4}
  \antonymref{从}{cong2}
  \antonymref{副}{fu4}
  \antonymref{客}{ke4}
  \antonymref{奴}{nu2}
\end{EntryWithPhonetic}

\begin{EntryWithPhonetic}{主办}{zhu3ban4}{5,4}{⼂,⼒}[HSK 5]
  \definition{v.}{manter; hospedar; dirigir; patrocinar}
  \synonymref{主持}{zhu3chi2}
\end{EntryWithPhonetic}

\begin{EntryWithPhonetic}{主持}{zhu3chi2}{5,9}{⼂,⼿}[HSK 3]
  \definition[位,名]{s.}{anfitrião; a pessoa responsável por administrar e lidar com uma determinada atividade}
  \definition{v.}{dirigir; administrar; assumir o comando; encarregar-se de; ser responsável por gerenciar, organizar uma determinada atividade ou lidar com um determinado assunto | defender; apoiar; preservar; manter}
  \synonymref{把持}{ba3chi2}
  \synonymref{主办}{zhu3ban4}
  \synonymref{主席}{zhu3xi2}
  \synonymref{总裁}{zong3cai2}
\end{EntryWithPhonetic}

\begin{EntryWithPhonetic}{主持人}{zhu3chi2ren2}{5,9,2}{⼂,⼿,⼈}[HSK 6]
  \definition[个,位]{s.}{anfitrião; âncora; apresentador}
\end{EntryWithPhonetic}

\begin{EntryWithPhonetic}{主导}{zhu3dao3}{5,6}{⼂,⼨}[HSK 5]
  \definition{adj.}{líder; dominante; guiado; principais e guias para que as coisas se desenvolvam em uma determinada direção}
  \definition{s.}{fator principal (ou orientador)}
  \synonymref{核心}{he2xin1}
  \synonymref{掌握}{zhang3wo4}
  \synonymref{支配}{zhi1pei4}
  \synonymref{主管}{zhu3guan3}
\end{EntryWithPhonetic}

\begin{EntryWithPhonetic}{主动}{zhu3dong4}{5,6}{⼂,⼒}[HSK 3]
  \definition{adj.}{ativo; positivo; agir sem esperar por um impulso externo | iniciativo; capaz de impulsionar as coisas por vontade própria; capaz de criar uma situação favorável e fazer as coisas acontecerem de acordo com suas próprias intenções}
  \synonymref{积极}{ji1ji2}
  \synonymref{自动}{zi4dong4}
  \antonymref{被动}{bei4dong4}
\end{EntryWithPhonetic}

\begin{EntryWithPhonetic}{主观}{zhu3guan1}{5,6}{⼂,⾒}[HSK 5]
  \definition{adj.}{subjetivo; não com base nas condições reais, mas com base nos próprios desejos | subjetivo; filosoficamente, refere"-se à consciência e aos aspectos espirituais dos seres humanos}
  \antonymref{客观}{ke4guan1}
\end{EntryWithPhonetic}

\begin{EntryWithPhonetic}{主管}{zhu3guan3}{5,14}{⼂,⽵}[HSK 5]
  \definition[位,名,个,些]{s.}{pessoa responsável, como supervisor, gerente, diretor, etc.}
  \definition{v.}{estar encarregado de; ser responsável por; ser o principal responsável pela gestão de um trabalho; assumir a responsabilidade primária pela gestão (um certo aspecto)}
  \synonymref{主导}{zhu3dao3}
  \synonymref{总裁}{zong3cai2}
\end{EntryWithPhonetic}

\begin{EntryWithPhonetic}{主角}{zhu3jue2}{5,7}{⼂,⾓}[HSK 6]
  \definition[个,位,名]{s.}{liderança; papel principal; protagonista; um papel importante em uma peça, filme, etc.; um ator que desempenha um papel importante | Figurativo: algo que tem grande influência em uma determinada área; refere"-se ao personagem principal}
  \synonymref{主人}{zhu3ren2}
\end{EntryWithPhonetic}

\begin{EntryWithPhonetic}{主流}{zhu3liu2}{5,10}{⼂,⽔}[HSK 6]
  \definition{s.}{corrente principal; corrente mãe; convencional | tendência principal; aspecto essencial ou principal; falando metaforicamente, os principais aspectos do desenvolvimento das coisas}
\end{EntryWithPhonetic}

\begin{EntryWithPhonetic}{主人}{zhu3ren2}{5,2}{⼂,⼈}[HSK 2]
  \definition[个,位]{s.}{mestre; uma pessoa que empregava tutores, contadores, etc. antigamente; uma pessoa que empregava empregados domésticos | anfitrião; Aaguém que entretém convidados | proprietário; uma pessoa que possui um certo tipo de bens ou poder}
  \synonymref{老板}{lao3ban3}
  \antonymref{客人}{ke4ren5}
\end{EntryWithPhonetic}

\begin{EntryWithPhonetic}{主任}{zhu3ren4}{5,6}{⼂,⼈}[HSK 3]
  \definition[个,位,名]{s.}{chefe; diretor; presidente; o principal responsável por um departamento ou instituição}
  \synonymref{主管}{zhu3guan3}
  \antonymref{职员}{zhi2yuan2}
\end{EntryWithPhonetic}

\begin{EntryWithPhonetic}{主题}{zhu3ti2}{5,15}{⼂,⾴}[HSK 4]
  \definition[个]{s.}{tema; assunto; motivo; lema; ideias básicas expressas em toda a obra de literatura e arte por meio de imagens artísticas concretas | pontos/conteúdos principais; referência geral ao conteúdo principal de artigos, discursos, conferências, etc.}
  \synonymref{草坪}{cao3ping2}
  \synonymref{核心}{he2xin1}
  \synonymref{焦点}{jiao1dian3}
  \synonymref{题材}{ti2cai2}
  \synonymref{中心}{zhong1xin1}
  \synonymref{中央}{zhong1yang1}
\end{EntryWithPhonetic}

\begin{EntryWithPhonetic}{主体}{zhu3ti3}{5,7}{⼂,⼈}[HSK 5]
  \definition[个,些,种,群]{s.}{corpo principal; parte principal; parte principal; esteio; a parte principal das coisas | Filosofia: sujeito}
  \synonymref{本身}{ben3shen1}
  \synonymref{首要}{shou3yao4}
\end{EntryWithPhonetic}

\begin{EntryWithPhonetic}{主席}{zhu3xi2}{5,10}{⼂,⼱}[HSK 4]
  \definition*{s.}{Presidente (da China)}
  \definition[个,位,名]{s.}{presidente, \emph{chairman} (de uma reunião) | chefe; presidente (de uma organização ou estado)}
  \synonymref{领导}{ling3dao3}
  \synonymref{主持}{zhu3chi2}
  \synonymref{总理}{zong3li3}
  \synonymref{总统}{zong3tong3}
\end{EntryWithPhonetic}

\begin{EntryWithPhonetic}{主席台}{zhu3xi2tai2}{5,10,5}{⼂,⼱,⼝}
  \definition[个]{s.}{plataforma | tribuna}
\end{EntryWithPhonetic}

\begin{EntryWithPhonetic}{主席团}{zhu3xi2tuan2}{5,10,6}{⼂,⼱,⼞}
  \definition{s.}{presídio}
\end{EntryWithPhonetic}

\begin{EntryWithPhonetic}{主要}{zhu3yao4}{5,9}{⼂,⾑}[HSK 2]
  \definition{adj.}{principal; chefe; o mais importante na questão; o decisivo | principal; núcleo; a raiz ou parte mais importante de algo}
  \synonymref{重点}{chong2dian3}
  \synonymref{关键}{guan1jian4}
  \synonymref{首要}{shou3yao4}
  \synonymref{严重}{yan2zhong4}
  \synonymref{重点}{zhong4dian3}
  \synonymref{重要}{zhong4yao4}
  \antonymref{辅助}{fu3zhu4}
\end{EntryWithPhonetic}

\begin{EntryWithPhonetic}{主义}{zhu3yi4}{5,3}{⼂,⼂}
  \definition[种]{s.}{doutrina; um determinado sistema social ou sistema político e econômico | estilo de pensamento; um certo ponto de vista ou estilo | ideologia; teorias e doutrinas sistemáticas sobre a natureza, a sociedade humana, etc.}
  \definition{suf.}{-ismo}
  \synonymref{办法}{ban4fa3}
  \synonymref{方针}{fang1zhen1}
  \synonymref{目标}{mu4biao1}
  \synonymref{目的}{mu4di4}
  \synonymref{想法}{xiang3fa5}
  \synonymref{主张}{zhu3zhang1}
\end{EntryWithPhonetic}

\begin{EntryWithPhonetic}{主意}{zhu3yi5}{5,13}{⼂,⼼}[HSK 3]
  \definition[个,种]{s.}{ideia; plano; decisão; método}
  \synonymref{办法}{ban4fa3}
  \synonymref{点子}{dian3zi5}
  \synonymref{方法}{fang1fa3}
  \synonymref{方针}{fang1zhen1}
  \synonymref{目的}{mu4di4}
  \synonymref{想法}{xiang3fa5}
  \synonymref{主张}{zhu3zhang1}
\end{EntryWithPhonetic}

\begin{EntryWithPhonetic}{主张}{zhu3zhang1}{5,7}{⼂,⼸}[HSK 3]
  \definition[个,项,些,种]{s.}{vista; posição; proposição}
  \definition{v.}{defender; apoiar; manter; representar; ter uma opinião sobre como agir, fazer uma sugestão}
  \synonymref{办法}{ban4fa3}
  \synonymref{倡导}{chang4dao3}
  \synonymref{观点}{guan1dian3}
  \synonymref{见解}{jian4jie3}
  \synonymref{看法}{kan4fa5}
  \synonymref{想法}{xiang3fa5}
  \synonymref{意见}{yi4jian5}
  \synonymref{主义}{zhu3yi4}
  \synonymref{主意}{zhu3yi5}
  \antonymref{禁止}{jin4zhi3}
\end{EntryWithPhonetic}

%%%%%%%%%% 属 %%%%%%%%%%
\subsection*{属}\addcontentsline{loh}{figure}{属 \dpy{zhu3}}

\begin{EntryWithPhonetic}{属}{zhu3}{12}{⼫}
  \definition{v.}{juntar; combinar | fixar (a mente) em; centrar (a atenção, etc.) em}
  \seeref{shu3}
\end{EntryWithPhonetic}

%%%%%%%%%% 煮 %%%%%%%%%%
\subsection*{煮}\addcontentsline{loh}{figure}{煮 \dpy{zhu3}}

\begin{EntryWithPhonetic}{煮}{zhu3}{12}{⽕}[HSK 6]
  \definition*{s.}{Sobrenome: Zhu}
  \definition{v.}{ferver; cozinhar; aquecer alimentos ou outros itens em água}
\end{EntryWithPhonetic}

%%%%%%%%%% 褚 %%%%%%%%%%
\subsection*{褚}\addcontentsline{loh}{figure}{褚 \dpy{zhu3}}

\begin{EntryWithPhonetic}{褚}{zhu3}{13}{⾐}
  \definition{s.}{acolchoamento (na vestimenta) | bolso}
  \definition{v.}{armazenar}
  \seeref{chu3}
\end{EntryWithPhonetic}

%%%%%%%%%% 嘱 %%%%%%%%%%
\subsection*{嘱}\addcontentsline{loh}{figure}{嘱 \dpy{zhu3}}

\begin{EntryWithPhonetic}{嘱}{zhu3}{15}{⼝}
  \definition{v.}{juntar-se | implorar | incitar}
\end{EntryWithPhonetic}

\begin{EntryWithPhonetic}{嘱咐}{zhu3fu5}{15,8}{⼝,⼝}
  \definition{v.}{ordenar | dizer | exortar}
\end{EntryWithPhonetic}

\begin{EntryWithPhonetic}{嘱托}{zhu3tuo1}{15,6}{⼝,⼿}
  \definition{v.}{confiar uma tarefa a alguém}
\end{EntryWithPhonetic}

%%%%%%%%%% 住 %%%%%%%%%%
\subsection*{住}\addcontentsline{loh}{figure}{住 \dpy{zhu4}}

\begin{EntryWithPhonetic}{住}{zhu4}{7}{⼈}[HSK 1]
  \definition{adv.}{firmemente; indica estabilidade ou firmeza}
  \definition{v.}{viver; residir; morar; ficar | parar; cessar | (após um verbo) com firmeza; até parar | hospedar; acomodar | parar; interromper | ser competente; ser qualificado; estar à altura; usado com 得 ou 不, indica que a força é suficiente (ou insuficiente)}
  \seealsoref{不}{bu4}
  \seealsoref{得}{de5}
\end{EntryWithPhonetic}

\begin{EntryWithPhonetic}{住处}{zhu4chu4}{7,5}{⼈,⼡}
  \definition{s.}{morada | habitação | residência}
\end{EntryWithPhonetic}

\begin{EntryWithPhonetic}{住房}{zhu4fang2}{7,8}{⼈,⼾}[HSK 2]
  \definition[套,处]{s.}{habitação; alojamento; casas para as pessoas morarem}
\end{EntryWithPhonetic}

\begin{EntryWithPhonetic}{住所}{zhu4suo3}{7,8}{⼈,⼾}
  \definition[处]{s.}{morada | habitação | residência}
\end{EntryWithPhonetic}

\begin{EntryWithPhonetic}{住院}{zhu4 yuan4}{7,9}{⼈,⾩}[HSK 2]
  \definition{v.}{estar hospitalizado; estar no hospital; ser internado no hospital para tratamento}
\end{EntryWithPhonetic}

\begin{EntryWithPhonetic}{住宅}{zhu4zhai2}{7,6}{⼈,⼧}[HSK 6]
  \definition[套,处,栋,座]{s.}{habitação; residência}
\end{EntryWithPhonetic}

\begin{EntryWithPhonetic}{住嘴}{zhu4zui3}{7,16}{⼈,⼝}
  \definition{interj.}{``Cale-se!''}
  \definition{v.}{calar | calar-se}
\end{EntryWithPhonetic}

%%%%%%%%%% 助 %%%%%%%%%%
\subsection*{助}\addcontentsline{loh}{figure}{助 \dpy{zhu4}}

\begin{EntryWithPhonetic}{助}{zhu4}{7}{⼒}
  \definition{v.}{ajudar; auxiliar; acudir; apoiar}
\end{EntryWithPhonetic}

\begin{EntryWithPhonetic}{助理}{zhu4li3}{7,11}{⼒,⽟}[HSK 5]
  \definition[个,位,名,些]{s.}{deputado; assistente; auxiliar do diretor responsável (geralmente usado em cargos) | ajudante; assistente; pessoa que auxilia o responsável a fazer as coisas}
\end{EntryWithPhonetic}

\begin{EntryWithPhonetic}{助手}{zhu4shou3}{7,4}{⼒,⼿}[HSK 5]
  \definition[个,位,种,名]{s.}{ajudante; auxiliar; assistente; alguém que ajuda os outros com seu trabalho}
\end{EntryWithPhonetic}

\begin{EntryWithPhonetic}{助兴}{zhu4/xing4}{7,6}{⼒,⼋}
  \definition{v.+compl.}{animar as coisas | juntar-se à diversão}
\end{EntryWithPhonetic}

%%%%%%%%%% 注 %%%%%%%%%%
\subsection*{注}\addcontentsline{loh}{figure}{注 \dpy{zhu4}}

\begin{EntryWithPhonetic}{注}{zhu4}{8}{⽔}
  \definition{s.}{apostas (em jogos de azar) | notas (em um texto)}
  \definition{v.}{derramar; encher | concentrar-se em; fixar-se em; focar em  | anotar; explicar com notas | registrar; gravar | irrigar | dar exegese ou explicação}
\end{EntryWithPhonetic}

\begin{EntryWithPhonetic}{注册}{zhu4/ce4}{8,5}{⽔,⼌}[HSK 5]
  \definition{v.+compl.}{inscrever"-se; matricular"-se; registrar"-se; registrar"-se junto à autoridade ou escola competente para obter status legal; refere"-se especificamente ao usuário de uma determinada rede de computadores que insere o nome de usuário, senha, etc. na rede para obter permissão para usar a rede}
\end{EntryWithPhonetic}

\begin{EntryWithPhonetic}{注册表}{zhu4ce4biao3}{8,5,8}{⽔,⼌,⾐}
  \definition[份,个,张]{s.}{registro do Windows}
\end{EntryWithPhonetic}

\begin{EntryWithPhonetic}{注册人}{zhu4ce4ren2}{8,5,2}{⽔,⼌,⼈}
  \definition{s.}{registrante}
\end{EntryWithPhonetic}

\begin{EntryWithPhonetic}{注册商标}{zhu4ce4 shang1biao1}{8,5,11,9}{⽔,⼌,⼝,⽊}
  \definition{s.}{marca registrada}
\end{EntryWithPhonetic}

\begin{EntryWithPhonetic}{注射}{zhu4she4}{8,10}{⽔,⼨}[HSK 5]
  \definition{v.}{injetar; usar uma seringa para administrar medicamento líquido em um organismo}
\end{EntryWithPhonetic}

\begin{EntryWithPhonetic}{注视}{zhu4shi4}{8,8}{⽔,⾒}[HSK 5]
  \definition{v.}{olhar atentamente para; observar atentamente}
\end{EntryWithPhonetic}

\begin{EntryWithPhonetic}{注意}{zhu4/yi4}{8,13}{⽔,⼼}[HSK 3]
  \definition{v.aux.}{prestar atenção; notar; ficar de olho; concentrar os pensamentos em um aspecto específico}
\end{EntryWithPhonetic}

\begin{EntryWithPhonetic}{注意地}{zhu4yi4di4}{8,13,6}{⽔,⼼,⼟}
  \definition{s.}{área de cuidado, de observação}
\end{EntryWithPhonetic}

\begin{EntryWithPhonetic}{注意力}{zhu4yi4li4}{8,13,2}{⽔,⼼,⼒}
  \definition{s.}{atenção}
\end{EntryWithPhonetic}

\begin{EntryWithPhonetic}{注意力缺失症}{zhu4yi4 li4 que1shi1 zheng4}{8,13,2,10,5,10}{⽔,⼼,⼒,⽸,⼤,⽧}
  \definition{s.}{transtorno de déficit de atenção}[他被诊断出注意力缺失症。===Ele foi diagnosticado com transtorno de déficit de atenção.]
\end{EntryWithPhonetic}

\begin{EntryWithPhonetic}{注重}{zhu4zhong4}{8,9}{⽔,⾥}[HSK 5]
  \definition{v.}{enfatizar; dar ênfase a; dar ênfase a; prestar atenção a; dar importância a}
\end{EntryWithPhonetic}

%%%%%%%%%% 驻 %%%%%%%%%%
\subsection*{驻}\addcontentsline{loh}{figure}{驻 \dpy{zhu4}}

\begin{EntryWithPhonetic}{驻}{zhu4}{8}{⾺}[HSK 6]
  \definition{v.}{parar; ficar | estar estacionado; acampar; (tropas ou pessoal) viver no local onde desempenham suas funções; (organização) estar localizada em um determinado lugar}
\end{EntryWithPhonetic}

\begin{EntryWithPhonetic}{驻军}{zhu4jun1}{8,6}{⾺,⼍}
  \definition{s.}{guarnição}
  \definition{v.}{guarcener ou prover uma tropa}
\end{EntryWithPhonetic}

%%%%%%%%%% 柱 %%%%%%%%%%
\subsection*{柱}\addcontentsline{loh}{figure}{柱 \dpy{zhu4}}

\begin{EntryWithPhonetic}{柱}{zhu4}{9}{⽊}
  \definition*{s.}{Sobrenome: Zhu}
  \definition[根]{s.}{poste; pilar; coluna | algo em forma de coluna | Matemática: cilindro}
\end{EntryWithPhonetic}

\begin{EntryWithPhonetic}{柱子}{zhu4zi5}{9,3}{⽊,⼦}[HSK 6]
  \definition{s.}{poste; pilar; coluna; estrutura de suporte vertical de um edifício, feita de madeira, pedra, aço, concreto armado, etc.}
\end{EntryWithPhonetic}

%%%%%%%%%% 祝 %%%%%%%%%%
\subsection*{祝}\addcontentsline{loh}{figure}{祝 \dpy{zhu4}}

\begin{EntryWithPhonetic}{祝}{zhu4}{9}{⽰}[HSK 3]
  \definition*{s.}{Sobrenome: Zhu}
  \definition{v.}{expressar bons votos; desejar; abençoar | rezar aos deuses ou espíritos para obter bênçãos}
\end{EntryWithPhonetic}

\begin{EntryWithPhonetic}{祝祷}{zhu4dao3}{9,11}{⽰,⽰}
  \definition{v.}{rezar | orar}
\end{EntryWithPhonetic}

\begin{EntryWithPhonetic}{祝福}{zhu4fu2}{9,13}{⽰,⽰}[HSK 4]
  \definition[个]{s.}{bênção; benzedura; benzimento; originalmente, referia"-se à oração para obter as bênçãos de Deus, mas, mais tarde, refere"-se ao desejo de paz e felicidade às pessoas}
  \definition{v.}{desejar boa sorte a alguém}
\end{EntryWithPhonetic}

\begin{EntryWithPhonetic}{祝好}{zhu4hao3}{9,6}{⽰,⼥}
  \definition{expr.}{desejo-lhe tudo de melhor! (ao encerrar uma correspondência)}
\end{EntryWithPhonetic}

\begin{EntryWithPhonetic}{祝贺}{zhu4he4}{9,9}{⽰,⾙}[HSK 5]
  \definition[个]{s.}{congratulações; felicitações}
  \definition{v.}{congratular; felicitar; parabenizar}
\end{EntryWithPhonetic}

\begin{EntryWithPhonetic}{祝酒}{zhu4jiu3}{9,10}{⽰,⾣}
  \definition{v.}{parabenizar e fazer um brinde | brindar}
\end{EntryWithPhonetic}

\begin{EntryWithPhonetic}{祝寿}{zhu4shou4}{9,7}{⽰,⼨}
  \definition{v.}{dar parabéns pelo aniversário (a uma pessoa idosa)}
\end{EntryWithPhonetic}

\begin{EntryWithPhonetic}{祝颂}{zhu4song4}{9,10}{⽰,⾴}
  \definition{v.}{expressar bons desejos}
\end{EntryWithPhonetic}

\begin{EntryWithPhonetic}{祝谢}{zhu4xie4}{9,12}{⽰,⾔}
  \definition{v.}{agradecer | dar parabéns}
\end{EntryWithPhonetic}

\begin{EntryWithPhonetic}{祝愿}{zhu4yuan4}{9,14}{⽰,⽕}[HSK 6]
  \definition{v.}{desejar; expressar bons desejos}
\end{EntryWithPhonetic}

%%%%%%%%%% 著 %%%%%%%%%%
\subsection*{著}\addcontentsline{loh}{figure}{著 \dpy{zhu4}}

\begin{EntryWithPhonetic}{著}{zhu4}{11}{⽬}
  \definition{adj.}{marcado; excelente; óbvio}
  \definition{s.}{livro; trabalho | nativo; pessoa/povo indígena; refere"-se a pessoas que se estabeleceram em um lugar por gerações}
  \definition{v.}{mostrar; provar; revelar | escrever}
\end{EntryWithPhonetic}

\begin{EntryWithPhonetic}{著名}{zhu4ming2}{11,6}{⽬,⼝}[HSK 4]
  \definition[位]{adj.}{famoso; bem conhecido; célebre}
\end{EntryWithPhonetic}

\begin{EntryWithPhonetic}{著作}{zhu4zuo4}{11,7}{⽬,⼈}[HSK 4]
  \definition[部,本,类]{s.}{obra; livro; escritos}
  \definition{v.}{escrever; usar palavras para expressar opiniões, conhecimentos, ideias, sentimentos, etc.}
\end{EntryWithPhonetic}

%%%%%%%%%% 抓 %%%%%%%%%%
\subsection*{抓}\addcontentsline{loh}{figure}{抓 \dpy{zhua1}}

\begin{EntryWithPhonetic}{抓}{zhua1}{7}{⼿}[HSK 3]
  \definition{v.}{agarrar; segurar; obter; apreender; juntar os dedos para segurar o objeto na mão | riscar; arranhar; usar as unhas, objetos com dentes ou garras de animais para riscar a superfície de um objeto | apanhar; capturar; controlar pessoas ou animais; fazer com que pessoas ou animais caiam nas mãos de alguém | compreender; saber onde está o ponto principal ou a chave de uma questão ou problema | concentrar-se em algo; reforçar a força para fazer (alguma coisa), controlar (algum aspecto) | chamar a atenção de alguém; atrair a atenção}
\end{EntryWithPhonetic}

\begin{EntryWithPhonetic}{抓紧}{zhua1/jin3}{7,10}{⼿,⽷}[HSK 4]
  \definition{v.+compl.}{agarrar com firmeza; segurar firme e não soltar | prestar muita atenção a}
\end{EntryWithPhonetic}

\begin{EntryWithPhonetic}{抓住}{zhua1 zhu4}{7,7}{⼿,⼈}[HSK 3]
  \definition{v.}{prender; deter; capturar (pessoas ou animais) e ter sucesso | segurar; agarrar; apreender; agarrar algo para que não se mova}
\end{EntryWithPhonetic}

%%%%%%%%%% 转 %%%%%%%%%%
\subsection*{转}\addcontentsline{loh}{figure}{转 \dpy{zhuai3}}

\begin{EntryWithPhonetic}{转}{zhuai3}{8}{⾞}
  \seeref{zhuan3}
  \seeref{zhuan4}
\end{EntryWithPhonetic}

%%%%%%%%%% 专 %%%%%%%%%%
\subsection*{专}\addcontentsline{loh}{figure}{专 \dpy{zhuan1}}

\begin{EntryWithPhonetic}{专}{zhuan1}{4}{⼀}
  \definition{adj.}{específico para; dedicado a um uso específico; dedicado a; especial}
  \definition{adv.}{especialmente; especificamente}
  \definition{v.}{monopolizar}
  \antonymref{博}{bo2}
\end{EntryWithPhonetic}

\begin{EntryWithPhonetic}{专辑}{zhuan1ji2}{4,13}{⼀,⾞}[HSK 5]
  \definition[张]{s.}{álbum (música) | registro (música) | coleção especial de material impresso ou transmitido}
\end{EntryWithPhonetic}

\begin{EntryWithPhonetic}{专家}{zhuan1jia1}{4,10}{⼀,⼧}[HSK 3]
  \definition[个,位]{s.}{perito; especialista; profissional; pessoa que se dedica ao estudo aprofundado de uma determinada disciplina; pessoa especializada em uma determinada técnica}
  \synonymref{大家}{da4jia1}
  \synonymref{大师}{da4shi1}
  \synonymref{行家}{hang2jia5}
  \synonymref{内行}{nei4hang2}
\end{EntryWithPhonetic}

\begin{EntryWithPhonetic}{专利}{zhuan1li4}{4,7}{⼀,⼑}[HSK 5]
  \definition[个,项,些]{s.}{patente; a garantia de que os criadores e inventores desfrutem exclusivamente dos benefícios decorrentes de suas criações e invenções durante um determinado período | direitos de patente; referência à patente}
  \synonymref{专用}{zhuan1yong4}
  \antonymref{公共}{gong1gong4}
  \antonymref{公用}{gong1yong4}
\end{EntryWithPhonetic}

\begin{EntryWithPhonetic}{专门}{zhuan1men2}{4,3}{⼀,⾨}[HSK 3]
  \definition{adj.}{especializado; dedicar-se exclusivamente a uma determinada tarefa; expressa ênfase em fazer frequentemente um determinado tipo de coisa}
  \definition{adv.}{especialmente}
  \synonymref{特地}{te4di4}
  \synonymref{特意}{te4yi4}
  \synonymref{专业}{zhuan1ye4}
  \antonymref{附带}{fu4dai4}
  \antonymref{顺便}{shun4bian4}
\end{EntryWithPhonetic}

\begin{EntryWithPhonetic}{专题}{zhuan1ti2}{4,15}{⼀,⾴}[HSK 3]
  \definition[个,些,种]{s.}{assunto especial; tópico especial; questões específicas}
  \synonymref{单元}{dan1yuan2}
\end{EntryWithPhonetic}

\begin{EntryWithPhonetic}{专心}{zhuan1xin1}{4,4}{⼀,⼼}[HSK 4]
  \definition{adj.}{absorto; concentrado}
  \synonymref{用心}{yong4 xin1}
\end{EntryWithPhonetic}

\begin{EntryWithPhonetic}{专业}{zhuan1ye4}{4,5}{⼀,⼀}[HSK 3]
  \definition{adj.}{profissional; descreve uma pessoa que possui um alto nível ou conhecimento profundo em determinada área}
  \definition[个,门]{s.}{profissão; área específica; comércio especializado; departamentos operacionais da divisão de produção | especialidade; disciplina; matéria especializada; área de estudo específica; em um departamento de uma instituição de ensino superior ou em uma escola profissionalizante de nível médio}
  \synonymref{那些}{na4xie1}
  \synonymref{专门}{zhuan1men2}
  \antonymref{业余}{ye4yu2}
\end{EntryWithPhonetic}

\begin{EntryWithPhonetic}{专业户}{zhuan1ye4hu4}{4,5,4}{⼀,⼀,⼾}
  \definition{s.}{indústria caseira | empresa familiar produzindo um produto especial}
\end{EntryWithPhonetic}

\begin{EntryWithPhonetic}{专业化}{zhuan1ye4hua4}{4,5,4}{⼀,⼀,⼔}
  \definition{s.}{especialização}
\end{EntryWithPhonetic}

\begin{EntryWithPhonetic}{专业教育}{zhuan1ye4jiao4yu4}{4,5,11,8}{⼀,⼀,⽁,⾁}
  \definition{s.}{educação especializada | escola técnica}
\end{EntryWithPhonetic}

\begin{EntryWithPhonetic}{专业人才}{zhuan1ye4ren2cai2}{4,5,2,3}{⼀,⼀,⼈,⼿}
  \definition{s.}{especialista (em uma área)}
\end{EntryWithPhonetic}

\begin{EntryWithPhonetic}{专业人士}{zhuan1ye4ren2shi4}{4,5,2,3}{⼀,⼀,⼈,⼠}
  \definition{s.}{profissional}
\end{EntryWithPhonetic}

\begin{EntryWithPhonetic}{专业性}{zhuan1ye4xing4}{4,5,8}{⼀,⼀,⼼}
  \definition{s.}{profissionalismo | expertise}
\end{EntryWithPhonetic}

\begin{EntryWithPhonetic}{专用}{zhuan1yong4}{4,5}{⼀,⽤}[HSK 6]
  \definition{adj.}{(reservado para) uso especial; para um propósito especial; dedicado a uma determinada necessidade ou a uma determinada pessoa}[他需要一个专用的工作空间。===Ele precisava de um espaço de trabalho dedicado.]
  \synonymref{专利}{zhuan1li4}
  \antonymref{公用}{gong1yong4}
  \antonymref{通用}{tong1yong4}
\end{EntryWithPhonetic}

%%%%%%%%%% 砖 %%%%%%%%%%
\subsection*{砖}\addcontentsline{loh}{figure}{砖 \dpy{zhuan1}}

\begin{EntryWithPhonetic}{砖}{zhuan1}{9}{⽯}
  \definition[块]{s.}{tijolo}
\end{EntryWithPhonetic}

%%%%%%%%%% 转 %%%%%%%%%%
\subsection*{转}\addcontentsline{loh}{figure}{转 \dpy{zhuan3}}

\begin{EntryWithPhonetic}{转}{zhuan3}{8}{⾞}[HSK 3]
  \definition{v.}{mudar; deslocar; transferir; virar; mudar de direção, posição, situação, circunstâncias, etc. | transmitir; transferir; passar adiante}
  \seeref{zhuai3}
  \seeref{zhuan4}
\end{EntryWithPhonetic}

\begin{EntryWithPhonetic}{转变}{zhuan3bian4}{8,8}{⾞,⼜}[HSK 3]
  \definition{v.}{mudar; converter; transformar}
\end{EntryWithPhonetic}

\begin{EntryWithPhonetic}{转产}{zhuan3chan3}{8,6}{⾞,⼇}
  \definition{v.}{mudar a produção | mudar para novos produtos}
\end{EntryWithPhonetic}

\begin{EntryWithPhonetic}{转递}{zhuan3di4}{8,10}{⾞,⾡}
  \definition{v.}{passar | retransmitir}
\end{EntryWithPhonetic}

\begin{EntryWithPhonetic}{转动}{zhuan3dong4}{8,6}{⾞,⼒}[HSK 4]
  \definition{v.}{girar; rodar; dar voltas; torcer | dar a volta em algo}
  \seeref{zhuan4dong4}
\end{EntryWithPhonetic}

\begin{EntryWithPhonetic}{转告}{zhuan3gao4}{8,7}{⾞,⼝}[HSK 4]
  \definition{v.}{passar adiante; comunicar; transmitir; ser instruído a dizer a outra parte o que uma pessoa diz, o que está acontecendo, etc.}
\end{EntryWithPhonetic}

\begin{EntryWithPhonetic}{转化}{zhuan3hua4}{8,4}{⾞,⼔}[HSK 5]
  \definition{v.}{mudar; transformar | inverter; converter}
\end{EntryWithPhonetic}

\begin{EntryWithPhonetic}{转换}{zhuan3huan4}{8,10}{⾞,⼿}[HSK 5]
  \definition{v.}{mudar; trocar; converter; transformar; alterar}
\end{EntryWithPhonetic}

\begin{EntryWithPhonetic}{转念}{zhuan3nian4}{8,8}{⾞,⼼}
  \definition{v.}{ter dúvidas sobre algo | pensar melhor}
\end{EntryWithPhonetic}

\begin{EntryWithPhonetic}{转让}{zhuan3rang4}{8,5}{⾞,⾔}[HSK 5]
  \definition{v.}{ceder; fazer a entrega; transferir a posse de; ceder seus bens ou direitos a outra pessoa}
\end{EntryWithPhonetic}

\begin{EntryWithPhonetic}{转身}{zhuan3 shen1}{8,7}{⾞,⾝}[HSK 4]
  \definition{adv.}{em um instante; em um piscar de olhos}
  \definition{v.}{dar a volta; dar meia"-volta; dar a volta por cima | virar; girar; refere"-se a uma mudança de direção, localização, natureza, etc.}
\end{EntryWithPhonetic}

\begin{EntryWithPhonetic}{转弯}{zhuan3/wan1}{8,9}{⾞,⼸}[HSK 4]
  \definition{s.}{esquina; curva}[小心急转弯。===Tenha cuidado em curvas fechadas.]
  \definition{v.+compl.}{rodar; desviar; virar uma esquina; fazer uma curva; fazer uma curva de 180º}
\end{EntryWithPhonetic}

\begin{EntryWithPhonetic}{转向}{zhuan3/xiang4}{8,6}{⾞,⼝}[HSK 5]
  \definition{v.+compl.}{desviar; desviar-se; mudar a direção do avanço | mudar a posição política de alguém | mudar de direção; virar-se para (a outra parte)}
  \seeref{zhuan4/xiang4}
\end{EntryWithPhonetic}

\begin{EntryWithPhonetic}{转移}{zhuan3yi2}{8,11}{⾞,⽲}[HSK 4]
  \definition{v.}{deslocar; desviar; transferir; redirecionar; reposicionar; reorientar | mudar; transformar}
\end{EntryWithPhonetic}

\begin{EntryWithPhonetic}{转账}{zhuan3/zhang4}{8,8}{⾞,⾙}
  \definition{v.+compl.}{transferir entre contas | trazer à frente | incluir uma soma de dinheiro do balanço anterior no seguinte}
\end{EntryWithPhonetic}

%%%%%%%%%% 传 %%%%%%%%%%
\subsection*{传}\addcontentsline{loh}{figure}{传 \dpy{zhuan4}}

\begin{EntryWithPhonetic}{传}{zhuan4}{6}{⼈}
  \definition{s.}{comentários sobre clássicos; obras que explicam as escrituras| biografia | romances sobre eventos históricos; obras que narram histórias históricas}
  \seeref{chuan2}
\end{EntryWithPhonetic}

%%%%%%%%%% 转 %%%%%%%%%%
\subsection*{转}\addcontentsline{loh}{figure}{转 \dpy{zhuan4}}

\begin{EntryWithPhonetic}{转}{zhuan4}{8}{⾞}[HSK 6]
  \definition{clas.}{usado para rotações (por minuto, por segundo, etc.): RPM}
  \definition{v.}{girar; rodar; revolver; movimento em torno de um centro | passear; dar uma volta}
  \seeref{zhuai3}
  \seeref{zhuan3}
\end{EntryWithPhonetic}

\begin{EntryWithPhonetic}{转动}{zhuan4dong4}{8,6}{⾞,⼒}[HSK 6]
  \definition{s.}{tambor; roda}
  \definition{v.}{girar; correr; rolar; revolver; rotacionar; torcer}
  \seeref{zhuan3dong4}
\end{EntryWithPhonetic}

\begin{EntryWithPhonetic}{转向}{zhuan4/xiang4}{8,6}{⾞,⼝}
  \definition{v.+compl.}{perder-se; perder o rumo; não consiguir distinguir a direção; estar perdido}
  \seeref{zhuan3/xiang4}
\end{EntryWithPhonetic}

\begin{EntryWithPhonetic}{转悠}{zhuan4you5}{8,11}{⾞,⼼}
  \definition{v.}{aparecer repetidamente | rolar | passear por aí}
\end{EntryWithPhonetic}

\begin{EntryWithPhonetic}{转游}{zhuan4you5}{8,12}{⾞,⽔}
  \seealsoref{转悠}{zhuan4you5}
\end{EntryWithPhonetic}

%%%%%%%%%% 赚 %%%%%%%%%%
\subsection*{赚}\addcontentsline{loh}{figure}{赚 \dpy{zhuan4}}

\begin{EntryWithPhonetic}{赚}{zhuan4}{14}{⾙}[HSK 6]
  \definition{s.}{lucro}
  \definition{v.}{ganhar (dinheiro); obter lucro com o negócio}
  \antonymref{赔}{pei2}
\end{EntryWithPhonetic}

\begin{EntryWithPhonetic}{赚钱}{zhuan4 qian2}{14,10}{⾙,⾦}[HSK 6]
  \definition{v.}{ganhar dinheiro; obter lucro ou recompensa}
\end{EntryWithPhonetic}

%%%%%%%%%% 妆 %%%%%%%%%%
\subsection*{妆}\addcontentsline{loh}{figure}{妆 \dpy{zhuang1}}

\begin{EntryWithPhonetic}{妆}{zhuang1}{6}{⼥}
  \definition{s.}{adornos femininos | enxoval; dote | adornos pessoais femininos; maquiagem e figurino de palco; costumava se referir às maquiagens em mulheres, mas agora se refere às maquiagens em atores}
  \definition{v.}{aplicar maquiagem; maquiar | arrumar-se; maquiar-se}
\end{EntryWithPhonetic}

\begin{EntryWithPhonetic}{妆扮}{zhuang1ban4}{6,7}{⼥,⼿}
  \variantof{装扮}
\end{EntryWithPhonetic}

%%%%%%%%%% 桩 %%%%%%%%%%
\subsection*{桩}\addcontentsline{loh}{figure}{桩 \dpy{zhuang1}}

\begin{EntryWithPhonetic}{桩}{zhuang1}{10}{⽊}
  \definition{clas.}{para eventos, casos, transações, assuntos, etc.}
  \definition{s.}{toco | estaca | pilha}
\end{EntryWithPhonetic}

%%%%%%%%%% 装 %%%%%%%%%%
\subsection*{装}\addcontentsline{loh}{figure}{装 \dpy{zhuang1}}

\begin{EntryWithPhonetic}{装}{zhuang1}{12}{⾐}[HSK 2]
  \definition*{s.}{Sobrenome: Zhuang}
  \definition{s.}{vestido; traje; vestimenta; roupa | maquiagem e figurino de palco; maquiagem de ator}
  \definition{v.}{enfeitar; adornar; vestir; decorar; vestir-se; vestir-se bem | fingir; fazer de conta | segurar; embalar; carregar; colocar as coisas em recipientes; colocar as coisas no transporte | encaixar; instalar; equipar; aparelhar; montar | embalar; encaixotar; embrulhar produtos ou colocá-los em caixas, garrafas, etc.}
\end{EntryWithPhonetic}

\begin{EntryWithPhonetic}{装扮}{zhuang1ban4}{12,7}{⾐,⼿}
  \definition{v.}{enfeitar | decorar | disfarçar-me | vestir-se}
\end{EntryWithPhonetic}

\begin{EntryWithPhonetic}{装备}{zhuang1bei4}{12,8}{⾐,⼡}[HSK 6]
  \definition[套]{s.}{equipamento; equipagem; traje}
  \definition{v.}{equipar}
\end{EntryWithPhonetic}

\begin{EntryWithPhonetic}{装配}{zhuang1pei4}{12,10}{⾐,⾣}
  \definition{v.}{montar | encaixar}
\end{EntryWithPhonetic}

\begin{EntryWithPhonetic}{装饰}{zhuang1shi4}{12,8}{⾐,⾷}[HSK 5]
  \definition[件,个]{s.}{decoração; acessórios decorativos}
  \definition{v.}{enfeitar; adornar; decorar; ornamentar; embelezar; destacar}
\end{EntryWithPhonetic}

\begin{EntryWithPhonetic}{装修}{zhuang1xiu1}{12,9}{⾐,⼈}[HSK 4]
  \definition{v.}{equipar; renovar; decorar (equipar uma sala ou prédio com equipamentos ou decorações); reboco, pintura e instalação de portas, janelas, encanamentos e outros equipamentos em projetos habitacionais}
\end{EntryWithPhonetic}

\begin{EntryWithPhonetic}{装置}{zhuang1zhi4}{12,13}{⾐,⽹}[HSK 4]
  \definition[个,台,种,些]{s.}{dispositivo; equipamento; máquinas, instrumentos ou outros equipamentos de construção mais complexa e com alguma função independente}
  \definition{v.}{instalar; ajustar; configurar; equipar; montar}
\end{EntryWithPhonetic}

%%%%%%%%%% 壮 %%%%%%%%%%
\subsection*{壮}\addcontentsline{loh}{figure}{壮 \dpy{zhuang4}}

\begin{EntryWithPhonetic}{壮}{zhuang4}{6}{⼠}
  \definition*{s.}{Grupo étnico Zhuang (ou Chuang) | Sobrenome: Zhuang}
  \definition{adj.}{forte; robusto | magnífico; grandioso; majestoso}
  \definition{v.}{fortalecer; tornar melhor | expandir}
\end{EntryWithPhonetic}

\begin{EntryWithPhonetic}{壮观}{zhuang4guan1}{6,6}{⼠,⾒}[HSK 6]
  \definition{adj.}{grandioso; magnífico; espetacular}
  \definition{s.}{grande vista; vista magnífica; espetáculo esplêndido}
\end{EntryWithPhonetic}

\begin{EntryWithPhonetic}{壮族}{zhuang4 zu2}{6,11}{⼠,⽅}
  \definition*{s.}{Grupo étnico Zhuang (ou Chuang) de Guangxi}
  \seealsoref{广东}{guang3dong1}
  \seealsoref{广西}{guang3xi1}
  \seealsoref{云南}{yun2nan2}
\end{EntryWithPhonetic}

%%%%%%%%%% 状 %%%%%%%%%%
\subsection*{状}\addcontentsline{loh}{figure}{状 \dpy{zhuang4}}

\begin{EntryWithPhonetic}{状}{zhuang4}{7}{⽝}
  \definition{s.}{forma | estado; condição | conta; registro | reclamação escrita; queixa; reclamação legal | certificado | situação; circunstâncias | documento oficial; documentos de elogio, nomeação, etc.}
  \definition{v.}{descrever | narrar}
\end{EntryWithPhonetic}

\begin{EntryWithPhonetic}{状况}{zhuang4kuang4}{7,7}{⽝,⼎}[HSK 3]
  \definition[个,种]{s.}{estado; \emph{status}; situação; condição; estado de coisas; a aparência ou o estado em que as coisas se apresentam}
\end{EntryWithPhonetic}

\begin{EntryWithPhonetic}{状态}{zhuang4tai4}{7,8}{⽝,⼼}[HSK 3]
  \definition[种,个]{s.}{\emph{status}; estado; condição; situação; estado de coisas; a forma manifestada por pessoas ou coisas}
\end{EntryWithPhonetic}

%%%%%%%%%% 僮 %%%%%%%%%%
\subsection*{僮}\addcontentsline{loh}{figure}{僮 \dpy{zhuang4}}

\begin{EntryWithPhonetic}{僮}{zhuang4}{14}{⼈}
  \variantof{壮}
  \seeref{tong2}
\end{EntryWithPhonetic}

%%%%%%%%%% 撞 %%%%%%%%%%
\subsection*{撞}\addcontentsline{loh}{figure}{撞 \dpy{zhuang4}}

\begin{EntryWithPhonetic}{撞}{zhuang4}{15}{⼿}[HSK 5]
  \definition{v.}{chocar-se contra; chocar-se com; bater; colidir | encontrar-se por acaso; esbarrar em; deparar-se com | apressar; correr; empurrar | aproveitar a chance | esbarrar de repente em |  encontrar | confiar em; tentar | agir precipitadamente; invadir}
\end{EntryWithPhonetic}

\begin{EntryWithPhonetic}{撞车}{zhuang4/che1}{15,4}{⼿,⾞}
  \definition{v.+compl.}{(figurativo) colidir (opiniões, cronogramas, etc.) | ser o mesmo (assunto) | colidir (com outro veículo)}
\end{EntryWithPhonetic}

\begin{EntryWithPhonetic}{撞运气}{zhuang4yun4qi5}{15,7,4}{⼿,⾡,⽓}
  \definition{v.}{confiar no destino | tentar a sorte}
\end{EntryWithPhonetic}

%%%%%%%%%% 獞 %%%%%%%%%%
\subsection*{獞}\addcontentsline{loh}{figure}{獞 \dpy{zhuang4}}

\begin{EntryWithPhonetic}{獞}{zhuang4}{15}{⽝}
  \variantof{壮}
  \seeref{tong2}
\end{EntryWithPhonetic}

%%%%%%%%%% 追 %%%%%%%%%%
\subsection*{追}\addcontentsline{loh}{figure}{追 \dpy{zhui1}}

\begin{EntryWithPhonetic}{追}{zhui1}{9}{⾡}[HSK 3]
  \definition{v.}{perseguir; correr atrás; seguir de perto | rastrear; investigar; chegar ao fundo de | procurar; ir atrás; esforçar-se para alcançar um determinado objetivo | recordar; relembrar | fazer depois do ocorrido; retrabalhar | cortejar (uma mulher)}
\end{EntryWithPhonetic}

\begin{EntryWithPhonetic}{追赶}{zhui1gan3}{9,10}{⾡,⾛}
  \definition{v.}{perseguir | acelerar | alcançar | ultrapassar}
\end{EntryWithPhonetic}

\begin{EntryWithPhonetic}{追究}{zhui1jiu1}{9,7}{⾡,⽳}[HSK 6]
  \definition{v.}{descobrir; investigar}
\end{EntryWithPhonetic}

\begin{EntryWithPhonetic}{追求}{zhui1qiu2}{9,7}{⾡,⽔}[HSK 4]
  \definition{s.}{perseguição (ações e metas positivas)}[她的追求是获得成功。===Sua meta é alcançar o sucesso.]
  \definition{v.}{procurar; aspirar; perseguir | cortejar; refere"-se especificamente ao namoro}
\end{EntryWithPhonetic}

%%%%%%%%%% 坠 %%%%%%%%%%
\subsection*{坠}\addcontentsline{loh}{figure}{坠 \dpy{zhui4}}

\begin{EntryWithPhonetic}{坠}{zhui4}{7}{⼟}
  \definition{s.}{peso; objeto pendurado | um objeto pendurado; um pingente}
  \definition{v.}{cair; derrubar | curvar para baixo}
\end{EntryWithPhonetic}

\begin{EntryWithPhonetic}{坠落}{zhui4luo4}{7,12}{⼟,⾋}
  \definition{v.}{cair}
\end{EntryWithPhonetic}

%%%%%%%%%% 惴 %%%%%%%%%%
\subsection*{惴}\addcontentsline{loh}{figure}{惴 \dpy{zhui4}}

\begin{EntryWithPhonetic}{惴}{zhui4}{12}{⼼}
  \definition{adj.}{ansioso e medroso; Literário: é ao mesmo tempo preocupante e assustador}
\end{EntryWithPhonetic}

%%%%%%%%%% 屯 %%%%%%%%%%
\subsection*{屯}\addcontentsline{loh}{figure}{屯 \dpy{zhun1}}

\begin{EntryWithPhonetic}{屯}{zhun1}{4}{⼬}
  \definition*{s.}{Sobrenome: Zhun}
  \definition{adj.}{difícil; árduo | lento; obtuso}
  \seeref{tun2}
\end{EntryWithPhonetic}

%%%%%%%%%% 准 %%%%%%%%%%
\subsection*{准}\addcontentsline{loh}{figure}{准 \dpy{zhun3}}

\begin{EntryWithPhonetic}{准}{zhun3}{10}{⼎}[HSK 3]
  \definition{adj.}{exato; preciso; algo determinado a ser imutável | preciso; exato; correto | perto; parcialmente; quase; próximo de algo em termos de padrão}
  \definition{adv.}{definitivamente; certamente}
  \definition{pref.}{quasi-; para-}
  \definition{prep.}{de acordo com; baseado em}
  \definition{s.}{norma; padrão; critério | confiança certa; uma ideia definida, certeza, etc. (geralmente usada depois de 有 ou 没有)}
  \definition{v.}{autorizar; conceder; consentir; permitir}
  \seealsoref{没有}{mei2you5}
  \seealsoref{有}{you3}
\end{EntryWithPhonetic}

\begin{EntryWithPhonetic}{准备}{zhun3bei4}{10,8}{⼎,⼡}[HSK 1]
  \definition{v.}{preparar; ficar pronto; planejar ou organizar com antecedência | pretender; planejar}
\end{EntryWithPhonetic}

\begin{EntryWithPhonetic}{准确}{zhun3que4}{10,12}{⼎,⽯}[HSK 2]
  \definition{adj.}{exato; preciso; acurado; os resultados da ação são completamente consistentes com os resultados reais ou esperados}
\end{EntryWithPhonetic}

\begin{EntryWithPhonetic}{准时}{zhun3shi2}{10,7}{⼎,⽇}[HSK 4]
  \definition{adj.}{pontual}
  \definition{adv.}{na hora certa; dentro do prazo; no horário especificado}
\end{EntryWithPhonetic}

%%%%%%%%%% 捉 %%%%%%%%%%
\subsection*{捉}\addcontentsline{loh}{figure}{捉 \dpy{zhuo1}}

\begin{EntryWithPhonetic}{捉}{zhuo1}{10}{⼿}[HSK 6]
  \definition{v.}{agarrar; segurar; apreender | pegar; capturar; aprisionar}
\end{EntryWithPhonetic}

%%%%%%%%%% 桌 %%%%%%%%%%
\subsection*{桌}\addcontentsline{loh}{figure}{桌 \dpy{zhuo1}}

\begin{EntryWithPhonetic}{桌}{zhuo1}{10}{⽊}
  \definition{clas.}{usado para mesas de convidados em um banquete etc.}
  \definition{s.}{mesa}
\end{EntryWithPhonetic}

\begin{EntryWithPhonetic}{桌布}{zhuo1bu4}{10,5}{⽊,⼱}
  \definition[条,块,张]{s.}{(computação) plano de fundo da área de trabalho | toalha de mesa | papel de parede}
\end{EntryWithPhonetic}

\begin{EntryWithPhonetic}{桌灯}{zhuo1deng1}{10,6}{⽊,⽕}
  \definition{s.}{luminária | lâmpada de mesa}
\end{EntryWithPhonetic}

\begin{EntryWithPhonetic}{桌机}{zhuo1ji1}{10,6}{⽊,⽊}
  \definition{s.}{computador \emph{desktop}}
\end{EntryWithPhonetic}

\begin{EntryWithPhonetic}{桌面}{zhuo1mian4}{10,9}{⽊,⾯}
  \definition{s.}{área de trabalho | mesa}
\end{EntryWithPhonetic}

\begin{EntryWithPhonetic}{桌球}{zhuo1qiu2}{10,11}{⽊,⽟}
  \definition{s.}{bilhar | sinuca | mesa de ping-pong}
\end{EntryWithPhonetic}

\begin{EntryWithPhonetic}{桌游}{zhuo1you2}{10,12}{⽊,⽔}
  \definition{s.}{jogo de tabuleiro}
\end{EntryWithPhonetic}

\begin{EntryWithPhonetic}{桌子}{zhuo1zi5}{10,3}{⽊,⼦}[HSK 1]
  \definition[张,套]{s.}{mesa; escrivaninha; móveis, com uma superfície plana na parte superior e uma estrutura de suporte na parte inferior, para colocar objetos ou realizar atividades}
\end{EntryWithPhonetic}

%%%%%%%%%% 棹 %%%%%%%%%%
\subsection*{棹}\addcontentsline{loh}{figure}{棹 \dpy{zhuo1}}

\begin{EntryWithPhonetic}{棹}{zhuo1}{12}{⽊}
  \variantof{桌}
\end{EntryWithPhonetic}

%%%%%%%%%% 浊 %%%%%%%%%%
\subsection*{浊}\addcontentsline{loh}{figure}{浊 \dpy{zhuo2}}

\begin{EntryWithPhonetic}{浊}{zhuo2}{9}{⽔}
  \definition*{s.}{Sobrenome: Zhuo}
  \definition{adj.}{turvo; lamacento; imundo | profundo e espesso | caótico; confuso; corrompido}
  \antonymref{清}{qing1}
\end{EntryWithPhonetic}

%%%%%%%%%% 着 %%%%%%%%%%
\subsection*{着}\addcontentsline{loh}{figure}{着 \dpy{zhuo2}}

\begin{EntryWithPhonetic}{着}{zhuo2}{11}{⽬}
  \definition{v.}{vestir (roupas); vestir-se | tocar; entrar em contato com; aproximar-se de; (contato físico) | enviar; despachar | expressão usada em documentos oficiais antigos, indicando um tom de ordem | aplicar; usar; adicionar; anexar}
  \seeref{zhao1}
  \seeref{zhao2}
  \seeref{zhe5}
\end{EntryWithPhonetic}

\begin{EntryWithPhonetic}{着花}{zhuo2hua1}{11,7}{⽬,⾋}
  \definition{s.}{floração}
  \definition{v.}{florescer}
  \seeref{zhao2hua1}
\end{EntryWithPhonetic}

\begin{EntryWithPhonetic}{着手}{zhuo2shou3}{11,4}{⽬,⼿}
  \definition{v.}{colocar a mão nisso | estabelecer | começar uma tarefa}
\end{EntryWithPhonetic}

\begin{EntryWithPhonetic}{着想}{zhuo2xiang3}{11,13}{⽬,⼼}
  \definition{v.}{considerar (as necessidades de outras pessoas) | pensar (para os outros)}
\end{EntryWithPhonetic}

\begin{EntryWithPhonetic}{着眼}{zhuo2yan3}{11,11}{⽬,⽬}
  \definition{v.}{ter seus olhos em (um objetivo) | ter algo em mente | concentrar-se}
\end{EntryWithPhonetic}

\begin{EntryWithPhonetic}{着装}{zhuo2zhuang1}{11,12}{⽬,⾐}
  \definition{s.}{roupa | vestimenta}
  \definition{v.}{vestir}
\end{EntryWithPhonetic}

%%%%%%%%%% 缴 %%%%%%%%%%
\subsection*{缴}\addcontentsline{loh}{figure}{缴 \dpy{zhuo2}}

\begin{EntryWithPhonetic}{缴}{zhuo2}{16}{⽷}
  \definition{s.}{corda de seda crua usada na antiguidade para amarrar flechas para caçar pássaros}
  \seeref{jiao3}
\end{EntryWithPhonetic}

%%%%%%%%%% 仔 %%%%%%%%%%
\subsection*{仔}\addcontentsline{loh}{figure}{仔 \dpy{zi1}}

\begin{EntryWithPhonetic}{仔}{zi1}{5}{⼈}
  \definition{s./v.}{usado em 仔肩}
  \seeref{zi3}
  \seealsoref{仔肩}{zi1jian1}
\end{EntryWithPhonetic}

\begin{EntryWithPhonetic}{仔肩}{zi1jian1}{5,8}{⼈,⾁}
  \definition{s.}{encargos oficiais (ou responsabilidades)}
  \definition{v.}{assumir a responsabilidade por algo}
\end{EntryWithPhonetic}

%%%%%%%%%% 吱 %%%%%%%%%%
\subsection*{吱}\addcontentsline{loh}{figure}{吱 \dpy{zi1}}

\begin{EntryWithPhonetic}{吱}{zi1}{7}{⼝}
  \definition{s.}{Onomatopéia: (ratos) guincho; (pássaros pequenos) chilreio; pio; descreve os sons de pequenos animais}
\end{EntryWithPhonetic}

%%%%%%%%%% 咨 %%%%%%%%%%
\subsection*{咨}\addcontentsline{loh}{figure}{咨 \dpy{zi1}}

\begin{EntryWithPhonetic}{咨}{zi1}{9}{⼝}
  \definition[行]{s.}{comunicação oficial; relatório entregue pelo chefe de um governo sobre assuntos de Estado}
  \definition{v.}{consultar; discutir com}
\end{EntryWithPhonetic}

\begin{EntryWithPhonetic}{咨询}{zi1xun2}{9,8}{⼝,⾔}[HSK 6]
  \definition{v.}{consultar; aconselhar-se com; buscar conselho de; pedir conselhos}
\end{EntryWithPhonetic}

%%%%%%%%%% 资 %%%%%%%%%%
\subsection*{资}\addcontentsline{loh}{figure}{资 \dpy{zi1}}

\begin{EntryWithPhonetic}{资}{zi1}{10}{⾙}
  \definition{s.}{recursos | capital | dinheiro | despesa}
  \definition{v.}{fornecer | suprir}
\end{EntryWithPhonetic}

\begin{EntryWithPhonetic}{资本}{zi1ben3}{10,5}{⾙,⽊}[HSK 5]
  \definition{s.}{capital; meios de produção ou moeda utilizados para fins lucrativos | o que é capitalizado; algo usado em benefício próprio; metáfora para obter benefícios}
\end{EntryWithPhonetic}

\begin{EntryWithPhonetic}{资本主义}{zi1ben3 zhu3yi4}{10,5,5,3}{⾙,⽊,⼂,⼂}
  \definition*{s.}{Capitalismo}
\end{EntryWithPhonetic}

\begin{EntryWithPhonetic}{资产}{zi1chan3}{10,6}{⾙,⼇}[HSK 5]
  \definition{s.}{propriedade; bens; patrimônio | capital; fundo de capital; recursos financeiros da empresa | ativos; na contabilidade, refere"-se à utilização de fundos}
\end{EntryWithPhonetic}

\begin{EntryWithPhonetic}{资格}{zi1ge5}{10,10}{⾙,⽊}[HSK 3]
  \definition{s.}{qualificação; condições e identidades necessárias para exercer uma determinada atividade | senioridade; identidade formada pelo tempo dedicado a um determinado trabalho ou atividade}
\end{EntryWithPhonetic}

\begin{EntryWithPhonetic}{资金}{zi1jin1}{10,8}{⾙,⾦}[HSK 3]
  \definition[笔]{s.}{fundo; capital; capital necessário para atividades comerciais, etc.}
\end{EntryWithPhonetic}

\begin{EntryWithPhonetic}{资料}{zi1liao4}{10,10}{⾙,⽃}[HSK 4]
  \definition[份,堆,本,个]{s.}{dados; material; material informativo para referência ou para ser considerado confiável | material de produção; meios de subsistência; requisitos de produção ou subsistência}
\end{EntryWithPhonetic}

\begin{EntryWithPhonetic}{资源}{zi1yuan2}{10,13}{⾙,⽔}[HSK 4]
  \definition{s.}{recurso; fontes naturais de meios de produção ou subsistência}
\end{EntryWithPhonetic}

\begin{EntryWithPhonetic}{资助}{zi1zhu4}{10,7}{⾙,⼒}[HSK 5]
  \definition{s.}{subsídio}
  \definition{v.}{subsidiar; patrocinar; ajudar financeiramente; ajudar com recursos financeiros}
\end{EntryWithPhonetic}

%%%%%%%%%% 子 %%%%%%%%%%
\subsection*{子}\addcontentsline{loh}{figure}{子 \dpy{zi3}}

\begin{EntryWithPhonetic}{子}{zi3}{3}{⼦}[Kangxi 39]
  \definition*{s.}{Sobrenome: Zi}
  \definition{adj.}{pequeno; jovem; tenro | subsidiário; subordinado; derivado}
  \definition{clas.}{usado para objetos finos que podem ser pinçados com os dedos}
  \definition{pron.}{você;  antigamente, era uma forma de tratamento respeitosa para se referir a outras pessoas, equivalente a 您}
  \definition[个,位,名]{s.}{filho, criança; antigamente, referia"-se aos filhos, mas atualmente refere"-se especificamente aos filhos homens | pessoa | antigo título de respeito para um homem culto ou virtuoso; na antiguidade, referia"-se especificamente a homens eruditos | visconde; o quarto posto na hierarquia dos cinco títulos feudais da nobreza | ovo | semente | coisas pequenas e duras; pequenos fragmentos ou grãos duros e sólidos | cobre; moeda de cobre | o primeiro dos doze ramos terrestres}
  \seeref{zi5}
  \seealsoref{您}{nin2}
\end{EntryWithPhonetic}

\begin{EntryWithPhonetic}{子弹}{zi3dan4}{3,11}{⼦,⼸}[HSK 5]
  \definition[粒,颗,发]{s.}{bala; cartucho; munição}
\end{EntryWithPhonetic}

\begin{EntryWithPhonetic}{子女}{zi3nv3}{3,3}{⼦,⼥}[HSK 3]
  \definition[个]{s.}{crianças; descendentes; filhos e filhas}
\end{EntryWithPhonetic}

%%%%%%%%%% 仔 %%%%%%%%%%
\subsection*{仔}\addcontentsline{loh}{figure}{仔 \dpy{zi3}}

\begin{EntryWithPhonetic}{仔}{zi3}{5}{⼈}
  \definition{adj.}{jovem | cuidadoso; pequeno; fino}
  \seeref{zi1}
\end{EntryWithPhonetic}

\begin{EntryWithPhonetic}{仔细}{zi3xi4}{5,8}{⼈,⽷}[HSK 5]
  \definition{adj.}{cuidadoso; atencioso; descreve alguém que é cuidadoso e meticuloso ao fazer as coisas; não é descuidado | frugal; econômico; descreve o uso moderado de dinheiro ou bens, sem desperdício}
  \definition{v.}{ter cuidado; prestar atenção; ter muito cuidado e evitar que aconteçam coisas ruins}
\end{EntryWithPhonetic}

%%%%%%%%%% 紫 %%%%%%%%%%
\subsection*{紫}\addcontentsline{loh}{figure}{紫 \dpy{zi3}}

\begin{EntryWithPhonetic}{紫}{zi3}{12}{⽷}[HSK 5]
  \definition*{s.}{Sobrenome: Zi}
  \definition{adj.}{roxo; púrpura; violeta; cor resultante da combinação do vermelho e do azul}
\end{EntryWithPhonetic}

\begin{EntryWithPhonetic}{紫色}{zi3 se4}{12,6}{⽷,⾊}
  \definition{s.}{cor púrpura | cor violeta}
\end{EntryWithPhonetic}

%%%%%%%%%% 字 %%%%%%%%%%
\subsection*{字}\addcontentsline{loh}{figure}{字 \dpy{zi4}}

\begin{EntryWithPhonetic}{字}{zi4}{6}{⼦}[HSK 1]
  \definition[个]{s.}{palavra; caractere; texto | pronúncia (de uma palavra ou caractere); som do caractere | tipo de impressão; estilo de caligrafia; forma de um caractere escrito ou impresso; refere"-se às diferentes formas dos caracteres chineses; também se refere às diferentes escolas de caligrafia | escritas; obras de caligrafia | recibo; compromisso por escrito; documento | nome de estilo masculino adotado aos vinte anos de idade | sobrenome | um número indicado num contador elétrico, contador de água, etc.; registrar dos números dos medidores de consumo de água e eletricidade}
  \definition{v.}{Arcaico: ficar noiva}
\end{EntryWithPhonetic}

\begin{EntryWithPhonetic}{字典}{zi4dian3}{6,8}{⼦,⼋}[HSK 2]
  \definition[本,册,部]{s.}{dicionário de caracteres chineses (contendo verbetes de caracteres únicos, em contraste com 词典 que contém verbetes para palavras com um ou mais caracteres)}
  \seealsoref{词典}{ci2dian3}
\end{EntryWithPhonetic}

\begin{EntryWithPhonetic}{字脚}{zi4jiao3}{6,11}{⼦,⾁}
  \definition[典]{s.}{gancho no final da pincelada | serifa}
\end{EntryWithPhonetic}

\begin{EntryWithPhonetic}{字母}{zi4mu3}{6,5}{⼦,⽏}[HSK 4]
  \definition[个,种]{s.}{letra; letras de um alfabeto | Fonologia: caractere que representa uma consoante inicial}
\end{EntryWithPhonetic}

\begin{EntryWithPhonetic}{字眼}{zi4yan3}{6,11}{⼦,⽬}
  \definition[个]{s.}{palavras | redação}
\end{EntryWithPhonetic}

\begin{EntryWithPhonetic}{字字珠玉}{zi4zi4zhu1yu4}{6,6,10,5}{⼦,⼦,⽟,⽟}
  \definition{expr.}{cada palavra é uma jóia}
  \definition{s.}{escrita magnífica}
\end{EntryWithPhonetic}

%%%%%%%%%% 自 %%%%%%%%%%
\subsection*{自}\addcontentsline{loh}{figure}{自 \dpy{zi4}}

\begin{EntryWithPhonetic}{自}{zi4}{6}{⾃}[HSK 4][Kangxi 132]
  \definition*{s.}{Sobrenome: Zi}
  \definition{adv.}{certamente; com certeza; é claro; naturalmente}
  \definition{prep.}{de; desde; a partir de; apresenta o ponto de partida, a fonte ou o horário de início do comportamento da ação, equivalente a 从 e 由}
  \definition{pron.}{si mesmo; próprio | próprio; indica que a ação é iniciada por e direcionada a si mesmo | por si mesmo; indica que a ação é autoiniciada e não é causada por uma força externa}
  \definition{v.}{iniciar}
  \seealsoref{从}{cong2}
  \seealsoref{由}{you2}
\end{EntryWithPhonetic}

\begin{EntryWithPhonetic}{自从}{zi4cong2}{6,4}{⾃,⼈}[HSK 3]
  \definition{prep.}{de; desde; a partir de; referir-se a um momento ou evento específico no passado}
\end{EntryWithPhonetic}

\begin{EntryWithPhonetic}{自动}{zi4dong4}{6,6}{⾃,⼒}[HSK 3]
  \definition{adj.}{automático; auto"-atuante; uso de dispositivos mecânicos, elétricos, etc, para funcionar automaticamente, sem necessidade de controle humano}
  \definition{adv.}{voluntariamente; por vontade própria; por iniciativa própria | automaticamente; espontaneamente; refere"-se a movimentos, mudanças, etc., que não são causados pela ação humana, mas sim pelo próprio objeto}
\end{EntryWithPhonetic}

\begin{EntryWithPhonetic}{自动化}{zi4dong4hua4}{6,6,4}{⾃,⼒,⼔}
  \definition{s.}{automação}
\end{EntryWithPhonetic}

\begin{EntryWithPhonetic}{自个儿}{zi4ge3r5}{6,3,2}{⾃,⼈,⼉}
  \definition{pron.}{(dialeto) a si mesmo, por si mesmo}
\end{EntryWithPhonetic}

\begin{EntryWithPhonetic}{自豪}{zi4hao2}{6,14}{⾃,⾗}[HSK 5]
  \definition{adj.}{orgulhar-se de; ter orgulho de; sentir-se honrado por possuir qualidades excelentes ou ter alcançado grandes conquistas, seja por si mesmo ou por um grupo ou indivíduo relacionado a si}
\end{EntryWithPhonetic}

\begin{EntryWithPhonetic}{自己}{zi4ji3}{6,3}{⾃,⼰}[HSK 2]
  \definition{pron.}{a si próprio; a si mesmo; refere"-se ao substantivo ou pronome precedente (enfatiza principalmente que não é devido a forças externas)}
\end{EntryWithPhonetic}

\begin{EntryWithPhonetic}{自己动手}{zi4ji3dong4shou3}{6,3,6,4}{⾃,⼰,⼒,⼿}
  \definition{v.}{fazer (algo) sozinho | ajudar-se a}
\end{EntryWithPhonetic}

\begin{EntryWithPhonetic}{自救}{zi4jiu4}{6,11}{⾃,⽁}
  \definition{s.}{autoajuda}
  \definition{v.}{salvar-se; prover-se e ajudar-se}
\end{EntryWithPhonetic}

\begin{EntryWithPhonetic}{自觉}{zi4jue2}{6,9}{⾃,⾒}[HSK 3]
  \definition{adj.}{autoconsciente; de livre e espontânea vontade; controlar o próprio comportamento e agir por iniciativa própria}
  \definition{v.}{estar ciente de}
\end{EntryWithPhonetic}

\begin{EntryWithPhonetic}{自来水}{zi4lai2shui3}{6,7,4}{⾃,⽊,⽔}[HSK 6]
  \definition{s.}{água da torneira; água corrente; água purificada e desinfetada fornecida por sistema hidráulico através de tubulações | equipamentos para transporte de água natural tratada}
\end{EntryWithPhonetic}

\begin{EntryWithPhonetic}{自燃}{zi4ran2}{6,16}{⾃,⽕}
  \definition{s.}{combustão espontânea}
\end{EntryWithPhonetic}

\begin{EntryWithPhonetic}{自然}{zi4ran5}{6,12}{⾃,⽕}[HSK 3]
  \definition{adj.}{natural; no curso normal dos eventos; formado ou desenvolvido sem intervenção humana; algo que se desenvolve livremente}
  \definition{adv.}{naturalmente; definitivamente; certamente, isso significa que, de acordo com a lógica, deve ser assim}
  \definition{conj.}{usado para ligar duas frases, com a segunda introduzindo informações adicionais ou adversativas; indica explicação complementar ou uma mudança de significado}
  \definition{s.}{natureza; mundo natural; tudo o que não foi criado pelo ser humano}
\end{EntryWithPhonetic}

\begin{EntryWithPhonetic}{自杀}{zi4sha1}{6,6}{⾃,⽊}[HSK 5]
  \definition{s.}{suicídio; auto-assassinato; auto-sacrifício}
  \definition{v.}{cometer suicídio; tentar suicídio; matar-se}
\end{EntryWithPhonetic}

\begin{EntryWithPhonetic}{自身}{zi4shen1}{6,7}{⾃,⾝}[HSK 3]
  \definition{pron.}{eu mesmo (enfatizando que não é outra pessoa ou outra coisa)}
\end{EntryWithPhonetic}

\begin{EntryWithPhonetic}{自我}{zi4wo3}{6,7}{⾃,⼽}[HSK 6]
  \definition{pref.}{auto-}
  \definition{pron.}{a si mesmo; eu próprio; geralmente usado antes de verbos dissílabos para indicar que a ação é realizada por alguém e dirigida a si mesmo | indicar o próprio caráter diferente dos outros; refere"-se às próprias características, personalidade, hobbies, etc.}
\end{EntryWithPhonetic}

\begin{EntryWithPhonetic}{自我安慰}{zi4wo3'an1wei4}{6,7,6,15}{⾃,⼽,⼧,⼼}
  \definition{v.}{confortar-se | consolar-se | tranquilizar-se}
\end{EntryWithPhonetic}

\begin{EntryWithPhonetic}{自我保存}{zi4wo3 bao3cun2}{6,7,9,6}{⾃,⼽,⼈,⼦}
  \definition{v.}{autopreservação}
\end{EntryWithPhonetic}

\begin{EntryWithPhonetic}{自我吹嘘}{zi4wo3 chui1xu1}{6,7,7,14}{⾃,⼽,⼝,⼝}
  \definition{expr.}{gabar-se}
  \definition{s.}{auto-ostentação; autoglorificação;}
\end{EntryWithPhonetic}

\begin{EntryWithPhonetic}{自我催眠}{zi4wo3cui1mian2}{6,7,13,10}{⾃,⼽,⼈,⽬}
  \definition{v.}{consolar-me | tranquilizar-me}
\end{EntryWithPhonetic}

\begin{EntryWithPhonetic}{自我的人}{zi4wo3de5ren2}{6,7,8,2}{⾃,⼽,⽩,⼈}
  \definition{s.}{(minha, sua) própria pessoa | (afirmar) a própria personalidade}
\end{EntryWithPhonetic}

\begin{EntryWithPhonetic}{自我防卫}{zi4wo3fang2wei4}{6,7,6,3}{⾃,⼽,⾩,⼙}
  \definition{s.}{defesa pessoal | auto-defesa}
\end{EntryWithPhonetic}

\begin{EntryWithPhonetic}{自我解嘲}{zi4wo3 jie3chao2}{6,7,13,15}{⾃,⼽,⾓,⼝}
  \definition{s.}{autodepreciação; referir"-se às próprias fraquezas ou falhas com humor autodepreciativo}
  \definition{v.}{encontrar desculpas para; consolar"-se}
\end{EntryWithPhonetic}

\begin{EntryWithPhonetic}{自我介绍}{zi4wo3jie4shao4}{6,7,4,8}{⾃,⼽,⼈,⽷}
  \definition{s.}{defesa pessoal | auto-defesa}
\end{EntryWithPhonetic}

\begin{EntryWithPhonetic}{自我批评}{zi4wo3 pi1ping2}{6,7,7,7}{⾃,⼽,⼿,⾔}
  \definition{s.}{autocrítica}
\end{EntryWithPhonetic}

\begin{EntryWithPhonetic}{自我实现}{zi4wo3shi2xian4}{6,7,8,8}{⾃,⼽,⼧,⾒}
  \definition{s.}{(psicologia) auto-realização}
\end{EntryWithPhonetic}

\begin{EntryWithPhonetic}{自我陶醉}{zi4wo3tao2zui4}{6,7,10,15}{⾃,⼽,⾩,⾣}
  \definition{s.}{narcisista | auto-imbuído | satisfeito consigo mesmo}
\end{EntryWithPhonetic}

\begin{EntryWithPhonetic}{自我意识}{zi4wo3 yi4shi2}{6,7,13,7}{⾃,⼽,⼼,⾔}
  \definition{s.}{autoconsciência; auto-consciente}
\end{EntryWithPhonetic}

\begin{EntryWithPhonetic}{自信}{zi4xin4}{6,9}{⾃,⼈}[HSK 4]
  \definition{adj.}{confiante; descreve a crença em suas próprias habilidades, decisões, etc., tendo confiança em si mesmo}
  \definition[份,种]{s.}{autoconfiança; confiança em si mesmo}
  \definition{v.}{acreditar em si mesmo}
\end{EntryWithPhonetic}

\begin{EntryWithPhonetic}{自行车}{zi4xing2che1}{6,6,4}{⾃,⾏,⾞}[HSK 2]
  \definition[辆]{s.}{bicicleta; um veículo de duas rodas que é impulsionado para a frente com os pedais}
\end{EntryWithPhonetic}

\begin{EntryWithPhonetic}{自行车馆}{zi4xing2che1guan3}{6,6,4,11}{⾃,⾏,⾞,⾷}
  \definition{s.}{estádio de ciclismo | velódromo}
\end{EntryWithPhonetic}

\begin{EntryWithPhonetic}{自行车架}{zi4xing2che1jia4}{6,6,4,9}{⾃,⾏,⾞,⽊}
  \definition{s.}{suporte para bicicleta | bicicletário}
\end{EntryWithPhonetic}

\begin{EntryWithPhonetic}{自行车赛}{zi4xing2che1sai4}{6,6,4,14}{⾃,⾏,⾞,⾙}
  \definition{s.}{corrida de bicicleta}
\end{EntryWithPhonetic}

\begin{EntryWithPhonetic}{自学}{zi4xue2}{6,8}{⾃,⼦}[HSK 6]
  \definition{s.}{auto-estudo; autodidata; autoaprendizagem}
  \definition{v.}{estudar por conta própria; estudar de forma independente; ensinar a si mesmo}
\end{EntryWithPhonetic}

\begin{EntryWithPhonetic}{自言自语}{zi4yan2-zi4yu3}{6,7,6,9}{⾃,⾔,⾃,⾔}[HSK 6]
  \definition{expr.}{falando sozinho; falar consigo mesmo; pensar em voz alta; solilóquio}
\end{EntryWithPhonetic}

\begin{EntryWithPhonetic}{自由}{zi4you2}{6,5}{⾃,⽥}[HSK 2]
  \definition{adj.}{livre; irrestrito}
  \definition[个]{s.}{liberdade; o direito de agir de acordo com a própria vontade dentro do âmbito da lei | liberdade; filosoficamente, liberdade é definida como o processo de as pessoas reconhecerem as leis que governam o desenvolvimento das coisas e aplicá-las conscientemente na prática}
\end{EntryWithPhonetic}

\begin{EntryWithPhonetic}{自由泳}{zi4you2yong3}{6,5,8}{⾃,⽥,⽔}
  \definition{s.}{natação de estilo livre}
\end{EntryWithPhonetic}

\begin{EntryWithPhonetic}{自愿}{zi4yuan4}{6,14}{⾃,⽕}[HSK 5]
  \definition{adv.}{voluntariamente; por iniciativa própria; por vontade própria}
  \definition{s.}{voluntário}
\end{EntryWithPhonetic}

\begin{EntryWithPhonetic}{自在}{zi4zai4}{6,6}{⾃,⼟}[HSK 6]
  \definition{adj.}{livre; irrestrito}
\end{EntryWithPhonetic}

\begin{EntryWithPhonetic}{自责}{zi4ze2}{6,8}{⾃,⾙}
  \definition{v.}{culpar"-se}
\end{EntryWithPhonetic}

\begin{EntryWithPhonetic}{自主}{zi4zhu3}{6,5}{⾃,⼂}[HSK 3]
  \definition{v.}{agir por conta própria; decidir por si mesmo; manter a iniciativa em suas próprias mãos; tomar suas próprias decisões}
\end{EntryWithPhonetic}

%%%%%%%%%% 子 %%%%%%%%%%
\subsection*{子}\addcontentsline{loh}{figure}{子 \dpy{zi5}}

\begin{EntryWithPhonetic}{子}{zi5}{3}{⼦}[HSK 1][Kangxi 39]
  \definition{suf.}{sufixo para substantivos | sufixos de palavras de medida individuais; anexado a certas palavras classificadoras}
  \seeref{zi3}
\end{EntryWithPhonetic}

%%%%%%%%%% 宗 %%%%%%%%%%
\subsection*{宗}\addcontentsline{loh}{figure}{宗 \dpy{zong1}}

\begin{EntryWithPhonetic}{宗}{zong1}{8}{⼧}
  \definition*{s.}{Sobrenome: Zong}
  \definition{adj.}{do mesmo clã; da mesma família}
  \definition{clas.}{usado para matérias, cargas, etc.}
  \definition{s.}{ancestral; antepassado | clã; família | seita; facção; escola | objetivo principal; propósito | modelo; grande mestre | Obsoleto: unidade administrativa no Tibete, equivalente a um condado | templo ancestral}
  \definition{v.}{(no trabalho acadêmico ou artístico) tomar como modelo; modelar"-se em}
\end{EntryWithPhonetic}

\begin{EntryWithPhonetic}{宗教}{zong1jiao4}{8,11}{⼧,⽁}[HSK 6]
  \definition[种]{s.}{religião; uma ideologia social é um reflexo ilusório do mundo objetivo, exigindo que as pessoas acreditem em Deus, no Xintoísmo, em espíritos, no carma, etc., e que depositem suas esperanças no chamado céu ou vida após a morte}
\end{EntryWithPhonetic}

%%%%%%%%%% 综 %%%%%%%%%%
\subsection*{综}\addcontentsline{loh}{figure}{综 \dpy{zong1}}

\begin{EntryWithPhonetic}{综}{zong1}{11}{⽷}
  \definition*{s.}{Sobrenome: Zong}
  \definition{v.}{reunir; resumir | combinar; reunir}
  \seeref{zeng4}
\end{EntryWithPhonetic}

\begin{EntryWithPhonetic}{综合}{zong1he2}{11,6}{⽷,⼝}[HSK 4]
  \definition{s.}{síntese}
  \definition{v.}{sintetizar; resumir as partes de uma coisa em um todo unificado após análise; reunir coisas de um tipo ou natureza diferente}
  \antonymref{分析}{fen1xi1}
\end{EntryWithPhonetic}

%%%%%%%%%% 棕 %%%%%%%%%%
\subsection*{棕}\addcontentsline{loh}{figure}{棕 \dpy{zong1}}

\begin{EntryWithPhonetic}{棕}{zong1}{12}{⽊}
  \definition{adj.}{marrom}
  \definition[个]{s.}{palmeira | fibra de palmeira; fibra de coco}
\end{EntryWithPhonetic}

\begin{EntryWithPhonetic}{棕褐色}{zong1he4 se4}{12,14,6}{⽊,⾐,⾊}
  \definition{s.}{cor sépia | bronzeado}
\end{EntryWithPhonetic}

%%%%%%%%%% 总 %%%%%%%%%%
\subsection*{总}\addcontentsline{loh}{figure}{总 \dpy{zong3}}

\begin{EntryWithPhonetic}{总}{zong3}{9}{⼼}[HSK 3]
  \definition{adj.}{total; geral; global | responsável (liderança)}
  \definition{adv.}{sempre; invariavelmente | de qualquer forma; afinal; eventualmente; mais cedo ou mais tarde; no fim das contas | certamente; provavelmente; com certeza; expressa estimativa; suposição; equivalente a 大概}
  \definition{v.}{reunir; resumir; juntar; compilar}
  \seealsoref{大概}{da4gai4}
\end{EntryWithPhonetic}

\begin{EntryWithPhonetic}{总部}{zong3bu4}{9,10}{⼼,⾢}[HSK 6]
  \definition{s.}{sede geral; escritório central}
\end{EntryWithPhonetic}

\begin{EntryWithPhonetic}{总裁}{zong3cai2}{9,12}{⼼,⾐}[HSK 5]
  \definition[位,名,个]{s.}{presidente (de uma empresa); nomes de certos líderes de partidos políticos ou grandes empresas}
\end{EntryWithPhonetic}

\begin{EntryWithPhonetic}{总长}{zong3chang2}{9,4}{⼼,⾧}
  \definition{s.}{comprimento total}
\end{EntryWithPhonetic}

\begin{EntryWithPhonetic}{总得}{zong3dei3}{9,11}{⼼,⼻}
  \definition{adv.}{prestes a}
  \definition{v.}{dever | precisar}
\end{EntryWithPhonetic}

\begin{EntryWithPhonetic}{总督}{zong3du1}{9,13}{⼼,⽬}
  \definition*{s.}{Governador-Geral | Governador | Vice-Rei}
\end{EntryWithPhonetic}

\begin{EntryWithPhonetic}{总共}{zong3gong4}{9,6}{⼼,⼋}[HSK 4]
  \definition{adv.}{em tudo; em todos; no total; completamente; totalmente; em conjunto}
\end{EntryWithPhonetic}

\begin{EntryWithPhonetic}{总价}{zong3jia4}{9,6}{⼼,⼈}
  \definition{s.}{preço total}
\end{EntryWithPhonetic}

\begin{EntryWithPhonetic}{总监}{zong3jian1}{9,10}{⼼,⽫}[HSK 6]
  \definition[名,位]{s.}{inspetor geral; inspetor-chefe}
\end{EntryWithPhonetic}

\begin{EntryWithPhonetic}{总结}{zong3jie2}{9,9}{⼼,⽷}[HSK 3]
  \definition[个,篇]{s.}{resumo; síntese; conclusão resumida}
  \definition{v.}{resumir; sumariar; sintetizar; analisar e estudar as experiências para chegar a conclusões}
\end{EntryWithPhonetic}

\begin{EntryWithPhonetic}{总经理}{zong3jing1li3}{9,8,11}{⼼,⽷,⽟}[HSK 6]
  \definition[位,名,个,些]{s.}{CEO; gerente geral; o mais alto executivo de uma empresa ou organização similar, que geralmente tem o poder de decidir políticas administrativas e de gestão}
\end{EntryWithPhonetic}

\begin{EntryWithPhonetic}{总理}{zong3li3}{9,11}{⼼,⽟}[HSK 4]
  \definition*[个,位,名]{s.}{Primeiro-Ministro do Conselho de Estado; Título do líder do Conselho de Estado da China | Título do chefe de governo em determinados países | Primeiro-Ministro; Título de líderes de determinados partidos políticos | Título dos chefes de determinadas instituições e empresas nos velhos tempos}
  \definition{v.}{assumir a responsabilidade total}
\end{EntryWithPhonetic}

\begin{EntryWithPhonetic}{总量}{zong3liang4}{9,12}{⼼,⾥}[HSK 6]
  \definition{s.}{capacidade total; quantidade bruta | valor total | total}
\end{EntryWithPhonetic}

\begin{EntryWithPhonetic}{总是}{zong3shi4}{9,9}{⼼,⽇}[HSK 3]
  \definition{adv.}{sempre; indica como tem sido durante um determinado período de tempo; um determinado estado permanece inalterado | afinal; significa que, independentemente do que acontecer, haverá ou será um resultado}
\end{EntryWithPhonetic}

\begin{EntryWithPhonetic}{总数}{zong3shu4}{9,13}{⼼,⽁}[HSK 5]
  \definition{s.}{soma; total; totalidade; inventário; número total; soma total}
\end{EntryWithPhonetic}

\begin{EntryWithPhonetic}{总算}{zong3suan4}{9,14}{⼼,⽵}[HSK 5]
  \definition{adv.}{finalmente; por fim; indica que, após um longo período de tempo, um desejo finalmente se tornou realidade | suficiente; considerando tudo; no geral; considerando todos os aspectos; significa que, em geral, está tudo bem}
\end{EntryWithPhonetic}

\begin{EntryWithPhonetic}{总台}{zong3tai2}{9,5}{⼼,⼝}
  \definition{s.}{recepção | balcão de recepção}
\end{EntryWithPhonetic}

\begin{EntryWithPhonetic}{总体}{zong3ti3}{9,7}{⼼,⼈}[HSK 5]
  \definition{s.}{total; geral; conjunto; totalidade; massa; população; o todo formado pela união de vários indivíduos; a totalidade das coisas}
\end{EntryWithPhonetic}

\begin{EntryWithPhonetic}{总统}{zong3tong3}{9,9}{⼼,⽷}[HSK 4]
  \definition*[个,位,名,家]{s.}{Presidente (de um país); Título dos líderes de determinadas repúblicas}
\end{EntryWithPhonetic}

\begin{EntryWithPhonetic}{总务}{zong3wu4}{9,5}{⼼,⼒}
  \definition{s.}{divisão de assuntos gerais | assuntos gerais | pessoa responsável geral}
\end{EntryWithPhonetic}

\begin{EntryWithPhonetic}{总线}{zong3xian4}{9,8}{⼼,⽷}
  \definition{s.}{barramento (computador) | \emph{computer bus}}
\end{EntryWithPhonetic}

\begin{EntryWithPhonetic}{总站}{zong3zhan4}{9,10}{⼼,⽴}
  \definition{s.}{terminal}
\end{EntryWithPhonetic}

\begin{EntryWithPhonetic}{总之}{zong3zhi1}{9,3}{⼼,⼂}[HSK 4]
  \definition{conj.}{em uma palavra; em suma; em resumo; indica que a declaração seguinte é uma declaração geral}
\end{EntryWithPhonetic}

\begin{EntryWithPhonetic}{总值}{zong3zhi2}{9,10}{⼼,⼈}
  \definition{s.}{valor total}
\end{EntryWithPhonetic}

%%%%%%%%%% 纵 %%%%%%%%%%
\subsection*{纵}\addcontentsline{loh}{figure}{纵 \dpy{zong4}}

\begin{EntryWithPhonetic}{纵}{zong4}{7}{⽷}
  \definition{adj.}{de norte a sul; geograficamente norte-sul | longitudinal | vertical; horizontal; paralelo ao lado longo do objeto | amassado; com rugas}
  \definition{conj.}{embora; mesmo que}
  \definition{v.}{libertar; deixar ir | entregar-se a; deixar-se levar | pular; saltar}
\end{EntryWithPhonetic}

\begin{EntryWithPhonetic}{纵向}{zong4xiang4}{7,6}{⽷,⼝}
  \definition{adj.}{vertical; de frente para trás; de norte para sul | longitudinal}
  \antonymref{横向}{heng2xiang4}
\end{EntryWithPhonetic}

%%%%%%%%%% 赱 %%%%%%%%%%
\subsection*{赱}\addcontentsline{loh}{figure}{赱 \dpy{zou3}}

\begin{EntryWithPhonetic}{赱}{zou3}{6}{⼟}
  \variantof{走}
\end{EntryWithPhonetic}

%%%%%%%%%% 走 %%%%%%%%%%
\subsection*{走}\addcontentsline{loh}{figure}{走 \dpy{zou3}}

\begin{EntryWithPhonetic}{走}{zou3}{7}{⾛}[HSK 1][Kangxi 156]
  \definition{v.}{andar; caminhar | correr | mover; movimentar; deslocar | sair; partir; ir embora | visitar; fazer uma visita; (entre amigos e familiares) troca de visitas | passar por; atravessar; ultrapassar | vazar; revelar; divulgar | afastar-se do original; alterar ou perder a forma, o sabor, a cor, etc. originais}
\end{EntryWithPhonetic}

\begin{EntryWithPhonetic}{走鬼}{zou3gui3}{7,9}{⾛,⿁}
  \definition{s.}{vendedor ambulante sem licença}
\end{EntryWithPhonetic}

\begin{EntryWithPhonetic}{走过}{zou3guo4}{7,6}{⾛,⾡}[HSK 2]
  \definition{v.}{passar por; perambular}
\end{EntryWithPhonetic}

\begin{EntryWithPhonetic}{走进}{zou3jin4}{7,7}{⾛,⾡}[HSK 2]
  \definition{v.}{entrar}
\end{EntryWithPhonetic}

\begin{EntryWithPhonetic}{走开}{zou3kai1}{7,4}{⾛,⼶}[HSK 2]
  \definition{v.}{ir embora; fugir; ir para outro lugar}
\end{EntryWithPhonetic}

\begin{EntryWithPhonetic}{走路}{zou3 lu4}{7,13}{⾛,⾜}[HSK 1]
  \definition{v.}{caminhar; ir a pé; andar em pé sobre a terra | sair; ir embora; partir}
\end{EntryWithPhonetic}

\begin{EntryWithPhonetic}{走去}{zou3qu4}{7,5}{⾛,⼛}
  \definition{v.}{caminhar até (para)}
\end{EntryWithPhonetic}

\begin{EntryWithPhonetic}{走绳}{zou3sheng2}{7,11}{⾛,⽷}
  \definition{v.}{andar na corda bamba; dança da corda; caminhada na corda; um tipo de acrobacia em que atores andam para frente e para trás em uma corda suspensa e realizam diversas ações}
  \seealsoref{走索}{zou3suo3}
\end{EntryWithPhonetic}

\begin{EntryWithPhonetic}{走势}{zou3shi4}{7,8}{⾛,⼒}
  \definition{s.}{caminho | tendência}
\end{EntryWithPhonetic}

\begin{EntryWithPhonetic}{走私}{zou3/si1}{7,7}{⾛,⽲}[HSK 6]
  \definition{v.+compl.}{contrabandear; ter um caso ilícito; violar regulamentos alfandegários; evadir inspeções alfandegárias; transportar mercadorias ilegalmente para dentro e para fora do país}
\end{EntryWithPhonetic}

\begin{EntryWithPhonetic}{走索}{zou3suo3}{7,10}{⾛,⽷}
  \definition{v.}{Acrobacia: andar na corda bamba; dança da corda; caminhada na corda}
  \seealsoref{走绳}{zou3sheng2}
\end{EntryWithPhonetic}

\begin{EntryWithPhonetic}{走秀}{zou3xiu4}{7,7}{⾛,⽲}
  \definition{s.}{desfile de moda}
  \definition{v.}{andar na passarela (em um desfile de moda)}
\end{EntryWithPhonetic}

\begin{EntryWithPhonetic}{走卒}{zou3zu2}{7,8}{⾛,⼗}
  \definition{s.}{lacaio (masculino) | peão (isto é, soldado de infantaria) | servo}
\end{EntryWithPhonetic}

%%%%%%%%%% 奏 %%%%%%%%%%
\subsection*{奏}\addcontentsline{loh}{figure}{奏 \dpy{zou4}}

\begin{EntryWithPhonetic}{奏}{zou4}{9}{⼤}[HSK 6]
  \definition{v.}{tocar (música); executar (em um instrumento musical)  | alcançar; produzir; alcançar ou estabelecer (desempenho ou realização) | (antigo) apresentar um memorial a um imperador; fazer uma petição}
\end{EntryWithPhonetic}

\begin{EntryWithPhonetic}{奏效}{zou4xiao4}{9,10}{⼤,⽁}
  \definition{v.}{mostrar resultados | ser eficaz}
\end{EntryWithPhonetic}

%%%%%%%%%% 租 %%%%%%%%%%
\subsection*{租}\addcontentsline{loh}{figure}{租 \dpy{zu1}}

\begin{EntryWithPhonetic}{租}{zu1}{10}{⽲}[HSK 2]
  \definition{s.}{aluguel | imposto sobre a terra; tributação; Antigo: refere"-se ao imposto predial}
  \definition{v.}{contratar; alugar; fretar | alugar; arrendar}
\end{EntryWithPhonetic}

\begin{EntryWithPhonetic}{租船}{zu1chuan2}{10,11}{⽲,⾈}
  \definition{v.}{fretar um navio | alugar um navio}
\end{EntryWithPhonetic}

\begin{EntryWithPhonetic}{租房}{zu1fang2}{10,8}{⽲,⼾}
  \definition{v.}{alugar um apartamento}
\end{EntryWithPhonetic}

\begin{EntryWithPhonetic}{租金}{zu1jin1}{10,8}{⽲,⾦}[HSK 6]
  \definition[笔]{s.}{aluguel; aluguer; o custo do aluguel de terras, casas ou itens; a renda do aluguel de terras, casas ou itens}
  \seealsoref{租钱}{zu1qian5}
\end{EntryWithPhonetic}

\begin{EntryWithPhonetic}{租赁}{zu1lin4}{10,10}{⽲,⾙}
  \definition{v.}{contratar | alugar}
\end{EntryWithPhonetic}

\begin{EntryWithPhonetic}{租钱}{zu1qian5}{10,10}{⽲,⾦}
  \definition{s.}{aluguel}
  \seealsoref{租金}{zu1jin1}
\end{EntryWithPhonetic}

\begin{EntryWithPhonetic}{租让}{zu1rang4}{10,5}{⽲,⾔}
  \definition{v.}{alugar | alugar (a propriedade de alguém para outra pessoa)}
\end{EntryWithPhonetic}

\begin{EntryWithPhonetic}{租用}{zu1yong4}{10,5}{⽲,⽤}
  \definition{v.}{contratar | alugar | alugar (algo de alguém)}
\end{EntryWithPhonetic}

\begin{EntryWithPhonetic}{租约}{zu1yue1}{10,6}{⽲,⽷}
  \definition{s.}{aluguel}
\end{EntryWithPhonetic}

%%%%%%%%%% 足 %%%%%%%%%%
\subsection*{足}\addcontentsline{loh}{figure}{足 \dpy{zu2}}

\begin{EntryWithPhonetic}{足}{zu2}{7}{⾜}[HSK 6][Kangxi 157]
  \definition{adj.}{amplo; bastante; suficiente}
  \definition{adv.}{totalmente; tanto quanto; indica suficiente até certo ponto ou grau | (geralmente no negativo) o suficiente; suficientemente; é totalmente possível; vale a pena}
  \definition{s.}{pé; um termo geral para os membros inferiores do corpo humano; especificamente a parte abaixo do tornozelo | futebol; refere"-se ao futebol ou a um time de futebol | pé (de utensílios); a parte inferior do objeto tem o formato de uma pé e serve de suporte}
  \seeref{ju4}
\end{EntryWithPhonetic}

\begin{EntryWithPhonetic}{足够}{zu2gou4}{7,11}{⾜,⼣}[HSK 3]
  \definition{adj.}{bastante; amplo; suficiente; atingir o nível adequado ou capaz de satisfazer as necessidades}
  \definition{v.}{satisfazer; ser suficiente; estar a contento}
\end{EntryWithPhonetic}

\begin{EntryWithPhonetic}{足球}{zu2qiu2}{7,11}{⾜,⽟}[HSK 3]
  \definition[个,只,颗,袋]{s.}{futebol | bola de futebol}
\end{EntryWithPhonetic}

\begin{EntryWithPhonetic}{足球场}{zu2qiu2chang3}{7,11,6}{⾜,⽟,⼟}
  \definition{s.}{campo de futebol}
\end{EntryWithPhonetic}

\begin{EntryWithPhonetic}{足球队}{zu2qiu2 dui4}{7,11,4}{⾜,⽟,⾩}
  \definition[个,支]{s.}{time de futebol}
\end{EntryWithPhonetic}

\begin{EntryWithPhonetic}{足球迷}{zu2qiu2mi2}{7,11,9}{⾜,⽟,⾡}
  \definition{s.}{fã (ou entusiasta) de futebol}
\end{EntryWithPhonetic}

\begin{EntryWithPhonetic}{足球赛}{zu2qiu2sai4}{7,11,14}{⾜,⽟,⾙}
  \definition{s.}{competição de futebol | partida de futebol}
\end{EntryWithPhonetic}

\begin{EntryWithPhonetic}{足球协会}{zu2qiu2xie2hui4}{7,11,6,6}{⾜,⽟,⼗,⼈}
  \definition*{s.}{Associação de Futebol}
\end{EntryWithPhonetic}

\begin{EntryWithPhonetic}{足以}{zu2yi3}{7,4}{⾜,⼈}[HSK 6]
  \definition{adj.}{suficiente; totalmente capaz de; o suficiente para fazer algo}
  \definition{v.}{bastar}
\end{EntryWithPhonetic}

\begin{EntryWithPhonetic}{足月}{zu2yue4}{7,4}{⾜,⽉}
  \definition{s.}{gestação completa}
\end{EntryWithPhonetic}

\begin{EntryWithPhonetic}{足足}{zu2zu2}{7,7}{⾜,⾜}
  \definition{adv.}{tanto quanto | extremamente | completamente | não menos que}
\end{EntryWithPhonetic}

%%%%%%%%%% 卒 %%%%%%%%%%
\subsection*{卒}\addcontentsline{loh}{figure}{卒 \dpy{zu2}}

\begin{EntryWithPhonetic}{卒}{zu2}{8}{⼗}
  \definition{adv.}{finalmente; enfim}
  \definition{s.}{Obsoleto: soldado; recruta | Obsoleto: servo | peão (uma das peças do xadrez chinês)}
  \definition{v.}{Literário: terminar; finalizar | morrer}
  \seeref{cu4}
\end{EntryWithPhonetic}

%%%%%%%%%% 族 %%%%%%%%%%
\subsection*{族}\addcontentsline{loh}{figure}{族 \dpy{zu2}}

\begin{EntryWithPhonetic}{族}{zu2}{11}{⽅}[HSK 6]
  \definition{s.}{clã; família | uma pena de morte na China antiga, imposta ao infrator e a toda a sua família, ou mesmo às famílias de sua mãe e esposa; uma antiga forma de tortura |
etnia; nacionalidade | uma grande categoria de coisas que compartilham algum atributo comum}
\end{EntryWithPhonetic}

%%%%%%%%%% 诅 %%%%%%%%%%
\subsection*{诅}\addcontentsline{loh}{figure}{诅 \dpy{zu3}}

\begin{EntryWithPhonetic}{诅}{zu3}{7}{⾔}
  \definition{v.}{amaldiçoar; xingar; imprecar; abusar; desejar algo maldoso | jurar; fazer um voto; prestar juramento}
\end{EntryWithPhonetic}

\begin{EntryWithPhonetic}{诅咒}{zu3zhou4}{7,8}{⾔,⼝}
  \definition{v.}{amaldiçoar}
\end{EntryWithPhonetic}

%%%%%%%%%% 阻 %%%%%%%%%%
\subsection*{阻}\addcontentsline{loh}{figure}{阻 \dpy{zu3}}

\begin{EntryWithPhonetic}{阻}{zu3}{7}{⾩}
  \definition{v.}{impedir; bloquear; obstruir; atrapalhar}
\end{EntryWithPhonetic}

\begin{EntryWithPhonetic}{阻碍}{zu3'ai4}{7,13}{⾩,⽯}[HSK 5]
  \definition{s.}{obstáculo; impedimento; barreira}
  \definition{v.}{bloquear; impedir; obstruir; impedir o bom andamento ou desenvolvimento}
\end{EntryWithPhonetic}

\begin{EntryWithPhonetic}{阻击}{zu3ji1}{7,5}{⾩,⼐}
  \definition{v.}{verificar | parar}
\end{EntryWithPhonetic}

\begin{EntryWithPhonetic}{阻止}{zu3zhi3}{7,4}{⾩,⽌}[HSK 4]
  \definition{v.}{parar; reter; conter; interromper; impedir o avanço; impedir o movimento; obstruir}
\end{EntryWithPhonetic}

%%%%%%%%%% 组 %%%%%%%%%%
\subsection*{组}\addcontentsline{loh}{figure}{组 \dpy{zu3}}

\begin{EntryWithPhonetic}{组}{zu3}{8}{⽷}[HSK 2]
  \definition{clas.}{usado para conjuntos, séries, suítes, baterias}
  \definition[个]{s.}{grupo; uma unidade composta por um pequeno número de pessoas}
  \definition{v.}{formar; organizar; combinar pessoas ou coisas dispersas em um todo ou sistema}
\end{EntryWithPhonetic}

\begin{EntryWithPhonetic}{组成}{zu3/cheng2}{8,6}{⽷,⼽}[HSK 2]
  \definition{v.+compl.}{formar; compor; inventar}
\end{EntryWithPhonetic}

\begin{EntryWithPhonetic}{组合}{zu3he2}{8,6}{⽷,⼝}[HSK 3]
  \definition{s.}{associação; combinação; o todo organizado | combinação; retirar n elementos diferentes de m elementos e agrupá-los, independentemente da ordem, em que cada grupo contenha pelo menos um elemento diferente, o resultado obtido é chamado de combinação de n elementos de m}
  \definition{v.}{compor; constituir; formar}
\end{EntryWithPhonetic}

\begin{EntryWithPhonetic}{组长}{zu3zhang3}{8,4}{⽷,⾧}[HSK 2]
  \definition[名,位,个]{s.}{líder de grupo; um supervisor de grupo}
\end{EntryWithPhonetic}

\begin{EntryWithPhonetic}{组织}{zu3zhi1}{8,8}{⽷,⽷}[HSK 5]
  \definition{s.}{organização; um coletivo ou grupo estabelecido de acordo com determinados objetivos e princípios | sistema organizado; vários fatores interligados de determinada maneira, formando um sistema | tecer; a combinação de linhas horizontais e verticais nos têxteis | tecido; os seres humanos, os animais, as plantas e outros seres vivos são compostos por uma combinação de células com formas e funções semelhantes, que formam os tecidos; os tecidos são as unidades que compõem os diversos órgãos}
\end{EntryWithPhonetic}

%%%%%%%%%% 祖 %%%%%%%%%%
\subsection*{祖}\addcontentsline{loh}{figure}{祖 \dpy{zu3}}

\begin{EntryWithPhonetic}{祖}{zu3}{9}{⽰}
  \definition*{s.}{Sobrenome: Zu}
  \definition{s.}{avô; geração anterior dos pais | ancestral; antepassado | fundador (de um negócio, facção, seita religiosa, etc.); originador; fundador; mestre fundador}
\end{EntryWithPhonetic}

\begin{EntryWithPhonetic}{祖父}{zu3fu4}{9,4}{⽰,⽗}[HSK 6]
  \definition{s.}{avô (paterno)}
\end{EntryWithPhonetic}

\begin{EntryWithPhonetic}{祖国}{zu3guo2}{9,8}{⽰,⼞}[HSK 6]
  \definition{s.}{país; pátria; próprio país}
\end{EntryWithPhonetic}

\begin{EntryWithPhonetic}{祖母}{zu3mu3}{9,5}{⽰,⽏}[HSK 6]
  \definition{s.}{avó (paterna)}
\end{EntryWithPhonetic}

%%%%%%%%%% 钻 %%%%%%%%%%
\subsection*{钻}\addcontentsline{loh}{figure}{钻 \dpy{zuan1}}

\begin{EntryWithPhonetic}{钻}{zuan1}{10}{⾦}[HSK 6]
  \definition{v.}{furar; perfurar; girar um objeto pontiagudo para perfurar outro objeto | perfurar; entrar; penetrar; passar por | aprofundar"-se; estudar intensivamente; fazer um estudo penetrante de | buscar ganho pessoal; tramar; refere"-se a esquemas}
  \seeref{zuan4}
\end{EntryWithPhonetic}

\begin{EntryWithPhonetic}{钻}{zuan4}{10}{⾦}
  \definition[把]{s.}{broca; pua; sonda; existem muitos tipos de ferramentas para perfuração, incluindo manivela, elétrica e pneumática | joia; diamante}
  \definition{v.}{furar; perfurar;  girar um objeto pontiagudo para perfurar outro objeto}
  \seeref{zuan1}
\end{EntryWithPhonetic}

\begin{EntryWithPhonetic}{钻戒}{zuan4jie4}{10,7}{⾦,⼽}
  \definition[枚]{s.}{anel de diamante}
\end{EntryWithPhonetic}

\begin{EntryWithPhonetic}{钻石}{zuan4shi2}{10,5}{⾦,⽯}
  \definition[颗]{s.}{diamante}
\end{EntryWithPhonetic}

%%%%%%%%%% 咀 %%%%%%%%%%
\subsection*{咀}\addcontentsline{loh}{figure}{咀 \dpy{zui3}}

\begin{EntryWithPhonetic}{咀}{zui3}{8}{⼝}
  \definition{s.}{boca; lábios; a palavra 嘴 é coloquialmente escrita como 咀 | boca (de um objeto)}
  \seeref{ju3}
  \seealsoref{嘴}{zui3}
\end{EntryWithPhonetic}

%%%%%%%%%% 嘴 %%%%%%%%%%
\subsection*{嘴}\addcontentsline{loh}{figure}{嘴 \dpy{zui3}}

\begin{EntryWithPhonetic}{嘴}{zui3}{16}{⼝}[HSK 2]
  \definition[张]{s.}{boca; boca humana ou animal | qualquer coisa com formato ou função semelhante a uma boca | fala | comida}
\end{EntryWithPhonetic}

\begin{EntryWithPhonetic}{嘴巴}{zui3ba5}{16,4}{⼝,⼰}[HSK 4]
  \definition[张]{s.}{boca}
\end{EntryWithPhonetic}

\begin{EntryWithPhonetic}{嘴巴子}{zui3ba5zi5}{16,4,3}{⼝,⼰,⼦}
  \definition{s.}{tapa | bofetada}
\end{EntryWithPhonetic}

%%%%%%%%%% 最 %%%%%%%%%%
\subsection*{最}\addcontentsline{loh}{figure}{最 \dpy{zui4}}

\begin{EntryWithPhonetic}{最}{zui4}{12}{⽈}[HSK 1]
  \definition{adv.}{(diante de um adjetivo ou verbo) o mais | (colocado antes de um substantivo de localidade ou de uma palavra que indica um lugar)  mais distante ou mais próximo de (um lugar) | mais; melhor; pior; primeiro; muito; menos; acima de tudo; indica que uma determinada característica excede todas as outras pessoas ou coisas do mesmo tipo}
  \definition{s.}{o máximo; o melhor (ou o mais alto, o maior, etc.)}
\end{EntryWithPhonetic}

\begin{EntryWithPhonetic}{最初}{zui4chu1}{12,7}{⽈,⾐}[HSK 4]
  \definition{adj.}{primordial; inicial; primeiro}
  \definition{adv.}{inicialmente; originalmente}
  \definition{s.}{o período mais antigo; início; começo}
\end{EntryWithPhonetic}

\begin{EntryWithPhonetic}{最多}{zui4duo1}{12,6}{⽈,⼣}
  \definition{adv.}{no máximo | máximo}
\end{EntryWithPhonetic}

\begin{EntryWithPhonetic}{最高}{zui4gao1}{12,10}{⽈,⾼}
  \definition{adj.}{altíssimo | supremo | mais alto}
\end{EntryWithPhonetic}

\begin{EntryWithPhonetic}{最好}{zui4hao3}{12,6}{⽈,⼥}[HSK 1]
  \definition{adj.}{melhor; de primeira qualidade; excelente}
  \definition{adv.}{seria melhor; seria o ideal; indica a escolha mais adequada entre várias possibilidades}
\end{EntryWithPhonetic}

\begin{EntryWithPhonetic}{最后}{zui4hou4}{12,6}{⽈,⼝}[HSK 1]
  \definition{s.}{último; final; definitivo; refere"-se ao tempo, local, etc. que vem depois de outros tempos, locais, etc. na ordem sequencial}
\end{EntryWithPhonetic}

\begin{EntryWithPhonetic}{最佳}{zui4jia1}{12,8}{⽈,⼈}[HSK 6]
  \definition{adj.}{o melhor (atleta, filme etc); ótimo}
\end{EntryWithPhonetic}

\begin{EntryWithPhonetic}{最近}{zui4jin4}{12,7}{⽈,⾡}[HSK 2]
  \definition{adj.}{mais próximo}
  \definition{s.}{recentemente; ultimamente; de tarde; refere"-se aos dias antes ou logo depois de um discurso | em breve; no futuro próximo; o futuro próximo}
\end{EntryWithPhonetic}

\begin{EntryWithPhonetic}{最善}{zui4shan4}{12,12}{⽈,⼝}
  \definition{adj.}{ótimo | o melhor}
\end{EntryWithPhonetic}

\begin{EntryWithPhonetic}{最少}{zui4shao3}{12,4}{⽈,⼩}
  \definition{adv.}{finalmente}
\end{EntryWithPhonetic}

\begin{EntryWithPhonetic}{最先}{zui4xian1}{12,6}{⽈,⼉}
  \definition{adv.}{o primeiro}
\end{EntryWithPhonetic}

\begin{EntryWithPhonetic}{最新}{zui4xin1}{12,13}{⽈,⽄}
  \definition{adv.}{mais recente | mais novo}
\end{EntryWithPhonetic}

\begin{EntryWithPhonetic}{最优}{zui4you1}{12,6}{⽈,⼈}
  \definition{adj.}{ótimo}
\end{EntryWithPhonetic}

\begin{EntryWithPhonetic}{最远}{zui4yuan3}{12,7}{⽈,⾡}
  \definition{adv.}{mais distante | mais longe}
\end{EntryWithPhonetic}

\begin{EntryWithPhonetic}{最终}{zui4zhong1}{12,8}{⽈,⽷}[HSK 6]
  \definition{adv.}{finalmente; por fim}
  \definition{s.}{final; definitivo}
\end{EntryWithPhonetic}

%%%%%%%%%% 罪 %%%%%%%%%%
\subsection*{罪}\addcontentsline{loh}{figure}{罪 \dpy{zui4}}

\begin{EntryWithPhonetic}{罪}{zui4}{13}{⽹}[HSK 6]
  \definition{s.}{crime; culpa | falha; culpa | sofrimento; dor; dificuldade | má conduta; transgressão; negligência | agonia; dor; sofrimento}
  \definition{v.}{colocar a culpa em alguém; culpar}
\end{EntryWithPhonetic}

\begin{EntryWithPhonetic}{罪恶}{zui4'e4}{13,10}{⽹,⼼}[HSK 6]
  \definition{s.}{pecado; mal; crime; comportamento criminoso grave}
\end{EntryWithPhonetic}

\begin{EntryWithPhonetic}{罪犯}{zui4fan4}{13,5}{⽹,⽝}
  \definition[名,个]{s.}{culpado; criminoso; infrator; pessoas que cometem crime}
\end{EntryWithPhonetic}

\begin{EntryWithPhonetic}{罪人}{zui4ren2}{13,2}{⽹,⼈}
  \definition{s.}{culpado; infrator; pecador}
  \antonymref{功臣}{gong1chen2}
\end{EntryWithPhonetic}

\begin{EntryWithPhonetic}{罪行}{zui4xing2}{13,6}{⽹,⾏}
  \definition{s.}{crime | ofensa}
\end{EntryWithPhonetic}

%%%%%%%%%% 醉 %%%%%%%%%%
\subsection*{醉}\addcontentsline{loh}{figure}{醉 \dpy{zui4}}

\begin{EntryWithPhonetic}{醉}{zui4}{15}{⾣}[HSK 5]
  \definition{v.}{embriagar-se; ficar bêbado; intoxicar-se; beber em excesso e perder o controle | (de certos alimentos) ser embebido em licor; ser mergulhado em vinho; marinar (alimentos) em vinho | entregar-se a; ser viciado em; gostar demais, a ponto de chegar à obsessão}
\end{EntryWithPhonetic}

%%%%%%%%%% 尊 %%%%%%%%%%
\subsection*{尊}\addcontentsline{loh}{figure}{尊 \dpy{zun1}}

\begin{EntryWithPhonetic}{尊}{zun1}{12}{⼨}
  \definition*{s.}{Sobrenome: Zun}
  \definition{adj.}{sênior; de uma geração sênior; alto status ou antiguidade}
  \definition{clas.}{usado para estátuas, canhões, etc.}
  \definition{pron.}{seu; vossa; antigamente, referia"-se a pessoas ou coisas relacionadas entre si}
  \definition{s.}{um tipo de recipiente para vinho usado nos tempos antigos}
  \definition{v.}{respeitar; reverenciar; venerar; honrar}
\end{EntryWithPhonetic}

\begin{EntryWithPhonetic}{尊敬}{zun1jing4}{12,12}{⼨,⽁}[HSK 5]
  \definition{adj.}{respeitoso; respeitável}
  \definition{v.}{respeitar; honrar; estimar}
\end{EntryWithPhonetic}

\begin{EntryWithPhonetic}{尊重}{zun1zhong4}{12,9}{⼨,⾥}[HSK 5]
  \definition{adj.}{sério; adequado; correto; (linguagem, comportamento) não ser descuidado; não ser leviano}
  \definition{v.}{respeitar; valorizar; estimar; tratar com educação; valorizar | tratar com seriedade; levar a sério e tratar com seriedade}
\end{EntryWithPhonetic}

%%%%%%%%%% 遵 %%%%%%%%%%
\subsection*{遵}\addcontentsline{loh}{figure}{遵 \dpy{zun1}}

\begin{EntryWithPhonetic}{遵}{zun1}{15}{⾡}
  \definition{v.}{cumprir; obedecer; observar; seguir}
\end{EntryWithPhonetic}

\begin{EntryWithPhonetic}{遵守}{zun1shou3}{15,6}{⾡,⼧}[HSK 5]
  \definition{v.}{obedecer; observar; cumprir; respeitar; atuar de acordo com as regras; não infringir}
\end{EntryWithPhonetic}

%%%%%%%%%% 作 %%%%%%%%%%
\subsection*{作}\addcontentsline{loh}{figure}{作 \dpy{zuo1}}

\begin{EntryWithPhonetic}{作}{zuo1}{7}{⼈}
  \definition{adj.}{(gíria) incômodo}
  \definition{s.}{trabalhador | oficina | (pessoa) de alta manutenção}
  \seeref{zuo4}
\end{EntryWithPhonetic}

%%%%%%%%%% 昨 %%%%%%%%%%
\subsection*{昨}\addcontentsline{loh}{figure}{昨 \dpy{zuo2}}

\begin{EntryWithPhonetic}{昨}{zuo2}{9}{⽇}
  \definition{s.}{ontem | o passado}
\end{EntryWithPhonetic}

\begin{EntryWithPhonetic}{昨日}{zuo2ri4}{9,4}{⽇,⽇}
  \definition{adv.}{ontem}
\end{EntryWithPhonetic}

\begin{EntryWithPhonetic}{昨天}{zuo2tian1}{9,4}{⽇,⼤}[HSK 1]
  \definition{s.}{ontem}
\end{EntryWithPhonetic}

\begin{EntryWithPhonetic}{昨晚}{zuo2wan3}{9,11}{⽇,⽇}
  \definition{adv.}{noite passada | ontem à noite}
\end{EntryWithPhonetic}

\begin{EntryWithPhonetic}{昨夜}{zuo2ye4}{9,8}{⽇,⼣}
  \definition{adv.}{noite passada}
\end{EntryWithPhonetic}

%%%%%%%%%% 左 %%%%%%%%%%
\subsection*{左}\addcontentsline{loh}{figure}{左 \dpy{zuo3}}

\begin{EntryWithPhonetic}{左}{zuo3}{5}{⼯}[HSK 1]
  \definition*{s.}{Sobrenome: Zuo}
  \definition{adj.}{estranho; herético; não ortodoxo | errado; incorreto | diferente; contrário; oposto | progressista; revolucionário; politicamente e ideologicamente progressista; radical}
  \definition{s.}{a esquerda; o lado esquerdo | leste; na antiguidade, referia"-se especificamente à direção leste (com base na orientação para o sul) | a esquerda; ala esquerda; refere"-se a uma posição inferior (na antiguidade, a direita era considerada superior e a esquerda, inferior)}
  \definition{v.}{assistir; auxiliar}
\end{EntryWithPhonetic}

\begin{EntryWithPhonetic}{左边}{zuo3bian5}{5,5}{⼯,⾡}[HSK 1]
  \definition{s.}{esquerda; o lado esquerdo}
\end{EntryWithPhonetic}

\begin{EntryWithPhonetic}{左面}{zuo3mian4}{5,9}{⼯,⾯}
  \definition{s.}{esquerda | lado esquerdo}
\end{EntryWithPhonetic}

\begin{EntryWithPhonetic}{左派}{zuo3pai4}{5,9}{⼯,⽔}
  \definition{s.}{(política) esquerda | esquerdista}
\end{EntryWithPhonetic}

\begin{EntryWithPhonetic}{左倾}{zuo3qing1}{5,10}{⼯,⼈}
  \definition{s.}{esquerdista | progressivo}
\end{EntryWithPhonetic}

\begin{EntryWithPhonetic}{左袒}{zuo3tan3}{5,10}{⼯,⾐}
  \definition{v.}{ser tendencioso | ser parcial para | favorecer um lado | tomar partido com}
\end{EntryWithPhonetic}

\begin{EntryWithPhonetic}{左舷}{zuo3xian2}{5,11}{⼯,⾈}
  \definition{s.}{porto (lado de um navio)}
\end{EntryWithPhonetic}

\begin{EntryWithPhonetic}{左翼}{zuo3yi4}{5,17}{⼯,⽻}
  \definition{s.}{esquerda (política)}
\end{EntryWithPhonetic}

\begin{EntryWithPhonetic}{左右}{zuo3you4}{5,5}{⼯,⼝}[HSK 3]
  \definition{s.}{os lados esquerdo e direito; esquerda e direita, também indicam os arredores | atendentes; acompanhantes; as pessoas que o acompanham | aproximadamente; mais ou menos; por aí; usado após números para indicar uma estimativa, com o mesmo significado de 上下}
  \definition{v.}{controlar; manipular; influenciar; dominar}
  \seealsoref{上下}{shang4xia4}
\end{EntryWithPhonetic}

%%%%%%%%%% 作 %%%%%%%%%%
\subsection*{作}\addcontentsline{loh}{figure}{作 \dpy{zuo4}}

\begin{EntryWithPhonetic}{作}{zuo4}{7}{⼈}[HSK 6]
  \definition*{s.}{Sobrenome: Zuo}
  \definition{s.}{trabalho; escritos; obras}
  \definition{v.}{subir; aumentar; crescer; aparecer | escrever; compor; criar | afetar; fingir; assumir; fingir deliberadamente ser uma determinada pessoa | considerar como; tomar algo ou alguém por | ter; sentir | ser; agir como; tornar-se; servir como | envolver-se em uma atividade; realizar alguma atividade | fazer; criar}
  \seeref{zuo1}
\end{EntryWithPhonetic}

\begin{EntryWithPhonetic}{作出}{zuo4chu1}{7,5}{⼈,⼐}[HSK 4]
  \definition{v.}{mostrar; tomar (decisões, conclusões, etc. por meio de consideração ou discussão); formar (uma conclusão, decisão, etc.) por meio de consideração ou discussão}
\end{EntryWithPhonetic}

\begin{EntryWithPhonetic}{作废}{zuo4fei4}{7,8}{⼈,⼴}[HSK 6]
  \definition{v.}{anular; tornar inválido; abandonar devido a falha}
\end{EntryWithPhonetic}

\begin{EntryWithPhonetic}{作家}{zuo4jia1}{7,10}{⼈,⼧}[HSK 2]
  \definition[位,名,个,些]{s.}{escritor; autor; pessoas que alcançaram sucesso na criação literária}
\end{EntryWithPhonetic}

\begin{EntryWithPhonetic}{作品}{zuo4pin3}{7,9}{⼈,⼝}[HSK 3]
  \definition[个,部,篇,幅]{s.}{obra de arte; obras literárias e artísticas}
\end{EntryWithPhonetic}

\begin{EntryWithPhonetic}{作为}{zuo4wei2}{7,4}{⼈,⼂}[HSK 4]
  \definition{prep.}{como; na capacidade de; no caráter de; no papel de; em termos de uma certa identidade de uma pessoa ou de uma certa natureza de uma coisa}
  \definition{s.}{ato; ação; conduta; feito; comportamento | conquista; realização; especificamente, uma boa ação}
  \definition{v.}{considerar como; tomar por; olhar como; tratar como | realizar; fazer conquistas; deixar uma marca}
\end{EntryWithPhonetic}

\begin{EntryWithPhonetic}{作文}{zuo4/wen2}{7,4}{⼈,⽂}[HSK 2]
  \definition[篇]{s.}{ensaio; composição; redação}
  \definition{v.+compl.}{(de alunos) escrever uma redação, artigo ou ensaio}
\end{EntryWithPhonetic}

\begin{EntryWithPhonetic}{作业}{zuo4ye4}{7,5}{⼈,⼀}[HSK 2]
  \definition[份,个]{s.}{tarefa escolar; tarefa de casa atribuída pelos professores aos alunos}
  \definition{v.}{trabalhar; executar tarefa}
\end{EntryWithPhonetic}

\begin{EntryWithPhonetic}{作用}{zuo4yong4}{7,5}{⼈,⽤}[HSK 2]
  \definition[副]{s.}{efeito; ação; função; a influência sobre as coisas; o efeito; a utilidade}
  \definition{v.}{afetar; agir sobre; realizar atividades que têm algum impacto nas coisas}
\end{EntryWithPhonetic}

\begin{EntryWithPhonetic}{作战}{zuo4 zhan4}{7,9}{⼈,⼽}[HSK 6]
  \definition{s.}{lutar; combater; batalhar}
\end{EntryWithPhonetic}

\begin{EntryWithPhonetic}{作者}{zuo4zhe3}{7,8}{⼈,⽼}[HSK 3]
  \definition[位,名,个]{s.}{autor; escritor; pessoas que escrevem artigos ou criam obras de arte}
\end{EntryWithPhonetic}

%%%%%%%%%% 坐 %%%%%%%%%%
\subsection*{坐}\addcontentsline{loh}{figure}{坐 \dpy{zuo4}}

\begin{EntryWithPhonetic}{坐}{zuo4}{7}{⼟}[HSK 1]
  \definition*{s.}{Sobrenome: Zuo}
  \definition{adv.}{sem motivo algum; sem causa ou razão; sem motivo aparente}
  \definition{prep.}{porque; pelo fato de que; pela razão de que; pelo motivo de que}
  \definition{s.}{assento; lugar; posição}
  \definition{v.}{sentar; sentar-se; ocupar um lugar; colocar os glúteos sobre um objeto para apoiar o peso corporal | pegar; viajar de; pegar carona | ter as costas voltadas para | colocar (uma panela, chaleira, etc.) no fogo | recuo; coice (de rifles, armas, etc.)  | produzir frutos; formar sementes | ser punido; ser acusado de crime | contrair (ou ter) uma doença; sofrer de uma doença | (um edifício) afundar; ceder}
\end{EntryWithPhonetic}

\begin{EntryWithPhonetic}{坐标}{zuo4biao1}{7,9}{⼟,⽊}
  \definition{s.}{Geometria: coordenada; um número ou conjunto de números que pode determinar a posição de um ponto no espaço}
\end{EntryWithPhonetic}

\begin{EntryWithPhonetic}{坐车}{zuo4che1}{7,4}{⼟,⾞}
  \definition{v.}{andar de carro, ônibus, trem, etc.}
\end{EntryWithPhonetic}

\begin{EntryWithPhonetic}{坐垫}{zuo4dian4}{7,9}{⼟,⼟}
  \definition[块]{s.}{assento (motocicleta) | almofada}
\end{EntryWithPhonetic}

\begin{EntryWithPhonetic}{坐好}{zuo4hao3}{7,6}{⼟,⼥}
  \definition{v.}{sentar-se corretamente | sentar direito}
\end{EntryWithPhonetic}

\begin{EntryWithPhonetic}{坐下}{zuo4xia4}{7,3}{⼟,⼀}[HSK 1]
  \definition{v.}{sentar-se; tomar um assento; passar da posição em pé para a posição sentada}
\end{EntryWithPhonetic}

\begin{EntryWithPhonetic}{坐享}{zuo4xiang3}{7,8}{⼟,⼇}
  \definition{v.}{curtir algo sem levantar um dedo}
\end{EntryWithPhonetic}

%%%%%%%%%% 座 %%%%%%%%%%
\subsection*{座}\addcontentsline{loh}{figure}{座 \dpy{zuo4}}

\begin{EntryWithPhonetic}{座}{zuo4}{10}{⼴}[HSK 2]
  \definition{clas.}{usado para montanhas, edifícios e objetos imóveis semelhantes}
  \definition{s.}{assento; lugar | suporte; pedestal; base | Astronomia: constelação | Arcaico: forma de tratamento a altos funcionários}
\end{EntryWithPhonetic}

\begin{EntryWithPhonetic}{座标}{zuo4biao1}{10,9}{⼴,⽊}
  \variantof{坐标}
\end{EntryWithPhonetic}

\begin{EntryWithPhonetic}{座谈会}{zuo4tan2hui4}{10,10,6}{⼴,⾔,⼈}[HSK 6]
  \definition{s.}{fórum; simpósio; discussão informal | conferência | sessão de rap}
\end{EntryWithPhonetic}

\begin{EntryWithPhonetic}{座位}{zuo4wei5}{10,7}{⼴,⼈}[HSK 2]
  \definition[个,排]{s.}{assento; lugar}
\end{EntryWithPhonetic}

\begin{EntryWithPhonetic}{座子}{zuo4zi5}{10,3}{⼴,⼦}
  \definition{s.}{suporte; pedestal; base | selim (de bicicleta, motocicleta, etc.)}
\end{EntryWithPhonetic}

%%%%%%%%%% 做 %%%%%%%%%%
\subsection*{做}\addcontentsline{loh}{figure}{做 \dpy{zuo4}}

\begin{EntryWithPhonetic}{做}{zuo4}{11}{⼈}[HSK 1]
  \definition{v.}{fabricar; produzir; criar | escrever; compor | fazer; trabalhar em; dedicar-se a; exercer uma determinada profissão ou atividade | realizar uma festa em família; comemorar | ser; tornar-se; agir como; atuar como | ser usado como | formar ou estabelecer um relacionamento; conectar-se (em algum tipo de relação) | fingir (alguma coisa) | cozinhar; preparar}
\end{EntryWithPhonetic}

\begin{EntryWithPhonetic}{做到}{zuo4 dao4}{11,8}{⼈,⼑}[HSK 2]
  \definition{v.}{alcançar; realizar; atingir um determinado objetivo; atingir um determinado padrão}
\end{EntryWithPhonetic}

\begin{EntryWithPhonetic}{做法}{zuo4fa5}{11,8}{⼈,⽔}[HSK 2]
  \definition[种,个]{s.}{método; maneira de fazer algo; métodos de lidar com coisas ou fazer coisas}
\end{EntryWithPhonetic}

\begin{EntryWithPhonetic}{做饭}{zuo4 fan4}{11,7}{⼈,⾷}[HSK 2]
  \definition{v.}{cozinhar; preparar uma refeição; cozinhar refeições e transformar alimentos crus em alimentos cozidos}
\end{EntryWithPhonetic}

\begin{EntryWithPhonetic}{做活}{zuo4huo2}{11,9}{⼈,⽔}
  \definition{v.}{trabalhar para ganhar a vida (especialmente de mulher costureira)}
\end{EntryWithPhonetic}

\begin{EntryWithPhonetic}{做客}{zuo4 ke4}{11,9}{⼈,⼧}[HSK 3]
  \definition{v.}{visitar; ser um convidado; ser hóspede}
\end{EntryWithPhonetic}

\begin{EntryWithPhonetic}{做梦}{zuo4 meng4}{11,11}{⼈,⼣}[HSK 4]
  \definition{s.}{sonho; ilusões e visões na consciência durante o sono}
  \definition{v.}{sonhar; ter um sonho | sonhar acordado, ter um sonho impossível; metáfora para fantasia irrealista}[别做梦了,她不会嫁给你的。===Pare de sonhar, ela não se casará com você.]
\end{EntryWithPhonetic}

\begin{EntryWithPhonetic}{做生活}{zuo4sheng1huo2}{11,5,9}{⼈,⽣,⽔}
  \definition{v.}{fazer tabalhos manuais}
\end{EntryWithPhonetic}

\begin{EntryWithPhonetic}{做戏}{zuo4xi4}{11,6}{⼈,⼽}
  \definition{v.}{atuar em uma peça | fazer uma peça}
\end{EntryWithPhonetic}

\begin{EntryWithPhonetic}{做眼}{zuo4yan3}{11,11}{⼈,⽬}
  \definition{v.}{agir como um guia | trabalhar como espião}
\end{EntryWithPhonetic}

\begin{EntryWithPhonetic}{做作}{zuo4zuo5}{11,7}{⼈,⼈}
  \definition{adj.}{afetado | artificial}
\end{EntryWithPhonetic}

%%%%%%%%%% 酢 %%%%%%%%%%
\subsection*{酢}\addcontentsline{loh}{figure}{酢 \dpy{zuo4}}

\begin{EntryWithPhonetic}{酢}{zuo4}{12}{⾣}
  \definition{s.}{brinde ao anfitrião feito pelo convidado}
\end{EntryWithPhonetic}

%%%%% EOF %%%%%


\end{DictionaryEntries} 

\pagestyle{plain}

\ifdraftdoc
%
%
%
\else

\clearpage
\chapter{Termos Gramaticais Chineses}
\input{include/termos.gramaticais.tex}

\clearpage
\chapter{Classificadores Nominais}
\input{include/classificadores.nominais.tex}

\clearpage
\chapter{Classificadores Verbais}
\input{include/classificadores.verbais.tex}

\clearpage
\chapter{Verbos Direcionais}
\input{include/verbos.direcionais.tex}

\clearpage
\chapter{Locativos}
%%%%%%%%%%%%%%%%%%%%%%%%%%%%%%%%%%%%%%%%%%%%%%%%%%%%%%%%%%%%%%%%%%%%%%%%%%%%%%%
%%%%%%%%%%%%%%%%%%%%%%%%%%%%%%%%%%%%%%%%%%%%%%%%%%%%%%%%%%%%%%%%%%%%%%%%%%%%%%%
%%%%%                                                                     %%%%%
%%%%% locativos.tex:                                                      %%%%%
%%%%% Tabela com os locativos chineses                                    %%%%%
%%%%%                                                                     %%%%%
%%%%%%%%%%%%%%%%%%%%%%%%%%%%%%%%%%%%%%%%%%%%%%%%%%%%%%%%%%%%%%%%%%%%%%%%%%%%%%%
%%%%%%%%%%%%%%%%%%%%%%%%%%%%%%%%%%%%%%%%%%%%%%%%%%%%%%%%%%%%%%%%%%%%%%%%%%%%%%%

%%% Ajustes para a tabela
\DefTblrTemplate{caption}{default}{}
\DefTblrTemplate{capcont}{default}{ \UseTblrTemplate{conthead-text}{default} }
\DefTblrTemplate{contfoot-text}{default}{Continua na próxima página.}
\DefTblrTemplate{conthead-text}{default}{(Continuação)}
\DefTblrTemplate{firsthead}{default}{ \UseTblrTemplate{caption}{default} }
\DefTblrTemplate{middlehead,lasthead}{default}{ \UseTblrTemplate{conthead}{default} }
\DefTblrTemplate{firstfoot,middlefoot}{default}{ \UseTblrTemplate{contfoot}{default} }
\DefTblrTemplate{lastfoot}{default}{ \UseTblrTemplate{note}{default} \UseTblrTemplate{remark}{default} }

%%% Tabela
\begin{longtblr}
{
 colspec = {cccccc},
 width = 1\linewidth,
 hlines = {white},
 vlines = {white},
 rowhead = 1, rowfoot = 0,
 row{1} = {font=\bfseries, bg=gray8, fg=black},
 column{1} = {font=\bfseries, bg=gray8, fg=black},
 cell{1}{1} = {bg=white},
 cell{2-Z}{2-Z} = {bg=gray9},
 cell{6}{5-6} = {bg=white},
 cell{7}{2-4} = {bg=white},
 cell{9}{2-5} = {bg=white},
 cell{10}{3-6} = {bg=white},
 cell{11}{2-5} = {bg=white},
 cell{12}{4-6} = {bg=white},
 cell{13}{4-6} = {bg=white},
}
                                           & {边\\   \normalsize\dpy{bian1}}        & {面\\   \normalsize\dpy{mian4}}        & {头\\   \normalsize\dpy{tou5}}        & {以\\   \normalsize\dpy{yi3}}        & {之\\   \normalsize\dpy{zhi1}}        \\
{上\\ \normalsize\dpy{shang4}\\ sobre}     & {上边\\ \normalsize\dpy{shang4 bian1}} & {上面\\ \normalsize\dpy{shang4 mian4}} & {上头\\ \normalsize\dpy{shang4 tou5}} & {以上\\ \normalsize\dpy{yi3 shang4}} & {之上\\ \normalsize\dpy{zhi1 shang4}} \\
{下\\ \normalsize\dpy{xia4}\\ sob}         & {下边\\ \normalsize\dpy{xia4 bian1}}   & {下面\\ \normalsize\dpy{xia4 mian4}}   & {下头\\ \normalsize\dpy{xia4 tou5}}   & {以下\\ \normalsize\dpy{yi3 xia4}}   & {之下\\ \normalsize\dpy{zhi1 xia4}}   \\
{前\\ \normalsize\dpy{qian2}\\ na frente}  & {前边\\ \normalsize\dpy{qian2 bian1}}  & {前面\\ \normalsize\dpy{qian2 mian4}}  & {前头\\ \normalsize\dpy{qian2 tou5}}  & {以前\\ \normalsize\dpy{yi3 qian2}}  & {之前\\ \normalsize\dpy{zhi1 qian2}}  \\
{后\\ \normalsize\dpy{hou4}\\ atrás}       & {后边\\ \normalsize\dpy{hou4 bian1}}   & {后面\\ \normalsize\dpy{hou4 mian4}}   & {后头\\ \normalsize\dpy{hou4 tou5}}   & {以后\\ \normalsize\dpy{yi3 hou4}}   & {之后\\ \normalsize\dpy{zhi1 hou4}}   \\
{里\\ \normalsize\dpy{li3}\\ dentro}       & {里边\\ \normalsize\dpy{li3 bian1}}    & {里面\\ \normalsize\dpy{li3 mian4}}    & {里头\\ \normalsize\dpy{li3 tou5}}    &                                      &                                       \\
{内\\ \normalsize\dpy{nei4}\\ no interior} &                                        &                                        &                                       & {以内\\ \normalsize\dpy{yi3 nei4}}   & {之内\\ \normalsize\dpy{zhi1 nei4}}   \\
{外\\ \normalsize\dpy{wai4}\\ no exterior} & {外边\\ \normalsize\dpy{wai4 bian1}}   & {外面\\ \normalsize\dpy{wai4 mian4}}   & {外头\\ \normalsize\dpy{wai4 tou5}}   & {以外\\ \normalsize\dpy{yi3 wai4}}   & {之外\\ \normalsize\dpy{zhi1 wai4}}   \\
{间\\ \normalsize\dpy{jian1}\\ entre}      &                                        &                                        &                                       &                                      & {之间\\ \normalsize\dpy{zhi1 jian1}}  \\
{旁\\ \normalsize\dpy{pang2}\\ ao lado}    & {旁边\\ \normalsize\dpy{pang2 bian1}}  &                                        &                                       &                                      &                                       \\
{中\\ \normalsize\dpy{zhong1}\\ no meio}   &                                        &                                        &                                       &                                      & {之中\\ \normalsize\dpy{zhi1 zhong1}} \\
{左\\ \normalsize\dpy{zuo3}\\ à esquerda}  & {左边\\ \normalsize\dpy{zuo3 bian1}}   & {左面\\ \normalsize\dpy{zuo3 mian4}}   &                                       &                                      &                                       \\
{右\\ \normalsize\dpy{you4}\\ à direita}   & {右边\\ \normalsize\dpy{you4 bian1}}   & {右面\\ \normalsize\dpy{you4 mian4}}   &                                       &                                      &                                       \\
\pagebreak
{东\\ \normalsize\dpy{dong1}\\ no leste}   & {东边\\ \normalsize\dpy{dong1 bian1}}  & {东面\\ \normalsize\dpy{dong1 mian4}}  & {东头\\ \normalsize\dpy{dong1 tou5}}  & {以东\\ \normalsize\dpy{yi3 dong1}}  & {之东\\ \normalsize\dpy{zhi1 dong1}}  \\
{南\\ \normalsize\dpy{nan2}\\ no sul}      & {南边\\ \normalsize\dpy{nan2 bian1}}   & {南面\\ \normalsize\dpy{nan2 mian4}}   & {南头\\ \normalsize\dpy{nan2 tou5}}   & {以南\\ \normalsize\dpy{yi3 nan2}}   & {之南\\ \normalsize\dpy{zhi1 nan2}}   \\
{西\\ \normalsize\dpy{xi1}\\ no oeste}     & {西边\\ \normalsize\dpy{xi1 bian1}}    & {西面\\ \normalsize\dpy{xi1 mian4}}    & {西头\\ \normalsize\dpy{xi1 tou5}}    & {以西\\ \normalsize\dpy{yi3 xi1}}    & {之西\\ \normalsize\dpy{zhi1 xi1}}    \\
{北\\ \normalsize\dpy{bei3}\\ n norte}     & {北边\\ \normalsize\dpy{bei3 bian1}}   & {北面\\ \normalsize\dpy{bei3 mian4}}   & {北头\\ \normalsize\dpy{bei3 tou5}}   & {以北\\ \normalsize\dpy{yi3 bei3}}   & {之北\\ \normalsize\dpy{zhi1 bei3}}   \\
\end{longtblr}

%%%%% EOF %%%%%


\clearpage
\chapter{Radicais Kangxi}
%%%%%%%%%%%%%%%%%%%%%%%%%%%%%%%%%%%%%%%%%%%%%%%%%%%%%%%%%%%%%%%%%%%%%%%%%%%%%%%
%%%%%%%%%%%%%%%%%%%%%%%%%%%%%%%%%%%%%%%%%%%%%%%%%%%%%%%%%%%%%%%%%%%%%%%%%%%%%%%
%%%%%                                                                     %%%%%
%%%%% radicais_kangxi.tex:                                                %%%%%
%%%%% Lista dos 214 radicais Kangxi utilizados nos caracteres chineses.   %%%%%
%%%%%                                                                     %%%%%
%%%%%%%%%%%%%%%%%%%%%%%%%%%%%%%%%%%%%%%%%%%%%%%%%%%%%%%%%%%%%%%%%%%%%%%%%%%%%%%
%%%%%%%%%%%%%%%%%%%%%%%%%%%%%%%%%%%%%%%%%%%%%%%%%%%%%%%%%%%%%%%%%%%%%%%%%%%%%%%

%%% Ajustes para a tabela
\DefTblrTemplate{caption}{default}{}
\DefTblrTemplate{capcont}{default}{ \UseTblrTemplate{conthead-text}{default} }
\DefTblrTemplate{contfoot-text}{default}{Continua na próxima página.}
\DefTblrTemplate{conthead-text}{default}{(Continuação)}
\DefTblrTemplate{firsthead}{default}{ \UseTblrTemplate{caption}{default} }
\DefTblrTemplate{middlehead,lasthead}{default}{ \UseTblrTemplate{conthead}{default} }
\DefTblrTemplate{firstfoot,middlefoot}{default}{ \UseTblrTemplate{contfoot}{default} }
\DefTblrTemplate{lastfoot}{default}{ \UseTblrTemplate{note}{default} \UseTblrTemplate{remark}{default} }

%%% Tabela
\begin{longtblr}
{
  colspec = {|r|ll|l|l|}, hlines,
  width = 1\linewidth,
  rowhead = 1, rowfoot = 0,
  row{1} = {font=\bfseries, fg=white, bg=black},
  row{2-Z} = {font=\normalfont},
}
\textbf{Nº} & \SetCell[c=2]{c}\textbf{Radical e\\Variantes} & 2-2 & \textbf{Tradução} & \textbf{Pinyin} \\
1 & 一 & & um & \dictpinyin{yi1} \\
2 & 丨 & & linha & \dictpinyin{shu4} \\
3 & 丶 & & ponto, indica um fim & \dictpinyin{dian3} \\
4 & 丿 & 乀,乁 & cortar, dobrar & \dictpinyin{pie3} \\
5 & 乙 & 乚、乛、⺄ & segundo, anzol & \dictpinyin{yi3} \\
6 & 亅 & & gancho & \dictpinyin{gou1} \\
7 & 二 & & dois & \dictpinyin{er4} \\
8 & 亠 & & tampa & \dictpinyin{tou2} \\
9 & 人 & 亻、𠆢 & pessoa & \dictpinyin{ren2} \\
10 & 儿 & & pernas & \dictpinyin{er2} \\
11 & 入 & & entrar, juntar-se & \dictpinyin{ru4} \\
12 & 八 & 丷 & oito & \dictpinyin{ba1} \\
13 & 冂 & & largo, exterior & \dictpinyin{jiong3} \\
14 & 冖 & & capa de pano & \dictpinyin{mi4} \\
15 & 冫 & & gelo & \dictpinyin{bing1} \\
16 & 几 & & mesa pequena & \dictpinyin{ji1},\dictpinyin{ji3} \\
17 & 凵 & & receptáculo, caixa aberta & \dictpinyin{qu3} \\
18 & 刀 & 刂、⺈ & faca & \dictpinyin{dao1} \\
19 & 力 & & poder, força & \dictpinyin{li4} \\
20 & 勹 & & invólucro & \dictpinyin{bao1} \\
21 & 匕 & & colher & \dictpinyin{bi3} \\
22 & 匚 & & caixa & \dictpinyin{fang1} \\
23 & 匸 & & compartimento oculto & \dictpinyin{xi3} \\
24 & 十 & & dez, completo, perfeito & \dictpinyin{shi2} \\
25 & 卜 & & advinhação, divinação & \dictpinyin{bu3} \\
26 & 卩 & 㔾 & foca & \dictpinyin{jie2} \\
27 & 厂 & & penhasco, precipício & \dictpinyin{han4} \\
28 & 厶 & & privado & \dictpinyin{si1} \\
29 & 又 & & mão direita, e, novamente & \dictpinyin{you4} \\
30 & 口 & & boca & \dictpinyin{kou3} \\
31 & 囗 & & compartimento, recinto & \dictpinyin{wei2} \\
32 & 土 & & terra & \dictpinyin{tu3} \\
33 & 士 & & acadêmico, bacharel & \dictpinyin{shi4} \\
34 & 夂 & & ir & \dictpinyin{zhi1} \\
35 & 夊 & & ir devagar & \dictpinyin{sui1} \\
36 & 夕 & & tarde, pôr do sol & \dictpinyin{xi1} \\
37 & 大 & & grande, muito & \dictpinyin{da4} \\
38 & 女 & & mulher, fêmea & \dictpinyin{nv3} \\
39 & 子 & & criança, semente & \dictpinyin{zi3} \\
40 & 宀 & & teto, telhado & \dictpinyin{mian2} \\
41 & 寸 & & polegar, polegada & \dictpinyin{cun4} \\
42 & 小 & ⺌、⺍ & pequeno, insignificante & \dictpinyin{xiao3} \\
43 & 尢 & 尣 & manco, coxo & \dictpinyin{you2} \\
44 & 尸 & & cadáver & \dictpinyin{shi1} \\
45 & 屮 & & brotar, germinar & \dictpinyin{che4} \\
46 & 山 & & montanha & \dictpinyin{shan1} \\
47 & 巛 & 川 & rio & \dictpinyin{chuan1} \\
48 & 工 & & trabalho & \dictpinyin{gong1} \\
49 & 己 & ⺒ & próprio, a si mesmo & \dictpinyin{ji3} \\
50 & 巾 & & turbante, cachecol & \dictpinyin{jin1} \\
51 & 干 & & oposto, seco & \dictpinyin{gan1} \\
52 & 幺 & 么 & baixo, minúsculo & \dictpinyin{yao1} \\
53 & 广 & & casa em um penhasco & \dictpinyin{guang3} \\
54 & 廴 & & passada longa & \dictpinyin{yin3} \\
55 & 廾 & & duas mãos, vinte, arco & \dictpinyin{gong3} \\
56 & 弋 & & tiro, flecha & \dictpinyin{yi4} \\
57 & 弓 & & arco & \dictpinyin{gong1} \\
58 & 彐 & 彑 & focinho de porco & \dictpinyin{ji4} \\
59 & 彡 & & cerda, barba & \dictpinyin{shan1} \\
60 & 彳 & & passo & \dictpinyin{chi4} \\
61 & 心 & 忄、⺗ & coração, mente & \dictpinyin{xin1} \\
62 & 戈 & & lança & \dictpinyin{ge1} \\
63 & 戶 & 户、戸 & porta, casa & \dictpinyin{hu4} \\
64 & 手 & 扌、龵 & mão & \dictpinyin{shou3} \\
65 & 支 & & ramo & \dictpinyin{zhi1} \\
66 & 攴 & 攵 & tocar, bater levemente & \dictpinyin{pu1} \\
67 & 文 & & escrita, literatura & \dictpinyin{wen2} \\
68 & 斗 & & objeto em forma de concha & \dictpinyin{dou3} \\
69 & 斤 & & machado & \dictpinyin{jin1} \\
70 & 方 & & quadrado & \dictpinyin{fang1} \\
71 & 无 & 旡 & não, nada, negativo & \dictpinyin{wu2} \\
72 & 日 & & sol, dia & \dictpinyin{ri4} \\
73 & 曰 & & dizer, falar & \dictpinyin{yue1} \\
74 & 月 & & lua, mês & \dictpinyin{yue4} \\
75 & 木 & & árvore & \dictpinyin{mu4} \\
76 & 欠 & & falta, não ter, hiato & \dictpinyin{qian4} \\
77 & 止 & & parar & \dictpinyin{zhi3} \\
78 & 歹 & 歺 & morte, decadência & \dictpinyin{dai3} \\
79 & 殳 & & arma, lança & \dictpinyin{shu1} \\
80 & 毋 & 母 & mãe, não faça & \dictpinyin{mu3} \\
81 & 比 & & comparar, competir & \dictpinyin{bi3} \\
82 & 毛 & & pelagem & \dictpinyin{mao2} \\
83 & 氏 & & clã, linhagem & \dictpinyin{shi4} \\
84 & 气 & & ar, vapor, respiração & \dictpinyin{qi4} \\
85 & 水 & 氵、氺 & água & \dictpinyin{shui3} \\
86 & 火 & 灬 & fogo & \dictpinyin{huo3} \\
87 & 爪 & 爫 & garra, unha & \dictpinyin{zhao3} \\
88 & 父 & & pai, luz & \dictpinyin{fu4} \\
89 & 爻 & & duplo x, trigramas & \dictpinyin{yao2} \\
90 & 爿 & 丬 & metade de um tronco, madeira rachada & \dictpinyin{pan2} \\
91 & 片 & & fatia, filme & \dictpinyin{pian4} \\
92 & 牙 & & dente, presa & \dictpinyin{ya2} \\
93 & 牛 & 牜、⺧ & boi, vaca & \dictpinyin{niu2} \\
94 & 犬 & 犭 & cão & \dictpinyin{quan3} \\
95 & 玄 & & escuro, profundo & \dictpinyin{xuan2} \\
96 & 玉 & 王、玊 & jade & \dictpinyin{yu4} \\
97 & 瓜 & & melão & \dictpinyin{gua1} \\
98 & 瓦 & & telha & \dictpinyin{wa3} \\
99 & 甘 & & doce & \dictpinyin{gan1} \\
100 & 生 & & vida & \dictpinyin{sheng1} \\
101 & 用 & & usar & \dictpinyin{yong4} \\
102 & 田 & & campo, arrozal & \dictpinyin{tian2} \\
103 & 疋 & ⺪& pedaço de pano & \dictpinyin{pi3} \\
104 & 疒 & & doença & \dictpinyin{ne4} \\
105 & 癶 & & pegadas, pernas & \dictpinyin{bo1} \\
106 & 白 & & branco & \dictpinyin{bai2} \\
107 & 皮 & & pele, couro & \dictpinyin{pi2} \\
108 & 皿 & & prato & \dictpinyin{min3} \\
109 & 目 & ⺫ & olho & \dictpinyin{mu4} \\
110 & 矛 & & lança & \dictpinyin{mao2} \\
111 & 矢 & & seta, flecha & \dictpinyin{shi3} \\
112 & 石 & & pedra & \dictpinyin{shi2} \\
113 & 示 & 礻& espírito, ancestral, veneração & \dictpinyin{shi4} \\
114 & 禸 & & trilha & \dictpinyin{rou2} \\
115 & 禾 & & grão & \dictpinyin{he2} \\
116 & 穴 & & caverna & \dictpinyin{xue2} \\
117 & 立 & & ficar em pé, ereto & \dictpinyin{li4} \\
118 & 竹 & ⺮ & bambu & \dictpinyin{zhu2} \\
119 & 米 & & arroz & \dictpinyin{mi3} \\
120 & 糸 & 纟、糹 & seda & \dictpinyin{mi4} \\
121 & 缶 & & pote, jarra & \dictpinyin{fou3} \\
122 & 网 & ⺲、罓、⺳ & rede & \dictpinyin{wang3} \\
123 & 羊 & ⺶、⺷ & ovelha, cabra & \dictpinyin{yang2} \\
124 & 羽 & & pena & \dictpinyin{yu3} \\
125 & 老 & 耂 & velho & \dictpinyin{lao3} \\
126 & 而 & & e, mas & \dictpinyin{er2} \\
127 & 耒 & & arado & \dictpinyin{lei3} \\
128 & 耳 & & orelha & \dictpinyin{er3} \\
129 & 聿 & ⺺、⺻ & escova & \dictpinyin{yu4} \\
130 & 肉 & 月、⺼ & carne & \dictpinyin{rou4} \\
131 & 臣 & & ministro, oficial & \dictpinyin{chen2} \\
132 & 自 & & próprio, auto-- & \dictpinyin{zi4} \\
133 & 至 & & chegar & \dictpinyin{zhi4} \\
134 & 臼 & & argamassa, ligação & \dictpinyin{jiu4} \\
135 & 舌 & & língua & \dictpinyin{she2} \\
136 & 舛 & & opor & \dictpinyin{chuan3} \\
137 & 舟 & & barco & \dictpinyin{zhou1} \\
138 & 艮 & & parada, quietude & \dictpinyin{gen3} \\
139 & 色 & & cor, forma & \dictpinyin{se4} \\
140 & 艸 & ⺿ & grama & \dictpinyin{cao3} \\
141 & 虍 & & tigre & \dictpinyin{hu1} \\
142 & 虫 & & inseto, verme & \dictpinyin{chong2} \\
143 & 血 & & sangue & \dictpinyin{xue4} \\
144 & 行 & & andar, ir, fazer & \dictpinyin{xing2} \\
145 & 衣 & ⻂& roupa & \dictpinyin{yi1} \\
146 & 襾 & 西、覀 & capa, oeste & \dictpinyin{ya4} \\
147 & 見 & 见 & ver & \dictpinyin{jian4} \\
148 & 角 & ⻆、⻇ & chifre & \dictpinyin{jiao3} \\
149 & 言 & 讠、訁 & palavra, linguagem & \dictpinyin{yan2} \\
150 & 谷 & & vale & \dictpinyin{gu3} \\
151 & 豆 & & feijão, fava & \dictpinyin{dou4} \\
152 & 豕 & & porco & \dictpinyin{shi3} \\
153 & 豸 & & texugo, inseto sem pernas & \dictpinyin{zhi4} \\
154 & 貝 & 贝 & concha & \dictpinyin{bei4} \\
155 & 赤 & & vermelho, nu & \dictpinyin{chi4} \\
156 & 走 & & correr & \dictpinyin{zou3} \\
157 & 足 & ⻊& pé & \dictpinyin{zu2} \\
158 & 身 & & corpo & \dictpinyin{shen1} \\
159 & 車 & 车 & carroça, carro & \dictpinyin{che1} \\
160 & 辛 & & amargo & \dictpinyin{xin1} \\
161 & 辰 & & manhã & \dictpinyin{chen2} \\
162 & 辵 & ⻌、⻍、⻎ & caminhar & \dictpinyin{chuo4} \\
163 & 邑 & ⻏ & cidade & \dictpinyin{yi4} \\
164 & 酉 & & vinho, álcool & \dictpinyin{you3} \\
165 & 釆 & & distinto & \dictpinyin{bian4} \\
166 & 里 & & aldeia, vila & \dictpinyin{li3} \\
167 & 金 & 钅、釒 & ouro, metal & \dictpinyin{jin1} \\
168 & 長 & 长、镸 & longo, crescer & \dictpinyin{zhang3} \\
169 & 門 & 门 & portão, porta & \dictpinyin{men2} \\
170 & 阜 & ⻖ & monte, barragem & \dictpinyin{fu4} \\
171 & 隶 & & escravo & \dictpinyin{li4} \\
172 & 隹 & & pássaro de cauda curta & \dictpinyin{zhui1} \\
173 & 雨 & & chuva & \dictpinyin{yu3} \\
174 & 靑 & 青 & azul, verde ou preto & \dictpinyin{qing1} \\
175 & 非 & & errado & \dictpinyin{fei1} \\
176 & 面 & 靣 & face & \dictpinyin{mian4} \\
177 & 革 & & couro, couro cru & \dictpinyin{ge2} \\
178 & 韋 & 韦 & couro tingido & \dictpinyin{wei2} \\
179 & 韭 & & alho-poró & \dictpinyin{jiu3} \\
180 & 音 & & som & \dictpinyin{yin1} \\
181 & 頁 & 页 & folha, página & \dictpinyin{ye4} \\
182 & 風 & 风 & vento & \dictpinyin{feng1} \\
183 & 飛 & 飞 & voar & \dictpinyin{fei1} \\
184 & 食 & 饣、飠 & alimento, comer & \dictpinyin{shi2} \\
185 & 首 & & cabeça & \dictpinyin{shou3} \\
186 & 香 & & perfume, aroma & \dictpinyin{xiang1} \\
187 & 馬 & 马 & cavalo & \dictpinyin{ma3} \\
188 & 骨 & ⻣ & osso & \dictpinyin{gu3} \\
189 & 高 & 髙 & alto & \dictpinyin{gao1} \\
190 & 髟 & & cabelo & \dictpinyin{biao1} \\
191 & 鬥 & & luta & \dictpinyin{dou4} \\
192 & 鬯 & & vinho sacrificial & \dictpinyin{chang4} \\
193 & 鬲 & & caldeirão, tripé & \dictpinyin{ge2} \\
194 & 鬼 & & fantasma, demônio & \dictpinyin{gui3} \\
195 & 魚 & 鱼 & peixe & \dictpinyin{yu2} \\
196 & 鳥 & 鸟 & pássaro & \dictpinyin{niao3} \\
197 & 鹵 & 卤 & sal & \dictpinyin{lu3} \\
198 & 鹿 & & corça, veado & \dictpinyin{lu4} \\
199 & 麥 & 麦 & trigo & \dictpinyin{mai4} \\
200 & 麻 & & cânhamo, linho & \dictpinyin{ma2} \\
201 & 黃 & 黄 & amarelo & \dictpinyin{huang4} \\
202 & 黍 & & milhete, painço & \dictpinyin{shu3} \\
203 & 黑 & & preto & \dictpinyin{hei1} \\
204 & 黹 & & bordado & \dictpinyin{zhi3} \\
205 & 黽 & 黾 & sapo, anfíbio & \dictpinyin{mian3} \\
206 & 鼎 & & tripé de sacrifício, caldeirão de três pernas & \dictpinyin{ding3} \\
207 & 鼓 & & tambor & \dictpinyin{gu3} \\
208 & 鼠 & 鼡 & rato, camundongo & \dictpinyin{shu3} \\
209 & 鼻 & & nariz & \dictpinyin{bi2} \\
210 & 齊 & 齐、斉 & mesmo, uniformemente & \dictpinyin{qi2} \\
211 & 齒 & 齿 & dente & \dictpinyin{chi3} \\
212 & 龍 & 龙 & dragão & \dictpinyin{long2} \\
213 & 龜 & 龟 & tartaruga & \dictpinyin{gui1} \\
214 & 龠 & & flauta & \dictpinyin{yue4} \\
\end{longtblr}

%%%%% EOF %%%%%


\fi

\clearpage
\chapter{Lista de Hanzis (somente o primeiro caracter)}
\begin{multicols}{5}
 \begin{KeepFromToc}
  \listoffirsthanzis
 \end{KeepFromToc}
\end{multicols}

\end{document}

%%%%% EOF %%%%
