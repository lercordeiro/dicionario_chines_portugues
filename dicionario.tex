
%%%%%%%%%%%%%%%%%%%%%%%%%%%%%%%%%%%%%%%%%
% XeTeX
%
% Dicionário Chinês -> Português
% Autor: Luiz Eduardo Roncato Cordeiro
%
% Licença:
% CC BY-NC-SA 3.0 (http://creativecommons.org/licenses/by-nc-sa/3.0/)
%%%%%%%%%%%%%%%%%%%%%%%%%%%%%%%%%%%%%%%%%

\documentclass[a4paper,10pt,twoside,openany]{memoir}

\usepackage[dvipsnames]{xcolor}
\usepackage[brazil]{babel}
\usepackage{fontspec}
\usepackage{xeCJK}
\usepackage{xpinyin}
\usepackage{xunicode}
\usepackage{xltxtra}
\usepackage{multicol}
\usepackage{fancyhdr}
\usepackage{imakeidx}
\usepackage{ifthen}
\usepackage{xparse}
\usepackage[inline]{enumitem}
\usepackage{zhnumber}
\usepackage{wasysym}
\usepackage[explicit]{titlesec}
\usepackage{tikz}
\usepackage[hyperindex,hidelinks]{hyperref}
\usepackage{listofitems}
\usepackage{numprint}
\usepackage{pifont}
\usepackage{xstring}
\usepackage{xpatch}
\usepackage{tabularray}

% Meus Comandos
%
% Dictionary environment and functions
%
\newbool{veja}
\newbool{exemplo}
\newcommand{\sectionbreak}{\phantomsection}
\renewcommand\stacktype{S}
\renewcommand\stackalignment{l}
\setstackgap{S}{2.5pt}

\DeclareRobustCommand{\&}%
{
    \ifdim\fontdimen1\font>0pt
        \textsl{\symbol{`\&}}%
    \else
        \symbol{`\&}%
    \fi
}

\NewDocumentCommand{\dictpinyin}{m}{\guillemotleft\pinyin{#1}\guillemotright} 

\NewDocumentCommand{\dpy}{m}%
{%
    \StrSubstitute{#1}{r5}{r}[\rA]%
    \StrSubstitute{\rA}{t5}{t}[\rZ]%
    \edef\py{\dictpinyin{\rZ}}%
    \mbox{}\py
}

\NewDocumentCommand{\dictenumerate}{>{\SplitList{;}}m}
{%
    \begin{enumerate*}[left=0pt,mode=unboxed,itemjoin={{; }}]
        \ProcessList{#1}{\insertitem}
    \end{enumerate*}
}
\NewDocumentCommand{\insertitem}{m}{\item #1}

% Lista Veja
\newenvironment{listaveja}%
{\list{}% empty label
    {
        \setlength{\topsep}{0ex}
        \setlength{\itemsep}{0ex}
        \setlength{\leftmargin}{0ex}
        \setlength{\labelsep}{0ex}
        \setlength{\parsep}{0pt}
        \setlength{\partopsep}{0pt}
        \setlength{\rightmargin}{0ex}
        \setlength{\listparindent}{0em}
        \setlength{\itemindent}{0ex}
        \setlength{\labelwidth}{0ex}
    }%
}%
{\endlist}
\newcommand{\vejaitem}[2]{\item[\addstackgap{\stackunder{#1}{\tiny\dpy{#2}}}]}

\ExplSyntaxOn

\newcommand\vejalst{}
\newcommand\exemplolst{}

\listadd{\vejalst}{}% Initialize list
\listadd{\exemplolst}{}% Initialize list

\NewDocumentEnvironment{verbete}{mO{}mO{}mo}%
{
    \leavevmode%
    \markboth{#1{\tiny\dpy{#3}}}{#1{\tiny\dpy{#3}}}
    \tl_set:Nn \l_hanzi_tl {#1}
    \tl_set:Nn \l_pinyin_tl {#3}
    \tl_set:Nn \l_strokes_tl {#5}
    \boolfalse{veja} \renewcommand\vejalst{} \listadd{\vejalst}{}% Initialize list
    \boolfalse{exemplo} \renewcommand\exemplolst{} \listadd{\exemplolst}{}% Initialize list
    \begin{minipage}[t][][t]{.485\textwidth}
        \baselineskip=1.1\baselineskip
        \label{#1:#3}
        \index[pinyin]{#1@\dpy{#3} #1}
        \index[stroke]{#1@#1 \dpy{#3}}
        \index[radical]{#1@#1 \dpy{#3}}
        \begin{flushleft}
            \begin{tcolorbox}[colframe=black,colback=white,boxrule=1pt,left=0mm,right=0mm,top=0mm,bottom=0mm]
                \textbf{\Large#1}\hfill\textsuperscript{\tiny(#5画)}\\
                {\footnotesize#2\dpy{#3}#4}\IfValueT{#6}{\hfill{\tiny[Kangxi\ #6]}}
            \end{tcolorbox}
}{%
    \ifbool{exemplo}%
    {% Há exemplos
        \renewcommand*{\do}[1]%
        {%
            \IfSubStr{##1}{::::}%
            {% Com tradução
                \StrCut{##1}{::::}{\sE}{\sT}%
                {\small\addstackgap{\stackunder{\StrSubstitute{\sE}{\l_hanzi_tl}{\CJKunderdot{\l_hanzi_tl}}}{\scriptsize(\sT)}}}\\%
            }%
            {%
                \IfSubStr{##1}{ERRO}{##1}%
                {% Sem tradução
                    {\small\StrSubstitute{##1}{\l_hanzi_tl}{\CJKunderdot{\l_hanzi_tl}}}\\%
                }%
            }%
        }%
            \par\vspace{1ex}
            \ding{46}\ \textit{Exemplos:}\linebreak
            \dolistloop{\exemplolst}
    }{}% DNGN
    \ifbool{veja}%
    {% Há 'vejas'
        \renewcommand*{\do}[1]%
        {%
            \IfBeginWith{##1}{eg:}
            {%
                \StrBehind{##1}{eg:}[\sHP]%
                \StrCut{\sHP}{:}{\sH}{\sP}%
                \vejaitem{\sH}{\sP}\mbox{}\linebreak%
                \mbox{}\dotfill(pág.~\pageref{\sHP})%
            }{%
                \StrCut{##1}{:}\sH\sP %
                \vejaitem{\sH}{\sP}\dotfill(pág.~\pageref{##1})%
            }%
        }
            \par\vspace{1ex}
            \ding{43}\ \textit{Veja\ também}:%
            \begin{listaveja}
                \dolistloop{\vejalst}
            \end{listaveja}
    }{}% DNGN
            \mbox{}\vspace{1ex}
        \end{flushleft}
    \end{minipage}\par
}

\NewDocumentEnvironment{verbete*}{mO{}mO{}mo}
{%
    \leavevmode%
    \markboth{#1{\tiny\dpy{#3}}}{#1{\tiny\dpy{#3}}}
    \tl_set:Nn \l_hanzi_tl {#1}
    \tl_set:Nn \l_pinyin_tl {#3}
    \tl_set:Nn \l_strokes_tl {#5}
    \boolfalse{veja}
    \renewcommand\vejalst{}
    \listadd{\vejalst}{}% Initialize list
    \boolfalse{exemplo}
    \renewcommand\exemplolst{}
    \begin{minipage}[t][][t]{.485\textwidth}
        \baselineskip=1.1\baselineskip
        \label{#1:#3}
        \index[pinyin]{#1@\dpy{#3}\ #1}
        \index[stroke]{#1@#1 \dpy{#3}}
        \index[radical]{#1@#1 \dpy{#3}}
        \begin{flushleft}
            \begin{tcolorbox}[colframe=black,colback=white,boxrule=1pt,left=0mm,right=0mm,top=0mm,bottom=0mm]
                \mbox{}\hfill\textsuperscript{\tiny(#5画)}\\
                \textbf{\Large#1}\\%
                {\footnotesize#2\dpy{#3}#4}%
                \IfValueT{#6}{\mbox{}\hfill{\tiny[Kangxi\ #6]}}
            \end{tcolorbox}
}{%
    \ifbool{exemplo}
    {%
        \renewcommand*{\do}[1]%
        {%
            \IfSubStr{##1}{::::}%
            {% Com tradução
                \StrCut{##1}{::::}{\sE}{\sT}%
                {\small\addstackgap{\stackunder{\StrSubstitute{\sE}{\l_hanzi_tl}{\CJKunderdot{\l_hanzi_tl}}}{\scriptsize(\sT)}}}\\%
            }{%
                \IfSubStr{##1}{ERRO}{##1}%
                {% Sem tradução
                    {\small\StrSubstitute{##1}{\l_hanzi_tl}{\CJKunderdot{\l_hanzi_tl}}}\\%
                }
            }
        }
            \par\vspace{1ex}
            \ding{46}\ \textit{Exemplos:}\linebreak
            \dolistloop{\exemplolst}
        }{}% DNGN
    \ifbool{veja}
    {%
        \renewcommand*{\do}[1]%
        {%
            \IfBeginWith{##1}{eg:}%
            {%
                \StrBehind{##1}{eg:}[\sHP]%
                \StrCut{\sHP}{:}{\sH}{\sP}%
                \vejaitem{\sH}{\sP}\mbox{}\linebreak%
                \mbox{}\dotfill(pág.~\pageref{\sHP})%
            }{%
                \StrCut{##1}{:}\sH\sP%
                \vejaitem{\sH}{\sP}\dotfill(pág.~\pageref{##1})%
            }%
        }
            \par\vspace{1ex}
            \ding{43}\ \textit{Veja\ também}:\\%
            \begin{listaveja}
                \dolistloop{\vejalst}
            \end{listaveja}
    }{}% DNGN
            \mbox{}\vspace{1ex}
        \end{flushleft}
    \end{minipage}\par
}

\NewDocumentCommand{\significado}{somm}%
{%
    \IfBooleanTF{#1}%
    {% Proper Name
        (\textit{Substantivo\ Próprio})\IfValueT{#2}{\ [p.c.:~#2]}{\ \dictenumerate{#4}}\\%
    }{%
        (\textit{#3})\IfValueT{#2}{\ [p.c.: #2]}{\ \dictenumerate{#4}}\\%
    }%
}

\NewDocumentCommand{\variante}{smm}%
{%
    \IfBooleanTF{#1}%
    {% Palavra Grande
        \ding{43}\ Variante\ de\ \addstackgap{\stackunder{#2}{\tiny\dpy{#3}}}\\%
        \hspace*{3ex}(pág.~\pageref{#2:#3})\\%
    }{%
        \ding{43}\ Variante\ de\ \addstackgap{\stackunder{#2}{\tiny\dpy{#3}}}\ (pág.~\pageref{#2:#3})\\%
    }%
}

\NewDocumentCommand{\exemplo}{mo}%
{%
    \booltrue{exemplo}
    \IfSubStr[1]{#1}{\l_hanzi_tl}%
    {%
        \IfValueTF{#2}%
        {% Com tradução
            \listgadd{\exemplolst}{#1::::#2}%
        }{% Sem tradução
            \listgadd{\exemplolst}{#1}%
        }%
    }{% Sem a palavra (hanzi)
       \listgadd{\exemplolst}{\textit{ERRO}: Não há \l_hanzi_tl\ em ``#1''.}%
    }%
}

\NewDocumentCommand{\veja}{smm}%
{%
    \booltrue{veja}
    \IfBooleanTF{#1}
    {% Veja extra grande
        \listgadd{\vejalst}{eg:#2:#3}%
    }{%
        \listgadd{\vejalst}{#2:#3}%
    }%
}

\ExplSyntaxOff



% Ajuste das fontes... No Tofu do Google
\setCJKmainfont{Noto Serif CJK SC ExtraLight}
\setCJKsansfont{Noto Sans CJK SC}
\setCJKmonofont{Noto Sans Mono CJK SC}

% Ajustes do PDF
\hypersetup{
  linktoc=page,
  colorlinks=true,
  urlcolor=green,
  linkcolor=blue,
  citecolor=blue,
  pdftitle={汉葡词典 - Dicionário Chinês-Português},
  pdfsubject={Dicionário Chinês-Português para o Curso de Chinês do Instituto Confúcio},
  pdfauthor={Luiz Eduardo Roncato Cordeiro AKA 罗学凯},
  pdfkeywords={dicionário, chinês, português, instituto confúcio}
}

% Sumário
%\renewcommand\cftchapteraftersnumb{\normalfont}
\renewcommand\cftbeforechapterskip{5pt plus 1pt}

% Índices Remissivos
\makeindex[title=Índice Remissivo por Pinyin,intoc=true,columns=3,columnsep=15pt,columnseprule=true,noautomatic=true,name=pinyin]
\makeindex[title=Índice Remissivo por Traço,intoc=true,columns=3,columnsep=15pt,columnseprule=true,noautomatic=true,name=stroke]
\makeindex[title=Índice Remissivo por Radical,intoc=true,columns=3,columnsep=15pt,columnseprule=true,noautomatic=true,name=radical]
\indexsetup{level=\chapter*,toclevel=chapter,headers={\indexname}{\indexname}}
\indexprologue{Para diferenciação, os nomes próprios estão em cor diferente.}

% Ajustes de espaçamento
\setlength{\parindent}{0cm}
\setlength{\parskip}{-1.5mm}
\setlength{\columnsep}{.5em}
\setlength{\columnseprule}{0.2mm}

% Ajustes do gerador de pinyin
\xpinyinsetup{ratio={.6},vsep={1em},pysep={\hspace{.2ex}}}

% Headers & Footers
\fancyhead[L]{\rightmark} % Top left header
\fancyhead[R]{\leftmark}  % Top right header
\renewcommand{\headrulewidth}{1pt} % Rule under the header
\fancyfoot[LE,RO]{\thepage} % Bottom left footer
\fancyfoot[C]{汉葡词典} % Bottom center footer
\renewcommand{\footrulewidth}{1pt} % Rule under the header
\setlength{\headheight}{16pt}
\addtolength{\topmargin}{-0.5pt}
\fancypagestyle{plain}{% 
  \fancyhead{} % clear all header and footer fields
  \fancyfoot[LE,RO]{\thepage}
  \fancyfoot[C]{汉葡词典}
}
\pagestyle{fancy} % Use the custom headers and footers throughout the document

% Seções têm uma caixa arredondada em volta do nome, para o A-Z dos pinyin.
\titlespacing{\section}{0pc}{2ex}{1ex}
\titleformat{\section}[hang]{\Large\bfseries}{}{0pt}{%
  \tikz{%
    \node[draw,rounded corners=1mm,text depth=0.2ex,line width=1pt, anchor=west]{\Large\bfseries#1};
  }%
}

% Até agora o melhor estilo para capítulos
\chapterstyle{verville}

%%%
%%% Documento começa aqui!
%%%

\begin{document}

\begin{titlingpage}
  \raggedleft
  \rule{1pt}{\textheight}
  \hspace{0.05\textwidth}
  \parbox[b]{0.75\textwidth}{
    {\Huge\bfseries 汉葡词典}\\[2\baselineskip] % Title
    {\large\textsc{Dicionário Chinês-Português para o\\%
                   Curso de Chinês do Instituto Confúcio}}\\%
    [4\baselineskip]
    {\Large\textsc{罗学凯}\\%
      \tiny Luiz Eduardo Roncato Cordeiro\\%
            Aluno do Instituto Confúcio da UNESP}

      \vspace{0.5\textheight}

      {\noindent \zhtoday}\\[\baselineskip]
    }
\end{titlingpage}

\newpage
\tableofcontents

\newpage
\chapter{汉葡词典}

%%%%%%%%%%%%%%%%%%%%%%%%
%
% Estou ordenando as palavras em ordem alfabética por pinyin, 
% depois por número de traços e, finalmente, por UTF-8 do hanzi.
%
% https://en.wikipedia.org/wiki/Chinese_character_orders
%
%%%%%%%%%%%%%%%%%%%%%%%%

\newpage
\begin{multicols}{2}
%%%
%%% A
%%%
%\section*{A}
\addcontentsline{toc}{section}{A}

\begin{verbete}{阿}{a1}{7}[Radical 阜]
  \significado{pref.}{utilizado para indicar familiaridade antes de nomes monossilábicos, termos de parentesco, etc.}
  \veja{阿}{e1}
\end{verbete}

\begin{verbete}{阿哥}{a1ge1}{7,10}
  \significado{s.}{irmão mais velho (familiar)}
\end{verbete}

\begin{verbete}{呵}{a1}{8}[Radical 口]
  \variante{啊}
  \veja{呵}{he1}
\end{verbete}

\begin{verbete}{啊}{a1}{10}[Radical 口]
  \significado{interj.}{Ah!; Oh! | interjeição de surpresa}
  \veja{啊}{a2}
  \veja{啊}{a3}
  \veja{啊}{a4}
  \veja{啊}{a5}
\end{verbete}

\begin{verbete}{啊呀}{a1ya1}{10,7}
  \significado{interj.}{Oh meu Deus! | interjeição de surpresa}
\end{verbete}

\begin{verbete}{啊哟}{a1yo5}{10,9}
  \significado{interj.}{Meu Deus!; Oh!; Ai! | interjeição de surpresa ou dor}
\end{verbete}

\begin{verbete}{啊}{a2}{10}[Radical 口]
  \significado{interj.}{Eh?; Que? | interjeição expressando dúvida ou exigindo resposta}
  \veja{啊}{a1}
  \veja{啊}{a3}
  \veja{啊}{a4}
  \veja{啊}{a5}
\end{verbete}

\begin{verbete}{嗄}{a2}{13}[Radical 口]
  \significado{adj.}{rouco}
  \variante{啊}
\end{verbete}

\begin{verbete}{啊}{a3}{10}[Radical 口]
  \significado{interj.}{Eh?; Meu!; E aí?; Que? | interjeição de surpresa ou dúvida}
  \veja{啊}{a1}
  \veja{啊}{a2}
  \veja{啊}{a4}
  \veja{啊}{a5}
\end{verbete}

\begin{verbete}{啊}{a4}{10}[Radical 口]
  \significado{interj.}{Ah!, OK!; Oh, é você!; Hum! | expressão de reconhecimento | interjeição de acordo}
  \veja{啊}{a1}
  \veja{啊}{a2}
  \veja{啊}{a3}
  \veja{啊}{a5}
\end{verbete}

\begin{verbete}{啊}{a5}{10}[Radical 口]
  \significado{adv.}{assim por diante}
  \significado{part.}{no final de sentença para expressar admiração | no final de sentença mostrando afirmação, aprovação, urgência, aconselhamento, etc. | no final de sentença para indicar uma pergunta | para pausar ligeiramente uma frase, chamando a atenção para as palavras seguintes | após cada um dos itens listados}
  \veja{啊}{a1}
  \veja{啊}{a2}
  \veja{啊}{a3}
  \veja{啊}{a4}
\end{verbete}

\begin{verbete}{矮}{ai3}{13}[Radical 矢]
  \significado{adj.}{baixo em estatura, dimensão, grau ou ranque | curto (em comprimento)}
\end{verbete}

\begin{verbete}{矮凳}{ai3deng4}{13,14}
  \significado{s.}{banquinho baixo; banqueta}
\end{verbete}

\begin{verbete}{矮林}{ai3lin2}{13,8}
  \significado{s.}{mato; mata}
\end{verbete}

\begin{verbete}{矮胖}{ai3pang4}{13,9}
  \significado{adj.}{atarracado |  gorducho; rechonchudo; roliço | curto e robusto}
\end{verbete}

\begin{verbete}{矮人}{ai3ren2}{13,2}
  \significado{s.}{anão | homúnculo | nanismo}
\end{verbete}

\begin{verbete}{矮树}{ai3shu4}{13,9}
  \significado{s.}{arbusto; árvore pequena}
\end{verbete}

\begin{verbete}{矮小}{ai3xiao3}{13,3}
  \significado{adj.}{baixo e pequeno; curto e pequeno; subdimensionado}
\end{verbete}

\begin{verbete}{矮星}{ai3xing1}{13,9}
  \significado{s.}{estrela anã}
\end{verbete}

\begin{verbete}{矮子}{ai3zi5}{13,3}
  \significado{s.}{pessoa baixa; anão}
\end{verbete}

\begin{verbete}{爱}{ai4}{10}[Radical 爪]
  \significado*{s.}{sobrenome Ai}
  \significado[个]{s.}{amor; afeição}
  \significado{v.}{amar; ser afeiçoado a; ter interesse em (alguém) | cuidar bem de | gostar de (fazer algo) | estar inclinado (a fazer algo); ter o hábito de (fazer algo)}
\end{verbete}

\begin{verbete}{爱爱}{ai4'ai5}{10,10}
  \significado{v.}{(coloquial)~fazer amor}
\end{verbete}

\begin{verbete}{爱抚}{ai4fu3}{10,7}
  \significado{s.}{cuidado afetuoso; carinho}
  \significado{v.}{acariciar; cuidar (com ternura)}
\end{verbete}

\begin{verbete}{爱国}{ai4guo2}{10,8}
  \significado{adj.}{patriótico}
  \significado{v.}{amar o país; ser patriota}
\end{verbete}

\begin{verbete}{爱好}{ai4hao4}{10,6}
  \significado[个]{s.}{passatempo; interesse}
  \significado{v.}{ter prazer em; gostar de; ter algo como hobby; apetite por}
\end{verbete}

\begin{verbete}{爱好者}{ai4hao4zhe3}{10,6,8}
  \significado{s.}{amador; entusiasta; fã; amante de arte, esportes, etc.}
\end{verbete}

\begin{verbete}{爱人}{ai4ren5}{10,2}
  \significado[个]{s.}{esposa; amor, amada}
\end{verbete}

\begin{verbete}{爱上}{ai4shang4}{10,3}
  \significado{v.}{apaixonar-se por; ser gentil com}
\end{verbete}

\begin{verbete}{碍事}{ai4shi4}{13,8}
  \significado{s.}{(usualmente em frases negativas) sem consequência; não importa}
  \significado{v.+compl.}{estar no caminho; ser um obstáculo}
\end{verbete}

\begin{verbete}{安家}{an1jia1}{6,10}
  \significado{v.+compl.}{montar uma casa; estabelecer-se}
\end{verbete}

\begin{verbete}{安静}{an1jing4}{6,14}
  \significado{adj.}{quieto; tranquilo; pacífico; calmo}
\end{verbete}

\begin{verbete}{安排}{an1pai2}{6,11}
  \significado{s.}{arranjos; planos}
  \significado{v.}{organizar; programar; fazer planos}
\end{verbete}

\begin{verbete}{安全}{an1quan2}{6,6}
  \significado{adj.}{seguro}
  \significado{s.}{segurança}
\end{verbete}

\begin{verbete}{安神}{an1shen2}{6,9}
  \significado{v.+compl.}{acalmar os nervos | aliviar a inquietação pela tranquilização da mente e do corpo}
\end{verbete}

\begin{verbete}{暗恋}{an4lian4}{13,10}
  \significado{s.}{amor secreto}
  \significado{v.}{estar secretamente apaixonado por}
\end{verbete}

\begin{verbete}{暗香}{an4xiang1}{13,9}
  \significado{s.}{fragrância sutil}
\end{verbete}

\begin{verbete}{奥}{ao4}{12}[Radical 大]
  \significado{adj.}{obscuro; misterioso}
\end{verbete}

\begin{verbete}{奥林匹克运动会}{ao4lin2pi3ke4 yun4dong4hui4}{12,8,4,7,7,6,6}
  \significado*{s.}{Jogos Olímpicos; Olimpíadas}
\end{verbete}

\begin{verbete}{奥特曼}{ao4te4man4}{12,10,11}
  \significado*{s.}{\emph{Ultraman},  super-herói de ficção científica japonesa}
\end{verbete}

\begin{verbete}{奥运}{ao4yun4}{12,7}
  \significado*{s.}{Jogos Olímpicos; Olimpíadas (abreviação de 奥林匹克运动会)}
  \veja{奥林匹克运动会}{ao4lin2pi3ke4 yun4dong4hui4}
\end{verbete}

\begin{verbete}{奥运会}{ao4yun4hui4}{12,7,6}
  \significado*{s.}{Jogos Olímpicos; Olimpíadas (abreviação de 奥林匹克运动会)}
  \veja{奥林匹克运动会}{ao4lin2pi3ke4 yun4dong4hui4}
\end{verbete}

\begin{verbete}{澳}{ao4}{15}[Radical 水]
  \significado*{s.}{Austrália; abreviação de 澳大利亚}
  \veja{澳大利亚}{ao4da4li4ya4}
\end{verbete}

\begin{verbete}{澳大利亚}{ao4da4li4ya4}{15,3,7,6}
  \significado*{s.}{Austrália}
\end{verbete}

%%%%% EOF %%%%%

%%%
%%% B
%%%
\section*{B}
\addcontentsline{toc}{section}{B}

\begin{verbete}{八}{ba1}{2}
  \significado{num.}{8, oito}
\end{verbete}
\begin{verbete}{八八六}{ba1ba1liu4}{2;2;4}
  \significado{expr.}{Bye bye! (em salas de bate-papo e mensagens de texto)}
\end{verbete}
\begin{verbete}{巴西}{ba1xi1}{4;6}
  \significado*{s.}{Brasil}
\end{verbete}
\begin{verbete}{巴西人}{ba1xi1ren2}{4;6;2}
  \significado[个,位]{s.}{brasileiro; nascido no Brasil}
  \exemplo{他是巴西人。}[Ele é brasileiro.]
\end{verbete}
\begin{verbete}{巴西战舞}{ba1xi1zhan4wu3}{4;6;9;14}
  \significado{s.}{capoeira}
\end{verbete}
\begin{verbete}{吧}{ba1}{7}
  \significado{s.}{bar (servindo bebidas ou fornecendo acesso à \textit{Internet}); onomatopéia: Bang!}
  \significado{v.}{soprar (em um cachimbo, etc.)}
  \veja{吧}{ba5}
  \veja{吧}{bia1}
\end{verbete}
\begin{verbete}{把}{ba3}{7}
  \significado{p.c.}{para objetos com alça; para objetos pequenos: punhado}
  \significado{v.}{conter; alcançar; segurar}
  \veja{把}{ba4}
\end{verbete}
\begin{verbete}{把柄}{ba3bing3}{7;9}
  \significado{s.}{figurativo: informações que podem ser usadas contra alguém}
\end{verbete}
\begin{verbete}{把持}{ba3chi2}{7;9}
  \significado{v.}{controlar; dominar; monopolizar}
\end{verbete}
\begin{verbete}{把风}{ba3feng1}{7;4}
  \significado{v.}{estar atento; vigiar (durante uma atividade clandestina)}
\end{verbete}
\begin{verbete}{把关}{ba3guan1}{7;6}
  \significado{v.}{verificar algo}
\end{verbete}
\begin{verbete}{把脉}{ba3mai4}{7;9}
  \significado{v.}{sentir ou tomar o pulso de alguém}
\end{verbete}
\begin{verbete}{把式}{ba3shi4}{7;6}
  \significado{s.}{pessoa qualificada em um comércio}
\end{verbete}
\begin{verbete}{把守}{ba3shou3}{7;6}
  \significado{v.}{vigiar; guardar}
\end{verbete}
\begin{verbete}{把玩}{ba3wan2}{7;8}
  \significado{v.}{brincar com; mexer com}
\end{verbete}
\begin{verbete}{把稳}{ba3wen3}{7;14}
  \significado{adj.}{confiável}
\end{verbete}
\begin{verbete}{把握}{ba3wo4}{7;12}
  \significado{s.}{seguro; garantia; certeza}
  \significado{v.}{agarrar; segurar; aproveitar}
\end{verbete}
\begin{verbete}{把戏}{ba3xi4}{7;6}
  \significado{s.}{acrobacia; malabarismo; truque barato}
\end{verbete}
\begin{verbete}{把}{ba4}{7}
  \significado{v.}{lidar}
  \veja{把}{ba3}
\end{verbete}
\begin{verbete}{爸}{ba4}{8}
  \significado[个,位]{s.}{pai}
\end{verbete}
\begin{verbete}{爸爸}{ba4ba5}{8;8}
  \significado[个,位]{s.}{papai, pai (informal)}
\end{verbete}
\begin{verbete}{爸妈}{ba4ma1}{8;6}
  \significado{s.}{pai e mãe}
\end{verbete}
\begin{verbete}{罢}{ba4}{10}
  \significado{v.}{parar; cessar; demitir; suspender; desistir; terminar}
  \veja{吧}{ba5}
  \veja{罢}{ba5}
\end{verbete}
\begin{verbete}{吧}{ba5}{7}
  \significado{part.}{partícula modal indicando sugestão ou suposição; ...eu presumo.; ...OK?; ...certo?}
  \veja{吧}{ba1}
  \veja{吧}{bia1}
\end{verbete}
\begin{verbete}{罢}{ba5}{10}
  \significado{part.}{partícula final, a mesma que 吧}
  \veja{罢}{ba4}
  \veja{吧}{ba5}
\end{verbete}
\begin{verbete}{白}{bai2}{5}
  \significado{adj.}{branco; claro; puro; límpido; simples; em branco; grátis}
  \significado{adv.}{em vão; sem propósito; por nada}
  \significado{s.}{parte falada na ópera; diálogo; dialeto}
  \significado*{s.}{sobrenome Bai}
\end{verbete}
\begin{verbete}{白菜}{bai2cai4}{5;11}
  \significado[棵,个]{s.}{acelga; repolho chinês}
\end{verbete}
\begin{verbete}{白痴}{bai2chi1}{5;13}
  \significado{adj.}{imbecil}
  \significado{s.}{estúpido; imbecil}
\end{verbete}
\begin{verbete}{白蛋白}{bai2dan4bai2}{5;11;5}
  \significado{s.}{albumina}
\end{verbete}
\begin{verbete}{白鹄}{bai2hu2}{5;12}
  \significado{s.}{cisne branco}
\end{verbete}
\begin{verbete}{白拣}{bai2jian3}{5;8}
  \significado{s.}{uma escolha barata}
  \significado{v.}{escolher algo que não custa nada}
\end{verbete}
\begin{verbete}{白萝卜}{bai2luo2bo5}{5;11;2}
  \significado{s.}{rabanete branco}
\end{verbete}
\begin{verbete}{白色}{bai2se4}{5;6}
  \significado{s.}{cor branca}
\end{verbete}
\begin{verbete}{白天}{bai2tian1}{5;4}
  \significado{p.t.}{dia; de dia}
  \significado[个]{s.}{dia}
\end{verbete}
\begin{verbete}{白苋}{bai2xian4}{5;7}
  \significado{s.}{amaranto branco; brotos e folhas tenras de espinafre chinês usados como alimento}
\end{verbete}
\begin{verbete}{百}{bai3}{6}
  \significado{num.}{100, cem; centena; cento}
  \significado*{s.}{sobrenome Bai}
\end{verbete}
\begin{verbete}{百分}{bai3fen1}{6;4}
  \significado{num.}{por cento}
  \significado{s.}{porcentagem}
\end{verbete}
\begin{verbete}{搬}{ban1}{13}
  \significado{v.}{copiar indiscriminadamente; mover-se (ou seja, mudar-se); mover-se (algo relativamente pesado ou volumoso); mudar; mudar-se}
\end{verbete}
\begin{verbete}{搬动}{ban1dong4}{13;6}
  \significado{v.}{mover-se (alguma coisa); se mudar}
\end{verbete}
\begin{verbete}{搬家}{ban1jia1}{13;10}
  \significado{s.}{mudança}
  \significado{v.+compl.}{mudar-se de casa}
\end{verbete}
\begin{verbete}{搬口}{ban1kou3}{13;3}
  \significado{v.}{tagarelar; transmitir histórias (idioma); semear dissensão; contar histórias}
\end{verbete}
\begin{verbete}{搬弄}{ban1nong4}{13;7}
  \significado{v.}{causar problemas; mexer com alguém; mostrar (o que se pode fazer)}
\end{verbete}
\begin{verbete}{搬运}{ban1yun4}{13;7}
  \significado{s.}{frete; transporte}
  \significado{v.}{carregar; transportar}
\end{verbete}
\begin{verbete}{搬走}{ban1zou3}{13;7}
  \significado{v.}{carregar}
\end{verbete}
\begin{verbete}{办}{ban4}{4}
  \significado{v.}{lidar com; lidar; gerenciar; configurar}
\end{verbete}
\begin{verbete}{办法}{ban4fa3}{4;8}
  \significado[条,个]{s.}{meio (de se fazer alguma coisa); método; medida}
\end{verbete}
\begin{verbete}{办公室}{ban4gong1shi4}{4;4;9}
  \significado[间]{s.}{gabinete; escritório}
\end{verbete}
\begin{verbete}{半}{ban4}{5}
  \significado{adj.}{incompleto}
  \significado{adv.}{prefixo semi}
  \significado{num.}{(depois de um número) ``e meio''}
  \significado{s.}{metade}
\end{verbete}
\begin{verbete}{半球}{ban4qiu2}{5;11}
  \significado{s.}{hemisfério}
\end{verbete}
\begin{verbete}{半音}{ban4yin1}{5;9}
  \significado{s.}{semitom}
\end{verbete}
\begin{verbete}{帮}{bang1}{9}
  \significado{p.c.}{para alguém (como uma ajuda)}
  \significado{s.}{gangue; grupo; contratado (como trabalhador); camada externa; festa; sociedade secreta}
  \significado{v.}{ajudar; apoiar}
\end{verbete}
\begin{verbete}{帮教}{bang1jiao4}{9;11}
  \significado{v.}{orientar}
\end{verbete}
\begin{verbete}{帮佣}{bang1yong1}{9;7}
  \significado{s.}{ajudante doméstico; servo}
\end{verbete}
\begin{verbete}{帮助}{bang1zhu4}{9;7}
  \significado[种]{s.}{ajuda; assistência}
  \significado{v.}{ajudar; dar assistência}
\end{verbete}
\begin{verbete}{包}{bao1}{5}
  \significado{p.c.}{pacotes, sacos, sacolas, embrulhos}
  \significado[个,只]{s.}{bolsa; pacote; recipiente; embrulho}
  \significado{v.}{contratar; cobrir; segurar ou abraçar; incluir; assumir o comando; embrulhar}
  \significado*{s.}{sobrenome Bao}
\end{verbete}
\begin{verbete}{包办}{bao1ban4}{5;4}
  \significado{v.}{comandar todo o show; comprometer-se a fazer tudo sozinho}
\end{verbete}
\begin{verbete}{包干}{bao1gan1}{5;3}
  \significado{s.}{tarefa alocada}
  \significado{v.}{ter a responsabilidade total sobre um trabalho}
\end{verbete}
\begin{verbete}{包括}{bao1kuo4}{5;9}
  \significado{v.}{compreender; consistir em; incluir; incorporar; envolver}
\end{verbete}
\begin{verbete}{包子}{bao1zi5}{5;3}
  \significado[个]{s.}{pão recheado cozido no vapor}
\end{verbete}
\begin{verbete}{包租}{bao1zu1}{5;10}
  \significado{s.}{aluguel fixo para terras agrícolas}
  \significado{v.}{fretar; alugar; alugar um terreno ou uma casa para subarrendar}
\end{verbete}
\begin{verbete}{保存}{bao3cun2}{9;6}
  \significado{v.}{conservar; preservar; computação: salvar (um arquivo, etc.)}
\end{verbete}
\begin{verbete}{保护}{bao3hu4}{9;7}
  \significado{s.}{proteção}
  \significado{v.}{proteger; defender; salvaguardar}
\end{verbete}
\begin{verbete}{保护国}{bao3hu4guo2}{9;7;8}
  \significado{s.}{protetorado}
\end{verbete}
\begin{verbete}{保护剂}{bao3hu4ji4}{9;7;8}
  \significado{s.}{agente protetor}
\end{verbete}
\begin{verbete}{保护区}{bao3hu4qu1}{9;7;4}
  \significado{s.}{área protegida; área de conservação}
\end{verbete}
\begin{verbete}{保护色}{bao3hu4se4}{9;7;6}
  \significado{s.}{camuflagem}
\end{verbete}
\begin{verbete}{保护神}{bao3hu4shen2}{9;7;9}
  \significado{s.}{anjo da guarda; santo patrono}
\end{verbete}
\begin{verbete}{保护物}{bao3hu4·wu4}{9;7;8}
  \significado{s.}{protetor}
\end{verbete}
\begin{verbete}{保护性}{bao3hu4xing4}{9;7;8}
  \significado{s.}{proteção}
\end{verbete}
\begin{verbete}{保护者}{bao3hu4zhe3}{9;7;8}
  \significado{s.}{protetor; segurador}
\end{verbete}
\begin{verbete}{报}{bao4}{7}
  \significado[份,张]{s.}{jornal; recompensa; relatório; vingança}
  \significado{v.}{anunciar; informar}
\end{verbete}
\begin{verbete}{报酬}{bao4chou5}{7;13}
  \significado{s.}{recompensa; remuneração}
\end{verbete}
\begin{verbete}{报纸}{bao4zhi3}{7;7}
  \significado[张]{s.}{jornal; diário}
\end{verbete}
\begin{verbete}{暴力}{bao4li4}{15;2}
  \significado{adj.}{violento}
  \significado{s.}{violência}
\end{verbete}
\begin{verbete}{暴雨}{bao4yu3}{15;8}
  \significado[场,阵]{s.}{tempestade; chuva torrencial}
\end{verbete}
\begin{verbete}{杯}{bei1}{8}
  \significado{p.c.}{para certos recipientes de líquidos: copo, xícara, etc.}
  \significado{s.}{copo; taça; xícara; copa troféu}
\end{verbete}
\begin{verbete}{杯具}{bei1ju4}{8;8}
  \significado{s.}{parachoque; fiasco; gíria: tragédia}
\end{verbete}
\begin{verbete}{杯子}{bei1zi5}{8;3}
  \significado[个,只]{s.}{copo; caneca; xícara; taça}
\end{verbete}
\begin{verbete}{背}{bei1}{9}
  \significado{v.}{estar sobrecarregado; carregar nas costas ou no ombro}
  \veja{背}{bei4}
\end{verbete}
\begin{verbete}{㮎}{bei1}{13}
  \variante{杯}{bei1}
\end{verbete}
\begin{verbete}{北}{bei3}{5}
  \significado{p.d.l.}{norte}
  \significado{v.}{ser derrotado (clássico)}
\end{verbete}
\begin{verbete}{北边}{bei3bian5}{5;5}
  \significado{p.l.}{lado norte; ao norte de}
\end{verbete}
\begin{verbete}{北方}{bei3fang1}{5;4}
  \significado{p.l.}{norte; a parte norte de um país}
\end{verbete}
\begin{verbete}{北京}{bei3jing1}{5;8}
  \significado*{s.}{Beijing (Pequim); Capital da China}
\end{verbete}
\begin{verbete}{北面}{bei3mian4}{5;9}
  \significado{p.l.}{lado norte}
\end{verbete}
\begin{verbete}{背}{bei4}{9}
  \significado{p.l.}{a parte de trás de um corpo ou objeto}
  \significado{s.}{costas; gíria: azarado}
  \significado{v.}{esconder algo de; decorar; recitar de memória; virar as costas}
  \veja{背}{bei1}
\end{verbete}
\begin{verbete}{被}{bei4}{10}
  \significado{prep.}{por}
\end{verbete}
\begin{verbete}{被单}{bei4dan1}{10;8}
  \significado[床]{s.}{lençol}
\end{verbete}
\begin{verbete}{被动}{bei4dong4}{10;6}
  \significado{adj.}{passivo}
\end{verbete}
\begin{verbete}{被告}{bei4gao4}{10;7}
  \significado{s.}{réu}
\end{verbete}
\begin{verbete}{被迫}{bei4po4}{10;8}
  \significado{v.}{ser compelido; ser forçado}
\end{verbete}
\begin{verbete}{被窝}{bei4wo1}{10;12}
  \significado{s.}{colcha}
\end{verbete}
\begin{verbete}{被子}{bei4zi5}{10;3}
  \significado[床]{s.}{colcha}
\end{verbete}
\begin{verbete}{本}{ben3}{5}
  \significado{adv.}{inerente; originalmente}
  \significado{p.c.}{para livros, dicionários, periódicos, arquivos, etc.}
  \significado{s.}{origem; fonte; raiz}
\end{verbete}
\begin{verbete}{本子}{ben3zi5}{5;3}
  \significado[本]{s.}{caderno}
\end{verbete}
\begin{verbete}{笨蛋}{ben4dan4}{11;11}
  \significado{s.}{bobalhão; cabeça-oca; cabeça-dura}
  \significado{v.}{iludir; enganar}
\end{verbete}
\begin{verbete}{甭}{beng2}{9}
  \significado{v.o.}{contração de 不用; não precisar}
  \veja{不用}{bu2yong4}
\end{verbete}
\begin{verbete}{鼻子}{bi2zi5}{14;3}
  \significado[个,只]{s.}{nariz}
\end{verbete}
\begin{verbete}{比}{bi3}{4}
  \significado{part.}{partícula usada para comparação (superioridade)}
  \significado{prep.}{que; do que}
  \significado{s.}{razão (taxa)}
  \significado{v.}{comparar; contrastar; gesticular (com as mãos)}
\end{verbete}
\begin{verbete}{比较}{bi3jiao4}{4;10}
  \significado{adv.}{comparativamente; relativamente}
  \significado{s.}{comparação; relativamente}
  \significado{v.}{comparar; contrastar}
\end{verbete}
\begin{verbete}{比萨饼}{bi3sa4bing3}{4;11;9}
  \significado[张]{s.}{pizza}
\end{verbete}
\begin{verbete}{比赛}{bi3sai4}{4;14}
  \significado[场,次]{s.}{competição; concurso}
  \significado{v.}{competir}
\end{verbete}
\begin{verbete}{笔}{bi3}{10}
  \significado{p.c.}{para somas de dinheiro, negócios}
  \significado[支,枝]{s.}{caneta; lápis}
\end{verbete}
\begin{verbete}{闭嘴}{bi4zui3}{6;16}
  \significado{expr.}{Cale-se!}
\end{verbete}
\begin{verbete}{壁纸}{bi4zhi3}{16;7}
  \significado{s.}{papel de parede}
\end{verbete}
\begin{verbete}{吧}{bia1}{7}
  \significado{s.}{bar (servindo bebidas ou fornecendo acesso à \textit{Internet}); onomatopéia: Smack! (para beijo)}
  \significado{v.}{soprar (em um cachimbo, etc.)}
  \veja{吧}{ba1}
  \veja{吧}{ba5}
\end{verbete}
\begin{verbete}{边}{bian1}{5}
  \significado{adv.}{simultaneamente}
  \significado[个]{s.}{fronteira; limite; borda; margem; lado}
  \veja{边}{bian5}
\end{verbete}
\begin{verbete}{编程}{bian1cheng2}{12;12}
  \significado{s.}{programa de computador}
  \significado{v.}{programar computador}
\end{verbete}
\begin{verbete}{邉}{bian1}{17}
  \variante{边}{bian1}
\end{verbete}
\begin{verbete}{变}{bian4}{8}
  \significado{v.}{mudar; transformar; variar}
\end{verbete}
\begin{verbete}{变更}{bian4geng1}{8;7}
  \significado{v.}{alterar; mudar; modificar}
\end{verbete}
\begin{verbete}{变节}{bian4jie2}{8;5}
  \significado{s.}{traição; deserção; vira-casaca}
  \significado{v.}{mudar de lado politicamente}
\end{verbete}
\begin{verbete}{变迁}{bian4qian1}{8;6}
  \significado{s.}{mudanças; vicissitudes}
\end{verbete}
\begin{verbete}{变数}{bian4shu4}{8;13}
  \significado{s.}{matemática: variável}
\end{verbete}
\begin{verbete}{变异}{bian4yi4}{8;6}
  \significado{s.}{variação; mutação}
\end{verbete}
\begin{verbete}{遍}{bian4}{12}
  \significado{p.l.}{em todos os lugares; por toda parte}
  \significado{p.c.}{para a repetição de ações de leitura, fala ou escrita}
\end{verbete}
\begin{verbete}{边}{bian5}{5}
  \significado{s.}{sufixo de uma palavra de localidade}
  \veja{边}{bian1}
\end{verbete}
\begin{verbete}{标准}{biao1zhun3}{9;10}
  \significado{adj.}{criterioso; padronizado; normatizado}
  \significado[个]{s.}{critério; padrão (oficial); norma}
\end{verbete}
\begin{verbete}{表演}{biao3yan3}{8;14}
  \significado[发,场]{s.}{representação; atuação}
  \significado{v.}{representar; atuar}
\end{verbete}
\begin{verbete}{表演赛}{biao3yan3sai4}{8;14;14}
  \significado{s.}{partida ou jogo de exibição}
\end{verbete}
\begin{verbete}{表演特技}{biao3yan3·te4ji4}{8;14;10;7}
  \significado{s.}{acrobacia; pirueta; façanha}
\end{verbete}
\begin{verbete}{表演艺术}{biao3yan3·yi4shu4}{8;14;4;5}
  \significado{s.}{arte performática}
\end{verbete}
\begin{verbete}{表演游戏}{biao3yan3·you2xi4}{8;14;12;6}
  \significado{s.}{exibição dramática}
\end{verbete}
\begin{verbete}{表演者}{biao3yan3·zhe3}{8;14;8}
  \significado{s.}{ator}
\end{verbete}
\begin{verbete}{表扬}{biao3yang2}{8;6}
  \significado{v.}{elogiar; louvar}
\end{verbete}
\begin{verbete}{表扬信}{biao3yang2·xin4}{8;6;9}
  \significado{s.}{carta de elogio; depoimento}
\end{verbete}
\begin{verbete}{别}{bie2}{7}
  \significado{adv.}{nada de (pedir a alguém para não fazer); não}
  \significado{pron.}{outro}
  \significado{v.}{classificar; separar; distinguir; partir; deixar; fixar; colar alguma coisa em}
  \significado*{s.}{sobrenome Bie}
  \veja{别}{bie4}
\end{verbete}
\begin{verbete}{别的}{bie2de5}{7;8}
  \significado{pron.}{outro}
\end{verbete}
\begin{verbete}{别人}{bie2ren5}{7;2}
  \significado{pron.}{outra pessoa; outro povo; outros}
\end{verbete}
\begin{verbete}{别}{bie4}{7}
  \significado{v.}{fazer com que alguém mude seus hábitos, opiniões, etc.}
  \veja{别}{bie2}
\end{verbete}
\begin{verbete}{宾馆}{bin1guan3}{10;11}
  \significado[个,家]{s.}{casa de hóspedes; hotel}
\end{verbete}
\begin{verbete}{冰}{bing1}{6}
  \significado{adj.}{hostil; gelado}
  \significado[块]{s.}{gelo; gíria: metanfetamina}
  \significado{v.}{sentir frio; relaxar algo}
\end{verbete}
\begin{verbete}{冰球}{bing1qiu2}{6;11}
  \significado{s.}{hóquei no gelo}
\end{verbete}
\begin{verbete}{冰天雪地}{bing1tian1-xue3di4}{6;4;11;6}
  \significado{expr.}{um mundo de gelo e neve}
\end{verbete}
\begin{verbete}{病}{bing4}{10}
  \significado[场]{s.}{doença}
  \significado{v.}{adoecer; estar doente}
\end{verbete}
\begin{verbete}{拨转}{bo1zhuan3}{8;8}
  \significado{v.}{transferir (fundos, etc.); virar; dar a volta}
\end{verbete}
\begin{verbete}{啵}{bo1}{11}
  \significado{s.}{onomatopéia: borbulhar}
  \veja{啵}{bo5}
\end{verbete}
\begin{verbete}{菠菜}{bo1cai4}{11;11}
  \significado[棵]{s.}{espinafre}
\end{verbete}
\begin{verbete}{脖子}{bo2zi5}{11;3}
  \significado[个]{s.}{pescoço}
\end{verbete}
\begin{verbete}{博物馆}{bo2wu4guan3}{12;8;11}
  \significado{s.}{museu}
\end{verbete}
\begin{verbete}{啵}{bo5}{11}
  \significado{part.}{partícula gramaticalmente equivalente a 吧}
  \veja{吧}{ba5}
  \veja{啵}{bo1}
\end{verbete}
\begin{verbete}{不}{bu2}[ (antes de quarto tom)]{4}
  \significado{adv.}{não}
  \veja{不}{bu4}
  \veja{不}{bu5}
\end{verbete}
\begin{verbete}{不错}{bu2cuo4}{4;13}
  \significado{adj.}{correto; não (é) mau; bastante bom; certo}
\end{verbete}
\begin{verbete}{不过}{bu2guo4}{4;6}
  \significado{conj.}{mas; contudo; no entanto}
\end{verbete}
\begin{verbete}{不客气}{bu2ke4qi5}{4;9;4}
  \significado{expr.}{de nada; não há de que}
\end{verbete}
\begin{verbete}{不是话}{bu2shi4hua4}{4;9;8}
  \veja{不像话}{bu2xiang4hua4}
\end{verbete}
\begin{verbete}{不成话}{bu2xheng2hua4}{4;6;8}
  \veja{不像话}{bu2xiang4hua4}
\end{verbete}
\begin{verbete}{不像话}{bu2xiang4hua4}{4;13;8}
  \significado{expr.}{sem razão; demasiado irracionável}
\end{verbete}
\begin{verbete}{不要}{bu2yao4}{4;9}
  \significado{adv.}{nada de (pedir a alguém para não fazer); não}
\end{verbete}
\begin{verbete}{不用}{bu2yong4}{4;5}
  \significado{v.o.}{não precisar}
  \veja{甭}{beng2}
\end{verbete}
\begin{verbete}{不大离}{bu2da4li2}{9;4;10}
  \significado{adj.}{bem perto; quase certo; nada mal}
\end{verbete}
\begin{verbete}{不}{bu4}{4}
  \significado{adv.}{não}
  \veja{不}{bu2}
  \veja{不}{bu5}
\end{verbete}
\begin{verbete}{不同}{bu4tong2}{4;6}
  \significado{adj.}{diferente; distinto}
\end{verbete}
\begin{verbete}{布署}{bu4shu3}{5;13}
  \variante{部署}{bu4shu3}
\end{verbete}
\begin{verbete}{部}{bu4}{10}
  \significado{p.c.}{para obras de literatura, filmes, máquinas etc.}
  \significado[根]{s.}{departamento; divisão; ministério; seção; parte; tropas}
\end{verbete}
\begin{verbete}{部分}{bu4fen5}{10;4}
  \significado[个]{s.}{parte; parte de; uma parte de; pedaço; secção}
\end{verbete}
\begin{verbete}{部门}{bu4men2}{10;3}
  \significado[个]{s.}{filial; departamento; divisão; seção}
\end{verbete}
\begin{verbete}{部属}{bu4shu3}{10;12}
  \significado{s.}{afiliado a um ministério; subordinado; tropas sob comando de alguém}
\end{verbete}
\begin{verbete}{部署}{bu4shu3}{10;13}
  \significado{s.}{implantação}
  \significado{v.}{implantar}
\end{verbete}
\begin{verbete}{部下}{bu4xia4}{10;3}
  \significado{s.}{subordinado; tropas sob comando de alguém}
\end{verbete}
\begin{verbete}{部族}{bu4zu2}{10;11}
  \significado{adj.}{tribal}
  \significado{s.}{tribo}
\end{verbete}
\begin{verbete}{不}{bu5}{4}
  \significado{adv.}{não (em expressões ``v.$+$不$+$v.'')}
  \veja{不}{bu2}
  \veja{不}{bu4}
\end{verbete}

%%%%% EOF %%%%%

%%%
%%% C
%%%
\section*{C}
\addcontentsline{toc}{section}{C}

\begin{verbete}{才}{cai2}{3}
  \significado{adv.}{apenas (seguido por uma cláusula numérica); só (indicando que algo está acontecendo mais tarde do que o esperado); não até (precedido por uma cláusula de condição ou razão); há um momento atrás}
  \significado{conj.}{apenas quando}
  \significado{s.}{um indivíduo capaz; habilidade; talento}
\end{verbete}

\begin{verbete}{才略}{cai2lve4}{3;11}
  \significado{s.}{habilidade e sagacidade}
\end{verbete}

\begin{verbete}{菜}{cai4}{11}
  \significado[棵]{s.}{hortaliça; verdura}
  \significado[样,道,盘]{s.}{prato (de comida)}
\end{verbete}

\begin{verbete}{菜单}{cai4dan1}{11;8}
  \significado[份,张]{s.}{menu; cardápio}
\end{verbete}

\begin{verbete}{参观}{can1guan3}{8;6}
  \significado{v.}{visitar}
\end{verbete}

\begin{verbete}{参加}{can1jia1}{8;5}
  \significado{v.}{participar em; tomar parte em; assistir}
\end{verbete}

\begin{verbete}{餐厅}{can1ting1}{16;4}
  \significado[家]{s.}{restaurante}
  \significado[间]{s.}{sala de jantar}
\end{verbete}

\begin{verbete}{蚕纸}{can2zhi3}{10;7}
  \significado{s.}{papel em que o bicho-da-seda põe seus ovos}
\end{verbete}

\begin{verbete}{草}{cao3}{9}
  \significado[棵,撮,株,根]{s.}{erva; grama}
\end{verbete}

\begin{verbete}{草地}{cao3di4}{9;6}
  \significado[片]{s.}{relva; pastagem}
\end{verbete}

\begin{verbete}{草纸}{cao3zhi3}{9;7}
  \significado{s.}{papel pardo; pergaminho; papel de palha áspero; papel higiênico}
\end{verbete}

\begin{verbete}{厕所}{ce4suo3}{8;8}
  \significado[间,处]{s.}{sanitário; toilette}
\end{verbete}

\begin{verbete}{厕纸}{ce4zhi3}{8;7}
  \significado{s.}{papel higiênico}
\end{verbete}

\begin{verbete}{层}{ceng2}{7}
  \significado{p.c.}{para andar, piso}
\end{verbete}

\begin{verbete}{茶}{cha2}{9}
  \significado[杯,壶]{s.}{chá}
\end{verbete}

\begin{verbete}{差不多}{cha4bu5duo1}{9;4;6}
  \significado{adj.}{mais ou menos}
\end{verbete}

\begin{verbete}{差点儿}{cha4dian3r5}{9;9;2}
  \significado{adv.}{por pouco; por um triz; quase}
\end{verbete}

\begin{verbete}{拆}{chai1}{8}
  \significado{v.}{remover; tirar do seu lugar; desfazer; desmontar}
\end{verbete}

\begin{verbete}{长}{chang2}{4}
  \significado{adj.}{comprido; longo}
  \veja{长}{zhang3}
\end{verbete}

\begin{verbete}{长成}{chang2cheng2}{4;6}
  \significado*{s.}{Grande Muralha}
\end{verbete}

\begin{verbete}{常常}{chang2chang2}{11;11}
  \significado{adv.}{frequentemente; com frequência}
\end{verbete}

\begin{verbete}{常问问题}{chang2wen4wen4ti2}{11;6;6;15}
  \significado{s.}{FAQ; perguntas frequentes}
\end{verbete}

\begin{verbete}{场}{chang3}{6}
  \significado{p.c.}{para número de exames; para atividades esportivas ou recreativas}
  \significado{s.}{local grande usado para um propósito específico; cena (de uma peça); palco}
\end{verbete}

\begin{verbete}{唱}{chang4}{11}
  \significado{v.}{cantar}
\end{verbete}

\begin{verbete}{唱歌}{chang4ge1}{11;14}
  \significado{v.+compl.}{cantar}
\end{verbete}

\begin{verbete}{超市}{chao1shi4}{12;5}
  \significado[家]{s.}{supermercado}
\end{verbete}

\begin{verbete}{吵}{chao3}{7}
  \significado{adj.}{barulhento; ruidoso}
\end{verbete}

\begin{verbete}{吵架}{chao3jia4}{7;9}
  \significado{v.+compl.}{brigar; ralhar; zangar-se}
\end{verbete}

\begin{verbete}{炒}{chao3}{8}
  \significado{v.}{saltear; demitir (alguém)}
\end{verbete}

\begin{verbete}{车}{che1}{4}
  \significado[辆]{s.}{carro; veículo; viatura}
  \significado*{s.}{sobrenome Che}
  \veja{车}{ju1}
\end{verbete}

\begin{verbete}{车次}{che1ci4}{4;6}
  \significado{s.}{número do trem}
\end{verbete}

\begin{verbete}{车库}{che1ku4}{4;7}
  \significado{s.}{garagem}
\end{verbete}

\begin{verbete}{车牌}{che1pai2}{4;12}
  \significado{s.}{matrícula; placa de carro}
\end{verbete}

\begin{verbete}{车水马龙}{che1shui3-ma3long2}{4;4;3;5}
  \significado{expr.}{tráfego engarrafado; engarrafamento; literalmente: ``fluxo interminável de cavalos e carruagens''}
\end{verbete}

\begin{verbete}{车站}{che1zhan4}{4;10}
  \significado[处,个]{s.}{estação; ponto de ônibus}
\end{verbete}

\begin{verbete}{衬衫}{chen4shan1}{8;8}
  \significado[件]{s.}{camisa; blusa}
\end{verbete}

\begin{verbete}{成}{cheng2}{6}
  \significado*{s.}{sobrenome Cheng}
  \significado{v.}{sair-se bem; ser bem sucedido}
\end{verbete}

\begin{verbete}{成都}{cheng2du1}{6;10}
  \significado*{s.}{Chengdu}
\end{verbete}

\begin{verbete}{成婚}{cheng2hun1}{6;11}
  \significado{v.}{casar-se}
\end{verbete}

\begin{verbete}{成活}{cheng2huo2}{6;9}
  \significado{v.}{sobreviver}
\end{verbete}

\begin{verbete}{成绩}{cheng2ji4}{6;11}
  \significado[项,个]{s.}{nota; classificação}
\end{verbete}

\begin{verbete}{成家}{cheng2jia1}{6;10}
  \significado{v.}{tornar-se um especialista reconhecido; estabelecer-se e casar-se (de um homem)}
\end{verbete}

\begin{verbete}{成批}{cheng2pi1}{6;7}
  \significado{s.}{em lotes; a granel}
\end{verbete}

\begin{verbete}{成器}{cheng2qi4}{6;16}
  \significado{v.}{tornar-se uma pessoa digna de respeito; fazer algo de si mesmo}
\end{verbete}

\begin{verbete}{成色}{cheng2se4}{6;6}
  \significado{v.}{sair-se bem; ser bem sucedido}
\end{verbete}

\begin{verbete}{成为}{cheng2wei2}{6;4}
  \significado{s.}{tornar-se; transformar-se em}
\end{verbete}

\begin{verbete}{诚实}{cheng2shi2}{8;8}
  \significado{adj.}{honesto}
\end{verbete}

\begin{verbete}{诚实地}{cheng2shi2·di5}{8;8;6}
  \significado{adv.}{honestamente}
\end{verbete}

\begin{verbete}{城市}{cheng2shi4}{9;5}
  \significado[座]{s.}{cidade}
\end{verbete}

\begin{verbete}{乘客}{cheng2ke2}{10;9}
  \significado{s.}{passageiro}
\end{verbete}

\begin{verbete}{乘客数}{cheng2ke2·shu4}{10;9;13}
  \significado{s.}{número de passageiros}
\end{verbete}

\begin{verbete}{惩处}{cheng2chu3}{12;5}
  \significado{v.}{administrar justiça; punir}
\end{verbete}

\begin{verbete}{惩罚}{cheng2fa2}{12;9}
  \significado{v.}{punir; penalizar}
\end{verbete}

\begin{verbete}{程控}{cheng2kong4}{12;11}
  \significado{s.}{programado; sob controle automático}
\end{verbete}

\begin{verbete}{程序}{cheng2xu4}{12;7}
  \significado{s.}{procedimento; sequência; ordem; programa de computador}
\end{verbete}

\begin{verbete}{程序库}{cheng2xu4ku4}{12;7;7}
  \significado{s.}{biblioteca de funções e procedimentos para programas de computador}
\end{verbete}

\begin{verbete}{程序设计}{cheng2xu4she4ji4}{12;7;6;4}
  \significado{s.}{programação de computadores}
\end{verbete}

\begin{verbete}{橙色}{cheng2se4}{16;6}
  \significado{s.}{cor de laranja}
\end{verbete}

\begin{verbete}{橙汁}{cheng2zhi1}{16;5}
  \significado[瓶,杯,罐,盒]{s.}{suco de laranja}
  \veja{橘子汁}{ju2zi5zhi1}
  \veja{柳橙汁}{liu3cheng2zhi1}
\end{verbete}

\begin{verbete}{吃}{chi1}{6}
  \significado{v.}{comer}
\end{verbete}

\begin{verbete}{吃屎}{chi1·shi3}{6;9}
  \significado{expr.}{Coma merda!}
\end{verbete}

\begin{verbete}{迟到}{chi1dao4}{7;8}
  \significado{v.}{chegar atrasado; tardar}
\end{verbete}

\begin{verbete}{斥骂}{chi4ma4}{5;9}
  \significado{v.}{repreender}
\end{verbete}

\begin{verbete}{憧憬}{chong1jing3}{15;15}
  \significado{v.}{ansiar por; esperar por}
\end{verbete}

\begin{verbete}{重}{chong2}{9}
  \significado{adv.}{de novo}
  \significado{p.c.}{camadas}
  \significado{s.}{repetição}
  \significado{v.}{repetir}
  \veja{重}{zhong4}
\end{verbete}

\begin{verbete}{重重}{chong2chong2}{9;9}
  \significado{adv.}{camada após camada; um após o outro}
  \veja{重重}{zhong4zhong4}
\end{verbete}

\begin{verbete}{重迭}{chong2die2}{9;8}
  \significado{s.}{sobreposição; redundância}
  \significado{v.}{duplicar; sobrepor}
\end{verbete}

\begin{verbete}{重阳节}{chong2yang1jie2}{9;6;5}
  \significado*{s.}{Festa do Duplo Nove, Festival Yang, dia de subir aos lugares mais altos para evitar calamidades e dia do culto aos antepassados (9º dia do nono mês lunar)}
\end{verbete}

\begin{verbete}{宠物}{chong3wu4}{8;8}
  \significado{s.}{animal de estimação}
\end{verbete}

\begin{verbete}{酬劳}{chou2lao2}{13;7}
  \significado{s.}{recompensa}
\end{verbete}

\begin{verbete}{臭}{chou4}{10}
  \significado{adj.}{fétido; repulsivo; repugnante; malcheiroso}
  \significado{s.}{fedor}
  \significado{v.}{feder}
  \veja{臭}{xiu4}
\end{verbete}

\begin{verbete}{臭气}{chou4qi4}{10;4}
  \significado{s.}{fedor}
\end{verbete}

\begin{verbete}{殠}{chou4}{14}
  \variante{臭}{chou4}
\end{verbete}

\begin{verbete}{出}{chu1}{5}
  \significado{p.c.}{para dramas, peças, óperas, etc.}
  \significado{v.d.}{sair; ir para fora; vir para fora}
\end{verbete}

\begin{verbete}{出版}{chu1ban3}{5;8}
  \significado{v.}{publicar; editar}
\end{verbete}

\begin{verbete}{出版社}{chu1ban3she4}{5;8;7}
  \significado{s.}{editora}
\end{verbete}

\begin{verbete}{出发}{chu1fa1}{5;5}
  \significado{v.}{partir; começar (uma jornada)}
\end{verbete}

\begin{verbete}{出口}{chu1kou3}{5;3}
  \significado[个]{s.}{exportação}
  \significado{v.}{exportar}
\end{verbete}

\begin{verbete}{出来}{chu1lai5}{5;7}
  \significado{v.d.}{sair; vir para fora (para a minha localização)}
\end{verbete}

\begin{verbete}{出去}{chu1qu5}{5;5}
  \significado{v.d.}{sair; ir para fora (a partir da minha localização)}
\end{verbete}

\begin{verbete}{出站}{chu1·zhan4}{5;10}
  \significado{s.}{saída da estação}
\end{verbete}

\begin{verbete}{出租}{chu1zu1}{5;10}
  \significado{v.}{alugar; arrendar}
\end{verbete}

\begin{verbete}{出租车}{chu1zu1che1}{5;10;4}
  \significado{s.}{táxi}
  \veja{出租汽车}{chu1zu1qi4che1}
\end{verbete}

\begin{verbete}{出租汽车}{chu1zu1qi4che1}{5;10;7;4}
  \significado[辆]{s.}{táxi}
  \veja{出租车}{chu1zu1che1}
\end{verbete}

\begin{verbete}{出租司机}{chu1zu1si1ji1}{5;10;5;6}
  \significado{s.}{motorista de táxi}
\end{verbete}

\begin{verbete}{厨房}{chu2fang2}{12;8}
  \significado[间]{s.}{cozinha}
\end{verbete}

\begin{verbete}{穿}{chuan1}{9}
  \significado{v.}{vestir}
\end{verbete}

\begin{verbete}{传真}{chuan2zhen1}{6;10}
  \significado{s.}{fax, facsímile}
\end{verbete}

\begin{verbete}{船}{chuan2}{11}
  \significado[条,艘,只]{s.}{barco; navio}
\end{verbete}

\begin{verbete}{床}{chuang2}{7}
  \significado{p.c.}{para camas}
  \significado[张]{s.}{cama}
\end{verbete}

\begin{verbete}{春天}{chun1tian1}{9;4}
  \significado[个]{p.t./s.}{primavera}
\end{verbete}

\begin{verbete}{绰号}{chuo4hao4}{11;5}
  \significado{s.}{apelido}
\end{verbete}

\begin{verbete}{词典}{ci2dian3}{7;8}
  \significado[部,本]{s.}{dicionário}
  \veja{字典}{zi4dian3}
\end{verbete}

\begin{verbete}{辞典}{ci2dian3}{13;8}
  \variante{词典}{ci2dian3}
\end{verbete}

\begin{verbete}{磁带}{ci2dai4}{14;9}
  \significado[盘,盒]{s.}{cassete; fita magnética}
\end{verbete}

\begin{verbete}{磁盘}{ci2pan2}{14;11}
  \significado{s.}{disquete}
\end{verbete}

\begin{verbete}{次}{ci4}{6}
  \significado{p.c.}{para frequência (número de vezes)}
\end{verbete}

\begin{verbete}{葱}{cong1}{12}
  \significado{s.}{cebolinha}
\end{verbete}

\begin{verbete}{聪慧}{cong1hui4}{15;15}
  \significado{adj.}{inteligente; brilhante}
\end{verbete}

\begin{verbete}{聪明}{cong1ming5}{15;8}
  \significado{adj.}{inteligente; brilhante; esperto}
\end{verbete}

\begin{verbete}{从}{cong2}{4}
  \significado{prep.}{de; desde; a partir de}
  \significado*{s.}{sobrenome Cong}
\end{verbete}

\begin{verbete}{粗心}{cu1xin1}{11;4}
  \significado{adj.}{descuido}
\end{verbete}

\begin{verbete}{粗心地做}{cu1xin1·di4·zuo4}{11;4;6;11}
  \significado{adj.}{feito descuidadamente}
\end{verbete}

\begin{verbete}{酢}{cu4}{12}
  \variante{醋}{cu4}
\end{verbete}

\begin{verbete}{醋}{cu4}{15}
  \significado{s.}{vinagre}
\end{verbete}

\begin{verbete}{窾}{cuan4}{17}
  \significado{v.}{esconder}
  \veja{窾}{kuan3}
\end{verbete}

\begin{verbete}{错}{cuo4}{13}
  \significado{adj.}{errado; enganado}
  \significado*{s.}{sobrenome Cuo}
\end{verbete}

%%%%% EOF %%%%%

%%%
%%% D
%%%
%\section*{D}
\addcontentsline{toc}{section}{D}

\begin{verbete}{搭配}{da1pei4}{12;10}
  \significado{v.}{emparelhar; combinar; organizar em pares; adicionar alguém em um grupo}
\end{verbete}

\begin{verbete}{搭讪}{da1shan4}{12;5}
  \significado{v.}{bater em alguém; incitar uma conversa; começar a conversar para acabar com um silêncio constrangedor ou uma situação embaraçosa}
\end{verbete}

\begin{verbete}{打}{da2}{5}[Radical 手][Componentes ⺘丁]
  \significado{s.}{(empréstimo linguístico) dúzia}
  \veja{打}{da3}
\end{verbete}

\begin{verbete}{答案}{da2'an4}{12;10}
  \significado[个]{s.}{resposta; solução}
\end{verbete}

\begin{verbete}{打}{da3}{5}[Radical 手][Componentes ⺘丁]
  \significado{adv.}{desde}
  \significado{v.}{jogar (um jogo); bater; atacar; acertar; quebrar; digitar; misturar; construir; lutar; pegar; fazer; amarrar; atirar; calcular}
  \veja{打}{da2}
\end{verbete}

\begin{verbete}{打扮}{da3ban5}{5;7}
  \significado{v.}{arranjar-se; enfeitar-se}
\end{verbete}

\begin{verbete}{打电话}{da3dian4hua4}{5;5;8}
  \significado{v.}{ligar; dar um telefonema}
  \veja{给……打电话}{gei3 da3dian4hua4}
\end{verbete}

\begin{verbete}{打工}{da3gong1}{5;3}
  \significado{v.}{(para alunos) ter um emprego fora do horário de aula ou durante as férias; trabalhar em um emprego temporá rio ou casual}
\end{verbete}

\begin{verbete}{打工人}{da3gong1ren2}{5;3;2}
  \significado{s.}{trabalhador}
\end{verbete}

\begin{verbete}{打搅}{da3jiao3}{5;12}
  \significado{v.}{perturbar; incomodar}
\end{verbete}

\begin{verbete}{打结}{da3jie2}{5;9}
  \significado{v.}{dar um nó; amarrar}
\end{verbete}

\begin{verbete}{打瞌睡}{da3ke1shui4}{5;15;13}
  \significado{v.}{cochilar}
\end{verbete}

\begin{verbete}{打猎}{da3lie4}{5;11}
  \significado{v.}{ir caçar}
\end{verbete}

\begin{verbete}{打骂}{da3ma4}{5;9}
  \significado{v.}{bater e repreender}
\end{verbete}

\begin{verbete}{打磨}{da3mo2}{5;16}
  \significado{v.}{polir; fazer brilhar}
\end{verbete}

\begin{verbete}{打屁股}{da3pi4gu5}{5;7;8}
  \significado{v.}{dar um tapa no bumbum de alguém}
\end{verbete}

\begin{verbete}{打球}{da3qiu2}{5;11}
  \significado{v.}{jogar bola; jogar (futebol, basquetebol, handbol, etc.)}
\end{verbete}

\begin{verbete}{打扰}{da3rao3}{5;7}
  \significado{v.}{perturbar; incomodar}
\end{verbete}

\begin{verbete}{打算}{da3suan4}{5;14}
  \significado[个]{s.}{plano; intenção}
  \significado{v.}{pensar; planejar; pretender}
\end{verbete}

\begin{verbete}{打压}{da3ya1}{5;6}
  \significado{v.}{reprimir; derrotar}
\end{verbete}

\begin{verbete}{打针}{da3zhen1}{5;7}
  \significado{v.+compl.}{dar injeção; levar injeção}
\end{verbete}

\begin{verbete}{大}{da4}{3}[Radical ⼈][Componentes ⼈⼀][Kangxi 37]
  \significado{adj.}{grande}
  \veja{大}{dai4}
\end{verbete}

\begin{verbete}{大胆}{da4dan3}{3;9}
  \significado{adj.}{audacioso; ousado; destemido}
\end{verbete}

\begin{verbete}{大豆}{da4dou4}{3;7}
  \significado{s.}{soja}
\end{verbete}

\begin{verbete}{大夫}{da4fu1}{3;4}
  \significado{s.}{oficial sênior (na China Imperial)}
  \veja{大夫}{dai4fu5}
\end{verbete}

\begin{verbete}{大概}{da4gai4}{3;13}
  \significado{adv.}{aproximadamente; por volta de}
\end{verbete}

\begin{verbete}{大规模}{da4gui1mo2}{3;8;14}
  \significado{adj.}{em grande escala; em larga escala; extenso}
\end{verbete}

\begin{verbete}{大海}{da4hai3}{3;10}
  \significado{s.}{mar; oceano}
\end{verbete}

\begin{verbete}{大后天}{da4hou4tian1}{3;6;4}
  \significado{p.t.}{daqui a três dias}
\end{verbete}

\begin{verbete}{大家}{da4jia1}{3;10}
  \significado{pron.}{todos}
\end{verbete}

\begin{verbete}{大口}{da4kou3}{3;3}
  \significado{s.}{grande bocado (de comida, bebida, fumo, etc.)}
\end{verbete}

\begin{verbete}{大马}{da4ma3}{3;3}
  \significado*{s.}{Malásia}
\end{verbete}

\begin{verbete}{大前天}{da4qian2tian1}{3;9;4}
  \significado{p.t.}{três dias atrás}
\end{verbete}

\begin{verbete}{大全}{da4quan2}{3;6}
  \significado{s.}{coleção abrangente}
\end{verbete}

\begin{verbete}{大人}{da4ren5}{3;2}
  \significado{s.}{adulto}
\end{verbete}

\begin{verbete}{大赛}{da4sai4}{3;14}
  \significado{s.}{grande concurso, competição}
\end{verbete}

\begin{verbete}{大神}{da4shen2}{3;9}
  \significado{s.}{deidade; (gíria da Internet) guru; \emph{expert}; gênio}
\end{verbete}

\begin{verbete}{大蒜}{da4suan4}{3;13}
  \significado[瓣,头]{s.}{alho}
\end{verbete}

\begin{verbete}{大腿}{da4tui3}{3;13}
  \significado{s.}{coxa}
\end{verbete}

\begin{verbete}{大戏}{da4xi4}{3;6}
  \significado*{s.}{Drama, Ópera Chinesa}
\end{verbete}

\begin{verbete}{大猩猩}{da4xing1xing5}{3;12;12}
  \significado{s.}{gorila}
\end{verbete}

\begin{verbete}{大学}{da4xue2}{3;8}
  \significado[所]{s.}{faculdade; universidade}
\end{verbete}

\begin{verbete}{大洋洲}{da4yang2zhou1}{3;9;9}
  \significado*{s.}{Oceania}
\end{verbete}

\begin{verbete}{大雨}{da4yu3}{3;8}
  \significado[场]{s.}{chuva pesada, forte}
\end{verbete}

\begin{verbete}{大约}{da4yue1}{3;6}
  \significado{adv.}{aproximadamente; provavelmente}
\end{verbete}

\begin{verbete}{大战}{da4zhan4}{3;9}
  \significado{s.}{guerra}
  \significado{v.}{guerrear; lutar em uma guerra}
\end{verbete}

\begin{verbete}{歹徒}{dai3tu2}{4;10}
  \significado{s.}{malfeitor; gangster; bandido}
\end{verbete}

\begin{verbete}{逮}{dai3}{11}[Radical 辵][Componentes ⻌隶]
  \significado{v.}{(coloquial) pegar, aproveitar, capturar}
  \veja{逮}{dai4}
\end{verbete}

\begin{verbete}{大}{dai4}{3}[Radical ⼈][Componentes ⼈⼀]
  \veja{大}{da4}
  \veja{大夫}{dai4fu5}
\end{verbete}

\begin{verbete}{大夫}{dai4fu5}{3;4}
  \significado{s.}{médico, doutor}
  \veja{大夫}{da4fu1}
\end{verbete}

\begin{verbete}{代表团}{dai4biao3tuan2}{5;8;6}
  \significado[个]{s.}{delegação}
\end{verbete}

\begin{verbete}{代称}{dai4cheng1}{5;10}
  \significado{s.}{nome alternativo; antonomásia}
  \significado{v.}{referir-se a algo ou alguém por outro nome}
\end{verbete}

\begin{verbete}{代价}{dai4jia4}{5;6}
  \significado{s.}{preço; custo}
\end{verbete}

\begin{verbete}{代言}{dai4yan2}{5;7}
  \significado{v.}{ser um porta-voz; ser um embaixador (para uma marca); endossar}
\end{verbete}

\begin{verbete}{带}{dai4}{9}[Radical ⼱][Componentes ⼍卅⼱]
  \significado{v.}{levar; trazer}
\end{verbete}

\begin{verbete}{带来}{dai4lai2}{9;7}
  \significado{v.}{trazer; (fig.) provocar, produzir}
\end{verbete}

\begin{verbete}{逮}{dai4}{11}[Radical 辵][Componentes ⻌隶]
  \significado{v.}{(literário) alcançar; usado em 逮捕}
  \veja{逮}{dai3}
  \veja{逮捕}{dai4bu3}
\end{verbete}

\begin{verbete}{逮捕}{dai4bu3}{11;10}
  \significado{v.}{prender, apreender, levar sob custódia}
\end{verbete}

\begin{verbete}{戴}{dai4}{17}[Radical ⼽][Componentes 𢦏異]
  \significado*{s.}{sobrenome Dai}
  \significado[条]{s.}{área; cinturão; região; zona}
  \significado{v.}{usar/vestir (óculos, gravata, relógio de pulso, luvas); trazer}
\end{verbete}

\begin{verbete}{单}{dan1}{8}[Radical 十][Componentes 一丷甲]
  \significado{adj.}{solteiro; único}
  \significado{adv.}{apenas}
  \significado[个]{s.}{conta; lista; formulário; número ímpar}
  \veja{单}{chan2}
  \veja{单}{shan4}
\end{verbete}

\begin{verbete}{单调}{dan1diao4}{8;10}
  \significado{adj.}{monótono}
\end{verbete}

\begin{verbete}{单脚滑行车}{dan1jiao3hua2xing2che1}{8;11;12;6;4}
  \significado{s.}{\emph{scooter}}
\end{verbete}

\begin{verbete}{单质}{dan1zhi4}{8;8}
  \significado{s.}{substância simples (consistindo puramente de um elemento, como diamante, grafite, etc.)}
\end{verbete}

\begin{verbete}{担心}{dan1xin1}{8;4}
  \significado{v.}{preocupar-se; estar preocupado}
\end{verbete}

\begin{verbete}{耽心}{dan1xin1}{10;4}
  \variante{担心}
\end{verbete}

\begin{verbete}{胆小鬼}{dan3xiao3gui3}{9;3;9}
  \significado{adj.}{covarde; medroso}
\end{verbete}

\begin{verbete}{但}{dan4}{7}[Radical 人][Componentes ⺅旦]
  \significado{conj.}{mas; ainda; no entanto; apenas}
\end{verbete}

\begin{verbete}{但是}{dan4shi4}{7;9}
  \significado{conj.}{mas; ainda; no entanto}
\end{verbete}

\begin{verbete}{蛋}{dan4}{11}[Radical 足][Componentes ⽦足]
  \significado[个,打]{s.}{ovo; objeto de formato oval}
\end{verbete}

\begin{verbete}{蛋糕}{dan4gao1}{11;16}
  \significado[块,个]{s.}{bolo}
\end{verbete}

\begin{verbete}{当初}{dang1chu1}{6;7}
  \significado{adv.}{naquela hora; originalmente}
\end{verbete}

\begin{verbete}{当然}{dang1ran2}{6;12}
  \significado{adv.}{claro; certamente}
\end{verbete}

\begin{verbete}{挡风玻璃}{dang3feng1bo1li5}{9;4;9;14}
  \significado{s.}{parabrisa}
\end{verbete}

\begin{verbete}{刀}{dao1}{2}[Radical ⼑][Componentes ㇆丿][Kangxi 18]
  \significado*{s.}{sobrenome Dao}
  \significado{p.c.}{para cortes de faca ou facadas}
  \significado[把]{s.}{faca; lâmina;  espada de fio único; cutelo; (gíria) dólar (empréstimo linguístico)}
\end{verbete}

\begin{verbete}{导弹}{dao3dan4}{6;11}
  \significado[枚]{s.}{míssil (guiado)}
\end{verbete}

\begin{verbete}{倒}{dao3}{10}[Radical 人][Componentes ⺅到]
  \significado{v.}{cair no chão; deitar-se no chão; colapsar; ir à falência}
  \veja{倒}{dao4}
\end{verbete}

\begin{verbete}{倒地}{dao3di4}{10;6}
  \significado{v.}{cair no chão}
\end{verbete}

\begin{verbete}{倒楣}{dao3mei2}{10;13}
  \variante{倒霉}
\end{verbete}

\begin{verbete}{倒霉}{dao3mei2}{10;15}
  \significado{adj.}{azarado}
  \significado{s.}{azar; má sorte}
  \significado{v.}{estar sem sorte; ter azar}
\end{verbete}

\begin{verbete}{倒血霉}{dao3xue4mei2}{10;6;15}
  \significado{v.}{ter muito azar (versão mais forte de 倒霉)}
  \veja{倒霉}{dao3mei2}
\end{verbete}

\begin{verbete}{到}{dao4}{8}[Radical 刀][Componentes 至⺉]
  \significado{prep.}{a; até; para}
  \significado{v.}{chegar}
\end{verbete}

\begin{verbete}{到处}{dao4chu4}{8;5}
  \significado{adv.}{em todos os lugares}
\end{verbete}

\begin{verbete}{到底}{dao4di3}{8;8}
  \significado{adv.}{na verdade; exatamente; são ou não são; afinal; no final; no final das contas; finalmente; quando tudo estiver dito e feito}
\end{verbete}

\begin{verbete}{倒}{dao4}{10}[Radical 人][Componentes ⺅到]
  \significado{adv.}{ao contrário da expectativa; ao contrário}
  \significado{v.}{inverter; colocar de cabeça para baixo ou de frente para trás; derramar; tombar}
  \veja{倒}{dao3}
\end{verbete}

\begin{verbete}{道理}{dao4li5}{12;11}
  \significado[个]{s.}{razão; argumento; sentido; princípio; base; justificativa}
\end{verbete}

\begin{verbete}{得}{de2}{11}[Radical 彳][Componentes 彳㝵]
  \significado{v.}{obter; ganhar; pegar (uma doença)}
  \veja{得}{de5}
  \veja{得}{dei3}
\end{verbete}

\begin{verbete}{得到}{de2dao4}{11;8}
  \significado{v.}{obter; receber}
\end{verbete}

\begin{verbete}{得了}{de2le5}{11;2}
  \significado{expr.}{Tudo bem!; É o bastante!}
  \veja{得了}{de2liao3}
\end{verbete}

\begin{verbete}{得了}{de2liao3}{11;2}
  \significado{adj.}{(enfaticamente, em perguntas retóricas) possível}
  \veja{得了}{de2le5}
\end{verbete}

\begin{verbete}{得意}{de2yi4}{11;13}
  \significado{adj.}{orgulhoso de si mesmo; satisfeito consigo mesmo; complacente}
\end{verbete}

\begin{verbete}{德}{de2}{15}[Radical 彳][Componentes 彳𢛳]
  \significado*{s.}{Alemanha, abreviação de~德国}
  \significado{s.}{virtude; bondade; moralidade; ética; personagem; tipo}
  \veja{德国}{de2guo2}
\end{verbete}

\begin{verbete}{德国}{de2guo2}{15;8}
  \significado*{s.}{Alemanha}
  \veja{德}{de2}
\end{verbete}

\begin{verbete}{德国人}{de2guo2ren2}{15;8;2}
  \significado{s.}{alemão; pessoa nascida na Alemanha}
\end{verbete}

\begin{verbete}{地}{de5}{6}[Radical 土][Componentes 土也]
  \significado{part.}{estrutural: utilizada antes de um verbo ou adjetivo, ligando-o ao adjunto adverbial modificador precedente}
  \veja{地}{di4}
\end{verbete}

\begin{verbete}{的}{de5}{8}[Radical 白][Componentes 白勺]
  \significado{part.}{utilizada em possessivos; utilizada entre adjetivos e substantivos (opcional se o adjetivo possui apenas um carácter); utilizada após um atributo; utilizada no final de uma frase declarativa para dar ênfase; para formar uma expressão nominal}
  \veja{的}{di1}
  \veja{的}{di2}
  \veja{的}{di4}
\end{verbete}

\begin{verbete}{得}{de5}{11}[Radical 彳][Componentes 彳㝵]
  \significado{part.}{estrutural:~ligando um verbo à frase seguinte indicando efeito, grau, possibilidade, etc.}
  \veja{得}{de2}
  \veja{得}{dei3}
\end{verbete}

\begin{verbete}{得}{dei3}{11}[Radical 彳][Componentes 彳㝵]
  \significado{v.}{haver de; ter de}
  \veja{得}{de2}
  \veja{得}{de5}
\end{verbete}

\begin{verbete}{灯}{deng1}{6}[Radical 火][Componentes 火丁]
  \significado[盏]{s.}{lâmpada; lanterna; luz}
\end{verbete}

\begin{verbete}{灯标}{deng1biao1}{6;9}
  \significado{s.}{farol; luz de farol}
\end{verbete}

\begin{verbete}{灯号}{deng1hao4}{6;5}
  \significado{s.}{sinal luminoso; luz indicadora}
\end{verbete}

\begin{verbete}{灯泡}{deng1pao4}{6;8}
  \significado[个]{s.}{lâmpada; terceiro indesejado estragando encontro de casal (gíria)}
  \veja{电灯泡}{dian4deng1pao4}
\end{verbete}

\begin{verbete}{灯丝}{deng1si1}{6;5}
  \significado{s.}{filamento (de uma lâmpada)}
\end{verbete}

\begin{verbete}{登}{deng1}{12}[Radical 癶][Componentes 癶豆]
  \significado{v.}{subir (montanha, cume)}
\end{verbete}

\begin{verbete}{等}{deng3}{12}[Radical 竹][Componentes ⺮寺]
  \significado{v.}{esperar; esperar por}
\end{verbete}

\begin{verbete}{等待}{deng3dai4}{12;9}
  \significado{v.}{esperar; esperar por}
\end{verbete}

\begin{verbete}{低}{di1}{7}[Radical 人][Componentes ⺅氐]
  \significado{adj.}{baixo}
  \significado{adv.}{abaixo}
  \significado{v.}{abaixar (a cabeça); deixar cair; pendurar; inclinar}
\end{verbete}

\begin{verbete}{的}{di1}{8}[Radical 白][Componentes 白勺]
  \significado{s.}{um táxi (abreviação de 的士)}
  \veja{的}{de5}
  \veja{的士}{di1shi4}
  \veja{的}{di2}
  \veja{的}{di4}
\end{verbete}

\begin{verbete}{的士}{di1shi4}{8;3}
  \significado{s.}{táxi (empréstimo linguístico)}
\end{verbete}

\begin{verbete}{堤坝}{di1ba4}{12;7}
  \significado{s.}{represa; dique; barragem}
\end{verbete}

\begin{verbete}{滴}{di1}{14}[Radical 水][Componentes ⺡啇]
  \significado{s.}{uma gota}
  \significado{v.}{pingar}
\end{verbete}

\begin{verbete}{的}{di2}{8}[Radical 白][Componentes 白勺]
  \significado{adv.}{realmente e verdadeiramente}
  \veja{的}{de5}
  \veja{的}{di1}
  \veja{的}{di4}
\end{verbete}

\begin{verbete}{的确}{di2que4}{8;12}
  \significado{adv.}{de fato; realmente}
\end{verbete}

\begin{verbete}{笛}{di2}{11}[Radical 竹][Componentes ⺮由]
  \significado{s.}{flauta}
\end{verbete}

\begin{verbete}{底气}{di3qi4}{8;4}
  \significado{s.}{capacidade pulmonar; ousadia, confiança, autoconfiança, vigor}
\end{verbete}

\begin{verbete}{抵抗}{di3kang4}{8;7}
  \significado{s.}{resistência}
  \significado{v.}{resistir}
\end{verbete}

\begin{verbete}{地}{di4}{6}[Radical 土][Componentes 土也]
  \significado[个,片]{s.}{mundo; campo; chão; terra; lugar}
  \veja{地}{de5}
\end{verbete}

\begin{verbete}{地点}{di4dian3}{6;9}
  \significado[个]{s.}{localização; lugar; local}
\end{verbete}

\begin{verbete}{地方}{di4fang1}{6;4}
  \significado{s.}{região; regional (longe da administração central); local}
  \veja{地方}{di4fang5}
\end{verbete}

\begin{verbete}{地方}{di4fang5}{6;4}
  \significado[处,个,块]{s.}{lugar; local; território}
  \veja{地方}{di4fang1}
\end{verbete}

\begin{verbete}{地核}{di4he2}{6;10}
  \significado{s.}{geologia:~núcleo da Terra}
\end{verbete}

\begin{verbete}{地理}{di4li3}{6;11}
  \significado{s.}{geografia}
\end{verbete}

\begin{verbete}{地球}{di4qiu2}{6;11}
  \significado{s.}{o planeta terra}
\end{verbete}

\begin{verbete}{地区}{di4qu1}{6;4}
  \significado{adj.}{regional}
  \significado{part.}{como sufixo do nome da cidade, significa prefeitura ou condado}
  \significado[个]{s.}{área; distrito (não necessariamente unidade administrativa formal); local; região}
\end{verbete}

\begin{verbete}{地铁}{di4tie3}{6;10}
  \significado{s.}{metrô; metropolitano}
\end{verbete}

\begin{verbete}{地图}{di4tu2}{6;8}
  \significado[张,本]{s.}{mapa}
\end{verbete}

\begin{verbete}{地下室}{di4xia4shi4}{6;3;9}
  \significado{s.}{subterrâneo; porão}
\end{verbete}

\begin{verbete}{地狱}{di4yu4}{6;9}
  \significado*{s.}{\emph{Naraka} (Budismo)}
  \significado{adj.}{infernal}
  \significado{s.}{inferno; submundo}
\end{verbete}

\begin{verbete}{地震}{di4zhen4}{6;15}
  \significado{s.}{terremoto; tremor de terra}
\end{verbete}

\begin{verbete}{地址}{di4zhi3}{6;7}
  \significado[个]{s.}{endereço}
\end{verbete}

\begin{verbete}{地砖}{di4zhuan1}{6;9}
  \significado{s.}{ladrilho de piso}
\end{verbete}

\begin{verbete}{弟}{di4}{7}[Radical 弓][Componentes 丷弓]
  \significado{s.}{irmão mais novo; júnior}
\end{verbete}

\begin{verbete}{弟弟}{di4di5}{7;7}
  \significado[个,位]{s.}{irmão mais novo}
\end{verbete}

\begin{verbete}{弟妹}{di4mei4}{7;8}
  \significado{s.}{esposa do irmão mais novo}
\end{verbete}

\begin{verbete}{的}{di4}{8}[Radical 白][Componentes 白勺]
  \significado{s.}{alvo; mosca (centro do alvo)}
  \veja{的}{de5}
  \veja{的}{di1}
  \veja{的}{di2}
\end{verbete}

\begin{verbete}{帝国}{di4guo2}{9;8}
  \significado{adj.}{imperial}
  \significado{s.}{império}
\end{verbete}

\begin{verbete}{第}{di4}{11}[Radical 竹][Componentes ⺮弟]
  \significado{num.}{prefixo para expressar números ordinais}
\end{verbete}

\begin{verbete}{墬}{di4}{14}
  \variante{地}
\end{verbete}

\begin{verbete}{点}{dian3}{9}[Radical 火][Componentes 占⺣]
  \significado{p.c.}{para itens; hora cheia}
  \significado{s.}{ponto; ponto (no espaço ou no tempo); gota; partícula}
\end{verbete}

\begin{verbete}{点火}{dian3huo3}{9;4}
  \significado{s.}{ignição}
  \significado{v.}{inflamar; acender um fogo; agitar; dar partida em um motor; (fig.) provocar problemas}
\end{verbete}

\begin{verbete}{点名}{dian3ming2}{9;6}
  \significado{v.}{mencionar alguém pelo nome; chamar, louvar ou criticar alguém pelo nome}
\end{verbete}

\begin{verbete}{点燃}{dian3ran2}{9;16}
  \significado{v.}{inflamar; incendiar}
\end{verbete}

\begin{verbete}{电冰箱}{dian4bing1xiang1}{5;6;15}
  \significado[台]{s.}{frigorífico; refrigerador}
\end{verbete}

\begin{verbete}{电车司机}{dian4che1 si1ji1}{5;4;5;6}
  \significado{s.}{motorista de bonde}
\end{verbete}

\begin{verbete}{电灯泡}{dian4deng1pao4}{5;6;8}
  \significado{s.}{lâmpada elétrica; (gíria) terceiro convidado indesejado}
\end{verbete}

\begin{verbete}{电动}{dian4dong4}{5;6}
  \significado{adj.}{movido a eletricidade; elétrico}
\end{verbete}

\begin{verbete}{电动车}{dian4dong4che1}{5;6;4}
  \significado{s.}{veículo elétrico (\emph{scooter}, bicicleta, carro, etc.)}
\end{verbete}

\begin{verbete}{电话}{dian4hua4}{5;8}
  \significado[部]{s.}{telefone}
  \significado[通]{s.}{chamada telefônica}
\end{verbete}

\begin{verbete}{电脑}{dian4nao3}{5;10}
  \significado[台]{s.}{computador}
\end{verbete}

\begin{verbete}{电脑语言}{dian4nao3yu3yan2}{5;10;9;7}
  \significado{s.}{linguagem de programação; linguagem de computador}
\end{verbete}

\begin{verbete}{电器}{dian4qi4}{5;16}
  \significado{s.}{aparelho elétrico}
\end{verbete}

\begin{verbete}{电视}{dian4shi4}{5;8}
  \significado[台,个]{s.}{televisão; TV; televisor}
\end{verbete}

\begin{verbete}{电视机}{dian4shi4ji1}{5;8;6}
  \significado[台]{s.}{aparelho de televisão; televisor}
\end{verbete}

\begin{verbete}{电梯}{dian4ti1}{5;11}
  \significado[台,部]{s.}{elevador; ascensor}
\end{verbete}

\begin{verbete}{电梯司机}{dian4ti1 si1ji1}{5;11;5;6}
  \significado{s.}{ascensorista}
\end{verbete}

\begin{verbete}{电影}{dian4ying3}{5;15}
  \significado[部,片,幕,场]{s.}{cinema; filme}
\end{verbete}

\begin{verbete}{电影奖}{dian4ying3jiang3}{5;15;9}
  \significado{s.}{premiações de cinema}
\end{verbete}

\begin{verbete}{电影节}{dian4ying3jie2}{5;15;5}
  \significado{s.}{festival de cinema}
\end{verbete}

\begin{verbete}{电影界}{dian4ying3jie4}{5;15;9}
  \significado{s.}{indústria cinematográfica}
\end{verbete}

\begin{verbete}{电影票}{dian4ying3piao4}{5;15;11}
  \significado{s.}{ingresso de filme}
\end{verbete}

\begin{verbete}{电影术}{dian4ying3 shu4}{5;15;5}
  \significado{s.}{cinematografia}
\end{verbete}

\begin{verbete}{电影艺术}{dian4ying3 yi4shu4}{5;15;4;5}
  \significado{s.}{arte cinematográfica}
\end{verbete}

\begin{verbete}{电影音乐}{dian4ying3 yin1yue4}{5;15;9;5}
  \significado{s.}{música cinematográfica}
\end{verbete}

\begin{verbete}{电影院}{dian4ying3yuan4}{5;15;9}
  \significado[次,家,座]{s.}{sala de cinema}
\end{verbete}

\begin{verbete}{电邮}{dian4you2}{5;7}
  \significado{s.}{correio eletrônico, \emph{e-mail}; abreviação de~电子邮件}
  \veja{电子邮件}{dian4zi3you2jian4}
\end{verbete}

\begin{verbete}{电子}{dian4zi3}{5;3}
  \significado{s.}{eletrônico; elétron}
\end{verbete}

\begin{verbete}{电子名片}{dian4zi3 ming2pian4}{5;3;6;4}
  \significado{s.}{cartão de visita eletrônico}
\end{verbete}

\begin{verbete}{电子邮件}{dian4zi3you2jian4}{5;3;7;6}
  \significado[封,份]{s.}{correio eletrônico, \emph{e-mail}}
  \veja{电邮}{dian4you2}
\end{verbete}

\begin{verbete}{店员}{dian4yuan2}{8;7}
  \significado{s.}{assistente de loja; balconista; vendedor}
\end{verbete}

\begin{verbete}{店主}{dian4zhu3}{8;5}
  \significado{s.}{lojista; dono de loja}
\end{verbete}

\begin{verbete}{垫子}{dian4zi5}{9;3}
  \significado{s.}{colchão, esteira, almofada}
\end{verbete}

\begin{verbete}{钿}{dian4}{10}[Radical 金][Componentes ⻐田]
  \significado{s.}{ornamento incrustado antigo em forma de flor}
  \significado{v.}{incrustar com ouro, prata, etc.}
  \veja{钿}{tian2}
\end{verbete}

\begin{verbete}{淀}{dian4}{11}[Radical 水][Componentes ⺡定]
  \significado{adj.}{pantanoso}
  \significado{s.}{lago raso; pântano}
  \significado{v.}{formar sedimentos; precipitar}
\end{verbete}

\begin{verbete}{貂}{diao1}{12}[Radical 豸][Componentes 豸召]
  \significado{s.}{marta; fuinha}
\end{verbete}

\begin{verbete}{雕刻}{diao1ke4}{16;8}
  \significado{s.}{escultura}
  \significado{v.}{esculpir; gravar}
\end{verbete}

\begin{verbete}{屌丝}{diao3si1}{9;5}
  \significado{adj.}{panaca; zé-ninguém; gíria de \emph{Internet}:~\emph{looser}}
\end{verbete}

\begin{verbete}{钓鱼}{diao4yu2}{8;8}
  \significado{v.}{pescar (com linha e anzol); (fig.) aprisionar}
\end{verbete}

\begin{verbete}{掉}{diao4}{11}[Radical 手][Componentes ⺘卓]
  \significado{v.}{cair; deixar cair}
\end{verbete}

\begin{verbete}{掉包}{diao4bao1}{11;5}
  \significado{v.}{vender uma falsificação pelo artigo genuíno; roubar o item valioso de alguém e substituí-lo por um item de aparência semelhante, mas sem valor}
\end{verbete}

\begin{verbete}{掉膘}{diao4biao1}{11;15}
  \significado{v.}{perder peso (gado)}
\end{verbete}

\begin{verbete}{掉队}{diao4dui4}{11;4}
  \significado{v.}{abandonar; ficar para trás}
\end{verbete}

\begin{verbete}{掉落}{diao4luo4}{11;12}
  \significado{v.}{derrubar}
\end{verbete}

\begin{verbete}{掉线}{diao4xian4}{11;8}
  \significado{v.}{desconectar-se (da \emph{Internet})}
\end{verbete}

\begin{verbete}{掉转}{diao4zhuan3}{11;8}
  \significado{v.}{dar a volta}
\end{verbete}

\begin{verbete}{叮嘱}{ding1zhu3}{5;15}
  \significado{v.}{exortar; avisar; insistir de novo e de novo}
\end{verbete}

\begin{verbete}{顶}{ding3}{8}[Radical 頁][Componentes 丁⻚]
  \significado{adv.}{mais; extremamente; melhor; muito (linguagem falada)}
\end{verbete}

\begin{verbete}{丢}{diu1}{6}[Radical 丿][Componentes 王厶]
  \significado{v.}{perder; perder-se}
\end{verbete}

\begin{verbete}{丢掉}{diu1diao4}{6;11}
  \significado{v.}{jogar fora; descartar; perder}
\end{verbete}

\begin{verbete}{丢官}{diu1guan1}{6;8}
  \significado{v.}{perder um cargo oficial}
\end{verbete}

\begin{verbete}{丢开}{diu1kai1}{6;4}
  \significado{v.}{jogar fora ou deixar de lado; esquecer por um tempo}
\end{verbete}

\begin{verbete}{丢脸}{diu1lian3}{6;11}
  \significado{adj.}{vergonhoso}
\end{verbete}

\begin{verbete}{丢弃}{diu1qi4}{6;7}
  \significado{v.}{jogar fora; descartar}
\end{verbete}

\begin{verbete}{丢失}{diu1shi1}{6;5}
  \significado{v.}{perder}
\end{verbete}

\begin{verbete}{丢下}{diu1xia4}{6;3}
  \significado{v.}{abandonar}
\end{verbete}

\begin{verbete}{东}{dong1}{5}[Radical ⼀][Componentes 七小]
  \significado*{s.}{sobrenome Dong}
  \significado{s.}{leste}
\end{verbete}

\begin{verbete}{东半球}{dong1ban4qiu2}{5;5;11}
  \significado*{s.}{Hemisfério Oriental}
\end{verbete}

\begin{verbete}{东北}{dong1bei3}{5;5}
  \significado*{s.}{Nordeste da China; Manchúria}
  \significado{p.l.}{nordeste}
\end{verbete}

\begin{verbete}{东边}{dong1bian5}{5;5}
  \significado{p.l.}{este; leste; lado leste; oriente}
\end{verbete}

\begin{verbete}{东部}{dong1bu4}{5;10}
  \significado{p.l.}{leste; oriente}
\end{verbete}

\begin{verbete}{东方}{dong1fang1}{5;4}
  \significado*{s.}{sobrenome Dongfang}
  \significado{p.l.}{leste; oriente}
\end{verbete}

\begin{verbete}{东方学院}{dong1fang1 xue2yuan4}{5;4;8;9}
  \significado*{s.}{Instituto Oriental}
\end{verbete}

\begin{verbete}{东面}{dong1mian4}{5;9}
  \significado{p.l.}{lado leste (de algo)}
\end{verbete}

\begin{verbete}{东西}{dong1xi1}{5;6}
  \significado{s.}{leste e oeste}
  \veja{东西}{dong1xi5}
\end{verbete}

\begin{verbete}{东西}{dong1xi5}{5;6}
  \significado[个,件]{s.}{coisa; materia; pessoa}
  \veja{东西}{dong1xi1}
\end{verbete}

\begin{verbete}{冬瓜}{dong1gua1}{5;5}
  \significado{s.}{melão de inverno}
\end{verbete}

\begin{verbete}{冬天}{dong1tian1}{5;4}
  \significado{p.t./s.}{inverno}
\end{verbete}

\begin{verbete}{懂}{dong3}{15}[Radical 心][Componentes ⺖董]
  \significado{v.}{compreender; entender}
\end{verbete}

\begin{verbete}{动}{dong4}{6}[Radical 力][Componentes 云力]
  \significado{v.}{mover; movimentar}
\end{verbete}

\begin{verbete}{动感}{dong4gan3}{6;13}
  \significado{adj.}{dinâmica; vívida}
  \significado{adv.}{dinamicamente}
  \significado{s.}{senso de movimento (geralmente em uma obra de arte estática)}
\end{verbete}

\begin{verbete}{动力}{dong4li4}{6;2}
  \significado{s.}{força motriz; força;  (fig.) motivação; ímpeto}
\end{verbete}

\begin{verbete}{动漫}{dong4man4}{6;14}
  \significado{s.}{desenhos animados; quadrinhos; anime; mangá}
\end{verbete}

\begin{verbete}{动物}{dong4wu4}{6;8}
  \significado[只,群,个]{s.}{animal}
\end{verbete}

\begin{verbete}{动物园}{dong4wu4yuan2}{6;8;7}
  \significado[个]{s.}{jardim zoológico; zoo}
\end{verbete}

\begin{verbete}{动作}{dong4zuo4}{6;7}
  \significado[个]{s.}{movimento, ação}
  \significado{v.}{mover, agir}
\end{verbete}

\begin{verbete}{洞穴}{dong4xue2}{9;5}
  \significado{s.}{caverna}
\end{verbete}

\begin{verbete}{都}{dou1}{10}[Radical 邑][Componentes 者⻏]
  \significado{adv.}{todo, todos}
  \veja{都}{du1}
\end{verbete}

\begin{verbete}{豆荚}{dou4jia2}{7;9}
  \significado{s.}{vagem (de legumes)}
\end{verbete}

\begin{verbete}{豆角}{dou4jiao3}{7;7}
  \significado{s.}{feijão verde}
\end{verbete}

\begin{verbete}{读}{dou4}{10}[Radical 言][Componentes ⻈卖]
  \significado{s.}{vírgula; frase marcada por pausa}
  \veja{读}{du2}
\end{verbete}

\begin{verbete}{都}{du1}{10}[Radical 邑][Componentes 者⻏]
  \significado*{s.}{sobrenome Du}
  \significado{s.}{capital; metrópole}
  \veja{都}{dou1}
\end{verbete}

\begin{verbete}{嘟}{du1}{13}[Radical ⼝][Componentes ⼝都]
  \significado{s.}{buzina; bip}
  \significado{v.}{fazer beicinho}
\end{verbete}

\begin{verbete}{毋}{du2}{9}[Radical ⺟][Componentes 龶⺟]
  \significado{adj.}{venenoso; tóxico}
  \significado{s.}{veneno; tóxico}
  \significado{v.}{intoxicar}
\end{verbete}

\begin{verbete}{毒害}{du2hai4}{9;10}
  \significado{s.}{envenenamento}
  \significado{v.}{envenenar (prejudicar com uma substância tóxica); envenenar (as mentes das pessoas)}
\end{verbete}

\begin{verbete}{毒杀}{du2sha1}{9;6}
  \significado{v.}{matar por envenenamento}
\end{verbete}

\begin{verbete}{毒蛇}{du2she2}{9;11}
  \significado{s.}{víbora; cobra venenosa}
\end{verbete}

\begin{verbete}{毒物}{du2wu4}{9;8}
  \significado{s.}{substância venenosa; toxina}
\end{verbete}

\begin{verbete}{独}{du2}{9}[Radical 犬][Componentes ⺨⾍]
  \significado{adj.}{sozinho; solitário; solteiro}
  \significado{adv.}{apenas}
\end{verbete}

\begin{verbete}{独自}{du2zi4}{9;6}
  \significado{adj.}{sozinho}
\end{verbete}

\begin{verbete}{读}{du2}{10}[Radical 言][Componentes ⻈卖]
  \significado{v.}{ler em voz alta; ler; frequentar (escola), estudar (uma matéria na escola); pronunciar}
  \veja{读}{dou4}
\end{verbete}

\begin{verbete}{肚}{du3}{7}[Radical 肉][Componentes ⺼⼟]
  \significado{s.}{tripa; entranhas}
  \veja{肚}{du4}
\end{verbete}

\begin{verbete}{堵车}{du3che1}{11;4}
  \significado{v.}{congestionar (trânsito)}
  \significado{v.+compl.}{congestionamento; engarrafamento (de trânsito)}
\end{verbete}

\begin{verbete}{杜鹃}{du4juan1}{7;12}
  \significado{s.}{cuco (pássaro)}
  \veja{布谷鸟}{bu4gu3niao3}
  \veja{杜鹃鸟}{du4juan1niao3}
  \veja{杜宇}{du4yu3}
\end{verbete}

\begin{verbete}{杜鹃鸟}{du4juan1niao3}{7;12;5}
  \significado{s.}{cuco (pássaro)}
  \veja{布谷鸟}{bu4gu3niao3}
  \veja{杜鹃}{du4juan1}
  \veja{杜宇}{du4yu3}
\end{verbete}

\begin{verbete}{杜宇}{du4yu3}{7;6}
  \significado{s.}{cuco (pássaro)}
  \veja{布谷鸟}{bu4gu3niao3}
  \veja{杜鹃}{du4juan1}
  \veja{杜鹃鸟}{du4juan1niao3}
\end{verbete}

\begin{verbete}{肚}{du4}{7}[Radical 肉][Componentes ⺼⼟]
  \significado{s.}{barriga}
  \veja{肚}{du3}
\end{verbete}

\begin{verbete}{肚子}{du4zi5}{7;3}
  \significado[个]{s.}{abdómen; barriga}
\end{verbete}

\begin{verbete}{度}{du4}{9}[Radical ⼴][Componentes ⼜⼴廿]
  \significado{p.c.}{para temperatura, etc.; para eventos e ocorrências}
  \significado{s.}{grau (ângulo, temperatura, etc.); kilowatt-hora}
  \veja{度}{duo2}
\end{verbete}

\begin{verbete}{渡过}{du4guo4}{12;6}
  \significado{v.}{atravessar; passar por}
\end{verbete}

\begin{verbete}{镀金}{du4jin1}{14;8}
  \significado{v.}{banhar a ouro; dourar; (fig.) fazer algo muito comum parecer especial}
\end{verbete}

\begin{verbete}{端午节}{duan1wu3jie2}{14;4;5}
  \significado*{s.}{Festa do Duplo Cinco, Festival dos Barcos-Dragão (5º~dia do quinto mês lunar)}
\end{verbete}

\begin{verbete}{短}{duan3}{12}[Radical ⽮][Componentes ⽮⾖]
  \significado{adj.}{curto; breve}
\end{verbete}

\begin{verbete}{短处}{duan3chu4}{12;5}
  \significado{s.}{defeito; falta; pontos fracos de alguém; deficiência}
\end{verbete}

\begin{verbete}{短促}{duan3cu4}{12;9}
  \significado{adj.}{curto (tom de voz); fugaz; ofegante (respiração); curto no tempo}
\end{verbete}

\begin{verbete}{短裤}{duan3ku4}{12;12}
  \significado{s.}{calção; shorts}
\end{verbete}

\begin{verbete}{短跑}{duan3pao3}{12;12}
  \significado{s.}{corrida}
\end{verbete}

\begin{verbete}{短期}{duan3qi1}{12;12}
  \significado{s.}{curto prazo}
\end{verbete}

\begin{verbete}{短缺}{duan3que1}{12;10}
  \significado{s.}{escassez}
\end{verbete}

\begin{verbete}{短少}{duan3shao3}{12;4}
  \significado{v.}{estar aquém do valor total}
\end{verbete}

\begin{verbete}{短视}{duan3shi4}{12;8}
  \significado{adj.}{míope}
\end{verbete}

\begin{verbete}{短暂}{duan3zan4}{12;12}
  \significado{adj.}{momentâneo; de curta duração}
\end{verbete}

\begin{verbete}{段}{duan4}{9}[Radical ⽎][Componentes ⽎]
  \significado*{s.}{sobrenome Duan}
  \significado{p.c.}{para histórias, períodos de tempo, desenvolvimento de um tópico, etc.}
  \significado{s.}{parágrafo; seção; segmento; estágio (de um processo)}
\end{verbete}

\begin{verbete}{锻炼}{duan4lian4}{14;9}
  \significado{v.}{fazer exercício físico; praticar esporte}
\end{verbete}

\begin{verbete}{队}{dui4}{4}[Radical 阜][Componentes ⻖⼈]
  \significado[个]{s.}{esquadrão; equipe; grupo}
\end{verbete}

\begin{verbete}{队友}{dui4you3}{4;4}
  \significado{s.}{companheiro de equipe}
\end{verbete}

\begin{verbete}{对}{dui4}{5}[Radical ⼨][Componentes ⼜⼨]
  \significado{adj.}{correto; sim}
  \significado{p.c.}{para casais}
  \significado{prep.}{com; para; para com}
\end{verbete}

\begin{verbete}{对不起}{dui4bu5qi3}{5;4;10}
  \significado{v.}{desculpar; pedir desculpas; perdoar}
\end{verbete}

\begin{verbete}{对得起}{dui4de5qi3}{5;11;10}
  \significado{v.}{não decepcionar alguém; tratar alguém de maneira justa; ser digno de}
\end{verbete}

\begin{verbete}{对……感兴趣}{dui4 gan3xing4qu4}{5;13;6;15}
  \significado{expr.}{estar interessado em\dots; ter interesse em\dots; interessar-se por\dots}
  \veja{对……有兴趣}{dui4 you3xing4qu4}
\end{verbete}

\begin{verbete}{对话}{dui4hua4}{5;8}
  \significado[个]{s.}{diálogo; conversa}
  \significado{v.}{dialogar; conversar}
\end{verbete}

\begin{verbete}{对面}{dui4mian4}{5;9}
  \significado{p.l.}{lado oposto}
\end{verbete}

\begin{verbete}{对手}{dui4shou3}{5;4}
  \significado{s.}{oponente; rival; concorrente; adversário}
\end{verbete}

\begin{verbete}{对……熟悉}{dui4 shu2xi1}{5;15;11}
  \significado{expr.}{estar familiarizado com\dots}
\end{verbete}

\begin{verbete}{对……说}{dui4 shuo5}{5;9}
  \significado{v.}{dizer a alguém}
\end{verbete}

\begin{verbete}{对……有兴趣}{dui4 you3xing4qu4}{5;6;6;15}
  \significado{expr.}{estar interessado em\dots; ter interesse em\dots; interessar-se por\dots}
  \veja{对……感兴趣}{dui4 gan3xing4qu4}
\end{verbete}

\begin{verbete}{蹲下}{dun1xia4}{19;3}
  \significado{v.}{agachar; agachar-se}
\end{verbete}

\begin{verbete}{顿}{dun4}{10}[Radical 頁][Componentes 屯⻚]
  \significado{p.c.}{para refeições, espancamentos, repreensões, etc.:~tempo, luta, feitiço, refeição}
  \significado{v.}{prostrar-se; pausar; bater (o pé)}
\end{verbete}

\begin{verbete}{多}{duo1}{6}[Radical ⼣][Componentes ⼣]
  \significado{adv.}{muito, muitos; em excesso; (prefixo) multi-, poli-}
  \significado{num.}{(após um número) ímpar}
\end{verbete}

\begin{verbete}{多重}{duo1chong2}{6;9}
  \significado{s.}{multi- (facetado, cultural, étnico, etc.)}
\end{verbete}

\begin{verbete}{多大}{duo1da4}{6;3}
  \significado{inter.}{quantos anos?; que idade?; quão grande?}
\end{verbete}

\begin{verbete}{多(么)}{duo1(me5)}{6;3}
  \significado{adv.}{como}
\end{verbete}

\begin{verbete}{多少}{duo1shao3}{6;4}
  \significado{num.}{número; quantidade}
  \veja{多少}{duo1shao5}
\end{verbete}

\begin{verbete}{多少}{duo1shao5}{6;4}
  \significado{inter.}{quanto?, quantos?, (para mais de 10 itens)}
  \veja{多少}{duo1shao3}
\end{verbete}

\begin{verbete}{多云}{duo1yun2}{6;4}
  \significado{adj.}{céu nublado}
\end{verbete}

\begin{verbete}{夺冠}{duo2guan4}{6;9}
  \significado{v.}{apoderar-se da coroa; (fig.) ganhar um campeonato; ganhar a medalha de ouro}
\end{verbete}

\begin{verbete}{度}{duo2}{9}[Radical ⼴][Componentes ⼜⼴廿]
  \significado{v.}{estimar}
  \veja{度}{du4}
\end{verbete}

\begin{verbete}{躲}{duo3}{13}[Radical ⾝][Componentes ⾝朵]
  \significado{v.}{esconder; esquivar; evitar}
\end{verbete}

\begin{verbete}{躲闪}{duo3shan3}{13;5}
  \significado{v.}{desviar; evadir; esquivar (para fora do caminho)}
\end{verbete}

%%%%% EOF %%%%%

%%%
%%% E
%%%
%\section*{E}
\addcontentsline{toc}{section}{E}

\begin{verbete}{阿}{e1}{7}[Radical 阜]
  \significado{adj.}{gracioso}
  \significado{pron.}{monte grande | canto, esquina}
  \significado{v.}{jogar; agradar; atender | ser injustamente parcial com | ser dobrado}
  \veja{阿}{a1}
\end{verbete}

\begin{verbete}{俄}{e2}{9}[Radical 人]
  \significado*{s.}{Rússia; abreviação de~俄罗斯}
  \veja{俄罗斯}{e2luo2si1}
\end{verbete}

\begin{verbete}{俄罗斯}{e2luo2si1}{9,8,12}
  \significado*{s.}{Rússia}
  \veja{俄}{e2}
\end{verbete}

\begin{verbete}{俄罗斯人}{e2luo2si1ren2}{9,8,12,2}
  \significado{s.}{russo; nascido na Rússia}
\end{verbete}

\begin{verbete}{哦}{e2}{10}[Radical ⼝]
  \significado{v.}{entoar cântico}
  \veja{哦}{o2}
  \veja{哦}{o4}
  \veja{哦}{o5}
\end{verbete}

\begin{verbete}{恶心}{e3xin1}{10,4}
  \significado{adj.}{nauseante; repugnante}
  \significado{s.}{enjôo; náusea; repugnância}
  \significado{v.}{envergonhar (deliberadamente); sentir-se doente}
  \veja{恶心}{e4xin1}
\end{verbete}

\begin{verbete}{恶心}{e4xin1}{10,4}
  \significado{s.}{mau hábito; hábito vicioso; vício}
  \veja{恶心}{e3xin1}
\end{verbete}

\begin{verbete}{饿}{e4}{10}[Radical 食]
  \significado{adj.}{faminto}
  \significado{s.}{fome}
  \significado{v.}{morrer de fome}
\end{verbete}

\begin{verbete}{鳄鱼}{e4yu2}{17,8}
  \significado[条]{s.}{jacaré; crocodilo}
\end{verbete}

\begin{verbete}{恩赐}{en1ci4}{10,12}
  \significado{s.}{favor; caridade}
  \significado{v.}{conceder (favor, caridade)}
\end{verbete}

\begin{verbete}{儿}{er2}{2}[Radical ⼉][Kangxi 10]
  \significado{s.}{criança; filho}
  \veja{儿}{r5}
  \veja{儿}{ren2}
\end{verbete}

\begin{verbete}{儿媳}{er2xi2}{2,13}
  \significado{s.}{esposa do filho}
\end{verbete}

\begin{verbete}{儿子}{er2zi5}{2,3}
  \significado{s.}{filho}
\end{verbete}

\begin{verbete}{而}{er2}{6}[Radical ⽽][Kangxi 126]
  \significado{conj.}{e (coordenação); e ainda (restrição)}
\end{verbete}

\begin{verbete}{而况}{er2kuang4}{6,7}
  \significado{conj.}{além disso; além do mais}
\end{verbete}

\begin{verbete}{而且}{er2qie3}{6,5}
  \significado{conj.}{muito menos; além disso; além do mais}
\end{verbete}

\begin{verbete}{耳朵}{er3duo5}{6,6}
  \significado[只,个,对]{s.}{orelha}
\end{verbete}

\begin{verbete}{二}{er4}{2}[Radical ⼆][Kangxi 7]
  \significado{num.}{dois, 2}
\end{verbete}

\begin{verbete}{二战}{er4zhan4}{2,9}
  \significado*{s.}{Segunda Guerra Mundial}
\end{verbete}

%%%%% EOF %%%%%

%%%
%%% F
%%%
%\section*{F}
\addcontentsline{toc}{section}{F}

\begin{verbete}{发}{fa1}{5}
  \significado{p.c.}{para tiros (rodadas)}
  \significado{v.}{enviar; mandar}
\end{verbete}

\begin{verbete}{发表}{fa1biao3}{5;8}
  \significado{v.}{emitir; publicar}
\end{verbete}

\begin{verbete}{发抖}{fa1dou3}{5;7}
  \significado{v.}{tremer; sacudir; estremecer}
\end{verbete}

\begin{verbete}{发明者}{fa1ming2zhe3}{5;8;8}
  \significado{s.}{inventor}
\end{verbete}

\begin{verbete}{发票}{fa1piao4}{5;11}
  \significado{s.}{fatura; recibo; conta}
\end{verbete}

\begin{verbete}{发烧}{fa1shao1}{5;10}
  \significado{v.}{ter febre}
\end{verbete}

\begin{verbete}{发生}{fa1sheng1}{5;5}
  \significado{v.}{acontecer; ocorrer}
\end{verbete}

\begin{verbete}{发现}{fa1xian4}{5;8}
  \significado{v.}{descobrir; encontrar}
\end{verbete}

\begin{verbete}{发现者}{fa1xian4 zhe3}{5;8;8}
  \significado{s.}{descobridor}
\end{verbete}

\begin{verbete}{发音}{fa1yin1}{5;9}
  \significado{s.}{pronúncia}
  \significado{v.}{pronunciar}
\end{verbete}

\begin{verbete}{发展}{fa1zhan3}{5;10}
  \significado{s.}{desenvolvimento}
  \significado{v.}{desenvolver}
\end{verbete}

\begin{verbete}{罚}{fa2}{9}
  \significado{v.}{castigar; punir}
\end{verbete}

\begin{verbete}{罚款}{fa2kuan3}{9;12}
  \significado{s.}{multa (monetária); pena}
  \significado{v.+compl.}{aplicar uma multa; multar}
\end{verbete}

\begin{verbete}{法}{fa3}{8}
  \significado*{s.}{França, abreviação de~法国}
  \veja{法国}{fa3guo2}
\end{verbete}

\begin{verbete}{法国}{fa3guo2}{8;8}
  \significado*{s.}{França}
  \veja{法}{fa3}
\end{verbete}

\begin{verbete}{法国人}{fa3guo2ren2}{8;8;2}
  \significado{s.}{francês; nascido na França}
\end{verbete}

\begin{verbete}{法网}{fa3wang3}{8;6}
  \significado*{s.}{Torneio de Roland Garros (French Open), torneio de tênis}
\end{verbete}

\begin{verbete}{法文}{fa3wen2}{8;4}
  \significado*{s.}{françês, língua francesa}
\end{verbete}

\begin{verbete}{法语}{fa3yu3}{8;9}
  \significado{s.}{françês, língua francesa}
\end{verbete}

\begin{verbete}{发簪}{fa4zan1}{5;18}
  \significado{s.}{grampo de cabelo}
\end{verbete}

\begin{verbete}{番茄}{fan1qie2}{12;8}
  \significado{s.}{tomate}
\end{verbete}

\begin{verbete}{蕃茄}{fan1qie2}{15;8}
  \variante{番茄}
\end{verbete}

\begin{verbete}{翻译}{fan1yi4}{18;7}
  \significado[个,位,名]{s.}{tradução; tradutor; interpretação; intérprete}
  \significado{v.}{traduzir; interpretar}
\end{verbete}

\begin{verbete}{反对}{fan3dui4}{4;5}
  \significado{v.}{contrariar; opor-se; lutar contra}
\end{verbete}

\begin{verbete}{反对党}{fan3dui4dang3}{4;5;10}
  \significado{s.}{partido de oposição}
\end{verbete}

\begin{verbete}{反对派}{fan3dui4pai4}{4;5;9}
  \significado{s.}{facção de oposição}
\end{verbete}

\begin{verbete}{反对票}{fan3dui4piao4}{4;5;11}
  \significado{s.}{voto dissidente}
\end{verbete}

\begin{verbete}{反复}{fan3fu4}{4;9}
  \significado{adv.}{de novo e de novo; repetidamente}
\end{verbete}

\begin{verbete}{反省}{fan3xing3}{4;9}
  \significado{v.}{examinar a consciência; questionar-se; refletir sobre si mesmo; sondar a alma}
\end{verbete}

\begin{verbete}{反正}{fan3zheng4}{4;5}
  \significado{adv.}{de qualquer maneira; em qualquer caso; aconteça o que acontecer}
\end{verbete}

\begin{verbete}{饭店}{fan4dian4}{7;8}
  \significado[家,个]{s.}{hotel; restaurante}
\end{verbete}

\begin{verbete}{方便}{fang1bian4}{4;9}
  \significado{adj.}{conveniente, adequado}
  \significado{v.}{facilitar, facilitar as coisas; ter dinheiro de sobra; (eufemismo) aliviar-se}
\end{verbete}

\begin{verbete}{方法}{fang1fa3}{4;8}
  \significado[个]{s.}{método; meio}
\end{verbete}

\begin{verbete}{方言}{fang1yan2}{4;7}
  \significado*{s.}{o primeiro dicionário de dialeto chinês, editado por Yang Xiong 扬雄 no século I, contendo mais de 9.000 caracteres}
  \significado{s.}{dialeto}
  \veja{扬雄}{yang2xiong2}
\end{verbete}

\begin{verbete}{防晒}{fang2shai4}{6;10}
  \significado{s.}{protetor solar}
\end{verbete}

\begin{verbete}{房东}{fang2dong1}{8;5}
  \significado{s.}{dono; proprietário; senhorio}
\end{verbete}

\begin{verbete}{房间}{fang2jian1}{8;7}
  \significado[间,个]{s.}{quarto}
\end{verbete}

\begin{verbete}{房子}{fang2zi5}{8;3}
  \significado[栋,幢,座,套,间,个]{s.}{apartamento; casa; quarto}
\end{verbete}

\begin{verbete}{访问}{fang3wen4}{6;6}
  \significado{v.}{visitar}
\end{verbete}

\begin{verbete}{放}{fang4}{8}
  \significado{v.}{liberar; libertar; deixar ir; colocar; por; detonar (fogos de artifício)}
\end{verbete}

\begin{verbete}{放出}{fang4chu1}{8;5}
  \significado{v.}{liberar; libertar}
\end{verbete}

\begin{verbete}{放大}{fang4da4}{8;3}
  \significado{v.}{ampliar}
\end{verbete}

\begin{verbete}{放电}{fang4dian4}{8;5}
  \significado{s.}{descarga elétrica}
\end{verbete}

\begin{verbete}{放飞}{fang4fei1}{8;3}
  \significado{s.}{deixar voar}
\end{verbete}

\begin{verbete}{放过}{fang4guo4}{8;6}
  \significado{v.}{deixar; deixar alguém escapar impune; passar despercebido}
\end{verbete}

\begin{verbete}{放假}{fang4jia4}{8;11}
  \significado{v.}{ter férias ou feriado}
\end{verbete}

\begin{verbete}{放弃}{fang4qi4}{8;7}
  \significado{v.}{abandonar; deistir de; renunciar}
\end{verbete}

\begin{verbete}{放弃权利}{fang4qi4 quan2li4}{8;7;6;7}
  \significado{s.}{renúncia}
\end{verbete}

\begin{verbete}{放弃者}{fang4qi4zhe3}{8;7;8}
  \significado{s.}{desistente}
\end{verbete}

\begin{verbete}{放任}{fang4ren4}{8;6}
  \significado{v.}{ignorar; saciar-se; deixar sozinho}
\end{verbete}

\begin{verbete}{放肆}{fang4si4}{8;13}
  \significado{adj.}{atrevido; pesunçoso; devasso}
\end{verbete}

\begin{verbete}{放松}{fang4song1}{8;8}
  \significado{adj.}{relaxado; afrouxado}
  \significado{v.}{relaxar; afrouxar}
\end{verbete}

\begin{verbete}{放心}{fang4xin1}{8;4}
  \significado{adj.}{descansado; despreocupado}
  \significado{v.}{sentir-se aliviado; sentir-se tranquilo; ficar à vontade}
  \significado{v.+compl.}{estar à vontade; ficar descansando; sentir-se tranquilo; sentir-se aliviado}
\end{verbete}

\begin{verbete}{放养}{fang4yang3}{8;9}
  \significado{v.}{criar (gado, peixes, culturas, etc.); crescer; criar}
\end{verbete}

\begin{verbete}{放走}{fang4zou3}{8;7}
  \significado{v.}{permitir (uma pessoa ou um animal) ir; liberar; libertar}
\end{verbete}

\begin{verbete}{飞船}{fei1chuan2}{3;11}
  \significado{s.}{espaçonave; dirigível; aeronave}
\end{verbete}

\begin{verbete}{飞碟}{fei1die2}{3;14}
  \significado{s.}{disco-voador; OVNI; \emph{frisbee}}
\end{verbete}

\begin{verbete}{飞机}{fei1ji1}{3;6}
  \significado[架]{s.}{avião}
\end{verbete}

\begin{verbete}{飞机票}{fei1ji1piao4}{3;6;11}
  \significado[张]{s.}{bilhete de avião}
  \veja{机票}{ji1piao4}
\end{verbete}

\begin{verbete}{非}{fei1}{8}[175]
  \significado*{s.}{África, abreviação de~非洲}
  \significado{adv.}{não ser; não é; não}
  \veja{非洲}{fei1zhou1}
\end{verbete}

\begin{verbete}{非常}{fei1chang2}{8;11}
  \significado{adv.}{extraordinário; altamente; muito}
\end{verbete}

\begin{verbete}{非洲}{fei1zhou1}{8;9}
  \significado*{s.}{África}
  \veja{非}{fei1}
\end{verbete}

\begin{verbete}{非洲人}{fei1zhou1ren2}{8;9;2}
  \significado{s.}{africano; nascido na África}
\end{verbete}

\begin{verbete}{分}{fen1}{4}
  \significado{p.c.}{minuto; centavo}
  \significado{s.}{um ponto (em esportes ou jogos); parte ou subdivisão; fração}
\end{verbete}

\begin{verbete}{分公司}{fen1gong1si1}{4;4;5}
  \significado{s.}{sucursal; filial de companhia}
\end{verbete}

\begin{verbete}{分量}{fen1liang4}{4;12}
  \significado{s.}{componente vetorial}
  \veja{分量}{fen4liang4}
  \veja{分量}{fen4liang5}
\end{verbete}

\begin{verbete}{分钟}{fen1zhong1}{4;9}
  \significado{p.t.}{minuto}
\end{verbete}

\begin{verbete}{分子}{fen1zi3}{4;3}
  \significado{s.}{molécula; (matemática) numerador de uma fração}
  \veja{分子}{fen4zi3}
\end{verbete}

\begin{verbete}{粉}{fen3}{10}
  \significado{s.}{pó; pó cosmético facial; alimento preparado a partir de amido, macarrão feito de qualquer tipo de farinha}
  \significado{v.}{tornar algo em pó; ser um fã de}
\end{verbete}

\begin{verbete}{粉色}{fen3se4}{10;6}
  \significado{s.}{cor-de-rosa}
\end{verbete}

\begin{verbete}{粉丝}{fen3si1}{10;5}
  \significado[把]{s.}{aletria de amido de feijão; aletria chinesa; macarrão de celofane ou macarrão de vidro (transparente)}
  \significado{s.}{fã (empréstimo linguístico); entusiasta de alguém ou alguma coisa}
\end{verbete}

\begin{verbete}{分量}{fen4liang4}{4;12}
  \significado{s.}{tamanho da porção (comida)}
  \veja{分量}{fen1liang4}
  \veja{分量}{fen4liang5}
\end{verbete}

\begin{verbete}{分量}{fen4liang5}{4;12}
  \significado{s.}{quantidade; peso; medida}
  \veja{分量}{fen1liang4}
  \veja{分量}{fen4liang4}
\end{verbete}

\begin{verbete}{分子}{fen4zi3}{4;3}
  \significado{s.}{membros de uma classe ou grupo; elementos políticos (como intelectuais ou extremistas)}
  \veja{分子}{fen1zi3}
\end{verbete}

\begin{verbete}{份}{fen4}{6}
  \significado{p.c.}{para presentes, jornais, revistas, papéis, relatórios, contratos, etc. ou pratos (refeição)}
\end{verbete}

\begin{verbete}{奋战}{fen4zhan4}{8;9}
  \significado{v.}{lutar bravamente; trabalhar duro}
\end{verbete}

\begin{verbete}{愤怒}{fen4nu4}{12;9}
  \significado{adj.}{zangado; indignado}
  \significado{s.}{ira}
\end{verbete}

\begin{verbete}{愤世嫉俗}{fen4shi4ji2su2}{12;5;13;9}
  \significado{v.}{ser cínico; ser amargurado}
\end{verbete}

\begin{verbete}{丰收}{feng1shou1}{4;6}
  \significado{s.}{colheita abundante}
\end{verbete}

\begin{verbete}{风}{feng1}{4}
  \significado[阵,丝]{s.}{vento}
\end{verbete}

\begin{verbete}{风扇}{feng1shan4}{4;10}
  \significado{s.}{ventilador elétrico}
\end{verbete}

\begin{verbete}{风筝}{feng1zheng5}{4;12}
  \significado{s.}{pipa; papagaio; pandorga}
\end{verbete}

\begin{verbete}{枫叶}{feng1ye4}{8;5}
  \significado{s.}{folha de bordo (maple, tipo de árvore)}
\end{verbete}

\begin{verbete}{封}{feng1}{9}
  \significado*{s.}{sobrenome Feng}
  \significado{p.c.}{para objetos selados, especialmente cartas}
  \significado{v.}{conceder um título; conferir; conceder; selar}
\end{verbete}

\begin{verbete}{封闭}{feng1bi4}{9;6}
  \significado{v.}{fechar; selar; confinado}
\end{verbete}

\begin{verbete}{封底}{feng1di3}{9;8}
  \significado{s.}{contracapa de um livro}
\end{verbete}

\begin{verbete}{封冻}{feng1dong4}{9;7}
  \significado{v.}{congelar (água ou terra)}
\end{verbete}

\begin{verbete}{封盖}{feng1gai4}{9;11}
  \significado{s.}{boné; capa; selo}
  \significado{v.}{cobrir}
\end{verbete}

\begin{verbete}{封建}{feng1jian4}{9;8}
  \significado{adj.}{feudal}
  \significado{s.}{feudalismo}
\end{verbete}

\begin{verbete}{封口}{feng1kou3}{9;3}
  \significado{v.}{selar; fechar; curar (uma ferida); manter os lábios selados}
\end{verbete}

\begin{verbete}{封面}{feng1mian4}{9;9}
  \significado{s.}{capa (de uma publicação); sobrecapa}
\end{verbete}

\begin{verbete}{封印}{feng1yin4}{9;5}
  \significado{s.}{selo (em envelopes)}
\end{verbete}

\begin{verbete}{封斋}{feng1zhai1}{9;10}
  \significado*{s.}{Ramadã (Islã)}
\end{verbete}

\begin{verbete}{疯狂}{feng1kuang2}{9;7}
  \significado{adj.}{louco, frenético, selvagem}
\end{verbete}

\begin{verbete}{缝纫}{feng2ren4}{13;6}
  \significado{v.}{costurar}
\end{verbete}

\begin{verbete}{缝纫机}{feng2ren4ji1}{13;6;6}
  \significado[架]{s.}{máquina de costura}
\end{verbete}

\begin{verbete}{凤凰}{feng4huang2}{4;11}
  \significado{s.}{fênix}
\end{verbete}

\begin{verbete}{否则}{fou3ze2}{7;6}
  \significado{conj.}{caso contrário; ou}
\end{verbete}

\begin{verbete}{夫妻}{fu1qi1}{4;8}
  \significado{s.}{casal; marido e eposa}
\end{verbete}

\begin{verbete}{扶梯}{fu2ti1}{7;11}
  \significado{s.}{escada rolante}
\end{verbete}

\begin{verbete}{服务员}{fu2wu4yuan2}{8;5;7}
  \significado{s.}{atendente; garçom; garçonete; pessoal de atendimento ao cliente}
\end{verbete}

\begin{verbete}{符合}{fu2he2}{11;6}
  \significado{conj.}{de acordo com; concordando com; contando com; alinhado com}
  \significado{v.}{concordar com; estar em conformidade com; corresponder com; gerenciar; lidar}
\end{verbete}

\begin{verbete}{福克斯}{fu2ke4si1}{13;7;12}
  \significado*{s.}{Fox (empresa de mídia); Focus (automóvel fabricado pela Ford)}
\end{verbete}

\begin{verbete}{父母亲}{fu4mu3qin1}{4;5;9}
  \significado{s.}{pais}
\end{verbete}

\begin{verbete}{父亲}{fu4qin1}{4;9}
  \significado[个]{s.}{pai}
\end{verbete}

\begin{verbete}{付}{fu4}{5}
  \significado*{s.}{sobrenome Fu}
  \significado{p.c.}{para pares ou conjuntos de coisas}
  \significado{v.}{pagar}
\end{verbete}

\begin{verbete}{附近}{fu4jin4}{7;7}
  \significado{p.l.}{aqui perto; perto daqui}
\end{verbete}

\begin{verbete}{复活节}{fu4huo2jie2}{9;9;5}
  \significado*{s.}{Páscoa}
\end{verbete}

\begin{verbete}{副}{fu4}{11}
  \significado{p.c.}{para pares, conjuntos de coisas e expressões faciais; para óculos, luvas, etc.}
\end{verbete}

%%%%% EOF %%%%%

%%%
%%% G
%%%
\section*{G}
\addcontentsline{toc}{section}{G}

\begin{verbete}[7;12]{改善}{gai3shan4}
  \significado{v.}{aperfeiçoar; melhorar}
\end{verbete}

\begin{verbete}[7;12;6;7]{改善关系}[\\]{gai3shan4guan1xi5}
  \significado{v.}{melhorar a relação}
\end{verbete}

\begin{verbete}[7;12;10;5]{改善通讯}[\\]{gai3shan4tong1xun4}
  \significado{v.}{melhorar a comunicação}
\end{verbete}

\begin{verbete}[3]{干}{gan1}
  \significado{v.}{preocupar; ignorar; interferir}
  \veja{干}{gan4}
\end{verbete}
\begin{verbete*}[3]{干}{gan1}
  \significado{s.}{sobrenome Gan}
\end{verbete*}

\begin{verbete}[3;8]{干杯}{gan1bei1}
  \significado{interj.}{Saúde!}
  \significado{v.+compl.}{fazer um brinde; brindar até a última gota}
\end{verbete}

\begin{verbete}[3;8]{干净}{gan1jing4}
  \significado{adj.}{limpo; arrumado}
\end{verbete}

\begin{verbete}[3;10]{干预}{gan1yu4}
  \significado{s.}{intervenção}
  \significado{v.}{intervir; intrometer-se}
\end{verbete}

\begin{verbete}[5;16]{甘薯}{gan1shu3}
  \significado{s.}{batata doce}
\end{verbete}

\begin{verbete}[10]{赶}{gan3}
  \significado{v.}{apressar; precipitar-se; conduzir (gado, etc.); aproveitar (uma oportunidade)}
\end{verbete}

\begin{verbete}[10;8]{赶到}{gan3dao4}
  \significado{v.}{apressar-se (para algum lugar)}
\end{verbete}

\begin{verbete}[10;10]{赶赴}{gan3fu4}
  \significado{v.}{apressar}
\end{verbete}

\begin{verbete}[10;12]{赶集}{gan3ji2}
  \significado{v.}{ir a uma feira; ir ao mercado}
\end{verbete}

\begin{verbete}[10;11]{赶脚}{gan3jiao3}
  \significado{v.}{transportar mercadorias para ganhar a vida (especialmente de burro); trabalhar como carroceiro ou porteiro}
\end{verbete}

\begin{verbete}[10;10]{赶紧}{gan3jin3}
  \significado{adv.}{apressadamente; sem demora}
\end{verbete}

\begin{verbete}[10;7]{赶快}{gan3kuai4}
  \significado{adv.}{imediatamente; de uma vez só}
\end{verbete}

\begin{verbete}[10;13]{赶路}{gan3lu4}
  \significado{v.}{apressar a jornada; apressar-se}
\end{verbete}

\begin{verbete}[10;6]{赶忙}{gan3mang2}
  \significado{v.}{acelerar; apressar; se apressar}
\end{verbete}

\begin{verbete}[10;12]{赶跑}{gan3pao3}
  \significado{v.}{afastar; forçar a saída; repelir}
\end{verbete}

\begin{verbete}[10;3]{赶上}{gan3shang4}
  \significado{adv.}{a tempo para}
  \significado{v.}{alcançar; ultrapassar}
\end{verbete}

\begin{verbete}[10;6]{赶早}{gan3zao3}
  \significado{adv.}{o mais breve possível; na primeira oportunidade; antes que seja tarde; quanto antes melhor}
\end{verbete}

\begin{verbete}[10;7]{赶走}{gan3zou3}
  \significado{v.}{expulsar; voltar atrás}
\end{verbete}

\begin{verbete}[13;9]{感冒}{gan3mao4}
  \significado{v.}{ficar resfriado; estar com resfriado}
\end{verbete}

\begin{verbete}[13;11]{感情}{gan3qing2}
  \significado{s.}{afeição; emoção; sentimento; sentimento amoroso}
\end{verbete}

\begin{verbete}[13;12]{感谢}{gan3xie4}
  \significado{v.}{agradecer}
\end{verbete}

\begin{verbete}[15;13;11]{橄榄球}{gan3lan3qiu2}
  \significado{s.}{futebol jogado com bola oval (rúgbi, futebol americano, regras australianas, etc.)}
\end{verbete}

\begin{verbete}[3]{干}{gan4}
  \significado{v.}{fazer; gerenciar; trabalhar; gíria: matar; vulgar: foder}
  \veja{干}{gan1}
\end{verbete}

\begin{verbete}[3;7;7;8]{干你屁事}[\\]{gan4·ni3.pi4shi4}
  \significado{expr.}{Foda-se!}
\end{verbete}

\begin{verbete}[6]{刚}{gang1}
  \significado{adj.}{duro (sentido de difícil); forte}
  \significado{adv.}{acabar de; por muito pouco; apenas}
\end{verbete}

\begin{verbete}[6;3]{刚才}{gang1cai2}
  \significado{p.t.}{ainda agora; há pouco tempo}
\end{verbete}

\begin{verbete}[10]{高}{gao1}
  \significado{adj.}{alto; acima da média}
\end{verbete}
\begin{verbete*}[10]{高}{gao1}
  \significado{s.}{sobrenome Gao}
\end{verbete*}

\begin{verbete}[10;6]{高兴}{gao1xing4}
  \significado{adj.}{feliz; alegre; contente; disposto (a fazer alguma coisa)}
\end{verbete}

\begin{verbete}[13]{搞}{gao3}
  \significado{v.}{fazer}
\end{verbete}

\begin{verbete}[13;13]{搞错}{gao3cuo4}
  \significado{v.}{cometer um erro}
\end{verbete}

\begin{verbete}[13;8]{搞定}{gao3ding4}
  \significado{v.}{consertar; resolver}
\end{verbete}

\begin{verbete}[13;9]{搞鬼}{gao3gui3}
  \significado{v.}{fazer travessuras; fazer truques}
\end{verbete}

\begin{verbete}[13;6]{搞好}{gao3hao3}
  \significado{v.}{fazer um ótimo trabalho}
\end{verbete}

\begin{verbete}[13;11]{搞混}{gao3hun4}
  \significado{v.}{confundir}
\end{verbete}

\begin{verbete}[13;7]{搞乱}{gao3luan4}
  \significado{v.}{estragar; confundir; bagunçar}
\end{verbete}

\begin{verbete}[13;10]{搞钱}{gao3qian2}
  \significado{v.}{fazer dinheiro; acumular dinheiro}
\end{verbete}

\begin{verbete}[13;10]{搞通}{gao3tong1}
  \significado{v.}{entender algo}
\end{verbete}

\begin{verbete}[13;10]{搞笑}{gao3xiao4}
  \significado{adj.}{engraçado; hilário}
  \significado{v.}{fazer as pessoas rirem}
\end{verbete}

\begin{verbete}[15;7]{稿纸}{gao3zhi3}
  \significado{s.}{rascunho; manuscrito}
\end{verbete}

\begin{verbete}[7;7]{告诉}{gao4su5}
  \significado{v.}{contar; dar a conhecer; informar}
\end{verbete}

\begin{verbete}[10;10]{哥哥}{ge1ge5}
  \significado[个,位]{s.}{irmão mais velho}
\end{verbete}

\begin{verbete}[14]{歌}{ge1}
  \significado[支,首]{s.}{canção; canto}
\end{verbete}

\begin{verbete}[3]{个}{ge4}
  \significado{p.c.}{para objetos e pessoas em geral}
\end{verbete}

\begin{verbete}[6;9]{各种}{ge4zhong3}
  \significado{adv.}{todas as espécies de; diversos gêneros de}
\end{verbete}

\begin{verbete}[9]{给}{gei3}
  \significado{pre.}{a; para}
  \significado{v.}{dar; permitir; fazer alguma coisa (para alguém)}
\end{verbete}

\begin{verbete}[9;5;5;8]{给……打电话}[\\]{gei3...da3dian4hua4}
  \significado{expr.}{telefonar para alguém}
  \veja{打电话}{da3dian4hua4}
\end{verbete}

\begin{verbete}[10;11]{根据}{gen1ju4}
  \significado{prep.}{de acordo com}
  \significado[个]{s.}{base; fundação}
\end{verbete}

\begin{verbete}[13]{跟}{gen1}
  \significado{prep.}{com}
  \significado{v.}{acompanhar junto; seguir de perto; ir com}
\end{verbete}

\begin{verbete}[7]{更}{geng4}
  \significado{adv.}{mais; ainda mais}
\end{verbete}

\begin{verbete}[3;4;9]{工艺品}{gong1yi3pin3}
  \significado[个]{s.}{artigo de artesanato; trabalho manual}
\end{verbete}

\begin{verbete}[3;7]{工作}{gong1zuo4}
  \significado[个,份,项]{s.}{trabalho; emprego; tarefa}
  \significado{v.}{trabalhar; operar (uma máquina)}
\end{verbete}

\begin{verbete}[4;6;7;4]{公共汽车}[\\]{gong1gong4qi4che1}
  \significado[辆,班]{s.}{ônibus}
\end{verbete}

\begin{verbete}[4;7]{公克}{gong1ke4}
  \significado{s.}{grama (medida de peso)}
\end{verbete}

\begin{verbete}[4;5]{公司}{gong1si1}
  \significado[家]{s.}{empresa; companhia; corporação; firma}
\end{verbete}

\begin{verbete}[4;5;8;11]{公司治理}[\\]{gong1si1zhi4li3}
  \significado{s.}{governança corporativa}
\end{verbete}

\begin{verbete}[4;5;5;8]{公用电话}[\\]{gong1yong4dian4hua4}
  \significado[部]{s.}{telefone público}
\end{verbete}

\begin{verbete}[4;12]{公寓}{gong1yu4}
  \significado[套]{s.}{prédio de apartamentos; pensão}
\end{verbete}

\begin{verbete}[4;4]{公元}{gong1yuan2}
  \significado{s.}{D.C. (por exemplo, 公元293年)}
  \veja{前}{qian2}
\end{verbete}

\begin{verbete}[4;7]{公园}{gong1yuan2}
  \significado[个,座]{s.}{parque (para recreação pública)}
\end{verbete}

\begin{verbete}[5;4]{功夫}{gong1fu5}
  \significado{s.}{Gongfu (Kung Fu), arte marcial; esforço; habilidade}
\end{verbete}

\begin{verbete}[8]{狗}{gou3}
  \significado[条,只]{s.}{cão; cachorro}
\end{verbete}

\begin{verbete}[8;9]{诟骂}{gou4ma4}
  \significado{v.}{abusar verbalmente; insultar; criticar}
\end{verbete}

\begin{verbete}[11]{够}{gou4}
  \significado{adv.}{(antes do adj.) realmente}
  \significado{adj.}{suficiente}
  \significado{v.}{bastar; chegar}
\end{verbete}

\begin{verbete}[11;5]{够本}{gou4ben3}
  \significado{v.}{empatar; fazer valer o dinheiro}
\end{verbete}

\begin{verbete}[11;4;11]{够不着}{gou4bu5zhao2}
  \significado{v.}{ser incapaz de alcançar}
\end{verbete}

\begin{verbete}[11;4;11]{够得着}[\\]{gou4de5zhao2}
  \significado{v.}{estar à altura; alcançar}
\end{verbete}

\begin{verbete}[11;10]{够格}{gou4ge2}
  \significado{adj.}{apto; qualificado; apresentável}
\end{verbete}

\begin{verbete}[11;8;4]{够朋友}{gou4peng2you3}
  \significado{v.}{ser um amigo verdadeiro}
\end{verbete}

\begin{verbete}[11;7]{够呛}{gou4qiang4}
  \significado{adj.}{suficiente; terrível; insuportável; improvável}
\end{verbete}

\begin{verbete}[11;8]{够味}{gou4wei4}
  \significado{adj.}{excelente; na medida}
\end{verbete}

\begin{verbete*}[9;9]{故宫}{gu4gong1}
  \significado{s.}{Palácio Imperial; Cidade Proibida}
\end{verbete*}

\begin{verbete}[8]{刮}{gua1}
  \significado{v.}{ventar; soprar (vento)}
\end{verbete}

\begin{verbete}[8;4]{刮风}{gua1feng1}
  \significado{v.+compl.}{ventar; fazer vento}
\end{verbete}

\begin{verbete}[9;5;9]{挂号信}{gua4hao4xin4}
  \significado{s.}{carta registrada}
\end{verbete}

\begin{verbete}[8]{拐}{guai3}
  \significado{v.}{virar (uma esquina, etc.); cortar}
\end{verbete}

\begin{verbete}[6;11]{光盘}{guang1pan2}
  \significado[片,张]{s.}{CD; DVD; disco compacto}
\end{verbete}

\begin{verbete}[6;6;9]{光污染}{guang1·wu1ran3}
  \significado{s.}{poluição luminosa}
\end{verbete}

\begin{verbete*}[3;5]{广东}{guang3dong1}
  \significado{s.}{Guangdong}
\end{verbete*}

\begin{verbete}[3;7]{广告}{guang3gao4}
  \significado[项]{s.}{publicidade; anúncio publicitário}
\end{verbete}

\begin{verbete}[8;8]{规定}{gui1ding4}
  \significado[个]{s.}{regulamento; regra}
  \significado{v.}{estipular}
\end{verbete}

\begin{verbete}[9]{贵}{gui4}
  \significado{adj.}{caro; nobre; precioso}
\end{verbete}

\begin{verbete}[9;8]{贵姓}{gui4xing4}
  \significado{interr.}{qual seu sobrenome?}
\end{verbete}

\begin{verbete}[8]{国}{guo2}
  \significado[个]{s.}{país; nação}
\end{verbete}
\begin{verbete*}[8]{国}{guo2}
  \significado{s.}{sobrenome Guo}
\end{verbete*}

\begin{verbete}[8;10;11]{国宾馆}[\\]{guo2bin1guan3}
  \significado{s.}{pousada estadual}
\end{verbete}

\begin{verbete*}[8;7;2;12;5]{国际儿童节}[\\]{guo2ji3·er2tong2jie2}
  \significado{s.}{Dia Internacional das Crianças (1 de junho)}
\end{verbete*}

\begin{verbete*}[8;7;6;3;5]{国际妇女节}[\\]{guo2ji4·fu4nv3jie2}
  \significado{s.}{Dia Internacional das Mulheres (8 de março)}
\end{verbete*}

\begin{verbete*}[8;7;7;6;5]{国际劳动节}[\\]{guo2ji4·lao2dong4·jie2}
  \significado{s.}{Dia Internacional dos Trabalhadores (1 de maio)}
\end{verbete*}

\begin{verbete}[8;10]{国家}{guo2jia1}
  \significado[个]{s.}{país; nação}
\end{verbete}

\begin{verbete*}[8;7;5]{国庆节}{guo2quing4jie2}
  \significado{s.}{Dia Nacional (1 de outubro)}
\end{verbete*}

\begin{verbete*}[8;9]{国语}{guo2yu3}
  \significado{s.}{Língua Chinesa (Mandarim), enfatizando sua natureza nacional}
\end{verbete*}

\begin{verbete}[8;13]{果酱}{guo3jiang4}
  \significado{s.}{geléia; compota ou doce (de frutas)}
\end{verbete}

\begin{verbete}[6]{过}{guo4}
  \significado{part.}{passado}
  \significado{v.}{atravessar; passar (tempo)}
\end{verbete}
\begin{verbete*}[6]{过}{guo4}
  \significado{s.}{sobrenome Guo}
\end{verbete*}

\begin{verbete}[6;4;11]{过不惯}{guo4·bu5·guan4}
  \significado{v.}{não se acostumar; não se habituar}
  \veja{过惯}{guo4guan4}
\end{verbete}

\begin{verbete}[6;11]{过惯}{guo4guan4}
  \significado{v.}{estar acostumado (a um certo estilo de vida, etc.)}
  \veja*{过不惯}{guo4·bu5·guan4}
\end{verbete}

\begin{verbete}[6;7]{过来}{guo4lai5}
  \significado{v.}{atravessar (para a minha localização); vir até aqui}
\end{verbete}

\begin{verbete}[6;6]{过年}{guo4nian2}
  \significado{v.}{festejar o Ano Novo Chinês}
\end{verbete}

\begin{verbete}[6;12]{过期}{guo4qi1}
  \significado{v.+compl.}{exceder a data; passar a data; expirar (passar a data de expiração)}
\end{verbete}

\begin{verbete}[6;5]{过去}{guo4qu5}
  \significado{v.}{atravessar (a partir da minha localização); ir até lá}
\end{verbete}


%%%%% EOF %%%%%

%%%
%%% H
%%%
%\section*{H}
\addcontentsline{toc}{section}{H}

\begin{verbete}{还}{hai2}{7}
  \significado{adv.}{ainda; também; ainda mais; razoavelmente; bastante}
  \veja{还}{huan2}
\end{verbete}

\begin{verbete}{还是}{hai2shi5}{7;9}
  \significado{adv.}{ainda, no entanto; tive melhores}
  \significado{conj.}{ou (somente para frases interrogativas)}
\end{verbete}

\begin{verbete}{孩子}{hai2zi5}{9;3}
  \significado{s.}{criança; filho}
\end{verbete}

\begin{verbete}{海}{hai3}{10}
  \significado*{s.}{sobrenome Hai}
  \significado[个,片]{s.}{mar; oceano}
\end{verbete}

\begin{verbete}{海边}{hai3bian1}{10;5}
  \significado{p.d.l.}{costa marítima; litoral}
\end{verbete}

\begin{verbete}{害怕}{hai4pa4}{10;8}
  \significado{v.}{ter medo; ficar com medo; temer}
\end{verbete}

\begin{verbete}{韩国}{han2guo2}{12;8}
  \significado*{s.}{Coréia do Sul}
\end{verbete}

\begin{verbete}{韩国人}{han2guo2ren2}{12;8;2}
  \significado{s.}{coreano; nascido na Coréia}
\end{verbete}

\begin{verbete}{汉堡包}{han4bao3bao1}{5;12;5}
  \significado[个]{s.}{hambúrguer}
\end{verbete}

\begin{verbete}{汉堡王}{han4bao3wang2}{5;12;4}
  \significado*{s.}{Burguer King (restaurante \emph{fast-food})}
\end{verbete}

\begin{verbete}{汉葡词典}{han4-pu2·ci2dian3}{5;12;7;8}
  \significado[部,本]{s.}{dicionário chinês-português}
  \veja{葡汉词典}{pu2-han4·ci2dian3}
\end{verbete}

\begin{verbete}{汉语}{han4yu3}{5;9}
  \significado[门]{s.}{chinês, língua chinesa, mandarim}
\end{verbete}

\begin{verbete}{行}{hang2}{6}
  \significado{s.}{firma comercial; linha de negócio; profissão; linha (de um tema); linha (em tabela de dados)}
  \veja{行}{xing2}
\end{verbete}

\begin{verbete}{航班}{hang2ban1}{10;10}
  \significado{s.}{voo; número de voo}
\end{verbete}

\begin{verbete}{号}{hao2}{5}
  \significado[个]{s.}{rugido; choro}
  \veja{号}{hao4}
\end{verbete}

\begin{verbete}{好}{hao3}{6}
  \significado{adj.}{bom, bem}
  \significado{adv.}{apropriadamente; cuidadosamente; muito (linguagem falada)}
  \veja{好}{hao4}
\end{verbete}

\begin{verbete}{好吃}{hao3chi1}{6;6}
  \significado{adj.}{delicioso; saboroso}
  \veja{好吃}{hao4chi1}
\end{verbete}

\begin{verbete}{好汉}{hao3han4}{6;5}
  \significado[条]{s.}{herói; pessoa forte e corajosa}
\end{verbete}

\begin{verbete}{好久}{hao3jiu3}{6;3}
  \significado{adv.}{por muito tempo; por eras (no passado)}
\end{verbete}

\begin{verbete}{好看}{hao3kan4}{6;9}
  \significado{adj.}{boa aparência; bom (um filme, livro, programa de TV, etc.)}
\end{verbete}

\begin{verbete}{好听}{hao3ting1}{6;7}
  \significado{adj.}{agradável de ouvir}
\end{verbete}

\begin{verbete}{好玩儿}{hao3wan2r5}{6;8;2}
  \significado{adj.}{divertido; prazeroso; interessante}
\end{verbete}

\begin{verbete}{好象}{hao3xiang4}{6;11}
  \variante{好像}{hao3xiang4}
\end{verbete}

\begin{verbete}{好像}{hao3xiang4}{6;13}
  \significado{adv.}{talvez fosse; parecer; ser como}
\end{verbete}

\begin{verbete}{好学}{hao3xue2}{6;8}
  \significado{adj.}{fácil de aprender}
  \veja{好学}{hao4xue2}
\end{verbete}

\begin{verbete}{号}{hao4}{5}
  \significado{p.c.}{dia do mês; usado para indicar o número de pessoas}
  \significado[个]{s.}{dia do mês; número}
  \veja{号}{hao2}
\end{verbete}

\begin{verbete}{号码}{hao4ma3}{5;8}
  \significado[堆,个]{s.}{número}
\end{verbete}

\begin{verbete}{好}{hao4}{6}
  \significado{v.}{gostar de; estar propenso a; ter tendência a}
  \veja{好}{hao3}
\end{verbete}

\begin{verbete}{好吃}{hao4chi1}{6;6}
  \significado{v.}{gostar de comer; ser guloso}
  \veja{好吃}{hao3chi1}
\end{verbete}

\begin{verbete}{好学}{hao4xue2}{6;8}
  \significado{s.}{estudioso; erudito}
  \veja{好学}{hao3xue2}
\end{verbete}

\begin{verbete}{欱}{he1}{10}
  \variante{喝}{he1}
\end{verbete}

\begin{verbete}{喝}{he1}{12}
  \significado{interj.}{Meu Deus!}
  \significado{v.}{beber}
  \veja{喝}{he4}
\end{verbete}

\begin{verbete}{喝醉}{he1zui4}{12;15}
  \significado{v.}{ficar bêbado}
\end{verbete}

\begin{verbete}{合同}{he2tong5}{6;6}
  \significado[个]{s.}{contrato (negócio)}
\end{verbete}

\begin{verbete}{合资}{he2zi1}{6;10}
  \significado{s.}{\emph{joint-venture} com capitais mistos}
\end{verbete}

\begin{verbete}{合作}{he2zuo4}{6;7}
  \significado[个]{s.}{cooperação}
  \significado{v.}{cooperar; colaborar}
\end{verbete}

\begin{verbete}{何况}{he2kuang4}{7;7}
  \significado{conj.}{além disso; muito menos}
\end{verbete}

\begin{verbete}{和}{he2}{8}
  \significado*{s.}{sobrenome He}
  \significado{conj.}{e (somente para palavras)}
  \veja{和}{he4}
  \veja{和}{hu2}
  \veja{和}{huo4}
\end{verbete}

\begin{verbete}{河}{he2}{8}
  \significado[条,道]{s.}{rio}
\end{verbete}

\begin{verbete}{盒}{he2}{11}
  \significado{p.c.}{caixa pequena}
  \significado{s.}{caixa pequena; estojo}
\end{verbete}

\begin{verbete}{和}{he4}{8}
  \significado{v.}{conversar com os outros; compor um poema em resposta (ao poema de alguém) usando a mesma sequência de rimas; juntar-se ao canto (canção)}
  \veja{和}{he2}
  \veja{和}{hu2}
  \veja{和}{huo4}
\end{verbete}

\begin{verbete}{喝}{he4}{12}
  \significado{v.}{gritar bem alto}
  \veja{喝}{he1}
\end{verbete}

\begin{verbete}{褐色}{he4se4}{14;6}
  \significado{s.}{cor marrom}
\end{verbete}

\begin{verbete}{黑}{hei1}{12}[203]
  \significado{adj.}{preto; escuro; ilegal; secreto; sombrio; sinistro}
  \significado{v.}{esconder (algo); difamar; hackear (computador)}
\end{verbete}

\begin{verbete}{黑板}{hei1ban3}{12;8}
  \significado[块,个]{s.}{quadro negro}
\end{verbete}

\begin{verbete}{黑色}{hei1se4}{12;6}
  \significado{s.}{cor preta}
\end{verbete}

\begin{verbete}{很}{hen3}{9}
  \significado{adv.}{muito; mui; advérbio de grau}
\end{verbete}

\begin{verbete}{横竖}{heng2shu5}{15;9}
  \significado{adv.}{de qualquer maneira; independentemente (linguagem falada)}
\end{verbete}

\begin{verbete}{红}{hong2}{6}
  \significado*{s.}{sobrenome Hong}
  \significado{adj.}{vermelho; popular; revolucionário}
  \significado{s.}{bônus}
\end{verbete}

\begin{verbete}{红茶}{hong2cha2}{6;9}
  \significado[杯,壶]{s.}{chá preto}
\end{verbete}

\begin{verbete}{红绿灯}{hong2lv4deng1}{6;11;6}
  \significado[个]{s.}{semáforo; sinal de trânsito}
\end{verbete}

\begin{verbete}{红色}{hong2se4}{6;6}
  \significado{s.}{cor vermelha}
\end{verbete}

\begin{verbete}{红烧}{hong2shao1}{6;10}
  \significado{s.}{guisado em molho de soja (prato)}
\end{verbete}

\begin{verbete}{后边}{hou4bian5}{6;5}
  \significado{p.l.}{atrás; detrás}
\end{verbete}

\begin{verbete}{后来}{hou4lai2}{6;7}
  \significado{p.t.}{mais tarde}
\end{verbete}

\begin{verbete}{后面}{hou4mian5}{6;9}
  \significado{p.l.}{atrás; detrás}
\end{verbete}

\begin{verbete}{后年}{hou4nian2}{6;6}
  \significado{p.t.}{daqui a dois anos}
\end{verbete}

\begin{verbete}{后天}{hou4tian1}{6;4}
  \significado{p.t.}{depois de amanhã}
\end{verbete}

\begin{verbete}{忽然}{hu1ran2}{8;12}
  \significado{adv.}{de repente}
\end{verbete}

\begin{verbete}{和}{hu2}{8}
  \significado{v.}{completar um conjunto de Mahjong ou cartas de baralho}
  \veja{和}{he2}
  \veja{和}{he4}
  \veja{和}{huo4}
\end{verbete}

\begin{verbete}{胡萝卜}{hu2luo2bo5}{9;11;2}
  \significado{s.}{cenoura}
\end{verbete}

\begin{verbete}{湖}{hu2}{12}
  \significado[个,片]{s.}{lago}
\end{verbete}

\begin{verbete}{湖南}{hu2nan2}{12;9}
  \significado*{s.}{Hunan}
\end{verbete}

\begin{verbete}{糊里糊涂}{hu2li5hu2tu5}{15;7;15;10}
  \significado{adj.}{desnorteado; perturbado}
\end{verbete}

\begin{verbete}{虎}{hu3}{8}
  \significado{s.}{tigre}
\end{verbete}

\begin{verbete}{虎虎}{hu3hu3}{8;8}
  \significado{adj.}{formidável; forte; vigoroso}
\end{verbete}

\begin{verbete}{虎口}{hu3kou3}{8;3}
  \significado{s.}{lugar perigoso; cova do tigre}
\end{verbete}

\begin{verbete}{虎鼬}{hu3you4}{8;18}
  \significado{s.}{doninha}
\end{verbete}

\begin{verbete}{互相}{hu4xiang1}{4;9}
  \significado{adv.}{mutuamente, um ao outro}
\end{verbete}

\begin{verbete}{护照}{hu4zhao4}{7;13}
  \significado[本,个]{s.}{passaporte}
\end{verbete}

\begin{verbete}{化}{hua1}{4}
  \variante{花}{hua1}
\end{verbete}

\begin{verbete}{花}{hua1}{7}
  \significado*{s.}{sobrenome Hua}
  \significado[朵,支,束,把,盆,簇]{s.}{flor}
\end{verbete}

\begin{verbete}{花茶}{hua1cha2}{7;9}
  \significado[杯,壶]{s.}{chá perfumado}
\end{verbete}

\begin{verbete}{花儿}{hua1r5}{7;2}
  \significado[朵,支,束,把,盆,簇]{s.}{flor}
\end{verbete}

\begin{verbete}{花生}{hua1sheng1}{7;5}
  \significado[粒]{s.}{amendoim}
\end{verbete}

\begin{verbete}{花椰菜}{hua1ye1cai4}{7;12;11}
  \significado{s.}{couve-flor}
\end{verbete}

\begin{verbete}{华盛顿}{hua2sheng4dun4}{6;11;10}
  \significado*{s.}{Washington}
\end{verbete}

\begin{verbete}{华氏}{hua2shi4}{6;4}
  \significado{s.}{graus Fahrenheit (°F)}
\end{verbete}

\begin{verbete}{华裔}{hua2yi4}{6;13}
  \significado{s.}{descendente de chinês}
\end{verbete}

\begin{verbete}{滑}{hua2}{12}
  \significado*{s.}{sobrenome Hua}
  \significado{adj.}{deslizado}
  \significado{v.}{deslizar}
\end{verbete}

\begin{verbete}{滑雪}{hua2xue3}{12;11}
  \significado{v.+compl.}{esquiar; fazer esqui}
\end{verbete}

\begin{verbete}{话}{hua4}{8}
  \significado[种,席,句,口,番]{s.}{fala; linguagem; dialeto}
\end{verbete}

\begin{verbete}{坏}{huai4}{7}
  \significado{adj.}{avariado; mau}
  \significado{v.}{perder o controle}
\end{verbete}

\begin{verbete}{坏蛋}{huai4dan4}{7;11}
  \significado{s.}{bastardo; canalha; pessoa má}
\end{verbete}

\begin{verbete}{欢迎}{huan1ying2}{6;7}
  \significado{v.}{dar as boas-vindas; ser bem-vindo}
\end{verbete}

\begin{verbete}{环境}{huan1jing4}{8;14}
  \significado[个]{s.}{ambiente; arredores; circunstâncias}
\end{verbete}

\begin{verbete}{还}{huan2}{7}
  \significado*{s.}{sobrenome Huan}
  \significado{v.}{devolver; restituir; pagar de volta}
  \veja{还}{hai2}
\end{verbete}

\begin{verbete}{换}{huan4}{10}
  \significado{v.}{mudar; trocar; substituir; converter (moedas)}
\end{verbete}

\begin{verbete}{黄}{huang2}{11}
  \significado*{s.}{sobrenome Huang ou Hwang}
  \significado{adj.}{amarelo; pornográfico}
\end{verbete}

\begin{verbete}{黄瓜}{huang2gua1}{11;5}
  \significado[条]{s.}{pepino}
\end{verbete}

\begin{verbete}{黄色}{huang2se4}{11;6}
  \significado{s.}{cor amarela}
\end{verbete}

\begin{verbete}{黄油}{huang2you2}{11;8}
  \significado[盒]{s.}{manteiga}
\end{verbete}

\begin{verbete}{灰色}{hui1se4}{6;6}
  \significado{s.}{cor cinza}
\end{verbete}

\begin{verbete}{囘}{hui2}{5}
  \variante{回}{hui2}
\end{verbete}

\begin{verbete}{回}{hui2}{6}
  \significado{v.d.}{regressar}
\end{verbete}

\begin{verbete}{回答}{hui2da2}{6;12}
  \significado{v.}{responder}
\end{verbete}

\begin{verbete}{回来}{hui2lai5}{6;7}
  \significado{v.d.}{regressar; voltar; estar de volta; (para a minha localização)}
\end{verbete}

\begin{verbete}{回去}{hui2qu5}{6;5}
  \significado{v.d.}{regressar; voltar; estar de volta; (a partir da minha localização)}
\end{verbete}

\begin{verbete}{廻}{hui2}{8}
  \variante{回}{hui2}
\end{verbete}

\begin{verbete}{会}{hui4}{6}
  \significado{v.}{saber; ter habilidade; saber como fazer}
  \veja{会}{kuai4}
\end{verbete}

\begin{verbete}{婚礼}{hun1li3}{11;5}
  \significado{s.}{casamento; núpcias; cerimônia de casamento}
\end{verbete}

\begin{verbete}{活动}{huo2dong4}{9;6}
  \significado[项,个]{s.}{atividade; evento; campanha}
  \significado{v.}{exercer; operar}
\end{verbete}

\begin{verbete}{火车}{huo3che1}{4;4}
  \significado[列,节,班,趟]{s.}{trem; comboio}
\end{verbete}

\begin{verbete}{火车司机}{huo3che1·si1ji1}{4;4;5;6}
  \significado{s.}{maquinista de trem}
\end{verbete}

\begin{verbete}{和}{huo4}{8}
  \significado{p.c.}{para fervuras de ervas medicinais; para enxágue de roupas}
  \significado{v.}{misturar; misturar (ingredientes) juntos}
  \veja{和}{he2}
  \veja{和}{he4}
  \veja{和}{hu2}
\end{verbete}

\begin{verbete}{或}{huo4}{8}
  \significado{conj.}{ou; ou... ou...}
\end{verbete}

\begin{verbete}{或者}{huo4zhe3}{8;8}
  \significado{conj.}{ou (usado em expressões afirmativas)}
\end{verbete}

\begin{verbete}{惑星}{huo4xing1}{12;9}
  \significado{s.}{planeta}
  \veja{行星}{xing2xing1}
\end{verbete}

%%%%% EOF %%%%%

%%%%%%%%%%%%%%%%%%%% Não existem palavras com pinyin iniciado em "I"
%%%
%%% J
%%%
%\section*{J}
\addcontentsline{toc}{section}{J}

\begin{verbete}{几}{ji1}{2}[16]
  \significado{adv.}{quase}
  \significado{s.}{mesa pequena}
  \veja{几}{ji3}
\end{verbete}

\begin{verbete}{几乎}{ji1hu1}{2;5}
  \significado{adv.}{quase}
\end{verbete}

\begin{verbete}{机场}{ji1chang3}{6;6}
  \significado[家,处]{s.}{aeroporto; aeródromo}
\end{verbete}

\begin{verbete}{机甲}{ji1jia3}{6;5}
  \significado{s.}{\emph{mecha} (robôs operados pelo homem em mangá japonês)}
\end{verbete}

\begin{verbete}{机票}{ji1piao4}{6;11}
  \significado[张]{s.}{bilhete de avião}
  \veja{飞机票}{fei1ji1piao4}
\end{verbete}

\begin{verbete}{机器}{ji1qi4}{6;16}
  \significado[台,部,个]{s.}{máquina}
\end{verbete}

\begin{verbete}{机器人}{ji1qi4ren2}{6;16;2}
  \significado{s.}{robô; androide}
\end{verbete}

\begin{verbete}{肌肉}{ji1rou4}{6;6}
  \significado{s.}{músculo, carne}
\end{verbete}

\begin{verbete}{鸡}{ji1}{7}
  \significado[只]{s.}{galo, galinha; gíria:~prostituta}
\end{verbete}

\begin{verbete}{鸡蛋}{ji1dan4}{7;11}
  \significado[个,打]{s.}{ovo de galinha}
\end{verbete}

\begin{verbete}{积木}{ji1mu4}{10;4}
  \significado{s.}{blocos de montar (brinquedo)}
\end{verbete}

\begin{verbete}{基本功}{ji1ben3gong1}{11;5;5}
  \significado{s.}{habilidades; fundamentos básicos}
\end{verbete}

\begin{verbete}{基督教}{ji1du1jiao4}{11;13;11}
  \significado*{s.}{Cristianismo; Cristão}
\end{verbete}

\begin{verbete}{基因}{ji1yin1}{11;6}
  \significado{s.}{gene}
\end{verbete}

\begin{verbete}{激动}{ji1dong4}{16;6}
  \significado{v.}{excitar;  mover-se emocionalmente; agitar (emoções)}
\end{verbete}

\begin{verbete}{鷄}{ji1}{21}
  \variante{鸡}
\end{verbete}

\begin{verbete}{及}{ji2}{3}
  \significado{conj.}{e; bem como}
\end{verbete}

\begin{verbete}{及格}{ji2ge2}{3;10}
  \significado{v.}{atender a um padrão mínimo; passar em um exame ou teste}
\end{verbete}

\begin{verbete}{吉他}{ji2ta1}{6;5}
  \significado[把]{s.}{guitarra (empréstimo linguístico)}
\end{verbete}

\begin{verbete}{即}{ji2}{7}
  \significado{conj.}{e; até; mesmo se/embora}
\end{verbete}

\begin{verbete}{即便}{ji2bian4}{7;9}
  \significado{conj.}{mesmo se/embora}
\end{verbete}

\begin{verbete}{即或}{ji2huo4}{7;8}
  \significado{conj.}{mesmo se/embora}
\end{verbete}

\begin{verbete}{即若}{ji2ruo4}{7;8}
  \significado{conj.}{mesmo se/embora}
\end{verbete}

\begin{verbete}{即使}{ji2shi3}{7;8}
  \significado{conj.}{mesmo se/embora}
\end{verbete}

\begin{verbete}{即是}{ji2shi4}{7;9}
  \significado{conj.}{aquilo é}
\end{verbete}

\begin{verbete}{……极了}{ji2le5}{7;2}
  \significado{expr.}{muito; extremamente}
\end{verbete}

\begin{verbete}{极其}{ji2qi2}{7;8}
  \significado{adv.}{extremamente, muito}
\end{verbete}

\begin{verbete}{集体}{ji2ti3}{12;7}
  \significado{s.}{coletivo (decisão); esforço (conjunto); um grupo; uma equipe}
\end{verbete}

\begin{verbete}{集团}{ji2tuan2}{12;6}
  \significado{s.}{grupo; bloco; corporação; conglomerado}
\end{verbete}

\begin{verbete}{几}{ji3}{2}
  \significado{interr.}{quantos?, (até 10 itens); alguns?}
  \veja{几}{ji1}
\end{verbete}

\begin{verbete}{给}{ji3}{9}
  \significado{v.}{fornecer; prover}
  \veja{给}{gei3}
\end{verbete}

\begin{verbete}{记得}{ji4de5}{5;11}
  \significado{v.}{lembrar; lembrar-se}
\end{verbete}

\begin{verbete}{记性}{ji4xing5}{5;8}
  \significado{s.}{memória (habilidade em reter informações)}
\end{verbete}

\begin{verbete}{记住}{ji4-zhu4}{5;7}
  \significado{v.}{decorar; memorizar; ter em mente}
\end{verbete}

\begin{verbete}{技俩}{ji4liang3}{7;9}
  \significado{s.}{truque; estratagema; ardil; esquema; estratégia; tática}
\end{verbete}

\begin{verbete}{技术}{ji4shu4}{7;5}
  \significado[门,种,项]{s.}{tecnologia; técnica; habilidade}
\end{verbete}

\begin{verbete}{季节}{ji4jie2}{8;5}
  \significado[个]{s.}{estação (clima)}
\end{verbete}

\begin{verbete}{既}{ji4}{9}
  \significado{conj.}{desde; como; agora isso; os dois e; assim como}
\end{verbete}

\begin{verbete}{既不……又不……}{ji4bu4 you4bu4}{9;4;2;4}
  \significado{conj.}{nem mesmo\dots}
\end{verbete}

\begin{verbete}{既然}{ji4ran2}{9;12}
  \significado{part.}{agora isso; desde; como}
\end{verbete}

\begin{verbete}{既又}{ji4you4}{9;2}
  \significado{conj.}{desde; como; agora isso; os dois e; assim como}
\end{verbete}

\begin{verbete}{寂寥}{ji4liao2}{11;14}
  \significado{s.}{solidão; vasto e vazio; quieto e desolado (literário)}
\end{verbete}

\begin{verbete}{寂寞}{ji4mo4}{11;13}
  \significado{adj.}{sozinho; solitário; (de um lugar) silencioso;}
\end{verbete}

\begin{verbete}{寄}{ji4}{11}
  \significado{v.}{enviar; mandar}
\end{verbete}

\begin{verbete}{寄存}{ji4cun2}{11;6}
  \significado{v.}{depositar; deixar algo com alguém; armazenar}
\end{verbete}

\begin{verbete}{寄递}{ji4di4}{11;10}
  \significado{s.}{entrega de correspondência}
\end{verbete}

\begin{verbete}{寄放}{ji4fang4}{11;8}
  \significado{v.}{deixar algo com alguém}
\end{verbete}

\begin{verbete}{寄居}{ji4ju1}{11;8}
  \significado{s.}{morar longe de casa}
\end{verbete}

\begin{verbete}{寄卖}{ji4mai4}{11;8}
  \significado{v.}{consignar para venda}
\end{verbete}

\begin{verbete}{寄生}{ji4sheng1}{11;5}
  \significado{s.}{parasita; parasitismo}
  \significado{v.}{viver tirando vantagem dos outros; viver dentro ou sobre outro organismo como um parasita}
\end{verbete}

\begin{verbete}{寄生生活}{ji4sheng1sheng1huo2}{11;5;5;9}
  \significado{s.}{parasitismo; vida parasitária}
\end{verbete}

\begin{verbete}{寄售}{ji4shou4}{11;11}
  \significado{v.}{venda em consignação}
\end{verbete}

\begin{verbete}{寄送}{ji4song4}{11;9}
  \significado{v.}{enviar; transmitir}
\end{verbete}

\begin{verbete}{寄宿}{ji4su4}{11;11}
  \significado{s.}{embarque}
  \significado{v.}{embarcar}
\end{verbete}

\begin{verbete}{寄托}{ji4tuo1}{11;6}
  \significado{v.}{investir (sua esperança, energia, etc.) em algo; confiar (a alguém); colocar (a esperança, a energia, etc.) em}
\end{verbete}

\begin{verbete}{寄望}{ji4wang4}{11;11}
  \significado{v.}{depositar esperanças em}
\end{verbete}

\begin{verbete}{寄养}{ji4yang3}{11;9}
  \significado{v.}{embarcar; promover; colocar sob os cuidados de alguém (uma criança, animal de estimação, etc.)}
\end{verbete}

\begin{verbete}{寄予}{ji4yu3}{11;4}
  \significado{v.}{expressar; colocar (esperança, importância, etc.) em; mostrar}
\end{verbete}

\begin{verbete}{旣}{ji4}{11}
  \variante{既}
\end{verbete}

\begin{verbete}{加}{jia1}{5}
  \significado*{s.}{Canadá, abreviação de~加拿大; sobrenome Jia}
  \veja{加拿大}{jia1na2da4}
\end{verbete}

\begin{verbete}{加拿大}{jia1na2da4}{5;10;3}
  \significado{s.}{Canadá}
  \veja{加}{jia1}
\end{verbete}

\begin{verbete}{加拿大人}{jia1na2da4ren2}{5;10;3;2}
  \significado{s.}{canadense; pessoa nascida no Canadá}
\end{verbete}

\begin{verbete}{家}{jia1}{10}
  \significado{p.c.}{para famílias ou empresas}
  \significado[个]{s.}{família; casa; sufixo de substantivos para designar um especialista em alguma atividade}
\end{verbete}

\begin{verbete}{家伙}{jia1huo5}{10;6}
  \significado{s.}{prato, implemento ou móvel doméstico; animal doméstico; (coloquial) o cara; indivíduo; arma}
\end{verbete}

\begin{verbete}{家具}{jia1ju4}{10;8}
  \significado[件,套]{s.}{móveis; mobiliário}
\end{verbete}

\begin{verbete}{家俱}{jia1ju4}{10;10}
  \variante{家具}
\end{verbete}

\begin{verbete}{家里}{jia1li3}{10;7}
  \significado{p.d.l.}{em casa}
\end{verbete}

\begin{verbete}{家乡}{jia1xiang1}{10;3}
  \significado[个]{s.}{terra natal; cidade natal}
\end{verbete}

\begin{verbete}{傢具}{jia1ju4}{12;8}
  \variante{家具}
\end{verbete}

\begin{verbete}{嘉年华}{jia1nian2hua2}{14;6;6}
  \significado{s.}{carnaval (empréstimo linguístico)}
\end{verbete}

\begin{verbete}{甲骨文}{jia3gu3wen2}{5;9;4}
  \significado{s.}{escrituras de oráculos; inscrições em ossos de oráculos (forma original de escritura chinesa)}
\end{verbete}

\begin{verbete}{假如}{jia3ru2}{11;6}
  \significado{conj.}{se; supondo; em caso}
\end{verbete}

\begin{verbete}{假声}{jia3sheng1}{11;7}
  \significado{s.}{falsete}
  \veja{真声}{zhen1sheng1}
\end{verbete}

\begin{verbete}{假使}{jia3shi3}{11;8}
  \significado{conj.}{se; supondo; em caso}
\end{verbete}

\begin{verbete}{假证件}{jia3zheng4jian4}{11;7;6}
  \significado{s.}{documentos falsos}
\end{verbete}

\begin{verbete}{驾照}{jia4zhao4}{8;13}
  \significado{s.}{carteira de motorista}
\end{verbete}

\begin{verbete}{架式}{jia4shi5}{9;6}
  \variante{架势}
\end{verbete}

\begin{verbete}{架势}{jia4shi5}{9;8}
  \significado{s.}{postura; atitude; posição (sobre um assunto, etc.)}
\end{verbete}

\begin{verbete}{坚持}{jian1chi2}{7;9}
  \significado{s.}{perseverar com; persistir em; insistir em}
\end{verbete}

\begin{verbete}{坚守}{jian1shou3}{7;6}
  \significado{v.}{agarrar-se}
\end{verbete}

\begin{verbete}{肩膀}{jian1bang3}{8;14}
  \significado{s.}{ombro}
\end{verbete}

\begin{verbete}{兼}{jian1}{10}
  \significado{conj.}{e (ocupando dois ou mais cargos (oficiais) ao memso tempo)}
\end{verbete}

\begin{verbete}{监狱}{jian1yu4}{10;9}
  \significado{s.}{prisão}
\end{verbete}

\begin{verbete}{煎}{jian1}{13}
  \significado{v.}{fritar; refogar}
\end{verbete}

\begin{verbete}{煎饼}{jian1bing3}{13;9}
  \significado[张]{s.}{jianbing, crepe chinês; panqueca}
\end{verbete}

\begin{verbete}{煎蛋}{jian1dan4}{13;11}
  \significado{s.}{ovos fritos}
\end{verbete}

\begin{verbete}{俭省}{jian3sheng3}{9;9}
  \significado{adj.}{econômico}
\end{verbete}

\begin{verbete}{捡}{jian3}{10}
  \significado{v.}{apanhar; recolher; coletar}
\end{verbete}

\begin{verbete}{检查}{jian3cha2}{11;9}
  \significado[次]{s.}{inspeção}
  \significado{v.}{examinar; inspecionar}
\end{verbete}

\begin{verbete}{简单}{jian3dan1}{13;8}
  \significado{adj.}{simples; sem complicações}
\end{verbete}

\begin{verbete}{简直}{jian3zhi2}{13;8}
  \significado{adv.}{simplesmente; realmente; absolutamente; em tudo}
\end{verbete}

\begin{verbete}{见}{jian4}{4}
  \significado{v.}{ver; entrevistar; encontrar alguém; parecer (ser alguma coisa)}
  \veja{见}{xian4}
\end{verbete}

\begin{verbete}{见面}{jian4mian4}{4;9}
  \significado{v.}{encontrar-se com alguém}
\end{verbete}

\begin{verbete}{件}{jian4}{6}
  \significado{p.c.}{para eventos, coisas, roupas etc.}
  \significado{s.}{item; componente}
\end{verbete}

\begin{verbete}{间或}{jian4huo4}{7;8}
  \significado{adv.}{às vezes; ocasionalmente; de vez em quando}
\end{verbete}

\begin{verbete}{间接}{jian4jie1}{7;11}
  \significado{adj.}{indireto}
  \veja{直接}{zhi2jie1}
\end{verbete}

\begin{verbete}{建立者}{jian4li4zhe3}{8;5;8}
  \significado{s.}{fundador}
\end{verbete}

\begin{verbete}{建设}{jian4she4}{8;6}
  \significado{s.}{construção}
  \significado{v.}{construir}
\end{verbete}

\begin{verbete}{建设性}{jian4she4xing4}{8;6;8}
  \significado{adj.}{construtivo}
  \significado{s.}{construtividade}
\end{verbete}

\begin{verbete}{建设者}{jian4she4zhe3}{8;6;8}
  \significado{s.}{construtor}
\end{verbete}

\begin{verbete}{建议}{jian4yi4}{8;5}
  \significado[个,点]{s.}{proposta; recomendação; sugestão}
  \significado{v.}{propor; recomendar; sugerir}
\end{verbete}

\begin{verbete}{建筑}{jian4zhu4}{8;12}
  \significado[个]{s.}{construção; prédio; edifício}
  \significado{v.}{construir}
\end{verbete}

\begin{verbete}{剑客}{jian4ke4}{9;9}
  \significado{s.}{espada; esgrimista, espadachim}
\end{verbete}

\begin{verbete}{健身}{jian4shen1}{10;7}
  \significado{s.}{exercício físico; \emph{fitness}}
  \significado{v.}{exercitar-se; manter a forma}
\end{verbete}

\begin{verbete}{渐渐}{jian4jian4}{11;11}
  \significado{adv.}{pouco a pouco; passo a passo; progressivamente}
\end{verbete}

\begin{verbete}{江南水乡}{jiang1nan2shui3xiang1}{6;9;4;3}
  \significado*{s.}{Vila Aquática de Jiangnan; Cidades Aquáticas}
\end{verbete}

\begin{verbete}{江西}{jiang1xi1}{6;6}
  \significado*{s.}{Jiangxi}
\end{verbete}

\begin{verbete}{姜}{jiang1}{9}
  \significado*{s.}{sobrenome Jiang}
  \significado{s.}{gengibre}
\end{verbete}

\begin{verbete}{将要}{jiang1yao4}{9;9}
  \significado{adv.}{vai, deve}
\end{verbete}

\begin{verbete}{讲述}{jiang3shu4}{6;8}
  \significado{v.}{falar sobre, narrar, descrever}
\end{verbete}

\begin{verbete}{强}{jiang4}{12}
  \significado{adj.}{teimoso; inflexível}
  \veja{强}{qiang2}
  \veja{强}{qiang3}
\end{verbete}

\begin{verbete}{酱}{jiang4}{13}
  \significado{s.}{pasta grossa de soja fermentada; marinada em pasta de soja; pasta; geléia}
\end{verbete}

\begin{verbete}{犟}{jiang4}{16}
  \variante{强}
\end{verbete}

\begin{verbete}{交}{jiao1}{6}
  \significado{v.}{entregar; dar}
\end{verbete}

\begin{verbete}{交班}{jiao1ban1}{6;10}
  \significado{v.}{passar para o próximo turno de trabalho}
\end{verbete}

\begin{verbete}{交杯酒}{jiao1bei1jiu3}{6;8;10}
  \significado{s.}{copo de vinho nupcial}
\end{verbete}

\begin{verbete}{交叉}{jiao1cha1}{6;3}
  \significado{v.}{cruzar; sobrepor}
\end{verbete}

\begin{verbete}{交叉点}{jiao1cha1dian3}{6;3;9}
  \significado{s.}{encruzilhada; cruzamento; junção}
\end{verbete}

\begin{verbete}{交叉口}{jiao1cha1kou3}{6;3;3}
  \significado{s.}{intersecção (rodovia)}
\end{verbete}

\begin{verbete}{交叠}{jiao1die2}{6;13}
  \significado{s.}{sobreposição}
\end{verbete}

\begin{verbete}{交给}{jiao1gei3}{6;9}
  \significado{v.}{entregar algo; dar algo}
\end{verbete}

\begin{verbete}{交媾}{jiao1gou4}{6;13}
  \significado{v.}{copular; ter relações sexuais}
\end{verbete}

\begin{verbete}{交界}{jiao1jie4}{6;9}
  \significado{p.l.}{fronteira comum; limite comum; interface}
\end{verbete}

\begin{verbete}{交警}{jiao1jing3}{6;19}
  \significado{s.}{policial de trânsito (abreviatura de 交通警察)}
  \veja{交通警察}{jiao1tong1jing3cha2}
\end{verbete}

\begin{verbete}{交通}{jiao1tong1}{6;10}
  \significado{s.}{transporte; tráfego; trânsito; comunicação; conexão}
  \significado{v.}{estar conectado; ser conectado}
\end{verbete}

\begin{verbete}{交通警察}{jiao1tong1jing3cha2}{6;10;19;14}
  \significado{s.}{policial de trânsito}
  \veja{交警}{jiao1jing3}
\end{verbete}

\begin{verbete}{交响}{jiao1xiang3}{6;9}
  \significado{s.}{sinfonia}
\end{verbete}

\begin{verbete}{交运}{jiao1yun4}{6;7}
  \significado{v.}{despachar (bagagem em um aeroporto, etc.); entregar para transporte}
\end{verbete}

\begin{verbete}{郊区}{jiao1qu1}{8;4}
  \significado[个]{s.}{subúrbio; distrito suburbano; arredores}
\end{verbete}

\begin{verbete}{胶卷}{jiao1juan3}{10;8}
  \significado{s.}{filme; rolo de filme}
\end{verbete}

\begin{verbete}{教}{jiao1}{11}
  \significado{v.}{ensinar; lecionar}
  \veja{教}{jiao4}
\end{verbete}

\begin{verbete}{教学}{jiao1xue2}{11;8}
  \significado{v.}{ensinar (como um professor)}
  \veja{教学}{jiao4xue2}
\end{verbete}

\begin{verbete}{焦虑}{jiao1lv4}{12;10}
  \significado{adj.}{ansioso; preocupado; apreensivo}
\end{verbete}

\begin{verbete}{角}{jiao3}{7}[148]
  \significado{p.c.}{1 jiao = 10 centavos}
  \significado[个]{s.}{ângulo; esquina; chifre; em forma de chifre}
  \veja{角}{jue2}
\end{verbete}

\begin{verbete}{饺子}{jiao3zi5}{9;3}
  \significado[个,只]{s.}{jiaozi; bolinhos chineses; bolinho de massa}
\end{verbete}

\begin{verbete}{脚}{jiao3}{11}
  \significado{p.c.}{para chutes}
  \significado[双,只]{s.}{pé; base (de um objeto); perna (de um animal ou objeto)}
\end{verbete}

\begin{verbete}{叫}{jiao4}{5}
  \significado{v.}{chamar-se; chamar; gritar; pedir (comida em um restaurante)}
\end{verbete}

\begin{verbete}{校}{jiao4}{10}
  \significado{v.}{verificar; comparar; revisar}
  \veja{校}{xiao4}
\end{verbete}

\begin{verbete}{敎}{jiao4}{11}
  \variante{教}
\end{verbete}

\begin{verbete}{教}{jiao4}{11}
  \significado*{s.}{sobrenome Jiao}
  \significado{s.}{religião; ensinamento}
  \significado{v.}{causar; como fazer algo; contar (explicar como fazer algo)}
  \veja{教}{jiao1}
\end{verbete}

\begin{verbete}{教导}{jiao4dao3}{11;6}
  \significado{s.}{orientação; ensino}
  \significado{v.}{instruir; ensinar}
\end{verbete}

\begin{verbete}{教官}{jiao4guan1}{11;8}
  \significado{s.}{instrutor militar}
\end{verbete}

\begin{verbete}{教练}{jiao4lian4}{11;8}
  \significado[个,位,名]{s.}{instrutor; treinador (esportes)}
\end{verbete}

\begin{verbete}{教师}{jiao4shi1}{11;6}
  \significado[个]{s.}{professor; mestre}
\end{verbete}

\begin{verbete}{教室}{jiao4shi4}{11;9}
  \significado[间]{s.}{sala de aula}
\end{verbete}

\begin{verbete}{教授}{jiao4shou4}{11;11}
  \significado[个,位]{s.}{professor (universitário)}
  \significado{v.}{instruir; palestrar sobre}
\end{verbete}

\begin{verbete}{教堂}{jiao4tang2}{11;11}
  \significado[间]{s.}{igreja; capela}
\end{verbete}

\begin{verbete}{教学}{jiao4xue2}{11;8}
  \significado[次]{s.}{ensino; instrução}
  \veja{教学}{jiao1xue2}
\end{verbete}

\begin{verbete}{教学楼}{jiao4xue2lou2}{11;8;13}
  \significado{s.}{edifício de salas de aula}
\end{verbete}

\begin{verbete}{教长}{jiao4zhang3}{11;4}
  \significado{s.}{imã (Islã); mulá}
\end{verbete}

\begin{verbete}{皆}{jie1}{9}
  \significado{adv.}{todos; em todos os casos}
\end{verbete}

\begin{verbete}{结}{jie1}{9}
  \significado{v.}{(uma planta) produzir (frutos ou sementes)}
  \veja{结}{jie2}
\end{verbete}

\begin{verbete}{结果}{jie1guo3}{9;8}
  \significado{v.}{dar frutos}
  \veja{结果}{jie2guo3}
\end{verbete}

\begin{verbete}{接}{jie1}{11}
  \significado{v.}{ir buscar (alguém); ir ao encontro de (alguém); receber}
\end{verbete}

\begin{verbete}{接班人}{jie1ban1ren2}{11;10;2}
  \significado{s.}{sucessor}
\end{verbete}

\begin{verbete}{接待}{jie1dai4}{11;9}
  \significado{v.}{receber (alguém); acolher; recepcionar}
\end{verbete}

\begin{verbete}{接(电话)}{jie1(dian4hua4)}{11;5;8}
  \significado{v.}{atender (o telefone)}
\end{verbete}

\begin{verbete}{街}{jie1}{12}
  \significado[条]{s.}{rua}
\end{verbete}

\begin{verbete}{节日}{jie2ri4}{5;4}
  \significado[个]{s.}{festival; feriado}
\end{verbete}

\begin{verbete}{节奏}{jie2zou4}{5;9}
  \significado{s.}{ritmo; cadência; batida}
\end{verbete}

\begin{verbete}{结}{jie2}{9}
  \significado{s.}{nó}
  \veja{结}{jie1}
\end{verbete}

\begin{verbete}{结果}{jie2guo3}{9;8}
  \significado{s.}{resultado; conclusão}
  \significado{v.}{despachar; matar}
  \veja{结果}{jie1guo3}
\end{verbete}

\begin{verbete}{结婚}{jie2hun1}{9;11}
  \significado{v.}{casar}
\end{verbete}

\begin{verbete}{结婚礼服}{jie2hun1 li3 fu2}{9;11;5;8}
  \significado{s.}{vestido de casamento}
\end{verbete}

\begin{verbete}{结束}{jie2shu4}{9;7}
  \significado{v.}{terminar; acabar; concluir}
\end{verbete}

\begin{verbete}{结束辩论}{jie2shu4 bian4 lun4}{9;7;16;6}
  \significado{s.}{debate de encerramento}
\end{verbete}

\begin{verbete}{结束工作}{jie2shu4gong1zuo4}{9;7;3;7}
  \significado{s.}{trabalho final}
  \significado{v.}{terminar o trabalho}
\end{verbete}

\begin{verbete}{结束剂}{jie2shu4 ji4}{9;7;8}
  \significado{s.}{finalizador}
\end{verbete}

\begin{verbete}{结束区}{jie2shu4 qu1}{9;7;4}
  \significado{s.}{zona final}
\end{verbete}

\begin{verbete}{结束文本}{jie2shu4 wen2ben3}{9;7;4;5}
  \significado{s.}{texto final}
\end{verbete}

\begin{verbete}{结束语}{jie2shu4yu3}{9;7;9}
  \significado{s.}{conclusões finais; considerações finais}
\end{verbete}

\begin{verbete}{姐夫}{jie3fu5}{8;4}
  \significado{s.}{marido da irmã mais velha}
\end{verbete}

\begin{verbete}{姐姐}{jie3jie5}{8;8}
  \significado[个]{s.}{irmã mais velha}
\end{verbete}

\begin{verbete}{解释}{jie3shi4}{13;12}
  \significado[个]{s.}{explicação}
  \significado{v.}{explicar; interpretar; resolver}
\end{verbete}

\begin{verbete}{解压}{jie3ya1}{13;6}
  \significado{v.}{aliviar o estresse; (computação) descomprimir}
\end{verbete}

\begin{verbete}{介绍}{jie4shao4}{4;8}
  \significado{s.}{introdução; apresentação}
  \significado{v.}{fazer uma apresentação; apresentar (alguém para alguém); apresentar (alguém para um emprego, etc.)}
\end{verbete}

\begin{verbete}{芥兰}{jie4lan2}{7;5}
  \significado{s.}{couve}
\end{verbete}

\begin{verbete}{界碑}{jie4bei1}{9;13}
  \significado{s.}{marco de fronteira}
\end{verbete}

\begin{verbete}{借}{jie4}{10}
  \significado{adv.}{por meio de}
  \significado{v.}{pedir emprestado; emprestar; aproveitar (uma oportunidade)}
\end{verbete}

\begin{verbete}{借书证}{jie4shu1zheng4}{10;4;7}
  \significado{s.}{cartão de biblioteca; literalmente:~cartão para pedir emprestado livros}
\end{verbete}

\begin{verbete}{今年}{jin1nian2}{4;6}
  \significado{p.t.}{este ano}
\end{verbete}

\begin{verbete}{今天}{jin1tian1}{4;4}
  \significado{p.t.}{hoje}
\end{verbete}

\begin{verbete}{金融}{jin1rong2}{8;16}
  \significado{s.}{finança}
\end{verbete}

\begin{verbete}{金色}{jin1se4}{8;6}
  \significado{s.}{dourado}
\end{verbete}

\begin{verbete}{仅}{jin3}{4}
  \significado{adv.}{apenas; meramente}
\end{verbete}

\begin{verbete}{仅仅}{jin3jin3}{4;4}
  \significado{adv.}{meramente; somente; apenas}
\end{verbete}

\begin{verbete}{尽管}{jin3guan3}{6;14}
  \significado{conj.}{no entanto; embora; apesar de}
\end{verbete}

\begin{verbete}{近}{jin4}{7}
  \significado{adj.}{perto; próximo}
\end{verbete}

\begin{verbete}{进}{jin4}{7}
  \significado{p.c.}{para seções em um edifício ou complexo residencial}
  \significado{s.}{matemática:~base de um sistema numérico}
  \significado{v.d.}{entrar}
\end{verbete}

\begin{verbete}{进步}{jin4bu4}{7;7}
  \significado[个]{s.}{progresso; melhora}
  \significado{v.}{progredir; melhorar}
\end{verbete}

\begin{verbete}{进出口}{jin4chu1kou3}{7;5;3}
  \significado{s.}{importação e exportação}
  \significado{v.}{importar e exportar}
\end{verbete}

\begin{verbete}{进口}{jin4kou3}{7;3}
  \significado{adj.}{importado}
  \significado{s.}{importação; entrada; entrada (para entrada de ar, água, etc.)}
  \significado{v.}{importar}
\end{verbete}

\begin{verbete}{进来}{jin4lai2}{7;7}
  \significado{v.d.}{entrar (para a minha localização)}
\end{verbete}

\begin{verbete}{进去}{jin4qu4}{7;5}
  \significado{v.d.}{entrar (a partir da minha localização)}
\end{verbete}

\begin{verbete}{进行编程}{jin4xing2bian1cheng2}{7;6;12;12}
  \significado{s.}{programa de computador executável}
\end{verbete}

\begin{verbete}{京}{jing1}{8}
  \significado*{s.}{Beijing, abreviação de~北京; sobrenome Jing}
  \significado{s.}{capital de um país}
  \veja{北京}{bei3jing1}
\end{verbete}

\begin{verbete}{京剧}{jing1ju4}{8;10}
  \significado*{s.}{Ópera de Beijing (Pequim)}
\end{verbete}

\begin{verbete}{经}{jing1}{8}
  \significado*{s.}{sobrenome Jing}
  \significado{s.}{livro sagrado; escritura; clássicos; longitude; menstruação; canal}
  \significado{v.}{passar; sofrer; suportar; deformar (têxtil)}
\end{verbete}

\begin{verbete}{经常}{jing1chang2}{8;11}
  \significado{adv.}{constantemente; diariamente; dia-a-dia; todo dia; frequentemente; sempre; regularmente}
\end{verbete}

\begin{verbete}{经过}{jing1guo4}{8;6}
  \significado[个]{s.}{processo; curso}
  \significado{v.}{passar; passar por}
\end{verbete}

\begin{verbete}{经济}{jing1ji4}{8;9}
  \significado{s.}{economia}
\end{verbete}

\begin{verbete}{经理}{jing1li3}{8;11}
  \significado[个,位,名]{s.}{diretor; gerente}
\end{verbete}

\begin{verbete}{精彩}{jing1cai3}{14;11}
  \significado{adj.}{espetacular; maravilhoso; brilhante}
\end{verbete}

\begin{verbete}{精品}{jing1pin3}{14;9}
  \significado{s.}{produtos de qualidade; produto premium; bom trabalho (de arte)}
\end{verbete}

\begin{verbete}{精致}{jing1zhi4}{14;10}
  \significado{adj.}{delicado; exótico; refinado}
\end{verbete}

\begin{verbete}{鲸鲨}{jing1sha1}{16;15}
  \significado{s.}{tubarão baleia}
\end{verbete}

\begin{verbete}{鲸鱼}{jing1yu2}{16;8}
  \significado{s.}{baleia}
\end{verbete}

\begin{verbete}{井}{jing3}{4}[7]
  \significado{adj.}{puro; ordenado}
  \significado[口]{s.}{poço}
\end{verbete}

\begin{verbete}{景色}{jing3se4}{12;6}
  \significado{s.}{paisagem; panorama; vista}
\end{verbete}

\begin{verbete}{警}{jing3}{19}
  \significado{s.}{policial}
  \significado{v.}{alertar; avisar}
\end{verbete}

\begin{verbete}{警察}{jing3cha2}{19;14}
  \significado[个]{s.}{polícia; oficial de polícia}
\end{verbete}

\begin{verbete}{纠葛}{jiu1ge2}{5;12}
  \significado{s.}{emaranhado; disputa}
\end{verbete}

\begin{verbete}{究竟}{jiu1jing4}{7;11}
  \significado{adv.}{afinal; no final; no final das contas; na verdade; exatamente; são ou não são}
\end{verbete}

\begin{verbete}{九}{jiu3}{2}
  \significado{num.}{nove, 9}
\end{verbete}

\begin{verbete}{韭菜}{jiu3cai4}{9;11}
  \significado{s.}{cebolinha chinesa; figurativo:~investidores de varejo que perdem seu dinheiro para operadores mais experientes (ou seja, são ``colhidos'' como cebolinhas)}
\end{verbete}

\begin{verbete}{酒}{jiu3}{10}
  \significado[杯,瓶,罐,桶,缸]{s.}{bebida alcoólica; vinho (especialmente vinho de arroz); aguardente; licor; espíritos}
\end{verbete}

\begin{verbete}{酒馆}{jiu3guan3}{10;11}
  \significado{s.}{bar; taverna; adega}
\end{verbete}

\begin{verbete}{酒鬼}{jiu3gui3}{10;9}
  \significado{adj.}{embriagado; ébrio}
  \significado{s.}{bêbado; alcoólatra; borracho}
\end{verbete}

\begin{verbete}{旧}{jiu4}{5}
  \significado{adj.}{velho; antigo; desgastado (com a idade)}
\end{verbete}

\begin{verbete}{救护车}{jiu4hu4che1}{11;7;4}
  \significado[辆]{s.}{ambulância}
\end{verbete}

\begin{verbete}{救命}{jiu4ming4}{11;8}
  \significado{interj.}{Socorro!; Salve-me!}
  \significado{v.}{salvar a vida de alguém}
\end{verbete}

\begin{verbete}{就}{jiu4}{12}
  \significado{adv.}{exatamente; justamente}
  \significado{v.}{realizar; se envolver em; acompanhar (em alimentos); aproveitar; avançar; empreender}
\end{verbete}

\begin{verbete}{就职}{jiu4zhi2}{12;11}
  \significado{v.}{assumir o cargo, assumir um posto}
\end{verbete}

\begin{verbete}{车}{ju1}{4}
  \significado{s.}{arcaico:~carruagem de guerra; torre (no xadrez)}
  \veja{车}{che1}
\end{verbete}

\begin{verbete}{居然}{ju1ran2}{8;12}
  \significado{adv.}{inesperadamente; na verdade; para surpresa de alguém}
\end{verbete}

\begin{verbete}{橘子汁}{ju2zi5zhi1}{16;3;5}
  \significado[瓶,杯,罐,盒]{s.}{suco de laranja}
  \veja{橙汁}{cheng2zhi1}
  \veja{柳橙汁}{liu3cheng2zhi1}
\end{verbete}

\begin{verbete}{举行}{ju3xing2}{9;6}
  \significado{v.}{realizar (uma reunião, cerimônia, etc.); ter lugar}
\end{verbete}

\begin{verbete}{句}{ju4}{5}
  \significado{p.c.}{para orações, frases ou linhas de versos}
  \significado{s.}{sentença; cláusula; frase}
  \veja{句}{gou4}
\end{verbete}

\begin{verbete}{句子}{ju4zi5}{5;3}
  \significado[个]{s.}{sentença; frase; oração}
\end{verbete}

\begin{verbete}{足}{ju4}{7}
  \significado{adj.}{excessivo}
  \veja{足}{zu2}
\end{verbete}

\begin{verbete}{聚散}{ju4san4}{14;12}
  \significado{s.}{juntos e separados; agregação e dissipação}
\end{verbete}

\begin{verbete}{角}{jue2}{7}
  \significado*{s.}{sobrenome Jue}
  \significado{s.}{papel (teatro)}
  \significado{v.}{competir}
  \veja{角}{jiao3}
\end{verbete}

\begin{verbete}{绝版}{jue2ban3}{9;8}
  \significado{adj.}{esgotado; fora de catálogo}
\end{verbete}

\begin{verbete}{绝不}{jue2bu4}{9;4}
  \significado{adv.}{definitivamente não; de forma alguma; sob nenhuma circunstância}
\end{verbete}

\begin{verbete}{绝对}{jue2dui4}{9;5}
  \significado{adv.}{absolutamente; totalmente; incondicionalmente; definitivamente}
\end{verbete}

\begin{verbete}{觉得}{jue2de5}{9;11}
  \significado{v.}{sentir; pensar; sentir (desconfortável, etc.)}
\end{verbete}

\begin{verbete}{脚}{jue2}{11}
  \variante{角}
\end{verbete}

\begin{verbete}{军人}{jun1ren2}{6;2}
  \significado{s.}{soldado; pessoal militar}
\end{verbete}

%%%%% EOF %%%%%

%%%
%%% K
%%%
%\section*{K}
\addcontentsline{toc}{section}{K}

\begin{verbete}{咖啡}{ka1fei1}{8;11}
  \significado[杯]{s.}{café}
\end{verbete}

\begin{verbete}{咖啡馆}{ka1fei1guan3}{8;11;11}
  \significado[家]{s.}{cafeteria}
\end{verbete}

\begin{verbete}{咖啡色}{ka1fei1se4}{8;11;6}
  \significado{s.}{cor café}
\end{verbete}

\begin{verbete}{卡车司机}{ka3che1·si1ji1}{5;4;5;6}
  \significado{s.}{motorista de caminhão}
\end{verbete}

\begin{verbete}{卡片}{ka3pian4}{5;4}
  \significado{s.}{cartão}
\end{verbete}

\begin{verbete}{卡片游戏}{ka3pian4·you2xi4}{5;4;12;6}
  \significado{s.}{carta de baralho}
\end{verbete}

\begin{verbete}{开}{kai1}{4}
  \significado{p.c.}{quilate (ouro)}
  \significado{v.}{abrir; ligar; dirigir; iniciar (alguma coisa); ferver; escrever  (uma receita, cheque, fatura, etc.)}
\end{verbete}

\begin{verbete}{开车}{kai1che1}{4;4}
  \significado{v.+compl.}{conduzir; dirigir}
\end{verbete}

\begin{verbete}{开尔文}{kai1'er3wen2}{4;5;4}
  \significado{s.}{Kelvin (K, escala de temperatura)}
\end{verbete}

\begin{verbete}{开发区}{kai1fa1qu1}{4;5;4}
  \significado{s.}{zona de desenvolvimento}
\end{verbete}

\begin{verbete}{开始}{kai1shi3}{4;8}
  \significado{adv.}{inicial}
  \significado[个]{s.}{começo; início}
  \significado{v.}{começar; iniciar}
\end{verbete}

\begin{verbete}{开夜车}{kai1ye4che1}{4;8;4}
  \significado{expr.}{trabalho noturno; literalmente:~``conduzir carro à noite''}
\end{verbete}

\begin{verbete}{看}{kan1}{9}
  \significado{v.}{cuidar; vigiar}
  \veja{看}{kan4}
\end{verbete}

\begin{verbete}{砍}{kan3}{9}
  \significado{v.}{cortar}
\end{verbete}

\begin{verbete}{砍刀}{kan3dao1}{9;2}
  \significado{s.}{facão; machete}
\end{verbete}

\begin{verbete}{砍掉}{kan3diao4}{9;11}
  \significado{v.}{amputar}
\end{verbete}

\begin{verbete}{砍断}{kan3duan4}{9;11}
  \significado{v.}{cortar}
\end{verbete}

\begin{verbete}{砍价}{kan3jia4}{9;6}
  \significado{v.}{barganhar; cortar ou derrubar um preço}
\end{verbete}

\begin{verbete}{砍杀}{kan3sha1}{9;6}
  \significado{v.}{atacar com arma branca}
\end{verbete}

\begin{verbete}{砍伤}{kan3shang1}{9;6}
  \significado{v.}{ferir com lâmina ou machado}
\end{verbete}

\begin{verbete}{砍树}{kan3shu4}{9;9}
  \significado{v.}{derrubar árvores}
\end{verbete}

\begin{verbete}{砍死}{kan3si3}{9;6}
  \significado{v.}{matar com um machado}
\end{verbete}

\begin{verbete}{砍头}{kan3tou2}{9;5}
  \significado{v.}{decapitar}
\end{verbete}

\begin{verbete}{看}{kan4}{9}
  \significado{interj.}{Cuidado! (para um perigo)}
  \significado{part.}{(depois de um verbo) tentar}
  \significado{v.}{olhar; ver; assistir; ler; visitar (pessoas)}
  \veja{看}{kan1}
\end{verbete}

\begin{verbete}{看见}{kan4jian4}{9;4}
  \significado{v.}{encontrar; enxergar; ver; avistar}
\end{verbete}

\begin{verbete}{考试}{kao3shi4}{6;8}
  \significado[次]{s.}{teste; prova; exame}
  \significado{v.+compl.}{submeter-se a uma prova; fazer um teste}
\end{verbete}

\begin{verbete}{烤}{kao3}{10}
  \significado{v.}{assar; grelhar}
\end{verbete}

\begin{verbete}{科技}{ke1ji4}{9;7}
  \significado*{s.}{Ciência e Tecnologia}
\end{verbete}

\begin{verbete}{颗}{ke1}{14}
  \significado{p.c.}{para grãos, pérolas, dentes, corações, satelites, pequenas esferas, etc.}
\end{verbete}

\begin{verbete}{咳嗽}{ke2sou5}{9;14}
  \significado{v.}{ter tosse; tossir}
\end{verbete}

\begin{verbete}{可}{ke3}{5}
  \significado{adv.}{muito; realmente}
\end{verbete}

\begin{verbete}{可爱}{ke3'ai4}{5;10}
  \significado{adj.}{adorável; querido; fofo}
\end{verbete}

\begin{verbete}{可编程}{ke3bian1cheng2}{5;12;12}
  \significado{adj.}{programável}
\end{verbete}

\begin{verbete*}{可擦写可编程只读存储器}{ke3ca1xie3ke3bian1cheng2zhi1du2cun2chu3qi4}{5;17;5;5;12;12;5;10;6;12;16}
  \significado{s.}{EPROM (\emph{erasable programmable read-only memory})}
\end{verbete*}

\begin{verbete}{可口可乐}{ke3kou3ke3le4}{5;3;5;5}
  \significado*{s.}{Coca-Cola}
\end{verbete}

\begin{verbete}{可能}{ke3neng2}{5;10}
  \significado{adj.}{possível; provável}
  \significado{adv.}{possivelmente; provavelmente}
  \significado[个]{s.}{possibilidade; probabilidade}
\end{verbete}

\begin{verbete}{可是}{ke3shi4}{5;9}
  \significado{adv.}{(usado para dar ênfase) de fato}
  \significado{conj.}{porém; contudo; mas}
\end{verbete}

\begin{verbete}{可惜}{ke3xi1}{5;11}
  \significado{adj.}{é uma pena; que pena}
  \significado{adv.}{infelizmente}
\end{verbete}

\begin{verbete}{可以}{ke3yi3}{5;4}
  \significado{v.o.}{ser capaz de; poder}
\end{verbete}

\begin{verbete}{刻}{ke4}{8}
  \significado{p.c.}{para curtos intervalos de tempo}
  \significado{p.t.}{quarto (de hora)}
  \significado{v.}{esculpir; cortar; gravar}
\end{verbete}

\begin{verbete}{刻钟}{ke4·zhong1}{8;9}
  \significado{p.t.}{um quarto de hora}
\end{verbete}

\begin{verbete}{客气}{ke4qi5}{9;4}
  \significado{adj.}{cortês; delicado; modesto; educado}
  \significado{v.}{fazer cerimônia}
\end{verbete}

\begin{verbete}{客厅}{ke4ting1}{9;4}
  \significado[间]{s.}{sala de estar; sala de visitas}
\end{verbete}

\begin{verbete}{课本}{ke4ben3}{10;5}
  \significado[本]{s.}{livro do aluno; manual}
\end{verbete}

\begin{verbete}{肯定}{ken3ding4}{8;8}
  \significado{adv.}{com certeza; certamente; definitivamente; afirmativo (resposta)}
  \significado{v.}{afirmar; ter a certeza; ser positivo; dar reconhecimento}
\end{verbete}

\begin{verbete}{空气}{kong1qi4}{8;4}
  \significado{s.}{ar; atmosfera}
\end{verbete}

\begin{verbete}{空调}{kong1tiao2}{8;10}
  \significado[台]{s.}{ar-condicionado; condicionador de ar}
\end{verbete}

\begin{verbete}{孔}{kong3}{4}
  \significado*{s.}{sobrenome Kong}
  \significado{p.c.}{para habitações em cavernas}
  \significado[个]{s.}{buraco}
\end{verbete}

\begin{verbete}{孔夫子}{kong3fu1zi3}{4;4;3}
  \significado*{s.}{Confúcio (551-479 aC), pensador e filósofo social chinês}
  \veja{孔子}{kong3zi3}
\end{verbete}

\begin{verbete}{孔雀}{kong3que4}{4;11}
  \significado{s.}{pavão}
\end{verbete}

\begin{verbete}{孔子}{kong3zi3}{4;3}
  \significado*{s.}{Confúcio (551-479 aC), pensador e filósofo social chinês}
  \veja{孔夫子}{kong3fu1zi3}
\end{verbete}

\begin{verbete}{孔子学院}{kong3zi3·xue2yuan4}{4;3;8;9}
  \significado*{s.}{Instituto Confúcio, organização estabelecida internacionalmente pela República Popular da China, que promove a língua e a cultura chinesas}
\end{verbete}

\begin{verbete}{恐怕}{kong3pa4}{10;8}
  \significado{adv.}{talvez; possivelmente; provavelmente; (em sentido não tão bom)}
  \significado{v.}{temer}
\end{verbete}

\begin{verbete}{空儿}{kong4r5}{8;2}
  \significado{s.}{tempo livre}
  \significado{v.}{ter tempo livre}
\end{verbete}

\begin{verbete}{口}{kou3}{3}[30]
  \significado{p.c.}{para coisas com bocas (pessoas, animais domésticos, canhões, etc.); para mordidas ou bocados}
  \significado{s.}{boca}
\end{verbete}

\begin{verbete}{口袋}{kou3dai4}{3;11}
  \significado{s.}{bolso; saco}
\end{verbete}

\begin{verbete}{口袋妖怪}{kou3dai4·yao1guai4}{3;11;7;8}
  \significado*{s.}{\emph{Pokémon}}
\end{verbete}

\begin{verbete}{口香糖}{kou3xiang1tang2}{3;9;16}
  \significado{s.}{goma de mascar; chiclete}
\end{verbete}

\begin{verbete}{口音}{kou3yin1}{3;9}
  \significado{s.}{linguística:~sons da fala oral}
  \veja{口音}{kou3yin5}
\end{verbete}

\begin{verbete}{口音}{kou3yin5}{3;9}
  \significado{s.}{sotaque; voz}
  \veja{口音}{kou3yin1}
\end{verbete}

\begin{verbete}{口语}{kou3yu3}{3;9}
  \significado[门]{s.}{linguagem oral; linguagem falada; fofoca; calúnia}
\end{verbete}

\begin{verbete}{苦瓜}{ku3gua1}{8;5}
  \significado{s.}{melão amargo (cabaça amarga, pêra bálsamo, maçã bálsamo, pepino amargo)}
\end{verbete}

\begin{verbete}{裤子}{ku4zi5}{12;3}
  \significado[条]{s.}{calças}
\end{verbete}

\begin{verbete}{会}{kuai4}{6}
  \significado{s.}{contador; contabilidade}
  \significado{v.}{equilibrar uma conta}
  \veja{会}{hui4}
\end{verbete}

\begin{verbete}{块}{kuai4}{7}
  \significado{p.c.}{coloquial:~para dinheiro e unidades monetárias; para peças ou pedaços de roupa, bolos, sabão, etc.}
  \significado{s.}{pedaço; pedaço (de terra); peça}
\end{verbete}

\begin{verbete}{快}{kuai4}{7}
  \significado{adj.}{quase; rápido; depressa}
  \significado{v.}{apressar-se}
\end{verbete}

\begin{verbete}{快乐}{kuai4le3}{7;5}
  \significado{adj.}{feliz; alegre}
  \significado{s.}{felicidade; alegria}
\end{verbete}

\begin{verbete}{款}{kuan3}{12}
  \significado{p.c.}{para versões ou modelos (de um produto)}
  \significado[笔,个]{s.}{montante de dinheiro; fundos; parágrafo; seção}
\end{verbete}

\begin{verbete}{窾}{kuan3}{17}
  \significado{adj.}{oco}
  \significado{s.}{rachadura; cavidade; onomatopéia:~água atingindo a rocha}
  \significado{v.}{escavar um buraco}
  \veja{窾}{cuan4}
\end{verbete}

\begin{verbete}{狂欢节}{kuang2huan1jie2}{7;6;5}
  \significado*{s.}{Carnaval}
\end{verbete}

\begin{verbete}{况且}{kuang4qie3}{7;5}
  \significado{conj.}{além disso; além do mais}
\end{verbete}

\begin{verbete}{矿泉水}{kuang4quan2shui3}{8;9;4}
  \significado[杯]{s.}{água mineral}
\end{verbete}

%%%%% EOF %%%%%

%%%
%%% L
%%%
%\section*{L}
\addcontentsline{toc}{section}{L}

\begin{verbete}{垃圾}{la1ji1}{8;6}
  \significado[把]{s.}{lixo}
\end{verbete}

\begin{verbete}{垃圾车}{la1ji1che1}{8;6;4}
  \significado{s.}{caminhão de lixo}
\end{verbete}

\begin{verbete}{垃圾电邮}{la1ji1dian4you2}{8;6;5;7}
  \significado{s.}{\emph{e-mail} de \emph{spam}}
\end{verbete}

\begin{verbete}{垃圾堆}{la1ji1dui1}{8;6;11}
  \significado{s.}{depósito de lixo}
\end{verbete}

\begin{verbete}{垃圾工}{la1ji1gong1}{8;6;3}
  \significado{s.}{lixeiro; gari}
\end{verbete}

\begin{verbete}{垃圾食品}{la1ji1shi2pin3}{8;6;9;9}
  \significado{s.}{\emph{junk food}}
\end{verbete}

\begin{verbete}{垃圾筒}{la1ji1tong3}{8;6;12}
  \significado{s.}{cesto de lixo}
\end{verbete}

\begin{verbete}{垃圾箱}{la1ji1xiang1}{8;6;15}
  \significado{s.}{cesto de lixo}
\end{verbete}

\begin{verbete}{垃圾邮件}{la1ji1you2jian4}{8;6;7;6}
  \significado{s.}{\emph{spam}; \emph{e-mail} não solicitado}
\end{verbete}

\begin{verbete}{拉拉队}{la1la1dui4}{8;8;4}
  \significado{s.}{claque; torcida}
\end{verbete}

\begin{verbete}{辣}{la4}{14}
  \significado{adj.}{picante; pungente}
\end{verbete}

\begin{verbete}{来}{lai2}{7}
  \significado{v.}{vir; chegar; trazer}
\end{verbete}

\begin{verbete}{蓝}{lan2}{13}
  \significado*{s.}{sobrenome Lan}
  \significado{adj.}{azul}
\end{verbete}

\begin{verbete}{蓝色}{lan2se4}{13;6}
  \significado{s.}{cor azul}
\end{verbete}

\begin{verbete}{篮球}{lan2qiu2}{16;11}
  \significado[个,只]{s.}{basquetebol}
\end{verbete}

\begin{verbete}{懒}{lan3}{16}
  \significado{adj.}{preguiçoso}
\end{verbete}

\begin{verbete}{懒虫}{lan3chong2}{16;6}
  \significado{s.}{desleixado ocioso; insulto:~sujeito preguiçoso}
\end{verbete}

\begin{verbete}{懒怠}{lan3dai4}{16;9}
  \significado{s.}{preguiça}
\end{verbete}

\begin{verbete}{懒得}{lan3de2}{16;11}
  \significado{adv.}{demasiado preguiçoso}
  \significado{v.}{não sentir vontade (de fazer algo)}
\end{verbete}

\begin{verbete}{懒惰}{lan3duo4}{16;12}
  \significado{adj.}{preguiçoso}
\end{verbete}

\begin{verbete}{懒鬼}{lan3gui3}{16;9}
  \significado{s.}{cara preguiçoso}
\end{verbete}

\begin{verbete}{懒汉}{lan3han4}{16;5}
  \significado{s.}{sujeito ocioso; vagabundo; preguiçosos}
\end{verbete}

\begin{verbete}{懒人}{lan3ren2}{16;2}
  \significado{s.}{pessoa preguiçosa}
\end{verbete}

\begin{verbete}{懒散}{lan3san3}{16;12}
  \significado{adj.}{inativo; indolente; preguiçoso; negligente}
\end{verbete}

\begin{verbete}{懒腰}{lan3yao1}{16;13}
  \significado[个]{s.}{alongamento (do corpo)}
\end{verbete}

\begin{verbete}{劳工同事}{lao2gong1 tong2shi4}{7;3;6;8}
  \significado{s.}{colaborador; colega de trabalho}
\end{verbete}

\begin{verbete}{老}{lao3}{6}[125]
  \significado{adj.}{velho (pessoa); venerável (pessoa); experiente; ultrapassado; duro (carne, etc.)}
  \significado{adv.}{de longa data; sempre; o tempo todo; do passado}
\end{verbete}

\begin{verbete}{老板}{lao3ban3}{6;8}
  \significado[个]{s.}{chefe; patrão; proprietário de empresa}
\end{verbete}

\begin{verbete}{老家}{lao3jia1}{6;10}
  \significado{s.}{estado ou região de origem; terra natal; lugar de origem}
\end{verbete}

\begin{verbete}{老人家}{lao3ren2jia5}{6;2;10}
  \significado{s.}{senhor ancião; madame, senhora; termo educado para mulher ou homem velho}
\end{verbete}

\begin{verbete}{老师}{lao3shi1}{6;6}
  \significado[个,位]{s.}{professor}
\end{verbete}

\begin{verbete}{了}{le5}{2}
  \significado{part.}{marcador de ação concluída; partícula modal indicando mudança de estado, situação; partícula modal intensificando a cláusula anterior}
  \veja{了}{liao3}
  \veja{了}{liao4}
\end{verbete}

\begin{verbete}{累}{lei2}{11}
  \significado*{s.}{sobrenome Lei}
  \significado{s.}{corda}
  \significado{v.}{amarrar; torcer}
  \veja{累}{lei3}
  \veja{累}{lei4}
\end{verbete}

\begin{verbete}{雷亚尔}{lei2ya4'er3}{13;6;5}
  \significado*{s.}{Real Brasileiro}
\end{verbete}

\begin{verbete}{累}{lei3}{11}
  \significado{adj.}{contínuo; repetido}
  \significado{v.}{acumular; envolver ou implicar}
  \veja{累}{lei2}
  \veja{累}{lei4}
\end{verbete}

\begin{verbete}{絫}{lei3}{12}
  \variante{累}{lei3}
\end{verbete}

\begin{verbete}{累}{lei4}{11}
  \significado{adj.}{cansado; fatigado}
  \significado{v.}{forçar; desgastar; trabalhar duro}
  \veja{累}{lei2}
  \veja{累}{lei3}
\end{verbete}

\begin{verbete}{冷}{leng3}{7}
  \significado*{s.}{sobrenome Leng}
  \significado{adj.}{frio}
\end{verbete}

\begin{verbete}{离}{li2}{10}
  \significado*{s.}{sobrenome Li}
  \significado{prep.}{(ser longe) de ... até...}
  \significado{v.}{ficar longe de; deixar; separar-se de}
\end{verbete}

\begin{verbete}{礼节}{li3jie2}{5;5}
  \significado{s.}{protocolo; cerimônia; etiqueta}
\end{verbete}

\begin{verbete}{礼物}{li3wu4}{5;8}
  \significado[件,个,份]{s.}{prenda; lembrança; presente}
\end{verbete}

\begin{verbete}{李四}{li3si4}{7;5}
  \significado*{s.}{Li Si; Zé Ninguém; nome para uma pessoa não especificada, 2 de 3}
  \veja{王五}{wan2wu3}
  \veja{张三}{zhang1san1}
\end{verbete}

\begin{verbete}{里}{li3}{7}[166]
  \significado*{s.}{sobrenome Li}
  \significado{p.l.}{em; dentro; interior}
  \significado{s.}{resina; vizinhança}
\end{verbete}

\begin{verbete}{里斯本}{li3si1ben3}{7;12;5}
  \significado*{s.}{Lisboa}
\end{verbete}

\begin{verbete}{里斯本大学}{li3si1ben3 da4xue2}{7;12;5;3;8}
  \significado*{s.}{Universidade de Lisboa}
\end{verbete}

\begin{verbete}{历史}{li4shi3}{4;5}
  \significado[门,段]{s.}{história}
\end{verbete}

\begin{verbete}{厉害}{li4hai5}{5;10}
  \significado{adj.}{severo; rigoroso; exigente; radical; violento; feroz}
\end{verbete}

\begin{verbete}{立刻}{li4ke4}{5;8}
  \significado{adv.}{imediatamente}
\end{verbete}

\begin{verbete}{例如}{li4ru2}{8;6}
  \significado{conj.}{por exemplo; como}
\end{verbete}

\begin{verbete}{詈骂}{li4ma4}{12;9}
  \significado{v.}{xingar; abusar}
\end{verbete}

\begin{verbete}{俩}{lia3}{9}
  \significado{adv.}{dois; ambos}
\end{verbete}

\begin{verbete}{俩钱}{lia3qian2}{9;10}
  \significado{s.}{uma pequena quantia de dinheiro}
\end{verbete}

\begin{verbete}{莲藕}{lian2'ou3}{10;18}
  \significado{s.}{raiz de Lotus}
\end{verbete}

\begin{verbete}{脸}{lian3}{11}
  \significado[张,个]{s.}{cara; rosto; face}
\end{verbete}

\begin{verbete}{练习}{lian4xi2}{8;3}
  \significado[个]{s.}{prática; exercício}
  \significado{v.}{praticar; exercitar}
\end{verbete}

\begin{verbete}{恋爱}{lian4'ai4}{10;10}
  \significado[个,场]{s.}{amor (romântico)}
  \significado{v.}{sentir-se profundamente apegado a}
\end{verbete}

\begin{verbete}{凉快}{liang2kuai5}{10;7}
  \significado{adj.}{agradável e frio; agradavelmente fresco}
\end{verbete}

\begin{verbete}{两}{liang3}{7}
  \significado{adv.}{ambos}
  \significado{num.}{dois (sempre usado antes de p.c.)}
\end{verbete}

\begin{verbete}{辆}{liang4}{11}
  \significado{p.c.}{para automóveis, veículos, etc.}
\end{verbete}

\begin{verbete}{了}{liao3}{2}
  \significado{v.}{terminar; alcançar; entender claramente}
  \veja{了}{le5}
  \veja{了}{liao4}
\end{verbete}

\begin{verbete}{了}{liao4}{2}
  \significado{adj.}{brilhantes (olhos)}
  \significado{v.}{observar; olhar para fora; olhar para baixo de um lugar mais alto; compreender claramente}
  \veja{了}{le5}
  \veja{了}{liao3}
\end{verbete}

\begin{verbete}{邻居}{lin2ju1}{7;8}
  \significado[个]{s.}{vizinho}
\end{verbete}

\begin{verbete}{菱角}{ling2jiao5}{11;7}
  \significado{s.}{castanha d'água}
\end{verbete}

\begin{verbete}{零/〇}{ling2}{13}
  \significado{adj.}{extra}
  \significado{num.}{zero, 0}
  \significado{s.}{matemática:~resto (após a divisão); fração; nada}
\end{verbete}

\begin{verbete}{领导}{ling3dao3}{11;6}
  \significado[位,个]{s.}{líder; liderança}
  \significado{v.}{liderar}
\end{verbete}

\begin{verbete}{另外}{ling4wai4}{5;5}
  \significado{adv./pron.}{além disso}
\end{verbete}

\begin{verbete}{流利}{liu2li4}{10;7}
  \significado{adj.}{fluente (em uma língua)}
\end{verbete}

\begin{verbete}{柳橙汁}{liu3cheng2zhi1}{9;16;5}
  \significado[瓶,杯,罐,盒]{s.}{suco de laranja}
  \veja{橙汁}{cheng2zhi1}
  \veja{橘子汁}{ju2zi5zhi1}
\end{verbete}

\begin{verbete}{六}{liu4}{4}
  \significado{num.}{seis, 6}
\end{verbete}

\begin{verbete}{遛狗}{liu4gou3}{13;8}
  \significado{v.+compl.}{passear com um cachorro}
\end{verbete}

\begin{verbete}{龙}{long2}{5}
  \significado*{s.}{sobrenome Long}
  \significado{adj.}{imperial}
  \significado[条]{s.}{dragão}
\end{verbete}

\begin{verbete}{龙山}{long2shan1}{5;3}
  \significado*{s.}{Longshan}
\end{verbete}

\begin{verbete}{楼}{lou2}{13}
  \significado*{s.}{sobrenome Lou}
  \significado{p.c.}{andar; piso}
  \significado[层,座,栋]{s.}{edifício; prédio; casa com mais de 1 andar}
\end{verbete}

\begin{verbete}{漏电}{lou4dian4}{14;5}
  \significado{v.}{vazar eletricidade}
\end{verbete}

\begin{verbete}{芦笋}{lu2sun3}{7;10}
  \significado{s.}{aspargos}
\end{verbete}

\begin{verbete}{录像带}{lu4xiang4dai4}{8;13;9}
  \significado[盘]{s.}{video-cassete}
\end{verbete}

\begin{verbete}{录像机}{lu4xiang4ji1}{8;13;6}
  \significado[台]{s.}{gravador de vídeo; VCR}
\end{verbete}

\begin{verbete}{录音}{lu4yin1}{8;9}
  \significado[个]{s.}{gravação de som}
  \significado{v.+compl.}{gravar (som)}
\end{verbete}

\begin{verbete}{录音机}{lu4yin1ji1}{8;9;6}
  \significado[台]{s.}{gravador de áudio}
\end{verbete}

\begin{verbete}{路}{lu4}{13}
  \significado*{s.}{sobrenome Lu}
  \significado[条]{s.}{caminho; estrada; via; jornada; linha (ônibus, etc.); rota}
\end{verbete}

\begin{verbete}{路口}{lu4kou3}{13;3}
  \significado{s.}{cruzamento; interseção (de estradas)}
\end{verbete}

\begin{verbete}{伦敦}{lun2dun1}{6;12}
  \significado*{s.}{Londres}
\end{verbete}

\begin{verbete}{罗}{luo2}{8}
  \significado*{s.}{sobrenome Luo}
  \significado{v.}{coletar; juntar; pegar; peneirar}
\end{verbete}

\begin{verbete}{旅行}{lv3xing2}{10;6}
  \significado{v.}{viajar}
\end{verbete}

\begin{verbete}{旅游}{lv3you2}{10;12}
  \significado[趟,次,个]{s.}{jornada; viagem}
  \significado{v.}{viajar}
\end{verbete}

\begin{verbete}{屡次}{lv3ci4}{12;6}
  \significado{adv.}{repetidamente; uma e outra vez; muitas vezes}
\end{verbete}

\begin{verbete}{绿}{lv4}{11}
  \significado{adj.}{verde}
\end{verbete}

\begin{verbete}{绿豆}{lv4dou4}{11;7}
  \significado{s.}{vagens}
\end{verbete}

\begin{verbete}{绿豆芽}{lv4dou4 ya2}{11;7;7}
  \significado{s.}{broto de feijão verde}
\end{verbete}

\begin{verbete}{绿色}{lv4se4}{11;6}
  \significado{s.}{cor verde}
\end{verbete}

\begin{verbete}{略}{lve4}{11}
  \significado{adv.}{ligeiramente; marginalmente; aproximadamente}
\end{verbete}

\begin{verbete}{略微}{lve4wei1}{11;13}
  \significado{adv.}{ligeiramente; marginalmente; aproximadamente}
\end{verbete}

%%%%% EOF %%%%%

%%%
%%% M
%%%
%\section*{M}
\addcontentsline{toc}{section}{M}

\begin{verbete}{妈妈}{ma1ma5}{6;6}
  \significado[个,位]{s.}{mamãe, mãe}
\end{verbete}

\begin{verbete}{麻烦}{ma2fan5}{11;10}
  \significado{adj.}{fastidioso; maçante; inconveniente; problemático}
  \significado{s.}{incômodo}
  \significado{v.}{incomodar alguém; colocar alguém em apuros}
\end{verbete}

\begin{verbete}{麻辣豆腐}{ma2la4 dou4fu5}{11;14;7;14}
  \significado{s.}{tofú guisado em molho picante (prato)}
\end{verbete}

\begin{verbete}{马路}{ma3lu4}{3;13}
  \significado[条]{s.}{rua; estrada}
\end{verbete}

\begin{verbete}{马马虎虎}{ma3ma3hu3hu3}{3;3;8;8}
  \significado{adj.}{descuidado; casual; tolerável; vago; mais ou menos}
\end{verbete}

\begin{verbete}{马上}{ma3shang4}{3;3}
  \significado{adv.}{já; imediatamente; de imediato; sem demora}
\end{verbete}

\begin{verbete}{㐷}{ma4}{5}
  \variante{骂}{ma4}
\end{verbete}

\begin{verbete}{骂}{ma4}{9}
  \significado{v.}{insultar; maldizer; ralhar}
\end{verbete}

\begin{verbete}{骂街}{ma4jie1}{9;12}
  \significado{v.}{gritar abusos na rua}
\end{verbete}

\begin{verbete}{骂名}{ma4ming2}{9;6}
  \significado{s.}{infâmia}
\end{verbete}

\begin{verbete}{吗}{ma5}{6}
  \significado{part.}{partícula interrogativa (usado em perguntas ``sim-não'')}
\end{verbete}

\begin{verbete}{买}{mai3}{6}
  \significado{v.}{comprar}
\end{verbete}

\begin{verbete}{买东西}{mai3dong1xi5}{6;5;6}
  \significado{v.}{fazer compras}
\end{verbete}

\begin{verbete}{麦当劳}{mai4dang1lao2}{7;6;7}
  \significado*{s.}{McDonald's (empresa de \emph{fast-food})}
  \veja{麦当劳叔叔}{mai4dang1lao2 shu1shu5}
\end{verbete}

\begin{verbete}{麦当劳叔叔}{mai4dang1lao2 shu1shu5}{7;6;7;8;8}
  \significado*{s.}{Ronald McDonald}
  \veja{麦当劳}{mai4dang1lao2}
\end{verbete}

\begin{verbete}{卖}{mai4}{8}
  \significado{v.}{vender}
\end{verbete}

\begin{verbete}{满意}{man3yi4}{13;13}
  \significado{adj.}{satisfatório}
\end{verbete}

\begin{verbete}{谩骂}{man4ma4}{13;9}
  \significado{v.}{ridicularizar; abusar}
\end{verbete}

\begin{verbete}{慢}{man4}{14}
  \significado{adj.}{devagar}
\end{verbete}

\begin{verbete}{漫骂}{man4ma4}{14;9}
  \variante{谩骂}{man4ma4}
\end{verbete}

\begin{verbete}{忙}{mang1}{6}
  \significado{adj.}{ocupado}
  \significado{s.}{apressar}
\end{verbete}

\begin{verbete}{猫}{mao1}{11}
  \significado[只]{s.}{gato; coloquial:~MODEM}
  \significado{v.}{dialeto: esconder-se}
\end{verbete}

\begin{verbete}{猫熊}{mao1xiong2}{11;14}
  \veja{熊猫}{xiong2mao1}
\end{verbete}

\begin{verbete}{毛}{mao2}{4}[82]
  \significado*{s.}{sobrenome Mao}
  \significado{p.c.}{1 mao = 10 centavos}
\end{verbete}

\begin{verbete}{贸易}{mao4yi4}{9;8}
  \significado[个]{s.}{transação comercial}
  \significado{v.}{fazer uma transação comercial}
\end{verbete}

\begin{verbete}{没}{mei2}{7}
  \significado{adv.}{não ter; não há; ficar sem; não (prefixo negativo para verbos, traduzido para outras línguas com verbos no pretérito)}
  \veja{没}{mo4}
\end{verbete}

\begin{verbete}{没关系}{mei2guan1xi5}{7;6;7}
  \significado{v.}{não ter problema; não ter importância; não fazer mal}
\end{verbete}

\begin{verbete}{没用}{mei2yong4}{7;5}
  \significado{adj.}{inútil}
\end{verbete}

\begin{verbete}{没有}{mei2you3}{7;6}
  \significado{v.}{não há; não tem; não existe}
\end{verbete}

\begin{verbete}{没有关系}{mei2you3guan1xi5}{7;6;6;7}
  \veja{没关系}{mei2guan1xi5}
\end{verbete}

\begin{verbete}{没有意思}{mei2you3yi4si5}{7;6;13;9}
  \significado{adj.}{tedioso; chato; sem interessante}
\end{verbete}

\begin{verbete}{眉毛}{mei2mao5}{9;4}
  \significado[根]{s.}{sobrancelha}
\end{verbete}

\begin{verbete}{每}{mei3}{7}
  \significado{pron.}{cada}
\end{verbete}

\begin{verbete}{每次}{mei3ci4}{7;6}
  \significado{adv.}{toda vez; cada vez}
\end{verbete}

\begin{verbete}{每天}{mei3tian1}{7;4}
  \significado{adv.}{todo dia; cada dia}
\end{verbete}

\begin{verbete}{美国}{mei3guo1}{9;8}
  \significado*{s.}{Estados Unidos da América}
\end{verbete}

\begin{verbete}{美国人}{mei3guo1ren2}{9;8;2}
  \significado{s.}{americano; nascido nos Estados Unidos da América}
\end{verbete}

\begin{verbete}{美丽}{mei3li4}{9;7}
  \significado{adj.}{bonito; lindo}
\end{verbete}

\begin{verbete}{美元}{mei3yuan2}{9;4}
  \significado*{s.}{Dólar Americano}
\end{verbete}

\begin{verbete}{美洲}{mei3zhou1}{9;9}
  \significado*{s.}{América (incluindo Norte, Central e Sul)}
\end{verbete}

\begin{verbete}{美洲人}{mei3zhou1ren2}{9;9;2}
  \significado{s.}{americano; nascido no continente Americano}
\end{verbete}

\begin{verbete}{妹夫}{mei4fu5}{8;4}
  \significado{s.}{marido da irmã mais nova}
\end{verbete}

\begin{verbete}{妹妹}{mei4mei5}{8;8}
  \significado[个]{s.}{irmã mais nova; mulher jovem}
\end{verbete}

\begin{verbete}{门口}{men2kou3}{3;3}
  \significado[个]{p.d.l.}{porta; portão}
\end{verbete}

\begin{verbete}{们}{men5}{5}
  \significado{part.}{sufixo para plural de pronomes e substantivos referentes a indivíduos}
\end{verbete}

\begin{verbete}{猛然}{meng3ran2}{11;12}
  \significado{adv.}{de repente; abruptamente}
\end{verbete}

\begin{verbete}{米饭}{mi3fan4}{6;7}
  \significado{s.}{arroz cozido}
\end{verbete}

\begin{verbete}{免得}{mian3de5}{7;11}
  \significado{conj.}{de modo a não; para evitar; para que não}
\end{verbete}

\begin{verbete}{靣}{mian4}{8}
  \variante{面}{mian4}
\end{verbete}

\begin{verbete}{面}{mian4}{9}[176]
  \significado{p.c.}{para objetos com superfície plana como tambores, espelhos, bandeiras, etc.}
  \significado{s.}{farinha; massa; gíria:~(uma pessoa) ineficaz}
\end{verbete}

\begin{verbete}{面包}{mian4bao1}{9;5}
  \significado[个,片,袋,块]{s.}{pão}
  \exemplo{我买八个面包了。}[Comprei oito pães.]
  \exemplo{他在吃两片面包。}[Ele está comendo duas fatias de pão.]
  \exemplo{我在家里带了一袋面包。}[Trouxe um saco de pão para casa.]
  \exemplo{我拿了一块面包。}[Peguei um pedaço de pão.]
\end{verbete}

\begin{verbete}{面积}{mian4ji1}{9;10}
  \significado{s.}{área (de um andar, pedaço de terreno, etc.); área de superfície; pedaço de terra}
\end{verbete}

\begin{verbete}{面条}{mian4tiao2}{9;7}
  \significado{s.}{macarrão; espaguete}
\end{verbete}

\begin{verbete}{糆}{mian4}{15}
  \variante{面}{mian4}
\end{verbete}

\begin{verbete}{麫}{mian4}{15}
  \variante{面}{mian4}
\end{verbete}

\begin{verbete}{名片}{ming2pian4}{6;4}
  \significado{s.}{cartão de visita}
\end{verbete}

\begin{verbete}{名字}{ming2zi5}{6;6}
  \significado[个]{s.}{nome (de uma pessoa ou coisa)}
\end{verbete}

\begin{verbete}{明白}{ming2bai5}{8;5}
  \significado{adj.}{compreendido; percebido; óbvio; inequívoco}
  \significado{v.}{compreender; perceber}
\end{verbete}

\begin{verbete}{明明}{ming2ming2}{8;8}
  \significado{interr.}{obviamente, claramente}
\end{verbete}

\begin{verbete}{明年}{ming2nian2}{8;6}
  \significado{p.t.}{próximo ano}
\end{verbete}

\begin{verbete}{明天}{ming2tian1}{8;4}
  \significado{p.t.}{amanhã}
\end{verbete}

\begin{verbete}{磨菇}{mo2gu5}{16;11}
  \variante{蘑菇}{mo2gu5}
\end{verbete}

\begin{verbete}{蘑菇}{mo2gu5}{19;11}
  \significado{s.}{cogumelo}
  \significado{v.}{mandriar; embromar; amofinar; incomodar alguém com solicitações ou interrupções frequentes ou persistentes}
\end{verbete}

\begin{verbete}{没}{mo4}{7}
  \significado{adj.}{afogado}
  \significado{v.}{acabar; morrer; inundar}
  \variante{没}{mei2}
\end{verbete}

\begin{verbete}{墨镜}{mo4jing4}{15;16}
  \significado[只,双,副]{s.}{óculos escuros}
\end{verbete}

\begin{verbete}{母亲}{mu3qin1}{5;9}
  \significado[个]{s.}{mãe}
  \veja{母亲}{mu3qin5}
\end{verbete}

\begin{verbete}{母亲}{mu3qin5}{5;9}
  \significado[个]{s.}{mãe}
  \veja{母亲}{mu3qin1}
\end{verbete}

\begin{verbete}{母语}{mu3yu3}{5;9}
  \significado{s.}{língua materna; língua nativa}
\end{verbete}

%%%%% EOF %%%%%

%%%
%%% N
%%%
\section*{N}
\addcontentsline{toc}{section}{N}

\begin{verbete*}[6]{那}{na1}
  \significado{s.}{sobrenome Na}
\end{verbete*}
\begin{verbete}[10]{拿}{na2}
  \significado{part.}{usado da mesma forma que 把: para marcar o seguinte substantivo seguinte como objeto direto}
  \significado{v.}{segurar; tomar; pegar em}
\end{verbete}
\begin{verbete}[6]{那}{na3}
  \variante{哪}{na3}
\end{verbete}
\begin{verbete}[9]{哪}{na3}
  \significado{interr.}{que?; qual?}
\end{verbete}
\begin{verbete}[9;8;2]{哪国人}{na3·guo2ren2}
  \significado{interr.}{de qual país?}
\end{verbete}
\begin{verbete}[9;7]{哪里}{na3li3}
  \significado{interr.}{onde?}
\end{verbete}
\begin{verbete}[9;2]{哪儿}{na3r5}
  \significado{interr.}{onde?}
\end{verbete}
\begin{verbete}[9;8]{哪些}{na3xie1}
  \significado{interr.}{quais?}
\end{verbete}
\begin{verbete}[6]{那}{na4}
  \significado{conj.}{nessa situação; nesse caso}
  \significado{pron.}{aquele; aquilo}
\end{verbete}
\begin{verbete}[6;7]{那里}{na4li5}
  \significado{pron.}{lá; ali}
\end{verbete}
\begin{verbete}[6;3]{那么}{na4me5}
  \significado{adv.}{então; como aquele; dessa maneira}
\end{verbete}
\begin{verbete}[6;5]{那末}{na4me5}
  \variante{那么}{na4me5}
\end{verbete}
\begin{verbete}[6;14]{那麽}{na4me5}
  \variante{那么}{na4me5}
\end{verbete}
\begin{verbete}[6;2]{那儿}{na4r5}
  \significado{pron.}{lá; ali}
\end{verbete}
\begin{verbete}[6;8]{那些}{na4xie1}
  \significado{pron.}{aqueles}
\end{verbete}
\begin{verbete}[5;5]{奶奶}{nai3nai5}
  \significado[位]{s.}{avó(paterna); respeitoso: dona da casa}
\end{verbete}
\begin{verbete}[7]{男}{nan2}
  \significado{adj.}{masculino}
  \significado{s.}{Barão, o mais baixo de cinco ordens de nobreza}
\end{verbete}
\begin{verbete}[7;9;2]{男孩儿}{nan2hai2r5}
  \significado{s.}{menino; rapaz}
\end{verbete}
\begin{verbete}[7;8;4]{男朋友}{nan2peng2you5}
  \significado{s.}{namorado}
\end{verbete}
\begin{verbete}[9;5]{南边}{nan2bian5}
  \significado{p.l.}{sul; lado sul; parte sul; ao sul de}
\end{verbete}
\begin{verbete}[9;4]{南方}{nan2fang1}
  \significado{p.l.}{sul; o Sul; a parte sul do país}
\end{verbete}
\begin{verbete}[9;9]{南面}{nan2mian4}
  \significado{p.l.}{sul; lado sul}
\end{verbete}
\begin{verbete}[10]{难}{nan2}
  \significado{adj.}{difícil}
  \significado{s.}{dificuldade}
  \veja{难}{nan4}
\end{verbete}
\begin{verbete}[10]{难}{nan4}
  \significado{s.}{desastre}
  \significado{v.}{repreender}
  \veja{难}{nan2}
\end{verbete}
\begin{verbete}[10]{孬}{nao1}
  \significado{adj.}{dialeto: não (é) bom (contração de 不+好)}
\end{verbete}
\begin{verbete}[8]{呢}{ne5}
  \significado{interr.}{(no final de uma frase declarativa) partícula que indica a continuação de um estado ou ação; partícula para perguntar sobre a localização (``Onde está...?''); partícula indicando afirmação forte; partícula indicando que uma pergunta feita anteriormente deve ser aplicada à palavra anterior (``E quanto a ...?'', ``E ...?''); partícula sinalizando uma pausa, para enfatizar as palavras anteriores e permitir que o ouvinte tenha tempo para compreendê-las (``ok?'', ``você está comigo?'')}
\end{verbete}
\begin{verbete}[4;6]{内存}{nei4cun2}
  \significado{s.}{armazenamento interno; memória do computador; RAM (\textit{random access memory})}
  \veja*{随机存取存储器}{sui2ji1cun2qu3cun2chu3qi4}
  \veja*{随机存取记忆体}{sui2ji1cun2qu3ji4yi4ti3}
\end{verbete}
\begin{verbete}[4;9]{内省}{nei4xing3}
  \significado{s.}{introspecção}
  \significado{v.}{refletir sobre si mesmo}
\end{verbete}
\begin{verbete}[10]{能}{neng2}
  \significado{adv.}{talvez}
  \significado{v.}{poder; ser capaz de}
  \significado{s.}{física: energia; habilidade}
\end{verbete}
\begin{verbete*}[10]{能}{neng2}
  \significado{s.}{sobrenome: Neng}
\end{verbete*}
\begin{verbete}[10;3;10;3]{能上能下}{neng2shang4neng2xia4}
  \significado{s.}{pronto para aceitar qualquer trabalho, alto ou baixo}
\end{verbete}
\begin{verbete}[7]{伲}{ni3}
  \variante{你}{ni3}
\end{verbete}
\begin{verbete}[7]{你}{ni3}
  \significado{pron.}{você (informal); tu; te; ti; contigo}
  \veja{您}{nin2}
\end{verbete}
\begin{verbete}[7;8]{你的}{ni3·de5}
  \significado{pron.}{seu}
\end{verbete}
\begin{verbete}[7;5]{你们}{ni3men5}
  \significado{pron.}{vocês (informal); vós; vos, convosco}
\end{verbete}
\begin{verbete}[7;5;8]{你们的}{ni3men5·de5}
  \significado{pron.}{vossos}
\end{verbete}
\begin{verbete}[10]{袮}{ni3}
  \significado{pron.}{Você; Tu (divindade)}
  \variante{你}{ni3}
\end{verbete}
\begin{verbete}[6]{年}{nian2}
  \significado{p.c.}{ano}
  \significado{p.t.}{ano}
  \significado[个]{s.}{ano}
\end{verbete}
\begin{verbete*}[6]{年}{nian2}
  \significado{s.}{sobrenome Nian}
\end{verbete*}
\begin{verbete}[6;8]{年货}{nian2huo4}
  \significado{s.}{mercadorias vendidas no Ano Novo Chinês}
\end{verbete}
\begin{verbete}[6;6]{年级}{nian2ji2}
  \significado[个]{s.}{classe; ano (escola)}
\end{verbete}
\begin{verbete}[6;6]{年纪}{nian2ji4}
  \significado[个]{s.}{grau; nota; classe; categoria; graduação; ano (na escola, faculdade, etc.)}
\end{verbete}
\begin{verbete}[6;9]{年轻}{nian2qing1}
  \significado{adj.}{jovem}
\end{verbete}
\begin{verbete}[5;2]{鸟儿}{niao3r5}
  \significado[只]{s.}{pássaro; ave}
\end{verbete}
\begin{verbete}[11]{您}{nin2}
  \significado{pron.}{você (formal); tu; te; ti, contigo}
  \veja{你}{ni3}
\end{verbete}
\begin{verbete}[4]{牛}{niu2}
  \significado[条,头]{s.}{boi; touro; vaca; gíria: incrível}
\end{verbete}
\begin{verbete*}[4]{牛}{niu2}
  \significado{s.}{sobrenome Niu}
\end{verbete*}
\begin{verbete*}[4;10]{牛顿}{niu2dun4}
  \significado{s.}{Newton (nome); newton (unidade de força do SI)}
\end{verbete*}
\begin{verbete}[4;5]{牛奶}{niu2nai3}
  \significado[瓶,杯]{s.}{leite de vaca}
\end{verbete}
\begin{verbete}[4;6]{牛肉}{niu2rou4}
  \significado{s.}{carne de vaca; bife}
\end{verbete}
\begin{verbete}[4;5;12]{牛仔裤}{niu2zai3ku4}
  \significado[条]{s.}{calça de ganga, jeans}
\end{verbete}
\begin{verbete}[6;7]{农村}{nong2cun1}
  \significado[个]{s.}{campo rural; aldeia; povoação rústica}
\end{verbete}
\begin{verbete}[7;2]{努力}{nu3li4}
  \significado{adj.}{diligente; aplicado}
  \significado{v.}{esforçar-se; se esforçar}
\end{verbete}
\begin{verbete}[9;9]{怒骂}{nu4ma4}
  \significado{v.}{praguejar de raiva}
\end{verbete}
\begin{verbete}[13;8]{暖和}{nuan3huo5}
  \significado{adj.}{morno; agradável e quente}
\end{verbete}
\begin{verbete}[13;4]{暖气}{nuan3qi4}
  \significado{s.}{aquecimento central; aquecedor; ar quente}
\end{verbete}
\begin{verbete*}[6]{那}{nuo2}
  \significado{s.}{sobrenome Nuo}
\end{verbete*}
\begin{verbete}[3]{女}{nv3}
  \significado{adj.}{feminino}
\end{verbete}
\begin{verbete}[3;2]{女儿}{nv3'er2}
  \significado{s.}{filha}
\end{verbete}
\begin{verbete}[3;9]{女孩}{nv3hai2}
  \significado{s.}{menina; garota}
\end{verbete}
\begin{verbete}[3;8;4]{女朋友}{nv3peng2you5}
  \significado{s.}{namorada}
\end{verbete}
\begin{verbete}[3;4]{女王}{nv3wang2}
  \significado{s.}{rainha}
\end{verbete}
\begin{verbete}[3;12]{女婿}{nv3xu5}
  \significado{s.}{marido da filha}
\end{verbete}

%%%%% EOF %%%%%

%%%
%%% O
%%%
%\section*{O}
\addcontentsline{toc}{section}{O}

\begin{verbete}{喔}{o1}{12}[Radical 口][Componentes 口屋]
  \significado{interj.}{Oh!, Entendi! (usado para indicar realização, compreensão)}
\end{verbete}

\begin{verbete}{哦}{o2}{10}[Radical ⼝][Componentes ⼝我]
  \significado{interj.}{Oh! (indicando dúvida ou surpresa)}
  \veja{哦}{e2}
  \veja{哦}{o4}
  \veja{哦}{o5}
\end{verbete}

\begin{verbete}{哦}{o4}{10}[Radical ⼝][Componentes ⼝我]
  \significado{interj.}{Oh! (indicando que acabou de aprender algo)}
  \veja{哦}{e2}
  \veja{哦}{o2}
  \veja{哦}{o5}
\end{verbete}

\begin{verbete}{哦}{o5}{10}[Radical ⼝][Componentes ⼝我]
  \significado{part.}{final da frase que transmite informalidade, calor, simpatia ou intimidade; também pode indicar que alguém está declarando um fato de que a outra pessoa não está ciente}
  \veja{哦}{e2}
  \veja{哦}{o2}
  \veja{哦}{o4}
\end{verbete}

\begin{verbete}{区}{ou1}{4}[Radical 匸][Componentes 匚乂]
  \significado*{s.}{sobrenome Ou}
  \veja{区}{qu1}
\end{verbete}

\begin{verbete}{欧}{ou1}{8}[Radical 欠][Componentes 区欠]
  \significado*{s.}{Europa, abreviação de~欧洲; sobrenome Ou}
  \veja{欧洲}{ou1zhou1}
\end{verbete}

\begin{verbete}{欧盟}{ou1meng2}{8;13}
  \significado*{s.}{Uniáo Europeia}
\end{verbete}

\begin{verbete}{欧洲}{ou1zhou1}{8;9}
  \significado*{s.}{Europa}
  \veja{欧}{ou1}
\end{verbete}

\begin{verbete}{欧洲共同体}{ou1zhou1 gong4tong2ti3}{8;9;6;6;7}
  \significado*{s.}{Comunidade Europeia}
\end{verbete}

\begin{verbete}{欧洲人}{ou1zhou1ren2}{8;9;2}
  \significado{s.}{europeu; nascido na Europa}
\end{verbete}

\begin{verbete}{偶然}{ou3ran2}{11;12}
  \significado{adv.}{por acaso; fortuitamente}
\end{verbete}

%%%%% EOF %%%%%

%%%
%%% P
%%%
\section*{P}
\addcontentsline{toc}{section}{P}

\begin{verbete}{扒犁}{pa2li2}{5;11}
  \significado{s.}{trenó}
  \veja{爬犁}{pa2li2}
\end{verbete}
\begin{verbete}{爬}{pa2}{8}
  \significado{v.}{escalar; subir; trepar; rastejar}
\end{verbete}
\begin{verbete}{爬山}{pa2chan1}{8;3}
  \significado{s.}{alpinista; montanhismo}
  \significado{v.}{escalar uma montanha}
\end{verbete}
\begin{verbete}{爬杆}{pa2gan1}{8;7}
  \significado{s.}{escalada em poste}
  \significado{v.}{escalar um poste}
\end{verbete}
\begin{verbete}{爬竿}{pa2gan1}{8;9}
  \significado{s.}{poste de escalada; escalada em poste (como ginástica ou ato de circo)}
\end{verbete}
\begin{verbete}{爬犁}{pa2li2}{8;11}
  \significado{s.}{trenó}
  \veja{扒犁}{pa2li2}
\end{verbete}
\begin{verbete}{爬墙}{pa2qiang2}{8;14}
  \significado{v.}{escalar uma parede}
\end{verbete}
\begin{verbete}{爬上}{pa2shang4}{8;3}
  \significado{v.}{escalar}
\end{verbete}
\begin{verbete}{爬升}{pa2sheng1}{8;4}
  \significado{v.}{ascender; ganhar promoção; subir (números de vendas, etc.); aumentar}
\end{verbete}
\begin{verbete}{爬梳}{pa2shu1}{8;11}
  \significado{v.}{vasculhar (documentos históricos, etc.); desvendar}
\end{verbete}
\begin{verbete}{爬行}{pa2xing2}{8;6}
  \significado{v.}{rastejar; arrastar; engatinhar}
\end{verbete}
\begin{verbete}{怕}{pa4}{8}
  \significado{v.}{ter medo; ser incapaz de suportar; temer}
  \significado*{s.}{sobrenome Pa}
\end{verbete}
\begin{verbete}{拍马}{pai1ma3}{8;3}
  \significado{v.}{instigar um cavalo dando tapinhas em seu traseiro; lisonjear; bajular}
  \veja{拍马屁}{pai1ma3pi4}
\end{verbete}
\begin{verbete}{拍马屁}{pai1ma3pi4}{8;3;7}
  \significado{s.}{puxa-saco; bajulador}
  \significado{v.}{puxar o saco; bajular}
  \veja{拍马}{pai1ma3}
\end{verbete}
\begin{verbete}{拍照}{pai1zhao4}{8;13}
  \significado{v.+compl.}{tirar fotografia}
\end{verbete}
\begin{verbete}{排球}{pai2qiu2}{11;11}
  \significado[个]{s.}{voleibol}
\end{verbete}
\begin{verbete}{盘}{pan2}{11}
  \significado{p.c.}{para bobinas de fio; de comida: pratos, serviços; para jogos de xadrez}
  \significado{s.}{tabuleiro; prato; bandeja; computação: disco rígido}
  \significado{v.}{construir; checar; enrolar; examinar; transferir (propriedade)}
\end{verbete}
\begin{verbete}{槃}{pan2}{14}
  \variante{盘}{pan2}
\end{verbete}
\begin{verbete}{胖}{pang2}{9}
  \significado{adj.}{saudável}
  \veja{胖}{pang4}
\end{verbete}
\begin{verbete}{旁边}{pang2bian1}{10;5}
  \significado{p.l.}{junto a; próximo de; ao lado}
\end{verbete}
\begin{verbete}{胖}{pang4}{9}
  \significado{adj.}{gordo}
  \veja{胖}{pang2}
\end{verbete}
\begin{verbete}{跑}{pao2}{12}
  \significado{v.}{(de um animal) dar patadas (no chão)}
  \veja{跑}{pao3}
\end{verbete}
\begin{verbete}{跑}{pao3}{12}
  \significado{v.}{vazar ou evaporar (sobre um gás ou líquido); escapar; correr; correr (em tarefas, etc.); fugir}
  \veja{跑}{pao2}
\end{verbete}
\begin{verbete}{跑步}{pao3bu4}{12;7}
  \significado{s.}{corrida}
  \significado{v.}{correr; militar: marchar em dupla}
\end{verbete}
\begin{verbete}{跑调}{pao3diao4}{12;10}
  \significado{v.}{coloquial: estar fora do tom ou desafinado (enquanto canta)}
\end{verbete}
\begin{verbete}{跑掉}{pao3diao4}{12;11}
  \significado{v.}{fugir}
\end{verbete}
\begin{verbete}{跑肚}{pao3du4}{12;7}
  \significado{v.}{coloquial: ter diarréia}
\end{verbete}
\begin{verbete}{跑马}{pao3ma3}{12;3}
  \significado{s.}{corrida de cavalos}
  \significado{v.}{andar a cavalo em ritmo acelerado}
\end{verbete}
\begin{verbete}{跑题}{pao3ti2}{12;15}
  \significado{v.}{divagar; fugir do assunto}
\end{verbete}
\begin{verbete}{跑腿}{pao3tui3}{12;13}
  \significado{v.}{realizar tarefas}
\end{verbete}
\begin{verbete}{陪}{pei2}{10}
  \significado{v.}{acompanhar; ajudar; fazer companhia a alguém}
\end{verbete}
\begin{verbete}{配}{pei4}{10}
  \significado{v.}{alocar; merecer; caber; juntar-se; compensar (uma receita); combinar; acasalar; misturar}
\end{verbete}
\begin{verbete}{盆友}{pen2you3}{9;4}
  \significado{s.}{Internet gíria: amigo (trocadilho com 朋友)}
  \veja{朋友}{peng2you5}
\end{verbete}
\begin{verbete}{朋友}{peng2you5}{8;4}
  \significado[个,位]{s.}{amigo}
\end{verbete}
\begin{verbete}{碰运气}{peng4yun4qi5}{13;7;4}
  \significado{v.}{deixar algo ao acaso; tentar a sorte}
\end{verbete}
\begin{verbete}{啤酒馆}{pi2jiu3guan3}{8;10;11}
  \significado{s.}{cervejaria}
\end{verbete}
\begin{verbete}{啤酒}{pi2jiu3}{11;10}
  \significado[杯,瓶,罐,桶,缸]{s.}{cerveja}
\end{verbete}
\begin{verbete}{脾气}{pi2qi5}{12;4}
  \significado{s.}{temperamento; humor; disposição; caráter}
\end{verbete}
\begin{verbete}{屁股}{pi4gu5}{7;8}
  \significado{s.}{nádega; quadris}
\end{verbete}
\begin{verbete}{屁话}{pi4hua4}{7;8}
  \significado{s.}{absurdo; tolice; besteira}
\end{verbete}
\begin{verbete}{便宜}{pian2yi5}{9;8}
  \significado{adj.}{barato}
  \significado{v.}{deixar alguém levemente de lado}
\end{verbete}
\begin{verbete}{片}{pian4}{4}
  \significado{adj.}{parcial; incompleto; que só tem um lado}
  \significado{p.c.}{classificador para CDs, filmes, DVDs, etc.; classificador para fatias, comprimidos, extensão de terra, área de água; usado com numeral 一: para cenário, cena, sentimento, atmosfera, som etc.}
  \significado{s.}{uma fatia; floco; filme; pedaço fino}
  \significado{v.}{fatiar; esculpir fino}
\end{verbete}
\begin{verbete}{票}{piao4}{11}
  \significado{p.c.}{para grupos, lotes, transações comerciais}
  \significado[张]{s.}{performance amadora de ópera chinesa; cédula eleitoral; nota; bilhete; pessoa detida por resgate}
\end{verbete}
\begin{verbete}{漂亮}{piao4liang5}{14;9}
  \significado{adj.}{bonita, linda; bonito, lindo (para objetos inanimados)}
\end{verbete}
\begin{verbete}{乒乓球}{ping1pang1qiu2}{6;6;11}
  \significado[个]{s.}{tênis de mesa; ping-pong}
\end{verbete}
\begin{verbete}{平时}{ping2shi2}{5;7}
  \significado{adv.}{normalmente}
  \significado{p.t.}{em tempos normais; em tempos de paz}
\end{verbete}
\begin{verbete}{苹果}{ping2guo3}{8;8}
  \significado[个,颗]{s.}{maçã}
\end{verbete}
\begin{verbete}{瓶}{ping2}{10}
  \significado[个]{s.}{garrafa; jarro; vaso}
  \significado{p.c.}{para vinho ou líquidos}
\end{verbete}
\begin{verbete}{甁}{ping2}{12}
  \variante{瓶}{ping2}
\end{verbete}
\begin{verbete}{破}{po4}{10}
  \significado{adj.}{partido; quebrado; roto; nojento; esgotado}
  \significado{v.}{romper com; quebrar, dividir ou clivar; capturar (uma cidade, etc.); derrotar; destruir; expor a verdade de; se livrar;  quebrado; roto}
\end{verbete}
\begin{verbete}{葡}{pu2}{12}
  \significado*{s.}{Portugal, abreviação de 葡萄牙}
  \veja{葡萄牙}{pu2tao2ya2}
\end{verbete}
\begin{verbete}{葡汉词典}{pu2-han4·ci2dian3}{12;5;7;8}
  \significado*{s.}{Dicionário Português-Chinês}
  \veja*{汉葡词典}{han4-pu2·ci2dian3}
\end{verbete}
\begin{verbete}{葡萄牙}{pu2tao2ya2}{12;11;4}
  \significado*{s.}{Portugal}
  \veja{葡}{pu2}
\end{verbete}
\begin{verbete}{葡萄牙文}{pu2tao2ya2wen2}{12;11;4;4}
  \significado{s.}{português, língua portuguesa}
  \veja{葡文}{pu2wen2}
\end{verbete}
\begin{verbete}{葡萄牙语}{pu2tao2ya2yu3}{12;11;4;9}
  \significado{s.}{português, língua portuguesa}
  \veja{葡语}{pu2yu3}
\end{verbete}
\begin{verbete}{葡文}{pu2wen2}{12;4}
  \significado{s.}{português, língua portuguesa}
  \veja{葡萄牙文}{pu2tao2ya2wen2}
\end{verbete}
\begin{verbete}{葡语}{pu2yu3}{12;9}
  \significado{s.}{português, língua portuguesa}
  \veja{葡萄牙语}{pu2tao2ya2yu3}
\end{verbete}
\begin{verbete}{普通话}{pu3tong1hua4}{12;10;8}
  \significado*{s.}{Mandarim (lit. ``linguagem comum''); Putonghua (fala comum da língua chinesa); discurso comum}
\end{verbete}

%%%%% EOF %%%%%

%%%
%%% Q
%%%
%\section*{Q}
\addcontentsline{toc}{section}{Q}

\begin{verbete}{七}{qi1}{2}
  \significado{num.}{sete, 7}
\end{verbete}

\begin{verbete}{七夕}{qi1xi1}{2;3}
  \significado*{s.}{Dia dos Namorados Chinês, quando o vaqueiro e a tecelã (牛郎织女) têm permissão para se reunirem anualmente; Festival das Meninas; Festival Duplo Sete, noite do sétimo mês lunar}
  \veja{牛郎织女}{niu2lang2zhi1nv3}
\end{verbete}

\begin{verbete}{其实}{qi2shi2}{8;8}
  \significado{adv.}{na verdade; de fato}
\end{verbete}

\begin{verbete}{其他}{qi2ta1}{8;5}
  \significado{pron.}{todos os outro(s); o resto}
\end{verbete}

\begin{verbete}{奇怪}{qi2guai4}{8;8}
  \significado{adj.}{estranho}
  \significado{v.}{ficar perplexo; maravilhar-se}
\end{verbete}

\begin{verbete}{奇迹}{qi2ji4}{8;9}
  \significado{adj.}{milagroso}
  \significado{s.}{milagre}
\end{verbete}

\begin{verbete}{骑}{qi2}{11}
  \significado{p.c.}{para cavalos de sela}
  \significado{v.}{andar (cavalo, bicicleta, etc.); sentar-se montado}
\end{verbete}

\begin{verbete}{骑车}{qi2che1}{11;4}
  \significado{v.}{andar de bicicleta; pedalar}
\end{verbete}

\begin{verbete}{旗}{qi2}{14}
  \significado[面]{s.}{bandeira}
\end{verbete}

\begin{verbete}{企业}{qi3ye4}{6;5}
  \significado[家]{s.}{empresa; corporação; empreendimento; firma}
\end{verbete}

\begin{verbete}{岂有此理}{qi3you3ci3li3}{6;6;6;11}
  \significado{interj.}{Que exorbitante!; Absurdo!; Como isso pode ser assim?; Ridículo!}
\end{verbete}

\begin{verbete}{起床}{qi3chuang2}{10;7}
  \significado{v.+compl.}{sair da cama; levantar-se}
\end{verbete}

\begin{verbete}{起来}{qi3lai5}{10;7}
  \significado{v.+compl.}{levantar-se}
\end{verbete}

\begin{verbete}{起跳}{qi3tiao4}{10;13}
  \significado{v.}{(atletismo) decolar (no início de um salto); (de preço, salário, etc.) começar (de um determinado nível)}
\end{verbete}

\begin{verbete}{气球}{qi4qiu2}{4;11}
  \significado{s.}{balão}
\end{verbete}

\begin{verbete}{气温}{qi4wen1}{4;12}
  \significado[个]{s.}{temperatura do ar}
\end{verbete}

\begin{verbete}{气质}{qi4zhi4}{4;8}
  \significado{s.}{traços de personalidade, temperamento, disposição; aura, ar, sentimento, \emph{vibe}; refinamento, sofisticação, classe}
\end{verbete}

\begin{verbete}{汽车}{qi4che1}{7;4}
  \significado[辆]{s.}{automóvel; carro; veículo motorizado}
\end{verbete}

\begin{verbete}{器}{qi4}{16}
  \significado[台]{s.}{dispositivo; ferramenta; utensílio}
\end{verbete}

\begin{verbete}{恰}{qia4}{9}
  \significado{adv.}{exatamente; apenas}
\end{verbete}

\begin{verbete}{恰到好处}{qia4dao4hao3chu4}{9;8;6;5}
  \significado{expr.}{é simplesmente perfeito; é simplesmente correto}
\end{verbete}

\begin{verbete}{恰好}{qia4hao3}{9;6}
  \significado{adv.}{certo; por sorte; ao que parece; por sorte coincidência}
\end{verbete}

\begin{verbete}{千}{qian1}{3}
  \significado{num.}{mil, 1.000}
\end{verbete}

\begin{verbete}{千古}{qian1gu3}{3;5}
  \significado{adv.}{por toda a eternidade; em todas as idades}
  \significado{s.}{eternidade (usada em um dístico elegíaco, coroa de flores, etc., dedicada aos mortos)}
\end{verbete}

\begin{verbete}{千年}{qian1nian2}{3;6}
  \significado{s.}{milênio}
\end{verbete}

\begin{verbete}{千千万万}{qian1qian1wan4wan4}{3;3;3;3}
  \significado{num.}{inumerável; números incontáveis; milhares e milhares}
\end{verbete}

\begin{verbete}{千万}{qian1wan4}{3;3}
  \significado{adv.}{absolutamente; por todos os meios; (quando usado negativamente) sob nehuma circunstância; nunca; pelo amor de Deus; por favor; não}
\end{verbete}

\begin{verbete}{签}{qian1}{13}
  \significado{s.}{vara de bambu com inscrição (usada em adivinhação, jogos de azar, sorteios, etc.); rótulo; pequena lasca de madeira; etiqueta}
  \significado{v.}{assinar}
\end{verbete}

\begin{verbete}{签名}{qian1ming2}{13;6}
  \significado{s.}{assinatura}
  \significado{v.}{autografar; assinar (o nome com uma caneta, etc.)}
\end{verbete}

\begin{verbete}{前}{qian2}{9}
  \significado{p.l.}{frente; em frente de; A.C. (Antes de~Cristo)}
  \veja{公元}{gong1yuan2}
  \exemplo{前293年}
\end{verbete}

\begin{verbete}{前边}{qian2bian5}{9;5}
  \significado{p.l.}{à frente; da frente}
\end{verbete}

\begin{verbete}{前面}{qian2mian4}{9;9}
  \significado{p.l.}{à frente; da frente}
\end{verbete}

\begin{verbete}{前年}{qian2nian2}{9;6}
  \significado{p.t.}{há dois anos}
\end{verbete}

\begin{verbete}{前天}{qian2tian1}{9;4}
  \significado{p.t.}{anteontem}
\end{verbete}

\begin{verbete}{钱}{qian2}{10}
  \significado*{s.}{sobrenome Qian}
  \significado[笔]{s.}{moeda; dinheiro}
\end{verbete}

\begin{verbete}{钱包}{qian2bao1}{10;5}
  \significado{s.}{carteira; bolsa}
\end{verbete}

\begin{verbete}{潜在}{qian2zai4}{15;6}
  \significado{adj.}{oculto; latente}
  \significado{s.}{potencial}
\end{verbete}

\begin{verbete}{强}{qiang2}{12}
  \significado*{s.}{sobrenome Qiang}
  \significado{adj.}{melhor em sua categoria; melhor; poderoso; forte; vigoroso; violento}
  \veja{强}{jiang4}
  \veja{强}{qiang3}
\end{verbete}

\begin{verbete}{墙}{qiang2}{14}
  \significado[面,堵]{s.}{parede}
  \significado{v.}{(gíria) bloquear (um website)(usado geralmente na voz passiva: 被墙)}
\end{verbete}

\begin{verbete}{墙纸}{qiang2zhi3}{14;7}
  \significado{s.}{papel de parede}
\end{verbete}

\begin{verbete}{抢掠}{qiang3lve4}{7;11}
  \significado{s.}{saque; pilhagem}
  \significado{v.}{saquear; pilhar}
\end{verbete}

\begin{verbete}{强}{qiang3}{12}
  \significado{v.}{obrigar; forçar; fazer um esforço; esforçar-se}
  \veja{强}{jiang4}
  \veja{强}{qiang2}
\end{verbete}

\begin{verbete}{桥}{qiao2}{10}
  \significado[座]{s.}{ponte}
\end{verbete}

\begin{verbete}{瞧}{qiao2}{17}
  \significado{v.}{olhar para; ver; ver (ir a um médico); visitar}
\end{verbete}

\begin{verbete}{巧合}{qiao3he2}{5;6}
  \significado{s.}{coincidência}
  \significado{v.}{coincidir}
\end{verbete}

\begin{verbete}{巧克力}{qiao3ke4li4}{5;7;2}
  \significado[块]{s.}{chocolate (empréstimo linguístico)}
\end{verbete}

\begin{verbete}{切割}{qie1ge1}{4;12}
  \significado{v.}{cortar}
\end{verbete}

\begin{verbete}{茄子}{qie2zi5}{8;3}
  \significado{s.}{berinjela chinesa; ``xis'' fonético (ao ser fotografado), equivale ao ``diga xis''}
\end{verbete}

\begin{verbete}{亲自}{qin1zi4}{9;6}
  \significado{adv.}{pessoalmente; a si mesmo}
\end{verbete}

\begin{verbete}{侵略}{qin1lve4}{9;11}
  \significado{s.}{invasão}
  \significado{v.}{invadir}
\end{verbete}

\begin{verbete}{芹菜}{qin2cai4}{7;11}
  \significado{s.}{salsão}
\end{verbete}

\begin{verbete}{琴键}{qin2jian4}{12;13}
  \significado{s.}{tecla de piano}
\end{verbete}

\begin{verbete}{擒获}{qin2huo4}{15;10}
  \significado{v.}{apreender; capturar}
\end{verbete}

\begin{verbete}{青菜}{qing1cai4}{8;11}
  \significado{s.}{verduras}
\end{verbete}

\begin{verbete}{青春}{qing1chun1}{8;9}
  \significado{s.}{juventude}
\end{verbete}

\begin{verbete}{青椒}{qing1jiao1}{8;12}
  \significado{s.}{pimenta verde}
\end{verbete}

\begin{verbete}{青年节}{qing1nian2jie2}{8;6;5}
  \significado*{s.}{Dia da Juventude (4 de maio)}
\end{verbete}

\begin{verbete}{青天}{qing1tian1}{8;4}
  \significado{s.}{céu claro, limpo ou azul}
\end{verbete}

\begin{verbete}{青铜}{qing1tong2}{8;11}
  \significado{s.}{bronze (liga de cobre, 銅, e estanho, 锡)}
\end{verbete}

\begin{verbete}{青蛙}{qing1wa1}{8;12}
  \significado{adj.}{(gíria velha) cara feio}
  \significado[只]{s.}{sapo;}
\end{verbete}

\begin{verbete}{青玉米}{qing1yu4mi3}{8;5;6}
  \significado{s.}{milho verde}
\end{verbete}

\begin{verbete}{轻松}{qing1song1}{9;8}
  \significado{adj.}{leve; gentil; relaxado; sem esforço; descomplicado}
  \significado{v.}{relaxar; levar as coisas menos a sério}
\end{verbete}

\begin{verbete}{轻易}{qing1yi4}{9;8}
  \significado{adj.}{fácil, simples}
  \significado{adv.}{impulsivamente, abruptamente}
\end{verbete}

\begin{verbete}{倾城}{qing1cheng2}{10;9}
  \significado{adj.}{sedutora (mulher)}
  \significado{adv.}{de todo o lugar; vindo de todos os lugares}
  \significado{v.}{arruinar e derrubar o estado}
\end{verbete}

\begin{verbete}{清}{qing1}{11}
  \significado*{s.}{sobrenome Qing}
  \significado{adj.}{claro; limpo (água, etc.); tranquilo, quieto, puro, não corrompido; distinto}
  \significado{v.}{limpar, resolver (contas)}
\end{verbete}

\begin{verbete}{清唱}{qing1chang4}{11;11}
  \significado{v.}{cantar à capela}
\end{verbete}

\begin{verbete}{清彻}{qing1che4}{11;7}
  \variante{清澈}
\end{verbete}

\begin{verbete}{清澈}{qing1che4}{11;15}
  \significado{adj.}{claro; límpido}
\end{verbete}

\begin{verbete}{清楚}{qing1chu5}{11;13}
  \significado{adj.}{claro; límpido}
  \significado{v.}{ser claro sobre; entender completamente}
\end{verbete}

\begin{verbete}{清理}{qing1li3}{11;11}
  \significado{v.}{limpar; arrumar; descartar}
\end{verbete}

\begin{verbete}{清凉}{qing1liang2}{11;10}
  \significado{adj.}{fresco; refrescante; (roupa) ousada, reveladora}
\end{verbete}

\begin{verbete}{清明节}{qing1ming2jie2}{11;8;5}
  \significado*{s.}{Dia Qingming, Dia dos Finados (uma das 24~divisões do ano solar no calendário lunar chinês:~dia~4 ou 5~de~abril solar)}
\end{verbete}

\begin{verbete}{清爽}{qing1shuang3}{11;11}
  \significado{adj.}{refrescante; relaxado}
\end{verbete}

\begin{verbete}{清晰}{qing1xi1}{11;12}
  \significado{adj.}{claro, distinto}
\end{verbete}

\begin{verbete}{蜻蜓}{qing1ting2}{14;12}
  \significado{s.}{libélula}
\end{verbete}

\begin{verbete}{蜻蝏}{qing1ting2}{14;15}
  \variante{蜻蜓}
\end{verbete}

\begin{verbete}{情感}{qing2gan3}{11;13}
  \significado{s.}{sentimento; emoção}
  \significado{v.}{mover-se (emocionalmente)}
\end{verbete}

\begin{verbete}{情况}{qing2kuang4}{11;7}
  \significado[个,种]{s.}{circunstância; situação; estado das coisas}
\end{verbete}

\begin{verbete}{情绪}{qing2xu4}{11;11}
  \significado[种]{s.}{humor; estado da mente; mau humor}
\end{verbete}

\begin{verbete}{请}{qing3}{10}
  \significado{v.}{por favor (fazer alguma coisa); perguntar; convidar; solicitar}
\end{verbete}

\begin{verbete}{请假条}{qing3jia4tiao2}{10;11;7}
  \significado{s.}{pedido de licença de ausência (do trabalho ou da escola)}
\end{verbete}

\begin{verbete}{请客}{qing3ke4}{10;9}
  \significado{v.+compl.}{entreter os convidados; dar um jantar; convidar para jantar}
\end{verbete}

\begin{verbete}{请求}{qing3qiu2}{10;7}
  \significado[个]{s.}{solicitação}
  \significado{v.}{solicitar; perguntar}
\end{verbete}

\begin{verbete}{请问}{qing3wen4}{10;6}
  \significado{expr.}{Com licença, posso perguntar\dots? (para perguntar por qualquer coisa)}
\end{verbete}

\begin{verbete}{丘陵}{qiu1ling2}{5;10}
  \significado{s.}{colinas}
\end{verbete}

\begin{verbete}{秋}{qiu1}{9}
  \significado*{s.}{sobrenome Qiu}
  \significado{s.}{outono; colheita}
\end{verbete}

\begin{verbete}{秋天}{qiu1tian1}{9;4}
  \significado[个]{p.t./s.}{outono}
\end{verbete}

\begin{verbete}{球}{qiu2}{11}
  \significado[个]{s.}{bola; esfera; globo}
  \significado[场]{s.}{jogo; partida de bola}
\end{verbete}

\begin{verbete}{球迷}{qiu2mi2}{11;9}
  \significado[个]{s.}{fã (esportes de bola)}
\end{verbete}

\begin{verbete}{球拍}{qiu2pai1}{11;8}
  \significado{s.}{raquete}
\end{verbete}

\begin{verbete}{区}{qu1}{4}
  \significado[个]{s.}{área; região; distrito}
  \veja{区}{ou1}
\end{verbete}

\begin{verbete}{区域}{qu1yu4}{4;11}
  \significado{s.}{área; região; distrito}
\end{verbete}

\begin{verbete}{曲棍球}{qu1gun4qiu2}{6;12;11}
  \significado{s.}{hóquei em campo}
\end{verbete}

\begin{verbete}{驱}{qu1}{7}
  \significado{v.}{expulsar; repelir}
\end{verbete}

\begin{verbete}{趋势}{qu1shi4}{12;8}
  \significado{s.}{tendência}
\end{verbete}

\begin{verbete}{取}{qu3}{8}
  \significado{v.}{buscar; obter; escolher}
\end{verbete}

\begin{verbete}{取胜}{qu3sheng4}{8;9}
  \significado{v.}{prevalecer sobre os oponentes; marcar uma vitória}
\end{verbete}

\begin{verbete}{取水}{qu3shui3}{8;4}
  \significado{v.}{obter água (de um poço, etc.)}
\end{verbete}

\begin{verbete}{取现}{qu3xian4}{8;8}
  \significado{v.}{sacar dinheiro}
\end{verbete}

\begin{verbete}{取悦}{qu3yue4}{8;10}
  \significado{v.}{tentar agradar}
\end{verbete}

\begin{verbete}{厺}{qu4}{5}
  \variante{去}
\end{verbete}

\begin{verbete}{去}{qu4}{5}
  \significado{v.}{ir; eufenismo:~morrer}
\end{verbete}

\begin{verbete}{去年}{qu4nian2}{5;6}
  \significado{p.t.}{ano passado}
\end{verbete}

\begin{verbete}{去死}{qu4si3}{5;6}
  \significado{interj.}{Caia morto!; Vá para o Inferno!}
\end{verbete}

\begin{verbete}{圈粉}{quan1fen3}{11;10}
  \significado{s.}{(neologismo, coloquial) ganhar alguém como fã; obter novos fãs}
\end{verbete}

\begin{verbete}{全}{quan2}{6}
  \significado*{s.}{sobrenome Quan}
  \significado{adv.}{completamente; totalmente}
\end{verbete}

\begin{verbete}{全部}{quan2bu4}{6;10}
  \significado{adv.}{todo, todos}
\end{verbete}

\begin{verbete}{全职}{quan2zhi2}{6;11}
  \significado{s.}{período integral; tempo inteiro, \emph{full-time} (trabalho)}
\end{verbete}

\begin{verbete}{拳法}{quan2fa3}{10;8}
  \significado{s.}{boxe; luta}
\end{verbete}

\begin{verbete}{拳王}{quan2wang2}{10;4}
  \significado{s.}{pugilista; boxeador}
\end{verbete}

\begin{verbete}{犬}{quan3}{4}[94]
  \significado{s.}{cachorro}
\end{verbete}

\begin{verbete}{却}{que4}{7}
  \significado{adv.}{mas; contudo; entretanto}
\end{verbete}

\begin{verbete}{却是}{que4shi4}{7;9}
  \significado{conj.}{no entanto; realmente; o fato é\dots; mas isso é\dots}
\end{verbete}

\begin{verbete}{确}{que4}{12}
  \significado{adj.}{autenticado; sólido; firme; real; verdadeiro}
\end{verbete}

\begin{verbete}{确实}{que4shi2}{12;8}
  \significado{adj.}{real; verdadeiro; confiável}
  \significado{adv.}{realmente}
\end{verbete}

\begin{verbete}{裙子}{qun2zi5}{12;3}
  \significado[条]{s.}{saia; vestido}
\end{verbete}

\begin{verbete}{群山}{qun2shan1}{13;3}
  \significado{s.}{montanhas; uma cadeia de colinas}
\end{verbete}

%%%%% EOF %%%%%

%%%
%%% R
%%%
\section*{R}
\addcontentsline{toc}{section}{R}

\begin{verbete}[12;6]{然后}{ran2hou4}
  \significado{conj.}{depois; logo; portanto}
\end{verbete}
\begin{verbete}[5]{让}{rang4}
  \significado{v.}{deixar alguém fazer alguma coisa; fazer alguém (sentir-se triste, etc.); permitir; conceder}
\end{verbete}
\begin{verbete}[10]{热}{re4}
  \significado{adj.}{quente (clima); fervente; ardente; fervoroso}
  \significado{v.}{aquecer; ferver}
\end{verbete}
\begin{verbete}[10;8]{热闹}{re4nao5}
  \significado{adj.}{animado; movimentado com barulho e excitação}
\end{verbete}
\begin{verbete}[2]{人}{ren2}
  \significado[个,位]{s.}{pessoa; gente}
\end{verbete}
\begin{verbete}[2;3]{人口}{ren2kou3}
  \significado{s.}{pessoas; população}
\end{verbete}
\begin{verbete}[2;5]{人民}{ren2min2}
  \significado[个]{s.}{povo; população}
\end{verbete}
\begin{verbete*}[2;5;4]{人民币}{ren2min2bi4}
  \significado{s.}{Renminbi (RMB); Yuan Chinês (CYN); nome da moeda chinesa}
\end{verbete*}
\begin{verbete}[2]{儿}{ren2}
  \variante{人}{ren2}
\end{verbete}
\begin{verbete}[4;7]{认识}{ren4shi5}
  \significado{s.}{conhecimento; saber; entendimento}
  \significado{v.}{estar familiarizado com; conhecer alguém; saber; reconhecer}
\end{verbete}
\begin{verbete}[4;10]{认真}{ren4zhen1}
  \significado{adj.}{sério; consciencioso}
  \significado{adv.}{seriamente}
  \significado{v.}{levar a sério}
\end{verbete}
\begin{verbete}[5]{扔}{reng1}
  \significado{v.}{lançar; atirar}
\end{verbete}
\begin{verbete}[5;11]{扔掉}{reng1diao4}
  \significado{v.}{jogar fora}
\end{verbete}
\begin{verbete}[5;7]{扔弃}{reng1qi4}
  \significado{v.}{abandonar; descartar; jogar fora}
\end{verbete}
\begin{verbete}[5;3]{扔下}{reng1xia4}
  \significado{v.}{lançar (uma bomba); derrubar}
\end{verbete}
\begin{verbete}[4]{日}{ri4}
  \significado{p.c.}{dia (mais usado em escrita); data, dia do mês}
\end{verbete}
\begin{verbete*}[4]{日}{ri4}
  \significado{s.}{Japão, abreviatura de 日本}
  \veja{日本}{ri4ben3}[sp]
\end{verbete*}
\begin{verbete*}[4;5]{日本}{ri4ben3}
  \significado{s.}{Japão}
  \veja{日}{ri4}
\end{verbete*}
\begin{verbete}[4;5;2]{日本人}{ri4ben3ren2}
  \significado{s.}{japonês; nascido no Japão}
\end{verbete}
\begin{verbete}[10;8]{容易}{rong2yi4}
  \significado{adj.}{fácil; responsável (por); provável}
\end{verbete}
\begin{verbete}[6]{肉}{rou4}
  \significado{s.}{carne; polpa de uma fruta}
\end{verbete}
\begin{verbete}[6;10]{肉桂}{rou4gui4}
  \significado{s.}{canela}
  \veja{官桂}{guan1gui4}
\end{verbete}
\begin{verbete}[6;8]{如果}{ru2guo3}
  \significado{conj.}{se; caso; no caso de}
\end{verbete}
\begin{verbete*}[16;11]{儒教}{ru2jiao4}
  \significado{v.}{Confucionismo}
\end{verbete*}
\begin{verbete}[8;8]{乳房}{ru3fang2}
  \significado{s.}{seio; mama; úbere}
\end{verbete}
\begin{verbete}[10;9]{辱骂}{ru3ma4}
  \significado{v.}{insultar; abusar}
\end{verbete}
\begin{verbete}[2;3;11;9]{入乡随俗}{ru4xiang1-sui2su2}
  \significado{expr.}{Em roma, faça como os romanos!}
\end{verbete}

%%%%% EOF %%%%%

%%%
%%% S
%%%
%\section*{S}
\addcontentsline{toc}{section}{S}

\begin{verbete}{撒旦}{sa1dan4}{15;5}
  \significado*{s.}{Satã}
\end{verbete}

\begin{verbete}{撒旦主义}{sa1dan4 zhu3yi4}{15;5;5;3}
  \significado*{s.}{Satanismo}
\end{verbete}

\begin{verbete}{撒但}{sa1dan4}{15;7}
  \variante{撒旦}
\end{verbete}

\begin{verbete}{洒水}{sa3shui3}{9;4}
  \significado{v.}{borrifar}
\end{verbete}

\begin{verbete}{飒飒}{sa4sa4}{9;9}
  \significado{s.}{o farfalhar; sussurro, murmúrio (do vento nas árvores, o mar, etc.)}
\end{verbete}

\begin{verbete}{赛}{sai4}{14}
  \significado{s.}{competição}
  \significado{v.}{competir; superar; destacar-se}
\end{verbete}

\begin{verbete}{赛车}{sai4che1}{14;4}
  \significado{s.}{corrida de automóvel; corrida de bicicleta; carro de corrida}
\end{verbete}

\begin{verbete}{三}{san1}{3}
  \significado*{s.}{sobrenome San}
  \significado{num.}{três, 3}
\end{verbete}

\begin{verbete}{三角恋爱}{san1jiao3lian4'ai4}{3;7;10;10}
  \significado{s.}{triângulo amoroso}
\end{verbete}

\begin{verbete}{三轮车}{san1lun2che1}{3;8;4}
  \significado{s.}{triciclo}
\end{verbete}

\begin{verbete}{三明治}{san1ming2zhi4}{3;8;8}
  \significado{s.}{sanduíche (empréstimo linguístico)}
\end{verbete}

\begin{verbete}{散步}{san4bu4}{12;7}
  \significado{v.+compl.}{dar uma volta; dar um passeio; passear}
\end{verbete}

\begin{verbete}{丧钟}{sang1zhong1}{8;9}
  \significado{s.}{sentença de morte}
\end{verbete}

\begin{verbete}{桑}{sang1}{10}
  \significado*{s.}{sobrenome Sang}
  \significado{s.}{amoreira}
\end{verbete}

\begin{verbete}{桑巴舞}{sang1ba1wu3}{10;4;14}
  \significado{s.}{samba}
\end{verbete}

\begin{verbete}{桑树}{sang1shu4}{10;9}
  \significado{s.}{amoreira, suas folhas são utilizadas para alimentar bichos-da-seda}
\end{verbete}

\begin{verbete}{骚乱}{sao1luan4}{12;7}
  \significado{s.}{rebelião; perturbação; tumulto}
  \significado{v.}{criar um distúrbio}
\end{verbete}

\begin{verbete}{嫂子}{sao3zi5}{12;3}
  \significado{s.}{esposa do irmão mais velho}
\end{verbete}

\begin{verbete}{色狼}{se4lang2}{6;10}
  \significado*{s.}{Sátiro}
  \significado{adj.}{depravado; tarado}
\end{verbete}

\begin{verbete}{森林}{sen1lin2}{12;8}
  \significado{s.}{floresta}
\end{verbete}

\begin{verbete}{僧}{seng1}{14}
  \significado{s.}{monge Budista (abreviatura de 僧伽)}
  \veja{僧伽}{seng1qie2}
\end{verbete}

\begin{verbete}{僧伽}{seng1qie2}{14;7}
  \significado{s.}{sangha ou sanga (Budismo); a comunidade monástica; monge}
\end{verbete}

\begin{verbete}{沙}{sha1}{7}
  \significado*{s.}{sobrenome Sha}
  \significado[粒]{s.}{areia; cascalho; grânulo; pó}
\end{verbete}

\begin{verbete}{沙漠}{sha1mo4}{7;13}
  \significado[个]{s.}{deserto}
\end{verbete}

\begin{verbete}{沙特}{sha1te4}{7;10}
  \significado*{s.}{Saudita; abreviação de 沙特阿拉伯}
  \veja{沙特阿拉伯}{sha1te4 a1la1bo2}
\end{verbete}

\begin{verbete}{沙特阿拉伯}{sha1te4 a1la1bo2}{7;10;7;8;7}
  \significado*{s.}{Arábia Saudita}
\end{verbete}

\begin{verbete}{沙鱼}{sha1yu2}{7;8}
  \variante{鲨鱼}
\end{verbete}

\begin{verbete}{刹}{sha1}{8}
  \significado{v.}{frear}
  \veja{刹}{cha4}
\end{verbete}

\begin{verbete}{砂}{sha1}{9}
  \variante{沙}
\end{verbete}

\begin{verbete}{莎莎舞}{sha1sha1wu3}{10;10;14}
  \significado{s.}{salsa (dança)}
\end{verbete}

\begin{verbete}{鲨鱼}{sha1yu2}{15;8}
  \significado{s.}{tubarão}
\end{verbete}

\begin{verbete}{啥}{sha2}{11}
  \significado{interr.}{Equivalente a 什么 (dialeto), também pronunciado como \dpy{sha4}}
  \veja{什么}{shen2me5}
\end{verbete}

\begin{verbete}{傻瓜}{sha3gua1}{13;5}
  \significado{adj.}{tolo; burro; simplório; idiota}
  \significado{v.}{enganar; iludir; lograr}
\end{verbete}

\begin{verbete}{傻眼}{sha3yan3}{13;11}
  \significado{adj.}{estupefato; atordoado}
\end{verbete}

\begin{verbete}{啥}{sha4}{11}
  \veja{啥}{sha2}
\end{verbete}

\begin{verbete}{晒干}{shai4gan1}{10;3}
  \significado{v.}{secar ao sol}
\end{verbete}

\begin{verbete}{山}{shan1}{3}[46]
  \significado*{s.}{sobrenome Shan}
  \significado[座]{s.}{montanha; monte; qualquer coisa que se assemelhe a uma montanha}
\end{verbete}

\begin{verbete}{山顶}{shan1ding3}{3;8}
  \significado{s.}{cume da montanha}
\end{verbete}

\begin{verbete}{山东}{shan1dong1}{3;5}
  \significado*{s.}{Shandong}
\end{verbete}

\begin{verbete}{山区}{shan1qu1}{3;4}
  \significado[个]{s.}{área montanhosa; montanhas}
\end{verbete}

\begin{verbete}{山体}{shan1ti3}{3;7}
  \significado{s.}{forma de uma montanha}
\end{verbete}

\begin{verbete}{山羊}{shan1yang2}{3;6}
  \significado{s.}{cabra; (ginástica) cavalo de salto de pequeno porte}
\end{verbete}

\begin{verbete}{闪存盘}{shan3cun2pan2}{5;6;11}
  \significado{s.}{unidade de memória USB; cartão de memória; \emph{pen drive}}
\end{verbete}

\begin{verbete}{扇子}{shan4zi5}{10;3}
  \significado[把]{s.}{leque; abano; abanador}
\end{verbete}

\begin{verbete}{善意}{shan4yi4}{12;13}
  \significado{s.}{boa vontade; benevolência; bondade}
\end{verbete}

\begin{verbete}{禅}{shan4}{12}
  \significado{v.}{abdicar}
  \veja{禅}{chan2}
\end{verbete}

\begin{verbete}{擅自}{shan4zi4}{16;6}
  \significado{adv.}{sem permissão ou autorização; por iniciativa própria}
\end{verbete}

\begin{verbete}{伤}{shang1}{6}
  \significado{s.}{ferida; ferimento}
  \significado{v.}{ferir; ferir-se}
\end{verbete}

\begin{verbete}{伤心}{shang1xin1}{6;4}
  \significado{v.}{sofrer; ter o coração partido; sentir-se profundamente magoado}
\end{verbete}

\begin{verbete}{汤}{shang1}{6}
  \significado{s.}{correnteza forte}
  \veja{汤}{tang1}
\end{verbete}

\begin{verbete}{商店}{shang1dian4}{11;8}
  \significado[家,个]{s.}{loja}
\end{verbete}

\begin{verbete}{商贸}{shang1mao4}{11;9}
  \significado{s.}{comércio}
\end{verbete}

\begin{verbete}{上声}{shang3sheng1}{3;7}
  \significado{s.}{tom descendente e ascendente; terceiro tom no mandarim moderno}
\end{verbete}

\begin{verbete}{赏赐}{shang3ci4}{12;12}
  \significado{s.}{recompensa; prêmio}
  \significado{v.}{recompensar; premiar}
\end{verbete}

\begin{verbete}{赏心悦目}{shang3xin1yue4mu4}{12;4;10;5}
  \significado{expr.}{"Aquece o coração e encanta os olhos."}
\end{verbete}

\begin{verbete}{上}{shang4}{3}
  \significado{p.l.}{acima; em cima; sobre}
  \significado{v.d.}{subir; entrar em; frequentar (aula ou universidade)}
\end{verbete}

\begin{verbete}{上班}{shang4ban1}{3;10}
  \significado{v.+compl.}{ir para o trabalho; ir para o emprego; estar de plantão}
\end{verbete}

\begin{verbete}{上边}{shang4bian5}{3;5}
  \significado{p.l.}{acima de; parte de cima; por cima}
\end{verbete}

\begin{verbete}{上车}{shang4che1}{3;4}
  \significado{v.}{entrar (em ônibus, trem, carro, etc.)}
\end{verbete}

\begin{verbete}{上访}{shang4fang3}{3;6}
  \significado{v.}{buscar uma audiência com superiores (especialmente funcionários do governo) para fazer uma petição por algo}
\end{verbete}

\begin{verbete}{上海}{shang4hai3}{3;10}
  \significado*{s.}{Shangai (Xangai)}
\end{verbete}

\begin{verbete}{上课}{shang4ke4}{3;10}
  \significado{v.}{assistir à aula; ir para a aula; ir dar uma aula}
\end{verbete}

\begin{verbete}{上来}{shang4lai2}{3;7}
  \significado{v.d.}{subir (para a minha localização)}
\end{verbete}

\begin{verbete}{上面}{shang4mian4}{3;9}
  \significado{p.l.}{acima de; parte de cima; por cima}
\end{verbete}

\begin{verbete}{上坡路}{shang4po1lu4}{3;8;13}
  \significado{s.}{aclive; progresso; (fig.) tendência ascendente}
\end{verbete}

\begin{verbete}{上去}{shang4qu4}{3;5}
  \significado{v.d.}{subir (a partir da minha localização)}
\end{verbete}

\begin{verbete}{上网}{shang4wang3}{3;6}
  \significado{v.}{conectar à \emph{Internet}; fazer \emph{upload}; ficar \emph{online}}
\end{verbete}

\begin{verbete}{上午}{shang4wu3}{3;4}
  \significado{p.t.}{manhã; de manhã; período antes do meio-dia}
\end{verbete}

\begin{verbete}{上询}{shang4 xun2}{3;8}
  \significado{p.t.}{primeira dezena do mês}
\end{verbete}

\begin{verbete}{尚且}{shang4qie3}{8;5}
  \significado{conj.}{até; ainda}
\end{verbete}

\begin{verbete}{尚且……何况……}{shang4qie3 he2kuang4}{8;5;7;7}
  \significado{conj.}{ainda que\dots, \dots}
\end{verbete}

\begin{verbete}{烧}{shao1}{10}
  \significado{s.}{febre}
  \significado{v.}{queimar; cozinhar; cozer; assar; aquecer; ferver (chá, água, etc.); ter febre; (coloquial) deixar as coisas subirem à cabeça}
\end{verbete}

\begin{verbete}{烧烤}{shao1kao3}{10;10}
  \significado{s.}{churrasco}
  \significado{v.}{assar}
\end{verbete}

\begin{verbete}{稍}{shao1}{12}
  \significado{adv.}{um pouco; ligeiramente; em vez de}
\end{verbete}

\begin{verbete}{稍微}{shao1wei1}{12;13}
  \significado{adv.}{um pouco}
\end{verbete}

\begin{verbete}{少}{shao3}{4}
  \significado{adj.}{pouco, poucos}
  \significado{v.}{sentir falta; faltar; parar (de fazer algo)}
  \veja{少}{shao4}
\end{verbete}

\begin{verbete}{少}{shao4}{4}
  \significado{s.}{jovem}
  \veja{少}{shao3}
\end{verbete}

\begin{verbete}{舌头}{she2tou5}{6;5}
  \significado[个]{s.}{língua; soldado inimigo capturado com o propósito de extrair informações}
\end{verbete}

\begin{verbete}{蛇}{she2}{11}
  \significado[条]{s.}{cobra; serpente}
\end{verbete}

\begin{verbete}{设备}{she4bei4}{6;8}
  \significado[个]{s.}{equipamento; instalações}
\end{verbete}

\begin{verbete}{设计}{she4ji4}{6;4}
  \significado{s.}{projeto; planejamento}
  \significado{v.}{projetar; planejar}
\end{verbete}

\begin{verbete}{射}{she4}{10}
  \significado{v.}{atirar; lançar}
\end{verbete}

\begin{verbete}{摄氏}{she4shi4}{13;4}
  \significado{s.}{graus Celsius (°C), centígrado}
\end{verbete}

\begin{verbete}{谁}{shei2}{10}
  \significado{interr.}{quem?}
  \veja{谁}{shui2}
\end{verbete}

\begin{verbete}{身体}{shen1ti3}{7;7}
  \significado[具,个]{s.}{em pessoa; saúde de alguém; o corpo}
\end{verbete}

\begin{verbete}{身体能力}{shen1ti3 neng2li4}{7;7;10;2}
  \significado{s.}{habilidade física}
\end{verbete}

\begin{verbete}{身体乳}{shen1ti3 ru3}{7;7;8}
  \significado{s.}{loção corporal}
\end{verbete}

\begin{verbete}{身亡}{shen1wang2}{7;3}
  \significado{v.}{morrer}
\end{verbete}

\begin{verbete}{深厚}{shen1hou4}{11;9}
  \significado{adj.}{profundo}
\end{verbete}

\begin{verbete}{深深}{shen1shen1}{11;11}
  \significado{adj.}{profundo}
  \significado{adv.}{profundamente}
\end{verbete}

\begin{verbete}{什么}{shen2me5}{4;3}
  \significado{interr.}{que?; o que?}
  \significado{pron.}{algo; qualquer coisa}
\end{verbete}

\begin{verbete}{什么时候}{shen2me5shi2hou5}{4;3;7;10}
  \significado{interr.}{quando?; a que horas?}
\end{verbete}

\begin{verbete}{神}{shen2}{9}
  \significado*{s.}{Deus}
  \significado{s.}{deus; divindade}
\end{verbete}

\begin{verbete}{神话}{shen2hua4}{9;8}
  \significado{s.}{lenda; conto de fadas; mito; mitologia}
\end{verbete}

\begin{verbete}{神经}{shen2jing1}{9;8}
  \significado{adj.}{desequilibrado; louco; insano}
  \significado{s.}{nervo}
\end{verbete}

\begin{verbete}{神经病的}{shen2jing1bing4de5}{9;8;10;8}
  \significado{adj.}{neurótico}
\end{verbete}

\begin{verbete}{神经病学}{shen2jing1bing4xue2}{9;8;10;8}
  \significado{s.}{neurologia}
\end{verbete}

\begin{verbete}{神明}{shen2ming2}{9;8}
  \significado{s.}{divindades; deuses}
\end{verbete}

\begin{verbete}{神奇}{shen2qi2}{9;8}
  \significado{adj.}{mágico; místico; milagroso}
  \significado{s.}{mágica; milagre}
\end{verbete}

\begin{verbete}{神器}{shen2qi4}{9;16}
  \significado{s.}{objeto mágico; objeto simbólico do poder imperial; arma fina; ferramenta muito útil}
\end{verbete}

\begin{verbete}{神兽}{shen2shou4}{9;11}
  \significado{s.}{animal mitológico; fera}
\end{verbete}

\begin{verbete}{甚而}{shen4'er2}{9;6}
  \significado{conj.}{(ir) tão longe quanto; tanto que}
\end{verbete}

\begin{verbete}{甚或}{shen4huo4}{9;8}
  \significado{conj.}{(ir) tão longe quanto; tanto que}
\end{verbete}

\begin{verbete}{甚至}{shen4zhi4}{9;6}
  \significado{conj.}{(ir) tão longe quanto; tanto que; mesmo (na medida em que)}
\end{verbete}

\begin{verbete}{升起}{sheng1qi3}{4;10}
  \significado{v.}{levantar; içar; subir}
\end{verbete}

\begin{verbete}{生}{sheng1}{5}[100]
  \significado{adj.}{vida; estudante; cru; não cozido}
  \significado{v.}{nascer; dar a luz; crescer}
\end{verbete}

\begin{verbete}{生菜}{sheng1cai4}{5;11}
  \significado{s.}{alface}
\end{verbete}

\begin{verbete}{生的}{sheng1de5}{5;8}
  \significado{conj.}{para evitar isso; para que\dots não\dots}
\end{verbete}

\begin{verbete}{生活}{sheng1huo2}{5;9}
  \significado[道]{s.}{vida; atividade; meios de subsistência}
  \significado{v.}{viver}
\end{verbete}

\begin{verbete}{生活垃圾}{sheng1huo2la1ji1}{5;9;8;6}
  \significado{s.}{lixo doméstico}
\end{verbete}

\begin{verbete}{生活型}{sheng1huo2 xing2}{5;9;9}
  \significado{s.}{forma de vida}
\end{verbete}

\begin{verbete}{生理}{sheng1li3}{5;11}
  \significado{adj.}{fisiológico}
  \significado{s.}{fisiologia}
\end{verbete}

\begin{verbete}{生气}{sheng1qi4}{5;4}
  \significado{adj.}{irritado; zangado}
  \significado{v.+compl.}{irritar-se; zangar-se}
\end{verbete}

\begin{verbete}{生日}{sheng1ri4}{5;4}
  \significado[个]{s.}{aniversário}
\end{verbete}

\begin{verbete}{生物}{sheng1wu4}{5;8}
  \significado{adj.}{biológico}
  \significado{s.}{biologia (disciplina); organismo; ser vivo}
\end{verbete}

\begin{verbete}{生意}{sheng1yi4}{5;13}
  \significado{s.}{força vital; vitalidade}
  \veja{生意}{sheng1yi5}
\end{verbete}

\begin{verbete}{生意}{sheng1yi5}{5;13}
  \significado{s.}{negócio}
  \veja{生意}{sheng1yi4}
\end{verbete}

\begin{verbete}{生鱼片}{sheng1yu2pian4}{5;8;4}
  \significado{s.}{fatias de peixe cru, \emph{sashimi}}
\end{verbete}

\begin{verbete}{声明}{sheng1ming2}{7;8}
  \significado[项,份]{s.}{declaração}
  \significado{v.}{declarar}
\end{verbete}

\begin{verbete}{绳子}{sheng2zi5}{11;3}
  \significado[条]{s.}{corda; cordão}
\end{verbete}

\begin{verbete}{省}{sheng3}{9}
  \significado{s.}{província}
  \significado{v.}{economizar; omitir; salvar}
  \veja{省}{xing3}
\end{verbete}

\begin{verbete}{省城}{sheng3cheng2}{9;9}
  \significado{s.}{capital da província}
\end{verbete}

\begin{verbete}{省会}{sheng3hui4}{9;6}
  \significado{s.}{capital da província}
\end{verbete}

\begin{verbete}{省俭}{sheng3jian3}{9;9}
  \significado{s.}{econômico; frugal}
  \significado{v.}{economizar}
\end{verbete}

\begin{verbete}{省力}{sheng3li4}{9;2}
  \significado{v.}{economizar esforço ou trabalho}
\end{verbete}

\begin{verbete}{省钱}{sheng3qian2}{9;10}
  \significado{v.}{economizar dinheiro}
\end{verbete}

\begin{verbete}{省却}{sheng3que4}{9;7}
  \significado{v.}{livrar-se (para economizar espaço); salvar}
\end{verbete}

\begin{verbete}{省心}{sheng3xin1}{9;4}
  \significado{adj.}{despreocupado}
  \significado{v.}{ser poupado de preocupações; despreocupar-se}
\end{verbete}

\begin{verbete}{省长}{sheng3zhang3}{9;4}
  \significado*{s.}{Governador; governador de uma província}
\end{verbete}

\begin{verbete}{圣诞节}{sheng4dan4jie2}{5;8;5}
  \significado*{s.}{Natal}
\end{verbete}

\begin{verbete}{圣地}{sheng4di4}{5;6}
  \significado{s.}{terra santa (de uma religião); lugar sagrado; santuário; cidade santa (como Jerusalém, Meca, etc.); centro de interesse histórico}
\end{verbete}

\begin{verbete}{胜利}{sheng4li4}{9;7}
  \significado[个]{s.}{vitória}
\end{verbete}

\begin{verbete}{胜算}{sheng4suan4}{9;14}
  \significado{s.}{probabilidade de sucesso; estratégia que garante o sucesso}
  \significado{v.}{ter certeza do sucesso}
\end{verbete}

\begin{verbete}{盛宴}{sheng4yan4}{11;10}
  \significado{s.}{celebração}
\end{verbete}

\begin{verbete}{失落}{shi1luo4}{5;12}
  \significado{s.}{frustração; decepção; perda}
  \significado{v.}{perder (algo); cair (algo); sentir uma sensação de perda}
\end{verbete}

\begin{verbete}{失眠}{shi1mian2}{5;10}
  \significado{s.}{insônia}
  \significado{v.}{ter insônia}
\end{verbete}

\begin{verbete}{失望}{shi1wang4}{5;11}
  \significado{adj.}{desapontado}
  \significado{v.}{perder a esperança; desesperar}
\end{verbete}

\begin{verbete}{师傅}{shi1fu5}{6;12}
  \significado[个,位,名]{s.}{técnico; mestre-trabalhador; forma respeitosa de tratamento para homens mais velhos}
\end{verbete}

\begin{verbete}{诗句}{shi1ju4}{8;5}
  \significado[行]{s.}{verso; versículo}
\end{verbete}

\begin{verbete}{诗意}{shi1yi4}{8;13}
  \significado{adj.}{poético}
  \significado{s.}{poesia}
\end{verbete}

\begin{verbete}{十}{shi2}{2}[24]
  \significado{num.}{dez, 10; dezena}
\end{verbete}

\begin{verbete}{十分}{shi2fen1}{2;4}
  \significado{adv.}{muito; extremamente; totalmente; absolutamente}
\end{verbete}

\begin{verbete}{时差}{shi2cha1}{7;9}
  \significado{s.}{diferença de tempo; \emph{jet lag}}
\end{verbete}

\begin{verbete}{时光}{shi2guang1}{7;6}
  \significado{s.}{tempo; época; período de tempo}
\end{verbete}

\begin{verbete}{时候}{shi2hou5}{7;10}
  \significado{interr.}{quando?}
  \significado{s.}{duração de tempo; momento; período; tempo}
\end{verbete}

\begin{verbete}{时间}{shi2jian1}{7;7}
  \significado{s.}{(conceito de, duração de, um ponto no) tempo}
\end{verbete}

\begin{verbete}{时刻}{shi2ke4}{7;8}
  \significado{adv.}{constantemente; sempre}
  \significado[个,段]{s.}{tempo; conjuntura; momento; período de tempo}
\end{verbete}

\begin{verbete}{时时}{shi2shi2}{7;7}
  \significado{adv.}{muitas vezes; constantemente}
\end{verbete}

\begin{verbete}{实力}{shi2li4}{8;2}
  \significado{s.}{força}
\end{verbete}

\begin{verbete}{实现}{shi2xian4}{8;8}
  \significado{v.}{alcançar, implementar, constatar}
\end{verbete}

\begin{verbete}{实在}{shi2zai4}{8;6}
  \significado{adv.}{realmente; verdadeiramente; de fato; na verdade}
\end{verbete}

\begin{verbete}{食品}{shi2pin3}{9;9}
  \significado[种]{s.}{comida; alimento; produtos alimentícios; provisões}
\end{verbete}

\begin{verbete}{食堂}{shi2tang2}{9;11}
  \significado[个,间]{s.}{sala de jantar}
\end{verbete}

\begin{verbete}{食物}{shi2wu4}{9;8}
  \significado[种]{s.}{comida}
\end{verbete}

\begin{verbete}{世代}{shi4dai4}{5;5}
  \significado{adv.}{por muitas gerações}
  \significado{s.}{geração; era}
\end{verbete}

\begin{verbete}{世界}{shi4jie4}{5;9}
  \significado[个]{s.}{mundo}
\end{verbete}

\begin{verbete}{世界杯}{shi4jie4bei1}{5;9;8}
  \significado*{s.}{Copa do Mundo}
\end{verbete}

\begin{verbete}{市场}{shi4chang3}{5;6}
  \significado{s.}{mercado (também no abstrato)}
\end{verbete}

\begin{verbete}{市区}{shi4qu1}{5;4}
  \significado{s.}{centro da cidade; distrito urbano}
\end{verbete}

\begin{verbete}{市中心}{shi4zhong1xin1}{5;4;4}
  \significado{s.}{centro da cidade}
\end{verbete}

\begin{verbete}{式}{shi4}{6}
  \significado{s.}{tipo; forma; padrão; estilo}
\end{verbete}

\begin{verbete}{事}{shi4}{8}
  \significado[件,桩,回]{s.}{coisa; assunto; item; matéria; coisa de trabalho; caso}
\end{verbete}

\begin{verbete}{事故}{shi4gu4}{8;9}
  \significado[桩,起,次]{s.}{acidente}
\end{verbete}

\begin{verbete}{事儿}{shi4r5}{8;2}
  \significado[件,桩]{s.}{o emprego; negócio; afazeres; assunto que precisa ser resolvido; matéria}
\end{verbete}

\begin{verbete}{视频}{shi4pin2}{8;13}
  \significado{s.}{vídeo}
\end{verbete}

\begin{verbete}{试}{shi4}{8}
  \significado{s.}{exame; experimento; prova; teste}
  \significado{v.}{experimentar; provar; teste}
\end{verbete}

\begin{verbete}{室}{shi4}{9}
  \significado*{s.}{sobrenome Shi}
  \significado{s.}{família ou clã; cova; cômodo; bainha; unidade de trabalho}
\end{verbete}

\begin{verbete}{是}{shi4}{9}
  \significado{v.}{ser}
\end{verbete}

\begin{verbete}{是的}{shi4de5}{9;8}
  \significado{adv.}{sim; está certo}
\end{verbete}

\begin{verbete}{收}{shou1}{6}
  \significado{expr.}{aos cuidados de (usado na linha de endereço após o nome)}
  \significado{v.}{receber; aceitar; coletar; colher; guardar}
\end{verbete}

\begin{verbete}{收到}{shou1dao4}{6;8}
  \significado{v.}{receber}
\end{verbete}

\begin{verbete}{收据}{shou1ju4}{6;11}
  \significado[张]{s.}{recibo; \emph{voucher}}
\end{verbete}

\begin{verbete}{收看}{shou1kan4}{6;9}
  \significado{v.}{assistir (a um programa de TV)}
\end{verbete}

\begin{verbete}{收买}{shou1mai3}{6;6}
  \significado{v.}{subornar; comprar}
\end{verbete}

\begin{verbete}{手}{shou3}{4}[64]
  \significado{adj.}{conveniente}
  \significado{p.c.}{de habilidade}
  \significado[双,只]{s.}{mão; pessoa envolvida em certos tipos de trabalho; pessoa qualificada em certos tipos de trabalho}
  \significado{v.}{segurar (formal)}
\end{verbete}

\begin{verbete}{手臂}{shou3bi4}{4;17}
  \significado{s.}{braço}
\end{verbete}

\begin{verbete}{手机}{shou3ji1}{4;6}
  \significado[部,支]{s.}{telefone celular, móvel}
\end{verbete}

\begin{verbete}{手指}{shou3zhi3}{4;9}
  \significado[个,只]{s.}{dedo}
\end{verbete}

\begin{verbete}{守门员}{shou3men2yuan2}{6;3;7}
  \significado{s.}{goleiro}
\end{verbete}

\begin{verbete}{首相}{shou3xiang4}{9;9}
  \significado*{s.}{Primeiro-Ministro (Japão, UK, etc.)}
\end{verbete}

\begin{verbete}{掱}{shou3}{12}
  \variante{手}
\end{verbete}

\begin{verbete}{受限}{shou4xian4}{8;8}
  \significado{v.}{ser limitado; ser restrito; ser constrangido}
\end{verbete}

\begin{verbete}{瘦}{shou4}{14}
  \significado{adj.}{magro; emagrecido; apertado (roupas); improdutivo (terras); magro (carne)}
  \significado{v.}{perder peso}
\end{verbete}

\begin{verbete}{书}{shu1}{4}
  \significado[本,册,部]{s.}{livro; carta; documento}
\end{verbete}

\begin{verbete}{舒服}{shu1fu5}{12;8}
  \significado{adj.}{estar confortável; bem disposto; (sentir-se) bem}
\end{verbete}

\begin{verbete}{熟练}{shu2lian4}{15;8}
  \significado{adj.}{especializado; proficiente; qualificado; habilidoso}
\end{verbete}

\begin{verbete}{熟悉}{shu2xi1}{15;11}
  \significado{v.}{conhecer bem; estar familiarizado com}
\end{verbete}

\begin{verbete}{属}{shu3}{12}
  \significado{s.}{categoria; gênero (taxonomia); familiares; dependentes}
  \significado{v.}{pertencer; subordinar; nascer no ano do signo de (um dos doze animais zodiacais); provar ser; constituir}
  \veja{属}{zhu3}
\end{verbete}

\begin{verbete}{属于}{shu3yu2}{12;3}
  \significado{v.}{ser classificado como; pertencer a; fazer parte de}
\end{verbete}

\begin{verbete}{暑假}{shu3jia4}{12;11}
  \significado[个]{s.}{férias de verão}
\end{verbete}

\begin{verbete}{薯}{shu3}{16}
  \significado{s.}{batata; inhame}
\end{verbete}

\begin{verbete}{束}{shu4}{7}
  \significado*{s.}{sobrenome Shu}
  \significado{p.c.}{para cachos, feixes, feixes de luz, etc.}
  \significado{s.}{monte; pacote; maço; feixe; cacho}
  \significado{v.}{vincular; controlar}
\end{verbete}

\begin{verbete}{树}{shu4}{9}
  \significado[棵]{s.}{árvore}
  \significado{v.}{cultivar}
\end{verbete}

\begin{verbete}{树莓}{shu4mei2}{9;10}
  \significado{s.}{framboesa}
\end{verbete}

\begin{verbete}{树木}{shu4mu4}{9;4}
  \significado{s.}{árvore}
\end{verbete}

\begin{verbete}{树叶}{shu4ye4}{9;5}
  \significado{s.}{folhas de árvores}
\end{verbete}

\begin{verbete}{数学}{shu4xue2}{13;8}
  \significado{s.}{matemática (disciplina)}
\end{verbete}

\begin{verbete}{刷子}{shua1zi5}{8;3}
  \significado[把]{s.}{pincel; escova; escovão}
\end{verbete}

\begin{verbete}{耍赖}{shua3lai4}{9;13}
  \significado{v.}{agir descaradamente; recusar -se a reconhecer que alguém perdeu o jogo ou fez uma promessa, etc.; agir como um idiota; agir como se algo nunca tivesse acontecido}
\end{verbete}

\begin{verbete}{摔}{shuai1}{14}
  \significado{v.}{cair; cair e quebrar; partir}
\end{verbete}

\begin{verbete}{帅}{shuai4}{5}
  \significado*{s.}{sobrenome Shuai}
  \significado{adj.}{elegante; agradável à vista; gracioso; inteligente}
  \significado{interj.}{Legal!}
  \significado{s.}{comandante em chefe}
\end{verbete}

\begin{verbete}{双层床}{shuang1ceng2chuang2}{4;7;7}
  \significado{s.}{beliche}
\end{verbete}

\begin{verbete}{双方同意}{shuang1fang1tong2yi4}{4;4;6;13}
  \significado{s.}{acordo bilateral}
\end{verbete}

\begin{verbete}{霜}{shuang1}{17}
  \significado{s.}{geada; pó branco ou creme espalhado por uma superfície; glacê; creme de pele}
\end{verbete}

\begin{verbete}{谁}{shui2}{10}
  \significado{interr.}{quem?}
  \veja{谁}{shei2}
\end{verbete}

\begin{verbete}{水}{shui3}{4}[85]
  \significado*{s.}{sobrenome Shui}
  \significado{p.c.}{para número de lavagens}
  \significado{s.}{água; líquido; encargos ou receitas adicionais}
\end{verbete}

\begin{verbete}{水边}{shui3bian1}{4;5}
  \significado{s.}{beira d'água; beira-mar; costa (de mar, lago ou rio)}
\end{verbete}

\begin{verbete}{水波}{shui3bo1}{4;8}
  \significado{s.}{ondulação (na água); onda}
\end{verbete}

\begin{verbete}{水果}{shui3guo3}{4;8}
  \significado[个]{s.}{fruta}
\end{verbete}

\begin{verbete}{水饺}{shui3jiao3}{4;9}
  \significado{s.}{\emph{dumplings}; pastéis chineses cozidos}
\end{verbete}

\begin{verbete}{水灵}{shui3ling2}{4;7}
  \significado{adj.}{cheio de vida (sobre uma pessoa, etc.); úmido e brilhante (sobre os olhos); fresco (sobre frutas; etc.); brilhante; aparência saudável}
\end{verbete}

\begin{verbete}{水培}{shui3pei2}{4;11}
  \significado{v.}{cultivar plantas hidroponicamente}
\end{verbete}

\begin{verbete}{水平}{shui3ping2}{4;5}
  \significado{s.}{nível (de realização, etc.); padrão; nível horizontal}
\end{verbete}

\begin{verbete}{水平尺}{shui3ping2chi3}{4;5;4}
  \significado{s.}{nível espiritual}
\end{verbete}

\begin{verbete}{水平度}{shui3ping2 du4}{4;5;9}
  \significado{s.}{nivelamento}
\end{verbete}

\begin{verbete}{水平面}{shui3ping2mian4}{4;5;9}
  \significado{s.}{plano horizontal; nível-da-água; superfície horizontal}
\end{verbete}

\begin{verbete}{水平视差}{shui3ping2 shi4cha1}{4;5;8;9}
  \significado{s.}{paralaxe horizontal}
\end{verbete}

\begin{verbete}{水平仪}{shui3ping2yi2}{4;5;5}
  \significado{s.}{nível (dispositivo para determinar horizontal); nível espiritual; nível de topógrafo}
\end{verbete}

\begin{verbete}{水平以下}{shui3ping2 yi3xia4}{4;5;4;3}
  \significado{s.}{sub-nível}
\end{verbete}

\begin{verbete}{水平轴}{shui3ping2zhou2}{4;5;9}
  \significado{s.}{eixo horizontal}
\end{verbete}

\begin{verbete}{水瓶}{shui3 ping2}{4;10}
  \significado{s.}{garrada de água}
\end{verbete}

\begin{verbete}{水豚}{shui3tun2}{4;11}
  \significado{s.}{capivara}
\end{verbete}

\begin{verbete}{水污染}{shui3wu1ran3}{4;6;9}
  \significado{s.}{poluição da água}
\end{verbete}

\begin{verbete}{说}{shui4}{9}
  \significado{v.}{persuadir}
  \veja{说}{shuo1}
\end{verbete}

\begin{verbete}{税}{shui4}{12}
  \significado{s.}{taxas; impostos}
\end{verbete}

\begin{verbete}{睡觉}{shui4jiao4}{13;9}
  \significado{v.}{ir para a cama; dormir; deitar-se}
\end{verbete}

\begin{verbete}{睡懒觉}{shui4lan3jiao4}{13;16;9}
  \significado{v.}{levantar-se tarde; passar o tempo a dormir}
\end{verbete}

\begin{verbete}{睡衣}{shui4yi1}{13;6}
  \significado{s.}{pijamas; roupas de dormir}
\end{verbete}

\begin{verbete}{顺}{shun4}{9}
  \significado{adj.}{correr bem; favorável}
\end{verbete}

\begin{verbete}{顺从}{shun4cong2}{9;4}
  \significado{v.}{obedecer; submeter-se}
\end{verbete}

\begin{verbete}{顺当}{shun4dang5}{9;6}
  \significado{adv.}{suavemente}
\end{verbete}

\begin{verbete}{顺耳}{shun4'er3}{9;6}
  \significado{adj.}{agradável ao ouvido}
\end{verbete}

\begin{verbete}{顺利}{shun4li4}{9;7}
  \significado{adv.}{suavemente; sem problemas}
\end{verbete}

\begin{verbete}{顺水}{shun4shui3}{9;4}
  \significado{v.}{ir com o fluxo}
\end{verbete}

\begin{verbete}{顺心}{shun4xin1}{9;4}
  \significado{adj.}{satisfatório; satisfeito}
\end{verbete}

\begin{verbete}{顺叙}{shun4xu4}{9;9}
  \significado{s.}{narrativa cronológica}
\end{verbete}

\begin{verbete}{顺延}{shun4yan2}{9;6}
  \significado{v.}{adiar; procrastinar}
\end{verbete}

\begin{verbete}{顺眼}{shun4yan3}{9;11}
  \significado{adj.}{agradável aos olhos}
\end{verbete}

\begin{verbete}{顺嘴}{shun4zui3}{9;16}
  \significado{v.}{deixar escapar (sem pensar); ler suavemente (texto); adequar-se  ao gosto (comida)}
\end{verbete}

\begin{verbete}{说}{shuo1}{9}
  \significado{s.}{uma teoria (normalmente o último caractere, como em 日心说, teoria heliocêntrica)}
  \significado{v.}{falar; dizer; explicar; contar}
  \veja{说}{shui4}
\end{verbete}

\begin{verbete}{说理}{shuo1li3}{9;11}
  \significado{v.}{racionalizar; discutir logicamente}
\end{verbete}

\begin{verbete}{说完}{shuo1-wan2}{9;7}
  \significado{expr.}{acabar/terminar palavras}
\end{verbete}

\begin{verbete}{司机}{si1ji1}{5;6}
  \significado{s.}{condutor; motorista}
\end{verbete}

\begin{verbete}{私人}{si1ren2}{7;2}
  \significado{adj.}{privado; interpessoal}
  \significado[些]{s.}{alguém com quem se tem um relacionamento pessoal próximo}
\end{verbete}

\begin{verbete}{私人信件}{si1ren2 xin4jian4}{7;2;9;6}
  \significado{s.}{carta pessoal}
\end{verbete}

\begin{verbete}{私人钥匙}{si1ren2yao4shi5}{7;2;9;11}
  \significado{s.}{criptografia:~chave privada}
\end{verbete}

\begin{verbete}{私人诊所}{si1ren2 zhen3suo3}{7;2;7;8}
  \significado[些]{s.}{clínica privada}
\end{verbete}

\begin{verbete}{私生活}{si1sheng1huo2}{7;5;9}
  \significado{s.}{vida privada}
\end{verbete}

\begin{verbete}{思想}{si1xiang3}{9;13}
  \significado[个]{s.}{pensamento; ideia; ideologia}
\end{verbete}

\begin{verbete}{死}{si3}{6}
  \significado{adj.}{maldito; intransitável; inflexível; rígido; intransponível}
  \significado{adv.}{extremamente}
  \significado{v.}{morrer; falecer}
\end{verbete}

\begin{verbete}{死亡}{si3wang2}{6;3}
  \significado{s.}{morte}
  \significado{v.}{morrer}
\end{verbete}

\begin{verbete}{四}{si4}{5}
  \significado{num.}{quatro, 4}
\end{verbete}

\begin{verbete}{四川}{si4chuan1}{5;3}
  \significado*{s.}{Sichuan}
\end{verbete}

\begin{verbete}{四季分明}{si4ji4-fen1ming2}{5;8;4;8}
  \significado{expr.}{as quatro estações são muito distintas}
\end{verbete}

\begin{verbete}{四季如春}{si4ji4-ru2chun1}{5;8;6;9}
  \significado{expr.}{é primavera todo o ano; clima favorável durante todo o ano; quatro estações como a primavera}
\end{verbete}

\begin{verbete}{似曾相识}{si4ceng2xiang1shi2}{6;12;9;7}
  \significado{s.}{\emph{déjà vu} (a experiência de ver exatamente a mesma situação pela segunda vez); situação aparentemente familiar}
\end{verbete}

\begin{verbete}{寺}{si4}{6}
  \significado{s.}{Templo Budista; Mesquita}
\end{verbete}

\begin{verbete}{寺庙}{si4miao4}{6;8}
  \significado{s.}{templo; mosteiro; santuário}
\end{verbete}

\begin{verbete}{松木}{song1mu4}{8;4}
  \significado{s.}{pinheiro}
\end{verbete}

\begin{verbete}{送}{song4}{9}
  \significado{v.}{distribuir; entregar; dar; oferecer (alguma coisa como presente); enviar; remeter}
\end{verbete}

\begin{verbete}{㮸}{song4}{14}
  \variante{送}
\end{verbete}

\begin{verbete}{苏格兰}{su1ge2lan2}{7;10;5}
  \significado*{s.}{Escócia}
\end{verbete}

\begin{verbete}{宿舍}{su4she4}{11;8}
  \significado[间]{s.}{dormitório; quarto de dormir; hostel}
\end{verbete}

\begin{verbete}{痠}{suan1}{12}
  \significado{v.}{doer; estar dolorido}
  \variante{酸}
\end{verbete}

\begin{verbete}{酸}{suan1}{14}
  \significado{adj.}{ácido; avinagrado}
\end{verbete}

\begin{verbete}{酸辣汤}{suan1la4tang1}{14;14;6}
  \significado{s.}{sopa avinagrada e picante (prato)}
\end{verbete}

\begin{verbete}{算了}{suan4le5}{14;2}
  \significado{v.}{deixar; deixe estar; deixe passar; esqueça isso}
\end{verbete}

\begin{verbete}{算命}{suan4ming4}{14;8}
  \significado{s.}{cartomante}
  \significado{v.}{ler a sorte; fazer advinhações}
\end{verbete}

\begin{verbete}{虽}{sui1}{9}
  \significado{conj.}{no entanto; embora; mesmo se/embora}
\end{verbete}

\begin{verbete}{虽然}{sui1ran2}{9;12}
  \significado{conj.}{embora (frequentemente usado correlativamente com 可是, 但是, etc)}
  \veja{但是}{dan4shi4}
  \veja{可是}{ke3shi4}
\end{verbete}

\begin{verbete}{随便}{sui2bian4}{11;9}
  \significado{adj.}{à vontade; como queira; como desejar; casual; negligente; devasso}
  \significado{adv.}{aleatoriamente}
\end{verbete}

\begin{verbete}{随处}{sui2chu4}{11;5}
  \significado{adv.}{em qualquer lugar}
\end{verbete}

\begin{verbete}{随地}{sui2di4}{11;6}
  \significado{adv.}{qualquer lugar; todo lugar}
\end{verbete}

\begin{verbete}{随机存取存储器}{sui2ji1cun2qu3cun2chu3qi4}{11;6;6;8;6;12;16}
  \significado{s.}{RAM (\emph{random access memory})}
  \veja{内存}{nei4cun2}
  \veja{随机存取记忆体}{sui2ji1cun2qu3ji4yi4ti3}
\end{verbete}

\begin{verbete}{随机存取记忆体}{sui2ji1cun2qu3ji4yi4ti3}{11;6;6;8;5;4;7}
  \significado{s.}{RAM (\emph{random access memory})}
  \veja{内存}{nei4cun2}
  \veja{随机存取存储器}{sui2ji1cun2qu3cun2chu3qi4}
\end{verbete}

\begin{verbete}{随时}{sui2shi2}{11;7}
  \significado{adv.}{a qualquer momento; sempre que necessário}
\end{verbete}

\begin{verbete}{岁}{sui4}{6}
  \significado{p.c.}{para anos (de idade)}
  \significado{s.}{idade; ano (idade ou colheita)}
\end{verbete}

\begin{verbete}{碎}{sui4}{13}
  \significado{adj.}{quebrato, fragmentado, espalhado; tagarela}
  \significado{v.}{(transitivo ou intransitivo) quebrar em pedaços, quebrar, desmoronar}
\end{verbete}

\begin{verbete}{隧道}{sui4dao4}{14;12}
  \significado{s.}{túnel}
\end{verbete}

\begin{verbete}{孙女}{sun1nv3}{6;3}
  \significado{s.}{filha do filho}
\end{verbete}

\begin{verbete}{孙武}{sun1wu3}{6;8}
  \significado*{s.}{Sun Wu, também conhecido por Sun Tzu (孙子), general, estrategista e filósofo autor do ``Arte da Guerra'' (孙子兵法)}
  \veja{孙子}{sun1zi3}
  \veja{孙子兵法}{sun1zi3 bing1fa3}
\end{verbete}

\begin{verbete}{孙子}{sun1zi3}{6;3}
  \significado*{s.}{Sun Tzu, também conhecido por Sun Wu (孙武), general, estrategista e filósofo autor do ``Arte da Guerra'' (孙子兵法)}
  \veja{孙武}{sun1wu3}
  \veja{孙子兵法}{sun1zi3 bing1fa3}
\end{verbete}

\begin{verbete}{孙子兵法}{sun1zi3 bing1fa3}{6;3;7;8}
  \significado*{s.}{``Arte da Guerra'', escrito por Sun Tzu (孫子)}
  \veja{孙武}{sun1wu3}
  \veja{孙子}{sun1zi3}
\end{verbete}

\begin{verbete}{孙子}{sun1zi5}{6;3}
  \significado{s.}{filho do filho}
\end{verbete}

\begin{verbete}{笋}{sun3}{10}
  \significado{s.}{broto de bambu}
\end{verbete}

\begin{verbete}{缩影卡片}{suo1ying3 ka3pian4}{14;15;5;4}
  \significado{s.}{cartão em miniatura}
\end{verbete}

\begin{verbete}{所以}{suo3yi3}{8;4}
  \significado{adv.}{portanto; então; como resultado}
  \significado{conj.}{por isso; como resultado; a razão porque}
\end{verbete}

\begin{verbete}{索性}{suo3xing4}{10;8}
  \significado{adv.}{poderia muito bem; simplesmente; apenas}
\end{verbete}

%%%%% EOF %%%%%

%%%
%%% T
%%%
\section*{T}
\addcontentsline{toc}{section}{T}

\begin{verbete}[0;9]{T-恤}{t5-xu4}
  \significado{s.}{camiseta; pulôver; suéter}
\end{verbete}

\begin{verbete}[5]{他}{ta1}
  \significado{pron.}{ele; se, o, lhe; si, consigo, ele}
\end{verbete}

\begin{verbete}[5;8]{他的}{ta1·de5}
  \significado{pron.}{dele}
\end{verbete}

\begin{verbete}[5;6;8]{他妈的}{ta1ma1de5}
  \significado{expr.}{Dane-se!; Foda-se!}
\end{verbete}

\begin{verbete}[5;5]{他们}{ta1men5}
  \significado{pron.}{eles; se, os, lhes; si, consigo, eles}
\end{verbete}

\begin{verbete}[5;5;8]{他们的}{ta1men5·de5}
  \significado{pron.}{deles}
\end{verbete}

\begin{verbete}[5]{它}{ta1}
  \significado{pron.}{ele (para objetos inanimados); se, o, lhe; si, consigo, eles}
\end{verbete}

\begin{verbete}[5;5]{它们}{ta1men5}
  \significado{pron.}{eles (para objetos inanimados); se, os, lhes; si, consigo, eles}
\end{verbete}

\begin{verbete}[6]{她}{ta1}
  \significado{pron.}{ela; se, a, lhe; si, consigo, ela}
\end{verbete}

\begin{verbete}[6;8]{她的}{ta1·de5}
  \significado{pron.}{dela}
\end{verbete}

\begin{verbete}[6;5]{她们}{ta1men5}
  \significado{pron.}{elas; se, as, lhes; si, consigo, elas}
\end{verbete}

\begin{verbete}[6;5;8]{她们的}{ta1men5·de5}
  \significado{pron.}{delas}
\end{verbete}

\begin{verbete}[5]{台}{tai2}
  \significado{p.c.}{para aparelhos e máquinas}
  \significado{s.}{Estação de transmissão; contador; \textit{help desk}; suporte técnico; plataforma; terraço; tufão}
\end{verbete}

\begin{verbete}[4]{太}{tai4}
  \significado{adv.}{excessivamente; demais; muito}
\end{verbete}

\begin{verbete*}[4;7;10]{太极拳}{tai4ji2quan2}
  \significado{s.}{Tai Chi Chuan; Taiji; T'aichi ou T'aichichuan; forma tradicional de exercício físico ou relaxamento}
\end{verbete*}

\begin{verbete}[4;4]{太太}{tai4tai5}
  \significado[个,位]{s.}{esposa; madame; mulher casada}
\end{verbete}

\begin{verbete}[4;6;12]{太阳窗}[\\]{tai4yang2chuang1}
  \significado{s.}{teto solar (de veículos)}
\end{verbete}

\begin{verbete}[4;6;6]{太阳灯}{tai4yang2deng1}
  \significado{s.}{lâmpada solar (com células fotovoltaicas)}
\end{verbete}

\begin{verbete}[4;6;4]{太阳风}{tai4yang2feng1}
  \significado{s.}{vento solar}
\end{verbete}

\begin{verbete}[4;6;16]{太阳镜}{tai4yang2jing4}
  \significado{s.}{óculos de sol}
\end{verbete}

\begin{verbete}[4;6;4]{太阳日}{tai4yang2ri4}
  \significado{s.}{dia solar}
\end{verbete}

\begin{verbete}[4;6;17]{太阳翼}{tai4yang2yi4}
  \significado{s.}{painel solar}
\end{verbete}

\begin{verbete}[4;6;8]{太阳雨}{tai4yang2yu3}
  \significado{s.}{banho de sol}
\end{verbete}

\begin{verbete}[4;6]{太阳}{tai4yang5}
  \significado[个]{s.}{sol}
\end{verbete}

\begin{verbete}[10;8]{谈话}{tan2hua4}
  \significado[次]{s.}{conversa; fala}
  \significado{v.+compl.}{conversar; falar}
\end{verbete}

\begin{verbete}[10;10;10]{谈恋爱}{tan2lian4'ai4}
  \significado{v.}{namorar; apaixonar-se}
\end{verbete}

\begin{verbete}[6]{汤}{tang1}
  \significado{s.}{sopa; caldo; decocção de ervas medicinais; água quente ou fervente; água em que algo foi fervido}
\end{verbete}
\begin{verbete*}[6]{汤}{tang1}
  \significado{s.}{sobrenome Tang}
\end{verbete*}

\begin{verbete*}[10;2;12]{唐人街}{tang2ren2jie1}
  \significado{s.}{Bairro Chinês; \textit{Chinatown}}
  \veja*{中国城}{zhong1guo2cheng2}[sp]
\end{verbete*}

\begin{verbete}[16]{糖}{tang2}
  \significado[颗,块]{s.}{açúcar; doces}
\end{verbete}

\begin{verbete}[16;15;8]{糖醋鱼}{tang2cu4yu2}
  \significado{s.}{peixe guisado em molho agridoce (prato)}
\end{verbete}

\begin{verbete}[5;5;9]{讨生活}{tao3sheng1huo2}
  \significado{v.}{ganhar a vida}
\end{verbete}

\begin{verbete}[10]{套}{tao4}
  \significado{p.c.}{para conjuntos, coleções}
  \significado{s.}{cobertura; fórmula; laço de corda}
  \significado{v.}{cobrir; envolver; intercalar; sobrepor}
\end{verbete}

\begin{verbete}[10;6]{套问}{tao4wen4}
  \significado{s.}{retórica}
  \significado{v.}{descobrir por meio de questionamento indireto diplomático}
\end{verbete}

\begin{verbete}[10;7]{特别}{te4bie2}
  \significado{adv.}{especialmente}
  \significado{adj.}{especial; paricular; incomum}
\end{verbete}

\begin{verbete}[10]{疼}{teng2}
  \significado{adj.}{dolorido; doído}
  \significado{v.}{doer; amar ternamente}
\end{verbete}

\begin{verbete}[15]{踢}{ti1}
  \significado{v.}{chutar; jogar (por exemplo, futebol); dar pontapés em}
\end{verbete}

\begin{verbete}[15;19]{踢爆}{ti1bao4}
  \significado{v.}{expor; revelar}
\end{verbete}

\begin{verbete}[15;17;14]{踢蹋舞}{ti1ta4wu3}
  \significado{s.}{sapateado; passo de dança}
\end{verbete}

\begin{verbete}[12;10]{提高}{ti2gao1}
  \significado{v.}{melhorar; aumentar; elevar}
\end{verbete}

\begin{verbete}[4]{天}{tian1}
  \significado{s.}{dia; céu; paraíso}
\end{verbete}

\begin{verbete}[4;12]{天鹅}{tian1'e2}
  \significado{s.}{cisne}
\end{verbete}

\begin{verbete}[4;4]{天气}{tian1qi4}
  \significado{s.}{clima; tempo}
\end{verbete}

\begin{verbete}[4;8]{天使}{tian1shi3}
  \significado{s.}{anjo}
\end{verbete}

\begin{verbete}[4;4]{天天}{tian1tian1}
  \significado{adv.}{todo dia}
\end{verbete}

\begin{verbete}[4;8]{天择}{tian1ze2}
  \significado{s.}{seleção natural}
\end{verbete}

\begin{verbete}[11]{甜}{tian2}
  \significado{adj.}{doce}
\end{verbete}

\begin{verbete}[11;10]{甜酒}{tian2jiu3}
  \significado{s.}{licor doce}
\end{verbete}

\begin{verbete}[11;11]{甜菊}{tian2ju2}
  \significado{s.}{estévia, arbusto cujas folhas produzem substituto do açúcar}
\end{verbete}

\begin{verbete}[11;9]{甜品}{tian2pin3}
  \significado{s.}{sobremesa}
\end{verbete}

\begin{verbete}[11;9]{甜食}{tian2shi2}
  \significado{s.}{doces; sobremesa}
\end{verbete}

\begin{verbete}[11;14]{甜酸}{tian2suan1}
  \significado{adj.}{agridoce}
\end{verbete}

\begin{verbete}[11;11;11]{甜甜圈}[\\]{tian2tian2quan1}
  \significado{s.}{rosquinha; \textit{doughnut}}
\end{verbete}

\begin{verbete}[11;12]{甜筒}{tian2tong3}
  \significado{s.}{sorvete de casquinha}
\end{verbete}

\begin{verbete}[11;5]{甜头}{tian2tou5}
  \significado{s.}{benefício; sabor doce (de poder, sucesso, etc.)}
\end{verbete}

\begin{verbete}[11;4]{甜心}{tian2xin1}
  \significado{s.}{querido}
\end{verbete}

\begin{verbete}[11;7]{甜言}{tian2yan2}
  \significado{s.}{boa conversa; palavras amáveis}
\end{verbete}

\begin{verbete}[11;5;6]{甜玉米}{tian2·yu4mi3}
  \significado{s.}{milho doce}
\end{verbete}

\begin{verbete}[11;13]{甜稚}{tian2zhi4}
  \significado{s.}{doce e inocente}
\end{verbete}

\begin{verbete}[7]{条}{tiao2}
  \significado{p.c.}{para coisas longas e finas (fita, rio, estrada, calças, etc.)}
  \significado{s.}{artigo; cláusula (de lei ou tratado); item; faixa}
\end{verbete}

\begin{verbete}[7;12]{条幅}{tiao2fu2}
  \significado{s.}{faixa; banner; pergaminho de parede (para pintura ou caligrafia)}
\end{verbete}

\begin{verbete}[7;9]{条贯}{tiao2guan4}
  \significado{s.}{ordem; procedimentos; sequência; sistema}
\end{verbete}

\begin{verbete}[7;6]{条件}{tiao2jian4}
  \significado[个]{s.}{circunstâncias; condição; fator; prerequisito; qualificação; requisito}
\end{verbete}

\begin{verbete}[7;8]{条例}{tiao2li4}
  \significado{s.}{código de conduta; ordenanças; regulamentos; regras; estatutos}
\end{verbete}

\begin{verbete}[7;5]{条目}{tiao2mu4}
  \significado{s.}{cláusulas e subcláusulas (em documento formal); verbete (em um dicionário, enciclopédia, etc.)}
\end{verbete}

\begin{verbete}[13]{跳}{tiao4}
  \significado{v.}{pular; saltar}
\end{verbete}

\begin{verbete}[13;9]{跳挡}{tiao4dang3}
  \significado{v.}{pular marcha (de um carro); perder a marcha}
\end{verbete}

\begin{verbete}[13;5]{跳电}{tiao4dian4}
  \significado{v.}{desarmar (um disjuntor ou interruptor)}
\end{verbete}

\begin{verbete}[13;13]{跳频}{tiao4pin2}
  \significado{s.}{FHSS, \textit{Frequency-Hopping Spread Spectrum}, método de transmissão de sinais de rádio}
\end{verbete}

\begin{verbete}[13;13;16]{跳跳糖}[\\]{tiao4tiao4tang2}
  \significado{s.}{Pop Rocks; \textit{popping candy}}
\end{verbete}

\begin{verbete}[13;14]{跳舞}{tiao4wu3}
  \significado{v.+compl.}{dançar}
\end{verbete}

\begin{verbete}[13;7]{跳远}{tiao4yuan3}
  \significado{v.+compl.}{salto em distância (atletismo)}
\end{verbete}

\begin{verbete}[13;9]{跳蚤}{tiao4zao5}
  \significado{s.}{pulga}
\end{verbete}

\begin{verbete}[7]{听}{ting1}
  \significado{p.c.}{para bebidas enlatadas }
  \significado{s.}{lata de bebida (do inglês ``tin'') }
  \significado{v.}{ouvir; escutar; obedecer}
\end{verbete}

\begin{verbete}[7;11]{听断}{ting1duan4}
  \significado{v.}{ouvir e decidir; julgar (ou seja, ouvir e julgar em um tribunal)}
\end{verbete}

\begin{verbete}[7;9]{听骨}{ting1gu3}
  \significado{v.}{ossículos (do ouvido médio)}
  \veja*{听小骨}{ting1xiao3gu3}
\end{verbete}

\begin{verbete}[7;6]{听会}{ting1hui4}
  \significado{v.}{participar de uma reunião (e ouvir o que é discutido)}
\end{verbete}

\begin{verbete}[7;7]{听来}{ting1lai2}
  \significado{v.}{ouvir de algum lugar; soar (antigo, estrangeiro, excitante, certo, etc.); soar como se (ou seja, dar uma impressão ao ouvinte)}
\end{verbete}

\begin{verbete}[7;2]{听力}{ting1li4}
  \significado{s.}{audição; capacidade de compreensão oral}
\end{verbete}

\begin{verbete}[7;2;11;13]{听力理解}[\\]{ting1li4li3jie3}
  \significado{s.}{compreensão auditiva}
\end{verbete}

\begin{verbete}[7;8]{听命}{ting1ming4}
  \significado{v.}{obedecer ordens; receber ordens}
\end{verbete}

\begin{verbete}[7;8]{听凭}{ting1ping2}
  \significado{v.}{permitir (alguém a fazer o que desejar)}
\end{verbete}

\begin{verbete}[7;9]{听说}{ting1shuo1}
  \significado{v.}{ouvir dizer}
\end{verbete}

\begin{verbete}[7;11]{听随}{ting1sui2}
  \significado{v.}{permitir; obedecer}
\end{verbete}

\begin{verbete}[7;6]{听戏}{ting1xi4}
  \significado{v.}{assistir a uma ópera; ver uma ópera}
\end{verbete}

\begin{verbete}[7;3;9]{听小骨}{ting1xiao3gu3}
  \significado{v.}{ossículos (do ouvido médio)}
  \veja{听骨}{ting1gu3}
\end{verbete}

\begin{verbete}[7;5]{听写}{ting1xie3}
  \significado{v.}{transcrever música de ouvido; escrever (em um exercício de ditado)}
  \significado{s.}{ditado}
\end{verbete}

\begin{verbete}[11]{停}{ting2}
  \significado{v.}{parar; estacionar (um carro)}
\end{verbete}

\begin{verbete}[11;4]{停办}{ting2ban4}
  \significado{v.}{cancelar; sair do negócio; desligar; terminar}
\end{verbete}

\begin{verbete}[11;4]{停车}{ting2che1}
  \significado{v.}{parar de trabalhar (uma máquina); estacionar; parar (um veículo); paralisar}
\end{verbete}

\begin{verbete}[11;4;6]{停车场}[\\]{ting2che1chang3}
  \significado{s.}{parque de estacionamento}
\end{verbete}

\begin{verbete}[11;6]{停当}{ting2dang4}
  \significado{adj.}{realizado; preparado; assentado}
\end{verbete}

\begin{verbete}[11;5]{停电}{ting2dian4}
  \significado{s.}{corte de energia}
  \significado{v.}{ter uma falha de energia}
\end{verbete}

\begin{verbete}[11;3]{停工}{ting2gong1}
  \significado{v.}{parar de trabalhar; parar a produção}
\end{verbete}

\begin{verbete}[11;4]{停火}{ting2huo3}
  \significado{v.}{cessar fogo}
  \significado{s.}{cessar-fogo}
\end{verbete}

\begin{verbete}[11;10]{停课}{ting2ke4}
  \significado{v.}{fechar (escola); parar as aulas}
\end{verbete}

\begin{verbete}[11;10]{停留}{ting2liu2}
  \significado{v.}{ficar em algum lugar temporariamente; demorar; permanecer}
\end{verbete}

\begin{verbete}[11;10]{停息}{ting2xi1}
  \significado{v.}{cessar; parar}
\end{verbete}

\begin{verbete}[11;13]{停歇}{ting2xie1}
  \significado{v.}{parar para descansar}
\end{verbete}

\begin{verbete}[11;5]{停业}{ting2ye4}
  \significado{v.}{cessar a negociação (temporária ou permanentemente); fechar}
\end{verbete}

\begin{verbete}[11;5]{停用}{ting2yong4}
  \significado{v.}{desabilitar; descontinuar; parar de usar; suspender}
\end{verbete}

\begin{verbete}[11;4]{停止}{ting2zhi3}
  \significado{v.}{cessar; encerrar; parar}
\end{verbete}

\begin{verbete}[9]{挺}{ting3}
  \significado{adv.}{bastante; muito}
  \significado{adj.}{ereto; fora do comum; direto}
  \significado{p.c.}{para metralhadoras}
  \significado{v.}{endireitar (fisicamente); sobressair (uma parte do corpo); dar suporte; resistir}
\end{verbete}

\begin{verbete}[9;8]{挺拔}{ting3ba2}
  \significado{adj.}{alto e reto}
\end{verbete}

\begin{verbete}[9;7]{挺杆}{ting3gan3}
  \significado{s.}{tucho (peça de máquina)}
\end{verbete}

\begin{verbete}[9;6]{挺过}{ting3·guo4}
  \significado{v.}{sobreviver}
\end{verbete}

\begin{verbete}[9;6]{挺好}{ting3hao3}
  \significado{adj.}{muito bom}
\end{verbete}

\begin{verbete}[9;7]{挺进}{ting3jin4}
  \significado{s.}{progresso; avanço}
  \significado{v.}{progredir; avançar}
\end{verbete}

\begin{verbete}[9;5]{挺立}{ting3li4}
  \significado{v.}{ficar ereto; ficar de pé}
\end{verbete}

\begin{verbete}[9;7]{挺身}{ting3shen1}
  \significado{v.}{endireitar as costas}
\end{verbete}

\begin{verbete}[9;13]{挺腰}{ting3yao1}
  \significado{v.}{arquear as costas; endireitar as costas}
\end{verbete}

\begin{verbete}[9;3]{挺尸}{ting3zhu4}
  \significado{v.}{coloquial: dormir; literalmente: ficar deitado duro como um cadáver}
\end{verbete}

\begin{verbete}[9;13]{挺住}{ting3zhu4}
  \significado{v.}{permanecer firme; manter-se firme (diante da adversidade ou da dor)}
\end{verbete}

\begin{verbete}[10]{通}{tong1}
  \significado{p.c.}{para cartas, telegramas, telefonemas, etc.}
  \significado{s.}{suffixo: especialista}
  \significado{v.}{ligar para; conseguir a ligação}
\end{verbete}

\begin{verbete}[10;13]{通牒}{tong1die2}
  \significado{s.}{nota diplomática}
\end{verbete}

\begin{verbete}[10;6]{通观}{tong1guan1}
  \significado{v.}{ter uma visão geral de algo}
\end{verbete}

\begin{verbete}[10;7]{通识}{tong1shi2}
  \significado{s.}{conhecimento comum; erudição; conhecimento geral; amplamente conhecido}
\end{verbete}

\begin{verbete}[6]{同}{tong2}
  \significado{adj.}{junto}
  \significado{adv.}{junto com}
\end{verbete}

\begin{verbete}[6;6]{同伙}{tong2huo3}
  \significado[个]{s.}{cúmplice; colega}
\end{verbete}

\begin{verbete}[6;8]{同事}{tong2shi4}
  \significado{s.}{colega; colega de trabalho; companheiro}
\end{verbete}

\begin{verbete}[6;9]{同屋}{tong2wu1}
  \significado[个]{s.}{companheiro de quarto; colega de quarto}
\end{verbete}

\begin{verbete}[6;8]{同学}{tong2xue2}
  \significado[位,个]{s.}{colega de classe; colega estudante}
\end{verbete}

\begin{verbete}[6;13]{同意}{tong2yi4}
  \significado{s.}{concordar; aprovar; consentir}
\end{verbete}

\begin{verbete}[6;9]{同砚}{tong2yuan4}
  \significado[位,个]{s.}{colega de classe; colega estudante}
\end{verbete}

\begin{verbete}[12;9]{痛骂}{tong4ma4}
  \significado{v.}{repreender severamente}
\end{verbete}

\begin{verbete}[11]{偷}{tou1}
  \significado{adv.}{furtivamente}
  \significado{v.}{furtar; roubar}
\end{verbete}

\begin{verbete}[11;6]{偷安}{tou1'an1}
  \significado{v.}{buscar facilidade temporária}
\end{verbete}

\begin{verbete}[11;12]{偷渡}{tou1du4}
  \significado{s.}{contrabando; imigração ilegal; clandestino (em um navio)}
  \significado{v.}{executar um bloqueio; roubar através da fronteira internacional}
\end{verbete}

\begin{verbete}[11;11]{偷窃}{tou1qie4}
  \significado{v.}{furtar; roubar}
\end{verbete}

\begin{verbete}[11;11]{偷情}{tou1qing2}
  \significado{v.}{manter um caso de amor clandestino}
\end{verbete}

\begin{verbete}[11;12]{偷税}{tou1shui4}
  \significado{s.}{evasão fiscal}
\end{verbete}

\begin{verbete}[11;7]{偷听}{tou1ting1}
  \significado{v.}{bisbilhotar; monitorar (secretamente)}
\end{verbete}

\begin{verbete}[11;11]{偷袭}{tou1xi2}
  \significado{s.}{ataque surpresa}
  \significado{v.}{montar um ataque furtivo; invadir}
\end{verbete}

\begin{verbete}[5]{头}{tou2}
  \significado[个]{s.}{cabeça}
  \significado{p.c.}{para suínos ou gado}
\end{verbete}

\begin{verbete}[5;5]{头发}{tou2fa5}
  \significado{s.}{cabelo}
\end{verbete}

\begin{verbete}[5;5]{头号}{tou2hao4}
  \significado{adj.}{primeira classe; número um; \textit{top rank}}
\end{verbete}

\begin{verbete}[5;5]{头头}{tou2tou2}
  \significado{s.}{chefe; o cabeça}
\end{verbete}

\begin{verbete}[7;10]{投资}{tou2zi1}
  \significado{s.}{investimento}
  \significado{v.}{investir}
\end{verbete}

\begin{verbete}[7;10;4;9]{投资风险}[\\]{tou2zi1feng1xian3}
  \significado{s.}{risco de investimento}
\end{verbete}

\begin{verbete}[7;10;6;7;11]{投资回报率}[\\]{tou2zi1hui2bao4lv4}
  \significado{s.}{retorno sobre o investimento (ROI)}
\end{verbete}

\begin{verbete}[7;10;10]{投资家}{tou2zi1jia1}
  \significado{s.}{investidor}
  \veja{投资人}{tou2zi1ren2}
  \veja{投资者}{tou2zi1zhe3}
\end{verbete}

\begin{verbete}[7;10;2]{投资人}{tou2zi1ren2}
  \significado{s.}{investidor}
  \veja{投资家}{tou2zi1jia1}
  \veja{投资者}{tou2zi1zhe3}
\end{verbete}

\begin{verbete}[7;10;8]{投资者}{tou2zi1zhe3}
  \significado{s.}{investidor}
  \veja{投资家}{tou2zi1jia1}
  \veja{投资人}{tou2zi1ren2}
\end{verbete}

\begin{verbete}[10]{透}{tou4}
  \significado{adj.}{completo; total}
  \significado{adv.}{completamente; totalmente}
  \significado{v.}{aparecer; passar através; penetrar}
\end{verbete}

\begin{verbete}[10;4]{透水}{tou4che4}
  \significado{adj.}{permeável}
  \significado{s.}{vazamento de água}
\end{verbete}

\begin{verbete}[10;7]{透彻}{tou4che4}
  \significado{adj.}{minucioso; incisivo; penetrante}
\end{verbete}

\begin{verbete}[10;8]{透顶}{tou4ding3}
  \significado{adv.}{completamente}
\end{verbete}

\begin{verbete}[10;6]{透过}{tou4guo4}
  \significado{v.}{passar através; penetrar}
\end{verbete}

\begin{verbete}[10;9]{透亮}{tou4liang4}
  \significado{adj.}{brilhante; claro como cristal}
\end{verbete}

\begin{verbete}[10;21]{透露}{tou4lu4}
  \significado{v.}{divulgar; vazar; revelar}
\end{verbete}

\begin{verbete}[10;8]{透明}{tou4ming2}
  \significado{adj.}{aberto (não-secreto); transparente}
\end{verbete}

\begin{verbete}[10;13]{透辟}{tou4pi4}
  \significado{adj.}{incisivo; penetrante}
\end{verbete}

\begin{verbete}[10;4]{透气}{tou4qi4}
  \significado{v.}{respirar (sobre tecido, etc.); fluir livremente (sobre ar); respirar ar fresco; ventilar}
\end{verbete}

\begin{verbete}[10;4]{透支}{tou4zhi1}
  \significado{v.}{cheque especial (bancário); saque a descoberto}
\end{verbete}

\begin{verbete}[8;4;11]{图书馆}{tu2shu1guan3}
  \significado[家,个]{s.}{biblioteca}
\end{verbete}

\begin{verbete}[3;7]{土豆}{tu3dou4}
  \significado[个,颗]{s.}{batata}
\end{verbete}

\begin{verbete}[3;7;8]{土豆泥}{tu3dou4ni2}
  \significado{s.}{purê de batatas}
\end{verbete}

\begin{verbete}[11;7]{推迟}{tui1chi2}
  \significado{v.}{adiar; deixar para mais tarde; tardar}
\end{verbete}

\begin{verbete}[13]{腿}{tui3}
  \significado[条]{s.}{perna; osso do quadril}
\end{verbete}

\begin{verbete}[13;5]{腿号}{tui3hao4}
  \significado{s.}{anilha numerada (por exemplo, usada para identificar pássaros}
\end{verbete}

\begin{verbete}[11;9]{唾骂}{tuo4ma4}
  \significado{v.}{insultar; amaldiçoar}
\end{verbete}


%%%%% EOF %%%%%

%%%%%%%%%%%%%%%%%%%% Não existem palavras com pinyin iniciado em "U"
%%%%%%%%%%%%%%%%%%%% Não existem palavras com pinyin iniciado em "V"
%%%
%%% W
%%%
\section*{W}
\addcontentsline{toc}{section}{W}

\begin{verbete}{外}{wai4}{5}
  \significado{p.l.}{fora; por fora; exterior; estrangeiro}
\end{verbete}
\begin{verbete}{外边}{wai4bian5}{5;5}
  \significado{p.l.}{fora do país; superfície externa; fora; lugar diferente de sua casa}
\end{verbete}
\begin{verbete}{外插}{wai4cha1}{5;12}
  \significado{s.}{extrapolar; computação: conectar (um dispositivo periférico, etc.)}
\end{verbete}
\begin{verbete}{外公}{wai4gong1}{5;4}
  \significado{s.}{avô materno}
\end{verbete}
\begin{verbete}{外国}{wai4guo2}{5;8}
  \significado[个]{s.}{país estrangeiro}
\end{verbete}
\begin{verbete}{外国人}{wai4guo2ren2}{5;8;2}
  \significado{s.}{estrangeiro; nascido fora do país}
\end{verbete}
\begin{verbete}{外海}{wai4hai3}{5;10}
  \significado{s.}{mar aberto}
\end{verbete}
\begin{verbete}{外号}{wai4hao4}{5;5}
  \significado{s.}{apelido}
\end{verbete}
\begin{verbete}{外积}{wai4ji1}{5;10}
  \significado{s.}{produto exterior; matemática: o produto vetorial de dois vetores}
\end{verbete}
\begin{verbete}{外交}{wai4jiao1}{5;6}
  \significado{adj.}{diplomático}
  \significado[个]{s.}{diplomacia; relações exteriores}
\end{verbete}
\begin{verbete}{外贸}{wai4mao4}{5;9}
  \significado{s.}{comércio exterior}
\end{verbete}
\begin{verbete}{外貌协会}{wai4mao4xie2hui4}{5;14;6;6}
  \significado{s.}{``o clube da boa aparência'': pessoas que dão grande importância à aparência de uma pessoa}
  \veja{外协}{wai4xie2}
\end{verbete}
\begin{verbete}{外面}{wai4mian4}{5;9}
  \significado{p.l.}{fora; por fora; exterior; superfície}
\end{verbete}
\begin{verbete}{外婆}{wai4po2}{5;11}
  \significado{s.}{avó materna}
\end{verbete}
\begin{verbete}{外事}{wai4shi4}{5;8}
  \significado{s.}{assuntos ou relações exteriores}
\end{verbete}
\begin{verbete}{外水}{wai4shui3}{5;4}
  \significado{s.}{renda extra}
\end{verbete}
\begin{verbete}{外孙}{wai4sun1}{5;6}
  \significado{s.}{filho da filha}
\end{verbete}
\begin{verbete}{外孙女}{wai4sun1nv3}{5;6;3}
  \significado{s.}{filha da filha}
\end{verbete}
\begin{verbete}{外围}{wai4wei2}{5;7}
  \significado{p.l.}{arredores}
\end{verbete}
\begin{verbete}{外协}{wai4xie2}{5;6}
  \significado{s.}{terceirização; pessoas que julgam os outros pela aparência}
  \veja*{外貌协会}{wai4mao4xie2hui4}
\end{verbete}
\begin{verbete}{外衣}{wai4yi1}{5;6}
  \significado{s.}{aparência; roupa de cima}
\end{verbete}
\begin{verbete}{外语}{wai4yu3}{5;9}
  \significado[门]{s.}{língua estrangeira}
\end{verbete}
\begin{verbete}{歪果仁}{wai4guo2ren2}{9;8;4}
  \significado{s.}{gíria na Internet para 外国人}
  \veja{外国人}{wai4guo2ren2}
\end{verbete}
\begin{verbete}{豌豆}{wan1dou4}{15;7}
  \significado{s.}{ervilha}
\end{verbete}
\begin{verbete}{王五}{wan2wu3}{4;4}
  \significado*{s.}{Wang Wu; Zé Ninguém; nome para uma pessoa não especificada, 3 de 3}
  \veja{李四}{li3si4}
  \veja{张三}{zhang1san1}
\end{verbete}
\begin{verbete}{完}{wan2}{7}
  \significado{v.}{acabar; completar; terminar}
\end{verbete}
\begin{verbete}{完备}{wan2bei4}{7;8}
  \significado{adj.}{completo; impecável; perfeito}
  \significado{v.}{não deixar nada a desejar}
\end{verbete}
\begin{verbete}{完毕}{wan2bi4}{7;6}
  \significado{v.}{completar; terminar; acabar}
\end{verbete}
\begin{verbete}{完成}{wan2cheng2}{7;6}
  \significado{v.}{realizar; completar}
\end{verbete}
\begin{verbete}{完满}{wan2man3}{7;13}
  \significado{adj.}{satisfatório; bem-sucedido}
\end{verbete}
\begin{verbete}{完美}{wan2mei3}{7;9}
  \significado{adj.}{perfeito}
  \significado{adv.}{perfeitamente}
  \significado{s.}{perfeição}
\end{verbete}
\begin{verbete}{完全}{wan2quan2}{7;6}
  \significado{adj.}{completo; todo}
  \significado{adv.}{inteiramente; totalmente}
\end{verbete}
\begin{verbete}{完人}{wan2ren2}{7;2}
  \significado{s.}{pessoa perfeita}
\end{verbete}
\begin{verbete}{完税}{wan2shui4}{7;12}
  \significado{v.}{pagar imposto}
\end{verbete}
\begin{verbete}{完完全全}{wan2wan2quan2quan2}{7;7;6;6}
  \significado{adv.}{completamente}
\end{verbete}
\begin{verbete}{玩}{wan2}{8}
  \significado{s.}{brinquedo; algo usado para diversão}
  \significado{v.}{divertir-se; manter algo para entretenimento; brincar com }
\end{verbete}
\begin{verbete}{玩伴}{wan2ban4}{8;7}
  \significado{s.}{parceiro de brincadeira}
\end{verbete}
\begin{verbete}{玩遍}{wan2bian4}{8;12}
  \significado{v.}{passear (todo o país, toda a cidade, etc.); visitar (um grande número de lugares)}
\end{verbete}
\begin{verbete}{玩家}{wan2jia1}{8;10}
  \significado{s.}{entusiasta (áudio, modelos de aviões, etc.); jogador (de um jogo)}
\end{verbete}
\begin{verbete}{玩儿}{wan2r5}{8;2}
  \significado{v.}{divertir-se}
\end{verbete}
\begin{verbete}{玩耍}{wan2shua3}{8;8}
  \significado{v.}{divertir-me; brincar (como as crianças fazem)}
\end{verbete}
\begin{verbete}{玩味}{wan2wei4}{8;8}
  \significado{v.}{ponderar sutilezas; ruminar (pensamentos)}
\end{verbete}
\begin{verbete}{玩艺}{wan2yi4}{8;4}
  \variante{玩意}{wan2yi4}
\end{verbete}
\begin{verbete}{玩味}{wan2yi4}{8;13}
  \significado{s.}{ato; brinquedo; coisa; truque (em uma performance, show de palco, acrobacias, etc.)}
\end{verbete}
\begin{verbete}{玩意}{wan2yi4}{8;13}
  \significado{s.}{ato; brinquedo; coisa; truque (em uma performance, show de palco, acrobacias, etc.)}
\end{verbete}
\begin{verbete}{玩者}{wan2zhe3}{8;8}
  \significado{s.}{jogador}
\end{verbete}
\begin{verbete}{埦}{wan3}{11}
  \variante{碗}{wan3}
\end{verbete}
\begin{verbete}{晚}{wan3}{11}
  \significado{adj.}{tarde; noite}
\end{verbete}
\begin{verbete}{晚报}{wan3bao4}{11;7}
  \significado{s.}{jornal da noite}
\end{verbete}
\begin{verbete}{晚餐}{wan3can1}{11;16}
  \significado[份,顿,次]{s.}{jantar; refeição noturna}
\end{verbete}
\begin{verbete}{晚点}{wan3dian3}{11;9}
  \significado{adj.}{atrasado}
  \significado{s.}{jantar leve}
\end{verbete}
\begin{verbete}{晚饭}{wan3fan4}{11;7}
  \significado[份,顿,次,餐]{s.}{jantar}
\end{verbete}
\begin{verbete}{晚会}{wan3hui4}{11;6}
  \significado[个]{s.}{festa noturna}
\end{verbete}
\begin{verbete}{晚近}{wan3jin4}{11;7}
  \significado{adv.}{ultimamente; recentemente}
  \significado{adj.}{recente; mais recente no passado}
\end{verbete}
\begin{verbete}{晚景}{wan3jing3}{11;12}
  \significado{s.}{circunstâncias dos anos de declínio de alguém; cena noturna}
\end{verbete}
\begin{verbete}{晚上}{wan3shang5}{11;3}
  \significado{p.t.}{noite; à noite}
\end{verbete}
\begin{verbete}{晚育}{wan3yu4}{11;8}
  \significado{n.}{parto tardio}
  \significado{v.}{ter um filho mais tarde}
\end{verbete}
\begin{verbete}{碗}{wan3}{13}
  \significado[只,个]{n}{tigela}
  \significado{p.c.}{tigelas}
\end{verbete}
\begin{verbete}{碗柜}{wan3gui4}{13;8}
  \significado{n}{armário}
\end{verbete}
\begin{verbete}{碗子}{wan3zi5}{13;3}
  \significado{n}{tigela}
\end{verbete}
\begin{verbete}{万}{wan4}{3}
  \significado{adj.}{um grande número}
  \significado{num.}{10.000, dez mil}
  \significado*{s.}{sobrenome Wan}
\end{verbete}
\begin{verbete}{万万}{wan4wan4}{3;3}
  \significado{adv.}{absolutamente; totalmente}
\end{verbete}
\begin{verbete}{网}{wang3}{6}
  \significado{s.}{rede}
\end{verbete}
\begin{verbete}{网罟}{wang3gu3}{6;10}
  \significado{s.}{figurativo: a rede da justiça; rede usada para capturar peixes (ou outros animais, como pássaros)}
\end{verbete}
\begin{verbete}{网际网路}{wang3ji4wang3lu4}{6;7;6;13}
  \significado*{s.}{Internet}
  \veja*{网际网络}{wang3ji4wang3luo4}
  \veja{网路}{wang3lu4}
\end{verbete}
\begin{verbete}{网际网络}{wang3ji4wang3luo4}{6;7;6;9}
  \significado*{s.}{Internet}
  \veja*{网际网路}{wang3ji4wang3lu4}
  \veja{网路}{wang3lu4}
\end{verbete}
\begin{verbete}{网路}{wang3lu4}{6;13}
  \significado*{s.}{Internet}
  \veja*{网际网路}{wang3ji4wang3lu4}
  \veja*{网际网络}{wang3ji4wang3luo4}
\end{verbete}
\begin{verbete}{网球}{wang3qiu2}{6;11}
  \significado[个]{s.}{tênis (esporte); bola de tênis}
\end{verbete}
\begin{verbete}{网上银行}{wang3shang4yin2hang2}{6;3;11;6}
  \significado[个]{s.}{banco online; acesso a operações bancárias via Internet}
  \veja{网银}{wang3yin2}
\end{verbete}
\begin{verbete}{网银}{wang3yin2}{6;11}
  \significado[个]{s.}{banco online; acesso a operações bancárias via Internet}
  \veja*{网上银行}{wang3shang4yin2hang2}
\end{verbete}
\begin{verbete}{往}{wang3}{8}
  \significado{prep.}{para; em direção a}
\end{verbete}
\begin{verbete}{往程}{wang3cheng2}{8;12}
  \significado{s.}{saída (de uma viagem de ônibus ou trem, etc.)}
\end{verbete}
\begin{verbete}{往返}{wang3fan3}{8;7}
  \significado{s.}{ida e volta}
  \significado{v.}{ir e voltar; ir e vir}
\end{verbete}
\begin{verbete}{往复}{wang3fu4}{8;9}
  \significado{s.}{para trás e para frente (por exemplo, da ação do pistão ou da bomba)}
  \significado{v.}{ir e voltar; fazer uma viagem de volta}
\end{verbete}
\begin{verbete}{往迹}{wang3ji4}{8;9}
  \significado{s.}{eventos passados}
\end{verbete}
\begin{verbete}{往来}{wang3lai2}{8;7}
  \significado{s.}{contatos; negociações}
\end{verbete}
\begin{verbete}{往例}{wang3li4}{8;8}
  \significado{s.}{prática (habitual) do passado; precedente}
\end{verbete}
\begin{verbete}{往日}{wang3ri4}{8;4}
  \significado{p.t.}{dias passados}
  \significado{s.}{o passado}
\end{verbete}
\begin{verbete}{往生}{wang3sheng1}{8;5}
  \significado{v.}{renascer; morrer; Budismo: viver no paraíso}
\end{verbete}
\begin{verbete}{往事}{wang3shi4}{8;8}
  \significado{s.}{acontecimentos anteriores; eventos passados}
\end{verbete}
\begin{verbete}{往往}{wang3wang3}{8;8}
  \significado{adv.}{em muitos casos; mais frequentes do que não; geralmente}
\end{verbete}
\begin{verbete}{往昔}{wang3xi1}{8;8}
  \significado{s.}{o passado}
\end{verbete}
\begin{verbete}{罔}{wang3}{8}
  \significado{v.}{enganar}
\end{verbete}
\begin{verbete}{忘}{wang4}{7}
  \significado{v.}{esquecer; negligenciar; ignorar}
\end{verbete}
\begin{verbete}{忘本}{wang4ben3}{7;5}
  \significado{v.}{esquecer as próprias raízes}
\end{verbete}
\begin{verbete}{忘餐}{wang4can1}{7;16}
  \significado{v.}{esquecer as refeições}
\end{verbete}
\begin{verbete}{忘掉}{wang4diao4}{7;11}
  \significado{v.}{esquecer}
\end{verbete}
\begin{verbete}{忘恩}{wang4'en1}{7;10}
  \significado{v.}{ser ingrato}
\end{verbete}
\begin{verbete}{忘怀}{wang4huai2}{7;7}
  \significado{v.}{esquecer}
\end{verbete}
\begin{verbete}{忘记}{wang4ji4}{7;5}
  \significado{v.}{esquecer}
\end{verbete}
\begin{verbete}{忘却}{wang4que4}{7;7}
  \significado{v.}{esquecer}
\end{verbete}
\begin{verbete}{为}{wei2}{4}
  \significado{prep.}{como (na capacidade de); por (na voz passiva)}
  \significado{v.}{tomar algo como;agir como;servir como;comportar-se como;tornar-se}
  \veja{为}{wei4}
\end{verbete}
\begin{verbete}{喂}{wei2}{12}
  \significado{interj.}{Alô!; Olá! (quando respondendo a um telefonema)}
  \veja{喂}{wei4}
\end{verbete}
\begin{verbete}{尾巴}{wei3ba5}{7;4}
  \significado{s.}{cauda}
\end{verbete}
\begin{verbete}{卫生}{wei4sheng1}{3;5}
  \significado{s.}{saúde; higiene; saneamento}
\end{verbete}
\begin{verbete}{卫生部}{wei4sheng1bu4}{3;5;10}
  \significado*{s.}{Ministério da Saúde}
\end{verbete}
\begin{verbete}{卫生防疫}{wei4sheng1·fang2yi4}{3;5;6;9}
  \significado{s.}{prevenção contra a epidemia}
\end{verbete}
\begin{verbete}{卫生间}{wei4sheng1jian1}{3;5;7}
  \significado[间]{s.}{banheiro; \textit{toilette}}
\end{verbete}
\begin{verbete}{卫生巾}{wei4sheng1jin1}{3;5;3}
  \significado{s.}{absorvente higiênico}
\end{verbete}
\begin{verbete}{卫生局}{wei4sheng1ju2}{3;5;7}
  \significado*{s.}{Departamento de Saúde; Escritório de Saúde}
\end{verbete}
\begin{verbete}{卫生棉}{wei4sheng1mian2}{3;5;12}
  \significado{s.}{absorvente; algodão absorvente esterilizado (usado para curativos ou limpeza de feridas); absorvente tampão}
\end{verbete}
\begin{verbete}{卫生球}{wei4sheng1qiu2}{3;5;11}
  \significado{s.}{naftalina}
\end{verbete}
\begin{verbete}{卫生署}{wei4sheng1shu3}{3;5;13}
  \significado*{s.}{Agência de Saúde (ou Escritório, ou Departamento)}
\end{verbete}
\begin{verbete}{卫生套}{wei4sheng1tao4}{3;5;10}
  \significado[只]{s.}{camisinha; preservativo}
\end{verbete}
\begin{verbete}{卫生厅}{wei4sheng1ting1}{3;5;4}
  \significado{s.}{Departamento de Saúde (da província)}
\end{verbete}
\begin{verbete}{卫生纸}{wei4sheng1zhi3}{3;5;7}
  \significado{s.}{papel higiênico}
\end{verbete}
\begin{verbete}{为}{wei4}{4}
  \significado{prep.}{para; porque}
  \veja{为}{wei2}
\end{verbete}
\begin{verbete}{为什么}{wei4shen2me5}{4;4;3}
  \significado{interr.}{por que?}
\end{verbete}
\begin{verbete}{位}{wei4}{7}
  \significado{p.c.}{para pessoas (com cortesia); classificador para bits binários (por exemplo, 十六位 16-bits ou 2 bytes)}
  \significado{s.}{física: potencial; localização; lugar; posição; assento}
\end{verbete}
\begin{verbete}{位居}{wei4ju1}{7;8}
  \significado{v.}{estar localizado em}
\end{verbete}
\begin{verbete}{位置}{wei4zhi5}{7;13}
  \significado[个]{s.}{lugar; posição; assento}
\end{verbete}
\begin{verbete}{味}{wei4}{8}
  \significado{p.c.}{para medicamentos}
  \significado{s.}{cheiro; gosto}
\end{verbete}
\begin{verbete}{味道}{wei4dao5}{8;12}
  \significado{s.}{sabor; odor}
\end{verbete}
\begin{verbete}{味儿}{wei4r5}{8;2}
  \significado{s.}{sabor}
\end{verbete}
\begin{verbete}{喂}{wei4}{12}
  \significado{interj.}{Ei!; chamar atenção}
  \significado{v.}{alimentar; alimentar (um animal, bebê, inválido, etc.)}
  \veja{喂}{wei2}
\end{verbete}
\begin{verbete}{喂哺}{wei4bu3}{12;10}
  \significado{v.}{alimentar (um bebê)}
\end{verbete}
\begin{verbete}{喂料}{wei4liao4}{12;10}
  \significado{v.}{alimentar (também no sentido figurativo)}
\end{verbete}
\begin{verbete}{喂母乳}{wei4mu3ru3}{12;5;8}
  \significado{s.}{amamentação}
\end{verbete}
\begin{verbete}{喂奶}{wei4nai3}{12;5}
  \significado{v.}{amamentar}
\end{verbete}
\begin{verbete}{喂食}{wei4shi2}{12;9}
  \significado{v.}{alimentar}
\end{verbete}
\begin{verbete}{喂养}{wei4yang3}{12;9}
  \significado{v.}{alimentar (uma criança, animal doméstico, etc.); manter; criar (um animal)}
\end{verbete}
\begin{verbete}{温度}{wen1du4}{12;9}
  \significado[个]{s.}{temperatura}
\end{verbete}
\begin{verbete}{温度表}{wen1du4biao3}{12;9;8}
  \significado{s.}{termômetro}
\end{verbete}
\begin{verbete}{温度计}{wen1du4ji4}{12;9;4}
  \significado{s.}{termógrafo; termômetro}
\end{verbete}
\begin{verbete}{温度梯度}{wen1du4ti1du4}{12;9;11;9}
  \significado{s.}{gradiente de temperatura}
\end{verbete}
\begin{verbete}{文化}{wen2hua4}{4;4}
  \significado[个,种]{s.}{cultura; civilização}
\end{verbete}
\begin{verbete}{文化层}{wen2hua4ceng2}{4;4;7}
  \significado{s.}{nível de cultura (em sítio arqueológico)}
\end{verbete}
\begin{verbete}{文化宫}{wen2hua4gong1}{4;4;9}
  \significado{s.}{palácio cultural}
\end{verbete}
\begin{verbete}{文化圈}{wen2hua4quan1}{4;4;11}
  \significado{s.}{esfera de influência cultural}
\end{verbete}
\begin{verbete}{文化热}{wen2hua4re4}{4;4;10}
  \significado{s.}{mania cultural; febre cultural}
\end{verbete}
\begin{verbete}{文化史}{wen2hua4shi3}{4;4;5}
  \significado*{s.}{História Cultural}
\end{verbete}
\begin{verbete}{文化障碍}{wen2xue2zhang4'ai4}{4;8;13;13}
  \significado{s.}{barreira cultural}
\end{verbete}
\begin{verbete}{文学系}{wen2xue2·xi4}{4;8;7}
  \significado*{s.}{Faculdade de Letras}
\end{verbete}
\begin{verbete}{问}{wen4}{6}
  \significado{v.}{perguntar}
\end{verbete}
\begin{verbete}{问安}{wen4'an1}{6;6}
  \significado{s.}{saudações}
  \significado{v.}{dar cumprimentos a; prestar homenagem}
\end{verbete}
\begin{verbete}{问鼎}{wen4ding3}{6;12}
  \significado{v.}{visar (o primeiro lugar, etc.); aspirar ao trono}
\end{verbete}
\begin{verbete}{问卷}{wen4juan3}{6;8}
  \significado[份]{s.}{questionário}
\end{verbete}
\begin{verbete}{问市}{wen4shi4}{6;5}
  \significado{v.}{chegar ao marcado; bater o mercado; atingir o mercado}
\end{verbete}
\begin{verbete}{问题}{wen4ti2}{6;15}
  \significado[个]{s.}{pergunta; questão; problema}
\end{verbete}
\begin{verbete}{我}{wo3}{7}
  \significado{pron.}{eu; me; mim, comigo}
\end{verbete}
\begin{verbete}{我的}{wo3·de5}{7;8}
  \significado{pron.}{meu, meus}
\end{verbete}
\begin{verbete}{我们}{wo3men5}{7;5}
  \significado{pron.}{nós; nos; nós, conosco}
\end{verbete}
\begin{verbete}{我们的}{wo3men5·de5}{7;5;8}
  \significado{pron.}{nosso, nossos}
\end{verbete}
\begin{verbete}{卧}{wo4}{8}
  \significado{v.}{agachar; deitar}
\end{verbete}
\begin{verbete}{卧病}{wo4bing4}{8;10}
  \significado{s.}{acamado; doente na cama}
\end{verbete}
\begin{verbete}{卧舱}{wo4cang1}{8;10}
  \significado{s.}{cabine de dormir em um barco ou trem}
\end{verbete}
\begin{verbete}{卧车}{wo4che1}{8;4}
  \significado{s.}{um carro-leito; vagão-leito}
\end{verbete}
\begin{verbete}{卧床}{wo4chuang2}{8;7}
  \significado{adj.}{acamado}
  \significado{s.}{cama}
  \significado{v.}{deitar na cama}
\end{verbete}
\begin{verbete}{卧倒}{wo4dao3}{8;10}
  \significado{v.}{cair no chão; deitar-se}
\end{verbete}
\begin{verbete}{卧式}{wo4shi4}{8;6}
  \significado{adj.}{horizontal}
\end{verbete}
\begin{verbete}{卧室}{wo4shi4}{8;9}
  \significado[间]{s.}{quarto de dormir}
\end{verbete}
\begin{verbete}{卧榻}{wo4ta4}{8;14}
  \significado{s.}{um sofá; uma cama estreita}
\end{verbete}
\begin{verbete}{卧推}{wo4tui1}{8;11}
  \significado{s.}{supino}
\end{verbete}
\begin{verbete}{污染}{wu1ran3}{6;9}
  \significado{s.}{poluição}
  \significado{v.}{poluir}
\end{verbete}
\begin{verbete}{污染区}{wu1ran3qu1}{6;9;4}
  \significado{s.}{área contaminada}
\end{verbete}
\begin{verbete}{污染物}{wu1ran3wu4}{6;9;8}
  \significado{s.}{poluente}
  \veja{污染物质}{wu1ran3·wu4hi2}
\end{verbete}
\begin{verbete}{污染物质}{wu1ran3·wu4hi2}{6;9;8;8}
  \significado{s.}{poluente}
  \veja{污染物}{wu1ran3wu4}
\end{verbete}
\begin{verbete}{五}{wu3}{4}
  \significado{num.}{5, cinco}
\end{verbete}
\begin{verbete}{五五}{wu3wu3}{4;4}
  \significado{num.}{50-50}
  \significado{s.}{igual (partilha, parceria, etc.)}
\end{verbete}
\begin{verbete}{午}{wu3}{4}
  \significado{p.t.}{11h00-13h00; meio-dia}
\end{verbete}
\begin{verbete}{午餐}{wu3can1}{4;16}
  \significado[份,顿,次]{s.}{almoço}
\end{verbete}
\begin{verbete}{午饭}{wu3fan4}{4;7}
  \significado[份,顿,次,餐]{s.}{almoço}
\end{verbete}
\begin{verbete}{午后}{wu3hou4}{4;6}
  \significado{p.t.}{tarde; período da tarde}
\end{verbete}
\begin{verbete}{午前}{wu3qian2}{4;9}
  \significado{p.t.}{A.M.; manhã; período da manhã}
\end{verbete}
\begin{verbete}{午睡}{wu3shui4}{4;14}
  \significado{s.}{siesta}
  \significado{v.}{tirar uma soneca}
\end{verbete}
\begin{verbete}{午休}{wu3xiu1}{4;6}
  \significado{s.}{pausa para almoço; cochilo na hora do almoço; intervalo do meio-dia}
\end{verbete}
\begin{verbete}{午宴}{wu3yan4}{4;10}
  \significado{s.}{banquete de almoço}
\end{verbete}
\begin{verbete}{午夜}{wu3ye4}{4;8}
  \significado{p.t.}{meia-noite}
\end{verbete}
\begin{verbete}{武}{wu3}{8}
  \significado{s.}{arte marcial}
  \significado*{s.}{sobrenome Wu}
\end{verbete}
\begin{verbete}{武大戏}{wu3·da3xi4}{8;3;6}
  \significado*{s.}{Drama de Luta Acrobática; Drama Wu}
\end{verbete}
\begin{verbete}{武断}{wu3duan4}{8;11}
  \significado{adj.}{arbitrário; dogmático; subjetivo}
\end{verbete}
\begin{verbete}{武官}{wu3guan1}{8;8}
  \significado{s.}{oficial militar}
\end{verbete}
\begin{verbete}{武力}{wu3li4}{8;2}
  \significado{s.}{forças armadas, militares}
\end{verbete}
\begin{verbete}{武器}{wu3qi4}{8;16}
  \significado[种]{s.}{arma}
\end{verbete}
\begin{verbete}{武士}{wu3shi4}{8;3}
  \significado{s.}{samurai; guerreiro}
\end{verbete}
\begin{verbete}{武艺}{wu3yi4}{8;4}
  \significado{s.}{arte marcial; habilidade militar}
\end{verbete}
\begin{verbete}{武装}{wu3zhuang1}{8;12}
  \significado{s.}{forças armadas; militar; arma}
  \significado{v.}{armar}
\end{verbete}
\begin{verbete}{舞}{wu3}{14}
  \significado{s.}{dança}
\end{verbete}
\begin{verbete}{舞抃}{wu3bian4}{14;7}
  \significado{s.}{dançar por prazer}
\end{verbete}
\begin{verbete}{舞会}{wu3hui4}{14;6}
  \significado{s.}{baile}
\end{verbete}
\begin{verbete}{舞会舞}{wu3hui4wu3}{14;6;14}
 \significado{s.}{baile}
\end{verbete}
\begin{verbete}{舞厅}{wu3ting1}{14;4}
  \significado[间]{s.}{salão de dança; salão de baile}
\end{verbete}
\begin{verbete}{舞厅舞}{wu3ting1wu3}{14;4;14}
  \significado{s.}{dança de salão}
\end{verbete}

%%%%% EOF %%%%%

%%%
%%% X
%%%
%\section*{X}
\addcontentsline{toc}{section}{X}

\begin{verbete}{夕阳}{xi1yang2}{3;6}
  \significado{s.}{pôr do sol}
  \veja{日出}{ri4chu1}
\end{verbete}

\begin{verbete}{吸管}{xi1guan3}{6;14}
  \significado[支]{s.}{canudo para beber; pipeta; conta-gotas; \emph{snorkel}}
\end{verbete}

\begin{verbete}{吸铁石}{xi1tie3shi2}{6;10;5}
  \significado{s.}{imã; magneto}
  \veja{磁铁}{ci2tie3}
\end{verbete}

\begin{verbete}{西}{xi1}{6}[Radical 襾][Componentes 兀囗]
  \significado{p.l.}{oeste}
\end{verbete}

\begin{verbete}{西安}{xi1'an1}{6;6}
  \significado*{s.}{Xi'an}
\end{verbete}

\begin{verbete}{西班牙文}{xi1ban1ya2wen2}{6;10;4;4}
  \significado{s.}{espanhol, língua espanhola}
  \veja{西文}{xi1wen2}
\end{verbete}

\begin{verbete}{西班牙语}{xi1ban1ya2yu3}{6;10;4;9}
  \significado{s.}{espanhol, língua espanhola}
  \veja{西语}{xi1yu3}
\end{verbete}

\begin{verbete}{西半球}{xi1ban4qiu2}{6;5;11}
  \significado{s.}{hemisfério oeste}
\end{verbete}

\begin{verbete}{西边}{xi1bian1}{6;5}
  \significado{p.l.}{ao oeste de; oeste; lado oeste; parte ocidental}
\end{verbete}

\begin{verbete}{西部}{xi1bu4}{6;10}
  \significado{p.l.}{parte ocidental}
\end{verbete}

\begin{verbete}{西方}{xi1fang1}{6;4}
  \significado{p.l.}{países ocidentais; o Ocidente; o Oeste}
\end{verbete}

\begin{verbete}{西瓜}{xi1gua1}{6;5}
  \significado[颗,个]{s.}{melancia}
\end{verbete}

\begin{verbete}{西兰花}{xi1lan2hua1}{6;5;7}
  \significado{s.}{brócolis}
\end{verbete}

\begin{verbete}{西蓝花}{xi1lan2hua1}{6;13;7}
  \variante{西兰花}
\end{verbete}

\begin{verbete}{西面}{xi1mian4}{6;9}
  \significado{p.l.}{oeste; lado oeste}
\end{verbete}

\begin{verbete}{西文}{xi1wen2}{6;4}
  \significado{s.}{espanhol, língua espanhola}
  \veja{西班牙文}{xi1ban1ya2wen2}
\end{verbete}

\begin{verbete}{西西}{xi1xi1}{6;6}
  \significado{num.}{centímetro cúbico}
\end{verbete}

\begin{verbete}{西语}{xi1yu3}{6;9}
  \significado{s.}{espanhol, língua espanhola}
  \veja{西班牙语}{xi1ban1ya2yu3}
\end{verbete}

\begin{verbete}{希望}{xi1wang4}{7;11}
  \significado[个]{s.}{desejo}
  \significado{v.}{desejar}
\end{verbete}

\begin{verbete}{昔日}{xi1ri4}{8;4}
  \significado{adj.}{passado}
\end{verbete}

\begin{verbete}{牺牲}{xi1sheng1}{10;9}
  \significado{s.}{abate de um animal como sacrifício}
  \significado{v.}{sacrificar a vida de alguém; sacrificar (algo de valor)}
\end{verbete}

\begin{verbete}{悉尼}{xi1ni2}{11;5}
  \significado*{s.}{Sidney}
\end{verbete}

\begin{verbete}{悉数}{xi1shu3}{11;13}
  \significado{adv.}{enumerar em detalhes; explicar claramente}
  \veja{悉数}{xi1shu4}
\end{verbete}

\begin{verbete}{悉数}{xi1shu4}{11;13}
  \significado{adv.}{todos; cada um; toda a soma}
  \veja{悉数}{xi1shu3}
\end{verbete}

\begin{verbete}{悉心}{xi1xin1}{11;4}
  \significado{adv.}{colocar o coração (e a alma) em algo; com muito cuidado}
\end{verbete}

\begin{verbete}{蜥易}{xi1yi4}{14;8}
  \variante{蜥蜴}
\end{verbete}

\begin{verbete}{蜥蜴}{xi1yi4}{14;14}
  \significado{s.}{lagarto}
\end{verbete}

\begin{verbete}{席卷}{xi2juan3}{10;8}
  \significado{v.}{engolfar; varrer; levar tudo para fora}
\end{verbete}

\begin{verbete}{袭击}{xi2ji1}{11;5}
  \significado{s.}{ataque (especialmente um ataque surpresa); invasão}
  \significado{v.}{atacar}
\end{verbete}

\begin{verbete}{洗}{xi3}{9}[Radical 水][Componentes 氵先]
  \significado{v.}{lavar; revelar (fotos); tomar banho}
\end{verbete}

\begin{verbete}{洗涤}{xi3di2}{9;10}
  \significado{s.}{enxágue; lava}
  \significado{v.}{enxaguar; lavar}
\end{verbete}

\begin{verbete}{洗涤间}{xi3di2jian1}{9;10;7}
  \significado{s.}{lavanderia}
\end{verbete}

\begin{verbete}{洗劫}{xi3jie2}{9;7}
  \significado{v.}{saquear; pilhar; roubar}
\end{verbete}

\begin{verbete}{洗净}{xi3jing4}{9;8}
  \significado{v.}{lavar (limpeza)}
\end{verbete}

\begin{verbete}{洗礼}{xi3li3}{9;5}
  \significado{s.}{batismo}
  \significado{v.}{batizar}
\end{verbete}

\begin{verbete}{洗手}{xi3shou3}{9;4}
  \significado{v.}{ir ao banheiro; lavar as mãos}
\end{verbete}

\begin{verbete}{洗手不干}{xi3shou3bu2gan4}{9;4;4;3}
  \significado{v.}{parar totalmente de fazer algo}
\end{verbete}

\begin{verbete}{洗手池}{xi3shou3chi2}{9;4;6}
  \significado{s.}{pia de banheiro; lavatório}
  \veja{洗手盆}{xi3shou3pen2}
\end{verbete}

\begin{verbete}{洗手间}{xi3shou3jian1}{9;4;7}
  \significado{s.}{sanitário; toilette; banheiro}
\end{verbete}

\begin{verbete}{洗手盆}{xi3shou3pen2}{9;4;9}
  \significado{s.}{pia de banheiro; lavatório}
  \veja{洗手池}{xi3shou3chi2}
\end{verbete}

\begin{verbete}{洗手乳}{xi3shou3ru3}{9;4;8}
  \significado{s.}{sabonete líquido para lavar as mãos}
  \veja{洗手液}{xi3shou3ye4}
\end{verbete}

\begin{verbete}{洗手液}{xi3shou3ye4}{9;4;11}
  \significado{s.}{sabonete líquido para lavar as mãos}
  \veja{洗手乳}{xi3shou3ru3}
\end{verbete}

\begin{verbete}{洗脱}{xi3tuo1}{9;11}
  \significado{v.}{limpar; purgar; lavar}
\end{verbete}

\begin{verbete}{洗碗}{xi3wan3}{9;13}
  \significado{v.}{lavar pratos}
\end{verbete}

\begin{verbete}{洗胃}{xi3wei4}{9;9}
  \significado{s.}{medicina:~lavagem gástrica}
  \significado{v.}{ter o estômago lavado}
\end{verbete}

\begin{verbete}{洗衣机}{xi3yi1ji1}{9;6;6}
  \significado[台]{s.}{máquina de lavar roupa}
\end{verbete}

\begin{verbete}{洗澡间}{xi3zao3jian1}{9;16;7}
  \significado[间]{s.}{banheiro}
\end{verbete}

\begin{verbete}{喜欢}{xi3huan5}{12;6}
  \significado{v.}{gostar}
\end{verbete}

\begin{verbete}{喜剧}{xi3ju4}{12;10}
  \significado[部,出]{s.}{uma comédia}
\end{verbete}

\begin{verbete}{戏}{xi4}{6}[Radical 戈][Componentes 又戈]
  \significado[出,场,台]{s.}{drama; peça de teatro; \emph{show}}
\end{verbete}

\begin{verbete}{戏法}{xi4fa3}{6;8}
  \significado{s.}{truque de mágica; prestidigitação}
\end{verbete}

\begin{verbete}{戏剧}{xi4ju4}{6;10}
  \significado{s.}{drama; suspense; teatro}
\end{verbete}

\begin{verbete}{戏剧般}{xi4ju4ban1}{6;10;10}
  \significado{adj.}{melodramático}
\end{verbete}

\begin{verbete}{戏剧编剧}{xi4ju4bian1ju4}{6;10;12;10}
  \significado{s.}{dramaturgo}
\end{verbete}

\begin{verbete}{戏剧化地}{xi4ju4hua4di4}{6;10;4;6}
  \significado{adv.}{dramaticamente; teatralmente}
\end{verbete}

\begin{verbete}{戏剧家}{xi4ju4jia1}{6;10;10}
  \significado{s.}{dramaturgo}
\end{verbete}

\begin{verbete}{戏剧效果}{xi4ju4xiao4guo3}{6;10;10;8}
  \significado{s.}{efeito dramático}
\end{verbete}

\begin{verbete}{戏剧性}{xi4ju4xing4}{6;10;8}
  \significado{adj.}{dramático}
\end{verbete}

\begin{verbete}{戏剧演出}{xi4ju4yan3chu1}{6;10;14;5}
  \significado{s.}{performance dramática}
\end{verbete}

\begin{verbete}{戏弄}{xi4nong4}{6;7}
  \significado{v.}{zombar de; pregar peças; provocar}
\end{verbete}

\begin{verbete}{戏耍}{xi4shua3}{6;9}
  \significado{v.}{divertir-me; brincar com; provocar}
\end{verbete}

\begin{verbete}{戏谑}{xi4xue4}{6;11}
  \significado{v.}{brincar; fazer piadas; ridicularizar}
\end{verbete}

\begin{verbete}{戏院}{xi4yuan4}{6;9}
  \significado{s.}{teatro}
\end{verbete}

\begin{verbete}{系}{xi4}{7}[Radical 糸][Componentes ㇒糸]
  \significado{s.}{faculdade (da universidade); departamento}
  \significado{v.}{prender; vincular; conectar; relacionar com; amarrar; se preocupar}
\end{verbete}

\begin{verbete}{系列}{xi4lie4}{7;6}
  \significado{s.}{série; conjunto}
\end{verbete}

\begin{verbete}{系囚}{xi4qiu2}{7;5}
  \significado{s.}{prisioneiro}
\end{verbete}

\begin{verbete}{系统}{xi4tong3}{7;9}
  \significado[个]{s.}{sistema}
\end{verbete}

\begin{verbete}{细菌战}{xi4jun1zhan4}{8;11;9}
  \significado{s.}{guerra biológica}
\end{verbete}

\begin{verbete}{繋}{xi4}{17}
  \variante{系}
\end{verbete}

\begin{verbete}{虾}{xia1}{9}[Radical 虫][Componentes 虫下]
  \significado{s.}{camarão}
\end{verbete}

\begin{verbete}{下}{xia4}{3}[Radical 一][Componentes 一卜]
  \significado{p.l.}{abaixo; em baixo de}
  \significado{v.d.}{descer; chegar a (uma decisão, conclusão, etc.); recusar}
\end{verbete}

\begin{verbete}{下巴}{xia4ba5}{3;4}
  \significado[个]{s.}{queixo}
\end{verbete}

\begin{verbete}{下边}{xia4bian5}{3;5}
  \significado{p.l.}{em baixo; abaixo; parte de baixo}
\end{verbete}

\begin{verbete}{下车}{xia4che1}{3;4}
  \significado{v.}{descer; sair (de ônibus, carro, etc.)}
\end{verbete}

\begin{verbete}{下蛋}{xia4dan4}{3;11}
  \significado{v.}{botar ovos}
\end{verbete}

\begin{verbete}{下课}{xia4ke4}{3;10}
  \significado{v.+compl.}{acabar a aula; terminar a aula}
\end{verbete}

\begin{verbete}{下来}{xia4lai5}{3;7}
  \significado{v.d.}{descer (para a minha localização)}
\end{verbete}

\begin{verbete}{下面}{xia4mian4}{3;9}
  \significado{p.l.}{em baixo; abaixo; parte de baixo}
  \significado{v.}{cozinhar macarrão}
\end{verbete}

\begin{verbete}{下去}{xia4qu4}{3;5}
  \significado{v.d.}{descer (a partir da minha localização)}
\end{verbete}

\begin{verbete}{下水道}{xia4shui3dao4}{3;4;12}
  \significado{s.}{esgoto}
\end{verbete}

\begin{verbete}{下午}{xia4wu3}{3;4}
  \significado{p.t.}{tarde; à tarde; período logo após o meio-dia}
\end{verbete}

\begin{verbete}{下午茶}{xia4wu3cha2}{3;4;9}
  \significado{s.}{chá da tarde (normalmente chás com doces);}
\end{verbete}

\begin{verbete}{下线}{xia4xian4}{3;8}
  \significado{v.}{ficar \emph{offline}; (um produto) sair da linha de montagem; pessoa abaixo de si em um esquema de pirâmide}
\end{verbete}

\begin{verbete}{下旬}{xia4xun2}{3;6}
  \significado{p.t.}{última dezena do mês}
\end{verbete}

\begin{verbete}{下雨}{xia4yu3}{3;8}
  \significado{v.+compl.}{chover}
\end{verbete}

\begin{verbete}{下载}{xia4zai3}{3;10}
  \significado{v.}{baixar; \emph{download}}
\end{verbete}

\begin{verbete}{下崽}{xia4zai3}{3;12}
  \significado{v.}{dar à luz (animais); parir}
\end{verbete}

\begin{verbete}{细节}{xia4jie2}{8;5}
  \significado{s.}{detalhe, particularidade}
\end{verbete}

\begin{verbete}{夏日}{xia4ri4}{10;4}
  \significado{s.}{horário de verão}
\end{verbete}

\begin{verbete}{夏天}{xia4tian1}{10;4}
  \significado[个]{p.t./s.}{verão}
\end{verbete}

\begin{verbete}{仙}{xian1}{5}[Radical 人][Componentes 亻山]
  \significado{s.}{imortal}
\end{verbete}

\begin{verbete}{先}{xian1}{6}[Radical 儿][Componentes ⺧儿]
  \significado{adv.}{em primeiro lugar; primeiramente; antes do tempo; de antemão}
\end{verbete}

\begin{verbete}{先不先}{xian1bu4xian1}{6;4;6}
  \significado{adv.}{dialeto:~antes de tudo; em primeiro lugar}
\end{verbete}

\begin{verbete}{先到先得}{xian1dao4xian1de2}{6;8;6;11}
  \significado{expr.}{primeiro a chegar, primeiro a ser servido}
\end{verbete}

\begin{verbete}{先烈}{xian1lie4}{6;10}
  \significado{s.}{mártir}
\end{verbete}

\begin{verbete}{先期}{xian1qi1}{6;12}
  \significado{adv.}{antecipadamente}
  \significado{s.}{prematuro; \emph{front-end}}
\end{verbete}

\begin{verbete}{先生}{xian1sheng5}{6;5}
  \significado[位]{s.}{senhor; marido; professor; dialeto:~doutor}
\end{verbete}

\begin{verbete}{先天}{xian1tian1}{6;4}
  \significado{adj.}{congênito; inato; natural}
  \significado{s.}{período embrionário}
\end{verbete}

\begin{verbete}{先验}{xian1yan4}{6;10}
  \significado{adj.}{filosofia:~a priori}
\end{verbete}

\begin{verbete}{先有}{xian1you3}{6;6}
  \significado{adj.}{preexistente; anterior}
\end{verbete}

\begin{verbete}{咸}{xian2}{9}[Radical 口][Componentes 戊]
  \significado*{s.}{sobrenome Xian}
  \significado{adj.}{salgado}
\end{verbete}

\begin{verbete}{咸菜}{xian2cai4}{9;11}
  \significado{s.}{legumes salgados; \emph{pickles}}
\end{verbete}

\begin{verbete}{咸淡}{xian2dan4}{9;11}
  \significado{s.}{água salobra; grau de salinidade; salgado e sem sal (sabores)}
\end{verbete}

\begin{verbete}{咸肉}{xian2rou4}{9;6}
  \significado{s.}{\emph{bacon}; carne curada com sal}
\end{verbete}

\begin{verbete}{咸涩}{xian2se4}{9;10}
  \significado{s.}{ácido; salgado e amargo}
\end{verbete}

\begin{verbete}{咸水}{xian2shui3}{9;4}
  \significado{s.}{salmora; água salgada}
\end{verbete}

\begin{verbete}{咸盐}{xian2yan2}{9;10}
  \significado{s.}{coloquial:~sal; sal de mesa}
\end{verbete}

\begin{verbete}{咸鱼}{xian2yu2}{9;8}
  \significado{s.}{peixe salgado}
\end{verbete}

\begin{verbete}{猃狁}{xian3yun3}{10;7}
  \significado*{s.}{Termo da dinastia Zhou para uma tribo nômade do norte mais tarde chamou o Xiongnu (匈奴) nas dinastias Qin e Han}
  \veja{匈奴}{xiong1nu2}
\end{verbete}

\begin{verbete}{见}{xian4}{4}[Radical 見][Componentes 冂儿]
  \significado{v.}{aparecer; também escrito como 现}
  \veja{见}{jian4}
  \veja{现}{xian4}
\end{verbete}

\begin{verbete}{现}{xian4}{8}[Radical 見][Componentes 王见]
  \significado{adj.}{presente; atual}
  \significado{v.}{aparecer}
  \veja{见}{xian4}
\end{verbete}

\begin{verbete}{现代}{xian4dai4}{8;5}
  \significado*{s.}{Hyundai, empresa sul-coreana}
  \significado{s.}{tempos modernos; era moderna}
\end{verbete}

\begin{verbete}{现货}{xian4huo4}{8;8}
  \significado{s.}{produtos à vista}
\end{verbete}

\begin{verbete}{现货的}{xian4huo4 de5}{8;8;8}
  \significado{s.}{produtos em estoque}
\end{verbete}

\begin{verbete}{现实}{xian4shi2}{8;8}
  \significado{adj.}{real; realístico}
  \significado{s.}{realidade}
\end{verbete}

\begin{verbete}{现象}{xian4xiang4}{8;11}
  \significado[个,种]{s.}{fenômeno}
\end{verbete}

\begin{verbete}{现有}{xian4you3}{8;6}
  \significado{adj.}{disponível atualmente; atualmente existente}
\end{verbete}

\begin{verbete}{现在}{xian4zai4}{8;6}
  \significado{p.t.}{agora; neste momento}
\end{verbete}

\begin{verbete}{现抓}{xian4zhua1}{8;7}
  \significado{v.}{improvisar}
\end{verbete}

\begin{verbete}{现做}{xian4zuo4}{8;11}
  \significado{adj.}{fresco}
  \significado{v.}{fazer (comida) no local}
\end{verbete}

\begin{verbete}{线香}{xian4xiang1}{8;9}
  \significado{s.}{bastão ou vareta de incenso}
\end{verbete}

\begin{verbete}{宪法法院}{xian4fa3fa3yuan4}{9;8;8;9}
  \significado{s.}{tribunal constitucional}
\end{verbete}

\begin{verbete}{宪政}{xian4zheng4}{9;9}
  \significado{s.}{governo constitucional}
\end{verbete}

\begin{verbete}{宪制}{xian4zhi4}{9;8}
  \significado{adj.}{constitucional}
  \significado{s.}{sistema de governo constitucional}
\end{verbete}

\begin{verbete}{陷入}{xian4ru4}{10;2}
  \significado{v.}{afundar; ser pego em; pousar (em uma situação)}
\end{verbete}

\begin{verbete}{羡慕}{xian4mu4}{12;14}
  \significado{v.}{invejar; admirar}
\end{verbete}

\begin{verbete}{乡巴佬}{xiang1ba1lao3}{3;4;8}
  \significado{s.}{aldeão; caipira}
\end{verbete}

\begin{verbete}{乡村}{xiang1cun1}{3;7}
  \significado{adj.}{rural; rústico}
  \significado{s.}{vila; campo}
\end{verbete}

\begin{verbete}{相处}{xiang1chu3}{9;5}
  \significado{v.}{entrar em contato (com alguém); associar; interagir; se dar bem (bem, mal)}
\end{verbete}

\begin{verbete}{相当}{xiang1dang1}{9;6}
  \significado{adv.}{bastante; consideravelmente}
\end{verbete}

\begin{verbete}{相聚}{xiang1ju4}{9;14}
  \significado{v.}{reunir-se; montar}
\end{verbete}

\begin{verbete}{相亲}{xiang1qin1}{9;9}
  \significado{s.}{encontro às cegas; entrevista arranjada para avaliar a proposta de um parceiro de casamento; apegar-se profundamente um ao outro}
\end{verbete}

\begin{verbete}{相思病}{xiang1si1bing4}{9;9;10}
  \significado{s.}{saudade de amor}
\end{verbete}

\begin{verbete}{相信}{xiang1xin4}{9;9}
  \significado{v.}{acreditar, estar convencido. aceitar como verdadeiro}
\end{verbete}

\begin{verbete}{相宜}{xiang1yi2}{9;8}
  \significado{adj.}{adequado; apropriado}
  \significado{v.}{ser adequado ou apropriado}
\end{verbete}

\begin{verbete}{相遇}{xiang1yu4}{9;12}
  \significado{v.}{encontrar (reunião, encontro, etc.)}
\end{verbete}

\begin{verbete}{香}{xiang1}{9}[Radical 香][Componentes 禾日]
  \significado{adj.}{perfumado; com cheiro doce; aromático; saboroso ou apetitoso;  (comer) com prazer; (para dormir) som}
  \significado[根]{s.}{perfume ou especiarias; bastão ou vareta de incenso}
\end{verbete}

\begin{verbete}{香槟酒}{xiang1bin1jiu3}{9;14;10}
  \significado[杯]{s.}{\emph{champagne} (empréstimo linguístico)}
\end{verbete}

\begin{verbete}{香波}{xiang1bo1}{9;8}
  \significado{s.}{xampu}
\end{verbete}

\begin{verbete}{香肠}{xiang1chang2}{9;7}
  \significado[根]{s.}{salsicha}
\end{verbete}

\begin{verbete}{香港}{xiang1gang3}{9;12}
  \significado*{s.}{Hong Kong}
  \veja{香港岛}{xiang1gang3 dao3}
\end{verbete}

\begin{verbete}{香港岛}{xiang1gang3 dao3}{9;12;7}
  \significado*{s.}{Ilha de Hong Kong}
  \veja{香港}{xiang1gang3}
\end{verbete}

\begin{verbete}{香蕉}{xiang1jiao1}{9;15}
  \significado[枝,根,个,把]{s.}{banana}
\end{verbete}

\begin{verbete}{香炉}{xiang1lu2}{9;8}
  \significado{s.}{incensário (para queimar incenso); queimador de incenso; insensório, turíbulo}
\end{verbete}

\begin{verbete}{香气}{xiang1qi4}{9;4}
  \significado{s.}{fragrância; aroma; incenso}
\end{verbete}

\begin{verbete}{香味}{xiang1wei4}{9;8}
  \significado[股]{s.}{fragrância; cheiro doce}
\end{verbete}

\begin{verbete}{香蕈}{xiang1xun4}{9;15}
  \significado{s.}{\emph{shiitake}, cogumelo comestível}
\end{verbete}

\begin{verbete}{香烟}{xiang1yan1}{9;10}
  \significado[支,条]{s.}{cigarro; fumaça de incenso queimado}
\end{verbete}

\begin{verbete}{香艳}{xiang1yan4}{9;10}
  \significado{adj.}{atraente; erótico; romântico}
\end{verbete}

\begin{verbete}{香皂}{xiang1zao4}{9;7}
  \significado{s.}{sabonete; sabonete perfumado}
\end{verbete}

\begin{verbete}{享受}{xiang3shou4}{8;8}
  \significado[种]{s.}{prazer}
  \significado{v.}{desfrutar; viver}
\end{verbete}

\begin{verbete}{想}{xiang3}{13}[Radical 心][Componentes 相心]
  \significado{v./v.o.}{acreditar; sentir falta (sentir-se melancólico com a ausência de alguém ou algo); supor; pensar; querer; desejar}
\end{verbete}

\begin{verbete}{想法}{xiang3fa3}{13;8}
  \significado[个]{s.}{noção; opinião; jeito de pensar}
  \significado{s.}{maneira de pensar, opinião, noção}
  \significado{v.}{pensar em uma maneira (de fazer algo)}
\end{verbete}

\begin{verbete}{想念}{xiang3nian4}{13;8}
  \significado{v.}{perder; sentir falta; lembrar com saudade}
\end{verbete}

\begin{verbete}{想想看}{xiang3xiang3kan4}{13;13;9}
  \significado{v.}{pensar sobre isso}
\end{verbete}

\begin{verbete}{想象}{xiang3xiang4}{13;11}
  \significado{v.}{imaginar}
\end{verbete}

\begin{verbete}{向}{xiang4}{6}[Radical 口][Componentes 丿冂口]
  \significado*{s.}{sobrenome Xiang}
  \significado{prep.}{para}
  \significado{v.}{enfrentar; virar para; apoiar}
\end{verbete}

\begin{verbete}{向汪}{xiang4wang1}{6;7}
  \significado{v.}{esperar que}
\end{verbete}

\begin{verbete}{向往}{xiang4wang3}{6;8}
  \significado{v.}{ansiar por; esperar ansiosamente por}
\end{verbete}

\begin{verbete}{像}{xiang4}{13}[Radical 人][Componentes 亻象]
  \significado{s.}{imagem; retrato; aparência}
  \significado{v.}{assemelhar -se; ser como}
\end{verbete}

\begin{verbete}{消防}{xiao1fang2}{10;6}
  \significado{s.}{combate a incêncios; controle de incêndios}
\end{verbete}

\begin{verbete}{消防员}{xiao1fang2yuan2}{10;6;7}
  \significado{s.}{bombeiro}
\end{verbete}

\begin{verbete}{消失}{xiao1shi1}{10;5}
  \significado{v.}{desaparecer; desvanecer}
\end{verbete}

\begin{verbete}{嚣张}{xiao1zhang1}{18;7}
  \significado{adj.}{desenfreado; arrogante; agressivo}
\end{verbete}

\begin{verbete}{小}{xiao3}{3}[Radical 小][Componentes 亅八][Kangxi 42]
  \significado{adj.}{pequeno; jovem}
\end{verbete}

\begin{verbete}{小白菜}{xiao3bai2cai4}{3;5;11}
  \significado[棵]{s.}{\emph{bok choy}; couve chinesa}
\end{verbete}

\begin{verbete}{小吃}{xiao3chi1}{3;6}
  \significado{s.}{refeição leve; petisco}
\end{verbete}

\begin{verbete}{小狗}{xiao3 gou3}{3;8}
  \significado{s.}{filhote de cachorro}
\end{verbete}

\begin{verbete}{小姐}{xiao3jie5}{3;8}
  \significado[个,位]{s.}{senhorita; jovem senhora; gíria:~prostituta}
\end{verbete}

\begin{verbete}{小气鬼}{xiao3qi4gui3}{3;4;9}
  \significado{adj.}{avarento; mão-de-vaca; miserável; pão-duro}
\end{verbete}

\begin{verbete}{小区}{xiao3qu1}{3;4}
  \significado{s.}{conjunto habitacional, comunidade, bairro; célula (telecomunicações)}
\end{verbete}

\begin{verbete}{小时}{xiao3shi2}{3;7}
  \significado{p.c.}{hora; para horas}
  \significado[个]{s.}{hora}
\end{verbete}

\begin{verbete}{小树}{xiao3shu4}{3;9}
  \significado[棵]{s.}{muda; arbusto; árvore pequena}
\end{verbete}

\begin{verbete}{小说}{xiao3shuo1}{3;9}
  \significado[本,部]{s.}{romance; ficção}
\end{verbete}

\begin{verbete}{小腿}{xiao3tui3}{3;13}
  \significado{s.}{perna (do joelho ao calcanhar); haste}
\end{verbete}

\begin{verbete}{小屋}{xiao3wu1}{3;9}
  \significado{s.}{cabana; chalé; cabine}
\end{verbete}

\begin{verbete}{小小}{xiao3xiao3}{3;3}
  \significado{adj.}{muito pequeno}
\end{verbete}

\begin{verbete}{小心}{xiao3xin1}{3;4}
  \significado{adj.}{cuidado}
\end{verbete}

\begin{verbete}{小学}{xiao3xue2}{3;8}
  \significado{s.}{escola ensino fundamental}
\end{verbete}

\begin{verbete}{小洋白菜}{xiao3 yang2bai2cai4}{3;9;5;11}
  \significado{s.}{couve de bruxelas}
\end{verbete}

\begin{verbete}{小众}{xiao3zhong4}{3;6}
  \significado{s.}{minoria da população; nicho (mercado, etc.)}
\end{verbete}

\begin{verbete}{哮喘}{xiao4chuan3}{10;12}
  \significado{s.}{asma}
\end{verbete}

\begin{verbete}{效果}{xiao4guo3}{10;8}
  \significado{s.}{resultado; efeito; eficácia; (teatro/cinema) efeitos sonoros ou visuais}
\end{verbete}

\begin{verbete}{校}{xiao4}{10}[Radical 木][Componentes 木交]
  \significado[所]{s.}{oficial militar; escola}
  \veja{校}{jiao4}
\end{verbete}

\begin{verbete}{校服}{xiao4fu2}{10;8}
  \significado{s.}{uniforme escolar}
\end{verbete}

\begin{verbete}{校规}{xiao4gui1}{10;8}
  \significado{s.}{regras e regulamentos escolares}
\end{verbete}

\begin{verbete}{校监}{xiao4jian1}{10;10}
  \significado{s.}{diretor; supervisor (de escola)}
\end{verbete}

\begin{verbete}{校园}{xiao4yuan2}{10;7}
  \significado{s.}{campus}
\end{verbete}

\begin{verbete}{校长}{xiao4zhang3}{10;4}
  \significado[个,位,名]{s.}{diretor de escola; reitor (universidade)}
\end{verbete}

\begin{verbete}{笑}{xiao4}{10}[Radical 竹][Componentes ⺮夭]
  \significado{v.}{sorrir, rir; rir de}
\end{verbete}

\begin{verbete}{笑话}{xiao4hua5}{10;8}
  \significado{adj.}{absurdo, ridículo}
  \significado[个]{s.}{piada, brincadeira}
  \significado{v.}{rir de algo, zombar, ridicularizar}
\end{verbete}

\begin{verbete}{笑容}{xiao4rong2}{10;10}
  \significado[副]{s.}{sorriso; expressão sorridente}
\end{verbete}

\begin{verbete}{些}{xie1}{8}[Radical 二][Componentes 此二]
  \significado{adv.}{uns; alguns; vários}
  \significado{p.c.}{que indica uma pequena quantidade ou pequeno número maior que 1}
\end{verbete}

\begin{verbete}{些许}{xie1xu3}{8;6}
  \significado{num.}{um pouco}
\end{verbete}

\begin{verbete}{斜阳}{xie2yang2}{11;6}
  \significado{s.}{sol poente}
\end{verbete}

\begin{verbete}{谐}{xie2}{11}[Radical 言][Componentes 讠皆]
  \significado{adj.}{harmonioso; humorístico}
\end{verbete}

\begin{verbete}{鞋}{xie2}{15}[Radical 革][Componentes 革圭]
  \significado[双,只]{s.}{sapatos}
\end{verbete}

\begin{verbete}{写}{xie3}{5}[Radical 冖][Componentes 冖与]
  \significado{v.}{escrever}
\end{verbete}

\begin{verbete}{写意}{xie3yi4}{5;13}
  \significado{s.}{estilo de pintura chinesa à mão livre, caracterizado por traços ousados em vez de detalhes precisos}
  \significado{v.}{sugerir (em vez de descrever em detalhes)}
  \veja{写意}{xie4yi4}
\end{verbete}

\begin{verbete}{写照}{xie3zhao4}{5;13}
  \significado{s.}{retrato}
\end{verbete}

\begin{verbete}{写真}{xie3zhen1}{5;10}
  \significado{s.}{retrato}
  \significado{v.}{descrever algo com precisão}
\end{verbete}

\begin{verbete}{写字}{xie3zi4}{5;6}
  \significado{v.}{escrever (à mão); praticar caligrafia}
\end{verbete}

\begin{verbete}{写字匠}{xie3zi4 jiang4}{5;6;6}
  \significado{s.}{calígrafo}
\end{verbete}

\begin{verbete}{写作}{xie3zuo4}{5;7}
  \significado{s.}{escrita; redação; composição}
  \significado{v.}{escrever}
\end{verbete}

\begin{verbete}{血}{xie3}{6}
  \veja{血}{xue4}
\end{verbete}

\begin{verbete}{写意}{xie4yi4}{5;13}
  \significado{adj.}{confortável; agradável; relaxado}
  \veja{写意}{xie3yi4}
\end{verbete}

\begin{verbete}{谢病}{xie4bing4}{12;10}
  \significado{v.}{desculpar-se por causa de doença}
\end{verbete}

\begin{verbete}{谢恩}{xie4'en1}{12;10}
  \significado{v.}{agradecer a alguém pelo favor (especialmente imperador ou oficial superior)}
\end{verbete}

\begin{verbete}{谢媒}{xie4mei2}{12;12}
  \significado{v.}{agradecer ao casamenteiro}
\end{verbete}

\begin{verbete}{谢世}{xie4shi4}{12;5}
  \significado{v.}{morrer; falecer}
\end{verbete}

\begin{verbete}{谢天谢地}{xie4tian1xie4di4}{12;4;12;6}
  \significado{expr.}{agradecer a Deus; agradecer aos céus}
\end{verbete}

\begin{verbete}{谢谢}{xie4xie5}{12;12}
  \significado{interj.}{Obrigado!}
  \significado{v.}{agradecer}
\end{verbete}

\begin{verbete}{谢意}{xie4yi4}{12;13}
  \significado{s.}{gratidão}
\end{verbete}

\begin{verbete}{心机}{xin1ji1}{4;6}
  \significado{s.}{pensamento; esquema}
\end{verbete}

\begin{verbete}{心声}{xin1sheng1}{4;7}
  \significado{s.}{desejo sincero, voz interior, aspiração}
\end{verbete}

\begin{verbete}{心疼}{xin1teng2}{4;10}
  \significado{adj.}{angustiado}
  \significado{v.}{sentir pena de alguém; arrepender-se; ressentir-se; ficar angustiado}
\end{verbete}

\begin{verbete}{心中}{xin1zhong1}{4;4}
  \significado{adv.}{nos pensamentos; no coração}
  \significado{s.}{ponto central}
\end{verbete}

\begin{verbete}{芯片}{xin1pian4}{7;4}
  \significado{s.}{chip de computador; microchip}
\end{verbete}

\begin{verbete}{辛苦}{xin1ku3}{7;8}
  \significado{adj.}{exaustivo; duro; árduo}
  \significado{s.}{dificuldades}
  \significado{v.}{trabalhar duro; ter muitos problemas}
\end{verbete}

\begin{verbete}{新}{xin1}{13}[Radical 斤][Componentes 亲斤]
  \significado*{s.}{sobrenome Xin; abreviatura de Xinjiang (新疆); abreviatura de Singapura (新加坡)}
  \significado{adj.}{novo; prefixo meso (química)}
  \significado{adv.}{recentemente}
  \veja{新加坡}{xin1jia1po1}
  \veja{新疆}{xin1jiang1}
\end{verbete}

\begin{verbete}{新加坡}{xin1jia1po1}{13;5;8}
  \significado*{s.}{Singapura}
\end{verbete}

\begin{verbete}{新疆}{xin1jiang1}{13;19}
  \significado*{s.}{Xinjiang}
\end{verbete}

\begin{verbete}{新疆维吾尔自治区}{xin1jiang1 wei2wu2'er3 zi4zhi4qu1}{13;19;11;7;5;6;8;4}
  \significado*{s.}{Região Autônoma Uigur de Xinjiang}
\end{verbete}

\begin{verbete}{新年}{xin1nian2}{13;6}
  \significado*[个]{s.}{Ano Novo}
\end{verbete}

\begin{verbete}{新娘}{xin1niang2}{13;10}
  \significado{s.}{noiva}
\end{verbete}

\begin{verbete}{新娘服装}{xin1niang2 fu2zhuang1}{13;10;8;12}
  \significado{s.}{roupas de noiva}
\end{verbete}

\begin{verbete}{新娘子}{xin1niang2zi5}{13;10;3}
  \veja{新娘}{xin1niang2}
\end{verbete}

\begin{verbete}{新闻}{xin1wen2}{13;9}
  \significado[条,个]{s.}{notícia}
\end{verbete}

\begin{verbete}{新鲜}{xin1xian1}{13;14}
  \significado{adj.}{fresco (experiência, alimento, etc.)}
  \significado{s.}{frescor}
\end{verbete}

\begin{verbete}{信}{xin4}{9}[Radical 人][Componentes 亻言]
  \significado[封]{s.}{carta; correspondência}
\end{verbete}

\begin{verbete}{信访}{xin4fang3}{9;6}
  \significado{s.}{carta de reclamação; carta de petição}
  \veja{上访}{shang4fang3}
\end{verbete}

\begin{verbete}{信封}{xin4feng1}{9;9}
  \significado[个]{s.}{envelope}
\end{verbete}

\begin{verbete}{信经}{xin4jing1}{9;8}
  \significado[个]{s.}{crença; credo (seção da missa católica)}
\end{verbete}

\begin{verbete}{信心}{xin4xin1}{9;4}
  \significado[个]{s.}{confiança; fé (em alguém ou algo)}
\end{verbete}

\begin{verbete}{信用}{xin4yong4}{9;5}
  \significado{s.}{crédito (comércio)}
\end{verbete}

\begin{verbete}{信用卡}{xin4yong4ka3}{9;5;5}
  \significado[些]{s.}{cartão de crédito}
\end{verbete}

\begin{verbete}{星表}{xing1biao3}{9;8}
  \significado{s.}{catálogo de estrelas}
\end{verbete}

\begin{verbete}{星辰}{xing1chen2}{9;7}
  \significado{s.}{estrelas}
\end{verbete}

\begin{verbete}{星火}{xing1huo3}{9;4}
  \significado{s.}{trilha de meteoro (usada principalmente em expressões como 急如星火); faísca}
\end{verbete}

\begin{verbete}{星期}{xing1qi1}{9;12}
  \significado[个]{s.}{semana}
\end{verbete}

\begin{verbete}{星期二}{xing1qi1'er4}{9;12;2}
  \significado{p.t.}{terça-feira}
\end{verbete}

\begin{verbete}{星期六}{xing1qi1liu4}{9;12;4}
  \significado{p.t.}{sábado}
\end{verbete}

\begin{verbete}{星期日}{xing1qi1ri4}{9;12;4}
  \significado{p.t.}{domingo}
  \veja{星期天}{xing1qi1tian1}
\end{verbete}

\begin{verbete}{星期三}{xing1qi1san1}{9;12;3}
  \significado{p.t.}{quarta-feira}
\end{verbete}

\begin{verbete}{星期四}{xing1qi1si4}{9;12;5}
  \significado{p.t.}{quinta-feira}
\end{verbete}

\begin{verbete}{星期天}{xing1qi1tian1}{9;12;4}
  \significado{p.t.}{domingo}
  \veja{星期日}{xing1qi1ri4}
\end{verbete}

\begin{verbete}{星期五}{xing1qi1wu3}{9;12;4}
  \significado{p.t.}{sexta-feira}
\end{verbete}

\begin{verbete}{星期一}{xing1qi1yi1}{9;12;1}
  \significado{p.t.}{segunda-feira}
\end{verbete}

\begin{verbete}{星星}{xing1xing5}{9;9}
  \significado{s.}{estrela}
\end{verbete}

\begin{verbete}{星座}{xing1zuo4}{9;10}
  \significado[张]{s.}{signo astrológico; constelação}
\end{verbete}

\begin{verbete}{猩猩}{xing1xing5}{12;12}
  \significado{s.}{orangotango}
\end{verbete}

\begin{verbete}{行}{xing2}{6}[Radical 行][Componentes 彳亍][Kangxi 144]
  \significado{adj.}{capaz; competente}
  \significado{expr.}{claro que sim; de acordo; está bem}
  \significado{interj.}{OK!}
  \significado{v.}{caminhar; ir; viajar; atuar}
  \veja{行}{hang2}
\end{verbete}

\begin{verbete}{行动}{xing2dong4}{6;6}
  \significado[个]{s.}{ação; operação}
  \significado{v.}{mover}
\end{verbete}

\begin{verbete}{行进}{xing2jin4}{6;7}
  \significado{s.}{avançar; movimentar-se para frente}
\end{verbete}

\begin{verbete}{行礼}{xing2li3}{6;5}
  \significado{v.}{saudar; fazer saudação}
\end{verbete}

\begin{verbete}{行李}{xing2li5}{6;7}
  \significado[件]{s.}{bagagem}
\end{verbete}

\begin{verbete}{行人}{xing2ren2}{6;2}
  \significado{s.}{transeunte; pedestre; viajante à pé}
\end{verbete}

\begin{verbete}{行驶}{xing2shi3}{6;8}
  \significado{v.}{viajar ao longo de uma rota (veículos, etc.)}
\end{verbete}

\begin{verbete}{行星}{xing2xing1}{6;9}
  \significado[颗]{s.}{planeta}
  \veja{惑星}{huo4xing1}
\end{verbete}

\begin{verbete}{形而上学}{xing2'er2shang4xue2}{7;6;3;8}
  \significado{s.}{metafísica}
\end{verbete}

\begin{verbete}{形容}{xing2rong2}{7;10}
  \significado{s.}{descrever}
  \significado{s.}{semblante (literário), aparência}
\end{verbete}

\begin{verbete}{形象}{xing2xiang4}{7;11}
  \significado[个]{s.}{imagem; forma; figura; vializuação}
\end{verbete}

\begin{verbete}{省}{xing3}{9}[Radical 目][Componentes 少目]
  \significado[个]{s.}{governadoria}
  \significado{v.}{examinar minuciosamente, refletir (sobre a conduta de alguém); realizar; fazer uma visita (aos pais ou idosos)}
  \veja{省}{sheng3}
\end{verbete}

\begin{verbete}{省悟}{xing3wu4}{9;10}
  \significado{v.}{voltar a si; constatar; ver a verdade; acordar para a realidade}
\end{verbete}

\begin{verbete}{兴趣}{xing4qu4}{6;15}
  \significado[个]{s.}{interesse (desejo de conhecer sobre alguma coisa ou coisa no qual está interessado); hobby}
\end{verbete}

\begin{verbete}{姓}{xing4}{8}[Radical 女][Componentes 女生]
  \significado[个]{s.}{sobrenome}
  \significado{v.}{ter o sobrenome}
\end{verbete}

\begin{verbete}{姓名}{xing4ming2}{8;6}
  \significado{s.}{nome completo}
\end{verbete}

\begin{verbete}{姓氏}{xing4shi4}{8;4}
  \significado{s.}{sobrenome}
\end{verbete}

\begin{verbete}{幸福}{xing4fu2}{8;13}
  \significado{adj.}{feliz; abençoado}
  \significado{s.}{felicidade}
\end{verbete}

\begin{verbete}{幸亏}{xing4kui1}{8;3}
  \significado{adv.}{felizmente}
\end{verbete}

\begin{verbete}{幸运}{xing4yun4}{8;7}
  \significado{adj.}{afortunado; feliz; sortudo}
\end{verbete}

\begin{verbete}{幸运抽奖}{xing4yun4chou1jiang3}{8;7;8;9}
  \significado{s.}{loteria; sorteio}
\end{verbete}

\begin{verbete}{幸运儿}{xing4yun4'er2}{8;7;2}
  \significado{s.}{pessoa de sorte}
\end{verbete}

\begin{verbete}{性}{xing4}{8}[Radical 心][Componentes 忄生]
  \significado{adj.}{sufixo formando adjetivo a partir de verbo}
  \significado[个]{s.}{natureza; carácter; propriedade; qualidade; atributo; sexualidade; sexo; gênero; essência}
  \significado{s.}{sufixo formando substantivo a partir de adjetivo}
\end{verbete}

\begin{verbete}{性侵}{xing4qin1}{8;9}
  \significado{s.}{agressão sexual}
  \significado{v.}{agredir sexualmente}
\end{verbete}

\begin{verbete}{性生活}{xing4sheng1huo2}{8;5;9}
  \significado{s.}{vida sexual}
\end{verbete}

\begin{verbete}{兄弟}{xiong1di4}{5;7}
  \significado{adj.}{fraternal; \emph{brotherly}}
  \significado{pron.}{eu, me (termo de uso humilde por homens em discurso público)}
  \significado[个]{s.}{irmãos; irmão mais novo; \emph{brothers}}
\end{verbete}

\begin{verbete}{匈奴}{xiong1nu2}{6;5}
  \significado*{s.}{Xiongnu, um povo da estepe oriental que criou um império que floresceu na época das dinastias Qin e Han}
\end{verbete}

\begin{verbete}{汹涌}{xiong1yong3}{7;10}
  \significado{adj.}{turbulento}
  \significado{v.}{aumentar ou emergir violentamente (oceano, rio, lago, etc.)}
\end{verbete}

\begin{verbete}{胸}{xiong1}{10}[Radical 肉][Componentes ⺼匈]
  \significado{s.}{peito; tórax}
\end{verbete}

\begin{verbete}{熊}{xiong2}{14}[Radical 火][Componentes 能灬]
  \significado*{s.}{sobrenome Xiong}
  \significado{adj.}{incapaz}
  \significado[把]{s.}{urso}
  \significado{v.}{repreender}
\end{verbete}

\begin{verbete}{熊猫}{xiong2mao1}{14;11}
  \significado[把,只]{s.}{panda gigante}
\end{verbete}

\begin{verbete}{休兵}{xiu1bing1}{6;7}
  \significado{s.}{armistício}
  \significado{v.}{cessar fogo}
\end{verbete}

\begin{verbete}{休憩}{xiu1qi4}{6;16}
  \significado{v.}{relaxar; descansar; dar um tempo}
\end{verbete}

\begin{verbete}{休息室}{xiu1xi1shi4}{6;10;9}
  \significado{s.}{saguão; salão}
\end{verbete}

\begin{verbete}{休息}{xiu1xi5}{6;10}
  \significado{s.}{descanço}
  \significado{v.}{descansar}
\end{verbete}

\begin{verbete}{休闲}{xiu1xian2}{6;7}
  \significado{s.}{ócio; lazer}
  \significado{v.}{desfrutar do lazer}
\end{verbete}

\begin{verbete}{休整}{xiu1zheng3}{6;16}
  \significado{v.}{militar:~descansar e reorganizar}
\end{verbete}

\begin{verbete}{修}{xiu1}{9}[Radical 人][Componentes 丨亻夂彡]
  \significado*{s.}{sobrenome Xiu}
  \significado{v.}{reparar; consertar; construir}
\end{verbete}

\begin{verbete}{修改}{xiu1gai3}{9;7}
  \significado{v.}{alterar; modificar; complementar}
\end{verbete}

\begin{verbete}{修规}{xiu1gui1}{9;8}
  \significado{s.}{plano de construção}
\end{verbete}

\begin{verbete}{绣}{xiu4}{10}[Radical 糸][Componentes 纟秀]
  \significado{s.}{bordado}
  \significado{v.}{bordar}
\end{verbete}

\begin{verbete}{臭}{xiu4}{10}[Radical ⾃][Componentes ⾃犬]
  \significado{s.}{olfato; cheiro ruim}
  \veja{臭}{chou4}
\end{verbete}

\begin{verbete}{袖}{xiu4}{10}[Radical 衣][Componentes 衤由]
  \significado{s.}{manga (de camisa, de camiseta, etc.)}
\end{verbete}

\begin{verbete}{虚伪}{xu1wei3}{11;6}
  \significado{adj.}{falso; hipócrita; artificial}
\end{verbete}

\begin{verbete}{需要}{xu1yao4}{14;9}
  \significado{s.}{necessidade}
  \significado{v.}{precisar; necessitar}
\end{verbete}

\begin{verbete}{许}{xu3}{6}[Radical 言][Componentes 讠午]
  \significado*{s.}{sobrenome Xu}
  \significado{adv.}{um pouco; talvez}
  \significado{v.}{permitir; prometer; elogiar}
\end{verbete}

\begin{verbete}{畜}{xu4}{10}[Radical ⽥][Componentes ⽞⽥]
  \significado{v.}{criar (animais)}
  \veja{畜}{chu4}
\end{verbete}

\begin{verbete}{宣布}{xuan1bu4}{9;5}
  \significado{v.}{declarar; anunciar; proclamar}
\end{verbete}

\begin{verbete}{宣扬}{xuan1yang2}{9;6}
  \significado{v.}{divulgar; anunciar; espalhar por toda parte}
\end{verbete}

\begin{verbete}{玄学}{xuan2xue2}{5;8}
  \significado{s.}{Escola Philosófica Wei e Jin amalgamando os ideais daoísta e confucionistas; tradução da metafísica (形而上学)}
  \veja{形而上学}{xing2'er2shang4xue2}
\end{verbete}

\begin{verbete}{悬挂}{xuan2gua4}{11;9}
  \significado{s.}{(veículo) suspensão}
  \significado{v.}{suspender}
\end{verbete}

\begin{verbete}{悬崖}{xuan2ya2}{11;11}
  \significado{s.}{precipício; penhasco}
\end{verbete}

\begin{verbete}{旋转}{xuan2zhuan3}{11;8}
  \significado{v.}{girar}
\end{verbete}

\begin{verbete}{选手}{xuan3shou3}{9;4}
  \significado{s.}{jogador; atleta; competidor}
\end{verbete}

\begin{verbete}{选择}{xuan3ze2}{9;8}
  \significado{s.}{escolha, opção, alternativa}
  \significado{v.}{selecionar, escolher}
\end{verbete}

\begin{verbete}{学}{xue2}{8}[Radical 子][Componentes 冖子𭕄]
  \significado{v.}{aprender; estudar}
\end{verbete}

\begin{verbete}{学费}{xue2fei4}{8;9}
  \significado[个]{s.}{mensalidade}
\end{verbete}

\begin{verbete}{学分}{xue2fen1}{8;4}
  \significado{s.}{créditos de um curso}
\end{verbete}

\begin{verbete}{学好}{xue2hao3}{8;6}
  \significado{v.}{seguir bons exemplos; aprender bem}
\end{verbete}

\begin{verbete}{学会}{xue2hui4}{8;6}
  \significado{s.}{instituto; associação (acadêmica); sociedade científica, douta ou erudita}
  \significado{v.}{aprender; dominar (um assunto)}
\end{verbete}

\begin{verbete}{学期}{xue2qi1}{8;12}
  \significado[个]{s.}{semestre}
\end{verbete}

\begin{verbete}{学生}{xue2sheng5}{8;5}
  \significado{s.}{estudante; aluno}
\end{verbete}

\begin{verbete}{学生证}{xue2sheng5zheng4}{8;5;7}
  \significado{s.}{cartão de identidade de estudante}
\end{verbete}

\begin{verbete}{学术}{xue2shu4}{8;5}
  \significado[个]{s.}{aprendizagem; ciência}
\end{verbete}

\begin{verbete}{学问}{xue2wen4}{8;6}
  \significado[个]{s.}{conhecimento; aprendizagem}
\end{verbete}

\begin{verbete}{学习}{xue2xi2}{8;3}
  \significado{v.}{estudar; aprender}
\end{verbete}

\begin{verbete}{学校}{xue2xiao4}{8;10}
  \significado{s.}{escola; instituição de ensino}
\end{verbete}

\begin{verbete}{学院}{xue2yuan4}{8;9}
  \significado[所]{s.}{instituto}
\end{verbete}

\begin{verbete}{雪}{xue3}{11}[Radical 雨][Componentes 雨彐]
  \significado*{s.}{sobrenome Xue}
  \significado[场]{s.}{neve}
\end{verbete}

\begin{verbete}{雪板}{xue3ban3}{11;8}
  \significado{s.}{prancha de \emph{snowboard}}
  \significado{v.}{praticar \textit{snowboard}}
\end{verbete}

\begin{verbete}{雪糕}{xue3gao1}{11;16}
  \significado{s.}{picolé}
\end{verbete}

\begin{verbete}{雪花}{xue3hua1}{11;7}
  \significado{s.}{floco de neve}
\end{verbete}

\begin{verbete}{雪葩}{xue3pa1}{11;12}
  \significado{s.}{sorvete}
\end{verbete}

\begin{verbete}{雪人}{xue3ren2}{11;2}
  \significado{s.}{boneco de neve; \emph{Yeti}}
\end{verbete}

\begin{verbete}{雪山}{xue3shan1}{11;3}
  \significado{s.}{montanha coberta de neve}
\end{verbete}

\begin{verbete}{雪鞋}{xue3xie2}{11;15}
  \significado[双]{s.}{sapatos de neve}
\end{verbete}

\begin{verbete}{血}{xue4}{6}[Radical 血][Componentes 皿丶]
  \significado[片]{s.}{sangue}
\end{verbete}

\begin{verbete}{血汗}{xue4han4}{6;6}
  \significado{s.}{(fig.) suor e labuta, trabalho duro}
\end{verbete}

\begin{verbete}{熏香}{xun1xiang1}{14;9}
  \significado{s.}{incenso}
\end{verbete}

\begin{verbete}{巡逻}{xun2luo2}{6;11}
  \significado{s.}{patrulha}
  \significado{v.}{patrulhar (polícia, exército ou marinha)}
\end{verbete}

%%%%% EOF %%%%%

%%%
%%% Y
%%%
%\section*{Y}
\addcontentsline{toc}{section}{Y}

\begin{verbete}{压岁钱}{ya1sui4qian2}{6;6;10}
  \significado{s.}{dinheiro da sorte; dinheiro dado às crianças como presente no Ano Novo Chinês}
\end{verbete}

\begin{verbete}{压韵}{ya1yun4}{6;13}
  \variante{押韵}{ya1yun4}
\end{verbete}

\begin{verbete}{押}{ya1}{8}
  \significado{v.}{deter sob custódia; escoltar e proteger; hipotecar; penhorar}
\end{verbete}

\begin{verbete}{押后}{ya1hou4}{8;6}
  \significado{v.}{encerrar; adiar}
\end{verbete}

\begin{verbete}{押金}{ya1jin1}{8;8}
  \significado{s.}{caução; sinal; depósito}
\end{verbete}

\begin{verbete}{押送}{ya1song4}{8;9}
  \significado{v.}{enviar sob escolta; transportar um detido}
\end{verbete}

\begin{verbete}{押运}{ya1yun4}{8;7}
  \significado{v.}{escoltar sob guarda; escoltar (bens ou fundos)}
\end{verbete}

\begin{verbete}{押韵}{ya1yun4}{8;13}
  \significado{v.}{rimar}
\end{verbete}

\begin{verbete}{押注}{ya1zhu4}{8;8}
  \significado{v.}{apostar}
\end{verbete}

\begin{verbete}{押租}{ya1zu1}{8;10}
  \significado{s.}{depósito de aluguel}
\end{verbete}

\begin{verbete}{鸭}{ya1}{10}
  \significado[只]{s.}{pato; gíria:~prostituto}
\end{verbete}

\begin{verbete}{鸭子}{ya1zi5}{10;3}
  \significado[只]{s.}{pato; gíria:~prostituto}
\end{verbete}

\begin{verbete}{牙}{ya2}{4}[92]
  \significado[颗]{s.}{dente; marfim}
\end{verbete}

\begin{verbete}{牙齿}{ya2chi3}{4;8}
  \significado{adv.}{dental}
  \significado[颗]{s.}{dente}
\end{verbete}

\begin{verbete}{牙膏}{ya2gao1}{4;14}
  \significado[管]{s.}{pasta de dente}
\end{verbete}

\begin{verbete}{牙行}{ya2hang2}{4;6}
  \significado{s.}{corretor; \emph{broker}}
\end{verbete}

\begin{verbete}{牙刷}{ya2shua1}{4;8}
  \significado[把]{s.}{escova de dentes}
\end{verbete}

\begin{verbete}{牙线}{ya2xian4}{4;8}
  \significado[条]{s.}{fio dental}
\end{verbete}

\begin{verbete}{牙医}{ya2yi1}{4;7}
  \significado{s.}{dentista}
\end{verbete}

\begin{verbete}{亚细亚洲}{ya4xi4ya4zhou1}{6;8;6;9}
  \significado*{s.}{Ásia}
  \veja{亚洲}{ya4zhou1}
\end{verbete}

\begin{verbete}{亚洲}{ya4zhou1}{6;9}
  \significado*{s.}{Ásia, abreviação de~亚细亚洲}
  \veja{亚细亚洲}{ya4xi4ya4zhou1}
\end{verbete}

\begin{verbete}{亚洲人}{ya4zhou1ren2}{6;9;2}
  \significado{s.}{asiático; nascido na Ásia}
\end{verbete}

\begin{verbete}{要不然}{yai4bu4ran2}{9;4;12}
  \significado{conj.}{de outra forma; se não; outro; ou}
\end{verbete}

\begin{verbete}{严重}{yan2zhong4}{7;9}
  \significado{adj.}{crítico; grave; sério; severo}
\end{verbete}

\begin{verbete}{严重打伤}{yan2zhong4·da3·shang1}{7;9;5;6}
  \significado{s.}{gravemente ferido}
\end{verbete}

\begin{verbete}{严重地}{yan2zhong4·di4}{7;9;6}
  \significado{adv.}{seriamente; gravemente}
\end{verbete}

\begin{verbete}{严重关切}{yan2zhong4guan1qie4}{7;9;6;4}
  \significado{s.}{preocupação séria}
\end{verbete}

\begin{verbete}{严重后果}{yan2zhong4hou4guo3}{7;9;6;8}
  \significado{s.}{consequências sérias; repercursões graves}
\end{verbete}

\begin{verbete}{严重破坏}{yan2zhong4·po4huai4}{7;9;10;7}
  \significado{s.}{destruição grave}
\end{verbete}

\begin{verbete}{严重伤害}{yan2zhong4·shang1hai4}{7;9;6;10}
  \significado{s.}{ferimento grave}
\end{verbete}

\begin{verbete}{严重危害}{yan2zhong4wei1hai4}{7;9;6;10}
  \significado{s.}{danos graves}
\end{verbete}

\begin{verbete}{严重问题}{yan2zhong4wen4ti2}{7;9;6;15}
  \significado{s.}{problema sério}
\end{verbete}

\begin{verbete}{严重性}{yan2zhong4xing4}{7;9;8}
  \significado{s.}{seriedade; gravidade}
\end{verbete}

\begin{verbete}{颜色}{yan2se4}{15;6}
  \significado{s.}{cor; pigmento; tintura}
\end{verbete}

\begin{verbete}{眼}{yan3}{11}
  \significado{p.c.}{para grandes coisas ocas (poços, fogões, panelas, etc.)}
  \significado[只,双]{s.}{ponto crucial (de um assunto); olho; pequeno buraco}
\end{verbete}

\begin{verbete}{眼柄}{yan3bing3}{11;9}
  \significado{s.}{pedúnculo ocular (de crustáceo, etc.)}
\end{verbete}

\begin{verbete}{眼镜}{yan3jing4}{11;16}
  \significado[副]{s.}{óculos}
\end{verbete}

\begin{verbete}{眼睛}{yan3jing5}{11;13}
  \significado[只,双]{s.}{olho(s)}
\end{verbete}

\begin{verbete}{眼泪}{yan3lei4}{11;8}
  \significado[滴]{s.}{choro; lágrimas}
\end{verbete}

\begin{verbete}{眼证}{yan3zheng4}{11;7}
  \significado{s.}{testemunha ocular}
\end{verbete}

\begin{verbete}{演员}{yan3yuan2}{14;7}
  \significado{s.}{ator; artista}
\end{verbete}

\begin{verbete}{扬雄}{yang2xiong2}{6;12}
  \significado*{s.}{Yang Xiong (53 AC-18 DC), estudioso, poeta e lexicógrafo, autor do primeiro dicionário de dialeto chinês 方言}
  \veja{方言}{fang1yan2}
\end{verbete}

\begin{verbete}{阳}{yang2}{6}
  \significado*{s.}{Yang (o princípio positivo de Yin e Yang)}
  \significado{s.}{positivo (eletricida de); sol}
  \veja{阴}{yin1}
  \veja{阴阳}{yin1yang2}
\end{verbete}

\begin{verbete}{洋葱}{yang2cong1}{9;12}
  \significado{s.}{cebola}
\end{verbete}

\begin{verbete}{养}{yang3}{9}
  \significado{v.}{criar (animais ou filhos), plantar (flores), etc.; dar a luz}
\end{verbete}

\begin{verbete}{养分}{yang3fen4}{9;4}
  \significado{s.}{nutriente}
\end{verbete}

\begin{verbete}{养料}{yang3liao4}{9;10}
  \significado{s.}{nutrição}
\end{verbete}

\begin{verbete}{样}{yang4}{10}
  \significado{s.}{aparência; forma; modelo}
\end{verbete}

\begin{verbete}{样品}{yang4pin3}{10;9}
  \significado{s.}{amostra; espécime}
\end{verbete}

\begin{verbete}{样儿}{yang4r5}{10;2}
  \significado{s.}{aparência; forma; modelo}
  \veja{样子}{yang4zi5}
\end{verbete}

\begin{verbete}{样样}{yang4yang4}{10;10}
  \significado{adv.}{todos os tipos}
\end{verbete}

\begin{verbete}{样章}{yang4zhang1}{10;11}
  \significado{s.}{capítulo de amostra}
\end{verbete}

\begin{verbete}{样子}{yang4zi5}{10;3}
  \significado{s.}{aparência; forma; modelo}
  \veja{样儿}{yang4r5}
\end{verbete}

\begin{verbete}{要}{yao1}{9}
  \significado{v.}{demandar; coagir}
  \veja{要}{yao4}
\end{verbete}

\begin{verbete}{要挟}{yao1xie2}{9;9}
  \significado{v.}{chantagear; ameaçar}
\end{verbete}

\begin{verbete}{腰}{yao1}{13}
  \significado{s.}{cintura}
\end{verbete}

\begin{verbete}{腰包}{yao1bao1}{13;5}
  \significado{s.}{pochete; bolso}
\end{verbete}

\begin{verbete}{腰椎}{yao1zhui1}{13;12}
  \significado{s.}{vértebra lombar (espinha dorsal inferior)}
\end{verbete}

\begin{verbete}{药}{yao4}{9}
  \significado[种,服,味]{s.}{medicamento; remédio; droga}
\end{verbete}

\begin{verbete}{药补}{yao4bu3}{9;7}
  \significado{s.}{suplemento dietético medicinal que ajuda a melhorar a saúde}
\end{verbete}

\begin{verbete}{药典}{yao4dian3}{9;8}
  \significado{s.}{farmacopéia}
\end{verbete}

\begin{verbete}{药罐}{yao4guan4}{9;23}
  \significado{s.}{frasco de remédio}
\end{verbete}

\begin{verbete}{药片}{yao4pian4}{9;4}
  \significado[片]{s.}{uma pílula ou comprimido (remédio)}
\end{verbete}

\begin{verbete}{药品}{yao4pin3}{9;9}
  \significado{s.}{medicamento; remédio; droga}
\end{verbete}

\begin{verbete}{药签}{yao4qian1}{9;13}
  \significado{s.}{cotonete médico}
\end{verbete}

\begin{verbete}{药膳}{yao4shan4}{9;16}
  \significado{s.}{dieta medicinal}
\end{verbete}

\begin{verbete}{药丸}{yao4wan2}{9;3}
  \significado[粒]{s.}{pílula}
\end{verbete}

\begin{verbete}{要}{yao4}{9}
  \significado{v./v.o.}{querer; precisar}
  \veja{要}{yao1}
\end{verbete}

\begin{verbete}{要不}{yao4bu4}{9;4}
  \significado{conj.}{de outra forma; se não; outro; ou}
\end{verbete}

\begin{verbete}{要点}{yao4dian3}{9;9}
  \significado{s.}{pontos principais; essencial}
\end{verbete}

\begin{verbete}{要好}{yao4hao3}{9;6}
  \significado{v.}{ser amigos íntimos; estar em boas condições}
\end{verbete}

\begin{verbete}{要谎}{yao4huang3}{9;11}
  \significado{v.}{pedir um preço enorme (como primeiro passo de negociação)}
\end{verbete}

\begin{verbete}{要求}{yao4qiu2}{9;7}
  \significado[点]{s.}{requerimento}
  \significado{v.}{pedir; exigir; solicitar; fazer uma reivindicação}
\end{verbete}

\begin{verbete}{要强}{yao4quiang2}{9;12}
  \significado{adj.}{ansioso para se destacar; ansioso para progredir na vida; obs tinado}
\end{verbete}

\begin{verbete}{要是}{yao4shi5}{9;9}
  \significado{conj.}{se; no caso de; no evento de; supondo que}
\end{verbete}

\begin{verbete}{要是……的话}{yao4shi5·de5hua5}{9;9;8;8}
  \significado{conj.}{se\dots no caso de}
\end{verbete}

\begin{verbete}{要死}{yao4si3}{9;6}
  \significado{adv.}{extremamente; muito}
\end{verbete}

\begin{verbete}{要义}{yao4yi4}{9;3}
  \significado{s.}{resumo; o essencial}
\end{verbete}

\begin{verbete}{钥匙}{yao4shi5}{9;11}
  \significado[把]{s.}{chave}
\end{verbete}

\begin{verbete}{钥匙洞孔}{yao4shi5dong4kong3}{9;11;9;4}
  \significado{s.}{buraco da fechadura}
\end{verbete}

\begin{verbete}{钥匙卡}{yao4shi5ka3}{9;11;5}
  \significado{s.}{cartão de acesso}
\end{verbete}

\begin{verbete}{钥匙孔}{yao4shi5kong3}{9;11;4}
  \significado{s.}{buraco da fechadura}
\end{verbete}

\begin{verbete}{钥匙圈}{yao4shi5quan1}{9;11;11}
  \significado{s.}{chaveiro}
\end{verbete}

\begin{verbete}{爷爷}{ye2ye5}{6;6}
  \significado[个]{s.}{avô (paterno)}
\end{verbete}

\begin{verbete}{也}{ye3}{3}
  \significado*{s.}{sobrenome Ye}
  \significado{adv.}{também; (em frases negativas) nem, tampouco}
\end{verbete}

\begin{verbete}{也就是}{ye3jiu4shi4}{3;12;9}
  \significado{adv.}{i.e.; isso é; ou seja}
\end{verbete}

\begin{verbete}{也就是说}{ye3jiu4shi4shuo1}{3;12;9;9}
  \significado{adv.}{em outras palavras; então; isto é; por isso}
\end{verbete}

\begin{verbete}{也许}{ye3xu3}{3;6}
  \significado{adv.}{possivelmente; talvez}
\end{verbete}

\begin{verbete}{也有今天}{ye3you3jin1tian1}{3;6;4;4}
  \significado{expr.}{obter apenas o que merece; todo cachorro tem seu dia; obter a sua parte (coisas boas ou ruins); servir alguém bem}
\end{verbete}

\begin{verbete}{夜}{ye4}{8}
  \significado{p.t.}{noite}
\end{verbete}

\begin{verbete}{夜店}{ye4dian4}{8;8}
  \significado{s.}{boate; \emph{nightclub}}
\end{verbete}

\begin{verbete}{夜里}{ye4li5}{8;7}
  \significado{p.t.}{à noite; durante a noite; período noturno}
\end{verbete}

\begin{verbete}{夜幕}{ye4mu4}{8;13}
  \significado{s.}{cortina da noite}
\end{verbete}

\begin{verbete}{夜鸟}{ye4niao3}{8;5}
  \significado{s.}{ave noturna}
\end{verbete}

\begin{verbete}{夜生活}{ye4sheng1huo2}{8;5;9}
  \significado{s.}{vida noturna}
\end{verbete}

\begin{verbete}{夜晚}{ye4wan3}{8;11}
  \significado[个]{s.}{noite}
\end{verbete}

\begin{verbete}{夜夜}{ye4ye4}{8;8}
  \significado{adv.}{toda noite}
\end{verbete}

\begin{verbete}{一}{yi1}[(quando usado sozinho)]{1}[1]
  \significado{num.}{um, 1; pronunciado como \dictpinyin{yao1} quando dito número a número}
  \veja{一}{yi2}
  \veja{一}{yi4}
\end{verbete}

\begin{verbete}{一……就……}{yi1·jiu4}{1;12}
  \significado{expr.}{logo que; uma vez que}
\end{verbete}

\begin{verbete}{一直}{yi1zhi2}{1;8}
  \significado{adv.}{direto; sempre em frente; o tempo todo; sempre; constantemente}
\end{verbete}

\begin{verbete}{伊马姆}{yi1ma3mu3}{6;3;8}
  \significado*{s.}{Islã}
  \veja{伊玛目}{yi1ma3mu4}
  \veja{伊曼}{yi1man4}
\end{verbete}

\begin{verbete}{伊玛目}{yi1ma3mu4}{6;7;5}
  \significado*{s.}{Islã}
  \veja{伊马姆}{yi1ma3mu3}
  \veja{伊曼}{yi1man4}
\end{verbete}

\begin{verbete}{伊曼}{yi1man4}{6;11}
  \significado*{s.}{Islã}
  \veja{伊马姆}{yi1ma3mu3}
  \veja{伊玛目}{yi1ma3mu4}
\end{verbete}

\begin{verbete}{衣}{yi1}{6}[145]
  \significado[件]{s.}{roupa}
  \veja{衣}{yi4}
\end{verbete}

\begin{verbete}{衣服}{yi1fu5}{6;8}
  \significado[件,套]{s.}{roupa; vestuário}
\end{verbete}

\begin{verbete}{衣柜}{yi1gui4}{6;8}
  \significado[个]{s.}{armário; guarda-roupa}
\end{verbete}

\begin{verbete}{衣甲}{yi1jia3}{6;5}
  \significado{s.}{armadura}
\end{verbete}

\begin{verbete}{医}{yi1}{7}
  \significado{s.}{médico; medicina}
  \significado{v.}{curar; tratar}
\end{verbete}

\begin{verbete}{医生}{yi1sheng1}{7;5}
  \significado[个,位,名]{s.}{médico; clínico}
\end{verbete}

\begin{verbete}{医院}{yi1yuan5}{7;9}
  \significado[所,家,座]{s.}{hospital}
\end{verbete}

\begin{verbete}{依然}{yi1ran2}{8;12}
  \significado{adv.}{como era antes; ainda}
\end{verbete}

\begin{verbete}{遗案}{yi1'an4}{12;10}
  \significado{s.}{lei: caso não resolvido}
\end{verbete}

\begin{verbete}{毉}{yi1}{18}
  \variante{医}{yi1}
\end{verbete}

\begin{verbete}{一}{yi2}[(antes de quarto tom)]{1}
  \significado{num.}{um, 1; um (artigo)}
  \veja{一}{yi1}
  \veja{一}{yi4}
\end{verbete}

\begin{verbete}{一半}{yi2ban4}{1;5}
  \significado{adj.}{metade}
\end{verbete}

\begin{verbete}{一道}{yi2dao4}{1;12}
  \significado{adv.}{juntos; ao lado}
\end{verbete}

\begin{verbete}{一定}{yi2ding4}{1;8}
  \significado{adv.}{certamente; definitivamente}
\end{verbete}

\begin{verbete}{一个样}{yi2ge5yang4}{1;3;10}
  \veja{一样}{yi2yang4}
\end{verbete}

\begin{verbete}{一共}{yi2gong4}{1;6}
  \significado{adv.}{completamente; no total; no todo; em suma}
\end{verbete}

\begin{verbete}{一会儿}{yi2hui4r5}{1;6;2}
  \significado{adv.}{daqui a pouco tempo; pouco tempo}
\end{verbete}

\begin{verbete}{一块}{yi2kuai4}{1;7}
  \significado{adv.}{(principalmente mandarim) juntos}
\end{verbete}

\begin{verbete}{一下}{yi2xia4}{1;3}
  \significado{adv.}{em um curto tempo; rapidamente}
\end{verbete}

\begin{verbete}{一样}{yi2yang4}{1;10}
  \significado{adj.}{igual; mesmo}
\end{verbete}

\begin{verbete}{遗产}{yi2chan3}{12;6}
  \significado[笔]{s.}{legado; herança}
\end{verbete}

\begin{verbete}{遗骸}{yi2hai2}{12;15}
  \significado{v.}{restos mortais}
\end{verbete}

\begin{verbete}{遗憾}{yi2han4}{12;16}
  \significado{v.}{ter pena de; lamentar}
\end{verbete}

\begin{verbete}{遗迹}{yi2ji4}{12;9}
  \significado{s.}{vestígios históricos; remanescente; vestígio}
\end{verbete}

\begin{verbete}{遗落}{yi2lou4}{12;12}
  \significado{v.}{esquecer; deixar para trás (inadvertidamente); deixar de fora; omitir}
\end{verbete}

\begin{verbete}{遗男}{yi2nan2}{12;7}
  \significado{s.}{órfão; filho póstumo}
\end{verbete}

\begin{verbete}{遗嘱}{yi2zhu3}{12;15}
  \significado{s.}{testamento}
\end{verbete}

\begin{verbete}{颐和园}{yi2he2yuan2}{13;8;7}
  \significado*{s.}{Palácio de Verão}
\end{verbete}

\begin{verbete}{已}{yi3}{3}
  \significado{adv.}{já; após; então}
\end{verbete}

\begin{verbete}{已故}{yi3gu4}{3;9}
  \significado{adj.}{morto; atrasado}
\end{verbete}

\begin{verbete}{已婚}{yi3hun1}{3;11}
  \significado{adj.}{casado}
\end{verbete}

\begin{verbete}{已经}{yi3jing1}{3;8}
  \significado{adv.}{já}
\end{verbete}

\begin{verbete}{已久}{yi3jiu3}{3;3}
  \significado{adv.}{já faz muito tempo}
\end{verbete}

\begin{verbete}{已灭}{yi3mie4}{3;5}
  \significado{adj.}{extinto}
\end{verbete}

\begin{verbete}{已然}{yi3ran2}{3;12}
  \significado{adv.}{já; já ser assim}
\end{verbete}

\begin{verbete}{已知}{yi3zhi1}{3;8}
  \significado{v.}{conhecido (ter ciência)}
\end{verbete}

\begin{verbete}{以便}{yi3bian4}{4;9}
  \significado{conj.}{a fim de; para que; assim como}
\end{verbete}

\begin{verbete}{以此}{yi3ci3}{4;6}
  \significado{adv.}{devido a esta; deste modo; por isso; com isso}
\end{verbete}

\begin{verbete}{以后}{yi3hou4}{4;6}
  \significado{p.t.}{depois de; depois; após}
\end{verbete}

\begin{verbete}{以及}{yi3ji2}{4;3}
  \significado{conj.}{assim como; juntamente como}
\end{verbete}

\begin{verbete}{以来}{yi3lai2}{4;7}
  \significado{prep.}{desde (um evento anterior)}
\end{verbete}

\begin{verbete}{以免}{yi3mian3}{4;7}
  \significado{conj.}{para evitar isso}
\end{verbete}

\begin{verbete}{以期}{yi3qi1}{4;12}
  \significado{v.}{tentando; esperando; esperando por}
\end{verbete}

\begin{verbete}{以前}{yi3qian2}{4;9}
  \significado{p.t.}{antes de; antes}
\end{verbete}

\begin{verbete}{以求}{yi3qiu2}{4;7}
  \significado{adv.}{a fim de}
\end{verbete}

\begin{verbete}{以至}{yi3zhi4}{4;6}
  \significado{adv.}{até}
  \significado{conj.}{a tal ponto que...}
  \veja{以至于}{yi3zhi4yu2}
\end{verbete}

\begin{verbete}{以至于}{yi3zhi4yu2}{4;6;3}
  \significado{adv.}{até}
  \significado{conj.}{na medida em que\dots}
  \veja{以至}{yi3zhi4}
\end{verbete}

\begin{verbete}{一}{yi4}{1}
  \significado{adv.}{uma vez; assim que; ao}
  \significado{num.}{um, 1; um (artigo)}
  \veja{一}{yi1}
  \veja{一}{yi2}
\end{verbete}

\begin{verbete}{一般}{yi4ban1}{1;10}
  \significado{adj.}{geral; comum; normal}
  \significado{adv.}{normalmente}
\end{verbete}

\begin{verbete}{一点儿}{yi4dian3r5}{1;9;2}
  \significado{adv.}{um pouco (``adj.$+$一点儿'' ou ``一点儿$+$s.''); um ponto}
\end{verbete}

\begin{verbete}{一齐}{yi4qi2}{1;6}
  \significado{adv.}{tudo ao mesmo tempo; em uníssono; junto}
\end{verbete}

\begin{verbete}{一起}{yi4qi3}{1;10}
  \significado{adv.}{juntamente; em conjunto; no mesmo lugar; completamente; em todos}
\end{verbete}

\begin{verbete}{一时}{yi4shi2}{1;7}
  \significado{adv.}{por pouco tempo; por um tempo; temporariamente; momentaneamente; uma vez; de tempos em tempos; ocasionalmente}
\end{verbete}

\begin{verbete}{一同}{yi4tong2}{1;6}
  \significado{adv.}{juntos, ao mesmo tempo}
\end{verbete}

\begin{verbete}{一些}{yi4xie1}{1;8}
  \significado{pron.}{uns; alguns}
\end{verbete}

\begin{verbete}{一直}{yi4zhi2}{1;8}
  \significado{adv.}{diretamente; sempre}
\end{verbete}

\begin{verbete}{亿}{yi4}{3}
  \significado{num.}{cem milhões, 100.000.000}
\end{verbete}

\begin{verbete}{以便}{yi4bian4}{4;9}
  \significado{conj.}{para que}
\end{verbete}

\begin{verbete}{亦}{yi4}{6}
  \significado{adv.}{também; igualmente; apenas; embora; já}
\end{verbete}

\begin{verbete}{异常}{yi4chang2}{6;11}
  \significado{adj.}{extraordinário; anormal}
  \significado{adv.}{extraordinariamente; excepcionalmente}
  \significado{s.}{anormalidade}
\end{verbete}

\begin{verbete}{衣}{yi4}{6}
  \significado{v.}{vestir; vestir-se}
  \veja{衣}{yi1}
\end{verbete}

\begin{verbete}{意见}{yi4jian4}{13;4}
  \significado[点,条]{s.}{reclamação; ideia; objeção; opinião; sugestão}
\end{verbete}

\begin{verbete}{意思}{yi4si5}{13;9}
  \significado[个]{s.}{interesse}
\end{verbete}

\begin{verbete}{意外}{yi4wai4}{13;5}
  \significado{adj.}{inesperado}
  \significado[个]{s.}{acidente}
\end{verbete}

\begin{verbete}{意义}{yi4yi4}{13;3}
  \significado[个]{s.}{importância; significado; senso; ; desejo; força de vontade}
\end{verbete}

\begin{verbete}{意译}{yi4yi4}{13;7}
  \significado{s.}{tradução livre; significado (de expressão estrangeira); paráfrase; tradução do significado (em oposição à tradução literal)}
  \veja{直译}{zhi2yi4}
\end{verbete}

\begin{verbete}{意指}{yi4zhi3}{13;9}
  \significado{v.}{implicar; significar}
\end{verbete}

\begin{verbete}{意志}{yi4zhi4}{13;7}
  \significado[个]{s.}{determinação; desejo; força de vontade}
\end{verbete}

\begin{verbete}{因此}{yin1ci3}{6;6}
  \significado{conj.}{então; portanto; por esta razão; consequentemente}
\end{verbete}

\begin{verbete}{因而}{yin1'er2}{6;6}
  \significado{conj.}{então; portanto; por esta razão; consequentemente}
\end{verbete}

\begin{verbete}{因为}{yin1wei4}{6;4}
  \significado{conj.}{porque}
\end{verbete}

\begin{verbete}{因为……所以……}{yin1wei4·suo3yi3}{6;4;8;4}
  \significado{conj.}{[porque]\dots portanto\dots}
\end{verbete}

\begin{verbete}{阴}{yin1}{6}
  \significado*{s.}{Yin (o princípio negativo de Yin e Yang); sobrenome Yin}
  \significado{adj.}{nublado; sombrio; escondido; implícito}
  \significado{s.}{negativo (eletricidade); lua}
  \veja{阳}{yang2}
  \veja{阴阳}{yin1yang2}
\end{verbete}

\begin{verbete}{阴天}{yin1tian1}{6;4}
  \significado{adj.}{céu nublado; céu cinzento}
\end{verbete}

\begin{verbete}{阴阳}{yin1yang2}{6;6}
  \significado*{s.}{Yin e Yang}
  \veja{阳}{yang2}
  \veja{阴}{yin1}
\end{verbete}

\begin{verbete}{音乐}{yin1yue4}{9;5}
  \significado[张,曲,段]{s.}{música}
\end{verbete}

\begin{verbete}{音乐光碟}{yin1yue4guang1die2}{9;5;6;14}
  \significado{s.}{CD de música}
\end{verbete}

\begin{verbete}{音乐会}{yin1yue4hui4}{9;5;6}
  \significado[场]{s.}{concerto}
\end{verbete}

\begin{verbete}{音乐家}{yin1yue4jia1}{9;5;10}
  \significado{s.}{músico}
\end{verbete}

\begin{verbete}{音乐节}{yin1yue4jie2}{9;5;5}
  \significado{s.}{festival de música}
\end{verbete}

\begin{verbete}{音乐厅}{yin1yue4ting1}{9;5;4}
  \significado{s.}{auditório; teatro; \emph{concert hall}}
\end{verbete}

\begin{verbete}{音乐学}{yin1yue4xue2}{9;5;8}
  \significado{s.}{musicologia}
\end{verbete}

\begin{verbete}{音乐学院}{yin1yue4xue2yuan4}{9;5;8;9}
  \significado{s.}{conservatório; academia de música}
\end{verbete}

\begin{verbete}{音乐院}{yin1yue4yuan4}{9;5;9}
  \significado{s.}{conservatório; instituto de música}
\end{verbete}

\begin{verbete}{银行}{yin2hang2}{11;6}
  \significado[家,个]{s.}{banco; agência bancária}
\end{verbete}

\begin{verbete}{银色}{yin2se4}{11;6}
  \significado{s.}{prateado}
\end{verbete}

\begin{verbete}{饮料}{yin3liao4}{7;10}
  \significado{s.}{bebida}
\end{verbete}

\begin{verbete}{应该}{ying1gai1}{7;8}
  \significado{v.}{dever; ter de}
\end{verbete}

\begin{verbete}{英国}{ying1guo2}{8;8}
  \significado*{s.}{Reino Unido}
\end{verbete}

\begin{verbete}{英国人}{ying1guo2ren2}{8;8;2}
  \significado{s.}{inglês; nascido no Reino Unido}
\end{verbete}

\begin{verbete}{英文}{ying1wen2}{8;4}
  \significado{s.}{inglês, língua inglesa}
\end{verbete}

\begin{verbete}{英语}{ying1yu3}{8;9}
  \significado{s.}{inglês, língua inglesa}
\end{verbete}

\begin{verbete}{应用程序}{ying4yong4cheng2xu4}{7;5;12;7}
  \significado{s.}{aplicativo; programa de computador}
\end{verbete}

\begin{verbete}{应用程序编程接口}{ying4yong4cheng2xu4bian1cheng2jie1kou3}{7;5;12;7;12;12;11;3}
  \significado{s.}{API (\emph{application programming interface})}
  \veja{应用程序接口}{ying4yong4cheng2xu4jie1kou3}
\end{verbete}

\begin{verbete}{应用程序接口}{ying4yong4cheng2xu4jie1kou3}{7;5;12;7;11;3}
  \significado{s.}{API (\emph{application programming interface})}
  \veja{应用程序编程接口}{ying4yong4cheng2xu4bian1cheng2jie1kou3}
\end{verbete}

\begin{verbete}{永远}{yong2yuan3}{5;7}
  \significado{adv.}{para sempre, sempre; permanentemente}
\end{verbete}

\begin{verbete}{用}{yong4}{5}[101]
  \significado{v.}{usar}
\end{verbete}

\begin{verbete}{用处}{yong4chu5}{5;5}
  \significado[个]{s.}{usabilidade; utilidade}
\end{verbete}

\begin{verbete}{用料}{yong4liao4}{5;10}
  \significado{s.}{ingredientes; materiais}
\end{verbete}

\begin{verbete}{优}{you1}{6}
  \significado{adj.}{excelente; superior}
\end{verbete}

\begin{verbete}{优等}{you1deng3}{6;12}
  \significado{adj.}{excelente; de primeira linha; alta classe; da mais alta ordem, superior}
\end{verbete}

\begin{verbete}{优点}{you1dian3}{6;9}
  \significado[个,项]{s.}{vantagem; benefício; mérito; ponto forte}
\end{verbete}

\begin{verbete}{优格}{you1ge2}{6;10}
  \significado{s.}{iogurte}
\end{verbete}

\begin{verbete}{优厚}{you1hou4}{6;9}
  \significado{adj.}{generoso}
\end{verbete}

\begin{verbete}{优伶}{you1ling2}{6;7}
  \significado{s.}{ator}
\end{verbete}

\begin{verbete}{优美}{you1mei3}{6;9}
  \significado{adj.}{gracioso; fino; elegante}
\end{verbete}

\begin{verbete}{优盘}{you1pan2}{6;11}
  \significado{s.}{unidade de memória USB}
  \veja{闪存盘}{shan3cun2pan2}
\end{verbete}

\begin{verbete}{优先}{you1xian1}{6;6}
  \significado{v.}{ter prioridade; ter precedência}
\end{verbete}

\begin{verbete}{优秀}{you1xiu4}{6;7}
  \significado{adj.}{excelente; fora do comum}
\end{verbete}

\begin{verbete}{优选}{you1xuan3}{6;9}
  \significado{v.}{otimizar}
\end{verbete}

\begin{verbete}{优于}{you1yu2}{6;3}
  \significado{v.}{superar}
\end{verbete}

\begin{verbete}{优裕}{you1yu4}{6;12}
  \significado{adj.}{abundante; bastante}
  \significado{s.}{abundância}
\end{verbete}

\begin{verbete}{优质}{you1zhi4}{6;8}
  \significado{adj.}{excelente qualidade}
\end{verbete}

\begin{verbete}{尤其}{you2qi2}{4;8}
  \significado{adv.}{especialmente; particularmente}
\end{verbete}

\begin{verbete}{邮包}{you2bao1}{7;5}
  \significado{s.}{encomenda postal}
\end{verbete}

\begin{verbete}{邮递}{you2di4}{7;10}
  \significado{v.}{enviar por correio}
\end{verbete}

\begin{verbete}{邮电}{you2dian4}{7;5}
  \significado*{s.}{Correios e Telecomunicações}
\end{verbete}

\begin{verbete}{邮费}{you2fei4}{7;9}
  \significado{s.}{postagem}
  \significado{v.}{postar}
\end{verbete}

\begin{verbete}{邮件}{you2jian4}{7;6}
  \significado{s.}{correspondência; \emph{e-mail}}
\end{verbete}

\begin{verbete}{邮局}{you2ju4}{7;7}
  \significado[家,个]{s.}{correio; agência dos correios}
\end{verbete}

\begin{verbete}{邮迷}{you2mi2}{7;9}
  \significado{s.}{filatelista; colecionador de selos}
\end{verbete}

\begin{verbete}{邮票}{you2piao4}{7;11}
  \significado[枚,张]{s.}{selo postal}
\end{verbete}

\begin{verbete}{邮市}{you2shi4}{7;5}
  \significado{s.}{mercado postal}
\end{verbete}

\begin{verbete}{邮资}{you2zi1}{7;10}
  \significado{s.}{postagem}
\end{verbete}

\begin{verbete}{游艇}{you2ting3}{12;12}
  \significado[只]{s.}{barcaça; iate}
\end{verbete}

\begin{verbete}{游泳}{you2yong3}{12;8}
  \significado{v.+compl.}{nadar}
\end{verbete}

\begin{verbete}{游泳池}{you2yong3chi2}{12;8;6}
  \significado[场]{s.}{piscina}
  \veja{游泳馆}{you2yong3guan3}
\end{verbete}

\begin{verbete}{游泳馆}{you2yong3guan3}{12;8;11}
  \significado[场]{s.}{piscina}
  \veja{游泳池}{you2yong3chi2}
\end{verbete}

\begin{verbete}{游泳镜}{you2yong3jing4}{12;8;16}
  \significado{s.}{óculos de natação}
\end{verbete}

\begin{verbete}{游泳衣}{you2yong3yi1}{12;8;6}
  \significado{s.}{roupa de banho}
\end{verbete}

\begin{verbete}{有}{you3}{6}
  \significado{v.}{ter; haver; existir}
\end{verbete}

\begin{verbete}{有的}{you3de5}{6;8}
  \significado{pron.}{algum, alguns}
\end{verbete}

\begin{verbete}{有的时候}{you3de5shi2hou5}{6;8;7;10}
  \significado{expr.}{às vezes; de vez em quando; de quando em quando}
  \veja{有时}{you3shi2}
  \veja{有时候}{you3shi2hou5}
\end{verbete}

\begin{verbete}{有点儿}{you3dian3r5}{6;9;2}
  \significado{adv.}{um pouco (``有点儿+s. ou v. mental'')}
\end{verbete}

\begin{verbete}{有名}{you3ming2}{6;6}
  \significado{adj.}{famoso; conhecido}
\end{verbete}

\begin{verbete}{有名无实}{you3ming2wu2shi2}{6;6;4;8}
  \significado{v.}{literal:~tem um nome, mas não tem realidade; existe apenas no nome}
\end{verbete}

\begin{verbete}{有时}{you3shi2}{6;7}
  \significado{expr.}{às vezes; de vez em quando; de quando em quando}
  \veja{有的时候}{you3de5shi2hou5}
  \veja{有时候}{you3shi2hou5}
\end{verbete}

\begin{verbete}{有时候}{you3shi2hou5}{6;7;10}
  \significado{expr.}{às vezes; de vez em quando; de quando em quando}
  \veja{有的时候}{you3de5shi2hou5}
  \veja{有时}{you3shi2}
\end{verbete}

\begin{verbete}{有意思}{you3yi4si5}{6;13;9}
  \significado{adj.}{interessante; agradável; significativo; divertido}
\end{verbete}

\begin{verbete}{有用}{you3yong4}{6;5}
  \significado{adj.}{útil}
\end{verbete}

\begin{verbete}{又}{you4}{2}[29]
  \significado{adv.}{mais uma vez; (usado para dar ênfase) de qualquer maneira; e ainda; e também}
\end{verbete}

\begin{verbete}{又称}{you4cheng1}{2;10}
  \significado{s.}{também conhecido como}
\end{verbete}

\begin{verbete}{又及}{you4ji2}{2;3}
  \significado{s.}{P.S.; \emph{postscript}}
\end{verbete}

\begin{verbete}{又名}{you4ming2}{2;6}
  \significado{s.}{também conhecido como; nome alternativo}
\end{verbete}

\begin{verbete}{又一次}{you4yi2ci4}{2;1;6}
  \significado{adv.}{outra vez; mais uma vez; de novo}
\end{verbete}

\begin{verbete}{右}{you4}{5}
  \significado{p.l.}{direita}
  \significado{s.}{política:~a Direita}
\end{verbete}

\begin{verbete}{右边}{you4bian5}{5;5}
  \significado{p.l.}{à direita; ao lado direito}
\end{verbete}

\begin{verbete}{右侧}{you4ce4}{5;8}
  \significado{p.l.}{lateral direita; lado direito}
\end{verbete}

\begin{verbete}{右面}{you4mian4}{5;9}
  \significado{p.l.}{lado direito}
\end{verbete}

\begin{verbete}{右倾}{you4qing1}{5;10}
  \significado{adj.}{conservador; reacionário}
\end{verbete}

\begin{verbete}{右手}{you4shou3}{5;4}
  \significado{s.}{mão direita; lado direito}
\end{verbete}

\begin{verbete}{右袒}{you4tan3}{5;10}
  \significado{v.}{ser tendencioso; ser parcial; favorecer um lado; tomar partido}
\end{verbete}

\begin{verbete}{右转}{you4zhuan3}{5;8}
  \significado{v.}{virar à direita}
\end{verbete}

\begin{verbete}{要么……要么……}{yqo4me5·yao4me5}{9;3;9;3}
  \significado{conj.}{ou\dots ou\dots}
\end{verbete}

\begin{verbete}{于是}{yu2shi4}{3;9}
  \significado{conj.}{então; portanto; é por isso}
\end{verbete}

\begin{verbete}{鱼}{yu2}{8}
  \significado*{s.}{sobrenome Yu}
  \significado[条,尾]{s.}{peixe}
\end{verbete}

\begin{verbete}{鱼船}{yu2chuan2}{8;11}
  \significado{s.}{barco de pesca}
  \veja{鱼船}{yu2chuan2}
\end{verbete}

\begin{verbete}{鱼具}{yu2ju4}{8;8}
  \variante{渔具}{yu2ju4}
\end{verbete}

\begin{verbete}{鱼片}{yu2pian4}{8;4}
  \significado{s.}{fatia de peixe; filé de peixe}
\end{verbete}

\begin{verbete}{鱼网}{yu2wang3}{8;6}
  \variante{渔网}{yu2wang3}
\end{verbete}

\begin{verbete}{鱼香}{yu2xiang1}{8;9}
  \significado{s.}{鱼香, um tempero da culinária chinesa que normalmente contém alho, cebolinha, gengibre, açúcar, sal, pimenta, etc.; Embora ``鱼香'' signifique literalmente ``fragrância de peixe'', não contém frutos do mar}
\end{verbete}

\begin{verbete}{鱼香肉丝}{yu2xiang1rou4si1}{8;9;6;5}
  \significado{s.}{tiras de carne de porco salteadas com molho picante (prato)}
  \variante{鱼香}{yu2xiang1}
\end{verbete}

\begin{verbete}{鱼汛}{yu2xun4}{8;6}
  \variante{渔汛}{yu2xun4}
\end{verbete}

\begin{verbete}{渔}{yu2}{11}
  \significado[条]{s.}{pescador}
  \significado{v.}{pescar}
\end{verbete}

\begin{verbete}{渔场}{yu2chang3}{11;6}
  \significado{s.}{área de pesca}
\end{verbete}

\begin{verbete}{渔船}{yu2chuan2}{11;11}
  \significado[条]{s.}{barco de pesca}
  \veja{鱼船}{yu2chuan2}
\end{verbete}

\begin{verbete}{渔船队}{yu2chuan2dui4}{11;11;4}
  \significado{s.}{frota pesqueira}
\end{verbete}

\begin{verbete}{渔夫}{yu2fu1}{11;4}
  \significado{s.}{pescador}
\end{verbete}

\begin{verbete}{渔具}{yu2ju4}{11;8}
  \significado{s.}{equipamento de pesca}
\end{verbete}

\begin{verbete}{渔捞}{yu2lao1}{11;10}
  \significado{s.}{pesca (como atividade comercial)}
\end{verbete}

\begin{verbete}{渔猎}{yu2lie4}{11;11}
  \significado{s.}{pesca e caça}
  \significado{v.}{saquear; pilhar}
\end{verbete}

\begin{verbete}{渔笼}{yu2long2}{11;11}
  \significado{s.}{gaiola de pesca; armadilha de pesca}
\end{verbete}

\begin{verbete}{渔轮}{yu2lun2}{11;8}
  \significado{s.}{navio de pesca}
\end{verbete}

\begin{verbete}{渔民}{yu2min2}{11;5}
  \significado{s.}{pescadores; povo pescador}
\end{verbete}

\begin{verbete}{渔网}{yu2wang3}{11;6}
  \significado{s.}{rede de pesca}
\end{verbete}

\begin{verbete}{渔汛}{yu2xun4}{11;6}
  \significado{s.}{temporada de pesca}
\end{verbete}

\begin{verbete}{与}{yu3}{3}
  \significado{conj.}{e, com}
  \veja{与}{yu4}
\end{verbete}

\begin{verbete}{与其}{yu3qi2}{3;8}
  \significado{conj.}{mais do que}
\end{verbete}

\begin{verbete}{与其……不如……}{yu3qi2·bu4ru2}{3;8;4;6}
  \significado{conj.}{ao invés de\dots melhor que\dots}
\end{verbete}

\begin{verbete}{与其……宁可……}{yu3qi2·ning4ke3}{3;8;5;5}
  \significado{conj.}{ao invés de\dots melhor que\dots}
\end{verbete}

\begin{verbete}{羽冠}{yu3guan1}{6;9}
  \significado{s.}{crista emplumada (de pássaro)}
\end{verbete}

\begin{verbete}{羽林}{yu3lin2}{6;8}
  \significado{s.}{escolta armada}
\end{verbete}

\begin{verbete}{羽流}{yu3liu2}{6;10}
  \significado{s.}{pluma}
\end{verbete}

\begin{verbete}{羽毛}{yu3mao2}{6;4}
  \significado{s.}{pena; plumagem; pluma}
\end{verbete}

\begin{verbete}{羽毛笔}{yu3mao2bi3}{6;4;10}
  \significado{s.}{caneta de pena}
\end{verbete}

\begin{verbete}{羽毛球}{yu3mao2qiu2}{6;4;11}
  \significado[个]{s.}{\emph{badminton}}
\end{verbete}

\begin{verbete}{雨}{yu3}{8}[173]
  \significado[阵,场]{s.}{chuva}
  \veja{雨}{yu4}
\end{verbete}

\begin{verbete}{雨伞}{yu3san3}{8;6}
  \significado[把]{s.}{guarda-chuva}
\end{verbete}

\begin{verbete}{雨蚀}{yu3shi2}{8;9}
  \significado{s.}{erosão da chuva}
\end{verbete}

\begin{verbete}{雨靴}{yu3xue1}{8;13}
  \significado[双]{s.}{botas de chuva}
\end{verbete}

\begin{verbete}{雨衣}{yu3yi1}{8;6}
  \significado[件]{s.}{impermeável}
\end{verbete}

\begin{verbete}{语}{yu3}{9}
  \significado{s.}{dialeto; linguagem; fala}
  \veja{语}{yu4}
\end{verbete}

\begin{verbete}{语调}{yu3diao4}{9;10}
  \significado[个]{s.}{entonação}
\end{verbete}

\begin{verbete}{语法}{yu3fa3}{9;8}
  \significado{s.}{gramática}
\end{verbete}

\begin{verbete}{语法术语}{yu3fa3shu4yu3}{9;8;5;9}
  \significado{s.}{termo gramatical}
\end{verbete}

\begin{verbete}{语气}{yu3qi4}{9;4}
  \significado[个]{s.}{maneira de falar; tom}
\end{verbete}

\begin{verbete}{语言}{yu3yan2}{9;7}
  \significado[门,种]{s.}{linguagem; língua}
\end{verbete}

\begin{verbete}{语言实验室}{yu3yan2shi2yan4shi4}{9;7;8;10;9}
  \significado{s.}{laboratório de línguas}
\end{verbete}

\begin{verbete}{与}{yu4}{3}
  \significado{v.}{fazer parte de}
  \veja{与}{yu3}
\end{verbete}

\begin{verbete}{玉}{yu4}{5}[96]
  \significado[块]{s.}{jade}
\end{verbete}

\begin{verbete}{玉米}{yu4mi3}{5;6}
  \significado[粒]{s.}{milho}
\end{verbete}

\begin{verbete}{玉米饼}{yu4mi3bing3}{5;6;9}
  \significado{s.}{tortilha mexicana; bolo de milho}
\end{verbete}

\begin{verbete}{玉米粉}{yu4mi3fen3}{5;6;10}
  \significado{s.}{amido de milho; farinha de milho}
\end{verbete}

\begin{verbete}{玉米糕}{yu4mi3gao1}{5;6;16}
  \significado{s.}{bolo de milho; polenta}
\end{verbete}

\begin{verbete}{玉米花}{yu4mi3hua1}{5;6;7}
  \significado{s.}{pipoca}
\end{verbete}

\begin{verbete}{玉米面}{yu4mi3mian4}{5;6;9}
  \significado{s.}{fubá; farinha de milho}
\end{verbete}

\begin{verbete}{玉米片}{yu4mi3pian4}{5;6;4}
  \significado{s.}{flocos de milho; chips de tortilha}
\end{verbete}

\begin{verbete}{玉米糁}{yu4mi3san3}{5;6;14}
  \significado{s.}{grãos de milho}
\end{verbete}

\begin{verbete}{玉米笋}{yu4mi3sun3}{5;6;10}
  \significado{s.}{broto de milho}
\end{verbete}

\begin{verbete}{芋头}{yu4tou5}{6;5}
  \significado{s.}{taro, similar ao inhame e batata doce}
\end{verbete}

\begin{verbete}{芋头色}{yu4tou5se4}{6;5;6}
  \significado{s.}{lilás (cor)}
\end{verbete}

\begin{verbete}{雨}{yu4}{8}
  \significado{v.}{cair (chuva, neve, etc.); precipitar; chover; molhar}
  \veja{雨}{yu3}
\end{verbete}

\begin{verbete}{语}{yu4}{9}
  \significado{v.}{contar para; falar para}
  \veja{语}{yu3}
\end{verbete}

\begin{verbete}{预}{yu4}{10}
  \significado{adv.}{antecipadamente}
  \significado{v.}{avançar; preparar}
\end{verbete}

\begin{verbete}{预报}{yu4bao4}{10;7}
  \significado{s.}{previsão (meteorológica); boletim meteorológico}
  \significado{v.}{prever (o tempo)}
\end{verbete}

\begin{verbete}{预定}{yu4ding4}{10;8}
  \significado{v.}{agendar com antecedência}
\end{verbete}

\begin{verbete}{预付}{yu4fu4}{10;5}
  \significado{s.}{pré-pago}
  \significado{v.}{pagar antecipadamente}
\end{verbete}

\begin{verbete}{预感}{yu4gan3}{10;13}
  \significado{s.}{premonição}
  \significado{v.}{ter uma premonição}
\end{verbete}

\begin{verbete}{预购}{yu4gou4}{10;8}
  \significado{s.}{compra antecipada}
  \significado{v.}{comprar antecipadamente}
\end{verbete}

\begin{verbete}{预见}{yu4jian4}{10;4}
  \significado{s.}{previsão; intuição; vislumbre}
  \significado{v.}{prever}
\end{verbete}

\begin{verbete}{预览}{yu4lan3}{10;9}
  \significado{s.}{visualização}
  \significado{v.}{visualizar}
\end{verbete}

\begin{verbete}{预留}{yu4liu2}{10;10}
  \significado{v.}{separar; reservar}
\end{verbete}

\begin{verbete}{预谋}{yu4mou2}{10;11}
  \significado{adj.}{premeditado}
  \significado{v.}{planejar algo com antecedência (especialmente um crime)}
\end{verbete}

\begin{verbete}{预配}{yu4pei4}{10;10}
  \significado{s.}{pré-alocado; pré-cabeado}
  \significado{v.}{pré-alocar; pré-cabear}
\end{verbete}

\begin{verbete}{预提}{yu4ti2}{10;12}
  \significado{s.}{retenção}
  \significado{v.}{reter (imposto)}
\end{verbete}

\begin{verbete}{预约}{yu4yue1}{10;6}
  \significado{s.}{reserva}
  \significado{v.}{agendar; marcar compromisso}
\end{verbete}

\begin{verbete}{预祝}{yu4zhu4}{10;9}
  \significado{v.}{parabenizar de antemão; oferecer os melhores votos para}
\end{verbete}

\begin{verbete}{愈}{yu4}{13}
  \significado{adv.}{mais e mais; ainda mais}
  \significado{v.}{recuperar; curar}
\end{verbete}

\begin{verbete}{豫}{yu4}{15}
  \significado{adj.}{feliz despreocupado; à vontade}
  \veja{预}{yu4}
\end{verbete}

\begin{verbete}{元}{yuan2}{4}
  \significado*{s.}{sobrenome Yuan; Dinastia Yuan (1279-1368)}
  \significado{p.c.}{unidade monetária da China}
\end{verbete}

\begin{verbete}{元旦}{yuan2dan4}{4;5}
  \significado*{s.}{Dia de Ano Novo (1 de janeiro)}
\end{verbete}

\begin{verbete}{元来}{yuan2lai2}{4;7}
  \variante{原来}{yuan2lai2}
\end{verbete}

\begin{verbete}{元宵}{yuan2xiao1}{4;10}
  \significado*{s.}{Festival das Lanternas}
  \veja{元宵节}{yuan2xiao1jie2}
  \veja{元夜}{yuan2ye4}
\end{verbete}

\begin{verbete}{元宵节}{yuan2xiao1jie2}{4;10;5}
  \significado*{s.}{Festival das Lanternas (15º~dia do primeiro mês lunar)}
  \veja{元宵}{yuan2xiao1}
  \veja{元夜}{yuan2ye4}
\end{verbete}

\begin{verbete}{元夜}{yuan2ye4}{4;8}
  \significado*{s.}{Festival das Lanternas}
  \veja{元宵}{yuan2xiao1}
  \veja{元宵节}{yuan2xiao1jie2}
\end{verbete}

\begin{verbete}{原来}{yuan2lai2}{10;7}
  \significado{adv.}{originalmente; como se vê; na verdade}
\end{verbete}

\begin{verbete}{原因}{yuan2yin1}{10;6}
  \significado[个]{s.}{causa; razão; motivo}
\end{verbete}

\begin{verbete}{远}{yuan3}{7}
  \significado{adj.}{longe; distante; remoto}
  \veja{远}{yuan4}
\end{verbete}

\begin{verbete}{远天}{yuan3tian1}{7;4}
  \significado{s.}{paraíso; o céu distante}
\end{verbete}

\begin{verbete}{远远}{yuan3yuan3}{7;7}
  \significado{adv.}{de longe}
\end{verbete}

\begin{verbete}{远}{yuan4}{7}
  \significado{v.}{distanciar-se de (clássico)}
  \veja{远}{yuan3}
\end{verbete}

\begin{verbete}{院}{yuan4}{9}
  \significado[个]{s.}{pátio; instituição}
\end{verbete}

\begin{verbete}{院长}{yuan4zhang3}{9;4}
  \significado[个]{s.}{presidente de um conselho; reitor; chefe de departamento; primeiro-ministro da República da China; presidente de uma universidade}
\end{verbete}

\begin{verbete}{院子}{yuan4zi5}{9;3}
  \significado[个]{s.}{pátio; jardim; quintal}
\end{verbete}

\begin{verbete}{约会}{yue1hui4}{6;6}
  \significado[次,个]{s.}{compromisso; encontro marcado}
\end{verbete}

\begin{verbete}{月}{yue4}{4}[74]
  \significado[个,轮]{s.}{mês}
\end{verbete}

\begin{verbete}{月径}{yue4jing4}{4;8}
  \significado{s.}{diâmetro da lua; diâmetro da órbita da lua; caminho iluminado pela lua}
\end{verbete}

\begin{verbete}{月亮}{yue4liang5}{4;9}
  \significado{s.}{lua}
\end{verbete}

\begin{verbete}{月相}{yue4xiang4}{4;9}
  \significado{s.}{fases da lua, a saber: lua nova 朔, lua crescente 上弦, lua cheia 望 e lua minguante 下弦}
\end{verbete}

\begin{verbete}{月月}{yue4yue4}{4;4}
  \significado{p.t.}{todo mês}
\end{verbete}

\begin{verbete}{阅读}{yue4du2}{10;10}
  \significado{s.}{leitura}
  \significado{v.}{ler}
\end{verbete}

\begin{verbete}{阅读广度}{yue4du2guang3du4}{10;10;3;9}
  \significado{s.}{intervalo de leitura}
\end{verbete}

\begin{verbete}{阅读理解}{yue4du2li3jie3}{10;10;11;13}
  \significado{s.}{compreensão de leitura}
\end{verbete}

\begin{verbete}{阅读器}{yue4du2qi4}{10;10;16}
  \significado{s.}{leitor (\emph{software})}
\end{verbete}

\begin{verbete}{阅读时间}{yue4du2shi2jian1}{10;10;7;7}
  \significado{s.}{tempo de leitura}
\end{verbete}

\begin{verbete}{阅读障碍}{yue4du2zhang4ai4}{10;10;13;13}
  \significado{s.}{dislexia}
\end{verbete}

\begin{verbete}{阅读装置}{yue4du2zhuang1zhi4}{10;10;12;13}
  \significado{s.}{dispositivo de leitura (por exemplo, para códigos de barras, etiquetas RFID, etc.)}
\end{verbete}

\begin{verbete}{阅览室}{yue4lan3shi4}{10;9;9}
  \significado[间]{s.}{sala de leitura}
\end{verbete}

\begin{verbete}{越}{yue4}{12}
  \significado{adv.}{quanto mais... mais}
  \significado{v.}{subir; exceder; superar}
\end{verbete}

\begin{verbete}{越境}{yue4jing4}{12;14}
  \significado{v.}{cruzar uma fronteira (geralmente ilegalmente); entrar ou sair furtivamente de um país}
\end{verbete}

\begin{verbete}{越来越……}{yue4lai2yue4}{12;7;12}
  \significado{adv.}{cada vez mais\dots}
\end{verbete}

\begin{verbete}{越……越……}{yue4·yue4}{12;12}
  \significado{expr.}{quanto mais\dots tanto mais\dots}
\end{verbete}

\begin{verbete}{越障}{yue4zhan4}{12;13}
  \significado{s.}{curso com obstáculos para treinamento de tropas}
  \significado{v.}{superar obstáculos}
\end{verbete}

\begin{verbete}{云}{yun2}{4}
  \significado*{s.}{sobrenome Yun}
  \significado[朵]{s.}{nuvem}
\end{verbete}

\begin{verbete}{云南}{yun2nan2}{4;9}
  \significado*{s.}{Yunnan}
\end{verbete}

\begin{verbete}{云云}{yun2yun2}{4;4}
  \significado{adv.}{e assim por diante; assim e assim}
\end{verbete}

\begin{verbete}{运动}{yun4dong4}{7;6}
  \significado[场]{s.}{esporte; desporto}
  \significado{v.}{exercitar; mover-se}
\end{verbete}

\begin{verbete}{运动病}{yun4dong4bing4}{7;6;10}
  \significado{s.}{enjôo (movimento, carro, etc.)}
\end{verbete}

\begin{verbete}{运动场}{yun4dong4chang3}{7;6;6}
  \significado{s.}{campo desportivo; campo de jogos}
\end{verbete}

\begin{verbete}{运动服}{yun4dong4fu4}{7;6;8}
  \significado{s.}{roupa para prática de esporte}
\end{verbete}

\begin{verbete}{运动会}{yun4dong4hui4}{7;6;6}
  \significado[个]{s.}{competição esportiva}
\end{verbete}

\begin{verbete}{运动家}{yun4dong4jia1}{7;6;10}
  \significado{s.}{ativista; atleta; esportista}
\end{verbete}

\begin{verbete}{运动衫}{yun4dong4shan1}{7;6;8}
  \significado[件]{s.}{moletom; camisa esportiva}
\end{verbete}

\begin{verbete}{运动鞋}{yun4dong4xie2}{7;6;15}
  \significado{s.}{tênis; sapatos esportivos}
\end{verbete}

\begin{verbete}{运动学}{yun4dong4xue2}{7;6;8}
  \significado{s.}{cinemática}
\end{verbete}

\begin{verbete}{运动员}{yun4dong4yuan2}{7;6;7}
  \significado[名,个]{s.}{jogador; atleta}
\end{verbete}

\begin{verbete}{运气}{yun4qi5}{7;4}
  \significado{s.}{sorte (boa ou má)}
\end{verbete}

%%%%% EOF %%%%%

%%%
%%% Z
%%%
%\section*{Z}
\addcontentsline{toc}{section}{Z}

\begin{verbete}{杂技}{za2ji4}{6,7}
  \significado[场]{s.}{acrobacia}
\end{verbete}

\begin{verbete}{杂志}{za2zhi4}{6,7}
  \significado[本,份,期]{s.}{revista}
\end{verbete}

\begin{verbete}{杂志社}{za2zhi4she4}{6,7,7}
  \significado{s.}{editora de revista}
\end{verbete}

\begin{verbete}{砸}{za2}{10}[Radical 石]
  \significado{v.}{esmagar; bater; falhar; estragar}
\end{verbete}

\begin{verbete}{栽}{zai1}{10}[Radical 木]
  \significado{v.}{cultivar; plantar}
\end{verbete}

\begin{verbete}{栽倒}{zai1dao3}{10,10}
  \significado{v.}{cair; sofrer uma queda}
\end{verbete}

\begin{verbete}{栽培}{zai1pei2}{10,11}
  \significado{v.}{cultivar; educar; patrocinar; treinar}
\end{verbete}

\begin{verbete}{栽培种}{zai1pei2 zhong3}{10,11,9}
  \significado{s.}{espécies cultivadas}
\end{verbete}

\begin{verbete}{栽赃}{zai1zang1}{10,10}
  \significado{v.}{enquadrar alguém (plantar provas nele)}
\end{verbete}

\begin{verbete}{栽植}{zai1zhi2}{10,12}
  \significado{v.}{plantar; transplantar}
\end{verbete}

\begin{verbete}{栽种}{zai1zhong4}{10,9}
  \significado{v.}{plantar}
\end{verbete}

\begin{verbete}{再}{zai4}{6}[Radical 冂]
  \significado{adv.}{de novo; outra vez; uma segunda vez; não importa como\dots (seguido por um adjetivo ou verbo, e então (normalmente) 也 ou 都 para dar ênfase)}
\end{verbete}

\begin{verbete}{再不}{zai4bu4}{6,4}
  \significado{adv.}{nunca mais}
\end{verbete}

\begin{verbete}{再读}{zai4du2}{6,10}
  \significado{v.}{ler novamente; rever (uma lição, etc.)}
\end{verbete}

\begin{verbete}{再度}{zai4du4}{6,9}
  \significado{adv.}{outra vez; mais uma vez}
\end{verbete}

\begin{verbete}{再发}{zai4fa1}{6,5}
  \significado{v.}{reenviar}
\end{verbete}

\begin{verbete}{再见}{zai4jian4}{6,4}
  \significado{v.}{adeus; até à vista; até à próxima; até logo}
\end{verbete}

\begin{verbete}{再临}{zai4lin2}{6,9}
  \significado{v.}{vir de novo}
\end{verbete}

\begin{verbete}{再三}{zai4san1}{6,3}
  \significado{adv.}{de novo e de novo; repetidamente}
\end{verbete}

\begin{verbete}{再审}{zai4shen3}{6,8}
  \significado{s.}{novo julgamento; revisão}
  \significado{v.}{ouvir um caso novamente}
\end{verbete}

\begin{verbete}{再生}{zai4sheng1}{6,5}
  \significado{s.}{reciclagem; regeneração}
  \significado{v.}{reciclar; renascer; regenerar}
\end{verbete}

\begin{verbete}{再说}{zai4shuo1}{6,9}
  \significado{conj.}{além do mais; além disso; o que mais}
  \significado{v.}{adiar uma discussão para mais tarde; dizer novamente}
\end{verbete}

\begin{verbete}{再育}{zai4yu4}{6,8}
  \significado{v.}{aumentar; multiplicar; proliferar}
\end{verbete}

\begin{verbete}{再者}{zai4zhe3}{6,8}
  \significado{conj.}{além do mais; além disso}
\end{verbete}

\begin{verbete}{在}{zai4}{6}[Radical 土]
  \significado{adv.}{para designar ações que estão passando; durante}
  \significado{prep.}{em}
  \significado{v.}{estar; ficar}
\end{verbete}

\begin{verbete}{在此}{zai4ci3}{6,6}
  \significado{adv.}{aqui}
\end{verbete}

\begin{verbete}{在地}{zai4di4}{6,6}
  \significado{s.}{local}
\end{verbete}

\begin{verbete}{在行}{zai4hang2}{6,6}
  \significado{v.}{ser adepto de algo; ser um especialista em um comércio ou profissão}
\end{verbete}

\begin{verbete}{在乎}{zai4hu5}{6,5}
  \significado{v.}{preocupar-se com}
\end{verbete}

\begin{verbete}{在教}{zai4jiao4}{6,11}
  \significado{v.}{ser um crente (em uma religião)}
\end{verbete}

\begin{verbete}{在下}{zai4xia4}{6,3}
  \significado{pron.}{eu mesmo (humildemente)}
\end{verbete}

\begin{verbete}{在线}{zai4xian4}{6,8}
  \significado{s.}{\emph{online}}
\end{verbete}

\begin{verbete}{在意}{zai4yi4}{6,13}
  \significado{v.+compl.}{preocupar-se; importar-se; levar a sério}
\end{verbete}

\begin{verbete}{在于}{zai4yu2}{6,3}
  \significado{v.}{descansar; deitar; ser devido a (um determinado atributo)/(de um assunto) a ser determinado; estar à altura de alguém}
\end{verbete}

\begin{verbete}{咱家}{zan2jia1}{9,10}
  \significado{pron.}{eu (frequentemente usado na literatura vernácula antiga); me; mim, comigo}
\end{verbete}

\begin{verbete}{咱俩}{zan2lia3}{9,9}
  \significado{pron.}{nós dois}
\end{verbete}

\begin{verbete}{咱们}{zan2men5}{9,5}
  \significado{pron.}{nós (incluindo o orador e a(s) pessoa(s) com quem se fala)}
\end{verbete}

\begin{verbete}{赞}{zan4}{16}[Radical 貝]
  \significado{v.}{patrocinar; apoiar; elogiar; (gíria na Internet) para curtir (uma postagem on-line)}
\end{verbete}

\begin{verbete}{赞扬}{zan4yang2}{16,6}
  \significado{v.}{elogiar; aprovar; demonstrar aprovação}
\end{verbete}

\begin{verbete}{赞助}{zan4zhu4}{16,7}
  \significado{s.}{patrocinador}
  \significado{v.}{apoiar; auxiliar; patrocinar}
\end{verbete}

\begin{verbete}{脏}{zang1}{10}[Radical 肉]
  \significado{adj.}{sujo; imundo}
  \veja{脏}{zang4}
\end{verbete}

\begin{verbete}{脏辫}{zang1bian4}{10,17}
  \significado{s.}{\emph{dreadlocks}}
\end{verbete}

\begin{verbete}{脏病}{zang1bing4}{10,10}
  \significado{s.}{doença venérea}
\end{verbete}

\begin{verbete}{脏煤}{zang1mei2}{10,13}
  \significado{s.}{carvão sujo; sujeira (de uma mina de carvão)}
\end{verbete}

\begin{verbete}{脏土}{zang1tu3}{10,3}
  \significado{s.}{solo sujo; lama; lixo}
\end{verbete}

\begin{verbete}{脏脏}{zang1zang1}{10,10}
  \significado{adj.}{sujo}
\end{verbete}

\begin{verbete}{脏字}{zang1zi4}{10,6}
  \significado{s.}{obscenidade}
\end{verbete}

\begin{verbete}{脏}{zang4}{10}[Radical 肉]
  \significado{s.}{órgão (anatomia); víscera}
  \veja{脏}{zang1}
\end{verbete}

\begin{verbete}{脏器}{zang4qi4}{10,16}
  \significado{s.}{órgãos internos}
\end{verbete}

\begin{verbete}{葬}{zang4}{12}[Radical 艸]
  \significado{v.}{enterrar (os mortos); sepultar}
\end{verbete}

\begin{verbete}{遭到}{zao1dao4}{14,8}
  \significado{v.}{sofrer; encontrar-se com (algo infeliz)}
\end{verbete}

\begin{verbete}{遭受}{zao1shou4}{14,8}
  \significado{v.}{sofrer, suportar (perda, infornúnio)}
\end{verbete}

\begin{verbete}{遭遇}{zao1yu4}{14,12}
  \significado{s.}{experiência (amarga)}
  \significado{v.}{encontrar-se com;}
\end{verbete}

\begin{verbete}{糟糕}{zao1gao1}{17,16}
  \significado{adj.}{muito mau; péssimo}
\end{verbete}

\begin{verbete}{早}{zao3}{6}[Radical 日]
  \significado{adj.}{prematuramente}
  \significado{adv.}{cedo; antecipadamante; breve}
  \significado{s.}{manhã}
\end{verbete}

\begin{verbete}{早安}{zao3'an1}{6,6}
  \significado{interj.}{Bom dia!}
\end{verbete}

\begin{verbete}{早餐}{zao3can1}{6,16}
  \significado[份,顿,次]{s.}{café da manhã}
\end{verbete}

\begin{verbete}{早车}{zao3che1}{6,4}
  \significado{s.}{trem matutino; ônibus matutino}
\end{verbete}

\begin{verbete}{早晨}{zao3chen2}{6,11}
  \significado{adv.}{manhã cedo; manhãzinha}
  \significado[个]{s.}{manhã}
\end{verbete}

\begin{verbete}{早饭}{zao3fan4}{6,7}
  \significado[份,顿,次,餐]{s.}{café da manhã}
\end{verbete}

\begin{verbete}{早就}{zao3jiu4}{6,12}
  \significado{adv.}{já em um momento anterior}
\end{verbete}

\begin{verbete}{早前}{zao3qian2}{6,9}
  \significado{adv.}{previamente}
\end{verbete}

\begin{verbete}{早上}{zao3shang5}{6,3}
  \significado{adv.}{manhã cedo; manhãzinha}
  \significado[个]{s.}{manhã}
\end{verbete}

\begin{verbete}{早亡}{zao3wang2}{6,3}
  \significado[个]{s.}{morte prematura}
  \significado{v.}{morrer prematuramente}
\end{verbete}

\begin{verbete}{早早儿}{zao3zao3r5}{6,6,2}
  \significado{adv.}{o mais cedo possível; o mais breve possível}
\end{verbete}

\begin{verbete}{早知}{zao3zhi1}{6,8}
  \significado{v.}{prever; se alguém soubesse antes, \dots}
\end{verbete}

\begin{verbete}{灶台}{zao4tai2}{7,5}
  \significado{s.}{fogão}
\end{verbete}

\begin{verbete}{造}{zao4}{10}[Radical 辵]
  \significado{clas.}{para colheitas, cultivos}
  \significado{v.}{criar; construir; fabricar; inventar}
\end{verbete}

\begin{verbete}{艁}{zao4}{13}
  \variante{造}
\end{verbete}

\begin{verbete}{责怪}{ze2guai4}{8,8}
  \significado{v.}{repreender; censurar}
\end{verbete}

\begin{verbete}{怎}{zen3}{9}[Radical 心]
  \significado{adv.}{como}
\end{verbete}

\begin{verbete}{怎么}{zen3me5}{9,3}
  \significado{pron.}{como?; o que?}
\end{verbete}

\begin{verbete}{怎么办}{zen3me5ban4}{9,3,4}
  \significado{adv.}{o que fazer?}
\end{verbete}

\begin{verbete}{怎么得了}{zen3me5de2liao3}{9,3,11,2}
  \significado{expr.}{Como isso pode ser?; Que bagunça horrível!; O que deve ser feito?}
\end{verbete}

\begin{verbete}{怎么搞的}{zen3me5gao3de5}{9,3,13,8}
  \significado{expr.}{Como isso aconteceu?; O que deu errado?;E aí?; O que está errado?}
\end{verbete}

\begin{verbete}{怎么回事}{zen3me5hui2shi4}{9,3,6,8}
  \significado{expr.}{O que aconteceu?; O que se passou?}
\end{verbete}

\begin{verbete}{怎么了}{zen3me5le5}{9,3,2}
  \significado{expr.}{O que aconteceu?; O que está acontecendo?; E aí?}
\end{verbete}

\begin{verbete}{怎么样}{zen3me5yang4}{9,3,10}
  \significado{adv.}{como?; que tal?}
\end{verbete}

\begin{verbete}{增速}{zeng1su4}{15,10}
  \significado{s.}{(economia) taxa de crescimento}
  \significado{v.}{acelerar;}
\end{verbete}

\begin{verbete}{闸门}{zha2men2}{8,3}
  \significado{s.}{eclusa; comporta}
\end{verbete}

\begin{verbete}{寨}{zhai4}{14}[Radical 宀]
  \significado{s.}{fortaleza; paliçada; acampamento; vila (paliçada)}
\end{verbete}

\begin{verbete}{斩获}{zhan3huo4}{8,10}
  \significado{v.}{matar ou capturar (em batalha); (fig.) (esportes) marcar (um gol), ganhar (uma medalha);(fig.) colher recompensas, obter ganhos}
\end{verbete}

\begin{verbete}{展示}{zhan3shi4}{10,5}
  \significado{v.}{revelar, mostrar, exibir}
\end{verbete}

\begin{verbete}{盏}{zhan3}{10}[Radical 皿]
  \significado{clas.}{para lâmpadas}
  \significado{s.}{copo pequeno}
\end{verbete}

\begin{verbete}{战}{zhan4}{9}[Radical 戈]
  \significado{s.}{luta; guerra; batalha}
  \significado{v.}{lutar}
\end{verbete}

\begin{verbete}{战士}{zhan4shi4}{9,3}
  \significado[个]{s.}{lutador; soldado; guerreiro}
\end{verbete}

\begin{verbete}{战争}{zhan4zheng1}{9,6}
  \significado[場,次]{s.}{guerra; conflito}
\end{verbete}

\begin{verbete}{站}{zhan4}{10}[Radical 立]
  \significado{s.}{estação; ponto; parada}
\end{verbete}

\begin{verbete}{站点}{zhan4dian3}{10,9}
  \significado{s.}{\emph{website}}
\end{verbete}

\begin{verbete}{站台}{zhan4tai2}{10,5}
  \significado{s.}{plataforma (em uma estação ferroviária)}
\end{verbete}

\begin{verbete}{站长}{zhan4zhang3}{10,4}
  \significado{s.}{pessoa responsável pela estação de trem; chefe da estação; \emph{webmaster}; gerente de centro de voluntariado}
\end{verbete}

\begin{verbete}{站姿}{zhan4zi1}{10,9}
  \significado{s.}{postura}
\end{verbete}

\begin{verbete}{张}{zhang1}{7}[Radical 弓]
  \significado*{s.}{sobrenome Zhang}
  \significado{clas.}{para folha de papéis, mapas, etc.; para votos}
  \significado{s.}{folha de papel}
  \significado{v.}{abrir; espalhar}
\end{verbete}

\begin{verbete}{张狂}{zhang1kuang2}{7,7}
  \significado{adj.}{impetuoso; frenético; insolente}
\end{verbete}

\begin{verbete}{张三}{zhang1san1}{7,3}
  \significado*{s.}{Zhang San; Zé Ninguém; nome para uma pessoa não especificada, 1 de 3}
  \veja{李四}{li3si4}
  \veja{王五}{wang2wu3}
\end{verbete}

\begin{verbete}{章}{zhang1}{11}[Radical 音]
  \significado*{s.}{sobrenome Zhang}
  \significado{s.}{capítulo; seção; cláusula;  movimento (de sinfonia); selo; crachá; regulamento}
\end{verbete}

\begin{verbete}{章鱼}{zhang1yu2}{11,8}
  \significado{s.}{polvo; octópode}
\end{verbete}

\begin{verbete}{长}{zhang3}{4}[Radical 長]
  \significado{s.}{chefe; ancião}
  \significado{v.}{crescer; desenvolver; aumentar; melhorar}
  \veja{长}{chang2}
\end{verbete}

\begin{verbete}{涨价}{zhang3jia4}{10,6}
  \significado{s.}{aumento de preços}
  \significado{v.+compl.}{avaliar (em valor); dar preço | aumentar o preço}
\end{verbete}

\begin{verbete}{掌}{zhang3}{12}[Radical 手]
  \significado{s.}{palma da mão; sola do pé; pata; ferradura}
  \significado{v.}{dar um tapa; segurar na mão; empunhar}
\end{verbete}

\begin{verbete}{招}{zhao1}{8}[Radical 手]
  \significado{adj.}{contagioso}
  \significado{s.}{um movimento (xadrez); uma manobra; dispositivo; truque}
  \significado{v.}{recrutar; provocar; acenar; incorrer; infectar; confessar}
\end{verbete}

\begin{verbete}{招手}{zhao1shou3}{8,4}
  \significado{v.+compl.}{acenar}
\end{verbete}

\begin{verbete}{招数}{zhao1shu4}{8,13}
  \significado{s.}{estratégia; movimento (no xadrez, no palco, nas artes marciais); esquema; truque}
\end{verbete}

\begin{verbete}{着}{zhao1}{11}[Radical 目]
  \significado{interj.}{Tudo bem!}
  \significado{s.}{movimento (xadrez); truque}
  \veja{着}{zhao2}
  \veja{着}{zhe5}
  \veja{着}{zhuo2}
\end{verbete}

\begin{verbete}{着数}{zhao1shu4}{11,13}
  \significado{s.}{estratégia; movimento (no xadrez, no palco, nas artes marciais); esquema; truque}
\end{verbete}

\begin{verbete}{朝}{zhao1}{12}[Radical 月]
  \significado{s.}{manhã}
  \veja{朝}{chao2}
\end{verbete}

\begin{verbete}{着}{zhao2}{11}[Radical 目]
  \significado{v.}{ser afetado por; queimar; pegar fogo; entrar em contato com; sentir; tocar}
  \veja{着}{zhao1}
  \veja{着}{zhe5}
  \veja{着}{zhuo2}
\end{verbete}

\begin{verbete}{着地}{zhao2di4}{11,6}
  \significado{v.}{pousar; tocar o chão}
\end{verbete}

\begin{verbete}{着花}{zhao2hua1}{11,7}
  \significado{v.}{florescer}
  \veja{着花}{zhuo2hua1}
\end{verbete}

\begin{verbete}{着急}{zhao2ji2}{11,9}
  \significado{adj.}{inquieto; ansioso}
  \significado{s.}{preocupação; ansiedade}
  \significado{v.+compl.}{preocupar-se; sentir-se ansioso | sentir uma sensação de urgência}
\end{verbete}

\begin{verbete}{着凉}{zhao2liang2}{11,10}
  \significado{v.}{pegar um resfriado}
\end{verbete}

\begin{verbete}{找}{zhao3}{7}[Radical 手]
  \significado{v.}{andar à procura de; procurar; tentar procurar; dar troco; retornar algo}
\end{verbete}

\begin{verbete}{找遍}{zhao3bian4}{7,12}
  \significado{v.}{pentear; pesquisar em todos os lugares}
\end{verbete}

\begin{verbete}{找到}{zhao3dao4}{7,8}
  \significado{v.}{encontrar}
\end{verbete}

\begin{verbete}{找回}{zhao3hui2}{7,6}
  \significado{v.}{recuperar algo}
\end{verbete}

\begin{verbete}{找见}{zhao3jian4}{7,4}
  \significado{v.}{encontrar (algo que está procurando)}
\end{verbete}

\begin{verbete}{找零}{zhao3ling2}{7,13}
  \significado{v.}{trocar dinheiro; dar troco}
\end{verbete}

\begin{verbete}{找钱}{zhao3qian2}{7,10}
  \significado{v.}{dar troco}
\end{verbete}

\begin{verbete}{找事}{zhao3shi4}{7,8}
  \significado{v.}{procurar emprego; começar uma briga}
\end{verbete}

\begin{verbete}{找寻}{zhao3xun2}{7,6}
  \significado{v.}{encontrar falhas; procurar; buscar}
\end{verbete}

\begin{verbete}{找着}{zhao3zhao2}{7,11}
  \significado{v.}{encontrar}
\end{verbete}

\begin{verbete}{找辙}{zhao3zhe2}{7,16}
  \significado{v.}{procurar um pretexto}
\end{verbete}

\begin{verbete}{兆}{zhao4}{6}[Radical 儿]
  \significado{num.}{trilhão, 1.000.000.000.000}
\end{verbete}

\begin{verbete}{照}{zhao4}{13}[Radical 火]
  \significado{adv.}{de acordo com; como antes; como pedido; conforme}
  \significado{s.}{foto}
  \significado{v.}{iluminar; olhar (o reflexo de alguém); refletir; brilhar; tirar uma foto}
\end{verbete}

\begin{verbete}{照亮}{zhao4liang4}{13,9}
  \significado{s.}{iluminação}
  \significado{v.}{iluminar}
\end{verbete}

\begin{verbete}{照片}{zhao4pian4}{13,4}
  \significado[张,套,幅]{s.}{fotografia; foto}
\end{verbete}

\begin{verbete}{照片底版}{zhao4pian4di3ban3}{13,4,8,8}
  \significado{s.}{placa fotográfica}
\end{verbete}

\begin{verbete}{照片子}{zhao4pian4zi5}{13,4,3}
  \significado{v.}{tirar um raio X}
\end{verbete}

\begin{verbete}{照骗}{zhao4pian4}{13,12}
  \significado{s.}{imagem ``photoshopada''}
\end{verbete}

\begin{verbete}{照相}{zhao4xiang4}{13,9}
  \significado{v.+compl.}{tirar fotografia}
\end{verbete}

\begin{verbete}{照相机}{zhao4xiang4ji1}{13,9,6}
  \significado[个,架,部,台,只]{s.}{câmera/máquina fotográfica}
\end{verbete}

\begin{verbete}{照像}{zhao4xiang4}{13,13}
  \variante{照相}
\end{verbete}

\begin{verbete}{照像机}{zhao4xiang4ji1}{13,13,6}
  \variante{照相机}
\end{verbete}

\begin{verbete}{照准}{zhao4zhun3}{13,10}
  \significado{s.}{solicitação concedida (uso formal em documento antigo)}
  \significado{v.}{mirar (arma)}
\end{verbete}

\begin{verbete}{折转}{zhe2zhuan3}{7,8}
  \significado{s.}{reflexo (ângulo)}
  \significado{v.}{voltar atrás}
\end{verbete}

\begin{verbete}{哲理}{zhe2li3}{10,11}
  \significado{s.}{filosofia; teoria filosófica}
\end{verbete}

\begin{verbete}{这}{zhe4}{7}[Radical 辵]
  \significado{pron.}{este, isto}
  \veja{这}{zhei4}
\end{verbete}

\begin{verbete}{这里}{zhe4li3}{7,7}
  \significado{pron.}{aqui}
\end{verbete}

\begin{verbete}{这么}{zhe4me5}{7,3}
  \significado{adv.}{como este; desta maneira}
\end{verbete}

\begin{verbete}{这末}{zhe4me5}{7,5}
  \variante{这么}
\end{verbete}

\begin{verbete}{这麽}{zhe4me5}{7,14}
  \variante{这么}
\end{verbete}

\begin{verbete}{这儿}{zhe4r5}{7,2}
  \significado{pron.}{aqui}
\end{verbete}

\begin{verbete}{这时}{zhe4shi2}{7,7}
  \significado{adv.}{neste momento}
\end{verbete}

\begin{verbete}{这些}{zhe4xie1}{7,8}
  \significado{pron.}{estes}
\end{verbete}

\begin{verbete}{这样}{zhe4yang4}{7,10}
  \significado{adv.}{assim; dessa maneira; deste modo}
\end{verbete}

\begin{verbete}{浙江}{zhe4jiang1}{10,6}
  \significado*{s.}{Zhejiang}
\end{verbete}

\begin{verbete}{着}{zhe5}{11}[Radical 目]
  \significado{part.}{indicando ação em andamento ou estado em andamento}
  \veja{着}{zhao1}
  \veja{着}{zhao2}
  \veja{着}{zhuo2}
\end{verbete}

\begin{verbete}{这}{zhei4}{7}[Radical 辵]
  \significado{pron.}{(coloquial) este}
  \veja{这}{zhe4}
\end{verbete}

\begin{verbete}{珍贵}{zhen1gui4}{9,9}
  \significado{adj.}{precioso}
\end{verbete}

\begin{verbete}{珍珠}{zhen1zhu1}{9,10}
  \significado[颗]{s.}{pérola}
\end{verbete}

\begin{verbete}{眞}{zhen1}{10}
  \variante{真}
\end{verbete}

\begin{verbete}{真}{zhen1}{10}[Radical 目]
  \significado{adj.}{genuíno}
  \significado{adv.}{que\dots tão\dots!; realmente}
\end{verbete}

\begin{verbete}{真理}{zhen1li3}{10,11}
  \significado[个]{s.}{verdade}
\end{verbete}

\begin{verbete}{真牛}{zhen1niu2}{10,4}
  \significado{adj.}{gíria:~muito legal, incrível}
\end{verbete}

\begin{verbete}{真切}{zhen1qie4}{10,4}
  \significado{adj.}{claro; distinto; honesto; sincero; vívido}
\end{verbete}

\begin{verbete}{真声}{zhen1sheng1}{10,7}
  \significado{s.}{voz natural; voz verdadeira}
  \veja{假声}{jia3sheng1}
\end{verbete}

\begin{verbete}{真释}{zhen1shi4}{10,12}
  \significado{s.}{razão genuína; explicação verdadeira}
\end{verbete}

\begin{verbete}{真心}{zhen1xin1}{10,4}
  \significado{adj.}{sincero}
  \significado[片]{s.}{sinceridade}
\end{verbete}

\begin{verbete}{真真}{zhen1zhen1}{10,10}
  \significado{adv.}{genuinamente; realmente; escrupulosamente}
\end{verbete}

\begin{verbete}{真珠}{zhen1zhu1}{10,10}
  \variante{珍珠}
\end{verbete}

\begin{verbete}{枕}{zhen3}{8}[Radical 木]
  \significado{s.}{travesseiro; almofada}
\end{verbete}

\begin{verbete}{阵地}{zhen4di4}{6,6}
  \significado{s.}{posição (militar); frente de batalha; \emph{front}}
\end{verbete}

\begin{verbete}{震撼}{zhen4han4}{15,16}
  \significado{v.}{sacudir; chocar; atordoar}
\end{verbete}

\begin{verbete}{正}{zheng1}{5}[Radical 止]
  \significado{s.}{primeiro mês do ano lunar}
  \veja{正}{zheng4}
\end{verbete}

\begin{verbete}{争霸}{zheng1ba4}{6,21}
  \significado{s.}{hegemonia; uma luta de poder}
  \significado{v.}{disputar a hegemonia}
\end{verbete}

\begin{verbete}{争风吃醋}{zheng1feng1chi1cu4}{6,4,6,15}
  \significado{v.}{rivalizar com alguém pelo carinho de um homem ou mulher;  estar com ciúmes de um rival em um caso de amor}
\end{verbete}

\begin{verbete}{争先}{zheng1xian1}{6,6}
  \significado{v.}{competir para ser o primeiro; contestar o primeiro lugar}
\end{verbete}

\begin{verbete}{挣扎}{zheng1zha2}{9,4}
  \significado{v.}{lutar}
\end{verbete}

\begin{verbete}{整天}{zheng3tian1}{16,4}
  \significado{adv.}{dia todo; o dia inteiro}
\end{verbete}

\begin{verbete}{正}{zheng4}{5}[Radical 止]
  \significado{adj.}{reto; vertical; adequado; principal; positivo (matemática)}
  \significado{adv.}{agora mesmo; no processo de}
  \significado{v.}{corrigir; retificar}
  \veja{正}{zheng1}
\end{verbete}

\begin{verbete}{正常}{zheng4chang2}{5,11}
  \significado{adj.}{regular; normal; ordinário}
\end{verbete}

\begin{verbete}{正在}{zheng4zai4}{5,6}
  \significado{adv.}{no processo de; atualmente; em andamento}
  \significado{v.}{estar a~+~v.inf.; estar~+~v.ger.}
\end{verbete}

\begin{verbete}{正正}{zheng4zheng4}{5,5}
  \significado{adv.}{na hora certa; ordenadamente}
\end{verbete}

\begin{verbete}{正宗}{zheng4zong1}{5,8}
  \significado{adj.}{autêntico; genuíno; \emph{old school}; (fig.) tradicional}
\end{verbete}

\begin{verbete}{证件}{zheng4jian4}{7,6}
  \significado{s.}{documento de identificação; credencial; certificado; comprovante}
\end{verbete}

\begin{verbete}{证据}{zheng4ju4}{7,11}
  \significado{s.}{evidência; prova; testemunho}
\end{verbete}

\begin{verbete}{证实}{zheng4shi2}{7,8}
  \significado{v.}{confirmar (algo como verdadeiro); verificar}
\end{verbete}

\begin{verbete}{挣}{zheng4}{9}[Radical 手]
  \significado{v.}{ganhar dinheiro; esforçar-se para adquirir; lutar para se libertar}
\end{verbete}

\begin{verbete}{挣得}{zheng4de2}{9,11}
  \significado{v.}{ganhar renda ou dinheiro}
\end{verbete}

\begin{verbete}{挣钱}{zheng4qian2}{9,10}
  \significado{v.+compl.}{ganhar dinheiro}
\end{verbete}

\begin{verbete}{政府}{zheng4fu3}{9,8}
  \significado[个]{s.}{governo}
\end{verbete}

\begin{verbete}{政纲}{zheng4gang1}{9,7}
  \significado{s.}{programa ou plataforma política}
\end{verbete}

\begin{verbete}{政治局}{zheng4zhi4ju2}{9,8,7}
  \significado{s.}{o principal comitê de políticas de um partido comunista}
\end{verbete}

\begin{verbete}{之外}{zhi1wai4}{3,5}
  \significado{adv.}{lado de fora}
\end{verbete}

\begin{verbete}{支}{zhi1}{4}[Radical 支][Kangxi 65]
  \significado*{s.}{sobrenome Zhi}
  \significado{clas.}{para varetas como canetas e armas, para divisões do exército e para canções ou composições}
  \significado{v.}{sacar dinheiro; erguer; criar; suportar; sustentar}
\end{verbete}

\begin{verbete}{支承}{zhi1cheng2}{4,8}
  \significado{v.}{suportar o peso de (um edifício); suportar}
\end{verbete}

\begin{verbete}{支持}{zhi1chi2}{4,9}
  \significado[个]{s.}{apoio; suporte}
  \significado{v.}{apoiar; ser a favor de; suportar}
\end{verbete}

\begin{verbete}{支根}{zhi1gen1}{4,10}
  \significado{s.}{raiz ramificada; raízes de apoio; radícula}
\end{verbete}

\begin{verbete}{支票}{zhi1piao4}{4,11}
  \significado[本]{s.}{cheque (banco)}
\end{verbete}

\begin{verbete}{支应}{zhi1ying4}{4,7}
  \significado{v.}{lidar com; fornecer}
\end{verbete}

\begin{verbete}{支支吾吾}{zhi1zhi1wu2wu2}{4,4,7,7}
  \significado{v.}{falhar; murmurar; paralisar; gaguejar}
\end{verbete}

\begin{verbete}{只}{zhi1}{5}[Radical 口]
  \significado{clas.}{para pássaros, gatos, cãezinhos, etc.}
  \veja{只}{zhi3}
\end{verbete}

\begin{verbete}{只身}{zhi1shen1}{5,7}
  \significado{adv.}{sozinho; por si só}
\end{verbete}

\begin{verbete}{芝麻}{zhi1ma5}{6,11}
  \significado{s.}{semente de gergelim}
\end{verbete}

\begin{verbete}{知道}{zhi1dao4}{8,12}
  \significado{v.}{conhecer, saber}
\end{verbete}

\begin{verbete}{知道了}{zhi1dao4le5}{8,12,2}
  \significado{interj.}{Entendi!; OK!}
\end{verbete}

\begin{verbete}{知识}{zhi1shi5}{8,7}
  \significado[门]{s.}{conhecimento}
  \significado{s.}{intelectual}
\end{verbete}

\begin{verbete}{织}{zhi1}{8}[Radical 糸]
  \significado{v.}{tecer; tricotar}
\end{verbete}

\begin{verbete}{脂麻}{zhi1ma5}{10,11}
  \variante{芝麻}
\end{verbete}

\begin{verbete}{蜘蛛}{zhi1zhu1}{14,12}
  \significado{s.}{aranha}
\end{verbete}

\begin{verbete}{蜘蛛网}{zhi1zhu1wang3}{14,12,6}
  \significado{s.}{teia de aranha}
\end{verbete}

\begin{verbete}{执着}{zhi2zhuo2}{6,11}
  \significado{s.}{(budismo) apego}
  \significado{v.}{estar fortemente apegado a; ser dedicado; apegar-se a}
\end{verbete}

\begin{verbete}{直播}{zhi2bo1}{8,15}
  \significado{s.}{transmissão ao vivo; (agricultura) semeadura direta}
  \significado{v.}{(TV, rádio, Internet) transmitir ao vivo}
\end{verbete}

\begin{verbete}{直接}{zhi2jie1}{8,11}
  \significado{adj.}{direto (oposto: indireto 间接); imediato}
  \veja{间接}{jian4jie1}
\end{verbete}

\begin{verbete}{直译}{zhi2yi4}{8,7}
  \significado{s.}{tradução literal}
  \veja{意译}{yi4yi4}
\end{verbete}

\begin{verbete}{直译器}{zhi2yi4qi4}{8,7,16}
  \significado{s.}{interpretador (computação)}
\end{verbete}

\begin{verbete}{职业}{zhi2ye4}{11,5}
  \significado{adj.}{profissional}
  \significado{s.}{ocupação, profissão, vocação}
\end{verbete}

\begin{verbete}{职员}{zhi2yuan2}{11,7}
  \significado[个,位]{s.}{empregado; trabalhador de escritório; membro da equipe}
\end{verbete}

\begin{verbete}{殖}{zhi2}{12}[Radical 歹]
  \significado{v.}{crescer; reproduzir}
\end{verbete}

\begin{verbete}{只}{zhi3}{5}[Radical 口]
  \significado{adv.}{apenas; só}
  \veja{只}{zhi1}
\end{verbete}

\begin{verbete}{只得}{zhi3de5}{5,11}
  \significado{v.}{ser obrigado a; não ter outra alternativa senão}
\end{verbete}

\begin{verbete}{只读}{zhi3du2}{5,10}
  \significado{s.}{somente leitura (computação); \emph{read-only}}
\end{verbete}

\begin{verbete}{只顾}{zhi3gu4}{5,10}
  \significado{adv.}{exclusivamente preocupado (com uma coisa)}
  \significado{v.}{cuidar de apenas um aspecto}
\end{verbete}

\begin{verbete}{只好}{zhi3hao3}{5,6}
  \significado{adv.}{ser forçado a; ter que; sem nenhuma opção melhor; não ter outro remédio senão}
\end{verbete}

\begin{verbete}{只怕}{zhi3pa4}{5,8}
  \significado{adv.}{receio que\dots; talvez; muito provavelmente}
\end{verbete}

\begin{verbete}{只消}{zhi3xiao1}{5,10}
  \significado{conj.}{desde que}
\end{verbete}

\begin{verbete}{只要}{zhi3yao4}{5,9}
  \significado{conj.}{se apenas; contanto que}
\end{verbete}

\begin{verbete}{只要……就……}{zhi3yao4 jiu4}{5,9,12}
  \significado{conj.}{contanto que/desde que/se somente\dots, então\dots}
\end{verbete}

\begin{verbete}{只有……才……}{zhi3you3 cai2}{5,6,3}
  \significado{conj.}{só se\dots então\dots}
\end{verbete}

\begin{verbete}{纸}{zhi3}{7}[Radical 糸]
  \significado{clas.}{para documentos, cartas, etc.}
  \significado[张,沓]{s.}{papel}
\end{verbete}

\begin{verbete}{纸币}{zhi3bi4}{7,4}
  \significado[张]{s.}{nota (dinheiro); cédula}
\end{verbete}

\begin{verbete}{纸巾}{zhi3jin1}{7,3}
  \significado[张,包]{s.}{lenço; guardanapo; papel toalha}
\end{verbete}

\begin{verbete}{纸尿裤}{zhi3niao4ku4}{7,7,12}
  \significado{s.}{fralda descartável}
\end{verbete}

\begin{verbete}{纸烟}{zhi3yan1}{7,10}
  \significado{s.}{cigarro}
\end{verbete}

\begin{verbete}{纸张}{zhi3zhang1}{7,7}
  \significado{s.}{papel}
\end{verbete}

\begin{verbete}{指挥}{zhi3hui1}{9,9}
  \significado[个]{s.}{condutor (de uma orquestra)}
  \significado{v.}{conduzir; comandar; direcionar}
\end{verbete}

\begin{verbete}{指甲}{zhi3jia5}{9,5}
  \significado{s.}{unha da mão}
\end{verbete}

\begin{verbete}{指南针}{zhi3nan2zhen1}{9,9,7}
  \significado{s.}{bússola}
\end{verbete}

\begin{verbete}{至于}{zhi4yu2}{6,3}
  \significado{conj.}{para; quanto a; a respeiro de}
\end{verbete}

\begin{verbete}{志愿}{zhi4yuan4}{7,14}
  \significado{s.}{aspiração; ambição}
  \significado{v.}{ser voluntário}
\end{verbete}

\begin{verbete}{制裁}{zhi4cai2}{8,12}
  \significado{s.}{punição; sanção (inclusive econômica)}
  \significado{v.}{punir}
\end{verbete}

\begin{verbete}{治理}{zhi4li3}{8,11}
  \significado{s.}{governança; governo}
  \significado{v.}{gerir para melhor; administrar; por em ordem}
\end{verbete}

\begin{verbete}{治愈}{zhi4yu4}{8,13}
  \significado{v.}{curar; restaurar a saúde}
\end{verbete}

\begin{verbete}{致敬}{zhi4jing4}{10,12}
  \significado{v.}{saudar; prestar respeitos a; prestar homenagem a}
\end{verbete}

\begin{verbete}{智慧}{zhi4hui4}{12,15}
  \significado{s.}{sabedoria; inteligência}
\end{verbete}

\begin{verbete}{智商}{zhi4shang1}{12,11}
  \significado{s.}{quociente de inteligência, QI}
\end{verbete}

\begin{verbete}{智障}{zhi4zhang4}{12,13}
  \significado{adj./s.}{retardado}
\end{verbete}

\begin{verbete}{置疑}{zhi4yi2}{13,14}
  \significado{v.}{duvidar}
\end{verbete}

\begin{verbete}{中东}{zhong1dong1}{4,5}
  \significado*{s.}{Oriente Médio}
\end{verbete}

\begin{verbete}{中国}{zhong1guo2}{4,8}
  \significado*{s.}{China}
\end{verbete}

\begin{verbete}{中国城}{zhong1guo2cheng2}{4,8,9}
  \significado*{s.}{Bairro Chinês; \emph{Chinatown}}
  \veja{唐人街}{tang2ren2 jie1}
\end{verbete}

\begin{verbete}{中国科学院}{zhong1guo2 ke1xue2yuan4}{4,8,9,8,9}
  \significado*{s.}{Academia Chinesa de Ciências}
\end{verbete}

\begin{verbete}{中国人}{zhong1guo2ren2}{4,8,2}
  \significado{s.}{chinês; nascido na China}
\end{verbete}

\begin{verbete}{中国通}{zhong1guo2tong1}{4,8,10}
  \significado*{s.}{Conhecedor da China; especialista em tudo sobre a China}
\end{verbete}

\begin{verbete}{中间}{zhong1jian1}{4,7}
  \significado{adv.}{central; centro; no meio}
\end{verbete}

\begin{verbete}{中情局}{zhong1qing2ju2}{4,11,7}
  \significado*{s.}{Agência Central de Inteligência dos EUA, CIA (abreviação de 中央情报局)}
  \veja{中央情报局}{zhong1yang1 qing2bao4ju2}
\end{verbete}

\begin{verbete}{中秋节}{zhong1qiu1jie2}{4,9,5}
  \significado*{s.}{Festival do Meio-Outono, Festival do Bolo Lunar (15º dia do oitavo mês lunar)}
\end{verbete}

\begin{verbete}{中文}{zhong1wen2}{4,4}
  \significado{s.}{chinês, língua chinesa}
\end{verbete}

\begin{verbete}{中午}{zhong1wu3}{4,4}
  \significado[个]{s.}{meio-dia}
\end{verbete}

\begin{verbete}{中性}{zhong1xing4}{4,8}
  \significado{adj.}{neutro}
\end{verbete}

\begin{verbete}{中学}{zhong1xue2}{4,8}
  \significado[个]{s.}{escola ensino médio}
\end{verbete}

\begin{verbete}{中学生}{zhong1xue2sheng1}{4,8,5}
  \significado{s.}{estudante da escola ensino médio}
\end{verbete}

\begin{verbete}{中询}{zhong1 xun2}{4,8}
  \significado{adv.}{segunda dezena do mês; meio do mês; em meados do mês}
\end{verbete}

\begin{verbete}{中央情报局}{zhong1yang1 qing2bao4ju2}{4,5,11,7,7}
  \significado*{s.}{Agência Central de Inteligência dos EUA, CIA}
\end{verbete}

\begin{verbete}{中药}{zhong1yao4}{4,9}
  \significado[服,种]{s.}{medicina tradicional chinesa}
\end{verbete}

\begin{verbete}{钟}{zhong1}{9}[Radical 金]
  \significado*{s.}{sobrenome Zhong}
  \significado{clas.}{hora}
\end{verbete}

\begin{verbete}{钟室}{zhong1shi4}{9,9}
  \significado{s.}{campanário; sala do relógio}
\end{verbete}

\begin{verbete}{钟罩}{zhong1zhao4}{9,13}
  \significado{s.}{redoma; dossel de sino}
\end{verbete}

\begin{verbete}{锺}{zhong1}{14}[Radical 金]
  \variante{钟}
\end{verbete}

\begin{verbete}{种}{zhong3}{9}[Radical 禾]
  \significado{clas.}{para tipos, espécies e gêneros}
  \significado{s.}{tipo; espécie}
\end{verbete}

\begin{verbete}{种麻}{zhong3ma2}{9,11}
  \significado{s.}{planta de cânhamo (feminina)}
\end{verbete}

\begin{verbete}{种薯}{zhong3shu3}{9,16}
  \significado{s.}{tubérculo semente}
\end{verbete}

\begin{verbete}{种种}{zhong3zhong3}{9,9}
  \significado{adj.}{todos os tipos de}
\end{verbete}

\begin{verbete}{种子}{zhong3zi5}{9,3}
  \significado[颗,粒]{s.}{semente}
\end{verbete}

\begin{verbete}{种族灭绝}{zhong3zu2mie4jue2}{9,11,5,9}
  \significado{s.}{genocídio; extinção étnica}
\end{verbete}

\begin{verbete}{中意}{zhong4yi4}{4,13}
  \significado{s.}{ser do seu agrado, começar a gostar muito de algo ou de alguém}
\end{verbete}

\begin{verbete}{众}{zhong4}{6}[Radical 人]
  \significado*{s.}{abreviatura de 众议院, Câmara dos Deputados}
  \significado{adj.}{numeroso}
  \significado{adv.}{muitos}
  \significado{s.}{multidão}
  \veja{众议院}{zhong4yi4yuan4}
\end{verbete}

\begin{verbete}{众议院}{zhong4yi4yuan4}{6,5,9}
  \significado*{s.}{Casa baixa da Assembléia Bicameral; Câmara dos Deputados}
\end{verbete}

\begin{verbete}{种地}{zhong4di4}{9,6}
  \significado{v.}{cultivar; trabalhar a terra}
\end{verbete}

\begin{verbete}{重}{zhong4}{9}[Radical ⾥]
  \significado{adj.}{pesado}
  \veja{重}{chong2}
\end{verbete}

\begin{verbete}{重量}{zhong4liang4}{9,12}
  \significado[个]{s.}{peso}
\end{verbete}

\begin{verbete}{重要}{zhong4yao4}{9,9}
  \significado{adj.}{importante, significativo, principal}
\end{verbete}

\begin{verbete}{重重}{zhong4zhong4}{9,9}
  \significado{adv.}{fortemente; severamente}
  \veja{重重}{chong2chong2}
\end{verbete}

\begin{verbete}{周}{zhou1}{8}[Radical 口]
  \significado*{s.}{sobrenome Zhou; Dinastia Zhou (1046-256 BC)}
  \significado{adv.}{semanalmente}
  \significado{s.}{círculo; circunferência; ciclo; uma volta (em um circuito); semana}
  \significado{v.}{fazer um circuito; circular; ajudar financeiramente}
\end{verbete}

\begin{verbete}{周末}{zhou1mo4}{8,5}
  \significado{s.}{final-de-semana}
\end{verbete}

\begin{verbete}{洲}{zhou1}{9}[Radical 水]
  \significado{s.}{continente; ilha em um rio}
\end{verbete}

\begin{verbete}{轴承}{zhou2cheng2}{9,8}
  \significado{s.}{(mecânico) rolamento}
\end{verbete}

\begin{verbete}{咒骂}{zhou4ma4}{8,9}
  \significado{v.}{xingar; amaldiçoar; execrar}
\end{verbete}

\begin{verbete}{珠子}{zhu1zi5}{10,3}
  \significado[粒,颗]{s.}{pérola; contas}
\end{verbete}

\begin{verbete}{猪}{zhu1}{11}[Radical 犬]
  \significado[口,头]{s.}{porco; suíno}
\end{verbete}

\begin{verbete}{猪窠}{zhu1ke1}{11,13}
  \significado{s.}{chiqueiro}
\end{verbete}

\begin{verbete}{猪柳}{zhu1liu3}{11,9}
  \significado{s.}{filé de porco}
\end{verbete}

\begin{verbete}{猪笼}{zhu1long2}{11,11}
  \significado{s.}{estrutura cilíndrica de bambu ou arame usada para restringir um porco durante o transporte}
\end{verbete}

\begin{verbete}{猪头}{zhu1tou2}{11,5}
  \significado{s.}{tolo; idiota}
\end{verbete}

\begin{verbete}{竹编}{zhu2bian1}{6,12}
  \significado{s.}{vime; tecelagem de bambu}
\end{verbete}

\begin{verbete}{竹马}{zhu2ma3}{6,3}
  \significado{s.}{cavalo de bambu; vara de bambu usada como cavalo de brinquedo}
\end{verbete}

\begin{verbete}{竹排}{zhu2pai2}{6,11}
  \significado{s.}{jangada de bambu}
\end{verbete}

\begin{verbete}{竹子}{zhu2zi5}{6,3}
  \significado[棵,支,根]{s.}{bambu}
\end{verbete}

\begin{verbete}{逐步}{zhu2bu4}{10,7}
  \significado{adv.}{pouco a pouco; passo a passo; progressivamente}
\end{verbete}

\begin{verbete}{逐渐}{zhu2jian4}{10,11}
  \significado{adv.}{pouco a pouco; passo a passo; progressivamente}
\end{verbete}

\begin{verbete}{主席}{zhu3xi2}{5,10}
  \significado*[个,位]{s.}{Presidente (da China); Primeiro-Ministro}
\end{verbete}

\begin{verbete}{主席台}{zhu3xi2tai2}{5,10,5}
  \significado[个]{s.}{plataforma; tribuna}
\end{verbete}

\begin{verbete}{主席团}{zhu3xi2tuan2}{5,10,6}
  \significado{s.}{presídio}
\end{verbete}

\begin{verbete}{主义}{zhu3yi4}{5,3}
  \significado{s.}{ideologia; sufixo "ismo"}
\end{verbete}

\begin{verbete}{属}{zhu3}{12}[Radical 尸]
  \significado{v.}{juntar-se; fixar a atenção em; concentrar-se em}
  \veja{属}{shu3}
\end{verbete}

\begin{verbete}{嘱}{zhu3}{15}[Radical 口]
  \significado{v.}{juntar-se; implorar; incitar}
\end{verbete}

\begin{verbete}{嘱咐}{zhu3fu5}{15,8}
  \significado{v.}{ordenar; dizer; exortar}
\end{verbete}

\begin{verbete}{嘱托}{zhu3tuo1}{15,6}
  \significado{v.}{confiar uma tarefa a alguém}
\end{verbete}

\begin{verbete}{住}{zhu4}{7}[Radical 人]
  \significado{v.}{habitar; residir; morar; alojar-se}
\end{verbete}

\begin{verbete}{住处}{zhu4chu4}{7,5}
  \significado{s.}{morada; habitação; residência}
\end{verbete}

\begin{verbete}{住房}{zhu4fang2}{7,8}
  \significado{s.}{habitação}
\end{verbete}

\begin{verbete}{住所}{zhu4suo3}{7,8}
  \significado[处]{s.}{morada; habitação; residência}
\end{verbete}

\begin{verbete}{住宅}{zhu4zhai2}{7,6}
  \significado{s.}{residência}
\end{verbete}

\begin{verbete}{住嘴}{zhu4zui3}{7,16}
  \significado{interj.}{Cale-se!}
  \significado{v.}{calar; calar-se}
\end{verbete}

\begin{verbete}{助兴}{zhu4xing4}{7,6}
  \significado{v.+compl.}{animar as coisas; juntar-se à diversão}
\end{verbete}

\begin{verbete}{注册}{zhu4ce4}{8,5}
  \significado{v.}{inscrever-se; matricular-se; registrar-se}
\end{verbete}

\begin{verbete}{注册表}{zhu4ce4biao3}{8,5,8}
  \significado*{s.}{Registro do Windows}
\end{verbete}

\begin{verbete}{注册人}{zhu4ce4ren2}{8,5,2}
  \significado{s.}{registrante}
\end{verbete}

\begin{verbete}{注册商标}{zhu4ce4shang1biao1}{8,5,11,9}
  \significado{s.}{marca registrada}
\end{verbete}

\begin{verbete}{注意}{zhu4yi4}{8,13}
  \significado{v.}{tomar nota de, prestar atenção em}
\end{verbete}

\begin{verbete}{注意地}{zhu4yi4di4}{8,13,6}
  \significado{s.}{área de cuidado, de observação}
\end{verbete}

\begin{verbete}{注意力}{zhu4yi4li4}{8,13,2}
  \significado{s.}{atenção}
\end{verbete}

\begin{verbete}{注意力缺失症}{zhu4yi4li4que1shi1zheng4}{8,13,2,10,5,10}
  \significado{s.}{transtorno de déficit de atenção}
\end{verbete}

\begin{verbete}{驻军}{zhu4jun1}{8,6}
  \significado{s.}{guarnição}
  \significado{v.}{guarcener ou prover uma tropa}
\end{verbete}

\begin{verbete}{祝}{zhu4}{9}[Radical 示]
  \significado*{s.}{sobrenome Zhu}
  \significado{v.}{desejar (exprimir um bom desejo); congratular; rezar}
\end{verbete}

\begin{verbete}{祝祷}{zhu4dao3}{9,11}
  \significado{v.}{rezar; orar}
\end{verbete}

\begin{verbete}{祝福}{zhu4fu2}{9,13}
  \significado{s.}{bênçãos}
  \significado{v.}{desejar boa sorte a alguém}
\end{verbete}

\begin{verbete}{祝好}{zhu4hao3}{9,6}
  \significado{expr.}{desejo-lhe tudo de melhor! (ao encerrar uma correspondência)}
\end{verbete}

\begin{verbete}{祝贺}{zhu4he4}{9,9}
  \significado[个]{s.}{congratulações}
  \significado{v.}{congratular}
\end{verbete}

\begin{verbete}{祝酒}{zhu4jiu3}{9,10}
  \significado{v.}{parabenizar e fazer um brinde; brindar}
\end{verbete}

\begin{verbete}{祝寿}{zhu4shou4}{9,7}
  \significado{v.}{dar parabéns pelo aniversário (a uma pessoa idosa)}
\end{verbete}

\begin{verbete}{祝颂}{zhu4song4}{9,10}
  \significado{v.}{expressar bons desejos}
\end{verbete}

\begin{verbete}{祝谢}{zhu4xie4}{9,12}
  \significado{v.}{agradecer; dar parabéns}
\end{verbete}

\begin{verbete}{祝愿}{zhu4yuan4}{9,14}
  \significado{v.}{desejar}
\end{verbete}

\begin{verbete}{专业}{zhuan1ye4}{4,5}
  \significado[门,个]{s.}{área de atuação; especialidade}
\end{verbete}

\begin{verbete}{专业户}{zhuan1ye4hu4}{4,5,4}
  \significado{s.}{indústria caseira; empresa familiar produzindo um produto especial}
\end{verbete}

\begin{verbete}{专业化}{zhuan1ye4hua4}{4,5,4}
  \significado{s.}{especialização}
\end{verbete}

\begin{verbete}{专业教育}{zhuan1ye4jiao4yu4}{4,5,11,8}
  \significado{s.}{educação especializada; escola técnica}
\end{verbete}

\begin{verbete}{专业人才}{zhuan1ye4ren2cai2}{4,5,2,3}
  \significado{s.}{especialista (em uma área)}
\end{verbete}

\begin{verbete}{专业人士}{zhuan1ye4ren2shi4}{4,5,2,3}
  \significado{s.}{profissional}
\end{verbete}

\begin{verbete}{专业性}{zhuan1ye4xing4}{4,5,8}
  \significado{s.}{profissionalismo; expertise}
\end{verbete}

\begin{verbete}{砖}{zhuan1}{9}[Radical 石]
  \significado[块]{s.}{tijolo}
\end{verbete}

\begin{verbete}{转}{zhuan3}{8}[Radical 車]
  \significado{v.}{mudar de direção; transferir; encaminhar (correio); virar}
  \veja{转}{zhuan4}
\end{verbete}

\begin{verbete}{转产}{zhuan3chan3}{8,6}
  \significado{v.}{mudar a produção; mudar para novos produtos}
\end{verbete}

\begin{verbete}{转递}{zhuan3di4}{8,10}
  \significado{v.}{passar; retransmitir}
\end{verbete}

\begin{verbete}{转告}{zhuan3gao4}{8,7}
  \significado{v.}{comunicar; transmitir}
\end{verbete}

\begin{verbete}{转念}{zhuan3nian4}{8,8}
  \significado{v.}{ter dúvidas sobre algo; pensar melhor}
\end{verbete}

\begin{verbete}{转账}{zhuan3zhang4}{8,8}
  \significado{v.+compl.}{transferir entre contas; trazer à frente; incluir uma soma de dinheiro do balanço anterior no seguinte}
\end{verbete}

\begin{verbete}{传}{zhuan4}{6}[Radical 人]
  \significado{s.}{biografia; narrativa histórica; comentários; estação de retransmissão}
  \veja{传}{chuan2}
\end{verbete}

\begin{verbete}{转}{zhuan4}{8}[Radical 車]
  \significado{clas.}{para ações repetidas; para rotações (por minuto, etc.): RPM}
  \significado{v.}{circular sobre; dar voltas; andar por aí}
  \veja{转}{zhuan3}
\end{verbete}

\begin{verbete}{转悠}{zhuan4you5}{8,11}
  \significado{v.}{aparecer repetidamente; rolar; passear por aí}
\end{verbete}

\begin{verbete}{转游}{zhuan4you5}{8,12}
  \variante{转悠}
\end{verbete}

\begin{verbete}{妆}{zhuang1}{6}[Radical 女]
  \significado{s.}{maquiagem; adorno; enxoval; maquiagem e figurino de palco}
  \significado{v.}{maquiar-se, enfeitar-se}
\end{verbete}

\begin{verbete}{妆扮}{zhuang1ban4}{6,7}
  \variante{装扮}
\end{verbete}

\begin{verbete}{桩}{zhuang1}{10}[Radical 木]
  \significado{clas.}{para eventos, casos, transações, assuntos, etc.}
  \significado{s.}{toco, estaca, pilha}
\end{verbete}

\begin{verbete}{装}{zhuang1}{12}[Radical 衣]
  \significado{s.}{adorno; roupa; traje (de um ator em uma peça)}
  \significado{v.}{adornar; vestir; desepenhar um papel; fingir; instalar; consertar; embrulhar (algo em um saco); empacotar}
\end{verbete}

\begin{verbete}{装扮}{zhuang1ban4}{12,7}
  \significado{v.}{enfeitar; decorar; disfarçar-me; vestir-se}
\end{verbete}

\begin{verbete}{装备}{zhuang1bei4}{12,8}
  \significado{s.}{equipamento}
  \significado{v.}{equipar}
\end{verbete}

\begin{verbete}{装配}{zhuang1pei4}{12,10}
  \significado{v.}{montar; encaixar}
\end{verbete}

\begin{verbete}{撞车}{zhuang4che1}{15,4}
  \significado{v.+compl.}{(figurativo) colidir (opiniões, cronogramas, etc.) | ser o mesmo (assunto) | colidir (com outro veículo)}
\end{verbete}

\begin{verbete}{撞运气}{zhuang4yun4qi5}{15,7,4}
  \significado{v.}{confiar no destino; tentar a sorte}
\end{verbete}

\begin{verbete}{追赶}{zhui1gan3}{9,10}
  \significado{v.}{perseguir; acelerar; alcançar; ultrapassar}
\end{verbete}

\begin{verbete}{坠}{zhui4}{7}[Radical 土]
  \significado{v.}{cair; pesar; fazer vergar com o peso}
\end{verbete}

\begin{verbete}{坠落}{zhui4luo4}{7,12}
  \significado{v.}{cair}
\end{verbete}

\begin{verbete}{准}{zhun3}{10}[Radical 冫]
  \significado{adv.}{certamente; de acordo com; à luz de}
  \significado{v.}{permitir; conceder}
\end{verbete}

\begin{verbete}{桌}{zhuo1}{10}[Radical 木]
  \significado{clas.}{para mesas de convidados em um banquete etc.}
  \significado{s.}{mesa}
\end{verbete}

\begin{verbete}{桌布}{zhuo1bu4}{10,5}
  \significado[条,块,张]{s.}{computação:~plano de fundo da área de trabalho; toalha de mesa; papel de parede}
\end{verbete}

\begin{verbete}{桌灯}{zhuo1deng1}{10,6}
  \significado{s.}{luminária; lâmpada de mesa}
\end{verbete}

\begin{verbete}{桌机}{zhuo1ji1}{10,6}
  \significado{s.}{computador \emph{desktop}}
\end{verbete}

\begin{verbete}{桌面}{zhuo1mian4}{10,9}
  \significado{s.}{área de trabalho; mesa}
\end{verbete}

\begin{verbete}{桌球}{zhuo1qiu2}{10,11}
  \significado{s.}{bilhar; sinuca; mesa de ping-pong}
\end{verbete}

\begin{verbete}{桌游}{zhuo1you2}{10,12}
  \significado{s.}{jogo de tabuleiro}
\end{verbete}

\begin{verbete}{桌子}{zhuo1zi5}{10,3}
  \significado[张,套]{s.}{mesa}
\end{verbete}

\begin{verbete}{棹}{zhuo1}{12}[Radical 木]
  \variante{桌}
\end{verbete}

\begin{verbete}{着}{zhuo2}{11}[Radical 目]
  \significado{v.}{aplicar; contactar; usar; vestir (roupas)}
  \veja{着}{zhao1}
  \veja{着}{zhao2}
  \veja{着}{zhe5}
\end{verbete}

\begin{verbete}{着花}{zhuo2hua1}{11,7}
  \significado{s.}{floração}
  \significado{v.}{florescer}
  \veja{着花}{zhao2hua1}
\end{verbete}

\begin{verbete}{着手}{zhuo2shou3}{11,4}
  \significado{v.}{colocar a mão nisso; estabelecer; começar uma tarefa}
\end{verbete}

\begin{verbete}{着想}{zhuo2xiang3}{11,13}
  \significado{v.}{considerar (as necessidades de outras pessoas); pensar (para os outros)}
\end{verbete}

\begin{verbete}{着眼}{zhuo2yan3}{11,11}
  \significado{v.}{ter seus olhos em (um objetivo); ter algo em mente; concentrar-se}
\end{verbete}

\begin{verbete}{着装}{zhuo2zhuang1}{11,12}
  \significado{s.}{roupa; vestimenta}
  \significado{v.}{vestir}
\end{verbete}

\begin{verbete}{资}{zi1}{10}[Radical 貝]
  \significado{s.}{recursos; capital; dinheiro; despesa}
  \significado{v.}{fornecer; suprir}
\end{verbete}

\begin{verbete}{资助}{zi1zhu4}{10,7}
  \significado{s.}{subsídio}
  \significado{v.}{subsidiar; fornecer ajuda financeira}
\end{verbete}

\begin{verbete}{子弹}{zi3dan4}{3,11}
  \significado[粒,颗,发]{s.}{bala (de revólver)}
\end{verbete}

\begin{verbete}{紫}{zi3}{12}[Radical 糸]
  \significado{adj.}{púrpura; violeta}
\end{verbete}

\begin{verbete}{紫色}{zi3se4}{12,6}
  \significado{s.}{cor púrpura; cor violeta}
\end{verbete}

\begin{verbete}{字}{zi4}{6}[Radical 子]
  \significado[个]{s.}{carácter; letra; símbolo; palavra}
\end{verbete}

\begin{verbete}{字典}{zi4dian3}{6,8}
  \significado[本]{s.}{dicionário de caracteres chineses (contendo verbetes de caracteres únicos, em contraste com 词典 que contém verbetes para palavras de um ou mais caracteres)}
\end{verbete}

\begin{verbete}{字脚}{zi4jiao3}{6,11}
  \significado[典]{s.}{gancho no final da pincelada; serifa}
\end{verbete}

\begin{verbete}{字母}{zi4mu3}{6,5}
  \significado[个]{s.}{letra (do alfabeto)}
\end{verbete}

\begin{verbete}{字眼}{zi4yan3}{6,11}
  \significado[个]{s.}{palavras; redação}
\end{verbete}

\begin{verbete}{字字珠玉}{zi4zi4zhu1yu4}{6,6,10,5}
  \significado{expr.}{cada palavra é uma jóia}
  \significado{s.}{escrita magnífica}
\end{verbete}

\begin{verbete}{自动化}{zi4dong4hua4}{6,6,4}
  \significado{s.}{automação}
\end{verbete}

\begin{verbete}{自个儿}{zi4ge3r5}{6,3,2}
  \significado{pron.}{(dialeto) a si mesmo, por si mesmo}
\end{verbete}

\begin{verbete}{自己}{zi4ji3}{6,3}
  \significado{pron.}{a si próprio; próprio}
\end{verbete}

\begin{verbete}{自己动手}{zi4ji3dong4shou3}{6,3,6,4}
  \significado{v.}{fazer (algo) sozinho; ajudar-se a}
\end{verbete}

\begin{verbete}{自救}{zi4jiu4}{6,11}
  \significado{v.}{sair a si mesmo de problemas}
\end{verbete}

\begin{verbete}{自来水}{zi4lai2shui3}{6,7,4}
  \significado{s.}{água corrente, água da torneira}
\end{verbete}

\begin{verbete}{自然}{zi4ran2}{6,12}
  \significado{adj.}{natural}
  \significado{adv.}{naturalmente; de maneira natural}
  \significado{s.}{natureza}
\end{verbete}

\begin{verbete}{自燃}{zi4ran2}{6,16}
  \significado{s.}{combustão espontânea}
\end{verbete}

\begin{verbete}{自我}{zi4wo3}{6,7}
  \significado{pron.}{a si mesmo; eu próprio; auto\dots; ego (psicologia)}
\end{verbete}

\begin{verbete}{自我安慰}{zi4wo3'an1wei4}{6,7,6,15}
  \significado{v.}{confortar-se; consolar-se; tranquilizar-se}
\end{verbete}

\begin{verbete}{自我保存}{zi4wo3 bao3cun2}{6,7,9,6}
  \significado{v.}{autopreservação}
\end{verbete}

\begin{verbete}{自我吹嘘}{zi4wo3chui1xu1}{6,7,7,14}
  \significado{expr.}{tocar a própria buzina}
\end{verbete}

\begin{verbete}{自我催眠}{zi4wo3cui1mian2}{6,7,13,10}
  \significado{v.}{consolar-me; tranquilizar-me}
\end{verbete}

\begin{verbete}{自我的人}{zi4wo3de5ren2}{6,7,8,2}
  \significado{s.}{(minha, sua) própria pessoa; (afirmar) a própria personalidade}
\end{verbete}

\begin{verbete}{自我防卫}{zi4wo3fang2wei4}{6,7,6,3}
  \significado{s.}{defesa pessoal; auto-defesa}
\end{verbete}

\begin{verbete}{自我解嘲}{zi4wo3jie3chao2}{6,7,13,15}
  \significado{s.}{referir-se às próprias fraquezas ou falhas com humor autodepreciativo}
\end{verbete}

\begin{verbete}{自我介绍}{zi4wo3jie4shao4}{6,7,4,8}
  \significado{s.}{defesa pessoal; auto-defesa}
\end{verbete}

\begin{verbete}{自我批评}{zi4wo3pi1ping2}{6,7,7,7}
  \significado{s.}{autocrítica}
\end{verbete}

\begin{verbete}{自我实现}{zi4wo3shi2xian4}{6,7,8,8}
  \significado{s.}{psicologia:~auto-atualização; auto-realização}
\end{verbete}

\begin{verbete}{自我陶醉}{zi4wo3tao2zui4}{6,7,10,15}
  \significado{s.}{narcisista; auto-imbuído; satisfeito consigo mesmo}
\end{verbete}

\begin{verbete}{自我意识}{zi4wo3yi4shi2}{6,7,13,7}
  \significado{s.}{autoapresentação}
  \significado{v.}{apresentar-se}
\end{verbete}

\begin{verbete}{自行车}{zi4xing2che1}{6,6,4}
  \significado[辆]{s.}{bicicleta}
\end{verbete}

\begin{verbete}{自行车馆}{zi4xing2che1guan3}{6,6,4,11}
  \significado{s.}{estádio de ciclismo; velódromo}
\end{verbete}

\begin{verbete}{自行车架}{zi4xing2che1jia4}{6,6,4,9}
  \significado{s.}{suporte para bicicleta; bicicletário}
\end{verbete}

\begin{verbete}{自行车赛}{zi4xing2che1sai4}{6,6,4,14}
  \significado{s.}{corrida de bicicleta}
\end{verbete}

\begin{verbete}{自由}{zi4you2}{6,5}
  \significado{adj.}{livre, irrestrito}
  \significado[种]{s.}{liberdade}
\end{verbete}

\begin{verbete}{自由泳}{zi4you2yong3}{6,5,8}
  \significado{s.}{natação de estilo livre}
\end{verbete}

\begin{verbete}{自责}{zi4ze2}{6,8}
  \significado{v.}{culpar-se}
\end{verbete}

\begin{verbete}{棕褐色}{zong1he4se4}{12,14,6}
  \significado{s.}{cor sépia; bronzeado}
\end{verbete}

\begin{verbete}{总}{zong3}{9}[Radical 心]
  \significado{adv.}{em geral; completamente}
\end{verbete}

\begin{verbete}{总长}{zong3chang2}{9,4}
  \significado{s.}{comprimento total}
\end{verbete}

\begin{verbete}{总得}{zong3dei3}{9,11}
  \significado{adv.}{prestes a}
  \significado{v.}{dever; precisar}
\end{verbete}

\begin{verbete}{总督}{zong3du1}{9,13}
  \significado*{s.}{Governador-Geral; Governador; Vice-Rei}
\end{verbete}

\begin{verbete}{总价}{zong3jia4}{9,6}
  \significado{s.}{preço total}
\end{verbete}

\begin{verbete}{总结}{zong3jie2}{9,9}
  \significado[个]{s.}{currículo; resumo}
  \significado{v.}{concluir; resumir}
\end{verbete}

\begin{verbete}{总理}{zong3li3}{9,11}
  \significado*{s.}{Primeiro-Ministro}
\end{verbete}

\begin{verbete}{总台}{zong3tai2}{9,5}
  \significado{s.}{recepção; balcão de recepção}
\end{verbete}

\begin{verbete}{总统}{zong3tong3}{9,9}
  \significado*[个,位,名,届]{s.}{Presidente (de um país)}
\end{verbete}

\begin{verbete}{总务}{zong3wu4}{9,5}
  \significado{s.}{divisão de assuntos gerais; assuntos gerais; pessoa responsável geral}
\end{verbete}

\begin{verbete}{总线}{zong3xian4}{9,8}
  \significado{s.}{barramento (computador); \emph{computer bus}}
\end{verbete}

\begin{verbete}{总站}{zong3zhan4}{9,10}
  \significado{s.}{terminal}
\end{verbete}

\begin{verbete}{总值}{zong3zhi2}{9,10}
  \significado{s.}{valor total}
\end{verbete}

\begin{verbete}{赱}{zou3}{6}
  \variante{走}
\end{verbete}

\begin{verbete}{走}{zou3}{7}[Radical 走][Kangxi 156]
  \significado{v.}{andar; caminhar}
\end{verbete}

\begin{verbete}{走鬼}{zou3gui3}{7,9}
  \significado{s.}{vendedor ambulante sem licença}
\end{verbete}

\begin{verbete}{走过}{zou3guo4}{7,6}
  \significado{v.}{passar}
\end{verbete}

\begin{verbete}{走去}{zou3qu4}{7,5}
  \significado{v.}{caminhar até (para)}
\end{verbete}

\begin{verbete}{走绳}{zou3sheng2}{7,11}
  \significado{v.}{andar na corda bamba}
  \veja{走索}{zou3suo3}
\end{verbete}

\begin{verbete}{走势}{zou3shi4}{7,8}
  \significado{s.}{caminho; tendência}
\end{verbete}

\begin{verbete}{走索}{zou3suo3}{7,10}
  \significado{v.}{andar na corda bamba}
  \veja{走绳}{zou3sheng2}
\end{verbete}

\begin{verbete}{走秀}{zou3xiu4}{7,7}
  \significado{s.}{desfile de moda}
  \significado{v.}{andar na passarela (em um desfile de moda)}
\end{verbete}

\begin{verbete}{走卒}{zou3zu2}{7,8}
  \significado{s.}{lacaio (masculino); peão (isto é, soldado de infantaria); servo}
\end{verbete}

\begin{verbete}{奏效}{zou4xiao4}{9,10}
  \significado{v.}{mostrar resultados, ser eficaz}
\end{verbete}

\begin{verbete}{租}{zu1}{10}[Radical 禾]
  \significado{s.}{imposto sobre propriedade urbana ou rural}
  \significado{v.}{alugar; tomar de aluguel}
\end{verbete}

\begin{verbete}{租船}{zu1chuan2}{10,11}
  \significado{v.}{fretar um navio; alugar um navio}
\end{verbete}

\begin{verbete}{租房}{zu1fang2}{10,8}
  \significado{v.}{alugar um apartamento}
\end{verbete}

\begin{verbete}{租金}{zu1jin1}{10,8}
  \significado{s.}{aluguel}
  \veja{租钱}{zu1qian5}
\end{verbete}

\begin{verbete}{租赁}{zu1lin4}{10,10}
  \significado{v.}{contratar; alugar}
\end{verbete}

\begin{verbete}{租钱}{zu1qian5}{10,10}
  \significado{s.}{aluguel}
  \veja{租金}{zu1jin1}
\end{verbete}

\begin{verbete}{租让}{zu1rang4}{10,5}
  \significado{v.}{alugar; alugar (a propriedade de alguém para outra pessoa)}
\end{verbete}

\begin{verbete}{租用}{zu1yong4}{10,5}
  \significado{v.}{contratar; alugar; alugar (algo de alguém)}
\end{verbete}

\begin{verbete}{租约}{zu1yue1}{10,6}
  \significado{s.}{aluguel}
\end{verbete}

\begin{verbete}{足}{zu2}{7}[Radical 足][Kangxi 157]
  \significado{adj.}{amplo}
  \significado{s.}{pé}
  \significado{v.}{ser suficiente}
  \veja{足}{ju4}
\end{verbete}

\begin{verbete}{足球}{zu2qiu2}{7,11}
  \significado[个]{s.}{futebol; bola de futebol}
\end{verbete}

\begin{verbete}{足球场}{zu2qiu2chang3}{7,11,6}
  \significado{s.}{campo de futebol}
\end{verbete}

\begin{verbete}{足球队}{zu2qiu2dui4}{7,11,4}
  \significado{s.}{time de futebol}
\end{verbete}

\begin{verbete}{足球迷}{zu2qiu2mi2}{7,11,9}
  \significado{s.}{fã de futebol}
\end{verbete}

\begin{verbete}{足球赛}{zu2qiu2sai4}{7,11,14}
  \significado{s.}{competição de futebol; partida de futebol}
\end{verbete}

\begin{verbete}{足球协会}{zu2qiu2xie2hui4}{7,11,6,6}
  \significado*{s.}{Associação de Futebol}
\end{verbete}

\begin{verbete}{足月}{zu2yue4}{7,4}
  \significado{s.}{gestação completa}
\end{verbete}

\begin{verbete}{足足}{zu2zu2}{7,7}
  \significado{adv.}{tanto quanto; extremamente; completamente; não menos que}
\end{verbete}

\begin{verbete}{族}{zu2}{11}[Radical 方]
  \significado{s.}{raça; nacionalidade; etnia; clã; por extensão, grupo social}
\end{verbete}

\begin{verbete}{诅咒}{zu3zhou4}{7,8}
  \significado{v.}{amaldiçoar}
\end{verbete}

\begin{verbete}{阻击}{zu3ji1}{7,5}
  \significado{v.}{verificar; parar}
\end{verbete}

\begin{verbete}{祖国}{zu3guo2}{9,8}
  \significado{s.}{pátria, terra natal}
\end{verbete}

\begin{verbete}{钻戒}{zuan4jie4}{10,7}
  \significado[只]{s.}{anel de diamante}
\end{verbete}

\begin{verbete}{钻石}{zuan4shi2}{10,5}
  \significado[颗]{s.}{diamante}
\end{verbete}

\begin{verbete}{嘴巴}{zui3ba5}{16,4}
  \significado[张]{s.}{boca}
  \significado[个]{s.}{bofetada na cara}
\end{verbete}

\begin{verbete}{嘴巴子}{zui3ba5zi5}{16,4,3}
  \significado{s.}{tapa; bofetada}
\end{verbete}

\begin{verbete}{最}{zui4}{12}[Radical 冂]
  \significado{adv.}{o mais; o melhor; a coisa mais\dots; grau superlativo relativo de superioridade}
\end{verbete}

\begin{verbete}{最初}{zui4chu1}{12,7}
  \significado{adj.}{inicial; original; primário}
  \significado{adv.}{inicialmente; originalmente}
\end{verbete}

\begin{verbete}{最多}{zui4duo1}{12,6}
  \significado{adv.}{no máximo; máximo}
\end{verbete}

\begin{verbete}{最高}{zui4gao1}{12,10}
  \significado{adj.}{altíssimo; supremo; mais alto}
\end{verbete}

\begin{verbete}{最好}{zui4hao3}{12,6}
  \significado{adv.}{ser melhor que}
  \significado{v.}{(você) estar melhor (faça o que sugerimos); querer ser o melhor}
\end{verbete}

\begin{verbete}{最后}{zui4hou4}{12,6}
  \significado{adj.}{final; último}
  \significado{adv.}{finalmente}
\end{verbete}

\begin{verbete}{最佳}{zui4jia1}{12,8}
  \significado{adj.}{melhor (atleta, filme etc); ótimo}
\end{verbete}

\begin{verbete}{最近}{zui4jin4}{12,7}
  \significado{adv.}{ultimamente; recentemente}
\end{verbete}

\begin{verbete}{最善}{zui4shan4}{12,12}
  \significado{adj.}{ótimo; o melhor}
\end{verbete}

\begin{verbete}{最少}{zui4shao3}{12,4}
  \significado{adv.}{finalmente}
\end{verbete}

\begin{verbete}{最先}{zui4xian1}{12,6}
  \significado{adv.}{o primeiro}
\end{verbete}

\begin{verbete}{最新}{zui4xin1}{12,13}
  \significado{adv.}{mais recente; mais novo}
\end{verbete}

\begin{verbete}{最优}{zui4you1}{12,6}
  \significado{adj.}{ótimo}
\end{verbete}

\begin{verbete}{最远}{zui4yuan3}{12,7}
  \significado{adv.}{mais distante; mais longe}
\end{verbete}

\begin{verbete}{最终}{zui4zhong1}{12,8}
  \significado{adv.}{pelo menos; finalmente;}
  \significado{s.}{final; ultimato}
\end{verbete}

\begin{verbete}{罪犯}{zui4fan4}{13,5}
  \significado{s.}{criminoso}
\end{verbete}

\begin{verbete}{罪行}{zui4xing2}{13,6}
  \significado{s.}{crime; ofensa}
\end{verbete}

\begin{verbete}{醉}{zui4}{15}[Radical 酉]
  \significado{v.}{embriagar-se; ficar bêbado}
\end{verbete}

\begin{verbete}{昨}{zuo2}{9}[Radical 日]
  \significado{s.}{ontem}
\end{verbete}

\begin{verbete}{昨日}{zuo2ri4}{9,4}
  \significado{adv.}{ontem}
\end{verbete}

\begin{verbete}{昨天}{zuo2tian1}{9,4}
  \significado{adv.}{ontem}
\end{verbete}

\begin{verbete}{昨晚}{zuo2wan3}{9,11}
  \significado{adv.}{noite passada; ontem à noite}
\end{verbete}

\begin{verbete}{昨夜}{zuo2ye4}{9,8}
  \significado{adv.}{noite passada}
\end{verbete}

\begin{verbete}{左}{zuo3}{5}[Radical 工]
  \significado*{s.}{sobrenome Zuo}
  \significado{p.l.}{esquerda}
\end{verbete}

\begin{verbete}{左边}{zuo3bian5}{5,5}
  \significado{s.}{esquerda; lado esquerdo}
\end{verbete}

\begin{verbete}{左面}{zuo3mian4}{5,9}
  \significado{s.}{esquerda; lado esquerdo}
\end{verbete}

\begin{verbete}{左派}{zuo3pai4}{5,9}
  \significado{s.}{esquerda (política); esquerdista}
\end{verbete}

\begin{verbete}{左倾}{zuo3qing1}{5,10}
  \significado{s.}{esquerdista; progressivo}
\end{verbete}

\begin{verbete}{左袒}{zuo3tan3}{5,10}
  \significado{v.}{ser tendencioso; ser parcial para; favorecer um lado; tomar partido com}
\end{verbete}

\begin{verbete}{左舷}{zuo3xian2}{5,11}
  \significado{s.}{porto (lado de um navio)}
\end{verbete}

\begin{verbete}{左翼}{zuo3yi4}{5,17}
  \significado{s.}{esquerda (política)}
\end{verbete}

\begin{verbete}{左右}{zuo3you4}{5,5}
  \significado{adv.}{cerca de; aproximadamente}
\end{verbete}

\begin{verbete}{坐}{zuo4}{7}[Radical 土]
  \significado*{s.}{sobrenome Zuo}
  \significado{v.}{sentar-se; andar de carro, ônibus, trem, avião, etc.}
\end{verbete}

\begin{verbete}{坐标}{zuo4biao1}{7,9}
  \significado{s.}{coordenada (geometria)}
\end{verbete}

\begin{verbete}{坐车}{zuo4che1}{7,4}
  \significado{v.}{andar de carro, ônibus, trem, etc.}
\end{verbete}

\begin{verbete}{坐垫}{zuo4dian4}{7,9}
  \significado[块]{s.}{assento (motocicleta); almofada}
\end{verbete}

\begin{verbete}{坐好}{zuo4hao3}{7,6}
  \significado{v.}{sentar-se corretamente; sentar direito}
\end{verbete}

\begin{verbete}{坐享}{zuo4xiang3}{7,8}
  \significado{v.}{curtir algo sem levantar um dedo}
\end{verbete}

\begin{verbete}{座标}{zuo4biao1}{10,9}
  \variante{坐标}
\end{verbete}

\begin{verbete}{座位}{zuo4wei4}{10,7}
  \significado[个]{s.}{assento; lugar}
\end{verbete}

\begin{verbete}{座子}{zuo4zi5}{10,3}
  \significado{s.}{soquete; pedestal; sela}
\end{verbete}

\begin{verbete}{做}{zuo4}{11}[Radical 人]
  \significado{v.}{fazer}
\end{verbete}

\begin{verbete}{做法}{zuo4fa3}{11,8}
  \significado[个]{s.}{método para fazer; prática; receita; maneira de lidar com algo; método de trabalho}
\end{verbete}

\begin{verbete}{做饭}{zuo4fan4}{11,7}
  \significado{v.}{preparar uma refeição; cozinhar}
\end{verbete}

\begin{verbete}{做活}{zuo4huo2}{11,9}
  \significado{v.}{trabalhar para ganhar a vida (especialmente de mulher costureira)}
\end{verbete}

\begin{verbete}{做生活}{zuo4sheng1huo2}{11,5,9}
  \significado{v.}{fazer tabalhos manuais}
\end{verbete}

\begin{verbete}{做戏}{zuo4xi4}{11,6}
  \significado{v.}{atuar em uma peça; fazer uma peça}
\end{verbete}

\begin{verbete}{做眼}{zuo4yan3}{11,11}
  \significado{v.}{agir como um guia; trabalhar como espião}
\end{verbete}

\begin{verbete}{做作}{zuo4zuo5}{11,7}
  \significado{adj.}{afetado; artificial}
\end{verbete}

\begin{verbete}{酢}{zuo4}{12}[Radical 酉]
  \significado{v.}{brindar o anfitrião com vinho}
\end{verbete}

%%%%% EOF %%%%%

\end{multicols}

\newpage
\pagestyle{plain}
\chapter{Termos Gramaticais Chineses}
\begin{center}
\begin{tblr}[m]{lll}
substantivo/nome  & \textbf{s.}        & 名词                     \\
pronome           & \textbf{pron.}     & 代词                     \\
numeral           & \textbf{num.}      & 数词                     \\
classificador     & \textbf{clas.}     & 量词                     \\
verbo             & \textbf{v.}        & 动词                     \\
verbo auxiliar    & \textbf{v.aux.}    & 助动词                   \\
verbo+complemento & \textbf{v.+compl.} & 动宾式\hspace{1em}离合词 \\
adjetivo          & \textbf{adj.}      & 形容词                   \\
advérbio          & \textbf{adv.}      & 副词                     \\
preposição        & \textbf{prep.}     & 介词                     \\
conjunção         & \textbf{conj.}     & 连词                     \\
partícula         & \textbf{part.}     & 助词                     \\
interjeição       & \textbf{interj.}   & 叹词                     \\
prefixo           & \textbf{pref.}     & 前缀                     \\
sufixo            & \textbf{suf.}      & 后缀                     \\
expressão         & \textbf{expr.}     & 熟语                     \\
\end{tblr}
\end{center}


\newpage
\chapter{Radicais Chineses}
\chapter{Radicais Chineses}

\begin{multicols}{3}
\begin{tabular}{rllll}
\hline
  Nº & Radical & Variante & Tradução & Pinyin \\
\hline
  1  & 一 && um           & \pinyin{yi1}         \\
  2  & 丨 && linha        & \pinyin{shu4}        \\
  3  & 丶 && ponto        & \pinyin{dian3}       \\
  4  & 丿 &乀,乁 & golpear & \pinyin{pie3}       \\
  5  & 乙 &乚,乛 & segundo & \pinyin{yi3}         \\
  6  & 亅 && gancho       & \pinyin{gou1}        \\
  7  & 二 && dois         & \pinyin{er4}         \\
  8  & 亠 && membro       & \pinyin{tou2}        \\
  9  & 人 &亻 & homem     & \pinyin{ren2}        \\
 10  & 儿 && pernas       & \pinyin{er2}         \\
 11  & 入 && entra        & \pinyin{ru4}         \\
 12  & 八 &丷 & oito      & \pinyin{ba1}         \\
 13  & 冂 && caixa de baixo & \pinyin{jiong3}    \\
 14  & 冖 && sobre        & \pinyin{mi4}         \\
 15  & 冫 && gelo         & \pinyin{bing1}       \\
 16  & 几 && mesa         & \pinyin{ji1},\pinyin{ji3} \\
 17  & 凵 && caixa aberta & \pinyin{qu3}         \\
 18  & 刀 &刂 & faca      & \pinyin{dao1}        \\
 19  & 力 && poder        & \pinyin{li4}         \\
 20  & 勹 && embrulho     & \pinyin{bao1}        \\
 21  & 匕 && colher       & \pinyin{bi3}         \\
 22  & 匚 && caixa aberta & \pinyin{fang1}       \\
 23  & 匸 && esconderijo anexo & \pinyin{xi3}    \\
 24  & 十 && dez          & \pinyin{shi2}        \\
 25  & 卜 && místico      & \pinyin{bu3}         \\
 26  & 卩 && foca         & \pinyin{jie2}        \\
 27  & 厂 && penhasco     & \pinyin{han4}        \\
 28  & 厶 && privado      & \pinyin{si1}         \\
 29  & 又 && novamente    & \pinyin{you4}        \\
 30  & 口 && boca         & \pinyin{kou3}        \\
 31  & 囗 && lugar        & \pinyin{wei2}        \\
 32  & 土 && Terra        & \pinyin{tu3}         \\
 33  & 士 && guerreiro    & \pinyin{shi4}        \\
 34  & 夂 && ir           & \pinyin{zhi1}        \\
 35  & 夊 && devagar      & \pinyin{sui1}        \\
 36  & 夕 && tarde        & \pinyin{xi1}         \\
 37  & 大 && grande       & \pinyin{da4}         \\
 38  & 女 && mulher       & \pinyin{nv3}         \\
 39  & 子 && criança      & \pinyin{zi3}         \\
 40  & 宀 && cobertura    & \pinyin{mian2}       \\
 41  & 寸 && polegada     & \pinyin{cun4}        \\
 42  & 小 && pequeno      & \pinyin{xiao3}       \\
 43  & 尢 &尣 & coxo      & \pinyin{you2}        \\
 44  & 尸 && cadáver      & \pinyin{shi1}        \\
 45  & 屮 && brotar       & \pinyin{che4}        \\
 46  & 山 && montanha     & \pinyin{shan1}       \\
 47  & 川 &巛,巜& rio     & \pinyin{chuan1}      \\
 48  & 工 && trabalho     & \pinyin{gong1}       \\
 49  & 己 && a si mesmo   & \pinyin{ji3}         \\
 50  & 巾 && turbante     & \pinyin{jin1}        \\
 51  & 干 && seco         & \pinyin{gan1}        \\
 52  & 幺 && fio curto    & \pinyin{yao1}        \\
 53  & 广 && vasto        & \pinyin{guang3}      \\
 54  & 廴 && passo longo  & \pinyin{yin3}        \\
 55  & 廾 && duas mãos    & \pinyin{gong3}       \\
 56  & 弋 && atirar flecha & \pinyin{yi4}        \\
 57  & 弓 && arco         & \pinyin{gong1}       \\
 58  & 彐 &彑 & focinho   & \pinyin{ji4}         \\
 59  & 彡 && cerdas       & \pinyin{shan1}       \\
 60  & 彳 && dupla        & \pinyin{chi4}        \\
 61  & 心 &忄& coração    & \pinyin{xin1}        \\
 62  & 戈 && lança        & \pinyin{ge1}         \\
 63  & 户 && por          & \pinyin{hu4}         \\
 64  & 手 &扌& mão        & \pinyin{shou3}       \\
 65  & 支 && ramo         & \pinyin{zhi1}        \\
 66  & 攴 &攵 & batida    & \pinyin{pu1}         \\
 67  & 文 && escrita      & \pinyin{wen2}        \\
 68  & 斗 && mergulhador  & \pinyin{dou3}        \\
 69  & 斤 && eixo         & \pinyin{jin1}        \\
 70  & 方 && quadrado     & \pinyin{fang1}       \\
 71  & 无 && não          & \pinyin{wu2}         \\
 72  & 日 && sol          & \pinyin{ri4}         \\
 73  & 曰 && dizer        & \pinyin{yue1}        \\
 74  & 月 && lua          & \pinyin{yue4}        \\
 75  & 木 && árvore       & \pinyin{mu4}         \\
 76  & 欠 && falta        & \pinyin{qian4}       \\
 77  & 止 && parar        & \pinyin{zhi3}        \\
 78  & 歹 && morte        & \pinyin{dai3}        \\
 79  & 殳 && arma         & \pinyin{shu1}        \\
 80  & 母 && mãe          & \pinyin{mu3}         \\
 81  & 比 && comparar     & \pinyin{bi3}         \\
 82  & 毛 && pelo & \pinyin{mao2}        \\
 83  & 氏 && clã & \pinyin{shi4}        \\
 84  & 气 && ar & \pinyin{qi4}         \\
 85  & 水 &氵 & água & \pinyin{shui3}       \\
 86  & 火 &灬 & fogo & \pinyin{huo3}        \\
 87  & 爪 &爫 & garra & \pinyin{zhao3}       \\
 88  & 父 && pai & \pinyin{fu4}         \\
 89  & 爻 && linha & \pinyin{yao2}        \\
 90  & 爿 &丬 & meio tronco & \pinyin{pan2}      \\
 91  & 片 && fatia & \pinyin{pian4}       \\
 92  & 牙 && dente & \pinyin{ya2}         \\
 93  & 牛 &牜 & vaca & \pinyin{niu2}        \\
 94  & 犬 &犭 & cão & \pinyin{quan3}       \\
 95  & 玄 && profundo & \pinyin{xuan2}       \\
 96  & 玉 &王 & jade & \pinyin{yu4}         \\
 97  & 瓜 && melão & \pinyin{gua1}         \\
 98  & 瓦 && telha & \pinyin{wa3}         \\
 99  & 甘 && doce & \pinyin{gan1}         \\
100  & 生 && vida & \pinyin{sheng1}         \\
101  & 用 && usar & \pinyin{yong4}         \\
102  & 田 && campo & \pinyin{tian2}         \\
103  & 疋 && roupa & \pinyin{pi3} \\
104  & 疒 && doença & \pinyin{ne4} \\
105  & 癶 && pegadas & \pinyin{bo1} \\
106  & 白 && branco & \pinyin{bai2} \\
107  & 皮 && pele & \pinyin{pi2} \\
108  & 皿 && prato & \pinyin{min3} \\
109  & 目 && olho & \pinyin{mu4} \\
110  & 矛 && lança & \pinyin{mao2} \\
111  & 矢 && seta & \pinyin{shi3} \\
112  & 石 && pedra & \pinyin{shi2} \\
113  & 示 &礻 & espírito & \pinyin{shi4} \\
114  & 禸 && rastrear & \pinyin{rou2} \\
115  & 禾 && grão & \pinyin{he2} \\
116  & 穴 && caverna & \pinyin{xue2} \\
117  & 立 && ficar em pé & \pinyin{li4} \\
118  & 竹 &⺮ & bambu & \pinyin{zhu2} \\
119  & 米 && arroz & \pinyin{mi3} \\
120  & 糸 &纟& seda & \pinyin{mi4} \\
121  & 缶 && pote & \pinyin{fou3} \\
122  & 网 &罒 & rede & \pinyin{wang3} \\
123  & 羊 && ovelha & \pinyin{yang2} \\
124  & 羽 && pena & \pinyin{yu3} \\
125  & 老 && velho & \pinyin{lao3} \\
126  & 而 && e & \pinyin{er2} \\
127  & 耒 && arado & \pinyin{lei3} \\
128  & 耳 && orelha & \pinyin{er3} \\
129  & 聿 && escova & \pinyin{yu4} \\
130  & 肉 && carne & \pinyin{rou4} \\
131  & 臣 && ministro & \pinyin{chen2} \\
132  & 自 && auto- & \pinyin{zi4} \\
133  & 至 && chegar & \pinyin{zhi4} \\
134  & 臼 && argamassa & \pinyin{jiu4} \\
135  & 舌 && língua & \pinyin{she2} \\
136  & 舛 && opor & \pinyin{chuan3} \\
137  & 舟 && barco & \pinyin{zhou1} \\
138  & 艮 && pausa & \pinyin{gen3} \\
139  & 色 && cor & \pinyin{se4} \\
140  & 艸 &艹 & grama & \pinyin{cao3} \\
141  & 虍 && tigre & \pinyin{hu1} \\
142  & 虫 && inseto & \pinyin{chong2} \\
143  & 血 && sangue & \pinyin{xue4} \\
144  & 行 && andar & \pinyin{xing2} \\
145  & 衣 &衤 & roupa & \pinyin{yi1} \\
146  & 襾 &覀 & oeste & \pinyin{ya4} \\
147  & 見 &见 & ver & \pinyin{jian4} \\
148  & 角 && chifre & \pinyin{jiao3} \\
149  & 言 &讠 & palavra & \pinyin{yan2} \\
150  & 谷 && vale & \pinyin{gu3} \\
151  & 豆 && grão & \pinyin{dou4} \\
152  & 豕 && porco & \pinyin{shi3} \\
153  & 豸 && texugo & \pinyin{zhi4} \\
154  & 貝 &贝 & concha & \pinyin{bei4} \\
155  & 赤 && vermelho & \pinyin{chi4} \\
156  & 走 && andar & \pinyin{zou3} \\
157  & 足 &⻊ & pé & \pinyin{zu2} \\
158  & 身 && corpo & \pinyin{shen1} \\
159  & 車 &车 & carro & \pinyin{che1} \\
160  & 辛 && amargo & \pinyin{xin1} \\
161  & 辰 && manhã & \pinyin{chen2} \\
162  & 辵 &辶 & caminhar & \pinyin{chuo4} \\
163  & 邑 &阝 & cidade & \pinyin{yi4} \\
164  & 酉 && vinho & \pinyin{you3} \\
165  & 釆 && distinto & \pinyin{bian4} \\
166  & 里 && aldeia & \pinyin{li3} \\
167  & 金 && ouro & \pinyin{jin1} \\
168  & 長 &长 & longo & \pinyin{zhang3} \\
169  & 門 &门 & portão & \pinyin{men2} \\
170  & 阜 &阝 & monte & \pinyin{fu4} \\
171  & 隶 && escravo & \pinyin{li4} \\
172  & 隹 && pássaro de cauda curta & \pinyin{zhui1} \\
173  & 雨 && chuva & \pinyin{yu3} \\
174  & 青 && azul & \pinyin{qing1} \\
175  & 非 && errado & \pinyin{fei1} \\
176  & 面 && face & \pinyin{mian4} \\
177  & 革 && couro & \pinyin{ge2} \\
178  & 韋 &韦 & couro tingido & \pinyin{wei2} \\
179  & 韭 && parecia & \pinyin{jiu3} \\
180  & 音 && som & \pinyin{yin1} \\
181  & 頁 &页 & folha & \pinyin{ye4} \\
182  & 風 &风 & vento & \pinyin{feng1} \\
183  & 飛 &飞 & mosca & \pinyin{fei1} \\
184  & 食 &饣,飠 & alimento & \pinyin{shi2} \\
185  & 首 && cabeça & \pinyin{shou3} \\
186  & 香 && perfume & \pinyin{xiang1} \\
187  & 馬 &马 & cavalo & \pinyin{ma3} \\
188  & 骨 && osso & \pinyin{gu3} \\
189  & 高 && alto & \pinyin{gao1} \\
190  & 髟 && cabelo & \pinyin{biao1} \\
191  & 鬥 && luta & \pinyin{dou4} \\
192  & 鬯 && vinho & \pinyin{chang4} \\
193  & 鬲 && separado & \pinyin{ge2} \\
194  & 鬼 && fantasma & \pinyin{gui3} \\
195  & 魚 &鱼 & peixe & \pinyin{yu2} \\
196  & 鳥 &鸟 & pássaro & \pinyin{niao3} \\
197  & 鹵 && sal & \pinyin{lu3} \\
198  & 鹿 && veado & \pinyin{lu4} \\
199  & 麥 &麦 & trigo & \pinyin{mai4} \\
200  & 麻 && cânhamo & \pinyin{ma2} \\
201  & 黃 && amarelo & \pinyin{huang4} \\
202  & 黍 && nação & \pinyin{shu3} \\
203  & 黑 && preto & \pinyin{hei1} \\
204  & 黹 && costura & \pinyin{zhi3} \\
205  & 黽 &黾 & rã & \pinyin{mian3} \\
206  & 鼎 && tripé & \pinyin{ding3} \\
207  & 鼓 && tambor & \pinyin{gu3} \\
208  & 鼠 &鼡 & rato & \pinyin{shu3} \\
209  & 鼻 && nariz & \pinyin{bi2} \\
210  & 齊 &齐 & até & \pinyin{qi2} \\
211  & 齒 &齿 & dente & \pinyin{chi3} \\
212  & 龍 &龙 & dragão & \pinyin{long2} \\
213  & 龜 &龟 & tartaruga & \pinyin{gui1} \\
214  & 龠 && flauta & \pinyin{yue4} \\
\end{tabular}
\end{multicols}


\printindex[stroke]
\printindex[radical]
\printindex[pinyin]

\end{document}

%%%%% EOF %%%%
