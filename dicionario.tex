%%%%%%%%%%%%%%%%%%%%%%%%%%%%%%%%%%%%%%%%%
% XeTeX
%
% Dicionário Chinês -> Português
% Autor: Luiz Eduardo Roncato Cordeiro
%
% Licença:
% CC BY-NC-SA 3.0 (http://creativecommons.org/licenses/by-nc-sa/3.0/)
%%%%%%%%%%%%%%%%%%%%%%%%%%%%%%%%%%%%%%%%%

%\documentclass[a4paper,12pt,twoside,openany,draft]{memoir}
\documentclass[a4paper,12pt,twoside,openany]{memoir}

\usepackage[usenames,dvipsnames]{color}
\usepackage[utf8]{inputenc}
\usepackage[brazil]{babel}
\usepackage{fontspec}
\usepackage{xltxtra}
\usepackage{xeCJK}
\usepackage{xpinyin}
\usepackage{xunicode}
\usepackage{xltxtra}
\usepackage{multicol}
\usepackage{fancyhdr}
\usepackage{imakeidx}
\usepackage{ifthen}
\usepackage{tocloft}
\usepackage{xparse}
\usepackage{enumitem}
\usepackage{zhnumber}
\usepackage{wasysym}

\setCJKmainfont{AR PL UKai CN}
\setCJKsansfont{AR PL UMing CN}

\makeindex[columns=3, title=Índice, intoc]

\setlength{\parindent}{0em}
\setlength{\parskip}{0.5em}
\setlength{\columnsep}{1em}
\setlength{\columnseprule}{0.2mm}

\xpinyinsetup{ratio={.5},vsep={1em},multiple={\color{Sepia}}}

% Headers & footers
\fancyhead[L]{\rightmark} % Top left header
\fancyhead[R]{\leftmark}  % Top right header
\renewcommand{\headrulewidth}{1.4pt} % Rule under the header
\fancyfoot[C]{\thepage} % Bottom center footer
\renewcommand{\footrulewidth}{1.4pt} % Rule under the header
\pagestyle{fancy} % Use the custom headers and footers throughout the document

\setlength{\headheight}{16pt}
\addtolength{\topmargin}{-0.5pt}

\NewDocumentCommand\mylist{>{\SplitList{|}}m}
{
\setlength{\leftmargin}{0em}
\begin{enumerate}[nosep,left=0em]
  \ProcessList{#1}{ \insertitem }
\end{enumerate}
}
\newcommand\insertitem[1]{\item #1}

\newenvironment{verbete}[2][]
  {\leavevmode
  \ifthenelse{\equal{#1}{}}%
    {\markboth{#2}{#2}\index{#2}}%
    {\markboth{#2«\pinyin{#1}»}{#2«\pinyin{#1}»}\index{#2«\pinyin{#1}»}}
  \begin{minipage}[t][][t]{\linewidth}
  {\LARGE \textbf{#2}}
  }
  {
  \end{minipage}
  }

%\newcommand{\entry}[4]{\markboth{#1}{#1}\textbf{#1}\ {(#2)}\ \textit{#3}\ $\bullet$\ {#4}}  
% Defines the command to print each word on the page, \markboth{}{} prints the first word on the page in the top left header and the last word in the top right

\newenvironment{pronuncia}[2][]
  {\ifthenelse{\equal{#1}{}}{}{#1}%
  «\pinyin{#2}»\\
  }
  {}

\newcommand{\significado}[3][]{%
\if\relax\detokenize{#2}\relax\else$\bullet$\ (\textit{#2})\fi%
\ifthenelse{\equal{#1}{}}{}{ [P.C.:\xpinyin*{#1}]}%
{\small \mylist{#3}}%
\vspace{1ex}
}

\newcommand{\e}[1]{\textcolor{OliveGreen}{#1}}
%\newcommand{\pc}[1]{P.C.:#1}

\makeatletter
\let\old@makechapterhead\@makechapterhead
% Taken from http://mirrors.ctan.org/macros/latex/unpacked/report.cls
\def\fake@makechapterhead#1{%
  \vspace*{50\p@}%
  {\parindent \z@ \raggedright \normalfont
    \ifnum \c@secnumdepth >\m@ne
        \huge\bfseries \strut%\@chapapp\space \thechapter
        \par\nobreak
        \vskip 20\p@
    \fi
    \interlinepenalty\@M
    \Huge \bfseries #1\par\nobreak
    \vskip 40\p@
  }
  \markboth{#1}{\thechapter}
}
\newcommand{\newchapterhead}{\let\@makechapterhead\fake@makechapterhead}
\newcommand{\restorechapterhead}{\let\@makechapterhead\old@makechapterhead}
\makeatother

\DeclareRobustCommand{\&}{%
    \ifdim\fontdimen1\font>0pt
        \textsl{\symbol{`\&}}%
    \else
        \symbol{`\&}%
    \fi
}

%%%
%%% Documento começa aqui!
%%%

\begin{document}

\newchapterhead

\begin{titlingpage} % Suppresses displaying the page number on the title page and the subsequent page counts as page 1
	
	\raggedleft % Right align the title page
	
	\rule{1pt}{\textheight} % Vertical line
	\hspace{0.05\textwidth} % Whitespace between the vertical line and title page text
	\parbox[b]{0.75\textwidth}{ % Paragraph box for holding the title page text, adjust the width to move the title page left or right on the page
		
		{\Huge\bfseries 汉葡词典}\\[2\baselineskip] % Title
		{\large\textsc{Dicionário Chinês-Português para o \\Curso de Chinês do Instituto Confúcio}}\\[4\baselineskip] % Subtitle or further description
        {\Large\textsc{罗学凯}\\\tiny Luiz Eduardo Roncato Cordeiro} % Author name, lower case for consistent small caps
		
		\vspace{0.5\textheight} % Whitespace between the title block and the publisher
		
		{\noindent \zhtoday}\\[\baselineskip] % Publisher and logo
	}

\end{titlingpage}

\tableofcontents

\newpage

\chapter{Termos Gramaticais Chineses}

\begin{tabular}{lll}
nome/substantivo       & \textbf{n.}        & 名词 \\
palavra de lugar       & \textbf{p.d.l.}    & 处所词 \\
palavra de localização & \textbf{p.l.}      & 方位词 \\
palavra de tempo       & \textbf{p.t.}      & 时间词 \\
verbo                  & \textbf{v.}        & 动词 \\
verbo direcional       & \textbf{v.d.}      & 趣向\hspace{1em}动词 \\
verbo optativo         & \textbf{v.o.}      & 能缘\hspace{1em}动词 \\
adjetivo               & \textbf{adj.}      & 形容词 \\
numeral                & \textbf{num.}      & 数词 \\
palavra classificadora & \textbf{p.c.}      & 两量词 \\
pronome                & \textbf{pron.}     & 代词 \\
interrogativo          & \textbf{interr.}   & 疑问词 \\
advérbio               & \textbf{adv.}      & 副词 \\
preposição             & \textbf{prep.}     & 介词 \\
conjunção              & \textbf{conj.}     & 连词 \\
partícula              & \textbf{part.}     & 助词 \\
sujeito                & \textbf{suj.}      & 主语 \\
objeto                 & \textbf{obj.}      & 宾语 \\
atributo               & \textbf{atrib.}    & 定语 \\
adjunto adverbial      & \textbf{a.adv.}    & 状语 \\
complemento            & \textbf{compl.}    & 补语 \\
verbo+complemento      & \textbf{v.+compl.} & 动宾式\hspace{1em}离合词 \\
expressão idiomática   & \textbf{expr.}     & \\
\end{tabular}

\newpage

\chapter{汉葡词典}

%%%
%%% Estou ordenando as palavras em ordem alfabética por pinyin.
%%% Obs: Para as palavras diferentes com o mesmo pinyin, a que tem o menor 
%%%      número de traços vem antes.
%%%

\clearpage
%%%
%%% A
%%%
\section*{A}
\addcontentsline{toc}{section}{A}
\begin{multicols}{2}

\begin{verbete}[啊]{a0}
\significado{a0}{part.}{
    ah!, oh!|
    no final da sentença para expressar entusiasmo|
    no final da sentença para expressar impaciência ou o que é óbvio|
    no final de uma ordem, aviso, etc|
    no final da sentença para expressar questionamento|
    para indicar uma pausa deliberada|
    para enumerar itens
}
\end{verbete}

\begin{verbete}[矮]{ai3}
\significado{ai3}{adj.}{
    baixo em estatura,dimensão,grau ou ranque
}
\end{verbete}

\begin{verbete}[爱]{ai4}
\significado{ai4}{n.}{
    amor; afeição
}
\significado{ai4}{v.}{
    amar; ter afeição; gostar; gostar de|
    inclinado a (fazer alguma coisa); tender (a acontecer)
}
\end{verbete}

\begin{verbete}[爱好]{ai4hao4}
\significado{ai4hao4}{n.}{
    passatempo; interesse|
    \pc{个}
}
\significado{ai4hao4}{v.}{
    ter algo como hobby;
    ter prazer em fazer algo; gostar de (fazer alguma coisa)
}
\end{verbete}

\begin{verbete}[爱人]{ai4ren0}
\significado{ai4ren0}{n.}{
    marido, esposa|
    amado, amada|
    querido, querida|
    \pc{个}
}
\end{verbete}

\begin{verbete}[安静]{an1jing4}
\significado{an1jing4}{adj.}{
    sossegado, sossegada; silencioso, silenciosa; pacato, pacata
}
\end{verbete}

\end{multicols}

%%%
%%% B
%%%
\section*{B}
\addcontentsline{toc}{section}{B}
%\begin{multicols*}{2}

\begin{verbete}[ba0]{吧}
\begin{pronuncia}{ba0}
\significado{part.}{
sugestão; requisição ou comando leve|
consentimento ou aprovação|
confirmação de uma suposição|
alguma dúvida|
pausa entre suposições em alternativas
}
\end{pronuncia}
\end{verbete}

\begin{verbete}[ba1]{八}
\begin{pronuncia}{ba1}
\significado{num.}{
oito; 8
}
\end{pronuncia}
\end{verbete}

\begin{verbete}[Ba1xi1]{巴西}
\begin{pronuncia}{Ba1xi1}
\significado{n.}{
Brasil
}
\end{pronuncia}
\end{verbete}

\begin{verbete}[ba4ba0]{爸爸}
\begin{pronuncia}{ba4ba0}
\significado[个,位]{n.}{
papai, pai
}
\end{pronuncia}
\end{verbete}

\begin{verbete}[bai2]{白}
\begin{pronuncia}{bai2}
\significado{adj.}{
branco; claro; puro; límpido; simples; em branco; grátis
}
\significado{adv.}{
em vão; sem propósito; por nada
}
\significado{n.}{
parte falada na ópera; diálogo|
dialeto
}
\end{pronuncia}
\end{verbete}

\begin{verbete}[bai2cai4]{白菜}
\begin{pronuncia}{bai2cai4}
\significado[棵,个]{n.}{
repolho chinês
}
\end{pronuncia}
\end{verbete}

\begin{verbete}[bai2se4]{白色}
\begin{pronuncia}{bai2se4}
\significado{n.}{
cor branca
}
\end{pronuncia}
\end{verbete}

\begin{verbete}[bai2tian1]{白天}
\begin{pronuncia}{bai2tian1}
\significado{p.t.}{
dia; de dia
}
\end{pronuncia}
\end{verbete}

\begin{verbete}[bai3]{百}
\begin{pronuncia}{bai3}
\significado{num.}{
cem; centena; cento; 100
}
\end{pronuncia}
\end{verbete}

\begin{verbete}[ban1jia1]{搬家}
\begin{pronuncia}{ban1jia1}
\significado{v.+compl.}{
mudar-se de casa
}
\end{pronuncia}
\end{verbete}

\begin{verbete}[ban4]{半}
\begin{pronuncia}{ban4}
\significado{num.}{
meio, meia
}
\end{pronuncia}
\end{verbete}

\begin{verbete}[ban4]{办}
\begin{pronuncia}{ban4}
\significado{v.}{
tratar|
fazer
}
\end{pronuncia}
\end{verbete}

\begin{verbete}[ban4gong1shi4]{办公室}
\begin{pronuncia}{ban4gong1shi4}
\significado[间]{n.}{
gabinete; escritório
}
\end{pronuncia}
\end{verbete}

\begin{verbete}[bang1zhu4]{帮助}
\begin{pronuncia}{bang1zhu4}
\significado{n.}{
ajuda
}
\significado{v.}{
ajudar
}
\end{pronuncia}
\end{verbete}

\begin{verbete}[bao1]{包}
\begin{pronuncia}{bao1}
\significado{p.c.}{
pacote; saco; sacola
}
\end{pronuncia}
\end{verbete}

\begin{verbete}[bao1zi0]{包子}
\begin{pronuncia}{bao1zi0}
\significado[个]{n.}{
pão recheado cozido no vapor
}
\end{pronuncia}
\end{verbete}

\begin{verbete}[bao4chou2]{报酬}
\begin{pronuncia}{bao4chou2}
\significado{n.}{
recompensa; remuneração
}
\end{pronuncia}
\end{verbete}

\begin{verbete}[bei1]{杯}
\begin{pronuncia}{bei1}
\significado{p.c.}{
para copos, canecas, xícaras, taças, etc
}
\end{pronuncia}
\end{verbete}

\begin{verbete}[bei1zi0]{杯子}
\begin{pronuncia}{bei1zi0}
\significado[个,只]{n.}{
copo; caneca; xícara; taça
}
\end{pronuncia}
\end{verbete}

\begin{verbete}[bei3bian0]{北边}
\begin{pronuncia}{bei3bian0}
\significado{p.l.}{
lado norte
}
\end{pronuncia}
\end{verbete}

\begin{verbete}[bei3fang1]{北方}
\begin{pronuncia}{bei3fang1}
\significado{p.l.}{
norte
}
\end{pronuncia}
\end{verbete}

\begin{verbete}[Bei3jing1]{北京}
\begin{pronuncia}{Bei3jing1}
\significado{n.}{
Beijing(Pequim); Capital da China
}
\end{pronuncia}
\end{verbete}

\begin{verbete}[bei3mian4]{北面}
\begin{pronuncia}{bei3mian4}
\significado{p.l.}{
lado norte
}
\end{pronuncia}
\end{verbete}

\begin{verbete}[bei4]{背}
\begin{pronuncia}{bei4}
\significado{n.}{
costas
}
\end{pronuncia}
\end{verbete}

\begin{verbete}[bei4zi0]{被子}
\begin{pronuncia}{bei4zi0}
\significado[床]{n.}{
colcha
}
\end{pronuncia}
\end{verbete}

\begin{verbete}[ben3]{本}
\begin{pronuncia}{ben3}
\significado{p.c.}{
para livros, dicionários, etc
}
\end{pronuncia}
\end{verbete}

\begin{verbete}[ben3zi0]{本子}
\begin{pronuncia}{ben3zi0}
\significado[本]{n.}{
caderno
}
\end{pronuncia}
\end{verbete}

\begin{verbete}[bi2zi0]{鼻子}
\begin{pronuncia}{bi2zi0}
\significado[个,只]{n.}{
nariz
}
\end{pronuncia}
\end{verbete}

\begin{verbete}[bi3]{笔}
\begin{pronuncia}{bi3}
\significado[支,枝]{n.}{
caneta; lápis
}
\end{pronuncia}
\end{verbete}

\begin{verbete}[bi3]{比}
\begin{pronuncia}{bi3}
\significado{prep.}{
que; do que
}
\end{pronuncia}
\end{verbete}

\begin{verbete}[bi3jiao4]{比较}
\begin{pronuncia}{bi3jiao4}
\significado{adv.}{
comparativamente; relativamente
}
\end{pronuncia}
\end{verbete}

\begin{verbete}[bi3sa4bing3]{比萨饼}
\begin{pronuncia}{bi3sa4bing3}
\significado[张]{n.}{
pizza
}
\end{pronuncia}
\end{verbete}

\begin{verbete}[bi3sai4]{比赛}
\begin{pronuncia}{bi3sai4}
\significado[场,次]{n.}{
competição; concurso
}
\end{pronuncia}
\end{verbete}

\begin{verbete}[biao1zhun3]{标准}
\begin{pronuncia}{biao1zhun3}
\significado{adj.}{
criterioso; padronizado
}
\significado{n.}{
critério; padrão
}
\end{pronuncia}
\end{verbete}

\begin{verbete}[bian4]{遍}
\begin{pronuncia}{bian4}
\significado{p.c.}{
para a repetição de ações de leitura, fala ou escrita
}
\end{pronuncia}
\end{verbete}

\begin{verbete}[bie2]{别}
\begin{pronuncia}{bie2}
\significado{adv.}{
nada de (pedir a alguém para não fazer); não
}
\end{pronuncia}
\end{verbete}

\begin{verbete}[bie2de0]{别的}
\begin{pronuncia}{bie2de0}
\significado{pron.}{
outro, outra
}
\end{pronuncia}
\end{verbete}

\begin{verbete}[bie2ren0]{别人}
\begin{pronuncia}{bie2ren0}
\significado{pron.}{
outrem; outra pessoa; outras pessoas
}
\end{pronuncia}
\end{verbete}

\begin{verbete}[bing1]{冰}
\begin{pronuncia}{bing1}
\significado[块]{n.}{
gelo
}
\end{pronuncia}
\end{verbete}

\begin{verbete}[bing1tian1-xue3di4]{冰天雪地}
\begin{pronuncia}{bing1tian1-xue3di4}
\significado{expr.}{
o mundo de gelo e neve
}
\end{pronuncia}
\end{verbete}

\begin{verbete}[bing1qiu2]{冰球}
\begin{pronuncia}{bing1qiu2}
\significado{n.}{
hóquei no gelo
}
\end{pronuncia}
\end{verbete}

\begin{verbete}[bing4]{病}
\begin{pronuncia}{bing4}
\significado[场]{n.}{
doença
}
\significado{v.}{
adoecer; estar doente
}
\end{pronuncia}
\end{verbete}

\begin{verbete}[bo2wu4guan3]{博物馆}
\begin{pronuncia}{bo2wu4guan3}
\significado{n.}{
museu
}
\end{pronuncia}
\end{verbete}

\begin{verbete}[bo2zi0]{脖子}
\begin{pronuncia}{bo2zi0}
\significado[个]{n.}{
pescoço
}
\end{pronuncia}
\end{verbete}

\begin{verbete}[bu0]{不}
\begin{pronuncia}{bu0}
\significado{adv.}{
não (em expressões ``v.$+$不$+$v.'')
}
\end{pronuncia}
\begin{pronuncia}{bu2}[(antes de quarto tom)]
\significado{adv.}{
não
}
\end{pronuncia}
\begin{pronuncia}{bu4}
\significado{adv.}{
não
}
\end{pronuncia}
\end{verbete}

\begin{verbete}[bu2]{不}
\begin{pronuncia}{bu2}[(antes de quarto tom)]
\significado{adv.}{
não
}
\end{pronuncia}
\begin{pronuncia}{bu0}
\significado{adv.}{
não (em expressões ``v.$+$不$+$v.'')
}
\end{pronuncia}
\begin{pronuncia}{bu4}
\significado{adv.}{
não
}
\end{pronuncia}
\end{verbete}

\begin{verbete}[bu2cuo4]{不错}
\begin{pronuncia}{bu2cuo4}
\significado{adj.}{
não (é) mau; bastante bom; bom, boa
}
\end{pronuncia}
\end{verbete}

\begin{verbete}[bu2guo4]{不过}
\begin{pronuncia}{bu2guo4}
\significado{conj.}{
mas; contudo
}
\end{pronuncia}
\end{verbete}

\begin{verbete}[bu2ke4qi0]{不客气}
\begin{pronuncia}{bu2ke4qi0}
\significado{expr.}{
de nada; não há de que
}
\end{pronuncia}
\end{verbete}

\begin{verbete}[bu2yao4]{不要}
\begin{pronuncia}{bu2yao4}
\significado{v.o.}{
nada de (pedir a alguém não fazer); não
}
\end{pronuncia}
\end{verbete}

\begin{verbete}[bu2yong4]{不用}
\begin{pronuncia}{bu2yong4}
\significado{v.o.}{
não precisar
}
\end{pronuncia}
\end{verbete}

\begin{verbete}[bu4]{不}
\begin{pronuncia}{bu4}
\significado{adv.}{
não
}
\end{pronuncia}
\begin{pronuncia}{bu0}
\significado{adv.}{
não (em expressões``v.$+$不$+$v.'')
}
\end{pronuncia}
\begin{pronuncia}{bu2}[(antes de quarto tom)]
\significado{adv.}{
não
}
\end{pronuncia}
\end{verbete}

\begin{verbete}[bu4tong2]{不同}
\begin{pronuncia}{bu4tong2}
\significado{adj.}{
diferente
}
\end{pronuncia}
\end{verbete}

%\end{multicols*}

%%%
%%% C
%%%
\section*{C}
\addcontentsline{toc}{section}{C}
%\begin{multicols*}{2}

\begin{verbete}[cai4]{菜}
\begin{pronuncia}{cai4}
\significado[棵]{n.}{
hortaliça; verdura
}
\significado[样,道,盘]{n.}{
prato (de comida)
}
\end{pronuncia}
\end{verbete}

\begin{verbete}[cai4dan1]{菜单}
\begin{pronuncia}{cai4dan1}
\significado[份,张]{n.}{
menu; ementa; cardápio
}
\end{pronuncia}
\end{verbete}

\begin{verbete}[cao3]{草}
\begin{pronuncia}{cao3}
\significado[棵,撮,株,根]{n.}{
erva
}
\end{pronuncia}
\end{verbete}

\begin{verbete}[cao3di4]{草地}
\begin{pronuncia}{cao3di4}
\significado[片]{n.}{
relva; pastagem
}
\end{pronuncia}
\end{verbete}

\begin{verbete}[can1guan3]{参观}
\begin{pronuncia}{can1guan3}
\significado{v.}{
visitar
}
\end{pronuncia}
\end{verbete}

\begin{verbete}[can1jia1]{参加}
\begin{pronuncia}{can1jia1}
\significado{v.}{
participar em; tomar parte em|
assistir
}
\end{pronuncia}
\end{verbete}

\begin{verbete}[can1ting1]{餐厅}
\begin{pronuncia}{can1ting1}
\significado[家]{n.}{
cantina
}
\significado[间]{n.}{
sala de jantar
}
\end{pronuncia}
\end{verbete}

\begin{verbete}[ce4suo3]{厕所}
\begin{pronuncia}{ce4suo3}
\significado[间,处]{n.}{
sanitário; toilette
}
\end{pronuncia}
\end{verbete}

\begin{verbete}[ceng2]{层}
\begin{pronuncia}{ceng2}
\significado{ceng2}{p.c.}{
para andar, piso
}
\end{pronuncia}
\end{verbete}

\begin{verbete}[ci2dai4]{磁带}
\begin{pronuncia}{ci2dai4}
\significado[盘,盒]{n.}{
cassete
}
\end{pronuncia}
\end{verbete}

\begin{verbete}[ci2pan2]{磁盘}
\begin{pronuncia}{ci2pan2}
\significado{n.}{
disquete
}
\end{pronuncia}
\end{verbete}

\begin{verbete}[ci2dian3]{词典}
\begin{pronuncia}{ci2dian3}
\significado[部,本]{n.}{
dicionário
}
\end{pronuncia}
\end{verbete}

\begin{verbete}[ci4]{次}
\begin{pronuncia}{ci4}
\significado{p.c.}{
para frequência (número de vezes)
}
\end{pronuncia}
\end{verbete}

\begin{verbete}[cha2]{茶}
\begin{pronuncia}{cha2}
\significado[杯,壶]{n.}{
chá
}
\end{pronuncia}
\end{verbete}

\begin{verbete}[cha4bu0duo1]{差不多}
\begin{pronuncia}{cha4bu0duo1}
\significado{adj.}{
mais ou menos
}
\end{pronuncia}
\end{verbete}

\begin{verbete}[chang2]{长}
\begin{pronuncia}{chang2}
\significado{adj.}{
comprido; longo
}
\end{pronuncia}
\end{verbete}

\begin{verbete}[chang2cheng2]{长成}
\begin{pronuncia}{chang2cheng2}
\significado{n.}{
Grande Muralha
}
\end{pronuncia}
\end{verbete}

\begin{verbete}[chang2chang2]{常常}
\begin{pronuncia}{chang2chang2}
\significado{adv.}{
frequentemente; com frequência
}
\end{pronuncia}
\end{verbete}

\begin{verbete}[chang4]{唱}
\begin{pronuncia}{chang4}
\significado{v.}{
cantar
}
\end{pronuncia}
\end{verbete}

\begin{verbete}[chang4ge1]{唱歌}
\begin{pronuncia}{chang4ge1}
\significado{v.+compl.}{
cantar
}
\end{pronuncia}
\end{verbete}

\begin{verbete}[chao1shi4]{超市}
\begin{pronuncia}{chao1shi4}
\significado[家]{n.}{
supermercado
}
\end{pronuncia}
\end{verbete}

\begin{verbete}[chao3]{吵}
\begin{pronuncia}{chao3}
\significado{adj.}{
barulhento, barulhenta; ruidoso, ruidosa;
}
\end{pronuncia}
\end{verbete}

\begin{verbete}[chao3]{炒}
\begin{pronuncia}{chao3}
\significado{v.}{
saltear
}
\end{pronuncia}
\end{verbete}

\begin{verbete}[che1]{车}
\begin{pronuncia}{che1}
\significado[辆]{n.}{
veículo; viatura
}
\end{pronuncia}
\end{verbete}

\begin{verbete}[che1ci4]{车次}
\begin{pronuncia}{che1ci4}
\significado{n.}{
número do trem
}
\end{pronuncia}
\end{verbete}

\begin{verbete}[che1ku4]{车库}
\begin{pronuncia}{che1ku4}
\significado{n.}{
garagem
}
\end{pronuncia}
\end{verbete}

\begin{verbete}[che1pai2]{车牌}
\begin{pronuncia}{che1pai2}
\significado{n.}{
matrícula; placa de carro
}
\end{pronuncia}
\end{verbete}

\begin{verbete}[che1shui3-ma3long2]{车水马龙}
\begin{pronuncia}{che1shui3-ma3long2}
\significado{expr.}{
tráfego engarrafado; engarrafamento
}
\end{pronuncia}
\end{verbete}

\begin{verbete}[che1zhan4]{车站}
\begin{pronuncia}{che1zhan4}
\significado[处,个]{n.}{
estação; paragem
}
\end{pronuncia}
\end{verbete}

\begin{verbete}[chen4shan1]{衬衫}
\begin{pronuncia}{chen4shan1}
\significado[件]{n.}{
camisa
}
\end{pronuncia}
\end{verbete}

\begin{verbete}[cheng2du1]{成都}
\begin{pronuncia}{cheng2du1}
\significado{n.}{
Chengdu
}
\end{pronuncia}
\end{verbete}

\begin{verbete}[cheng2ji4]{成绩}
\begin{pronuncia}{cheng2ji4}
\significado[项,个]{n.}{
nota; classificação
}
\end{pronuncia}
\end{verbete}

\begin{verbete}[cheng2shi4]{城市}
\begin{pronuncia}{cheng2shi4}
\significado[座]{n.}{
cidade
}
\end{pronuncia}
\end{verbete}

\begin{verbete}[cheng2se4]{橙色}
\begin{pronuncia}{cheng2se4}
\significado{n.}{
cor de laranja
}
\end{pronuncia}
\end{verbete}

\begin{verbete}[cheng2zhi1]{橙汁}
\begin{pronuncia}{cheng2zhi1}
\significado[瓶,杯,罐,盒]{n.}{
suco de laranja
}
\end{pronuncia}
\end{verbete}

\begin{verbete}[cheng2fa2]{惩罚}
\begin{pronuncia}{cheng2fa2}
\significado{v.}{
punir; penalizar
}
\end{pronuncia}
\end{verbete}

\begin{verbete}[cheng2chu3]{惩处}
\begin{pronuncia}{cheng2chu3}
\significado{v.}{
punir; penalizar
}
\end{pronuncia}
\end{verbete}

\begin{verbete}[chi1]{吃}
\begin{pronuncia}{chi1}
\significado{v.}{
comer
}
\end{pronuncia}
\end{verbete}

\begin{verbete}[chi1dao4]{迟到}
\begin{pronuncia}{chi1dao4}
\significado{v.}{
chegar atrasado; tardar
}
\end{pronuncia}
\end{verbete}

\begin{verbete}[chong1jing3]{憧憬}
\begin{pronuncia}{chong1jing3}
\significado{v.}{
ansiar por; esperar por
}
\end{pronuncia}
\end{verbete}

\begin{verbete}[chong3wu4]{宠物}
\begin{pronuncia}{chong3wu4}
\significado{n.}{
animal de estimação
}
\end{pronuncia}
\end{verbete}

\begin{verbete}[chou2lao2]{酬劳}
\begin{pronuncia}{chou2lao2}
\significado{n.}{
recompensa
}
\end{pronuncia}
\end{verbete}

\begin{verbete}[chu1]{出}
\begin{pronuncia}{chu1}
\significado{v.d.}{
sair
}
\end{pronuncia}
\end{verbete}

\begin{verbete}[chu1ban3]{出版}
\begin{pronuncia}{chu1ban3}
\significado{v.}{
publicar; editar
}
\end{pronuncia}
\end{verbete}

\begin{verbete}[chu1ban3she4]{出版社}
\begin{pronuncia}{chu1ban3she4}
\significado{n.}{
editora
}
\end{pronuncia}
\end{verbete}

\begin{verbete}[chu1kou3]{出口}
\begin{pronuncia}{chu1kou3}
\significado{n.}{
exportação
}
\significado{v.}{
exportar
}
\end{pronuncia}
\end{verbete}

\begin{verbete}[chu1qu0]{出来}
\begin{pronuncia}{chu1lai0}
\significado{v.d.}{
sair; ir para fora (para a minha localização)
}
\end{pronuncia}
\end{verbete}

\begin{verbete}[chu1qu0]{出去}
\begin{pronuncia}{chu1qu0}
\significado{v.d.}{
sair; ir para fora (a partir da minha localização)
}
\end{pronuncia}
\end{verbete}

\begin{verbete}[chu1zhan4]{出站}
\begin{pronuncia}{chu1zhan4}
\significado{n.}{
saída da estação
}
\end{pronuncia}
\end{verbete}

\begin{verbete}[chu1zu1qi4che1]{出租汽车}
\begin{pronuncia}{chu1zu1qi4che1}
\significado[辆]{n.}{
táxi
}
\end{pronuncia}
\end{verbete}

\begin{verbete}[chuan1]{穿}
\begin{pronuncia}{chuan1}
\significado{v.}{
vestir
}
\end{pronuncia}
\end{verbete}

\begin{verbete}[chuan2]{船}
\begin{pronuncia}{chuan2}
\significado{v.}{
barco; navio
}
\end{pronuncia}
\end{verbete}

\begin{verbete}[chuan2zhen1]{传真}
\begin{pronuncia}{chuan2zhen1}
\significado{n.}{
fax; facsímile
}
\end{pronuncia}
\end{verbete}

\begin{verbete}[chuang2]{床}
\begin{pronuncia}{chuang2}
\significado[张]{n.}{
cama
}
\end{pronuncia}
\end{verbete}

\begin{verbete}[chun1tian1]{春天}
\begin{pronuncia}{chun1tian1}
\significado[个]{n.}{
primavera
}
\significado{p.t.}{
primavera
}
\end{pronuncia}
\end{verbete}

\begin{verbete}[chuo4hao4]{绰号}
\begin{pronuncia}{chuo4hao4}
\significado{n.}{
apelido
}
\end{pronuncia}
\end{verbete}

\begin{verbete}[cong1ming2]{聪明}
\begin{pronuncia}{cong1ming2}
\significado{adj.}{
inteligente; brilhante; esperto
}
\end{pronuncia}
\end{verbete}

\begin{verbete}[cong1hui4]{聪慧}
\begin{pronuncia}{cong1hui4}
\significado{adj.}{
inteligente; brilhante
}
\end{pronuncia}
\end{verbete}

\begin{verbete}[cong2]{从}
\begin{pronuncia}{cong2}
\significado{prep.}{
de; desde; a partir de
}
\end{pronuncia}
\end{verbete}

\begin{verbete}[cu4]{醋}
\begin{pronuncia}{cu4}
\significado{n.}{
vinagre
}
\end{pronuncia}
\end{verbete}

\begin{verbete}[cuo4]{错}
\begin{pronuncia}{cuo4}
\significado{adj.}{
errado; enganado
}
\end{pronuncia}
\end{verbete}

%\end{multicols*}

%%%
%%% D
%%%
\section*{D}
\addcontentsline{toc}{section}{D}
%\begin{multicols*}{2}

\begin{verbete}[da3]{打}
\begin{pronuncia}{da3}
\significado{v.}{
jogar
}
\end{pronuncia}
\end{verbete}

\begin{verbete}[da3ban0]{打扮}
\begin{pronuncia}{da3ban0}
\significado{v.}{
arranjar-se; enfeitar-se
}
\end{pronuncia}
\end{verbete}

\begin{verbete}[da3\ dian4hua4]{打电话}
\begin{pronuncia}{da3\ dian4hua4}
\significado{v.}{
ligar; dar um telefonema
}
\end{pronuncia}
\begin{pronuncia}{gei3\ ...\ da3\ dian4hua4}
\significado{expr.}{
telefonar para alguém|veja: 给\ ······\ 打\ 电话
}
\end{pronuncia}
\end{verbete}

\begin{verbete}[da3gong1]{打工}
\begin{pronuncia}{da3gong1}
\significado{v.}{
trabalhar temporariamente para alguém; trabalhar por conta de alguém
}
\end{pronuncia}
\end{verbete}

\begin{verbete}[da3jiao3]{打搅}
\begin{pronuncia}{da3jiao3}
\significado{v.}{
perturbar; incomodar
}
\end{pronuncia}
\end{verbete}

\begin{verbete}[da3qiu2]{打球}
\begin{pronuncia}{da3qiu2}
\significado{v.}{
jogar bola; jogar (futebol, basquetebol, handbol, etc)
}
\end{pronuncia}
\end{verbete}

\begin{verbete}[da3rao3]{打扰}
\begin{pronuncia}{da3rao3}
\significado{v.}{
perturbar; incomodar
}
\end{pronuncia}
\end{verbete}

\begin{verbete}[da3suan4]{打算}
\begin{pronuncia}{da3suan4}
\significado[个]{n.}{
plano
}
\significado{v.}{
pensar; planejar; pretender
}
\end{pronuncia}
\end{verbete}

\begin{verbete}[da3zhen1]{打针}
\begin{pronuncia}{da3zhen1}
\significado{v.+compl.}{
dar injeção; levar injeção
}
\end{pronuncia}
\end{verbete}

\begin{verbete}[da4]{大}
\begin{pronuncia}{da4}
\significado{adj.}{
grande
}
\end{pronuncia}
\end{verbete}

\begin{verbete}[da4fu0]{大夫}
\begin{pronuncia}{da4fu0}
\significado{n.}{
médico; doutor
}
\end{pronuncia}
\end{verbete}

\begin{verbete}[da4gai4]{大概}
\begin{pronuncia}{da4gai4}
\significado{adv.}{
aproximadamente; por volta de
}
\end{pronuncia}
\end{verbete}

\begin{verbete}[da4hai3]{大海}
\begin{pronuncia}{da4hai3}
\significado{n.}{
mar; oceano
}
\end{pronuncia}
\end{verbete}

\begin{verbete}[da4jia1]{大海}
\begin{pronuncia}{da4jia1}
\significado{pron.}{
todos; todas
}
\end{pronuncia}
\end{verbete}

\begin{verbete}[da4suan4]{大蒜}
\begin{pronuncia}{da4suan4}
\significado[瓣,头]{n.}{
alho
}
\end{pronuncia}
\end{verbete}

\begin{verbete}[da4tui3]{大腿}
\begin{pronuncia}{da4tui3}
\significado{n.}{
coxa
}
\end{pronuncia}
\end{verbete}

\begin{verbete}[da4xue2]{大学}
\begin{pronuncia}{da4xue2}
\significado[所]{n.}{
universidade
}
\end{pronuncia}
\end{verbete}

\begin{verbete}[Da4yang2zhou1]{大洋洲}
\begin{pronuncia}{Da4yang2zhou1}
\significado{n.}{
Oceania
}
\end{pronuncia}
\end{verbete}

\begin{verbete}[dai4]{带}
\begin{pronuncia}{dai4}
\significado{v.}{
levar; trazer
}
\end{pronuncia}
\end{verbete}

\begin{verbete}[dai4]{戴}
\begin{pronuncia}{dai4}
\significado{v.}{
usar/vestir (óculos, gravata, relógio de pulso, luvas)|
trazer
}
\end{pronuncia}
\end{verbete}

\begin{verbete}[dan1xin1]{担心}
\begin{pronuncia}{dan1xin1}
\significado{v.}{
preocupar-se; estar preocupado
}
\end{pronuncia}
\end{verbete}

\begin{verbete}[dan4gao1]{蛋糕}
\begin{pronuncia}{dan4gao1}
\significado[块,个]{n.}{
bolo
}
\end{pronuncia}
\end{verbete}

\begin{verbete}[dan4shi4]{但是}
\begin{pronuncia}{dan4shi4}
\significado{conj.}{
mas; contudo
}
\end{pronuncia}
\end{verbete}

\begin{verbete}[dang1ran2]{当然}
\begin{pronuncia}{dang1ran2}
\significado{adv.}{
claro; certamente; com certeza
}
\end{pronuncia}
\end{verbete}

\begin{verbete}[dao3]{倒}
\begin{pronuncia}{dao3}
\significado{v.}{
cair no chão; deitar-se no chão
}
\end{pronuncia}
\end{verbete}

\begin{verbete}[dao4]{到}
\begin{pronuncia}{dao4}
\significado{prep.}{
a; até; para
}
\significado{v.}{
chegar
}
\end{pronuncia}
\end{verbete}

\begin{verbete}[de0]{的}
\begin{pronuncia}{de0}
\significado{part.}{
partícula utilizada em possessivos|
partícula utilizada entre adjetivos e substantivos|
(opcional se substantivo possui apenas um carácter)
}
\end{pronuncia}
\end{verbete}

\begin{verbete}[de0]{得}
\begin{pronuncia}{de0}
\significado{part.}{
partícula estrutural: ligando um verbo à frase seguinte indicando efeito,
grau, possibilidade, etc
}
\end{pronuncia}
\begin{pronuncia}{de2}
\significado{v.}{
obter; ganhar; pegar (uma doença)
}
\end{pronuncia}
\begin{pronuncia}{dei3}
\significado{v.}{
haver de; ter de
}
\end{pronuncia}
\end{verbete}

\begin{verbete}[De2guo2]{德国}
\begin{pronuncia}{De2guo2}
\significado{n.}{
Alemanha
}
\end{pronuncia}
\end{verbete}

\begin{verbete}[de2]{得}
\begin{pronuncia}{de2}
\significado{v.}{
obter; ganhar; pegar (uma doença)
}
\end{pronuncia}
\begin{pronuncia}{de0}
\significado{part.}{
partícula estrutural: ligando um verbo à frase seguinte indicando efeito,
grau, possibilidade, etc
}
\end{pronuncia}
\begin{pronuncia}{dei3}
\significado{v.}{
haver de; ter de
}
\end{pronuncia}
\end{verbete}

\begin{verbete}[de2dao4]{得到}
\begin{pronuncia}{de2dao4}
\significado{v.}{
obter
}
\end{pronuncia}
\end{verbete}

\begin{verbete}[dei3]{得}
\begin{pronuncia}{dei3}
\significado{v.}{
haver de; ter de
}
\end{pronuncia}
\begin{pronuncia}{de0}
\significado{part.}{
partícula estrutural: ligando um verbo à frase seguinte indicando efeito,
grau, possibilidade, etc
}
\end{pronuncia}
\begin{pronuncia}{de2}
\significado{v.}{
obter; ganhar; pegar (uma doença)
}
\end{pronuncia}
\end{verbete}

\begin{verbete}[deng1]{登}
\begin{pronuncia}{deng1}
\significado{v.}{
subir (montanha, cume)
}
\end{pronuncia}
\end{verbete}

\begin{verbete}[deng3]{等}
\begin{pronuncia}{deng3}
\significado{v.}{
esperar
}
\end{pronuncia}
\end{verbete}

\begin{verbete}[di1]{低}
\begin{pronuncia}{di1}
\significado{adj.}{
baixo, baixa
}
\end{pronuncia}
\end{verbete}

\begin{verbete}[di4fang1]{地方}
\begin{pronuncia}{di4fang1}
\significado[处,个,块]{n.}{
lugar; local; sítio
}
\end{pronuncia}
\end{verbete}

\begin{verbete}[di4zhi3]{地址}
\begin{pronuncia}{di4zhi3}
\significado[个]{n.}{
endereço
}
\end{pronuncia}
\end{verbete}

\begin{verbete}[di4tie3]{地铁}
\begin{pronuncia}{di4tie3}
\significado{n.}{
Metrô; metropolitano
}
\end{pronuncia}
\end{verbete}

\begin{verbete}[di4tu2]{地图}
\begin{pronuncia}{di4tu2}
\significado[张,本]{n.}{
mapa
}
\end{pronuncia}
\end{verbete}

\begin{verbete}[di4xia4shi4]{地下室}
\begin{pronuncia}{di4tu2}
\significado{n.}{
subterrâneo; porão
}
\end{pronuncia}
\end{verbete}

\begin{verbete}[di4di0]{弟弟}
\begin{pronuncia}{di4di0}
\significado[个,位]{n.}{
irmão mais novo
}
\end{pronuncia}
\end{verbete}

\begin{verbete}[di4mei4]{弟妹}
\begin{pronuncia}{di4mei4}
\significado{n.}{
esposa do irmão mais novo
}
\end{pronuncia}
\end{verbete}

\begin{verbete}[di4]{第}
\begin{pronuncia}{di4}
\significado{num.}{
prefixo para expressar números ordinais
}
\end{pronuncia}
\end{verbete}

\begin{verbete}[dian3]{点}
\begin{pronuncia}{dian3}
\significado{p.c.}{
hora
}
\end{pronuncia}
\end{verbete}

\begin{verbete}[dian4hua4]{电话}
\begin{pronuncia}{dian4hua4}
\significado[部]{n.}{
telefone
}
\significado[通]{n.}{
chamada telefônica
}
\end{pronuncia}
\end{verbete}

\begin{verbete}[dian4nao3]{电脑}
\begin{pronuncia}{dian4nao3}
\significado[台]{n.}{
computador
}
\end{pronuncia}
\end{verbete}

\begin{verbete}[dian4shi4]{电视}
\begin{pronuncia}{dian4shi4}
\significado[台,个]{n.}{
televisão; TV; televisor
}
\end{pronuncia}
\end{verbete}

\begin{verbete}[dian4ti1]{电梯}
\begin{pronuncia}{dian4ti1}
\significado[台,部]{n.}{
elevador
}
\end{pronuncia}
\end{verbete}

\begin{verbete}[dian4ying3]{电影}
\begin{pronuncia}{dian4ying3}
\significado[部,片,幕,场]{n.}{
cinema; filme
}
\end{pronuncia}
\end{verbete}

\begin{verbete}[dian4zi3]{电子}
\begin{pronuncia}{dian4zi3}
\significado{n.}{
eletrônico, eletrônica|
elétron
}
\end{pronuncia}
\end{verbete}

\begin{verbete}[dian4zi3you2jian4]{电子邮件}
\begin{pronuncia}{dian4zi3you2jian4}
\significado[封,份]{n.}{
correio eletrônico; \textit{e-mail}
}
\end{pronuncia}
\end{verbete}

\begin{verbete}[ding1zhu3]{叮嘱}
\begin{pronuncia}{ding1zhu3}
\significado{v.}{
exortar; avisar; insistir de novo e de novo
}
\end{pronuncia}
\end{verbete}

\begin{verbete}[dong1]{东}
\begin{pronuncia}{dong1}
\significado{n.}{
leste
}
\end{pronuncia}
\end{verbete}

\begin{verbete}[dong1ban4qiu2]{东半球}
\begin{pronuncia}{dong1ban4qiu2}
\significado{n.}{
hemisfério leste
}
\end{pronuncia}
\end{verbete}

\begin{verbete}[dong1bei3]{东北}
\begin{pronuncia}{dong1bei3}
\significado{p.l.}{
nordeste
}
\end{pronuncia}
\end{verbete}

\begin{verbete}[dong1bian0]{东边}
\begin{pronuncia}{dong1bian0}
\significado{p.l.}{
este; leste; lado leste; oriente
}
\end{pronuncia}
\end{verbete}

\begin{verbete}[dong1bu4]{东部}
\begin{pronuncia}{dong1bu4}
\significado{p.l.}{
leste; oriente
}
\end{pronuncia}
\end{verbete}

\begin{verbete}[dong1fang1]{东方}
\begin{pronuncia}{dong1fang1}
\significado{p.l.}{
leste; oriente
}
\end{pronuncia}
\end{verbete}

\begin{verbete}[Dong1fang1\ Xue2yuan4]{东方学院}
\begin{pronuncia}[\\]{Dong1fang1\ Xue2yuan4}
\significado{n.}{
Instituto Oriental
}
\end{pronuncia}
\end{verbete}

\begin{verbete}[dong1tian1]{东天}
\begin{pronuncia}{dong1tian1}
\significado{n.}{
inverno
}
\significado{p.t.}{
inverno
}
\end{pronuncia}
\end{verbete}

\begin{verbete}[dong1mia4]{东面}
\begin{pronuncia}{dong1mian4}
\significado{}{p.l.}{
este; leste; lado leste; oriente
}
\end{pronuncia}
\end{verbete}

\begin{verbete}[dong1xi0]{东西}
\begin{pronuncia}{dong1xi0}
\significado[个,件]{n.}{
coisa
}
\end{pronuncia}
\end{verbete}

\begin{verbete}[dong4wu4]{动物}
\begin{pronuncia}{dong4wu4}
\significado[只,群,个]{n.}{
animal
}
\end{pronuncia}
\end{verbete}

\begin{verbete}[dong4wu4yuan2]{动物园}
\begin{pronuncia}{dong4wu4yuan2}
\significado[个]{n.}{
jardim zoológico
}
\end{pronuncia}
\end{verbete}

\begin{verbete}[dou1]{都}
\begin{pronuncia}{dou1}
\significado{adv.}{
todo, toda, todos, todas
}
\end{pronuncia}
\end{verbete}

\begin{verbete}[du2]{读}
\begin{pronuncia}{du2}
\significado{v.}{
ler
}
\end{pronuncia}
\end{verbete}

\begin{verbete}[du4]{度}
\begin{pronuncia}{du4}
\significado{p.c.}{
para temperatura, etc
}
\significado{v.}{
grau
}
\end{pronuncia}
\end{verbete}

\begin{verbete}[du4zi0]{肚子}
\begin{pronuncia}{du4zi0}
\significado[个]{n.}{
abdómen; barriga
}
\end{pronuncia}
\end{verbete}

\begin{verbete}[duan3]{短}
\begin{pronuncia}{duan3}
\significado{adj.}{
curto, curta
}
\end{pronuncia}
\end{verbete}

\begin{verbete}[duan3ku4]{短裤}
\begin{pronuncia}{duan3ku4}
\significado{n.}{
calção, shorts
}
\end{pronuncia}
\end{verbete}

\begin{verbete}[duan4lian4]{锻炼}
\begin{pronuncia}{duan4lian4}
\significado{v.}{
fazer exercício físico
}
\end{pronuncia}
\end{verbete}

\begin{verbete}[dui4]{对}
\begin{pronuncia}{dui4}
\significado{adj.}{
correto; sim
}
\significado{prep.}{
com; para; para com
}
\end{pronuncia}
\end{verbete}

\begin{verbete}[dui4 ...\  gan3shing4qu4]{对······感兴趣}
\begin{pronuncia}[\\]{dui4 ...\  gan3shing4qu4}
\significado{expr.}{
estar interessado em...; ter interesse em...; interessar-se por...
}
\end{pronuncia}
\end{verbete}

\begin{verbete}[dui4 ...\  shu2xi1]{对······熟悉}
\begin{pronuncia}{dui4 ...\  shu2xi1}
\significado{expr.}{
estar familiarizado com...
}
\end{pronuncia}
\end{verbete}

\begin{verbete}[dui4 ...\  you3xing4qu4]{对······有兴趣}
\begin{pronuncia}[\\]{dui4 ...\  you3shing4qu4}
\significado{expr.}{
estar interessado em...; ter interesse em...; interessar-se por...
}
\end{pronuncia}
\end{verbete}

\begin{verbete}[dui4bu0qi3]{对不起}
\begin{pronuncia}{dui4bu0qi3}
\significado{v.}{
desculpar; pedir desculpa; perdoar
}
\end{pronuncia}
\end{verbete}

\begin{verbete}[dui4hua4]{对话}
\begin{pronuncia}{dui4hua4}
\significado{n.}{
diálogo; conversa
}
\significado{v.}{
dialogar; conversar
}
\end{pronuncia}
\end{verbete}

\begin{verbete}[dui4mian4]{对面}
\begin{pronuncia}{dui4mian4}
\significado{p.l.}{
lado oposto
}
\end{pronuncia}
\end{verbete}

\begin{verbete}[duo1]{多}
\begin{pronuncia}{duo1}
\significado{adj.}{
muito, muita, muitos, muitas
}
\end{pronuncia}
\end{verbete}

\begin{verbete}[duo1da4]{多大}
\begin{pronuncia}{duo1da4}
\significado{interr.}{
quantos anos?; que idade?
}
\end{pronuncia}
\end{verbete}

\begin{verbete}[duo1(me0)]{多(么)}
\begin{pronuncia}{duo1(me0)}
\significado{adv.}{
como
}
\end{pronuncia}
\end{verbete}

\begin{verbete}[duo1shao0]{多少}
\begin{pronuncia}{duo1shao0}
\significado{interr.}{
quanto?, quanta?, quantos?, quantas?; (para mais de 10 itens)
}
\end{pronuncia}
\end{verbete}

\begin{verbete}[duo1yun2]{多云}
\begin{pronuncia}{duo1yun2}
\significado{adj.}{
céu nublado
}
\end{pronuncia}
\end{verbete}

%\end{multicols*}

%%%
%%% E
%%%
\section*{E}
\addcontentsline{toc}{section}{E}
\begin{multicols}{2}

\begin{verbete}[俄罗斯]{É\ luo2si1}
\entry{É\ luo2si1}{n.}{
    Rússia
}
\end{verbete}

\begin{verbete}[恩赐]{en1ci4}
\entry{en1ci4}{n.}{
    favor; caridade
}
\entry{en1ci4}{v.}{
    conceder (favor, caridade);
}
\end{verbete}

\begin{verbete}[儿媳]{er2xi2}
\entry{er2xi2}{n.}{
    esposa do filho
}
\end{verbete}

\begin{verbete}[儿子]{er2zi0}
\entry{er2zi0}{n.}{
    filho
}
\end{verbete}

\begin{verbete}[耳朵]{er3duo0}
\entry{er3duo0}{n.}{
    orelha
}
\end{verbete}

\begin{verbete}[二]{er4}
\entry{er4}{num.}{
    2|
    dois
}
\end{verbete}

\end{multicols}

%%%
%%% F
%%%
\section*{F}
\addcontentsline{toc}{section}{F}
\begin{multicols*}{2}

\begin{verbete}[发]{fa1}
\significado{fa1}{v.}{
    enviar; mandar
}
\end{verbete}

\begin{verbete}[发烧]{fa1shao1}
\significado{fa1shao1}{v.}{
    ter febre
}
\end{verbete}

\begin{verbete}[罚]{fa2}
\significado{fa2}{v.}{
    castigar; punir
}
\end{verbete}

\begin{verbete}[罚款]{fa2kuan3}
\significado{fa2kuan3}{v.+compl.}{
    aplicar uma multa; multar
}
\end{verbete}

\begin{verbete}[法国]{Fa3guo2}
\significado{Fa3guo2}{n.}{
    França
}
\end{verbete}

\begin{verbete}[法语]{fa3yu3}
\significado{fa3yu3}{n.}{
    françês, língua francesa
}
\end{verbete}

\begin{verbete}[法文]{fa3wen2}
\significado{fa3wen2}{n.}{
    françês, língua francesa
}
\end{verbete}

\begin{verbete}[番茄]{fan4qie2}
\significado{fan4qie2}{n.}{
    tomate
}
\end{verbete}

\begin{verbete}[饭店]{fan4dian4}
\significado{fan4dian4}{n.}{
    restaurante
}
\end{verbete}

\begin{verbete}[方便]{fang1bian4}
\significado{fang1bian4}{adj.}{
    conveniente
}
\end{verbete}

\begin{verbete}[访问]{fang3wen4}
\significado{fang3wen4}{v.}{
    visitar
}
\end{verbete}

\begin{verbete}[放假]{fang4jia4}
\significado{fang4jia4}{v.}{
    ter férias
}
\end{verbete}

\begin{verbete}[房间]{fang4jian1}
\significado{fang4jian1}{n.}{
    quarto
}
\end{verbete}

\begin{verbete}[放心]{fang4xin1}
\significado{fang4xin1}{adj.}{
    descansado; despreocupado
}
\end{verbete}

\begin{verbete}[非]{fei1}
\significado{fei1}{adv.}{
    não; nem
}
\end{verbete}

\begin{verbete}[非常]{fei1chang2}
\significado{fei1chang2}{adv.}{
    muito
}
\end{verbete}

\begin{verbete}[非洲]{Fei1zhou1}
\significado{Fei1zhou1}{n.}{
    África
}
\end{verbete}

\begin{verbete}[飞机]{fei1ji1}
\significado{fei1ji1}{n.}{
    avião
}
\end{verbete}

\begin{verbete}[(飞)机票]{(fei1)\ ji1piao4}
\significado{(fei1)\ ji1piao4}{n.}{
    bilhete de avião
}
\end{verbete}

\begin{verbete}[分]{fen1}
\significado{fen1}{p.c.}{fen1}{
    minuto|
    centavo
}
\end{verbete}

\begin{verbete}[分公司]{fen1gong1si1}
\significado{fen1gong1si1}{n.}{fen1gong1si1}{
    sucursal; filial de companhia
}
\end{verbete}

\begin{verbete}[分钟]{fen1zhong1}
\significado{fen1zhong1}{n.}{fen1zhong1}{
    minuto
}
\end{verbete}

\begin{verbete}[份]{fen4}
\significado{fen4}{p.c.}{fen4}{
    dose
}
\end{verbete}

\begin{verbete}[分量]{fen4liang4}
\significado{fen4liang4}{p.}{fen4liang4}{
    peso; componente vetorial; física
}
\end{verbete}

\begin{verbete}[风]{feng1}
\significado{feng1}{n.}{
    vento
}
\end{verbete}

\begin{verbete}[枫叶]{feng1ye4}
\significado{feng1ye4}{n.}{
    folha de bordo (tipo de árvore/arbusto)
}
\end{verbete}

\begin{verbete}[副]{fu4}
\significado{fu4}{p.c.}{
    par; para óculos, luvas, etc
}
\end{verbete}

\begin{verbete}[父亲]{fu4qin0}
\significado{fu4quin0}{n.}{
    pai
}
\end{verbete}

\begin{verbete}[父母亲]{fu4mu3qin0}
\significado{fu4mu3quin0}{n.}{
    pais
}
\end{verbete}

\begin{verbete}[附近]{fu4jin4}
\significado{fu4jin4}{p.l.}{
    aqui perto; perto daqui
}
\end{verbete}

\end{multicols*}

%%%
%%% G
%%%
\section*{G}
\addcontentsline{toc}{section}{G}

\begin{verbete}[gan1]{干}[3]
\begin{pronuncia}{gan1}
\significado{v.}{ preocupar; ignorar; interferir }
\end{pronuncia}
\begin{pronuncia}{gan4}
\significado*{}{ 干\p{gan4} }
\end{pronuncia}
\end{verbete}

\begin{verbete}[gan1bei1]{干杯}[3;8]
\begin{pronuncia}{gan1bei1}
\significado{interj.}{ Saúde! }
\significado{v.+compl.}{ fazer um brinde; brindar até a última gota }
\end{pronuncia}
\end{verbete}

\begin{verbete}[gan1jing4]{干净}[3;8]
\begin{pronuncia}{gan1jing4}
\significado{adj.}{ limpo; arrumado }
\end{pronuncia}
\end{verbete}

\begin{verbete}[gan1shu3]{甘薯}[5;16]
\begin{pronuncia}{gan1shu3}
\significado{s.}{ batata doce }
\end{pronuncia}
\end{verbete}

\begin{verbete}[gan3kuai4]{赶快}[10;7]
\begin{pronuncia}{gan3kuai4}
\significado{adv.}{ imediatamente; de uma vez só }
\end{pronuncia}
\end{verbete}

\begin{verbete}[gan3lan3qiu2]{橄榄球}[15;13;11]
\begin{pronuncia}{gan3lan3qiu2}
\significado{s.}{ futebol jogado com bola oval (rúgbi, futebol americano, regras australianas, etc) }
\end{pronuncia}
\end{verbete}

\begin{verbete}[gan3mao4]{感冒}[13;9]
\begin{pronuncia}{gan3mao4}
\significado{v.}{ ficar resfriado; estar com resfriado }
\end{pronuncia}
\end{verbete}

\begin{verbete}[gan4]{干}[3]
\begin{pronuncia}{gan4}
\significado{v.}{ fazer; gerenciar; trabalhar; gíria: matar; vulgar: foder }
\end{pronuncia}
\begin{pronuncia}{gan1}
\significado*{}{ 干\p{gan1} }
\end{pronuncia}
\end{verbete}

\begin{verbete}[gang1]{刚}[6]
\begin{pronuncia}{gang1}
\significado{adj.}{ duro (sentido de difícil); forte }
\significado{adv.}{ acabar de; por muito pouco; apenas }
\end{pronuncia}
\end{verbete}

\begin{verbete}[gao1]{高}[10]
\begin{pronuncia}{gao1}
\significado{adj.}{ alto; acima da média }
\end{pronuncia}
\end{verbete}

\begin{verbete}[gao1xing4]{高兴}[10;6]
\begin{pronuncia}{gao1xing4}
\significado{adj.}{ feliz; alegre; contente; disposto (a fazer alguma coisa) }
\end{pronuncia}
\end{verbete}

\begin{verbete}[gao4su0]{告诉}[7;7]
\begin{pronuncia}{gao4su0}
\significado{v.}{ contar; dar a conhecer; informar }
\end{pronuncia}
\end{verbete}

\begin{verbete}[ge1ge0]{哥哥}[10;10]
\begin{pronuncia}{ge1ge0}
\significado[个,位]{s.}{ irmão mais velho }
\end{pronuncia}
\end{verbete}

\begin{verbete}[ge1]{歌}[14]
\begin{pronuncia}{ge1}
\significado[支,首]{s.}{ canção; canto }
\end{pronuncia}
\end{verbete}

\begin{verbete}[ge4]{个}[3]
\begin{pronuncia}{ge4}
\significado{p.c.}{ para objetos e pessoas em geral }
\end{pronuncia}
\end{verbete}

\begin{verbete}[gei3]{给}[9]
\begin{pronuncia}{gei3}
\significado{pre.}{ a; para }
\significado{v.}{ dar; permitir; fazer alguma coisa (para alguém) }
\end{pronuncia}
\end{verbete}

\begin{verbete}[gei3\ da3dian4hua4]{给(alguém)打电话}[9;5;5;8]
\begin{pronuncia}[\\]{gei3\ da3dian4hua4}
\significado{expr.}{ telefonar para alguém }
\end{pronuncia}
\end{verbete}

\begin{verbete}[gen1]{跟}[13]
\begin{pronuncia}{gen1}
\significado{prep.}{ com }
\significado{v.}{ acompanhar junto; seguir de perto; ir com }
\end{pronuncia}
\end{verbete}

\begin{verbete}[gen1ju4]{根据}[10;11]
\begin{pronuncia}{gen1ju4}
\significado{prep.}{ de acordo com }
\significado[个]{s.}{ base; fundação }
\end{pronuncia}
\end{verbete}

\begin{verbete}[geng4]{更}[7]
\begin{pronuncia}{geng4}
\significado{adv.}{ mais; ainda mais }
\end{pronuncia}
\end{verbete}

\begin{verbete}[gong1zuo4]{工作}[3;7]
\begin{pronuncia}{gong1zuo4}
\significado[个,份,项]{s.}{ trabalho; emprego; tarefa }
\significado{v.}{ trabalhar; operar (uma máquina) }
\end{pronuncia}
\end{verbete}

\begin{verbete}[gong1yi3pin3]{工艺品}[3;4;9]
\begin{pronuncia}{gong1yi3pin3}
\significado[个]{s.}{ artigo de artesanato; trabalho manual }
\end{pronuncia}
\end{verbete}

\begin{verbete}[gong1gong4qi4che1]{公共汽车}[4;6;7;4]
\begin{pronuncia}[\\]{gong1gong4qi4che1}
\significado[辆,班]{s.}{ ônibus }
\end{pronuncia}
\end{verbete}

\begin{verbete}[gong1ke4]{公克}[4;7]
\begin{pronuncia}{gong1ke4}
\significado{s.}{ grama (medida de peso) }
\end{pronuncia}
\end{verbete}

\begin{verbete}[gong1si1]{公司}[4;5]
\begin{pronuncia}{gong1si1}
\significado[家]{s.}{ empresa; companhia; corporação; firma }
\end{pronuncia}
\end{verbete}

\begin{verbete}[gong1yong4dian4hua4]{公用电话}[4;5;5;8]
\begin{pronuncia}[\\]{gong1yong4dian4hua4}
\significado[部]{s.}{ telefone público }
\end{pronuncia}
\end{verbete}

\begin{verbete}[gong1yu4]{公寓}[4;12]
\begin{pronuncia}{gong1yu4}
\significado[套]{s.}{ prédio de apartamentos; pensão }
\end{pronuncia}
\end{verbete}

\begin{verbete}[gong1yuan2]{公园}[4;7]
\begin{pronuncia}{gong1yuan2}
\significado[个,座]{s.}{ parque (para recreação pública) }
\end{pronuncia}
\end{verbete}

\begin{verbete}[gou3]{狗}[8]
\begin{pronuncia}{gou3}
\significado[条,只]{s.}{ cão; cachorro }
\end{pronuncia}
\end{verbete}

\begin{verbete}[Gu4gong1]{故宫}[9;9]
\begin{pronuncia}{Gu4gong1}
\significado{s.}{ Palácio Imperial; Cidade Proibida }
\end{pronuncia}
\end{verbete}

\begin{verbete}[gua1]{刮}[8]
\begin{pronuncia}{gua1}
\significado{v.}{ ventar; soprar (vento) }
\end{pronuncia}
\end{verbete}

\begin{verbete}[gua1feng1]{刮风}[8;4]
\begin{pronuncia}{gua1feng1}
\significado{v.+compl.}{ ventar; fazer vento }
\end{pronuncia}
\end{verbete}

\begin{verbete}[guai3]{拐}[8]
\begin{pronuncia}{guai3}
\significado{v.}{ virar (uma esquina, etc); cortar }
\end{pronuncia}
\end{verbete}

\begin{verbete}[guang1pan2]{光盘}[6;11]
\begin{pronuncia}{guang1pan2}
\significado[片,张]{s.}{ CD; DVD; disco compacto }
\end{pronuncia}
\end{verbete}

\begin{verbete}[guang3dong1]{广东}[3;5]
\begin{pronuncia}{guang3dong1}
\significado{s.}{ Guangdong }
\end{pronuncia}
\end{verbete}

\begin{verbete}[guang3gao4]{广告}[3;7]
\begin{pronuncia}{guang3gao4}
\significado[项]{s.}{ publicidade; anúncio publicitário }
\end{pronuncia}
\end{verbete}

\begin{verbete}[gui1ding4]{规定}[8;8]
\begin{pronuncia}{gui1ding4}
\significado[个]{s.}{ regulamento; regra }
\significado{v.}{ estipular }
\end{pronuncia}
\end{verbete}

\begin{verbete}[gui4]{贵}[9]
\begin{pronuncia}{gui4}
\significado{adj.}{ caro; nobre; precioso }
\end{pronuncia}
\end{verbete}

\begin{verbete}[gui4xing4]{贵姓}[9;8]
\begin{pronuncia}{gui4xing4}
\significado{interr.}{ qual seu sobrenome? }
\end{pronuncia}
\end{verbete}

\begin{verbete}[guo2]{国}[8]
\begin{pronuncia}{guo2}
\significado[个]{s.}{ país; nação }
\end{pronuncia}
\end{verbete}

\begin{verbete}[guo2jia1]{国家}[8;10]
\begin{pronuncia}{guo2jia1}
\significado[个]{s.}{ país; nação }
\end{pronuncia}
\end{verbete}

\begin{verbete}[guo2yu3]{国语}[8;9]
\begin{pronuncia}{guo2yu3}
\significado{s.}{ Língua Chinesa (Mandarim), enfatizando sua natureza nacional }
\end{pronuncia}
\end{verbete}

\begin{verbete}[guo3jiang4]{果酱}[8;13]
\begin{pronuncia}{guo3jiang4}
\significado{s.}{ geléia; compota ou doce (de frutas) }
\end{pronuncia}
\end{verbete}

\begin{verbete}[guo4]{过}[6]
\begin{pronuncia}{guo4}
\significado{part.}{ passado }
\significado{v.}{ atravessar; passar (tempo) }
\end{pronuncia}
\end{verbete}

\begin{verbete}[guo4lai0]{过来}[6;7]
\begin{pronuncia}{guo4lai0}
\significado{v.}{ atravessar (para a minha localização); vir até aqui }
\end{pronuncia}
\end{verbete}

\begin{verbete}[guo4nian2]{过年}[6;6]
\begin{pronuncia}{guo4nian2}
\significado{v.}{ festejar o Ano Novo Chinês }
\end{pronuncia}
\end{verbete}

\begin{verbete}[guo4qi1]{过期}[6;12]
\begin{pronuncia}{guo4qi1}
\significado{v.+compl.}{ exceder a data; passar a data; expirar (passar a data de expiração) }
\end{pronuncia}
\end{verbete}

\begin{verbete}[guo4qu0]{过去}[6;5]
\begin{pronuncia}{guo4qu0}
\significado{v.}{ atravessar (a partir da minha localização); ir até lá }
\end{pronuncia}
\end{verbete}

%%%%% EOF %%%%%

%%%
%%% H
%%%
\section*{H}
\addcontentsline{toc}{section}{H}
%\begin{multicols*}{2}

\begin{verbete}[hai2]{还}
\begin{pronuncia}{hai2}
\significado{adv.}{
ainda; também
}
\end{pronuncia}
\begin{pronuncia}{huan4}
\significado{v.}{
devolver; restituir
}
\end{pronuncia}
\end{verbete}

\begin{verbete}[hai2shi0]{还是}
\begin{pronuncia}{hai2shi0}
\significado{conj.}{
ou (somente para frases interrogativas)
}
\end{pronuncia}
\end{verbete}

\begin{verbete}[hai2zi0]{孩子}
\begin{pronuncia}{hai2zi0}
\significado{n.}{
criança|
filho, filha
}
\end{pronuncia}
\end{verbete}

\begin{verbete}[hai3]{海}
\begin{pronuncia}{hai3}
\significado[个,片]{n.}{
mar; oceano
}
\end{pronuncia}
\end{verbete}

\begin{verbete}[hai3bian1]{海边}
\begin{pronuncia}{hai3bian1}
\significado{p.d.l.}{
costa marítima; litoral
}
\end{pronuncia}
\end{verbete}

\begin{verbete}[hai4pa4]{害怕}
\begin{pronuncia}{hai4pa4}
\significado{v.}{
ter medo de; temer
}
\end{pronuncia}
\end{verbete}

\begin{verbete}[Han2guo2]{汉国}
\begin{pronuncia}{Han2guo2}
\significado{n.}{
Coréia do Sul
}
\end{pronuncia}
\end{verbete}

\begin{verbete}[han4pu2ci2dian3]{汉葡词典}
\begin{pronuncia}{han4pu2ci2dian3}
\significado[部,本]{n.}{
dicionário chinês-português
}
\end{pronuncia}
\end{verbete}

\begin{verbete}[han4yu3]{汉语}
\begin{pronuncia}{han4yu3}
\significado[门]{n.}{
chinês, língua chinesa, mandarim
}
\end{pronuncia}
\end{verbete}

\begin{verbete}[hang2ban1]{航班}
\begin{pronuncia}{hang2ban1}
\significado{n.}{
voo; número de voo
}
\end{pronuncia}
\end{verbete}

\begin{verbete}[hao3]{好}
\begin{pronuncia}{hao3}
\significado{adj.}{
bom, boa, bem
}
\end{pronuncia}
\end{verbete}

\begin{verbete}[hao3chi1]{好吃}
\begin{pronuncia}{hao3chi1}
\significado{adj.}{
delicioso; saboroso
}
\end{pronuncia}
\end{verbete}

\begin{verbete}[hao3han4]{好汉}
\begin{pronuncia}{hao3han4}
\significado[条]{n.}{
herói; homem verdadeiro
}
\end{pronuncia}
\end{verbete}

\begin{verbete}[hao3kan4]{好看}
\begin{pronuncia}{hao3kan4}
\significado{adj.}{
bonito; com bom aspecto
}
\end{pronuncia}
\end{verbete}

\begin{verbete}[hao3ting1]{好听}
\begin{pronuncia}{hao3ting1}
\significado{adj.}{
agradável ao ouvido
}
\end{pronuncia}
\end{verbete}

\begin{verbete}[hao3wanr2]{好玩儿}
\begin{pronuncia}{hao3wanr2}
\significado{adj.}{
divertido, interessante
}
\end{pronuncia}
\end{verbete}

\begin{verbete}[hao3xue2]{好学}
\begin{pronuncia}{hao3xue2}
\significado{adj.}{
fácil de aprender
}
\end{pronuncia}
\end{verbete}

\begin{verbete}[hao4]{号}
\begin{pronuncia}{hao4}
\significado[个]{n.}{
dia; número
}
\significado{p.c.}{
dia
}
\end{pronuncia}
\end{verbete}

\begin{verbete}[hao4ma3]{号码}
\begin{pronuncia}{hao4ma3}
\significado[堆,个]{n.}{
número
}
\end{pronuncia}
\end{verbete}

\begin{verbete}[he1]{喝}
\begin{pronuncia}{he1}
\significado{v.}{
beber
}
\end{pronuncia}
\end{verbete}

\begin{verbete}[he2zi1]{合资}
\begin{pronuncia}{he2zi1}
\significado{n.}{
joint-venture com capitais mistos
}
\end{pronuncia}
\end{verbete}

\begin{verbete}[he2zuo4]{合作}
\begin{pronuncia}{he2zuo4}
\significado[个]{n.}{
cooperação
}
\significado{v.}{
cooperar; colaborar
}
\end{pronuncia}
\end{verbete}

\begin{verbete}[he2]{和}
\begin{pronuncia}{he2}
\significado{conj.}{
e (somente para palavras)
}
\end{pronuncia}
\end{verbete}

\begin{verbete}[he2]{河}
\begin{pronuncia}{he2}
\significado[条,道]{n.}{
rio
}
\end{pronuncia}
\end{verbete}

\begin{verbete}[he2]{盒}
\begin{pronuncia}{he2}
\significado{p.c.}{
caixa
}
\end{pronuncia}
\end{verbete}

\begin{verbete}[hei1]{黑}
\begin{pronuncia}{hei1}
\significado{adj.}{
preto, preta
}
\end{pronuncia}
\end{verbete}

\begin{verbete}[hei1ban3]{黑板}
\begin{pronuncia}{hei1ban3}
\significado[块,个]{n.}{
quadro negro
}
\end{pronuncia}
\end{verbete}

\begin{verbete}[hei1se4]{黑色}
\begin{pronuncia}{hei1se4}
\significado{n.}{
cor preta
}
\end{pronuncia}
\end{verbete}

\begin{verbete}[hen3]{很}
\begin{pronuncia}{hen3}
\significado{adv.}{
muito; mui
}
\end{pronuncia}
\end{verbete}

\begin{verbete}[hong2]{红}
\begin{pronuncia}{hong2}
\significado{adj.}{
vermelho, vermelha
}
\end{pronuncia}
\end{verbete}

\begin{verbete}[hong2se4]{红色}
\begin{pronuncia}{hong2se4}
\significado{n.}{
cor vermelha
}
\end{pronuncia}
\end{verbete}

\begin{verbete}[hong2shao1]{红烧}
\begin{pronuncia}{hong2shao1}
\significado{n.}{
guisado em molho de soja
}
\end{pronuncia}
\end{verbete}

\begin{verbete}[hou4bian0]{后边}
\begin{pronuncia}{hou4bian0}
\significado{p.l.}{
atrás; detrás
}
\end{pronuncia}
\end{verbete}

\begin{verbete}[hou4mian0]{后面}
\begin{pronuncia}{hou4mian0}
\significado{p.l.}{
atrás; detrás
}
\end{pronuncia}
\end{verbete}

\begin{verbete}[hou4nian2]{后年}
\begin{pronuncia}{hou4nian2}
\significado{p.t.}{
daqui a dois anos
}
\end{pronuncia}
\end{verbete}

\begin{verbete}[hou4tian1]{后天}
\begin{pronuncia}{hou4tian1}
\significado{p.t.}{
depois de amanhã
}
\end{pronuncia}
\end{verbete}

\begin{verbete}[hu2]{湖}
\begin{pronuncia}{hu2}
\significado[个,片]{n.}{
lago
}
\end{pronuncia}
\end{verbete}

\begin{verbete}[Hu2nan2]{湖南}
\begin{pronuncia}{Hu2nan2}
\significado{n.}{
Hunan
}
\end{pronuncia}
\end{verbete}

\begin{verbete}[hu2li0hu2tu0]{糊里糊涂}
\begin{pronuncia}{hu2li0hu2tu0}
\significado{adj.}{
desnorteado; perturbado
}
\end{pronuncia}
\end{verbete}

\begin{verbete}[hu4xiang1]{互相}
\begin{pronuncia}{hu4xiang1}
\significado{adv.}{
mutuamente
}
\end{pronuncia}
\end{verbete}

\begin{verbete}[hua1]{花}
\begin{pronuncia}{hua1}
\significado[朵,支,束,把,盆,簇]{n.}{
flor
}
\end{pronuncia}
\end{verbete}

\begin{verbete}[Hua2sheng4dun4]{华盛顿}
\begin{pronuncia}{Hua2sheng4dun4}
\significado{n.}{
Washington
}
\end{pronuncia}
\end{verbete}

\begin{verbete}[hua2shi4]{花氏}
\begin{pronuncia}{hua2shi4}
\significado{n.}{
Fahrenheit
}
\end{pronuncia}
\end{verbete}

\begin{verbete}[hua2yi4]{华裔}
\begin{pronuncia}{hua2yi4}
\significado{n.}{
descendente de chinês
}
\end{pronuncia}
\end{verbete}

\begin{verbete}[hua2]{滑}
\begin{pronuncia}{hua2}
\significado{adj.}{
deslizado
}
\significado{v.}{
deslizar
}
\end{pronuncia}
\end{verbete}

\begin{verbete}[hua2xue3]{滑雪}
\begin{pronuncia}{hua2xue3}
\significado{v.+compl.}{
esquiar; fazer esqui
}
\end{pronuncia}
\end{verbete}

\begin{verbete}[hua4]{话}
\begin{pronuncia}{hua4}
\significado[种,席,句,口,番]{n.}{
palavra; fala
}
\end{pronuncia}
\end{verbete}

\begin{verbete}[huai4]{坏}
\begin{pronuncia}{huai4}
\significado{adj.}{
avariado|
mau
}
\end{pronuncia}
\end{verbete}

\begin{verbete}[huan1ying2]{欢迎}
\begin{pronuncia}{huan1ying2}
\significado{v.}{
dar as boas-vindas; ser bem-vindo
}
\end{pronuncia}
\end{verbete}

\begin{verbete}[huan1jing4]{环境}
\begin{pronuncia}{huan1jing4}
\significado[个]{n.}{
ambiente; meio ambiente
}
\end{pronuncia}
\end{verbete}

\begin{verbete}[huan4]{还}
\begin{pronuncia}{huan4}
\significado{v.}{
devolver; restituir
}
\end{pronuncia}
\begin{pronuncia}{hai2}
\significado{adv.}{
ainda; também
}
\end{pronuncia}
\end{verbete}

\begin{verbete}[huan4]{换}
\begin{pronuncia}{huan4}
\significado{v.}{
mudar
}
\end{pronuncia}
\end{verbete}

\begin{verbete}[huang2]{黄}
\begin{pronuncia}{huang2}
\significado{adj.}{
amarelo, amarela
}
\end{pronuncia}
\end{verbete}

\begin{verbete}[huang2se4]{黄色}
\begin{pronuncia}{huang2se4}
\significado{n.}{
cor amarela
}
\end{pronuncia}
\end{verbete}

\begin{verbete}[huang2you2]{黄油}
\begin{pronuncia}{huang2you2}
\significado[盒]{n.}{
manteiga
}
\end{pronuncia}
\end{verbete}

\begin{verbete}[huar1]{花儿}
\begin{pronuncia}{huar1}
\significado[朵,支,束,把,盆,簇]{n.}{
flor
}
\end{pronuncia}
\end{verbete}

\begin{verbete}[hui2]{回}
\begin{pronuncia}{hui2}
\significado{v.d.}{
regressar
}
\end{pronuncia}
\end{verbete}

\begin{verbete}[hui2da2]{回答}
\begin{pronuncia}{hui2da2}
\significado{v.}{
responder
}
\end{pronuncia}
\end{verbete}

\begin{verbete}[hui2lai0]{回来}
\begin{pronuncia}{hui2lai0}
\significado{v.d.}{
regressar; voltar; estar de volta (para a minha localização)
}
\end{pronuncia}
\end{verbete}

\begin{verbete}[hui2qu0]{回去}
\begin{pronuncia}{hui2qu0}
\significado{v.d.}{
regressar; voltar; estar de volta (a partir da minha localização)
}
\end{pronuncia}
\end{verbete}

\begin{verbete}[hui4]{会}
\begin{pronuncia}{hui4}
\significado{v.}{
saber
}
\end{pronuncia}
\end{verbete}

\begin{verbete}[huo2dong4]{活动}
\begin{pronuncia}{huo2dong4}
\significado[项,个]{n.}{
atividade; evento
}
\significado{v.}{
exercer
}
\end{pronuncia}
\end{verbete}

\begin{verbete}[huo3che1]{火车}
\begin{pronuncia}{huo3che1}
\significado[列,节,班,趟]{n.}{
trem; comboio
}
\end{pronuncia}
\end{verbete}

\begin{verbete}[huo4zhe3]{或者}
\begin{pronuncia}{huo4zhe3}
\significado{conj.}{
ou (usado em expressões afirmativas)
}
\end{pronuncia}
\end{verbete}

%\end{multicols*}

%%%%
%%% I
%%%
%\section*{I}
%\addcontentsline{toc}{section}{I}
%\begin{multicols*}{2}
%\end{multicols*}
 %%%%% Não tem palavras com pinyin iniciado em "I" em chinês
%%%
%%% J
%%%
\section*{J}
\addcontentsline{toc}{section}{J}

\begin{verbete}[ji1]{鸡}[7]
\begin{pronuncia}{ji1}
\significado[只]{s.}{ galo, galinha; gíria: prostituta }
\end{pronuncia}
\end{verbete}

\begin{verbete}[ji1dan4]{鸡蛋}[7;11]
\begin{pronuncia}{ji1dan4}
\significado[个,打]{s.}{ ovo de galinha }
\end{pronuncia}
\end{verbete}

\begin{verbete}[ji1chang3]{机场}[6;6]
\begin{pronuncia}{ji1chang3}
\significado[家,处]{s.}{ aeroporto }
\end{pronuncia}
\end{verbete}

\begin{verbete}[...ji2le0]{······极了}[7;2]
\begin{pronuncia}{...ji2le0}
\significado{expr.}{ muito; extremamente }
\end{pronuncia}
\end{verbete}

\begin{verbete}[ji2ge2]{及格}[3;10]
\begin{pronuncia}{ji2ge2}
\significado{v.}{ aprovar; passar no exame }
\end{pronuncia}
\end{verbete}

\begin{verbete}[ji3]{几}[2]
\begin{pronuncia}{ji3}
\significado{interr.}{ quantos? (até 10 itens); alguns? (até 10 itens) }
\end{pronuncia}
\end{verbete}

\begin{verbete}[ji4jie2]{季节}[8;5]
\begin{pronuncia}{ji4jie2}
\significado[个]{s.}{ estação (clima) }
\end{pronuncia}
\end{verbete}

\begin{verbete}[jia1]{家}[10]
\begin{pronuncia}{jia1}
\significado[个]{s.}{ família; casa }
\end{pronuncia}
\end{verbete}

\begin{verbete}[jia1li0]{家里}[10;7]
\begin{pronuncia}{jia1li0}
\significado{p.d.l.}{ em casa }
\end{pronuncia}
\end{verbete}

\begin{verbete}[jia1xiang1]{家乡}[10;3]
\begin{pronuncia}{jia1xiang1}
\significado[个]{s.}{ terra natal }
\end{pronuncia}
\end{verbete}

\begin{verbete}[Jia1na2da4]{加拿大}[5;10;3]
\begin{pronuncia}{Jia1na2da4}
\significado{s.}{ Canadá }
\end{pronuncia}
\end{verbete}

\begin{verbete}[jia1na2da4ren2]{加拿大人}[5;10;3;2]
\begin{pronuncia}[\\]{jia1na2da4ren2}
\significado{s.}{ canadense; pessoa nascida no Canadá }
\end{pronuncia}
\end{verbete}

\begin{verbete}[jian1bang3]{肩膀}[8;14]
\begin{pronuncia}{jian1bang3}
\significado{s.}{ ombro }
\end{pronuncia}
\end{verbete}

\begin{verbete}[jian3cha2]{检查}[11;9]
\begin{pronuncia}{jian3cha2}
\significado[次]{s.}{ inspeção }
\significado{v.}{ examinar; verificar; inspecionar }
\end{pronuncia}
\end{verbete}

\begin{verbete}[jian3dan1]{简单}[13;8]
\begin{pronuncia}{jian3dan1}
\significado{adj.}{ simples }
\end{pronuncia}
\end{verbete}

\begin{verbete}[jian4]{见}[4]
\begin{pronuncia}{jian4}
\significado{v.}{ ver; encontrar alguém }
\end{pronuncia}
\end{verbete}

\begin{verbete}[jian4mian4]{见面}[4;9]
\begin{pronuncia}{jian4mian4}
\significado{v.}{ encontrar-se com alguém }
\end{pronuncia}
\end{verbete}

\begin{verbete}[jian4]{件}[6]
\begin{pronuncia}{jian4}
\significado{p.c.}{ para roupas }
\end{pronuncia}
\end{verbete}

\begin{verbete}[jian4yi4]{建议}[8;5]
\begin{pronuncia}{jian4yi4}
\significado[个,点]{s.}{ sugestão }
\significado{v.}{ sugerir }
\end{pronuncia}
\end{verbete}

\begin{verbete}[Jiang1xi1]{江西}[6;6]
\begin{pronuncia}{Jiang1xi1}
\significado{s.}{ Jiangxi }
\end{pronuncia}
\end{verbete}

\begin{verbete}[jiao1tong1]{交通}[6;10]
\begin{pronuncia}{jiao1tong1}
\significado{s.}{ transporte; tráfego; trânsito }
\end{pronuncia}
\end{verbete}

\begin{verbete}[jiao1qu1]{郊区}
\begin{pronuncia}{jiao1qu1}
\significado[个]{s.}{ subúrbio; arredores }
\end{pronuncia}
\end{verbete}

\begin{verbete}[jiao1juan3]{胶卷}[8;4]
\begin{pronuncia}{jiao1juan3}
\significado{s.}{ filme; película; rolo }
\end{pronuncia}
\end{verbete}

\begin{verbete}[jiao3]{脚}[11]
\begin{pronuncia}{jiao3}
\significado[双,只]{s.}{ pé }
\significado{p.c.}{ para crianças }
\end{pronuncia}
\end{verbete}

\begin{verbete}[jiao3]{角}[7]
\begin{pronuncia}{jiao3}
\significado{p.c.}{ 1 jiao = 10 centavos }
\end{pronuncia}
\end{verbete}

\begin{verbete}[jiao3zi0]{饺子}[9;3]
\begin{pronuncia}{jiao3zi0}
\significado[个,只]{s.}{ jiaozi; raviólis chineses; bolinho de massa }
\end{pronuncia}
\end{verbete}

\begin{verbete}[jiao4]{叫}[5]
\begin{pronuncia}{jiao4}
\significado{v.}{ chamar-se; chamar }
\end{pronuncia}
\end{verbete}

\begin{verbete}[jiao4]{教}[11]
\begin{pronuncia}{jiao4}
\significado{v.}{ ensinar }
\end{pronuncia}
\end{verbete}

\begin{verbete}[jiao4lian4]{教练}[11;8]
\begin{pronuncia}{jiao4lian4}
\significado[个,位,名]{s.}{ treinador }
\end{pronuncia}
\end{verbete}

\begin{verbete}[jiao4shou4]{教授}[11;11]
\begin{pronuncia}{jiao4shou4}
\significado[个,位]{s.}{ professor }
\end{pronuncia}
\end{verbete}

\begin{verbete}[jiao4shi1]{教师}[11;6]
\begin{pronuncia}{jiao4shi1}
\significado[个]{s.}{ professor; mestre }
\end{pronuncia}
\end{verbete}

\begin{verbete}[jiao4shi4]{教室}[11;9]
\begin{pronuncia}{jiao4shi4}
\significado[间]{s.}{ sala de aula }
\end{pronuncia}
\end{verbete}

\begin{verbete}[jiao4xue2lou2]{教学楼}[11;8;13]
\begin{pronuncia}{jiao4xue2lou2}
\significado{s.}{ edifício de salas de aula }
\end{pronuncia}
\end{verbete}

\begin{verbete}[jie1]{街}[12]
\begin{pronuncia}{jie1}
\significado[条]{s.}{ rua }
\end{pronuncia}
\end{verbete}

\begin{verbete}[jie1]{接}[11]
\begin{pronuncia}{jie1}
\significado{v.}{ ir buscar (alguém); ir ao encontro de (alguém); receber }
\end{pronuncia}
\end{verbete}

\begin{verbete}[jie1]{接(电话)}[5;5;8]
\begin{pronuncia}{jie1}
\significado{v.}{ atender (o telefone) }
\end{pronuncia}
\end{verbete}

\begin{verbete}[jie1dai4]{接待}[11;9]
\begin{pronuncia}{jie1dai4}
\significado{v.}{ receber (alguém); acolher; recepcionar }
\end{pronuncia}
\end{verbete}

\begin{verbete}[jie2ri4]{节日}[5;4]
\begin{pronuncia}{jie2ri4}
\significado[个]{s.}{ festa }
\end{pronuncia}
\end{verbete}

\begin{verbete}[jie2guo3]{结果}[9;8]
\begin{pronuncia}{jie2guo3}
\significado{s.}{ resultado; consequência }
\end{pronuncia}
\end{verbete}

\begin{verbete}[jie3jie0]{姐姐}[8;8]
\begin{pronuncia}{jie3jie0}
\significado[个]{s.}{ irmã mais velha }
\end{pronuncia}
\end{verbete}

\begin{verbete}[jie3fu0]{姐夫}[8;4]
\begin{pronuncia}{jie3fu0}
\significado{s.}{ marido da irmã mais velha }
\end{pronuncia}
\end{verbete}

\begin{verbete}[jie4shao4]{介绍}[4;8]
\begin{pronuncia}{jie4shao4}
\significado{s.}{ apresentação }
\significado{v.}{ apresentar }
\end{pronuncia}
\end{verbete}

\begin{verbete}[jie4]{借}[10]
\begin{pronuncia}{jie4}
\significado{v.}{ pedir emprestado; emprestar }
\end{pronuncia}
\end{verbete}

\begin{verbete}[jie4shu1zheng4]{借书证}[10;4;7]
\begin{pronuncia}{jie4shu1zheng4}
\significado{s.}{ cartão de biblioteca; literalmente: cartão para pedir emprestado livros }
\end{pronuncia}
\end{verbete}

\begin{verbete}[jin1nian2]{今年}[4;6]
\begin{pronuncia}{jin1nian2}
\significado{p.t.}{ este ano }
\end{pronuncia}
\end{verbete}

\begin{verbete}[jin1tian1]{今天}[4;4]
\begin{pronuncia}{jin1tian1}
\significado{p.t.}{ hoje }
\end{pronuncia}
\end{verbete}

\begin{verbete}[jin1rong2]{金融}[8;16]
\begin{pronuncia}{jin1rong2}
\significado{s.}{ finança }
\end{pronuncia}
\end{verbete}

\begin{verbete}[jin4]{近}[7]
\begin{pronuncia}{jin4}
\significado{adj.}{ perto; próximo }
\end{pronuncia}
\end{verbete}

\begin{verbete}[jin4]{进}[7]
\begin{pronuncia}{jin4}
\significado{v.d.}{ entrar }
\end{pronuncia}
\end{verbete}

\begin{verbete}[jin4chu1kou3]{进出口}[7;5;3]
\begin{pronuncia}{jin4chu1kou3}
\significado{s.}{ importação e exportação }
\end{pronuncia}
\end{verbete}

\begin{verbete}[jin4kou3]{进口}[7;3]
\begin{pronuncia}{jin4kou3}
\significado{s.}{ importação }
\significado{v.}{ importar }
\end{pronuncia}
\end{verbete}

\begin{verbete}[jin4lai0]{进来}[7;7]
\begin{pronuncia}{jin4lai0}
\significado{v.d.}{ entrar (para a minha localização) }
\end{pronuncia}
\end{verbete}

\begin{verbete}[jin4qu0]{进去}[7;5]
\begin{pronuncia}{jin4qu0}
\significado{v.d.}{ entrar (a partir da minha localização) }
\end{pronuncia}
\end{verbete}

\begin{verbete}[jing1chang2]{经常}[8;11]
\begin{pronuncia}{jing1chang2}
\significado{adv.}{ muitas vezes }
\end{pronuncia}
\end{verbete}

\begin{verbete}[jing1ji4]{经济}[8;9]
\begin{pronuncia}{jing1ji4}
\significado{s.}{ economia }
\end{pronuncia}
\end{verbete}

\begin{verbete}[jing1li3]{经理}[8;11]
\begin{pronuncia}{jing1li3}
\significado[个,位,名]{s.}{ gerente }
\end{pronuncia}
\end{verbete}

\begin{verbete}[jing3cha2]{警察}[19;14]
\begin{pronuncia}{jing3cha2}
\significado[个]{s.}{ polícia; agente de polícia }
\end{pronuncia}
\end{verbete}

\begin{verbete}[jiu3]{九}[2]
\begin{pronuncia}{jiu3}
\significado{num.}{ nove; 9 }
\end{pronuncia}
\end{verbete}

\begin{verbete}[jiu3cai4]{韭菜}[9;11]
\begin{pronuncia}{jiu3cai4}
\significado{s.}{ cebolinha chinesa }
\end{pronuncia}
\end{verbete}

\begin{verbete}[jiu3]{酒}[10]
\begin{pronuncia}{jiu3}
\significado[杯,瓶,罐,桶,缸]{s.}{ bebida alcoólica; vinho; aguardente; licor }
\end{pronuncia}
\end{verbete}

\begin{verbete}[jiu3guan3]{酒馆}[10;11]
\begin{pronuncia}{jiu3guan3}
\significado{s.}{ bar }
\end{pronuncia}
\end{verbete}

\begin{verbete}[jiu4]{就}[12]
\begin{pronuncia}{jiu4}
\significado{adv.}{ exatamente; justamente }
\end{pronuncia}
\end{verbete}

\begin{verbete}[ju4]{句}[5]
\begin{pronuncia}{ju4}
\significado{s.}{ sentença; cláusula; frase }
\significado{p.c.}{ para orações, frases ou linhas de versos}
\end{pronuncia}
\end{verbete}

\begin{verbete}[ju4zi0]{句子}[5;3]
\begin{pronuncia}{ju4zi0}
\significado[个]{n}{ sentença }
\end{pronuncia}
\end{verbete}

\begin{verbete}[ju3xing2]{举行}[9;6]
\begin{pronuncia}{ju3xing2}
\significado{v.}{ realizar; ter lugar }
\end{pronuncia}
\end{verbete}

\begin{verbete}[jue2de2]{觉得}[9;11]
\begin{pronuncia}{jue2de2}
\significado{v.}{ achar; sentir }
\end{pronuncia}
\end{verbete}

%%%%% EOF %%%%

%%%
%%% K
%%%
\section*{K}
\addcontentsline{toc}{section}{K}

\begin{verbete}[8;11]{咖啡}{ka1fei1}
\significado[杯]{s.}{ café }
\end{verbete}

\begin{verbete}[8;11;11]{咖啡馆}[\\]{ka1fei1guan3}
\significado[家]{s.}{ cafeteria }
\end{verbete}

\begin{verbete}[4]{开}{kai1}
\significado{p.c.}{ quilate (ouro) }
\significado{v.}{ abrir; ligar; dirigir; iniciar (alguma coisa); ferver; escrever (uma receita, cheque, fatura, etc.) }
\end{verbete}

\begin{verbete}[4;4]{开车}{kai1che1}
\significado{v.}{ conduzir; dirigir }
\end{verbete}

\begin{verbete}[4;5;4]{开尔文}{kai1er3wen2}
\significado{s.}{ Kelvin (escala de temperatura) }
\end{verbete}

\begin{verbete}[4;5;4]{开发区}{kai1fa1qu1}
\significado{s.}{ zona de desenvolvimento }
\end{verbete}

\begin{verbete}[4;8]{开始}{kai1shi3}
\significado{adv.}{ inicial }
\significado[个]{s.}{ começo; início }
\significado{v.}{ começar; iniciar }
\end{verbete}

\begin{verbete}[9]{看}{kan4}
\significado{interj.}{ Cuidado! (para um perigo) }
\significado{part.}{ (depois de um verbo) tentar }
\significado{v.}{ olhar; ver; assistir; ler; visitar (pessoas) }
\end{verbete}

\begin{verbete}[9;4]{看见}{kan4jian4}
\significado{v.}{ encontrar; enxergar; ver; avistar }
\end{verbete}

\begin{verbete}[6;8]{考试}{kao3shi4}
\significado[次]{s.}{ teste; prova; exame }
\significado{v.+compl.}{ submeter-se a uma prova; fazer um teste }
\end{verbete}

\begin{verbete}[10]{烤}{kao3}
\significado{v.}{ assar; grelhar }
\end{verbete}

\begin{verbete}[9;7]{科技}{Ke1ji4}
\significado{s.}{ Ciência e Tecnologia }
\end{verbete}

\begin{verbete}[14]{颗}{ke1}
\significado{p.c.}{ para grãos, pérolas, dentes, corações, satelites, pequenas esferas, etc }
\end{verbete}

\begin{verbete}[9;14]{咳嗽}{ke2sou0}
\significado{v.}{ ter tosse; tossir }
\end{verbete}

\begin{verbete}[5;10]{可爱}{ke3'ai4}
\significado{adj.}{ adorável; querido; fofo }
\end{verbete}

\begin{verbete}[5;3;5;5]{可口可乐}[\\]{ke3kou3ke3le3}
\significado{s.}{ Coca-Cola }
\end{verbete}

\begin{verbete}[5;10]{可能}{ke3neng2}
\significado{adj.}{ possível; provável }
\significado{adv.}{ possivelmente; provavelmente }
\significado[个]{s.}{ possibilidade; probabilidade }
\end{verbete}

\begin{verbete}[5;9]{可是}{ke3shi4}
\significado{adv.}{ (usado para dar ênfase) de fato }
\significado{conj.}{ porém; contudo; mas }
\end{verbete}

\begin{verbete}[5;11]{可惜}{ke3xi1}
\significado{adj.}{ é uma pena; que pena }
\significado{adv.}{ infelizmente }
\end{verbete}

\begin{verbete}[5;4]{可以}{ke3yi3}
\significado{v.o.}{ ser capaz de; poder }
\end{verbete}

\begin{verbete}[8]{刻}{ke4}
\significado{p.c.}{ para curtos intervalos de tempo }
\significado{p.t.}{ quarto (de hora) }
\significado{v.}{ esculpir; cortar; gravar }
\end{verbete}

\begin{verbete}[8;9]{刻钟}{ke4zhong1}
\significado{p.t.}{ um quarto de hora }
\end{verbete}

\begin{verbete}[9;4]{客气}{ke4qi0}
\significado{adj.}{ cortês; delicado; modesto; educado }
\significado{v.}{ fazer cerimônia }
\end{verbete}

\begin{verbete}[9;4]{客厅}{ke4ting1}
\significado[间]{s.}{ sala de estar; sala de visitas }
\end{verbete}

\begin{verbete}[10;5]{课本}{ke4ben3}
\significado[本]{s.}{ livro do aluno; manual }
\end{verbete}

\begin{verbete}[8;8]{肯定}{ken3ding4}
\significado{adv.}{ com certeza; certamente; definitivamente; afirmativo (resposta) }
\significado{v.}{ afirmar; ter a certeza; ser positivo; dar reconhecimento }
\end{verbete}

\begin{verbete}[8;4]{空气}{kong1qi4}
\significado{s.}{ ar; atmosfera }
\end{verbete}

\begin{verbete}[8;10]{空调}{kong1tiao2}
\significado[台]{s.}{ ar-condicionado; condicionador de ar }
\end{verbete}

\begin{verbete}[10;8]{恐怕}{kong3pa4}
\significado{adv.}{ talvez; possivelmente; provavelmente; (em sentido não tão bom) }
\significado{v.}{ temer }
\end{verbete}

\begin{verbete}[8;2]{空儿}{kongr4}
\significado{s.}{ tempo livre }
\significado{v.}{ ter tempo livre }
\end{verbete}

\begin{verbete}[3]{口}{kou3}
\significado{p.c.}{ para coisas com bocas (pessoas, animais domésticos, canhões, etc); para mordidas ou bocados }
\significado{s.}{ boca }
\end{verbete}

\begin{verbete}[3;9;16]{口香糖}[\\]{kou3xiang1tang2}
\significado{s.}{ goma de mascar; chiclete }
\end{verbete}

\begin{verbete}[3;9]{口音}{kou3yin1}
\significado{s.}{ sotaque; voz }
\end{verbete}

\begin{verbete}[3;9]{口语}{kou3yu3}
\significado[门]{s.}{ linguagem oral; linguagem falada; fofoca; calúnia }
\end{verbete}

\begin{verbete}[8;5]{苦瓜}{ku3gua1}
\significado{s.}{ melão amargo (cabaça amarga, pêra bálsamo, maçã bálsamo, pepino amargo) }
\end{verbete}

\begin{verbete}[12;3]{裤子}{ku4zi0}
\significado[条]{s.}{ calças }
\end{verbete}

\begin{verbete}[7]{块}{kuai4}
\significado{p.c.}{ coloquial: para dinheiro e unidades monetárias; para peças ou pedaços de roupa, bolos, sabão, etc }
\significado{s.}{ pedaço; pedaço (de terra); peça }
\end{verbete}

\begin{verbete}[7]{快}{kuai4}
\significado{adj.}{ quase; rápido; depressa }
\significado{v.}{ apressar-se }
\end{verbete}

\begin{verbete}[7;5]{快乐}{kuai4le3}
\significado{adj.}{ feliz; alegre }
\significado{s.}{ felicidade; alegria }
\end{verbete}

\begin{verbete}[12]{款}{kuan3}
\significado{p.c.}{ para versões ou modelos (de um produto) }
\significado[笔,个]{s.}{ montante de dinheiro; fundos; parágrafo; seção }
\end{verbete}                                                                     

%%%%% EOF %%%%%

%%%
%%% L
%%%
\section*{L}
\addcontentsline{toc}{section}{L}
\begin{multicols}{2}

\begin{verbete}[拉拉队]{la1la1dui4}
\entry{la1la1dui4}{n.}{
    claque; torcida
}
\end{verbete}

\begin{verbete}[辣]{la4}
\entry{la4}{adj.}{
    picante
}
\end{verbete}

\begin{verbete}[来]{lai2}
\entry{lai2}{v.}{
    vir; trazer
}
\end{verbete}

\begin{verbete}[蓝]{lan2}
\entry{lan2}{adj.}{
    azul
}
\end{verbete}

\begin{verbete}[蓝色]{lan2se4}
\entry{lan2se4}{n.}{
    cor azul
}
\end{verbete}

\begin{verbete}[篮球]{lan2qiu2}
\entry{lan2qiu2}{n.}{
    basquetebol
}
\end{verbete}

\begin{verbete}[老板]{lao3ban3}
\entry{lao3ban3}{n.}{
    patrão, patroa
}
\end{verbete}

\begin{verbete}[老家]{lao3jia1}
\entry{lao3jia1}{n.}{
    terra natal
}
\end{verbete}

\begin{verbete}[老人家]{lao3ren2jia0}
\entry{lao3ren2jia0}{n.}{
    senhor ancião; madame; senhora
}
\end{verbete}

\begin{verbete}[老师]{lao3shi1}
\entry{lao3shi1}{n.}{
    professor, professora
}
\end{verbete}

\begin{verbete}[了]{le0}
\entry{le0}{part.}{
    partícula para denotar mudança
}
\end{verbete}

\begin{verbete}[冷]{leng3}
\entry{leng3}{adj.}{
    frio, fria
}
\end{verbete}

\begin{verbete}[累]{lei4}
\entry{lei4}{adj.}{
    cansado
}
\end{verbete}

\begin{verbete}[离]{li2}
\entry{li2}{n.}{
    (ser longe) de ... até...
}
\end{verbete}

\begin{verbete}[里]{li3}
\entry{li3}{p.l.}{
    em; dentro; interior
}
\end{verbete}

\begin{verbete}[里斯本]{Li3si1ben3}
\entry{Li3si1ben3}{n.}{
    Lisboa
}
\end{verbete}

\begin{verbete}[里斯本大学]{Li3si1ben3\ Da4xue2}
\entry{Li3si1ben3\ Da4xue2}{n.}{
    Universidade de Lisboa
}
\end{verbete}

\begin{verbete}[礼节]{li3jie2}
\entry{li3jie2}{n.}{
    cortesia; protocolo; cerimônia
}
\end{verbete}

\begin{verbete}[礼物]{li3wu4}
\entry{li3wu4}{n.}{
    prenda; lembrança; presente|
    \pc{件}
}
\end{verbete}

\begin{verbete}[厉害]{li4hai0}
\entry{li4hai0}{adj.}{
    severo; rigoroso; exigente
}
\end{verbete}

\begin{verbete}[历史]{li4shi3}
\entry{li4shi3}{n.}{
    história
}
\end{verbete}

\begin{verbete}[脸]{lian3}
\entry{lian3}{n.}{
    cara; rosto; face
}
\end{verbete}

\begin{verbete}[凉快]{liang2kuai0}
\entry{liang2kuai0}{adj.}{
    agradável; fresco, fresca
}
\end{verbete}

\begin{verbete}[两]{liang3}
\entry{liang3}{num.}{
    2|
    dois, duas|
    (sempre usado antes de p.c.)
}
\end{verbete}

\begin{verbete}[辆]{liang4}
\entry{liang4}{p.c.}{
    palavra classificadora para automóveis
}
\end{verbete}

\begin{verbete}[邻居]{lin2ju1}
\entry{lin2ju1}{n.}{
    vizinho
}
\end{verbete}

\begin{verbete}[零/\Circle]{ling2}
\entry{ling2}{num.}{
    0|
    zero
}
\end{verbete}

\begin{verbete}[领导]{ling3dao3}
\entry{ling3dao3}{n.}{
    chefe; dirigente
}
\end{verbete}

\begin{verbete}[流利]{liu2li4}
\entry{liu2li4}{adj.}{
    fluente
}
\end{verbete}

\begin{verbete}[六]{liu4}
\entry{liu4}{num.}{
    6|
    seis
}
\end{verbete}

\begin{verbete}[遛狗]{liu4gou3}
\entry{liu4gou3}{v.+compl.}{
    passear o cão
}
\end{verbete}

\begin{verbete}[伦敦]{lun2dun1}
\entry{lun2dun1}{n.}{
    Londres
}
\end{verbete}

\begin{verbete}[龙]{long2}
\entry{long2}{n.}{
    dragão
}
\end{verbete}

\begin{verbete}[龙山]{Long2shan1}
\entry{Long2shan1}{n.}{
    Longshan
}
\end{verbete}

\begin{verbete}[楼]{lou2}
\entry{lou2}{n.}{
    edifício; prédio
}
\end{verbete}

\begin{verbete}[路]{lu4}
\entry{lu4}{n.}{
    caminho; via
}
\end{verbete}

\begin{verbete}[路口]{lu4kou3}
\entry{lu4kou3}{n.}{
    cruzamento; interseção
}
\end{verbete}

\begin{verbete}[录像带]{lu4xiang4dai4}
\entry{lu4xiang4dai4}{n.}{
    video-cassete|
    \pc{盘}
}
\end{verbete}

\begin{verbete}[录像机]{lu4xiang4ji1}
\entry{lu4xiang4ji1}{n.}{
    gravador de vídeo|
    \pc{台}
}
\end{verbete}

\begin{verbete}[录音机]{lu4yin1ji1}
\entry{lu4yin1ji1}{n.}{
    gravador|
    \pc{台}
}
\end{verbete}

\begin{verbete}[旅行]{lv3xing2}
\entry{lv3xing2}{v.}{
    viajar
}
\end{verbete}

\begin{verbete}[旅游]{lv3you2}
\entry{lv3you2}{v.}{
    viajar
}
\end{verbete}

\begin{verbete}[绿]{lv4}
\entry{lv4}{adj.}{
    verde
}
\end{verbete}

\begin{verbete}[绿色]{lv4se4}
\entry{lv4se4}{n.}{lv4se4}{
    cor verde
}
\end{verbete}

\end{multicols}

%%%
%%% M
%%%
\section*{M}
\addcontentsline{toc}{section}{M}

\begin{verbete}[ma0]{吗}[6]
\begin{pronuncia}{ma0}
\significado{part.}{ partícula interrogativa (usado em perguntas ``sim-não'') }
\end{pronuncia}
\end{verbete}

\begin{verbete}[ma1ma0]{妈妈}[6;6]
\begin{pronuncia}{ma1ma0}
\significado[个,位]{s.}{ma1ma0}{ mamãe, mãe }
\end{pronuncia}
\end{verbete}

\begin{verbete}[ma2fan0]{麻烦}[11;10]
\begin{pronuncia}{ma2fan0}
\significado{adj.}{ enfastidioso; maçante; inconveniente; problemático }
\significado{s.}{ incômodo }
\significado{v.}{ incomodar alguém; colocar alguém em apuros }
\end{pronuncia}
\end{verbete}

\begin{verbete}[ma2la4dou4fu0]{麻辣豆腐}[11;14;7;14]
\begin{pronuncia}[\\]{ma2la4dou4fu0}
  \significado{s.}{ tofú guisado em molho picante (prato) }
\end{pronuncia}
\end{verbete}

\begin{verbete}[ma3lu4]{马路}[3;13]
\begin{pronuncia}{ma3lu4}
\significado[条]{s.}{ rua; estrada }
\end{pronuncia}
\end{verbete}

\begin{verbete}[ma3shang4]{马上}[3;3]
\begin{pronuncia}{ma3shang4}
\significado{adv.}{ já; imediatamente; de imediato; sem demora }
\end{pronuncia}
\end{verbete}

\begin{verbete}[mai3]{买}[6]
\begin{pronuncia}{mai3}
\significado{v.}{ comprar }
\end{pronuncia}
\end{verbete}

\begin{verbete}[mai3dong1xi0]{买东西}[6;5;6]
\begin{pronuncia}{mai3dong1xi0}
\significado{v.}{ fazer compras }
\end{pronuncia}
\end{verbete}

\begin{verbete}[mai4]{卖}[8]
\begin{pronuncia}{mai4}
\significado{v.}{ vender }
\end{pronuncia}
\end{verbete}

\begin{verbete}[mao1]{猫}[11]
\begin{pronuncia}{mao1}
\significado[只]{s.}{ gato; coloquial: MODEM }
\significado{v.}{ dialeto: esconder-se }
\end{pronuncia}
\end{verbete}

\begin{verbete}[mao2]{毛}[4]
\begin{pronuncia}{mao2}
\significado{p.c.}{ 1 mao = 10 centavos }
\end{pronuncia}
\end{verbete}

\begin{verbete}[mao4yi4]{贸易}[9;8]
\begin{pronuncia}{mao4yi4}
\significado[个]{s.}{ transação comercial }
\significado{v.}{ fazer uma transação comercial }
\end{pronuncia}
\end{verbete}

\begin{verbete}[man3yi4]{满意}[13;13]
\begin{pronuncia}{man3yi4}
\significado{adj.}{ satisfeito }
\end{pronuncia}
\end{verbete}

\begin{verbete}[man4]{慢}[14]
\begin{pronuncia}{man4}
\significado{adj.}{ lento; devagar }
\significado{adv.}{ lentamente }
\end{pronuncia}
\end{verbete}

\begin{verbete}[mang1]{忙}[6]
\begin{pronuncia}{mang1}
\significado{adj.}{ ocupado }
\significado{v.}{ apressar }
\end{pronuncia}
\end{verbete}

\begin{verbete}[mei2guan1xi0]{没关系}[7;6;7]
\begin{pronuncia}{mei2guan1xi0}
\significado{v.}{ não ter problema; não ter importância; não fazer mal }
\end{pronuncia}
\end{verbete}

\begin{verbete}[mei2you3]{没有}[7;6]
\begin{pronuncia}{mei2you3}
\significado{v.}{ não há; não tem; não existe }
\end{pronuncia}
\end{verbete}

\begin{verbete}[mei2mao0]{眉毛}[9;4]
\begin{pronuncia}{mei2mao0}
\significado[根]{s.}{ sobrancelha }
\end{pronuncia}
\end{verbete}

\begin{verbete}[mei3]{每}[7]
\begin{pronuncia}{mei3}
\significado{pron.}{ cada }
\end{pronuncia}
\end{verbete}

\begin{verbete}[mei3ci4]{每次}[7;6]
\begin{pronuncia}{mei3ci4}
\significado{adv.}{ toda vez; cada vez }
\end{pronuncia}
\end{verbete}

\begin{verbete}[mei3tian1]{每天}[7;4]
\begin{pronuncia}{mei3tian1}
\significado{adv.}{ todo dia; cada dia }
\end{pronuncia}
\end{verbete}

\begin{verbete}[Mei3guo1]{美国}[9;8]
\begin{pronuncia}{Mei3guo1}
\significado{s.}{ Estados Unidos da América }
\end{pronuncia}
\end{verbete}

\begin{verbete}[mei3guo1ren2]{美国人}[9;8;2]
\begin{pronuncia}{mei3guo1ren2}
\significado{s.}{ americano; nascido nos Estados Unidos da América }
\end{pronuncia}
\end{verbete}

\begin{verbete}[mei3li4]{美丽}[9;7]
\begin{pronuncia}{mei3li4}
\significado{adj.}{ bonito; lindo }
\end{pronuncia}
\end{verbete}

\begin{verbete}[Mei3zhou1]{美洲}[9;9]
\begin{pronuncia}{Mei3zhou1}
\significado{s.}{ América }
\end{pronuncia}
\end{verbete}

\begin{verbete}[mei3zhou1ren2]{美洲人}[9;9;2]
\begin{pronuncia}{mei3zhou1ren2}
\significado{s.}{ americano; nascido no continente Americano }
\end{pronuncia}
\end{verbete}

\begin{verbete}[mei4fu0]{妹夫}[8;4]
\begin{pronuncia}{mei4fu0}
\significado{s.}{ marido da irmã mais nova }
\end{pronuncia}
\end{verbete}

\begin{verbete}[mei4mei0]{妹妹}[8;8]
\begin{pronuncia}{mei4mei0}
\significado[个]{s.}{ irmã mais nova; mulher jovem }
\end{pronuncia}
\end{verbete}

\begin{verbete}[men2kou3]{门口}[3;3]
\begin{pronuncia}{men2kou3}
\significado{p.d.l.}{ porta; portão }
\end{pronuncia}
\end{verbete}

\begin{verbete}[men0]{们}[5]
\begin{pronuncia}{men0}
\significado{part.}{ sufixo para plural de pronomes e substantivos referentes a indivíduos }
\end{pronuncia}
\end{verbete}

\begin{verbete}[mi3fan4]{米饭}[6;7]
\begin{pronuncia}{mi3fan4}
\significado{s.}{ arroz cozido }
\end{pronuncia}
\end{verbete}

\begin{verbete}[mian4]{面}[9]
\begin{pronuncia}{mian4}
\significado{p.c.}{ para objetos com superfície plana como tambores, espelhos, bandeiras, etc }
\significado{s.}{ farinha; massa; gíria: (uma pessoa) ineficaz }
\end{pronuncia}
\end{verbete}

\begin{verbete}[mian4bao1]{面包}[9;5]
\begin{pronuncia}{mian4bao1}
\significado[片,袋,块]{s.}{ pão }
\end{pronuncia}
\end{verbete}

\begin{verbete}[mian4ji1]{面积}[9;10]
\begin{pronuncia}{mian4ji1}
\significado{s.}{ área (de um andar, pedaço de terreno, etc.); área de superfície; pedaço de terra }
\end{pronuncia}
\end{verbete}

\begin{verbete}[mian4tiao2]{面条}[9;7]
\begin{pronuncia}{mian4tiao2}
\significado{s.}{ macarrão; espaguete }
\end{pronuncia}
\end{verbete}

\begin{verbete}[ming2pian4]{名片}[6;4]
\begin{pronuncia}{ming2pian4}
\significado{s.}{ cartão de visita }
\end{pronuncia}
\end{verbete}

\begin{verbete}[ming2zi0]{名字}[6;6]
\begin{pronuncia}{ming2zi0}
\significado[个]{s.}{ nome (de uma pessoa ou coisa) }
\end{pronuncia}
\end{verbete}

\begin{verbete}[ming2bai0]{明白}[8;5]
\begin{pronuncia}{ming2bai0}
\significado{adj.}{ compreendido; percebido; óbvio; inequívoco }
\significado{v.}{ compreender; perceber }
\end{pronuncia}
\end{verbete}

\begin{verbete}[ming2tian1]{明天}[8;4]
\begin{pronuncia}{ming2tian1}
\significado{p.t.}{ amanhã }
\end{pronuncia}
\end{verbete}

\begin{verbete}[ming2nian2]{明年}[8;6]
\begin{pronuncia}{ming2nian2}
\significado{s.}{ próximo ano }
\end{pronuncia}
\end{verbete}

\begin{verbete}[mo2gu0]{蘑菇}[19;11]
\begin{pronuncia}{mo2gu0}
\significado{s.}{ cogumelo }
\significado{v.}{ mandriar; embromar; amofinar; incomodar alguém com solicitações ou interrupções frequentes ou persistentes }
\end{pronuncia}
\end{verbete}

\begin{verbete}[mo4jing4]{墨镜}[15;16]
\begin{pronuncia}{mo4jing4}
\significado[只,双,副]{s.}{ óculos escuros }
\end{pronuncia}
\end{verbete}

\begin{verbete}[mu3qin0]{母亲}[5;9]
\begin{pronuncia}{mu3qin0}
\significado[个]{s.}{ mãe }
\end{pronuncia}
\begin{pronuncia}{mu3qin1}
\significado*{}{ 母亲\p{mu3qin1} }
\end{pronuncia}
\end{verbete}                                                                     

\begin{verbete}[mu3qin1]{母亲}[5;9]
\begin{pronuncia}{mu3qin1}
\significado[个]{s.}{ mãe }
\end{pronuncia}
\begin{pronuncia}{mu3qin0}
\significado*{}{ 母亲\p{mu3qin0} }
\end{pronuncia}
\end{verbete}                                                                     

\begin{verbete}[mu3yu3]{母语}[5;9]
\begin{pronuncia}{mu3yu3}
\significado{s.}{ língua materna; língua nativa }
\end{pronuncia}
\end{verbete}                                                                     

%%%%% EOF %%%%%

%%%
%%% N
%%%
\section*{N}
\addcontentsline{toc}{section}{N}
\begin{multicols}{2}

\begin{verbete}[拿]{na2}
\entry{na2}{v.}{
    segurar; tomar; pegar em
}
\end{verbete}

\begin{verbete}[哪]{na3}
\entry{na3}{interr.}{
    que?; qual?
}
\end{verbete}

\begin{verbete}[哪儿]{nar3}
\entry{nar3}{interr.}{
    onde?
}
\end{verbete}

\begin{verbete}[哪国人]{na3guo2ren2}
\entry{na3guo2ren2}{interr.}{
    de qual país ... ?
}
\end{verbete}

\begin{verbete}[哪里]{na3li0}
\entry{na3li0}{interr.}{na3li0}{
    onde?
}
\end{verbete}

\begin{verbete}[哪些]{na3xie1}
\entry{na3xie1}{interr.}{
    quais?
}
\end{verbete}

\begin{verbete}[那]{na4}
\entry{na4}{conj.}{
    nessa situação; nesse caso
}
\entry{na4}{pron.}{
    aquele; aquilo; aquela
}
\end{verbete}

\begin{verbete}[那里]{na4li0}
\entry{na4li0}{pron.}{
    lá; ali
}
\end{verbete}

\begin{verbete}[那么]{na4me0}
\entry{na4me0}{adv.}{
    então; como aquele; dessa maneira
}
\end{verbete}

\begin{verbete}[那些]{na4xie1}
\entry{na4xie1}{pron.}{
    aqueles, aquelas
}
\end{verbete}

\begin{verbete}[那儿]{nar4}
\entry{nar4}{pron.}{
    lá; ali
}
\end{verbete}

\begin{verbete}[奶奶]{nai3nai0}
\entry{nai3nai0}{n.}{
    avó(paterna)
}
\end{verbete}

\begin{verbete}[男]{nan2}
\entry{nan2}{adj.}{
    masculino
}
\end{verbete}

\begin{verbete}[男朋友]{nan2peng2you0}
\entry{nan2peng2you0}{n.}{
    namorado
}
\end{verbete}

\begin{verbete}[男孩儿]{nan2hair2}
\entry{nan2hair2}{n.}{
    menino; rapaz
}
\end{verbete}

\begin{verbete}[南边]{nan2bian0}
\entry{nan2bian0}{p.l.}{
    sul
}
\end{verbete}

\begin{verbete}[南方]{nan2fang1}
\entry{nan2fang1}{p.l.}{
    sul
}
\end{verbete}

\begin{verbete}[南面]{nan2mian0}
\entry{nan2mian0}{p.l.}{
    sul
}
\end{verbete}

\begin{verbete}[呢]{ne0}
\entry{ne0}{interr.}{
    partícula interrogativa enfática
}
\end{verbete}

\begin{verbete}[能]{neng2}
\entry{neng2}{v.}{
    poder
}
\end{verbete}

\begin{verbete}[你]{ni3}
\entry{ni3}{pron.}{
    você (informal); tu
}
\end{verbete}

\begin{verbete}[你的]{ni3de0}
\entry{ni3de0}{pron.}{
    seu, sua
}
\end{verbete}

\begin{verbete}[你们]{ni3men0}
\entry{ni3men0}{pron.}{
    vocês (informal); vós
}
\end{verbete}

\begin{verbete}[你们的]{ni3men0de0}
\entry{ni3men0de0}{pron.}{
    vossos, vossas
}
\end{verbete}

\begin{verbete}[年]{nian2}
\entry{nian2}{p.t.}{
    ano
}
\end{verbete}

\begin{verbete}[年纪]{nian2ji4}
\entry{nian2ji4}{n.}{
    idade
}
\end{verbete}

\begin{verbete}[年货]{nian2huo4}
\entry{nian2huo4}{n.}{
    mercadorias de Ano Novo Chinês
}
\end{verbete}

\begin{verbete}[年轻]{nian2qing1}
\entry{nian2qing1}{adj.}{
    jovem
}
\end{verbete}

\begin{verbete}[鸟儿]{niaor3}
\entry{niaor3}{n.}{
    pássaro; ave|
    \pc{只}
}
\end{verbete}

\begin{verbete}[您]{nin2}
\entry{nin2}{pron.}{
    você (formal); tu
}
\end{verbete}

\begin{verbete}[牛]{niu2}
\entry{niu2}{n.}{
    boi, vaca
}
\end{verbete}

\begin{verbete}[牛奶]{niu2nai3}
\entry{niu2nai3}{n.}{niu2nai3}{
    leite|
    \pc{杯}
}
\end{verbete}

\begin{verbete}[牛肉]{niu2rou4}
\entry{niu2rou4}{n.}{
    carne de vaca
}
\end{verbete}

\begin{verbete}[牛仔裤]{niu2zai3ku4}
\entry{niu2zai3ku4}{n.}{
    calça de ganga, jeans|
    \pc{条}
}
\end{verbete}

\begin{verbete}[农村]{nong2cun1}
\entry{nong2cun1}{n.}{
    campo rural; aldeia; povoação rústica
}
\end{verbete}

\begin{verbete}[暖和]{nuan3huo0}
\entry{nuan3huo0}{adj.}{
    morno, morna; quente
}
\end{verbete}

\begin{verbete}[暖气]{nuan3qi4}
\entry{nuan3qi4}{n.}{
    aquecimento
}
\end{verbete}

\begin{verbete}[女]{nv3}
\entry{nv3}{adj.}{
    feminino
}
\end{verbete}

\begin{verbete}[女儿]{nv3'er2}
\entry{nv3'er2}{n.}{
    filha
}
\end{verbete}

\begin{verbete}[女孩儿]{nv3hair2}
\entry{nv3hair2}{n.}{
    menina; garota
}
\end{verbete}

\begin{verbete}[女朋友]{nv3peng2you0}
\entry{nv3peng2you0}{n.}{
    namorada
}
\end{verbete}

\begin{verbete}[女王]{nv3wang2}
\entry{nv3wang2}{n.}{
    rainha
}
\end{verbete}

\begin{verbete}[女婿]{nv3xu4}
\entry{nv3xu4}{n.}{
    marido da filha
}
\end{verbete}

\end{multicols}

%%%
%%% O
%%%
\section*{O}
\addcontentsline{toc}{section}{O}
\begin{multicols}{2}

\begin{verbete}[欧洲]{Ou1zhou1}
\entry{Ou1zhou1}{n.}{
    Europa
}
\end{verbete}

\end{multicols}

%%%
%%% P
%%%
\section*{P}
\addcontentsline{toc}{section}{P}

\begin{verbete}[pa2]{爬}[8]
\begin{pronuncia}{pa2}
\significado{v.}{ escalar; trepar }
\end{pronuncia}
\end{verbete}

\begin{verbete}[pa4]{怕}[8]
\begin{pronuncia}{pa4}
\significado{v.}{ ter medo de }
\end{pronuncia}
\end{verbete}

\begin{verbete}[pai1zhao4]{拍照}[8;13]
\begin{pronuncia}{pai1zhao4}
\significado{v.+compl.}{ tirar fotografia }
\end{pronuncia}
\end{verbete}

\begin{verbete}[pai2qiu2]{排球}[11;11]
\begin{pronuncia}{pai2qiu2}
\significado[个]{s.}{ voleibol }
\end{pronuncia}
\end{verbete}

\begin{verbete}[pan2]{盘}[11]
\begin{pronuncia}{pan2}
\significado{p.c.}{ para cassete; vídeo-cassete }
\end{pronuncia}
\end{verbete}

\begin{verbete}[pang2bian1]{旁边}[10;5]
\begin{pronuncia}{pang2bian1}
\significado{p.l.}{ junto a; próximo de; ao lado }
\end{pronuncia}
\end{verbete}

\begin{verbete}[pang4]{胖}[9]
\begin{pronuncia}{pang4}
\significado{adj.}{ gordo }
\end{pronuncia}
\end{verbete}

\begin{verbete}[pao3bu4]{跑步}[12;7]
\begin{pronuncia}{pao3bu4}
\significado{v.}{ correr }
\end{pronuncia}
\end{verbete}

\begin{verbete}[pei2]{陪}[10]
\begin{pronuncia}{pei2}
\significado{v.}{ acompanhar }
\end{pronuncia}
\end{verbete}

\begin{verbete}[pei3]{配}[10]
\begin{pronuncia}{pei3}
\significado{v.}{ combinar }
\end{pronuncia}
\end{verbete}

\begin{verbete}[peng2you0]{朋友}[8;4]
\begin{pronuncia}{peng2you0}
\significado[个,位]{s.}{ namorado; amigo }
\end{pronuncia}
\end{verbete}

\begin{verbete}[pi2jiu3]{啤酒}[11;10]
\begin{pronuncia}{pi2jiu3}
\significado[杯,瓶,罐,桶,缸]{s.}{ cerveja }
\end{pronuncia}
\end{verbete}

\begin{verbete}[pi2jiu3guan3]{啤酒馆}[8;10;11]
\begin{pronuncia}{pi2jiu3guan3}
\significado{s.}{ cervejaria }
\end{pronuncia}
\end{verbete}

\begin{verbete}[pi4gu]{屁股}[7;8]
\begin{pronuncia}{pi4gu}
\significado{s.}{ nádega; quadris }
\end{pronuncia}
\end{verbete}

\begin{verbete}[piao4]{票}[11]
\begin{pronuncia}{piao4}
\significado[张]{s.}{ bilhete }
\end{pronuncia}
\end{verbete}

\begin{verbete}[piao4liang0]{漂亮}[14;9]
\begin{pronuncia}{piao4liang0}
\significado{adj.}{ bonita, linda; bonito, lindo (para objetos inanimados) }
\end{pronuncia}
\end{verbete}

\begin{verbete}[pian4]{片}[4]
\begin{pronuncia}{pian4}
\significado{p.c.}{ para algumas coisas finas, com pouca espessura; fatia, rodela }
\end{pronuncia}
\end{verbete}

\begin{verbete}[ping2]{瓶}[10]
\begin{pronuncia}{ping2}
\significado[个]{s.}{ garrafa }
\significado{p.c.}{ para vinho ou líquidos }
\end{pronuncia}
\end{verbete}

\begin{verbete}[ping2shi2]{平时}[5;7]
\begin{pronuncia}{ping2shi2}
\significado{p.t.}{ normalmente; numa época normal }
\end{pronuncia}
\end{verbete}

\begin{verbete}[ping2guo3]{苹果}[8;8]
\begin{pronuncia}{ping2guo3}
\significado[个,颗]{s.}{ maçã }
\end{pronuncia}
\end{verbete}

\begin{verbete}[po4]{破}[10]
\begin{pronuncia}{po4}
\significado{adj.}{ partido; quebrado; roto }
\end{pronuncia}
\end{verbete}

\begin{verbete}[pu2han4ci2dian3]{葡汉词典}[12;5;7;8]
\begin{pronuncia}[\\]{pu2han4ci2dian3}
\significado{s.}{ dicionário português-chinês }
\end{pronuncia}
\end{verbete}

\begin{verbete}[Pu2tao2ya2]{葡萄牙}[12;11;4]
\begin{pronuncia}{Pu2tao2ya2}
\significado{s.}{ Portugal }
\end{pronuncia}
\end{verbete}

\begin{verbete}[pu2tao2ya2yu3]{葡萄牙语}[12;11;4;9]
\begin{pronuncia}[\\]{pu2tao2ya2yu3}
\significado{s.}{ português, língua portuguesa }
\end{pronuncia}
\end{verbete}

\begin{verbete}[pu2yu3]{葡语}[12;9]
\begin{pronuncia}{pu2yu3}
\significado{s.}{ português, língua portuguesa }
\end{pronuncia}
\end{verbete}

\begin{verbete}[pu2wen2]{葡文}[12;4]
\begin{pronuncia}{pu2wen2}
\significado{s.}{ português, língua portuguesa }
\end{pronuncia}
\end{verbete}

\begin{verbete}[pu3tong1hua4]{普通话}[12;10;8]
\begin{pronuncia}{pu3tong1hua4}
\significado{s.}{ mandarim (lit. ``linguagem comum'') }
\end{pronuncia}
\end{verbete}

\begin{verbete}[pian2yi0]{便宜}[9;8]
\begin{pronuncia}{pian2yi0}
\significado{adj.}{ barato }
\end{pronuncia}
\end{verbete}

\begin{verbete}[ping1pang1qiu2]{乒乓球}[6;6;11]
\begin{pronuncia}[\\]{ping1pang1qiu2}
\significado[个]{s.}{ tênis de mesa; ping-pong }
\end{pronuncia}
\end{verbete}

%%%%% EOF %%%%%

%%%
%%% Q
%%%
\section*{Q}
\addcontentsline{toc}{section}{Q}
\begin{multicols}{2}

\begin{verbete}[七]{qi1}
\entry{qi1}{num.}{
    7|
    sete
}
\end{verbete}

\begin{verbete}[骑]{qi2}
\entry{qi2}{v.}{
    andar de
}
\end{verbete}

\begin{verbete}[骑车]{qi2che1}
\entry{qi2che1}{v.}{
    andar de bicicleta
}
\end{verbete}

\begin{verbete}[起]{qi3}
\entry{qi3}{v.}{qi3}{
    levantar
}
\end{verbete}

\begin{verbete}[起床]{qi3chuang2}
\entry{qi3chuang2}{v.+compl.}{
    levantar-se
}
\end{verbete}

\begin{verbete}[汽车]{qi4che1}
\entry{qi4che1}{n.}{
    automóvel; veículo motorizado
}
\end{verbete}

\begin{verbete}[气温]{qi4wen1}
\entry{qi4wen1}{n.}{
    temperatura do ar
}
\end{verbete}

\begin{verbete}[千]{qian1}
\entry{qian1}{num.}{
    1.000|
    mil
}
\end{verbete}

\begin{verbete}[钱]{qian2}
\entry{qian2}{n.}{
    dinheiro
}
\end{verbete}

\begin{verbete}[前]{qian2}
\entry{qian2}{p.l.}{
    frente; em frente de
}
\end{verbete}

\begin{verbete}[前年]{qian2nian2}
\entry{qian2nian2}{p.t.}{
    há dois anos
}
\end{verbete}

\begin{verbete}[前面]{qian2bian0}
\entry{qian2bian0}{p.l.}{
    à frente; da frente
}
\end{verbete}

\begin{verbete}[前面]{qian2mian0}
\entry{qian2mian0}{p.l.}{
    à frente; da frente
}
\end{verbete}

\begin{verbete}[前天]{qian2tian1}
\entry{qian2tian1}{p.t.}{
    anteontem
}
\end{verbete}

\begin{verbete}[钱包]{qian2bao1}
\entry{qian2bao1}{n.}{
    carteira
}
\end{verbete}

\begin{verbete}[强]{qiang2}
\entry{qiang2}{adj.}{
    forte
}
\end{verbete}

\begin{verbete}[巧克力]{qiao3ke4li4}
\entry{qiao3ke4li4}{n.}{
    chocolate
}
\end{verbete}

\begin{verbete}[青菜]{qing1cai4}
\entry{qing1cai4}{n.}{
    verduras
}
\end{verbete}

\begin{verbete}[青椒]{qing1jiao1}
\entry{qing1jiao1}{n.}{
    pimenta verde
}
\end{verbete}

\begin{verbete}[青天]{qing2tian1}
\entry{qing2tian1}{n.}{
    céu claro; céu limpo; céu azul
}
\end{verbete}

\begin{verbete}[请]{qing3}
\entry{qing3}{v.}{
    fazer o favor
}
\end{verbete}

\begin{verbete}[请客]{qing3ke4}
\entry{qing3ke4}{v.+compl.}{
    convidar para comer e beber; banquetear
}
\end{verbete}

\begin{verbete}[请问]{qing3wen4}
\entry{qing3wen4}{}{
    Desculpe... (para perguntar por qualquer coisa)
}
\end{verbete}

\begin{verbete}[请假条]{qing3jia4tiao2}
\entry{qing3jia4tiao2}{n.}{
    pedido de licença para ausência
}
\end{verbete}

\begin{verbete}[秋天]{qiu1tian1}
\entry{qiu1tian1}{n.}{
    outono
}
\entry{qiu1tian1}{p.t.}{
    outono
}
\end{verbete}

\begin{verbete}[球]{qiu2}
\entry{qiu2}{n.}{
    bola (futebol, basquetebol, handbol, etc)
}
\end{verbete}

\begin{verbete}[曲棍球]{qu1gun4qiu2}
\entry{qu1gun4qiu2}{n.}{
    hóquei em campo
}
\end{verbete}

\begin{verbete}[去]{qu4}
\entry{qu4}{v.}{
    ir
}
\end{verbete}

\begin{verbete}[去年]{qu4nian2}
\entry{qu4nian2}{n.}{
    ano passado
}
\end{verbete}

\begin{verbete}[裙子]{qun2zi0}
\entry{qun2zi0}{n.}{
    saia; vestido
}
\end{verbete}

\end{multicols}

%%%
%%% R
%%%
\section*{R}
\addcontentsline{toc}{section}{R}

\begin{verbete}[ran2hou4]{然后}[12;6]
\begin{pronuncia}{ran2hou4}
\significado{conj.}{ depois; logo; portanto }
\end{pronuncia}
\end{verbete}

\begin{verbete}[rang4]{让}[5]
\begin{pronuncia}{rang4}
\significado{v.}{ deixar; permitir; conceder }
\end{pronuncia}
\end{verbete}

\begin{verbete}[re4]{热}[10]
\begin{pronuncia}{re4}
\significado{adj.}{ quente }
\end{pronuncia}
\end{verbete}

\begin{verbete}[re4nao0]{热闹}[10;8]
\begin{pronuncia}{re4nao0}
\significado{adj.}{ animado; movimentado }
\end{pronuncia}
\end{verbete}

\begin{verbete}[ren2]{人}[2]
\begin{pronuncia}{ren2}
\significado[个,位]{s.}{ pessoa; gente }
\end{pronuncia}
\end{verbete}

\begin{verbete}[ren2kou3]{人口}[2;3]
\begin{pronuncia}{ren2kou3}
\significado{s.}{ população }
\end{pronuncia}
\end{verbete}

\begin{verbete}[Ren2min2bi4]{人民币}[2;5;4]
\begin{pronuncia}{Ren2min2bi4}
\significado{s.}{ RMB; CYN; nome da moeda chinesa }
\end{pronuncia}
\end{verbete}

\begin{verbete}[ren4shi0]{认识}[4;7]
\begin{pronuncia}{ren4shi0}
\significado{v.}{ conhecer }
\end{pronuncia}
\end{verbete}

\begin{verbete}[ri4]{日}[4]
\begin{pronuncia}{ri4}
\significado{p.c.}{ dia (mais usado em escrita) }
\end{pronuncia}
\end{verbete}

\begin{verbete}[Ri4ben3]{日本}[4;5]
\begin{pronuncia}{Ri4ben3}
\significado{s.}{ Japão }
\end{pronuncia}
\end{verbete}

\begin{verbete}[ri4ben3ren2]{日本人}[4;5;2]
\begin{pronuncia}{ri4ben3ren2}
\significado{s.}{ japonês; nascido no Japão }
\end{pronuncia}
\end{verbete}

\begin{verbete}[rong2yi4]{容易}[10;8]
\begin{pronuncia}{rong2yi4}
\significado{adj.}{ fácil }
\end{pronuncia}
\end{verbete}

\begin{verbete}[rou4]{肉}[6]
\begin{pronuncia}{rou4}
\significado{s.}{ carne; polpa de uma fruta }
\end{pronuncia}
\end{verbete}

\begin{verbete}[ru2guo3]{如果}[6;8]
\begin{pronuncia}{ru2guo3}
\significado{conj.}{ se; caso; no caso de }
\end{pronuncia}
\end{verbete}

\begin{verbete}[ru3fang2]{乳房}[8;8]
\begin{pronuncia}{ru3fang2}
\significado{s.}{ seio; mama }
\end{pronuncia}
\end{verbete}

%%%%% EOF %%%%%

%%%
%%% S
%%%
\section*{S}
\addcontentsline{toc}{section}{S}
\begin{multicols}{2}

\begin{verbete}[三]{san1}
\significado{san1}{num.}{
    3|
    três
}
\end{verbete}

\begin{verbete}[散步]{san4bu4}
\significado{san4bu4}{v.+compl.}{
    dar um passeio; passear
}
\end{verbete}

\begin{verbete}[嫂子]{sao3zi0}
\significado{sao3zi0}{n.}{
    esposa do irmão mais velho
}
\end{verbete}

\begin{verbete}[森林]{sen1lin2}
\significado{sen1lin2}{n.}{
    floresta
}
\end{verbete}

\begin{verbete}[沙漠]{sha1mo4}
\significado{sha1mo4}{n.}{
    deserto
}
\end{verbete}

\begin{verbete}[山]{shan1}
\significado{shan1}{n.}{
    montanha; monte
}
\end{verbete}

\begin{verbete}[山东]{Shan4dong3}
\significado{Shan4dong3}{n.}{
    Shandong
}
\end{verbete}

\begin{verbete}[山区]{shan1qu1}
\significado{shan1qu1}{n.}{
    área montanhosa; montanhas
}
\end{verbete}

\begin{verbete}[商店]{shang1dian4}
\significado{shang1dian4}{n.}{
    loja
}
\end{verbete}

\begin{verbete}[商贸]{shang1mao4}
\significado{shang1mao4}{n.}{
    comércio
}
\end{verbete}

\begin{verbete}[赏赐]{shang3ci4}
\significado{shang3ci4}{n.}{
    recompensa; prêmio
}
\significado{shang3ci4}{v.}{
    recompensar; premiar
}
\end{verbete}

\begin{verbete}[上]{shang4}
\significado{shang4}{p.l.}{
    acima; em cima de
}
\significado{shang4}{v.d.}{
    subir
}
\end{verbete}

\begin{verbete}[上班]{shang4ban1}
\significado{shang4ban1}{v.+compl.}{
    ir para o trabalho
}
\end{verbete}

\begin{verbete}[上边]{shang4bian0}
\significado{shang4bian0}{p.l.}{
    acima de; parte de cima; por cima
}
\end{verbete}

\begin{verbete}[上车]{shang4che0}
\significado{shang4che0}{v.}{
    entrar (em ônibus)
}
\end{verbete}

\begin{verbete}[上海]{Shang4hai3}
\significado{Shang4hai3}{n.}{
    Shangai (Xangai)
}
\end{verbete}

\begin{verbete}[上课]{shang4ke4}
\significado{shang4ke4}{v.}{
    ter aulas
}
\end{verbete}

\begin{verbete}[上面]{shang4mian0}
\significado{shang4mian0}{p.l.}{
    acima de; parte de cima; por cima
}
\end{verbete}

\begin{verbete}[上网]{shang4wang3}
\significado{shang4wang3}{v.}{
    acessar a Internet
}
\end{verbete}

\begin{verbete}[上午]{shang4wu3}
\significado{shang4wu3}{p.t.}{
    manhã; de manhã; período antes do meio-dia
}
\end{verbete}

\begin{verbete}[上询]{shang4xun2}
\significado{shang4xun2}{p.t.}{
    primeira dezena do mês
}
\end{verbete}

\begin{verbete}[少]{shao3}
\significado{shao3}{adj.}{
    pouco, poucos
}
\end{verbete}

\begin{verbete}[舌头]{she2tou0}
\significado{she2tou0}{n.}{
    língua
}
\end{verbete}

\begin{verbete}[摄氏]{she2shi4}
\significado{she4shi4}{n.}{
    Celsius, centígrado
}
\end{verbete}

\begin{verbete}[谁]{shei2}
\significado{shei2}{interr.}{
    quem?
}
\significado{shui2}{interr.}{
    quem?
}
\end{verbete}

\begin{verbete}[身体]{shen1ti3}
\significado{shen1ti3}{n.}{
    corpo; saúde
}
\end{verbete}

\begin{verbete}[什么]{shen2me0}
\significado{shen2me0}{interr.}{
    que?; o que?
}
\end{verbete}

\begin{verbete}[什么时候]{shen2me0shi2hou0}
\significado{shen2me0shi2hou0}{interr.}{
    quando?; a que horas?
}
\end{verbete}

\begin{verbete}[生]{sheng1}
\significado{sheng1}{adj.}{
    cru; não cozido
}
\end{verbete}

\begin{verbete}[生日]{sheng1ri0}
\significado{sheng1ri0}{n.}{
    aniversário; dia de anos
}
\end{verbete}

\begin{verbete}[生意]{sheng1yi0}
\significado{sheng1yi0}{n.}{
    negócio
}
\end{verbete}

\begin{verbete}[生鱼片]{sheng1yu2pian4}
\significado{sheng1yu2pian4}{}{
    fatias de peixe cru, \textit{sashimi}
}
\end{verbete}

\begin{verbete}[省长]{Sheng4zhang3}
\significado{Sheng4zhang3}{n.}{
    Governador (país socialista)|
    presidente da província
}
\end{verbete}

\begin{verbete}[圣诞节]{Sheng4dan4jie2}
\significado{Sheng4dan4jie2}{n.}{
    Natal
}
\end{verbete}

\begin{verbete}[十]{shi2}
\significado{shi2}{num.}{
    10|
    dez, dezena
}
\end{verbete}

\begin{verbete}[时候]{shi2hou0}
\significado{shi2hou0}{n.}{
    horas; tempo
}
\significado{shi2hou0}{interr.}{
    quando
}
\end{verbete}

\begin{verbete}[时间]{shi2jian1}
\significado{shi2jian1}{n.}{
    tempo
}
\end{verbete}

\begin{verbete}[食品]{shi2pin3}
\significado{shi2pin3}{n.}{
    comida
}
\end{verbete}

\begin{verbete}[食堂]{shi2tang2}
\significado{shi2tang2}{n.}{
    cantina
}
\end{verbete}

\begin{verbete}[事]{shi4}
\significado{shi4}{n.}{
    coisa; assunto
}
\end{verbete}

\begin{verbete}[事儿]{shir4}
\significado{shir4}{n.}{
    afazeres; assunto; coisa; matéria|
    \pc{件}
}
\end{verbete}

\begin{verbete}[试]{shi4}
\significado{shi4}{v.}{
    experimentar; provar
}
\end{verbete}

\begin{verbete}[室]{shi4}
\significado{shi4}{n.}{
    quarto
}
\end{verbete}

\begin{verbete}[是]{shi4}
\significado{shi4}{v.}{
    ser
}
\end{verbete}

\begin{verbete}[是的]{shi4de}
\significado{shi4de}{adv.}{shi4de}{
    sim
}
\end{verbete}

\begin{verbete}[市区]{shi4qu1}
\significado{shi4qu1}{n.}{
    cidade própria; distrito urbano;
}
\end{verbete}

\begin{verbete}[市中心]{shi4zhong1xin1}
\significado{shi4zhong1xin1}{n.}{
    centro da cidade
}
\end{verbete}

\begin{verbete}[收到]{shou1dao4}
\significado{shou1dao4}{v.}{
    receber
}
\end{verbete}

\begin{verbete}[手]{shou3}
\significado{shou3}{n.}{
    mão
}
\end{verbete}

\begin{verbete}[手臂]{shou3bi4}
\significado{shou3bi4}{n.}{
    braço
}
\end{verbete}

\begin{verbete}[首相]{shou3xiang4}
\significado{shou3xiang4}{n.}{
    Primeiro-Ministro (país socialista)
}
\end{verbete}

\begin{verbete}[瘦]{shou4}
\significado{shou4}{adj.}{
    magro; emagrecido
}
\end{verbete}

\begin{verbete}[书]{shu1}
\significado{shu1}{n.}{
    livro|
    \pc{本}
}
\end{verbete}

\begin{verbete}[舒服]{shu1fu2}
\significado{shu1fu2}{adj.}{
    estar confortável; bem disposto; (sentir-se) bem
}
\end{verbete}

\begin{verbete}[熟悉]{shu2xi1}
\significado{shu2xi1}{v.}{
    conhecer bem
}
\end{verbete}

\begin{verbete}[属]{shu3}
\significado{shu3}{v.}{
    nascer no ano do signo de (um dos doze animais zodiacais)
}
\end{verbete}

\begin{verbete}[暑假]{shu3jia4}
\significado{shu3jia4}{n.}{
    férias de verão
}
\end{verbete}

\begin{verbete}[树]{shu4}
\significado{shu4}{n.}{
    árvore
}
\end{verbete}

\begin{verbete}[树木]{shu4mu4}
\significado{shu4mu4}{n.}{
    árvore
}
\end{verbete}

\begin{verbete}[谁]{shui2}
\significado{shui2}{interr.}{
    quem?
}
\significado{shei2}{interr.}{
    quem?
}
\end{verbete}

\begin{verbete}[水]{shui3}
\significado{shui3}{n.}{água}
\end{verbete}

\begin{verbete}[水果]{shui3guo3}
\significado{shui3guo3}{n.}{
    fruta
}
\end{verbete}

\begin{verbete}[水饺]{shui3jiao3}
\significado{shui3jiao3}{n.}{
    dumplings; raviólis chineses
}
\end{verbete}

\begin{verbete}[睡觉]{shui4jiao4}
\significado{shui4jiao4}{v.}{
    ir para a cama; dormir; deitar-se
}
\end{verbete}

\begin{verbete}[说]{shuo1}
\significado{shuo1}{v.}{
    falar; dizer
}
\end{verbete}

\begin{verbete}[说完]{shuo1-wan2}
\significado{shuo1-wan2}{}{
    acabar/terminar palavras
}
\end{verbete}

\begin{verbete}[帅]{shuai4}
\significado{shuai4}{adj.}{
    elegante; agradável à vista
}
\end{verbete}

\begin{verbete}[死]{si3}
\significado{si3}{v.}{
    morrer; falecer
}
\end{verbete}

\begin{verbete}[四]{si4}
\significado{si4}{num.}{
    4|
    quatro
}
\end{verbete}

\begin{verbete}[四川]{Si4chuan1}
\significado{Si4chuan1}{n.}{
    Sichuan
}
\end{verbete}

\begin{verbete}[四季如春]{si4ji4-ru2chun1}
\significado{si4ji4-ru2chun1}{}{
    é primavera todo o ano
}
\end{verbete}

\begin{verbete}[四季分明]{si4ji4-fen1ming2}
\significado{si4ji4-fen1ming2}{}{
    as quatro estações são muito distintas
}
\end{verbete}

\begin{verbete}[送]{song4}
\significado{song4}{v.}{
    distribuir; entregar; dar; oferecer (alguma coisa como presente)
}
\end{verbete}

\begin{verbete}[宿舍]{su4she4}
\significado{su4she4}{n.}{
    dormitório
}
\end{verbete}

\begin{verbete}[酸]{suan1}
\significado{suan1}{adj.}{
    ácido, ácida|
    avinagrado, avinagrada
}
\end{verbete}

\begin{verbete}[酸辣汤]{suan1la4tang1}
\significado{suan1la4tang1}{n.}{
    sopa avinagrada e picante
}
\end{verbete}

\begin{verbete}[算了]{suan4le0}
\significado{suan4le0}{v.}{
    deixar
}
\end{verbete}

\begin{verbete}[随便]{sui2bian4}
\significado{sui2bian4}{adj.}{
    à vontade; como queira
}
\end{verbete}

\begin{verbete}[岁]{sui4}
\significado{sui4}{n.}{
    anos de idade
}
\end{verbete}

\begin{verbete}[孙女]{sun1nur3}
\significado{sun1nur3}{n.}{
    filha do filho
}
\end{verbete}

\begin{verbete}[孙子]{sun1zi0}
\significado{sun1zi0}{n.}{
    filho do filho
}
\end{verbete}

\begin{verbete}[所以]{suo3yi3}
\significado{suo3yi3}{conj.}{
    por isso
}
\end{verbete}

\end{multicols}

%%%
%%% T
%%%
\section*{T}
\addcontentsline{toc}{section}{T}
\begin{multicols*}{2}

\begin{verbete}[T-恤]{T-xu4}
\significado{T-xu4}{n.}{
    camiseta; pulôver; suéter
}
\end{verbete}

\begin{verbete}[它]{ta1}
\significado{ta1}{pron.}{
    ele, ela|
    objetos e animais
}
\end{verbete}

\begin{verbete}[它们]{ta1men0}
\significado{ta1men0}{pron.}{
    eles, elas|
    objetos e animais
}
\end{verbete}

\begin{verbete}[他]{ta1}
\significado{ta1}{pron.}{
    ele
}
\end{verbete}

\begin{verbete}[他的]{ta1de0}
\significado{ta1de0}{pron.}{
    dele
}
\end{verbete}

\begin{verbete}[他们]{ta1men0}
\significado{ta1men0}{pron.}{
    eles
}
\end{verbete}

\begin{verbete}[他们的]{ta1men0de0}
\significado{ta1men0de0}{pron.}{
    deles
}
\end{verbete}

\begin{verbete}[她]{ta1}
\significado{ta1}{pron.}{
    ela
}
\end{verbete}

\begin{verbete}[她的]{ta1de0}
\significado{ta1de0}{pron.}{
    dela
}
\end{verbete}

\begin{verbete}[她们]{ta1men0}
\significado{ta1men0}{pron.}{
    elas
}
\end{verbete}

\begin{verbete}[她们的]{ta1men0de0}
\significado{ta1men0de0}{pron.}{
    delas
}
\end{verbete}

\begin{verbete}[特别]{te4bie2}
\significado{te4bie2}{adv.}{
    especialmente
}
\significado{te4bie2}{adj.}{
    especial
}
\end{verbete}

\begin{verbete}[台]{tai2}
\significado{tai2}{p.c.}{
    para computador, gravador, gravador de vídeo, etc
}
\end{verbete}

\begin{verbete}[太]{tai4}
\significado{tai4}{adv.}{
    excessivamente; demais; muito
}
\end{verbete}

\begin{verbete}[太太]{tai4tai0}
\significado{tai4tai0}{n.}{
    esposa; mulher
}
\end{verbete}

\begin{verbete}[太阳]{tai4yang2}
\significado{tai4yang2}{n.}{
    sol
}
\end{verbete}

\begin{verbete}[汤]{tang1}
\significado{tang1}{n.}{
    sopa; caldo
}
\end{verbete}

\begin{verbete}[糖]{tang2}
\significado{tang2}{n.}{
    açúcar
}
\end{verbete}

\begin{verbete}[糖醋鱼]{tang2cu4yu2}
\significado{tang2cu4yu2}{n.}{    
    peixe guisado em molho agridoce
}
\end{verbete}

\begin{verbete}[特别]{te4bie2}
\significado{te4bie2}{adv.}{
    especialmente
}
\end{verbete}

\begin{verbete}[疼]{teng2}
\significado{teng2}{adj.}{
    doloroso; doído
}
\significado{teng2}{v.}{doer}
\end{verbete}

\begin{verbete}[踢]{ti1}
\significado{ti1}{v.}{
    jogar; dar pontapés em
}
\end{verbete}

\begin{verbete}[天]{tian1}
\significado{tian1}{n.}{
    dia
}
\end{verbete}

\begin{verbete}[天气]{tian1qi4}
\significado{tian1qi4}{n.}{
    clima; tempo
}
\end{verbete}

\begin{verbete}[甜]{tian2}
\significado{tian2}{adj.}{
    doce
}
\end{verbete}

\begin{verbete}[条]{tiao2}
\significado{tiao2}{p.c.}{
    para calças, saia, rio, peixe, etc
}
\end{verbete}

\begin{verbete}[跳]{tiao4}
\significado{tiao4}{v.}{
    dançar; saltar
}
\end{verbete}

\begin{verbete}[跳舞]{tiao4wu3}
\significado{tiao4wu3}{v.+compl.}{
    dançar
}
\end{verbete}

\begin{verbete}[听]{ting1}
\significado{ting1}{v.}{
    ouvir; escutar
}
\end{verbete}

\begin{verbete}[听说]{ting1shuo1}
\significado{ting1shuo1}{v.}{
    ouvir dizer
}
\end{verbete}

\begin{verbete}[停车场]{ting2che1chang3}
\significado{ting2che1chang3}{n.}{
    parque de estacionamento
}
\end{verbete}

\begin{verbete}[同]{tong2}
\significado{tong2}{adj.}{
    junto
}
\significado{tong2}{adv.}{
    junto com
}
\end{verbete}

\begin{verbete}[同学]{tong2xue2}
\significado{tong2xue2}{n.}{
    aluno; aluna; colega de classe
}
\end{verbete}

\begin{verbete}[头]{tou2}
\significado{tou2}{n.}{
    cabeça
}
\end{verbete}

\begin{verbete}[头发]{tou2fa0}
\significado{tou2fa0}{n.}{
    cabelo
}
\end{verbete}

\begin{verbete}[图书馆]{tu2shu1guan3}
\significado{tu2shu1guan3}{n.}{
    biblioteca
}
\end{verbete}

\begin{verbete}[腿]{tui3}
\significado{tui3}{n.}{
    perna
}
\end{verbete}

\end{multicols*}

%%%%
%%% U
%%%
%\section*{U}
%\addcontentsline{toc}{section}{U}
%\begin{multicols*}{2}
%\end{multicols*}
 %%%%% Não tem palavras com pinyin iniciado em "U" em chinês
%%%%
%%% V
%%%
%\section*{V}
%\addcontentsline{toc}{section}{V}
%\begin{multicols}{2}
%\end{multicols}
 %%%%% Não tem palavras com pinyin iniciado em "V" em chinês
%%%
%%% W
%%%
\section*{W}
\addcontentsline{toc}{section}{W}

\begin{verbete}[wai4bian0]{外边}[5;5]
\begin{pronuncia}{wai4bian0}
\significado{p.l.}{ fora; por fora; exterior }
\end{pronuncia}
\end{verbete}

\begin{verbete}[wai4gong1]{外公}[5;4]
\begin{pronuncia}{wai4gong1}
\significado{s.}{ avô materno }
\end{pronuncia}
\end{verbete}

\begin{verbete}[wai4guo2]{外国}[5;8]
\begin{pronuncia}{wai4guo2}
\significado[个]{s.}{ país estrangeiro }
\end{pronuncia}
\end{verbete}

\begin{verbete}[wai4hao4]{外号}[5;5]
\begin{pronuncia}{wai4hao4}
\significado{s.}{ apelido }
\end{pronuncia}
\end{verbete}

\begin{verbete}[wai4mao4]{外贸}[5;9]
\begin{pronuncia}{wai4mao4}
\significado{s.}{ comércio exterior }
\end{pronuncia}
\end{verbete}

\begin{verbete}[wai4mian0]{外面}[5;9]
\begin{pronuncia}{wai4mian0}
\significado{p.l.}{ fora; por fora; exterior }
\end{pronuncia}
\end{verbete}

\begin{verbete}[wai4po2]{外婆}[5;11]
\begin{pronuncia}{wai4po2}
\significado{s.}{ avó materna }
\end{pronuncia}
\end{verbete}

\begin{verbete}[wai4shi4]{外事}[5;8]
\begin{pronuncia}{wai4shi4}
\significado{s.}{ assuntos ou relações exteriores }
\end{pronuncia}
\end{verbete}

\begin{verbete}[wai4sun1]{外孙}[5;6]
\begin{pronuncia}{wai4sun1}
\significado{s.}{ filho da filha }
\end{pronuncia}
\end{verbete}

\begin{verbete}[wai4sun1nv3]{外孙女}[5;6;3]
\begin{pronuncia}{wai4sun1nv3}
\significado{s.}{ filha da filha }
\end{pronuncia}
\end{verbete}

\begin{verbete}[wai4yu3]{外语}[5;9]
\begin{pronuncia}{wai4yu3}
\significado[门]{s.}{ língua estrangeira }
\end{pronuncia}
\end{verbete}

\begin{verbete}[wan1dou4]{豌豆}[15;7]
\begin{pronuncia}{wan1dou4}
\significado{s.}{ ervilha }
\end{pronuncia}
\end{verbete}

\begin{verbete}[wan2]{完}[7]
\begin{pronuncia}{wan2}
\significado{v.}{ acabar; terminar }
\end{pronuncia}
\end{verbete}

\begin{verbete}[wan2]{玩}[8]
\begin{pronuncia}{wan2}
\significado{v.}{ brincar; tocar (intrumento musical) }
\end{pronuncia}
\end{verbete}

\begin{verbete}[wanr2]{玩儿}[8;2]
\begin{pronuncia}{wanr2}
\significado{v.}{ divertir-se }
\end{pronuncia}
\end{verbete}

\begin{verbete}[wan3]{晚}[11]
\begin{pronuncia}{wan3}
\significado{adj.}{ tarde }
\end{pronuncia}
\end{verbete}

\begin{verbete}[wan3fan4]{晚饭}[11;7]
\begin{pronuncia}{wan3fan4}
\significado[份,顿,次,餐]{s.}{ jantar }
\end{pronuncia}
\end{verbete}

\begin{verbete}[wan3shang0]{晚上}[11;3]
\begin{pronuncia}{wan3shang0}
\significado{p.t.}{ noite; à noite }
\end{pronuncia}
\end{verbete}

\begin{verbete}[wan3]{碗}[13]
\begin{pronuncia}{wan3}
\significado[只,个]{n}{ tigela }
\significado{p.c.}{ tigelas }
\end{pronuncia}
\end{verbete}

\begin{verbete}[wan3zi0]{碗子}[13;3]
\begin{pronuncia}{wan3zi0}
\significado{n}{ tigela }
\end{pronuncia}
\end{verbete}

\begin{verbete}[wan4]{万}[3]
\begin{pronuncia}{wan4}
\significado{num.}{ dez mil; 10.000 }
\end{pronuncia}
\end{verbete}

\begin{verbete}[wang3]{往}[8]
\begin{pronuncia}{wang3}
\significado{prep.}{ para; em direção a }
\end{pronuncia}
\end{verbete}

\begin{verbete}[wang3qiu2]{网球}[6;11]
\begin{pronuncia}{wang3qiu2}
\significado[个]{s.}{ tênis (esporte); bola de tênis }
\end{pronuncia}
\end{verbete}

\begin{verbete}[wang4]{忘}[7]
\begin{pronuncia}{wang4}
\significado{v.}{ esquecer }
\end{pronuncia}
\end{verbete}

\begin{verbete}[wen1du4]{温度}[12;9]
\begin{pronuncia}{wen1du4}
\significado[个]{s.}{ temperatura }
\end{pronuncia}
\end{verbete}

\begin{verbete}[wei2]{喂}[12]
\begin{pronuncia}{wei2}
\significado{interj.}{ ei!; chamar atenção (alô, telefone) }
\end{pronuncia}
\begin{pronuncia}{wei4}
\significado*{}{ 喂\p{wei4} }
\end{pronuncia}
\end{verbete}

\begin{verbete}[wei4sheng1jian1]{卫生间}[3;5;7]
\begin{pronuncia}{wei4sheng1jian1}
\significado[间]{s.}{ banheiro; toilette }
\end{pronuncia}
\end{verbete}

\begin{verbete}[wei4]{为}[4]
\begin{pronuncia}{wei4}
\significado{prep.}{ para }
\end{pronuncia}
\end{verbete}

\begin{verbete}[wei4shen2me0]{为什么}[4;4;3]
\begin{pronuncia}{wei4shen2me0}
\significado{interr.}{ por que? }
\end{pronuncia}
\end{verbete}

\begin{verbete}[wei4]{位}[7]
\begin{pronuncia}{wei4}
\significado{p.c.}{ para pessoas (com cortesia) }
\end{pronuncia}
\end{verbete}

\begin{verbete}[wei4dao0]{味道}[8;12]
\begin{pronuncia}{wei4dao0}
\significado{s.}{ sabor }
\end{pronuncia}
\end{verbete}

\begin{verbete}[wei4]{喂}
\begin{pronuncia}{wei4}
\significado{interj.}{ ei!; chamar atenção (alô, telefone) }
\end{pronuncia}
\begin{pronuncia}{wei2}
\significado*{}{ 喂\p{wei2} }
\end{pronuncia}
\end{verbete}

\begin{verbete}[wen2hua4]{文化}[4;4]
\begin{pronuncia}{wen2hua4}
\significado[个,种]{s.}{ cultura; civilização }
\end{pronuncia}
\end{verbete}

\begin{verbete}[Wen2xue2xi4]{文学系}[4;8;7]
\begin{pronuncia}{Wen2xue2xi4}
\significado{s.}{ Faculdade de Letras }
\end{pronuncia}
\end{verbete}

\begin{verbete}[wen4]{问}[6]
\begin{pronuncia}{wen4}
\significado{v.}{ perguntar }
\end{pronuncia}
\end{verbete}

\begin{verbete}[wen4ti2]{问题}[6;15]
\begin{pronuncia}{wen4ti2}
\significado[个]{s.}{ pergunta; questão; problema }
\end{pronuncia}
\end{verbete}

\begin{verbete}[wo3]{我}[7]
\begin{pronuncia}{wo3}
\significado{pron.}{ eu }
\end{pronuncia}
\end{verbete}

\begin{verbete}[wo3de0]{我的}[7;8]
\begin{pronuncia}{wo3de0}
\significado{pron.}{ meu, meus }
\end{pronuncia}
\end{verbete}

\begin{verbete}[wo3men0]{我们}[7;5]
\begin{pronuncia}{wo3men0}
\significado{pron.}{ nós }
\end{pronuncia}
\end{verbete}

\begin{verbete}[wo3men0de0]{我们的}[7;5;8]
\begin{pronuncia}{wo3men0de0}
\significado{pron.}{ nosso, nossos }
\end{pronuncia}
\end{verbete}

\begin{verbete}[wo3shi4]{卧室}[8;9]
\begin{pronuncia}{wo3shi4}
\significado{s.}{ quarto de dormir }
\end{pronuncia}
\end{verbete}

\begin{verbete}[wu3fan4]{午饭}[4;7]
\begin{pronuncia}{wu3fan4}
\significado[份,顿,次,餐]{s.}{ almoço }
\end{pronuncia}
\end{verbete}

\begin{verbete}[wu3]{五}[4]
\begin{pronuncia}{wu3}
\significado{num.}{ cinco; 5 }
\end{pronuncia}
\end{verbete}

\begin{verbete}[wu3]{舞}[14]
\begin{pronuncia}{wu3}
\significado{s.}{ dança }
\end{pronuncia}
\end{verbete}

%%%%% EOF %%%%%

%%%
%%% X
%%%
\section*{X}
\addcontentsline{toc}{section}{X}
\begin{multicols*}{2}

\begin{verbete}[西]{xi1}
\significado{xi1}{n.}{
    oeste
}
\end{verbete}

\begin{verbete}[西安]{xi1'an1}
\significado{xi1'an1}{n.}{
    Xi'an
}
\end{verbete}

\begin{verbete}[西半球]{xi1ban4qiu2}
\significado{xi1ban4qiu2}{n.}{
    hemisfério oeste
}
\end{verbete}

\begin{verbete}[西边]{xi1bian0}
\significado{xi1bian0}{p.l.}{
    oeste; lado oeste; ocidente
}
\end{verbete}

\begin{verbete}[西部]{xi1bu4}
\significado{xi1bu4}{p.l.}{
    oeste; ocidente
}
\end{verbete}

\begin{verbete}[西方]{xi1fang1}
\significado{xi1fang1}{p.l.}{
    oeste; ocidente
}
\end{verbete}

\begin{verbete}[西面]{xi1mian4}
\significado{xi1mian4}{p.l.}{
    oeste; lado oeste; ocidente
}
\end{verbete}

\begin{verbete}[西语]{xi1yu3}
\significado{xi1yu3}{n.}{
    espanhol, língua espanhola
}
\end{verbete}

\begin{verbete}[西文]{xi1wen2}
\significado{xi1wen2}{n.}{
    espanhol, língua espanhola
}
\end{verbete}

\begin{verbete}[悉尼]{Xi1ni2}
\significado{Xi1ni2}{n.}{
    Sidney
}
\end{verbete}

\begin{verbete}[喜欢]{xi3huan0}
\significado{xi3huan0}{v.}{
    gostar
}
\end{verbete}

\begin{verbete}[洗手间]{xi3shou3jian1}
\significado{xi3shou3jian1}{n.}{
    sanitário; toilette
}
\end{verbete}

\begin{verbete}[系]{xi4}
\significado{xi4}{n.}{
    faculdade (da universidade)
}
\end{verbete}

\begin{verbete}[下]{xia4}
\significado{xia4}{p.l.}{
    abaixo; em baixo de
}
\significado{xia4}{v.d.}{
    descer
}
\end{verbete}

\begin{verbete}[下巴]{xia4ba0}
\significado{xia4ba0}{n.}{
    queixo
}
\end{verbete}

\begin{verbete}[下边]{xia4bian0}
\significado{xia4bian0}{p.l.}{
    em baixo; abaixo; parte de baixo
}
\end{verbete}

\begin{verbete}[下车]{xia4che1}
\significado{xia4che1}{n.}{
    descer; sair (de ônibus)
}
\end{verbete}

\begin{verbete}[下课]{xia4ke4}
\significado{xia4ke4}{v.+compl.}{
    acabar a aula; terminar a aula
}
\end{verbete}

\begin{verbete}[下面]{xia4mian0}
\significado{xia4mian0}{p.l.}{
    em baixo; abaixo; parte de baixo
}
\end{verbete}

\begin{verbete}[下午]{xia4wu3}
\significado{xia4wu3}{p.t.}{
    tarde; à tarde; período logo após o meio-dia
}
\end{verbete}

\begin{verbete}[下旬]{xia4xun2}
\significado{xia4xun2}{p.t.}{
    última dezena do mês
}
\end{verbete}

\begin{verbete}[下雨]{xia4yu3}
\significado{xia4yu3}{v.+compl.}{
    chover
}
\end{verbete}

\begin{verbete}[夏天]{xia4tian1}
\significado{xia4tian1}{n.}{
    verão
}
\significado{xia4tian1}{p.t.}{
    verão
}
\end{verbete}

\begin{verbete}[先]{xian1}
\significado{xian1}{adv.}{
    em primeiro lugar; primeiramente
}
\end{verbete}

\begin{verbete}[先生]{xian1sheng0}
\significado{xian1sheng0}{n.}{
    senhor; marido
}
\end{verbete}

\begin{verbete}[咸]{xian2}
\significado{xian2}{adj.}{
    salgado; salgada
}
\end{verbete}

\begin{verbete}[现在]{xian4zai4}
\significado{xian4zai4}{p.t.}{
    agora
}
\end{verbete}

\begin{verbete}[香港]{xiang1gang3}
\significado{xiang1gang3}{n.}{
    Hong Kong
}
\end{verbete}

\begin{verbete}[想]{xiang3}
\significado{xiang3}{v./v.o.}{
    pensar; querer; achar
}
\end{verbete}

\begin{verbete}[向]{xiang4}
\significado{xiang4}{prep.}{
    para
}
\end{verbete}

\begin{verbete}[向汪]{xiang4wang3}
\significado{xiang4wang3}{v.}{
    esperar que
}
\end{verbete}

\begin{verbete}[小]{xiao3}
\significado{xiao3}{adj.}{
    pequeno, pequena
}
\end{verbete}

\begin{verbete}[小姐]{xiao3jie0}
\significado{xiao3jie0}{n.}{
    senhorita|
    empregada
}
\end{verbete}

\begin{verbete}[小时]{xiao3shi2}
\significado{xiao3shi2}{p.c.}{
    hora; para horas
}
\end{verbete}

\begin{verbete}[小腿]{xiao3tui3}
\significado{xiao3tui3}{n.}{
    perna
}
\end{verbete}

\begin{verbete}[小学]{xiao3xue2}
\significado{xiao3xue2}{n.}{
    escola ensino fundamental
}
\end{verbete}

\begin{verbete}[校长]{xiao4zhang3}
\significado{xiao4zhang3}{n.}{
    diretor de escola|
    reitor (universidade)
}
\end{verbete}

\begin{verbete}[些]{xie1}
\significado{xie1}{adv.}{
    uns, umas|
    alguns, algumas
}
\end{verbete}

\begin{verbete}[写]{xie3}
\significado{xie3}{v.}{
    escrever
}
\end{verbete}

\begin{verbete}[谢谢]{xie4xie0}
\significado{xie4xie0}{v.}{
    agradecer|
    obrigado, obrigada
}
\end{verbete}

\begin{verbete}[谢天谢地]{xie4tian1xie4di4}
\significado{xie4tian1xie4di4}{}{
    agradecer a Deus; agradecer aos céus
}
\end{verbete}

\begin{verbete}[新]{xin1}
\significado{xin1}{adj.}{
    novo
}
\end{verbete}

\begin{verbete}[新年]{xin1nian2}
\significado{xin1nian2}{n.}{
    Ano Novo
}
\end{verbete}

\begin{verbete}[新鲜]{xin1xian1}
\significado{xin1xian1}{adj.}{
    fresco, fresca
}
\end{verbete}

\begin{verbete}[信]{xin4}
\significado{xin4}{n.}{
    carta
}
\end{verbete}

\begin{verbete}[星期]{xing1qi1}
\significado{xing1qi1}{n.}{
    semana
}
\end{verbete}

\begin{verbete}[星期一]{xing1qi1yi4}
\significado{xing1qi1yi4}{n.}{
    segunda-feira
}
\end{verbete}

\begin{verbete}[星期二]{xing1qi1}
\significado{xing1qi1}{n.}{
    terça-feira
}
\end{verbete}

\begin{verbete}[星期三]{xing1qi1san1}
\significado{xing1qi1san1}{n.}{
    quarta-feira
}
\end{verbete}

\begin{verbete}[星期四]{xing1qi1si4}
\significado{xing1qi1si4}{n.}{
    quinta-feira
}
\end{verbete}

\begin{verbete}[星期五]{xing1qi1wu3}
\significado{xing1qi1wu3}{n.}{
    sexta-feira
}
\end{verbete}

\begin{verbete}[星期六]{xing1qi1liu4}
\significado{xing1qi1liu4}{n.}{
    sábado
}
\end{verbete}

\begin{verbete}[星期天]{xing1qi1tian1}
\significado{xing1qi1tian1}{n.}{
    domingo
}
\end{verbete}

\begin{verbete}[星期日]{xing1qi1ri4}
\significado{xing1qi1ri4}{n.}{
    domingo
}
\end{verbete}

\begin{verbete}[星星]{xing1xing0}
\significado{xing1xing0}{n.}{
    estrela
}
\end{verbete}

\begin{verbete}[行]{xing2}
\significado{xing2}{v.}{
    claro que sim; de acordo; está bem
}
\end{verbete}

\begin{verbete}[行人]{xing2ren2}
\significado{xing2ren2}{n.}{
    transeunte|
    peão
}
\end{verbete}

\begin{verbete}[姓]{xing4}
\significado{xing4}{n.}{
    sobrenome
}
\significado{xing4}{v.}{
    ter o sobrenome
}
\end{verbete}

\begin{verbete}[兴趣]{xing4qu4}
\significado{xing4qu4}{n.}{
    interesse
}
\end{verbete}

\begin{verbete}[胸]{xiong1}
\significado{xiong1}{n.}{
    peito
}
\end{verbete}

\begin{verbete}[休息]{xiu1xi0}
\significado{xiu1xi0}{v.}{
    descansar
}
\end{verbete}

\begin{verbete}[需要]{xu1yao4}
\significado{xu1yao4}{n.}{
    necessidade
}
\significado{xu1yao4}{v.}{
    precisar; necessitar
}
\end{verbete}

\begin{verbete}[学]{xue2}
\significado{xue2}{v.}{
    estudar
}
\end{verbete}

\begin{verbete}[学生]{xue2sheng0}
\significado{xue2sheng0}{n.}{
    estudante|
    aluno, aluna
}
\end{verbete}

\begin{verbete}[学生证]{xue2sheng0zheng4}
\significado{xue2sheng0zheng4}{n.}{
    cartão de estudante
}
\end{verbete}

\begin{verbete}[学习]{xue2xi2}
\significado{xue2xi2}{v.}{
    estudar; aprender
}
\end{verbete}

\begin{verbete}[学校]{xue2xiao4}
\significado{xue2xiao4}{n.}{
    escola; instituição de ensino
}
\end{verbete}

\begin{verbete}[学院]{xue2yuan4}
\significado{xue2yuan4}{n.}{
    instituto
}
\end{verbete}

\begin{verbete}[雪]{xue3}
\significado{xue3}{n.}{
    neve
}
\end{verbete}

\end{multicols*}

%%%
%%% Y
%%%
\section*{Y}
\addcontentsline{toc}{section}{Y}
%\begin{multicols*}{2}

\begin{verbete}[ya1sui4qian2]{压岁钱}
\begin{pronuncia}{ya1sui4qian2}
\significado{n.}{
dinheiro da sorte|
dinheiro dado às crianças como presente no Ano Novo Chinês
}
\end{pronuncia}
\end{verbete}

\begin{verbete}[ya1]{鸭}
\begin{pronuncia}{ya1}
\significado[只]{n.}{
pato|
gíria: prostituto
}
\end{pronuncia}
\end{verbete}

\begin{verbete}[ya2]{牙}
\begin{pronuncia}{ya2}
\significado[颗]{n.}{
dente; marfim
}
\end{pronuncia}
\end{verbete}

\begin{verbete}[ya2chi3]{牙齿}
\begin{pronuncia}{ya2chi3}
\significado{adv.}{
dental
}
\significado[颗]{n.}{
dente
}
\end{pronuncia}
\end{verbete}

\begin{verbete}[Ya4zhou1]{亚洲}
\begin{pronuncia}{Ya4zhou1}
\significado{n.}{
Ásia
}
\end{pronuncia}
\end{verbete}

\begin{verbete}[yan2se4]{颜色}
\begin{pronuncia}{yan2se4}
\significado{n.}{
cor; pigmento; tintura
}
\end{pronuncia}
\end{verbete}

\begin{verbete}[yan3jing4]{眼镜}
\begin{pronuncia}{yan3jing4}
\significado[副]{n.}{
óculos
}
\end{pronuncia}
\end{verbete}

\begin{verbete}[yan3jing0]{眼睛}
\begin{pronuncia}{yan3jing0}
\significado[只,双]{n.}{
olho(s)
}
\end{pronuncia}
\end{verbete}

\begin{verbete}[yang3]{养}
\begin{pronuncia}{yang3}
\significado{v.}{
criar (animais ou filhos), plantar (flores), etc
}
\end{pronuncia}
\end{verbete}

\begin{verbete}[yang4zi0]{样子}
\begin{pronuncia}{yang4zi0}
\significado{n.}{
aparência; forma; modelo
}
\end{pronuncia}
\end{verbete}

\begin{verbete}[yao1]{腰}
\begin{pronuncia}{yao1}
\significado{n.}{
cintura
}
\end{pronuncia}
\end{verbete}

\begin{verbete}[yao4]{药}
\begin{pronuncia}{yao4}
\significado[种,服,味]{n.}{
medicamento; remédio; droga
}
\end{pronuncia}
\end{verbete}

\begin{verbete}[yao4]{要}
\begin{pronuncia}{yao4}
\significado{v./v.o.}{
querer; precisar
}
\end{pronuncia}
\end{verbete}

\begin{verbete}[yao4shi0]{要是}
\begin{pronuncia}{yao4shi0}
\significado{conj.}{
se
}
\end{pronuncia}
\end{verbete}

\begin{verbete}[yao4shi0 ...\  de0hua0]{要是······的话}
\begin{pronuncia}[\\]{yao4shi0 ...\  de0hua0}
\significado{conj.}{
se ... no caso de
}
\end{pronuncia}
\end{verbete}

\begin{verbete}[ye2ye0]{爷爷}
\begin{pronuncia}{ye2ye0}
\significado[个]{n.}{
avô (paterno)
}
\end{pronuncia}
\end{verbete}

\begin{verbete}[ye3]{也}
\begin{pronuncia}{ye3}
\significado{adv.}{
também
}
\end{pronuncia}
\end{verbete}

\begin{verbete}[ye4li0]{夜里}
\begin{pronuncia}{ye4li0}
\significado{p.t.}{
noite
}
\end{pronuncia}
\end{verbete}

\begin{verbete}[yi1]{一}
\begin{pronuncia}{yi1}[(quando usado sozinho)]
\significado{num.}{
um, uma; 1
}
\end{pronuncia}
\begin{pronuncia}{yi2}[(antes de quarto tom)]
\significado{num.}{
um, uma; 1|
um, uma (artigo)
}
\end{pronuncia}
\begin{pronuncia}{yi4}
\significado{num.}{
um, uma; 1|
um, uma (artigo)
}
\end{pronuncia}
\end{verbete}

\begin{verbete}[yi2]{一}
\begin{pronuncia}{yi2}[(antes de quarto tom)]
\significado{num.}{
um, uma; 1|
um, uma (artigo)
}
\end{pronuncia}
\begin{pronuncia}{yi1}[(quando usado sozinho)]
\significado{num.}{
um, uma; 1
}
\end{pronuncia}
\begin{pronuncia}{yi4}
\significado{num.}{
um, uma; 1|
um, uma (artigo)
}
\end{pronuncia}
\end{verbete}

\begin{verbete}[yi2ban4]{一半}
\begin{pronuncia}{yi2ban4}
\significado{adj.}{
metade
}
\end{pronuncia}
\end{verbete}

\begin{verbete}[yi2ding4]{一定}
\begin{pronuncia}{yi2ding4}
\significado{adv.}{
certamente; definitivamente
}
\end{pronuncia}
\end{verbete}

\begin{verbete}[yi2gong4]{一共}
\begin{pronuncia}{yi2gong4}
\significado{adv.}{
tudo; no local
}
\end{pronuncia}
\end{verbete}

\begin{verbete}[yi2xia4]{一下}
\begin{pronuncia}{yi2xia4}
\significado{adv.}{
em um curto tempo; rapidamente
}
\end{pronuncia}
\end{verbete}

\begin{verbete}[yi2yang4]{一样}
\begin{pronuncia}{yi2yang4}
\significado{adj.}{
igual; mesmo, mesma
}
\end{pronuncia}
\end{verbete}

\begin{verbete}[yi4]{一}
\begin{pronuncia}{yi4}
\significado{num.}{
um, uma; 1|
um, uma (artigo)
}
\end{pronuncia}
\begin{pronuncia}{yi2}[(antes de quarto tom)]
\significado{num.}{
um, uma; 1|
um, uma (artigo)
}
\end{pronuncia}
\begin{pronuncia}{yi1}[(quando usado sozinho)]
\significado{num.}{
um, uma; 1
}
\end{pronuncia}
\end{verbete}

\begin{verbete}[yi4ban1]{一般}
\begin{pronuncia}{yi4ban1}
\significado{adj.}{
geral; comum; normal
}
\significado{adv.}{
normalmente
}
\end{pronuncia}
\end{verbete}

\begin{verbete}[yi4dianr3]{一点儿}
\begin{pronuncia}{yi4dianr3}
\significado{adv.}{
um pouco (``adj.+一点儿'' ou ``一点儿+n.'')
}
\end{pronuncia}
\end{verbete}

\begin{verbete}[yi4huir4]{一会儿}
\begin{pronuncia}{yi4huir4}
\significado{adv.}{
daqui a pouco tempo; pouco tempo
}
\end{pronuncia}
\end{verbete}

\begin{verbete}[yi4qi3]{一起}
\begin{pronuncia}{yi4qi3}
\significado{adv.}{
juntamente; em conjunto
}
\end{pronuncia}
\end{verbete}

\begin{verbete}[yi4zhi2]{一直}
\begin{pronuncia}{yi4zhi2}
\significado{adv.}{
diretamente; sempre
}
\end{pronuncia}
\end{verbete}

\begin{verbete}[yi4xie1]{一些}
\begin{pronuncia}{yi4xie1}
\significado{pron.}{
uns, umas; alguns, algumas
}
\end{pronuncia}
\end{verbete}

\begin{verbete}[yi1fu0]{衣服}
\begin{pronuncia}{yi1fu0}
\significado[件,套]{n.}{
roupa, vestuário
}
\end{pronuncia}
\end{verbete}

\begin{verbete}[yi1sheng1]{医生}
\begin{pronuncia}{yi1sheng1}
\significado[个,位,名]{n.}{
médico; clínico
}
\end{pronuncia}
\end{verbete}

\begin{verbete}[yi1yuan0]{医院}
\begin{pronuncia}{yi1yuan0}
\significado[所,家,座]{n.}{
hospital
}
\end{pronuncia}
\end{verbete}

\begin{verbete}[yi2he2yuan2]{颐和园}
\begin{pronuncia}{yi2he2yuan2}
\significado{n.}{
Palácio de Verão
}
\end{pronuncia}
\end{verbete}

\begin{verbete}[yi2han4]{遗憾}
\begin{pronuncia}{yi2han4}
\significado{v.}{
ter pena de
}
\end{pronuncia}
\end{verbete}

\begin{verbete}[yi3jing1]{已经}
\begin{pronuncia}{yi3jing1}
\significado{adv.}{
já
}
\end{pronuncia}
\end{verbete}

\begin{verbete}[yi3hou4]{以后}
\begin{pronuncia}{yi3hou4}
\significado{p.t.}{
depois de; depois; após
}
\end{pronuncia}
\end{verbete}

\begin{verbete}[yi3qian2]{以前}
\begin{pronuncia}{yi3qian2}
\significado{p.t.}{
antes de; antes
}
\end{pronuncia}
\end{verbete}

\begin{verbete}[yi4]{亿}
\begin{pronuncia}{yi4}
\significado{num.}{
cem milhões; 100.000.000
}
\end{pronuncia}
\end{verbete}

\begin{verbete}[yi4si0]{意思}
\begin{pronuncia}{yi4si0}
\significado[个]{n.}{
interesse
}
\end{pronuncia}
\end{verbete}

\begin{verbete}[yin1tian1]{阴天}
\begin{pronuncia}{yin1tian1}
\significado{adj.}{
céu muito nublado; céu cinzento
}
\end{pronuncia}
\end{verbete}

\begin{verbete}[yin1wei4]{因为}
\begin{pronuncia}{yin1wei4}
\significado{conj.}{
porque
}
\end{pronuncia}
\end{verbete}

\begin{verbete}[yin1yue4音乐]{音乐}
\begin{pronuncia}{yin1yue4}
\significado[张,曲,段]{n.}{
música
}
\end{pronuncia}
\end{verbete}

\begin{verbete}[yin2hang2]{银行}
\begin{pronuncia}{yin2hang2}
\significado[家,个]{n.}{
banco; agência bancária
}
\end{pronuncia}
\end{verbete}

\begin{verbete}[yin3liao4]{饮料}
\begin{pronuncia}{yin3liao4}
\significado{n.}{
bebida
}
\end{pronuncia}
\end{verbete}

\begin{verbete}[ying1gai1]{应该}
\begin{pronuncia}{ying1gai1}
\significado{v.}{
dever; ter de
}
\end{pronuncia}
\end{verbete}

\begin{verbete}[Ying1guo2]{英国}
\begin{pronuncia}{Ying1guo2}
\significado{n.}{
Reino Unido
}
\end{pronuncia}
\end{verbete}

\begin{verbete}[ying1yu3]{英语}
\begin{pronuncia}{ying1yu3}
\significado{n.}{
inglês, língua inglesa
}
\end{pronuncia}
\end{verbete}

\begin{verbete}[ying1wen2]{英文}
\begin{pronuncia}{ying1wen2}
\significado{n.}{
inglês, língua inglesa
}
\end{pronuncia}
\end{verbete}

\begin{verbete}[you1mei3]{优美}
\begin{pronuncia}{you1mei3}
\significado{adj.}{
gracioso; fino; elegante
}
\end{pronuncia}
\end{verbete}

\begin{verbete}[you2jian4]{邮件}
\begin{pronuncia}{you2jian4}
\significado{n.}{
correspondência; \emph{email}
}
\end{pronuncia}
\end{verbete}

\begin{verbete}[you2ju4]{邮局}
\begin{pronuncia}{you2ju4}
\significado[家,个]{n.}{
correio; agência dos correios
}
\end{pronuncia}
\end{verbete}

\begin{verbete}[you2]{游}
\begin{pronuncia}{you2}
\significado{v.}{
nadar
}
\end{pronuncia}
\end{verbete}

\begin{verbete}[you2yong3]{游泳}
\begin{pronuncia}{you2yong3}
\significado{v.+compl.}{
nadar
}
\end{pronuncia}
\end{verbete}

\begin{verbete}[you2yong3chi2]{游泳池}
\begin{pronuncia}{you2yong3chi2}
\significado[场]{n.}{
piscina
}
\end{pronuncia}
\end{verbete}

\begin{verbete}[you3]{有}
\begin{pronuncia}{you3}
\significado{v.}{
ter; haver
}
\end{pronuncia}
\end{verbete}

\begin{verbete}[you3de0]{有的}
\begin{pronuncia}{you3de0}
\significado{pron.}{
algum, alguma, alguns, algumas
}
\end{pronuncia}
\end{verbete}

\begin{verbete}[you3de0\ shi2hou0]{有的时候}
\begin{pronuncia}{you3de0\ shi2hou0}
\significado{expr.}{
às vezes;
de vez em quando;
de quando em quando
}
\end{pronuncia}
\end{verbete}

\begin{verbete}[you3dianr3]{有点儿}
\begin{pronuncia}{you3dianr3}
\significado{adv.}{
um pouco (``有点儿+n. ou v. mental'')
}
\end{pronuncia}
\end{verbete}

\begin{verbete}[you3ming2]{有名}
\begin{pronuncia}{you3ming2}
\significado{adj.}{
famoso, famosa
}
\end{pronuncia}
\end{verbete}

\begin{verbete}[you3shi2]{有时}
\begin{pronuncia}{you3shi2}
\significado{expr.}{
às vezes;
de vez em quando;
de quando em quando
}
\end{pronuncia}
\end{verbete}

\begin{verbete}[you3shi2hou0]{有时候}
\begin{pronuncia}{you3shi2hou0}
\significado{expr.}{
às vezes;
de vez em quando;
de quando em quando
}
\end{pronuncia}
\end{verbete}

\begin{verbete}[you3yi2si0]{有意思}
\begin{pronuncia}{you3yi2si0}
\significado{adj.}{
interessante
}
\end{pronuncia}
\end{verbete}

\begin{verbete}[you3yong4]{有用}
\begin{pronuncia}{you3yong4}
\significado{adj.}{
útil
}
\end{pronuncia}
\end{verbete}

\begin{verbete}[you4]{右}
\begin{pronuncia}{you4}
\significado{p.l.}{
direita
}
\end{pronuncia}
\end{verbete}

\begin{verbete}[you4bian0]{右边}
\begin{pronuncia}{you4bian0}
\significado{p.l.}{
à direita; ao lado direito
}
\end{pronuncia}
\end{verbete}

\begin{verbete}[you4mian0]{右面}
\begin{pronuncia}{you4mian0}
\significado{p.l.}{
à direita; ao lado direito
}
\end{pronuncia}
\end{verbete}

\begin{verbete}[yong4]{用}
\begin{pronuncia}{yong4}
\significado{v.}{
usar
}
\end{pronuncia}
\end{verbete}

\begin{verbete}[yu2]{鱼}
\begin{pronuncia}{yu2}
\significado[条,尾]{n.}{
peixe
}
\end{pronuncia}
\end{verbete}

\begin{verbete}[yu2pian4]{鱼片}
\begin{pronuncia}{yu2pian4}
\significado{n.}{
fatia de peixe; filé de peixe
}
\end{pronuncia}
\end{verbete}

\begin{verbete}[yu2xiang1rou4si1]{鱼香肉丝}
\begin{pronuncia}{yu2xiang1rou4si1}
\significado{n.}{
tiras de carne de porco salteadas com molho picante
}
\end{pronuncia}
\end{verbete}

\begin{verbete}[yu3]{玉}
\begin{pronuncia}{yu3}
\significado[块]{n.}{
jade
}
\end{pronuncia}
\end{verbete}

\begin{verbete}[yu3mao2qiu2]{羽毛球}
\begin{pronuncia}{yu3mao2qiu2}
\significado{n.}{
badminton
}
\end{pronuncia}
\end{verbete}

\begin{verbete}[yu3]{雨}
\begin{pronuncia}{yu3}
\significado[阵,场]{n.}{
chuva
}
\end{pronuncia}
\end{verbete}

\begin{verbete}[yu3san3]{雨伞}
\begin{pronuncia}{yu3san3}
\significado[把]{n.}{
guarda-chuva
}
\end{pronuncia}
\end{verbete}

\begin{verbete}[yu3yi1]{雨衣}
\begin{pronuncia}{yu3yi1}
\significado[件]{n.}{
impermeável
}
\end{pronuncia}
\end{verbete}

\begin{verbete}[yu3fa3]{语法}
\begin{pronuncia}{yu3fa3}
\significado{n.}{
gramática
}
\end{pronuncia}
\end{verbete}

\begin{verbete}[yu3yan2shi2yan4shi4]{语言实验室}
\begin{pronuncia}[\\]{yu3yan2shi2yan4shi4}
\significado{n.}{
laboratório de línguas
}
\end{pronuncia}
\end{verbete}

\begin{verbete}[yu4bao4]{预报}
\begin{pronuncia}{yu4bao4}
\significado{n.}{
previsão (meteorológica); boletim meteorológico
}
\significado{v.}{
prever (o tempo)
}
\end{pronuncia}
\end{verbete}

\begin{verbete}[yuan2]{元}
\begin{pronuncia}{yuan2}
\significado{p.c.}{
unidade monetária da China
}
\end{pronuncia}
\end{verbete}

\begin{verbete}[yuan3]{远}
\begin{pronuncia}{yuan3}
\significado{adj.}{
longe; longo, longa
}
\end{pronuncia}
\end{verbete}

\begin{verbete}[yuan4zi0]{院子}
\begin{pronuncia}{yuan4zi0}
\significado[个]{n.}{
pátio; jardim; quintal
}
\end{pronuncia}
\end{verbete}

\begin{verbete}[yue1hui4]{约会}
\begin{pronuncia}{yue1hui4}
\significado[次,个]{n.}{
compromisso; encontro marcado
}
\end{pronuncia}
\end{verbete}

\begin{verbete}[yue4]{月}
\begin{pronuncia}{yue4}
\significado[个,轮]{n.}{
mês
}
\end{pronuncia}
\end{verbete}

\begin{verbete}[yue4liang0]{月亮}
\begin{pronuncia}{yue4liang0}
\significado{n.}{
lua
}
\end{pronuncia}
\end{verbete}

\begin{verbete}[yue4du2]{阅读}
\begin{pronuncia}{yue4du2}
\significado{n.}{
leitura
}
\significado{v.}{
ler
}
\end{pronuncia}
\end{verbete}

\begin{verbete}[yue4...\ yue4...]{越······越······}
\begin{pronuncia}{yue4...\ yue4...}
\significado{expr.}{
quanto mais... tanto mais...
}
\end{pronuncia}
\end{verbete}

\begin{verbete}[yue4lai2yue4...]{越来越······}
\begin{pronuncia}{yue4lai2yue4...}
\significado{expr.}{
cada vez mais...
}
\end{pronuncia}
\end{verbete}

\begin{verbete}[yue4lan3shi4]{阅览室}
\begin{pronuncia}{yue4lan3shi4}
\significado[间]{n.}{
sala de leitura
}
\end{pronuncia}
\end{verbete}

\begin{verbete}[Yun2nan2]{云南}
\begin{pronuncia}{Yun2nan2}
\significado{n.}{
Yunnan
}
\end{pronuncia}
\end{verbete}

\begin{verbete}[yun4dong4]{运动}
\begin{pronuncia}{yun4dong4}
\significado[场]{n.}{
esporte; desporto
}
\end{pronuncia}
\end{verbete}

\begin{verbete}[yun4dong4chang3]{运动场}
\begin{pronuncia}{yun4dong4chang3}
\significado{n.}{
campo desportivo; campo de jogos
}
\end{pronuncia}
\end{verbete}

\begin{verbete}[yun4dong4hui4]{运动会}
\begin{pronuncia}{yun4dong4hui4}
\significado[个]{n.}{
jogos desportivos
}
\end{pronuncia}
\end{verbete}

\begin{verbete}[yun4dong4yuan2]{运动员}
\begin{pronuncia}{yun4dong4yuan2}
\significado[名,个]{n.}{
jogador, jogadora; atleta
}
\end{pronuncia}
\end{verbete}

%\end{multicols*}

%%%
%%% Z
%%%
\section*{Z}
\addcontentsline{toc}{section}{Z}
\begin{multicols*}{2}

\begin{verbete}[zai4]{在}
\begin{pronuncia}{zai4}
\significado{adv.}{
para designar ações que estão passando
}
\significado{prep.}{
em
}
\significado{v.}{
estar; ficar
}
\end{pronuncia}
\end{verbete}

\begin{verbete}[zai4]{再}
\begin{pronuncia}{zai4}
\significado{adv.}{
de novo; outra vez
}
\end{pronuncia}
\end{verbete}

\begin{verbete}[zai4jian4]{再见}
\begin{pronuncia}{zai4jian4}
\significado{v.}{
adeus; até à vista; até à próxima; até logo
}
\end{pronuncia}
\end{verbete}

\begin{verbete}[zan2men0]{咱们}
\begin{pronuncia}{zan2men0}
\significado{pron.}{
nós (eu e você)
}
\end{pronuncia}
\end{verbete}

\begin{verbete}[zang1]{脏}
\begin{pronuncia}{zang1}
\significado{adj.}{
sujo
}
\end{pronuncia}
\end{verbete}

\begin{verbete}[zao3]{早}
\begin{pronuncia}{zao3}
\significado{adj.}{
cedo
}
\end{pronuncia}
\end{verbete}

\begin{verbete}[zao3fan4]{早反}
\begin{pronuncia}{zao3fan4}
\significado{n.}{
café da manhã
}
\end{pronuncia}
\end{verbete}

\begin{verbete}[zao3shang0]{早上}
\begin{pronuncia}{zao3shang0}
\significado{p.t.}{
manhã cedo; manhãzinha
}
\end{pronuncia}
\end{verbete}

\begin{verbete}[zen3me0]{怎么}
\begin{pronuncia}{zen3me0}
\significado{interr.}{
como?
}
\end{pronuncia}
\end{verbete}

\begin{verbete}[zen3me0yang4]{怎么样}
\begin{pronuncia}{zen3me0yang4}
\significado{interr.}{
como?; que tal?
}
\end{pronuncia}
\end{verbete}

\begin{verbete}[zhan4]{站}
\begin{pronuncia}{zhan4}
\significado{n.}{
estação; ponto; paragem
}
\end{pronuncia}
\end{verbete}

\begin{verbete}[zhang1]{张}
\begin{pronuncia}{zhang1}
\significado{p.c.}{
para folha de papéis, mapas, etc
}
\end{pronuncia}
\end{verbete}

\begin{verbete}[zhao2ji2]{着急}
\begin{pronuncia}{zhao2ji2}
\significado{adj.}{
inquieto; ansioso
}
\end{pronuncia}
\end{verbete}

\begin{verbete}[zhao3]{找}
\begin{pronuncia}{zhao3}
\significado{v.}{
andar à procura de; procurar; dar troco
}
\end{pronuncia}
\end{verbete}

\begin{verbete}[zhao4pian4]{照片}
\begin{pronuncia}{zhao4pian4}
\significado[张,套,幅]{n.}{
fotografia; foto
}
\end{pronuncia}
\end{verbete}

\begin{verbete}[zhao4xiang4]{照相}
\begin{pronuncia}{zhao4xiang4}
\significado{v.+compl.}{
tirar fotografia
}
\end{pronuncia}
\end{verbete}

\begin{verbete}[zhao4xiang4ji1]{照相机}
\begin{pronuncia}{zhao4xiang4ji1}
\significado[个,架,部,台,只]{n.}{
câmera/máquina fotográfica
}
\end{pronuncia}
\end{verbete}

\begin{verbete}[zhe4]{这}
\begin{pronuncia}{zhe4}
\significado{pron.}{
este, esta, isto
}
\end{pronuncia}
\end{verbete}

\begin{verbete}[zhe4li0]{这里}
\begin{pronuncia}{zhe4li0}
\significado{pron.}{
aqui
}
\end{pronuncia}
\end{verbete}

\begin{verbete}[zhe4me0]{这么}
\begin{pronuncia}{zhe4me0}
\significado{adv.}{
como este; desta maneira
}
\end{pronuncia}
\end{verbete}

\begin{verbete}[zhe4xie1]{这些}
\begin{pronuncia}{zhe4xie1}
\significado{pron.}{
estes, estas
}
\end{pronuncia}
\end{verbete}

\begin{verbete}[zher4]{这儿}
\begin{pronuncia}{zher4}
\significado{pron.}{
aqui
}
\end{pronuncia}
\end{verbete}

\begin{verbete}[Zhe4jiang1]{浙江}
\begin{pronuncia}{Zhe4jiang1}
\significado{n.}{
Zhejiang
}
\end{pronuncia}
\end{verbete}

\begin{verbete}[zhen1]{真}
\begin{pronuncia}{zhen1}
\significado{adv.}{
que...tão...!|
realmente
}
\end{pronuncia}
\end{verbete}

\begin{verbete}[zheng4qian2]{挣钱}
\begin{pronuncia}{zheng4qian2}
\significado{v.+compl.}{
ganhar dinheiro
}
\end{pronuncia}
\end{verbete}

\begin{verbete}[zheng4zai4]{正在}
\begin{pronuncia}{zheng4zai4}
\significado{adv.}{
estar a + inf.; estar + ger.
}
\end{pronuncia}
\end{verbete}

\begin{verbete}[zhi1]{支}
\begin{pronuncia}{zhi1}
\significado{p.c.}{
para caneta, lápis, etc
}
\end{pronuncia}
\end{verbete}

\begin{verbete}[zhi1]{只}
\begin{pronuncia}{zhi1}
\significado{p.c.}{
para pássaros, gatos, cãezinhos, etc
}
\end{pronuncia}
\begin{pronuncia}{zhi3}
\significado{adv.}{
apenas; só
}
\end{pronuncia}
\end{verbete}

\begin{verbete}[zhi1dao0]{知道}
\begin{pronuncia}{zhi1dao0}
\significado{v.}{
conhecer; saber
}
\end{pronuncia}
\end{verbete}

\begin{verbete}[zhi2yuan2]{职员}
\begin{pronuncia}{zhi2yuan2}
\significado{n.}{
empregado, empregada
}
\end{pronuncia}
\end{verbete}

\begin{verbete}[zhi3]{只}
\begin{pronuncia}{zhi3}
\significado{adv.}{
apenas; só
}
\end{pronuncia}
\begin{pronuncia}{zhi1}
\significado{p.c.}{
para pássaros, gatos, cãezinhos, etc
}
\end{pronuncia}
\end{verbete}

\begin{verbete}[Zhong1guo2]{中国}
\begin{pronuncia}{Zhong1guo2}
\significado{n.}{
China
}
\end{pronuncia}
\end{verbete}

\begin{verbete}[Zhong1guo2tong1]{中国通}
\begin{pronuncia}{Zhong1guo2tong1}
\significado{n.}{
conhecedor da China; especialista em tudo sobre a China
}
\end{pronuncia}
\end{verbete}

\begin{verbete}[zhong1jian1]{中间}
\begin{pronuncia}{zhong1jian1}
\significado{p.l.}{
central; centro; no meio
}
\end{pronuncia}
\end{verbete}

\begin{verbete}[zhong1wen2]{中文}
\begin{pronuncia}{zhong1wen2}
\significado{n.}{
chinês; língua chinesa
}
\end{pronuncia}
\end{verbete}

\begin{verbete}[zhong1xue2]{中学}
\begin{pronuncia}{zhong1xue2}
\significado[个]{n.}{
escola ensino médio
}
\end{pronuncia}
\end{verbete}

\begin{verbete}[zhong1xue2sheng1]{中学生}
\begin{pronuncia}{zhong1xue2sheng1}
\significado{n.}{
estudante da escola ensino médio
}
\end{pronuncia}
\end{verbete}

\begin{verbete}[zhong1xun2]{中询}
\begin{pronuncia}{zhong1xun2}
\significado{p.t.}{
segunda dezena do mês; meio do mês; em meados do mês
}
\end{pronuncia}
\end{verbete}

\begin{verbete}[zhong1]{钟}
\begin{pronuncia}{zhong1}
\significado{p.c.}{
hora
}
\end{pronuncia}
\end{verbete}

\begin{verbete}[zhong3]{种}
\begin{pronuncia}{zhong3}
\significado{p.c.}{
para tipos, espécies e gêneros
}
\end{pronuncia}
\end{verbete}

\begin{verbete}[zhong4]{重}
\begin{pronuncia}{zhong4}
\significado{adj.}{
pesado, pesada
}
\end{pronuncia}
\end{verbete}

\begin{verbete}[zhong4liang4]{重量}
\begin{pronuncia}{zhong4liang4}
\significado[个]{n.}{
peso
}
\end{pronuncia}
\end{verbete}

\begin{verbete}[zhou1mo4]{周末}
\begin{pronuncia}{zhou1mo4}
\significado{n.}{
final-de-semana
}
\end{pronuncia}
\end{verbete}

\begin{verbete}[zhu1]{猪}
\begin{pronuncia}{zhu1}
\significado[口,头]{n.}{
porco, porca
}
\end{pronuncia}
\end{verbete}

\begin{verbete}[Zhu3xi2]{主席}
\begin{pronuncia}{Zhu3xi2}
\significado[个,位]{n.}{
Presidente (da China); Primeiro-Ministro
}
\end{pronuncia}
\end{verbete}

\begin{verbete}[zhu4]{住}
\begin{pronuncia}{zhu4}
\significado{v.}{
morar; viver; alojar-se
}
\end{pronuncia}
\end{verbete}

\begin{verbete}[zhu4zhai2]{住宅}
\begin{pronuncia}{zhu4zhai2}
\significado{n.}{
residência
}
\end{pronuncia}
\end{verbete}

\begin{verbete}[zhu4ce4]{注册}
\begin{pronuncia}{zhu4ce4}
\significado{v.}{
inscrever-se; matricular-se
}
\end{pronuncia}
\end{verbete}

\begin{verbete}[zhu4]{祝}
\begin{pronuncia}{zhu4}
\significado{v.}{
desejar (exprimir um bom desejo);
congratular
}
\end{pronuncia}
\end{verbete}

\begin{verbete}[zhu4fu4]{嘱咐}
\begin{pronuncia}{zhu4fu4}
\significado{v.}{
ordenar; dizer; exortar
}
\end{pronuncia}
\end{verbete}

\begin{verbete}[zhuan1ye4]{专业}
\begin{pronuncia}{zhuan1ye4}
\significado[门,个]{n.}{
área de atuação; especialidade
}
\end{pronuncia}
\end{verbete}

\begin{verbete}[zhuo1zi0]{桌子}
\begin{pronuncia}{zhuo1zi0}
\significado[张,套]{n.}{
mesa
}
\end{pronuncia}
\end{verbete}

\begin{verbete}[zi3se4]{紫色}
\begin{pronuncia}{zi3se4}
\significado{n.}{
cor roxa
}
\end{pronuncia}
\end{verbete}

\begin{verbete}[zi4]{字}
\begin{pronuncia}{zi4}
\significado[个]{n.}{
carácter; letra; símbolo; palavra
}
\end{pronuncia}
\end{verbete}

\begin{verbete}[zi4ji3]{自己}
\begin{pronuncia}{zi4ji3}
\significado{pron.}{
a si próprio; próprio
}
\end{pronuncia}
\end{verbete}

\begin{verbete}[zi4xing2che1]{自行车}
\begin{pronuncia}{zi4xing2che1}
\significado[辆]{n.}{
bicicleta
}
\end{pronuncia}
\end{verbete}

\begin{verbete}[zi4wo3]{自我}
\begin{pronuncia}{zi4wo3}
\significado{pron.}{
a si mesmo; eu próprio|
auto-...
}
\end{pronuncia}
\end{verbete}

\begin{verbete}[Zong3du1]{总督}
\begin{pronuncia}{Zong3du1}
\significado{n.}{
Governador; Governador-Geral; Vice-Rei
}
\end{pronuncia}
\end{verbete}

\begin{verbete}[Zong3li3]{总理}
\begin{pronuncia}{Zong3li3}
\significado[个,位,名]{n.}{
Primeiro-Ministro
}
\end{pronuncia}
\end{verbete}

\begin{verbete}[Zong3tong3]{总统}
\begin{pronuncia}{Zong3tong3}
\significado[个,位,名,届]{n.}{
Presidente (de um país)
}
\end{pronuncia}
\end{verbete}

\begin{verbete}[zou3]{走}
\begin{pronuncia}{zou3}
\significado{v.}{
andar; caminhar
}
\end{pronuncia}
\end{verbete}

\begin{verbete}[zu2qiu2]{足球}
\begin{pronuncia}{zu2qiu2}
\significado[个]{n.}{
futebol; bola de futebol
}
\end{pronuncia}
\end{verbete}

\begin{verbete}[zui3ba0]{嘴巴}
\begin{pronuncia}{zui3ba0}
\significado[张]{n.}{
boca
}
\significado[个]{n.}{
bofetada na cara
}
\end{pronuncia}
\end{verbete}

\begin{verbete}[zui4]{最}
\begin{pronuncia}{zui4}
\significado{adv.}{
o mais, a mais|
grau superlativo relativo de superioridade
}
\end{pronuncia}
\end{verbete}

\begin{verbete}[zui4hou4]{最后}
\begin{pronuncia}{zui4hou4}
\significado{adj.}{
final; último
}
\end{pronuncia}
\end{verbete}

\begin{verbete}[zui4jin4]{最近}
\begin{pronuncia}{zui4jin4}
\significado{adv.}{
ultimamente; recentemente
}
\end{pronuncia}
\end{verbete}

\begin{verbete}[zuo2tian1]{昨天}
\begin{pronuncia}{zuo2tian1}
\significado{p.t.}{
ontem
}
\end{pronuncia}
\end{verbete}

\begin{verbete}[zuo3]{左}
\begin{pronuncia}{zuo3}
\significado{p.l.}{
esquerda
}
\end{pronuncia}
\end{verbete}

\begin{verbete}[zuo3bian0]{左边}
\begin{pronuncia}{zuo3bian0}
\significado{p.l.}{
esquerda; lado esquerdo
}
\end{pronuncia}
\end{verbete}

\begin{verbete}[zuo3mian0]{左面}
\begin{pronuncia}{zuo3mian0}
\significado{p.l.}{
esquerda; lado esquerdo
}
\end{pronuncia}
\end{verbete}

\begin{verbete}[zuo3you4]{左右}
\begin{pronuncia}{zuo3you4}
\significado{part.}{
cerca de; aproximadamente
}
\end{pronuncia}
\end{verbete}

\begin{verbete}[zuo4]{坐}
\begin{pronuncia}{zuo4}
\significado{v.}{
sentar-se|
andar de carro, ônibus, trem, avião, etc
}
\end{pronuncia}
\end{verbete}

\begin{verbete}[zuo4]{做}
\begin{pronuncia}{zuo4}
\significado{v.}{
fazer
}
\end{pronuncia}
\end{verbete}

\end{multicols*}


\printindex

\end{document}
