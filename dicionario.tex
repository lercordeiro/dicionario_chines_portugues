%%%%%%%%%%%%%%%%%%%%%%%%%%%%%%%%%%%%%%%%%
% XeTeX
%
% Vocabulário
% Autor: Luiz Eduardo Roncato Cordeiro
%
% Licença:
% CC BY-NC-SA 3.0 (http://creativecommons.org/licenses/by-nc-sa/3.0/)
%%%%%%%%%%%%%%%%%%%%%%%%%%%%%%%%%%%%%%%%%

%\documentclass[a4paper,12pt,twoside,openany,draft]{memoir}
\documentclass[a4paper,12pt,twoside,openany]{memoir}

\usepackage{xltxtra}
\usepackage[brazil]{babel}
\usepackage{xeCJK}
\usepackage{xpinyin}
\usepackage{fontspec}
\usepackage{xunicode}
\usepackage{xltxtra}
\usepackage[usenames,dvipsnames]{color}
\usepackage{multicol}
\usepackage{fancyhdr}
\usepackage{imakeidx}
\usepackage{ifthen}
\usepackage{tocloft}
\usepackage{xparse}
\usepackage{enumitem}
\usepackage{wasysym}

%\setlength{\cftbeforesecskip}{2pt}
%\renewcommand{\cftsecleader}{\cftdotfill{\cftdotsep}}

\setCJKmainfont{AR PL New Kai}
\setCJKsansfont{AR PL UMing CN}

\makeindex[columns=3, title=Índice, intoc]

\setlength{\parindent}{0em}
\setlength{\parskip}{0.5em}
\setlength{\columnseprule}{0.5pt}

\xpinyinsetup{ratio={.5},vsep={1.2em},multiple={\color{Sepia}}}

% Headers & footers
\fancyhead[L]{\textsf{\rightmark}} % Top left header
\fancyhead[R]{\textsf{\leftmark}} % Top right header
\renewcommand{\headrulewidth}{1.4pt} % Rule under the header
\fancyfoot[C]{{\thepage}} % Bottom center footer
\renewcommand{\footrulewidth}{1.4pt} % Rule under the header
\pagestyle{fancy} % Use the custom headers and footers throughout the document

\setlength{\headheight}{16pt}
\addtolength{\topmargin}{-0.5pt}

\NewDocumentCommand\mylist{>{\SplitList{|}}m}
{
\setlength{\leftmargin}{0em}
\begin{itemize}[nosep,left=0em]
  \ProcessList{#1}{ \insertitem }
\end{itemize}
}
\newcommand\insertitem[1]{\item #1}

\newenvironment{verbete}[2][none]
  {\vspace{1ex}
  \markboth{#1«\pinyin{#2}»}{#1«\pinyin{#2}»}\index{#1«\pinyin{#2}»}
  \begin{minipage}[t][][t]{\linewidth}
  {\Huge \textbf{#1}} \\
  }
  {\end{minipage}
  }

\newcommand{\entry}[3]{%
«\pinyin{#1}» \if\relax\detokenize{#2}\relax\else\ (\textit{#2})\fi%
{\small \mylist{#3}}%
\vspace{1ex}
}

\newcommand{\e}[1]{\textcolor{OliveGreen}{#1}}
\newcommand{\pc}[1]{P.C.:#1}

\makeatletter
\let\old@makechapterhead\@makechapterhead
% Taken from http://mirrors.ctan.org/macros/latex/unpacked/report.cls
\def\fake@makechapterhead#1{%
  \vspace*{50\p@}%
  {\parindent \z@ \raggedright \normalfont
    \ifnum \c@secnumdepth >\m@ne
        \huge\bfseries \strut%\@chapapp\space \thechapter
        \par\nobreak
        \vskip 20\p@
    \fi
    \interlinepenalty\@M
    \Huge \bfseries #1\par\nobreak
    \vskip 40\p@
  }
  \markboth{#1}{\thechapter}
}
\newcommand{\newchapterhead}{\let\@makechapterhead\fake@makechapterhead}
\newcommand{\restorechapterhead}{\let\@makechapterhead\old@makechapterhead}
\makeatother

%%%
%%% Documento começa aqui!
%%%

\begin{document}

\newchapterhead

\begin{titlingpage} % Suppresses displaying the page number on the title page and the subsequent page counts as page 1
	
	\raggedleft % Right align the title page
	
	\rule{1pt}{\textheight} % Vertical line
	\hspace{0.05\textwidth} % Whitespace between the vertical line and title page text
	\parbox[b]{0.75\textwidth}{ % Paragraph box for holding the title page text, adjust the width to move the title page left or right on the page
		
		{\Huge\bfseries 汉葡词典}\\[2\baselineskip] % Title
		{\large\textit{O Minúsculo Dicionário do Curso de Chinês}}\\[4\baselineskip] % Subtitle or further description
		{\Large\textsc{罗学凯}} % Author name, lower case for consistent small caps
		
		\vspace{0.5\textheight} % Whitespace between the title block and the publisher
		
		{\noindent \today}\\[\baselineskip] % Publisher and logo
	}

\end{titlingpage}

\tableofcontents

\newpage

\chapter{Termos Gramaticais Chineses}

\begin{tabular}{lll}
nome/substantivo       & \textbf{n.}        & 名词 \\
palavra de lugar       & \textbf{p.d.l.}    & 处所词 \\
palavra de localização & \textbf{p.l.}      & 方位词 \\
palavra de tempo       & \textbf{p.t.}      & 时间词 \\
verbo                  & \textbf{v.}        & 动词 \\
verbo direcional       & \textbf{v.d.}      & 趣向\hspace{1em}动词 \\
verbo optativo         & \textbf{v.o.}      & 能缘\hspace{1em}动词 \\
adjetivo               & \textbf{adj.}      & 形容词 \\
numeral                & \textbf{num.}      & 数词 \\
palavra classificadora & \textbf{p.c.}      & 两量词 \\
pronome                & \textbf{pron.}     & 代词 \\
interrogativo          & \textbf{interr.}   & 疑问词 \\
advérbio               & \textbf{adv.}      & 副词 \\
preposição             & \textbf{prep.}     & 介词 \\
conjunção              & \textbf{conj.}     & 连词 \\
partícula              & \textbf{part.}     & 助词 \\
sujeito                & \textbf{suj.}      & 主语 \\
objeto                 & \textbf{obj.}      & 宾语 \\
atributo               & \textbf{atrib.}    & 定语 \\
adjunto adverbial      & \textbf{a.adv.}    & 状语 \\
complemento            & \textbf{compl.}    & 补语 \\
verbo+complemento      & \textbf{v.+compl.} & 动宾式\hspace{1em}离合词 \\
\end{tabular}

\newpage

\chapter{汉葡词典}

%%%
%%% Estou ordenando as palavras em ordem alfabética por pinyin.
%%% Obs: Para as palavras diferentes com o mesmo pinyin, a que tem o menor 
%%%      número de traços vem antes.
%%%

%%%
%%% A
%%%

\section*{A}\addcontentsline{toc}{section}{A}

\begin{entry}{阿}{a1}{7}{⾩}
  \definition{pref.}{em dialetos do sul para formar termos carinhosos, antes de nomes de animais de estimação, sobrenomes monossilábicos ou números que denotam ordem de antiguidade em uma; anexado a 大, 二, 三\dots\ para indicar classificação (e, às vezes, intimidade) | antes dos termos de parentesco; na frente de um sobrenome, de um nome próprio ou de um determinado título, com uma conotação de intimidade | em alguns contextos, pode soar infantil ou muito informal (por exemplo, chamar um colega de trabalho por ``阿 + Nome'' sem intimidade)}[阿妈 (mamãe) | 阿明 (forma carinhosa de chamar alguém chamado Ming)]
  \seeref{阿}{e1}
\end{entry}

\begin{entry}{阿哥}{a1ge1}{7,10}{⾩、⼝}
  \definition{s.}{irmão mais velho (afetivo)}[阿哥,帮我拿一下书包!(Irmão, ajude-me com minha mochila escolar!)]
\end{entry}

\begin{entry}{阿姨}{a1yi2}{7,9}{⾩、⼥}[HSK 4]
  \definition[个,位]{s.}{tia; uma forma de tratamento para uma mulher da geração dos pais; dirigir-se a uma mulher que tem aproximadamente a mesma idade da sua mãe, geralmente não é parente | babá em uma família; professora em um jardim de infância | tia; irmã da mãe (mais comum no sul da China)}[阿姨,生日快乐!(Tia, feliz aniversário!) | 阿姨,这个苹果多少钱一斤?(Tia/Senhora, quanto custa o quilo dessas maçãs?) | 阿姨,我想喝水。(Tia/Babá, eu quero beber água.)]
\end{entry}

\begin{entry}{呵}{a1}{8}{⼝}
  \variantof{啊}
  \seeref{呵}{he1}
\end{entry}

\begin{entry}{啊}{a1}{10}{⼝}[HSK 2]
  \definition{interj.}{Ah!; Oh!; expressar surpresa ou admiração}
  \seeref{啊}{a2}
  \seeref{啊}{a3}
  \seeref{啊}{a4}
  \seeref{啊}{a5}
\end{entry}

\begin{entry}{啊呀}{a1ya1}{10,7}{⼝、⼝}
  \definition{interj.}{Oh meu Deus! | interjeição de surpresa}
\end{entry}

\begin{entry}{啊哟}{a1yo5}{10,9}{⼝、⼝}
  \definition{interj.}{Meu Deus! | Oh! | Ai! | interjeição de surpresa ou dor}
\end{entry}

\begin{entry}{啊}{a2}{10}{⼝}[HSK 2]
  \definition{interj.}{Eh?; Ei?; Que?; Por que?; expressar questionamento, dúvida ou solicitar opinião}
  \seeref{啊}{a1}
  \seeref{啊}{a3}
  \seeref{啊}{a4}
  \seeref{啊}{a5}
\end{entry}

\begin{entry}{嗄}{a2}{13}{⼝}
  \definition{adj.}{rouco}
  \variantof{啊}
\end{entry}

\begin{entry}{啊}{a3}{10}{⼝}[HSK 2]
  \definition{interj.}{Eh?; Meu!; E aí?; Que?; expressar surpresa e dúvida}
  \seeref{啊}{a1}
  \seeref{啊}{a2}
  \seeref{啊}{a4}
  \seeref{啊}{a5}
\end{entry}

\begin{entry}{啊}{a4}{10}{⼝}[HSK 2]
  \definition{interj.}{Bem!; Sim!; expressa concordância, pronúncia mais curta | Oh!; Ah!; indica que compreendeu, com pronúncia mais longa | Oh!; expressa surpresa ou admiração, com pronúncia mais longa | Desgraça!; expressar tristeza, pesar ou pesar}
  \seeref{啊}{a1}
  \seeref{啊}{a2}
  \seeref{啊}{a3}
  \seeref{啊}{a5}
\end{entry}

\begin{entry}{啊}{a5}{10}{⼝}[HSK 2,4]
  \definition{part.}{usado no final da frase para expressar admiração | usado no final da frase para expressar afirmação, justificativa, insistência, recomendação, etc. | usado no final da frase para indicar dúvida | usado para fazer uma pequena pausa na frase, chamando a atenção para o que vem a seguir | usado após os itens enumerados | usado após verbos repetitivos, indica um processo longo}
  \seeref{啊}{a1}
  \seeref{啊}{a2}
  \seeref{啊}{a3}
  \seeref{啊}{a4}
\end{entry}

\begin{entry}{矮}{ai3}{13}{⽮}[HSK 4]
  \definition{adj.}{baixo em estatura, dimensão, grau ou ranque | curto (em comprimento)}[他比我矮。(Ele é mais baixo que eu.) | 这栋楼很矮,只有三层。(Esse prédio é baixo, tem só três andares.) | 她虽然矮,但是跑得很快!(Ela pode ser baixinha, mas corre muito rápido!)]
\end{entry}

\begin{entry}{矮凳}{ai3deng4}{13,14}{⽮、⼏}
  \definition{s.}{banquinho baixo | banqueta}[这个矮凳是木制的,很结实。(Este banquinho é de madeira e bem resistente.)]
\end{entry}

\begin{entry}{矮林}{ai3lin2}{13,8}{⽮、⽊}
  \definition{s.}{mata rasteira | bosque baixo}[这片矮林里有很多野兔和鸟类。(Neste bosque baixo há muitos coelhos selvagens e pássaros.) | 山坡上长满了矮林,远看像绿色的地毯。(A encosta está coberta de mata rasteira, que de longe parece um tapete verde.)]
\end{entry}

\begin{entry}{矮胖}{ai3pang4}{13,9}{⽮、⾁}
  \definition{adj.}{atarracado; gorducho; rechonchudo; roliço; baixo e robusto | chamar alguém diretamente de 矮胖 pode ser ofensivo}[我家猫矮胖矮胖的,像个毛球。(Meu gato é baixinho e gordinho, parece uma bolinha de pelo.)]
\end{entry}

\begin{entry}{矮人}{ai3ren2}{13,2}{⽮、⼈}
  \definition{s.}{anão; pessoa de baixa estatura (indivíduo) | homúnculo; figuras criadas artificialmente pelos alquimistas em frascos de destilação | nanismo}[他虽然是矮人,但很有力气。(Embora ele seja baixo, é muito forte.) | 北欧神话中的矮人是技艺高超的工匠。(Na mitologia nórdica, os anões são artesãos habilidosos.) | 他因为身高被嘲笑为‘矮人’,这让他很伤心。(Ele foi zombado por ser chamado de ‘anão’ devido à sua altura, o que o magoou.)]
\end{entry}

\begin{entry}{矮树}{ai3shu4}{13,9}{⽮、⽊}
  \definition{s.}{arbusto | árvore pequena, baixa}[矮树比高树更容易修剪。(Árvores baixas são mais fáceis de podar do que árvores altas.) | 我们种了些矮树作为花园的边界。(Plantamos alguns arbustos como cerca natural do jardim.)]
\end{entry}

\begin{entry}{矮小}{ai3 xiao3}{13,3}{⽮、⼩}[HSK 4]
  \definition{adj.}{subdimensionado; curto e pequeno; baixo e pequeno | quando usado para pessoas, pode soar depreciativo se não for em contexto neutro ou afetuoso}[这位矮小的老人是村里的智者。(Este idoso baixinho é o sábio da vila.) | 这种矮小的灌木适合盆栽。(Este tipo de arbusto pequeno é ideal para vasos.) | 山脚下有一片矮小的房屋,显得格外宁静。(Ao pé da montanha, havia casas baixas que transmitiam uma tranquilidade única.)]
\end{entry}

\begin{entry}{矮星}{ai3xing1}{13,9}{⽮、⽇}
  \definition{s.}{estrela anã}[白矮星是恒星演化的最终阶段之一。(Anãs brancas são um dos estágios finais da evolução estelar.)]
\end{entry}

\begin{entry}{矮子}{ai3zi5}{13,3}{⽮、⼦}
  \definition{s.}{pessoa baixa; anão; baixinho}[白雪公主和七个小矮子住在森林里。(Branca de Neve e os sete anões vivem na floresta.) | 用`矮子'称呼他人是不礼貌的。(Chamar alguém de ``baixinho'' é falta de educação.)]
\end{entry}

\begin{entry}{爱}{ai4}{10}{⽖}[HSK 1]
  \definition*{s.}{sobrenome Ai}
  \definition[个]{s.}{amor; afeição; afeição profunda; preocupação profunda; especialmente amor entre pessoas}[爱是理解和包容。(O amor é compreensão e tolerância.)]
  \definition{v.}{amar; ter sentimentos profundos por pessoas ou coisas | gostar; gostar de; estar interessado em |  cuidar; valorizar; ter em alta estima; cuidar bem de | estar apto a; ter o hábito de}[他们深深爱着对方。(Eles se amam profundamente.) | 我爱我的家人。(Eu amo minha família.) | 我爱旅行。(Eu adoro viajar.)]
\end{entry}

\begin{entry}{爱爱}{ai4'ai5}{10,10}{⽖、⽖}
  \definition{v.}{(coloquial) fazer amor ou relações íntimas | pode ser usado como um apelido entre casais, transmitindo ternura | pode soar vulgar se usado em contextos inadequados}[他们俩刚结婚,天天都想爱爱。(Eles acabaram de se casar e querem fazer amor todo dia.) | 爱爱,你今天好漂亮!(Amor, você está linda hoje!)]
\end{entry}

\begin{entry}{爱抚}{ai4fu3}{10,7}{⽖、⼿}
  \definition{v.}{acariciar; afagar; cuidar (com ternura)}[他轻轻爱抚她的头发。(Ele afagou suavemente o cabelo dela.) | 母亲爱抚婴儿的脸颊。(A mãe acaricia a bochecha do bebê.) | 她爱抚着小猫的耳朵。(Ela acariciou as orelhas do gatinho.)]
\end{entry}

\begin{entry}{爱国}{ai4 guo2}{10,8}{⽖、⼞}[HSK 4]
  \definition{adj.}{patriótico; patriotismo}[爱国是每个公民的责任。(O patriotismo é o dever de todo cidadão.) | 这部电影讲述了英雄的爱国故事。(Este filme conta a história patriótica de um herói.)]
  \definition{v.}{ser patriota; amar o seu país}
\end{entry}

\begin{entry}{爱好}{ai4 hao4}{10,6}{⽖、⼥}[HSK 1]
  \definition[个,种]{s.}{passatempo; interesse; \emph{hobby}; sentimentos de interesse especial ou afeição por algo | 爱好 é mais usado para atividades regulares (esportes, música), enquanto 喜欢 é para preferências gerais}[他的爱好是收集邮票。(Seu hobby era colecionar selos.)  | 我的爱好是读书和旅行。(Meus hobbies são ler e viajar.)]
  \definition{v.}{estar interessado em; ter prazer em; ter um forte interesse em algo; ter sentimentos profundos por alguém ou algo}
  \seealsoref{喜欢}{xi3huan5}
\end{entry}

\begin{entry}{爱好者}{ai4 hao4 zhe3}{10,6,8}{⽖、⼥、⽼}
  \definition{s.}{hobbista; amador; entusiasta; fã; amante (de arte, esportes, etc.)}[他是一位摄影爱好者。(Ele é um entusiasta de fotografia.) | 她是位潜水爱好者,经常去东南亚潜水。(Ela é uma mergulhadora amadora e frequentemente mergulha no Sudeste Asiático.)  | 我们为书法爱好者创建了一个微信群。(Criamos um grupo no WeChat para amantes de caligrafia.)]
\end{entry}

\begin{entry}{爱护}{ai4hu4}{10,7}{⽖、⼿}[HSK 4]
  \definition{v.}{acalentar; valorizar; salvaguardar; cuidar bem de}[全社会都应爱护老年人。(Toda a sociedade deve tratar os idosos com cuidado e respeito.) | 请爱护公园里的小动物。(Por favor, tratem os animais do parque com cuidado.)]
\end{entry}

\begin{entry}{爱情}{ai4qing2}{10,11}{⽖、⼼}[HSK 2]
  \definition{s.}{amor (entre pessoas); afeição}[爱情是盲目的。(O amor é cego.) | 爱情如同玫瑰,美丽却带刺。(O amor é como uma rosa, bela mas com espinhos.)  | 这首歌讲述了破碎的爱情故事。(Esta música conta uma história de amor fracassado.)]
\end{entry}

\begin{entry}{爱人}{ai4 ren5}{10,2}{⽖、⼈}[HSK 2]
  \definition[个]{s.}{amante; \emph{dollbaby}; namorado(a) | marido ou esposa; mais usado em ocasiões formais}[这是我的爱人。(Este é o meu/minha esposo/companheiro.) | 她是我一生的爱人。(Ela é o amor da minha vida.) | 请携带爱人出席晚宴。(Por favor, traga seu cônjuge para o jantar.)]
\end{entry}

\begin{entry}{爱上}{ai4shang4}{10,3}{⽖、⼀}
  \definition{v.}{perder o coração por; apaixonar-se por}[他在旅行时爱上了一位法国女孩。(Ele se apaixonou por uma garota francesa durante a viagem.)  | 来到杭州后,我爱上了龙井茶。(Depois de chegar em Hangzhou, me apaixonei pelo chá Longjing.) | 我从来没想过自己会爱上健身。(Eu nunca imaginei que iria me apaixonar por academia.)]
\end{entry}

\begin{entry}{爱心}{ai4xin1}{10,4}{⽖、⼼}[HSK 3]
  \definition[片]{s.}{amor | cuidado | compaixão}
\end{entry}

\begin{entry}{碍事}{ai4shi4}{13,8}{⽯、⼅}
  \definition{s.}{(usualmente em frases negativas) sem consequência, não importa}
  \definition{v.+compl.}{estar no caminho | ser um obstáculo}
\end{entry}

\begin{entry}{广}{an1}{3}{⼴}
  \definition{s.}{mais comum em nomes de pessoas; o mesmo que 庵}[广安是我的朋友。(An'an é meu amigo.)]
  \seeref{广}{guang3}
  \seeref{广}{yan3}
  \seealsoref{庵}{an1}
\end{entry}

\begin{entry}{安}{an1}{6}{⼧}[HSK 4]
  \definition{adj.}{pacífico; quieto; tranquilo; calmo; estáve; sem perturbação | seguro; protegido; com boa saúde; em paz; bem}
  \definition{pron.}{onde; como}
  \definition{s.}{segurança; proteção; paz; conforto | ampère; (eletricidade) abreviação de ampère}
  \definition{v.}{deixar (a mente de alguém) à vontade; acalmar; estabilizar | satisfazer; estar satisfeito; sentir-se satisfeito e à vontade | colocar em uma posição adequada; encontrar um lugar para | instalar; consertar; encaixar; configurar | trazer (uma acusação contra alguém); dar (a alguém um apelido) | abrigar (uma intenção); manter; segurar}
\end{entry}

\begin{entry}{安家}{an1jia1}{6,10}{⼧、⼧}
  \definition{v.+compl.}{montar uma casa | estabelecer-se}
\end{entry}

\begin{entry}{安静}{an1jing4}{6,14}{⼧、⾭}[HSK 2]
  \definition{adj.}{silencioso; tranquilo; sem som; sem barulho e sem algazarra}
\end{entry}

\begin{entry}{安排}{an1pai2}{6,11}{⼧、⼿}[HSK 3]
  \definition{s.}{arranjos | planos}
  \definition{v.}{organizar | programar | fazer planos}
\end{entry}

\begin{entry}{安全}{an1quan2}{6,6}{⼧、⼊}[HSK 2]
  \definition{adj.}{seguro; protegido; sem perigo; sem ameaças; sem acidentes}
  \definition{s.}{segurança; proteção; refere-se a um estado ou conceito, geralmente indicando ausência de ameaças ou perigo}
\end{entry}

\begin{entry}{安神}{an1shen2}{6,9}{⼧、⽰}
  \definition{v.+compl.}{acalmar os nervos | aliviar a inquietação pela tranquilização da mente e do corpo}
\end{entry}

\begin{entry}{安慰}{an1wei4}{6,15}{⼧、⼼}[HSK 5]
  \definition{adj.}{confortar; tranquilizar; consolar; apaziguar;}
  \definition[个]{s.}{conforto; consolo; comportamento que alivia a dor de alguém e o acalma com palavras ou gestos}
  \definition{v.}{confortar; consolar; acalmar e confortar; deixar a mente tranquila}
\end{entry}

\begin{entry}{安置}{an1zhi4}{6,13}{⼧、⽹}[HSK 4]
  \definition{v.}{providenciar; encontrar um lugar para; ajudar a estabelecer-se; colocar pessoas ou coisas em uma determinada posição ou organizá-las adequadamente}
\end{entry}

\begin{entry}{安装}{an1zhuang1}{6,12}{⼧、⾐}[HSK 3]
  \definition{v.}{instalar | consertar | configurar}
\end{entry}

\begin{entry}{庵}{an1}{11}{⼴}
  \definition*{s.}{sobrenome An}
  \definition[个,座]{s.}{cabana | convento de freiras; templos budistas, principalmente onde vivem as freiras}
\end{entry}

\begin{entry}{岸}{an4}{8}{⼭}[HSK 5]
  \definition{adj.}{elevado; grandioso (de maneira sombria ou condescendente)}
  \definition[个]{s.}{margem; costa; litoral; terreno à beira da água}
\end{entry}

\begin{entry}{岸上}{an4 shang4}{8,3}{⼭、⼀}[HSK 5]
  \definition{s.}{em terra; costa; margem | na margem do rio; na beira do rio}
\end{entry}

\begin{entry}{按}{an4}{9}{⼿}[HSK 3]
  \definition{v.}{pressionar | empurrar para baixo | deixar de lado | arquivar | restringir | controlar}
\end{entry}

\begin{entry}{按摩}{an4mo2}{9,15}{⼿、⼿}[HSK 5]
  \definition{s.}{massagem; empurrar, pressionar, beliscar e amassar o corpo de uma pessoa com as mãos para promover a circulação sanguínea, aumentar a resistência da pele e regular a função dos nervos}
\end{entry}

\begin{entry}{按时}{an4shi2}{9,7}{⼿、⽇}[HSK 4]
  \definition{adv.}{na hora; no horário; pontualmente; de acordo com o tempo estipulado}
\end{entry}

\begin{entry}{按照}{an4zhao4}{9,13}{⼿、⽕}[HSK 3]
  \definition{prep.}{de acordo com; em conformidade com; à luz de; com base em}
\end{entry}

\begin{entry}{暗}{an4}{13}{⽇}[HSK 4]
  \definition{adj.}{escuro; opaco; sem graça; pouca luz | escondido; secreto; não revelado | pouco claro; nebuloso; vago; confuso | subterrâneo}
  \definition{adv.}{secretamente | no escuro}
\end{entry}

\begin{entry}{暗恋}{an4lian4}{13,10}{⽇、⼼}
  \definition{s.}{amor secreto}
  \definition{v.}{estar secretamente apaixonado por}
\end{entry}

\begin{entry}{暗示}{an4shi4}{13,5}{⽇、⽰}[HSK 4]
  \definition[个]{s.}{sugestão; insinuação; intimação; (psicologia) refere-se ao uso de palavras, gestos, expressões, etc. para fazer as pessoas aceitarem involuntariamente uma determinada opinião ou fazerem algo}
  \definition{v.}{dar uma dica; sugerir secretamente; indicar algo a alguém usando outras palavras, expressões faciais ou gestos sem dizer em voz alta}
\end{entry}

\begin{entry}{暗香}{an4xiang1}{13,9}{⽇、⾹}
  \definition{s.}{fragrância sutil}
\end{entry}

\begin{entry}{奥}{ao4}{12}{⼤}
  \definition{adj.}{obscuro | misterioso}
\end{entry}

\begin{entry}{奥林匹克运动会}{ao4lin2pi3ke4 yun4dong4hui4}{12,8,4,7,7,6,6}{⼤、⽊、⼖、⼗、⾡、⼒、⼈}
  \definition*{s.}{Jogos Olímpicos, Olimpíadas}
\end{entry}

\begin{entry}{奥特曼}{ao4te4man4}{12,10,11}{⼤、⽜、⽈}
  \definition*{s.}{\emph{Ultraman},  super-herói de ficção científica japonesa}
\end{entry}

\begin{entry}{奥运}{ao4yun4}{12,7}{⼤、⾡}
  \definition*{s.}{Jogos Olímpicos, Olimpíadas, abreviação de 奥林匹克运动会}
  \seealsoref{奥林匹克运动会}{ao4lin2pi3ke4 yun4dong4hui4}
\end{entry}

\begin{entry}{奥运会}{ao4yun4hui4}{12,7,6}{⼤、⾡、⼈}
  \definition*{s.}{Jogos Olímpicos, Olimpíadas, abreviação de 奥林匹克运动会}
  \seealsoref{奥林匹克运动会}{ao4lin2pi3ke4 yun4dong4hui4}
\end{entry}

\begin{entry}{澳}{ao4}{15}{⽔}
  \definition*{s.}{Austrália, abreviação de 澳大利亚}
  \seealsoref{澳大利亚}{ao4da4li4ya4}
\end{entry}

\begin{entry}{澳大利亚}{ao4da4li4ya4}{15,3,7,6}{⽔、⼤、⼑、⼆}
  \definition*{s.}{Austrália}
\end{entry}

%%%%% EOF %%%%%


%%%
%%% B
%%%
\section*{B}
\addcontentsline{toc}{section}{B}

\begin{verbete}{八}{ba1}{2}
  \significado{num.}{8, oito}
\end{verbete}
\begin{verbete}{八八六}{ba1ba1liu4}{2;2;4}
  \significado{expr.}{Bye bye! (em salas de bate-papo e mensagens de texto)}
\end{verbete}
\begin{verbete}{巴西}{ba1xi1}{4;6}
  \significado*{s.}{Brasil}
\end{verbete}
\begin{verbete}{巴西人}{ba1xi1ren2}{4;6;2}
  \significado[个,位]{s.}{brasileiro; nascido no Brasil}
  \exemplo{他是巴西人。}[Ele é brasileiro.]
\end{verbete}
\begin{verbete}{巴西战舞}{ba1xi1zhan4wu3}{4;6;9;14}
  \significado{s.}{capoeira}
\end{verbete}
\begin{verbete}{吧}{ba1}{7}
  \significado{s.}{bar (servindo bebidas ou fornecendo acesso à \textit{Internet}); onomatopéia: Bang!}
  \significado{v.}{soprar (em um cachimbo, etc.)}
  \veja{吧}{ba5}
  \veja{吧}{bia1}
\end{verbete}
\begin{verbete}{把}{ba3}{7}
  \significado{p.c.}{para objetos com alça; para objetos pequenos: punhado}
  \significado{v.}{conter; alcançar; segurar}
  \veja{把}{ba4}
\end{verbete}
\begin{verbete}{把柄}{ba3bing3}{7;9}
  \significado{s.}{figurativo: informações que podem ser usadas contra alguém}
\end{verbete}
\begin{verbete}{把持}{ba3chi2}{7;9}
  \significado{v.}{controlar; dominar; monopolizar}
\end{verbete}
\begin{verbete}{把风}{ba3feng1}{7;4}
  \significado{v.}{estar atento; vigiar (durante uma atividade clandestina)}
\end{verbete}
\begin{verbete}{把关}{ba3guan1}{7;6}
  \significado{v.}{verificar algo}
\end{verbete}
\begin{verbete}{把脉}{ba3mai4}{7;9}
  \significado{v.}{sentir ou tomar o pulso de alguém}
\end{verbete}
\begin{verbete}{把式}{ba3shi4}{7;6}
  \significado{s.}{pessoa qualificada em um comércio}
\end{verbete}
\begin{verbete}{把守}{ba3shou3}{7;6}
  \significado{v.}{vigiar; guardar}
\end{verbete}
\begin{verbete}{把玩}{ba3wan2}{7;8}
  \significado{v.}{brincar com; mexer com}
\end{verbete}
\begin{verbete}{把稳}{ba3wen3}{7;14}
  \significado{adj.}{confiável}
\end{verbete}
\begin{verbete}{把握}{ba3wo4}{7;12}
  \significado{s.}{seguro; garantia; certeza}
  \significado{v.}{agarrar; segurar; aproveitar}
\end{verbete}
\begin{verbete}{把戏}{ba3xi4}{7;6}
  \significado{s.}{acrobacia; malabarismo; truque barato}
\end{verbete}
\begin{verbete}{把}{ba4}{7}
  \significado{v.}{lidar}
  \veja{把}{ba3}
\end{verbete}
\begin{verbete}{爸}{ba4}{8}
  \significado[个,位]{s.}{pai}
\end{verbete}
\begin{verbete}{爸爸}{ba4ba5}{8;8}
  \significado[个,位]{s.}{papai, pai (informal)}
\end{verbete}
\begin{verbete}{爸妈}{ba4ma1}{8;6}
  \significado{s.}{pai e mãe}
\end{verbete}
\begin{verbete}{罢}{ba4}{10}
  \significado{v.}{parar; cessar; demitir; suspender; desistir; terminar}
  \veja{吧}{ba5}
  \veja{罢}{ba5}
\end{verbete}
\begin{verbete}{吧}{ba5}{7}
  \significado{part.}{partícula modal indicando sugestão ou suposição; ...eu presumo.; ...OK?; ...certo?}
  \veja{吧}{ba1}
  \veja{吧}{bia1}
\end{verbete}
\begin{verbete}{罢}{ba5}{10}
  \significado{part.}{partícula final, a mesma que 吧}
  \veja{罢}{ba4}
  \veja{吧}{ba5}
\end{verbete}
\begin{verbete}{白}{bai2}{5}
  \significado{adj.}{branco; claro; puro; límpido; simples; em branco; grátis}
  \significado{adv.}{em vão; sem propósito; por nada}
  \significado{s.}{parte falada na ópera; diálogo; dialeto}
  \significado*{s.}{sobrenome Bai}
\end{verbete}
\begin{verbete}{白菜}{bai2cai4}{5;11}
  \significado[棵,个]{s.}{acelga; repolho chinês}
\end{verbete}
\begin{verbete}{白痴}{bai2chi1}{5;13}
  \significado{adj.}{imbecil}
  \significado{s.}{estúpido; imbecil}
\end{verbete}
\begin{verbete}{白蛋白}{bai2dan4bai2}{5;11;5}
  \significado{s.}{albumina}
\end{verbete}
\begin{verbete}{白鹄}{bai2hu2}{5;12}
  \significado{s.}{cisne branco}
\end{verbete}
\begin{verbete}{白拣}{bai2jian3}{5;8}
  \significado{s.}{uma escolha barata}
  \significado{v.}{escolher algo que não custa nada}
\end{verbete}
\begin{verbete}{白萝卜}{bai2luo2bo5}{5;11;2}
  \significado{s.}{rabanete branco}
\end{verbete}
\begin{verbete}{白色}{bai2se4}{5;6}
  \significado{s.}{cor branca}
\end{verbete}
\begin{verbete}{白天}{bai2tian1}{5;4}
  \significado{p.t.}{dia; de dia}
  \significado[个]{s.}{dia}
\end{verbete}
\begin{verbete}{白苋}{bai2xian4}{5;7}
  \significado{s.}{amaranto branco; brotos e folhas tenras de espinafre chinês usados como alimento}
\end{verbete}
\begin{verbete}{百}{bai3}{6}
  \significado{num.}{100, cem; centena; cento}
  \significado*{s.}{sobrenome Bai}
\end{verbete}
\begin{verbete}{百分}{bai3fen1}{6;4}
  \significado{num.}{por cento}
  \significado{s.}{porcentagem}
\end{verbete}
\begin{verbete}{搬}{ban1}{13}
  \significado{v.}{copiar indiscriminadamente; mover-se (ou seja, mudar-se); mover-se (algo relativamente pesado ou volumoso); mudar; mudar-se}
\end{verbete}
\begin{verbete}{搬动}{ban1dong4}{13;6}
  \significado{v.}{mover-se (alguma coisa); se mudar}
\end{verbete}
\begin{verbete}{搬家}{ban1jia1}{13;10}
  \significado{s.}{mudança}
  \significado{v.+compl.}{mudar-se de casa}
\end{verbete}
\begin{verbete}{搬口}{ban1kou3}{13;3}
  \significado{v.}{tagarelar; transmitir histórias (idioma); semear dissensão; contar histórias}
\end{verbete}
\begin{verbete}{搬弄}{ban1nong4}{13;7}
  \significado{v.}{causar problemas; mexer com alguém; mostrar (o que se pode fazer)}
\end{verbete}
\begin{verbete}{搬运}{ban1yun4}{13;7}
  \significado{s.}{frete; transporte}
  \significado{v.}{carregar; transportar}
\end{verbete}
\begin{verbete}{搬走}{ban1zou3}{13;7}
  \significado{v.}{carregar}
\end{verbete}
\begin{verbete}{办}{ban4}{4}
  \significado{v.}{lidar com; lidar; gerenciar; configurar}
\end{verbete}
\begin{verbete}{办法}{ban4fa3}{4;8}
  \significado[条,个]{s.}{meio (de se fazer alguma coisa); método; medida}
\end{verbete}
\begin{verbete}{办公室}{ban4gong1shi4}{4;4;9}
  \significado[间]{s.}{gabinete; escritório}
\end{verbete}
\begin{verbete}{半}{ban4}{5}
  \significado{adj.}{incompleto}
  \significado{adv.}{prefixo semi}
  \significado{num.}{(depois de um número) ``e meio''}
  \significado{s.}{metade}
\end{verbete}
\begin{verbete}{半球}{ban4qiu2}{5;11}
  \significado{s.}{hemisfério}
\end{verbete}
\begin{verbete}{半音}{ban4yin1}{5;9}
  \significado{s.}{semitom}
\end{verbete}
\begin{verbete}{帮}{bang1}{9}
  \significado{p.c.}{para alguém (como uma ajuda)}
  \significado{s.}{gangue; grupo; contratado (como trabalhador); camada externa; festa; sociedade secreta}
  \significado{v.}{ajudar; apoiar}
\end{verbete}
\begin{verbete}{帮教}{bang1jiao4}{9;11}
  \significado{v.}{orientar}
\end{verbete}
\begin{verbete}{帮佣}{bang1yong1}{9;7}
  \significado{s.}{ajudante doméstico; servo}
\end{verbete}
\begin{verbete}{帮助}{bang1zhu4}{9;7}
  \significado[种]{s.}{ajuda; assistência}
  \significado{v.}{ajudar; dar assistência}
\end{verbete}
\begin{verbete}{包}{bao1}{5}
  \significado{p.c.}{pacotes, sacos, sacolas, embrulhos}
  \significado[个,只]{s.}{bolsa; pacote; recipiente; embrulho}
  \significado{v.}{contratar; cobrir; segurar ou abraçar; incluir; assumir o comando; embrulhar}
  \significado*{s.}{sobrenome Bao}
\end{verbete}
\begin{verbete}{包办}{bao1ban4}{5;4}
  \significado{v.}{comandar todo o show; comprometer-se a fazer tudo sozinho}
\end{verbete}
\begin{verbete}{包干}{bao1gan1}{5;3}
  \significado{s.}{tarefa alocada}
  \significado{v.}{ter a responsabilidade total sobre um trabalho}
\end{verbete}
\begin{verbete}{包括}{bao1kuo4}{5;9}
  \significado{v.}{compreender; consistir em; incluir; incorporar; envolver}
\end{verbete}
\begin{verbete}{包子}{bao1zi5}{5;3}
  \significado[个]{s.}{pão recheado cozido no vapor}
\end{verbete}
\begin{verbete}{包租}{bao1zu1}{5;10}
  \significado{s.}{aluguel fixo para terras agrícolas}
  \significado{v.}{fretar; alugar; alugar um terreno ou uma casa para subarrendar}
\end{verbete}
\begin{verbete}{保存}{bao3cun2}{9;6}
  \significado{v.}{conservar; preservar; computação: salvar (um arquivo, etc.)}
\end{verbete}
\begin{verbete}{保护}{bao3hu4}{9;7}
  \significado{s.}{proteção}
  \significado{v.}{proteger; defender; salvaguardar}
\end{verbete}
\begin{verbete}{保护国}{bao3hu4guo2}{9;7;8}
  \significado{s.}{protetorado}
\end{verbete}
\begin{verbete}{保护剂}{bao3hu4ji4}{9;7;8}
  \significado{s.}{agente protetor}
\end{verbete}
\begin{verbete}{保护区}{bao3hu4qu1}{9;7;4}
  \significado{s.}{área protegida; área de conservação}
\end{verbete}
\begin{verbete}{保护色}{bao3hu4se4}{9;7;6}
  \significado{s.}{camuflagem}
\end{verbete}
\begin{verbete}{保护神}{bao3hu4shen2}{9;7;9}
  \significado{s.}{anjo da guarda; santo patrono}
\end{verbete}
\begin{verbete}{保护物}{bao3hu4·wu4}{9;7;8}
  \significado{s.}{protetor}
\end{verbete}
\begin{verbete}{保护性}{bao3hu4xing4}{9;7;8}
  \significado{s.}{proteção}
\end{verbete}
\begin{verbete}{保护者}{bao3hu4zhe3}{9;7;8}
  \significado{s.}{protetor; segurador}
\end{verbete}
\begin{verbete}{报}{bao4}{7}
  \significado[份,张]{s.}{jornal; recompensa; relatório; vingança}
  \significado{v.}{anunciar; informar}
\end{verbete}
\begin{verbete}{报酬}{bao4chou5}{7;13}
  \significado{s.}{recompensa; remuneração}
\end{verbete}
\begin{verbete}{报纸}{bao4zhi3}{7;7}
  \significado[张]{s.}{jornal; diário}
\end{verbete}
\begin{verbete}{暴力}{bao4li4}{15;2}
  \significado{adj.}{violento}
  \significado{s.}{violência}
\end{verbete}
\begin{verbete}{暴雨}{bao4yu3}{15;8}
  \significado[场,阵]{s.}{tempestade; chuva torrencial}
\end{verbete}
\begin{verbete}{杯}{bei1}{8}
  \significado{p.c.}{para certos recipientes de líquidos: copo, xícara, etc.}
  \significado{s.}{copo; taça; xícara; copa troféu}
\end{verbete}
\begin{verbete}{杯具}{bei1ju4}{8;8}
  \significado{s.}{parachoque; fiasco; gíria: tragédia}
\end{verbete}
\begin{verbete}{杯子}{bei1zi5}{8;3}
  \significado[个,只]{s.}{copo; caneca; xícara; taça}
\end{verbete}
\begin{verbete}{背}{bei1}{9}
  \significado{v.}{estar sobrecarregado; carregar nas costas ou no ombro}
  \veja{背}{bei4}
\end{verbete}
\begin{verbete}{㮎}{bei1}{13}
  \variante{杯}{bei1}
\end{verbete}
\begin{verbete}{北}{bei3}{5}
  \significado{p.d.l.}{norte}
  \significado{v.}{ser derrotado (clássico)}
\end{verbete}
\begin{verbete}{北边}{bei3bian5}{5;5}
  \significado{p.l.}{lado norte; ao norte de}
\end{verbete}
\begin{verbete}{北方}{bei3fang1}{5;4}
  \significado{p.l.}{norte; a parte norte de um país}
\end{verbete}
\begin{verbete}{北京}{bei3jing1}{5;8}
  \significado*{s.}{Beijing (Pequim); Capital da China}
\end{verbete}
\begin{verbete}{北面}{bei3mian4}{5;9}
  \significado{p.l.}{lado norte}
\end{verbete}
\begin{verbete}{背}{bei4}{9}
  \significado{p.l.}{a parte de trás de um corpo ou objeto}
  \significado{s.}{costas; gíria: azarado}
  \significado{v.}{esconder algo de; decorar; recitar de memória; virar as costas}
  \veja{背}{bei1}
\end{verbete}
\begin{verbete}{被}{bei4}{10}
  \significado{prep.}{por}
\end{verbete}
\begin{verbete}{被单}{bei4dan1}{10;8}
  \significado[床]{s.}{lençol}
\end{verbete}
\begin{verbete}{被动}{bei4dong4}{10;6}
  \significado{adj.}{passivo}
\end{verbete}
\begin{verbete}{被告}{bei4gao4}{10;7}
  \significado{s.}{réu}
\end{verbete}
\begin{verbete}{被迫}{bei4po4}{10;8}
  \significado{v.}{ser compelido; ser forçado}
\end{verbete}
\begin{verbete}{被窝}{bei4wo1}{10;12}
  \significado{s.}{colcha}
\end{verbete}
\begin{verbete}{被子}{bei4zi5}{10;3}
  \significado[床]{s.}{colcha}
\end{verbete}
\begin{verbete}{本}{ben3}{5}
  \significado{adv.}{inerente; originalmente}
  \significado{p.c.}{para livros, dicionários, periódicos, arquivos, etc.}
  \significado{s.}{origem; fonte; raiz}
\end{verbete}
\begin{verbete}{本子}{ben3zi5}{5;3}
  \significado[本]{s.}{caderno}
\end{verbete}
\begin{verbete}{笨蛋}{ben4dan4}{11;11}
  \significado{s.}{bobalhão; cabeça-oca; cabeça-dura}
  \significado{v.}{iludir; enganar}
\end{verbete}
\begin{verbete}{甭}{beng2}{9}
  \significado{v.o.}{contração de 不用; não precisar}
  \veja{不用}{bu2yong4}
\end{verbete}
\begin{verbete}{鼻子}{bi2zi5}{14;3}
  \significado[个,只]{s.}{nariz}
\end{verbete}
\begin{verbete}{比}{bi3}{4}
  \significado{part.}{partícula usada para comparação (superioridade)}
  \significado{prep.}{que; do que}
  \significado{s.}{razão (taxa)}
  \significado{v.}{comparar; contrastar; gesticular (com as mãos)}
\end{verbete}
\begin{verbete}{比较}{bi3jiao4}{4;10}
  \significado{adv.}{comparativamente; relativamente}
  \significado{s.}{comparação; relativamente}
  \significado{v.}{comparar; contrastar}
\end{verbete}
\begin{verbete}{比萨饼}{bi3sa4bing3}{4;11;9}
  \significado[张]{s.}{pizza}
\end{verbete}
\begin{verbete}{比赛}{bi3sai4}{4;14}
  \significado[场,次]{s.}{competição; concurso}
  \significado{v.}{competir}
\end{verbete}
\begin{verbete}{笔}{bi3}{10}
  \significado{p.c.}{para somas de dinheiro, negócios}
  \significado[支,枝]{s.}{caneta; lápis}
\end{verbete}
\begin{verbete}{闭嘴}{bi4zui3}{6;16}
  \significado{expr.}{Cale-se!}
\end{verbete}
\begin{verbete}{壁纸}{bi4zhi3}{16;7}
  \significado{s.}{papel de parede}
\end{verbete}
\begin{verbete}{吧}{bia1}{7}
  \significado{s.}{bar (servindo bebidas ou fornecendo acesso à \textit{Internet}); onomatopéia: Smack! (para beijo)}
  \significado{v.}{soprar (em um cachimbo, etc.)}
  \veja{吧}{ba1}
  \veja{吧}{ba5}
\end{verbete}
\begin{verbete}{边}{bian1}{5}
  \significado{adv.}{simultaneamente}
  \significado[个]{s.}{fronteira; limite; borda; margem; lado}
  \veja{边}{bian5}
\end{verbete}
\begin{verbete}{编程}{bian1cheng2}{12;12}
  \significado{s.}{programa de computador}
  \significado{v.}{programar computador}
\end{verbete}
\begin{verbete}{邉}{bian1}{17}
  \variante{边}{bian1}
\end{verbete}
\begin{verbete}{变}{bian4}{8}
  \significado{v.}{mudar; transformar; variar}
\end{verbete}
\begin{verbete}{变更}{bian4geng1}{8;7}
  \significado{v.}{alterar; mudar; modificar}
\end{verbete}
\begin{verbete}{变节}{bian4jie2}{8;5}
  \significado{s.}{traição; deserção; vira-casaca}
  \significado{v.}{mudar de lado politicamente}
\end{verbete}
\begin{verbete}{变迁}{bian4qian1}{8;6}
  \significado{s.}{mudanças; vicissitudes}
\end{verbete}
\begin{verbete}{变数}{bian4shu4}{8;13}
  \significado{s.}{matemática: variável}
\end{verbete}
\begin{verbete}{变异}{bian4yi4}{8;6}
  \significado{s.}{variação; mutação}
\end{verbete}
\begin{verbete}{遍}{bian4}{12}
  \significado{p.l.}{em todos os lugares; por toda parte}
  \significado{p.c.}{para a repetição de ações de leitura, fala ou escrita}
\end{verbete}
\begin{verbete}{边}{bian5}{5}
  \significado{s.}{sufixo de uma palavra de localidade}
  \veja{边}{bian1}
\end{verbete}
\begin{verbete}{标准}{biao1zhun3}{9;10}
  \significado{adj.}{criterioso; padronizado; normatizado}
  \significado[个]{s.}{critério; padrão (oficial); norma}
\end{verbete}
\begin{verbete}{表演}{biao3yan3}{8;14}
  \significado[发,场]{s.}{representação; atuação}
  \significado{v.}{representar; atuar}
\end{verbete}
\begin{verbete}{表演赛}{biao3yan3sai4}{8;14;14}
  \significado{s.}{partida ou jogo de exibição}
\end{verbete}
\begin{verbete}{表演特技}{biao3yan3·te4ji4}{8;14;10;7}
  \significado{s.}{acrobacia; pirueta; façanha}
\end{verbete}
\begin{verbete}{表演艺术}{biao3yan3·yi4shu4}{8;14;4;5}
  \significado{s.}{arte performática}
\end{verbete}
\begin{verbete}{表演游戏}{biao3yan3·you2xi4}{8;14;12;6}
  \significado{s.}{exibição dramática}
\end{verbete}
\begin{verbete}{表演者}{biao3yan3·zhe3}{8;14;8}
  \significado{s.}{ator}
\end{verbete}
\begin{verbete}{表扬}{biao3yang2}{8;6}
  \significado{v.}{elogiar; louvar}
\end{verbete}
\begin{verbete}{表扬信}{biao3yang2·xin4}{8;6;9}
  \significado{s.}{carta de elogio; depoimento}
\end{verbete}
\begin{verbete}{别}{bie2}{7}
  \significado{adv.}{nada de (pedir a alguém para não fazer); não}
  \significado{pron.}{outro}
  \significado{v.}{classificar; separar; distinguir; partir; deixar; fixar; colar alguma coisa em}
  \significado*{s.}{sobrenome Bie}
  \veja{别}{bie4}
\end{verbete}
\begin{verbete}{别的}{bie2de5}{7;8}
  \significado{pron.}{outro}
\end{verbete}
\begin{verbete}{别人}{bie2ren5}{7;2}
  \significado{pron.}{outra pessoa; outro povo; outros}
\end{verbete}
\begin{verbete}{别}{bie4}{7}
  \significado{v.}{fazer com que alguém mude seus hábitos, opiniões, etc.}
  \veja{别}{bie2}
\end{verbete}
\begin{verbete}{宾馆}{bin1guan3}{10;11}
  \significado[个,家]{s.}{casa de hóspedes; hotel}
\end{verbete}
\begin{verbete}{冰}{bing1}{6}
  \significado{adj.}{hostil; gelado}
  \significado[块]{s.}{gelo; gíria: metanfetamina}
  \significado{v.}{sentir frio; relaxar algo}
\end{verbete}
\begin{verbete}{冰球}{bing1qiu2}{6;11}
  \significado{s.}{hóquei no gelo}
\end{verbete}
\begin{verbete}{冰天雪地}{bing1tian1-xue3di4}{6;4;11;6}
  \significado{expr.}{um mundo de gelo e neve}
\end{verbete}
\begin{verbete}{病}{bing4}{10}
  \significado[场]{s.}{doença}
  \significado{v.}{adoecer; estar doente}
\end{verbete}
\begin{verbete}{拨转}{bo1zhuan3}{8;8}
  \significado{v.}{transferir (fundos, etc.); virar; dar a volta}
\end{verbete}
\begin{verbete}{啵}{bo1}{11}
  \significado{s.}{onomatopéia: borbulhar}
  \veja{啵}{bo5}
\end{verbete}
\begin{verbete}{菠菜}{bo1cai4}{11;11}
  \significado[棵]{s.}{espinafre}
\end{verbete}
\begin{verbete}{脖子}{bo2zi5}{11;3}
  \significado[个]{s.}{pescoço}
\end{verbete}
\begin{verbete}{博物馆}{bo2wu4guan3}{12;8;11}
  \significado{s.}{museu}
\end{verbete}
\begin{verbete}{啵}{bo5}{11}
  \significado{part.}{partícula gramaticalmente equivalente a 吧}
  \veja{吧}{ba5}
  \veja{啵}{bo1}
\end{verbete}
\begin{verbete}{不}{bu2}[ (antes de quarto tom)]{4}
  \significado{adv.}{não}
  \veja{不}{bu4}
  \veja{不}{bu5}
\end{verbete}
\begin{verbete}{不错}{bu2cuo4}{4;13}
  \significado{adj.}{correto; não (é) mau; bastante bom; certo}
\end{verbete}
\begin{verbete}{不过}{bu2guo4}{4;6}
  \significado{conj.}{mas; contudo; no entanto}
\end{verbete}
\begin{verbete}{不客气}{bu2ke4qi5}{4;9;4}
  \significado{expr.}{de nada; não há de que}
\end{verbete}
\begin{verbete}{不是话}{bu2shi4hua4}{4;9;8}
  \veja{不像话}{bu2xiang4hua4}
\end{verbete}
\begin{verbete}{不成话}{bu2xheng2hua4}{4;6;8}
  \veja{不像话}{bu2xiang4hua4}
\end{verbete}
\begin{verbete}{不像话}{bu2xiang4hua4}{4;13;8}
  \significado{expr.}{sem razão; demasiado irracionável}
\end{verbete}
\begin{verbete}{不要}{bu2yao4}{4;9}
  \significado{adv.}{nada de (pedir a alguém para não fazer); não}
\end{verbete}
\begin{verbete}{不用}{bu2yong4}{4;5}
  \significado{v.o.}{não precisar}
  \veja{甭}{beng2}
\end{verbete}
\begin{verbete}{不大离}{bu2da4li2}{9;4;10}
  \significado{adj.}{bem perto; quase certo; nada mal}
\end{verbete}
\begin{verbete}{不}{bu4}{4}
  \significado{adv.}{não}
  \veja{不}{bu2}
  \veja{不}{bu5}
\end{verbete}
\begin{verbete}{不同}{bu4tong2}{4;6}
  \significado{adj.}{diferente; distinto}
\end{verbete}
\begin{verbete}{布署}{bu4shu3}{5;13}
  \variante{部署}{bu4shu3}
\end{verbete}
\begin{verbete}{部}{bu4}{10}
  \significado{p.c.}{para obras de literatura, filmes, máquinas etc.}
  \significado[根]{s.}{departamento; divisão; ministério; seção; parte; tropas}
\end{verbete}
\begin{verbete}{部分}{bu4fen5}{10;4}
  \significado[个]{s.}{parte; parte de; uma parte de; pedaço; secção}
\end{verbete}
\begin{verbete}{部门}{bu4men2}{10;3}
  \significado[个]{s.}{filial; departamento; divisão; seção}
\end{verbete}
\begin{verbete}{部属}{bu4shu3}{10;12}
  \significado{s.}{afiliado a um ministério; subordinado; tropas sob comando de alguém}
\end{verbete}
\begin{verbete}{部署}{bu4shu3}{10;13}
  \significado{s.}{implantação}
  \significado{v.}{implantar}
\end{verbete}
\begin{verbete}{部下}{bu4xia4}{10;3}
  \significado{s.}{subordinado; tropas sob comando de alguém}
\end{verbete}
\begin{verbete}{部族}{bu4zu2}{10;11}
  \significado{adj.}{tribal}
  \significado{s.}{tribo}
\end{verbete}
\begin{verbete}{不}{bu5}{4}
  \significado{adv.}{não (em expressões ``v.$+$不$+$v.'')}
  \veja{不}{bu2}
  \veja{不}{bu4}
\end{verbete}

%%%%% EOF %%%%%

%%%
%%% C
%%%
\section*{C}
\addcontentsline{toc}{section}{C}

\begin{verbete}{才}{cai2}{3}
  \significado{adv.}{apenas (seguido por uma cláusula numérica); só (indicando que algo está acontecendo mais tarde do que o esperado); não até (precedido por uma cláusula de condição ou razão); há um momento atrás}
  \significado{conj.}{apenas quando}
  \significado{s.}{um indivíduo capaz; habilidade; talento}
\end{verbete}

\begin{verbete}{才略}{cai2lve4}{3;11}
  \significado{s.}{habilidade e sagacidade}
\end{verbete}

\begin{verbete}{菜}{cai4}{11}
  \significado[棵]{s.}{hortaliça; verdura}
  \significado[样,道,盘]{s.}{prato (de comida)}
\end{verbete}

\begin{verbete}{菜单}{cai4dan1}{11;8}
  \significado[份,张]{s.}{menu; cardápio}
\end{verbete}

\begin{verbete}{参观}{can1guan3}{8;6}
  \significado{v.}{visitar}
\end{verbete}

\begin{verbete}{参加}{can1jia1}{8;5}
  \significado{v.}{participar em; tomar parte em; assistir}
\end{verbete}

\begin{verbete}{餐厅}{can1ting1}{16;4}
  \significado[家]{s.}{restaurante}
  \significado[间]{s.}{sala de jantar}
\end{verbete}

\begin{verbete}{蚕纸}{can2zhi3}{10;7}
  \significado{s.}{papel em que o bicho-da-seda põe seus ovos}
\end{verbete}

\begin{verbete}{草}{cao3}{9}
  \significado[棵,撮,株,根]{s.}{erva; grama}
\end{verbete}

\begin{verbete}{草地}{cao3di4}{9;6}
  \significado[片]{s.}{relva; pastagem}
\end{verbete}

\begin{verbete}{草纸}{cao3zhi3}{9;7}
  \significado{s.}{papel pardo; pergaminho; papel de palha áspero; papel higiênico}
\end{verbete}

\begin{verbete}{厕所}{ce4suo3}{8;8}
  \significado[间,处]{s.}{sanitário; toilette}
\end{verbete}

\begin{verbete}{厕纸}{ce4zhi3}{8;7}
  \significado{s.}{papel higiênico}
\end{verbete}

\begin{verbete}{层}{ceng2}{7}
  \significado{p.c.}{para andar, piso}
\end{verbete}

\begin{verbete}{茶}{cha2}{9}
  \significado[杯,壶]{s.}{chá}
\end{verbete}

\begin{verbete}{差不多}{cha4bu5duo1}{9;4;6}
  \significado{adj.}{mais ou menos}
\end{verbete}

\begin{verbete}{差点儿}{cha4dian3r5}{9;9;2}
  \significado{adv.}{por pouco; por um triz; quase}
\end{verbete}

\begin{verbete}{拆}{chai1}{8}
  \significado{v.}{remover; tirar do seu lugar; desfazer; desmontar}
\end{verbete}

\begin{verbete}{长}{chang2}{4}
  \significado{adj.}{comprido; longo}
  \veja{长}{zhang3}
\end{verbete}

\begin{verbete}{长成}{chang2cheng2}{4;6}
  \significado*{s.}{Grande Muralha}
\end{verbete}

\begin{verbete}{常常}{chang2chang2}{11;11}
  \significado{adv.}{frequentemente; com frequência}
\end{verbete}

\begin{verbete}{常问问题}{chang2wen4wen4ti2}{11;6;6;15}
  \significado{s.}{FAQ; perguntas frequentes}
\end{verbete}

\begin{verbete}{场}{chang3}{6}
  \significado{p.c.}{para número de exames; para atividades esportivas ou recreativas}
  \significado{s.}{local grande usado para um propósito específico; cena (de uma peça); palco}
\end{verbete}

\begin{verbete}{唱}{chang4}{11}
  \significado{v.}{cantar}
\end{verbete}

\begin{verbete}{唱歌}{chang4ge1}{11;14}
  \significado{v.+compl.}{cantar}
\end{verbete}

\begin{verbete}{超市}{chao1shi4}{12;5}
  \significado[家]{s.}{supermercado}
\end{verbete}

\begin{verbete}{吵}{chao3}{7}
  \significado{adj.}{barulhento; ruidoso}
\end{verbete}

\begin{verbete}{吵架}{chao3jia4}{7;9}
  \significado{v.+compl.}{brigar; ralhar; zangar-se}
\end{verbete}

\begin{verbete}{炒}{chao3}{8}
  \significado{v.}{saltear; demitir (alguém)}
\end{verbete}

\begin{verbete}{车}{che1}{4}
  \significado[辆]{s.}{carro; veículo; viatura}
  \significado*{s.}{sobrenome Che}
  \veja{车}{ju1}
\end{verbete}

\begin{verbete}{车次}{che1ci4}{4;6}
  \significado{s.}{número do trem}
\end{verbete}

\begin{verbete}{车库}{che1ku4}{4;7}
  \significado{s.}{garagem}
\end{verbete}

\begin{verbete}{车牌}{che1pai2}{4;12}
  \significado{s.}{matrícula; placa de carro}
\end{verbete}

\begin{verbete}{车水马龙}{che1shui3-ma3long2}{4;4;3;5}
  \significado{expr.}{tráfego engarrafado; engarrafamento; literalmente: ``fluxo interminável de cavalos e carruagens''}
\end{verbete}

\begin{verbete}{车站}{che1zhan4}{4;10}
  \significado[处,个]{s.}{estação; ponto de ônibus}
\end{verbete}

\begin{verbete}{衬衫}{chen4shan1}{8;8}
  \significado[件]{s.}{camisa; blusa}
\end{verbete}

\begin{verbete}{成}{cheng2}{6}
  \significado*{s.}{sobrenome Cheng}
  \significado{v.}{sair-se bem; ser bem sucedido}
\end{verbete}

\begin{verbete}{成都}{cheng2du1}{6;10}
  \significado*{s.}{Chengdu}
\end{verbete}

\begin{verbete}{成婚}{cheng2hun1}{6;11}
  \significado{v.}{casar-se}
\end{verbete}

\begin{verbete}{成活}{cheng2huo2}{6;9}
  \significado{v.}{sobreviver}
\end{verbete}

\begin{verbete}{成绩}{cheng2ji4}{6;11}
  \significado[项,个]{s.}{nota; classificação}
\end{verbete}

\begin{verbete}{成家}{cheng2jia1}{6;10}
  \significado{v.}{tornar-se um especialista reconhecido; estabelecer-se e casar-se (de um homem)}
\end{verbete}

\begin{verbete}{成批}{cheng2pi1}{6;7}
  \significado{s.}{em lotes; a granel}
\end{verbete}

\begin{verbete}{成器}{cheng2qi4}{6;16}
  \significado{v.}{tornar-se uma pessoa digna de respeito; fazer algo de si mesmo}
\end{verbete}

\begin{verbete}{成色}{cheng2se4}{6;6}
  \significado{v.}{sair-se bem; ser bem sucedido}
\end{verbete}

\begin{verbete}{成为}{cheng2wei2}{6;4}
  \significado{s.}{tornar-se; transformar-se em}
\end{verbete}

\begin{verbete}{诚实}{cheng2shi2}{8;8}
  \significado{adj.}{honesto}
\end{verbete}

\begin{verbete}{诚实地}{cheng2shi2·di5}{8;8;6}
  \significado{adv.}{honestamente}
\end{verbete}

\begin{verbete}{城市}{cheng2shi4}{9;5}
  \significado[座]{s.}{cidade}
\end{verbete}

\begin{verbete}{乘客}{cheng2ke2}{10;9}
  \significado{s.}{passageiro}
\end{verbete}

\begin{verbete}{乘客数}{cheng2ke2·shu4}{10;9;13}
  \significado{s.}{número de passageiros}
\end{verbete}

\begin{verbete}{惩处}{cheng2chu3}{12;5}
  \significado{v.}{administrar justiça; punir}
\end{verbete}

\begin{verbete}{惩罚}{cheng2fa2}{12;9}
  \significado{v.}{punir; penalizar}
\end{verbete}

\begin{verbete}{程控}{cheng2kong4}{12;11}
  \significado{s.}{programado; sob controle automático}
\end{verbete}

\begin{verbete}{程序}{cheng2xu4}{12;7}
  \significado{s.}{procedimento; sequência; ordem; programa de computador}
\end{verbete}

\begin{verbete}{程序库}{cheng2xu4ku4}{12;7;7}
  \significado{s.}{biblioteca de funções e procedimentos para programas de computador}
\end{verbete}

\begin{verbete}{程序设计}{cheng2xu4she4ji4}{12;7;6;4}
  \significado{s.}{programação de computadores}
\end{verbete}

\begin{verbete}{橙色}{cheng2se4}{16;6}
  \significado{s.}{cor de laranja}
\end{verbete}

\begin{verbete}{橙汁}{cheng2zhi1}{16;5}
  \significado[瓶,杯,罐,盒]{s.}{suco de laranja}
  \veja{橘子汁}{ju2zi5zhi1}
  \veja{柳橙汁}{liu3cheng2zhi1}
\end{verbete}

\begin{verbete}{吃}{chi1}{6}
  \significado{v.}{comer}
\end{verbete}

\begin{verbete}{吃屎}{chi1·shi3}{6;9}
  \significado{expr.}{Coma merda!}
\end{verbete}

\begin{verbete}{迟到}{chi1dao4}{7;8}
  \significado{v.}{chegar atrasado; tardar}
\end{verbete}

\begin{verbete}{斥骂}{chi4ma4}{5;9}
  \significado{v.}{repreender}
\end{verbete}

\begin{verbete}{憧憬}{chong1jing3}{15;15}
  \significado{v.}{ansiar por; esperar por}
\end{verbete}

\begin{verbete}{重}{chong2}{9}
  \significado{adv.}{de novo}
  \significado{p.c.}{camadas}
  \significado{s.}{repetição}
  \significado{v.}{repetir}
  \veja{重}{zhong4}
\end{verbete}

\begin{verbete}{重重}{chong2chong2}{9;9}
  \significado{adv.}{camada após camada; um após o outro}
  \veja{重重}{zhong4zhong4}
\end{verbete}

\begin{verbete}{重迭}{chong2die2}{9;8}
  \significado{s.}{sobreposição; redundância}
  \significado{v.}{duplicar; sobrepor}
\end{verbete}

\begin{verbete}{重阳节}{chong2yang1jie2}{9;6;5}
  \significado*{s.}{Festa do Duplo Nove, Festival Yang, dia de subir aos lugares mais altos para evitar calamidades e dia do culto aos antepassados (9º dia do nono mês lunar)}
\end{verbete}

\begin{verbete}{宠物}{chong3wu4}{8;8}
  \significado{s.}{animal de estimação}
\end{verbete}

\begin{verbete}{酬劳}{chou2lao2}{13;7}
  \significado{s.}{recompensa}
\end{verbete}

\begin{verbete}{臭}{chou4}{10}
  \significado{adj.}{fétido; repulsivo; repugnante; malcheiroso}
  \significado{s.}{fedor}
  \significado{v.}{feder}
  \veja{臭}{xiu4}
\end{verbete}

\begin{verbete}{臭气}{chou4qi4}{10;4}
  \significado{s.}{fedor}
\end{verbete}

\begin{verbete}{殠}{chou4}{14}
  \variante{臭}{chou4}
\end{verbete}

\begin{verbete}{出}{chu1}{5}
  \significado{p.c.}{para dramas, peças, óperas, etc.}
  \significado{v.d.}{sair; ir para fora; vir para fora}
\end{verbete}

\begin{verbete}{出版}{chu1ban3}{5;8}
  \significado{v.}{publicar; editar}
\end{verbete}

\begin{verbete}{出版社}{chu1ban3she4}{5;8;7}
  \significado{s.}{editora}
\end{verbete}

\begin{verbete}{出发}{chu1fa1}{5;5}
  \significado{v.}{partir; começar (uma jornada)}
\end{verbete}

\begin{verbete}{出口}{chu1kou3}{5;3}
  \significado[个]{s.}{exportação}
  \significado{v.}{exportar}
\end{verbete}

\begin{verbete}{出来}{chu1lai5}{5;7}
  \significado{v.d.}{sair; vir para fora (para a minha localização)}
\end{verbete}

\begin{verbete}{出去}{chu1qu5}{5;5}
  \significado{v.d.}{sair; ir para fora (a partir da minha localização)}
\end{verbete}

\begin{verbete}{出站}{chu1·zhan4}{5;10}
  \significado{s.}{saída da estação}
\end{verbete}

\begin{verbete}{出租}{chu1zu1}{5;10}
  \significado{v.}{alugar; arrendar}
\end{verbete}

\begin{verbete}{出租车}{chu1zu1che1}{5;10;4}
  \significado{s.}{táxi}
  \veja{出租汽车}{chu1zu1qi4che1}
\end{verbete}

\begin{verbete}{出租汽车}{chu1zu1qi4che1}{5;10;7;4}
  \significado[辆]{s.}{táxi}
  \veja{出租车}{chu1zu1che1}
\end{verbete}

\begin{verbete}{出租司机}{chu1zu1si1ji1}{5;10;5;6}
  \significado{s.}{motorista de táxi}
\end{verbete}

\begin{verbete}{厨房}{chu2fang2}{12;8}
  \significado[间]{s.}{cozinha}
\end{verbete}

\begin{verbete}{穿}{chuan1}{9}
  \significado{v.}{vestir}
\end{verbete}

\begin{verbete}{传真}{chuan2zhen1}{6;10}
  \significado{s.}{fax, facsímile}
\end{verbete}

\begin{verbete}{船}{chuan2}{11}
  \significado[条,艘,只]{s.}{barco; navio}
\end{verbete}

\begin{verbete}{床}{chuang2}{7}
  \significado{p.c.}{para camas}
  \significado[张]{s.}{cama}
\end{verbete}

\begin{verbete}{春天}{chun1tian1}{9;4}
  \significado[个]{p.t./s.}{primavera}
\end{verbete}

\begin{verbete}{绰号}{chuo4hao4}{11;5}
  \significado{s.}{apelido}
\end{verbete}

\begin{verbete}{词典}{ci2dian3}{7;8}
  \significado[部,本]{s.}{dicionário}
  \veja{字典}{zi4dian3}
\end{verbete}

\begin{verbete}{辞典}{ci2dian3}{13;8}
  \variante{词典}{ci2dian3}
\end{verbete}

\begin{verbete}{磁带}{ci2dai4}{14;9}
  \significado[盘,盒]{s.}{cassete; fita magnética}
\end{verbete}

\begin{verbete}{磁盘}{ci2pan2}{14;11}
  \significado{s.}{disquete}
\end{verbete}

\begin{verbete}{次}{ci4}{6}
  \significado{p.c.}{para frequência (número de vezes)}
\end{verbete}

\begin{verbete}{葱}{cong1}{12}
  \significado{s.}{cebolinha}
\end{verbete}

\begin{verbete}{聪慧}{cong1hui4}{15;15}
  \significado{adj.}{inteligente; brilhante}
\end{verbete}

\begin{verbete}{聪明}{cong1ming5}{15;8}
  \significado{adj.}{inteligente; brilhante; esperto}
\end{verbete}

\begin{verbete}{从}{cong2}{4}
  \significado{prep.}{de; desde; a partir de}
  \significado*{s.}{sobrenome Cong}
\end{verbete}

\begin{verbete}{粗心}{cu1xin1}{11;4}
  \significado{adj.}{descuido}
\end{verbete}

\begin{verbete}{粗心地做}{cu1xin1·di4·zuo4}{11;4;6;11}
  \significado{adj.}{feito descuidadamente}
\end{verbete}

\begin{verbete}{酢}{cu4}{12}
  \variante{醋}{cu4}
\end{verbete}

\begin{verbete}{醋}{cu4}{15}
  \significado{s.}{vinagre}
\end{verbete}

\begin{verbete}{窾}{cuan4}{17}
  \significado{v.}{esconder}
  \veja{窾}{kuan3}
\end{verbete}

\begin{verbete}{错}{cuo4}{13}
  \significado{adj.}{errado; enganado}
  \significado*{s.}{sobrenome Cuo}
\end{verbete}

%%%%% EOF %%%%%

%%%
%%% D
%%%
%\section*{D}
\addcontentsline{toc}{section}{D}

\begin{verbete}{搭配}{da1pei4}{12;10}
  \significado{v.}{emparelhar; combinar; organizar em pares; adicionar alguém em um grupo}
\end{verbete}

\begin{verbete}{搭讪}{da1shan4}{12;5}
  \significado{v.}{bater em alguém; incitar uma conversa; começar a conversar para acabar com um silêncio constrangedor ou uma situação embaraçosa}
\end{verbete}

\begin{verbete}{打}{da2}{5}[Radical 手][Componentes ⺘丁]
  \significado{s.}{(empréstimo linguístico) dúzia}
  \veja{打}{da3}
\end{verbete}

\begin{verbete}{答案}{da2'an4}{12;10}
  \significado[个]{s.}{resposta; solução}
\end{verbete}

\begin{verbete}{打}{da3}{5}[Radical 手][Componentes ⺘丁]
  \significado{adv.}{desde}
  \significado{v.}{jogar (um jogo); bater; atacar; acertar; quebrar; digitar; misturar; construir; lutar; pegar; fazer; amarrar; atirar; calcular}
  \veja{打}{da2}
\end{verbete}

\begin{verbete}{打扮}{da3ban5}{5;7}
  \significado{v.}{arranjar-se; enfeitar-se}
\end{verbete}

\begin{verbete}{打电话}{da3dian4hua4}{5;5;8}
  \significado{v.}{ligar; dar um telefonema}
  \veja{给……打电话}{gei3 da3dian4hua4}
\end{verbete}

\begin{verbete}{打工}{da3gong1}{5;3}
  \significado{v.}{(para alunos) ter um emprego fora do horário de aula ou durante as férias; trabalhar em um emprego temporá rio ou casual}
\end{verbete}

\begin{verbete}{打工人}{da3gong1ren2}{5;3;2}
  \significado{s.}{trabalhador}
\end{verbete}

\begin{verbete}{打搅}{da3jiao3}{5;12}
  \significado{v.}{perturbar; incomodar}
\end{verbete}

\begin{verbete}{打结}{da3jie2}{5;9}
  \significado{v.}{dar um nó; amarrar}
\end{verbete}

\begin{verbete}{打瞌睡}{da3ke1shui4}{5;15;13}
  \significado{v.}{cochilar}
\end{verbete}

\begin{verbete}{打猎}{da3lie4}{5;11}
  \significado{v.}{ir caçar}
\end{verbete}

\begin{verbete}{打骂}{da3ma4}{5;9}
  \significado{v.}{bater e repreender}
\end{verbete}

\begin{verbete}{打磨}{da3mo2}{5;16}
  \significado{v.}{polir; fazer brilhar}
\end{verbete}

\begin{verbete}{打屁股}{da3pi4gu5}{5;7;8}
  \significado{v.}{dar um tapa no bumbum de alguém}
\end{verbete}

\begin{verbete}{打球}{da3qiu2}{5;11}
  \significado{v.}{jogar bola; jogar (futebol, basquetebol, handbol, etc.)}
\end{verbete}

\begin{verbete}{打扰}{da3rao3}{5;7}
  \significado{v.}{perturbar; incomodar}
\end{verbete}

\begin{verbete}{打算}{da3suan4}{5;14}
  \significado[个]{s.}{plano; intenção}
  \significado{v.}{pensar; planejar; pretender}
\end{verbete}

\begin{verbete}{打压}{da3ya1}{5;6}
  \significado{v.}{reprimir; derrotar}
\end{verbete}

\begin{verbete}{打针}{da3zhen1}{5;7}
  \significado{v.+compl.}{dar injeção; levar injeção}
\end{verbete}

\begin{verbete}{大}{da4}{3}[Radical ⼈][Componentes ⼈⼀][Kangxi 37]
  \significado{adj.}{grande}
  \veja{大}{dai4}
\end{verbete}

\begin{verbete}{大胆}{da4dan3}{3;9}
  \significado{adj.}{audacioso; ousado; destemido}
\end{verbete}

\begin{verbete}{大豆}{da4dou4}{3;7}
  \significado{s.}{soja}
\end{verbete}

\begin{verbete}{大夫}{da4fu1}{3;4}
  \significado{s.}{oficial sênior (na China Imperial)}
  \veja{大夫}{dai4fu5}
\end{verbete}

\begin{verbete}{大概}{da4gai4}{3;13}
  \significado{adv.}{aproximadamente; por volta de}
\end{verbete}

\begin{verbete}{大规模}{da4gui1mo2}{3;8;14}
  \significado{adj.}{em grande escala; em larga escala; extenso}
\end{verbete}

\begin{verbete}{大海}{da4hai3}{3;10}
  \significado{s.}{mar; oceano}
\end{verbete}

\begin{verbete}{大后天}{da4hou4tian1}{3;6;4}
  \significado{p.t.}{daqui a três dias}
\end{verbete}

\begin{verbete}{大家}{da4jia1}{3;10}
  \significado{pron.}{todos}
\end{verbete}

\begin{verbete}{大口}{da4kou3}{3;3}
  \significado{s.}{grande bocado (de comida, bebida, fumo, etc.)}
\end{verbete}

\begin{verbete}{大马}{da4ma3}{3;3}
  \significado*{s.}{Malásia}
\end{verbete}

\begin{verbete}{大前天}{da4qian2tian1}{3;9;4}
  \significado{p.t.}{três dias atrás}
\end{verbete}

\begin{verbete}{大全}{da4quan2}{3;6}
  \significado{s.}{coleção abrangente}
\end{verbete}

\begin{verbete}{大人}{da4ren5}{3;2}
  \significado{s.}{adulto}
\end{verbete}

\begin{verbete}{大赛}{da4sai4}{3;14}
  \significado{s.}{grande concurso, competição}
\end{verbete}

\begin{verbete}{大神}{da4shen2}{3;9}
  \significado{s.}{deidade; (gíria da Internet) guru; \emph{expert}; gênio}
\end{verbete}

\begin{verbete}{大蒜}{da4suan4}{3;13}
  \significado[瓣,头]{s.}{alho}
\end{verbete}

\begin{verbete}{大腿}{da4tui3}{3;13}
  \significado{s.}{coxa}
\end{verbete}

\begin{verbete}{大戏}{da4xi4}{3;6}
  \significado*{s.}{Drama, Ópera Chinesa}
\end{verbete}

\begin{verbete}{大猩猩}{da4xing1xing5}{3;12;12}
  \significado{s.}{gorila}
\end{verbete}

\begin{verbete}{大学}{da4xue2}{3;8}
  \significado[所]{s.}{faculdade; universidade}
\end{verbete}

\begin{verbete}{大洋洲}{da4yang2zhou1}{3;9;9}
  \significado*{s.}{Oceania}
\end{verbete}

\begin{verbete}{大雨}{da4yu3}{3;8}
  \significado[场]{s.}{chuva pesada, forte}
\end{verbete}

\begin{verbete}{大约}{da4yue1}{3;6}
  \significado{adv.}{aproximadamente; provavelmente}
\end{verbete}

\begin{verbete}{大战}{da4zhan4}{3;9}
  \significado{s.}{guerra}
  \significado{v.}{guerrear; lutar em uma guerra}
\end{verbete}

\begin{verbete}{歹徒}{dai3tu2}{4;10}
  \significado{s.}{malfeitor; gangster; bandido}
\end{verbete}

\begin{verbete}{逮}{dai3}{11}[Radical 辵][Componentes ⻌隶]
  \significado{v.}{(coloquial) pegar, aproveitar, capturar}
  \veja{逮}{dai4}
\end{verbete}

\begin{verbete}{大}{dai4}{3}[Radical ⼈][Componentes ⼈⼀]
  \veja{大}{da4}
  \veja{大夫}{dai4fu5}
\end{verbete}

\begin{verbete}{大夫}{dai4fu5}{3;4}
  \significado{s.}{médico, doutor}
  \veja{大夫}{da4fu1}
\end{verbete}

\begin{verbete}{代表团}{dai4biao3tuan2}{5;8;6}
  \significado[个]{s.}{delegação}
\end{verbete}

\begin{verbete}{代称}{dai4cheng1}{5;10}
  \significado{s.}{nome alternativo; antonomásia}
  \significado{v.}{referir-se a algo ou alguém por outro nome}
\end{verbete}

\begin{verbete}{代价}{dai4jia4}{5;6}
  \significado{s.}{preço; custo}
\end{verbete}

\begin{verbete}{代言}{dai4yan2}{5;7}
  \significado{v.}{ser um porta-voz; ser um embaixador (para uma marca); endossar}
\end{verbete}

\begin{verbete}{带}{dai4}{9}[Radical ⼱][Componentes ⼍卅⼱]
  \significado{v.}{levar; trazer}
\end{verbete}

\begin{verbete}{带来}{dai4lai2}{9;7}
  \significado{v.}{trazer; (fig.) provocar, produzir}
\end{verbete}

\begin{verbete}{逮}{dai4}{11}[Radical 辵][Componentes ⻌隶]
  \significado{v.}{(literário) alcançar; usado em 逮捕}
  \veja{逮}{dai3}
  \veja{逮捕}{dai4bu3}
\end{verbete}

\begin{verbete}{逮捕}{dai4bu3}{11;10}
  \significado{v.}{prender, apreender, levar sob custódia}
\end{verbete}

\begin{verbete}{戴}{dai4}{17}[Radical ⼽][Componentes 𢦏異]
  \significado*{s.}{sobrenome Dai}
  \significado[条]{s.}{área; cinturão; região; zona}
  \significado{v.}{usar/vestir (óculos, gravata, relógio de pulso, luvas); trazer}
\end{verbete}

\begin{verbete}{单}{dan1}{8}[Radical 十][Componentes 一丷甲]
  \significado{adj.}{solteiro; único}
  \significado{adv.}{apenas}
  \significado[个]{s.}{conta; lista; formulário; número ímpar}
  \veja{单}{chan2}
  \veja{单}{shan4}
\end{verbete}

\begin{verbete}{单调}{dan1diao4}{8;10}
  \significado{adj.}{monótono}
\end{verbete}

\begin{verbete}{单脚滑行车}{dan1jiao3hua2xing2che1}{8;11;12;6;4}
  \significado{s.}{\emph{scooter}}
\end{verbete}

\begin{verbete}{单质}{dan1zhi4}{8;8}
  \significado{s.}{substância simples (consistindo puramente de um elemento, como diamante, grafite, etc.)}
\end{verbete}

\begin{verbete}{担心}{dan1xin1}{8;4}
  \significado{v.}{preocupar-se; estar preocupado}
\end{verbete}

\begin{verbete}{耽心}{dan1xin1}{10;4}
  \variante{担心}
\end{verbete}

\begin{verbete}{胆小鬼}{dan3xiao3gui3}{9;3;9}
  \significado{adj.}{covarde; medroso}
\end{verbete}

\begin{verbete}{但}{dan4}{7}[Radical 人][Componentes ⺅旦]
  \significado{conj.}{mas; ainda; no entanto; apenas}
\end{verbete}

\begin{verbete}{但是}{dan4shi4}{7;9}
  \significado{conj.}{mas; ainda; no entanto}
\end{verbete}

\begin{verbete}{蛋}{dan4}{11}[Radical 足][Componentes ⽦足]
  \significado[个,打]{s.}{ovo; objeto de formato oval}
\end{verbete}

\begin{verbete}{蛋糕}{dan4gao1}{11;16}
  \significado[块,个]{s.}{bolo}
\end{verbete}

\begin{verbete}{当初}{dang1chu1}{6;7}
  \significado{adv.}{naquela hora; originalmente}
\end{verbete}

\begin{verbete}{当然}{dang1ran2}{6;12}
  \significado{adv.}{claro; certamente}
\end{verbete}

\begin{verbete}{挡风玻璃}{dang3feng1bo1li5}{9;4;9;14}
  \significado{s.}{parabrisa}
\end{verbete}

\begin{verbete}{刀}{dao1}{2}[Radical ⼑][Componentes ㇆丿][Kangxi 18]
  \significado*{s.}{sobrenome Dao}
  \significado{p.c.}{para cortes de faca ou facadas}
  \significado[把]{s.}{faca; lâmina;  espada de fio único; cutelo; (gíria) dólar (empréstimo linguístico)}
\end{verbete}

\begin{verbete}{导弹}{dao3dan4}{6;11}
  \significado[枚]{s.}{míssil (guiado)}
\end{verbete}

\begin{verbete}{倒}{dao3}{10}[Radical 人][Componentes ⺅到]
  \significado{v.}{cair no chão; deitar-se no chão; colapsar; ir à falência}
  \veja{倒}{dao4}
\end{verbete}

\begin{verbete}{倒地}{dao3di4}{10;6}
  \significado{v.}{cair no chão}
\end{verbete}

\begin{verbete}{倒楣}{dao3mei2}{10;13}
  \variante{倒霉}
\end{verbete}

\begin{verbete}{倒霉}{dao3mei2}{10;15}
  \significado{adj.}{azarado}
  \significado{s.}{azar; má sorte}
  \significado{v.}{estar sem sorte; ter azar}
\end{verbete}

\begin{verbete}{倒血霉}{dao3xue4mei2}{10;6;15}
  \significado{v.}{ter muito azar (versão mais forte de 倒霉)}
  \veja{倒霉}{dao3mei2}
\end{verbete}

\begin{verbete}{到}{dao4}{8}[Radical 刀][Componentes 至⺉]
  \significado{prep.}{a; até; para}
  \significado{v.}{chegar}
\end{verbete}

\begin{verbete}{到处}{dao4chu4}{8;5}
  \significado{adv.}{em todos os lugares}
\end{verbete}

\begin{verbete}{到底}{dao4di3}{8;8}
  \significado{adv.}{na verdade; exatamente; são ou não são; afinal; no final; no final das contas; finalmente; quando tudo estiver dito e feito}
\end{verbete}

\begin{verbete}{倒}{dao4}{10}[Radical 人][Componentes ⺅到]
  \significado{adv.}{ao contrário da expectativa; ao contrário}
  \significado{v.}{inverter; colocar de cabeça para baixo ou de frente para trás; derramar; tombar}
  \veja{倒}{dao3}
\end{verbete}

\begin{verbete}{道理}{dao4li5}{12;11}
  \significado[个]{s.}{razão; argumento; sentido; princípio; base; justificativa}
\end{verbete}

\begin{verbete}{得}{de2}{11}[Radical 彳][Componentes 彳㝵]
  \significado{v.}{obter; ganhar; pegar (uma doença)}
  \veja{得}{de5}
  \veja{得}{dei3}
\end{verbete}

\begin{verbete}{得到}{de2dao4}{11;8}
  \significado{v.}{obter; receber}
\end{verbete}

\begin{verbete}{得了}{de2le5}{11;2}
  \significado{expr.}{Tudo bem!; É o bastante!}
  \veja{得了}{de2liao3}
\end{verbete}

\begin{verbete}{得了}{de2liao3}{11;2}
  \significado{adj.}{(enfaticamente, em perguntas retóricas) possível}
  \veja{得了}{de2le5}
\end{verbete}

\begin{verbete}{得意}{de2yi4}{11;13}
  \significado{adj.}{orgulhoso de si mesmo; satisfeito consigo mesmo; complacente}
\end{verbete}

\begin{verbete}{德}{de2}{15}[Radical 彳][Componentes 彳𢛳]
  \significado*{s.}{Alemanha, abreviação de~德国}
  \significado{s.}{virtude; bondade; moralidade; ética; personagem; tipo}
  \veja{德国}{de2guo2}
\end{verbete}

\begin{verbete}{德国}{de2guo2}{15;8}
  \significado*{s.}{Alemanha}
  \veja{德}{de2}
\end{verbete}

\begin{verbete}{德国人}{de2guo2ren2}{15;8;2}
  \significado{s.}{alemão; pessoa nascida na Alemanha}
\end{verbete}

\begin{verbete}{地}{de5}{6}[Radical 土][Componentes 土也]
  \significado{part.}{estrutural: utilizada antes de um verbo ou adjetivo, ligando-o ao adjunto adverbial modificador precedente}
  \veja{地}{di4}
\end{verbete}

\begin{verbete}{的}{de5}{8}[Radical 白][Componentes 白勺]
  \significado{part.}{utilizada em possessivos; utilizada entre adjetivos e substantivos (opcional se o adjetivo possui apenas um carácter); utilizada após um atributo; utilizada no final de uma frase declarativa para dar ênfase; para formar uma expressão nominal}
  \veja{的}{di1}
  \veja{的}{di2}
  \veja{的}{di4}
\end{verbete}

\begin{verbete}{得}{de5}{11}[Radical 彳][Componentes 彳㝵]
  \significado{part.}{estrutural:~ligando um verbo à frase seguinte indicando efeito, grau, possibilidade, etc.}
  \veja{得}{de2}
  \veja{得}{dei3}
\end{verbete}

\begin{verbete}{得}{dei3}{11}[Radical 彳][Componentes 彳㝵]
  \significado{v.}{haver de; ter de}
  \veja{得}{de2}
  \veja{得}{de5}
\end{verbete}

\begin{verbete}{灯}{deng1}{6}[Radical 火][Componentes 火丁]
  \significado[盏]{s.}{lâmpada; lanterna; luz}
\end{verbete}

\begin{verbete}{灯标}{deng1biao1}{6;9}
  \significado{s.}{farol; luz de farol}
\end{verbete}

\begin{verbete}{灯号}{deng1hao4}{6;5}
  \significado{s.}{sinal luminoso; luz indicadora}
\end{verbete}

\begin{verbete}{灯泡}{deng1pao4}{6;8}
  \significado[个]{s.}{lâmpada; terceiro indesejado estragando encontro de casal (gíria)}
  \veja{电灯泡}{dian4deng1pao4}
\end{verbete}

\begin{verbete}{灯丝}{deng1si1}{6;5}
  \significado{s.}{filamento (de uma lâmpada)}
\end{verbete}

\begin{verbete}{登}{deng1}{12}[Radical 癶][Componentes 癶豆]
  \significado{v.}{subir (montanha, cume)}
\end{verbete}

\begin{verbete}{等}{deng3}{12}[Radical 竹][Componentes ⺮寺]
  \significado{v.}{esperar; esperar por}
\end{verbete}

\begin{verbete}{等待}{deng3dai4}{12;9}
  \significado{v.}{esperar; esperar por}
\end{verbete}

\begin{verbete}{低}{di1}{7}[Radical 人][Componentes ⺅氐]
  \significado{adj.}{baixo}
  \significado{adv.}{abaixo}
  \significado{v.}{abaixar (a cabeça); deixar cair; pendurar; inclinar}
\end{verbete}

\begin{verbete}{的}{di1}{8}[Radical 白][Componentes 白勺]
  \significado{s.}{um táxi (abreviação de 的士)}
  \veja{的}{de5}
  \veja{的士}{di1shi4}
  \veja{的}{di2}
  \veja{的}{di4}
\end{verbete}

\begin{verbete}{的士}{di1shi4}{8;3}
  \significado{s.}{táxi (empréstimo linguístico)}
\end{verbete}

\begin{verbete}{堤坝}{di1ba4}{12;7}
  \significado{s.}{represa; dique; barragem}
\end{verbete}

\begin{verbete}{滴}{di1}{14}[Radical 水][Componentes ⺡啇]
  \significado{s.}{uma gota}
  \significado{v.}{pingar}
\end{verbete}

\begin{verbete}{的}{di2}{8}[Radical 白][Componentes 白勺]
  \significado{adv.}{realmente e verdadeiramente}
  \veja{的}{de5}
  \veja{的}{di1}
  \veja{的}{di4}
\end{verbete}

\begin{verbete}{的确}{di2que4}{8;12}
  \significado{adv.}{de fato; realmente}
\end{verbete}

\begin{verbete}{笛}{di2}{11}[Radical 竹][Componentes ⺮由]
  \significado{s.}{flauta}
\end{verbete}

\begin{verbete}{底气}{di3qi4}{8;4}
  \significado{s.}{capacidade pulmonar; ousadia, confiança, autoconfiança, vigor}
\end{verbete}

\begin{verbete}{抵抗}{di3kang4}{8;7}
  \significado{s.}{resistência}
  \significado{v.}{resistir}
\end{verbete}

\begin{verbete}{地}{di4}{6}[Radical 土][Componentes 土也]
  \significado[个,片]{s.}{mundo; campo; chão; terra; lugar}
  \veja{地}{de5}
\end{verbete}

\begin{verbete}{地点}{di4dian3}{6;9}
  \significado[个]{s.}{localização; lugar; local}
\end{verbete}

\begin{verbete}{地方}{di4fang1}{6;4}
  \significado{s.}{região; regional (longe da administração central); local}
  \veja{地方}{di4fang5}
\end{verbete}

\begin{verbete}{地方}{di4fang5}{6;4}
  \significado[处,个,块]{s.}{lugar; local; território}
  \veja{地方}{di4fang1}
\end{verbete}

\begin{verbete}{地核}{di4he2}{6;10}
  \significado{s.}{geologia:~núcleo da Terra}
\end{verbete}

\begin{verbete}{地理}{di4li3}{6;11}
  \significado{s.}{geografia}
\end{verbete}

\begin{verbete}{地球}{di4qiu2}{6;11}
  \significado{s.}{o planeta terra}
\end{verbete}

\begin{verbete}{地区}{di4qu1}{6;4}
  \significado{adj.}{regional}
  \significado{part.}{como sufixo do nome da cidade, significa prefeitura ou condado}
  \significado[个]{s.}{área; distrito (não necessariamente unidade administrativa formal); local; região}
\end{verbete}

\begin{verbete}{地铁}{di4tie3}{6;10}
  \significado{s.}{metrô; metropolitano}
\end{verbete}

\begin{verbete}{地图}{di4tu2}{6;8}
  \significado[张,本]{s.}{mapa}
\end{verbete}

\begin{verbete}{地下室}{di4xia4shi4}{6;3;9}
  \significado{s.}{subterrâneo; porão}
\end{verbete}

\begin{verbete}{地狱}{di4yu4}{6;9}
  \significado*{s.}{\emph{Naraka} (Budismo)}
  \significado{adj.}{infernal}
  \significado{s.}{inferno; submundo}
\end{verbete}

\begin{verbete}{地震}{di4zhen4}{6;15}
  \significado{s.}{terremoto; tremor de terra}
\end{verbete}

\begin{verbete}{地址}{di4zhi3}{6;7}
  \significado[个]{s.}{endereço}
\end{verbete}

\begin{verbete}{地砖}{di4zhuan1}{6;9}
  \significado{s.}{ladrilho de piso}
\end{verbete}

\begin{verbete}{弟}{di4}{7}[Radical 弓][Componentes 丷弓]
  \significado{s.}{irmão mais novo; júnior}
\end{verbete}

\begin{verbete}{弟弟}{di4di5}{7;7}
  \significado[个,位]{s.}{irmão mais novo}
\end{verbete}

\begin{verbete}{弟妹}{di4mei4}{7;8}
  \significado{s.}{esposa do irmão mais novo}
\end{verbete}

\begin{verbete}{的}{di4}{8}[Radical 白][Componentes 白勺]
  \significado{s.}{alvo; mosca (centro do alvo)}
  \veja{的}{de5}
  \veja{的}{di1}
  \veja{的}{di2}
\end{verbete}

\begin{verbete}{帝国}{di4guo2}{9;8}
  \significado{adj.}{imperial}
  \significado{s.}{império}
\end{verbete}

\begin{verbete}{第}{di4}{11}[Radical 竹][Componentes ⺮弟]
  \significado{num.}{prefixo para expressar números ordinais}
\end{verbete}

\begin{verbete}{墬}{di4}{14}
  \variante{地}
\end{verbete}

\begin{verbete}{点}{dian3}{9}[Radical 火][Componentes 占⺣]
  \significado{p.c.}{para itens; hora cheia}
  \significado{s.}{ponto; ponto (no espaço ou no tempo); gota; partícula}
\end{verbete}

\begin{verbete}{点火}{dian3huo3}{9;4}
  \significado{s.}{ignição}
  \significado{v.}{inflamar; acender um fogo; agitar; dar partida em um motor; (fig.) provocar problemas}
\end{verbete}

\begin{verbete}{点名}{dian3ming2}{9;6}
  \significado{v.}{mencionar alguém pelo nome; chamar, louvar ou criticar alguém pelo nome}
\end{verbete}

\begin{verbete}{点燃}{dian3ran2}{9;16}
  \significado{v.}{inflamar; incendiar}
\end{verbete}

\begin{verbete}{电冰箱}{dian4bing1xiang1}{5;6;15}
  \significado[台]{s.}{frigorífico; refrigerador}
\end{verbete}

\begin{verbete}{电车司机}{dian4che1 si1ji1}{5;4;5;6}
  \significado{s.}{motorista de bonde}
\end{verbete}

\begin{verbete}{电灯泡}{dian4deng1pao4}{5;6;8}
  \significado{s.}{lâmpada elétrica; (gíria) terceiro convidado indesejado}
\end{verbete}

\begin{verbete}{电动}{dian4dong4}{5;6}
  \significado{adj.}{movido a eletricidade; elétrico}
\end{verbete}

\begin{verbete}{电动车}{dian4dong4che1}{5;6;4}
  \significado{s.}{veículo elétrico (\emph{scooter}, bicicleta, carro, etc.)}
\end{verbete}

\begin{verbete}{电话}{dian4hua4}{5;8}
  \significado[部]{s.}{telefone}
  \significado[通]{s.}{chamada telefônica}
\end{verbete}

\begin{verbete}{电脑}{dian4nao3}{5;10}
  \significado[台]{s.}{computador}
\end{verbete}

\begin{verbete}{电脑语言}{dian4nao3yu3yan2}{5;10;9;7}
  \significado{s.}{linguagem de programação; linguagem de computador}
\end{verbete}

\begin{verbete}{电器}{dian4qi4}{5;16}
  \significado{s.}{aparelho elétrico}
\end{verbete}

\begin{verbete}{电视}{dian4shi4}{5;8}
  \significado[台,个]{s.}{televisão; TV; televisor}
\end{verbete}

\begin{verbete}{电视机}{dian4shi4ji1}{5;8;6}
  \significado[台]{s.}{aparelho de televisão; televisor}
\end{verbete}

\begin{verbete}{电梯}{dian4ti1}{5;11}
  \significado[台,部]{s.}{elevador; ascensor}
\end{verbete}

\begin{verbete}{电梯司机}{dian4ti1 si1ji1}{5;11;5;6}
  \significado{s.}{ascensorista}
\end{verbete}

\begin{verbete}{电影}{dian4ying3}{5;15}
  \significado[部,片,幕,场]{s.}{cinema; filme}
\end{verbete}

\begin{verbete}{电影奖}{dian4ying3jiang3}{5;15;9}
  \significado{s.}{premiações de cinema}
\end{verbete}

\begin{verbete}{电影节}{dian4ying3jie2}{5;15;5}
  \significado{s.}{festival de cinema}
\end{verbete}

\begin{verbete}{电影界}{dian4ying3jie4}{5;15;9}
  \significado{s.}{indústria cinematográfica}
\end{verbete}

\begin{verbete}{电影票}{dian4ying3piao4}{5;15;11}
  \significado{s.}{ingresso de filme}
\end{verbete}

\begin{verbete}{电影术}{dian4ying3 shu4}{5;15;5}
  \significado{s.}{cinematografia}
\end{verbete}

\begin{verbete}{电影艺术}{dian4ying3 yi4shu4}{5;15;4;5}
  \significado{s.}{arte cinematográfica}
\end{verbete}

\begin{verbete}{电影音乐}{dian4ying3 yin1yue4}{5;15;9;5}
  \significado{s.}{música cinematográfica}
\end{verbete}

\begin{verbete}{电影院}{dian4ying3yuan4}{5;15;9}
  \significado[次,家,座]{s.}{sala de cinema}
\end{verbete}

\begin{verbete}{电邮}{dian4you2}{5;7}
  \significado{s.}{correio eletrônico, \emph{e-mail}; abreviação de~电子邮件}
  \veja{电子邮件}{dian4zi3you2jian4}
\end{verbete}

\begin{verbete}{电子}{dian4zi3}{5;3}
  \significado{s.}{eletrônico; elétron}
\end{verbete}

\begin{verbete}{电子名片}{dian4zi3 ming2pian4}{5;3;6;4}
  \significado{s.}{cartão de visita eletrônico}
\end{verbete}

\begin{verbete}{电子邮件}{dian4zi3you2jian4}{5;3;7;6}
  \significado[封,份]{s.}{correio eletrônico, \emph{e-mail}}
  \veja{电邮}{dian4you2}
\end{verbete}

\begin{verbete}{店员}{dian4yuan2}{8;7}
  \significado{s.}{assistente de loja; balconista; vendedor}
\end{verbete}

\begin{verbete}{店主}{dian4zhu3}{8;5}
  \significado{s.}{lojista; dono de loja}
\end{verbete}

\begin{verbete}{垫子}{dian4zi5}{9;3}
  \significado{s.}{colchão, esteira, almofada}
\end{verbete}

\begin{verbete}{钿}{dian4}{10}[Radical 金][Componentes ⻐田]
  \significado{s.}{ornamento incrustado antigo em forma de flor}
  \significado{v.}{incrustar com ouro, prata, etc.}
  \veja{钿}{tian2}
\end{verbete}

\begin{verbete}{淀}{dian4}{11}[Radical 水][Componentes ⺡定]
  \significado{adj.}{pantanoso}
  \significado{s.}{lago raso; pântano}
  \significado{v.}{formar sedimentos; precipitar}
\end{verbete}

\begin{verbete}{貂}{diao1}{12}[Radical 豸][Componentes 豸召]
  \significado{s.}{marta; fuinha}
\end{verbete}

\begin{verbete}{雕刻}{diao1ke4}{16;8}
  \significado{s.}{escultura}
  \significado{v.}{esculpir; gravar}
\end{verbete}

\begin{verbete}{屌丝}{diao3si1}{9;5}
  \significado{adj.}{panaca; zé-ninguém; gíria de \emph{Internet}:~\emph{looser}}
\end{verbete}

\begin{verbete}{钓鱼}{diao4yu2}{8;8}
  \significado{v.}{pescar (com linha e anzol); (fig.) aprisionar}
\end{verbete}

\begin{verbete}{掉}{diao4}{11}[Radical 手][Componentes ⺘卓]
  \significado{v.}{cair; deixar cair}
\end{verbete}

\begin{verbete}{掉包}{diao4bao1}{11;5}
  \significado{v.}{vender uma falsificação pelo artigo genuíno; roubar o item valioso de alguém e substituí-lo por um item de aparência semelhante, mas sem valor}
\end{verbete}

\begin{verbete}{掉膘}{diao4biao1}{11;15}
  \significado{v.}{perder peso (gado)}
\end{verbete}

\begin{verbete}{掉队}{diao4dui4}{11;4}
  \significado{v.}{abandonar; ficar para trás}
\end{verbete}

\begin{verbete}{掉落}{diao4luo4}{11;12}
  \significado{v.}{derrubar}
\end{verbete}

\begin{verbete}{掉线}{diao4xian4}{11;8}
  \significado{v.}{desconectar-se (da \emph{Internet})}
\end{verbete}

\begin{verbete}{掉转}{diao4zhuan3}{11;8}
  \significado{v.}{dar a volta}
\end{verbete}

\begin{verbete}{叮嘱}{ding1zhu3}{5;15}
  \significado{v.}{exortar; avisar; insistir de novo e de novo}
\end{verbete}

\begin{verbete}{顶}{ding3}{8}[Radical 頁][Componentes 丁⻚]
  \significado{adv.}{mais; extremamente; melhor; muito (linguagem falada)}
\end{verbete}

\begin{verbete}{丢}{diu1}{6}[Radical 丿][Componentes 王厶]
  \significado{v.}{perder; perder-se}
\end{verbete}

\begin{verbete}{丢掉}{diu1diao4}{6;11}
  \significado{v.}{jogar fora; descartar; perder}
\end{verbete}

\begin{verbete}{丢官}{diu1guan1}{6;8}
  \significado{v.}{perder um cargo oficial}
\end{verbete}

\begin{verbete}{丢开}{diu1kai1}{6;4}
  \significado{v.}{jogar fora ou deixar de lado; esquecer por um tempo}
\end{verbete}

\begin{verbete}{丢脸}{diu1lian3}{6;11}
  \significado{adj.}{vergonhoso}
\end{verbete}

\begin{verbete}{丢弃}{diu1qi4}{6;7}
  \significado{v.}{jogar fora; descartar}
\end{verbete}

\begin{verbete}{丢失}{diu1shi1}{6;5}
  \significado{v.}{perder}
\end{verbete}

\begin{verbete}{丢下}{diu1xia4}{6;3}
  \significado{v.}{abandonar}
\end{verbete}

\begin{verbete}{东}{dong1}{5}[Radical ⼀][Componentes 七小]
  \significado*{s.}{sobrenome Dong}
  \significado{s.}{leste}
\end{verbete}

\begin{verbete}{东半球}{dong1ban4qiu2}{5;5;11}
  \significado*{s.}{Hemisfério Oriental}
\end{verbete}

\begin{verbete}{东北}{dong1bei3}{5;5}
  \significado*{s.}{Nordeste da China; Manchúria}
  \significado{p.l.}{nordeste}
\end{verbete}

\begin{verbete}{东边}{dong1bian5}{5;5}
  \significado{p.l.}{este; leste; lado leste; oriente}
\end{verbete}

\begin{verbete}{东部}{dong1bu4}{5;10}
  \significado{p.l.}{leste; oriente}
\end{verbete}

\begin{verbete}{东方}{dong1fang1}{5;4}
  \significado*{s.}{sobrenome Dongfang}
  \significado{p.l.}{leste; oriente}
\end{verbete}

\begin{verbete}{东方学院}{dong1fang1 xue2yuan4}{5;4;8;9}
  \significado*{s.}{Instituto Oriental}
\end{verbete}

\begin{verbete}{东面}{dong1mian4}{5;9}
  \significado{p.l.}{lado leste (de algo)}
\end{verbete}

\begin{verbete}{东西}{dong1xi1}{5;6}
  \significado{s.}{leste e oeste}
  \veja{东西}{dong1xi5}
\end{verbete}

\begin{verbete}{东西}{dong1xi5}{5;6}
  \significado[个,件]{s.}{coisa; materia; pessoa}
  \veja{东西}{dong1xi1}
\end{verbete}

\begin{verbete}{冬瓜}{dong1gua1}{5;5}
  \significado{s.}{melão de inverno}
\end{verbete}

\begin{verbete}{冬天}{dong1tian1}{5;4}
  \significado{p.t./s.}{inverno}
\end{verbete}

\begin{verbete}{懂}{dong3}{15}[Radical 心][Componentes ⺖董]
  \significado{v.}{compreender; entender}
\end{verbete}

\begin{verbete}{动}{dong4}{6}[Radical 力][Componentes 云力]
  \significado{v.}{mover; movimentar}
\end{verbete}

\begin{verbete}{动感}{dong4gan3}{6;13}
  \significado{adj.}{dinâmica; vívida}
  \significado{adv.}{dinamicamente}
  \significado{s.}{senso de movimento (geralmente em uma obra de arte estática)}
\end{verbete}

\begin{verbete}{动力}{dong4li4}{6;2}
  \significado{s.}{força motriz; força;  (fig.) motivação; ímpeto}
\end{verbete}

\begin{verbete}{动漫}{dong4man4}{6;14}
  \significado{s.}{desenhos animados; quadrinhos; anime; mangá}
\end{verbete}

\begin{verbete}{动物}{dong4wu4}{6;8}
  \significado[只,群,个]{s.}{animal}
\end{verbete}

\begin{verbete}{动物园}{dong4wu4yuan2}{6;8;7}
  \significado[个]{s.}{jardim zoológico; zoo}
\end{verbete}

\begin{verbete}{动作}{dong4zuo4}{6;7}
  \significado[个]{s.}{movimento, ação}
  \significado{v.}{mover, agir}
\end{verbete}

\begin{verbete}{洞穴}{dong4xue2}{9;5}
  \significado{s.}{caverna}
\end{verbete}

\begin{verbete}{都}{dou1}{10}[Radical 邑][Componentes 者⻏]
  \significado{adv.}{todo, todos}
  \veja{都}{du1}
\end{verbete}

\begin{verbete}{豆荚}{dou4jia2}{7;9}
  \significado{s.}{vagem (de legumes)}
\end{verbete}

\begin{verbete}{豆角}{dou4jiao3}{7;7}
  \significado{s.}{feijão verde}
\end{verbete}

\begin{verbete}{读}{dou4}{10}[Radical 言][Componentes ⻈卖]
  \significado{s.}{vírgula; frase marcada por pausa}
  \veja{读}{du2}
\end{verbete}

\begin{verbete}{都}{du1}{10}[Radical 邑][Componentes 者⻏]
  \significado*{s.}{sobrenome Du}
  \significado{s.}{capital; metrópole}
  \veja{都}{dou1}
\end{verbete}

\begin{verbete}{嘟}{du1}{13}[Radical ⼝][Componentes ⼝都]
  \significado{s.}{buzina; bip}
  \significado{v.}{fazer beicinho}
\end{verbete}

\begin{verbete}{毋}{du2}{9}[Radical ⺟][Componentes 龶⺟]
  \significado{adj.}{venenoso; tóxico}
  \significado{s.}{veneno; tóxico}
  \significado{v.}{intoxicar}
\end{verbete}

\begin{verbete}{毒害}{du2hai4}{9;10}
  \significado{s.}{envenenamento}
  \significado{v.}{envenenar (prejudicar com uma substância tóxica); envenenar (as mentes das pessoas)}
\end{verbete}

\begin{verbete}{毒杀}{du2sha1}{9;6}
  \significado{v.}{matar por envenenamento}
\end{verbete}

\begin{verbete}{毒蛇}{du2she2}{9;11}
  \significado{s.}{víbora; cobra venenosa}
\end{verbete}

\begin{verbete}{毒物}{du2wu4}{9;8}
  \significado{s.}{substância venenosa; toxina}
\end{verbete}

\begin{verbete}{独}{du2}{9}[Radical 犬][Componentes ⺨⾍]
  \significado{adj.}{sozinho; solitário; solteiro}
  \significado{adv.}{apenas}
\end{verbete}

\begin{verbete}{独自}{du2zi4}{9;6}
  \significado{adj.}{sozinho}
\end{verbete}

\begin{verbete}{读}{du2}{10}[Radical 言][Componentes ⻈卖]
  \significado{v.}{ler em voz alta; ler; frequentar (escola), estudar (uma matéria na escola); pronunciar}
  \veja{读}{dou4}
\end{verbete}

\begin{verbete}{肚}{du3}{7}[Radical 肉][Componentes ⺼⼟]
  \significado{s.}{tripa; entranhas}
  \veja{肚}{du4}
\end{verbete}

\begin{verbete}{堵车}{du3che1}{11;4}
  \significado{v.}{congestionar (trânsito)}
  \significado{v.+compl.}{congestionamento; engarrafamento (de trânsito)}
\end{verbete}

\begin{verbete}{杜鹃}{du4juan1}{7;12}
  \significado{s.}{cuco (pássaro)}
  \veja{布谷鸟}{bu4gu3niao3}
  \veja{杜鹃鸟}{du4juan1niao3}
  \veja{杜宇}{du4yu3}
\end{verbete}

\begin{verbete}{杜鹃鸟}{du4juan1niao3}{7;12;5}
  \significado{s.}{cuco (pássaro)}
  \veja{布谷鸟}{bu4gu3niao3}
  \veja{杜鹃}{du4juan1}
  \veja{杜宇}{du4yu3}
\end{verbete}

\begin{verbete}{杜宇}{du4yu3}{7;6}
  \significado{s.}{cuco (pássaro)}
  \veja{布谷鸟}{bu4gu3niao3}
  \veja{杜鹃}{du4juan1}
  \veja{杜鹃鸟}{du4juan1niao3}
\end{verbete}

\begin{verbete}{肚}{du4}{7}[Radical 肉][Componentes ⺼⼟]
  \significado{s.}{barriga}
  \veja{肚}{du3}
\end{verbete}

\begin{verbete}{肚子}{du4zi5}{7;3}
  \significado[个]{s.}{abdómen; barriga}
\end{verbete}

\begin{verbete}{度}{du4}{9}[Radical ⼴][Componentes ⼜⼴廿]
  \significado{p.c.}{para temperatura, etc.; para eventos e ocorrências}
  \significado{s.}{grau (ângulo, temperatura, etc.); kilowatt-hora}
  \veja{度}{duo2}
\end{verbete}

\begin{verbete}{渡过}{du4guo4}{12;6}
  \significado{v.}{atravessar; passar por}
\end{verbete}

\begin{verbete}{镀金}{du4jin1}{14;8}
  \significado{v.}{banhar a ouro; dourar; (fig.) fazer algo muito comum parecer especial}
\end{verbete}

\begin{verbete}{端午节}{duan1wu3jie2}{14;4;5}
  \significado*{s.}{Festa do Duplo Cinco, Festival dos Barcos-Dragão (5º~dia do quinto mês lunar)}
\end{verbete}

\begin{verbete}{短}{duan3}{12}[Radical ⽮][Componentes ⽮⾖]
  \significado{adj.}{curto; breve}
\end{verbete}

\begin{verbete}{短处}{duan3chu4}{12;5}
  \significado{s.}{defeito; falta; pontos fracos de alguém; deficiência}
\end{verbete}

\begin{verbete}{短促}{duan3cu4}{12;9}
  \significado{adj.}{curto (tom de voz); fugaz; ofegante (respiração); curto no tempo}
\end{verbete}

\begin{verbete}{短裤}{duan3ku4}{12;12}
  \significado{s.}{calção; shorts}
\end{verbete}

\begin{verbete}{短跑}{duan3pao3}{12;12}
  \significado{s.}{corrida}
\end{verbete}

\begin{verbete}{短期}{duan3qi1}{12;12}
  \significado{s.}{curto prazo}
\end{verbete}

\begin{verbete}{短缺}{duan3que1}{12;10}
  \significado{s.}{escassez}
\end{verbete}

\begin{verbete}{短少}{duan3shao3}{12;4}
  \significado{v.}{estar aquém do valor total}
\end{verbete}

\begin{verbete}{短视}{duan3shi4}{12;8}
  \significado{adj.}{míope}
\end{verbete}

\begin{verbete}{短暂}{duan3zan4}{12;12}
  \significado{adj.}{momentâneo; de curta duração}
\end{verbete}

\begin{verbete}{段}{duan4}{9}[Radical ⽎][Componentes ⽎]
  \significado*{s.}{sobrenome Duan}
  \significado{p.c.}{para histórias, períodos de tempo, desenvolvimento de um tópico, etc.}
  \significado{s.}{parágrafo; seção; segmento; estágio (de um processo)}
\end{verbete}

\begin{verbete}{锻炼}{duan4lian4}{14;9}
  \significado{v.}{fazer exercício físico; praticar esporte}
\end{verbete}

\begin{verbete}{队}{dui4}{4}[Radical 阜][Componentes ⻖⼈]
  \significado[个]{s.}{esquadrão; equipe; grupo}
\end{verbete}

\begin{verbete}{队友}{dui4you3}{4;4}
  \significado{s.}{companheiro de equipe}
\end{verbete}

\begin{verbete}{对}{dui4}{5}[Radical ⼨][Componentes ⼜⼨]
  \significado{adj.}{correto; sim}
  \significado{p.c.}{para casais}
  \significado{prep.}{com; para; para com}
\end{verbete}

\begin{verbete}{对不起}{dui4bu5qi3}{5;4;10}
  \significado{v.}{desculpar; pedir desculpas; perdoar}
\end{verbete}

\begin{verbete}{对得起}{dui4de5qi3}{5;11;10}
  \significado{v.}{não decepcionar alguém; tratar alguém de maneira justa; ser digno de}
\end{verbete}

\begin{verbete}{对……感兴趣}{dui4 gan3xing4qu4}{5;13;6;15}
  \significado{expr.}{estar interessado em\dots; ter interesse em\dots; interessar-se por\dots}
  \veja{对……有兴趣}{dui4 you3xing4qu4}
\end{verbete}

\begin{verbete}{对话}{dui4hua4}{5;8}
  \significado[个]{s.}{diálogo; conversa}
  \significado{v.}{dialogar; conversar}
\end{verbete}

\begin{verbete}{对面}{dui4mian4}{5;9}
  \significado{p.l.}{lado oposto}
\end{verbete}

\begin{verbete}{对手}{dui4shou3}{5;4}
  \significado{s.}{oponente; rival; concorrente; adversário}
\end{verbete}

\begin{verbete}{对……熟悉}{dui4 shu2xi1}{5;15;11}
  \significado{expr.}{estar familiarizado com\dots}
\end{verbete}

\begin{verbete}{对……说}{dui4 shuo5}{5;9}
  \significado{v.}{dizer a alguém}
\end{verbete}

\begin{verbete}{对……有兴趣}{dui4 you3xing4qu4}{5;6;6;15}
  \significado{expr.}{estar interessado em\dots; ter interesse em\dots; interessar-se por\dots}
  \veja{对……感兴趣}{dui4 gan3xing4qu4}
\end{verbete}

\begin{verbete}{蹲下}{dun1xia4}{19;3}
  \significado{v.}{agachar; agachar-se}
\end{verbete}

\begin{verbete}{顿}{dun4}{10}[Radical 頁][Componentes 屯⻚]
  \significado{p.c.}{para refeições, espancamentos, repreensões, etc.:~tempo, luta, feitiço, refeição}
  \significado{v.}{prostrar-se; pausar; bater (o pé)}
\end{verbete}

\begin{verbete}{多}{duo1}{6}[Radical ⼣][Componentes ⼣]
  \significado{adv.}{muito, muitos; em excesso; (prefixo) multi-, poli-}
  \significado{num.}{(após um número) ímpar}
\end{verbete}

\begin{verbete}{多重}{duo1chong2}{6;9}
  \significado{s.}{multi- (facetado, cultural, étnico, etc.)}
\end{verbete}

\begin{verbete}{多大}{duo1da4}{6;3}
  \significado{inter.}{quantos anos?; que idade?; quão grande?}
\end{verbete}

\begin{verbete}{多(么)}{duo1(me5)}{6;3}
  \significado{adv.}{como}
\end{verbete}

\begin{verbete}{多少}{duo1shao3}{6;4}
  \significado{num.}{número; quantidade}
  \veja{多少}{duo1shao5}
\end{verbete}

\begin{verbete}{多少}{duo1shao5}{6;4}
  \significado{inter.}{quanto?, quantos?, (para mais de 10 itens)}
  \veja{多少}{duo1shao3}
\end{verbete}

\begin{verbete}{多云}{duo1yun2}{6;4}
  \significado{adj.}{céu nublado}
\end{verbete}

\begin{verbete}{夺冠}{duo2guan4}{6;9}
  \significado{v.}{apoderar-se da coroa; (fig.) ganhar um campeonato; ganhar a medalha de ouro}
\end{verbete}

\begin{verbete}{度}{duo2}{9}[Radical ⼴][Componentes ⼜⼴廿]
  \significado{v.}{estimar}
  \veja{度}{du4}
\end{verbete}

\begin{verbete}{躲}{duo3}{13}[Radical ⾝][Componentes ⾝朵]
  \significado{v.}{esconder; esquivar; evitar}
\end{verbete}

\begin{verbete}{躲闪}{duo3shan3}{13;5}
  \significado{v.}{desviar; evadir; esquivar (para fora do caminho)}
\end{verbete}

%%%%% EOF %%%%%

%%%
%%% E
%%%
\section*{E}
\addcontentsline{toc}{section}{E}
\begin{multicols*}{2}

\begin{verbete}[É\ luo2si1]{俄罗斯}
\begin{pronuncia}{É\ luo2si1}
\significado{}{n.}{
Rússia
}
\end{pronuncia}
\end{verbete}

\begin{verbete}[en1ci4]{恩赐}
\begin{pronuncia}{en1ci4}
\significado{n.}{
favor; caridade
}
\significado{v.}{
conceder (favor, caridade);
}
\end{pronuncia}
\end{verbete}

\begin{verbete}[er2xi2]{儿媳}
\begin{pronuncia}{er2xi2}
\significado{n.}{
esposa do filho
}
\end{pronuncia}
\end{verbete}

\begin{verbete}[er2zi0]{儿子}
\begin{pronuncia}{er2zi0}
\significado{n.}{
filho
}
\end{pronuncia}
\end{verbete}

\begin{verbete}[er3duo0]{耳朵}
\begin{pronuncia}{er3duo0}
\significado[只,个,对]{n.}{
orelha
}
\end{pronuncia}
\end{verbete}

\begin{verbete}[er4]{二}
\begin{pronuncia}{er4}
\significado{num.}{
dois; 2
}
\end{pronuncia}
\end{verbete}

\end{multicols*}

%%%
%%% F
%%%

\section*{F}\addcontentsline{toc}{section}{F}

\begin{entry}{发}{fa1}{5}[Radical ⼜][HSK 2]
  \definition{clas.}{para tiros (rodadas)}
  \definition{v.}{enviar | mandar}
  \seeref{发}{fa4}
\end{entry}

\begin{entry}{发表}{fa1biao3}{5,8}[HSK 3]
  \definition{v.}{publicar; entregar; emitir; expressar; anunciar | publicar}
\end{entry}

\begin{entry}{发财}{fa1cai2}{5,7}
  \definition{v.+compl.}{ficar rico | fazer fortuna}
\end{entry}

\begin{entry}{发愁}{fa1chou2}{5,13}
  \definition{v.+compl.}{preocupar-se | ficar ansioso | ficar triste}
\end{entry}

\begin{entry}{发出}{fa1 chu1}{5,5}[HSK 3]
  \definition{v.}{fazer; produzir; deixar sair | emitir; anunciar | enviar; partir | dar; emitir}
\end{entry}

\begin{entry}{发达}{fa1da2}{5,6}[HSK 3]
  \definition{adj.}{desenvolvido; florescente}
  \definition{v.}{desenvolver; promover; florescer}
\end{entry}

\begin{entry}{发动}{fa1dong4}{5,6}[HSK 3]
  \definition{v.}{iniciar; lançar; ligar motor; dar a partida (motor de combustão interna) | chamar à ação; mobilizar; estimular; despertar}
\end{entry}

\begin{entry}{发动机}{fa1dong4ji1}{5,6,6}
  \definition[台]{s.}{motor}
\end{entry}

\begin{entry}{发抖}{fa1dou3}{5,7}
  \definition{v.}{tremer | sacudir | estremecer}
\end{entry}

\begin{entry}{发明}{fa1ming2}{5,8}[HSK 3]
  \definition[个]{s.}{invenção}
  \definition{v.}{inventar | expor; explicar}
\end{entry}

\begin{entry}{发明者}{fa1ming2zhe3}{5,8,8}
  \definition{s.}{inventor}
\end{entry}

\begin{entry}{发票}{fa1piao4}{5,11}
  \definition{s.}{fatura | recibo | conta}
\end{entry}

\begin{entry}{发烧}{fa1shao1}{5,10}
  \definition{v.}{ter febre}
\end{entry}

\begin{entry}{发生}{fa1sheng1}{5,5}[HSK 3]
  \definition{v.}{ocorrer; acontecer; tomar lugar}
\end{entry}

\begin{entry}{发送}{fa1 song4}{5,9}[HSK 3]
  \definition{v.}{enviar; despachar | transmitir; enviar}
\end{entry}

\begin{entry}{发现}{fa1xian4}{5,8}[HSK 2]
  \definition{s.}{descoberta}
  \definition{v.}{perceber, tornar-se ciente de | descobrir, encontrar, detectar}
\end{entry}

\begin{entry}{发现者}{fa1xian4 zhe3}{5,8,8}
  \definition{s.}{descobridor}
\end{entry}

\begin{entry}{发言}{fa1yan2}{5,7}[HSK 3]
  \definition[个]{s.}{discurso; declaração; palestra}
  \definition{v.+compl.}{falar; fazer uma declaração (discurso)}
\end{entry}

\begin{entry}{发音}{fa1yin1}{5,9}
  \definition{s.}{pronúncia}
  \definition{v.}{pronunciar}
\end{entry}

\begin{entry}{发展}{fa1zhan3}{5,10}[HSK 3]
  \definition{s.}{desenvolvimento}
  \definition{v.}{crescer; expandir; avançar; desenvolver | recrutar; expandir; admitir}
\end{entry}

\begin{entry}{罚}{fa2}{9}[Radical 网]
  \definition{v.}{castigar | punir}
\end{entry}

\begin{entry}{罚款}{fa2kuan3}{9,12}
  \definition{s.}{multa (monetária) | pena}
  \definition{v.+compl.}{aplicar uma multa | multar}
\end{entry}

\begin{entry}{筏}{fa2}{12}[Radical 竹]
  \definition{s.}{jangada (de troncos, bambus, etc.)}
\end{entry}

\begin{entry}{法}{fa3}{8}[Radical 水]
  \definition*{s.}{França, abreviação de~法国}
  \seealsoref{法国}{fa3guo2}
\end{entry}

\begin{entry}{法国}{fa3guo2}{8,8}
  \definition*{s.}{França}
\end{entry}

\begin{entry}{法国人}{fa3guo2ren2}{8,8,2}
  \definition{s.}{francês | pessoa ou povo da França}
\end{entry}

\begin{entry}{法网}{fa3wang3}{8,6}
  \definition*{s.}{Torneio de Roland Garros (French Open), torneio de tênis}
\end{entry}

\begin{entry}{法文}{fa3wen2}{8,4}
  \definition*{s.}{françês, língua francesa}
\end{entry}

\begin{entry}{法语}{fa3yu3}{8,9}
  \definition{s.}{françês, língua francesa}
\end{entry}

\begin{entry}{发}{fa4}{5}[Radical ⼜]
  \definition{s.}{cabelo}
  \seeref{发}{fa1}
\end{entry}

\begin{entry}{发型}{fa4xing2}{5,9}
  \definition{s.}{penteado}
\end{entry}

\begin{entry}{发簪}{fa4zan1}{5,18}
  \definition{s.}{grampo de cabelo}
\end{entry}

\begin{entry}{番茄}{fan1qie2}{12,8}
  \definition{s.}{tomate}
\end{entry}

\begin{entry}{蕃茄}{fan1qie2}{15,8}
  \variantof{番茄}
\end{entry}

\begin{entry}{翻过}{fan1guo4}{18,6}
  \definition{v.}{virar |  transformar}
\end{entry}

\begin{entry}{翻脸}{fan1lian3}{18,11}
  \definition{v.+compl.}{brigar com alguém | tornar-se hostil}
\end{entry}

\begin{entry}{翻译}{fan1yi4}{18,7}
  \definition[个,位,名]{s.}{tradução | tradutor | interpretação | intérprete}
  \definition{v.}{traduzir; interpretar}
\end{entry}

\begin{entry}{反对}{fan3dui4}{4,5}[HSK 3]
  \definition{v.}{lutar; opor-se; objetar a; ser contra}
\end{entry}

\begin{entry}{反对党}{fan3dui4dang3}{4,5,10}
  \definition{s.}{partido de oposição}
\end{entry}

\begin{entry}{反对派}{fan3dui4pai4}{4,5,9}
  \definition{s.}{facção de oposição}
\end{entry}

\begin{entry}{反对票}{fan3dui4piao4}{4,5,11}
  \definition{s.}{voto dissidente}
\end{entry}

\begin{entry}{反复}{fan3fu4}{4,9}[HSK 3]
  \definition{adv.}{repetidamente; de ​​novo e de novo}
  \definition{s.}{reversão; recaída}
  \definition{v.}{recuar; cortar e mudar}
\end{entry}

\begin{entry}{反省}{fan3xing3}{4,9}
  \definition{v.}{examinar a consciência | questionar-se | refletir sobre si mesmo | sondar a alma}
\end{entry}

\begin{entry}{反应}{fan3ying4}{4,7}[HSK 3]
  \definition[个]{s.}{reação; resposta}
  \definition{v.}{reagir; responder}
\end{entry}

\begin{entry}{反正}{fan3zheng4}{4,5}[HSK 3]
  \definition{adv.}{de qualquer forma | tudo igual; em qualquer caso}
\end{entry}

\begin{entry}{犯法}{fan4fa3}{5,8}
  \definition{v.}{violar (quebrar) a lei}
\end{entry}

\begin{entry}{犯罪}{fan4zui4}{5,13}
  \definition{v.+compl.}{cometer  um crime (uma ofensa)}
\end{entry}

\begin{entry}{饭}{fan4}{7}[Radical 食][HSK 1]
  \definition[碗]{s.}{arroz cozido}
  \definition[顿]{s.}{refeição}
  \definition{s.}{(empréstimo linguístico) fã, devoto}
\end{entry}

\begin{entry}{饭店}{fan4dian4}{7,8}[HSK 1]
  \definition[家,个]{s.}{restaurante | hotel}
\end{entry}

\begin{entry}{饭馆}{fan4 guan3}{7,11}[HSK 2]
  \definition[家,个]{s.}{restaurante | lanchonete}
\end{entry}

\begin{entry}{范围}{fan4wei2}{9,7}[HSK 3]
  \definition[个]{s.}{escopo; limite; alcance}
  \definition{v.}{estabelecer limites para; limitar o escopo de}
\end{entry}

\begin{entry}{方案}{fang1'an4}{4,10}
  \definition[个,套]{s.}{plano | programa (para uma ação, etc.) | proposta | proposta de projeto de lei}
\end{entry}

\begin{entry}{方便}{fang1bian4}{4,9}[HSK 2]
  \definition{adj.}{conveniente | adequado}
  \definition{v.}{facilitar, facilitar as coisas | ter dinheiro de sobra | (eufemismo) aliviar-se}
\end{entry}

\begin{entry}{方便面}{fang1 bian4 mian4}{4,9,9}[HSK 2]
  \definition{s.}{macarrão instantâneo}
\end{entry}

\begin{entry}{方法}{fang1fa3}{4,8}[HSK 2]
  \definition[个]{s.}{método | meio}
\end{entry}

\begin{entry}{方面}{fang1mian4}{4,9}[HSK 2]
  \definition[个]{s.}{lado | campo | aspecto}
\end{entry}

\begin{entry}{方式}{fang1shi4}{4,6}[HSK 3]
  \definition[种,个]{s.}{maneira; método}
\end{entry}

\begin{entry}{方向}{fang1xiang4}{4,6}[HSK 2]
  \definition[个]{s.}{direção | orientação | alvo | meta | objetivo}
\end{entry}

\begin{entry}{方言}{fang1yan2}{4,7}
  \definition*{s.}{o primeiro dicionário de dialeto chinês, editado por Yang Xiong 扬雄 no século I, contendo mais de 9.000 caracteres}
  \definition{s.}{dialeto}
  \seealsoref{扬雄}{yang2xiong2}
\end{entry}

\begin{entry}{防}{fang2}{6}[HSK 3]
  \definition*{s.}{sobrenome Fang}
  \definition{s.}{defesa | barragem; dique; aterro}
  \definition{v.}{prover contra; defender contra; proteger contra}
\end{entry}

\begin{entry}{防护}{fang2hu4}{6,7}
  \definition{v.}{defender | proteger}
\end{entry}

\begin{entry}{防晒}{fang2shai4}{6,10}
  \definition{s.}{protetor solar}
\end{entry}

\begin{entry}{防止}{fang2zhi3}{6,4}[HSK 3]
  \definition{v.}{evitar; prevenir; prevenir; proteger contra}
\end{entry}

\begin{entry}{房东}{fang2dong1}{8,5}[HSK 3]
  \definition[个,位]{s.}{dono;  proprietário; senhorio}
\end{entry}

\begin{entry}{房间}{fang2jian1}{8,7}[HSK 1]
  \definition[间,个]{s.}{quarto}
\end{entry}

\begin{entry}{房屋}{fang2 wu1}{8,9}[HSK 3]
  \definition[间,所,套]{s.}{casas; habitação; edifícios}
\end{entry}

\begin{entry}{房主}{fang2zhu3}{8,5}
  \definition{s.}{proprietário | dono de um imóvel}
\end{entry}

\begin{entry}{房子}{fang2zi5}{8,3}[HSK 1]
  \definition[栋,幢,座,套,间,个]{s.}{apartamento | casa | quarto}
\end{entry}

\begin{entry}{房租}{fang2 zu1}{8,10}[HSK 3]
  \definition[笔]{s.}{aluguel}
\end{entry}

\begin{entry}{访问}{fang3wen4}{6,6}[HSK 3]
  \definition{v.}{visitar; ligar; entrevistar | visitar um \emph{site}}
\end{entry}

\begin{entry}{放}{fang4}{8}[Radical 攴][HSK 1]
  \definition{v.}{liberar | libertar | deixar ir | colocar | por | detonar (fogos de artifício)}
\end{entry}

\begin{entry}{放鞭炮}{fang4bian1pao4}{8,18,9}
  \definition{s.}{um conjunto de bombinhas ou traques}
\end{entry}

\begin{entry}{放出}{fang4chu1}{8,5}
  \definition{v.}{liberar | libertar}
\end{entry}

\begin{entry}{放大}{fang4da4}{8,3}
  \definition{v.}{ampliar}
\end{entry}

\begin{entry}{放到}{fang4 dao4}{8,8}[HSK 3]
  \definition{v.}{colocar em; meter}
\end{entry}

\begin{entry}{放电}{fang4dian4}{8,5}
  \definition{s.}{descarga elétrica}
\end{entry}

\begin{entry}{放飞}{fang4fei1}{8,3}
  \definition{s.}{deixar voar}
\end{entry}

\begin{entry}{放过}{fang4guo4}{8,6}
  \definition{v.}{deixar | deixar alguém escapar impune | passar despercebido}
\end{entry}

\begin{entry}{放假}{fang4 jia4}{8,11}[HSK 1]
  \definition{v.}{ter férias ou feriado}
\end{entry}

\begin{entry}{放弃}{fang4qi4}{8,7}
  \definition{v.}{abandonar | desistir de | renunciar}
\end{entry}

\begin{entry}{放弃权利}{fang4qi4 quan2li4}{8,7,6,7}
  \definition{s.}{renúncia}
\end{entry}

\begin{entry}{放弃者}{fang4qi4zhe3}{8,7,8}
  \definition{s.}{desistente}
\end{entry}

\begin{entry}{放任}{fang4ren4}{8,6}
  \definition{v.}{ignorar | saciar-se | deixar sozinho}
\end{entry}

\begin{entry}{放肆}{fang4si4}{8,13}
  \definition{adj.}{atrevido | pesunçoso | devasso}
\end{entry}

\begin{entry}{放松}{fang4song1}{8,8}
  \definition{adj.}{relaxado | afrouxado}
  \definition{v.}{relaxar | afrouxar}
\end{entry}

\begin{entry}{放下}{fang4 xia4}{8,3}[HSK 2]
  \definition{v.}{deitar | colocar para baixo | deixar ir | liberar | desistir | colocar em algum lugar}
\end{entry}

\begin{entry}{放心}{fang4xin1}{8,4}[HSK 2]
  \definition{adj.}{despreocupado}
  \definition{v.}{sentir-se aliviado | sentir-se tranquilo | ficar à vontade}
  \definition{v.+compl.}{confiar | ter confiança em alguém | estar à vontade | sentir-se aliviado}
\end{entry}

\begin{entry}{放学}{fang4 xue2}{8,8}[HSK 1]
  \definition{v.+compl.}{sair da escola | acabar as aulas | terminar a aula (por hoje)}
\end{entry}

\begin{entry}{放养}{fang4yang3}{8,9}
  \definition{v.}{criar (gado, peixes, culturas, etc.) | crescer | criar}
\end{entry}

\begin{entry}{放走}{fang4zou3}{8,7}
  \definition{v.}{permitir (uma pessoa ou um animal) ir | liberar | libertar}
\end{entry}

\begin{entry}{飞}{fei1}{3}[Radical 飛][HSK 1]
  \definition*{s.}{sobrenome Fei}
  \definition{adj.}{inesperado | acidental | infundado | sem fundamento}
  \definition{adv.}{rapidamente}
  \definition{s.}{roda livre de uma bicicleta}
  \definition{v.}{voar | esvoaçar | flutuar no ar | volatilizar}
\end{entry}

\begin{entry}{飞船}{fei1chuan2}{3,11}
  \definition{s.}{espaçonave | dirigível | aeronave}
\end{entry}

\begin{entry}{飞碟}{fei1die2}{3,14}
  \definition{s.}{disco-voador, OVNI, \emph{UFO} | \emph{frisbee}}
\end{entry}

\begin{entry}{飞机}{fei1ji1}{3,6}[HSK 1]
  \definition[架]{s.}{avião}
\end{entry}

\begin{entry}{飞机票}{fei1ji1piao4}{3,6,11}
  \definition[张]{s.}{bilhete de avião}
  \seealsoref{机票}{ji1piao4}
\end{entry}

\begin{entry}{飞行}{fei1 xing2}{3,6}[HSK 3]
  \definition{s.}{voo | aviação}
  \definition{v.}{voar; fazer um voo | (aviões, foguetes, etc.) voar no ar}
\end{entry}

\begin{entry}{非}{fei1}{8}[Radical ⾮][Kangxi 175]
  \definition*{s.}{África, abreviação de 非洲}
  \definition{adv.}{não ser | não é | não}
  \seealsoref{非洲}{fei1zhou1}
\end{entry}

\begin{entry}{非常}{fei1chang2}{8,11}[HSK 1]
  \definition{adv.}{extraordinário | altamente | muito}
\end{entry}

\begin{entry}{非洲}{fei1zhou1}{8,9}
  \definition*{s.}{África}
\end{entry}

\begin{entry}{非洲人}{fei1zhou1ren2}{8,9,2}
  \definition{s.}{africano | pessoa ou povo da África}
\end{entry}

\begin{entry}{狒狒}{fei4fei4}{8,8}
  \definition{s.}{babuíno}
\end{entry}

\begin{entry}{费}{fei4}{9}[Radical 貝][HSK 3]
  \definition*{s.}{Fei}
  \definition{s.}{taxa; despesa; encargo}
  \definition{v.}{custar; gastar; desperdiçar}
\end{entry}

\begin{entry}{费用}{fei4 yong4}{9,5}[HSK 3]
  \definition[笔,个]{s.}{custo; despesa; desembolso}
\end{entry}

\begin{entry}{分}{fen1}{4}[Radical ⼑][HSK 1]
  \definition{s.}{parte ou subdivisão | fração | um décimo (de certas unidades) | unidade de comprimento equivalente a 0,33cm | minuto (unidade de tempo) | minuto (unidade de medida angular) | um ponto (em esportes e jogos) | 0,01 yuan (unidade de dinheiro)}
  \definition{v.}{dividir | separar | distribuir | atribuir | distinguir (bom e mau)}
  \seeref{分}{fen4}
\end{entry}

\begin{entry}{分别}{fen1bie2}{4,7}[HSK 3]
  \definition{adv.}{diferentemente; de ​​maneiras diferentes}
  \definition{s.}{diferença}
  \definition{v.}{partir; deixar um ao outro | distinguir; diferenciar}
\end{entry}

\begin{entry}{分公司}{fen1gong1si1}{4,4,5}
  \definition{s.}{sucursal | filial de companhia}
\end{entry}

\begin{entry}{分开}{fen1 kai1}{4,4}[HSK 2]
  \definition{v.+compl.}{separar | dividir | desacoplar | desempacotar | quebrar | desmembrar | romper | desfazer | desvincular | distribuir | separar de (em) | dividir \dots de \dots | separar de}
\end{entry}

\begin{entry}{分量}{fen1liang4}{4,12}
  \definition{s.}{componente vetorial}
  \seeref{分量}{fen4liang4}
  \seeref{分量}{fen4liang5}
\end{entry}

\begin{entry}{分配}{fen1pei4}{4,10}[HSK 3]
  \definition{v.}{atribuir; dispor | atribuir; compartilhar; distribuir}
\end{entry}

\begin{entry}{分手}{fen1shou3}{4,4}
  \definition{v.+compl.}{separar | separar-se do companheiro | dizer adeus}
\end{entry}

\begin{entry}{分数}{fen1 shu4}{4,13}[HSK 2]
  \definition{s.}{fração | número fracionário | marca | nota | ponto}
\end{entry}

\begin{entry}{分钟}{fen1zhong1}{4,9}[HSK 2]
  \definition{s.}{minuto (usado em intervalos de tempo)}
\end{entry}

\begin{entry}{分子}{fen1zi3}{4,3}
  \definition{s.}{molécula | (matemática) numerador de uma fração}
  \seeref{分子}{fen4zi3}
\end{entry}

\begin{entry}{分组}{fen1 zu3}{4,8}[HSK 3]
  \definition{v.}{agrupar; dividir em grupos}
\end{entry}

\begin{entry}{焚香}{fen2xiang1}{12,9}
  \definition{v.}{queimar incenso}
\end{entry}

\begin{entry}{粉}{fen3}{10}[Radical ⽶]
  \definition{s.}{pó | pó cosmético facial | alimento preparado a partir de amido | macarrão feito de qualquer tipo de farinha}
  \definition{v.}{tornar algo em pó | ser um fã de}
\end{entry}

\begin{entry}{粉色}{fen3 se4}{10,6}
  \definition{s.}{cor-de-rosa}
\end{entry}

\begin{entry}{粉丝}{fen3si1}{10,5}
  \definition{s.}{(empréstimo linguístico) fã | entusiasta de alguém ou alguma coisa}
  \definition[把]{s.}{aletria de amido de feijão | aletria chinesa | macarrão de celofane ou macarrão de vidro (transparente)}
\end{entry}

\begin{entry}{分}{fen4}{4}[Radical 刀][HSK 2]
  \definition{s.}{parte | ingrediente | componente}
  \seeref{分}{fen1}
\end{entry}

\begin{entry}{分量}{fen4liang4}{4,12}
  \definition{s.}{tamanho da porção (comida)}
  \seeref{分量}{fen1liang4}
  \seeref{分量}{fen4liang5}
\end{entry}

\begin{entry}{分量}{fen4liang5}{4,12}
  \definition{s.}{quantidade | peso | medida}
  \seeref{分量}{fen1liang4}
  \seeref{分量}{fen4liang4}
\end{entry}

\begin{entry}{分子}{fen4zi3}{4,3}
  \definition{s.}{membros de uma classe ou grupo | elementos políticos (como intelectuais ou extremistas)}
  \seeref{分子}{fen1zi3}
\end{entry}

\begin{entry}{份}{fen4}{6}[Radical 人][HSK 2]
  \definition{clas.}{para presentes, jornais, revistas, papéis, relatórios, contratos, etc. ou pratos (refeição)}
\end{entry}

\begin{entry}{奋战}{fen4zhan4}{8,9}
  \definition{v.}{lutar bravamente | trabalhar duro}
\end{entry}

\begin{entry}{愤怒}{fen4nu4}{12,9}
  \definition{adj.}{zangado | indignado}
  \definition{s.}{ira}
\end{entry}

\begin{entry}{愤世嫉俗}{fen4shi4ji2su2}{12,5,13,9}
  \definition{v.}{ser cínico | ser amargurado}
\end{entry}

\begin{entry}{丰富}{feng1fu4}{4,12}[HSK 3]
  \definition{adj.}{rico; abundante; pleno}
  \definition{v.}{enriquecer}
\end{entry}

\begin{entry}{丰收}{feng1shou1}{4,6}
  \definition{s.}{colheita abundante}
\end{entry}

\begin{entry}{风}{feng1}{4}[Radical 風][Kangxi 182][HSK 1]
  \definition[阵,丝]{s.}{vento}
\end{entry}

\begin{entry}{风景}{feng1jing3}{4,12}
  \definition{s.}{cenário | paisagem}
\end{entry}

\begin{entry}{风扇}{feng1shan4}{4,10}
  \definition{s.}{ventilador elétrico}
\end{entry}

\begin{entry}{风险}{feng1xian3}{4,9}[HSK 3]
  \definition[个,种,项,类]{s.}{risco; perigo}
\end{entry}

\begin{entry}{风筝}{feng1zheng5}{4,12}
  \definition{s.}{pipa | papagaio | pandorga}
\end{entry}

\begin{entry}{枫叶}{feng1ye4}{8,5}
  \definition{s.}{folha de bordo (maple, tipo de árvore)}
\end{entry}

\begin{entry}{封}{feng1}{9}[Radical 寸][HSK 2]
  \definition*{s.}{sobrenome Feng}
  \definition{clas.}{para objetos selados, especialmente cartas}
  \definition{v.}{conceder um título | conferir | conceder | selar}
\end{entry}

\begin{entry}{封闭}{feng1bi4}{9,6}
  \definition{v.}{fechar | selar | confinado}
\end{entry}

\begin{entry}{封底}{feng1di3}{9,8}
  \definition{s.}{contracapa de um livro}
\end{entry}

\begin{entry}{封冻}{feng1dong4}{9,7}
  \definition{v.}{congelar (água ou terra)}
\end{entry}

\begin{entry}{封盖}{feng1gai4}{9,11}
  \definition{s.}{boné | capa | selo}
  \definition{v.}{cobrir}
\end{entry}

\begin{entry}{封建}{feng1jian4}{9,8}
  \definition{adj.}{feudal}
  \definition{s.}{feudalismo}
\end{entry}

\begin{entry}{封口}{feng1kou3}{9,3}
  \definition{v.}{selar | fechar | curar (uma ferida) | manter os lábios selados}
\end{entry}

\begin{entry}{封面}{feng1mian4}{9,9}
  \definition{s.}{capa (de uma publicação) | sobrecapa}
\end{entry}

\begin{entry}{封印}{feng1yin4}{9,5}
  \definition{s.}{selo (em envelopes)}
\end{entry}

\begin{entry}{封斋}{feng1zhai1}{9,10}
  \definition*{s.}{Ramadã (Islã)}
\end{entry}

\begin{entry}{疯狂}{feng1kuang2}{9,7}
  \definition{adj.}{louco | frenético | selvagem}
\end{entry}

\begin{entry}{缝纫}{feng2ren4}{13,6}
  \definition{v.}{costurar}
\end{entry}

\begin{entry}{缝纫机}{feng2ren4ji1}{13,6,6}
  \definition[架]{s.}{máquina de costura}
\end{entry}

\begin{entry}{凤凰}{feng4huang2}{4,11}
  \definition{s.}{fênix}
\end{entry}

\begin{entry}{佛}{fo2}{7}[Radical 人]
  \definition*{s.}{Buda, abreviação de 佛陀 | Budismo}
  \seeref{佛}{fu2}
  \seealsoref{佛陀}{fo2tuo2}
\end{entry}

\begin{entry}{佛陀}{fo2tuo2}{7,7}
  \definition{s.}{Buda (uma pessoa que atingiu a Budeidade, ou especificamente Siddhartha Gautama)}
\end{entry}

\begin{entry}{否定}{fou3ding4}{7,8}[HSK 3]
  \definition{adj.}{negativo}
  \definition{s.}{negativo (resposta); negação}
  \definition{v.}{rejeitar; negar}
\end{entry}

\begin{entry}{否认}{fou3ren4}{7,4}[HSK 3]
  \definition{v.}{negar; repudiar}
\end{entry}

\begin{entry}{否则}{fou3ze2}{7,6}
  \definition{conj.}{caso contrário | ou}
\end{entry}

\begin{entry}{夫妻}{fu1qi1}{4,8}
  \definition{s.}{casal | marido e eposa}
\end{entry}

\begin{entry}{佛}{fu2}{7}[Radical 人]
  \definition{adv.}{aparentemente}
  \definition{s.}{ornamento da cabeça (feminino)}
  \seeref{佛}{fo2}
\end{entry}

\begin{entry}{扶梯}{fu2ti1}{7,11}
  \definition{s.}{escada rolante}
\end{entry}

\begin{entry}{服}{fu2}{8}[Radical ⽉]
  \definition{s.}{roupas | vestido | vestuário | roupa de luto}
  \definition{v.}{servir (nas forças armadas, uma sentença de prisão, etc.) | obedecer | ser convencido (por um argumento) | convencer | admirar | aclimatar | tomar (medicamento) | usar roupas de luto}
  \seeref{服}{fu4}
\end{entry}

\begin{entry}{服务}{fu2 wu4}{8,5}[HSK 2]
  \definition{v.}{prestar serviço a | estar a serviço de | servir | trabalhar | servir}
\end{entry}

\begin{entry}{服务员}{fu2wu4yuan2}{8,5,7}
  \definition{s.}{atendente | garçom | garçonete | pessoal de atendimento ao cliente}
\end{entry}

\begin{entry}{服装}{fu2zhuang1}{8,12}[HSK 3]
  \definition[套,件,身]{s.}{roupas; trajes; fantasias}
\end{entry}

\begin{entry}{浮力}{fu2li4}{10,2}
  \definition{s.}{flutuabilidade}
\end{entry}

\begin{entry}{浮图}{fu2tu2}{10,8}
  \definition*{s.}{Termo alternativo para 佛陀}
  \variantof{浮屠}
  \seealsoref{佛陀}{fo2tuo2}
\end{entry}

\begin{entry}{浮屠}{fu2tu2}{10,11}
  \definition*{s.}{Buda | Templo (Stupa) Budista (transliteração de Pali Thuo)}
\end{entry}

\begin{entry}{符合}{fu2he2}{11,6}
  \definition{conj.}{de acordo com | concordando com | contando com | alinhado com}
  \definition{v.}{concordar com | estar em conformidade com | corresponder com | gerenciar | lidar}
\end{entry}

\begin{entry}{福}{fu2}{13}[Radical 示][HSK 3]
  \definition*{s.}{sobrenome Fu}
  \definition{s.}{benção; felicidade; boa sorte; boa fortuna}
  \definition{v.}{curvar-se; reverenciar}
\end{entry}

\begin{entry}{福克斯}{fu2ke4si1}{13,7,12}
  \definition*{s.}{Fox (empresa de mídia) | Focus (automóvel fabricado pela Ford)}
\end{entry}

\begin{entry}{福泽}{fu2ze2}{13,8}
  \definition{s.}{boa sorte}
\end{entry}

\begin{entry}{父母}{fu4 mu3}{4,5}[HSK 3]
  \definition{s.}{pai e mãe; pais}
\end{entry}

\begin{entry}{父母亲}{fu4mu3qin1}{4,5,9}
  \definition{s.}{pais}
\end{entry}

\begin{entry}{父亲}{fu4qin1}{4,9}[HSK 3]
  \definition[个,位]{s.}{pai}
\end{entry}

\begin{entry}{付}{fu4}{5}[Radical 人][HSK 3]
  \definition*{s.}{sobrenome Fu}
  \definition{clas.}{para pares ou conjuntos de coisas | para expressões faciais}
  \definition{v.}{comprometer-se a; entregar (entregar) a; entregar | pagar}
\end{entry}

\begin{entry}{付款}{fu4kuan3}{5,12}
  \definition{s.}{pagamento}
  \definition{v.+compl.}{pagar uma quantia em dinheiro}
\end{entry}

\begin{entry}{负责}{fu4ze2}{6,8}[HSK 3]
  \definition{adj.}{consciencioso}
  \definition{v.}{ser responsável por; estar encarregado de}
\end{entry}

\begin{entry}{附近}{fu4jin4}{7,7}
  \definition{adv.}{aqui perto | perto daqui}
\end{entry}

\begin{entry}{服}{fu4}{8}[Radical ⽉]
  \definition{clas.}{(para remédio) dose}
  \seeref{服}{fu2}
\end{entry}

\begin{entry}{复活节}{fu4huo2jie2}{9,9,5}
  \definition*{s.}{Páscoa}
\end{entry}

\begin{entry}{复刻}{fu4ke4}{9,8}
  \definition{v.}{reimprimir (um trabalho que esteve fora do catálogo) | reeditar (um disco de vinil, um CD, etc.) | replicar | recriar | (empréstimo linguístico) (computação) \emph{fork}}
\end{entry}

\begin{entry}{复习}{fu4xi2}{9,3}[HSK 2]
  \definition{s.}{revisão}
  \definition{v.}{rever | revisar}
\end{entry}

\begin{entry}{复印}{fu4yin4}{9,5}[HSK 3]
  \definition{v.}{fotografar; fotocopiar; duplicar}
\end{entry}

\begin{entry}{复杂}{fu4za1}{9,6}[HSK 3]
  \definition{adj.}{complexo; complicado}
\end{entry}

\begin{entry}{副}{fu4}{11}[Radical 刀]
  \definition{clas.}{para pares, conjuntos de coisas e expressões faciais | para óculos, luvas, etc.}
\end{entry}

\begin{entry}{富}{fu4}{12}[Radical 宀][HSK 3]
  \definition*{s.}{sobrenome Fu}
  \definition{adj.}{rico; póspero | rico; abundante}
  \definition{s.}{fortuna; riqueza}
\end{entry}

\begin{entry}{覆盆子}{fu4pen2zi5}{18,9,3}
  \definition{s.}{framboesa}
\end{entry}

%%%%% EOF %%%%%


%%%
%%% G
%%%
\section*{G}
\addcontentsline{toc}{section}{G}
\begin{multicols}{2}

\begin{verbete}[干杯]{gan1bei1}
\significado{gan1bei1}{v.+compl.}{
    brindar até a última gota; ``saúde!''
}
\end{verbete}

\begin{verbete}[干净]{gan1jing4}
\significado{gan1jing4}{adj.}{
    limpo
}
\end{verbete}

\begin{verbete}[赶快]{gan3kuai4}
\significado{gan3kuai4}{adv.}{
    rapidamente, imediatamente
}
\end{verbete}

\begin{verbete}[橄榄球]{gan3lan3qiu2}
\significado{gan3lan3qiu2}{n.}{
    rúgbi
}
\end{verbete}

\begin{verbete}[感冒]{gan3mao4}
\significado{gan3mao4}{v.}{
    ficar resfriado; estar com resfriado
}
\end{verbete}

\begin{verbete}[干]{gan4}
\significado{gan4}{v.}{
    fazer
}
\end{verbete}

\begin{verbete}[刚]{gang1}
\significado{gang1}{adv.}{
    acabar de
}
\end{verbete}

\begin{verbete}[高]{gao1}
\significado{gao1}{adj.}{
    alto
}
\end{verbete}

\begin{verbete}[高兴]{gao1xing4}
\significado{gao1xing4}{adj.}{
    feliz; alegre; contente
}
\end{verbete}

\begin{verbete}[告诉]{gao4su0}
\significado{gao4su0}{v.}{
    contar; dar a conhecer; dizer
}
\end{verbete}

\begin{verbete}[歌]{ge1}
\significado{ge1}{n.}{
    canção; canto
}
\end{verbete}

\begin{verbete}[哥哥]{ge1ge0}
\significado{ge1ge0}{n.}{
    irmão mais velho
}
\end{verbete}

\begin{verbete}[个]{ge4}
\significado{ge4}{p.c.}{
    de uso geral
}
\end{verbete}

\begin{verbete}[给]{gei3}
\significado{gei3}{pre.}{
    a; para
}
\significado{gei3}{v.}{
    dar
}
\end{verbete}

\begin{verbete}[给\ ······\ 打\ 电话]{gei3\ ...\ da3\ dian4hua4}
\significado{gei3\ ...\ da3\ dian4hua4}{}{
    telefonar para alguém
}
\end{verbete}

\begin{verbete}[跟]{gen1}
\significado{gen1}{prep.}{
    com
}
\end{verbete}

\begin{verbete}[根据]{gen1ju4}
\significado{gen1ju4}{prep.}{
    de acordo com
}
\end{verbete}

\begin{verbete}[更]{geng4}
\significado{geng4}{adv.}{
    mais
}
\end{verbete}

\begin{verbete}[工作]{gong1zuo4}
\significado{gong1zuo4}{n.}{
    trabalho
}
\significado{gong1zuo4}{v.}{
    trabalhar
}
\end{verbete}

\begin{verbete}[公共汽车]{gong1gong4qi4che1}
\significado{gong1gong4qi4che1}{n.}{
    ônibus
}
\end{verbete}

\begin{verbete}[公克]{gong1ke4}
\significado{gong1ke4}{n.}{
    trabalho escolar; trabalho de casa
}
\end{verbete}

\begin{verbete}[公司]{gong1si1}
\significado{gong1si1}{n.}{
    empresa; companhia
}
\end{verbete}

\begin{verbete}[公园]{gong1yuan2}
\significado{gong1yuan2}{n.}{
    parque
}
\end{verbete}

\begin{verbete}[狗]{gou3}
\significado{gou3}{n.}{
    cão; cachorro|
    \pc{条/只}
}
\end{verbete}

\begin{verbete}[故宫]{Gu4gong1}
\significado{Gu4gong1}{n.}{
    Palácio Imperial
}
\end{verbete}

\begin{verbete}[刮]{gua1}
\significado{gua1}{v.}{
    ventar, soprar (vento)
}
\end{verbete}

\begin{verbete}[刮风]{gua1feng1}
\significado{gua1feng1}{v.+compl.}{
    ventanejar; fazer vento
}
\end{verbete}

\begin{verbete}[拐]{guai3}
\significado{guai3}{v.}{
    virar; cortar
}
\end{verbete}

\begin{verbete}[光盘]{guang1pan2}
\significado{guang1pan2}{n.}{
    CD; disco compacto
}
\end{verbete}

\begin{verbete}[广东]{guang3dong1}
\significado{guang3dong1}{n.}{
    Guangdong
}
\end{verbete}

\begin{verbete}[规定]{gui1ding4}
\significado{gui1ding4}{n.}{
    regulamento
}
\significado{gui1ding4}{v.}{
    estipular
}
\end{verbete}

\begin{verbete}[贵]{gui4}
\significado{gui4}{adj.}{
    caro
}
\end{verbete}

\begin{verbete}[贵姓]{gui4xing4}
\significado{gui4xing4}{interr.}{
    seu sobrenome
}
\end{verbete}

\begin{verbete}[国]{guo2}
\significado{guo2}{n.}{
    país
}
\end{verbete}

\begin{verbete}[国家]{guo2jia1}
\significado{guo2jia1}{n.}{
    país
}
\end{verbete}

\begin{verbete}[果酱]{guo3jiang4}
\significado{guo3jiang4}{n.}{
    geléia; compota ou doce (de frutas)
}
\end{verbete}

\begin{verbete}[过]{guo4}
\significado{guo4}{v.}{
    passar
}
\significado{guo4}{part.}{
    passado
}
\end{verbete}

\begin{verbete}[过年]{guo4nian2}
\significado{guo4nian2}{v.}{
    festejar o Ano Novo Chinês
}
\end{verbete}

\begin{verbete}[过期]{guo4qi1}
\significado{guo4qi1}{v.+compl.}{
    exceder a data; passar a data
}
\end{verbete}

\end{multicols}

%%%
%%% H
%%%

\section*{H}\addcontentsline{toc}{section}{H}

\begin{entry}{哈哈}{ha1 ha1}{9,9}{⼝、⼝}[HSK 3]
  \definition{expr.}{(onomatopéia)  ha ha; o som de uma risada alta}
\end{entry}

\begin{entry}{哈马斯}{ha1ma3si1}{9,3,12}{⼝、⾺、⽄}
  \definition*{s.}{Hamas (Grupo Palestino)}
\end{entry}

\begin{entry}{咳}{hai1}{9}{⼝}
  \definition{interj.}{expressa tristeza, arrependimento ou espanto}
\end{entry}

\begin{entry}{还}{hai2}{7}{⾡}[HSK 1]
  \definition{adv.}{ainda; indica que a ação ou estado permanece inalterado, equivalente a 仍然 | também; além disso; em adição; indica que há um aumento ou suplemento além do escopo já indicado | ainda mais; usado com 比 para indicar que as características e o grau das coisas comparadas aumentaram, o que é equivalente a 更加 razoavelmente; medianamente; usado antes de um adjetivo, indica que algo atinge apenas o nível mínimo exigido | mesmo; usado na primeira parte da frase como complemento, e na segunda parte como conclusão, equivalente a 尚且 | que expressa realização ou descoberta; expressa surpresa por algo que não se esperava, mas que acabou acontecendo | tão cedo quanto; por um curto período de tempo; indica que já era assim há muito tempo | para dar ênfase; para reforçar o tom}
  \seeref{还}{huan2}
  \seealsoref{比}{bi3}
  \seealsoref{更加}{geng4 jia1}
  \seealsoref{仍然}{reng2ran2}
  \seealsoref{尚且}{shang4qie3}
\end{entry}

\begin{entry}{还是}{hai2shi5}{7,9}{⾡、⽇}[HSK 1]
  \definition{adv.}{ainda; ainda assim; não é a continuação de um determinado estado, fenômeno ou ação; o resultado é o mesmo de antes, sem mudanças  |que expressa uma preferência por uma alternativa; expressa comparação ou escolha feita após consideração cuidadosa, frequentemente usado para fazer sugestões | que expressa realização ou descoberta; indica que o resultado final foi inesperado}
  \definition{conj.}{ou (somente para frases interrogativas); indica várias opções, geralmente usado em perguntas | tudo; se; não importa; independentemente de; significa que, independentemente das mudanças que ocorram, o resultado permanecerá o mesmo}
\end{entry}

\begin{entry}{还有}{hai2 you3}{7,6}{⾡、⽉}[HSK 1]
  \definition{adv.}{também; ainda; além disso; então novamente; enfatizar as partes complementares, excedentes ou não mencionadas além do que já é conhecido}
\end{entry}

\begin{entry}{孩子}{hai2 zi5}{9,3}{⼦、⼦}[HSK 1]
  \definition[个]{s.}{criança; crianças; pessoas com idade entre alguns anos ou na adolescência, geralmente com menos de 14 anos | crianças; filho ou filha}
\end{entry}

\begin{entry}{海}{hai3}{10}{⽔}[HSK 2]
  \definition*{s.}{sobrenome Hai}
  \definition{adj.}{extragrande; de grande capacidade; descreve capacidade, tom de voz, etc.}
  \definition{adv.}{aleatoriamente; sem rumo; sem limites; sem restrições}
  \definition[片]{s.}{mar; grande lago; a parte do oceano próxima à costa, alguns grandes lagos também são chamados de mar | grande número de pessoas ou coisas reunidas; metáfora para muitas coisas semelhantes que formam um grande conjunto}
\end{entry}

\begin{entry}{海边}{hai3 bian1}{10,5}{⽔、⾡}[HSK 2]
  \definition{s.}{praia; costa; litoral; orla marítima; a parte marginal do oceano e as grandes áreas de água salgada cercadas por terra firme, onde a terra e a água se encontram, formam a costa}
\end{entry}

\begin{entry}{海底}{hai3di3}{10,8}{⽔、⼴}
  \definition{adj.}{submarino}
  \definition{s.}{fundo do mar | solo oceânico | fundo do oceano}
\end{entry}

\begin{entry}{海风}{hai3feng1}{10,4}{⽔、⾵}
  \definition{s.}{brisa do mar | vento que vem do mar}
\end{entry}

\begin{entry}{海关}{hai3guan1}{10,6}{⽔、⼋}[HSK 3]
  \definition{s.}{alfândega}
\end{entry}

\begin{entry}{海浪}{hai3lang4}{10,10}{⽔、⽔}
  \definition{s.}{ondas do mar}
\end{entry}

\begin{entry}{海里}{hai3li3}{10,7}{⽔、⾥}
  \definition{s.}{milha náutica}
\end{entry}

\begin{entry}{海绵}{hai3mian2}{10,11}{⽔、⽷}
  \definition{s.}{(zoologia) esponja do mar | esponja (feita de poliéster ou celulose, etc.) | espuma de borracha}
\end{entry}

\begin{entry}{海鸥}{hai3'ou1}{10,9}{⽔、⿃}
  \definition{s.}{gaivota}
\end{entry}

\begin{entry}{海水}{hai3 shui3}{10,4}{⽔、⽔}[HSK 4]
  \definition[把]{s.}{água do mar; salmoura}
\end{entry}

\begin{entry}{海棠}{hai3tang2}{10,12}{⽔、⽊}
  \definition{s.}{begônia}
\end{entry}

\begin{entry}{海鲜}{hai3xian1}{10,14}{⽔、⿂}[HSK 4]
  \definition[种,份,桌,批,些]{s.}{frutos do mar; mariscos; peixes marinhos frescos, camarões, etc., para consumo |}
\end{entry}

\begin{entry}{害}{hai4}{10}{⼧}[HSK 5]
  \definition{adj.}{prejudicial; destrutivo; injurioso; nocivo}
  \definition{s.}{mal; maldade; dano; calamidade}
  \definition{v.}{prejudicar; fazer mal a; causar problemas a | matar; assassinar | sofrer de; contrair (uma doença) | sentir-se (envergonhado, com medo, etc.); despertar (um sentimento ou uma emoção)}
\end{entry}

\begin{entry}{害怕}{hai4pa4}{10,8}{⼧、⼼}[HSK 3]
  \definition{v.}{estar assustado; ter medo}
\end{entry}

\begin{entry}{害羞}{hai4xiu1}{10,10}{⼧、⽺}
  \definition{adj.}{tímido | envergonhado}
\end{entry}

\begin{entry}{汗}{han2}{6}{⽔}
  \definition*{s.}{abreviação de Khan}[他是成吉思汗。(Ele é Genghis Khan.)]
  \seeref{汗}{han4}
\end{entry}

\begin{entry}{含}{han2}{7}{⼝}[HSK 4]
  \definition{v.}{manter na boca (sem engolir ou cuspir) | conter; incluir | cuidar; acalentar; abrigar}
\end{entry}

\begin{entry}{含金量}{han2jin1liang4}{7,8,12}{⼝、⾦、⾥}
  \definition{adj.}{conteúdo de ouro | (fig.) valioso}
\end{entry}

\begin{entry}{含量}{han2 liang4}{7,12}{⼝、⾥}[HSK 4]
  \definition{s.}{conteúdo; a quantidade de um componente contido em uma substância}
\end{entry}

\begin{entry}{含义}{han2yi4}{7,3}{⼝、⼂}[HSK 4]
  \definition[个,种,层]{s.}{sentido; mensagem; significado; implicação}
\end{entry}

\begin{entry}{含有}{han2 you3}{7,6}{⼝、⽉}[HSK 4]
  \definition{v.}{conter; ter; incluir}
\end{entry}

\begin{entry}{函数}{han2shu4}{8,13}{⼐、⽁}
  \definition{s.}{função (matemática)}
\end{entry}

\begin{entry}{涵}{han2}{11}{⽔}
  \definition{s.}{bueiro | galeria}
  \definition{v.}{conter | incluir | entupir}
\end{entry}

\begin{entry}{寒假}{han2jia4}{12,11}{⼧、⼈}[HSK 4]
  \definition[个]{s.}{férias de inverno (feriados); férias escolares no meio do inverno, em janeiro e fevereiro (na China)}
\end{entry}

\begin{entry}{寒冷}{han2 leng3}{12,7}{⼧、⼎}[HSK 4]
  \definition{adj.}{frio; frígido; gélido; gelado}
\end{entry}

\begin{entry}{韩国}{han2guo2}{12,8}{⾱、⼞}
  \definition*{s.}{Coréia do Sul}
\end{entry}

\begin{entry}{韩国人}{han2guo2ren2}{12,8,2}{⾱、⼞、⼈}
  \definition{s.}{coreano | pessoa ou povo da Coréia}
\end{entry}

\begin{entry}{厂}{han3}{2}{⼚}[Kangxi 27]
  \definition{s.}{radical ``penhasco'' em caracteres chineses (radical Kangxi 27)}
  \seeref{厂}{chang3}
\end{entry}

\begin{entry}{喊}{han3}{12}{⼝}[HSK 2]
  \definition{v.}{gritar; clamar; berrar | chamar (uma pessoa) | chamar; dirigir-se a}
\end{entry}

\begin{entry}{汉}{han4}{5}{⽔}
  \definition{s.}{grupo étnico Han | chinês (língua) | dinastia Han (206 a.C.-220d.C.) | homem}
\end{entry}

\begin{entry}{汉堡包}{han4bao3bao1}{5,12,5}{⽔、⼟、⼓}
  \definition[个]{s.}{hambúrguer}
\end{entry}

\begin{entry}{汉堡王}{han4bao3wang2}{5,12,4}{⽔、⼟、⽟}
  \definition*{s.}{Burguer King (restaurante \emph{fast-food})}
\end{entry}

\begin{entry}{汉服}{han4fu2}{5,8}{⽔、⽉}
  \definition{s.}{vestido chinês tradicional Han}
\end{entry}

\begin{entry}{汉葡词典}{han4-pu2 ci2dian3}{5,12,7,8}{⽔、⾋、⾔、⼋}
  \definition[部,本]{s.}{dicionário chinês-português}
  \seealsoref{葡汉词典}{pu2-han4 ci2dian3}
\end{entry}

\begin{entry}{汉语}{han4yu3}{5,9}{⽔、⾔}[HSK 1]
  \definition[门]{s.}{língua chinesa, mandarim}
\end{entry}

\begin{entry}{汉字}{han4 zi4}{5,6}{⽔、⼦}[HSK 1]
  \definition[个]{s.}{caractere chinês; ideograma chinês; sinograma; com pouquíssimas exceções, os caracteres chineses representam uma sílaba cada um}
\end{entry}

\begin{entry}{汗}{han4}{6}{⽔}[HSK 5]
  \definition{s.}{suor; transpiração; perspiração}
  \seeref{汗}{han2}
\end{entry}

\begin{entry}{汗水}{han4shui3}{6,4}{⽔、⽔}
  \definition*{s.}{Rio Han (Hanshui)}
  \definition{s.}{suor | transpiração}
\end{entry}

\begin{entry}{汗腺}{han4xian4}{6,13}{⽔、⾁}
  \definition{s.}{glândula sudorípara}
\end{entry}

\begin{entry}{汗液}{han4ye4}{6,11}{⽔、⽔}
  \definition{s.}{suor}
\end{entry}

\begin{entry}{焊}{han4}{11}{⽕}
  \definition{v.}{soldar}
\end{entry}

\begin{entry}{撼}{han4}{16}{⼿}
  \definition{v.}{sacudir | vibrar}
\end{entry}

\begin{entry}{行}{hang2}{6}{⾏}[HSK 3]
  \definition{adj.}{temporário; improvisado | capaz; competente}
  \definition{adv.}{logo; em breve}
  \definition{clas.}{linha; fileira; coisas usadas para formar filas, linhas}
  \definition{s.}{comportamento; conduta | linha; fileira | empresa comercial; certas instituições comerciais | comércio; profissão; ramo de atividade | especialista; conhecedor; refere-se ao conhecimento e experiência em um determinado setor}
  \definition{v.}{ir; caminhar; viajar | estar atualizado; circular | fazer; executar; realizar | (antes de um verbo dissílabo, indicando a realização de alguma ação) | ficar bem; vai dar certo | (remédio) fazer efeito | classificar (entre irmãos e irmãs por ordem de idade)}
  \seeref{行}{heng2}
  \seeref{行}{xing2}
\end{entry}

\begin{entry}{行业}{hang2ye4}{6,5}{⾏、⼀}[HSK 4]
  \definition[种,个]{s.}{comércio; indústria; setor; profissão; categorias em negócios e indústria referem-se a ocupações em geral}
\end{entry}

\begin{entry}{航班}{hang2ban1}{10,10}{⾈、⽟}[HSK 4]
  \definition[个,次]{s.}{número do voo; voo programado}
\end{entry}

\begin{entry}{航空}{hang2kong1}{10,8}{⾈、⽳}[HSK 4]
  \definition{s.}{viagem; aviação; refere-se ao voo de uma aeronave no ar}
\end{entry}

\begin{entry}{航天员}{hang2tian1yuan2}{10,4,7}{⾈、⼤、⼝}
  \definition{s.}{astronauta}
\end{entry}

\begin{entry}{号}{hao2}{5}{⼝}
  \definition[个]{s.}{rugido | choro}
  \definition{v.}{uivar; gritar; gritar em voz alta e prolongada |
lamentar; chorar alto |
uivar; (vento) assobiar, assoviar}
  \seeref{号}{hao4}
\end{entry}

\begin{entry}{蚝}{hao2}{10}{⾍}
  \definition{s.}{ostra}
\end{entry}

\begin{entry}{毫不费力}{hao2bu2fei4li4}{11,4,9,2}{⽊、⼀、⾙、⼒}
  \definition{adj.}{sem esforço | não gastando o menor esforço}
\end{entry}

\begin{entry}{毫米}{hao2mi3}{11,6}{⽊、⽶}[HSK 4]
  \definition{clas.}{milímetro; unidade legal de medida de comprimento, 1 mm equivale a 0,1 cm}
\end{entry}

\begin{entry}{毫升}{hao2 sheng1}{11,4}{⽊、⼗}[HSK 4]
  \definition{clas.}{mililitro; unidade de volume, milésimo de um litro (ml)}
\end{entry}

\begin{entry}{豪华}{hao2hua2}{14,6}{⾗、⼗}
  \definition{adj.}{luxuoso}
\end{entry}

\begin{entry}{好}{hao3}{6}{⼥}[HSK 1,2,4]
  \definition{adj.}{bom; ótimo; agradável; vantajoso; satisfatório | amigável; gentil; amistoso; amável | saudável; bem | pronto; concluído; usado após um verbo para indicar conclusão ou perfeição | fácil (de fazer); conveniente; responsável (por)}
  \definition{adv.}{muito; bastante; tão; usado na frente de uma palavra de quantidade ou uma palavra de tempo para indicar muito ou por muito tempo | em que medida; como; usado antes de adjetivos e verbos para indicar profundidade e com exclamação}
  \definition{interj.}{O.K.; tudo bem; aprovação, acordo ou encerramento | (no início de uma frase ou oração) expressa concordância (ou desaprovação, surpresa, etc.)}
  \definition{prep.}{de modo a; para que}
  \definition{s.}{referindo-se a palavras de elogio ou aplauso | saudações; cumprimentos}
  \definition{suf.}{sufixo que indica conclusão ou prontidão | depois de um pronome significa ``olá''}
  \definition{v.}{dever; precisar; ter que | apaixonar-se}
  \seeref{好}{hao4}
\end{entry}

\begin{entry}{好吃}{hao3chi1}{6,6}{⼥、⼝}[HSK 1]
  \definition{adj.}{bom; saboroso; delicioso; descreve o sabor agradável de algo, que as pessoas gostam de comer}
  \seeref{好吃}{hao4chi1}
\end{entry}

\begin{entry}{好处}{hao3chu4}{6,5}{⼥、⼡}[HSK 2]
  \definition[个]{s.}{bom; benefício; vantagem; fatores favoráveis a pessoas ou coisas | ganho; lucro; algo que não se deveria receber, dado por outra pessoa ou obtido através de uma oportunidade; geralmente tem conotação pejorativa}
\end{entry}

\begin{entry}{好多}{hao3 duo1}{6,6}{⼥、⼣}[HSK 2]
  \definition{adj.}{muitos; uma boa quantidade; uma grande quantidade; uma quantidade enorme}
  \definition{pron.}{quantos?; quanto?; frequentemente usado para perguntar sobre quantidade}
\end{entry}

\begin{entry}{好汉}{hao3han4}{6,5}{⼥、⽔}
  \definition[条]{s.}{herói | pessoa forte e corajosa}
\end{entry}

\begin{entry}{好好}{hao3 hao3}{6,6}{⼥、⼥}[HSK 3]
  \definition{adj.}{realmente bom/bem; em perfeitas condições; quando tudo está bem}
  \definition{adv.}{diretamente; seriamente; cuidadosamente}
\end{entry}

\begin{entry}{好久}{hao3jiu3}{6,3}{⼥、⼃}[HSK 2]
  \definition{adv.}{por muito tempo | por eras (no passado)}
\end{entry}

\begin{entry}{好看}{hao3 kan4}{6,9}{⼥、⽬}[HSK 1]
  \definition{adj.}{de boa aparência; agradável; bonito | interessante; descreve o enredo ou conteúdo de filmes, romances, performances, etc., como sendo cativante, agradável ou apreciável}
\end{entry}

\begin{entry}{好人}{hao3 ren2}{6,2}{⼥、⼈}[HSK 2]
  \definition[个,位,名]{s.}{pessoa boa (ou excelente) (oposto de 坏人) | pessoa saudável | pessoa gentil que tenta se dar bem com todos (muitas vezes em detrimento dos princípios)}
  \seealsoref{坏人}{huai4 ren2}
\end{entry}

\begin{entry}{好生}{hao3sheng1}{6,5}{⼥、⽣}
  \definition{adv.}{bastante; extremamente | cuidadosamente; apropriadamente}
\end{entry}

\begin{entry}{好事}{hao3 shi4}{6,8}{⼥、⼅}[HSK 2]
  \definition[个,件]{s.}{boa ação; gentileza | (antigo) obra de caridade | acontecimento feliz; evento festivo}
  \seeref{好事}{hao4 shi4}
\end{entry}

\begin{entry}{好听}{hao3 ting1}{6,7}{⼥、⼝}[HSK 1]
  \definition{adj.}{agradável de ouvir (de som ou voz) | bom; palatável; satisfatório (de palavras)  | decente; honrado (de ações, etc.); descreve uma coisa que parece prestigiosa | interessante; descreve palavras, histórias e outras coisas interessantes}
\end{entry}

\begin{entry}{好玩儿}{hao3 wan2r5}{6,8,2}{⼥、⽟、⼉}[HSK 1]
  \definition{adj.}{divertido; interessante; capaz de despertar interesse}
\end{entry}

\begin{entry}{好象}{hao3xiang4}{6,11}{⼥、⾗}
  \variantof{好像}
\end{entry}

\begin{entry}{好像}{hao3xiang4}{6,13}{⼥、⼈}[HSK 2]
  \definition{adv.}{como se; um pouco parecido; como se fosse}
  \definition{v.}{parecer; ser como; parecer-se com}
\end{entry}

\begin{entry}{好心}{hao3xin1}{6,4}{⼥、⼼}
  \definition{s.}{bondade | boas intenções}
\end{entry}

\begin{entry}{好学}{hao3xue2}{6,8}{⼥、⼦}
  \definition{adj.}{fácil de aprender}
  \seeref{好学}{hao4xue2}
\end{entry}

\begin{entry}{好用}{hao3yong4}{6,5}{⼥、⽤}
  \definition{adj.}{fácil de usar | adequado ao uso}
\end{entry}

\begin{entry}{好友}{hao3you3}{6,4}{⼥、⼜}[HSK 4]
  \definition[位,个]{s.}{bom amigo; amigo próximo}
\end{entry}

\begin{entry}{好运}{hao3 yun4}{6,7}{⼥、⾡}[HSK 5]
  \definition{s.}{boa sorte, fortuna ou oportunidade}
\end{entry}

\begin{entry}{号}{hao4}{5}{⼝}[HSK 1]
  \definition{clas.}{usado para o número de pessoas |  tipo; espécie; classificação | para pessoas ou negócios; número de vezes utilizado para transações}
  \definition{num.}{dia do mês | usado para indicar o número de pessoas}
  \definition[个]{s.}{nome | nome presumido; nome alternativo; pseudônimo; apelido | casa de negócios; loja | marca; sinal; sinalização | número | data | ordem; no exército, as ordens são transmitidas verbalmente ou por meio de clarins | qualquer instrumento de sopro e latão; trombeta usada no exército ou em bandas | qualquer coisa usada como buzina | chamada de corneta; qualquer chamada feita em uma corneta; usar um apito para emitir um som com um significado específico | pessoa em uma condição especial; pessoas que se encontram em uma situação especial}
  \definition{suf.}{sufixo de navio}
  \definition{v.}{marcar; fazer uma marca | sentir; colocar a mão no pulso do paciente e avaliar a situação através do fluxo sanguíneo}
  \seeref{号}{hao2}
\end{entry}

\begin{entry}{号角}{hao4jiao3}{5,7}{⼝、⾓}
  \definition{s.}{corneta | trombeta}
\end{entry}

\begin{entry}{号码}{hao4ma3}{5,8}{⼝、⽯}[HSK 4]
  \definition[个,组,串]{s.}{número}
\end{entry}

\begin{entry}{号召}{hao4zhao4}{5,5}{⼝、⼝}[HSK 5]
  \definition{s.}{chamado; apelo; desejo ou pedido solene (de um governo, partido político, organização etc.) para que as massas façam algo}
  \definition{v.}{chamar;  (governo, partido político, organização, etc.) fazer um pedido solene às massas para que façam algo, na esperança de que todos se esforcem para alcançá-lo}
\end{entry}

\begin{entry}{好}{hao4}{6}{⼥}
  \definition*{s.}{sobrenome Hao}
  \definition{adv.}{algo que acontece com frequência, que é fácil de acontecer}
  \definition{v.}{gostar; amar; ter afeição por}
  \seeref{好}{hao3}
\end{entry}

\begin{entry}{好吃}{hao4chi1}{6,6}{⼥、⼝}
  \definition{v.}{ser guloso; gostar de comer boa comida}
  \seeref{好吃}{hao3chi1}
\end{entry}

\begin{entry}{好奇}{hao4qi2}{6,8}{⼥、⼤}[HSK 3]
  \definition{adj.}{curioso}
  \definition{s.}{curiosidade}
  \definition{v.}{ser ou estar curioso}
\end{entry}

\begin{entry}{好事}{hao4 shi4}{6,8}{⼥、⼅}
  \definition[个,件]{s.}{intrometido; gostar de se meter na vida dos outros}
  \seeref{好事}{hao3 shi4}
\end{entry}

\begin{entry}{好学}{hao4xue2}{6,8}{⼥、⼦}
  \definition{s.}{estudioso | erudito}
  \seeref{好学}{hao3xue2}
\end{entry}

\begin{entry}{呵}{he1}{8}{⼝}
  \definition{expr.}{Meu Deus! | expelir a respiração}
  \seeref{呵}{a1}
\end{entry}

\begin{entry}{欱}{he1}{10}{⽋}
  \variantof{喝}
\end{entry}

\begin{entry}{喝}{he1}{12}{⼝}[HSK 1]
  \definition{interj.}{Meu Deus!; Oh!; Ah!; Uau!}
  \definition{s.}{bebida; especificamente, vinho}
  \definition{v.}{beber; engolir líquidos ou alimentos líquidos | beber bebida alcoólica; referência específica ao consumo de álcool}
  \seeref{喝}{he4}
\end{entry}

\begin{entry}{喝醉}{he1zui4}{12,15}{⼝、⾣}
  \definition{v.}{ficar bêbado}
\end{entry}

\begin{entry}{合}{he2}{6}{⼝}[HSK 3]
  \definition*{s.}{sobrenome He}
  \definition{adj.}{todo; completo; inteiro}
  \definition{clas.}{para rodadas}
  \definition{s.}{conjunção}
  \definition{v.}{fechar | juntar; combinar | adequar-se; concordar; conformar-se a | ser igual a; somar}
\end{entry}

\begin{entry}{合并}{he2bing4}{6,6}{⼝、⼲}[HSK 5]
  \definition{v.}{fundir; amalgamar; combinar várias coisas em uma coisa só | (doença) ser complicada por outra doença; uma doença levar a outra, ataques simultâneos (de várias doenças)}
\end{entry}

\begin{entry}{合成}{he2cheng2}{6,6}{⼝、⼽}[HSK 5]
  \definition{s.}{compor; integrar; combinar; misturar | sintetizar, reação química para transformar uma substância com uma composição simples em uma substância com uma composição complexa}
\end{entry}

\begin{entry}{合法}{he2fa3}{6,8}{⼝、⽔}[HSK 3]
  \definition{adj.}{legal; legítimo; correto}
\end{entry}

\begin{entry}{合格}{he2ge2}{6,10}{⼝、⽊}[HSK 3]
  \definition{adj.}{qualificado; de acordo com o padrão}
\end{entry}

\begin{entry}{合理}{he2li3}{6,11}{⼝、⽟}[HSK 3]
  \definition{adj.}{racional; razoável; equitativo}
\end{entry}

\begin{entry}{合适}{he2shi4}{6,9}{⼝、⾡}[HSK 2]
  \definition{adj.}{correto; adequado; apropriado; conveniente; em conformidade com a realidade ou com os requisitos objetivos}
\end{entry}

\begin{entry}{合同}{he2tong5}{6,6}{⼝、⼝}[HSK 4]
  \definition[个,份]{s.}{contrato; acordo; uma disposição para observância mútua por duas ou mais partes na condução de um assunto com o objetivo de determinar seus respectivos direitos e obrigações.}
\end{entry}

\begin{entry}{合宪性}{he2xian4xing4}{6,9,8}{⼝、⼧、⼼}
  \definition{s.}{constitucionalismo}
\end{entry}

\begin{entry}{合资}{he2zi1}{6,10}{⼝、⾙}
  \definition{s.}{\emph{joint-venture} com capitais mistos}
\end{entry}

\begin{entry}{合作}{he2zuo4}{6,7}{⼝、⼈}[HSK 3]
  \definition[个]{s.}{cooperação; colaboração}
  \definition{v.}{cooperar; colaborar; trabalhar em conjunto}
\end{entry}

\begin{entry}{何}{he2}{7}{⼈}
  \definition*{s.}{sobrenome He}
  \definition{adv.}{expressa exclamação, equivalente a 多么}
  \definition{pron.}{O que?; Onde?; Por que? | expressa uma pergunta retórica, equivalente a 岂, 怎}
  \seealsoref{多么}{duo1me5}
  \seealsoref{岂}{qi3}
  \seealsoref{怎}{zen3}
\end{entry}

\begin{entry}{何况}{he2kuang4}{7,7}{⼈、⼎}
  \definition{conj.}{além disso | muito menos}
\end{entry}

\begin{entry}{和}{he2}{8}{⼝}[HSK 1]
  \definition*{s.}{sobrenome He}
  \definition{adj.}{gentil; suave; amável | harmonioso; em boas condições}
  \definition{conj.}{e (somente para palavras) | junto com}
  \definition{prep.}{relacionado com | para; com; indica relação; comparação, etc.}
  \definition{s.}{soma; soma total | japonês; refere-se ao Japão}
  \definition{v.}{disputar; reconciliar; acabar com a guerra ou a disputa | empatar; (próxima edição ou torneio) sem vencedor}
  \seeref{和}{he4}
  \seeref{和}{hu2}
  \seeref{和}{huo2}
  \seeref{和}{huo4}
\end{entry}

\begin{entry}{和平}{he2ping2}{8,5}{⼝、⼲}[HSK 3]
  \definition{adj.}{pacífico; não violento}
  \definition{s.}{paz}
\end{entry}

\begin{entry}{和平共处}{he2ping2gong4chu3}{8,5,6,5}{⼝、⼲、⼋、⼡}
  \definition{s.}{coexistência pacífica de nações, sociedades, etc.}
\end{entry}

\begin{entry}{和谐}{he2xie2}{8,11}{⼝、⾔}
  \definition{adj.}{harmonioso}
  \definition{s.}{harmonia}
  \definition{v.}{(eufemismo) censurar}
\end{entry}

\begin{entry}{河}{he2}{8}{⽔}[HSK 2]
  \definition*{s.}{o sistema da Via Láctea | o rio Amarelo; o rio Huanghe}
  \definition*{s.}{sobrenome He}
  \definition[条,道]{s.}{rio; refere-se a grandes cursos de água}
\end{entry}

\begin{entry}{河蚌}{he2bang4}{8,10}{⽔、⾍}
  \definition{s.}{mexilhões | bivalves cultivados em rios e lagos}
\end{entry}

\begin{entry}{核}{he2}{10}{⽊}
  \definition{adj.}{nuclear}
  \definition{s.}{poço | pedra | núcleo}
  \definition{v.}{examinar | checar | verificar}
\end{entry}

\begin{entry}{荷}{he2}{10}{⾋}
  \definition{s.}{lótus}
  \seeref{荷}{he4}
\end{entry}

\begin{entry}{荷花}{he2hua1}{10,7}{⾋、⾋}
  \definition{s.}{lótus}
\end{entry}

\begin{entry}{盒}{he2}{11}{⽫}[HSK 5]
  \definition{clas.}{caixa (de pequena dimensão)}
  \definition{s.}{caixa; estojo; recipiente; receptáculo}
\end{entry}

\begin{entry}{盒饭}{he2 fan4}{11,7}{⽫、⾷}[HSK 5]
  \definition{s.}{refeição embalada; marmita; \emph{fast-food} vendida em caixas}
\end{entry}

\begin{entry}{盒子}{he2zi5}{11,3}{⽫、⼦}[HSK 5]
  \definition[个]{s.}{caixa; recipiente que têm tampas na parte superior e podem conter coisas dentro, geralmente é pequeno e plano}
\end{entry}

\begin{entry}{和}{he4}{8}{⼝}
  \definition{v.}{compor um poema em resposta (ao poema de alguém) usando a mesma sequência de rimas | juntar-se à cantoria | cantar junto com outros}
  \seeref{和}{he2}
  \seeref{和}{hu2}
  \seeref{和}{huo2}
  \seeref{和}{huo4}
\end{entry}

\begin{entry}{贺}{he4}{9}{⾙}
  \definition*{s.}{sobrenome He}
  \definition{v.}{parabenizar | congratular}
\end{entry}

\begin{entry}{贺卡}{he4 ka3}{9,5}{⾙、⼘}[HSK 5]
  \definition[张]{s.}{cartão de felicitações; pedaço de papel para parabenizar amigos e parentes em seu casamento, aniversário ou festivais, geralmente impresso com palavras e desenhos de felicitações}
\end{entry}

\begin{entry}{荷}{he4}{10}{⾋}
  \definition{s.}{carga | responsabilidade}
  \definition{v.}{carregar no ombro ou nas costas}
  \seeref{荷}{he2}
\end{entry}

\begin{entry}{喝}{he4}{12}{⼝}
  \definition{v.}{gritar bem alto}
  \seeref{喝}{he1}
\end{entry}

\begin{entry}{喝彩}{he4cai3}{12,11}{⼝、⼺}
  \definition{s.}{aclamar | torcer}
\end{entry}

\begin{entry}{褐色}{he4 se4}{14,6}{⾐、⾊}
  \definition{s.}{cor marrom}
\end{entry}

\begin{entry}{鹤}{he4}{15}{⿃}
  \definition{s.}{grou (ave)}
\end{entry}

\begin{entry}{黑}{hei1}{12}{⿊}[HSK 2][Kangxi 203]
  \definition*{s.}{sobrenome Hei}
  \definition*{s.}{abreviação de Província de Heilongjiang, 黑龙江}
  \definition{adj.}{preto; cor semelhante à do carvão | escuro | obscuro; secreto | perverso; sinistro; ruim; cruel | reacionário}
  \definition{s.}{noite}
  \definition{v.}{fazer algo ilegalmente ou de forma desonesta; enganar; desviar dinheiro ilegalmente | invadir (uma rede, sites, computador, etc.)}
  \seealsoref{黑龙江}{hei1long2jiang1}
\end{entry}

\begin{entry}{黑暗}{hei1 an4}{12,13}{⿊、⽇}[HSK 4]
  \definition{adj.}{escuro; sombrio; sem luz | maligno; corrupto; reacionário}
\end{entry}

\begin{entry}{黑板}{hei1ban3}{12,8}{⿊、⽊}[HSK 2]
  \definition[块,个]{s.}{quadro negro; quadro de giz; uma placa, na qual se pode escrever com giz}
\end{entry}

\begin{entry}{黑客}{hei1ke4}{12,9}{⿊、⼧}
  \definition{s.}{(empréstimo linguístico) (computação) \emph{hacker}}
\end{entry}

\begin{entry}{黑龙江}{hei1long2jiang1}{12,5,6}{⿊、⿓、⽔}
  \definition*{s.}{Heilongjiang (Província)}
  \definition*{s.}{Rio Heilong Jiang; (na Rússia) o rio Amur}
\end{entry}

\begin{entry}{黑色}{hei1 se4}{12,6}{⿊、⾊}[HSK 2]
  \definition{adj.}{metafórico: suspeito, ilegal}
  \definition{s.}{cor preta}
\end{entry}

\begin{entry}{很}{hen3}{9}{⼻}[HSK 1]
  \definition{adv.}{muito; bastante; terrivelmente; indica um grau bastante elevado; definitivo; o mais alto}
\end{entry}

\begin{entry}{恨}{hen4}{9}{⼼}[HSK 5]
  \definition{s.}{ódio; resentimento}
  \definition{v.}{odiar}
\end{entry}

\begin{entry}{行}{heng2}{6}{⾏}
  \definition{s.}{usado em 道行}
  \seeref{行}{hang2}
  \seeref{行}{xing2}
  \seealsoref{道行}{dao4 heng2}
\end{entry}

\begin{entry}{恒星系}{heng2xing1xi4}{9,9,7}{⼼、⽇、⽷}
  \definition{s.}{sistema estelar | galáxia}
\end{entry}

\begin{entry}{横竖}{heng2shu5}{15,9}{⽊、⽴}
  \definition{adv.}{de qualquer maneira | independentemente (linguagem falada)}
\end{entry}

\begin{entry}{轰鸣}{hong1ming2}{8,8}{⾞、⿃}
  \definition{s.}{bum (som de explosão) | estrondo}
\end{entry}

\begin{entry}{轰炸机}{hong1zha4ji1}{8,9,6}{⾞、⽕、⽊}
  \definition{s.}{bombardeiro (aeronave)}
\end{entry}

\begin{entry}{哄}{hong1}{9}{⼝}
  \definition{s.}{gargalhadas | risadas ruidosas | algazarra | rugido | clamor}
  \seeref{哄}{hong3}
  \seeref{哄}{hong4}
\end{entry}

\begin{entry}{红}{hong2}{6}{⽷}[HSK 2]
  \definition*{s.}{sobrenome Hong}
  \definition{adj.}{vermelho | popular; bem-sucedido; símbolo de sucesso ou valorização | vermelho; revolucionário; símbolo da revolução | festivo; símbolo de alegria}
  \definition{s.}{tecido vermelho, bandeirinhas, etc. usados em ocasiões festivas | bônus; dividendo}
\end{entry}

\begin{entry}{红包}{hong2 bao1}{6,5}{⽷、⼓}[HSK 4]
  \definition[个]{s.}{saco de papel vermelho ou envelope contendo dinheiro como presente, gorjeta ou bônus | suborno; propina}
\end{entry}

\begin{entry}{红宝石}{hong2bao3shi2}{6,8,5}{⽷、⼧、⽯}
  \definition{s.}{rubi}
\end{entry}

\begin{entry}{红茶}{hong2 cha2}{6,9}{⽷、⾋}[HSK 3]
  \definition[杯,壶,斤,种]{s.}{chá preto}
\end{entry}

\begin{entry}{红酒}{hong2 jiu3}{6,10}{⽷、⾣}[HSK 3]
  \definition{s.}{vinho tinto}
\end{entry}

\begin{entry}{红绿灯}{hong2lv4deng1}{6,11,6}{⽷、⽷、⽕}
  \definition[个]{s.}{semáforo | sinal de trânsito}
\end{entry}

\begin{entry}{红色}{hong2 se4}{6,6}{⽷、⾊}[HSK 2]
  \definition{adj.}{vermelho; revolucionário; símbolo da revolução ou da consciência política elevada}
  \definition{s.}{cor vermelha}
\end{entry}

\begin{entry}{红烧}{hong2shao1}{6,10}{⽷、⽕}
  \definition{s.}{guisado em molho de soja (prato)}
\end{entry}

\begin{entry}{红薯}{hong2shu3}{6,16}{⽷、⾋}
  \definition{s.}{batata doce}
\end{entry}

\begin{entry}{红线}{hong2xian4}{6,8}{⽷、⽷}
  \definition{s.}{linha vermelha}
\end{entry}

\begin{entry}{洪水}{hong2shui3}{9,4}{⽔、⽔}
  \definition{s.}{enchente | inundação | dilúvio}
\end{entry}

\begin{entry}{哄}{hong3}{9}{⼝}
  \definition{v.}{enganar | persuadir | divertir (uma criança)}
  \seeref{哄}{hong1}
  \seeref{哄}{hong4}
\end{entry}

\begin{entry}{哄}{hong4}{9}{⼝}
  \definition{s.}{tumulto | agitação | perturbação}
  \seeref{哄}{hong1}
  \seeref{哄}{hong3}
\end{entry}

\begin{entry}{猴}{hou2}{12}{⽝}[HSK 5]
  \definition{adj.}{esperto; inteligente; perspicaz}
  \definition[只,群]{s.}{macaco}
\end{entry}

\begin{entry}{猴子}{hou2zi5}{12,3}{⽝、⼦}
  \definition[只]{s.}{macaco}
\end{entry}

\begin{entry}{后}{hou4}{6}{⼝}[HSK 1]
  \definition*{s.}{sobrenome Hou}
  \definition{s.}{atrás; traseiro; a direção oposta àquela para a qual a pessoa está voltada; a direção oposta àquela para a qual a parte de trás de uma casa está voltada (o oposto de 前)  | depois; mais tarde no tempo; futuro (em oposição a 先 ou 前) | último | posteridade; descendência | rainha; imperatriz | governante; soberano; monarca antigo}
  \seealsoref{前}{qian2}
  \seealsoref{先}{xian1}
\end{entry}

\begin{entry}{后边}{hou4 bian5}{6,5}{⼝、⾡}[HSK 1]
  \definition{adv.}{costas; traseira; atrás}
\end{entry}

\begin{entry}{后果}{hou4guo3}{6,8}{⼝、⽊}[HSK 3]
  \definition{s.}{consequência; resultado}
\end{entry}

\begin{entry}{后悔}{hou4hui3}{6,10}{⼝、⼼}[HSK 5]
  \definition{v.}{lamentar; ter remorso; arrepender-se}
\end{entry}

\begin{entry}{后来}{hou4lai2}{6,7}{⼝、⽊}[HSK 2]
  \definition{adv.}{mais tarde; depois; refere-se a um período posterior a um determinado momento no passado}
\end{entry}

\begin{entry}{后面}{hou4mian4}{6,9}{⼝、⾯}[HSK 3]
  \definition{adv.}{parte de trás; retaguarda; atrás | atrás; perto do fim; na parte de trás | mais tarde; depois}
  \seeref{后面}{hou4mian5}
\end{entry}

\begin{entry}{后面}{hou4mian5}{6,9}{⼝、⾯}[HSK 3]
  \definition{adv.}{parte de trás; retaguarda; atrás | atrás; perto do fim; na parte de trás | mais tarde; depois}
  \seeref{后面}{hou4mian4}
\end{entry}

\begin{entry}{后年}{hou4nian2}{6,6}{⼝、⼲}[HSK 3]
  \definition{s.}{o ano que vem; daqui a dois anos}
\end{entry}

\begin{entry}{后天}{hou4 tian1}{6,4}{⼝、⼤}[HSK 1]
  \definition{s.}{depois de amanhã; período em que uma pessoa ou animal vive e cresce sozinho após deixar o útero materno (em oposição a 先天)}
  \seealsoref{先天}{xian1tian1}
\end{entry}

\begin{entry}{后头}{hou4 tou5}{6,5}{⼝、⼤}[HSK 4]
  \definition{adv.}{posteriormente | atrás | mais tarde}
  \definition{s.}{a parte de trás | a parte traseira}
\end{entry}

\begin{entry}{厚}{hou4}{9}{⼚}[HSK 4]
  \definition*{s.}{sobrenome Hou}
  \definition{adj.}{grosso; espesso | profundo | bondoso; gentil; magnânimo | grande; generoso | rico ou forte em sabor}
  \definition{s.}{espessura; profundidade}
  \definition{v.}{favorecer; enfatizar}
\end{entry}

\begin{entry}{呼吸}{hu1xi1}{8,6}{⼝、⼝}[HSK 4]
  \definition{s.}{um suspiro; metáfora para um período de tempo muito curto}
  \definition{v.}{respirar}
\end{entry}

\begin{entry}{呼啸}{hu1xiao4}{8,11}{⼝、⼝}
  \definition{v.}{assobiar}
\end{entry}

\begin{entry}{忽然}{hu1ran2}{8,12}{⼼、⽕}[HSK 2]
  \definition{adv.}{repentinamente; de repente; sem aviso prévio; significa que algo aconteceu de forma rápida e inesperada}
\end{entry}

\begin{entry}{忽视}{hu1shi4}{8,8}{⼼、⾒}[HSK 4]
  \definition{v.}{ignorar; negligenciar; menosprezar; desprezar; dar de ombros}
\end{entry}

\begin{entry}{和}{hu2}{8}{⼝}
  \definition{v.}{completar um conjunto de Mahjong ou cartas de baralho}
  \seeref{和}{he2}
  \seeref{和}{he4}
  \seeref{和}{huo2}
  \seeref{和}{huo4}
\end{entry}

\begin{entry}{胡萝卜}{hu2luo2bo5}{9,11,2}{⾁、⾋、⼘}
  \definition{s.}{cenoura}
\end{entry}

\begin{entry}{胡同儿}{hu2 tong4r5}{9,6,2}{⾁、⼝、⼉}[HSK 5]
  \definition{s.}{beco; via; rua}
\end{entry}

\begin{entry}{胡子}{hu2 zi5}{9,3}{⾁、⼦}[HSK 5]
  \definition[团,根,个]{s.}{barba; bigode | bandido; salteador}
\end{entry}

\begin{entry}{湖}{hu2}{12}{⽔}[HSK 2]
  \definition*{s.}{abreviação de Huzhou, 湖州 | um nome que se refere às províncias de Hunan, 湖南,  e Hubei, 湖北}
  \definition[个,片]{s.}{lago}
  \seealsoref{湖北}{hu2bei3}
  \seealsoref{湖南}{hu2nan2}
  \seealsoref{湖州}{hu2zhou1}
\end{entry}

\begin{entry}{湖北}{hu2bei3}{12,5}{⽔、⼔}
  \definition*{s.}{Província de Hubei (Hupeh), na China central}
\end{entry}

\begin{entry}{湖南}{hu2nan2}{12,9}{⽔、⼗}
  \definition*{s.}{Província de Hunan (província do sul da China)}
\end{entry}

\begin{entry}{湖州}{hu2zhou1}{12,6}{⽔、⼮}
  \definition*{s.}{Cidade de Huzhou, em Zhejiang}
\end{entry}

\begin{entry}{葫芦}{hu2lu5}{12,7}{⾋、⾋}
  \definition{adj.}{confuso}
  \definition{s.}{cabaça | termo genérico para bloco e equipamento (ou partes dele)}
\end{entry}

\begin{entry}{糊里糊涂}{hu2li5hu2tu5}{15,7,15,10}{⽶、⾥、⽶、⽔}
  \definition{adj.}{desnorteado | perturbado}
\end{entry}

\begin{entry}{蝴蝶}{hu2die2}{15,15}{⾍、⾍}
  \definition[只]{s.}{borboleta}
\end{entry}

\begin{entry}{虎}{hu3}{8}{⾌}[HSK 5]
  \definition*{s.}{sobrenome Hu}
  \definition{adj.}{corajoso; bravo; valente; vigoroso}
  \definition{s.}{tigre}
  \definition{v.}{blefar; o mesmo que 唬 | parecer feroz; mostrar a aparência feroz de alguém}
  \seealsoref{唬}{hu3}
  \seealsoref{老虎}{lao3hu3}
\end{entry}

\begin{entry}{虎虎}{hu3hu3}{8,8}{⾌、⾌}
  \definition{adj.}{formidável | forte | vigoroso}
\end{entry}

\begin{entry}{虎口}{hu3kou3}{8,3}{⾌、⼝}
  \definition{s.}{lugar perigoso | cova do tigre}
\end{entry}

\begin{entry}{虎鼬}{hu3you4}{8,18}{⾌、⿏}
  \definition{s.}{doninha}
\end{entry}

\begin{entry}{唬}{hu3}{11}{⼝}
  \definition{v.}{blefar, exagerar para assustar ou confundir}
\end{entry}

\begin{entry}{互}{hu4}{4}{⼆}
  \definition{adj.}{mútuo | recíproco}
\end{entry}

\begin{entry}{互动}{hu4dong4}{4,6}{⼆、⼒}
  \definition{s.}{interativo}
  \definition{v.}{interagir}
\end{entry}

\begin{entry}{互利}{hu4li4}{4,7}{⼆、⼑}
  \definition{s.}{benefício mútuo}
\end{entry}

\begin{entry}{互联网}{hu4lian2wang3}{4,12,6}{⼆、⽿、⽹}[HSK 3]
  \definition{s.}{\emph{Internet}}
  \seealsoref{网际网路}{wang3ji4wang3lu4}
  \seealsoref{网际网络}{wang3ji4wang3luo4}
  \seealsoref{网路}{wang3lu4}
\end{entry}

\begin{entry}{互相}{hu4xiang1}{4,9}{⼆、⽬}[HSK 3]
  \definition{adv.}{mutuamente; um ao outro}
\end{entry}

\begin{entry}{户}{hu4}{4}{⼾}[HSK 4][Kangxi 63]
  \definition*{s.}{sobrenome Hu}
  \definition[个]{s.}{porta com um painel; porta | domicílio; residência; família | status familiar | conta (banco)}
\end{entry}

\begin{entry}{护士}{hu4shi5}{7,3}{⼿、⼠}[HSK 4]
  \definition[名,位]{s.}{enfermeiro; pessoas especializadas em enfermagem em hospitais ou instituições epidemiológicas}
\end{entry}

\begin{entry}{护照}{hu4zhao4}{7,13}{⼿、⽕}[HSK 2]
  \definition[本,个]{s.}{passaporte; documento emitido pela autoridade competente do país para comprovar a nacionalidade e a identidade dos cidadãos que viajam para o exterior}
\end{entry}

\begin{entry}{化}{hua1}{4}{⼔}[HSK 3]
  \definition*{s.}{sobrenome Hua}
  \definition{s.}{químico}
  \definition{suf.}{modernizar; modernização}
  \definition{v.}{mudar; converter; transformar | converter; influenciar | derreter; dissolver | digerir | queimar | morrer | pedir esmola}
  \variantof{花}
\end{entry}

\begin{entry}{花}{hua1}{7}{⾋}[HSK 1,2,4]
  \definition*{s.}{sobrenome Hua}
  \definition{adj.}{multicolorido; colorido | embaçado; obscuro; deslumbrado e confuso | extravagante; florido; vistoso}
  \definition[朵,支,束,把,盆,簇]{s.}{flor; órgãos de reprodução sexual de plantas com sementes | flor; planta ornamental |  qualquer coisa que se assemelhe a uma flor | fogos de artifício | padrão; design; design decorativo | flor; metáfora para a essência de uma causa | prostituta; cortesã; referindo-se a prostitutas ou a assuntos relacionados a prostitutas | algodão | varíola | ferimento; ferida; lesões traumáticas sofridas em combate}
  \definition{v.}{gastar; despender; consumir}
\end{entry}

\begin{entry}{花茶}{hua1cha2}{7,9}{⾋、⾋}
  \definition[杯,壶]{s.}{chá perfumado}
\end{entry}

\begin{entry}{花店}{hua1dian4}{7,8}{⾋、⼴}
  \definition{s.}{floricultura}
\end{entry}

\begin{entry}{花儿}{hua1r5}{7,2}{⾋、⼉}
  \definition[朵,支,束,把,盆,簇]{s.}{flor}
\end{entry}

\begin{entry}{花生}{hua1sheng1}{7,5}{⾋、⽣}
  \definition[粒]{s.}{amendoim}
\end{entry}

\begin{entry}{花样游泳}{hua1yang4you2yong3}{7,10,12,8}{⾋、⽊、⽔、⽔}
  \definition{s.}{nado sincronizado}
\end{entry}

\begin{entry}{花椰菜}{hua1ye1cai4}{7,12,11}{⾋、⽊、⾋}
  \definition{s.}{couve-flor}
\end{entry}

\begin{entry}{花园}{hua1 yuan2}{7,7}{⾋、⼞}[HSK 2]
  \definition[个,座]{s.}{jardim; um local onde se plantam flores e árvores para passear e descansar}
\end{entry}

\begin{entry}{划}{hua2}{6}{⼑}[HSK 4]
  \definition{adj.}{rentável; vale (o esforço); compensa (fazer alguma coisa)}
  \definition{v.}{remar | ser vantajoso para alguém; ser uma pechincha | arranhar; cortar a superfície de; cortar em outra coisa com um objeto pontiagudo | arranhar; golpear;  esfregar uma coisa ou varrer sobre outra}
  \seeref{划}{hua4}
\end{entry}

\begin{entry}{划船}{hua2 chuan2}{6,11}{⼑、⾈}[HSK 3]
  \definition[次,回]{s.}{remo (ato de remar); passeios de barco}
  \definition{v.}{remar um barco}
\end{entry}

\begin{entry}{划艇}{hua2ting3}{6,12}{⼑、⾈}
  \definition{s.}{barco a remo}
\end{entry}

\begin{entry}{华人}{hua2 ren2}{6,2}{⼗、⼈}[HSK 3]
  \definition{s.}{Chinês; chinês étnico | cidadãos estrangeiros de ascendência chinesa que adquiriram nacionalidade no seu país de residência}
\end{entry}

\begin{entry}{华盛顿}{hua2sheng4dun4}{6,11,10}{⼗、⽫、⾴}
  \definition*{s.}{Washington}
\end{entry}

\begin{entry}{华氏}{hua2shi4}{6,4}{⼗、⽒}
  \definition{s.}{graus Fahrenheit (°F)}
\end{entry}

\begin{entry}{华夏}{hua2xia4}{6,10}{⼗、⼢}
  \definition*{s.}{Huaxia, nome antigo da China | Catai}
\end{entry}

\begin{entry}{华裔}{hua2yi4}{6,13}{⼗、⾐}
  \definition{s.}{descendente de chinês}
\end{entry}

\begin{entry}{华语}{hua2 yu3}{6,9}{⼗、⾔}[HSK 5]
  \definition*{s.}{Chinês (idioma)}
\end{entry}

\begin{entry}{滑}{hua2}{12}{⽔}[HSK 5]
  \definition*{s.}{sobrenome Hua}
  \definition{adj.}{escorregadio; liso; objetos com superfícies lisas e baixo atrito | astuto; ardiloso; escorregadio}
  \definition{v.}{escorregar; deslizar | se atrapalhar; se safar de algo}
\end{entry}

\begin{entry}{滑雪}{hua2xue3}{12,11}{⽔、⾬}
  \definition{v.+compl.}{esquiar | praticar esqui}
\end{entry}

\begin{entry}{化石}{hua4shi2}{4,5}{⼔、⽯}[HSK 5]
  \definition{s.}{fóssil; restos, relíquias ou vestígios de organismos antigos enterrados no solo e transformados em objetos semelhantes a pedras}
\end{entry}

\begin{entry}{化学}{hua4xue2}{4,8}{⼔、⼦}
  \definition{s.}{química (disciplina)}
\end{entry}

\begin{entry}{划}{hua4}{6}{⼑}[HSK 4]
  \definition{s.}{traço de um caracter chinês}
  \definition{v.}{delimitar; diferenciar; delinear | transferir; ceder | planejar; programar | desenhar; marcar; delinear; fazer linhas ou escrever como marcadores com uma caneta ou objeto semelhante a uma caneta}
  \seeref{划}{hua2}
\end{entry}

\begin{entry}{划分}{hua4fen1}{6,4}{⼑、⼑}[HSK 5]
  \definition{v.}{dividir; particionar; reparticionar | diferenciar; encontrar aspectos diferentes}
\end{entry}

\begin{entry}{画}{hua4}{8}{⽥}[HSK 2]
  \definition*{s.}{sobrenome Hua}
  \definition{clas.}{traços (de um caractere chinês)}
  \definition[张,幅]{s.}{desenho; pintura; imagem; figura desenhada | traço horizontal (em caracteres chineses)}
  \definition{v.}{desenhar; pintar | desenhar; marcar; assinar}
  \seealsoref{划}{hua4}
\end{entry}

\begin{entry}{画地为牢}{hua4di4wei2lao2}{8,6,4,7}{⽥、⼟、⼂、⼧}
  \definition{expr.}{(literalmente) ser confinado dentro de um círculo desenhado no chão | (figurativo) limitar-se a uma gama restrita de atividades}
\end{entry}

\begin{entry}{画家}{hua4 jia1}{8,10}{⽥、⼧}[HSK 2]
  \definition[个,位,名,些]{s.}{pintor; pessoa com talento para pintura}
\end{entry}

\begin{entry}{画面}{hua4 mian4}{8,9}{⽥、⾯}[HSK 5]
  \definition[个,幅]{s.}{quadro; aparência geral de uma imagem; imagem apresentada no quadro, na tela, etc.}
\end{entry}

\begin{entry}{画儿}{hua4r5}{8,2}{⽥、⼉}[HSK 2]
  \definition[幅,张]{s.}{imagem; desenho; pintura; obra de arte pintada}
\end{entry}

\begin{entry}{话}{hua4}{8}{⾔}[HSK 1]
  \definition[种,席,句,口,番]{s.}{palavra; conversa; a voz que expressa os pensamentos quando falada, ou os caracteres que registram essa voz}
  \definition{v.}{falar sobre; falar a respeito}
\end{entry}

\begin{entry}{话剧}{hua4 ju4}{8,10}{⾔、⼑}[HSK 3]
  \definition[台,部]{s.}{drama moderno; peça de teatro}
\end{entry}

\begin{entry}{话题}{hua4ti2}{8,15}{⾔、⾴}[HSK 3]
  \definition[个,种,项]{s.}{assunto de uma palestra; tópico de uma conversa}
\end{entry}

\begin{entry}{怀旧}{huai2jiu4}{7,5}{⼼、⽇}
  \definition{s.}{nostalgia}
  \definition{v.}{sentir-se nostálgico}
\end{entry}

\begin{entry}{怀念}{huai2nian4}{7,8}{⼼、⼼}[HSK 4]
  \definition{v.}{pensar em; valorizar a memória de}
\end{entry}

\begin{entry}{怀疑}{huai2yi2}{7,14}{⼼、⽦}[HSK 4]
  \definition{v.}{duvidar; suspeitar | supor}
\end{entry}

\begin{entry}{坏}{huai4}{7}{⼟}[HSK 1]
  \definition{adj.}{ruim; prejudicial; insatisfatório; péssimo | mal; extremamente; indica um grau profundo, geralmente usado após verbos ou adjetivos que expressam estado psicológico | podre; estragado; impróprio; prejudicial ao uso}
  \definition[种]{s.}{ideia maligna; truque sujo; péssima ideia}
  \definition{v.}{estragar; destruir; corromper}
\end{entry}

\begin{entry}{坏处}{huai4 chu4}{7,5}{⼟、⼡}[HSK 2]
  \definition[个]{s.}{dano; prejuízo; desvantagem; fatores prejudiciais a pessoas ou coisas}
\end{entry}

\begin{entry}{坏蛋}{huai4dan4}{7,11}{⼟、⾍}
  \definition{s.}{bastardo | canalha | pessoa má}
\end{entry}

\begin{entry}{坏人}{huai4 ren2}{7,2}{⼟、⼈}[HSK 2]
  \definition[个,种]{s.}{malfeitor; canalha; pessoa má; pessoa de má qualidade; pessoa que faz coisas ruins}
\end{entry}

\begin{entry}{欢快}{huan1kuai4}{6,7}{⽋、⼼}
  \definition{adj.}{feliz e sem ansiedade | vívido}
\end{entry}

\begin{entry}{欢乐}{huan1le4}{6,5}{⽋、⼃}[HSK 3]
  \definition{adj.}{feliz; alegre}
\end{entry}

\begin{entry}{欢迎}{huan1ying2}{6,7}{⽋、⾡}[HSK 2]
  \definition{adj.}{bem-vindo}
  \definition{v.}{dar as boas-vindas; cumprimentar; receber com alegria | dar as boas-vindas; receber favoravelmente (bem)}
\end{entry}

\begin{entry}{还}{huan2}{7}{⾡}[HSK 1]
  \definition*{s.}{sobrenome Huan}
  \definition{v.}{voltar; retornar; voltar ao lugar original ou restaurar o estado original | retribuir; devolver; reembolsar; devolver o dinheiro ou os bens emprestados ao seu proprietário | dar ou fazer algo em troca; retribuir as ações dos outros}
  \seeref{还}{hai2}
\end{entry}

\begin{entry}{环}{huan2}{8}{⽟}[HSK 3]
  \definition*{s.}{sobrenome Huan}
  \definition{clas.}{para anéis}
  \definition{s.}{anel; arco | \emph{link}; ligação}
  \definition{v.}{cercar; rodear; circular; circundar}
\end{entry}

\begin{entry}{环保}{huan2 bao3}{8,9}{⽟、⼈}[HSK 3]
  \definition{adj.}{bom para o meio ambiente; não danifica o meio ambiente}
  \definition{s.}{proteção ambiental}
\end{entry}

\begin{entry}{环节}{huan2jie2}{8,5}{⽟、⾋}[HSK 5]
  \definition{s.}{\emph{link}; ligação; vínculo; uma das muitas coisas que estão inter-relacionadas | segmento; estrutura anelar de alguns animais inferiores}
\end{entry}

\begin{entry}{环境}{huan2jing4}{8,14}{⽟、⼟}[HSK 3]
  \definition[个]{s.}{ambiente | arredores; circunstâncias}
\end{entry}

\begin{entry}{环境卫生}{huan2jing4wei4sheng1}{8,14,3,5}{⽟、⼟、⼙、⽣}
  \definition{s.}{saneamento ambiental}
  \seealsoref{环卫}{huan2wei4}
\end{entry}

\begin{entry}{环卫}{huan2wei4}{8,3}{⽟、⼙}
  \definition{s.}{limpeza pública | saneamento urbano | saneamento ambiental | abreviação de 环境卫生}
  \seealsoref{环境卫生}{huan2jing4wei4sheng1}
\end{entry}

\begin{entry}{缓}{huan3}{12}{⽶}
  \definition{adj.}{lento; sem pressa | sem tensão; relaxado}
  \definition{v.}{atrasar; adiar; protelar | recuperar; reviver; voltar a si}
\end{entry}

\begin{entry}{缓解}{huan3jie3}{12,13}{⽶、⾓}[HSK 4]
  \definition{v.}{facilitar; aliviar; atenuar; amenizar; reduzir}
\end{entry}

\begin{entry}{幻觉}{huan4jue2}{4,9}{⼳、⾒}
  \definition{s.}{ilusão | alucinação}
\end{entry}

\begin{entry}{换}{huan4}{10}{⼿}[HSK 2]
  \definition{v.}{negociar; trocar; permutar; dar algo a alguém e, ao mesmo tempo, obter algo dele em troca | mudar; transformar; substituir | trocar dinheiro (câmbio) | transfundir (sangue) | transplantar (um órgão)}
\end{entry}

\begin{entry}{换钱}{huan4qian2}{10,10}{⼿、⾦}
  \definition{v.+compl.}{trocar dinheiro (em pequenas valores ou em outra moeda) | trocar (mercadorias) por dinheiro | vender}
\end{entry}

\begin{entry}{荒芜}{huang1wu2}{9,7}{⾋、⾋}
  \definition{adj.}{estéril}
\end{entry}

\begin{entry}{慌}{huang1}{12}{⼼}[HSK 5]
  \definition{adj.}{agitado; confuso; que inspira terror}
  \definition{v.}{ficar com medo; ficar nervoso}
\end{entry}

\begin{entry}{慌忙}{huang1 mang2}{12,6}{⼼、⼼}[HSK 5]
  \definition{adj.}{apressado; afobado; com muita pressa}
  \definition{adv.}{apressadamente}
\end{entry}

\begin{entry}{皇帝}{huang2di4}{9,9}{⽩、⼱}
  \definition[个]{s.}{imperador}
\end{entry}

\begin{entry}{黄}{huang2}{11}{⿈}[HSK 2][Kangxi 201]
  \definition*{s.}{sobrenome Huang ou Hwang}
  \definition*{s.}{abreviação de Rio Huanghe | refere-se ao Imperador Amarelo, um imperador da mitologia chinesa antiga}
  \definition{adj.}{amarelo | obsceno; indecente; pornográfico; símbolo de corrupção e decadência, referindo-se especificamente à pornografia}
  \definition{s.}{gema; ovas de caranguejo; refere-se a certas coisas de cor amarela}
  \definition{v.}{fracassar; dar errado}
  \seealsoref{黄河}{huang2he2}
\end{entry}

\begin{entry}{黄瓜}{huang2 gua1}{11,5}{⿈、⽠}[HSK 4]
  \definition[根,棵,株]{s.}{pepino}
\end{entry}

\begin{entry}{黄河}{huang2he2}{11,8}{⿈、⽔}
  \definition*{s.}{Rio Amarelo | Rio Huang He}
\end{entry}

\begin{entry}{黄昏}{huang2hun1}{11,8}{⿈、⽇}
  \definition{s.}{anoitecer}
\end{entry}

\begin{entry}{黄金}{huang2jin1}{11,8}{⿈、⾦}[HSK 4]
  \definition{adj.}{de primeira qualidade; dourado;}
  \definition[块,克,两]{s.}{ouro; \emph{aurum}; um tipo de metal, de cor amarela, mais precioso, abreviação de 金, frequentemente falado como 金子}
  \seealsoref{金}{jin1}
  \seealsoref{金子}{jin1zi5}
\end{entry}

\begin{entry}{黄色}{huang2 se4}{11,6}{⿈、⾊}[HSK 2]
  \definition{adj.}{decadente; obsceno; erótico; pornográfico; símbolo de corrupção e decadência, referindo-se especificamente à pornografia}
  \definition[种]{s.}{cor amarela}
\end{entry}

\begin{entry}{黄油}{huang2you2}{11,8}{⿈、⽔}
  \definition[盒]{s.}{manteiga}
\end{entry}

\begin{entry}{谎话}{huang3hua4}{11,8}{⾔、⾔}
  \definition{s.}{mentira}
\end{entry}

\begin{entry}{灰色}{hui1 se4}{6,6}{⽕、⾊}[HSK 5]
  \definition{adj.}{obscuro; ambíguo | sombrio; pessimista}
  \definition[个]{s.}{cor cinza; acinzentado}
\end{entry}

\begin{entry}{恢复}{hui1fu4}{9,9}{⼼、⼢}[HSK 5]
  \definition{v.}{retomar; renovar; restaurar; voltar a | reviver; recuperar; reencontrar | restaurar; restabelecer; reabilitar; regenerar; ressurgir; restabelecer alguém em; recuperar o que foi perdido}
\end{entry}

\begin{entry}{挥汗如雨}{hui1han4ru2yu3}{9,6,6,8}{⼿、⽔、⼥、⾬}
  \definition{s.}{suor derramado}
  \definition{v.}{pingar com suor}
\end{entry}

\begin{entry}{囘}{hui2}{5}{⼞}
  \variantof{回}
\end{entry}

\begin{entry}{回}{hui2}{6}{⼞}[HSK 1,2]
  \definition{clas.}{usado para coisas, ações, número de vezes |  um trecho de um conto; um capítulo de um romance em capítulos}
  \definition{s.}{sobrenome Hui | seção ou capítulo (de um livro clássico) | grupo étnico Hui (mulçumanos chineses)}
  \definition{v.}{circular; enrolar | retornar; voltar; voltar ao lugar de origem | dar meia-volta | responder; contestar | relatar; reportar; responder}
\end{entry}

\begin{entry}{回报}{hui2bao4}{6,7}{⼞、⼿}[HSK 5]
  \definition{s.}{recompensa; pagamento; benefícios recebidos como resultado de assistência, esforço ou afeto | retornos; benefícios recebidos por meio de investimentos}
  \definition{v.}{pagar de volta; beneficiar pessoas ou organizações que os ajudaram ou cuidaram deles de alguma forma}
\end{entry}

\begin{entry}{回避}{hui2bi4}{6,16}{⼞、⾌}
  \definition{v.}{fugir (de um problema); em direito, refere-se especificamente à não participação nos procedimentos de um caso de um oficial de justiça, etc., que tenha interesse no caso ou nas partes do caso | esquivar-se; evadir-se; evitar (encontrar alguém)}
\end{entry}

\begin{entry}{回答}{hui2da2}{6,12}{⼞、⽵}[HSK 1]
  \definition[个]{s.}{resposta}
  \definition{v.}{responder; explicar a questão; expressar opinião sobre a solicitação}
\end{entry}

\begin{entry}{回到}{hui2 dao4}{6,8}{⼞、⼑}[HSK 1]
  \definition{v.}{retornar para; voltar e chegar (ao lugar onde estava originalmente); (após uma mudança nas circunstâncias) retornar ao estado original}
\end{entry}

\begin{entry}{回复}{hui2 fu4}{6,9}{⼞、⼢}[HSK 4]
  \definition{v.}{responder (a uma carta) | retornar ao estado normal; restaurar algo ao seu estado original}
\end{entry}

\begin{entry}{回顾}{hui2gu4}{6,10}{⼞、⾴}[HSK 5]
  \definition{v.}{olhar para trás | revisar; fazer uma retrospectiva; olhar para trás, pensar no passado}
\end{entry}

\begin{entry}{回国}{hui2 guo2}{6,8}{⼞、⼞}[HSK 2]
  \definition{v.}{regressar ao seu país (terra natal); referindo-se a voltar do exterior}
\end{entry}

\begin{entry}{回家}{hui2 jia1}{6,10}{⼞、⼧}[HSK 1]
  \definition{v.}{ir (voltar) para casa; estar em casa; voltar para casa}
\end{entry}

\begin{entry}{回来}{hui2 lai5}{6,7}{⼞、⽊}[HSK 1]
  \definition{v.}{voltar; regressar (para a minha localização) | retornar; usado após um verbo, significa ``vir ao lugar original''}
\end{entry}

\begin{entry}{回去}{hui2 qu4}{6,5}{⼞、⼛}[HSK 1]
  \definition{v.}{retornar; voltar; estar de volta ; (a partir da minha localização)}
\end{entry}

\begin{entry}{回收}{hui2shou1}{6,6}{⼞、⽁}[HSK 5]
  \definition{v.}{reciclar; reciclar itens (geralmente resíduos ou produtos antigos) para reutilização | recuperar; recolher; recuperar o que foi emitido ou demitido}
\end{entry}

\begin{entry}{回头}{hui2 tou2}{6,5}{⼞、⼤}[HSK 5]
  \definition{adv.}{mais tarde; depois de um tempo}
  \definition{conj.}{ou então; usado no início da segunda metade de uma frase para indicar o que acontecerá se você não fizer o que fez na primeira metade da frase}
  \definition{v.}{dar a meia-volta; virar a cabeça; virar a cabeça para trás | retornar; voltar | arrepender-se; corrigir seu caminho; reconhecer e corrigir erros}
\end{entry}

\begin{entry}{回信}{hui2 xin4}{6,9}{⼞、⼈}[HSK 5]
  \definition[封]{s.}{uma carta em resposta; uma mensagem verbal em resposta}
  \definition{v.+compl.}{escrever em resposta; escrever de volta; responder uma carta; responder verbalmente uma mensagem}
\end{entry}

\begin{entry}{回旋}{hui2xuan2}{6,11}{⼞、⽅}
  \definition{v.}{circular | rodar | dar a volta}
\end{entry}

\begin{entry}{回忆}{hui2yi4}{6,4}{⼞、⼼}[HSK 5]
  \definition[个,段]{s.}{memória; lembrança de eventos ou experiências passadas}
  \definition{v.}{lembrar; recordar}
\end{entry}

\begin{entry}{廻}{hui2}{8}{⼵}
  \variantof{回}
\end{entry}

\begin{entry}{汇}{hui4}{5}{⽔}[HSK 4]
  \definition{s.}{coisas coletadas; conjunto; coleção}
  \definition{v.}{convergir | reunir-se | remeter; transferir por meio de agências postais e telegráficas, bancos}
\end{entry}

\begin{entry}{汇报}{hui4bao4}{5,7}{⽔、⼿}[HSK 4]
  \definition[份,次]{s.}{relatório; referindo-se ao conteúdo de declarações escritas ou orais feitas a um superior ou pessoa relevante para apresentar uma situação ou refletir um problema}
  \definition{v.}{relatar; fazer um relato de}
\end{entry}

\begin{entry}{汇款}{hui4 kuan3}{5,12}{⽔、⽋}[HSK 5]
  \definition[笔,个]{s.}{remessa; dinheiro enviado ou recebido}
  \definition{v.+compl.}{remeter dinheiro; fazer uma remessa; enviar dinheiro}
\end{entry}

\begin{entry}{汇率}{hui4lv4}{5,11}{⽔、⽞}[HSK 4]
  \definition[个]{s.}{taxa de câmbio; relação entre a moeda de um país e a de outro}
\end{entry}

\begin{entry}{会}{hui4}{6}{⼈}[HSK 1,2]
  \definition{adv.}{um momento}
  \definition{clas.}{momento; um curto período de tempo}
  \definition{s.}{reunião; festa; conferência; reunião com um objetivo específico | reunião; reunião no trabalho | feira do templo; festival religioso | associação; sociedade; sindicato; certas organizações públicas | oportunidade; ocasião; momento oportuno | cidade principal; capital; cidade central}
  \definition{suf.}{união; grupo; associação}
  \definition{v.}{ser provável que; ter certeza de; indica que é possível realizar (é possível responder à pergunta separadamente) |  poder; ser capaz de; significa saber como fazer ou ter a capacidade de fazer (geralmente se refere a coisas que precisam ser aprendidas) | saber; compreender; entender | encontrar; ver | reunir-se; reunir; agregar; juntar | destacar-se em; ser bom em; ser hábil em; indica proficiência | pagar (ou custear) contas}
  \seeref{会}{kuai4}
\end{entry}

\begin{entry}{会首}{hui4shou3}{6,9}{⼈、⾸}
  \definition{s.}{chefe de uma sociedade | patrocinador de uma organização}
\end{entry}

\begin{entry}{会谈}{hui4 tan2}{6,10}{⼈、⾔}[HSK 5]
  \definition{v.}{manter conversações; comumente usado em assuntos internacionais ou atividades diplomáticas}
\end{entry}

\begin{entry}{会议}{hui4yi4}{6,5}{⼈、⾔}[HSK 3]
  \definition[场,届,个]{s.}{reunião; conferência | conselho; congresso}
\end{entry}

\begin{entry}{会员}{hui4 yuan2}{6,7}{⼈、⼝}[HSK 3]
  \definition[位]{s.}{membro; associado | filiação}
\end{entry}

\begin{entry}{婚礼}{hun1li3}{11,5}{⼥、⽰}[HSK 4]
  \definition[场]{s.}{casamento; núpcias; cerimônia de casamento}
\end{entry}

\begin{entry}{魂}{hun2}{13}{⿁}
  \definition{s.}{alma | espírito | alma imortal (que pode ser separada do corpo)}
\end{entry}

\begin{entry}{混饭}{hun4fan4}{11,7}{⽔、⾷}
  \definition{v.+compl.}{trabalhar para viver}
\end{entry}

\begin{entry}{混乱}{hun4luan4}{11,7}{⽔、⼄}
  \definition{adj.}{confuso | caótico | desordenado}
  \definition{s.}{caos}
\end{entry}

\begin{entry}{和}{huo2}{8}{⼝}
  \definition{v.}{combinar uma substância em pó (farinha, gesso, etc.) com água; adicionar líquido ao pó e mexer ou amassar até ficar pegajoso}
  \seeref{和}{he2}
  \seeref{和}{he4}
  \seeref{和}{hu2}
  \seeref{和}{huo4}
\end{entry}

\begin{entry}{活}{huo2}{9}{⽔}[HSK 3]
  \definition{adj.}{vivo; vivendo | vívido; animado; ativo | móvel; em movimento}
  \definition{adv.}{exatamente; simplesmente}
  \definition{s.}{trabalho | produto}
  \definition{v.}{viver | salvar (a vida de uma pessoa)}
\end{entry}

\begin{entry}{活动}{huo2dong4}{9,6}{⽔、⼒}[HSK 2]
  \definition{adj.}{móvel; flexível para alterações ou mudanças}
  \definition[些,个,种,类,次]{s.}{atividade; ação tomada com o objetivo de alcançar um determinado objetivo}
  \definition{v.}{fazer exercício; movimentar-se | usar influência pessoal; usar meios irregulares | mover-se}
\end{entry}

\begin{entry}{活力}{huo2li4}{9,2}{⽔、⼒}[HSK 5]
  \definition{s.}{vigor; vitalidade; energia; muito forte, geralmente usado para descrever pessoas, cidades, empresas, economias, etc.}
\end{entry}

\begin{entry}{活路}{huo2lu4}{9,13}{⽔、⾜}
  \definition{s.}{maneira de sobreviver | meio de subsistência}
  \seeref{活路}{huo2lu5}
\end{entry}

\begin{entry}{活路}{huo2lu5}{9,13}{⽔、⾜}
  \definition{s.}{labor | trabalho físico}
  \seeref{活路}{huo2lu4}
\end{entry}

\begin{entry}{活泼}{huo2po1}{9,8}{⽔、⽔}[HSK 5]
  \definition{adj.}{vívido; ativo; animado; brilhante; vivaz; cheio de vida | reativo; (química) significa que a substância é ativa e reage facilmente com outras substâncias}
\end{entry}

\begin{entry}{活着}{huo2zhe5}{9,11}{⽔、⽬}
  \definition{adj.}{vivo}
\end{entry}

\begin{entry}{火}{huo3}{4}{⽕}[HSK 3,4][Kangxi 86]
  \definition*{s.}{sobrenome Huo}
  \definition{adj.}{ardente; flamejante; vermelho como fogo | efervescente; próspero}
  \definition{adv.}{urgentemente}
  \definition{clas.}{para unidades militares (antigo)}
  \definition{s.}{fogo | armas de fogo; munições | calor interno (uma das seis causas de doenças) | a ação de lutar}
  \definition{v.}{ficar com raiva; perder a paciência}
\end{entry}

\begin{entry}{火柴}{huo3chai2}{4,10}{⽕、⽊}[HSK 5]
  \definition[根,盒]{s.}{fósforo (palito de fósforo); fósforo de segurança; iniciador de fogo feito de uma tira fina de madeira mergulhada em um composto de fósforo ou enxofre}
\end{entry}

\begin{entry}{火车}{huo3 che1}{4,4}{⽕、⾞}[HSK 1]
  \definition[个,列,节,班,趟]{s.}{trem; comboio}
\end{entry}

\begin{entry}{火车司机}{huo3che1 si1ji1}{4,4,5,6}{⽕、⾞、⼝、⽊}
  \definition{s.}{maquinista de trem}
\end{entry}

\begin{entry}{火海}{huo3hai3}{4,10}{⽕、⽔}
  \definition{s.}{um mar de chamas}
\end{entry}

\begin{entry}{火腿}{huo3 tui3}{4,13}{⽕、⾁}[HSK 5]
  \definition[道,个]{s.}{presunto; as pernas de porco marinadas mais famosas são produzidas em Jinhua, na província de Zhejiang, e em Xuanwei, na província de Yunnan.}
\end{entry}

\begin{entry}{火灾}{huo3 zai1}{4,7}{⽕、⽕}[HSK 5]
  \definition[场]{s.}{fogo (como um desastre); conflagração; desastres causados por incêndios}
\end{entry}

\begin{entry}{伙}{huo3}{6}{⼈}[HSK 4]
  \definition{clas.}{grupo; multidão; banda}
  \definition{s.}{iguaria; alimentação; refeições | parceiro; companheiro | coletivo de colegas}
  \definition{v.}{combinar; unir}
\end{entry}

\begin{entry}{伙伴}{huo3ban4}{6,7}{⼈、⼈}[HSK 4]
  \definition[个,位,群]{s.}{parceiro; companheiro; antigo sistema militar de dez pessoas para uma fogueira, o chefe da fogueira, uma pessoa encarregada de cozinhar, com a fogueira é chamado de parceiro da fogueira, agora se refere à participação comum em uma determinada organização ou engajada em certas atividades}
\end{entry}

\begin{entry}{和}{huo4}{8}{⼝}
  \definition{clas.}{para enxágues de roupas | para fervuras de ervas medicinais}
  \definition{v.}{misturar (ingredientes); misturar pós ou grãos; misturar com água para obter uma consistência mais líquida}
  \seeref{和}{he2}
  \seeref{和}{he4}
  \seeref{和}{hu2}
  \seeref{和}{huo2}
\end{entry}

\begin{entry}{或}{huo4}{8}{⼽}[HSK 2]
  \definition{adv.}{talvez; possivelmente; provavelmente | (geralmente na forma negativa) um pouco; ligeiramente}
  \definition{conj.}{ou (indicando escolha); ou\dots ou\dots}
  \definition{pron.}{alguém; algumas pessoas; refere-se a alguém ou algo, equivalente a 有人 ou 有的}
  \seealsoref{有的}{you3 de5}
  \seealsoref{有人}{you3 ren2}
\end{entry}

\begin{entry}{或是}{huo4 shi4}{8,9}{⼽、⽇}[HSK 5]
  \definition{adv.}{um ou outro; o outro}
  \definition{conj.}{ou; às vezes, é apenas uma de duas coisas}
\end{entry}

\begin{entry}{或许}{huo4xu3}{8,6}{⼽、⾔}[HSK 4]
  \definition{adv.}{talvez; possivelmente; receio; não tenho certeza}
\end{entry}

\begin{entry}{或者}{huo4zhe3}{8,8}{⼽、⽼}[HSK 2]
  \definition{adv.}{talvez; possivelmente}
  \definition{conj.}{ou (usado em expressões afirmativas); ou\dots ou\dots; usado em frases narrativas para indicar uma relação de escolha | ou (usado para indicar equação); indica relação de equivalência, indicando que os objetos anterior e posterior são iguais}
\end{entry}

\begin{entry}{货}{huo4}{8}{⾙}[HSK 4]
  \definition{s.}{dinheiro; moeda | bens; mercadorias; \emph{commodity} | palavras insultuosas dirigidas a alguém; maldição; xingamento}
\end{entry}

\begin{entry}{货车}{huo4che1}{8,4}{⾙、⾞}
  \definition{s.}{caminhão | van | vagão de carga}
\end{entry}

\begin{entry}{获}{huo4}{10}{⾋}[HSK 4]
  \definition*{s.}{sobrenome Huo}
  \definition{v.}{capturar; pegar | obter; ganhar; colher | colher; ceifar}
\end{entry}

\begin{entry}{获得}{huo4de2}{10,11}{⾋、⼻}[HSK 4]
  \definition{v.}{adquirir; ganhar; obter; alcançar}
\end{entry}

\begin{entry}{获奖}{huo4 jiang3}{10,9}{⾋、⼤}[HSK 4]
  \definition{v.}{ganhar prêmio; ser recompensado; ganhar um prêmio; receber um prêmio}
\end{entry}

\begin{entry}{获取}{huo4 qu3}{10,8}{⾋、⼜}[HSK 4]
  \definition{v.}{adquirir; obter; ganhar; colher}
\end{entry}

\begin{entry}{惑星}{huo4xing1}{12,9}{⼼、⽇}
  \definition{s.}{planeta}
  \seealsoref{行星}{xing2xing1}
\end{entry}

%%%%% EOF %%%%%


%%%
%%% I
%%%
%\section*{I}
%\addcontentsline{toc}{section}{I}
%\begin{multicols*}{2}
%\end{multicols*}

%%%
%%% J
%%%

\section*{J}\addcontentsline{toc}{section}{J}

\begin{verbete}{几}{ji1}{2}[Radical 几][Kangxi 16]
  \significado{adv.}{quase}
  \significado{s.}{mesa pequena}
  \veja{几}{ji3}
\end{verbete}

\begin{verbete}{几乎}{ji1hu1}{2,5}
  \significado{adv.}{quase}
\end{verbete}

\begin{verbete}{机场}{ji1chang3}{6,6}
  \significado[家,处]{s.}{aeroporto; aeródromo}
\end{verbete}

\begin{verbete}{机甲}{ji1jia3}{6,5}
  \significado{s.}{\emph{mecha} (robôs operados pelo homem em mangá japonês)}
\end{verbete}

\begin{verbete}{机票}{ji1piao4}{6,11}
  \significado[张]{s.}{bilhete de avião}
  \veja{飞机票}{fei1ji1piao4}
\end{verbete}

\begin{verbete}{机器}{ji1qi4}{6,16}
  \significado[台,部,个]{s.}{máquina}
\end{verbete}

\begin{verbete}{机器人}{ji1qi4ren2}{6,16,2}
  \significado{s.}{robô; androide}
\end{verbete}

\begin{verbete}{机械}{ji1xie4}{6,11}
  \significado{s.}{máquina; maquinaria; mecânica}
\end{verbete}

\begin{verbete}{肌肉}{ji1rou4}{6,6}
  \significado{s.}{músculo, carne}
\end{verbete}

\begin{verbete}{鸡}{ji1}{7}[Radical 鳥]
  \significado[只]{s.}{galo, galinha; gíria:~prostituta}
\end{verbete}

\begin{verbete}{鸡蛋}{ji1dan4}{7,11}
  \significado[个,打]{s.}{ovo de galinha}
\end{verbete}

\begin{verbete}{积木}{ji1mu4}{10,4}
  \significado{s.}{blocos de montar (brinquedo)}
\end{verbete}

\begin{verbete}{基本法}{ji1ben3fa3}{11,5,8}
  \significado{s.}{lei básica (constituição)}
\end{verbete}

\begin{verbete}{基本功}{ji1ben3gong1}{11,5,5}
  \significado{s.}{habilidades; fundamentos básicos}
\end{verbete}

\begin{verbete}{基督教}{ji1du1jiao4}{11,13,11}
  \significado*{s.}{Cristianismo; Cristão}
\end{verbete}

\begin{verbete}{基因}{ji1yin1}{11,6}
  \significado{s.}{gene}
\end{verbete}

\begin{verbete}{激动}{ji1dong4}{16,6}
  \significado{v.}{excitar;  mover-se emocionalmente; agitar (emoções)}
\end{verbete}

\begin{verbete}{鷄}{ji1}{21}
  \variante{鸡}
\end{verbete}

\begin{verbete}{及}{ji2}{3}[Radical 又]
  \significado{conj.}{e; bem como}
\end{verbete}

\begin{verbete}{及格}{ji2ge2}{3,10}
  \significado{v.}{atender a um padrão mínimo; passar em um exame ou teste}
\end{verbete}

\begin{verbete}{吉他}{ji2ta1}{6,5}
  \significado[把]{s.}{guitarra (empréstimo linguístico)}
\end{verbete}

\begin{verbete}{即}{ji2}{7}[Radical 卩]
  \significado{conj.}{e; até; mesmo se/embora}
\end{verbete}

\begin{verbete}{即便}{ji2bian4}{7,9}
  \significado{conj.}{mesmo se/embora}
\end{verbete}

\begin{verbete}{即或}{ji2huo4}{7,8}
  \significado{conj.}{mesmo se/embora}
\end{verbete}

\begin{verbete}{即若}{ji2ruo4}{7,8}
  \significado{conj.}{mesmo se/embora}
\end{verbete}

\begin{verbete}{即使}{ji2shi3}{7,8}
  \significado{conj.}{mesmo se/embora}
\end{verbete}

\begin{verbete}{即是}{ji2shi4}{7,9}
  \significado{conj.}{aquilo é}
\end{verbete}

\begin{verbete}{……极了}{ji2le5}{7,2}
  \significado{expr.}{muito; extremamente, excessivamente}
\end{verbete}

\begin{verbete}{极其}{ji2qi2}{7,8}
  \significado{adv.}{extremamente, muito}
\end{verbete}

\begin{verbete}{急救}{ji2jiu4}{9,11}
  \significado{s.}{primeiros socorros}
  \significado{v.}{dar tratamento de emergência}
\end{verbete}

\begin{verbete}{集体}{ji2ti3}{12,7}
  \significado{s.}{coletivo (decisão); esforço (conjunto); um grupo; uma equipe}
\end{verbete}

\begin{verbete}{集团}{ji2tuan2}{12,6}
  \significado{s.}{grupo; bloco; corporação; conglomerado}
\end{verbete}

\begin{verbete}{嫉妒}{ji2du4}{13,7}
  \significado{v.}{estar com ciúmes de, invejar}
\end{verbete}

\begin{verbete}{几}{ji3}{2}[Radical 几]
  \significado{interr.}{quantos?, (até 10 itens); alguns?}
  \veja{几}{ji1}
\end{verbete}

\begin{verbete}{几何}{ji3he2}{2,7}
  \significado{s.}{geometria}
\end{verbete}

\begin{verbete}{给}{ji3}{9}[Radical 糸]
  \significado{v.}{fornecer; prover}
  \veja{给}{gei3}
\end{verbete}

\begin{verbete}{计划}{ji4hua4}{4,6}
  \significado[个,项]{s.}{plano; projeto; programa}
  \significado{v.}{planejar; mapear}
\end{verbete}

\begin{verbete}{记得}{ji4de5}{5,11}
  \significado{v.}{lembrar; lembrar-se}
\end{verbete}

\begin{verbete}{记性}{ji4xing5}{5,8}
  \significado{s.}{memória (habilidade em reter informações)}
\end{verbete}

\begin{verbete}{记住}{ji4-zhu4}{5,7}
  \significado{v.}{decorar; memorizar; ter em mente}
\end{verbete}

\begin{verbete}{技俩}{ji4liang3}{7,9}
  \significado{s.}{truque; estratagema; ardil; esquema; estratégia; tática}
\end{verbete}

\begin{verbete}{技术}{ji4shu4}{7,5}
  \significado[门,种,项]{s.}{tecnologia; técnica; habilidade}
\end{verbete}

\begin{verbete}{季节}{ji4jie2}{8,5}
  \significado[个]{s.}{estação (clima)}
\end{verbete}

\begin{verbete}{既}{ji4}{9}[Radical 无]
  \significado{conj.}{desde; como; agora isso; os dois e; assim como}
\end{verbete}

\begin{verbete}{既不……又不……}{ji4bu4 you4bu4}{9,4,2,4}
  \significado{conj.}{nem mesmo\dots}
\end{verbete}

\begin{verbete}{既然}{ji4ran2}{9,12}
  \significado{part.}{agora isso; desde; como}
\end{verbete}

\begin{verbete}{既又}{ji4you4}{9,2}
  \significado{conj.}{desde; como; agora isso; os dois e; assim como}
\end{verbete}

\begin{verbete}{寂寥}{ji4liao2}{11,14}
  \significado{s.}{solidão; vasto e vazio; quieto e desolado (literário)}
\end{verbete}

\begin{verbete}{寂寞}{ji4mo4}{11,13}
  \significado{adj.}{sozinho; solitário; (de um lugar) silencioso;}
\end{verbete}

\begin{verbete}{寄}{ji4}{11}[Radical 宀]
  \significado{v.}{enviar; mandar}
\end{verbete}

\begin{verbete}{寄存}{ji4cun2}{11,6}
  \significado{v.}{depositar; deixar algo com alguém; armazenar}
\end{verbete}

\begin{verbete}{寄递}{ji4di4}{11,10}
  \significado{s.}{entrega de correspondência}
\end{verbete}

\begin{verbete}{寄放}{ji4fang4}{11,8}
  \significado{v.}{deixar algo com alguém}
\end{verbete}

\begin{verbete}{寄居}{ji4ju1}{11,8}
  \significado{s.}{morar longe de casa}
\end{verbete}

\begin{verbete}{寄卖}{ji4mai4}{11,8}
  \significado{v.}{consignar para venda}
\end{verbete}

\begin{verbete}{寄生}{ji4sheng1}{11,5}
  \significado{s.}{parasita; parasitismo}
  \significado{v.}{viver tirando vantagem dos outros; viver dentro ou sobre outro organismo como um parasita}
\end{verbete}

\begin{verbete}{寄生生活}{ji4sheng1sheng1huo2}{11,5,5,9}
  \significado{s.}{parasitismo; vida parasitária}
\end{verbete}

\begin{verbete}{寄售}{ji4shou4}{11,11}
  \significado{v.}{venda em consignação}
\end{verbete}

\begin{verbete}{寄送}{ji4song4}{11,9}
  \significado{v.}{enviar; transmitir}
\end{verbete}

\begin{verbete}{寄宿}{ji4su4}{11,11}
  \significado{s.}{embarque}
  \significado{v.}{embarcar}
\end{verbete}

\begin{verbete}{寄托}{ji4tuo1}{11,6}
  \significado{v.}{investir (sua esperança, energia, etc.) em algo; confiar (a alguém); colocar (a esperança, a energia, etc.) em}
\end{verbete}

\begin{verbete}{寄望}{ji4wang4}{11,11}
  \significado{v.}{depositar esperanças em}
\end{verbete}

\begin{verbete}{寄养}{ji4yang3}{11,9}
  \significado{v.}{embarcar; promover; colocar sob os cuidados de alguém (uma criança, animal de estimação, etc.)}
\end{verbete}

\begin{verbete}{寄予}{ji4yu3}{11,4}
  \significado{v.}{expressar; colocar (esperança, importância, etc.) em; mostrar}
\end{verbete}

\begin{verbete}{旣}{ji4}{11}
  \variante{既}
\end{verbete}

\begin{verbete}{加}{jia1}{5}[Radical 力]
  \significado*{s.}{Canadá, abreviação de~加拿大; sobrenome Jia}
  \veja{加拿大}{jia1na2da4}
\end{verbete}

\begin{verbete}{加工}{jia1gong1}{5,3}
  \significado{s.}{processo; trabalho (de uma máquina)}
  \significado{v.}{processar}
\end{verbete}

\begin{verbete}{加拿大}{jia1na2da4}{5,10,3}
  \significado{s.}{Canadá}
  \veja{加}{jia1}
\end{verbete}

\begin{verbete}{加拿大人}{jia1na2da4ren2}{5,10,3,2}
  \significado{s.}{canadense; pessoa nascida no Canadá}
\end{verbete}

\begin{verbete}{加入}{jia1ru4}{5,2}
  \significado{v.}{tornar-se um membro; juntar-se; participar de; adicionar em}
\end{verbete}

\begin{verbete}{加速}{jia1su4}{5,10}
  \significado{v.}{acelerar; agilizar}
\end{verbete}

\begin{verbete}{加速度}{jia1su4du4}{5,10,9}
  \significado{s.}{aceleração}
\end{verbete}

\begin{verbete}{加油}{jia1you2}{5,8}
  \significado{v.+compl.}{lubrificar; encher o tanque de combustível | fazer um esforço maior; fazer um esforço extra}
\end{verbete}

\begin{verbete}{家}{jia1}{10}[Radical 宀]
  \significado{clas.}{para famílias ou empresas}
  \significado[个]{s.}{família; casa; sufixo de substantivos para designar um especialista em alguma atividade}
\end{verbete}

\begin{verbete}{家伙}{jia1huo5}{10,6}
  \significado{s.}{prato, implemento ou móvel doméstico; animal doméstico; (coloquial) o cara; indivíduo; arma}
\end{verbete}

\begin{verbete}{家具}{jia1ju4}{10,8}
  \significado[件,套]{s.}{móveis; mobiliário}
\end{verbete}

\begin{verbete}{家俱}{jia1ju4}{10,10}
  \variante{家具}
\end{verbete}

\begin{verbete}{家里}{jia1li3}{10,7}
  \significado{adv.}{em casa}
\end{verbete}

\begin{verbete}{家乡}{jia1xiang1}{10,3}
  \significado[个]{s.}{terra natal; cidade natal}
\end{verbete}

\begin{verbete}{傢具}{jia1ju4}{12,8}
  \variante{家具}
\end{verbete}

\begin{verbete}{嘉年华}{jia1nian2hua2}{14,6,6}
  \significado{s.}{carnaval (empréstimo linguístico)}
\end{verbete}

\begin{verbete}{甲骨文}{jia3gu3wen2}{5,9,4}
  \significado{s.}{escrituras de oráculos; inscrições em ossos de oráculos (forma original de escritura chinesa)}
\end{verbete}

\begin{verbete}{假}{jia3}{11}[Radical 人]
  \significado{adj.}{falso; artificial}
  \significado{v.}{emprestar}
  \veja{假}{jia4}
\end{verbete}

\begin{verbete}{假的}{jia3de5}{11,8}
  \significado{adj.}{falso; substituto; simulado}
\end{verbete}

\begin{verbete}{假如}{jia3ru2}{11,6}
  \significado{conj.}{se; supondo; em caso}
\end{verbete}

\begin{verbete}{假声}{jia3sheng1}{11,7}
  \significado{s.}{falsete}
  \veja{真声}{zhen1sheng1}
\end{verbete}

\begin{verbete}{假使}{jia3shi3}{11,8}
  \significado{conj.}{se; supondo; em caso}
\end{verbete}

\begin{verbete}{假证件}{jia3zheng4jian4}{11,7,6}
  \significado{s.}{documentos falsos}
\end{verbete}

\begin{verbete}{驾照}{jia4zhao4}{8,13}
  \significado{s.}{carteira de motorista}
\end{verbete}

\begin{verbete}{架式}{jia4shi5}{9,6}
  \variante{架势}
\end{verbete}

\begin{verbete}{架势}{jia4shi5}{9,8}
  \significado{s.}{postura; atitude; posição (sobre um assunto, etc.)}
\end{verbete}

\begin{verbete}{假}{jia4}{11}[Radical 人]
  \significado{s.}{férias}
  \veja{假}{jia3}
\end{verbete}

\begin{verbete}{奸夫}{jian1fu1}{6,4}
  \significado{s.}{homem adúltero}
\end{verbete}

\begin{verbete}{坚持}{jian1chi2}{7,9}
  \significado{s.}{perseverar com; persistir em; insistir em}
\end{verbete}

\begin{verbete}{坚守}{jian1shou3}{7,6}
  \significado{v.}{agarrar-se}
\end{verbete}

\begin{verbete}{间}{jian1}{7}[Radical 門]
  \significado{adv.}{entre; dentro de um tempo ou espaço definidos}
  \significado{clas.}{para salas}
  \significado{s.}{sala; seção de uma sala ou espaço lateral entre dois pares de pilares}
  \veja{间}{jian4}
\end{verbete}

\begin{verbete}{肩膀}{jian1bang3}{8,14}
  \significado{s.}{ombro}
\end{verbete}

\begin{verbete}{兼}{jian1}{10}[Radical 八]
  \significado{conj.}{e (ocupando dois ou mais cargos (oficiais) ao memso tempo)}
\end{verbete}

\begin{verbete}{监狱}{jian1yu4}{10,9}
  \significado{s.}{prisão}
\end{verbete}

\begin{verbete}{煎}{jian1}{13}[Radical 火]
  \significado{v.}{fritar; refogar}
\end{verbete}

\begin{verbete}{煎饼}{jian1bing3}{13,9}
  \significado[张]{s.}{jianbing, crepe chinês; panqueca}
\end{verbete}

\begin{verbete}{煎蛋}{jian1dan4}{13,11}
  \significado{s.}{ovos fritos}
\end{verbete}

\begin{verbete}{俭省}{jian3sheng3}{9,9}
  \significado{adj.}{econômico}
\end{verbete}

\begin{verbete}{柬埔寨}{jian3pu3zhai4}{9,10,14}
  \significado*{s.}{Camboja}
\end{verbete}

\begin{verbete}{捡}{jian3}{10}[Radical 手]
  \significado{v.}{apanhar; recolher; coletar}
\end{verbete}

\begin{verbete}{检查}{jian3cha2}{11,9}
  \significado[次]{s.}{inspeção}
  \significado{v.}{examinar; inspecionar}
\end{verbete}

\begin{verbete}{简单}{jian3dan1}{13,8}
  \significado{adj.}{simples; sem complicações}
\end{verbete}

\begin{verbete}{简直}{jian3zhi2}{13,8}
  \significado{adv.}{simplesmente; realmente; absolutamente; em tudo}
\end{verbete}

\begin{verbete}{见}{jian4}{4}[Radical 見]
  \significado{s.}{opinião, visão}
  \significado{v.}{ver; entrevistar; encontrar alguém; parecer (ser alguma coisa)}
  \veja{见}{xian4}
\end{verbete}

\begin{verbete}{见面}{jian4mian4}{4,9}
  \significado{v.+compl.}{encontrar-se com alguém; ver alguém face-a-face}
\end{verbete}

\begin{verbete}{件}{jian4}{6}[Radical 人]
  \significado{clas.}{para eventos, coisas, roupas etc.}
  \significado{s.}{item; componente}
\end{verbete}

\begin{verbete}{间}{jian4}{7}[Radical 門]
  \significado{s.}{lacuna}
  \significado{v.}{separar; podar (mudas); semear descontentamento}
  \veja{间}{jian1}
\end{verbete}

\begin{verbete}{间或}{jian4huo4}{7,8}
  \significado{adv.}{às vezes; ocasionalmente; de vez em quando}
\end{verbete}

\begin{verbete}{间接}{jian4jie1}{7,11}
  \significado{adj.}{indireto}
  \veja{直接}{zhi2jie1}
\end{verbete}

\begin{verbete}{建立者}{jian4li4zhe3}{8,5,8}
  \significado{s.}{fundador}
\end{verbete}

\begin{verbete}{建设}{jian4she4}{8,6}
  \significado{s.}{construção}
  \significado{v.}{construir}
\end{verbete}

\begin{verbete}{建设性}{jian4she4xing4}{8,6,8}
  \significado{adj.}{construtivo}
  \significado{s.}{construtividade}
\end{verbete}

\begin{verbete}{建设者}{jian4she4zhe3}{8,6,8}
  \significado{s.}{construtor}
\end{verbete}

\begin{verbete}{建议}{jian4yi4}{8,5}
  \significado[个,点]{s.}{proposta, recomendação, sugestão}
  \significado{v.}{propor, recomendar, sugerir}
\end{verbete}

\begin{verbete}{建筑}{jian4zhu4}{8,12}
  \significado[个]{s.}{construção; prédio; edifício}
  \significado{v.}{construir}
\end{verbete}

\begin{verbete}{剑}{jian4}{9}[Radical 刀]
  \significado{clas.}{para golpes de uma espada}
  \significado[口,把]{s.}{espada de dois gumes}
\end{verbete}

\begin{verbete}{剑客}{jian4ke4}{9,9}
  \significado{s.}{espada; esgrimista, espadachim}
\end{verbete}

\begin{verbete}{健身}{jian4shen1}{10,7}
  \significado{s.}{exercício físico; \emph{fitness}}
  \significado{v.}{exercitar-se; manter a forma}
\end{verbete}

\begin{verbete}{渐渐}{jian4jian4}{11,11}
  \significado{adv.}{pouco a pouco; passo a passo; progressivamente}
\end{verbete}

\begin{verbete}{键}{jian4}{13}[Radical 金]
  \significado{s.}{tecla (em um teclado de piano ou computador); botão (em um mouse ou outro dispositivo); ligação química; cavilha de roda, chaveta}
\end{verbete}

\begin{verbete}{江南水乡}{jiang1nan2shui3xiang1}{6,9,4,3}
  \significado*{s.}{Vila Aquática de Jiangnan; Cidades Aquáticas}
\end{verbete}

\begin{verbete}{江水}{jiang1shui3}{6,4}
  \significado{s.}{água do rio}
\end{verbete}

\begin{verbete}{江西}{jiang1xi1}{6,6}
  \significado*{s.}{Jiangxi}
\end{verbete}

\begin{verbete}{姜}{jiang1}{9}[Radical 女]
  \significado*{s.}{sobrenome Jiang}
  \significado{s.}{gengibre}
\end{verbete}

\begin{verbete}{将要}{jiang1yao4}{9,9}
  \significado{adv.}{vai, deve}
\end{verbete}

\begin{verbete}{讲话}{jiang3hua4}{6,8}
  \significado{s.}{discurso | guia; introdução}
  \significado{v.+compl.}{falar; conversar; abordar}
\end{verbete}

\begin{verbete}{讲述}{jiang3shu4}{6,8}
  \significado{v.}{falar sobre, narrar, descrever}
\end{verbete}

\begin{verbete}{匠}{jiang4}{6}[Radical 匚]
  \significado{s.}{artesão}
\end{verbete}

\begin{verbete}{强}{jiang4}{12}[Radical 弓]
  \significado{adj.}{teimoso; inflexível}
  \veja{强}{qiang2}
  \veja{强}{qiang3}
\end{verbete}

\begin{verbete}{酱}{jiang4}{13}[Radical 酉]
  \significado{s.}{pasta grossa de soja fermentada; marinada em pasta de soja; pasta; geléia}
\end{verbete}

\begin{verbete}{犟}{jiang4}{16}[Radical 牛]
  \variante{强}
\end{verbete}

\begin{verbete}{交}{jiao1}{6}[Radical 亠]
  \significado{v.}{entregar; dar}
\end{verbete}

\begin{verbete}{交班}{jiao1ban1}{6,10}
  \significado{v.}{passar para o próximo turno de trabalho}
\end{verbete}

\begin{verbete}{交杯酒}{jiao1bei1jiu3}{6,8,10}
  \significado{s.}{copo de vinho nupcial}
\end{verbete}

\begin{verbete}{交叉}{jiao1cha1}{6,3}
  \significado{v.}{cruzar; sobrepor}
\end{verbete}

\begin{verbete}{交叉点}{jiao1cha1dian3}{6,3,9}
  \significado{s.}{encruzilhada; cruzamento; junção}
\end{verbete}

\begin{verbete}{交叉口}{jiao1cha1kou3}{6,3,3}
  \significado{s.}{intersecção (rodovia)}
\end{verbete}

\begin{verbete}{交叠}{jiao1die2}{6,13}
  \significado{s.}{sobreposição}
\end{verbete}

\begin{verbete}{交给}{jiao1gei3}{6,9}
  \significado{v.}{entregar algo; dar algo}
\end{verbete}

\begin{verbete}{交媾}{jiao1gou4}{6,13}
  \significado{v.}{copular; ter relações sexuais}
\end{verbete}

\begin{verbete}{交界}{jiao1jie4}{6,9}
  \significado{s.}{fronteira comum; limite comum; interface}
\end{verbete}

\begin{verbete}{交警}{jiao1jing3}{6,19}
  \significado{s.}{policial de trânsito (abreviatura de 交通警察)}
  \veja{交通警察}{jiao1tong1jing3cha2}
\end{verbete}

\begin{verbete}{交通}{jiao1tong1}{6,10}
  \significado{s.}{transporte; tráfego; trânsito; comunicação; conexão}
  \significado{v.}{estar conectado; ser conectado}
\end{verbete}

\begin{verbete}{交通警察}{jiao1tong1jing3cha2}{6,10,19,14}
  \significado{s.}{policial de trânsito}
  \veja{交警}{jiao1jing3}
\end{verbete}

\begin{verbete}{交响}{jiao1xiang3}{6,9}
  \significado{s.}{sinfonia}
\end{verbete}

\begin{verbete}{交运}{jiao1yun4}{6,7}
  \significado{v.}{despachar (bagagem em um aeroporto, etc.); entregar para transporte}
\end{verbete}

\begin{verbete}{郊区}{jiao1qu1}{8,4}
  \significado[个]{s.}{subúrbio; distrito suburbano; arredores}
\end{verbete}

\begin{verbete}{胶卷}{jiao1juan3}{10,8}
  \significado{s.}{filme; rolo de filme}
\end{verbete}

\begin{verbete}{教}{jiao1}{11}[Radical 攴]
  \significado{v.}{ensinar; lecionar}
  \veja{教}{jiao4}
\end{verbete}

\begin{verbete}{教会}{jiao1hui4}{11,6}
  \significado{v.}{mostrar; ensinar}
  \veja{教会}{jiao4hui4}
\end{verbete}

\begin{verbete}{教学}{jiao1xue2}{11,8}
  \significado{v.}{ensinar (como um professor)}
  \veja{教学}{jiao4xue2}
\end{verbete}

\begin{verbete}{焦虑}{jiao1lv4}{12,10}
  \significado{adj.}{ansioso; preocupado; apreensivo}
\end{verbete}

\begin{verbete}{角}{jiao3}{7}[Radical 角][Kangxi 148]
  \significado{clas.}{1 jiao = 10 centavos}
  \significado[个]{s.}{ângulo; esquina; chifre; em forma de chifre}
  \veja{角}{jue2}
\end{verbete}

\begin{verbete}{角度}{jiao3du4}{7,9}
  \significado{s.}{ângulo; ponto de vista}
\end{verbete}

\begin{verbete}{饺子}{jiao3zi5}{9,3}
  \significado[个,只]{s.}{jiaozi; bolinhos chineses; bolinho de massa}
\end{verbete}

\begin{verbete}{脚}{jiao3}{11}[Radical 肉]
  \significado{clas.}{para chutes}
  \significado[双,只]{s.}{pé; base (de um objeto); perna (de um animal ou objeto)}
\end{verbete}

\begin{verbete}{叫}{jiao4}{5}[Radical 口]
  \significado{v.}{chamar-se; chamar; gritar; pedir (comida em um restaurante)}
\end{verbete}

\begin{verbete}{校}{jiao4}{10}[Radical 木]
  \significado{v.}{verificar; comparar; revisar}
  \veja{校}{xiao4}
\end{verbete}

\begin{verbete}{敎}{jiao4}{11}
  \variante{教}
\end{verbete}

\begin{verbete}{教}{jiao4}{11}[Radical 攴]
  \significado*{s.}{sobrenome Jiao}
  \significado{s.}{religião; ensinamento}
  \significado{v.}{causar; como fazer algo; contar (explicar como fazer algo)}
  \veja{教}{jiao1}
\end{verbete}

\begin{verbete}{教导}{jiao4dao3}{11,6}
  \significado{s.}{orientação; ensino}
  \significado{v.}{instruir; ensinar}
\end{verbete}

\begin{verbete}{教官}{jiao4guan1}{11,8}
  \significado{s.}{instrutor militar}
\end{verbete}

\begin{verbete}{教会}{jiao4hui4}{11,6}
  \significado{s.}{igreja cristã}
  \veja{教会}{jiao1hui4}
\end{verbete}

\begin{verbete}{教练}{jiao4lian4}{11,8}
  \significado[个,位,名]{s.}{instrutor; treinador (esportes)}
\end{verbete}

\begin{verbete}{教师}{jiao4shi1}{11,6}
  \significado[个]{s.}{professor; mestre}
\end{verbete}

\begin{verbete}{教室}{jiao4shi4}{11,9}
  \significado[间]{s.}{sala de aula}
\end{verbete}

\begin{verbete}{教授}{jiao4shou4}{11,11}
  \significado[个,位]{s.}{professor (universitário)}
  \significado{v.}{instruir; palestrar sobre}
\end{verbete}

\begin{verbete}{教堂}{jiao4tang2}{11,11}
  \significado[间]{s.}{igreja; capela}
\end{verbete}

\begin{verbete}{教学}{jiao4xue2}{11,8}
  \significado[次]{s.}{ensino; instrução}
  \veja{教学}{jiao1xue2}
\end{verbete}

\begin{verbete}{教学楼}{jiao4xue2lou2}{11,8,13}
  \significado{s.}{edifício de salas de aula}
\end{verbete}

\begin{verbete}{教长}{jiao4zhang3}{11,4}
  \significado{s.}{imã (Islã); mulá}
\end{verbete}

\begin{verbete}{皆}{jie1}{9}[Radical 白]
  \significado{adv.}{todos; em todos os casos}
\end{verbete}

\begin{verbete}{结}{jie1}{9}[Radical 糸]
  \significado{v.}{(uma planta) produzir (frutos ou sementes)}
  \veja{结}{jie2}
\end{verbete}

\begin{verbete}{结果}{jie1guo3}{9,8}
  \significado{v.}{dar frutos}
  \veja{结果}{jie2guo3}
\end{verbete}

\begin{verbete}{接}{jie1}{11}[Radical 手]
  \significado{v.}{ir buscar (alguém); ir ao encontro de (alguém); receber}
\end{verbete}

\begin{verbete}{接班人}{jie1ban1ren2}{11,10,2}
  \significado{s.}{sucessor}
\end{verbete}

\begin{verbete}{接待}{jie1dai4}{11,9}
  \significado{v.}{receber (alguém); acolher; recepcionar}
\end{verbete}

\begin{verbete}{接(电话)}{jie1(dian4hua4)}{11,5,8}
  \significado{v.}{atender (o telefone)}
\end{verbete}

\begin{verbete}{街}{jie1}{12}[Radical 行]
  \significado[条]{s.}{rua}
\end{verbete}

\begin{verbete}{街舞}{jie1wu3}{12,14}
  \significado{s.}{dança de rua, \emph{street dance} (por exemplo, \emph{breakdance})}
\end{verbete}

\begin{verbete}{节日}{jie2ri4}{5,4}
  \significado[个]{s.}{festival; feriado}
\end{verbete}

\begin{verbete}{节奏}{jie2zou4}{5,9}
  \significado{s.}{ritmo; cadência; batida}
\end{verbete}

\begin{verbete}{结}{jie2}{9}[Radical 糸]
  \significado{s.}{nó}
  \veja{结}{jie1}
\end{verbete}

\begin{verbete}{结果}{jie2guo3}{9,8}
  \significado{s.}{resultado; conclusão}
  \significado{v.}{despachar; matar}
  \veja{结果}{jie1guo3}
\end{verbete}

\begin{verbete}{结婚}{jie2hun1}{9,11}
  \significado{v.+compl.}{casar; casar-se}
\end{verbete}

\begin{verbete}{结婚礼服}{jie2hun1 li3 fu2}{9,11,5,8}
  \significado{s.}{vestido de casamento}
\end{verbete}

\begin{verbete}{结局}{jie2ju2}{9,7}
  \significado{s.}{conclusão; fim; final}
\end{verbete}

\begin{verbete}{结社自由}{jie2she4zi4you2}{9,7,6,5}
  \significado{s.}{(constitucional) liberdade de associação}
\end{verbete}

\begin{verbete}{结束}{jie2shu4}{9,7}
  \significado{v.}{terminar; acabar; concluir}
\end{verbete}

\begin{verbete}{结束辩论}{jie2shu4 bian4 lun4}{9,7,16,6}
  \significado{s.}{debate de encerramento}
\end{verbete}

\begin{verbete}{结束工作}{jie2shu4gong1zuo4}{9,7,3,7}
  \significado{s.}{trabalho final}
  \significado{v.}{terminar o trabalho}
\end{verbete}

\begin{verbete}{结束剂}{jie2shu4 ji4}{9,7,8}
  \significado{s.}{finalizador}
\end{verbete}

\begin{verbete}{结束区}{jie2shu4 qu1}{9,7,4}
  \significado{s.}{zona final}
\end{verbete}

\begin{verbete}{结束文本}{jie2shu4 wen2ben3}{9,7,4,5}
  \significado{s.}{texto final}
\end{verbete}

\begin{verbete}{结束语}{jie2shu4yu3}{9,7,9}
  \significado{s.}{conclusões finais; considerações finais}
\end{verbete}

\begin{verbete}{捷径}{jie2jing4}{11,8}
  \significado{s.}{atalho}
\end{verbete}

\begin{verbete}{姐夫}{jie3fu5}{8,4}
  \significado{s.}{marido da irmã mais velha}
\end{verbete}

\begin{verbete}{姐姐}{jie3jie5}{8,8}
  \significado[个]{s.}{irmã mais velha}
\end{verbete}

\begin{verbete}{解雇}{jie3gu4}{13,12}
  \significado{v.}{demitir}
\end{verbete}

\begin{verbete}{解救}{jie3jiu4}{13,11}
  \significado{v.}{resgatar; ajudar a sair de dificuldades; salvar a situação}
\end{verbete}

\begin{verbete}{解释}{jie3shi4}{13,12}
  \significado[个]{s.}{explicação}
  \significado{v.}{explicar; interpretar; resolver}
\end{verbete}

\begin{verbete}{解压}{jie3ya1}{13,6}
  \significado{v.}{aliviar o estresse; (computação) descomprimir}
\end{verbete}

\begin{verbete}{介绍}{jie4shao4}{4,8}
  \significado{s.}{introdução; apresentação}
  \significado{v.}{fazer uma apresentação; apresentar (alguém para alguém); apresentar (alguém para um emprego, etc.)}
\end{verbete}

\begin{verbete}{芥}{jie4}{7}[Radical 艸]
  \significado{s.}{mostarda}
  \veja{芥}{gai4}
\end{verbete}

\begin{verbete}{芥兰}{jie4lan2}{7,5}
  \significado{s.}{couve}
\end{verbete}

\begin{verbete}{界碑}{jie4bei1}{9,13}
  \significado{s.}{marco de fronteira}
\end{verbete}

\begin{verbete}{借}{jie4}{10}[Radical 人]
  \significado{adv.}{por meio de}
  \significado{v.}{pedir emprestado; emprestar; aproveitar (uma oportunidade)}
\end{verbete}

\begin{verbete}{借书证}{jie4shu1zheng4}{10,4,7}
  \significado{s.}{cartão de biblioteca; literalmente:~cartão para pedir emprestado livros}
\end{verbete}

\begin{verbete}{今年}{jin1nian2}{4,6}
  \significado{adv.}{este ano}
\end{verbete}

\begin{verbete}{今天}{jin1tian1}{4,4}
  \significado{adv.}{hoje}
\end{verbete}

\begin{verbete}{金刚石}{jin1gang1shi2}{8,6,5}
  \significado{s.}{diamante; também chamado de 钻石}
  \veja{钻石}{zuan4shi2}
\end{verbete}

\begin{verbete}{金融}{jin1rong2}{8,16}
  \significado{s.}{finança}
\end{verbete}

\begin{verbete}{金色}{jin1se4}{8,6}
  \significado{s.}{dourado}
\end{verbete}

\begin{verbete}{仅}{jin3}{4}[Radical 人]
  \significado{adv.}{apenas; meramente}
\end{verbete}

\begin{verbete}{仅仅}{jin3jin3}{4,4}
  \significado{adv.}{meramente; somente; apenas}
\end{verbete}

\begin{verbete}{尽管}{jin3guan3}{6,14}
  \significado{conj.}{no entanto; embora; apesar de}
\end{verbete}

\begin{verbete}{紧急}{jin3ji2}{10,9}
  \significado{adj.}{urgente}
  \significado{s.}{emergência}
\end{verbete}

\begin{verbete}{锦上添花}{jin3shang4tian1hua1}{13,3,11,7}
  \significado{expr.}{A cereja do bolo; (literalmente) adicione flores ao brocato}
  \significado{v.}{dar a alguém esplendor adicional; fornecer o toque final}
\end{verbete}

\begin{verbete}{近}{jin4}{7}[Radical 辵]
  \significado{adj.}{perto; próximo}
\end{verbete}

\begin{verbete}{进}{jin4}{7}[Radical 辵]
  \significado{clas.}{para seções em um edifício ou complexo residencial}
  \significado{s.}{matemática:~base de um sistema numérico}
  \significado{v.}{entrar}
\end{verbete}

\begin{verbete}{进步}{jin4bu4}{7,7}
  \significado[个]{s.}{progresso; melhora}
  \significado{v.}{progredir; melhorar}
\end{verbete}

\begin{verbete}{进出口}{jin4chu1kou3}{7,5,3}
  \significado{s.}{importação e exportação}
  \significado{v.}{importar e exportar}
\end{verbete}

\begin{verbete}{进口}{jin4kou3}{7,3}
  \significado{adj.}{importado}
  \significado{s.}{importação; entrada; entrada (para entrada de ar, água, etc.)}
  \significado{v.}{importar}
\end{verbete}

\begin{verbete}{进来}{jin4lai2}{7,7}
  \significado{v.}{entrar (para a minha localização)}
\end{verbete}

\begin{verbete}{进去}{jin4qu4}{7,5}
  \significado{v.}{entrar (a partir da minha localização)}
\end{verbete}

\begin{verbete}{进行编程}{jin4xing2bian1cheng2}{7,6,12,12}
  \significado{s.}{programa de computador executável}
\end{verbete}

\begin{verbete}{京}{jing1}{8}[Radical 亠]
  \significado*{s.}{Beijing, abreviação de~北京; sobrenome Jing}
  \significado{s.}{capital de um país}
  \veja{北京}{bei3jing1}
\end{verbete}

\begin{verbete}{京剧}{jing1ju4}{8,10}
  \significado*{s.}{Ópera de Beijing (Pequim)}
\end{verbete}

\begin{verbete}{经}{jing1}{8}[Radical 糸]
  \significado*{s.}{sobrenome Jing}
  \significado{s.}{livro sagrado; escritura; clássicos; longitude; menstruação; canal}
  \significado{v.}{passar; sofrer; suportar; deformar (têxtil)}
\end{verbete}

\begin{verbete}{经常}{jing1chang2}{8,11}
  \significado{adv.}{constantemente; diariamente; dia-a-dia; todo dia; frequentemente; sempre; regularmente}
\end{verbete}

\begin{verbete}{经过}{jing1guo4}{8,6}
  \significado[个]{s.}{processo; curso}
  \significado{v.}{passar; passar por}
\end{verbete}

\begin{verbete}{经济}{jing1ji4}{8,9}
  \significado{s.}{economia}
\end{verbete}

\begin{verbete}{经理}{jing1li3}{8,11}
  \significado[个,位,名]{s.}{diretor; gerente}
\end{verbete}

\begin{verbete}{惊呆}{jing1dai1}{11,7}
  \significado{adj.}{estupefato; chocado}
\end{verbete}

\begin{verbete}{惊喜}{jing1xi3}{11,12}
  \significado{s.}{boa surpresa}
  \significado{v.}{ser agradavelmente surpreendido}
\end{verbete}

\begin{verbete}{精彩}{jing1cai3}{14,11}
  \significado{adj.}{espetacular; maravilhoso; brilhante}
\end{verbete}

\begin{verbete}{精灵}{jing1ling2}{14,7}
  \significado{s.}{espírito; fada; elfo; duende; gênio}
\end{verbete}

\begin{verbete}{精品}{jing1pin3}{14,9}
  \significado{s.}{produtos de qualidade; produto premium; bom trabalho (de arte)}
\end{verbete}

\begin{verbete}{精致}{jing1zhi4}{14,10}
  \significado{adj.}{delicado; exótico; refinado}
\end{verbete}

\begin{verbete}{鲸鲨}{jing1sha1}{16,15}
  \significado{s.}{tubarão baleia}
\end{verbete}

\begin{verbete}{鲸鱼}{jing1yu2}{16,8}
  \significado{s.}{baleia}
\end{verbete}

\begin{verbete}{井}{jing3}{4}[Radical 二][Kangxi 7]
  \significado{adj.}{puro; ordenado}
  \significado[口]{s.}{poço}
\end{verbete}

\begin{verbete}{景色}{jing3se4}{12,6}
  \significado{s.}{paisagem; panorama; vista}
\end{verbete}

\begin{verbete}{警}{jing3}{19}[Radical 言]
  \significado{s.}{policial}
  \significado{v.}{alertar; avisar}
\end{verbete}

\begin{verbete}{警察}{jing3cha2}{19,14}
  \significado[个]{s.}{polícia; oficial de polícia}
\end{verbete}

\begin{verbete}{警官}{jing3guan1}{19,8}
  \significado{s.}{polícia, policial}
\end{verbete}

\begin{verbete}{竞赛}{jing4sai4}{10,14}
  \significado{s.}{concurso, competição, partida, corrida}
  \significado{v.}{competir, correr}
\end{verbete}

\begin{verbete}{敬礼}{jing4li3}{12,5}
  \significado{s.}{saudação}
  \significado{v.}{saudar}
\end{verbete}

\begin{verbete}{纠葛}{jiu1ge2}{5,12}
  \significado{s.}{emaranhado; disputa}
\end{verbete}

\begin{verbete}{究竟}{jiu1jing4}{7,11}
  \significado{adv.}{afinal; no final; no final das contas; na verdade; exatamente; são ou não são}
\end{verbete}

\begin{verbete}{九}{jiu3}{2}[Radical 乙]
  \significado{num.}{nove, 9}
\end{verbete}

\begin{verbete}{韭菜}{jiu3cai4}{9,11}
  \significado{s.}{cebolinha chinesa; figurativo:~investidores de varejo que perdem seu dinheiro para operadores mais experientes (ou seja, são ``colhidos'' como cebolinhas)}
\end{verbete}

\begin{verbete}{酒}{jiu3}{10}[Radical 酉]
  \significado[杯,瓶,罐,桶,缸]{s.}{bebida alcoólica; vinho (especialmente vinho de arroz); aguardente; licor; espíritos}
\end{verbete}

\begin{verbete}{酒馆}{jiu3guan3}{10,11}
  \significado{s.}{bar; taverna; adega}
\end{verbete}

\begin{verbete}{酒鬼}{jiu3gui3}{10,9}
  \significado{adj.}{embriagado; ébrio}
  \significado{s.}{bêbado; alcoólatra; borracho}
\end{verbete}

\begin{verbete}{旧}{jiu4}{5}[Radical 日]
  \significado{adj.}{velho; antigo; desgastado (com a idade)}
\end{verbete}

\begin{verbete}{救出}{jiu4chu1}{11,5}
  \significado{v.}{resgatar; tirar do perigo}
\end{verbete}

\begin{verbete}{救护车}{jiu4hu4che1}{11,7,4}
  \significado[辆]{s.}{ambulância}
\end{verbete}

\begin{verbete}{救命}{jiu4ming4}{11,8}
  \significado{interj.}{Socorro!; Salve-me!}
  \significado{v.+compl.}{salvar a vida de alguém}
\end{verbete}

\begin{verbete}{就}{jiu4}{12}[Radical 尢]
  \significado{adv.}{exatamente; justamente}
  \significado{v.}{realizar; se envolver em; acompanhar (em alimentos); aproveitar; avançar; empreender}
\end{verbete}

\begin{verbete}{就是}{jiu4shi4}{12,9}
  \significado{adv.}{exatamente, precisamente; apenas, simplesmente; (usado correlativamente com 也) mesmo, mesmo se}
\end{verbete}

\begin{verbete}{就业}{jiu4ye4}{12,5}
  \significado{v.+compl.}{obter emprego; assumir uma ocupação; conseguir um emprego}
\end{verbete}

\begin{verbete}{就职}{jiu4zhi2}{12,11}
  \significado{v.}{assumir o cargo, assumir um posto}
\end{verbete}

\begin{verbete}{车}{ju1}{4}[Radical 車]
  \significado{s.}{arcaico:~carruagem de guerra; torre (no xadrez)}
  \veja{车}{che1}
\end{verbete}

\begin{verbete}{居然}{ju1ran2}{8,12}
  \significado{adv.}{inesperadamente; na verdade; para surpresa de alguém}
\end{verbete}

\begin{verbete}{橘子汁}{ju2zi5zhi1}{16,3,5}
  \significado[瓶,杯,罐,盒]{s.}{suco de laranja}
  \veja{橙汁}{cheng2zhi1}
  \veja{柳橙汁}{liu3cheng2zhi1}
\end{verbete}

\begin{verbete}{举行}{ju3xing2}{9,6}
  \significado{v.}{realizar (uma reunião, cerimônia, etc.); ter lugar}
\end{verbete}

\begin{verbete}{句}{ju4}{5}[Radical 口]
  \significado{clas.}{para orações, frases ou linhas de versos}
  \significado{s.}{sentença; cláusula; frase}
  \veja{句}{gou4}
\end{verbete}

\begin{verbete}{句子}{ju4zi5}{5,3}
  \significado[个]{s.}{sentença; frase; oração}
\end{verbete}

\begin{verbete}{足}{ju4}{7}[Radical 足]
  \significado{adj.}{excessivo}
  \veja{足}{zu2}
\end{verbete}

\begin{verbete}{距离}{ju4li2}{11,10}
  \significado[个]{s.}{distância}
  \significado{v.}{estar distante de}
\end{verbete}

\begin{verbete}{聚散}{ju4san4}{14,12}
  \significado{s.}{juntos e separados; agregação e dissipação}
\end{verbete}

\begin{verbete}{卷}{juan3}{8}[Radical 卩]
  \significado{clas.}{para pequenas coisas enroladas (maço de papel dinheiro, carretel de filme, etc.); para rolos, carretéis, bobinas, etc.}
  \significado{s.}{rolo; carretel; bobina}
  \significado{v.}{rolar; varrer; carregar}
  \veja{卷}{juan4}
\end{verbete}

\begin{verbete}{卷}{juan4}{8}[Radical 卩]
  \significado{clas.}{para livros, pinturas: volumes, rolos}
  \significado{s.}{rolo com texto; livro; volume; capítulo; artigo}
  \veja{卷}{juan3}
\end{verbete}

\begin{verbete}{角}{jue2}{7}[Radical 角]
  \significado*{s.}{sobrenome Jue}
  \significado{s.}{papel (teatro)}
  \significado{v.}{competir}
  \veja{角}{jiao3}
\end{verbete}

\begin{verbete}{绝版}{jue2ban3}{9,8}
  \significado{adj.}{esgotado; fora de catálogo}
\end{verbete}

\begin{verbete}{绝不}{jue2bu4}{9,4}
  \significado{adv.}{definitivamente não; de forma alguma; sob nenhuma circunstância}
\end{verbete}

\begin{verbete}{绝对}{jue2dui4}{9,5}
  \significado{adv.}{absolutamente; totalmente; incondicionalmente; definitivamente}
\end{verbete}

\begin{verbete}{绝招}{jue2zhao1}{9,8}
  \significado{s.}{habilidade única; movimento delicado inesperado (como último recurso); golpe de mestre; golpe final}
\end{verbete}

\begin{verbete}{觉得}{jue2de5}{9,11}
  \significado{v.}{pensar que\dots, sentir que\dots; sentir (desconfortável, etc.)}
\end{verbete}

\begin{verbete}{脚}{jue2}{11}[Radical 肉]
  \variante{角}
\end{verbete}

\begin{verbete}{军人}{jun1ren2}{6,2}
  \significado{s.}{soldado; pessoal militar}
\end{verbete}

\begin{verbete}{军装}{jun1zhuang1}{6,12}
  \significado{s.}{uniforme militar}
\end{verbete}

\begin{verbete}{君主立宪制}{jun1zhu3li4xian4zhi4}{7,5,5,9,8}
  \significado{s.}{monarquia constitucional}
\end{verbete}

%%%%% EOF %%%%%


%%%
%%% K
%%%
\section*{K}
\addcontentsline{toc}{section}{K}

\begin{verbete}[ka1fei1]{咖啡}[8;11]
\begin{pronuncia}{ka1fei1}
\significado[杯]{s.}{ café }
\end{pronuncia}
\end{verbete}

\begin{verbete}[ka1fei1guan3]{咖啡馆}[8;11;11]
\begin{pronuncia}{ka1fei1guan3}
\significado[家]{s.}{ cafeteria }
\end{pronuncia}
\end{verbete}

\begin{verbete}[kai1]{开}[4]
\begin{pronuncia}{kai1}
\significado{v.}{ abrir; ligar; dirigir }
\end{pronuncia}
\end{verbete}

\begin{verbete}[kai1che1]{开车}[4;4]
\begin{pronuncia}{kai1che1}
\significado{v.}{ conduzir; dirigir }
\end{pronuncia}
\end{verbete}

\begin{verbete}[kai1fa1qu1]{开发区}[4;5;4]
\begin{pronuncia}{kai1fa1qu1}
\significado{s.}{ zona de desenvolvimento }
\end{pronuncia}
\end{verbete}

\begin{verbete}[kai1shi3]{开始}[4;8]
\begin{pronuncia}{kai1shi3}
\significado{v.}{ começar; iniciar }
\end{pronuncia}
\end{verbete}

\begin{verbete}[kan4]{看}[9]
\begin{pronuncia}{kan4}
\significado{v.}{ olhar; ver; assistir }
\end{pronuncia}
\end{verbete}

\begin{verbete}[kan4jian4]{看见}[9;4]
\begin{pronuncia}{kan4jian4}
\significado{v.}{ encontrar; enxergar; ver }
\end{pronuncia}
\end{verbete}

\begin{verbete}[kao3shi4]{考试}[6;8]
\begin{pronuncia}{kao3shi4}
\significado[次]{s.}{ teste; prova; exame }
\significado{v.+compl.}{ submeter-se a uma prova; fazer um teste }
\end{pronuncia}
\end{verbete}

\begin{verbete}[kao3]{烤}[10]
\begin{pronuncia}{kao3}
\significado{v.}{ assar }
\end{pronuncia}
\end{verbete}

\begin{verbete}[Ke1ji4]{科技}[9;7]
\begin{pronuncia}{Ke1ji4}
\significado{s.}{ Ciência e Tecnologia }
\end{pronuncia}
\end{verbete}

\begin{verbete}[ke3]{颗}[14]
\begin{pronuncia}{ke3}
\significado{p.c.}{ para grãos, pérolas, dentes, corações, satelites, pequenas esferas, etc }
\end{pronuncia}
\end{verbete}

\begin{verbete}[ke3sou0]{咳嗽}[9;14]
\begin{pronuncia}{ke3sou0}
\significado{v.}{ ter tosse; tussir }
\end{pronuncia}
\end{verbete}

\begin{verbete}[ke3'ai4]{可爱}[5;10]
\begin{pronuncia}{ke3'ai4}
\significado{adj.}{ querido; fofo }
\end{pronuncia}
\end{verbete}

\begin{verbete}[ke3kou3ke3le3]{可口可乐}[5;3;5;5]
\begin{pronuncia}[\\]{ke3kou3ke3le3}
\significado{s.}{ Coca-Cola }
\end{pronuncia}
\end{verbete}

\begin{verbete}[ke3neng2]{可能}[5;10]
\begin{pronuncia}{ke3neng2}
\significado{adj.}{ possível }
\significado{adv.}{ possivelmente; provavelmente }
\end{pronuncia}
\end{verbete}

\begin{verbete}[ke3shi4]{可是}[5;9]
\begin{pronuncia}{ke3shi4}
\significado{conj.}{ porém; contudo; mas }
\end{pronuncia}
\end{verbete}

\begin{verbete}[ke3xi1]{可惜}[5;11]
\begin{pronuncia}{ke3xi1}
\significado{adj.}{ é pena }
\end{pronuncia}
\end{verbete}

\begin{verbete}[ke3yi3]{可以}[5;4]
\begin{pronuncia}{ke3yi3}
\significado{v.o.}{ poder }
\end{pronuncia}
\end{verbete}

\begin{verbete}[ke4]{刻}[8]
\begin{pronuncia}{ke4}
\significado{s.}{ quarto (de hora) }
\significado{p.c.}{ para curtos intervalos de tempo }
\end{pronuncia}
\end{verbete}

\begin{verbete}[ke4zhong1]{刻钟}[8;9]
\begin{pronuncia}{ke4zhong1}
\significado{p.c.}{ um quarto de hora }
\end{pronuncia}
\end{verbete}

\begin{verbete}[ke4qi0]{客气}[9;4]
\begin{pronuncia}{ke4qi0}
\significado{adj.}{ cortês; delicado; educado }
\significado{v.}{ fazer cerimônia }
\end{pronuncia}
\end{verbete}

\begin{verbete}[ke4ting1]{客厅}[9;4]
\begin{pronuncia}{ke4ting1}
\significado[间]{s.}{ sala de estar; sala de visitas }
\end{pronuncia}
\end{verbete}

\begin{verbete}[ke4ben3]{课本}[10;5]
\begin{pronuncia}{ke4ben3}
\significado[本]{s.}{ livro do aluno; manual }
\end{pronuncia}
\end{verbete}

\begin{verbete}[ken3ding4]{肯定}[8;8]
\begin{pronuncia}{ken3ding4}
\significado{adv.}{ com certeza; certamente }
\end{pronuncia}
\end{verbete}

\begin{verbete}[kong1qi4]{空气}[8;4]
\begin{pronuncia}{kong1qi4}
\significado{s.}{ ar }
\end{pronuncia}
\end{verbete}

\begin{verbete}[kong1tiao2]{空调}[8;10]
\begin{pronuncia}{kong1tiao2}
\significado[台]{s.}{ ar-condicionado; condicionador de ar }
\end{pronuncia}
\end{verbete}

\begin{verbete}[kong3pa4]{恐怕}[10;8]
\begin{pronuncia}{kong3pa4}
\significado{adv.}{ talvez; possivelmente; provavelmente; (em sentido não tão bom) }
\end{pronuncia}
\end{verbete}

\begin{verbete}[kongr4]{空儿}[8;2]
\begin{pronuncia}{kongr4}
\significado{s.}{ tempo livre }
\end{pronuncia}
\end{verbete}

\begin{verbete}[kou3]{口}[3]
\begin{pronuncia}{kou3}
\significado{p.c.}{ para coisas com bocas (pessoas, animais domésticos, canhões, etc); 
  para mordidas ou bocados }
\end{pronuncia}
\end{verbete}

\begin{verbete}[kou3xiang1tang2]{口香糖}[3;9;16]
\begin{pronuncia}[\\]{kou3xiang1tang2}
\significado{s.}{ goma de mascar; chiclete }
\end{pronuncia}
\end{verbete}

\begin{verbete}[kou3yin1]{口音}[3;9]
\begin{pronuncia}{kou3yin1}
\significado{s.}{ sotaque }
\end{pronuncia}
\end{verbete}

\begin{verbete}[kou3yu3]{口语}[3;9]
\begin{pronuncia}{kou3yu3}
\significado[门]{s.}{ linguagem oral; linguagem falada }
\end{pronuncia}
\end{verbete}

\begin{verbete}[ku3gua1]{苦瓜}[8;5]
\begin{pronuncia}{ku3gua1}
\significado{s.}{ melão amargo (cabaça amarga, pêra bálsamo, maçã bálsamo, pepino amargo) }
\end{pronuncia}
\end{verbete}

\begin{verbete}[ku4zi0]{裤子}[12;3]
\begin{pronuncia}{ku4zi0}
\significado[条]{s.}{ calças }
\end{pronuncia}
\end{verbete}

\begin{verbete}[kuai4]{块}[7]
\begin{pronuncia}{kuai4}
\significado{p.c.}{ para unidades de Reminbi, dinheiro; para peças ou pedaços de roupa, bolos, sabão, etc }
\end{pronuncia}
\end{verbete}

\begin{verbete}[kuai4]{快}[7]
\begin{pronuncia}{kuai4}
\significado{adj.}{ quase; rápido; depressa }
\end{pronuncia}
\end{verbete}

\begin{verbete}[kuai4le4]{快乐}[7;5]
\begin{pronuncia}{kuai4le3}
\significado{s.}{ felicidade }
\significado{adj.}{ feliz }
\end{pronuncia}
\end{verbete}

\begin{verbete}[kuan3]{款}[12]
\begin{pronuncia}{kuan3}
\significado[笔,个]{s.}{ montante de dinheiro }
\end{pronuncia}
\end{verbete}                                                                     

%%%%% EOF %%%%%

%%%
%%% L
%%%
%\section*{L}
\addcontentsline{toc}{section}{L}

\begin{verbete}{垃圾}{la1ji1}{8;6}
  \significado[把]{s.}{lixo}
\end{verbete}

\begin{verbete}{垃圾车}{la1ji1che1}{8;6;4}
  \significado{s.}{caminhão de lixo}
\end{verbete}

\begin{verbete}{垃圾电邮}{la1ji1dian4you2}{8;6;5;7}
  \significado{s.}{\emph{e-mail} de \emph{spam}}
\end{verbete}

\begin{verbete}{垃圾堆}{la1ji1dui1}{8;6;11}
  \significado{s.}{depósito de lixo}
\end{verbete}

\begin{verbete}{垃圾工}{la1ji1gong1}{8;6;3}
  \significado{s.}{lixeiro; gari}
\end{verbete}

\begin{verbete}{垃圾食品}{la1ji1shi2pin3}{8;6;9;9}
  \significado{s.}{\emph{junk food}}
\end{verbete}

\begin{verbete}{垃圾筒}{la1ji1tong3}{8;6;12}
  \significado{s.}{cesto de lixo}
\end{verbete}

\begin{verbete}{垃圾箱}{la1ji1xiang1}{8;6;15}
  \significado{s.}{cesto de lixo}
\end{verbete}

\begin{verbete}{垃圾邮件}{la1ji1you2jian4}{8;6;7;6}
  \significado{s.}{\emph{spam}; \emph{e-mail} não solicitado}
\end{verbete}

\begin{verbete}{拉拉队}{la1la1dui4}{8;8;4}
  \significado{s.}{claque; torcida}
\end{verbete}

\begin{verbete}{辣}{la4}{14}
  \significado{adj.}{picante; pungente}
\end{verbete}

\begin{verbete}{来}{lai2}{7}
  \significado{v.}{vir; chegar; trazer}
\end{verbete}

\begin{verbete}{蓝}{lan2}{13}
  \significado*{s.}{sobrenome Lan}
  \significado{adj.}{azul}
\end{verbete}

\begin{verbete}{蓝色}{lan2se4}{13;6}
  \significado{s.}{cor azul}
\end{verbete}

\begin{verbete}{篮球}{lan2qiu2}{16;11}
  \significado[个,只]{s.}{basquetebol}
\end{verbete}

\begin{verbete}{懒}{lan3}{16}
  \significado{adj.}{preguiçoso}
\end{verbete}

\begin{verbete}{懒虫}{lan3chong2}{16;6}
  \significado{s.}{desleixado ocioso; insulto:~sujeito preguiçoso}
\end{verbete}

\begin{verbete}{懒怠}{lan3dai4}{16;9}
  \significado{s.}{preguiça}
\end{verbete}

\begin{verbete}{懒得}{lan3de2}{16;11}
  \significado{adv.}{demasiado preguiçoso}
  \significado{v.}{não sentir vontade (de fazer algo)}
\end{verbete}

\begin{verbete}{懒惰}{lan3duo4}{16;12}
  \significado{adj.}{preguiçoso}
\end{verbete}

\begin{verbete}{懒鬼}{lan3gui3}{16;9}
  \significado{s.}{cara preguiçoso}
\end{verbete}

\begin{verbete}{懒汉}{lan3han4}{16;5}
  \significado{s.}{sujeito ocioso; vagabundo; preguiçosos}
\end{verbete}

\begin{verbete}{懒人}{lan3ren2}{16;2}
  \significado{s.}{pessoa preguiçosa}
\end{verbete}

\begin{verbete}{懒散}{lan3san3}{16;12}
  \significado{adj.}{inativo; indolente; preguiçoso; negligente}
\end{verbete}

\begin{verbete}{懒腰}{lan3yao1}{16;13}
  \significado[个]{s.}{alongamento (do corpo)}
\end{verbete}

\begin{verbete}{劳工同事}{lao2gong1 tong2shi4}{7;3;6;8}
  \significado{s.}{colaborador; colega de trabalho}
\end{verbete}

\begin{verbete}{老}{lao3}{6}[125]
  \significado{adj.}{velho (pessoa); venerável (pessoa); experiente; ultrapassado; duro (carne, etc.)}
  \significado{adv.}{de longa data; sempre; o tempo todo; do passado}
\end{verbete}

\begin{verbete}{老板}{lao3ban3}{6;8}
  \significado[个]{s.}{chefe; patrão; proprietário de empresa}
\end{verbete}

\begin{verbete}{老家}{lao3jia1}{6;10}
  \significado{s.}{estado ou região de origem; terra natal; lugar de origem}
\end{verbete}

\begin{verbete}{老人家}{lao3ren2jia5}{6;2;10}
  \significado{s.}{senhor ancião; madame, senhora; termo educado para mulher ou homem velho}
\end{verbete}

\begin{verbete}{老师}{lao3shi1}{6;6}
  \significado[个,位]{s.}{professor}
\end{verbete}

\begin{verbete}{了}{le5}{2}
  \significado{part.}{marcador de ação concluída; partícula modal indicando mudança de estado, situação; partícula modal intensificando a cláusula anterior}
  \veja{了}{liao3}
  \veja{了}{liao4}
\end{verbete}

\begin{verbete}{累}{lei2}{11}
  \significado*{s.}{sobrenome Lei}
  \significado{s.}{corda}
  \significado{v.}{amarrar; torcer}
  \veja{累}{lei3}
  \veja{累}{lei4}
\end{verbete}

\begin{verbete}{雷亚尔}{lei2ya4'er3}{13;6;5}
  \significado*{s.}{Real Brasileiro}
\end{verbete}

\begin{verbete}{累}{lei3}{11}
  \significado{adj.}{contínuo; repetido}
  \significado{v.}{acumular; envolver ou implicar}
  \veja{累}{lei2}
  \veja{累}{lei4}
\end{verbete}

\begin{verbete}{絫}{lei3}{12}
  \variante{累}{lei3}
\end{verbete}

\begin{verbete}{累}{lei4}{11}
  \significado{adj.}{cansado; fatigado}
  \significado{v.}{forçar; desgastar; trabalhar duro}
  \veja{累}{lei2}
  \veja{累}{lei3}
\end{verbete}

\begin{verbete}{冷}{leng3}{7}
  \significado*{s.}{sobrenome Leng}
  \significado{adj.}{frio}
\end{verbete}

\begin{verbete}{离}{li2}{10}
  \significado*{s.}{sobrenome Li}
  \significado{prep.}{(ser longe) de ... até...}
  \significado{v.}{ficar longe de; deixar; separar-se de}
\end{verbete}

\begin{verbete}{礼节}{li3jie2}{5;5}
  \significado{s.}{protocolo; cerimônia; etiqueta}
\end{verbete}

\begin{verbete}{礼物}{li3wu4}{5;8}
  \significado[件,个,份]{s.}{prenda; lembrança; presente}
\end{verbete}

\begin{verbete}{李四}{li3si4}{7;5}
  \significado*{s.}{Li Si; Zé Ninguém; nome para uma pessoa não especificada, 2 de 3}
  \veja{王五}{wan2wu3}
  \veja{张三}{zhang1san1}
\end{verbete}

\begin{verbete}{里}{li3}{7}[166]
  \significado*{s.}{sobrenome Li}
  \significado{p.l.}{em; dentro; interior}
  \significado{s.}{resina; vizinhança}
\end{verbete}

\begin{verbete}{里斯本}{li3si1ben3}{7;12;5}
  \significado*{s.}{Lisboa}
\end{verbete}

\begin{verbete}{里斯本大学}{li3si1ben3 da4xue2}{7;12;5;3;8}
  \significado*{s.}{Universidade de Lisboa}
\end{verbete}

\begin{verbete}{历史}{li4shi3}{4;5}
  \significado[门,段]{s.}{história}
\end{verbete}

\begin{verbete}{厉害}{li4hai5}{5;10}
  \significado{adj.}{severo; rigoroso; exigente; radical; violento; feroz}
\end{verbete}

\begin{verbete}{立刻}{li4ke4}{5;8}
  \significado{adv.}{imediatamente}
\end{verbete}

\begin{verbete}{例如}{li4ru2}{8;6}
  \significado{conj.}{por exemplo; como}
\end{verbete}

\begin{verbete}{詈骂}{li4ma4}{12;9}
  \significado{v.}{xingar; abusar}
\end{verbete}

\begin{verbete}{俩}{lia3}{9}
  \significado{adv.}{dois; ambos}
\end{verbete}

\begin{verbete}{俩钱}{lia3qian2}{9;10}
  \significado{s.}{uma pequena quantia de dinheiro}
\end{verbete}

\begin{verbete}{莲藕}{lian2'ou3}{10;18}
  \significado{s.}{raiz de Lotus}
\end{verbete}

\begin{verbete}{脸}{lian3}{11}
  \significado[张,个]{s.}{cara; rosto; face}
\end{verbete}

\begin{verbete}{练习}{lian4xi2}{8;3}
  \significado[个]{s.}{prática; exercício}
  \significado{v.}{praticar; exercitar}
\end{verbete}

\begin{verbete}{恋爱}{lian4'ai4}{10;10}
  \significado[个,场]{s.}{amor (romântico)}
  \significado{v.}{sentir-se profundamente apegado a}
\end{verbete}

\begin{verbete}{凉快}{liang2kuai5}{10;7}
  \significado{adj.}{agradável e frio; agradavelmente fresco}
\end{verbete}

\begin{verbete}{两}{liang3}{7}
  \significado{adv.}{ambos}
  \significado{num.}{dois (sempre usado antes de p.c.)}
\end{verbete}

\begin{verbete}{辆}{liang4}{11}
  \significado{p.c.}{para automóveis, veículos, etc.}
\end{verbete}

\begin{verbete}{了}{liao3}{2}
  \significado{v.}{terminar; alcançar; entender claramente}
  \veja{了}{le5}
  \veja{了}{liao4}
\end{verbete}

\begin{verbete}{了}{liao4}{2}
  \significado{adj.}{brilhantes (olhos)}
  \significado{v.}{observar; olhar para fora; olhar para baixo de um lugar mais alto; compreender claramente}
  \veja{了}{le5}
  \veja{了}{liao3}
\end{verbete}

\begin{verbete}{邻居}{lin2ju1}{7;8}
  \significado[个]{s.}{vizinho}
\end{verbete}

\begin{verbete}{菱角}{ling2jiao5}{11;7}
  \significado{s.}{castanha d'água}
\end{verbete}

\begin{verbete}{零/〇}{ling2}{13}
  \significado{adj.}{extra}
  \significado{num.}{zero, 0}
  \significado{s.}{matemática:~resto (após a divisão); fração; nada}
\end{verbete}

\begin{verbete}{领导}{ling3dao3}{11;6}
  \significado[位,个]{s.}{líder; liderança}
  \significado{v.}{liderar}
\end{verbete}

\begin{verbete}{另外}{ling4wai4}{5;5}
  \significado{adv./pron.}{além disso}
\end{verbete}

\begin{verbete}{流利}{liu2li4}{10;7}
  \significado{adj.}{fluente (em uma língua)}
\end{verbete}

\begin{verbete}{柳橙汁}{liu3cheng2zhi1}{9;16;5}
  \significado[瓶,杯,罐,盒]{s.}{suco de laranja}
  \veja{橙汁}{cheng2zhi1}
  \veja{橘子汁}{ju2zi5zhi1}
\end{verbete}

\begin{verbete}{六}{liu4}{4}
  \significado{num.}{seis, 6}
\end{verbete}

\begin{verbete}{遛狗}{liu4gou3}{13;8}
  \significado{v.+compl.}{passear com um cachorro}
\end{verbete}

\begin{verbete}{龙}{long2}{5}
  \significado*{s.}{sobrenome Long}
  \significado{adj.}{imperial}
  \significado[条]{s.}{dragão}
\end{verbete}

\begin{verbete}{龙山}{long2shan1}{5;3}
  \significado*{s.}{Longshan}
\end{verbete}

\begin{verbete}{楼}{lou2}{13}
  \significado*{s.}{sobrenome Lou}
  \significado{p.c.}{andar; piso}
  \significado[层,座,栋]{s.}{edifício; prédio; casa com mais de 1 andar}
\end{verbete}

\begin{verbete}{漏电}{lou4dian4}{14;5}
  \significado{v.}{vazar eletricidade}
\end{verbete}

\begin{verbete}{芦笋}{lu2sun3}{7;10}
  \significado{s.}{aspargos}
\end{verbete}

\begin{verbete}{录像带}{lu4xiang4dai4}{8;13;9}
  \significado[盘]{s.}{video-cassete}
\end{verbete}

\begin{verbete}{录像机}{lu4xiang4ji1}{8;13;6}
  \significado[台]{s.}{gravador de vídeo; VCR}
\end{verbete}

\begin{verbete}{录音}{lu4yin1}{8;9}
  \significado[个]{s.}{gravação de som}
  \significado{v.+compl.}{gravar (som)}
\end{verbete}

\begin{verbete}{录音机}{lu4yin1ji1}{8;9;6}
  \significado[台]{s.}{gravador de áudio}
\end{verbete}

\begin{verbete}{路}{lu4}{13}
  \significado*{s.}{sobrenome Lu}
  \significado[条]{s.}{caminho; estrada; via; jornada; linha (ônibus, etc.); rota}
\end{verbete}

\begin{verbete}{路口}{lu4kou3}{13;3}
  \significado{s.}{cruzamento; interseção (de estradas)}
\end{verbete}

\begin{verbete}{伦敦}{lun2dun1}{6;12}
  \significado*{s.}{Londres}
\end{verbete}

\begin{verbete}{罗}{luo2}{8}
  \significado*{s.}{sobrenome Luo}
  \significado{v.}{coletar; juntar; pegar; peneirar}
\end{verbete}

\begin{verbete}{旅行}{lv3xing2}{10;6}
  \significado{v.}{viajar}
\end{verbete}

\begin{verbete}{旅游}{lv3you2}{10;12}
  \significado[趟,次,个]{s.}{jornada; viagem}
  \significado{v.}{viajar}
\end{verbete}

\begin{verbete}{屡次}{lv3ci4}{12;6}
  \significado{adv.}{repetidamente; uma e outra vez; muitas vezes}
\end{verbete}

\begin{verbete}{绿}{lv4}{11}
  \significado{adj.}{verde}
\end{verbete}

\begin{verbete}{绿豆}{lv4dou4}{11;7}
  \significado{s.}{vagens}
\end{verbete}

\begin{verbete}{绿豆芽}{lv4dou4 ya2}{11;7;7}
  \significado{s.}{broto de feijão verde}
\end{verbete}

\begin{verbete}{绿色}{lv4se4}{11;6}
  \significado{s.}{cor verde}
\end{verbete}

\begin{verbete}{略}{lve4}{11}
  \significado{adv.}{ligeiramente; marginalmente; aproximadamente}
\end{verbete}

\begin{verbete}{略微}{lve4wei1}{11;13}
  \significado{adv.}{ligeiramente; marginalmente; aproximadamente}
\end{verbete}

%%%%% EOF %%%%%

%%%
%%% M
%%%
%\section*{M}
\addcontentsline{toc}{section}{M}

\begin{verbete}{妈妈}{ma1ma5}{6;6}
  \significado[个,位]{s.}{mamãe, mãe}
\end{verbete}

\begin{verbete}{麻烦}{ma2fan5}{11;10}
  \significado{adj.}{fastidioso; maçante; inconveniente; problemático}
  \significado{s.}{incômodo}
  \significado{v.}{incomodar alguém; colocar alguém em apuros}
\end{verbete}

\begin{verbete}{麻辣豆腐}{ma2la4 dou4fu5}{11;14;7;14}
  \significado{s.}{tofú guisado em molho picante (prato)}
\end{verbete}

\begin{verbete}{马路}{ma3lu4}{3;13}
  \significado[条]{s.}{rua; estrada}
\end{verbete}

\begin{verbete}{马马虎虎}{ma3ma3hu3hu3}{3;3;8;8}
  \significado{adj.}{descuidado; casual; tolerável; vago; mais ou menos}
\end{verbete}

\begin{verbete}{马上}{ma3shang4}{3;3}
  \significado{adv.}{já; imediatamente; de imediato; sem demora}
\end{verbete}

\begin{verbete}{㐷}{ma4}{5}
  \variante{骂}{ma4}
\end{verbete}

\begin{verbete}{骂}{ma4}{9}
  \significado{v.}{insultar; maldizer; ralhar}
\end{verbete}

\begin{verbete}{骂街}{ma4jie1}{9;12}
  \significado{v.}{gritar abusos na rua}
\end{verbete}

\begin{verbete}{骂名}{ma4ming2}{9;6}
  \significado{s.}{infâmia}
\end{verbete}

\begin{verbete}{吗}{ma5}{6}
  \significado{part.}{partícula interrogativa (usado em perguntas ``sim-não'')}
\end{verbete}

\begin{verbete}{买}{mai3}{6}
  \significado{v.}{comprar}
\end{verbete}

\begin{verbete}{买东西}{mai3dong1xi5}{6;5;6}
  \significado{v.}{fazer compras}
\end{verbete}

\begin{verbete}{麦当劳}{mai4dang1lao2}{7;6;7}
  \significado*{s.}{McDonald's (empresa de \emph{fast-food})}
  \veja{麦当劳叔叔}{mai4dang1lao2 shu1shu5}
\end{verbete}

\begin{verbete}{麦当劳叔叔}{mai4dang1lao2 shu1shu5}{7;6;7;8;8}
  \significado*{s.}{Ronald McDonald}
  \veja{麦当劳}{mai4dang1lao2}
\end{verbete}

\begin{verbete}{卖}{mai4}{8}
  \significado{v.}{vender}
\end{verbete}

\begin{verbete}{满意}{man3yi4}{13;13}
  \significado{adj.}{satisfatório}
\end{verbete}

\begin{verbete}{谩骂}{man4ma4}{13;9}
  \significado{v.}{ridicularizar; abusar}
\end{verbete}

\begin{verbete}{慢}{man4}{14}
  \significado{adj.}{devagar}
\end{verbete}

\begin{verbete}{漫骂}{man4ma4}{14;9}
  \variante{谩骂}{man4ma4}
\end{verbete}

\begin{verbete}{忙}{mang1}{6}
  \significado{adj.}{ocupado}
  \significado{s.}{apressar}
\end{verbete}

\begin{verbete}{猫}{mao1}{11}
  \significado[只]{s.}{gato; coloquial:~MODEM}
  \significado{v.}{dialeto: esconder-se}
\end{verbete}

\begin{verbete}{猫熊}{mao1xiong2}{11;14}
  \veja{熊猫}{xiong2mao1}
\end{verbete}

\begin{verbete}{毛}{mao2}{4}[82]
  \significado*{s.}{sobrenome Mao}
  \significado{p.c.}{1 mao = 10 centavos}
\end{verbete}

\begin{verbete}{贸易}{mao4yi4}{9;8}
  \significado[个]{s.}{transação comercial}
  \significado{v.}{fazer uma transação comercial}
\end{verbete}

\begin{verbete}{没}{mei2}{7}
  \significado{adv.}{não ter; não há; ficar sem; não (prefixo negativo para verbos, traduzido para outras línguas com verbos no pretérito)}
  \veja{没}{mo4}
\end{verbete}

\begin{verbete}{没关系}{mei2guan1xi5}{7;6;7}
  \significado{v.}{não ter problema; não ter importância; não fazer mal}
\end{verbete}

\begin{verbete}{没用}{mei2yong4}{7;5}
  \significado{adj.}{inútil}
\end{verbete}

\begin{verbete}{没有}{mei2you3}{7;6}
  \significado{v.}{não há; não tem; não existe}
\end{verbete}

\begin{verbete}{没有关系}{mei2you3guan1xi5}{7;6;6;7}
  \veja{没关系}{mei2guan1xi5}
\end{verbete}

\begin{verbete}{没有意思}{mei2you3yi4si5}{7;6;13;9}
  \significado{adj.}{tedioso; chato; sem interessante}
\end{verbete}

\begin{verbete}{眉毛}{mei2mao5}{9;4}
  \significado[根]{s.}{sobrancelha}
\end{verbete}

\begin{verbete}{每}{mei3}{7}
  \significado{pron.}{cada}
\end{verbete}

\begin{verbete}{每次}{mei3ci4}{7;6}
  \significado{adv.}{toda vez; cada vez}
\end{verbete}

\begin{verbete}{每天}{mei3tian1}{7;4}
  \significado{adv.}{todo dia; cada dia}
\end{verbete}

\begin{verbete}{美国}{mei3guo1}{9;8}
  \significado*{s.}{Estados Unidos da América}
\end{verbete}

\begin{verbete}{美国人}{mei3guo1ren2}{9;8;2}
  \significado{s.}{americano; nascido nos Estados Unidos da América}
\end{verbete}

\begin{verbete}{美丽}{mei3li4}{9;7}
  \significado{adj.}{bonito; lindo}
\end{verbete}

\begin{verbete}{美元}{mei3yuan2}{9;4}
  \significado*{s.}{Dólar Americano}
\end{verbete}

\begin{verbete}{美洲}{mei3zhou1}{9;9}
  \significado*{s.}{América (incluindo Norte, Central e Sul)}
\end{verbete}

\begin{verbete}{美洲人}{mei3zhou1ren2}{9;9;2}
  \significado{s.}{americano; nascido no continente Americano}
\end{verbete}

\begin{verbete}{妹夫}{mei4fu5}{8;4}
  \significado{s.}{marido da irmã mais nova}
\end{verbete}

\begin{verbete}{妹妹}{mei4mei5}{8;8}
  \significado[个]{s.}{irmã mais nova; mulher jovem}
\end{verbete}

\begin{verbete}{门口}{men2kou3}{3;3}
  \significado[个]{p.d.l.}{porta; portão}
\end{verbete}

\begin{verbete}{们}{men5}{5}
  \significado{part.}{sufixo para plural de pronomes e substantivos referentes a indivíduos}
\end{verbete}

\begin{verbete}{猛然}{meng3ran2}{11;12}
  \significado{adv.}{de repente; abruptamente}
\end{verbete}

\begin{verbete}{米饭}{mi3fan4}{6;7}
  \significado{s.}{arroz cozido}
\end{verbete}

\begin{verbete}{免得}{mian3de5}{7;11}
  \significado{conj.}{de modo a não; para evitar; para que não}
\end{verbete}

\begin{verbete}{靣}{mian4}{8}
  \variante{面}{mian4}
\end{verbete}

\begin{verbete}{面}{mian4}{9}[176]
  \significado{p.c.}{para objetos com superfície plana como tambores, espelhos, bandeiras, etc.}
  \significado{s.}{farinha; massa; gíria:~(uma pessoa) ineficaz}
\end{verbete}

\begin{verbete}{面包}{mian4bao1}{9;5}
  \significado[个,片,袋,块]{s.}{pão}
  \exemplo{我买八个面包了。}[Comprei oito pães.]
  \exemplo{他在吃两片面包。}[Ele está comendo duas fatias de pão.]
  \exemplo{我在家里带了一袋面包。}[Trouxe um saco de pão para casa.]
  \exemplo{我拿了一块面包。}[Peguei um pedaço de pão.]
\end{verbete}

\begin{verbete}{面积}{mian4ji1}{9;10}
  \significado{s.}{área (de um andar, pedaço de terreno, etc.); área de superfície; pedaço de terra}
\end{verbete}

\begin{verbete}{面条}{mian4tiao2}{9;7}
  \significado{s.}{macarrão; espaguete}
\end{verbete}

\begin{verbete}{糆}{mian4}{15}
  \variante{面}{mian4}
\end{verbete}

\begin{verbete}{麫}{mian4}{15}
  \variante{面}{mian4}
\end{verbete}

\begin{verbete}{名片}{ming2pian4}{6;4}
  \significado{s.}{cartão de visita}
\end{verbete}

\begin{verbete}{名字}{ming2zi5}{6;6}
  \significado[个]{s.}{nome (de uma pessoa ou coisa)}
\end{verbete}

\begin{verbete}{明白}{ming2bai5}{8;5}
  \significado{adj.}{compreendido; percebido; óbvio; inequívoco}
  \significado{v.}{compreender; perceber}
\end{verbete}

\begin{verbete}{明明}{ming2ming2}{8;8}
  \significado{interr.}{obviamente, claramente}
\end{verbete}

\begin{verbete}{明年}{ming2nian2}{8;6}
  \significado{p.t.}{próximo ano}
\end{verbete}

\begin{verbete}{明天}{ming2tian1}{8;4}
  \significado{p.t.}{amanhã}
\end{verbete}

\begin{verbete}{磨菇}{mo2gu5}{16;11}
  \variante{蘑菇}{mo2gu5}
\end{verbete}

\begin{verbete}{蘑菇}{mo2gu5}{19;11}
  \significado{s.}{cogumelo}
  \significado{v.}{mandriar; embromar; amofinar; incomodar alguém com solicitações ou interrupções frequentes ou persistentes}
\end{verbete}

\begin{verbete}{没}{mo4}{7}
  \significado{adj.}{afogado}
  \significado{v.}{acabar; morrer; inundar}
  \variante{没}{mei2}
\end{verbete}

\begin{verbete}{墨镜}{mo4jing4}{15;16}
  \significado[只,双,副]{s.}{óculos escuros}
\end{verbete}

\begin{verbete}{母亲}{mu3qin1}{5;9}
  \significado[个]{s.}{mãe}
  \veja{母亲}{mu3qin5}
\end{verbete}

\begin{verbete}{母亲}{mu3qin5}{5;9}
  \significado[个]{s.}{mãe}
  \veja{母亲}{mu3qin1}
\end{verbete}

\begin{verbete}{母语}{mu3yu3}{5;9}
  \significado{s.}{língua materna; língua nativa}
\end{verbete}

%%%%% EOF %%%%%

%%%
%%% N
%%%

\section*{N}\addcontentsline{toc}{section}{N}

\begin{entry}{那}{na1}{6}[Radical 邑]
  \definition*{s.}{sobrenome Na}
\end{entry}

\begin{entry}{拿}{na2}{10}[Radical 手]
  \definition{part.}{usado da mesma forma que 把: para marcar o seguinte substantivo seguinte como objeto direto}
  \definition{v.}{segurar | tomar | pegar em}
\end{entry}

\begin{entry}{那}{na3}{6}[Radical 邑]
  \variantof{哪}
\end{entry}

\begin{entry}{哪}{na3}{9}[Radical 口]
  \definition{prep.}{que? | qual?}
\end{entry}

\begin{entry}{哪国人}{na3 guo2ren2}{9,8,2}
  \definition{expr.}{de qual país?}
\end{entry}

\begin{entry}{哪里}{na3li3}{9,7}
  \definition{adv.}{onde?}
\end{entry}

\begin{entry}{哪怕}{na3pa4}{9,8}
  \definition{conj.}{mesmo se/embora | até | não importa como}
\end{entry}

\begin{entry}{哪儿}{na3r5}{9,2}
  \definition{adv.}{onde?}
\end{entry}

\begin{entry}{哪些}{na3xie1}{9,8}
  \definition{pron.}{quais?}
\end{entry}

\begin{entry}{那}{na4}{6}[Radical 邑]
  \definition{conj.}{nessa situação | nesse caso}
  \definition{pron.}{aquele | aquilo}
\end{entry}

\begin{entry}{那里}{na4li5}{6,7}
  \definition{pron.}{lá | ali}
\end{entry}

\begin{entry}{那么}{na4me5}{6,3}
  \definition{adv.}{então | como aquele | dessa maneira}
\end{entry}

\begin{entry}{那末}{na4me5}{6,5}
  \variantof{那么}
\end{entry}

\begin{entry}{那麽}{na4me5}{6,14}
  \variantof{那么}
\end{entry}

\begin{entry}{那儿}{na4r5}{6,2}
  \definition{pron.}{lá | ali}
\end{entry}

\begin{entry}{那些}{na4xie1}{6,8}
  \definition{pron.}{aqueles}
\end{entry}

\begin{entry}{奶奶}{nai3nai5}{5,5}
  \definition[位]{s.}{avó (paterna) | (respeitoso) dona da casa}
\end{entry}

\begin{entry}{耐心}{nai4xin1}{9,4}
  \definition{s.}{paciência}
  \definition{v.}{ser paciente}
\end{entry}

\begin{entry}{男}{nan2}{7}[Radical 田]
  \definition{adj.}{masculino}
  \definition{s.}{Barão, o mais baixo de cinco ordens de nobreza}
\end{entry}

\begin{entry}{男孩儿}{nan2hai2r5}{7,9,2}
  \definition{s.}{menino | rapaz}
\end{entry}

\begin{entry}{男朋友}{nan2peng2you5}{7,8,4}
  \definition{s.}{namorado}
\end{entry}

\begin{entry}{南边}{nan2bian5}{9,5}
  \definition{adv.}{sul | lado sul | parte sul | ao sul de}
\end{entry}

\begin{entry}{南方}{nan2fang1}{9,4}
  \definition{s.}{sul | o Sul | a parte sul do país}
\end{entry}

\begin{entry}{南极}{nan2ji2}{9,7}
  \definition*{s.}{Antártico | Pólo Sul}
  \definition{s.}{pólo sul magnético}
\end{entry}

\begin{entry}{南面}{nan2mian4}{9,9}
  \definition{s.}{sul | lado sul}
\end{entry}

\begin{entry}{难}{nan2}{10}[Radical 隹]
  \definition{adj.}{difícil}
  \definition{s.}{dificuldade}
  \seeref{难}{nan4}
\end{entry}

\begin{entry}{难道}{nan2dao4}{10,12}
  \definition{adv.}{indica uma pergunta retórica | certamente não significa que\dots | é possível que\dots}
\end{entry}

\begin{entry}{难度}{nan2du4}{10,9}
  \definition{s.}{grau de dificuldade}
\end{entry}

\begin{entry}{难}{nan4}{10}[Radical 隹]
  \definition{s.}{desastre}
  \definition{v.}{repreender}
  \seeref{难}{nan2}
\end{entry}

\begin{entry}{孬}{nao1}{10}[Radical 子]
  \definition{adj.}{(dialeto) não (é) bom (contração de 不+好)}
\end{entry}

\begin{entry}{脑袋}{nao3dai5}{10,11}
  \definition[颗,个]{s.}{cabeça | crânio | cérebro | capacidade mental}
\end{entry}

\begin{entry}{脑瓜}{nao3gua1}{10,5}
  \definition{s.}{crânio | cérebro | cabeça | mente | mentalidade | ideia}
  \seealsoref{脑瓜子}{nao3gua1zi5}
\end{entry}

\begin{entry}{脑瓜子}{nao3gua1zi5}{10,5,3}
  \definition{s.}{crânio | cérebro | cabeça | mente | mentalidade | ideia}
  \seealsoref{脑瓜}{nao3gua1}
\end{entry}

\begin{entry}{呢}{ne5}{8}[Radical 口]
  \definition{part.}{(no final de uma frase declarativa) partícula que indica a continuação de um estado ou ação |  partícula para perguntar sobre a localização (``Onde está\dots?'') | partícula indicando  afirmação forte | partícula indicando que uma pergunta feita anteriormente deve ser aplicada à palavra anterior (``E quanto a\dots?'', ``E\dots?'') | partícula sinalizando uma pausa, para enfatizar as palavras anteriores e permitir que o ouvinte tenha tempo para compreendê-las (``ok?'', ``você está comigo ?'')}
  \seeref{呢}{ni2}
\end{entry}

\begin{entry}{内存}{nei4cun2}{4,6}
  \definition{s.}{armazenamento interno | memória do computador | RAM (\emph{random access memory})}
  \seealsoref{随机存取存储器}{sui2ji1cun2qu3cun2chu3qi4}
  \seealsoref{随机存取记忆体}{sui2ji1cun2qu3ji4yi4ti3}
\end{entry}

\begin{entry}{内燃机}{nei4ran2ji1}{4,16,6}
  \definition{s.}{motor de combustão interna}
\end{entry}

\begin{entry}{内省}{nei4xing3}{4,9}
  \definition{s.}{introspecção}
  \definition{v.}{refletir sobre si mesmo}
\end{entry}

\begin{entry}{能}{neng2}{10}[Radical 肉]
  \definition*{s.}{sobrenome Neng}
  \definition{adv.}{talvez}
  \definition{s.}{(física)nenergia | habilidade}
  \definition{v.}{poder | ser capaz de}
\end{entry}

\begin{entry}{能干}{neng2gan4}{10,3}
  \definition{adj.}{capaz | competente}
\end{entry}

\begin{entry}{能够}{neng2gou4}{10,11}
  \definition{v.}{ser capaz de}
\end{entry}

\begin{entry}{能上能下}{neng2shang4neng2xia4}{10,3,10,3}
  \definition{s.}{pronto para aceitar qualquer trabalho, alto ou baixo}
\end{entry}

\begin{entry}{呢}{ni2}{8}[Radical 口]
  \definition{s.}{material de lã}
  \seeref{呢}{ne5}
\end{entry}

\begin{entry}{泥}{ni2}{8}[Radical 水]
  \definition{s.}{lama | argila | pasta | polpa}
  \seeref{泥}{ni4}
\end{entry}

\begin{entry}{泥潭}{ni2tan2}{8,15}
  \definition{s.}{atoleiro | lamaçal | charco | pântano}
\end{entry}

\begin{entry}{伲}{ni3}{7}[Radical 人]
  \variantof{你}
\end{entry}

\begin{entry}{你}{ni3}{7}[Radical 人]
  \definition{pron.}{você (informal) | tu | te | ti | contigo}
  \seeref{您}{nin2}
\end{entry}

\begin{entry}{你的}{ni3 de5}{7,8}
  \definition{pron.}{seu}
\end{entry}

\begin{entry}{你好}{ni3hao3}{7,6}
  \definition{interj.}{Olá! | Oi!}
\end{entry}

\begin{entry}{你们}{ni3men5}{7,5}
  \definition{pron.}{vocês (informal) | vós | vos | convosco}
\end{entry}

\begin{entry}{你们的}{ni3men5 de5}{7,5,8}
  \definition{pron.}{vossos}
\end{entry}

\begin{entry}{袮}{ni3}{10}[Radical 衣]
  \definition{pron.}{Você, Tu (divindade)}
  \variantof{你}
\end{entry}

\begin{entry}{泥}{ni4}{8}[Radical 水]
  \definition{adj.}{contido}
  \seeref{泥}{ni2}
\end{entry}

\begin{entry}{逆境}{ni4jing4}{9,14}
  \definition{s.}{adversidade | tribulação}
\end{entry}

\begin{entry}{年}{nian2}{6}[Radical 干]
  \definition*{s.}{sobrenome Nian}
\end{entry}

\begin{entry}{年货}{nian2huo4}{6,8}
  \definition{s.}{mercadorias vendidas no Ano Novo Chinês}
\end{entry}

\begin{entry}{年级}{nian2ji2}{6,6}
  \definition[个]{s.}{classe | ano (escola)}
\end{entry}

\begin{entry}{年纪}{nian2ji4}{6,6}
  \definition[个]{s.}{grau | nota | classe | categoria | graduação | ano (na escola, faculdade, etc.)}
\end{entry}

\begin{entry}{年轻}{nian2qing1}{6,9}
  \definition{adj.}{jovem}
\end{entry}

\begin{entry}{碾碎}{nian3sui4}{15,13}
  \definition{v.}{pulverizar | esmagar}
\end{entry}

\begin{entry}{鸟儿}{niao3r5}{5,2}
  \definition[只]{s.}{pássaro | ave}
\end{entry}

\begin{entry}{尿}{niao4}{7}[Radical 尸]
  \definition[泡]{s.}{urina}
  \definition{v.}{urinar}
  \seeref{尿}{sui1}
\end{entry}

\begin{entry}{您}{nin2}{11}[Radical 心]
  \definition{pron.}{você (formal) | tu | te | ti | contigo}
  \seeref{你}{ni3}
\end{entry}

\begin{entry}{宁}{ning2}{5}[Radical 宀]
  \definition*{s.}{sobrenome Ning}
  \definition{adj.}{calmo, pacífico, sereno | saudável}
  \seeref{宁}{ning4}
\end{entry}

\begin{entry}{柠檬}{ning2meng2}{9,17}
  \definition{s.}{limão}
\end{entry}

\begin{entry}{拧开}{ning3kai1}{8,4}
  \definition{v.}{desaparafusar | desatarrachar | torcer (uma tampa) | abrir (uma torneira) | ligar (girando um botão) | girar (maçaneta da porta)}
\end{entry}

\begin{entry}{宁}{ning4}{5}[Radical 宀]
  \definition{conj.}{mais\dots do que\dots, melhor\dots do que\dots}
  \seeref{宁}{ning2}
\end{entry}

\begin{entry}{宁可}{ning4ke3}{5,5}
  \definition{conj.}{mais\dots do que\dots | melhor\dots do que\dots}
\end{entry}

\begin{entry}{宁可……也不……}{ning4ke3 ye3bu4}{5,5,3,4}
  \definition{conj.}{em vez de\dots}
\end{entry}

\begin{entry}{宁可……也要……}{ning4ke3 ye3yao4}{5,5,3,9}
  \definition{conj.}{mesmo que tenhamos que\dots nós iremos\dots}
\end{entry}

\begin{entry}{宁肯}{ning4ken3}{5,8}
  \definition{conj.}{mais\dots do que\dots, melhor\dots do que\dots}
\end{entry}

\begin{entry}{宁愿}{ning4yuan4}{5,14}
  \definition{conj.}{mais\dots do que\dots, melhor\dots do que\dots}
\end{entry}

\begin{entry}{牛}{niu2}{4}[Radical 牛][Kangxi 93]
  \definition*{s.}{sobrenome Niu}
  \definition[条,头]{s.}{boi | touro | vaca | (gíria) incrível}
\end{entry}

\begin{entry}{牛顿}{niu2dun4}{4,10}
  \definition*{s.}{Newton (nome) | newton (N, unidade de força do SI)}
\end{entry}

\begin{entry}{牛郎织女}{niu2lang2zhi1nv3}{4,8,8,3}
  \definition*{s.}{Vaqueiro e Tecelã (personagens de contos folclóricos) | amantes separados | Altair e Vega (estrelas)}
\end{entry}

\begin{entry}{牛奶}{niu2nai3}{4,5}
  \definition[瓶,杯]{s.}{leite de vaca}
\end{entry}

\begin{entry}{牛人}{niu2ren2}{4,2}
  \definition{s.}{(coloquial) o cara | verdadeiro especialista | \emph{badass}}
\end{entry}

\begin{entry}{牛肉}{niu2rou4}{4,6}
  \definition{s.}{carne de vaca | bife}
\end{entry}

\begin{entry}{牛仔裤}{niu2zai3ku4}{4,5,12}
  \definition[条]{s.}{calça de ganga, jeans}
\end{entry}

\begin{entry}{农村}{nong2cun1}{6,7}
  \definition[个]{s.}{campo rural | aldeia | povoação rústica}
\end{entry}

\begin{entry}{浓}{nong2}{9}[Radical 水]
  \definition{adj.}{concentrado | denso | forte (cheiro, etc.)}
\end{entry}

\begin{entry}{努力}{nu3li4}{7,2}
  \definition{adj.}{diligente | aplicado}
  \definition{s.}{esforçar-se | se esforçar}
\end{entry}

\begin{entry}{怒骂}{nu4ma4}{9,9}
  \definition{v.}{praguejar de raiva}
\end{entry}

\begin{entry}{暖}{nuan3}{13}[Radical 日]
  \definition{adj.}{quente}
  \definition{v.}{esquentar}
\end{entry}

\begin{entry}{暖和}{nuan3huo5}{13,8}
  \definition{adj.}{morno; agradável e quente}
\end{entry}

\begin{entry}{暖气}{nuan3qi4}{13,4}
  \definition{s.}{aquecimento central | aquecedor | ar quente}
\end{entry}

\begin{entry}{那}{nuo2}{6}[Radical 邑]
  \definition*{s.}{sobrenome Nuo}
\end{entry}

\begin{entry}{诺贝尔奖}{nuo4bei4'er3 jiang3}{10,4,5,9}
  \definition*{s.}{Prêmio Nobel}
\end{entry}

\begin{entry}{诺奖}{nuo4jiang3}{10,9}
  \definition*{s.}{Prêmio Nobel, abreviação de 诺贝尔奖}
  \seeref{诺贝尔奖}{nuo4bei4'er3 jiang3}
\end{entry}

\begin{entry}{女}{nv3}{3}[Radical 女][Kangxi 38]
  \definition{adj.}{feminino}
\end{entry}

\begin{entry}{女儿}{nv3'er2}{3,2}
  \definition{s.}{filha}
  \seealsoref{儿子}{er2zi5}
\end{entry}

\begin{entry}{女孩}{nv3hai2}{3,9}
  \definition{s.}{menina | garota}
\end{entry}

\begin{entry}{女朋友}{nv3peng2you5}{3,8,4}
  \definition{s.}{namorada}
\end{entry}

\begin{entry}{女王}{nv3wang2}{3,4}
  \definition{s.}{rainha}
\end{entry}

\begin{entry}{女婿}{nv3xu5}{3,12}
  \definition{s.}{marido da filha}
\end{entry}

%%%%% EOF %%%%%


%%%
%%% O
%%%
%\section*{O}
\addcontentsline{toc}{section}{O}

\begin{verbete}{喔}{o1}{12}[Radical 口][Componentes 口屋]
  \significado{interj.}{Oh!, Entendi! (usado para indicar realização, compreensão)}
\end{verbete}

\begin{verbete}{哦}{o2}{10}[Radical ⼝][Componentes ⼝我]
  \significado{interj.}{Oh! (indicando dúvida ou surpresa)}
  \veja{哦}{e2}
  \veja{哦}{o4}
  \veja{哦}{o5}
\end{verbete}

\begin{verbete}{哦}{o4}{10}[Radical ⼝][Componentes ⼝我]
  \significado{interj.}{Oh! (indicando que acabou de aprender algo)}
  \veja{哦}{e2}
  \veja{哦}{o2}
  \veja{哦}{o5}
\end{verbete}

\begin{verbete}{哦}{o5}{10}[Radical ⼝][Componentes ⼝我]
  \significado{part.}{final da frase que transmite informalidade, calor, simpatia ou intimidade; também pode indicar que alguém está declarando um fato de que a outra pessoa não está ciente}
  \veja{哦}{e2}
  \veja{哦}{o2}
  \veja{哦}{o4}
\end{verbete}

\begin{verbete}{区}{ou1}{4}[Radical 匸][Componentes 匚乂]
  \significado*{s.}{sobrenome Ou}
  \veja{区}{qu1}
\end{verbete}

\begin{verbete}{欧}{ou1}{8}[Radical 欠][Componentes 区欠]
  \significado*{s.}{Europa, abreviação de~欧洲; sobrenome Ou}
  \veja{欧洲}{ou1zhou1}
\end{verbete}

\begin{verbete}{欧盟}{ou1meng2}{8;13}
  \significado*{s.}{Uniáo Europeia}
\end{verbete}

\begin{verbete}{欧洲}{ou1zhou1}{8;9}
  \significado*{s.}{Europa}
  \veja{欧}{ou1}
\end{verbete}

\begin{verbete}{欧洲共同体}{ou1zhou1 gong4tong2ti3}{8;9;6;6;7}
  \significado*{s.}{Comunidade Europeia}
\end{verbete}

\begin{verbete}{欧洲人}{ou1zhou1ren2}{8;9;2}
  \significado{s.}{europeu; nascido na Europa}
\end{verbete}

\begin{verbete}{偶然}{ou3ran2}{11;12}
  \significado{adv.}{por acaso; fortuitamente}
\end{verbete}

%%%%% EOF %%%%%

%%%
%%% P
%%%

\section*{P}\addcontentsline{toc}{section}{P}

\begin{entry}{扒犁}{pa2li2}{5,11}{⼿、⽜}
  \definition{s.}{trenó}
  \seealsoref{爬犁}{pa2li2}
\end{entry}

\begin{entry}{爬}{pa2}{8}{⽖}[HSK 2]
  \definition{v.}{rastejar; arrastar-se; engatinhar | escalar; trepar; subir com dificuldade | sentar-se; levantar-se; levantar-se da posição deitada ou sentada}
\end{entry}

\begin{entry}{爬杆}{pa2gan1}{8,7}{⽖、⽊}
  \definition{s.}{escalada em poste}
  \definition{v.}{escalar um poste}
\end{entry}

\begin{entry}{爬竿}{pa2gan1}{8,9}{⽖、⽵}
  \definition{s.}{poste de escalada | escalada em poste (como ginástica ou ato de circo)}
\end{entry}

\begin{entry}{爬犁}{pa2li2}{8,11}{⽖、⽜}
  \definition{s.}{trenó}
  \seealsoref{扒犁}{pa2li2}
\end{entry}

\begin{entry}{爬墙}{pa2qiang2}{8,14}{⽖、⼟}
  \definition{v.}{escalar uma parede}
\end{entry}

\begin{entry}{爬山}{pa2shan1}{8,3}{⽖、⼭}[HSK 2]
  \definition{v.+compl.}{escalar uma montanha;}
\end{entry}

\begin{entry}{爬上}{pa2shang4}{8,3}{⽖、⼀}
  \definition{v.}{escalar}
\end{entry}

\begin{entry}{爬升}{pa2sheng1}{8,4}{⽖、⼗}
  \definition{v.}{ascender | ganhar promoção | subir (números de vendas, etc.) | aumentar}
\end{entry}

\begin{entry}{爬梳}{pa2shu1}{8,11}{⽖、⽊}
  \definition{v.}{vasculhar (documentos históricos, etc.) | desvendar}
\end{entry}

\begin{entry}{爬行}{pa2xing2}{8,6}{⽖、⾏}
  \definition{v.}{rastejar | arrastar | engatinhar}
\end{entry}

\begin{entry}{怕}{pa4}{8}{⼼}[HSK 2]
  \definition{adv.}{(expressando suposição, julgamento, estimativa, etc.) talvez; suponho; receio (que)}
  \definition{adv.}{por medo; talvez; suponho}
  \definition{v.}{temer; ter medo; recear; sentir medo, ficar nervoso | estar preocupado com; estar preocupado por (ou sobre); ter medo de que algo possa acontecer | ser afetado por; não conseguir suportar; não aguentar mais}
\end{entry}

\begin{entry}{拍}{pai1}{8}{⼿}[HSK 3]
  \definition[只,把]{s.}{bastão; raquete | batida; tempo}
  \definition{v.}{bater palmas; bater; dar um tapa | chicotear; açoitar; bater | enviar (um telegrama, etc.) | tirar (uma foto); fotografar | bajular; lisonjear; adular}
\end{entry}

\begin{entry}{拍马}{pai1ma3}{8,3}{⼿、⾺}
  \definition{v.}{instigar um cavalo dando tapinhas em seu traseiro | lisonjear | bajular}
  \seealsoref{拍马屁}{pai1ma3pi4}
\end{entry}

\begin{entry}{拍马屁}{pai1ma3pi4}{8,3,7}{⼿、⾺、⼫}
  \definition{s.}{puxa-saco | bajulador}
  \definition{v.}{puxar o saco | bajular}
  \seealsoref{拍马}{pai1ma3}
\end{entry}

\begin{entry}{拍摄}{pai1 she4}{8,13}{⼿、⼿}[HSK 5]
  \definition{s.}{fotografar; tirar (uma foto); usar uma câmera fotográfica para capturar imagens de pessoas e objetos}
\end{entry}

\begin{entry}{拍照}{pai1 zhao4}{8,13}{⼿、⽕}[HSK 4]
  \definition{v.+compl.}{fotografar; tirar uma foto}
\end{entry}

\begin{entry}{排}{pai2}{11}{⼿}[HSK 2,3]
  \definition{clas.}{usado para linhas, filas; coisas usadas para formar filas}
  \definition{s.}{linha; fileira; fileiras horizontais | pelotão; unidade militar, abaixo do nível de companhia, acima do nível de pelotão | jangada; balsa; um meio de transporte aquático feito de bambu e madeira unidos lado a lado; também se refere a bambu e madeira amarrados em fileiras para facilitar o transporte aquático | torta; bolo de carne; bolinho assado; comida cozida no vapor}
  \definition{v.}{organizar; alinhar; colocar em ordem; posicionar ou organizar em uma determinada ordem; ordenar | ensaiar | ejetar; excluir; dispensar; remover; eliminar | empurrar o obstáculo para fora do caminho}
\end{entry}

\begin{entry}{排除}{pai2chu2}{11,9}{⼿、⾩}[HSK 5]
  \definition{v.}{remover; superar; excluir; eliminar; livrar-se de}
\end{entry}

\begin{entry}{排队}{pai2dui4}{11,4}{⼿、⾩}[HSK 2]
  \definition{v.+compl.}{formar uma fila; alinhar-se; enfileirar-se; organizar em sequência | listar; classificar}
\end{entry}

\begin{entry}{排列}{pai2lie4}{11,6}{⼿、⼑}[HSK 4]
  \definition{v.}{classificar; colocar; variar; organizar; pôr em ordem}
\end{entry}

\begin{entry}{排名}{pai2 ming2}{11,6}{⼿、⼝}[HSK 3]
  \definition{s.}{classificação; resultado}
\end{entry}

\begin{entry}{排球}{pai2 qiu2}{11,11}{⼿、⽟}[HSK 2]
  \definition[场,只,个]{s.}{voleibol; bola de voleibol}
\end{entry}

\begin{entry}{排水}{pai2shui3}{11,4}{⼿、⽔}
  \definition{v.}{drenar}
\end{entry}

\begin{entry}{牌}{pai2}{12}{⽚}[HSK 4]
  \definition[块]{s.}{placa; tabuleta; quadro; placar | marca; marca registrada; marca comercial | cartas, dominó, etc. | a tonalidade de uma música}
\end{entry}

\begin{entry}{牌子}{pai2 zi5}{12,3}{⽚、⼦}[HSK 3]
  \definition[个,种,块]{s.}{sinal; placa | marca; marca registrada}
\end{entry}

\begin{entry}{派}{pai4}{9}{⽔}[HSK 3]
  \definition{adj.}{elegante; bonito}
  \definition{clas.}{para grupos, escolas de pensamento ou arte, etc. | para um discursos, atmosferas, cenas, etc.}
  \definition{s.}{panelinha; grupo exclusivo; facção | torta | estilo | afluente; braço de rio}
  \definition{v.}{enviar; despachar | alocar; repartir; distribuir}
\end{entry}

\begin{entry}{攀爬}{pan1pa2}{19,8}{⼿、⽖}
  \definition{v.}{escalar}
\end{entry}

\begin{entry}{攀岩}{pan1yan2}{19,8}{⼿、⼭}
  \definition{s.}{alpinista}
  \definition{v.}{escalar uma montanha}
\end{entry}

\begin{entry}{爿}{pan2}{4}{⽙}[Kangxi 90]
  \definition{clas.}{para faixas de terra ou bambu, lojas, fábricas etc.}
\end{entry}

\begin{entry}{胖}{pan2}{9}{⾁}
  \definition{adj.}{saudável}
  \seeref{胖}{pang4}
\end{entry}

\begin{entry}{般}{pan2}{10}{⾈}
  \definition{s.}{utilizado em 般乐}
  \seealsoref{般乐}{pan2le4}
\end{entry}

\begin{entry}{般乐}{pan2le4}{10,5}{⾈、⼃}
  \definition{v.}{jogar | divertir-se}
\end{entry}

\begin{entry}{盘}{pan2}{11}{⽫}[HSK 4]
  \definition*{s.}{sobrenome Pan}
  \definition{clas.}{para pratos, pedras de moer, etc. | para jogos de xadrez e de bola | para as coisas que estão entrelaçadas, emaranhadas}
  \definition{s.}{bandeja; tabuleiro | recipiente plano e raso, como uma bandeja, prato, travessa etc.  | preço atual; cotação de mercado; refere-se ao preço básico pelo qual as commodities são negociadas}
  \definition{v.}{enrolar; torcer; enrolar (para cima); entrelaçar; cercar | construir (assentando tijolos, pedras, etc.) | checar; examinar; interrogar; verificar um por um ou repetidamente (quantidade, situação, etc.) | transferir a propriedade de; passar para outra pessoa | carregar; transportar}
\end{entry}

\begin{entry}{盘子}{pan2zi5}{11,3}{⽫、⼦}[HSK 4]
  \definition[个,叠,套,只]{s.}{prato; utensílio de fundo raso para guardar objetos, maior do que um pires, geralmente de formato redondo | situação de mercado; taxa de mercado; transação comercial}
\end{entry}

\begin{entry}{槃}{pan2}{14}{⽊}
  \variantof{盘}
\end{entry}

\begin{entry}{判断}{pan4duan4}{7,11}{⼑、⽄}[HSK 3]
  \definition[个]{s.}{julgamento}
  \definition{v.}{julgar; decidir}
\end{entry}

\begin{entry}{旁}{pang2}{10}{⽅}[HSK 5]
  \definition{adj.}{outro | abundante; abrangente}
  \definition{s.}{lado | radical lateral de um caractere chinês}
\end{entry}

\begin{entry}{旁边}{pang2bian1}{10,5}{⽅、⾡}[HSK 1]
  \definition{s.}{junto a; próximo de; ao lado}
\end{entry}

\begin{entry}{胖}{pang4}{9}{⾁}[HSK 3]
  \definition{adj.}{gordo; robusto; rechonchudo}
  \seeref{胖}{pan2}
\end{entry}

\begin{entry}{胖子}{pang4 zi5}{9,3}{⾁、⼦}[HSK 4]
  \definition{s.}{obeso; gordo; pessoa gorda}
\end{entry}

\begin{entry}{泡}{pao1}{8}{⽔}
  \definition{adj.}{estufado | inchado | esponjoso}
  \definition{clas.}{para urina ou fezes}
  \definition{s.}{pequeno lago (especialmente em nomes de lugares)}
  \seeref{泡}{pao4}
\end{entry}

\begin{entry}{跑}{pao2}{12}{⾜}
  \definition{v.}{(de animais) bater com a pata (no chão); (de animais) escavar o solo com suas garras ou cascos}
  \seeref{跑}{pao3}
\end{entry}

\begin{entry}{跑}{pao3}{12}{⾜}[HSK 1]
  \definition{v.}{correr; pessoas ou animais que se movem rapidamente para a frente com as pernas e os pés | caminhar; passear | fugir; escapar | correr de um lado para outro; fazer rondas; correr atrás de algo | de um líquido ou gás) vazar; evaporar | (como complemento de um verbo) fora; longe | participar de uma corrida}
  \seeref{跑}{pao2}
\end{entry}

\begin{entry}{跑步}{pao3bu4}{12,7}{⾜、⽌}[HSK 3]
  \definition{s.}{corrida}
  \definition{v.+compl.}{correr; trotar}
\end{entry}

\begin{entry}{跑调}{pao3diao4}{12,10}{⾜、⾔}
  \definition{v.}{(coloquial) estar fora do tom ou desafinado (enquanto canta)}
\end{entry}

\begin{entry}{跑掉}{pao3diao4}{12,11}{⾜、⼿}
  \definition{v.}{fugir}
\end{entry}

\begin{entry}{跑肚}{pao3du4}{12,7}{⾜、⾁}
  \definition{v.}{(coloquial) ter diarréia}
\end{entry}

\begin{entry}{跑酷}{pao3ku4}{12,14}{⾜、⾣}
  \definition*{s.}{(empréstimo linguístico) \emph{Parkour}}
\end{entry}

\begin{entry}{跑马}{pao3ma3}{12,3}{⾜、⾺}
  \definition{s.}{corrida de cavalos}
  \definition{v.}{andar a cavalo em ritmo acelerado}
\end{entry}

\begin{entry}{跑题}{pao3ti2}{12,15}{⾜、⾴}
  \definition{v.}{divagar | fugir do assunto | tergiversar}
\end{entry}

\begin{entry}{跑腿}{pao3tui3}{12,13}{⾜、⾁}
  \definition{v.}{realizar tarefas}
\end{entry}

\begin{entry}{泡}{pao4}{8}{⽔}
  \definition{clas.}{para ocorrências de uma ação | para número de infusões}
  \definition{s.}{bolha | espuma}
  \definition{v.}{encharcar | infundir | pegar (uma garota) | sair com (um parceiro sexual)}
  \seeref{泡}{pao1}
\end{entry}

\begin{entry}{胚}{pei1}{9}{⾁}
  \definition{s.}{embrião}
\end{entry}

\begin{entry}{陪}{pei2}{10}{⾩}[HSK 5]
  \definition{v.}{servir; acompanhar; cuidar; fazer companhia a alguém | auxiliar; ajudar}
\end{entry}

\begin{entry}{培训}{pei2xun4}{11,5}{⼟、⾔}[HSK 4]
  \definition{v.}{treinar (trabalhadores técnicos, quadros profissionais, etc.)}
\end{entry}

\begin{entry}{培训班}{pei2 xun4 ban1}{11,5,10}{⼟、⾔、⽟}[HSK 4]
  \definition{s.}{aula de treinamento; curso de treinamento}
\end{entry}

\begin{entry}{培养}{pei2yang3}{11,9}{⼟、⼋}[HSK 4]
  \definition{v.}{cultivar (plantas, microorganismos) | promover; treinar ou desenvolver; educar e treinar para um determinado propósito durante um longo período de tempo; fazer crescer | progredir gradualmente; desenvolver ou cultivar gradualmente (hábito, qualidade, caráter, emoção, estilo, interesse, habilidade, etc.)}
\end{entry}

\begin{entry}{培育}{pei2yu4}{11,8}{⼟、⾁}[HSK 4]
  \definition{v.}{criar; fomentar; educar; procriar; nutrir; cultivar}
\end{entry}

\begin{entry}{赔}{pei2}{12}{⾙}[HSK 5]
  \definition{v.}{compensar; pagar por; indenizar | sofrer uma perda; fazer negócios e perder dinheiro}
\end{entry}

\begin{entry}{赔偿}{pei2chang2}{12,11}{⾙、⼈}[HSK 5]
  \definition{v.}{indenizar; compensar; pagar por; indenizar outras pessoas ou grupos por perdas causadas por suas próprias ações}
\end{entry}

\begin{entry}{赔钱}{pei2qian2}{12,10}{⾙、⾦}
  \definition{v.+compl.}{perder dinheiro | compensar; compensar com dinheiro os prejuízos causados a terceiros}
\end{entry}

\begin{entry}{佩服}{pei4fu2}{8,8}{⼈、⽉}
  \definition{v.}{admirar}
\end{entry}

\begin{entry}{配}{pei4}{10}{⾣}[HSK 3]
  \definition{s.}{esposa}
  \definition{v.}{unir-se em matrimônio | acasalar (animais) | compor; combinar; mesclar; amalgamar; misturar |distribuir de acordo com o plano; repartir | encontrar algo para encaixar ou substituir outra coisa | corresponder; combinar; equiparar | merecer; ser digno de; ser qualificado}
\end{entry}

\begin{entry}{配备}{pei4bei4}{10,8}{⾣、⼡}[HSK 5]
  \definition{s.}{equipamento; material; conjunto completo de utensílios, etc.}
  \definition{v.}{fornecer; alocar; equipar; distribuir conforme necessário | posicionar; dispor (tropas, etc.)}
\end{entry}

\begin{entry}{配合}{pei4he2}{10,6}{⾣、⼝}[HSK 3]
  \definition{s.}{coordenação}
  \definition{v.}{cooperar; coordenar}
\end{entry}

\begin{entry}{配套}{pei4tao4}{10,10}{⾣、⼤}[HSK 5]
  \definition{v.+compl.}{formar um conjunto ou sistema completo; combinar vários elementos relacionados em um conjunto completo}
\end{entry}

\begin{entry}{喷}{pen1}{12}{⼝}[HSK 5]
  \definition{v.}{jorrar; esguichar; expelir sob pressão | borrifar; espalhar; pulverizar}
  \seeref{喷}{pen4}
\end{entry}

\begin{entry}{盆}{pen2}{9}{⽫}[HSK 5]
  \definition*{s.}{sobrenome Pen}
  \definition[个]{s.}{bacia; banheira; panela; utensílios para guardar ou lavar coisas}
\end{entry}

\begin{entry}{盆友}{pen2you3}{9,4}{⽫、⼜}
  \definition{s.}{(gíria na \emph{Internet}) amigo (trocadilho com 朋友)}
  \seealsoref{朋友}{peng2you5}
\end{entry}

\begin{entry}{喷}{pen4}{12}{⼝}
  \definition{s.}{na época; tempo no mercado; época em que frutas, peixes e camarões são comercializados em grande quantidade | colheita; número de vezes que floresceu e frutificou; número de vezes que foi colhido na maturação}
  \seeref{喷}{pen1}
\end{entry}

\begin{entry}{朋友}{peng2you5}{8,4}{⽉、⼜}[HSK 1]
  \definition[个,位,帮,群]{s.}{amigo; pessoas que têm um bom relacionamento, uma boa relação, se entendem e se ajudam mutuamente | namorado; namorada}
\end{entry}

\begin{entry}{膨胀}{peng2zhang4}{16,8}{⾁、⾁}
  \definition{v.}{expandir | inflar | inchar}
\end{entry}

\begin{entry}{碰}{peng4}{13}{⽯}[HSK 2]
  \definition{v.}{tocar; bater; esbarrar | encontrar; esbarrar | arriscar; tentar | tentar a sorte | reunir-se para discutir; ter uma reunião curta}
\end{entry}

\begin{entry}{碰到}{peng4 dao4}{13,8}{⽯、⼑}[HSK 2]
  \definition{v.}{encontrar (com); esbarrar; cruzar}
\end{entry}

\begin{entry}{碰见}{peng4 jian4}{13,4}{⽯、⾒}[HSK 2]
  \definition{v.}{encontrar; encontrar-se; sem combinar, encontrar-se por acaso}
\end{entry}

\begin{entry}{碰头}{peng4tou2}{13,5}{⽯、⼤}
  \definition{s.}{colisão | conflito}
  \definition{v.}{colidir}
  \definition{v.+compl.}{conhecer e discutir | juntar ideias | ver-se}
\end{entry}

\begin{entry}{碰运气}{peng4yun4qi5}{13,7,4}{⽯、⾡、⽓}
  \definition{v.}{deixar algo ao acaso | tentar a sorte}
\end{entry}

\begin{entry}{批}{pi1}{7}{⼿}[HSK 4]
  \definition{adj.}{(compra ou venda) atacado; a granel; em grandes quantidades}
  \definition{clas.}{para mercadorias a granel, grande número de pessoas}
  \definition{s.}{fibras de algodão, linho, etc., prontas para serem estiradas e torcidas | anotação; comentário}
  \definition{v.}{escrever comentários ou críticas sobre documentos subordinados, textos de outras pessoas, tarefas etc. | refutar; criticar | dar um tapa}
\end{entry}

\begin{entry}{批评}{pi1ping2}{7,7}{⼿、⾔}[HSK 3]
  \definition{s.}{crítica}
  \definition{v.}{criticar; comentar sobre}
\end{entry}

\begin{entry}{批准}{pi1zhun3}{7,10}{⼿、⼎}[HSK 3]
  \definition{v.}{aprovar}
\end{entry}

\begin{entry}{披}{pi1}{8}{⼿}[HSK 5]
  \definition{v.}{colocar sobre os ombros; enrolar em volta; cobrir ou colocar sobre os ombros | abrir; desenrolar; espalhar | abrir-se; rachar}
\end{entry}

\begin{entry}{皮}{pi2}{5}{⽪}[HSK 3][Kangxi 107]
  \definition*{s.}{sobrenome Pi}
  \definition{adj.}{macios e encharcados; não mais crocantes | danadinho; travesso | endurecido; não se importa mais}
  \definition{pref.}{pico- (um trilhonésimo)}
  \definition[张]{s.}{pele | couro cru; couro | pelagem | capa; envoltório | superfície | uma peça larga e plana (de algum material fino) | borracha}
\end{entry}

\begin{entry}{皮包}{pi2 bao1}{5,5}{⽪、⼓}[HSK 3]
  \definition[个,只,款]{s.}{bolsa; pasta; portfólio}
\end{entry}

\begin{entry}{皮肤}{pi2fu1}{5,8}{⽪、⾁}[HSK 5]
  \definition{adj.}{superficial}
  \definition[层,块]{s.}{pele; couro; derme}
\end{entry}

\begin{entry}{皮卡}{pi2ka3}{5,5}{⽪、⼘}
  \definition{s.}{(empréstimo linguístico) \emph{pick-up} | caminhonete}
\end{entry}

\begin{entry}{皮卡丘}{pi2ka3qiu1}{5,5,5}{⽪、⼘、⼀}
  \definition*{s.}{\emph{Pikachu} (Pokémon, 口袋妖怪)}
  \seealsoref{口袋妖怪}{kou3dai4 yao1guai4}
\end{entry}

\begin{entry}{皮下}{pi2xia4}{5,3}{⽪、⼀}
  \definition{adj.}{(injeção) subcutâneo | sob a pele}
\end{entry}

\begin{entry}{皮鞋}{pi2xie2}{5,15}{⽪、⾰}[HSK 5]
  \definition[双,只,款]{s.}{sapatos feitos de couro}
\end{entry}

\begin{entry}{啤酒}{pi2jiu3}{11,10}{⼝、⾣}[HSK 3]
  \definition[杯,瓶,罐,桶,缸]{s.}{(empréstimo linguístico) cerveja}
\end{entry}

\begin{entry}{啤酒馆}{pi2jiu3guan3}{11,10,11}{⼝、⾣、⾷}
  \definition{s.}{cervejaria}
\end{entry}

\begin{entry}{脾气}{pi2qi5}{12,4}{⾁、⽓}[HSK 5]
  \definition[发,个]{s.}{temperamento; disposição; referindo-se ao caráter de uma pessoa | mau humor; temperamento irascível}
\end{entry}

\begin{entry}{匹}{pi3}{4}{⼖}[HSK 5]
  \definition{adj.}{solitário}
  \definition{clas.}{para cavalos, mulas, etc. | para rolos inteiros de seda ou tecido}
  \definition{v.}{ser igual a; ser compatível com}
\end{entry}

\begin{entry}{屁股}{pi4gu5}{7,8}{⼫、⾁}
  \definition{s.}{nádega | quadris}
\end{entry}

\begin{entry}{屁话}{pi4hua4}{7,8}{⼫、⾔}
  \definition{s.}{absurdo | tolice | besteira}
\end{entry}

\begin{entry}{譬如}{pi4ru2}{20,6}{⾔、⼥}
  \definition{conj.}{por exemplo | como}
\end{entry}

\begin{entry}{片}{pian1}{4}{⽚}
  \definition{s.}{película; filme; refere-se a filmes com imagens, paisagens ou imagens gravadas com som}
  \seeref{片}{pian4}
\end{entry}

\begin{entry}{片儿}{pian1r5}{4,2}{⽚、⼉}
  \definition{s.}{folha | película; filme}
\end{entry}

\begin{entry}{偏偏}{pian1pian1}{11,11}{⼈、⼈}
  \definition{adv.}{voluntariamente | insistentemente | persistentemente | ao contrário da expectativa | infelizmente (indicando que alguma coisa aconteceu ao contrário do que se esperava) | teimosamente (indicando que algo é o oposto ao que seria normal ou razoável) | precisamente (indicando que alguém ou um grupo é escolhido)}
\end{entry}

\begin{entry}{篇}{pian1}{15}{⽵}[HSK 2]
  \definition*{s.}{sobrenome Pian}
  \definition{clas.}{usado para folhas de papel, páginas de livros, artigos, etc.}
  \definition{s.}{um pedaço de escrita | folha (de papel, etc.) | (para papel, folhas de livros, artigos, etc.) folha; página; pedaço}
\end{entry}

\begin{entry}{便}{pian2}{9}{⼈}
  \definition*{s.}{sobrenome Pian}
  \definition{adj.}{silencioso e confortável}
  \seeref{便}{bian4}
\end{entry}

\begin{entry}{便宜}{pian2yi5}{9,8}{⼈、⼧}[HSK 2]
  \definition{adj.}{barato; acessível}
  \definition[个,份,件]{s.}{vantagem em algum aspecto | ganho; lucro; vantagem; benefício indevido}
  \definition{v.}{deixar alguém escapar impune; obter algum benefício}
  \seeref{便宜}{bian4yi2}
\end{entry}

\begin{entry}{片}{pian4}{4}{⽚}[HSK 2][Kangxi 91]
  \definition*{s.}{sobrenome Pian}
  \definition{adj.}{breve; parcial; incompleto; fragmentário; esporádico; breve | unilateral}
  \definition{clas.}{usado para coisas em forma de lâminas | usado para terrenos ou superfícies aquáticas com a mesma paisagem e que estão conectados entre si | usado para paisagens, clima, sons, linguagem, intenções, etc. (usado em conjunto com o numeral 一)}
  \definition{s.}{plano, fatia; floco; pedaço fino; algo plano e fino | seção; parte de uma grande área; uma pequena parte do todo ou uma área menor dividida dentro de uma área maior | filme; peça de TV; referência ao filme}
  \definition{v.}{fatiar; cortar em fatias; cortar em fatias finas com uma faca | abrir; cortar; separar}
  \seeref{片}{pian1}
  \seealsoref{一}{yi1}
\end{entry}

\begin{entry}{片面}{pian4mian4}{4,9}{⽚、⾯}[HSK 4]
  \definition{adj.}{unilateral (em oposição a 全面)}
  \seealsoref{全面}{quan2mian4}
\end{entry}

\begin{entry}{骗}{pian4}{12}{⾺}[HSK 5]
  \definition{v.}{enganar; trapacear; iludir; ludibriar; usar mentiras ou meios fraudulentos para fazer alguém acreditar ou ser enganado | enganar; fraudar | montar (um cavalo); balançar (ou saltar) para a sela}
\end{entry}

\begin{entry}{骗子}{pian4 zi5}{12,3}{⾺、⼦}[HSK 5]
  \definition[个]{s.}{trapaceiro; vigarista; fraudador; impostor; golpista; pessoa que obtém bens de forma fraudulenta}
\end{entry}

\begin{entry}{漂}{piao1}{14}{⽔}
  \definition{v.}{flutuar | estar a deriva}
  \seeref{漂}{piao3}
  \seeref{漂}{piao4}
\end{entry}

\begin{entry}{漂流}{piao1liu2}{14,10}{⽔、⽔}
  \definition{s.}{\emph{rafting}}
  \definition{v.}{ser levado pela correnteza | flutuar ao longo ou sobre}
\end{entry}

\begin{entry}{飘}{piao1}{15}{⾵}
  \definition{adj.}{complacente | frívolo | fraco | instável | bambo | cambaleante}
  \definition{v.}{flutuar (no ar) | esvoaçar | tremular}
\end{entry}

\begin{entry}{漂}{piao3}{14}{⽔}
  \definition{v.}{alvejar | branquear}
  \seeref{漂}{piao1}
  \seeref{漂}{piao4}
\end{entry}

\begin{entry}{票}{piao4}{11}{⽰}[HSK 1]
  \definition{clas.}{para grupos, lotes, transações comerciais}
  \definition[张]{s.}{bilhete; passagem; ingresso | cédula | nota bancária; conta | pessoa mantida em cativeiro por sequestradores para obter resgate; refém | apresentação amadora (de ópera de Pequim, etc.); peças teatrais amadoras}
  \definition{v.}{atuar como amador (na ópera de Pequim)}
\end{entry}

\begin{entry}{票价}{piao4 jia4}{11,6}{⽰、⼈}[HSK 3]
  \definition[个]{s.}{o preço de um bilhete; taxa de admissão; taxa de entrada}
\end{entry}

\begin{entry}{漂}{piao4}{14}{⽔}
  \definition{adj.}{usado em 漂亮}
  \seeref{漂}{piao1}
  \seeref{漂}{piao3}
  \seealsoref{漂亮}{piao4liang5}
\end{entry}

\begin{entry}{漂亮}{piao4liang5}{14,9}{⽔、⼇}[HSK 2]
  \definition{adj.}{bonito; lindo; atraente; de boa aparência; esteticamente agradável | excelente; notável | não pode ser utilizado para descrever homens}
\end{entry}

\begin{entry}{拼}{pin1}{9}{⼿}[HSK 5]
  \definition{v.}{montar; juntar as peças | dar tudo de si no trabalho; estar disposto a arriscar a vida (em lutas, no trabalho, etc.); fazer tudo o que for preciso; arriscar tudo}
\end{entry}

\begin{entry}{拼命}{pin1ming4}{9,8}{⼿、⼝}
  \definition{adv.}{com toda a força | desesperadamente}
  \definition{v.+compl.}{arriscar a vida de alguém | desafiar a morte | colocar-se em uma luta desesperada | fazer algo desesperadamente | exercer a maior força}
\end{entry}

\begin{entry}{拼音}{pin1yin1}{9,9}{⼿、⾳}
  \definition{s.}{escrita fonética | pinyin (romanização chinesa)}
\end{entry}

\begin{entry}{贫}{pin2}{8}{⾙}
  \definition{adj.}{pobre; empobrecido | inadequado; deficiente; insuficiente | tagarela; loquaz; falante; chato e irritante}
\end{entry}

\begin{entry}{贫民窟}{pin2min2ku1}{8,5,13}{⾙、⽒、⽳}
  \definition{s.}{favela}
\end{entry}

\begin{entry}{频道}{pin2dao4}{13,12}{⾴、⾡}[HSK 5]
  \definition[个]{s.}{canal; canal de frequência; televisão e rádio, os sinais de som e imagem ocupam um determinado canal de frequência}
\end{entry}

\begin{entry}{频繁}{pin2fan2}{13,17}{⾴、⽷}[HSK 5]
  \definition{adj.}{frequentemente}
  \definition{adj.}{frequente}
\end{entry}

\begin{entry}{品}{pin3}{9}{⼝}[HSK 5]
  \definition*{s.}{sobrenome Pin}
  \definition{s.}{artigo; produto | grau; classe; classificação; nível | caráter; qualidade | classificação; os graus dos funcionários públicos antigos, num total de nove graus}
  \definition{v.}{provar; saborear; degustar algo com discernimento | soprar; tocar (instrumentos de sopro) | avaliar; distinguir o bom do ruim}
\end{entry}

\begin{entry}{品德}{pin3de2}{9,15}{⼝、⼻}
  \definition{s.}{caráter moral | moralidade}
\end{entry}

\begin{entry}{品质}{pin3zhi4}{9,8}{⼝、⾙}[HSK 4]
  \definition[个,种]{s.}{qualidade; caráter; natureza do pensamento, da compreensão, do caráter, etc., conforme expresso no comportamento, no estilo, etc. | qualidade (de produtos, mercadorias, etc.)}
\end{entry}

\begin{entry}{品种}{pin3zhong3}{9,9}{⼝、⽲}[HSK 5]
  \definition[个]{s.}{raça; linhagem; variedade; refere-se a um grupo de organismos com características genéticas comuns, formados por meio da seleção e cultivo artificiais de culturas, gado, aves, etc. | variedade; sortimento; referência geral ao tipo de item}
\end{entry}

\begin{entry}{乒乓球}{ping1pang1qiu2}{6,6,11}{⼃、⼃、⽟}
  \definition[个]{s.}{tênis de mesa |ping-pong}
\end{entry}

\begin{entry}{平}{ping2}{5}{⼲}[HSK 2]
  \definition*{s.}{sobrenome Ping}
  \definition{adj.}{plano; nivelado; uniforme; liso | igual; justo | mesma pontuação; empatado | médio; comum | silencioso; tranquilo | no mesmo nível; altura igual; sem diferença | imparcial; médio; equitativo | calmo; estável; tranquilo | comum;  frequente}
  \definition{s.}{no mesmo nível; em pé de igualdade com; igual | tom nivelado, um dos quatro tons do chinês clássico}
  \definition{v.}{tornar nivelado ou uniforme; nivelar | reprimir; suprimir | acalmar; tornar pacífico; silenciar (acalmar); conter a raiva | estar no mesmo nível | acalmar; amenizar; controlar a raiva}
\end{entry}

\begin{entry}{平安}{ping2'an1}{5,6}{⼲、⼧}[HSK 2]
  \definition{s.}{seguro; bem; sem contratempos; sem acidentes; são e salvo}
\end{entry}

\begin{entry}{平常}{ping2chang2}{5,11}{⼲、⼱}[HSK 2]
  \definition{adj.}{comum; normal; ordinário; nada de especial}
  \definition{adv.}{normalmente; geralmente; como regra geral}
\end{entry}

\begin{entry}{平等}{ping2deng3}{5,12}{⼲、⽵}[HSK 2]
  \definition{adj.}{igual; igualdade; refere-se ao fato de as pessoas gozarem de tratamento igualitário nos aspectos sociais, políticos, econômicos e jurídicos}
\end{entry}

\begin{entry}{平地}{ping2di4}{5,6}{⼲、⼟}
  \definition{v.}{nivelar a terra | aplanar}
\end{entry}

\begin{entry}{平方}{ping2fang1}{5,4}{⼲、⽅}[HSK 4]
  \definition{s.}{quadrado}
\end{entry}

\begin{entry}{平方米}{ping2fang1 mi3}{5,4,6}{⼲、⽅、⽶}
  \definition{clas.}{unidade de medida de área, 1 metro quadrado equivale a 10.000 centímetros quadrados}
\end{entry}

\begin{entry}{平静}{ping2jing4}{5,14}{⼲、⾭}[HSK 4]
  \definition{adj.}{(humor, ambiente, etc.) calmo; quieto; pacífico; tranquilo}
\end{entry}

\begin{entry}{平均}{ping2jun1}{5,7}{⼲、⼟}[HSK 4]
  \definition{adj.}{igual; médio}
  \definition{s.}{média}
  \definition{v.}{calcular a média de um conjunto de números}
\end{entry}

\begin{entry}{平时}{ping2shi2}{5,7}{⼲、⽇}[HSK 2]
  \definition{s.}{em tempos normais; em tempos comuns | em tempo de paz; refere-se a períodos normais}
\end{entry}

\begin{entry}{平台}{ping2tai2}{5,5}{⼲、⼝}
  \definition{s.}{plataforma | terraço | edifício de telhado plano}
\end{entry}

\begin{entry}{平坦}{ping2tan3}{5,8}{⼲、⼟}[HSK 5]
  \definition{adj.}{plano; uniforme; nivelado; liso; sem elevações ou depressões (referindo-se principalmente ao relevo)}
\end{entry}

\begin{entry}{平稳}{ping2 wen3}{5,14}{⼲、⽲}[HSK 4]
  \definition{adj.}{firme; estável; suave e constante; sem oscilações ou flutuações}
\end{entry}

\begin{entry}{平原}{ping2yuan2}{5,10}{⼲、⼚}[HSK 5]
  \definition[片]{s.}{campo; planície; terreno plano e extenso}
\end{entry}

\begin{entry}{评估}{ping2gu1}{7,7}{⾔、⼈}[HSK 5]
  \definition{v.}{estimar; avaliar; apreciar; avaliar e estimar (coisas abstratas)}
\end{entry}

\begin{entry}{评价}{ping2jia4}{7,6}{⾔、⼈}[HSK 3]
  \definition[个,项,条,份]{s.}{avaliação; apreciação}
  \definition{v.}{estimar; avaliar}
\end{entry}

\begin{entry}{评论}{ping2lun4}{7,6}{⾔、⾔}[HSK 5]
  \definition[篇]{s.}{revisão; comentário; artigos ou comentários críticos}
  \definition{v.}{discutir; comentar sobre algo ou alguém}
\end{entry}

\begin{entry}{凭}{ping2}{8}{⼏}[HSK 5]
  \definition{prep.}{introduzir a ação ou o comportamento com base em algo; quando a frase nominal após 凭 é longa, pode-se adicionar 着 após 凭}
  \definition[张]{s.}{prova; evidência}
  \definition{v.}{apoiar-se; encostar-se | confiar em; depender de | basear-se em; tomar como base}
  \seealsoref{着}{zhe5}
\end{entry}

\begin{entry}{苹果}{ping2guo3}{8,8}{⾋、⽊}[HSK 3]
  \definition[个,颗]{s.}{maçã}
\end{entry}

\begin{entry}{瓶}{ping2}{10}{⽡}[HSK 2]
  \definition*{s.}{sobrenome Ping}
  \definition{clas.}{usado para coisas que são engarrafadas; quantidade contida em um frasco, vaso, garrafa}
  \definition[个]{s.}{jarra; vaso; frasco; garrafa;}
\end{entry}

\begin{entry}{瓶盖}{ping2gai4}{10,11}{⽡、⽫}
  \definition{s.}{tampa de garrafa}
\end{entry}

\begin{entry}{瓶装}{ping2zhuang1}{10,12}{⽡、⾐}
  \definition{adj.}{engarrafado}
\end{entry}

\begin{entry}{瓶子}{ping2zi5}{10,3}{⽡、⼦}[HSK 2]
  \definition[个,只,种]{s.}{garrafa; recipientes com gargalo feitos de cerâmica, vidro, plástico, etc., geralmente em forma cilíndrica}
\end{entry}

\begin{entry}{甁}{ping2}{12}{⽡}
  \variantof{瓶}
\end{entry}

\begin{entry}{泼}{po1}{8}{⽔}[HSK 5]
  \definition{adj.}{rude e irracional; mal-humorado}
  \definition{v.}{espalhar; salpicar; derramar; derramar ou espalhar o líquido com força para fora |}
\end{entry}

\begin{entry}{颇}{po1}{11}{⽪}
  \definition*{s.}{sobrenome Po}
  \definition{adv.}{muito, bastante (linguagem escrita)}
\end{entry}

\begin{entry}{迫切}{po4qie4}{8,4}{⾡、⼑}[HSK 4]
  \definition{adj.}{urgente; premente; muito ansiosamente, a ponto de ser difícil esperar}
\end{entry}

\begin{entry}{破}{po4}{10}{⽯}[HSK 3]
  \definition{adj.}{quebrado; danificado; rasgado; desgastado | pobre; ruim; insignificante; péssimo; miserável}
  \definition{v.}{estar quebrado; estar danificado | quebrar; avariar; danificar | quebrar; dividir; cortar; cinzelar | trocar (dinheiro) | romper; quebrar (avanço) | livrar-se de; destruir; romper com | derrotar; capturar (uma cidade, etc.) | despender; gastar (dinheiro) | expor a verdade de; desnudar}
\end{entry}

\begin{entry}{破产}{po4chan3}{10,6}{⽯、⼇}[HSK 4]
  \definition{v.+compl.}{falir; ir à falência; tornar-se insolvente; entrar em liquidação; perder todo o patrimônio | falhar; fracassar; não dar em nada; figura de linguagem (geralmente com uma conotação depreciativa)}
\end{entry}

\begin{entry}{破坏}{po4huai4}{10,7}{⽯、⼟}[HSK 3]
  \definition{s.}{destruição | dano}
  \definition{v.}{demolir; naufragar; soçobrar; destruir; obliterar | quebrar; violar (um acordo, regulamento, etc.) | prejudicar; perturbar; sabotar; causar grande dano | reverter; mudar (um sistema social, costume, etc.) completamente ou violentamente | destruir; decompor}
\end{entry}

\begin{entry}{破坏性}{po4huai4xing4}{10,7,8}{⽯、⼟、⼼}
  \definition{adj.}{destrutivo}
  \definition{s.}{poder destrutivo}
\end{entry}

\begin{entry}{扑克}{pu1ke4}{5,7}{⼿、⼗}
  \definition{s.}{(empréstimo linguístico) (jogo) \emph{poker}  | baralho}
\end{entry}

\begin{entry}{铺}{pu1}{12}{⾦}
  \definition{v.}{espalhar | exibir | montar}
  \seeref{铺}{pu4}
\end{entry}

\begin{entry}{铺垫}{pu1dian4}{12,9}{⾦、⼟}
  \definition{s.}{cobre leito | colcha | roupa de cama}
  \definition{v.}{pavimentar}
\end{entry}

\begin{entry}{葡}{pu2}{12}{⾋}
  \definition*{s.}{Portugal, abreviação de 葡萄牙}
  \seealsoref{葡萄牙}{pu2tao2ya2}
\end{entry}

\begin{entry}{葡汉词典}{pu2-han4 ci2dian3}{12,5,7,8}{⾋、⽔、⾔、⼋}
  \definition{s.}{dicionário português-chinês}
  \seealsoref{汉葡词典}{han4-pu2 ci2dian3}
\end{entry}

\begin{entry}{葡萄酒}{pu2 tao2 jiu3}{12,11,10}{⾋、⾋、⾣}[HSK 5]
  \definition[瓶]{s.}{vinho (de uva)}
\end{entry}

\begin{entry}{葡萄牙}{pu2tao2ya2}{12,11,4}{⾋、⾋、⽛}
  \definition{s.}{Portugal}
\end{entry}

\begin{entry}{葡萄牙文}{pu2tao2ya2wen2}{12,11,4,4}{⾋、⾋、⽛、⽂}
  \definition{s.}{português, língua portuguesa}
  \seealsoref{葡文}{pu2wen2}
\end{entry}

\begin{entry}{葡萄牙语}{pu2tao2ya2yu3}{12,11,4,9}{⾋、⾋、⽛、⾔}
  \definition{s.}{português, língua portuguesa}
  \seealsoref{葡语}{pu2yu3}
\end{entry}

\begin{entry}{葡萄}{pu2tao5}{12,11}{⾋、⾋}[HSK 5]
  \definition[棵,串]{s.}{parreira | uva}
\end{entry}

\begin{entry}{葡文}{pu2wen2}{12,4}{⾋、⽂}
  \definition{s.}{português, língua portuguesa}
  \seealsoref{葡萄牙文}{pu2tao2ya2wen2}
\end{entry}

\begin{entry}{葡语}{pu2yu3}{12,9}{⾋、⾔}
  \definition{s.}{português, língua portuguesa}
  \seealsoref{葡萄牙语}{pu2tao2ya2yu3}
\end{entry}

\begin{entry}{普遍}{pu3bian4}{12,12}{⽇、⾡}[HSK 3]
  \definition{adj.}{geral; comum; universal; difundido}
\end{entry}

\begin{entry}{普及}{pu3ji2}{12,3}{⽇、⼃}[HSK 3]
  \definition{adj.}{popular; universal; onipresente}
  \definition{v.}{popularizar; disseminar; espalhar entre o povo}
\end{entry}

\begin{entry}{普通}{pu3 tong1}{12,10}{⽇、⾡}[HSK 2]
  \definition{adj.}{comum; normal; geral; médio; em geral, nada de especial, como a maioria das pessoas ou coisas}
\end{entry}

\begin{entry}{普通话}{pu3tong1hua4}{12,10,8}{⽇、⾡、⾔}[HSK 2]
  \definition*{s.}{Mandarim (literalmente "linguagem comum") | Putonghua (fala comum da língua chinesa) | Língua oficial da China}
\end{entry}

\begin{entry}{铺}{pu4}{12}{⾦}
  \definition{s.}{cama de tábua | lugar para dormir | loja | depósito}
  \seeref{铺}{pu1}
\end{entry}

\begin{entry}{瀑布}{pu4bu4}{18,5}{⽔、⼱}
  \definition{s.}{queda de água | cachoeira | cascata | catarata}
\end{entry}

%%%%% EOF %%%%%


%%%
%%% Q
%%%
%\section*{Q}
\addcontentsline{toc}{section}{Q}

\begin{verbete}{七}{qi1}{2}
  \significado{num.}{sete, 7}
\end{verbete}

\begin{verbete}{七夕}{qi1xi1}{2;3}
  \significado*{s.}{Dia dos Namorados Chinês, quando o vaqueiro e a tecelã (牛郎织女) têm permissão para se reunirem anualmente; Festival das Meninas; Festival Duplo Sete, noite do sétimo mês lunar}
  \veja{牛郎织女}{niu2lang2zhi1nv3}
\end{verbete}

\begin{verbete}{其实}{qi2shi2}{8;8}
  \significado{adv.}{na verdade; de fato}
\end{verbete}

\begin{verbete}{其他}{qi2ta1}{8;5}
  \significado{pron.}{todos os outro(s); o resto}
\end{verbete}

\begin{verbete}{奇怪}{qi2guai4}{8;8}
  \significado{adj.}{estranho}
  \significado{v.}{ficar perplexo; maravilhar-se}
\end{verbete}

\begin{verbete}{奇迹}{qi2ji4}{8;9}
  \significado{adj.}{milagroso}
  \significado{s.}{milagre}
\end{verbete}

\begin{verbete}{骑}{qi2}{11}
  \significado{p.c.}{para cavalos de sela}
  \significado{v.}{andar (cavalo, bicicleta, etc.); sentar-se montado}
\end{verbete}

\begin{verbete}{骑车}{qi2che1}{11;4}
  \significado{v.}{andar de bicicleta; pedalar}
\end{verbete}

\begin{verbete}{旗}{qi2}{14}
  \significado[面]{s.}{bandeira}
\end{verbete}

\begin{verbete}{企业}{qi3ye4}{6;5}
  \significado[家]{s.}{empresa; corporação; empreendimento; firma}
\end{verbete}

\begin{verbete}{岂有此理}{qi3you3ci3li3}{6;6;6;11}
  \significado{interj.}{Que exorbitante!; Absurdo!; Como isso pode ser assim?; Ridículo!}
\end{verbete}

\begin{verbete}{起床}{qi3chuang2}{10;7}
  \significado{v.+compl.}{sair da cama; levantar-se}
\end{verbete}

\begin{verbete}{起来}{qi3lai5}{10;7}
  \significado{v.+compl.}{levantar-se}
\end{verbete}

\begin{verbete}{起跳}{qi3tiao4}{10;13}
  \significado{v.}{(atletismo) decolar (no início de um salto); (de preço, salário, etc.) começar (de um determinado nível)}
\end{verbete}

\begin{verbete}{气球}{qi4qiu2}{4;11}
  \significado{s.}{balão}
\end{verbete}

\begin{verbete}{气温}{qi4wen1}{4;12}
  \significado[个]{s.}{temperatura do ar}
\end{verbete}

\begin{verbete}{气质}{qi4zhi4}{4;8}
  \significado{s.}{traços de personalidade, temperamento, disposição; aura, ar, sentimento, \emph{vibe}; refinamento, sofisticação, classe}
\end{verbete}

\begin{verbete}{汽车}{qi4che1}{7;4}
  \significado[辆]{s.}{automóvel; carro; veículo motorizado}
\end{verbete}

\begin{verbete}{器}{qi4}{16}
  \significado[台]{s.}{dispositivo; ferramenta; utensílio}
\end{verbete}

\begin{verbete}{恰}{qia4}{9}
  \significado{adv.}{exatamente; apenas}
\end{verbete}

\begin{verbete}{恰到好处}{qia4dao4hao3chu4}{9;8;6;5}
  \significado{expr.}{é simplesmente perfeito; é simplesmente correto}
\end{verbete}

\begin{verbete}{恰好}{qia4hao3}{9;6}
  \significado{adv.}{certo; por sorte; ao que parece; por sorte coincidência}
\end{verbete}

\begin{verbete}{千}{qian1}{3}
  \significado{num.}{mil, 1.000}
\end{verbete}

\begin{verbete}{千古}{qian1gu3}{3;5}
  \significado{adv.}{por toda a eternidade; em todas as idades}
  \significado{s.}{eternidade (usada em um dístico elegíaco, coroa de flores, etc., dedicada aos mortos)}
\end{verbete}

\begin{verbete}{千年}{qian1nian2}{3;6}
  \significado{s.}{milênio}
\end{verbete}

\begin{verbete}{千千万万}{qian1qian1wan4wan4}{3;3;3;3}
  \significado{num.}{inumerável; números incontáveis; milhares e milhares}
\end{verbete}

\begin{verbete}{千万}{qian1wan4}{3;3}
  \significado{adv.}{absolutamente; por todos os meios; (quando usado negativamente) sob nehuma circunstância; nunca; pelo amor de Deus; por favor; não}
\end{verbete}

\begin{verbete}{签}{qian1}{13}
  \significado{s.}{vara de bambu com inscrição (usada em adivinhação, jogos de azar, sorteios, etc.); rótulo; pequena lasca de madeira; etiqueta}
  \significado{v.}{assinar}
\end{verbete}

\begin{verbete}{签名}{qian1ming2}{13;6}
  \significado{s.}{assinatura}
  \significado{v.}{autografar; assinar (o nome com uma caneta, etc.)}
\end{verbete}

\begin{verbete}{前}{qian2}{9}
  \significado{p.l.}{frente; em frente de; A.C. (Antes de~Cristo)}
  \veja{公元}{gong1yuan2}
  \exemplo{前293年}
\end{verbete}

\begin{verbete}{前边}{qian2bian5}{9;5}
  \significado{p.l.}{à frente; da frente}
\end{verbete}

\begin{verbete}{前面}{qian2mian4}{9;9}
  \significado{p.l.}{à frente; da frente}
\end{verbete}

\begin{verbete}{前年}{qian2nian2}{9;6}
  \significado{p.t.}{há dois anos}
\end{verbete}

\begin{verbete}{前天}{qian2tian1}{9;4}
  \significado{p.t.}{anteontem}
\end{verbete}

\begin{verbete}{钱}{qian2}{10}
  \significado*{s.}{sobrenome Qian}
  \significado[笔]{s.}{moeda; dinheiro}
\end{verbete}

\begin{verbete}{钱包}{qian2bao1}{10;5}
  \significado{s.}{carteira; bolsa}
\end{verbete}

\begin{verbete}{潜在}{qian2zai4}{15;6}
  \significado{adj.}{oculto; latente}
  \significado{s.}{potencial}
\end{verbete}

\begin{verbete}{强}{qiang2}{12}
  \significado*{s.}{sobrenome Qiang}
  \significado{adj.}{melhor em sua categoria; melhor; poderoso; forte; vigoroso; violento}
  \veja{强}{jiang4}
  \veja{强}{qiang3}
\end{verbete}

\begin{verbete}{墙}{qiang2}{14}
  \significado[面,堵]{s.}{parede}
  \significado{v.}{(gíria) bloquear (um website)(usado geralmente na voz passiva: 被墙)}
\end{verbete}

\begin{verbete}{墙纸}{qiang2zhi3}{14;7}
  \significado{s.}{papel de parede}
\end{verbete}

\begin{verbete}{抢掠}{qiang3lve4}{7;11}
  \significado{s.}{saque; pilhagem}
  \significado{v.}{saquear; pilhar}
\end{verbete}

\begin{verbete}{强}{qiang3}{12}
  \significado{v.}{obrigar; forçar; fazer um esforço; esforçar-se}
  \veja{强}{jiang4}
  \veja{强}{qiang2}
\end{verbete}

\begin{verbete}{桥}{qiao2}{10}
  \significado[座]{s.}{ponte}
\end{verbete}

\begin{verbete}{瞧}{qiao2}{17}
  \significado{v.}{olhar para; ver; ver (ir a um médico); visitar}
\end{verbete}

\begin{verbete}{巧合}{qiao3he2}{5;6}
  \significado{s.}{coincidência}
  \significado{v.}{coincidir}
\end{verbete}

\begin{verbete}{巧克力}{qiao3ke4li4}{5;7;2}
  \significado[块]{s.}{chocolate (empréstimo linguístico)}
\end{verbete}

\begin{verbete}{切割}{qie1ge1}{4;12}
  \significado{v.}{cortar}
\end{verbete}

\begin{verbete}{茄子}{qie2zi5}{8;3}
  \significado{s.}{berinjela chinesa; ``xis'' fonético (ao ser fotografado), equivale ao ``diga xis''}
\end{verbete}

\begin{verbete}{亲自}{qin1zi4}{9;6}
  \significado{adv.}{pessoalmente; a si mesmo}
\end{verbete}

\begin{verbete}{侵略}{qin1lve4}{9;11}
  \significado{s.}{invasão}
  \significado{v.}{invadir}
\end{verbete}

\begin{verbete}{芹菜}{qin2cai4}{7;11}
  \significado{s.}{salsão}
\end{verbete}

\begin{verbete}{琴键}{qin2jian4}{12;13}
  \significado{s.}{tecla de piano}
\end{verbete}

\begin{verbete}{擒获}{qin2huo4}{15;10}
  \significado{v.}{apreender; capturar}
\end{verbete}

\begin{verbete}{青菜}{qing1cai4}{8;11}
  \significado{s.}{verduras}
\end{verbete}

\begin{verbete}{青春}{qing1chun1}{8;9}
  \significado{s.}{juventude}
\end{verbete}

\begin{verbete}{青椒}{qing1jiao1}{8;12}
  \significado{s.}{pimenta verde}
\end{verbete}

\begin{verbete}{青年节}{qing1nian2jie2}{8;6;5}
  \significado*{s.}{Dia da Juventude (4 de maio)}
\end{verbete}

\begin{verbete}{青天}{qing1tian1}{8;4}
  \significado{s.}{céu claro, limpo ou azul}
\end{verbete}

\begin{verbete}{青铜}{qing1tong2}{8;11}
  \significado{s.}{bronze (liga de cobre, 銅, e estanho, 锡)}
\end{verbete}

\begin{verbete}{青蛙}{qing1wa1}{8;12}
  \significado{adj.}{(gíria velha) cara feio}
  \significado[只]{s.}{sapo;}
\end{verbete}

\begin{verbete}{青玉米}{qing1yu4mi3}{8;5;6}
  \significado{s.}{milho verde}
\end{verbete}

\begin{verbete}{轻松}{qing1song1}{9;8}
  \significado{adj.}{leve; gentil; relaxado; sem esforço; descomplicado}
  \significado{v.}{relaxar; levar as coisas menos a sério}
\end{verbete}

\begin{verbete}{轻易}{qing1yi4}{9;8}
  \significado{adj.}{fácil, simples}
  \significado{adv.}{impulsivamente, abruptamente}
\end{verbete}

\begin{verbete}{倾城}{qing1cheng2}{10;9}
  \significado{adj.}{sedutora (mulher)}
  \significado{adv.}{de todo o lugar; vindo de todos os lugares}
  \significado{v.}{arruinar e derrubar o estado}
\end{verbete}

\begin{verbete}{清}{qing1}{11}
  \significado*{s.}{sobrenome Qing}
  \significado{adj.}{claro; limpo (água, etc.); tranquilo, quieto, puro, não corrompido; distinto}
  \significado{v.}{limpar, resolver (contas)}
\end{verbete}

\begin{verbete}{清唱}{qing1chang4}{11;11}
  \significado{v.}{cantar à capela}
\end{verbete}

\begin{verbete}{清彻}{qing1che4}{11;7}
  \variante{清澈}
\end{verbete}

\begin{verbete}{清澈}{qing1che4}{11;15}
  \significado{adj.}{claro; límpido}
\end{verbete}

\begin{verbete}{清楚}{qing1chu5}{11;13}
  \significado{adj.}{claro; límpido}
  \significado{v.}{ser claro sobre; entender completamente}
\end{verbete}

\begin{verbete}{清理}{qing1li3}{11;11}
  \significado{v.}{limpar; arrumar; descartar}
\end{verbete}

\begin{verbete}{清凉}{qing1liang2}{11;10}
  \significado{adj.}{fresco; refrescante; (roupa) ousada, reveladora}
\end{verbete}

\begin{verbete}{清明节}{qing1ming2jie2}{11;8;5}
  \significado*{s.}{Dia Qingming, Dia dos Finados (uma das 24~divisões do ano solar no calendário lunar chinês:~dia~4 ou 5~de~abril solar)}
\end{verbete}

\begin{verbete}{清爽}{qing1shuang3}{11;11}
  \significado{adj.}{refrescante; relaxado}
\end{verbete}

\begin{verbete}{清晰}{qing1xi1}{11;12}
  \significado{adj.}{claro, distinto}
\end{verbete}

\begin{verbete}{蜻蜓}{qing1ting2}{14;12}
  \significado{s.}{libélula}
\end{verbete}

\begin{verbete}{蜻蝏}{qing1ting2}{14;15}
  \variante{蜻蜓}
\end{verbete}

\begin{verbete}{情感}{qing2gan3}{11;13}
  \significado{s.}{sentimento; emoção}
  \significado{v.}{mover-se (emocionalmente)}
\end{verbete}

\begin{verbete}{情况}{qing2kuang4}{11;7}
  \significado[个,种]{s.}{circunstância; situação; estado das coisas}
\end{verbete}

\begin{verbete}{情绪}{qing2xu4}{11;11}
  \significado[种]{s.}{humor; estado da mente; mau humor}
\end{verbete}

\begin{verbete}{请}{qing3}{10}
  \significado{v.}{por favor (fazer alguma coisa); perguntar; convidar; solicitar}
\end{verbete}

\begin{verbete}{请假条}{qing3jia4tiao2}{10;11;7}
  \significado{s.}{pedido de licença de ausência (do trabalho ou da escola)}
\end{verbete}

\begin{verbete}{请客}{qing3ke4}{10;9}
  \significado{v.+compl.}{entreter os convidados; dar um jantar; convidar para jantar}
\end{verbete}

\begin{verbete}{请求}{qing3qiu2}{10;7}
  \significado[个]{s.}{solicitação}
  \significado{v.}{solicitar; perguntar}
\end{verbete}

\begin{verbete}{请问}{qing3wen4}{10;6}
  \significado{expr.}{Com licença, posso perguntar\dots? (para perguntar por qualquer coisa)}
\end{verbete}

\begin{verbete}{丘陵}{qiu1ling2}{5;10}
  \significado{s.}{colinas}
\end{verbete}

\begin{verbete}{秋}{qiu1}{9}
  \significado*{s.}{sobrenome Qiu}
  \significado{s.}{outono; colheita}
\end{verbete}

\begin{verbete}{秋天}{qiu1tian1}{9;4}
  \significado[个]{p.t./s.}{outono}
\end{verbete}

\begin{verbete}{球}{qiu2}{11}
  \significado[个]{s.}{bola; esfera; globo}
  \significado[场]{s.}{jogo; partida de bola}
\end{verbete}

\begin{verbete}{球迷}{qiu2mi2}{11;9}
  \significado[个]{s.}{fã (esportes de bola)}
\end{verbete}

\begin{verbete}{球拍}{qiu2pai1}{11;8}
  \significado{s.}{raquete}
\end{verbete}

\begin{verbete}{区}{qu1}{4}
  \significado[个]{s.}{área; região; distrito}
  \veja{区}{ou1}
\end{verbete}

\begin{verbete}{区域}{qu1yu4}{4;11}
  \significado{s.}{área; região; distrito}
\end{verbete}

\begin{verbete}{曲棍球}{qu1gun4qiu2}{6;12;11}
  \significado{s.}{hóquei em campo}
\end{verbete}

\begin{verbete}{驱}{qu1}{7}
  \significado{v.}{expulsar; repelir}
\end{verbete}

\begin{verbete}{趋势}{qu1shi4}{12;8}
  \significado{s.}{tendência}
\end{verbete}

\begin{verbete}{取}{qu3}{8}
  \significado{v.}{buscar; obter; escolher}
\end{verbete}

\begin{verbete}{取胜}{qu3sheng4}{8;9}
  \significado{v.}{prevalecer sobre os oponentes; marcar uma vitória}
\end{verbete}

\begin{verbete}{取水}{qu3shui3}{8;4}
  \significado{v.}{obter água (de um poço, etc.)}
\end{verbete}

\begin{verbete}{取现}{qu3xian4}{8;8}
  \significado{v.}{sacar dinheiro}
\end{verbete}

\begin{verbete}{取悦}{qu3yue4}{8;10}
  \significado{v.}{tentar agradar}
\end{verbete}

\begin{verbete}{厺}{qu4}{5}
  \variante{去}
\end{verbete}

\begin{verbete}{去}{qu4}{5}
  \significado{v.}{ir; eufenismo:~morrer}
\end{verbete}

\begin{verbete}{去年}{qu4nian2}{5;6}
  \significado{p.t.}{ano passado}
\end{verbete}

\begin{verbete}{去死}{qu4si3}{5;6}
  \significado{interj.}{Caia morto!; Vá para o Inferno!}
\end{verbete}

\begin{verbete}{圈粉}{quan1fen3}{11;10}
  \significado{s.}{(neologismo, coloquial) ganhar alguém como fã; obter novos fãs}
\end{verbete}

\begin{verbete}{全}{quan2}{6}
  \significado*{s.}{sobrenome Quan}
  \significado{adv.}{completamente; totalmente}
\end{verbete}

\begin{verbete}{全部}{quan2bu4}{6;10}
  \significado{adv.}{todo, todos}
\end{verbete}

\begin{verbete}{全职}{quan2zhi2}{6;11}
  \significado{s.}{período integral; tempo inteiro, \emph{full-time} (trabalho)}
\end{verbete}

\begin{verbete}{拳法}{quan2fa3}{10;8}
  \significado{s.}{boxe; luta}
\end{verbete}

\begin{verbete}{拳王}{quan2wang2}{10;4}
  \significado{s.}{pugilista; boxeador}
\end{verbete}

\begin{verbete}{犬}{quan3}{4}[94]
  \significado{s.}{cachorro}
\end{verbete}

\begin{verbete}{却}{que4}{7}
  \significado{adv.}{mas; contudo; entretanto}
\end{verbete}

\begin{verbete}{却是}{que4shi4}{7;9}
  \significado{conj.}{no entanto; realmente; o fato é\dots; mas isso é\dots}
\end{verbete}

\begin{verbete}{确}{que4}{12}
  \significado{adj.}{autenticado; sólido; firme; real; verdadeiro}
\end{verbete}

\begin{verbete}{确实}{que4shi2}{12;8}
  \significado{adj.}{real; verdadeiro; confiável}
  \significado{adv.}{realmente}
\end{verbete}

\begin{verbete}{裙子}{qun2zi5}{12;3}
  \significado[条]{s.}{saia; vestido}
\end{verbete}

\begin{verbete}{群山}{qun2shan1}{13;3}
  \significado{s.}{montanhas; uma cadeia de colinas}
\end{verbete}

%%%%% EOF %%%%%

%%%
%%% R
%%%
\section*{R}
\addcontentsline{toc}{section}{R}
\begin{multicols*}{2}

\begin{verbete}[ran2hou4]{然后}
\begin{pronuncia}{ran2hou4}
\significado{conj.}{
depois; logo; portanto
}
\end{pronuncia}
\end{verbete}

\begin{verbete}[rang4]{让}
\begin{pronuncia}{rang4}
\significado{v.}{
deixar; permitir
}
\end{pronuncia}
\end{verbete}

\begin{verbete}[re4]{热}
\begin{pronuncia}{re4}
\significado{adj.}{
quente
}
\end{pronuncia}
\end{verbete}

\begin{verbete}[re4nao0]{热闹}
\begin{pronuncia}{re4nao0}
\significado{adj.}{
animado; movimentado
}
\end{pronuncia}
\end{verbete}

\begin{verbete}[ren2]{人}
\begin{pronuncia}{ren2}
\significado[个,位]{n.}{
pessoa; gente
}
\end{pronuncia}
\end{verbete}

\begin{verbete}[ren2kou3]{人口}
\begin{pronuncia}{ren2kou3}
\significado{n.}{
população
}
\end{pronuncia}
\end{verbete}

\begin{verbete}[Ren2min2bi4]{人民币}
\begin{pronuncia}{Ren2min2bi4}
\significado{n.}{
RMB; CYN|
nome da moeda chinesa
}
\end{pronuncia}
\end{verbete}

\begin{verbete}[ren4shi0]{认识}
\begin{pronuncia}{ren4shi0}
\significado{v.}{
conhecer
}
\end{pronuncia}
\end{verbete}

\begin{verbete}[ri4]{日}
\begin{pronuncia}{ri4}
\significado{p.c.}{
dia (mais usado em escrita)
}
\end{pronuncia}
\end{verbete}

\begin{verbete}[Ri4ben3]{日本}
\begin{pronuncia}{Ri4ben3}
\significado{n.}{
Japão
}
\end{pronuncia}
\end{verbete}

\begin{verbete}[rong2yi4]{容易}
\begin{pronuncia}{rong2yi4}
\significado{adj.}{
fácil
}
\end{pronuncia}
\end{verbete}

\begin{verbete}[rou4]{肉}
\begin{pronuncia}{rou4}
\significado{n.}{
carne; polpa de uma fruta
}
\end{pronuncia}
\end{verbete}

\begin{verbete}[ru2guo3]{如果}
\begin{pronuncia}{ru2guo3}
\significado{conj.}{
se; caso; no caso de
}
\end{pronuncia}
\end{verbete}

\begin{verbete}[ru3fang2]{乳房}
\begin{pronuncia}{ru3fang2}
\significado{n.}{
seio; mama
}
\end{pronuncia}
\end{verbete}

\end{multicols*}

%%%
%%% S
%%%

\section*{S}\addcontentsline{toc}{section}{S}

\begin{entry}{撒旦}{sa1dan4}{15,5}{⼿、⽇}
  \definition*{s.}{Satã}
\end{entry}

\begin{entry}{撒旦主义}{sa1dan4 zhu3yi4}{15,5,5,3}{⼿、⽇、⼂、⼂}
  \definition*{s.}{Satanismo}
\end{entry}

\begin{entry}{撒但}{sa1dan4}{15,7}{⼿、⼈}
  \variantof{撒旦}
\end{entry}

\begin{entry}{洒水}{sa3shui3}{9,4}{⽔、⽔}
  \definition{v.}{borrifar}
\end{entry}

\begin{entry}{飒飒}{sa4sa4}{9,9}{⾵、⾵}
  \definition{s.}{o farfalhar | sussurro | murmúrio (do vento nas árvores, o mar, etc.)}
\end{entry}

\begin{entry}{赛}{sai4}{14}{⾙}
  \definition{s.}{competição}
  \definition{v.}{competir | superar | destacar-se}
\end{entry}

\begin{entry}{赛车}{sai4che1}{14,4}{⾙、⾞}
  \definition{s.}{corrida de automóvel | corrida de bicicleta | carro de corrida}
\end{entry}

\begin{entry}{三}{san1}{3}{⼀}[HSK 1]
  \definition*{s.}{sobrenome San}
  \definition{num.}{três; 3}
\end{entry}

\begin{entry}{三角}{san1jiao3}{3,7}{⼀、⾓}
  \definition{s.}{triângulo}
\end{entry}

\begin{entry}{三角恋爱}{san1jiao3lian4'ai4}{3,7,10,10}{⼀、⾓、⼼、⽖}
  \definition{s.}{triângulo amoroso}
\end{entry}

\begin{entry}{三轮车}{san1lun2che1}{3,8,4}{⼀、⾞、⾞}
  \definition{s.}{triciclo}
\end{entry}

\begin{entry}{三明治}{san1ming2zhi4}{3,8,8}{⼀、⽇、⽔}
  \definition{s.}{(empréstimo linguístico) sanduíche}
\end{entry}

\begin{entry}{伞}{san3}{6}{⼈}[HSK 4]
  \definition*{s.}{sobrenome San}
  \definition[把]{s.}{guarda-chuva; proteção contra chuva ou sol | algo que tem o formato de um guarda-chuva}
\end{entry}

\begin{entry}{散}{san3}{12}{⽁}
  \definition{adj.}{disperso; fragmentado; não integrado}
  \definition{s.}{medicamento em forma de pó}
  \definition{v.}{divergir; espalhar-se; separar-se; soltar-se; não se manter unido;  desintegrar}
  \seeref{散}{san4}
\end{entry}

\begin{entry}{散}{san4}{12}{⽁}
  \definition{v.}{quebrar; fragmentar; dispersar | dar; distribuir; disseminar; divulgar | dissipar; deixar sai  | terminar um acordo ou contrato; demitir}
  \seeref{散}{san3}
\end{entry}

\begin{entry}{散步}{san4bu4}{12,7}{⽁、⽌}[HSK 3]
  \definition{v.+compl.}{dar uma volta; passear; dar uma caminhada}
\end{entry}

\begin{entry}{散心}{san4xin1}{12,4}{⽁、⼼}
  \definition{v.+compl.}{aliviar o tédio | desfrutar de uma diversão | estar despreocupado}
\end{entry}

\begin{entry}{丧钟}{sang1zhong1}{8,9}{⼗、⾦}
  \definition{s.}{sentença de morte}
\end{entry}

\begin{entry}{桑}{sang1}{10}{⽊}
  \definition*{s.}{sobrenome Sang}
  \definition{s.}{amoreira}
\end{entry}

\begin{entry}{桑巴舞}{sang1ba1wu3}{10,4,14}{⽊、⼰、⾇}
  \definition{s.}{samba}
\end{entry}

\begin{entry}{桑树}{sang1shu4}{10,9}{⽊、⽊}
  \definition{s.}{amoreira, suas folhas são utilizadas para alimentar bichos-da-seda}
\end{entry}

\begin{entry}{骚乱}{sao1luan4}{12,7}{⾺、⼄}
  \definition{s.}{rebelião | perturbação | tumulto}
  \definition{v.}{criar um distúrbio}
\end{entry}

\begin{entry}{扫}{sao3}{6}{⼿}[HSK 4]
  \definition{v.}{varrer; limpar | passar rapidamente ao longo ou sobre; varrer | juntar tudo}
  \seeref{扫}{sao4}
\end{entry}

\begin{entry}{扫兴}{sao3xing4}{6,6}{⼿、⼋}
  \definition{v.+compl.}{sentir-se decepcionado | entristecer alguém}
\end{entry}

\begin{entry}{嫂子}{sao3zi5}{12,3}{⼥、⼦}
  \definition{s.}{esposa do irmão mais velho}
\end{entry}

\begin{entry}{扫}{sao4}{6}{⼿}
  \seeref{扫}{sao3}
  \seeref{扫帚}{sao4zhou5}
\end{entry}

\begin{entry}{扫帚}{sao4zhou5}{6,8}{⼿、⼱}
  \definition[把]{s.}{vassoura; ferramenta de varredura feita de varas de bambu, etc., maior que uma vassora}
\end{entry}

\begin{entry}{色}{se4}{6}{⾊}[HSK 4][Kangxi 139]
  \definition[种]{s.}{cor | aparência; semblante; expressão | tipo; gênero; descrição | cena; cenário;  paisagem | qualidade (de metais preciosos, mercadorias, etc.) | aparência feminina; beleza feminina}
  \seeref{色}{shai3}
\end{entry}

\begin{entry}{色彩}{se4cai3}{6,11}{⾊、⼺}[HSK 4]
  \definition[种,丝]{s.}{cor; matiz; tonalidade | cor; sabor; característica; metáfora para um determinado estado de espírito ou tendência de pensamento}
\end{entry}

\begin{entry}{色狼}{se4lang2}{6,10}{⾊、⽝}
  \definition*{s.}{Sátiro}
  \definition{adj.}{depravado | tarado}
\end{entry}

\begin{entry}{森林}{sen1lin2}{12,8}{⽊、⽊}[HSK 4]
  \definition[片,座,处]{s.}{floresta; bosque; normalmente, refere-se a uma grande área de árvores em crescimento; na silvicultura, refere-se a um grande número de árvores que crescem em uma área razoavelmente grande de terra, juntamente com os animais e outras plantas}
\end{entry}

\begin{entry}{僧}{seng1}{14}{⼈}
  \definition{s.}{monge Budista, abreviação de 僧伽}
  \seeref{僧伽}{seng1qie2}
\end{entry}

\begin{entry}{僧伽}{seng1qie2}{14,7}{⼈、⼈}
  \definition{s.}{sangha ou sanga (Budismo) | a comunidade monástica | monge}
\end{entry}

\begin{entry}{杀气}{sha1qi4}{6,4}{⽊、⽓}
  \definition{s.}{espírito assassino | aura de morte}
  \definition{v.}{desabafar a raiva de alguém}
\end{entry}

\begin{entry}{沙}{sha1}{7}{⽔}
  \definition*{s.}{sobrenome Sha}
  \definition[粒]{s.}{areia | cascalho | grânulo | pó}
\end{entry}

\begin{entry}{沙发}{sha1fa1}{7,5}{⽔、⼜}[HSK 3]
  \definition[套,组,个,张]{s.}{sofá; divã}
\end{entry}

\begin{entry}{沙漠}{sha1mo4}{7,13}{⽔、⽔}
  \definition[个]{s.}{deserto}
\end{entry}

\begin{entry}{沙特}{sha1te4}{7,10}{⽔、⽜}
  \definition*{s.}{Saudita | abreviação de 沙特阿拉伯}
  \seeref{沙特阿拉伯}{sha1te4 a1la1bo2}
\end{entry}

\begin{entry}{沙特阿拉伯}{sha1te4 a1la1bo2}{7,10,7,8,7}{⽔、⽜、⾩、⼿、⼈}
  \definition*{s.}{Arábia Saudita}
\end{entry}

\begin{entry}{沙鱼}{sha1yu2}{7,8}{⽔、⿂}
  \variantof{鲨鱼}
\end{entry}

\begin{entry}{沙子}{sha1 zi5}{7,3}{⽔、⼦}[HSK 3]
  \definition[粒,把]{s.}{areia; grão | \emph{pellets}; grãos pequenos}
\end{entry}

\begin{entry}{刹}{sha1}{8}{⼑}
  \definition{v.}{frear}
  \seeref{刹}{cha4}
\end{entry}

\begin{entry}{砂}{sha1}{9}{⽯}
  \variantof{沙}
\end{entry}

\begin{entry}{莎莎舞}{sha1sha1wu3}{10,10,14}{⾋、⾋、⾇}
  \definition{s.}{salsa (dança)}
\end{entry}

\begin{entry}{鲨鱼}{sha1yu2}{15,8}{⿂、⿂}
  \definition{s.}{tubarão}
\end{entry}

\begin{entry}{啥}{sha2}{11}{⼝}
  \definition{adv.}{Equivalente a 什么 (dialeto)}
\end{entry}

\begin{entry}{傻瓜}{sha3gua1}{13,5}{⼈、⽠}
  \definition{adj.}{tolo | burro | simplório | idiota}
  \definition{v.}{enganar | iludir | lograr}
\end{entry}

\begin{entry}{傻眼}{sha3yan3}{13,11}{⼈、⽬}
  \definition{adj.}{estupefato | atordoado}
\end{entry}

\begin{entry}{色}{shai3}{6}{⾊}
  \definition[4]{s.}{cor}
  \seeref{色}{se4}
\end{entry}

\begin{entry}{晒}{shai4}{10}{⽇}[HSK 4]
  \definition{v.}{(sol) brilhar sobre | aquecer-se; secar ao sol; colocar algo sob a luz do sol para secar | ignorar (alguém) | mostrar; divulgar o conteúdo de sua vida privada na Internet}
\end{entry}

\begin{entry}{晒干}{shai4gan1}{10,3}{⽇、⼲}
  \definition{v.}{secar ao sol}
\end{entry}

\begin{entry}{山}{shan1}{3}{⼭}[HSK 1][Kangxi 46]
  \definition*{s.}{sobrenome Shan}
  \definition[座]{s.}{montanha | monte | qualquer coisa que se assemelhe a uma montanha}
\end{entry}

\begin{entry}{山顶}{shan1ding3}{3,8}{⼭、⾴}
  \definition{s.}{cume da montanha}
\end{entry}

\begin{entry}{山东}{shan1dong1}{3,5}{⼭、⼀}
  \definition*{s.}{Shandong}
\end{entry}

\begin{entry}{山谷}{shan1gu3}{3,7}{⼭、⾕}
  \definition{s.}{vale | ravina}
\end{entry}

\begin{entry}{山区}{shan1qu1}{3,4}{⼭、⼖}
  \definition[个]{s.}{área montanhosa | montanhas}
\end{entry}

\begin{entry}{山体}{shan1ti3}{3,7}{⼭、⼈}
  \definition{s.}{forma de uma montanha}
\end{entry}

\begin{entry}{山羊}{shan1yang2}{3,6}{⼭、⽺}
  \definition{s.}{cabra | (ginástica) cavalo de salto de pequeno porte}
\end{entry}

\begin{entry}{山寨}{shan1zhai4}{3,14}{⼭、⼧}
  \definition{s.}{fortaleza fortificada da vila | fortaleza da montanha (especialmente de bandidos) | falsificação | imitação | (fig.) pechincha}
\end{entry}

\begin{entry}{闪}{shan3}{5}{⾨}[HSK 4]
  \definition*{s.}{sobrenome Shan}
  \definition{s.}{relâmpago}
  \definition{v.}{esquivar-se; desviar; sair do caminho | torcer; distender | surgir de repente | cintilar; brilhar | deixar para trás; abandonar | (corpo) oscilar dramaticamente}
\end{entry}

\begin{entry}{闪存盘}{shan3cun2pan2}{5,6,11}{⾨、⼦、⽫}
  \definition{s.}{unidade de memória \emph{USB} | cartão de memória}
  \seealsoref{优盘}{you1pan2}
\end{entry}

\begin{entry}{闪电}{shan3dian4}{5,5}{⾨、⽥}[HSK 4]
  \definition[道]{s.}{relâmpago; descargas elétricas entre nuvens ou entre nuvens e o solo}
  \seealsoref{雷电}{lei2dian4}
\end{entry}

\begin{entry}{单}{shan4}{8}{⼗}
  \definition*{s.}{sobrenome Shan}
  \seeref{单}{chan2}
  \seeref{单}{dan1}
\end{entry}

\begin{entry}{扇子}{shan4zi5}{10,3}{⼾、⼦}
  \definition[把]{s.}{leque | abano | abanador}
\end{entry}

\begin{entry}{善良}{shan4liang2}{12,7}{⼝、⾉}[HSK 4]
  \definition{adj.}{de bom coração; bom e honesto; de bom coração e cheio de boa vontade}
\end{entry}

\begin{entry}{善意}{shan4yi4}{12,13}{⼝、⼼}
  \definition{s.}{boa vontade | benevolência | bondade}
\end{entry}

\begin{entry}{善于}{shan4yu2}{12,3}{⼝、⼆}[HSK 4]
  \definition{adv./v.}{ser bom em; ser hábil em}
\end{entry}

\begin{entry}{禅}{shan4}{12}{⽰}
  \definition{v.}{abdicar}
  \seeref{禅}{chan2}
\end{entry}

\begin{entry}{擅自}{shan4zi4}{16,6}{⼿、⾃}
  \definition{adv.}{sem permissão ou autorização | por iniciativa própria}
\end{entry}

\begin{entry}{伤}{shang1}{6}{⼈}[HSK 3]
  \definition*{s.}{sobrenome Shang}
  \definition{s.}{ferida; ferimento}
  \definition{v.}{ferir; machucar | estar angustiado | enjoar de algo; desenvolver aversão a algo. |ser prejudicial a; entravar}
\end{entry}

\begin{entry}{伤害}{shang1hai4}{6,10}{⼈、⼧}[HSK 4]
  \definition{v.}{ferir; prejudicar; machucar; magoar; causar danos físicos ou mentais}
\end{entry}

\begin{entry}{伤心}{shang1xin1}{6,4}{⼈、⼼}[HSK 3]
  \definition{v.+compl.}{estar triste; lamentar; estar com o coração partido}
\end{entry}

\begin{entry}{汤}{shang1}{6}{⽔}
  \definition{s.}{correnteza forte}
  \seeref{汤}{tang1}
\end{entry}

\begin{entry}{商场}{shang1chang3}{11,6}{⼝、⼟}[HSK 1]
  \definition[家]{s.}{mercado | shopping | loja de departamentos | o mundo dos negócios}
\end{entry}

\begin{entry}{商店}{shang1dian4}{11,8}{⼝、⼴}[HSK 1]
  \definition[家,个]{s.}{loja}
\end{entry}

\begin{entry}{商量}{shang1liang5}{11,12}{⼝、⾥}[HSK 2]
  \definition{v.}{consultar | discutir | falar sobre}
\end{entry}

\begin{entry}{商贸}{shang1mao4}{11,9}{⼝、⾙}
  \definition{s.}{comércio}
\end{entry}

\begin{entry}{商品}{shang1pin3}{11,9}{⼝、⼝}[HSK 3]
  \definition[种,个,件,批]{s.}{bens; mercadoria; \emph{merchandising}}
\end{entry}

\begin{entry}{商人}{shang1 ren2}{11,2}{⼝、⼈}[HSK 2]
  \definition[位,名]{s.}{comerciante | mercador | homem de negócios}
\end{entry}

\begin{entry}{商务}{shang1wu4}{11,5}{⼝、⼒}[HSK 4]
  \definition[种,类,项]{s.}{negócios; assuntos de negócios; assuntos comerciais}
\end{entry}

\begin{entry}{商业}{shang1ye4}{11,5}{⼝、⼀}[HSK 3]
  \definition[个]{s.}{barganha; negócio; comércio}
\end{entry}

\begin{entry}{上声}{shang3sheng1}{3,7}{⼀、⼠}
  \definition{s.}{tom descendente e ascendente | terceiro tom no mandarim moderno}
\end{entry}

\begin{entry}{赏}{shang3}{12}{⾙}[HSK 4]
  \definition*{s.}{sobrenome Shang}
  \definition{s.}{recompensa; prêmio}
  \definition{v.}{conceder (outorgar) uma recompensa; recompensar; premiar | admirar; desfrutar; apreciar; valorizar}
\end{entry}

\begin{entry}{赏赐}{shang3ci4}{12,12}{⾙、⾙}
  \definition{s.}{recompensa | prêmio}
  \definition{v.}{recompensar | premiar}
\end{entry}

\begin{entry}{赏心悦目}{shang3xin1yue4mu4}{12,4,10,5}{⾙、⼼、⼼、⽬}
  \definition{expr.}{``Aquece o coração e encanta os olhos.''}
\end{entry}

\begin{entry}{上}{shang4}{3}{⼀}[HSK 1]
  \definition{adv.}{acima | em cima | sobre}
  \definition{v.}{subir | entrar em | frequentar (aula ou universidade)}
\end{entry}

\begin{entry}{上班}{shang4 ban1}{3,10}{⼀、⽟}[HSK 1]
  \definition{v.+compl.}{ir para o trabalho | ir para o emprego | estar de plantão}
\end{entry}

\begin{entry}{上边}{shang4bian5}{3,5}{⼀、⾡}[HSK 1]
  \definition{adv.}{acima de | parte de cima | por cima}
\end{entry}

\begin{entry}{上车}{shang4 che1}{3,4}{⼀、⾞}[HSK 1]
  \definition{v.}{entrar (em ônibus, trem, carro, etc.)}
\end{entry}

\begin{entry}{上次}{shang4 ci4}{3,6}{⼀、⽋}[HSK 1]
  \definition{adv.}{última vez}
\end{entry}

\begin{entry}{上当}{shang4dang4}{3,6}{⼀、⼹}
  \definition{v.+compl.}{ser enganado | morder uma isca | ser manipulado | ser joguete nas mãos de alguém}
\end{entry}

\begin{entry}{上访}{shang4fang3}{3,6}{⼀、⾔}
  \definition{v.}{buscar uma audiência com superiores (especialmente funcionários do governo) para fazer uma petição por algo}
\end{entry}

\begin{entry}{上个月}{shang4 ge4 yue4}{3,3,4}{⼀、⼈、⽉}[HSK 4]
  \definition{s.}{mês passado; refere-se à hora de um mês atrás, ou seja, o mês passado mais próximo da hora atual}
\end{entry}

\begin{entry}{上古}{shang4gu3}{3,5}{⼀、⼝}
  \definition{s.}{o passado distante | tempos antigos | antiguidade}
\end{entry}

\begin{entry}{上海}{shang4hai3}{3,10}{⼀、⽔}
  \definition*{s.}{Shangai (Xangai)}
\end{entry}

\begin{entry}{上课}{shang4 ke4}{3,10}{⼀、⾔}[HSK 1]
  \definition{v.}{assistir à aula | ir para a aula | ir dar uma aula}
\end{entry}

\begin{entry}{上来}{shang4 lai2}{3,7}{⼀、⽊}[HSK 3]
  \definition{v.}{subir (para a minha localização) | estar no começo | vir à tona | usado depois de um verbo para indicar sucesso em fazer algo}
\end{entry}

\begin{entry}{上楼}{shang4 lou2}{3,13}{⼀、⽊}[HSK 4]
  \definition{v.}{subir as escadas; ir para o andar de cima}
\end{entry}

\begin{entry}{上门}{shang4 men2}{3,3}{⼀、⾨}[HSK 4]
  \definition{v.}{chamar; visitar; aparecer; ir ou vir para ver alguém; ir até a porta; ir até a casa de alguém | trancar a porta; fechar a porta durante a noite | casar-se e morar com a família da noiva}
\end{entry}

\begin{entry}{上面}{shang4 mian4}{3,9}{⼀、⾯}[HSK 3]
  \definition{s.}{uma posição mais alta que algo; uma posição acima/acima de algo | superfície do objeto | aspecto | a parte acima mencionada | autoridades superiores | os mais velhos; a geração mais velha da família}
\end{entry}

\begin{entry}{上坡路}{shang4po1lu4}{3,8,13}{⼀、⼟、⾜}
  \definition{s.}{aclive | progresso | (fig.) tendência ascendente}
\end{entry}

\begin{entry}{上去}{shang4 qu4}{3,5}{⼀、⼛}[HSK 3]
  \definition{v.}{subir (a partir da minha localização) | ascender a um lugar (ou estado) considerado mais elevado (ou acima)}
\end{entry}

\begin{entry}{上升}{shang4 sheng1}{3,4}{⼀、⼗}[HSK 3]
  \definition{v.}{elevar; subir; mover-se para cima}
\end{entry}

\begin{entry}{上网}{shang4 wang3}{3,6}{⼀、⽹}[HSK 1]
  \definition{v.}{conectar à \emph{Internet} | fazer \emph{upload} | ficar \emph{online}}
\end{entry}

\begin{entry}{上午}{shang4wu3}{3,4}{⼀、⼗}[HSK 1]
  \definition{adv.}{manhã | de manhã | período antes do meio-dia}
\end{entry}

\begin{entry}{上学}{shang4 xue2}{3,8}{⼀、⼦}[HSK 1]
  \definition{v.}{ir à escola | frequentar a escola | estar na escola | iniciar as aulas}
\end{entry}

\begin{entry}{上询}{shang4 xun2}{3,8}{⼀、⾔}
  \definition{adv.}{primeira dezena do mês}
\end{entry}

\begin{entry}{上演}{shang4yan3}{3,14}{⼀、⽔}
  \definition{s.}{exibição | encenação}
  \definition{v.}{exibir (um filme) | encenar (uma peça)}
\end{entry}

\begin{entry}{上衣}{shang4 yi1}{3,6}{⼀、⾐}[HSK 3]
  \definition{s.}{jaqueta; vestimenta externa superior}
\end{entry}

\begin{entry}{上周}{shang4 zhou1}{3,8}{⼀、⼝}[HSK 2]
  \definition{s.}{semana passada}
\end{entry}

\begin{entry}{尚且}{shang4qie3}{8,5}{⼩、⼀}
  \definition{conj.}{até | ainda}
\end{entry}

\begin{entry}{尚且……何况……}{shang4qie3 he2kuang4}{8,5,7,7}{⼩、⼀、⼈、⼎}
  \definition{conj.}{ainda que\dots, \dots}
\end{entry}

\begin{entry}{烧}{shao1}{10}{⽕}[HSK 4]
  \definition[次]{s.}{febre; temperatura corporal mais alta do que o normal}
  \definition{v.}{queimar; pegar fogo | cozinhar; aquecer; assar | guisar depois de fritar ou fritar depois de guisar | assar; grelhar os ingredientes dos alimentos diretamente sobre o fogo | ter febre; estar com febre | danificar (matar ou murchar) as plantas pelo uso excessivo (ou inadequado) de fertilizantes | tornar-se arrogante ou presunçoso; metáfora de estar em uma boa posição e se deixar levar}
\end{entry}

\begin{entry}{烧烤}{shao1kao3}{10,10}{⽕、⽕}
  \definition{s.}{churrasco}
  \definition{v.}{assar}
\end{entry}

\begin{entry}{稍}{shao1}{12}{⽲}
  \definition{adv.}{um pouco | ligeiramente | em vez de}
\end{entry}

\begin{entry}{稍微}{shao1wei1}{12,13}{⽲、⼻}
  \definition{adv.}{um pouco}
\end{entry}

\begin{entry}{少}{shao3}{4}{⼩}[HSK 1]
  \definition{adj.}{pouco, poucos}
  \definition{v.}{sentir falta | faltar | parar (de fazer algo)}
  \seeref{少}{shao4}
\end{entry}

\begin{entry}{少数}{shao3 shu4}{4,13}{⼩、⽁}[HSK 2]
  \definition{s.}{pequeno número | poucos | minoria}
\end{entry}

\begin{entry}{少}{shao4}{4}{⼩}
  \definition{s.}{jovem}
  \seeref{少}{shao3}
\end{entry}

\begin{entry}{少年}{shao4 nian2}{4,6}{⼩、⼲}[HSK 2]
  \definition[个]{s.}{adolescente; juventude; atualmente, a faixa etária geralmente referida é de 10 anos ou mais a 18 anos ou mais | menor; jovem; juvenil; refere-se a menores na faixa etária anterior | jovem; adolescente; rapaz}
\end{entry}

\begin{entry}{舌头}{she2tou5}{6,5}{⾆、⼤}
  \definition[个]{s.}{língua | soldado inimigo capturado com o propósito de extrair informações}
\end{entry}

\begin{entry}{折}{she2}{7}{⼿}
  \definition{v.}{estalar; quebrar | perder dinheiro em um negócio}
  \seeref{折}{zhe1}
  \seeref{折}{zhe2}
\end{entry}

\begin{entry}{蛇}{she2}{11}{⾍}
  \definition[条]{s.}{cobra | serpente}
\end{entry}

\begin{entry}{设备}{she4bei4}{6,8}{⾔、⼡}[HSK 3]
  \definition[个]{s.}{facilidade; equipamento; instalação}
\end{entry}

\begin{entry}{设计}{she4ji4}{6,4}{⾔、⾔}[HSK 3]
  \definition[份]{s.}{plano; esquema}
  \definition{v.}{planejar; projetar | inventar}
\end{entry}

\begin{entry}{设立}{she4li4}{6,5}{⾔、⽴}[HSK 3]
  \definition{v.}{encontrar; estabelecer; configurar}
\end{entry}

\begin{entry}{设施}{she4shi1}{6,9}{⾔、⽅}[HSK 4]
  \definition{s.}{facilidade; instalação; instituições, sistemas, organizações, edifícios, etc., estabelecidos para realizar um trabalho ou atender a uma necessidade}
\end{entry}

\begin{entry}{设置}{she4zhi4}{6,13}{⾔、⽹}[HSK 4]
  \definition{v.}{estabelecer; colocar em prática; estabelecer ou criar instituições, empregos, profissões ou códigos, etc. | encaixar; ajustar; instalar; configurar; colocar}
\end{entry}

\begin{entry}{社会}{she4hui4}{7,6}{⽰、⼈}[HSK 3]
  \definition[个,种]{s.}{sociedade | comunidade}
\end{entry}

\begin{entry}{射}{she4}{10}{⼨}
  \definition{v.}{atirar | lançar}
\end{entry}

\begin{entry}{摄氏}{she4shi4}{13,4}{⼿、⽒}
  \definition{s.}{graus Celsius (°C), centígrado}
\end{entry}

\begin{entry}{谁}{shei2}{10}{⾔}[HSK 1]
  \definition{pron.}{quem?}
  \seeref{谁}{shui2}
\end{entry}

\begin{entry}{申请}{shen1qing3}{5,10}{⽥、⾔}[HSK 4]
  \definition[份,批,项]{s.}{pedido de; solicitação de; requerimento para; solicitações escritas para serem mostradas a superiores ou autoridades}
  \definition{v.}{solicitar; apresentar uma solicitação; fazer representações e solicitações a uma autoridade superior ou às autoridades relevantes}
\end{entry}

\begin{entry}{身边}{shen1 bian1}{7,5}{⾝、⾡}[HSK 2]
  \definition{adv.}{ao redor | ao lado de alguém | em mãos}
\end{entry}

\begin{entry}{身材}{shen1cai2}{7,7}{⾝、⽊}[HSK 4]
  \definition[副,种,个,具]{s.}{figura; estatura; altura e peso corporal}
\end{entry}

\begin{entry}{身份}{shen1fen4}{7,6}{⾝、⼈}[HSK 4]
  \definition[种]{s.}{status; capacidade; identidade; refere-se à origem, ao status e às qualificações de uma pessoa | dignidade; posição honrada; referência especial ao status respeitável}
\end{entry}

\begin{entry}{身份证}{shen1 fen4 zheng4}{7,6,7}{⾝、⼈、⾔}[HSK 3]
  \definition[张]{s.}{ID; bilhete de identidade; carteira de identidade}
\end{entry}

\begin{entry}{身高}{shen1 gao1}{7,10}{⾝、⾼}[HSK 4]
  \definition[个,种,段]{s.}{estatura; altura (de uma pessoa);}
\end{entry}

\begin{entry}{身上}{shen1shang5}{7,3}{⾝、⼀}[HSK 1]
  \definition{adv.}{no corpo de alguém | em um | com um}
\end{entry}

\begin{entry}{身体}{shen1ti3}{7,7}{⾝、⼈}[HSK 1]
  \definition[具,个]{s.}{em pessoa | saúde de alguém | o corpo}
\end{entry}

\begin{entry}{身体能力}{shen1ti3 neng2li4}{7,7,10,2}{⾝、⼈、⾁、⼒}
  \definition{s.}{habilidade física}
\end{entry}

\begin{entry}{身体乳}{shen1ti3 ru3}{7,7,8}{⾝、⼈、⼄}
  \definition{s.}{loção corporal}
\end{entry}

\begin{entry}{身亡}{shen1wang2}{7,3}{⾝、⼇}
  \definition{v.}{morrer}
\end{entry}

\begin{entry}{深}{shen1}{11}{⽔}[HSK 3]
  \definition*{s.}{sobrenome Shen}
  \definition{adj.}{profundo | difícil; intenso; profundo | completo; penetrante; intenso; profundo | próximo; íntimo | escuro; profundo | tardio}
  \definition{adv.}{muito; grandemente; profundamente}
  \definition{s.}{profundidade}
  \seealsoref{浅}{qian3}
\end{entry}

\begin{entry}{深厚}{shen1hou4}{11,9}{⽔、⼚}[HSK 4]
  \definition{adj.}{profundo; sentimentos fortes | sólido; profundamente enraizado; fundação sólida}
\end{entry}

\begin{entry}{深刻}{shen1ke4}{11,8}{⽔、⼑}[HSK 3]
  \definition{adj.}{profundo; instenso}
\end{entry}

\begin{entry}{深入}{shen1 ru4}{11,2}{⽔、⼊}[HSK 3]
  \definition{adj.}{minucioso; meticuloso; profundo}
  \definition{v.}{ir fundo em; penetrar em}
\end{entry}

\begin{entry}{深深}{shen1shen1}{11,11}{⽔、⽔}
  \definition{adj.}{profundo}
  \definition{adv.}{profundamente}
\end{entry}

\begin{entry}{深夜}{shen1ye4}{11,8}{⽔、⼣}
  \definition{adv.}{tarde da noite}
\end{entry}

\begin{entry}{什么}{shen2me5}{4,3}{⼈、⼃}[HSK 1]
  \definition{pron.}{que? | o que?}
  \definition{pron.}{algo | qualquer coisa}
\end{entry}

\begin{entry}{什么时候}{shen2me5shi2hou5}{4,3,7,10}{⼈、⼃、⽇、⼈}
  \definition{adv.}{quando? | a que horas?}
\end{entry}

\begin{entry}{什么样}{shen2 me5 yang4}{4,3,10}{⼈、⼃、⽊}[HSK 2]
  \definition{pron.}{que tipo? | o quê? | que tipo?}
\end{entry}

\begin{entry}{神}{shen2}{9}{⽰}
  \definition*{s.}{Deus}
  \definition{s.}{deus | divindade}
\end{entry}

\begin{entry}{神话}{shen2hua4}{9,8}{⽰、⾔}[HSK 4]
  \definition[个]{s.}{mito; mitologia; conto de fadas; refere-se a deuses e deusas lendários e histórias de heróis antigos deificados | lorota; refere-se a alegações ridículas e infundadas}
\end{entry}

\begin{entry}{神经}{shen2jing1}{9,8}{⽰、⽷}
  \definition{adj.}{desequilibrado | louco | insano}
  \definition{s.}{nervo}
\end{entry}

\begin{entry}{神经病的}{shen2jing1bing4de5}{9,8,10,8}{⽰、⽷、⽧、⽩}
  \definition{adj.}{neurótico}
\end{entry}

\begin{entry}{神经病学}{shen2jing1bing4xue2}{9,8,10,8}{⽰、⽷、⽧、⼦}
  \definition{s.}{neurologia}
\end{entry}

\begin{entry}{神秘}{shen2mi4}{9,10}{⽰、⽲}[HSK 4]
  \definition{adj.}{místico; misterioso}
\end{entry}

\begin{entry}{神明}{shen2ming2}{9,8}{⽰、⽇}
  \definition{s.}{divindades | deuses}
\end{entry}

\begin{entry}{神奇}{shen2qi2}{9,8}{⽰、⼤}
  \definition{adj.}{mágico | místico | milagroso}
  \definition{s.}{mágica | milagre}
\end{entry}

\begin{entry}{神器}{shen2qi4}{9,16}{⽰、⼝}
  \definition{s.}{objeto mágico | objeto simbólico do poder imperial | arma fina | ferramenta muito útil}
\end{entry}

\begin{entry}{神兽}{shen2shou4}{9,11}{⽰、⼋}
  \definition{s.}{animal mitológico | fera}
\end{entry}

\begin{entry}{甚而}{shen4'er2}{9,6}{⽢、⽽}
  \definition{conj.}{(ir) tão longe quanto | tanto que}
\end{entry}

\begin{entry}{甚或}{shen4huo4}{9,8}{⽢、⼽}
  \definition{conj.}{(ir) tão longe quanto | tanto que}
\end{entry}

\begin{entry}{甚至}{shen4zhi4}{9,6}{⽢、⾄}[HSK 4]
  \definition{conj.}{e até mesmo; nem mesmo; para apresentar uma situação típica e especial, para enfatizar a profundidade e a seriedade de uma situação}
\end{entry}

\begin{entry}{升}{sheng1}{4}{⼗}[HSK 3]
  \definition*{s.}{sobrenome Sheng}
  \definition{clas.}{litro (l)}
  \definition{s.}{sheng, uma unidade de medida seca para grãos (= 1 litro)}
  \definition{v.}{elevar; içar; subir; ascender | promover}
\end{entry}

\begin{entry}{升起}{sheng1qi3}{4,10}{⼗、⾛}
  \definition{v.}{levantar | içar | subir}
\end{entry}

\begin{entry}{生}{sheng1}{5}{⽣}[HSK 2,3][Kangxi 100]
  \definition*{s.}{sobrenome Sheng}
  \definition{adj.}{vivo | imaturo; verde | cru; não cozido | não processado; não refinado; bruto | desconhecido; estranho | rígido; mecânico}
  \definition{adv.}{muito | usado antes de certas palavras que expressam emoções ou sentimentos}
  \definition{s.}{vida | meio de vida; sustento | aluno; estudante; pupilo | erudito | o tipo de personagem masculino na ópera de Pequim, etc.}
  \definition{suf.}{certos sufixos substantivos que se referem a pessoas (学生) | sufixos de certos advérbios (好生)}
  \definition{v.}{dar à luz; suportar | nascer | crescer | viver; existir | obter; ter | acender (um fogo)}
  \seealsoref{好生}{hao3sheng1}
  \seealsoref{学生}{xue2sheng5}
\end{entry}

\begin{entry}{生病}{sheng1 bing4}{5,10}{⽣、⽧}[HSK 1]
  \definition{v.}{ficar doente | estar adoecido}
\end{entry}

\begin{entry}{生菜}{sheng1cai4}{5,11}{⽣、⾋}
  \definition{s.}{alface}
\end{entry}

\begin{entry}{生产}{sheng1chan3}{5,6}{⽣、⼇}[HSK 3]
  \definition{v.}{produzir; fabricar | dar à luz uma criança}
\end{entry}

\begin{entry}{生词}{sheng1 ci2}{5,7}{⽣、⾔}[HSK 2]
  \definition[个]{s.}{nova palavra}
\end{entry}

\begin{entry}{生存}{sheng1cun2}{5,6}{⽣、⼦}[HSK 3]
  \definition{v.}{viver; sobreviver; subsistir}
\end{entry}

\begin{entry}{生的}{sheng1de5}{5,8}{⽣、⽩}
  \definition{conj.}{para evitar isso | para que\dots não\dots}
\end{entry}

\begin{entry}{生动}{sheng1dong4}{5,6}{⽣、⼒}[HSK 3]
  \definition{adj.}{vívido; animado}
\end{entry}

\begin{entry}{生活}{sheng1huo2}{5,9}{⽣、⽔}[HSK 2]
  \definition[道]{s.}{vida | atividade | meios de subsistência}
  \definition{v.}{viver}
\end{entry}

\begin{entry}{生活垃圾}{sheng1huo2la1ji1}{5,9,8,6}{⽣、⽔、⼟、⼟}
  \definition{s.}{lixo doméstico}
\end{entry}

\begin{entry}{生活型}{sheng1huo2 xing2}{5,9,9}{⽣、⽔、⼟}
  \definition{s.}{forma de vida}
\end{entry}

\begin{entry}{生理}{sheng1li3}{5,11}{⽣、⽟}
  \definition{adj.}{fisiológico}
  \definition{s.}{fisiologia}
\end{entry}

\begin{entry}{生命}{sheng1ming4}{5,8}{⽣、⼝}[HSK 3]
  \definition{s.}{vida | não envolve apenas a existência e as atividades dos organismos, mas também inclui experiências de vida humana, valores e elementos-chave da sobrevivência e do desenvolvimento de várias coisas}
\end{entry}

\begin{entry}{生气}{sheng1 qi4}{5,4}{⽣、⽓}[HSK 1]
  \definition{s.}{vitalidade | vigor}
  \definition{v.+compl.}{irritar-se | zangar-se | ofender-se | ficar com raiva}
\end{entry}

\begin{entry}{生日}{sheng1ri4}{5,4}{⽣、⽇}[HSK 1]
  \definition[个]{s.}{aniversário}
\end{entry}

\begin{entry}{生态}{sheng1tai4}{5,8}{⽣、⼼}
  \definition{adj.}{ecológico}
  \definition{s.}{ecologia}
\end{entry}

\begin{entry}{生物}{sheng1wu4}{5,8}{⽣、⽜}
  \definition{adj.}{biológico}
  \definition{s.}{biologia (disciplina) | organismo | ser vivo}
\end{entry}

\begin{entry}{生意}{sheng1yi4}{5,13}{⽣、⼼}
  \definition{s.}{tendência a crescer; vida e vitalidade}
  \seealsoref{生意}{sheng1yi5}
\end{entry}

\begin{entry}{生意}{sheng1yi5}{5,13}{⽣、⼼}[HSK 3]
  \definition[笔,种,次]{s.}{comércio; negócios}
  \seealsoref{生意}{sheng1yi4}
\end{entry}

\begin{entry}{生鱼片}{sheng1yu2pian4}{5,8,4}{⽣、⿂、⽚}
  \definition{s.}{fatias de peixe cru, \emph{sashimi}}
\end{entry}

\begin{entry}{生长}{sheng1zhang3}{5,4}{⽣、⾧}[HSK 3]
  \definition{v.}{cresçer | nascer e crescer}
\end{entry}

\begin{entry}{声明}{sheng1ming2}{7,8}{⼠、⽇}[HSK 3]
  \definition[项,份]{s.}{declaração}
  \definition{v.}{declarar}
  \definition{v.}{declarar; anunciar}
\end{entry}

\begin{entry}{声音}{sheng1yin1}{7,9}{⼠、⾳}[HSK 2]
  \definition[个,种]{s.}{som | voz}
\end{entry}

\begin{entry}{绳子}{sheng2zi5}{11,3}{⽷、⼦}
  \definition[条]{s.}{corda | cordão}
\end{entry}

\begin{entry}{省}{sheng3}{9}{⽬}[HSK 2]
  \definition{s.}{província | capital provincial}
  \definition{v.}{economizar | guardar | ser frugal | omitir | excluir | deixar de fora}
  \seeref{省}{xing3}
\end{entry}

\begin{entry}{省城}{sheng3cheng2}{9,9}{⽬、⼟}
  \definition{s.}{capital da província}
\end{entry}

\begin{entry}{省会}{sheng3hui4}{9,6}{⽬、⼈}
  \definition{s.}{capital da província}
\end{entry}

\begin{entry}{省俭}{sheng3jian3}{9,9}{⽬、⼈}
  \definition{s.}{econômico | frugal}
  \definition{v.}{economizar}
\end{entry}

\begin{entry}{省力}{sheng3li4}{9,2}{⽬、⼒}
  \definition{v.}{economizar esforço ou trabalho}
\end{entry}

\begin{entry}{省钱}{sheng3qian2}{9,10}{⽬、⾦}
  \definition{v.}{economizar dinheiro}
\end{entry}

\begin{entry}{省却}{sheng3que4}{9,7}{⽬、⼙}
  \definition{v.}{livrar-se (para economizar espaço) | salvar}
\end{entry}

\begin{entry}{省心}{sheng3xin1}{9,4}{⽬、⼼}
  \definition{adj.}{despreocupado}
  \definition{v.}{ser poupado de preocupações | despreocupar-se}
\end{entry}

\begin{entry}{省长}{sheng3zhang3}{9,4}{⽬、⾧}
  \definition*{s.}{Governador | governador de uma província}
\end{entry}

\begin{entry}{圣诞节}{sheng4dan4jie2}{5,8,5}{⼟、⾔、⾋}
  \definition*{s.}{Natal}
\end{entry}

\begin{entry}{圣地}{sheng4di4}{5,6}{⼟、⼟}
  \definition{s.}{terra santa (de uma religião) | lugar sagrado | santuário | cidade santa (como Jerusalém, Meca, etc.) | centro de interesse histórico}
\end{entry}

\begin{entry}{胜}{sheng4}{9}{⾁}[HSK 3]
  \definition{adj.}{soberbo; maravilhoso; adorável}
  \definition[场]{s.}{vitória; sucesso | penteado de mulher}
  \definition{v.}{ganhar; derrotar; vencer; ter sucesso | superar; ser superior a; levar a melhor sobre | ser igual a; poder suportar}
\end{entry}

\begin{entry}{胜利}{sheng4li4}{9,7}{⾁、⼑}[HSK 3]
  \definition{adv.}{com sucesso; triunfantemente}
  \definition[场,个]{s.}{vitória; triunfo; sucesso}
  \definition{v.}{ganhar; vencer; triunfar; ter sucesso}
\end{entry}

\begin{entry}{胜算}{sheng4suan4}{9,14}{⾁、⽵}
  \definition{s.}{probabilidade de sucesso | estratégia que garante o sucesso}
  \definition{v.}{ter certeza do sucesso}
\end{entry}

\begin{entry}{乘}{sheng4}{10}{⽲}
  \definition{clas.}{para carruagens de guerra puxada por quatro cavalos}
  \definition{s.}{obras históricas; livros de história geral}
  \seeref{乘}{cheng2}
\end{entry}

\begin{entry}{盛宴}{sheng4yan4}{11,10}{⽫、⼧}
  \definition{s.}{celebração}
\end{entry}

\begin{entry}{失败}{shi1bai4}{5,8}{⼤、⾒}[HSK 4]
  \definition{adj.}{insatisfatório; a maneira como as coisas aconteceram deixou muito a desejar; o resultado final deixou muito a desejar}
  \definition{v.}{perder; ser derrotado; não vencer em uma guerra ou competição | falhar; fracassar; não dar em nada; falhar em atingir um objetivo ou meta desejada (trabalho, carreira, etc.)}
\end{entry}

\begin{entry}{失落}{shi1luo4}{5,12}{⼤、⾋}
  \definition{s.}{frustração | decepção | perda}
  \definition{v.}{perder (algo) | cair (algo) | sentir uma sensação de perda}
\end{entry}

\begin{entry}{失眠}{shi1mian2}{5,10}{⼤、⽬}
  \definition{s.}{insônia}
  \definition{v.}{ter insônia}
\end{entry}

\begin{entry}{失去}{shi1qu4}{5,5}{⼤、⼛}[HSK 3]
  \definition{v.}{perder}
\end{entry}

\begin{entry}{失望}{shi1wang4}{5,11}{⼤、⽉}[HSK 4]
  \definition{adj.}{desapontado; decepcionado}
  \definition{v.}{ficar desapontado; ficar decepcionado; estar desapontado;}
\end{entry}

\begin{entry}{失业}{shi1ye4}{5,5}{⼤、⼀}[HSK 4]
  \definition{v.}{não ter emprego; estar desempregado; estar sem trabalho}
\end{entry}

\begin{entry}{失意}{shi1yi4}{5,13}{⼤、⼼}
  \definition{adj.}{desapontado | frustrado}
\end{entry}

\begin{entry}{师}{shi1}{6}{⼱}
  \definition*{s.}{sobrenome Shi}
  \definition{s.}{professor | mestre | especialista | modelo | divisão do exército}
  \definition{v.}{despachar tropas}
\end{entry}

\begin{entry}{师傅}{shi1fu5}{6,12}{⼱、⼈}
  \definition[个,位,名]{s.}{técnico | mestre-trabalhador | forma respeitosa de tratamento para homens mais velhos}
\end{entry}

\begin{entry}{诗}{shi1}{8}{⾔}[HSK 4]
  \definition*{s.}{O Livro das Canções《诗经》| sobrenome Shi}
  \definition{s.}{poesia; verso; poema;}
  \seealsoref{诗经}{shi1jing1}
\end{entry}

\begin{entry}{诗词}{shi1ci2}{8,7}{⾔、⾔}
  \definition{s.}{verso}
\end{entry}

\begin{entry}{诗经}{shi1jing1}{8,8}{⾔、⽷}
  \definition*{s.}{Shijing, o Livro das Canções, antiga coleção de poemas chineses e um dos Cinco Clássicos do Confucionismo}
\end{entry}

\begin{entry}{诗句}{shi1ju4}{8,5}{⾔、⼝}
  \definition[行]{s.}{verso | versículo}
\end{entry}

\begin{entry}{诗人}{shi1 ren2}{8,2}{⾔、⼈}[HSK 4]
  \definition{s.}{poeta; escritor de poesia}
\end{entry}

\begin{entry}{诗意}{shi1yi4}{8,13}{⾔、⼼}
  \definition{adj.}{poético}
  \definition{s.}{poesia}
\end{entry}

\begin{entry}{湿}{shi1}{12}{⽔}[HSK 4]
  \definition{adj.}{molhado; úmido; algo com água ou com muita água dentro}
\end{entry}

\begin{entry}{十}{shi2}{2}{⼗}[HSK 1][Kangxi 24]
  \definition{num.}{dez; 10 | dezena}
\end{entry}

\begin{entry}{十分}{shi2fen1}{2,4}{⼗、⼑}[HSK 2]
  \definition{adv.}{muito | extremamente | totalmente | absolutamente}
\end{entry}

\begin{entry}{十足}{shi2zu2}{2,7}{⼗、⾜}
  \definition{adj.}{amplo | completo | cento por cento | tom puro (de alguma cor)}
\end{entry}

\begin{entry}{石头}{shi2tou5}{5,5}{⽯、⼤}[HSK 3]
  \definition[块,堆,些]{s.}{rocha; pedra}
\end{entry}

\begin{entry}{石油}{shi2you2}{5,8}{⽯、⽔}[HSK 3]
  \definition[桶,吨,升]{s.}{óleo; óleo fóssil; petróleo}
\end{entry}

\begin{entry}{时}{shi2}{7}{⽇}[HSK 3]
  \definition*{s.}{sobrenome Shi}
  \definition{adj.}{atual; presente | a tempo; feito a tempo}
  \definition{adv.}{de vez em quando; ocasionalmente; de ​​tempos em tempos | às vezes\dots às vezes\dots}
  \definition{clas.}{hora; horas}
  \definition{s.}{dias; tempos; longo período de tempo | tempo; tempo fixo | hora; hora do dia | temporada | chance; oportunidade | atualidade; presente | tempo verbal}
\end{entry}

\begin{entry}{时差}{shi2cha1}{7,9}{⽇、⼯}
  \definition{s.}{diferença de tempo | \emph{jet lag}}
\end{entry}

\begin{entry}{时代}{shi2dai4}{7,5}{⽇、⼈}[HSK 3]
  \definition[个]{s.}{idade; era; tempos; época | um período na vida de alguém}
\end{entry}

\begin{entry}{时光}{shi2guang1}{7,6}{⽇、⼉}
  \definition{s.}{tempo | época | período de tempo}
\end{entry}

\begin{entry}{时候}{shi2hou5}{7,10}{⽇、⼈}[HSK 1]
  \definition{adv.}{quando?}
  \definition{s.}{duração de tempo | momento | período | tempo}
\end{entry}

\begin{entry}{时间}{shi2jian1}{7,7}{⽇、⾨}[HSK 1]
  \definition{s.}{(conceito de, duração de, um ponto no) tempo}
\end{entry}

\begin{entry}{时刻}{shi2ke4}{7,8}{⽇、⼑}[HSK 3]
  \definition{adv.}{constantemente; sempre}
  \definition[个,段]{s.}{tempo; hora; momento; conjuntura}
\end{entry}

\begin{entry}{时时}{shi2shi2}{7,7}{⽇、⽇}
  \definition{adv.}{muitas vezes | constantemente}
\end{entry}

\begin{entry}{实际}{shi2ji4}{8,7}{⼧、⾩}[HSK 2]
  \definition{adj.}{real | atual | concreto | prático | factual | realista}
  \definition{s.}{realidade | prática}
\end{entry}

\begin{entry}{实际上}{shi2 ji4 shang4}{8,7,3}{⼧、⾩、⼀}[HSK 3]
  \definition{adv.}{de fato; na verdade; como aliás}
\end{entry}

\begin{entry}{实力}{shi2li4}{8,2}{⼧、⼒}[HSK 3]
  \definition{s.}{força | geralmente se refere à força militar e econômica de um país, grupo ou indivíduo, e também se refere à habilidade de um indivíduo ou grupo em um jogo.}
\end{entry}

\begin{entry}{实施}{shi2shi1}{8,9}{⼧、⽅}[HSK 4]
  \definition{v.}{colocar em vigor; implementar (leis, políticas, etc.); executar; trazer (colocar) algo em vigor; fazer cumprir; colocar algo em (prática)}
\end{entry}

\begin{entry}{实习}{shi2xi2}{8,3}{⼧、⼄}[HSK 2]
  \definition{s.}{estagiário | estágio}
\end{entry}

\begin{entry}{实现}{shi2xian4}{8,8}{⼧、⾒}[HSK 2]
  \definition{v.}{alcançar | implementar | constatar}
\end{entry}

\begin{entry}{实行}{shi2xing2}{8,6}{⼧、⾏}[HSK 3]
  \definition{v.}{praticar; implementar; executar; pôr em prática}
\end{entry}

\begin{entry}{实验}{shi2yan4}{8,10}{⼧、⾺}[HSK 3]
  \definition[个,次]{s.}{teste; experimento; trabalho de laboratório}
  \definition{v.}{testar; experimentar}
\end{entry}

\begin{entry}{实验室}{shi2 yan4 shi4}{8,10,9}{⼧、⾺、⼧}[HSK 3]
  \definition[个,间]{s.}{laboratório}
\end{entry}

\begin{entry}{实用}{shi2yong4}{8,5}{⼧、⽤}[HSK 4]
  \definition{adj.}{prático; pragmático; funcional; atende aos requisitos reais da aplicação}
  \definition{v.}{colocar em uso prático}
\end{entry}

\begin{entry}{实在}{shi2zai4}{8,6}{⼧、⼟}[HSK 2]
  \definition{adv.}{realmente | verdadeiramente | de fato | na verdade}
\end{entry}

\begin{entry}{食品}{shi2 pin3}{9,9}{⾷、⼝}[HSK 3]
  \definition[种]{s.}{comida; gêneros alimentícios; provisões}
\end{entry}

\begin{entry}{食堂}{shi2 tang2}{9,11}{⾷、⼟}[HSK 4]
  \definition[个,间]{s.}{cantina; refeitório}
\end{entry}

\begin{entry}{食物}{shi2wu4}{9,8}{⾷、⽜}[HSK 2]
  \definition[种]{s.}{comida}
\end{entry}

\begin{entry}{使}{shi3}{8}{⼈}[HSK 3]
  \definition*{s.}{sobrenome Shi}
  \definition{conj.}{se; supondo}
  \definition{s.}{enviado; mensageiro}
  \definition{v.}{enviar; dizer a alguém para fazer algo | usar; empregar; aplicar | fazer; causar; habilitar}
\end{entry}

\begin{entry}{使劲}{shi3 jin4}{8,7}{⼈、⼒}[HSK 4]
  \definition{v.+compl.}{colocar energia; exercer toda a sua força | esforçar-se para ajudar; colocar energia para ajudar}
\end{entry}

\begin{entry}{使用}{shi3yong4}{8,5}{⼈、⽤}[HSK 2]
  \definition{v.}{usar | empregar | aplicar}
\end{entry}

\begin{entry}{始终}{shi3zhong1}{8,8}{⼥、⽷}[HSK 3]
  \definition{adv.}{sempre; o tempo todo; durante todo; do começo ao fim}
  \definition{s.}{todo o processo do começo ao fim}
\end{entry}

\begin{entry}{屎}{shi3}{9}{⼫}
  \definition{s.}{fezes | excrementos | (forma ligada) secreção (do ouvido, olho, etc.)}
\end{entry}

\begin{entry}{士兵}{shi4bing1}{3,7}{⼠、⼋}[HSK 4]
  \definition[名,个]{s.}{soldado; militar; termo coletivo para oficiais não comissionados e soldados; os membros mais jovens do exército}
\end{entry}

\begin{entry}{世代}{shi4dai4}{5,5}{⼀、⼈}
  \definition{adv.}{por muitas gerações, eras}
  \definition{s.}{geração | era}
\end{entry}

\begin{entry}{世纪}{shi4ji4}{5,6}{⼀、⽷}[HSK 3]
  \definition[个]{s.}{século}
\end{entry}

\begin{entry}{世界}{shi4jie4}{5,9}{⼀、⽥}[HSK 3]
  \definition[个]{s.}{mundo | a soma da natureza e da sociedade humana | o universo sem limites | situação social}
\end{entry}

\begin{entry}{世界杯}{shi4jie4bei1}{5,9,8}{⼀、⽥、⽊}[HSK 3]
  \definition*{s.}{Copa do Mundo}
\end{entry}

\begin{entry}{世锦赛}{shi4jin3sai4}{5,13,14}{⼀、⾦、⾙}
  \definition*{s.}{Campeonato Mundial}
\end{entry}

\begin{entry}{市}{shi4}{5}{⼱}[HSK 2]
  \definition*{s.}{sobrenome Shi}
  \definition{s.}{mercado | cidade | município | referente ao sistema chinês de pesos e medidas}
  \definition{v.}{comprar | vender | negociar}
\end{entry}

\begin{entry}{市场}{shi4chang3}{5,6}{⼱、⼟}[HSK 3]
  \definition[家]{s.}{mercado (também no abstrato) | área de \emph{marketing} | âmbito de influência (figurado)}
\end{entry}

\begin{entry}{市区}{shi4 qu1}{5,4}{⼱、⼖}[HSK 4]
  \definition[个]{s.}{\emph{downtown}; centro da cidade; distrito urbano; áreas que ficam dentro dos limites da cidade e geralmente têm uma alta concentração de população e estoque de moradias.}
\end{entry}

\begin{entry}{市长}{shi4 zhang3}{5,4}{⼱、⾧}[HSK 2]
  \definition[个]{s.}{prefeito}
\end{entry}

\begin{entry}{市中心}{shi4zhong1xin1}{5,4,4}{⼱、⼁、⼼}
  \definition{s.}{centro da cidade}
\end{entry}

\begin{entry}{似的}{shi4de5}{6,8}{⼈、⽩}[HSK 4]
  \definition{part.}{como; como\dots como; como se (embora); usada após uma palavra ou frase para indicar uma semelhança com algo ou uma situação | usada para indicar alto grau}
\end{entry}

\begin{entry}{式}{shi4}{6}{⼷}
  \definition{s.}{tipo | forma | padrão | estilo}
\end{entry}

\begin{entry}{事}{shi4}{8}{⼅}[HSK 1]
  \definition[件,桩,回]{s.}{coisa | assunto | item | matéria | coisa de trabalho | caso}
\end{entry}

\begin{entry}{事故}{shi4gu4}{8,9}{⼅、⽁}[HSK 3]
  \definition[桩,起,次]{s.}{acidente}
\end{entry}

\begin{entry}{事件}{shi4jian4}{8,6}{⼅、⼈}[HSK 3]
  \definition[个,件,次]{s.}{evento; incidente}
\end{entry}

\begin{entry}{事情}{shi4qing5}{8,11}{⼅、⼼}[HSK 2]
  \definition[件,桩]{s.}{assunto | coisa | erro | acidente | trabalho; emprego}
\end{entry}

\begin{entry}{事儿}{shi4r5}{8,2}{⼅、⼉}
  \definition[件,桩]{s.}{o emprego | negócio | afazeres | assunto que precisa ser resolvido | matéria}
\end{entry}

\begin{entry}{事实}{shi4shi2}{8,8}{⼅、⼧}[HSK 3]
  \definition{s.}{mito; lenda}
  \definition{v.}{dizer; contar; ser dito}
\end{entry}

\begin{entry}{事实上}{shi4 shi2 shang4}{8,8,3}{⼅、⼧、⼀}[HSK 3]
  \definition{adv.}{realmente; de ​​fato; na verdade}
\end{entry}

\begin{entry}{事物}{shi4wu4}{8,8}{⼅、⽜}[HSK 4]
  \definition{s.}{coisa; objeto; todos os objetos e fenômenos que existem objetivamente}
\end{entry}

\begin{entry}{事先}{shi4xian1}{8,6}{⼅、⼉}[HSK 4]
  \definition{adv.}{antes; de antemão; com antecedência; antecipadamente}
\end{entry}

\begin{entry}{事业}{shi4ye4}{8,5}{⼅、⼀}[HSK 3]
  \definition[个]{s.}{causa; carreira; empreendimento | instituição; instalações; unidade de trabalho apoiada financeiramente pelo governo}
\end{entry}

\begin{entry}{视角}{shi4jiao3}{8,7}{⾒、⾓}
  \definition{s.}{ângulo do qual se observa um objeto | (figurativo) perspectiva, ponto de vista, quadro de referência | (cinematografia) ângulo da câmera | (percepção visual) ângulo visual (o ângulo que um objeto visto subtende no olho) | (fotografia) ângulo de visão}
\end{entry}

\begin{entry}{视频}{shi4pin2}{8,13}{⾒、⾴}
  \definition{s.}{vídeo}
\end{entry}

\begin{entry}{试}{shi4}{8}{⾔}[HSK 1]
  \definition{s.}{exame | experimento | prova | teste}
  \definition{v.}{experimentar | provar | testar}
\end{entry}

\begin{entry}{试卷}{shi4juan4}{8,8}{⾔、⼙}[HSK 4]
  \definition[分,张]{s.}{folha de teste; folha de exame; papel usado para escrever as respostas nos exames}
\end{entry}

\begin{entry}{试题}{shi4 ti2}{8,15}{⾔、⾴}[HSK 3]
  \definition[道]{s.}{perguntas de teste; perguntas de exame}
\end{entry}

\begin{entry}{试验}{shi4yan4}{8,10}{⾔、⾺}[HSK 3]
  \definition{v.}{testar; fazer um teste; fazer um experimento}
\end{entry}

\begin{entry}{室}{shi4}{9}{⼧}[HSK 3]
  \definition*{s.}{sobrenome Shi | Shi, uma das mansões lunares}
  \definition{s.}{sala; aposento; cômodo |  seção; escritório | esposa}
\end{entry}

\begin{entry}{是}{shi4}{9}{⽇}[HSK 1]
  \definition{adj.}{correto | certo | verdadeiro | (reconhecimento respeitoso de um comando) muito bem}
  \definition{adv.}{(advérbio para afirmação enfática)}
  \definition{v.}{ser (somente seguido por substantivos)}
\end{entry}

\begin{entry}{是的}{shi4de5}{9,8}{⽇、⽩}
  \definition{adv.}{sim | está certo}
\end{entry}

\begin{entry}{是否}{shi4fou3}{9,7}{⽇、⼝}[HSK 4]
  \definition{adv.}{se; se ou não}
\end{entry}

\begin{entry}{适合}{shi4he2}{9,6}{⾡、⼝}[HSK 3]
  \definition{v.}{servir (uma roupa); caber; se adequar}
\end{entry}

\begin{entry}{适应}{shi4ying4}{9,7}{⾡、⼴}[HSK 3]
  \definition{v.}{ajustar-se; adequar-se; adaptar-se}
\end{entry}

\begin{entry}{适用}{shi4 yong4}{9,5}{⾡、⽤}[HSK 3]
  \definition{adj.}{adequado; aplicável}
  \definition{v.}{ser aplicável; ser adequado}
\end{entry}

\begin{entry}{收}{shou1}{6}{⽁}[HSK 2]
  \definition{expr.}{aos cuidados de (usado na linha de endereço após o nome)}
  \definition{v.}{receber | aceitar | coletar | colher | guardar}
\end{entry}

\begin{entry}{收到}{shou1 dao4}{6,8}{⽁、⼑}[HSK 2]
  \definition{v.}{receber}
\end{entry}

\begin{entry}{收费}{shou1 fei4}{6,9}{⽁、⾙}[HSK 3]
  \definition{v.}{cobrar; cobrar taxas}
\end{entry}

\begin{entry}{收回}{shou1 hui2}{6,6}{⽁、⼞}[HSK 4]
  \definition{v.}{retomar; recuperar; relembrar; recordar; receber de volta o que foi enviado ou emprestado, ou o dinheiro que foi emprestado ou usado | sacar; retirar; recolher; rescindir; cancelar (uma opinião, ordem, etc.)}
\end{entry}

\begin{entry}{收获}{shou1huo4}{6,10}{⽁、⾋}[HSK 4]
  \definition[次,番,份]{s.}{resultados; ganhos; metaforicamente falando, conhecimento, experiência, etc. obtidos em estudo ou trabalho; os resultados obtidos por meio de trabalho árduo | colheita; colheita de safras}
  \definition{v.}{colher; juntar as colheitas;}
\end{entry}

\begin{entry}{收据}{shou1ju4}{6,11}{⽁、⼿}
  \definition[张]{s.}{recibo | \emph{voucher}}
\end{entry}

\begin{entry}{收看}{shou1 kan4}{6,9}{⽁、⽬}[HSK 3]
  \definition{v.}{assistir (a um programa de TV)}
\end{entry}

\begin{entry}{收敛}{shou1lian3}{6,11}{⽁、⽁}
  \definition{v.}{diminuir | desaparecer | fazer desaparecer | exercer restrição | conter (alegria, arrogância, etc.) | constringir | (matemática) convergir}
\end{entry}

\begin{entry}{收买}{shou1mai3}{6,6}{⽁、⼄}
  \definition{v.}{subornar | comprar}
\end{entry}

\begin{entry}{收入}{shou1ru4}{6,2}{⽁、⼊}[HSK 2]
  \definition[笔,个]{s.}{renda | salário}
  \definition{v.}{receber dinheiro |  coletar | receber}
\end{entry}

\begin{entry}{收听}{shou1 ting1}{6,7}{⽁、⼝}[HSK 3]
  \definition{v.}{ouvir; escutar}
\end{entry}

\begin{entry}{收益}{shou1yi4}{6,10}{⽁、⽫}[HSK 4]
  \definition{s.}{lucro; renda; benefício; ganhos; vantagens ou benefícios obtidos}
\end{entry}

\begin{entry}{收音机}{shou1yin1ji1}{6,9,6}{⽁、⾳、⽊}[HSK 3]
  \definition[部,台]{s.}{rádio; sem fio}
\end{entry}

\begin{entry}{手}{shou3}{4}{⼿}[HSK 1][Kangxi 64]
  \definition{adj.}{conveniente}
  \definition{clas.}{de habilidade}
  \definition[双,只]{s.}{mão | pessoa envolvida em certos tipos de trabalho | pessoa qualificada para certos tipos de trabalho}
  \definition{v.}{segurar (formal)}
\end{entry}

\begin{entry}{手臂}{shou3bi4}{4,17}{⼿、⾁}
  \definition{s.}{braço}
\end{entry}

\begin{entry}{手边}{shou3bian1}{4,5}{⼿、⾡}
  \definition{adv.}{à mão | na mão}
\end{entry}

\begin{entry}{手表}{shou3biao3}{4,8}{⼿、⾐}[HSK 2]
  \definition[块,只,个]{s.}{relógio de pulso}
\end{entry}

\begin{entry}{手工}{shou3gong1}{4,3}{⼿、⼯}[HSK 4]
  \definition{s.}{trabalho manual; trabalho feito à mão | método de operação manual; método manual, sem máquina | remuneração por trabalho manual, braçal; custo de mão de obra braçal}
\end{entry}

\begin{entry}{手工艺人}{shou3gong1 yi4ren2}{4,3,4,2}{⼿、⼯、⾋、⼈}
  \definition{s.}{artesão}
\end{entry}

\begin{entry}{手机}{shou3ji1}{4,6}{⼿、⽊}[HSK 1]
  \definition[部,支]{s.}{telefone celular ou móvel}
\end{entry}

\begin{entry}{手里}{shou3 li3}{4,7}{⼿、⾥}[HSK 4]
  \definition[个]{s.}{(uma situação está) nas mãos de alguém | em mãos}
\end{entry}

\begin{entry}{手刹}{shou3sha1}{4,8}{⼿、⼑}
  \definition{s.}{freio de mão}
\end{entry}

\begin{entry}{手术}{shou3shu4}{4,5}{⼿、⽊}[HSK 4]
  \definition[个]{s.}{cirurgia; operação (cirúrgica); método de tratamento no qual o médico usa uma faca, tesoura etc. para fazer uma incisão em uma parte do corpo do paciente}
  \definition{v.}{realizar uma cirurgia}
\end{entry}

\begin{entry}{手套}{shou3tao4}{4,10}{⼿、⼤}[HSK 4]
  \definition[副,套,双,种]{s.}{luvas; itens usados ​​nas mãos, feitos de algodão, lã, couro, etc., para proteger as mãos ou manter o frio longe}
\end{entry}

\begin{entry}{手续}{shou3xu4}{4,11}{⼿、⽷}[HSK 3]
  \definition[个]{s.}{processo; formalidade; procedimento}
\end{entry}

\begin{entry}{手指}{shou3zhi3}{4,9}{⼿、⼿}[HSK 3]
  \definition[根,个]{s.}{dedo da mão}
\end{entry}

\begin{entry}{守}{shou3}{6}{⼧}[HSK 4]
  \definition*{s.}{sobrenome Shou}
  \definition{adv.}{próximo; perto de; perto de algum lugar em posição, perto de algum lugar}
  \definition{v.}{guardar; defender; estar presente para cuidar; não ir embora | manter vigilância; defender do ataque do oponente em uma luta ou confronto | observar; cumprir; respeitar; fazer as coisas como elas devem ser feitas | manter, observar a integridade; honrar a palavra de alguém; manter a palavra de alguém}
\end{entry}

\begin{entry}{守门员}{shou3men2yuan2}{6,3,7}{⼧、⾨、⼝}
  \definition{s.}{goleiro}
\end{entry}

\begin{entry}{首}{shou3}{9}{⾸}[HSK 4][Kangxi 185]
  \definition*{s.}{sobrenome Shou}
  \definition{adj.}{primeiro}
  \definition{adv.}{inicialmente; como o primeiro; em primeiro lugar}
  \definition{clas.}{para canções e poemas}
  \definition{s.}{cabeça | cabeça; chefe; líder | capital (cidade)}
  \definition{v.}{apresentar acusações contra alguém}
\end{entry}

\begin{entry}{首都}{shou3du1}{9,10}{⾸、⾢}[HSK 3]
  \definition[个]{s.}{capital (cidade)}
\end{entry}

\begin{entry}{首席执行官}{shou3xi2 zhi2xing2 guan1}{9,10,6,6,8}{⾸、⼱、⼿、⾏、⼧}
  \definition{s.}{\emph{chief executive officer}, CEO}
\end{entry}

\begin{entry}{首先}{shou3xian1}{9,6}{⾸、⼉}[HSK 3]
  \definition{adv.}{primeiramente; antes de todos os outros}
  \definition{conj.}{acima de tudo; primeiramente; em primeiro lugar}
\end{entry}

\begin{entry}{首相}{shou3xiang4}{9,9}{⾸、⽬}
  \definition*{s.}{Primeiro-Ministro (Japão, UK, etc.)}
\end{entry}

\begin{entry}{掱}{shou3}{12}{⼿}
  \variantof{手}
\end{entry}

\begin{entry}{受}{shou4}{8}{⼜}[HSK 3]
  \definition{v.}{receber; aceitar | sofrer; ser submetido a | aguentar; suportar; tolerar | ser agradável}
\end{entry}

\begin{entry}{受不了}{shou4bu5liao3}{8,4,2}{⼜、⼀、⼅}[HSK 4]
  \definition{adj.}{intolerável; insuportável}
  \definition{v.}{ser insuportável; não poder suportar algo; não suportar algo}
\end{entry}

\begin{entry}{受到}{shou4dao4}{8,8}{⼜、⼑}[HSK 2]
  \definition{v.}{receber (elogio, educação, punição, etc.) | ser elogiado, educado, punido, etc.}
\end{entry}

\begin{entry}{受得了}{shou4de5liao3}{8,11,2}{⼜、⼻、⼅}
  \definition{v.}{suportar | aguentar}
\end{entry}

\begin{entry}{受伤}{shou4shang1}{8,6}{⼜、⼈}[HSK 3]
  \definition{v.}{ser ferido; sofrer uma lesão}
\end{entry}

\begin{entry}{受限}{shou4xian4}{8,8}{⼜、⾩}
  \definition{v.}{ser limitado | ser restrito | ser constrangido}
\end{entry}

\begin{entry}{售货员}{shou4huo4yuan2}{11,8,7}{⼝、⾙、⼝}[HSK 4]
  \definition[个]{s.}{vendedor; balconista; assistente de loja; equipe que vende produtos em lojas}
\end{entry}

\begin{entry}{瘦}{shou4}{14}{⽧}
  \definition{adj.}{magro | emagrecido | apertado (roupas) | improdutivo (terras) | magro (carne)}
  \definition{v.}{perder peso}
\end{entry}

\begin{entry}{书}{shu1}{4}{⼄}[HSK 1]
  \definition[本,册,部]{s.}{livro | carta | documento}
\end{entry}

\begin{entry}{书包}{shu1bao1}{4,5}{⼄、⼓}[HSK 1]
  \definition[个,款]{s.}{mochila escolar}
\end{entry}

\begin{entry}{书店}{shu1dian4}{4,8}{⼄、⼴}[HSK 1]
  \definition[家]{s.}{livraria}
\end{entry}

\begin{entry}{书记}{shu1ji5}{4,5}{⼄、⾔}
  \definition{s.}{secretário (chefe de um ramo de um partido socialista ou comunista) | atendente | balconista | escriturário}
\end{entry}

\begin{entry}{书架}{shu1jia4}{4,9}{⼄、⽊}[HSK 3]
  \definition[个]{s.}{estante de livros}
\end{entry}

\begin{entry}{叔叔}{shu1shu5}{8,8}{⼜、⼜}
  \definition[个]{s.}{tio; irmão mais novo do pai | tio, dirigindo-se a um homem da mesma geração que o pai e mais jovem em idade}
\end{entry}

\begin{entry}{舒服}{shu1fu5}{12,8}{⾆、⽉}[HSK 2]
  \definition{adj.}{estar confortável | bem disposto | sentir-se bem}
\end{entry}

\begin{entry}{舒适}{shu1shi4}{12,9}{⾆、⾡}[HSK 4]
  \definition{adj.}{aconchegante; confortável; acolhedor; cômodo}
\end{entry}

\begin{entry}{输}{shu1}{13}{⾞}[HSK 3]
  \definition{v.}{transportar; transmitir | contribuir com dinheiro; doar | perder; ser batido; ser derrotado}
\end{entry}

\begin{entry}{输入}{shu1ru4}{13,2}{⾞、⼊}[HSK 3]
  \definition{v.}{introduzir; importar  (de fora para dentro) | inserir informações, programas, dados, sinais, etc. em uma máquina}
\end{entry}

\begin{entry}{熟}{shu2}{15}{⽕}[HSK 2]
  \definition{adj.}{maduro | cozido | feito | processado | familiar | qualificado | experiente | profundo}
\end{entry}

\begin{entry}{熟练}{shu2lian4}{15,8}{⽕、⽷}[HSK 4]
  \definition{adj.}{especializado; proficiente; qualificado; habilidoso}
\end{entry}

\begin{entry}{熟人}{shu2 ren2}{15,2}{⽕、⼈}[HSK 3]
  \definition[位]{s.}{amigo; conhecido}
\end{entry}

\begin{entry}{熟悉}{shu2xi1}{15,11}{⽕、⼼}
  \definition{v.}{conhecer bem | estar familiarizado com}
\end{entry}

\begin{entry}{属}{shu3}{12}{⼫}[HSK 3]
  \definition{s.}{categoria
gênero
membros da família; dependentes}
  \definition{v.}{estar sob; subordinado a | pertencer a | nascer no ano de (um dos doze animais do zodíaco)}
  \seeref{属}{zhu3}
\end{entry}

\begin{entry}{属于}{shu3yu2}{12,3}{⼫、⼆}[HSK 3]
  \definition{v.}{pertencer a; fazer parte de; ser classificado como}
\end{entry}

\begin{entry}{暑假}{shu3 jia4}{12,11}{⽇、⼈}[HSK 4]
  \definition[个]{s.}{férias de verão; feriado de verão; férias escolares de verão, na China, durante o sétimo e o oitavo meses do calendário gregoriano}
\end{entry}

\begin{entry}{黍}{shu3}{12}{⿉}[Kangxi 202]
  \definition{s.}{painço}
\end{entry}

\begin{entry}{数}{shu3}{13}{⽁}[HSK 2]
  \definition{v.}{contar
ser considerado excepcionalmente (bom, ruim, etc.)
enumerar; listar}
  \seeref{数}{shu4}
  \seeref{数}{shuo4}
\end{entry}

\begin{entry}{鼠}{shu3}{13}{⿏}[Kangxi 208]
  \definition[只]{s.}{rato; camundongo}
\end{entry}

\begin{entry}{薯}{shu3}{16}{⾋}
  \definition{s.}{batata | inhame}
\end{entry}

\begin{entry}{束}{shu4}{7}{⽊}[HSK 3]
  \definition*{s.}{sobrenome Shu}
  \definition{clas.}{para cachos, molhos, feixes, feixes de luz, etc.}
  \definition{s.}{monte; pacote; maço; feixe; cacho}
  \definition{v.}{atar; amarrar; vincular | controlar; restringir}
\end{entry}

\begin{entry}{束腰}{shu4yao1}{7,13}{⽊、⾁}
  \definition{s.}{cinto | cinta | cinturão}
\end{entry}

\begin{entry}{树}{shu4}{9}{⽊}[HSK 1]
  \definition[棵]{s.}{árvore}
  \definition{v.}{cultivar}
\end{entry}

\begin{entry}{树林}{shu4 lin2}{9,8}{⽊、⽊}[HSK 4]
  \definition{s.}{bosque; muitas árvores que crescem em fragmentos, menores que as florestas}
\end{entry}

\begin{entry}{树莓}{shu4mei2}{9,10}{⽊、⾋}
  \definition{s.}{framboesa}
\end{entry}

\begin{entry}{树木}{shu4mu4}{9,4}{⽊、⽊}
  \definition{s.}{árvore}
\end{entry}

\begin{entry}{树叶}{shu4ye4}{9,5}{⽊、⼝}[HSK 4]
  \definition[片,枚,堆]{s.}{folha; folhagem;}
\end{entry}

\begin{entry}{数}{shu4}{13}{⽁}
  \definition{num.}{vários | alguns}
  \definition{s.}{número | figura | destino}
  \seeref{数}{shu3}
  \seeref{数}{shuo4}
\end{entry}

\begin{entry}{数据}{shu4ju4}{13,11}{⽁、⼿}[HSK 4]
  \definition[些,个]{s.}{dados; valores com base nos quais são realizadas estatísticas, cálculos, pesquisas científicas ou projetos técnicos}
\end{entry}

\begin{entry}{数量}{shu4liang4}{13,12}{⽁、⾥}[HSK 3]
  \definition[个]{s.}{quantidade; quantum; quantia; magnitude; número}
\end{entry}

\begin{entry}{数码}{shu4ma3}{13,8}{⽁、⽯}[HSK 4]
  \definition{s.}{dígito; numeral; algarismo | número; quantidade (usado principalmente na linguagem falada)}
  \definition{v.}{digitalizar}
\end{entry}

\begin{entry}{数学}{shu4xue2}{13,8}{⽁、⼦}
  \definition{s.}{matemática (disciplina)}
\end{entry}

\begin{entry}{数字}{shu4zi4}{13,6}{⽁、⼦}[HSK 2]
  \definition{adj.}{digital}
  \definition[个]{s.}{dígito | figura | número | numeral | quantidade | montante}
\end{entry}

\begin{entry}{刷}{shua1}{8}{⼑}[HSK 4]
  \definition{s.}{escova; pincel | (onomatopéia) farfalhar; descreve o som de uma passagem rápida}
  \definition{v.}{escovar; esfregar; remover com uma escova | borrar; colar; aplicar com um pincel | eliminar; remover; limpar}
  \seeref{刷}{shua4}
\end{entry}

\begin{entry}{刷牙}{shua1ya2}{8,4}{⼑、⽛}[HSK 4]
  \definition{s.}{escovar os dentes}
\end{entry}

\begin{entry}{刷子}{shua1zi5}{8,3}{⼑、⼦}[HSK 4]
  \definition[把]{s.}{escova; escovão; utensílio feito de lã, fio de plástico, fio de metal, etc., para remover sujeira ou aplicar óleo de unção, etc., geralmente longo ou oval, alguns com alças}
\end{entry}

\begin{entry}{耍}{shua3}{9}{⽽}
  \definition{v.}{brincar com | empunhar | agir (legal, calmo, tranquilo, descolado, etc.) | exibir (uma habilidade, o temperamento de alguém, etc.)}
\end{entry}

\begin{entry}{耍赖}{shua3lai4}{9,13}{⽽、⾙}
  \definition{v.}{agir descaradamente | recusar -se a reconhecer que alguém perdeu o jogo ou fez uma promessa, etc. | agir como um idiota | agir como se algo nunca tivesse acontecido}
\end{entry}

\begin{entry}{刷}{shua4}{8}{⼑}
  \definition{v.}{selecionar}
  \seeref{刷}{shua1}
\end{entry}

\begin{entry}{摔}{shuai1}{14}{⼿}
  \definition{v.}{cair | cair e quebrar | partir}
\end{entry}

\begin{entry}{帅}{shuai4}{5}{⼱}[HSK 4]
  \definition*{s.}{sobrenome Shuai}
  \definition{adj.}{bonito; arrojado; elegante; inteligente}
  \definition{interj.}{Legal!}
  \definition[位,名]{s.}{comandante em chefe; o mais alto comandante do exército | comandante em chefe, a peça principal no xadrez chinês}
\end{entry}

\begin{entry}{帅哥}{shuai4 ge1}{5,10}{⼱、⼝}[HSK 4]
  \definition[个,位]{s.}{rapaz bonito; um garoto que é bonito e atraente na aparência}
\end{entry}

\begin{entry}{率先}{shuai4 xian1}{11,6}{⽞、⼉}[HSK 4]
  \definition{v.}{tomar a iniciativa de fazer algo; ser o primeiro a fazer algo; assumir a liderança}
\end{entry}

\begin{entry}{双}{shuang1}{4}{⼜}[HSK 3]
  \definition*{s.}{sobrenome Shuang}
  \definition{adj.}{dois; gêmeo; par; dual | par (número) | duplo; dublê; duplicata; cópia; sósia}
  \definition{clas.}{par}
\end{entry}

\begin{entry}{双层床}{shuang1ceng2chuang2}{4,7,7}{⼜、⼫、⼴}
  \definition{s.}{beliche}
\end{entry}

\begin{entry}{双打}{shuang1da3}{4,5}{⼜、⼿}
  \definition[场]{s.}{duplas (em esportes)}
\end{entry}

\begin{entry}{双方}{shuang1fang1}{4,4}{⼜、⽅}[HSK 3]
  \definition{s.}{ambos os lados; as duas partes}
\end{entry}

\begin{entry}{双方同意}{shuang1fang1tong2yi4}{4,4,6,13}{⼜、⽅、⼝、⼼}
  \definition{s.}{acordo bilateral}
\end{entry}

\begin{entry}{霜}{shuang1}{17}{⾬}
  \definition{s.}{geada | pó branco ou creme espalhado por uma superfície | glacê | creme de pele}
\end{entry}

\begin{entry}{爽}{shuang3}{11}{⽘}
  \definition{adj.}{claro; nítido; brilhante |franco; de coração aberto; direto | relaxado; confortável}
  \definition{v.}{desviar; afastar | tornar confortável; ficar confortável}
\end{entry}

\begin{entry}{谁}{shui2}{10}{⾔}[HSK 1]
  \definition{pron.}{quem?}
  \seeref{谁}{shei2}
\end{entry}

\begin{entry}{水}{shui3}{4}{⽔}[HSK 1][Kangxi 85]
  \definition*{s.}{sobrenome Shui}
  \definition{clas.}{para número de lavagens}
  \definition{s.}{água | líquido | encargos ou receitas adicionais}
\end{entry}

\begin{entry}{水边}{shui3bian1}{4,5}{⽔、⾡}
  \definition{s.}{beira d'água | beira-mar | costa (de mar, lago ou rio)}
\end{entry}

\begin{entry}{水波}{shui3bo1}{4,8}{⽔、⽔}
  \definition{s.}{ondulação (na água) | onda}
\end{entry}

\begin{entry}{水槽}{shui3cao2}{4,15}{⽔、⽊}
  \definition{s.}{pia (de cozinha)}
\end{entry}

\begin{entry}{水果}{shui3guo3}{4,8}{⽔、⽊}[HSK 1]
  \definition[个]{s.}{fruta}
\end{entry}

\begin{entry}{水饺}{shui3jiao3}{4,9}{⽔、⾷}
  \definition{s.}{\emph{dumplings} | pastéis chineses cozidos}
\end{entry}

\begin{entry}{水灵}{shui3ling2}{4,7}{⽔、⽕}
  \definition{adj.}{cheio de vida (sobre uma pessoa, etc.) | úmido e brilhante (sobre os olhos) | fresco (sobre frutas, etc.) | brilhante | aparência saudável}
\end{entry}

\begin{entry}{水路}{shui3lu4}{4,13}{⽔、⾜}
  \definition{s.}{hidrovia}
\end{entry}

\begin{entry}{水培}{shui3pei2}{4,11}{⽔、⼟}
  \definition{v.}{cultivar plantas hidroponicamente}
\end{entry}

\begin{entry}{水平}{shui3ping2}{4,5}{⽔、⼲}[HSK 2]
  \definition{s.}{nível (de realização, etc.) | padrão | nível horizontal}
\end{entry}

\begin{entry}{水平尺}{shui3ping2chi3}{4,5,4}{⽔、⼲、⼫}
  \definition{s.}{nível espiritual}
\end{entry}

\begin{entry}{水平度}{shui3ping2 du4}{4,5,9}{⽔、⼲、⼴}
  \definition{s.}{nivelamento}
\end{entry}

\begin{entry}{水平面}{shui3ping2mian4}{4,5,9}{⽔、⼲、⾯}
  \definition{s.}{plano horizontal | nível-da-água | superfície horizontal}
\end{entry}

\begin{entry}{水平视差}{shui3ping2 shi4cha1}{4,5,8,9}{⽔、⼲、⾒、⼯}
  \definition{s.}{paralaxe horizontal}
\end{entry}

\begin{entry}{水平仪}{shui3ping2yi2}{4,5,5}{⽔、⼲、⼈}
  \definition{s.}{nível (dispositivo para determinar horizontal) | nível espiritual | nível de topógrafo}
\end{entry}

\begin{entry}{水平以下}{shui3ping2 yi3xia4}{4,5,4,3}{⽔、⼲、⼈、⼀}
  \definition{s.}{sub-nível}
\end{entry}

\begin{entry}{水平轴}{shui3ping2zhou2}{4,5,9}{⽔、⼲、⾞}
  \definition{s.}{eixo horizontal}
\end{entry}

\begin{entry}{水瓶}{shui3 ping2}{4,10}{⽔、⽡}
  \definition{s.}{garrada de água}
\end{entry}

\begin{entry}{水豚}{shui3tun2}{4,11}{⽔、⾗}
  \definition{s.}{capivara}
\end{entry}

\begin{entry}{水污染}{shui3wu1ran3}{4,6,9}{⽔、⽔、⽊}
  \definition{s.}{poluição da água}
\end{entry}

\begin{entry}{说}{shui4}{9}{⾔}
  \definition{v.}{persuadir}
  \seeref{说}{shuo1}
\end{entry}

\begin{entry}{税}{shui4}{12}{⽲}
  \definition{s.}{taxas | impostos}
\end{entry}

\begin{entry}{睡}{shui4}{13}{⽬}[HSK 1]
  \definition{v.}{dormir}
\end{entry}

\begin{entry}{睡觉}{shui4jiao4}{13,9}{⽬、⾒}[HSK 1]
  \definition{v.+compl.}{ir para a cama | dormir | deitar-se}
\end{entry}

\begin{entry}{睡懒觉}{shui4lan3jiao4}{13,16,9}{⽬、⼼、⾒}
  \definition{v.}{levantar-se tarde | passar o tempo a dormir}
\end{entry}

\begin{entry}{睡衣}{shui4yi1}{13,6}{⽬、⾐}
  \definition{s.}{pijamas | roupas de dormir}
\end{entry}

\begin{entry}{睡着}{shui4 zhao2}{13,11}{⽬、⽬}[HSK 4]
  \definition{v.}{dormir; adormecer; cair no sono}
\end{entry}

\begin{entry}{顺}{shun4}{9}{⾴}
  \definition{adj.}{correr bem | favorável}
\end{entry}

\begin{entry}{顺便}{shun4bian4}{9,9}{⾴、⼈}
  \definition{adv.}{convenientemente | de passagem | sem muito esforço extra}
\end{entry}

\begin{entry}{顺畅}{shun4chang4}{9,8}{⾴、⽥}
  \definition{adj.}{liso e sem obstáculos | fluente}
\end{entry}

\begin{entry}{顺从}{shun4cong2}{9,4}{⾴、⼈}
  \definition{v.}{obedecer | submeter-se}
\end{entry}

\begin{entry}{顺当}{shun4dang5}{9,6}{⾴、⼹}
  \definition{adv.}{suavemente}
\end{entry}

\begin{entry}{顺耳}{shun4'er3}{9,6}{⾴、⽿}
  \definition{adj.}{agradável ao ouvido}
\end{entry}

\begin{entry}{顺境}{shun4jing4}{9,14}{⾴、⼟}
  \definition{s.}{circunstâncias favoráveis}
\end{entry}

\begin{entry}{顺利}{shun4li4}{9,7}{⾴、⼑}[HSK 2]
  \definition{adv.}{suavemente | sem problemas}
\end{entry}

\begin{entry}{顺水}{shun4shui3}{9,4}{⾴、⽔}
  \definition{v.}{ir com o fluxo}
\end{entry}

\begin{entry}{顺心}{shun4xin1}{9,4}{⾴、⼼}
  \definition{adj.}{satisfatório | satisfeito}
\end{entry}

\begin{entry}{顺序}{shun4xu4}{9,7}{⾴、⼴}[HSK 4]
  \definition{adv.}{por sua vez; na ordem correta; na devida ordem; na ordem adequada; na ordem apropriada}
  \definition[个]{s.}{ordem; sequência; sucessão; subsequência; sequência simples; ordem de prioridade}
\end{entry}

\begin{entry}{顺叙}{shun4xu4}{9,9}{⾴、⼜}
  \definition{s.}{narrativa cronológica}
\end{entry}

\begin{entry}{顺延}{shun4yan2}{9,6}{⾴、⼵}
  \definition{v.}{adiar | procrastinar}
\end{entry}

\begin{entry}{顺眼}{shun4yan3}{9,11}{⾴、⽬}
  \definition{adj.}{agradável aos olhos}
\end{entry}

\begin{entry}{顺嘴}{shun4zui3}{9,16}{⾴、⼝}
  \definition{v.}{deixar escapar (sem pensar) | ler suavemente (texto) | adequar-se  ao gosto (comida)}
\end{entry}

\begin{entry}{说}{shuo1}{9}{⾔}[HSK 1]
  \definition{s.}{uma teoria (normalmente o último caractere, como em 日心说, teoria heliocêntrica)}
  \definition{v.}{falar | dizer | explicar | contar}
  \seeref{说}{shui4}
\end{entry}

\begin{entry}{说不定}{shuo1bu5ding4}{9,4,8}{⾔、⼀、⼧}[HSK 4]
  \definition{adv.}{talvez; indica uma estimativa, possivelmente, provavelmente}
  \definition{v.}{não ter certeza; não estar certo; ser impreciso}
\end{entry}

\begin{entry}{说服}{shuo1fu2}{9,8}{⾔、⽉}[HSK 4]
  \definition{v.}{persuadir; convencer; convencer a outra parte com palavras bem fundamentadas}
\end{entry}

\begin{entry}{说好}{shuo1hao3}{9,6}{⾔、⼥}
  \definition{v.}{chegar a um acordo | concluir negociações}
\end{entry}

\begin{entry}{说话}{shuo1 hua4}{9,8}{⾔、⾔}[HSK 1]
  \definition{adv.}{imediatamente | em um minuto}
  \definition{v.}{falar | dizer | bater-papo | conversar | fofocar}
\end{entry}

\begin{entry}{说谎}{shuo1huang3}{9,11}{⾔、⾔}
  \definition{v.+compl.}{mentir | contar uma mentira}
\end{entry}

\begin{entry}{说理}{shuo1li3}{9,11}{⾔、⽟}
  \definition{v.}{racionalizar | discutir logicamente}
\end{entry}

\begin{entry}{说明}{shuo1ming2}{9,8}{⾔、⽇}[HSK 2]
  \definition[本,个]{s.}{legenda | instrução | explicação}
  \definition{v.}{mostrar | explicar | ilustrar | indicar | provar | demonstrar}
\end{entry}

\begin{entry}{说完}{shuo1-wan2}{9,7}{⾔、⼧}
  \definition{expr.}{acabar/terminar palavras}
\end{entry}

\begin{entry}{数}{shuo4}{13}{⽁}
  \definition{adv.}{frequentemente | repetidamente}
  \seeref{数}{shu3}
  \seeref{数}{shu4}
\end{entry}

\begin{entry}{丝}{si1}{5}{⼀}
  \definition{adj.}{filiforme | delgado como um fio | que se assemelha a um fio}
  \definition{clas.}{um traço (de fumaça, etc.) | um pouquinho, etc.}
  \definition{s.}{seda | (cozinha) pedaços ou tiras de julienne, tiras cortadas finas}
\end{entry}

\begin{entry}{司机}{si1ji1}{5,6}{⼝、⽊}[HSK 2]
  \definition{s.}{condutor | motorista | chofer}
\end{entry}

\begin{entry}{私人}{si1ren2}{7,2}{⽲、⼈}
  \definition{adj.}{privado | interpessoal}
  \definition[些]{s.}{alguém com quem se tem um relacionamento pessoal próximo}
\end{entry}

\begin{entry}{私人信件}{si1ren2 xin4jian4}{7,2,9,6}{⽲、⼈、⼈、⼈}
  \definition{s.}{carta pessoal}
\end{entry}

\begin{entry}{私人钥匙}{si1ren2yao4shi5}{7,2,9,11}{⽲、⼈、⾦、⼔}
  \definition{s.}{(criptografia) chave privada}
\end{entry}

\begin{entry}{私人诊所}{si1ren2 zhen3suo3}{7,2,7,8}{⽲、⼈、⾔、⼾}
  \definition[些]{s.}{clínica privada}
\end{entry}

\begin{entry}{私生活}{si1sheng1huo2}{7,5,9}{⽲、⽣、⽔}
  \definition{s.}{vida privada}
\end{entry}

\begin{entry}{私自}{si1zi4}{7,6}{⽲、⾃}
  \definition{adj.}{privado | pessoal}
  \definition{adv.}{secretamente | sem aprovação explícita}
\end{entry}

\begin{entry}{思考}{si1kao3}{9,6}{⼼、⽼}[HSK 4]
  \definition{v.}{pensar; ponderar; considerar; deliberar; envolver-se em atividades de pensamento, como análise, síntese, julgamento, raciocínio e generalização}
\end{entry}

\begin{entry}{思想}{si1xiang3}{9,13}{⼼、⼼}[HSK 3]
  \definition[个]{s.}{reflexão; pensamento; ideologia | ideia}
\end{entry}

\begin{entry}{斯巴达}{si1ba1da2}{12,4,6}{⽄、⼰、⾡}
  \definition*{s.}{Esparta}
\end{entry}

\begin{entry}{死}{si3}{6}{⽍}[HSK 3]
  \definition{adj.}{até a morte | implacável; mortal | fixo; rígido; inflexível | intransitável; fechado}
  \definition{adv.}{extremamente; até a morte}
  \definition{v.}{morrer; falecer}
\end{entry}

\begin{entry}{死亡}{si3wang2}{6,3}{⽍、⼇}
  \definition{s.}{morte}
  \definition{v.}{morrer}
\end{entry}

\begin{entry}{四}{si4}{5}{⼞}[HSK 1]
  \definition{num.}{quatro; 4}
\end{entry}

\begin{entry}{四川}{si4chuan1}{5,3}{⼞、⼮}
  \definition*{s.}{Sichuan}
\end{entry}

\begin{entry}{四季分明}{si4ji4-fen1ming2}{5,8,4,8}{⼞、⼦、⼑、⽇}
  \definition{expr.}{as quatro estações são muito distintas}
\end{entry}

\begin{entry}{四季如春}{si4ji4-ru2chun1}{5,8,6,9}{⼞、⼦、⼥、⽇}
  \definition{expr.}{é primavera todo o ano | clima favorável durante todo o ano | quatro estações como a primavera}
\end{entry}

\begin{entry}{似曾相识}{si4ceng2xiang1shi2}{6,12,9,7}{⼈、⽈、⽬、⾔}
  \definition{s.}{\emph{déjà vu} (a experiência de ver exatamente a mesma situação pela segunda vez) | situação aparentemente familiar}
\end{entry}

\begin{entry}{似乎}{si4hu1}{6,5}{⼈、⼃}[HSK 4]
  \definition{adv.}{como se; aparentemente; se parece como}
\end{entry}

\begin{entry}{寺}{si4}{6}{⼨}
  \definition{s.}{Templo Budista | Mesquita}
\end{entry}

\begin{entry}{寺庙}{si4miao4}{6,8}{⼨、⼴}
  \definition{s.}{templo | mosteiro | santuário}
\end{entry}

\begin{entry}{肆}{si4}{13}{⾀}
  \definition*{s.}{sobrenome Si}
  \definition{adj.}{arbitrário; desenfreado; sem limites; descuidado; imprudente}
  \definition{num.}{quatro (usado para o numeral 四 em cheques, etc., para evitar erros ou alterações)}
  \definition{s.}{loja}
  \seealsoref{四}{si4}
\end{entry}

\begin{entry}{松}{song1}{8}{⽊}[HSK 4]
  \definition*{s.}{sobrenome Song}
  \definition{adj.}{solto; frouxo; folgado | abastado; rico; próspero | leve e crocante; macio}
  \definition[棵]{s.}{pinheiro | fio de carne seca; carne moída seca; alimentos macios ou quebradiços |}
  \definition{v.}{afrouxar; relaxar; soltar}
\end{entry}

\begin{entry}{松木}{song1mu4}{8,4}{⽊、⽊}
  \definition{s.}{pinheiro}
\end{entry}

\begin{entry}{松树}{song1 shu4}{8,9}{⽊、⽊}[HSK 4]
  \definition[棵]{s.}{pinheiro; conífera comum, geralmente com folhas longas e pontiagudas e cones lenhosos}
\end{entry}

\begin{entry}{宋}{song4}{7}{⼧}
  \definition*{s.}{sobrenome Song}
  \definition{s.}{Dinastia Song (960-1279) | Song das dinastias do sul (420-479)}
\end{entry}

\begin{entry}{送}{song4}{9}{⾡}[HSK 1]
  \definition{v.}{distribuir | entregar | dar | oferecer (alguma coisa como presente) | enviar | remeter}
\end{entry}

\begin{entry}{送到}{song4 dao4}{9,8}{⾡、⼑}[HSK 2]
  \definition{v.}{enviar para (lugar)}
\end{entry}

\begin{entry}{送给}{song4 gei3}{9,9}{⾡、⽷}[HSK 2]
  \definition{v.}{dar a (alguém ou organização)}
\end{entry}

\begin{entry}{㮸}{song4}{14}{⽊}
  \variantof{送}
\end{entry}

\begin{entry}{苏格兰}{su1ge2lan2}{7,10,5}{⾋、⽊、⼋}
  \definition*{s.}{Escócia}
\end{entry}

\begin{entry}{速度}{su4du4}{10,9}{⾡、⼴}[HSK 3]
  \definition[个,种]{s.}{velocidade; taxa; ritmo; andamento | velocidade; rapidez}
\end{entry}

\begin{entry}{宿舍}{su4she4}{11,8}{⼧、⾆}
  \definition[间]{s.}{dormitório | quarto de dormir | hostel}
\end{entry}

\begin{entry}{塑料}{su4 liao4}{13,10}{⼟、⽃}[HSK 4]
  \definition[块,种]{s.}{plástico; compostos de polímeros feitos de resinas naturais ou sintéticas como componente principal}
\end{entry}

\begin{entry}{塑料袋}{su4liao4dai4}{13,10,11}{⼟、⽃、⾐}[HSK 4]
  \definition{s.}{saco plástico; sacola de plástico}
\end{entry}

\begin{entry}{痠}{suan1}{12}{⽧}
  \definition{v.}{doer | estar dolorido}
  \variantof{酸}
\end{entry}

\begin{entry}{酸}{suan1}{14}{⾣}[HSK 4]
  \definition{adj.}{azedo; ácido | aflito; angustiado; doente do coração | pedante; descreve uma pessoa que finge ser culta e também descreve uma pessoa que é muito inflexível com suas próprias ideias e não está disposta a mudá-las para atender às exigências da época, é usado principalmente para satirizar intelectuais que fingem ser capazes de escrever poemas e artigos | ciumento; invejoso; sentimentos desconfortáveis porque outra pessoa é melhor do que você e, em geral, também apresenta comportamento hostil}
  \definition{s.}{ácido; produto químico que tem um sabor ácido quando misturado com água}
  \definition{v.}{estar dolorido (devido à fadiga ou doença); descreve a sensação de não ter força muscular e um pouco de dor por estar doente ou muito cansado}
\end{entry}

\begin{entry}{酸辣汤}{suan1la4tang1}{14,14,6}{⾣、⾟、⽔}
  \definition{s.}{sopa avinagrada e picante (prato)}
\end{entry}

\begin{entry}{酸奶}{suan1 nai3}{14,5}{⾣、⼥}[HSK 4]
  \definition[瓶,杯,盒,袋]{s.}{iogurte; produto lácteo fermentado por bactérias de ácido láctico}
\end{entry}

\begin{entry}{算}{suan4}{14}{⽵}[HSK 2]
  \definition{adv.}{finalmente; no fim; finalmente}
  \definition{v.}{calcular | contar | computar | figurar | incluir | planejar | calcular | pensar | supor | considerar | considerar como | contar como | carregar peso | deixar estar | deixar passar}
\end{entry}

\begin{entry}{算了}{suan4le5}{14,2}{⽵、⼅}
  \definition{v.}{deixar | deixe estar | deixe passar | esqueça isso}
\end{entry}

\begin{entry}{算命}{suan4ming4}{14,8}{⽵、⼝}
  \definition{s.}{cartomante}
  \definition{v.}{ler a sorte | fazer advinhações}
\end{entry}

\begin{entry}{尿}{sui1}{7}{⼫}
  \definition{s.}{(coloquial) urina}
  \seeref{尿}{niao4}
\end{entry}

\begin{entry}{虽}{sui1}{9}{⾍}
  \definition{conj.}{no entanto | embora | mesmo se/embora}
\end{entry}

\begin{entry}{虽然}{sui1 ran2}{9,12}{⾍、⽕}[HSK 2]
  \definition{conj.}{embora (frequentemente usado correlativamente com 可是, 但是, etc); geralmente é usado no início de uma frase para indicar que o fato anterior foi reconhecido, mas não mudará o que acontecerá em seguida}
  \seealsoref{但是}{dan4 shi4}
  \seealsoref{可是}{ke3shi4}
\end{entry}

\begin{entry}{随}{sui2}{11}{⾩}[HSK 3]
  \definition*{s.}{sobrenome Sui}
  \definition{v.}{seguir | cumprir com; adaptar-se a | deixar (alguém fazer o que ele gosta) | parecer-se com; assemelhar-se a}
\end{entry}

\begin{entry}{随便}{sui2bian4}{11,9}{⾩、⼈}[HSK 2]
  \definition{adj.}{à vontade | como queira | como desejar | casual | negligente | devasso}
  \definition{adv.}{aleatoriamente}
\end{entry}

\begin{entry}{随处}{sui2chu4}{11,5}{⾩、⼡}
  \definition{adv.}{em qualquer lugar}
\end{entry}

\begin{entry}{随地}{sui2di4}{11,6}{⾩、⼟}
  \definition{adv.}{qualquer lugar | todo lugar}
\end{entry}

\begin{entry}{随机存取存储器}{sui2ji1cun2qu3cun2chu3qi4}{11,6,6,8,6,12,16}{⾩、⽊、⼦、⼜、⼦、⼈、⼝}
  \definition{s.}{RAM (\emph{random access memory})}
  \seealsoref{内存}{nei4cun2}
  \seealsoref{随机存取记忆体}{sui2ji1cun2qu3ji4yi4ti3}
\end{entry}

\begin{entry}{随机存取记忆体}{sui2ji1cun2qu3ji4yi4ti3}{11,6,6,8,5,4,7}{⾩、⽊、⼦、⼜、⾔、⼼、⼈}
  \definition{s.}{RAM (\emph{random access memory})}
  \seealsoref{内存}{nei4cun2}
  \seealsoref{随机存取存储器}{sui2ji1cun2qu3cun2chu3qi4}
\end{entry}

\begin{entry}{随时}{sui2shi2}{11,7}{⾩、⽇}[HSK 2]
  \definition{adv.}{a qualquer momento | sempre que necessário}
\end{entry}

\begin{entry}{随手}{sui2shou3}{11,4}{⾩、⼿}[HSK 4]
  \definition{adv.}{convenientemente; sem problemas adicionais; casualmente}
\end{entry}

\begin{entry}{岁}{sui4}{6}{⼭}[HSK 1]
  \definition{clas.}{para anos (de idade)}
  \definition{s.}{idade | ano (idade ou colheita)}
\end{entry}

\begin{entry}{碎}{sui4}{13}{⽯}
  \definition{adj.}{quebrado | fragmentado | espalhado | tagarela}
  \definition{v.}{(transitivo ou intransitivo) quebrar em pedaços, quebrar, desmoronar}
\end{entry}

\begin{entry}{隧道}{sui4dao4}{14,12}{⾩、⾡}
  \definition{s.}{túnel}
\end{entry}

\begin{entry}{孙女}{sun1nv3}{6,3}{⼦、⼥}[HSK 4]
  \definition{s.}{filha do filho; neta}
\end{entry}

\begin{entry}{孙武}{sun1wu3}{6,8}{⼦、⽌}
  \definition*{s.}{Sun Wu, também conhecido por Sun Tzu (孙子), general, estrategista e filósofo autor do ``Arte da Guerra'' (孙子兵法)}
  \seeref{孙子}{sun1zi3}
  \seealsoref{孙子兵法}{sun1zi3 bing1fa3}
\end{entry}

\begin{entry}{孙子}{sun1zi3}{6,3}{⼦、⼦}
  \definition*{s.}{Sun Tzu, também conhecido por Sun Wu (孙武), general, estrategista e filósofo autor do ``Arte da Guerra'' (孙子兵法)}
  \seeref{孙武}{sun1wu3}
  \seeref{孙子}{sun1zi5}
  \seealsoref{孙子兵法}{sun1zi3 bing1fa3}
\end{entry}

\begin{entry}{孙子兵法}{sun1zi3 bing1fa3}{6,3,7,8}{⼦、⼦、⼋、⽔}
  \definition*{s.}{``Arte da Guerra'', escrito por Sun Tzu (孫子)}
  \seeref{孙武}{sun1wu3}
  \seeref{孙子}{sun1zi3}
\end{entry}

\begin{entry}{孙子}{sun1zi5}{6,3}{⼦、⼦}[HSK 4]
  \definition{s.}{filho do filho; neto}
  \seeref{孙子}{sun1zi3}
\end{entry}

\begin{entry}{笋}{sun3}{10}{⽵}
  \definition{s.}{broto de bambu}
\end{entry}

\begin{entry}{缩短}{suo1duan3}{14,12}{⽷、⽮}[HSK 4]
  \definition{v.}{encurtar; reduzir; diminuir}
\end{entry}

\begin{entry}{缩小}{suo1 xiao3}{14,3}{⽷、⼩}[HSK 4]
  \definition{v.}{reduzir, estreitar, encolher;  tornar menor (em oposição a ``放大'')}
  \seealsoref{放大}{fang4da4}
\end{entry}

\begin{entry}{缩影卡片}{suo1ying3 ka3pian4}{14,15,5,4}{⽷、⼺、⼘、⽚}
  \definition{s.}{cartão em miniatura}
\end{entry}

\begin{entry}{所}{suo3}{8}{⼾}[HSK 3]
  \definition*{s.}{sobrenome Suo | usado como nome de uma agência ou outro escritório}
  \definition{clas.}{para casas, etc.}
  \definition{part.}{usado com ``为'' ou ``被'' para indicar voz passiva | usado com verbos, representa a entidade que recebe a ação | usado em conjunto com verbos, seguido de um substantivo que recebe a ação | usado com verbos, ``者'' ou ``的'' depois do verbo representa a entidade que recebe a ação}
  \definition{s.}{lugar}
\end{entry}

\begin{entry}{所长}{suo3 chang2}{8,4}{⼾、⾧}
  \definition{s.}{aquilo em que alguém é bom; o ponto forte de alguém; o forte de alguém}
  \seealsoref{所长}{suo3 zhang3}
\end{entry}

\begin{entry}{所以}{suo3 yi3}{8,4}{⼾、⼈}[HSK 2]
  \definition{adv.}{portanto | então | como resultado}
  \definition{conj.}{por isso | como resultado | a razão porque}
\end{entry}

\begin{entry}{所有}{suo3you3}{8,6}{⼾、⽉}[HSK 2]
  \definition{adj.}{todo | tudo}
  \definition{s.}{posses}
  \definition{v.}{possuir | ser dono de}
\end{entry}

\begin{entry}{所长}{suo3 zhang3}{8,4}{⼾、⾧}[HSK 3]
  \definition{s.}{chefe de um instituto, etc. | superintendente}
  \seeref{所长}{suo3 chang2}
\end{entry}

\begin{entry}{索性}{suo3xing4}{10,8}{⽷、⼼}
  \definition{adv.}{poderia muito bem | simplesmente | apenas}
\end{entry}

%%%%% EOF %%%%%


%%%
%%% T
%%%

\section*{T}\addcontentsline{toc}{section}{T}

\begin{entry}{T-恤}{t5 xu4}{∅,9}{∅、⼼}
  \definition{s.}{camiseta | pulôver | suéter}
\end{entry}

\begin{entry}{他}{ta1}{5}{⼈}[HSK 1]
  \definition{pron.}{ele | outro; referindo-se a outro; diferente | usado após o verbo, indica referência vaga | alguém; todos; usado em conjunto com 你, significa qualquer pessoa ou muitas pessoas | em outro lugar; outro lugar}
  \seealsoref{你}{ni3}
  \seealsoref{怹}{tan1}
\end{entry}

\begin{entry}{他的}{ta1 de5}{5,8}{⼈、⽩}
  \definition{pron.}{dele}
\end{entry}

\begin{entry}{他妈的}{ta1ma1de5}{5,6,8}{⼈、⼥、⽩}
  \definition{interj.}{Dane-se! | Foda-se!}
\end{entry}

\begin{entry}{他们}{ta1men5}{5,5}{⼈、⼈}[HSK 1]
  \definition{pron.}{eles}
\end{entry}

\begin{entry}{他们的}{ta1men5 de5}{5,5,8}{⼈、⼈、⽩}
  \definition{pron.}{deles}
\end{entry}

\begin{entry}{它}{ta1}{5}{⼧}[HSK 2]
  \definition*{s.}{Sobrenome Ta}
  \definition{pron.}{ele; referência a algo além da pessoa (para objetos inanimados) | ele; usado após o verbo, indica referência vaga}
\end{entry}

\begin{entry}{它们}{ta1 men5}{5,5}{⼧、⼈}[HSK 2]
  \definition{pron.}{eles; usado para se referir a mais de uma coisa não humana; geralmente se refere a animais, objetos ou conceitos abstratos}
\end{entry}

\begin{entry}{她}{ta1}{6}{⼥}[HSK 1]
  \definition{pron.}{ela | ela; referir-se a coisas que se ama ou aprecia, como a pátria, a bandeira nacional, etc.}
\end{entry}

\begin{entry}{她的}{ta1 de5}{6,8}{⼥、⽩}
  \definition{pron.}{dela}
\end{entry}

\begin{entry}{她们}{ta1men5}{6,5}{⼥、⼈}[HSK 1]
  \definition{pron.}{elas; referindo-se a várias mulheres: em textos escritos, use 她们 quando todas as pessoas forem mulheres e 他们 quando houver homens e mulheres}
  \seealsoref{他们}{ta1men5}
\end{entry}

\begin{entry}{她们的}{ta1men5 de5}{6,5,8}{⼥、⼈、⽩}
  \definition{pron.}{delas}
\end{entry}

\begin{entry}{踏}{ta1}{15}{⾜}
  \definition{part.}{Caracter formador de palavras}
  \seeref{踏}{ta4}
\end{entry}

\begin{entry}{塔}{ta3}{12}{⼟}[HSK 6]
  \definition*{s.}{Sobrenome Ta}
  \definition[个,座]{s.}{pagode budista; pagode | torre | (química) coluna; torre}[蒸馏塔___torre de destilação]
\end{entry}

\begin{entry}{踏}{ta4}{15}{⾜}[HSK 6]
  \definition{v.}{por os pés em; pisar em; esmagar com o pé | fazer uma investigação ou levantamento no local}
  \seeref{踏}{ta1}
\end{entry}

\begin{entry}{踏板}{ta4ban3}{15,8}{⾜、⽊}
  \definition{s.}{pedal (em um carro, em um piano, etc.) |  apoio para os pés | estribo}
\end{entry}

\begin{entry}{台}{tai2}{5}{⼝}[HSK 3]
  \definition*{s.}{Sobrenome Tai}
  \definition{clas.}{usado para certas máquinas, aparelhos, instrumentos, etc | usado para uma performance completa, como drama, música e dança}
  \definition{s.}{torre | plataforma; palco | suporte; pedestal | qualquer coisa em forma de plataforma ou palco | mesa; escrivaninha | estação de transmissão; refere-se a estações de rádio | um serviço telefônico especial; refere-se à estação telefônica | ``seu'' (um termo respeitoso usado antigamente para se dirigir a alguém) | tufão}
\end{entry}

\begin{entry}{台风}{tai2feng1}{5,4}{⼝、⾵}[HSK 5]
  \definition[场,阵,级]{s.}{tufão; classificação de um ciclone tropical ocorrido no oeste do Pacífico Norte | postura; presença de palco; comportamento ou estilo que os atores demonstram no palco}
\end{entry}

\begin{entry}{台阶}{tai2jie1}{5,6}{⼝、⾩}[HSK 4]
  \definition[个,级]{s.}{escada; escadaria | passos; metáfora para uma maneira ou oportunidade de evitar constrangimentos causados ​​por um impasse | nova fase; novo nível; novo patamar; metáfora para novas conquistas ou novos patamares alcançados no estudo ou no trabalho}
\end{entry}

\begin{entry}{台上}{tai2 shang4}{5,3}{⼝、⼀}[HSK 4]
  \definition{s.}{no palco}
\end{entry}

\begin{entry}{台下}{tai2xia4}{5,3}{⼝、⼀}
  \definition{s.}{platéia | fora do palco}
\end{entry}

\begin{entry}{抬}{tai2}{8}{⼿}[HSK 5]
  \definition{clas.}{para objetos que precisam ser carregados por pessoas quando transportados (por exemplo, móveis)}
  \definition{v.}{levantar; elevar; puxar para cima | (por duas ou mais pessoas) carregar; transportar; duas ou mais pessoas carregando algo com as mãos ou nos ombros | discutir, debater (geralmente sem sentido ou sem importância)}
\end{entry}

\begin{entry}{抬杠}{tai2gang4}{8,7}{⼿、⽊}
  \definition{v.+compl.}{discutir pelo prazer em discutir | discutir obstinadamente | brigar}
\end{entry}

\begin{entry}{抬头}{tai2 tou2}{8,5}{⼿、⼤}[HSK 5]
  \definition{s.}{(em recibos, contas, etc.) nome do comprador ou beneficiário, ou espaço para preencher esse nome | nome do comprador ou beneficiário; refere-se ao cabeçalho do documento ou da fatura}
  \definition{v.}{levantar a cabeça | ganhar terreno; olhar para cima; subir | começar uma nova linha, como sinal de respeito, ao mencionar o destinatário em cartas, correspondência oficial, etc.}
\end{entry}

\begin{entry}{太}{tai4}{4}{⼤}[HSK 1]
  \definition*{s.}{Sobrenome Tai}
  \definition{adj.}{mais alto; maior; mais distante | maior; extremo | bisavô; mais velho ou mais antigo; o de maior posição social ou hierarquia}
  \definition{adv.}{demais; expressa um grau excessivo (usado principalmente para coisas indesejáveis) | muito; extremamente; excessivamente; indica um grau extremamente elevado | muito; usado após o advérbio negativo 不, enfraquece o grau de negação e contém um tom diplomático}
\end{entry}

\begin{entry}{太极拳}{tai4ji2quan2}{4,7,10}{⼤、⽊、⼿}
  \definition*{s.}{Tai Chi Chuan, Taiji, T'aichi ou T'aichichuan; forma tradicional de exercício físico ou relaxamento}
\end{entry}

\begin{entry}{太空}{tai4kong1}{4,8}{⼤、⽳}[HSK 5]
  \definition[把]{s.}{firmamento; espaço sideral; espaço além da atmosfera terrestre; o céu vasto e infinito}
\end{entry}

\begin{entry}{太平洋}{tai4ping2 yang2}{4,5,9}{⼤、⼲、⽔}
  \definition*{s.}{Oceano Pacífico}
\end{entry}

\begin{entry}{太太}{tai4tai5}{4,4}{⼤、⼤}[HSK 2]
  \definition[位,名,个,些]{s.}{senhora; madame; títulos para mulheres casadas | esposa; senhora; madame; referir-se à própria esposa ou à esposa de outra pessoa}
\end{entry}

\begin{entry}{太阳窗}{tai4yang2chuang1}{4,6,12}{⼤、⾩、⽳}
  \definition{s.}{teto solar (de veículos)}
\end{entry}

\begin{entry}{太阳灯}{tai4yang2deng1}{4,6,6}{⼤、⾩、⽕}
  \definition{s.}{lâmpada solar (com células fotovoltaicas)}
\end{entry}

\begin{entry}{太阳风}{tai4yang2feng1}{4,6,4}{⼤、⾩、⾵}
  \definition{s.}{vento solar}
\end{entry}

\begin{entry}{太阳镜}{tai4yang2jing4}{4,6,16}{⼤、⾩、⾦}
  \definition{s.}{óculos de sol}
\end{entry}

\begin{entry}{太阳日}{tai4yang2ri4}{4,6,4}{⼤、⾩、⽇}
  \definition{s.}{dia solar}
\end{entry}

\begin{entry}{太阳穴}{tai4yang2xue2}{4,6,5}{⼤、⾩、⽳}
  \definition{s.}{têmpora (nas laterais da cabeça humana)}
\end{entry}

\begin{entry}{太阳翼}{tai4yang2yi4}{4,6,17}{⼤、⾩、⽻}
  \definition{s.}{painel solar}
\end{entry}

\begin{entry}{太阳雨}{tai4yang2yu3}{4,6,8}{⼤、⾩、⾬}
  \definition{s.}{banho de sol}
\end{entry}

\begin{entry}{太阳}{tai4yang5}{4,6}{⼤、⾩}[HSK 2]
  \definition[个,轮,枚,颗,盏]{s.}{o Sol | luz do sol; luz solar}
\end{entry}

\begin{entry}{态}{tai4}{8}{⼼}
  \definition{s.}{forma; aparência; condição | (física) estado | (linguística) voz}[气态___estado gasoso | 被动态___voz passiva]
\end{entry}

\begin{entry}{态度}{tai4du5}{8,9}{⼼、⼴}[HSK 2]
  \definition[种,个]{s.}{maneira; comportamento; atitude; comportamento e expressão facial das pessoas | atitude; abordagem; opinião sobre o assunto e medidas tomadas}
\end{entry}

\begin{entry}{贪}{tan1}{8}{⾙}
  \definition{adj.}{corrupto; venal | ganancioso; avarento; ambicioso}
  \definition{v.}{apropriar-se indevidamente; praticar corrupção; ser corrupto | ter um desejo insaciável por; ter um desejo voraz por | cobiçar; ansiar por; ser ganancioso por}
\end{entry}

\begin{entry}{贪婪}{tan1lan2}{8,11}{⾙、⼥}
  \definition{adj.}{avaro | ambicioso | voraz | insaciável}
\end{entry}

\begin{entry}{怹}{tan1}{9}{⼼}
  \definition{pron.}{ele, ela (cortês, em oposição a 他)}
  \seealsoref{他}{ta1}
\end{entry}

\begin{entry}{谈}{tan2}{10}{⾔}[HSK 3]
  \definition*{s.}{Sobrenome Tan}
  \definition{s.}{o que é dito ou falado; discurso}
  \definition{v.}{falar; bater papo; discutir}
\end{entry}

\begin{entry}{谈话}{tan2 hua4}{10,8}{⾔、⾔}[HSK 3]
  \definition[次]{s.}{declaração; opiniões (principalmente políticas) expressas na forma de conversas}
  \definition{v.+compl.}{conversar; discutir | falar; refere-se especificamente ao uso da conversa para entender a situação, fazer trabalho ideológico, etc. (usado principalmente por superiores para subordinados)}
\end{entry}

\begin{entry}{谈恋爱}{tan2lian4'ai4}{10,10,10}{⾔、⼼、⽖}
  \definition{v.}{namorar | apaixonar-se}
\end{entry}

\begin{entry}{谈判}{tan2pan4}{10,7}{⾔、⼑}[HSK 3]
  \definition{v.}{negociar; manter conversações; para resolver um grande problema, as partes relevantes trocaram opiniões entre si, na esperança de encontrar uma solução com a qual todos pudessem concordar}
\end{entry}

\begin{entry}{弹}{tan2}{11}{⼸}[HSK 5]
  \definition{s.}{bola; pelota; pequenas bolas disparadas com um estilingue | bomba; bala; explosivos que podem ser lançados ou arremessados, com poder destrutivo e letal}
\end{entry}

\begin{entry}{坦}{tan3}{8}{⼟}
  \definition*{s.}{Sobrenome Tan}
  \definition{adj.}{nivelado; suave; plano | calmo; composto | aberto; sincero; franco}
\end{entry}

\begin{entry}{坦克}{tan3ke4}{8,7}{⼟、⼗}
  \definition{s.}{(empréstimo linguístico) tanque (veículo militar)}
\end{entry}

\begin{entry}{探}{tan4}{11}{⼿}
  \definition[个,位,名]{s.}{batedor; espião; detetive}
  \definition{v.}{tentar descobrir; explorar; soar | explorar; espionar | visitar; fazer uma visita em | se destacar | preocupar-se com; envolver-se em | ver; invocar}
\end{entry}

\begin{entry}{探亲}{tan4qin1}{11,9}{⼿、⼇}
  \definition{v.+compl.}{ir para casa para visitar a família}
\end{entry}

\begin{entry}{碳}{tan4}{14}{⽯}
  \definition{s.}{carbono (elemento químico)}
\end{entry}

\begin{entry}{汤}{tang1}{6}{⽔}[HSK 3]
  \definition*{s.}{Sobrenome Tang}
  \definition[勺,碗,杯,锅]{s.}{água quente; água fervente | fontes termais | água utilizada para ferver algo| sopa; caldo | uma preparação líquida de ervas medicinais; decocção}
  \seeref{汤}{shang1}
\end{entry}

\begin{entry}{趟}{tang1}{15}{⾛}
  \definition{v.}{atravessar; andar na grama ou onde não haja caminho | usar arados, capinadores, etc. para virar o solo e remover ervas daninhas | vadear; atravessar a vau; caminhar por águas rasas}[我们趟水去那小岛。___Nós vadeamos até a ilha.]
  \seeref{趟}{tang4}
\end{entry}

\begin{entry}{唐}{tang2}{10}{⼝}
  \definition*{s.}{Dinastia estabelecida pelo Imperador Yao, 尧, no período lendário da história chinesa | Dinastia Tang (618-907) | Dinastia Tang posterior (923-936), uma das cinco dinastias | Sobrenome Tang}
  \definition{adj.}{exagerado; bombástico; orgulhoso | em vão; por nada}
  \seealsoref{尧}{yao2}
\end{entry}

\begin{entry}{唐人街}{tang2ren2 jie1}{10,2,12}{⼝、⼈、⾏}
  \definition*[条,座]{s.}{Bairro Chinês; Chinatown; refere-se ao mercado de rua onde os chineses do exterior vivem e abrem muitas lojas com características chinesas}
  \seealsoref{中国城}{zhong1guo2cheng2}
\end{entry}

\begin{entry}{糖}{tang2}{16}{⽶}[HSK 3]
  \definition[包,斤,勺,袋,块]{s.}{açúcar; um tipo de açúcar; um tipo de composto orgânico, que pode ser dividido em três tipos: monossacarídeos, dissacarídeos e polissacarídeos; é a principal substância que produz energia térmica no corpo humano, como glicose, sacarose, lactose, amido, etc. | açúcar; açúcar comestível; termo geral para açúcar | doces; balas | carboidrato; algo doce e calórico}
\end{entry}

\begin{entry}{糖醋鱼}{tang2cu4yu2}{16,15,8}{⽶、⾣、⿂}
  \definition{s.}{peixe guisado em molho agridoce (prato)}
\end{entry}

\begin{entry}{倘}{tang3}{10}{⼈}
  \definition{conj.}{se; supondo; no caso}
  \seeref{倘}{chang2}
\end{entry}

\begin{entry}{倘或}{tang3huo4}{10,8}{⼈、⼽}
  \definition{conj.}{se | supondo que | no caso}
\end{entry}

\begin{entry}{倘若}{tang3ruo4}{10,8}{⼈、⾋}
  \definition{conj.}{se | supondo que | no caso}
\end{entry}

\begin{entry}{倘使}{tang3shi3}{10,8}{⼈、⼈}
  \definition{conj.}{se | supondo que | no caso}
\end{entry}

\begin{entry}{躺}{tang3}{15}{⾝}[HSK 4]
  \definition{v.}{deitar; reclinar}
\end{entry}

\begin{entry}{趟}{tang4}{15}{⾛}[HSK 6]
  \definition{clas.}{usado para o número de vezes de viagens de ida e volta |  usado para coisas dispostas em fileiras ou tiras | usado para a programação de veículos, navios, etc. que circulam em uma determinada ordem | usado em conjuntos de movimentos de artes marciais}
  \definition{s.}{marcha; procissão; jornada; viagem}
\end{entry}

\begin{entry}{掏}{tao1}{11}{⼿}[HSK 6]
  \definition{v.}{extrair; retirar; pescar | cavar (um buraco, etc.); escavar; retirar | (coloquial) roubar do bolso de alguém | tirar}
\end{entry}

\begin{entry}{滔}{tao1}{13}{⽔}
  \definition{adj.}{(de água) transbordando | arrogante | turbulento | largo e longo; grande}
  \definition{v.}{inundar; alagar}
\end{entry}

\begin{entry}{滔天}{tao1tian1}{13,4}{⽔、⼤}
  \definition{adj.}{(ondas, raiva, desastres, crimes, etc.) imponente, avassalador, imenso}
\end{entry}

\begin{entry}{逃}{tao2}{9}{⾡}[HSK 5]
  \definition{v.}{fugir; escapar; correr; dar no pé | evadir; esquivar-se; escapar}
\end{entry}

\begin{entry}{逃跑}{tao2 pao3}{9,12}{⾡、⾜}[HSK 5]
  \definition{v.}{fugir; escapar; correr; partir para fugir de um ambiente ou de coisas que não lhe são favoráveis}
\end{entry}

\begin{entry}{逃走}{tao2 zou3}{9,7}{⾡、⾛}[HSK 5]
  \definition{v.}{escapar; afastar-se de pessoas, coisas ou lugares que não são bons para você ou que você não gosta}
\end{entry}

\begin{entry}{桃}{tao2}{10}{⽊}[HSK 5]
  \definition*{s.}{Sobrenome Tao}
  \definition{s.}{pêssego | em forma de pêssego | pessegueiro}
\end{entry}

\begin{entry}{桃花}{tao2 hua1}{10,7}{⽊、⾋}[HSK 5]
  \definition{s.}{(figurativo) caso amoroso | flor de pessegueiro}
\end{entry}

\begin{entry}{桃树}{tao2 shu4}{10,9}{⽊、⽊}[HSK 5]
  \definition[株]{s.}{pêssego (árvore) | pessegueiro; pêssegos}
\end{entry}

\begin{entry}{讨}{tao3}{5}{⾔}
  \definition{v.}{enviar forças armadas para suprimir; enviar uma expedição punitiva contra; enviar exército ou despachar tropas para suprimir ou atacar | denunciar; condenar; censurar | exigir; pedir; implorar por | casar (com uma mulher) | incorrer; convidar | discutir; estudar | provocar; cortejar}
\end{entry}

\begin{entry}{讨论}{tao3lun4}{5,6}{⾔、⾔}[HSK 2]
  \definition{v.}{discutir; conversar sobre; trocar opiniões ou debater as questões levantadas}
\end{entry}

\begin{entry}{讨生活}{tao3sheng1huo2}{5,5,9}{⾔、⽣、⽔}
  \definition{v.}{ganhar a vida}
\end{entry}

\begin{entry}{讨厌}{tao3yan4}{5,6}{⾔、⼚}[HSK 5]
  \definition{adj.}{desagradável; repugnante; repulsivo; irritante; incômodo |}
  \definition{v.}{odiar; não gostar; sentir repulsa por}
\end{entry}

\begin{entry}{套}{tao4}{10}{⼤}[HSK 2]
  \definition{clas.}{usado para coisas agrupadas: conjuntos, coleções, séries, etc.}
  \definition{s.}{estojo; capa; bainha | local onde o rio ou a cordilheira faz uma curva (usado principalmente em nomes de lugares) | enchimento de algodão em roupas e edredons | arreios; corda para amarrar animais | nó; laço; um objeto circular feito com corda ou algo semelhante | cortersia; convenção; conversa fiada; métodos repetitivos | armadilha; truque; conspiração}
  \definition{v.}{sobrepor; interligar | deslizar sobre; cobrir por fora | atrelar; engatar; usar um cinto de segurança | copiar; imitar; seguir o modelo de | extrair; induzir a falar; persuadir alguém a revelar um segredo; induzir; provocar | tentar vencer; aproximar-se de; aproximar-se intencionalmente de outras pessoas para algum propósito | fazer a rosca de um parafuso; usar um macho de rosca ou uma chave de rosca para fazer roscas}
\end{entry}

\begin{entry}{套餐}{tao4 can1}{10,16}{⼤、⾷}[HSK 4]
  \definition{s.}{combo; pacote de produtos; pacote de serviços; metaforicamente, bens ou projetos que são combinados e levados ao mercado | refeição preparada; pacotes de refeições completos}
\end{entry}

\begin{entry}{套问}{tao4wen4}{10,6}{⼤、⾨}
  \definition{s.}{retórica}
  \definition{v.}{descobrir por meio de questionamento indireto diplomático}
\end{entry}

\begin{entry}{特}{te4}{10}{⽜}[HSK 6]
  \definition{adj.}{especial; incomum; particular; excepcional; diferente do geral | especial; solteiro; solitário}
  \definition{adv.}{muito; extremamente | especialmente; para um propósito especial |mas; somente}
  \definition{clas.}{TEX; abreviação para unidades de medida como TEX; a unidade de medida TEX indica a espessura de um fio têxtil através do seu peso}
  \definition{s.}{espião; agente secreto}
\end{entry}

\begin{entry}{特别}{te4bie2}{10,7}{⽜、⼑}[HSK 2]
  \definition{adj.}{especial; particular; fora do comum; diferente dos outros, com características próprias}
  \definition{adv.}{especialmente; particularmente | ainda mais; em particular; frequentemente usado com 是 | especialmente; deliberadamente; para um propósito específico}
  \seealsoref{是}{shi4}
\end{entry}

\begin{entry}{特地}{te4di4}{10,6}{⽜、⼟}
  \definition{adv.}{especialmente | propositalmente}
\end{entry}

\begin{entry}{特点}{te4dian3}{10,9}{⽜、⽕}[HSK 2]
  \definition[个,大]{s.}{característica; peculiaridade; traço distintivo; a singularidade de uma pessoa ou coisa}
\end{entry}

\begin{entry}{特定}{te4ding4}{10,8}{⽜、⼧}[HSK 5]
  \definition{adj.}{específico; especialmente designado | dado; especificado; específico (pessoa, hora, lugar, local, ambiente, etc.)}
\end{entry}

\begin{entry}{特技}{te4ji4}{10,7}{⽜、⼿}
  \definition{s.}{efeito especial | dublê}
\end{entry}

\begin{entry}{特价}{te4 jia4}{10,6}{⽜、⼈}[HSK 4]
  \definition{s.}{oferta especial; preço de barganha; preço especial reduzido}
\end{entry}

\begin{entry}{特色}{te4se4}{10,6}{⽜、⾊}[HSK 3]
  \definition{s.}{característica; característica distintiva; a cor única, estilo, etc. de um objeto}
\end{entry}

\begin{entry}{特殊}{te4shu1}{10,10}{⽜、⽍}[HSK 4]
  \definition{adj.}{especial; particular; peculiar; excepcional; incomum}
\end{entry}

\begin{entry}{特性}{te4 xing4}{10,8}{⽜、⼼}[HSK 5]
  \definition[个]{s.}{propriedade específica (ou característica) | característica; sabores | propriedade}
\end{entry}

\begin{entry}{特有}{te4 you3}{10,6}{⽜、⽉}[HSK 5]
  \definition{adj.}{específico; peculiar; característico; único; exclusivo; especial}
\end{entry}

\begin{entry}{特征}{te4zheng1}{10,8}{⽜、⼻}[HSK 4]
  \definition[个,种]{s.}{característica; aparência ou o fenômeno característico de uma pessoa ou coisa que pode ser visto de fora}
\end{entry}

\begin{entry}{疼}{teng2}{10}{⽧}[HSK 2]
  \definition{adj.}{dolorido; doído; sensação de extremo desconforto causada por ferimentos, doenças, etc.}
  \definition{v.}{ferir; machucar | adorar; amar profundamente; gostar muito; cuidar}
\end{entry}

\begin{entry}{梯}{ti1}{11}{⽊}
  \definition*{s.}{Sobrenome Ti}
  \definition{adj.}{em forma de escada; em socalcos}
  \definition[个]{s.}{escada; degrau; socalco (são plataformas niveladas, semelhantes a degraus, cortadas em encostas de morros para permitir o cultivo agrícola e evitar a erosão do solo)}
\end{entry}

\begin{entry}{梯恩梯}{ti1'en1ti1}{11,10,11}{⽊、⼼、⽊}
  \definition{s.}{(empréstimo linguístico) TNT, trinitrotolueno}
\end{entry}

\begin{entry}{踢}{ti1}{15}{⾜}[HSK 6]
  \definition{v.}{chutar | jogar (por exemplo, futebol)}
\end{entry}

\begin{entry}{踢爆}{ti1bao4}{15,19}{⾜、⽕}
  \definition{v.}{expor | revelar}
\end{entry}

\begin{entry}{踢蹋舞}{ti1ta4wu3}{15,17,14}{⾜、⾜、⾇}
  \definition{s.}{sapateado | passo de dança}
\end{entry}

\begin{entry}{提}{ti2}{12}{⼿}[HSK 2]
  \definition*{s.}{Sobrenome Ti}
  \definition{s.}{concha; utensílio para servir óleo ou vinho | traço ascendente (em caracteres chineses)}
  \definition{v.}{carregar (na mão, com o braço para baixo) ; segurar com as mãos para baixo | elevar; levantar; promover | avançar; antecipar uma data; mudar para uma data anterior; adiar o prazo previsto | levantar; apresentar; indicar ou citar | extrair; retirar (tirar) | (prisioneiros) trazer; entregar | mencionar; referir-se a; abordar}
\end{entry}

\begin{entry}{提倡}{ti2chang4}{12,10}{⼿、⼈}[HSK 5]
  \definition{v.}{promover; incentivar; recomendar; apresentar as vantagens de algo para incentivar as pessoas a usá-lo ou implementá-lo}
\end{entry}

\begin{entry}{提出}{ti2 chu1}{12,5}{⼿、⼐}[HSK 2]
  \definition{v.}{levantar; propor; apresentar; expressar seus desejos, ideias, sugestões, etc. por meio de palavras ou textos}
\end{entry}

\begin{entry}{提到}{ti2 dao4}{12,8}{⼿、⼑}[HSK 2]
  \definition{v.}{mencionar; referir-se a; levantar (assunto)}
\end{entry}

\begin{entry}{提高}{ti2gao1}{12,10}{⼿、⾼}[HSK 2]
  \definition{v.}{elevar; aprimorar; aumentar; melhorar a posição, o nível, a quantidade, a qualidade e outros aspectos em relação ao estado original}
\end{entry}

\begin{entry}{提供}{ti2gong1}{12,8}{⼿、⼈}[HSK 4]
  \definition{v.}{oferecer; fornecer; suprir; prover; proporcionar}
\end{entry}

\begin{entry}{提及}{ti2ji2}{12,3}{⼿、⼃}
  \definition{v.}{mencionar | levantar (um assunto) | chamar a atenção de alguém}
\end{entry}

\begin{entry}{提起}{ti2 qi3}{12,10}{⼿、⾛}[HSK 5]
  \definition{v.}{mencionar; falar sobre; abordar | levantar; despertar; estimular; revigorar; alegrar/animar | iniciar; instituir; propor | levantar; pegar}
\end{entry}

\begin{entry}{提前}{ti2qian2}{12,9}{⼿、⼑}[HSK 3]
  \definition{adv.}{antecipadamente; faça uma coisa antes de fazer outra}
  \definition{v.}{avançar; adiantar; mudar para uma data anterior; trazer para frente}
\end{entry}

\begin{entry}{提升}{ti2sheng1}{12,4}{⼿、⼗}
  \definition{v.}{promover (para uma posição de classificação mais alta) | levantar | içar | (figurativo) elevar, levantar, melhorar}
\end{entry}

\begin{entry}{提示}{ti2shi4}{12,5}{⼿、⽰}[HSK 5]
  \definition[个]{s.}{dica; lembrete; pistas ou informações fornecidas para chamar a atenção, fazer com que a outra pessoa pense ou compreenda}
  \definition{v.}{solicitar; lembrar; indicar; alertar; levantar questões que o outro não tenha pensado ou não tenha imaginado, para chamar a atenção dele}
\end{entry}

\begin{entry}{提问}{ti2wen4}{12,6}{⼿、⾨}[HSK 3]
  \definition{v.}{\emph{quiz}; fazer uma pergunta; colocar questões para}
\end{entry}

\begin{entry}{提醒}{ti2xing3}{12,16}{⼿、⾣}[HSK 4]
  \definition{v.+compl.}{alertar; avisar; advertir; lembrar; apontar para ou chamar a atenção para}
\end{entry}

\begin{entry}{题}{ti2}{15}{⾴}[HSK 2]
  \definition*{s.}{Sobrenome Ti}
  \definition[个,道]{s.}{tópico; título; assunto; problema; frases que indicam o conteúdo de poemas ou discursos | questão; questões que devem ser respondidas durante os exercícios ou exames | antigamente, referia-se à testa}
  \definition{v.}{inscrever; escrever; assinar}
\end{entry}

\begin{entry}{题材}{ti2cai2}{15,7}{⾴、⽊}[HSK 5]
  \definition{s.}{tema; assunto; material que compõe as obras literárias e artísticas, ou seja, os eventos ou fenômenos da vida descritos concretamente nas obras}
\end{entry}

\begin{entry}{题目}{ti2mu4}{15,5}{⾴、⽬}[HSK 3]
  \definition[个,道]{s.}{título; assunto; tópico; o título de um poema ou discurso | quebra-cabeça; problema de exercício; questões a serem respondidas em exercícios ou provas}
\end{entry}

\begin{entry}{体}{ti3}{7}{⼈}
  \definition{s.}{corpo; parte do corpo | substância; objeto; estado de uma substância | estilo; forma | sistema | estilo de caligrafia | tipo de letra; fonte | (linguística) aspecto (de um verbo) | estrutura; a forma escrita do texto; o gênero da obra}
  \definition{v.}{fazer ou vivenciar algo pessoalmente | colocar-se na posição de outro; colocar-se mentalmente na posição do outro; colocar-se no lugar do outro}
\end{entry}

\begin{entry}{体操}{ti3 cao1}{7,16}{⼈、⼿}[HSK 4]
  \definition{s.}{ginástica; esportes, exercícios ou performances de vários movimentos, sem armas ou com o auxílio de determinados equipamentos}
\end{entry}

\begin{entry}{体会}{ti3hui4}{7,6}{⼈、⼈}[HSK 3]
  \definition[个,些,种]{s.}{conhecimento; compreensão; experiência pessoal}
  \definition{v.}{perceber; saber (ou aprender) com a experiência}
\end{entry}

\begin{entry}{体积}{ti3ji1}{7,10}{⼈、⽲}[HSK 5]
  \definition[个]{s.}{volume; quantidade; o tamanho do espaço ocupado pelo objeto}
\end{entry}

\begin{entry}{体检}{ti3 jian3}{7,11}{⼈、⽊}[HSK 4]
  \definition{s.}{exame clínico}
  \definition{v.}{fazer um exame médico}
\end{entry}

\begin{entry}{体力}{ti3 li4}{7,2}{⼈、⼒}[HSK 5]
  \definition{s.}{força física; vigor físico (ou corporal); a força do corpo humano para sustentar suas próprias atividades}
\end{entry}

\begin{entry}{体内}{ti3nei4}{7,4}{⼈、⼌}
  \definition{adj.}{dentro do corpo | \emph{in vivo} (versus \emph{in vitro} | interno a}
\end{entry}

\begin{entry}{体现}{ti3xian4}{7,8}{⼈、⾒}[HSK 3]
  \definition{v.}{refletir; incorporar; encarnar; uma certa qualidade ou fenômeno se manifesta especificamente em uma determinada coisa}
\end{entry}

\begin{entry}{体验}{ti3yan4}{7,10}{⼈、⾺}[HSK 3]
  \definition[种]{s.}{experiência; a sensação adquirida pela experiência pessoal}
  \definition{v.}{aprender através da prática; aprender através da experiência pessoal; entender as coisas através da experiência pessoal}
\end{entry}

\begin{entry}{体育}{ti3yu4}{7,8}{⼈、⾁}[HSK 2]
  \definition{s.}{cultura física; treinamento físico; educação cuja principal tarefa é desenvolver a capacidade física e fortalecer a constituição física, alcançada através da participação em várias atividades esportivas | esportes; atividades esportivas; refere-se a esportes}
\end{entry}

\begin{entry}{体育场}{ti3 yu4 chang3}{7,8,6}{⼈、⾁、⼟}[HSK 2]
  \definition[个,座]{s.}{estádio; campo esportivo; espaço ao ar livre para a prática de exercícios físicos ou competições esportivas}
\end{entry}

\begin{entry}{体育馆}{ti3 yu4 guan3}{7,8,11}{⼈、⾁、⾷}[HSK 2]
  \definition[个,座,家]{s.}{ginásio; locais esportivos ou competições em ambientes fechados geralmente têm arquibancadas fixas}
\end{entry}

\begin{entry}{体重}{ti3 zhong4}{7,9}{⼈、⾥}[HSK 4]
  \definition{s.}{peso corporal}
\end{entry}

\begin{entry}{替}{ti4}{12}{⽈}[HSK 4]
  \definition{prep.}{para; em nome de}
  \definition{s.}{decadência; declínio; enfraquecimento}
  \definition{v.}{substituir; substituir por; tomar o lugar de}
\end{entry}

\begin{entry}{替代}{ti4 dai4}{12,5}{⽈、⼈}[HSK 4]
  \definition{v.}{substituir; suplantar}
\end{entry}

\begin{entry}{天}{tian1}{4}{⼤}[HSK 1]
  \definition*{s.}{Sobrenome Tian}
  \definition{adj.}{localizado no topo; suspenso no ar | inato; natural}
  \definition{clas.}{usado para contar dias}
  \definition{s.}{céu; paraíso; espaço onde se encontram o sol, a lua e as estrelas | dia; as 24 horas do dia, às vezes referindo-se especificamente ao período diurno | um período de tempo em um dia; em algum momento do dia | temporada; estação do ano | clima | natureza | Deus; céu; o criador | paraíso; refere-se ao local onde residem os deuses, budas e imortais}
\end{entry}

\begin{entry}{天才}{tian1cai2}{4,3}{⼤、⼿}[HSK 5]
  \definition{adj.}{talentoso | superdotado | genial}
  \definition[个]{s.}{dom; genialidade; talento natural; inteligência e sabedoria acima da média}
\end{entry}

\begin{entry}{天鹅}{tian1'e2}{4,12}{⼤、⿃}
  \definition{s.}{cisne}
\end{entry}

\begin{entry}{天公}{tian1gong1}{4,4}{⼤、⼋}
  \definition{s.}{céu, paraíso | senhor do céu}
\end{entry}

\begin{entry}{天花板}{tian1hua1ban3}{4,7,8}{⼤、⾋、⽊}
  \definition{s.}{teto}
\end{entry}

\begin{entry}{天空}{tian1kong1}{4,8}{⼤、⽳}[HSK 3]
  \definition{s.}{o céu; o firmamento}
\end{entry}

\begin{entry}{天气}{tian1qi4}{4,4}{⼤、⽓}[HSK 1]
  \definition{s.}{clima, tempo; mudanças meteorológicas que ocorrem na atmosfera em uma determinada área e durante um determinado período de tempo, tais como temperatura, umidade, pressão atmosférica, precipitação, vento, nuvens, etc.}
\end{entry}

\begin{entry}{天然}{tian1ran2}{4,12}{⼤、⽕}
  \definition{adj.}{natural}
\end{entry}

\begin{entry}{天然气}{tian1ran2qi4}{4,12,4}{⼤、⽕、⽓}[HSK 5]
  \definition{s.}{gás; gás natural; gás combustível produzido em campos petrolíferos, carboníferos e pântanos}
\end{entry}

\begin{entry}{天上}{tian1 shang4}{4,3}{⼤、⼀}[HSK 2]
  \definition[期]{s.}{o céu; o paraíso}
\end{entry}

\begin{entry}{天使}{tian1shi3}{4,8}{⼤、⼈}
  \definition{s.}{anjo}
\end{entry}

\begin{entry}{天堂}{tian1tang2}{4,11}{⼤、⼟}
  \definition{s.}{paraíso, céu}
\end{entry}

\begin{entry}{天天}{tian1tian1}{4,4}{⼤、⼤}
  \definition{adv.}{todo dia}
\end{entry}

\begin{entry}{天文}{tian1wen2}{4,4}{⼤、⽂}[HSK 5]
  \definition[对]{s.}{astronomia; a distribuição e o movimento dos corpos celestes, como o sol, a lua e as estrelas, no universo}
\end{entry}

\begin{entry}{天下}{tian1xia4}{4,3}{⼤、⼀}
  \definition{s.}{terra sob o céu | o mundo todo | toda a China | reino}
\end{entry}

\begin{entry}{天择}{tian1ze2}{4,8}{⼤、⼿}
  \definition{s.}{seleção natural}
\end{entry}

\begin{entry}{天真}{tian1zhen1}{4,10}{⼤、⼗}[HSK 4]
  \definition{adj.}{ingênuo; inocente; ignorante; simples de coração, direto por natureza, livre de fingimento e hipocrisia}
\end{entry}

\begin{entry}{天柱}{tian1zhu4}{4,9}{⼤、⽊}
  \definition{s.}{pilar celestial, que sustenta o céu}
\end{entry}

\begin{entry}{兲}{tian1}{6}{⼋}
  \variantof{天}
\end{entry}

\begin{entry}{添}{tian1}{11}{⽔}[HSK 6]
  \definition{v.}{adicionar; aumentar | dar à luz}
\end{entry}

\begin{entry}{田}{tian2}{5}{⽥}[HSK 6][Kangxi 102]
  \definition*{s.}{Sobrenome Tian}
  \definition[亩,块,片]{s.}{campo; terra; terra de cultivo | área aberta rica em algum produto natural; campo}
  \definition{v.}{(arcaico) caçar}
\end{entry}

\begin{entry}{田园}{tian2yuan2}{5,7}{⽥、⼞}
  \definition{adj.}{bucólico}
  \definition{s.}{campo | interior | rural}
\end{entry}

\begin{entry}{钿}{tian2}{10}{⾦}
  \definition{s.}{(dialeto) moeda | dinheiro; moeda | uma quantia de dinheiro}
  \seeref{钿}{dian4}
\end{entry}

\begin{entry}{甜}{tian2}{11}{⽢}[HSK 3]
  \definition{adj.}{doce; melado | agradável; confortável; fazer as pessoas se sentirem confortáveis e felizes | (sono) profundo | feliz; descreve o sentimento de felicidade}
\end{entry}

\begin{entry}{甜酒}{tian2jiu3}{11,10}{⽢、⾣}
  \definition{s.}{licor doce}
\end{entry}

\begin{entry}{甜菊}{tian2ju2}{11,11}{⽢、⾋}
  \definition{s.}{estévia, arbusto cujas folhas produzem um substituto para o açúcar}
\end{entry}

\begin{entry}{甜品}{tian2pin3}{11,9}{⽢、⼝}
  \definition{s.}{sobremesa}
\end{entry}

\begin{entry}{甜食}{tian2shi2}{11,9}{⽢、⾷}
  \definition{s.}{doces | sobremesa}
\end{entry}

\begin{entry}{甜酸}{tian2suan1}{11,14}{⽢、⾣}
  \definition{adj.}{agridoce}
\end{entry}

\begin{entry}{甜甜圈}{tian2tian2quan1}{11,11,11}{⽢、⽢、⼞}
  \definition{s.}{rosquinha | \emph{doughnut}}
\end{entry}

\begin{entry}{甜筒}{tian2tong3}{11,12}{⽢、⽵}
  \definition{s.}{sorvete de casquinha}
\end{entry}

\begin{entry}{甜头}{tian2tou5}{11,5}{⽢、⼤}
  \definition{s.}{benefício | sabor doce (de poder, sucesso, etc.)}
\end{entry}

\begin{entry}{甜心}{tian2xin1}{11,4}{⽢、⼼}
  \definition{s.}{querido}
\end{entry}

\begin{entry}{甜言}{tian2yan2}{11,7}{⽢、⾔}
  \definition{s.}{boa conversa | palavras amáveis}
\end{entry}

\begin{entry}{甜玉米}{tian2 yu4mi3}{11,5,6}{⽢、⽟、⽶}
  \definition{s.}{milho doce}
\end{entry}

\begin{entry}{甜稚}{tian2zhi4}{11,13}{⽢、⽲}
  \definition{s.}{doce e inocente}
\end{entry}

\begin{entry}{填}{tian2}{13}{⼟}
  \definition{v.}{encher; rechear | reabastecer; suplementar; complementar | preencher; escrever dados em uma caixa (em um questionário ou formulário da \emph{Web})}
\end{entry}

\begin{entry}{填空}{tian2kong4}{13,8}{⼟、⽳}[HSK 4]
  \definition{v.}{preencher o espaço em branco (por exemplo, em um teste)}
\end{entry}

\begin{entry}{挑}{tiao1}{9}{⼿}[HSK 4]
  \definition{clas.}{para coisas que são escolhidas ou selecionadas | para coisas que podem ser usadas como palhetas}
  \definition{s.}{vara comprida com algo pendurado nas pontas; haste de transporte}
  \definition{v.}{escolher; selecionar | fazer picuinhas; ser hipercrítico; ser meticuloso; ser excessivamente rigoroso nos detalhes | carregar com uma haste de transporte; carregar no ombro; pendurar coisas nas pontas de varas longas e carregá-las em seus ombros}
  \seeref{挑}{tiao3}
\end{entry}

\begin{entry}{挑选}{tiao1 xuan3}{9,9}{⼿、⾡}[HSK 4]
  \definition{v.}{escolher; optar; selecionar; escolher a pessoa ou coisa certa para o trabalho}
\end{entry}

\begin{entry}{条}{tiao2}{7}{⽊}[HSK 2]
  \definition*{s.}{Sobrenome Tiao}
  \definition{clas.}{usado para objetos longos e finos; usado para sintetizar certas coisas longas e retangulares em quantidades fixas | usado para itemização | aplicado ao corpo humano}
  \definition{s.}{galho; galhos finos e longos | tira; faixa | item; artigo | ordem; método | nota; anotação em papel}
\end{entry}

\begin{entry}{条幅}{tiao2fu2}{7,12}{⽊、⼱}
  \definition{s.}{faixa | banner | pergaminho de parede (para pintura ou caligrafia)}
\end{entry}

\begin{entry}{条贯}{tiao2guan4}{7,8}{⽊、⾙}
  \definition{s.}{ordem | procedimentos | sequência | sistema}
\end{entry}

\begin{entry}{条件}{tiao2jian4}{7,6}{⽊、⼈}[HSK 2]
  \definition[个,项,些]{s.}{condição; termo; fator; fatores que restringem a ocorrência, existência ou desenvolvimento das coisas | requisito; pré-requisito; qualificação; requisitos ou padrões estabelecidos para determinadas coisas | situação; estado; condição}
\end{entry}

\begin{entry}{条例}{tiao2li4}{7,8}{⽊、⼈}
  \definition{s.}{código de conduta | ordenanças | regulamentos | regras | estatutos}
\end{entry}

\begin{entry}{条目}{tiao2mu4}{7,5}{⽊、⽬}
  \definition{s.}{cláusulas e subcláusulas (em documento formal) | verbete (em um dicionário, enciclopédia, etc.)}
\end{entry}

\begin{entry}{调}{tiao2}{10}{⾔}[HSK 3]
  \definition{adj.}{harmonioso; boa coordenação}
  \definition{v.}{misturar; ajustar; fazer o ajuste uniforme e apropriado | provocar; importunar; fazer pouco de | incitar; instigar; provocar; semear discórdia | mediar; trazer harmonia}
  \seeref{调}{diao4}
\end{entry}

\begin{entry}{调节}{tiao2jie2}{10,5}{⾔、⾋}[HSK 5]
  \definition{v.}{regular; ajustar; ajustar e controlar de várias maneiras para atender aos requisitos}
\end{entry}

\begin{entry}{调解}{tiao2jie3}{10,13}{⾔、⾓}[HSK 5]
  \definition{v.}{mediar; fazer as pazes; resolver conflitos através da persuasão}
\end{entry}

\begin{entry}{调律}{tiao2lv4}{10,9}{⾔、⼻}
  \definition{v.}{afinar (por exemplo, um piano)}
\end{entry}

\begin{entry}{调皮}{tiao2pi2}{10,5}{⾔、⽪}[HSK 4]
  \definition{adj.}{travesso; malicioso; malandro | indisciplinado; desordeiro; indomável; astuto | inteligente e desonesto}
\end{entry}

\begin{entry}{调整}{tiao2zheng3}{10,16}{⾔、⽁}[HSK 3]
  \definition{v.}{ajustar; revisar; regularizar; fazer as alterações apropriadas no estado original para se adaptar à nova situação}
\end{entry}

\begin{entry}{挑}{tiao3}{9}{⼿}[HSK 4]
  \definition{s.}{um dos traços dos caracteres chineses; inclinado para cima da esquerda para a direita}
  \definition{v.}{levantar; elevar; erguer | levantar ou apoiar com uma extremidade de uma vara ou objeto semelhante; segurar ou apoiar com a ponta de uma vara etc. | causar conflitos deliberadamente; provocar deliberadamente um conflito | (método de bordado) usar uma agulha para levantar os fios de urdidura ou trama, com a agulha e a linha passando por baixo para formar padrões e desenhos}
  \seeref{挑}{tiao1}
\end{entry}

\begin{entry}{挑衅}{tiao3xin4}{9,11}{⼿、⾎}
  \definition{s.}{provocação}
  \definition{v.}{provocar}
\end{entry}

\begin{entry}{挑战}{tiao3zhan4}{9,9}{⼿、⼽}[HSK 4]
  \definition{v.}{desafiar; deixar um oponente deliberadamente irritado e sair para lutar ou lutar consigo mesmo; estimular um oponente a lutar consigo mesmo}
\end{entry}

\begin{entry}{跳}{tiao4}{13}{⾜}[HSK 3]
  \definition{v.}{pular; saltar | mover para cima e para baixo | pular (por cima); fazer omissões | quicar; a força elástica faz com que o objeto se mova repentinamente para cima | pulsar; palpitar; contrair-se | pular sobre;  saltar sobre; cruzar}
\end{entry}

\begin{entry}{跳挡}{tiao4dang3}{13,9}{⾜、⼿}
  \definition{v.}{pular marcha (de um carro) | perder a marcha}
\end{entry}

\begin{entry}{跳电}{tiao4dian4}{13,5}{⾜、⽥}
  \definition{v.}{desarmar (um disjuntor ou interruptor)}
\end{entry}

\begin{entry}{跳高}{tiao4 gao1}{13,10}{⾜、⾼}[HSK 3]
  \definition{s.}{salto em altura (atletismo)}
  \definition{v.}{saltar em altura}
\end{entry}

\begin{entry}{跳频}{tiao4pin2}{13,13}{⾜、⾴}
  \definition{s.}{FHSS, \emph{Frequency-Hopping Spread Spectrum}, método de transmissão de sinais de rádio}
\end{entry}

\begin{entry}{跳伞}{tiao4san3}{13,6}{⾜、⼈}
  \definition{s.}{paraquedas}
  \definition{v.}{saltar de paraquedas}
\end{entry}

\begin{entry}{跳绳}{tiao4sheng2}{13,11}{⾜、⽷}
  \definition{v.}{pular corda}
\end{entry}

\begin{entry}{跳水}{tiao4shui3}{13,4}{⾜、⽔}
  \definition{s.}{mergulho esportivo}
  \definition{v.}{mergulhar (na água) | cometer suicídio pulando na água | (figurativo, preços das ações, etc.) cair dramaticamente}
\end{entry}

\begin{entry}{跳跳糖}{tiao4tiao4tang2}{13,13,16}{⾜、⾜、⽶}
  \definition{s.}{\emph{Pop Rocks}, \emph{popping candy}}
\end{entry}

\begin{entry}{跳舞}{tiao4wu3}{13,14}{⾜、⾇}[HSK 3]
  \definition{v.+compl.}{dançar (como performance); executar dança, especialmente dança de salão}
\end{entry}

\begin{entry}{跳远}{tiao4 yuan3}{13,7}{⾜、⾡}[HSK 3]
  \definition{s.}{salto em distância (atletismo)}
\end{entry}

\begin{entry}{跳蚤}{tiao4zao5}{13,9}{⾜、⾍}
  \definition{s.}{pulga}
\end{entry}

\begin{entry}{贴}{tie1}{9}{⾙}[HSK 4]
  \definition{adj.}{submisso; obediente}
  \definition{clas.}{para uso em gessos, emplastros}
  \definition{s.}{subsídio; subvenção}
  \definition{v.}{grudar; colar | aninhar-se a; aconchegar-se a | subsidiar; ajudar financeiramente}
\end{entry}

\begin{entry}{铁}{tie3}{10}{⾦}[HSK 3]
  \definition*{s.}{Sobrenome Tie}
  \definition{adj.}{duro; forte; sólido como ferro; metáfora para natureza dura; vontade forte | violento | inabalável; inalterável; determinado; metáfora para violência ou crueldade}
  \definition{s.}{ferro (Fe) | arma; armamento; refere-se a facas, armas de fogo, etc.}
  \definition{v.}{resolver; determinar}
\end{entry}

\begin{entry}{铁轨}{tie3gui3}{10,6}{⾦、⾞}
  \definition[根]{s.}{trilho | trilho ferroviário}
\end{entry}

\begin{entry}{铁路}{tie3 lu4}{10,13}{⾦、⾜}[HSK 3]
  \definition[条,公里]{s.}{ferrovia; estrada de ferro; uma estrada com trilhos de aço dispostos no leito da estrada para a circulação de trens}
\end{entry}

\begin{entry}{厅}{ting1}{4}{⼚}[HSK 5]
  \definition{s.}{salão; sala grande para reuniões ou receber convidados | escritório; nome de um departamento administrativo de uma grande organização | departamento governamental a nível provincial; nomes de alguns órgãos estaduais}
\end{entry}

\begin{entry}{听}{ting1}{7}{⼝}[HSK 1]
  \definition{clas.}{latas; usado para bebidas e alimentos para levar consigo}
  \definition{s.}{lata; embalagem metálica; recipiente cilíndrico utilizado para armazenar bebidas, alimentos, etc.}
  \definition{v.}{ouvir; escutar | obedecer; dar ouvidos; aceitar | supervisionar; administrar; tratar (assuntos políticos); julgar (casos) | permitir; deixar ser; deixar fazer}
  \seeref{听}{yin3}
\end{entry}

\begin{entry}{听到}{ting1 dao4}{7,8}{⼝、⼑}[HSK 1]
  \definition{v.}{ouvir, escutar; ouvir atentamente, escutar atentamente}
\end{entry}

\begin{entry}{听断}{ting1duan4}{7,11}{⼝、⽄}
  \definition{v.}{ouvir e decidir | julgar (ou seja, ouvir e julgar em um tribunal)}
\end{entry}

\begin{entry}{听骨}{ting1gu3}{7,9}{⼝、⾻}
  \definition{s.}{ossículos (do ouvido médio)}
  \seealsoref{听小骨}{ting1xiao3gu3}
\end{entry}

\begin{entry}{听会}{ting1hui4}{7,6}{⼝、⼈}
  \definition{v.}{participar de uma reunião (e ouvir o que é discutido)}
\end{entry}

\begin{entry}{听见}{ting1 jian4}{7,4}{⼝、⾒}[HSK 1]
  \definition{v.}{ouvir; conseguir ouvir}
\end{entry}

\begin{entry}{听讲}{ting1 jiang3}{7,6}{⼝、⾔}[HSK 2]
  \definition{v.+compl.}{assistir a uma palestra; ouvir palestras ou discursos}
\end{entry}

\begin{entry}{听来}{ting1lai2}{7,7}{⼝、⽊}
  \definition{v.}{ouvir de algum lugar | soar (antigo, estrangeiro, excitante, certo, etc.) | soar como se (ou seja, dar uma impressão ao ouvinte)}
\end{entry}

\begin{entry}{听力}{ting1li4}{7,2}{⼝、⼒}[HSK 3]
  \definition{s.}{audição; capacidade auditiva | compreensão auditiva (na aprendizagem de línguas)}
\end{entry}

\begin{entry}{听力理解}{ting1li4li3jie3}{7,2,11,13}{⼝、⼒、⽟、⾓}
  \definition{s.}{compreensão auditiva}
\end{entry}

\begin{entry}{听命}{ting1ming4}{7,8}{⼝、⼝}
  \definition{v.}{obedecer ordens | receber ordens}
\end{entry}

\begin{entry}{听凭}{ting1ping2}{7,8}{⼝、⼏}
  \definition{v.}{permitir (alguém a fazer o que desejar)}
\end{entry}

\begin{entry}{听说}{ting1 shuo1}{7,9}{⼝、⾔}[HSK 2]
  \definition{v.}{ser informado; ouvir falar de; ouvir dizer | ouvir e falar}
\end{entry}

\begin{entry}{听随}{ting1sui2}{7,11}{⼝、⾩}
  \definition{v.}{permitir | obedecer}
\end{entry}

\begin{entry}{听戏}{ting1xi4}{7,6}{⼝、⼽}
  \definition{v.}{assistir a uma ópera | ver uma ópera}
\end{entry}

\begin{entry}{听小骨}{ting1xiao3gu3}{7,3,9}{⼝、⼩、⾻}
  \definition{s.}{ossículos (do ouvido médio)}
  \seealsoref{听骨}{ting1gu3}
\end{entry}

\begin{entry}{听写}{ting1 xie3}{7,5}{⼝、⼍}[HSK 1]
  \definition{s.}{ditado}
  \definition{v.}{ouvir e escrever}
\end{entry}

\begin{entry}{听众}{ting1 zhong4}{7,6}{⼝、⼈}[HSK 3]
  \definition[位,名,个]{s.}{audiência; ouvintes; pessoas que ouvem palestras, música ou transmissões}
\end{entry}

\begin{entry}{聼}{ting1}{19}{⼼}
  \variantof{听}
\end{entry}

\begin{entry}{亭}{ting2}{9}{⼇}
  \definition{s.}{pavilhão | cabine | quiosque}
\end{entry}

\begin{entry}{停}{ting2}{11}{⼈}[HSK 2]
  \definition{adj.}{pronto; resolvido; bem organizado}
  \definition{clas.}{usado para partes (de um total); porções}
  \definition{v.}{parar; interromper; cessar; fazer uma pausa | permanecer; ficar; fazer uma parada (para descansar) | estacionar; ancorar; atracar}
\end{entry}

\begin{entry}{停办}{ting2ban4}{11,4}{⼈、⼒}
  \definition{v.}{cancelar | sair do negócio | desligar | terminar}
\end{entry}

\begin{entry}{停车}{ting2 che1}{11,4}{⼈、⾞}[HSK 2]
  \definition{v.}{(veículo) parar; frear | estacionar o veículo | parar; deixar de funcionar}
\end{entry}

\begin{entry}{停车场}{ting2 che1 chang3}{11,4,6}{⼈、⾞、⼟}[HSK 2]
  \definition[个]{s.}{estacionamento; área de estacionamento; local para estacionamento de veículos}
\end{entry}

\begin{entry}{停当}{ting2dang5}{11,6}{⼈、⼹}
  \definition{adj.}{realizado | preparado | assentado}
\end{entry}

\begin{entry}{停电}{ting2dian4}{11,5}{⼈、⽥}
  \definition{s.}{corte de energia}
  \definition{v.}{ter uma falha de energia}
\end{entry}

\begin{entry}{停工}{ting2gong1}{11,3}{⼈、⼯}
  \definition{v.}{parar de trabalhar | parar a produção}
\end{entry}

\begin{entry}{停火}{ting2huo3}{11,4}{⼈、⽕}
  \definition{s.}{cessar-fogo}
  \definition{v.+compl.}{cessar fogo}
\end{entry}

\begin{entry}{停课}{ting2ke4}{11,10}{⼈、⾔}
  \definition{v.}{fechar (escola) | parar as aulas}
\end{entry}

\begin{entry}{停留}{ting2 liu2}{11,10}{⼈、⽥}[HSK 5]
  \definition{v.}{permanecer; ficar por muito tempo; parar temporariamente em algum lugar, sem continuar avançando | permanecer; parar por um longo tempo; parar em um determinado estágio ou nível, sem evoluir}
\end{entry}

\begin{entry}{停息}{ting2xi1}{11,10}{⼈、⼼}
  \definition{v.}{cessar | parar}
\end{entry}

\begin{entry}{停下}{ting2 xia4}{11,3}{⼈、⼀}[HSK 4]
  \definition{v.}{encerrar; desligar; parar}
\end{entry}

\begin{entry}{停歇}{ting2xie1}{11,13}{⼈、⽋}
  \definition{v.}{parar para descansar}
\end{entry}

\begin{entry}{停业}{ting2ye4}{11,5}{⼈、⼀}
  \definition{v.}{cessar a negociação (temporária ou permanentemente) | fechar}
\end{entry}

\begin{entry}{停用}{ting2yong4}{11,5}{⼈、⽤}
  \definition{v.}{desabilitar | descontinuar | parar de usar | suspender}
\end{entry}

\begin{entry}{停止}{ting2 zhi3}{11,4}{⼈、⽌}[HSK 3]
  \definition{v.}{parar; suspender; cessar; cancelar}
\end{entry}

\begin{entry}{挺}{ting3}{9}{⼿}[HSK 2,4]
  \definition{adj.}{rígido; ereto; vertical; reto | notável; destacado; distinto}
  \definition{adv.}{muito; bastante}
  \definition{clas.}{usado para metralhadoras}
  \definition{v.}{sobressair; endireitar-se; protrudir (protuberância ou saliência) | suportar; aguentar; resistir; perseverar}
\end{entry}

\begin{entry}{挺拔}{ting3ba2}{9,8}{⼿、⼿}
  \definition{adj.}{alto e reto}
\end{entry}

\begin{entry}{挺杆}{ting3gan3}{9,7}{⼿、⽊}
  \definition{s.}{tucho (peça de máquina)}
\end{entry}

\begin{entry}{挺过}{ting3guo4}{9,6}{⼿、⾡}
  \definition{s.}{sobreviver}
\end{entry}

\begin{entry}{挺好}{ting3 hao3}{9,6}{⼿、⼥}[HSK 2]
  \definition{adj.}{nada mal; surpreendentemente bom}
\end{entry}

\begin{entry}{挺进}{ting3jin4}{9,7}{⼿、⾡}
  \definition{s.}{progresso | avanço}
  \definition{v.}{progredir | avançar}
\end{entry}

\begin{entry}{挺立}{ting3li4}{9,5}{⼿、⽴}
  \definition{v.}{ficar ereto | ficar de pé}
\end{entry}

\begin{entry}{挺身}{ting3shen1}{9,7}{⼿、⾝}
  \definition{v.}{endireitar as costas}
\end{entry}

\begin{entry}{挺尸}{ting3shi1}{9,3}{⼿、⼫}
  \definition{v.}{(coloquial) dormir | (literalmente) ficar deitado duro como um cadáver}
\end{entry}

\begin{entry}{挺腰}{ting3yao1}{9,13}{⼿、⾁}
  \definition{v.}{arquear as costas | endireitar as costas}
\end{entry}

\begin{entry}{挺住}{ting3zhu4}{9,7}{⼿、⼈}
  \definition{v.}{permanecer firme | manter-se firme (diante da adversidade ou da dor)}
\end{entry}

\begin{entry}{通}{tong1}{10}{⾡}[HSK 2]
  \definition*{s.}{Sobrenome Tong}
  \definition{adj.}{lógico; coerente | geral; comum | tudo; inteiro | aberto; através de | total}
  \definition{clas.}{(antigo) usado para cartas, telegramas, documentos oficiais, etc.}
  \definition{s.}{autoridade; especialista}
  \definition{suf.}{especialista}
  \definition{v.}{abrir; atravessar | abrir ou limpar cutucando ou espetando | levar a; ir a | conectar; comunicar | notificar; informar | compreender; saber | cutucar; dar uma pancada | transmitir; conectar; interagir | dominar; compreender; entender}
  \seeref{通}{tong4}
\end{entry}

\begin{entry}{通常}{tong1chang2}{10,11}{⾡、⼱}[HSK 3]
  \definition{adj.}{usual; normal; geral}
  \definition{adv.}{habitualmente; usualmente; geralmente; ordinariamente}
\end{entry}

\begin{entry}{通牒}{tong1die2}{10,13}{⾡、⽚}
  \definition{s.}{nota diplomática}
\end{entry}

\begin{entry}{通观}{tong1guan1}{10,6}{⾡、⾒}
  \definition{v.}{ter uma visão geral de algo}
\end{entry}

\begin{entry}{通过}{tong1guo4}{10,6}{⾡、⾡}[HSK 2]
  \definition{prep.}{por; através de; por meio de; por meio de; meios, métodos, etc. para introduzir ações}
  \definition{v.}{atravessar; passar por; transitar | aprovar; adotar | solicitar o consentimento ou aprovação de}
\end{entry}

\begin{entry}{通识}{tong1shi2}{10,7}{⾡、⾔}
  \definition{s.}{conhecimento comum | erudição | conhecimento geral | amplamente conhecido}
\end{entry}

\begin{entry}{通信}{tong1 xin4}{10,9}{⾡、⼈}[HSK 3]
  \definition{v.+compl.}{corresponder; comunicar por carta; comunicar situações e informações escrevendo cartas | transmitir (ou transportar) mensagem; passar (ou transmitir) informação; usar ondas de rádio e outros sinais para transmitir texto, imagens, etc.}
\end{entry}

\begin{entry}{通用}{tong1yong4}{10,5}{⾡、⽤}[HSK 5]
  \definition{adj.}{de uso comum; universal; (em um determinado âmbito) de uso generalizado | intercambiável; alguns caracteres chineses com grafia diferente, mas pronúncia igual, podem ser usados indistintamente (alguns limitados a um determinado significado)}
\end{entry}

\begin{entry}{通知}{tong1zhi1}{10,8}{⾡、⽮}[HSK 2]
  \definition[份,个,张]{s.}{aviso; circular; notificação por escrito ou verbal}
  \definition{v.}{aconselhar; notificar; informar; dar aviso prévio}
\end{entry}

\begin{entry}{通知书}{tong1 zhi1 shu1}{10,8,4}{⾡、⽮、⼄}[HSK 4]
  \definition{s.}{aviso; observação; notificação}
\end{entry}

\begin{entry}{同}{tong2}{6}{⼝}[HSK 6]
  \definition{adj.}{como; igual; parecido; similar; o mesmo; sem diferença}
  \definition{adv.}{juntos; em comum; indica que diferentes atores realizam uma determinada ação juntos ou estão na mesma situação, o que equivale a 一同 ou 一起}
  \definition{v.}{ser o mesmo que}
  \seeref{同}{tong4}
  \seealsoref{一起}{yi4qi3}
  \seealsoref{一同}{yi4tong2}
\end{entry}

\begin{entry}{同伙}{tong2huo3}{6,6}{⼝、⼈}
  \definition[个]{s.}{cúmplice | colega}
\end{entry}

\begin{entry}{同流合污}{tong2liu2he2wu1}{6,10,6,6}{⼝、⽔、⼝、⽔}
  \definition{expr.}{chafurdar na lama com alguém | seguir o mau exemplo dos outros}
\end{entry}

\begin{entry}{同情}{tong2qing2}{6,11}{⼝、⼼}[HSK 4]
  \definition{s.}{simpatia}
  \definition{v.}{simpatizar com; solidarizar-se; compadecer-se; ter empatia emocional pelo que os outros estão passando}
\end{entry}

\begin{entry}{同时}{tong2shi2}{6,7}{⼝、⽇}[HSK 2]
  \definition{conj.}{além disso; além do mais; ainda mais; indica uma relação de equivalência, geralmente com um significado mais profundo}
  \definition{s.}{enquanto isso; ao mesmo tempo}
\end{entry}

\begin{entry}{同事}{tong2shi4}{6,8}{⼝、⼅}[HSK 2]
  \definition[个,位,名]{s.}{companheiro; colega; colega de trabalho; pessoas que trabalham juntas}
  \definition{v.}{trabalhar no mesmo lugar; trabalhar juntos; trabalhar na mesma unidade}
\end{entry}

\begin{entry}{同屋}{tong2wu1}{6,9}{⼝、⼫}
  \definition[个]{s.}{companheiro de quarto | colega de quarto}
\end{entry}

\begin{entry}{同性恋}{tong2xing4lian4}{6,8,10}{⼝、⼼、⼼}
  \definition{s.}{homossexualidade | pessoa gay | amor gay}
\end{entry}

\begin{entry}{同学}{tong2xue2}{6,8}{⼝、⼦}[HSK 1]
  \definition[位,个,些]{s.}{colega de escola; colega de turma; colega de estudos; pessoas que estudam na mesma escola}
\end{entry}

\begin{entry}{同砚}{tong2yan4}{6,9}{⼝、⽯}
  \definition[位,个]{s.}{colega de classe | colega estudante}
\end{entry}

\begin{entry}{同样}{tong2 yang4}{6,10}{⼝、⽊}[HSK 2]
  \definition{adj.}{igual; semelhante; similar; idêntico; sem diferença}
\end{entry}

\begin{entry}{同意}{tong2yi4}{6,13}{⼝、⼼}[HSK 3]
  \definition{v.}{concordar; consentir; aprovar; concordar com; dizer sim}
\end{entry}

\begin{entry}{童}{tong2}{12}{⽴}
  \definition*{s.}{Sobrenome Tong}
  \definition{adj.}{virgem; solteira | nu; careca | árido; estéril}
  \definition{s.}{criança | jovem servo; antigamente, referia-se a um servo menor de idade.}
\end{entry}

\begin{entry}{童话}{tong2hua4}{12,8}{⽴、⾔}[HSK 4]
  \definition[个,部]{s.}{conto de fadas; gênero de literatura infantil no qual as histórias adequadas para a diversão das crianças são escritas com muita imaginação, fantasia e exagero}
\end{entry}

\begin{entry}{童年}{tong2 nian2}{12,6}{⽴、⼲}[HSK 4]
  \definition{s.}{infância; primeiros anos de vida}
\end{entry}

\begin{entry}{统}{tong3}{9}{⽷}
  \definition{adv.}{todos; juntos; de forma unificada | inteiramente; totalmente}
  \definition{s.}{interligado; inter-relacionado | sistema interconectado | qualquer parte em forma de tubo de uma peça de roupa, etc.; o mesmo que 筒}
  \definition{v.}{reunir em um; unir | unir; liderar; comandar}
  \seealsoref{筒}{tong3}
\end{entry}

\begin{entry}{统计}{tong3ji4}{9,4}{⽷、⾔}[HSK 4]
  \definition{v.}{compilar estatísticas; refere-se à realização de trabalho estatístico, ou seja, coletar, reunir, analisar e extrapolar dados sobre um fenômeno | somar; adicionar; contar}
\end{entry}

\begin{entry}{统一}{tong3yi1}{9,1}{⽷、⼀}[HSK 4]
  \definition{adj.}{unificado; unitário; centralizado; consistente}
  \definition{v.}{unificar; unir; integrar; padronizar}
\end{entry}

\begin{entry}{筒}{tong3}{12}{⽵}
  \definition[个]{s.}{seção de bambu grosso; tubo grosso de bambu | objeto em forma de tubo largo | a parte em forma de tubo das roupas etc.}
\end{entry}

\begin{entry}{同}{tong4}{6}{⼝}
  \definition[条,处]{s.}{beco; rua estreita}
  \seealsoref{胡同}{hu2tong5}
\end{entry}

\begin{entry}{通}{tong4}{10}{⾡}
  \definition{clas.}{usado para uma atividade, tomada em sua totalidade (discurso de abuso, período de reprodução de música, bebedeira, etc.)}
  \seeref{通}{tong1}
\end{entry}

\begin{entry}{痛}{tong4}{12}{⽧}[HSK 3]
  \definition{adv.}{extremamente; profundamente; amargamente}
  \definition{s.}{dor; sofrimento | tristeza; pesar}
\end{entry}

\begin{entry}{痛苦}{tong4ku3}{12,8}{⽧、⾋}[HSK 3]
  \definition{adj.}{doloroso; angustiado; sentindo-se muito desconfortável física ou mentalmente}
  \definition[降,种]{s.}{dor; agonia; sofrimento; refere-se a um estado ou sentimento de extremo desconforto físico ou mental}
\end{entry}

\begin{entry}{痛快}{tong4kuai4}{12,7}{⽧、⼼}[HSK 4]
  \definition{adj.}{encantado; alegre; muito feliz; confortável | franco; direto; simples e direto}
\end{entry}

\begin{entry}{痛骂}{tong4ma4}{12,9}{⽧、⾺}
  \definition{v.}{repreender severamente}
\end{entry}

\begin{entry}{偷}{tou1}{11}{⼈}[HSK 5]
  \definition{adv.}{furtivamente; secretamente; às escondidas}
  \definition{s.}{ladrão; furtador}
  \definition{v.}{roubar; furtar; levar sem pagar; roubar os bens alheios às escondidas | encontrar (tempo) | deixar-se levar; viver apenas para o presente, sem se preocupar com o futuro}
\end{entry}

\begin{entry}{偷安}{tou1'an1}{11,6}{⼈、⼧}
  \definition{v.}{buscar facilidade temporária}
\end{entry}

\begin{entry}{偷渡}{tou1du4}{11,12}{⼈、⽔}
  \definition{s.}{contrabando | imigração ilegal | clandestino (em um navio)}
  \definition{v.}{executar um bloqueio | roubar através da fronteira internacional}
\end{entry}

\begin{entry}{偷窃}{tou1qie4}{11,9}{⼈、⽳}
  \definition{v.}{furtar | roubar}
\end{entry}

\begin{entry}{偷情}{tou1qing2}{11,11}{⼈、⼼}
  \definition{v.}{manter um caso de amor clandestino}
\end{entry}

\begin{entry}{偷税}{tou1shui4}{11,12}{⼈、⽲}
  \definition{s.}{evasão fiscal}
\end{entry}

\begin{entry}{偷听}{tou1ting1}{11,7}{⼈、⼝}
  \definition{v.}{bisbilhotar; monitorar (secretamente)}
\end{entry}

\begin{entry}{偷偷}{tou1 tou1}{11,11}{⼈、⼈}[HSK 5]
  \definition{adv.}{secretamente; dissimuladamente; furtivamente; às escondidas}
\end{entry}

\begin{entry}{偷袭}{tou1xi2}{11,11}{⼈、⾐}
  \definition{s.}{ataque surpresa}
  \definition{v.}{montar um ataque furtivo | invadir}
\end{entry}

\begin{entry}{偸}{tou1}{11}{⼈}
  \variantof{偷}
\end{entry}

\begin{entry}{头}{tou2}{5}{⼤}[HSK 2,3]
  \definition{adj.}{(antes de um numeral) primeiro | (antes de 年 ou 天) último; anterior}
  \definition{clas.}{usado para suínos ou gado (animais de criação) | usado para cabeças de alho ou coisas com formato de cabeça}
  \definition{num.}{primeiro}
  \definition{prep.}{antes de; perto de; introduz o tempo de uma ação, equivalente a  在……之前 ou 临近 | (entre dois algarismos, indicando um número aproximado) cerca de}
  \definition[个,颗]{s.}{cabeça; a parte do corpo humano ou animal que possui órgãos como boca, nariz, olhos e ouvidos | cabelo ou penteado | topo; fim; a parte superior ou final de um objeto | começo ou fim; o ponto inicial ou final de algo | fim; remanescente; os restos de algo | cabeça; chefe; líder | lado; aspecto}
  \seeref{头}{tou5}
  \seealsoref{临近}{lin2jin4}
  \seealsoref{年}{nian2}
  \seealsoref{天}{tian1}
  \seealsoref{在}{zai4}
  \seealsoref{之前}{zhi1 qian2}
\end{entry}

\begin{entry}{头发}{tou2fa5}{5,5}{⼤、⼜}[HSK 2]
  \definition[根,缕,头]{s.}{cabelo}
\end{entry}

\begin{entry}{头号}{tou2hao4}{5,5}{⼤、⼝}
  \definition{adj.}{primeira classe | número um | \emph{top rank}}
\end{entry}

\begin{entry}{头脑}{tou2 nao3}{5,10}{⼤、⾁}[HSK 3]
  \definition{s.}{inteligência; mente | pista; tópicos principais | chefe; líder; capitão}
\end{entry}

\begin{entry}{头脑风暴}{tou2nao3feng1bao4}{5,10,4,15}{⼤、⾁、⾵、⽇}
  \definition{s.}{\emph{brainstorm}}
\end{entry}

\begin{entry}{头头}{tou2tou2}{5,5}{⼤、⼤}
  \definition{s.}{chefe | o cabeça}
\end{entry}

\begin{entry}{头像}{tou2xiang4}{5,13}{⼤、⼈}
  \definition{s.}{retrato | busto | avatar | imagem de perfil (computação)}
\end{entry}

\begin{entry}{投}{tou2}{7}{⼿}[HSK 4]
  \definition*{s.}{Sobrenome Tou}
  \definition{pron.}{para; indica tempo, equivalente a 到, 临 | para; em direção a; indica orientação, direção, equivalente a 朝 ou 向}
  \definition{s.}{um jogo durante uma festa em que o vencedor era decidido pelo número de flechas lançadas em um pote distante | jogo de dados}
  \definition{v.}{lançar; arremessar; atirar | deixar cair; colocar em; lançar | mergulhar em; lançar-se em; pular dentro | lançar; projetar; sombrear | entregar; postar; enviar | ir até; ir para; buscar; juntar-se | sentir-se atraído por; adaptar-se a; concordar com; atender a}
  \seealsoref{朝}{chao2}
  \seealsoref{到}{dao4}
  \seealsoref{临}{lin2}
  \seealsoref{向}{xiang4}
\end{entry}

\begin{entry}{投递}{tou2di4}{7,10}{⼿、⾡}
  \definition{v.}{despachar | enviar}
\end{entry}

\begin{entry}{投票}{tou2piao4}{7,11}{⼿、⽰}
  \definition{v.+compl.}{votar | depositar um voto}
\end{entry}

\begin{entry}{投入}{tou2ru4}{7,2}{⼿、⼊}[HSK 4]
  \definition{adj.}{sisudo; dedicado; devotado; absorto}
  \definition{s.}{investimento; insumo; refere-se à aplicação de recursos}
  \definition{v.}{lançar em; colocar em; jogar em; por em | entrar em uma situação; participar de | aplicar; investir; colocar fundos em}
\end{entry}

\begin{entry}{投诉}{tou2su4}{7,7}{⼿、⾔}[HSK 4]
  \definition{v.}{reclamar; queixar-se; reclamar às autoridades ou pessoas envolvidas}
\end{entry}

\begin{entry}{投资}{tou2zi1}{7,10}{⼿、⾙}[HSK 4]
  \definition[次]{s.}{investimento}
  \definition{v.}{investir; aplicar dinheiro; investir dinheiro em negócios}
\end{entry}

\begin{entry}{投资风险}{tou2zi1feng1xian3}{7,10,4,9}{⼿、⾙、⾵、⾩}
  \definition{s.}{risco de investimento}
\end{entry}

\begin{entry}{投资回报率}{tou2zi1hui2bao4lv4}{7,10,6,7,11}{⼿、⾙、⼞、⼿、⽞}
  \definition{s.}{retorno sobre o investimento (ROI)}
\end{entry}

\begin{entry}{投资家}{tou2zi1jia1}{7,10,10}{⼿、⾙、⼧}
  \definition{s.}{investidor}
  \seealsoref{投资人}{tou2zi1ren2}
  \seealsoref{投资者}{tou2zi1zhe3}
\end{entry}

\begin{entry}{投资人}{tou2zi1ren2}{7,10,2}{⼿、⾙、⼈}
  \definition{s.}{investidor}
  \seealsoref{投资家}{tou2zi1jia1}
  \seealsoref{投资者}{tou2zi1zhe3}
\end{entry}

\begin{entry}{投资者}{tou2zi1zhe3}{7,10,8}{⼿、⾙、⽼}
  \definition{s.}{investidor}
  \seealsoref{投资家}{tou2zi1jia1}
  \seealsoref{投资人}{tou2zi1ren2}
\end{entry}

\begin{entry}{透}{tou4}{10}{⾡}[HSK 4]
  \definition{adv.}{totalmente; completamente; minuciosamente | profundamente; extremamente}
  \definition{v.}{penetrar; passar através de; infiltrar-se através de | revelar; deixar transparecer; contar secretamente |mostrar; aparecer}
\end{entry}

\begin{entry}{透彻}{tou4che4}{10,7}{⾡、⼻}
  \definition{adj.}{minucioso | incisivo | penetrante}
\end{entry}

\begin{entry}{透澈}{tou4che4}{10,15}{⾡、⽔}
  \variantof{透彻}
\end{entry}

\begin{entry}{透顶}{tou4ding3}{10,8}{⾡、⾴}
  \definition{adv.}{completamente}
\end{entry}

\begin{entry}{透过}{tou4guo4}{10,6}{⾡、⾡}
  \definition{v.}{passar através | penetrar}
\end{entry}

\begin{entry}{透亮}{tou4liang4}{10,9}{⾡、⼇}
  \definition{adj.}{brilhante | claro como cristal}
\end{entry}

\begin{entry}{透露}{tou4lu4}{10,21}{⾡、⾬}
  \definition{v.}{divulgar | vazar | revelar}
\end{entry}

\begin{entry}{透明}{tou4ming2}{10,8}{⾡、⽇}[HSK 4]
  \definition{adj.}{transparente; diáfano; capaz de transmitir luz | evidente; transparente; situação ou assunto que seja aberto e não oculto | transparente; diáfano; indica pureza, ausência de impurezas}
\end{entry}

\begin{entry}{透辟}{tou4pi4}{10,13}{⾡、⾟}
  \definition{adj.}{incisivo | penetrante}
\end{entry}

\begin{entry}{透气}{tou4qi4}{10,4}{⾡、⽓}
  \definition{v.}{respirar (sobre tecido, etc.) | fluir livremente (sobre ar) | respirar ar fresco | ventilar}
\end{entry}

\begin{entry}{透水}{tou4shui3}{10,4}{⾡、⽔}
  \definition{adj.}{permeável}
  \definition{s.}{vazamento de água}
\end{entry}

\begin{entry}{透支}{tou4zhi1}{10,4}{⾡、⽀}
  \definition{v.}{cheque especial (bancário) | saque a descoberto}
\end{entry}

\begin{entry}{头}{tou5}{5}{⼤}
  \definition{suf.}{adicionado após componentes nominais comuns | adicionado após o componente verbal, forma um substantivo abstrato, geralmente indicando que vale a pena realizar essa ação | adicionado após um componente adjetival, forma um substantivo, geralmente indicando algo abstrato | adicionado após o componente substantivo que indica a direção}
  \seeref{头}{tou2}
\end{entry}

\begin{entry}{突}{tu1}{9}{⽳}
  \definition{adv.}{de repente; abruptamente; inesperadamente}
  \definition{s.}{chaminé}
  \definition{v.}{avançar rapidamente; atacar | projetar; destacar-se | romper | projetar-se; inchar; fazer bojo}
\end{entry}

\begin{entry}{突出}{tu1chu1}{9,5}{⽳、⼐}[HSK 3]
  \definition{adj.}{proeminente; excelente; mais que a média}
  \definition{v.}{romper | enfatizar; destacar; dar destaque a | sobressair; projetar-se; destacar-se}
\end{entry}

\begin{entry}{突破}{tu1po4}{9,10}{⽳、⽯}[HSK 5]
  \definition{v.}{romper; fazer uma descoberta revolucionária; concentrar esforços em um único ponto de ataque, reunir o sucesso | quebrar (limite); superar (dificuldade); superar dificuldades; ultrapassar os números ou limites anteriores, superar recordes anteriores, etc.; romper com as limitações e restrições anteriores}
\end{entry}

\begin{entry}{突然}{tu1ran2}{9,12}{⽳、⽕}[HSK 3]
  \definition{adj.}{repentino; abrupto; inesperado}
  \definition{adv.}{de repente; abruptamente; inesperadamente; subitamente}
\end{entry}

\begin{entry}{图}{tu2}{8}{⼞}[HSK 3]
  \definition*{s.}{Sobrenome Tu}
  \definition[张]{s.}{mapa; gráfico; imagem; desenho | plano; esquema; tentativa}
  \definition{v.}{procurar; perseguir; esperar obter| desenhar; retratar; pintar | imaginar; planejar; pensar; maquinar}
\end{entry}

\begin{entry}{图案}{tu2'an4}{8,10}{⼞、⽊}[HSK 4]
  \definition{s.}{padrão; desenho; padrões e gráficos usados para decoração de edifícios, tecidos, artes e artesanato, etc.}
\end{entry}

\begin{entry}{图画}{tu2 hua4}{8,8}{⼞、⽥}[HSK 3]
  \definition[幅,张,套]{s.}{desenho; imagem; pintura}
\end{entry}

\begin{entry}{图片}{tu2 pian4}{8,4}{⼞、⽚}[HSK 2]
  \definition[张,幅]{s.}{imagem; fotografia; um termo geral para imagens, fotografias, decalques, etc. usados para ilustrar algo}
\end{entry}

\begin{entry}{图书馆}{tu2shu1guan3}{8,4,11}{⼞、⼄、⾷}[HSK 1]
  \definition[个,家]{s.}{biblioteca; instituição que coleta, organiza e armazena livros e materiais para leitura e consulta}
\end{entry}

\begin{entry}{徒}{tu2}{10}{⼻}
  \definition{adj.}{vazio; nu}
  \definition{adv.}{somente; meramente; apenas | a pé | em vão; sem sucesso; sem sucesso}
  \definition{s.}{aprendiz; aluno | seguidor; crente | (pejorativo) pessoas da mesma facção | (pejorativo) pessoa; companheiro | (prisão) pena; prisão; sentença | estudante}
  \definition{v.}{estar a pé | andar}
\end{entry}

\begin{entry}{徒手}{tu2shou3}{10,4}{⼻、⼿}
  \definition{adj.}{com as mãos vazias | desarmado | mão livre (desenho) | lutando mão-a-mão}
\end{entry}

\begin{entry}{途}{tu2}{10}{⾡}
  \definition[条]{s.}{caminho; estrada; rota | jornada; caminho}
\end{entry}

\begin{entry}{途中}{tu2 zhong1}{10,4}{⾡、⼁}[HSK 4]
  \definition{adv.}{no caminho; ao longo do caminho}
\end{entry}

\begin{entry}{土}{tu3}{3}{⼟}[HSK 3,6][Kangxi 32]
  \definition*{s.}{Sobrenome Tu}
  \definition{adj.}{local; nativo; local com características regionais| caseiro; indígena; o que é tradicional no país; popular | não refinado; não esclarecido; não está na moda; não é popular}
  \definition[堆,捧,层]{s.}{solo; terra | terra; território | ópio bruto | cidade natal; terra natal; pátria}
\end{entry}

\begin{entry}{土地}{tu3di4}{3,6}{⼟、⼟}[HSK 4]
  \definition[片,块]{s.}{terra; solo; chão; superfície terrestre da Terra usada para cultivar, construir edifícios e viver | território}
  \seeref{土地}{tu3di5}
\end{entry}

\begin{entry}{土地}{tu3di5}{3,6}{⼟、⼟}
  \definition{s.}{deus da audeia; deus local; \emph{genius loci} deidade protetora de um local; (superstição) refere-se ao deus da terra que governa uma pequena área}
  \seeref{土地}{tu3di4}
\end{entry}

\begin{entry}{土豆}{tu3dou4}{3,7}{⼟、⾖}[HSK 5]
  \definition[个,片,块,斤]{s.}{batata; denominação comum da batata}
\end{entry}

\begin{entry}{土豆泥}{tu3dou4ni2}{3,7,8}{⼟、⾖、⽔}
  \definition{s.}{purê de batatas}
\end{entry}

\begin{entry}{土鸡}{tu3ji1}{3,7}{⼟、⿃}
  \definition{s.}{galinha caipira}
\end{entry}

\begin{entry}{吐}{tu3}{6}{⼝}[HSK 5]
  \definition{v.}{cuspir; sair pela boca | surgir ou aparecer pela boca ou por uma fenda | dizer; contar; falar abertamente}
  \seeref{吐}{tu4}
\end{entry}

\begin{entry}{吐}{tu4}{6}{⼝}[HSK 5]
  \definition{v.}{vomitar; sair pela boca | vomitar; expelir; metáfora para ser forçado a devolver bens usurpados}
  \seeref{吐}{tu3}
\end{entry}

\begin{entry}{兔}{tu4}{8}{⼉}[HSK 5]
  \definition[只]{s.}{lebre; coelho}
\end{entry}

\begin{entry}{兔子}{tu4zi5}{8,3}{⼉、⼦}
  \definition[只]{s.}{coelho | lebre}
\end{entry}

\begin{entry}{团}{tuan2}{6}{⼞}[HSK 3]
  \definition*{s.}{Liga da Juventude Comunista da China; Liga}
  \definition{adj.}{redondo; circular}
  \definition{clas.}{usado para algo em forma de bola}
  \definition[个]{s.}{bolinho de massa; comida em forma de bola feita de arroz ou farinha | algo em forma de bola | grupo; corpo; sociedade; organização; um grupo envolvido em um determinado trabalho ou atividade | regimento; unidade organizacional militar, geralmente abaixo do nível de divisão e acima do nível de batalhão}
  \definition{v.}{enrolar algo para formar uma bola; rolar | reunir; unir; conglomerar}
\end{entry}

\begin{entry}{团队}{tuan2dui4}{6,4}{⼞、⾩}
  \definition{s.}{equipe}
\end{entry}

\begin{entry}{团结}{tuan2jie2}{6,9}{⼞、⽷}[HSK 3]
  \definition{adj.}{unido; amigável; harmonioso; relação harmoniosa e coexistência harmoniosa}
  \definition{v.}{unir; reunir}
\end{entry}

\begin{entry}{团体}{tuan2ti3}{6,7}{⼞、⼈}[HSK 3]
  \definition[种,个]{s.}{equipe; grupo; organização; um grupo de pessoas com objetivos e interesses comuns}
\end{entry}

\begin{entry}{团长}{tuan2 zhang3}{6,4}{⼞、⾧}[HSK 5]
  \definition{s.}{comandante do regimento | chefe (ou presidente) de uma delegação, trupe, etc. | líder de uma delegação}
\end{entry}

\begin{entry}{推}{tui1}{11}{⼿}[HSK 2]
  \definition{v.}{empurrar; dar um encontrão | girar um moinho ou uma pedra de amolar; moer | cortar; aparar | impulsionar; promover; avançar | inferir; deduzir | afastar; fugir; deslocar | adiar | eleger; escolher | ter em alta estima; elogiar muito | declinar | selecionar | elogiar muito}
\end{entry}

\begin{entry}{推迟}{tui1chi2}{11,7}{⼿、⾡}[HSK 4]
  \definition{v.}{adiar; postergar; tardar; deixar para mais tarde}
\end{entry}

\begin{entry}{推动}{tui1 dong4}{11,6}{⼿、⼒}[HSK 3]
  \definition{v.}{promover; atuar; impulsionar; empurrar para a frente; dar ímpeto a; começar ou avançar algo (com alguma força); começar a trabalhar}
\end{entry}

\begin{entry}{推广}{tui1guang3}{11,3}{⼿、⼴}[HSK 3]
  \definition{v.}{espalhar; estender; promover; popularizar; expandir o escopo de uso ou função de algo}
\end{entry}

\begin{entry}{推介}{tui1jie4}{11,4}{⼿、⼈}
  \definition{s.}{promoção}
  \definition{v.}{promover | introduzir e recomendar}
\end{entry}

\begin{entry}{推进}{tui1 jin4}{11,7}{⼿、⾡}[HSK 3]
  \definition{v.}{avançar; empurrar; levar adiante; dar ímpeto a; promover o trabalho e fazê-lo avançar | empurrar; dirigir; avançar; seguir em frente; seguir em frente}
\end{entry}

\begin{entry}{推开}{tui1 kai1}{11,4}{⼿、⼶}[HSK 3]
  \definition{v.}{declinar; rejeitar | empurrar para longe; aplicar força em uma determinada direção para mover uma pessoa ou objeto para longe de seu lugar original | empurrar para abrir (um portão, etc.); empurrar para fora para abrir algo que está fechado | estender; popularizar; promover para um alcance mais amplo e realizar em uma escala mais ampla}
\end{entry}

\begin{entry}{推销}{tui1xiao1}{11,12}{⼿、⾦}[HSK 4]
  \definition{v.}{vender; comercializar; promover vendas; promover a comercialização de mercadorias}
\end{entry}

\begin{entry}{推行}{tui1 xing2}{11,6}{⼿、⾏}[HSK 5]
  \definition{v.}{realizar; prosseguir; praticar | implementar; praticar; implementação generalizada; divulgar (experiências, métodos, etc.)}
\end{entry}

\begin{entry}{腿}{tui3}{13}{⾁}[HSK 2]
  \definition[条,双]{s.}{perna; as partes dos humanos e dos animais que sustentam o corpo e permitem caminhar | um suporte em forma de perna; a parte inferior de um objeto que atua como uma perna e serve de suporte | presunto}
\end{entry}

\begin{entry}{腿号}{tui3hao4}{13,5}{⾁、⼝}
  \definition{s.}{anilha numerada (por exemplo, usada para identificar pássaros)}
  \seealsoref{腿号箍}{tui3hao4gu1}
\end{entry}

\begin{entry}{腿号箍}{tui3hao4gu1}{13,5,14}{⾁、⼝、⽵}
  \definition{s.}{anilha numerada (por exemplo, usada para identificar pássaros)}
  \seealsoref{腿号}{tui3hao4}
\end{entry}

\begin{entry}{退}{tui4}{9}{⾡}[HSK 3]
  \definition{v.}{recuar; mover-se para trás  (oposto de 進) | remover; retirar; fazer recuar; mover para trás | desistir; retirar-se de | refluir; declinar; retroceder | aposentar-se; deixar o emprego por atingir a idade estipulada ou por problemas de saúde | retornar; reembolsar; devolver | romper; cancelar o que foi decidido}
  \seealsoref{进}{jin4}
\end{entry}

\begin{entry}{退出}{tui4 chu1}{9,5}{⾡、⼐}[HSK 3]
  \definition{v.}{desistir; retirar-se; separar-se; retirar-se de; abandonar o local ou outro lugar e parar de participar; abandonaar o grupo ou organização}
\end{entry}

\begin{entry}{退休}{tui4xiu1}{9,6}{⾡、⼈}[HSK 3]
  \definition{v.+compl.}{aposentar-se; os trabalhadores que deixarem o emprego por velhice ou invalidez causada pelo trabalho receberão as despesas de subsistência conforme o cronograma}
\end{entry}

\begin{entry}{吞}{tun1}{7}{⼝}[HSK 6]
  \definition*{s.}{Sobrenome Tun}
  \definition{v.}{engolir; engolir em seco | tomar posse de; anexar | engolir; tragar; devorar; engolir inteiro ou em pedaços | absorver; engolir; engolfar}
\end{entry}

\begin{entry}{屯}{tun2}{4}{⼬}
  \definition*{s.}{Sobrenome Tun}
  \definition{s.}{vila (geralmente usado em nomes de vilas); vilarejos; aldeias; povoados}
  \definition{v.}{coletar; estocar; armazenar; acumular | estacionar (tropas); aquartelar}
  \seeref{屯}{zhun1}
\end{entry}

\begin{entry}{托}{tuo1}{6}{⼿}[HSK 6]
  \definition{clas.}{torr, uma unidade de pressão, 1 torr é igual à pressão de 1 mmHg, ou 133,322 Pa}
  \definition{s.}{algo servindo como suporte | fantoche; cúmplice; pessoas que ajudam golpistas a enganar outras pessoas}
  \definition{v.}{segurar na palma; apoiar com a mão ou palma; suportar (um objeto) com um objeto ou com a palma da mão | destacar; servir como contraste | pedir; confiar | implorar; dar como pretexto | dever a; confiar em}
\end{entry}

\begin{entry}{拖}{tuo1}{8}{⼿}[HSK 6]
  \definition{v.}{puxar; arrastar; transportar; puxar um objeto para movê-lo contra o solo ou outra superfície | esfregar; limpar o chão com uma ferramenta especial para esfregar | atrasar; prolongar; procrastinar; arrastar; coisas que deveriam ser feitas nunca são iniciadas ou concluídas; uma certa nota é prolongada por um longo tempo | atrasar; conter; segurar; restringir}
\end{entry}

\begin{entry}{拖拉机}{tuo1la1ji1}{8,8,6}{⼿、⼿、⽊}
  \definition[台]{s.}{trator}
\end{entry}

\begin{entry}{拖鞋}{tuo1xie2}{8,15}{⼿、⾰}
  \definition[双,只]{s.}{chinelos | sandálias}
\end{entry}

\begin{entry}{脱}{tuo1}{11}{⾁}[HSK 4]
  \definition{conj.}{se; no caso;}
  \definition{v.}{(cabelo, pele) soltar-se; desprender-se; cair | retirar peça de roupa do corpo | sair de; escapar de | perder (palavras) | livrar-se de algo}
\end{entry}

\begin{entry}{脱离}{tuo1li2}{11,10}{⾁、⼇}[HSK 5]
  \definition{v.}{separar-se; divorciar-se; afastar-se; sair (de um determinado ambiente ou situação); romper (uma determinada relação)}
\end{entry}

\begin{entry}{脱毛}{tuo1mao2}{11,4}{⾁、⽑}
  \definition{s.}{depilação}
  \definition{v.}{perder cabelo ou penas | depilar | fazer a barba}
\end{entry}

\begin{entry}{脱险}{tuo1xian3}{11,9}{⾁、⾩}
  \definition{v.}{sair do perigo}
\end{entry}

\begin{entry}{鸵}{tuo2}{10}{⿃}
  \definition[只]{s.}{avestruz}
\end{entry}

\begin{entry}{鸵鸟}{tuo2niao3}{10,5}{⿃、⿃}
  \definition{s.}{avestruz}
\end{entry}

\begin{entry}{唾}{tuo4}{11}{⼝}
  \definition[口]{s.}{saliva; cuspe}
  \definition{v.}{cuspir (mostrar desprezo) | rejeitar}
\end{entry}

\begin{entry}{唾骂}{tuo4ma4}{11,9}{⼝、⾺}
  \definition{v.}{insultar | amaldiçoar}
\end{entry}

%%%%% EOF %%%%%


%%%
%%% U
%%%
%\section*{U}
%\addcontentsline{toc}{section}{U}
%\begin{multicols}{2}
%\end{multicols}

%%%
%%% V
%%%
%\section*{V}
%\addcontentsline{toc}{section}{V}
%\begin{multicols}{2}
%\end{multicols}

%%%
%%% W
%%%

\section*{W}\addcontentsline{toc}{section}{W}

\begin{entry}{哇塞}{wa1sai1}{9,13}{⼝、⼟}
  \definition{interj.}{(gíria) Uau!}
\end{entry}

\begin{entry}{哇噻}{wa1sai1}{9,16}{⼝、⼝}
  \variantof{哇塞}
\end{entry}

\begin{entry}{挖}{wa1}{9}{⼿}
  \definition{v.}{cavar | escavar}
\end{entry}

\begin{entry}{挖掘机}{wa1jue2ji1}{9,11,6}{⼿、⼿、⽊}
  \definition{s.}{escavadeira | escavador | escavadora | pá mecânica}
\end{entry}

\begin{entry}{瓦}{wa3}{4}{⽡}[Kangxi 98]
  \definition{s.}{telha | abreviação de 瓦特}
  \seeref{瓦特}{wa3te4}
\end{entry}

\begin{entry}{瓦努阿图}{wa3nu3'a1tu2}{4,7,7,8}{⽡、⼒、⾩、⼞}
  \definition*{s.}{Vanuatu, país do sudoeste do Oceano Pacífico}
\end{entry}

\begin{entry}{瓦特}{wa3te4}{4,10}{⽡、⽜}
  \definition{s.}{(empréstimo linguístico) watt | medida de potência}
\end{entry}

\begin{entry}{袜子}{wa4zi5}{10,3}{⾐、⼦}[HSK 4]
  \definition[双,只,对]{s.}{meias; peúgas; meias-calças}
\end{entry}

\begin{entry}{歪}{wai1}{9}{⽌}
  \definition{adj.}{torto | tortuoso | nocivo}
\end{entry}

\begin{entry}{歪果仁}{wai1guo3ren2}{9,8,4}{⽌、⽊、⼈}
  \definition{s.}{gíria na \emph{Internet} para 外国人}
  \seeref{外国人}{wai4guo2ren2}
\end{entry}

\begin{entry}{外}{wai4}{5}{⼣}[HSK 1]
  \definition{s.}{fora | por fora | exterior | estrangeiro}
\end{entry}

\begin{entry}{外边}{wai4bian5}{5,5}{⼣、⾡}[HSK 1]
  \definition{adv.}{fora do país | superfície externa | fora | lugar diferente de sua casa}
\end{entry}

\begin{entry}{外插}{wai4cha1}{5,12}{⼣、⼿}
  \definition{v.}{extrapolar | (computação) conectar (um dispositivo periférico, etc.)}
\end{entry}

\begin{entry}{外地}{wai4 di4}{5,6}{⼣、⼟}[HSK 2]
  \definition{s.}{não local | outros lugares}
\end{entry}

\begin{entry}{外公}{wai4gong1}{5,4}{⼣、⼋}
  \definition{s.}{avô materno}
\end{entry}

\begin{entry}{外国}{wai4guo2}{5,8}{⼣、⼞}[HSK 1]
  \definition[个]{s.}{país estrangeiro}
\end{entry}

\begin{entry}{外国人}{wai4guo2ren2}{5,8,2}{⼣、⼞、⼈}
  \definition{s.}{estrangeiro | pessoa de fora do país}
\end{entry}

\begin{entry}{外海}{wai4hai3}{5,10}{⼣、⽔}
  \definition{s.}{mar aberto}
\end{entry}

\begin{entry}{外号}{wai4hao4}{5,5}{⼣、⼝}
  \definition{s.}{apelido}
\end{entry}

\begin{entry}{外汇}{wai4 hui4}{5,5}{⼣、⽔}[HSK 4]
  \definition{s.}{câmbio estrangeiro; moeda estrangeira; moedas estrangeiras e títulos, como cheques, letras de câmbio, notas promissórias, etc., conversíveis em moedas estrangeiras, usados na compensação do comércio internacional}
\end{entry}

\begin{entry}{外积}{wai4ji1}{5,10}{⼣、⽲}
  \definition{s.}{produto exterior | (matemática) o produto vetorial de dois vetores}
\end{entry}

\begin{entry}{外交}{wai4jiao1}{5,6}{⼣、⼇}[HSK 3]
  \definition{adj.}{diplomático}
  \definition[个]{s.}{diplomacia; relações exteriores}
\end{entry}

\begin{entry}{外交官}{wai4 jiao1 guan1}{5,6,8}{⼣、⼇、⼧}[HSK 4]
  \definition{s.}{diplomata}
\end{entry}

\begin{entry}{外界}{wai4jie4}{5,9}{⼣、⽥}[HSK 5]
  \definition{s.}{o exterior; o mundo externo; área fora de um determinado âmbito; sociedade externa}
\end{entry}

\begin{entry}{外卖}{wai4 mai4}{5,8}{⼣、⼗}[HSK 2]
  \definition{s.}{para viagem | para fora}
  \definition{v.}{entregar | oferecer}
\end{entry}

\begin{entry}{外贸}{wai4mao4}{5,9}{⼣、⾙}
  \definition{s.}{comércio exterior}
\end{entry}

\begin{entry}{外貌协会}{wai4mao4xie2hui4}{5,14,6,6}{⼣、⾘、⼗、⼈}
  \definition{s.}{``o clube da boa aparência'': pessoas que dão grande importância à aparência de uma pessoa}
  \seealsoref{外协}{wai4xie2}
\end{entry}

\begin{entry}{外面}{wai4 mian4}{5,9}{⼣、⾯}[HSK 3]
  \definition{s.}{o lado de fora | exterior; aparência externa}
\end{entry}

\begin{entry}{外婆}{wai4po2}{5,11}{⼣、⼥}
  \definition{s.}{avó materna}
\end{entry}

\begin{entry}{外事}{wai4shi4}{5,8}{⼣、⼅}
  \definition{s.}{assuntos ou relações exteriores}
\end{entry}

\begin{entry}{外水}{wai4shui3}{5,4}{⼣、⽔}
  \definition{s.}{renda extra}
\end{entry}

\begin{entry}{外孙}{wai4sun1}{5,6}{⼣、⼦}
  \definition{s.}{filho da filha}
\end{entry}

\begin{entry}{外孙女}{wai4sun1nv3}{5,6,3}{⼣、⼦、⼥}
  \definition{s.}{filha da filha}
\end{entry}

\begin{entry}{外套}{wai4 tao4}{5,10}{⼣、⼤}[HSK 4]
  \definition[件,套]{s.}{casaco; jaqueta; paletó; sobretudo}
\end{entry}

\begin{entry}{外围}{wai4wei2}{5,7}{⼣、⼞}
  \definition{adv.}{arredores}
\end{entry}

\begin{entry}{外文}{wai4 wen2}{5,4}{⼣、⽂}[HSK 3]
  \definition{s.}{língua estrangeira (escrita)}
\end{entry}

\begin{entry}{外协}{wai4xie2}{5,6}{⼣、⼗}
  \definition{s.}{terceirização | pessoas que julgam os outros pela aparência}
  \seealsoref{外貌协会}{wai4mao4xie2hui4}
\end{entry}

\begin{entry}{外衣}{wai4yi1}{5,6}{⼣、⾐}
  \definition{s.}{aparência | roupa de cima}
\end{entry}

\begin{entry}{外语}{wai4yu3}{5,9}{⼣、⾔}[HSK 1]
  \definition[门]{s.}{língua estrangeira}
\end{entry}

\begin{entry}{弯}{wan1}{9}{⼸}[HSK 4]
  \definition{adj.}{curvo; dobrado; torto; flexível; tortuoso}
  \definition{s.}{curva; dobra}
  \definition{v.}{curvar; dobrar; flexionar}
\end{entry}

\begin{entry}{豌豆}{wan1dou4}{15,7}{⾖、⾖}
  \definition{s.}{ervilha}
\end{entry}

\begin{entry}{完}{wan2}{7}{⼧}[HSK 2]
  \definition{adj.}{completo | inteiro}
  \definition{adv.}{todo}
  \definition{v.}{acabar | completar | terminar}
\end{entry}

\begin{entry}{完备}{wan2bei4}{7,8}{⼧、⼡}
  \definition{adj.}{completo | impecável | perfeito}
  \definition{v.}{não deixar nada a desejar}
\end{entry}

\begin{entry}{完毕}{wan2bi4}{7,6}{⼧、⽐}
  \definition{v.}{completar | terminar | acabar}
\end{entry}

\begin{entry}{完成}{wan2cheng2}{7,6}{⼧、⼽}[HSK 2]
  \definition{v.}{realizar | completar}
\end{entry}

\begin{entry}{完了}{wan2 le5}{7,2}{⼧、⼅}[HSK 5]
  \definition{v.}{acabar; terminar; concluir; chegar ao fim}
\end{entry}

\begin{entry}{完满}{wan2man3}{7,13}{⼧、⽔}
  \definition{adj.}{satisfatório | bem-sucedido}
\end{entry}

\begin{entry}{完美}{wan2mei3}{7,9}{⼧、⽺}[HSK 3]
  \definition{adj.}{perfeito; impecável; consumado}
  \definition{adv.}{perfeitamente}
  \definition{s.}{perfeição}
\end{entry}

\begin{entry}{完全}{wan2quan2}{7,6}{⼧、⼊}[HSK 2]
  \definition{adj.}{completo | todo}
  \definition{adv.}{inteiramente | totalmente}
\end{entry}

\begin{entry}{完人}{wan2ren2}{7,2}{⼧、⼈}
  \definition{s.}{pessoa perfeita}
\end{entry}

\begin{entry}{完善}{wan2shan4}{7,12}{⼧、⼝}[HSK 3]
  \definition{adj.}{perfeito; consumado}
  \definition{v.}{refinar; melhorar; tornar perfeito}
\end{entry}

\begin{entry}{完税}{wan2shui4}{7,12}{⼧、⽲}
  \definition{v.}{pagar imposto}
\end{entry}

\begin{entry}{完完全全}{wan2wan2quan2quan2}{7,7,6,6}{⼧、⼧、⼊、⼊}
  \definition{adv.}{completamente}
\end{entry}

\begin{entry}{完整}{wan2zheng3}{7,16}{⼧、⽁}[HSK 3]
  \definition{adj.}{intacto; inteiro; completo; integrado}
\end{entry}

\begin{entry}{玩}{wan2}{8}{⽟}
  \definition{s.}{brinquedo | algo usado para diversão}
  \definition{v.}{divertir-se | manter algo para entretenimento | brincar com}
\end{entry}

\begin{entry}{玩伴}{wan2ban4}{8,7}{⽟、⼈}
  \definition{s.}{parceiro de brincadeira}
\end{entry}

\begin{entry}{玩遍}{wan2bian4}{8,12}{⽟、⾡}
  \definition{v.}{passear (todo o país, toda a cidade, etc.) | visitar (um grande número de lugares)}
\end{entry}

\begin{entry}{玩家}{wan2jia1}{8,10}{⽟、⼧}
  \definition{s.}{entusiasta (áudio, modelos de aviões, etc.) | jogador (de um jogo)}
\end{entry}

\begin{entry}{玩具}{wan2ju4}{8,8}{⽟、⼋}[HSK 3]
  \definition[个,件,套,只,辆]{s.}{brinquedo; brincadeira}
\end{entry}

\begin{entry}{玩具厂}{wan2ju4chang3}{8,8,2}{⽟、⼋、⼚}
  \definition{s.}{fábrica de brinquedos}
\end{entry}

\begin{entry}{玩具车}{wan2ju4 che1}{8,8,4}{⽟、⼋、⾞}
  \definition{s.}{carrinho de brinquedo}
\end{entry}

\begin{entry}{玩偶}{wan2'ou3}{8,11}{⽟、⼈}
  \definition{s.}{estatueta de brinquedo | boneco de ação | bicho de pelúcia | boneca}
\end{entry}

\begin{entry}{玩儿}{wan2r5}{8,2}{⽟、⼉}[HSK 1]
  \definition{v.}{divertir-se}
\end{entry}

\begin{entry}{玩耍}{wan2shua3}{8,9}{⽟、⽽}
  \definition{v.}{divertir-me | brincar (como as crianças fazem)}
\end{entry}

\begin{entry}{玩味}{wan2wei4}{8,8}{⽟、⼝}
  \definition{v.}{ponderar sutilezas | ruminar (pensamentos)}
\end{entry}

\begin{entry}{玩艺}{wan2yi4}{8,4}{⽟、⾋}
  \variantof{玩意}
\end{entry}

\begin{entry}{玩意}{wan2yi4}{8,13}{⽟、⼼}
  \definition{s.}{ato | brinquedo | coisa | truque (em uma performance, show de palco, acrobacias, etc.)}
\end{entry}

\begin{entry}{玩者}{wan2zhe3}{8,8}{⽟、⽼}
  \definition{s.}{jogador}
\end{entry}

\begin{entry}{顽强}{wan2qiang2}{10,12}{⾴、⼸}
  \definition{adj.}{persistente | tenaz | difícil de derrotar}
\end{entry}

\begin{entry}{埦}{wan3}{11}{⼟}
  \variantof{碗}
\end{entry}

\begin{entry}{晚}{wan3}{11}{⽇}[HSK 1]
  \definition{adj.}{tarde | noite}
\end{entry}

\begin{entry}{晚安}{wan3'an1}{11,6}{⽇、⼧}[HSK 2]
  \definition{v.}{boa noite}
\end{entry}

\begin{entry}{晚报}{wan3 bao4}{11,7}{⽇、⼿}[HSK 2]
  \definition{s.}{jornal da noite}
\end{entry}

\begin{entry}{晚餐}{wan3can1}{11,16}{⽇、⾷}[HSK 2]
  \definition[份,顿,次]{s.}{jantar | refeição noturna}
\end{entry}

\begin{entry}{晚点}{wan3 dian3}{11,9}{⽇、⽕}[HSK 4]
  \definition{adj.}{atrasado}
  \definition{s.}{jantar leve}
  \definition{v.}{atrasar; retardar; adiar; (carro, navio, avião) partir, correr ou chegar mais tarde do que o horário especificado}
\end{entry}

\begin{entry}{晚饭}{wan3fan4}{11,7}{⽇、⾷}[HSK 1]
  \definition[份,顿,次,餐]{s.}{jantar}
\end{entry}

\begin{entry}{晚会}{wan3hui4}{11,6}{⽇、⼈}[HSK 2]
  \definition[个]{s.}{festa noturna}
\end{entry}

\begin{entry}{晚近}{wan3jin4}{11,7}{⽇、⾡}
  \definition{adj.}{recente | mais recente no passado}
  \definition{adv.}{ultimamente | recentemente}
\end{entry}

\begin{entry}{晚景}{wan3jing3}{11,12}{⽇、⽇}
  \definition{s.}{circunstâncias dos anos de declínio de alguém | cena noturna}
\end{entry}

\begin{entry}{晚上}{wan3shang5}{11,3}{⽇、⼀}[HSK 1]
  \definition{adv.}{noite | à noite}
\end{entry}

\begin{entry}{晚育}{wan3yu4}{11,8}{⽇、⾁}
  \definition{s.}{parto tardio}
  \definition{v.}{ter um filho mais tarde}
\end{entry}

\begin{entry}{碗}{wan3}{13}{⽯}[HSK 2]
  \definition{clas.}{tigelas}
  \definition[只,个]{s.}{tigela}
\end{entry}

\begin{entry}{碗柜}{wan3gui4}{13,8}{⽯、⽊}
  \definition{s.}{armário}
\end{entry}

\begin{entry}{碗子}{wan3zi5}{13,3}{⽯、⼦}
  \definition{s.}{tigela}
\end{entry}

\begin{entry}{万}{wan4}{3}{⼀}[HSK 2]
  \definition*{s.}{sobrenome Wan}
  \definition{adj.}{um grande número}
  \definition{num.}{dez mil; 10.000; 1.0000}
\end{entry}

\begin{entry}{万圣节}{wan4sheng4jie2}{3,5,5}{⼀、⼟、⾋}
  \definition*{s.}{Dia de Todos os Santos}
  \seealsoref{万圣节前夕}{wan4sheng4jie2qian2xi1}
\end{entry}

\begin{entry}{万圣节前夕}{wan4sheng4jie2qian2xi1}{3,5,5,9,3}{⼀、⼟、⾋、⼑、⼣}
  \definition*{s.}{Véspera do Dia de Todos os Santos | \emph{Halloween}}
  \seealsoref{万圣节}{wan4sheng4jie2}
\end{entry}

\begin{entry}{万万}{wan4wan4}{3,3}{⼀、⼀}
  \definition{adv.}{absolutamente | totalmente}
\end{entry}

\begin{entry}{万一}{wan4yi1}{3,1}{⼀、⼀}[HSK 4]
  \definition{conj.}{por via das dúvidas; se por acaso; só por precaução; expressa uma suposição muito improvável (usado para coisas desagradáveis)}
  \definition{num.}{um décimo milionésimo; uma porcentagem muito pequena}
  \definition{s.}{contingência; eventualidade; contingências muito improváveis}
\end{entry}

\begin{entry}{王}{wang2}{4}{⽟}[HSK 4]
  \definition*{s.}{sobrenome Wang}
  \definition{adj.}{grande; ótimo; honoríficos antigos para avós}
  \definition{s.}{rei; monarca; imperador; governante supremo de uma monarquia | cabeça; chefe; líder | o primeiro, maior ou mais forte de seu tipo | duque; príncipe; o título mais alto da sociedade feudal após a dinastia Han}
  \seeref{王}{wang4}
\end{entry}

\begin{entry}{王朝}{wang2chao2}{4,12}{⽟、⽉}
  \definition{s.}{dinastia}
\end{entry}

\begin{entry}{王五}{wang2wu3}{4,4}{⽟、⼆}
  \definition{s.}{Wang Wu | Zé Ninguém | nome para uma pessoa não especificada, 3 de 3}
  \seealsoref{李四}{li3si4}
  \seealsoref{张三}{zhang1san1}
\end{entry}

\begin{entry}{网}{wang3}{6}{⽹}[HSK 2][Kangxi 122]
  \definition{s.}{rede}
\end{entry}

\begin{entry}{网罟}{wang3gu3}{6,10}{⽹、⽹}
  \definition{s.}{(fig.) a rede da justiça | rede usada para capturar peixes (ou outros animais, como pássaros)}
\end{entry}

\begin{entry}{网际网路}{wang3ji4wang3lu4}{6,7,6,13}{⽹、⾩、⽹、⾜}
  \definition*{s.}{\emph{Internet}}
  \seealsoref{互联网}{hu4lian2wang3}
  \seealsoref{网际网络}{wang3ji4wang3luo4}
  \seealsoref{网路}{wang3lu4}
\end{entry}

\begin{entry}{网际网络}{wang3ji4wang3luo4}{6,7,6,9}{⽹、⾩、⽹、⽷}
  \definition*{s.}{\emph{Internet}}
  \seealsoref{互联网}{hu4lian2wang3}
  \seealsoref{网际网路}{wang3ji4wang3lu4}
  \seealsoref{网路}{wang3lu4}
\end{entry}

\begin{entry}{网路}{wang3lu4}{6,13}{⽹、⾜}
  \definition{s.}{\emph{Internet}}
  \seealsoref{互联网}{hu4lian2wang3}
  \seealsoref{网际网路}{wang3ji4wang3lu4}
  \seealsoref{网际网络}{wang3ji4wang3luo4}
\end{entry}

\begin{entry}{网络}{wang3luo4}{6,9}{⽹、⽷}[HSK 4]
  \definition{s.}{rede; um sistema que consiste em ramificações interconectadas; em um sistema elétrico, um circuito ou parte de um circuito que consiste em vários elementos que permitem a transmissão de sinais elétricos de acordo com determinados requisitos | rede; rede de computadores}
\end{entry}

\begin{entry}{网球}{wang3qiu2}{6,11}{⽹、⽟}[HSK 2]
  \definition{s.}{tênis (esporte)}
  \definition[个]{s.}{bola de tênis}
\end{entry}

\begin{entry}{网上}{wang3 shang4}{6,3}{⽹、⼀}[HSK 1]
  \definition{s.}{\emph{online}}
\end{entry}

\begin{entry}{网上银行}{wang3shang4yin2hang2}{6,3,11,6}{⽹、⼀、⾦、⾏}
  \definition[个]{s.}{banco \emph{online} | acesso a operações bancárias via \emph{Internet}}
  \seealsoref{网银}{wang3yin2}
\end{entry}

\begin{entry}{网银}{wang3yin2}{6,11}{⽹、⾦}
  \definition{s.}{banco \emph{online} | acesso a operações bancárias via \emph{Internet}}
  \seealsoref{网上银行}{wang3shang4yin2hang2}
\end{entry}

\begin{entry}{网友}{wang3you3}{6,4}{⽹、⼜}[HSK 1]
  \definition{s.}{internauta | usuário da \emph{Internet}}
\end{entry}

\begin{entry}{网站}{wang3zhan4}{6,10}{⽹、⽴}[HSK 2]
  \definition[个,家]{s.}{\emph{website}}
\end{entry}

\begin{entry}{网址}{wang3 zhi3}{6,7}{⽹、⼟}[HSK 4]
  \definition{s.}{\emph{website}; endereço da \emph{web}; endereço de um \emph{site} na \emph{Internet}, que os usuários podem acessar, consultar e obter recursos de informações nesse \emph{site} clicando nele}
\end{entry}

\begin{entry}{往}{wang3}{8}{⼻}[HSK 2]
  \definition{prep.}{para | em direção a}
\end{entry}

\begin{entry}{往程}{wang3cheng2}{8,12}{⼻、⽲}
  \definition{s.}{saída (de uma viagem de ônibus ou trem, etc.)}
\end{entry}

\begin{entry}{往返}{wang3fan3}{8,7}{⼻、⾡}
  \definition{s.}{ida e volta}
  \definition{v.}{ir e voltar | ir e vir}
\end{entry}

\begin{entry}{往复}{wang3fu4}{8,9}{⼻、⼢}
  \definition{s.}{para trás e para frente (por exemplo, da ação do pistão ou da bomba)}
  \definition{v.}{ir e voltar | fazer uma viagem de volta}
\end{entry}

\begin{entry}{往迹}{wang3ji4}{8,9}{⼻、⾡}
  \definition{s.}{eventos passados}
\end{entry}

\begin{entry}{往来}{wang3lai2}{8,7}{⼻、⽊}
  \definition{s.}{contatos | negociações}
\end{entry}

\begin{entry}{往例}{wang3li4}{8,8}{⼻、⼈}
  \definition{s.}{prática (habitual) do passado | precedente}
\end{entry}

\begin{entry}{往日}{wang3ri4}{8,4}{⼻、⽇}
  \definition{adv.}{dias passados}
  \definition{s.}{o passado}
\end{entry}

\begin{entry}{往生}{wang3sheng1}{8,5}{⼻、⽣}
  \definition{v.}{renascer | morrer | (Budismo) viver no paraíso}
\end{entry}

\begin{entry}{往事}{wang3shi4}{8,8}{⼻、⼅}
  \definition{s.}{acontecimentos anteriores | eventos passados}
\end{entry}

\begin{entry}{往往}{wang3wang3}{8,8}{⼻、⼻}[HSK 3]
  \definition{adv.}{frequentemente; muitas vezes; mais frequentemente do que não}
\end{entry}

\begin{entry}{往昔}{wang3xi1}{8,8}{⼻、⽇}
  \definition{s.}{o passado}
\end{entry}

\begin{entry}{罔}{wang3}{8}{⼌}
  \definition{v.}{enganar}
\end{entry}

\begin{entry}{王}{wang4}{4}{⽟}
  \definition{v.}{reger; governar; reinar; dominar}
  \seeref{王}{wang2}
\end{entry}

\begin{entry}{忘}{wang4}{7}{⼼}[HSK 1]
  \definition{v.}{esquecer | negligenciar | ignorar}
\end{entry}

\begin{entry}{忘本}{wang4ben3}{7,5}{⼼、⽊}
  \definition{v.}{esquecer as próprias raízes}
\end{entry}

\begin{entry}{忘餐}{wang4can1}{7,16}{⼼、⾷}
  \definition{v.}{esquecer as refeições}
\end{entry}

\begin{entry}{忘掉}{wang4diao4}{7,11}{⼼、⼿}
  \definition{v.}{esquecer}
\end{entry}

\begin{entry}{忘恩}{wang4'en1}{7,10}{⼼、⼼}
  \definition{v.}{ser ingrato}
\end{entry}

\begin{entry}{忘怀}{wang4huai2}{7,7}{⼼、⼼}
  \definition{v.}{esquecer}
\end{entry}

\begin{entry}{忘记}{wang4ji4}{7,5}{⼼、⾔}[HSK 1]
  \definition{v.}{esquecer}
\end{entry}

\begin{entry}{忘却}{wang4que4}{7,7}{⼼、⼙}
  \definition{v.}{esquecer}
\end{entry}

\begin{entry}{危害}{wei1hai4}{6,10}{⼙、⼧}[HSK 3]
  \definition{s.}{prejuízo; perigo; dano}
  \definition{v.}{prejudicar; pôr em perigo; pôr em risco}
\end{entry}

\begin{entry}{危急}{wei1ji2}{6,9}{⼙、⼼}
  \definition{adj.}{crítico | desesperadora (situação)}
\end{entry}

\begin{entry}{危难}{wei1nan4}{6,10}{⼙、⾫}
  \definition{s.}{calamidade}
\end{entry}

\begin{entry}{危险}{wei1xian3}{6,9}{⼙、⾩}[HSK 3]
  \definition{adj.}{arriscado; perigoso}
\end{entry}

\begin{entry}{微博}{wei1 bo2}{13,12}{⼻、⼗}[HSK 5]
  \definition*{s.}{Weibo (um aplicativo de mídia social chinês)}
  \definition[条]{s.}{\emph{microblog}}
\end{entry}

\begin{entry}{微风}{wei1feng1}{13,4}{⼻、⾵}
  \definition{s.}{brisa | vento leve}
\end{entry}

\begin{entry}{微软}{wei1ruan3}{13,8}{⼻、⾞}
  \definition*{s.}{\emph{Microsoft Corporation}}
\end{entry}

\begin{entry}{微笑}{wei1xiao4}{13,10}{⼻、⽵}[HSK 4]
  \definition[个,丝]{s.}{sorriso;}
  \definition{v.}{sorrir}
\end{entry}

\begin{entry}{微信}{wei1 xin4}{13,9}{⼻、⼈}[HSK 4]
  \definition*{s.}{\emph{WeChat}; aplicativo gratuito lançado pela Tencent em 21 de janeiro de 2011 para fornecer serviços de mensagens instantâneas para terminais inteligentes}
\end{entry}

\begin{entry}{微型}{wei1xing2}{13,9}{⼻、⼟}
  \definition{pref.}{micro-}
  \definition{s.}{miniatura}
\end{entry}

\begin{entry}{为}{wei2}{4}{⼂}[HSK 3]
  \definition*{s.}{sobrenome Wei}
  \definition{part.}{frequentemente usado com “何” para expressar dúvida}
  \definition{prep.}{como (na capacidade de) | por (na voz passiva)}
  \definition{suf.}{anexado a certos adjetivos monossilábicos, indicando grau ou alcançe | anexado a certos advérbios de grau para fortalecer o tom}
  \definition{v.}{fazer; agir | servir como; agir como; desempenhar o papel de | tornar-se; transformar-se em | ser; significar}
  \seeref{为}{wei4}
  \seealsoref{何}{he2}
\end{entry}

\begin{entry}{为难}{wei2nan2}{4,10}{⼂、⾫}[HSK 5]
  \definition{adj.}{envergonhado; sentir-se constrangido; sentir-se sobrecarregado; sentir-se incapaz de lidar com algo}
  \definition{v.}{dificultar as coisas para; dificultar; contrariar}
\end{entry}

\begin{entry}{为期}{wei2qi1}{4,12}{⼂、⽉}[HSK 5]
  \definition{s.}{tempo restante}
  \definition{v.}{a ser concluído (até uma data definida, por um determinado período de tempo)}
\end{entry}

\begin{entry}{为止}{wei2 zhi3}{4,4}{⼂、⽌}[HSK 5]
  \definition{adv.}{até; até um determinado momento}
\end{entry}

\begin{entry}{为主}{wei2 zhu3}{4,5}{⼂、⼂}[HSK 5]
  \definition{v.}{dar prioridade a; dar preferência a; dar importância a}
\end{entry}

\begin{entry}{围}{wei2}{7}{⼞}[HSK 3]
  \definition*{s.}{sobrenome Wei}
  \definition{clas.}{o comprimento dos dois polegares e indicadores ou o comprimento de ambos os braços quando unidos}
  \definition{s.}{em volta de tudo; ao redor}
  \definition{v.}{cercar; rodear; circundar; encurralar | enrolar; envolver}
\end{entry}

\begin{entry}{围巾}{wei2jin1}{7,3}{⼞、⼱}[HSK 4]
  \definition[条]{s.}{lenço; cachecol; echarpe; gravata; tiras longas de malha ou tecido usadas ao redor do pescoço para aquecimento, proteção do colarinho ou decoração}
\end{entry}

\begin{entry}{围绕}{wei2rao4}{7,9}{⼞、⽷}[HSK 5]
  \definition{v.}{girar; circundar; dar voltas; girar em torno de algo; cercar | concentrar-se em; centrar-se em; centrar-se em uma questão ou evento (para realizar atividades)}
\end{entry}

\begin{entry}{违法}{wei2 fa3}{7,8}{⾡、⽔}[HSK 5]
  \definition{v.}{ser ilegal; infringir a lei; violar a lei ou os regulamentos}
\end{entry}

\begin{entry}{违反}{wei2fan3}{7,4}{⾡、⼜}[HSK 5]
  \definition{v.}{violar; transgredir; contrariar; não estar em conformidade (com as regras, regulamentos, etc.)}
\end{entry}

\begin{entry}{违规}{wei2 gui1}{7,8}{⾡、⾒}[HSK 5]
  \definition{v.}{violar (regras); infringir as regras e regulamentos}
\end{entry}

\begin{entry}{违宪}{wei2xian4}{7,9}{⾡、⼧}
  \definition{adj.}{inconstitucional}
\end{entry}

\begin{entry}{唯一}{wei2yi1}{11,1}{⼝、⼀}[HSK 5]
  \definition{adj.}{único; exclusivo; singular}
\end{entry}

\begin{entry}{维持}{wei2chi2}{11,9}{⽷、⼿}[HSK 4]
  \definition{v.}{manter; conservar; guardar; manter vivo}
\end{entry}

\begin{entry}{维护}{wei2hu4}{11,7}{⽷、⼿}[HSK 4]
  \definition{v.}{defender; proteger; manter; preservar}
\end{entry}

\begin{entry}{维吾尔}{wei2wu2'er3}{11,7,5}{⽷、⼝、⼩}
  \definition*{s.}{Grupo étnico Uigur de Xinjiang}
\end{entry}

\begin{entry}{维修}{wei2xiu1}{11,9}{⽷、⼈}[HSK 4]
  \definition{v.}{prestar serviços; manter; reparar; manter em (bom) estado de conservação}
\end{entry}

\begin{entry}{伟}{wei3}{6}{⼈}
  \definition{adj.}{grande | ótimo}
\end{entry}

\begin{entry}{伟大}{wei3da4}{6,3}{⼈、⼤}[HSK 3]
  \definition{adj.}{ótimo; importante (contribuição, etc.) | ótimo; magnífico; digno da maior admiração}
\end{entry}

\begin{entry}{尾巴}{wei3ba5}{7,4}{⼫、⼰}[HSK 4]
  \definition{s.}{cauda; projeções na extremidade do corpo de certos animais | parte semelhante a uma cauda; refere-se, em geral, ao final de algo | apêndice; anexo; adepto servil; pessoa que segue ou concorda com outra pessoa | (figura de linguagem) alguém que faz sombra a outro | fim; remanescente; parte restante (ou inacabada)}
\end{entry}

\begin{entry}{委内瑞拉}{wei3nei4rui4la1}{8,4,13,8}{⼥、⼌、⽟、⼿}
  \definition*{s.}{Venezuela}
\end{entry}

\begin{entry}{委托}{wei3tuo1}{8,6}{⼥、⼿}[HSK 5]
  \definition{v.}{confiar; confiar uma tarefa a outra pessoa ou instituição (para que seja realizada)}
\end{entry}

\begin{entry}{卫生}{wei4 sheng1}{3,5}{⼙、⽣}[HSK 3]
  \definition{adj.}{bom para a saúde; higiênico}
  \definition{s.}{higiene; saneamento}
\end{entry}

\begin{entry}{卫生部}{wei4sheng1bu4}{3,5,10}{⼙、⽣、⾢}
  \definition*{s.}{Ministério da Saúde}
\end{entry}

\begin{entry}{卫生防疫}{wei4sheng1 fang2yi4}{3,5,6,9}{⼙、⽣、⾩、⽧}
  \definition{s.}{prevenção contra a epidemia}
\end{entry}

\begin{entry}{卫生间}{wei4sheng1jian1}{3,5,7}{⼙、⽣、⾨}[HSK 3]
  \definition[间,个]{s.}{banheiro; sanitário; \emph{toilette}}
\end{entry}

\begin{entry}{卫生巾}{wei4sheng1jin1}{3,5,3}{⼙、⽣、⼱}
  \definition{s.}{absorvente higiênico}
\end{entry}

\begin{entry}{卫生局}{wei4sheng1ju2}{3,5,7}{⼙、⽣、⼫}
  \definition*{s.}{Departamento de Saúde | Escritório de Saúde}
\end{entry}

\begin{entry}{卫生棉}{wei4sheng1mian2}{3,5,12}{⼙、⽣、⽊}
  \definition{s.}{absorvente | algodão absorvente esterilizado (usado para curativos ou limpeza de feridas) | absorvente tampão}
\end{entry}

\begin{entry}{卫生球}{wei4sheng1qiu2}{3,5,11}{⼙、⽣、⽟}
  \definition{s.}{naftalina}
\end{entry}

\begin{entry}{卫生署}{wei4sheng1shu3}{3,5,13}{⼙、⽣、⽹}
  \definition*{s.}{Agência de Saúde (ou Escritório, ou Departamento)}
\end{entry}

\begin{entry}{卫生套}{wei4sheng1tao4}{3,5,10}{⼙、⽣、⼤}
  \definition[只]{s.}{preservativo | camisinha}
\end{entry}

\begin{entry}{卫生厅}{wei4sheng1ting1}{3,5,4}{⼙、⽣、⼚}
  \definition*{s.}{Departamento de Saúde (da província)}
\end{entry}

\begin{entry}{卫生纸}{wei4sheng1zhi3}{3,5,7}{⼙、⽣、⽷}
  \definition{s.}{papel higiênico}
\end{entry}

\begin{entry}{卫星}{wei4xing1}{3,9}{⼙、⽇}[HSK 5]
  \definition[个,颗]{s.}{satélite; lua; corpos celestes orbitando planetas | satélite artificial | algo que gira em torno de um centro}
\end{entry}

\begin{entry}{为}{wei4}{4}{⼂}[HSK 2,3]
  \definition{prep.}{objeto da ação | indicando propósito | indicando razões | para; em direção a}
  \definition{v.}{apoiar; defender}
  \seeref{为}{wei2}
\end{entry}

\begin{entry}{为了}{wei4le5}{4,2}{⼂、⼅}[HSK 3]
  \definition{conj.}{para; por causa de; a fim de}
\end{entry}

\begin{entry}{为什么}{wei4shen2me5}{4,4,3}{⼂、⼈、⼃}[HSK 2]
  \definition{adv.}{por que?}
\end{entry}

\begin{entry}{未}{wei4}{5}{⽊}
  \definition{adv.}{não ter | ainda não}
\end{entry}

\begin{entry}{未必}{wei4bi4}{5,5}{⽊、⼼}[HSK 4]
  \definition{adv.}{não tenho certeza; talvez não; não necessariamente}
\end{entry}

\begin{entry}{未来}{wei4lai2}{5,7}{⽊、⽊}[HSK 4]
  \definition{adj.}{próximo (refere-se ao tempo)}
  \definition[个]{s.}{futuro; o amanhã}
\end{entry}

\begin{entry}{位}{wei4}{7}{⼈}[HSK 2]
  \definition{clas.}{para pessoas (com cortesia) | para bits binários}[十六位 (16 bits)]
  \definition{s.}{(física) potencial | localização | lugar | posição | assento}
\end{entry}

\begin{entry}{位居}{wei4ju1}{7,8}{⼈、⼫}
  \definition{v.}{estar localizado em}
\end{entry}

\begin{entry}{位于}{wei4yu2}{7,3}{⼈、⼆}[HSK 4]
  \definition{v.}{estar localizado; estar situado}
\end{entry}

\begin{entry}{位置}{wei4zhi4}{7,13}{⼈、⽹}[HSK 4]
  \definition[通,个]{s.}{assento; lugar; localização | lugar; posição; \emph{status} | posição (por exemplo: cargo no escritório)}
\end{entry}

\begin{entry}{位子}{wei4zi5}{7,3}{⼈、⼦}
  \definition{s.}{lugar | assento}
\end{entry}

\begin{entry}{味}{wei4}{8}{⼝}
  \definition{clas.}{para medicamentos}
  \definition{s.}{cheiro | gosto}
\end{entry}

\begin{entry}{味道}{wei4dao5}{8,12}{⼝、⾡}[HSK 2]
  \definition{s.}{sabor | (dialeto) odor, cheiro | (figurativo) sentimento (de…), dica (de…) | (figurativo) interesse, prazer}
\end{entry}

\begin{entry}{味儿}{wei4r5}{8,2}{⼝、⼉}[HSK 4]
  \definition{s.}{gosto; sabor; propriedade de uma substância que dá à língua uma determinada sensação de sabor | cheiro; odor; propriedade de uma substância que dá ao nariz um determinado sentido de cheiro | interesse; significado; deleite}
\end{entry}

\begin{entry}{胃}{wei4}{9}{⾁}[HSK 5]
  \definition*{s.}{Wei, uma das mansões lunares; uma das vinte e oito constelações}
  \definition{s.}{estômago; parte do aparelho digestivo}
\end{entry}

\begin{entry}{胃口}{wei4kou3}{9,3}{⾁、⼝}
  \definition{s.}{apetite}
\end{entry}

\begin{entry}{喂}{wei4}{12}{⼝}[HSK 2,4]
  \definition{interj.}{Ei!, Olá!, para chamar atenção | Alô? (quando respondendo uma chamada telefônica, pronuncia-se como \dpy{wei2})}
  \definition{v.}{criar; alimentar (animais); dar comida a um animal |
alimentar (pessoas); colocar alimentos, medicamentos, etc. na boca de alguém}
\end{entry}

\begin{entry}{喂哺}{wei4bu3}{12,10}{⼝、⼝}
  \definition{v.}{alimentar (um bebê)}
\end{entry}

\begin{entry}{喂料}{wei4liao4}{12,10}{⼝、⽃}
  \definition{v.}{alimentar (também no sentido figurativo)}
\end{entry}

\begin{entry}{喂母乳}{wei4mu3ru3}{12,5,8}{⼝、⽏、⼄}
  \definition{s.}{amamentação}
\end{entry}

\begin{entry}{喂奶}{wei4nai3}{12,5}{⼝、⼥}
  \definition{v.}{amamentar}
\end{entry}

\begin{entry}{喂食}{wei4shi2}{12,9}{⼝、⾷}
  \definition{v.}{alimentar}
\end{entry}

\begin{entry}{喂养}{wei4yang3}{12,9}{⼝、⼋}
  \definition{v.}{alimentar (uma criança, animal doméstico, etc.) | manter | criar (um animal)}
\end{entry}

\begin{entry}{慰问}{wei4wen4}{15,6}{⼼、⾨}[HSK 5]
  \definition{v.}{visitar; consolar; expressar simpatia por; confortar e cumprimentar com palavras e presentes;  enfatizar o conforto e o cumprimento, frequentemente usado por superiores para subordinados}
\end{entry}

\begin{entry}{温度}{wen1du4}{12,9}{⽔、⼴}[HSK 2]
  \definition[个]{s.}{temperatura}
\end{entry}

\begin{entry}{温度表}{wen1du4biao3}{12,9,8}{⽔、⼴、⾐}
  \definition{s.}{termômetro}
\end{entry}

\begin{entry}{温度计}{wen1du4ji4}{12,9,4}{⽔、⼴、⾔}
  \definition{s.}{termógrafo | termômetro}
\end{entry}

\begin{entry}{温度梯度}{wen1du4ti1du4}{12,9,11,9}{⽔、⼴、⽊、⼴}
  \definition{s.}{gradiente de temperatura}
\end{entry}

\begin{entry}{温和}{wen1he2}{12,8}{⽔、⼝}[HSK 5]
  \definition{adj.}{gentil; suave; moderado}
\end{entry}

\begin{entry}{温暖}{wen1nuan3}{12,13}{⽔、⽇}[HSK 3]
  \definition{adj.}{caloroso; gentil}
  \definition{v.}{aquecer (fazer você se sentir aquecido)}
\end{entry}

\begin{entry}{温柔}{wen1rou2}{12,9}{⽔、⽊}
  \definition{adj.}{gentil e suave | terno | doce (comumente usado para descrever uma menina ou mulher)}
\end{entry}

\begin{entry}{文化}{wen2hua4}{4,4}{⽂、⼔}[HSK 3]
  \definition[个,种]{s.}{cultura; civilização | cultura; alfabetização; escolaridade; educação}
\end{entry}

\begin{entry}{文化层}{wen2hua4ceng2}{4,4,7}{⽂、⼔、⼫}
  \definition{s.}{nível de cultura (em sítio arqueológico)}
\end{entry}

\begin{entry}{文化宫}{wen2hua4gong1}{4,4,9}{⽂、⼔、⼧}
  \definition{s.}{palácio cultural}
\end{entry}

\begin{entry}{文化圈}{wen2hua4quan1}{4,4,11}{⽂、⼔、⼞}
  \definition{s.}{esfera de influência cultural}
\end{entry}

\begin{entry}{文化热}{wen2hua4re4}{4,4,10}{⽂、⼔、⽕}
  \definition{s.}{mania cultural | febre cultural}
\end{entry}

\begin{entry}{文化史}{wen2hua4shi3}{4,4,5}{⽂、⼔、⼝}
  \definition*{s.}{História Cultural}
\end{entry}

\begin{entry}{文化水平}{wen2hua4 shui3ping2}{4,4,4,5}{⽂、⼔、⽔、⼲}
  \definition{s.}{nível educacional}
\end{entry}

\begin{entry}{文化障碍}{wen2hua4zhang4'ai4}{4,4,13,13}{⽂、⼔、⾩、⽯}
  \definition{s.}{barreira cultural}
\end{entry}

\begin{entry}{文件}{wen2jian4}{4,6}{⽂、⼈}[HSK 3]
  \definition[份,分]{s.}{documentos oficiais; papéis; instrumentos | os arquivos no computador | artigos ou trabalhos sobre teorias políticas, atualidades, pesquisas acadêmicas, etc.}
\end{entry}

\begin{entry}{文明}{wen2ming2}{4,8}{⽂、⽇}[HSK 3]
  \definition{adj.}{civilizado}
  \definition[个]{s.}{cultura; civilização}
\end{entry}

\begin{entry}{文学}{wen2xue2}{4,8}{⽂、⼦}[HSK 3]
  \definition[个,种]{s.}{literatura}
\end{entry}

\begin{entry}{文学系}{wen2xue2 xi4}{4,8,7}{⽂、⼦、⽷}
  \definition*{s.}{Faculdade de Letras}
\end{entry}

\begin{entry}{文艺}{wen2yi4}{4,4}{⽂、⾋}[HSK 5]
  \definition{s.}{termo genérico para literatura e arte | performance (arte); refere-se especificamente às artes performativas, como música e dança}
\end{entry}

\begin{entry}{文章}{wen2zhang1}{4,11}{⽂、⾳}[HSK 3]
  \definition[篇,段,页]{s.}{ensaio; dissertação; artigo | significado oculto; significado implícito | trabalho (coisas para fazer)}
\end{entry}

\begin{entry}{文字}{wen2zi4}{4,6}{⽂、⼦}[HSK 3]
  \definition[种,类,段,行,篇]{s.}{personagens; roteiro; escrita
linguagem escrita}
\end{entry}

\begin{entry}{纹路}{wen2lu4}{7,13}{⽷、⾜}
  \definition{s.}{padrão de linhas | rugas | veias | veias (em mármore ou impressão digital) | grãos (em madeira, etc.)}
\end{entry}

\begin{entry}{闻}{wen2}{9}{⾨}[HSK 2]
  \definition*{s.}{sobrenome Wen}
  \definition{s.}{notícias | reputação | fama}
  \definition{v.}{ouvir | cheirar | farejar}
\end{entry}

\begin{entry}{蚊香}{wen2xiang1}{10,9}{⾍、⾹}
  \definition{s.}{incenso ou espiral repelente de mosquitos}
\end{entry}

\begin{entry}{蚊子}{wen2zi5}{10,3}{⾍、⼦}
  \definition{s.}{pernilongo}
\end{entry}

\begin{entry}{稳}{wen3}{14}{⽲}[HSK 4]
  \definition{adj.}{constante; estável; firme | estável; estático; sedado | seguro; confiável; certo}
  \definition{adv.}{certamente; com certeza; seguramente; sem dúvida}
  \definition{v.}{estabilizar, manter estável}
\end{entry}

\begin{entry}{稳定}{wen3ding4}{14,8}{⽲、⼧}[HSK 4]
  \definition{adj.}{estável; firme; descreve uma natureza, um estado, etc. relativamente fixo; não muda significativamente}
  \definition{s.}{estabilidade}
  \definition{v.}{manter estável; estabilizar; liquidar; resolver a situação}
\end{entry}

\begin{entry}{问}{wen4}{6}{⾨}[HSK 1]
  \definition{v.}{perguntar}
\end{entry}

\begin{entry}{问安}{wen4'an1}{6,6}{⾨、⼧}
  \definition{s.}{saudações}
  \definition{v.}{dar cumprimentos a | prestar homenagem}
\end{entry}

\begin{entry}{问鼎}{wen4ding3}{6,12}{⾨、⿍}
  \definition{v.}{visar (o primeiro lugar, etc.) | aspirar ao trono}
\end{entry}

\begin{entry}{问候}{wen4hou4}{6,10}{⾨、⼈}[HSK 4]
  \definition{s.}{homenagem | saudação}
  \definition{v.}{prestar homenagem; enviar uma saudação;  dar os respeitos (cumprimentos) a alguém | (fig.) (coloquial) fazer referência ofensiva a (alguém querido pela pessoa com quem se está falando)}
\end{entry}

\begin{entry}{问卷}{wen4juan4}{6,8}{⾨、⼙}
  \definition[份]{s.}{questionário}
\end{entry}

\begin{entry}{问路}{wen4 lu4}{6,13}{⾨、⾜}[HSK 2]
  \definition{v.}{perguntar sobre o caminho | pedir por direções}
\end{entry}

\begin{entry}{问市}{wen4shi4}{6,5}{⾨、⼱}
  \definition{v.}{chegar ao mercado | bater o mercado | atingir o mercado}
\end{entry}

\begin{entry}{问题}{wen4ti2}{6,15}{⾨、⾴}[HSK 2]
  \definition[个]{s.}{pergunta | questão | problema}
\end{entry}

\begin{entry}{嗡嗡}{weng1weng1}{13,13}{⼝、⼝}
  \definition{s.}{zumbido}
  \definition{v.}{zumbir}
\end{entry}

\begin{entry}{蕹菜}{weng4cai4}{16,11}{⾋、⾋}
  \definition{s.}{espinafre aquático | \emph{ong choy} | repolho do pântano | convolvulus aquático | glória-da-manhã aquática}
  \seealsoref{空心菜}{kong1xin1cai4}
\end{entry}

\begin{entry}{我}{wo3}{7}{⼽}[HSK 1]
  \definition{pron.}{eu | me | mim | comigo}
\end{entry}

\begin{entry}{我的}{wo3 de5}{7,8}{⼽、⽩}
  \definition{pron.}{meu, meus}
\end{entry}

\begin{entry}{我们}{wo3men5}{7,5}{⼽、⼈}[HSK 1]
  \definition{pron.}{nós | nos | conosco}
\end{entry}

\begin{entry}{我们的}{wo3men5 de5}{7,5,8}{⼽、⼈、⽩}
  \definition{pron.}{nosso, nossos}
\end{entry}

\begin{entry}{我去}{wo3qu4}{7,5}{⼽、⼛}
  \definition{interj.}{(gíria) O que\dots!! | Oh meu Deus! | Isso é insano!}
\end{entry}

\begin{entry}{卧}{wo4}{8}{⾂}
  \definition{v.}{agachar | deitar}
\end{entry}

\begin{entry}{卧病}{wo4bing4}{8,10}{⾂、⽧}
  \definition{s.}{acamado | doente na cama}
\end{entry}

\begin{entry}{卧舱}{wo4cang1}{8,10}{⾂、⾈}
  \definition{s.}{cabine de dormir em um barco ou trem}
\end{entry}

\begin{entry}{卧车}{wo4che1}{8,4}{⾂、⾞}
  \definition{s.}{um carro-leito | vagão-leito}
\end{entry}

\begin{entry}{卧床}{wo4chuang2}{8,7}{⾂、⼴}
  \definition{adj.}{acamado}
  \definition{s.}{cama}
  \definition{v.}{deitar na cama}
\end{entry}

\begin{entry}{卧倒}{wo4dao3}{8,10}{⾂、⼈}
  \definition{v.}{cair no chão | deitar-se}
\end{entry}

\begin{entry}{卧式}{wo4shi4}{8,6}{⾂、⼷}
  \definition{adj.}{horizontal}
\end{entry}

\begin{entry}{卧室}{wo4shi4}{8,9}{⾂、⼧}[HSK 5]
  \definition[间,个]{s.}{quarto de dormir; quarto de uma casa usado para dormir}
\end{entry}

\begin{entry}{卧榻}{wo4ta4}{8,14}{⾂、⽊}
  \definition{s.}{um sofá | uma cama estreita}
\end{entry}

\begin{entry}{卧推}{wo4tui1}{8,11}{⾂、⼿}
  \definition{s.}{supino}
\end{entry}

\begin{entry}{握}{wo4}{12}{⼿}[HSK 5]
  \definition{v.}{segurar; agarrar | agarrar; segurar; empunhar; controlar | pegar pela mão}
\end{entry}

\begin{entry}{握手}{wo4shou3}{12,4}{⼿、⼿}[HSK 3]
  \definition{v.+compl.}{apertar as mãos}
\end{entry}

\begin{entry}{斡旋}{wo4xuan2}{14,11}{⽃、⽅}
  \definition{v.}{mediar (um conflito, etc.)}
\end{entry}

\begin{entry}{乌龟}{wu1gui1}{4,7}{⼃、⿔}
  \definition{s.}{tartaruga}
\end{entry}

\begin{entry}{乌克兰}{wu1ke4lan2}{4,7,5}{⼃、⼗、⼋}
  \definition*{s.}{Ucrânia}
\end{entry}

\begin{entry}{污染}{wu1ran3}{6,9}{⽔、⽊}[HSK 5]
  \definition{s.}{poluição}
  \definition{v.}{poluir; contaminar com substâncias nocivas e prejudiciais; refere-se especificamente à destruição do ambiente natural causada por substâncias nocivas, tais como gases, líquidos e resíduos emitidos por indústrias, minas, veículos, etc. | contaminar; metáfora de que pensamentos prejudiciais causam efeitos negativos nas pessoas}
\end{entry}

\begin{entry}{污染区}{wu1ran3qu1}{6,9,4}{⽔、⽊、⼖}
  \definition{s.}{área contaminada}
\end{entry}

\begin{entry}{污染物}{wu1ran3wu4}{6,9,8}{⽔、⽊、⽜}
  \definition{s.}{poluente}
  \seealsoref{污染物质}{wu1ran3 wu4zhi4}
\end{entry}

\begin{entry}{污染物质}{wu1ran3 wu4zhi4}{6,9,8,8}{⽔、⽊、⽜、⾙}
  \definition{s.}{poluente}
  \seealsoref{污染物}{wu1ran3wu4}
\end{entry}

\begin{entry}{污水}{wu1shui3}{6,4}{⽔、⽔}[HSK 5]
  \definition{s.}{água suja (ou poluída, residual); esgoto; lodo | efluente; drenagem; água suja; água poluída; água residual}
\end{entry}

\begin{entry}{屋}{wu1}{9}{⼫}[HSK 5]
  \definition[间,座]{s.}{casa | quarto}
\end{entry}

\begin{entry}{屋子}{wu1zi5}{9,3}{⼫、⼦}[HSK 3]
  \definition[间,座,栋]{s.}{casa}
\end{entry}

\begin{entry}{无}{wu2}{4}{⽆}[HSK 4][Kangxi 71]
  \definition{adv.}{não; não ter algo; não há\dots}
  \definition{conj.}{independentemente de; não importa se, o que, etc.}
  \definition{v.}{não ter; estar sem; não existir;}
\end{entry}

\begin{entry}{无敌}{wu2di2}{4,10}{⽆、⾆}
  \definition{adj.}{invencível | inigualável}
\end{entry}

\begin{entry}{无法}{wu2 fa3}{4,8}{⽆、⽔}[HSK 4]
  \definition{adj.}{incapaz; incapacitado}
  \definition{v.}{não há nada a ser feito}
\end{entry}

\begin{entry}{无骨}{wu2 gu3}{4,9}{⽆、⾻}
  \definition{adj.}{desossado}
\end{entry}

\begin{entry}{无故}{wu2gu4}{4,9}{⽆、⽁}
  \definition{adv.}{sem causa ou razão | sem motivo}
\end{entry}

\begin{entry}{无聊}{wu2liao2}{4,11}{⽆、⽿}[HSK 4]
  \definition{adj.}{entediado; aborrecido; sentir-se desinteressado porque não há nada para fazer | tolo; bobo; sem sentido; descreve palavras ou coisas ditas ou feitas como sem sentido e irritantes; descreve pessoas ou coisas como sem sentido e pouco atraentes}
\end{entry}

\begin{entry}{无论}{wu2lun4}{4,6}{⽆、⾔}[HSK 4]
  \definition{conj.}{não importa o quê; não importa como; independentemente de; indica que as condições são diferentes, mas resultado é o mesmo |}
  \seealsoref{无论……也……}{wu2lun4 ye3}
\end{entry}

\begin{entry}{无论……也……}{wu2lun4 ye3}{4,6,3}{⽆、⾔、⼄}
  \definition{conj.}{não apenas\dots, (o que, quem, como, etc.), \dots}
\end{entry}

\begin{entry}{无奈}{wu2nai4}{4,8}{⽆、⼤}[HSK 5]
  \definition{conj.}{mas (infelizmente); no entanto}
  \definition{v.}{não poder evitar; não ter alternativa; não ter escolha; não haver nada a fazer}
\end{entry}

\begin{entry}{无人}{wu2ren2}{4,2}{⽆、⼈}
  \definition{adj.}{não tripulado | desabitado}
\end{entry}

\begin{entry}{无人机}{wu2ren2ji1}{4,2,6}{⽆、⼈、⽊}
  \definition{s.}{\emph{drone} | veículo aéreo não tripulado}
\end{entry}

\begin{entry}{无视}{wu2shi4}{4,8}{⽆、⾒}
  \definition{v.}{ignorar | desconsiderar}
\end{entry}

\begin{entry}{无数}{wu2shu4}{4,13}{⽆、⽁}[HSK 4]
  \definition{adj.}{incontável; inumerável | inseguro; incerto; não conhecer a história ou os detalhes internos; não ter certeza}
\end{entry}

\begin{entry}{无所谓}{wu2suo3wei4}{4,8,11}{⽆、⼾、⾔}[HSK 4]
  \definition{v.}{não pode ser designado como; não merece o nome de; ser incapaz de dizer ou contar | não ter importância; ser indiferente;}
\end{entry}

\begin{entry}{无限}{wu2 xian4}{4,8}{⽆、⾩}[HSK 4]
  \definition{adj.}{infinito; ilimitado; sem limites; sem fim à vista}
\end{entry}

\begin{entry}{无氧}{wu2yang3}{4,10}{⽆、⽓}
  \definition{adj.}{anaeróbico}
\end{entry}

\begin{entry}{无疑}{wu2 yi2}{4,14}{⽆、⽦}[HSK 5]
  \definition{adv.}{indubitavelmente; sem dúvida; sem sombra de dúvida}
\end{entry}

\begin{entry}{吾}{wu2}{7}{⼝}
  \definition*{s.}{sobrenome Wu}
  \definition{pron.}{eu | (antigo) meu}
\end{entry}

\begin{entry}{五}{wu3}{4}{⼆}[HSK 1]
  \definition{num.}{cinco; 5}
\end{entry}

\begin{entry}{五体投地}{wu3ti3tou2di4}{4,7,7,6}{⼆、⼈、⼿、⼟}
  \definition{expr.}{prostrar-se em admiração | adular alguém}
\end{entry}

\begin{entry}{五五}{wu3wu3}{4,4}{⼆、⼆}
  \definition{num.}{50-50}
  \definition{s.}{igual (partilha, parceria, etc.)}
\end{entry}

\begin{entry}{五颜六色}{wu3 yan2 liu4 se4}{4,15,4,6}{⼆、⾴、⼋、⾊}[HSK 4]
  \definition{adj.}{todas as cores sob o sol; multicolorido; colorido}
\end{entry}

\begin{entry}{午}{wu3}{4}{⼗}
  \definition{s.}{período entre 11h00 e 13h00, meio-dia}
\end{entry}

\begin{entry}{午餐}{wu3 can1}{4,16}{⼗、⾷}[HSK 2]
  \definition[份,顿,次]{s.}{almoço}
  \seealsoref{午饭}{wu3fan4}
\end{entry}

\begin{entry}{午饭}{wu3fan4}{4,7}{⼗、⾷}[HSK 1]
  \definition[份,顿,次,餐]{s.}{almoço}
  \seealsoref{午餐}{wu3 can1}
\end{entry}

\begin{entry}{午后}{wu3hou4}{4,6}{⼗、⼝}
  \definition{s.}{tarde | período da tarde}
\end{entry}

\begin{entry}{午前}{wu3qian2}{4,9}{⼗、⼑}
  \definition{s.}{\emph{A.M.} | manhã | período da manhã}
\end{entry}

\begin{entry}{午睡}{wu3 shui4}{4,13}{⼗、⽬}[HSK 2]
  \definition{s.}{siesta}
  \definition{v.}{tirar uma soneca}
\end{entry}

\begin{entry}{午休}{wu3xiu1}{4,6}{⼗、⼈}
  \definition{s.}{pausa para almoço | cochilo na hora do almoço | intervalo do meio-dia}
\end{entry}

\begin{entry}{午宴}{wu3yan4}{4,10}{⼗、⼧}
  \definition{s.}{banquete de almoço}
\end{entry}

\begin{entry}{午夜}{wu3ye4}{4,8}{⼗、⼣}
  \definition{s.}{meia-noite}
\end{entry}

\begin{entry}{武}{wu3}{8}{⽌}
  \definition*{s.}{sobrenome Wu}
  \definition{s.}{arte marcial}
\end{entry}

\begin{entry}{武大戏}{wu3 da4xi4}{8,3,6}{⽌、⼤、⼽}
  \definition*{s.}{Drama de Luta Acrobática | Drama Wu}
\end{entry}

\begin{entry}{武断}{wu3duan4}{8,11}{⽌、⽄}
  \definition{adj.}{arbitrário | dogmático | subjetivo}
\end{entry}

\begin{entry}{武官}{wu3guan1}{8,8}{⽌、⼧}
  \definition{s.}{oficial militar}
\end{entry}

\begin{entry}{武力}{wu3li4}{8,2}{⽌、⼒}
  \definition{s.}{forças armadas | militares}
\end{entry}

\begin{entry}{武器}{wu3qi4}{8,16}{⽌、⼝}[HSK 3]
  \definition[批,种]{s.}{arma; armamento}
\end{entry}

\begin{entry}{武士}{wu3shi4}{8,3}{⽌、⼠}
  \definition{s.}{samurai | guerreiro}
\end{entry}

\begin{entry}{武术}{wu3shu4}{8,5}{⽌、⽊}[HSK 3]
  \definition[种,套,门]{s.}{arte marcial; autodefesa; \emph{wushu}}
\end{entry}

\begin{entry}{武艺}{wu3yi4}{8,4}{⽌、⾋}
  \definition{s.}{arte marcial | habilidade militar}
\end{entry}

\begin{entry}{武装}{wu3zhuang1}{8,12}{⽌、⾐}
  \definition{s.}{forças armadas | militar | arma}
  \definition{v.}{armar}
\end{entry}

\begin{entry}{舞}{wu3}{14}{⾇}[HSK 5]
  \definition{s.}{dança | palco; metáfora do domínio das atividades sociais}
  \definition{v.}{mover-se como numa dança | dançar com algo nas mãos; brincar com | florescer; empunhar; brandir | esvoaçar | fazer malabarismos; brincar com}
\end{entry}

\begin{entry}{舞抃}{wu3bian4}{14,7}{⾇、⼿}
  \definition{s.}{dançar por prazer}
\end{entry}

\begin{entry}{舞蹈}{wu3dao3}{14,17}{⾇、⾜}
  \definition{s.}{dança (ato performático)}
\end{entry}

\begin{entry}{舞会}{wu3hui4}{14,6}{⾇、⼈}
  \definition{s.}{baile}
\end{entry}

\begin{entry}{舞会舞}{wu3hui4wu3}{14,6,14}{⾇、⼈、⾇}
  \definition{s.}{baile}
\end{entry}

\begin{entry}{舞台}{wu3 tai2}{14,5}{⾇、⼝}[HSK 3]
  \definition[个]{s.}{palco; arena}
\end{entry}

\begin{entry}{舞厅}{wu3ting1}{14,4}{⾇、⼚}
  \definition[间]{s.}{salão de dança | salão de baile}
\end{entry}

\begin{entry}{舞厅舞}{wu3ting1wu3}{14,4,14}{⾇、⼚、⾇}
  \definition{s.}{dança de salão}
\end{entry}

\begin{entry}{务实}{wu4shi2}{5,8}{⼒、⼧}
  \definition{adj.}{pragmático}
  \definition{v.}{lidar com assuntos concretos}
\end{entry}

\begin{entry}{物价}{wu4 jia4}{8,6}{⽜、⼈}[HSK 5]
  \definition[个]{s.}{preços das commodities; preços das matérias-primas; preço das mercadorias}
\end{entry}

\begin{entry}{物理}{wu4li3}{8,11}{⽜、⽟}
  \definition{s.}{física (disciplina)}
\end{entry}

\begin{entry}{物业}{wu4ye4}{8,5}{⽜、⼀}[HSK 5]
  \definition[处]{s.}{propriedade; gestão de propriedades; gestão patrimonial; administração de imóveis | empresa de administração de imóveis; empresa de gestão imobiliária; empresa de administração de bens imóveis}
\end{entry}

\begin{entry}{物质}{wu4zhi4}{8,8}{⽜、⾙}[HSK 5]
  \definition[个]{s.}{matéria; substância; algo que existe além do espírito, que pode ser visto, tocado, cheirado ou detectado por instrumentos científicos | material; meios de subsistência; coisas que permitem às pessoas viver ou viver melhor, como comida, roupas, casas, dinheiro, etc.}
\end{entry}

\begin{entry}{误点}{wu4dian3}{9,9}{⾔、⽕}
  \definition{v.+compl.}{atrasar | chegar tarde}
\end{entry}

\begin{entry}{误会}{wu4hui4}{9,6}{⾔、⼈}
  \definition[场]{s.}{mal-entendido; desentendimentos ou conflitos decorrentes de mal-entendidos}
  \definition{v.}{entender mal; entender errado; interpretar mal; não entender; não entender corretamente o significado}
\end{entry}

\begin{entry}{误解}{wu4jie3}{9,13}{⾔、⾓}[HSK 5]
  \definition[种]{s.}{equívoco; mal-entendido; desentendimento}
  \definition{v.}{interpretar mal; interpretar erroneamente; não compreender corretamente}
\end{entry}

\begin{entry}{雾气}{wu4qi4}{13,4}{⾬、⽓}
  \definition{s.}{nevoeiro | névoa | vapor}
\end{entry}

%%%%% EOF %%%%%


%%%
%%% X
%%%

\section*{X}\addcontentsline{toc}{section}{X}

\begin{EntryWithPhonetic}{夕}{xi1}{3}{⼣}[Kangxi 36]
  \definition*{s.}{Sobrenome Xi}
  \definition{s.}{pôr do sol; crepúsculo | tarde; noite}
\end{EntryWithPhonetic}

\begin{EntryWithPhonetic}{夕阳}{xi1yang2}{3,6}{⼣、⾩}
  \definition{s.}{pôr do sol}
  \seealsoref{日出}{ri4chu1}
\end{EntryWithPhonetic}

\begin{EntryWithPhonetic}{吸}{xi1}{6}{⼝}[HSK 4]
  \definition{v.}{inalar; inspirar; aspirar (oposto a 呼) | sugar (líquidos) | absorver; sugar | atrair; atrair para si mesmo | aspirar; introdução de líquidos, gases, etc. no corpo}
  \seealsoref{呼}{hu1}
\end{EntryWithPhonetic}

\begin{EntryWithPhonetic}{吸毒}{xi1 du2}{6,9}{⼝、⽏}[HSK 6]
  \definition{s.}{droga}
  \definition{v.}{usar drogas viciantes; ser viciado em um narcótico; consumir drogas}
\end{EntryWithPhonetic}

\begin{EntryWithPhonetic}{吸管}{xi1 guan3}{6,14}{⼝、⽵}[HSK 4]
  \definition[根,个,支]{s.}{tubo de sucção; sugador; canudo (para beber); refere-se ao tubo fino usado para sugar bebidas | conta-gotas; pipeta; cateter para bombeamento de líquidos usando pressão de ar}
\end{EntryWithPhonetic}

\begin{EntryWithPhonetic}{吸收}{xi1shou1}{6,6}{⼝、⽁}[HSK 4]
  \definition{v.}{imbuir; absorver; assimilar; sugar;  chupar; (animais, plantas, etc.) extrair material de fora dos tecidos para o interior dos tecidos | absorver; chupar;  sugar alguma substância de fora para dentro | recrutar; alistar; inscrever-se; matricular-se; admitir; (organizações ou coletivos) aceitar novos membros | absorver; aproveitar e usar a experiência, o conhecimento, o dinheiro e outras coisas valiosas de outras pessoas | absorver; diminuir, atenuar ou eliminar determinados efeitos ou fenômenos}
\end{EntryWithPhonetic}

\begin{EntryWithPhonetic}{吸铁石}{xi1tie3shi2}{6,10,5}{⼝、⾦、⽯}
  \definition{s.}{imã | magneto}
  \seealsoref{磁铁}{ci2tie3}
\end{EntryWithPhonetic}

\begin{EntryWithPhonetic}{吸烟}{xi1/yan1}{6,10}{⼝、⽕}[HSK 4]
  \definition{v.+compl.}{fumar}
\end{EntryWithPhonetic}

\begin{EntryWithPhonetic}{吸引}{xi1yin3}{6,4}{⼝、⼸}[HSK 4]
  \definition{v.}{atrair; apelar para; chamar a atenção de outros objetos, forças ou pessoas para si mesmo}
\end{EntryWithPhonetic}

\begin{EntryWithPhonetic}{西}{xi1}{6}{⾑}[HSK 1][Kangxi 146]
  \definition*{s.}{Espanha, abreviatura de 西班牙 | Paraíso Ocidental | Sobrenome Xi}
  \definition{s.}{oeste; uma das quatro direções básicas, o lado onde o sol se põe (oposto ao 东) | ocidental; refere-se ao Ocidente (principalmente aos países europeus e americanos) | aqui e ali; em contraposição a 东, significa 到处 ou 零散, 没有次序}
  \seealsoref{到处}{dao4chu4}
  \seealsoref{东}{dong1}
  \seealsoref{零散}{ling2san3}
  \seealsoref{没有次序}{mei2you3 ci4xu4}
  \seealsoref{西班牙}{xi1ban1ya2}
\end{EntryWithPhonetic}

\begin{EntryWithPhonetic}{西安}{xi1'an1}{6,6}{⾑、⼧}
  \definition*{s.}{Xi'an, Capital da Província de Shaanxi}
\end{EntryWithPhonetic}

\begin{EntryWithPhonetic}{西班牙}{xi1ban1ya2}{6,10,4}{⾑、⽟、⽛}
  \definition*{s.}{Espanha}
\end{EntryWithPhonetic}

\begin{EntryWithPhonetic}{西班牙文}{xi1ban1ya2wen2}{6,10,4,4}{⾑、⽟、⽛、⽂}
  \definition{s.}{espanhol, língua espanhola}
  \seealsoref{西文}{xi1wen2}
\end{EntryWithPhonetic}

\begin{EntryWithPhonetic}{西班牙语}{xi1 ban1 ya2 yu3}{6,10,4,9}{⾑、⽟、⽛、⾔}[HSK 6]
  \definition[句]{s.}{espanhol | língua espanhola}
  \seealsoref{西语}{xi1yu3}
\end{EntryWithPhonetic}

\begin{EntryWithPhonetic}{西半球}{xi1ban4qiu2}{6,5,11}{⾑、⼗、⽟}
  \definition{s.}{hemisfério oeste}
\end{EntryWithPhonetic}

\begin{EntryWithPhonetic}{西北}{xi1 bei3}{6,5}{⾑、⼔}[HSK 2]
  \definition{s.}{noroeste | noroeste da China; o Noroeste}
\end{EntryWithPhonetic}

\begin{EntryWithPhonetic}{西边}{xi1bian1}{6,5}{⾑、⾡}[HSK 1]
  \definition{s.}{lado oeste; (oeste) Uma das quatro direções principais; uma das direções cardeais, oposta ao 东方}
  \seealsoref{东方}{dong1 fang1}
\end{EntryWithPhonetic}

\begin{EntryWithPhonetic}{西部}{xi1 bu4}{6,10}{⾑、⾢}[HSK 3]
  \definition{s.}{(EUA) filme de faroeste; filme de \emph{cowboys}; um faroeste | filme da região ocidental (China) | parte ocidental; região oeste da China}
\end{EntryWithPhonetic}

\begin{EntryWithPhonetic}{西餐}{xi1 can1}{6,16}{⾑、⾷}[HSK 2]
  \definition[份,顿,桌]{s.}{comida ocidental; comida de estilo ocidental, comida com garfo e faca (diferente da 中餐)}
  \seealsoref{中餐}{zhong1 can1}
\end{EntryWithPhonetic}

\begin{EntryWithPhonetic}{西方}{xi1 fang1}{6,4}{⾑、⽅}[HSK 2]
  \definition{s.}{oeste | o Ocidente; o Oeste; países europeus e americanos | Paraíso Ocidental, termo budista}
\end{EntryWithPhonetic}

\begin{EntryWithPhonetic}{西瓜}{xi1gua1}{6,5}{⾑、⽠}[HSK 4]
  \definition[个,颗,粒]{s.}{melancia; fruto que é uma baga de formato grande, globular ou oval, com muita polpa aguada e doce}
\end{EntryWithPhonetic}

\begin{EntryWithPhonetic}{西红柿}{xi1hong2shi4}{6,6,9}{⾑、⽷、⽊}[HSK 5]
  \definition[种,只,株]{s.}{tomate}
\end{EntryWithPhonetic}

\begin{EntryWithPhonetic}{西兰花}{xi1lan2hua1}{6,5,7}{⾑、⼋、⾋}
  \definition{s.}{brócolis}
\end{EntryWithPhonetic}

\begin{EntryWithPhonetic}{西蓝花}{xi1lan2hua1}{6,13,7}{⾑、⾋、⾋}
  \variantof{西兰花}
\end{EntryWithPhonetic}

\begin{EntryWithPhonetic}{西面}{xi1mian4}{6,9}{⾑、⾯}
  \definition{s.}{oeste | lado oeste}
\end{EntryWithPhonetic}

\begin{EntryWithPhonetic}{西南}{xi1 nan2}{6,9}{⾑、⼗}[HSK 2]
  \definition{s.}{sudoeste | o Sudoeste; Sudoeste da China}
\end{EntryWithPhonetic}

\begin{EntryWithPhonetic}{西文}{xi1wen2}{6,4}{⾑、⽂}
  \definition{s.}{espanhol | língua espanhola}
  \seealsoref{西班牙文}{xi1ban1ya2wen2}
\end{EntryWithPhonetic}

\begin{EntryWithPhonetic}{西西}{xi1xi1}{6,6}{⾑、⾑}
  \definition{num.}{centímetro cúbico}
\end{EntryWithPhonetic}

\begin{EntryWithPhonetic}{西药}{xi1 yao4}{6,9}{⾑、⾋}
  \definition[片,粒]{s.}{medicina ocidental; refere-se aos medicamentos usados ​​na medicina ocidental, geralmente feitos por métodos sintéticos ou extraídos de produtos naturais, como comprimidos anti-inflamatórios, aspirina, tintura de iodo, penicilina, etc.}
\end{EntryWithPhonetic}

\begin{EntryWithPhonetic}{西医}{xi1 yi1}{6,7}{⾑、⼖}[HSK 2]
  \definition[名,位]{s.}{medicina ocidental; medicina introduzida na China a partir da Europa e da América | um médico treinado em medicina ocidental}
\end{EntryWithPhonetic}

\begin{EntryWithPhonetic}{西语}{xi1yu3}{6,9}{⾑、⾔}
  \definition{s.}{línguas ocidentais | espanhol | língua espanhola}
  \seealsoref{西班牙语}{xi1 ban1 ya2 yu3}
\end{EntryWithPhonetic}

\begin{EntryWithPhonetic}{西藏}{xi1zang4}{6,17}{⾑、⾋}
  \definition*{s.}{Xizang; Região Autônoma do Tibete, 西藏自治区}
  \seealsoref{西藏自治区}{xi1zang4 zi4zhi4qu1}
\end{EntryWithPhonetic}

\begin{EntryWithPhonetic}{西藏自治区}{xi1zang4 zi4zhi4qu1}{6,17,6,8,4}{⾑、⾋、⾃、⽔、⼖}
  \definition*{s.}{Região Autônoma do Tibete}
\end{EntryWithPhonetic}

\begin{EntryWithPhonetic}{西装}{xi1 zhuang1}{6,12}{⾑、⾐}[HSK 5]
  \definition[件,套,个]{s.}{terno; roupas de estilo ocidental; roupas ocidentais, divididas em masculinas e femininas}
\end{EntryWithPhonetic}

\begin{EntryWithPhonetic}{希}{xi1}{7}{⼱}
  \definition*{s.}{Sobrenome Xi}
  \definition{v.}{ter esperança}
\end{EntryWithPhonetic}

\begin{EntryWithPhonetic}{希望}{xi1wang4}{7,11}{⼱、⽉}[HSK 3]
  \definition[个,丝,点]{s.}{esperança; desejo; expectativa; a possibilidade de alcançar um determinado objetivo ou de ocorrer uma determinada situação ideal no futuro | aquilo em que a esperança é depositada; o objeto da esperança}
  \definition{v.}{ter esperança; desejar; esperar; pensar em alcançar algum objetivo ou que alguma situação ocorra}
\end{EntryWithPhonetic}

\begin{EntryWithPhonetic}{昔}{xi1}{8}{⽇}
  \definition{s.}{tempos antigos; o passado; era uma vez}
\end{EntryWithPhonetic}

\begin{EntryWithPhonetic}{昔日}{xi1ri4}{8,4}{⽇、⽇}
  \definition{adj.}{passado}
\end{EntryWithPhonetic}

\begin{EntryWithPhonetic}{牺}{xi1}{10}{⽜}
  \definition{s.}{um animal de cor uniforme para sacrifício; sacrifício; gado com pelagem pura usado para sacrifício}
\end{EntryWithPhonetic}

\begin{EntryWithPhonetic}{牺牲}{xi1sheng1}{10,9}{⽜、⽜}[HSK 6]
  \definition[份]{s.}{sacrifício; um animal abatido para sacrifício; refere-se ao sacrifício da própria vida ou dos próprios interesses por um propósito justo, ou refere-se ao preço pago por um determinado propósito}
  \definition{v.}{sacrificar-se; morrer como mártir; dar a própria vida; sacrificar sua vida pela justiça | sacrificar; desistir; fazer algo às custas de; geralmente se refere a pagar um preço ou sofrer danos por alguém ou algo}
\end{EntryWithPhonetic}

\begin{EntryWithPhonetic}{悉}{xi1}{11}{⼼}
  \definition*{s.}{Sobrenome Xi}
  \definition{adj.}{tudo; inteiro; total | detalhado}
  \definition{v.}{saber; aprender; ser informado de}
\end{EntryWithPhonetic}

\begin{EntryWithPhonetic}{悉尼}{xi1ni2}{11,5}{⼼、⼫}
  \definition*{s.}{Sidney}
\end{EntryWithPhonetic}

\begin{EntryWithPhonetic}{悉数}{xi1shu3}{11,13}{⼼、⽁}
  \definition{adv.}{enumerar em detalhes | explicar claramente}
  \seeref{xi1shu4}
\end{EntryWithPhonetic}

\begin{EntryWithPhonetic}{悉数}{xi1shu4}{11,13}{⼼、⽁}
  \definition{adv.}{todos | cada um | toda a soma}
  \seeref{xi1shu3}
\end{EntryWithPhonetic}

\begin{EntryWithPhonetic}{悉心}{xi1xin1}{11,4}{⼼、⼼}
  \definition{adv.}{colocar o coração (e a alma) em algo | com muito cuidado}
\end{EntryWithPhonetic}

\begin{EntryWithPhonetic}{蜥}{xi1}{14}{⾍}
  \definition{s.}{lagarto}
\end{EntryWithPhonetic}

\begin{EntryWithPhonetic}{蜥易}{xi1yi4}{14,8}{⾍、⽇}
  \variantof{蜥蜴}
\end{EntryWithPhonetic}

\begin{EntryWithPhonetic}{蜥蜴}{xi1yi4}{14,14}{⾍、⾍}
  \definition{s.}{lagarto}
\end{EntryWithPhonetic}

\begin{EntryWithPhonetic}{习}{xi2}{3}{⼄}
  \definition*{s.}{Sobrenome Xi}
  \definition{s.}{hábito; costume; prática usual; um comportamento que se desenvolve inconscientemente por meio de ações repetidas ao longo de um longo período de tempo}
  \definition{v.}{revisar; praticar; exercitar | acostumado a; familiarizado com; familiarizado com algo por meio de contato frequente | estudar; aprender (pássaro)}
\end{EntryWithPhonetic}

\begin{EntryWithPhonetic}{习惯}{xi2guan4}{3,11}{⼄、⼼}[HSK 2]
  \definition[个,种]{s.}{hábito; costume; prática usual; comportamentos, tendências ou tendências sociais que se desenvolvem gradualmente ao longo de um longo período de tempo e são difíceis de mudar}
  \definition{v.}{estar acostumado a; ter o hábito de}
\end{EntryWithPhonetic}

\begin{EntryWithPhonetic}{席}{xi2}{10}{⼱}
  \definition*{s.}{Sobrenome Xi}
  \definition[卷,张]{s.}{esteira | assento; lugar; caixa | assento (em uma assembleia legislativa) | festim; banquete; jantar}
\end{EntryWithPhonetic}

\begin{EntryWithPhonetic}{席卷}{xi2juan3}{10,8}{⼱、⼙}
  \definition{v.}{engolfar | varrer | levar tudo para fora}
\end{EntryWithPhonetic}

\begin{EntryWithPhonetic}{袭}{xi2}{11}{⾐}
  \definition*{s.}{Sobrenome Xi}
  \definition{clas.}{usado para conjuntos completos de roupas}
  \definition{v.}{fazer um ataque surpresa a; invadir | seguir o padrão de; continuar como antes; fazer o mesmo}
\end{EntryWithPhonetic}

\begin{EntryWithPhonetic}{袭击}{xi2ji1}{11,5}{⾐、⼐}
  \definition{s.}{ataque (especialmente um ataque surpresa) | invasão}
  \definition{v.}{atacar}
\end{EntryWithPhonetic}

\begin{EntryWithPhonetic}{洗}{xi3}{9}{⽔}[HSK 1]
  \definition[个]{s.}{pequeno recipiente contendo água para enxaguar os pincéis de escrever | batismo}
  \definition{v.}{lavar; tomar banho; remover a sujeira do objeto com água ou outro solvente | batizar | eliminar; corrigir; reparar | saquear; matar e pilhar; matar ou roubar tudo, como se tivesse sido lavado | revelar filmes; imprimir fotos | apagar; limpar (uma gravação, etc.) | embaralhar (cartas, etc.)}
\end{EntryWithPhonetic}

\begin{EntryWithPhonetic}{洗涤}{xi3di2}{9,10}{⽔、⽔}
  \definition{s.}{enxágue | lava}
  \definition{v.}{enxaguar | lavar}
\end{EntryWithPhonetic}

\begin{EntryWithPhonetic}{洗涤间}{xi3di2jian1}{9,10,7}{⽔、⽔、⾨}
  \definition{s.}{lavanderia}
\end{EntryWithPhonetic}

\begin{EntryWithPhonetic}{洗劫}{xi3jie2}{9,7}{⽔、⼒}
  \definition{v.}{saquear | pilhar | roubar}
\end{EntryWithPhonetic}

\begin{EntryWithPhonetic}{洗净}{xi3jing4}{9,8}{⽔、⼎}
  \definition{v.}{lavar (limpeza)}
\end{EntryWithPhonetic}

\begin{EntryWithPhonetic}{洗礼}{xi3li3}{9,5}{⽔、⽰}
  \definition{s.}{batismo}
  \definition{v.}{batizar}
\end{EntryWithPhonetic}

\begin{EntryWithPhonetic}{洗手}{xi3shou3}{9,4}{⽔、⼿}
  \definition{v.}{ir ao banheiro | lavar as mãos}
\end{EntryWithPhonetic}

\begin{EntryWithPhonetic}{洗手不干}{xi3shou3bu2gan4}{9,4,4,3}{⽔、⼿、⼀、⼲}
  \definition{v.}{parar totalmente de fazer algo}
\end{EntryWithPhonetic}

\begin{EntryWithPhonetic}{洗手池}{xi3shou3chi2}{9,4,6}{⽔、⼿、⽔}
  \definition{s.}{pia de banheiro | lavatório}
  \seealsoref{洗手盆}{xi3shou3pen2}
\end{EntryWithPhonetic}

\begin{EntryWithPhonetic}{洗手间}{xi3shou3jian1}{9,4,7}{⽔、⼿、⾨}[HSK 1]
  \definition[个]{s.}{banheiro; lavatório; lavabo}
\end{EntryWithPhonetic}

\begin{EntryWithPhonetic}{洗手盆}{xi3shou3pen2}{9,4,9}{⽔、⼿、⽫}
  \definition{s.}{pia de banheiro | lavatório}
  \seealsoref{洗手池}{xi3shou3chi2}
\end{EntryWithPhonetic}

\begin{EntryWithPhonetic}{洗手乳}{xi3shou3ru3}{9,4,8}{⽔、⼿、⼄}
  \definition{s.}{sabonete líquido para lavar as mãos}
  \seealsoref{洗手液}{xi3shou3ye4}
\end{EntryWithPhonetic}

\begin{EntryWithPhonetic}{洗手液}{xi3shou3ye4}{9,4,11}{⽔、⼿、⽔}
  \definition{s.}{sabonete líquido para lavar as mãos}
  \seealsoref{洗手乳}{xi3shou3ru3}
\end{EntryWithPhonetic}

\begin{EntryWithPhonetic}{洗脱}{xi3tuo1}{9,11}{⽔、⾁}
  \definition{v.}{limpar | purgar | lavar}
\end{EntryWithPhonetic}

\begin{EntryWithPhonetic}{洗碗}{xi3wan3}{9,13}{⽔、⽯}
  \definition{v.}{lavar pratos}
\end{EntryWithPhonetic}

\begin{EntryWithPhonetic}{洗胃}{xi3wei4}{9,9}{⽔、⾁}
  \definition{s.}{(medicina) lavagem gástrica}
  \definition{v.}{ter o estômago lavado}
\end{EntryWithPhonetic}

\begin{EntryWithPhonetic}{洗衣粉}{xi3 yi1 fen3}{9,6,10}{⽔、⾐、⽶}[HSK 6]
  \definition[袋,包,勺]{s.}{sabão em pó; detergente para roupa (em pó); detergente em pó sintetizado quimicamente, específico para uso em lavanderia}
\end{EntryWithPhonetic}

\begin{EntryWithPhonetic}{洗衣机}{xi3 yi1 ji1}{9,6,6}{⽔、⾐、⽊}[HSK 2]
  \definition[台]{s.}{máquina de lavar roupa; eletrodomésticos para lavagem automática ou semiautomática de roupas}
\end{EntryWithPhonetic}

\begin{EntryWithPhonetic}{洗澡}{xi3/zao3}{9,16}{⽔、⽔}[HSK 2]
  \definition{v.+compl.}{tomar banho; tomar banho de chuveiro; lavar-se}
\end{EntryWithPhonetic}

\begin{EntryWithPhonetic}{洗澡间}{xi3zao3jian1}{9,16,7}{⽔、⽔、⾨}
  \definition[间]{s.}{banheiro}
\end{EntryWithPhonetic}

\begin{EntryWithPhonetic}{喜}{xi3}{12}{⼝}
  \definition{adj.}{feliz; satisfeito; encantado}
  \definition[桩,件]{s.}{evento feliz (especialmente casamento); ocasião para celebração; algo para comemorar | gravidez | casamento ou coisas relacionadas a ele}
  \definition{v.}{gostar; fonte de; ter inclinação para | precisa; requer; combina melhor com; (um certo organismo) precisa ou é adequado para (um certo ambiente ou algo)}
\end{EntryWithPhonetic}

\begin{EntryWithPhonetic}{喜爱}{xi3 ai4}{12,10}{⼝、⽖}[HSK 4]
  \definition{v.}{gostar; amar; ter afeição por; estar interessado em; ter uma queda ou sentir interesse por pessoas ou coisas}
\end{EntryWithPhonetic}

\begin{EntryWithPhonetic}{喜欢}{xi3huan5}{12,6}{⼝、⽋}[HSK 1]
  \definition{adj.}{feliz; encantado; exultante; cheio de alegria}
  \definition{v.}{gostar; amar; ter afeição por; estar interessado em; ter uma boa impressão ou interesse por alguém ou algo}
\end{EntryWithPhonetic}

\begin{EntryWithPhonetic}{喜剧}{xi3 ju4}{12,10}{⼝、⼑}[HSK 5]
  \definition[部,出]{s.}{comédia (oposto de 悲剧) | comédia; uma das principais categorias do teatro; usa o exagero para satirizar e ridicularizar o feio; fenômenos retrógrados; destaca as contradições inerentes a esses fenômenos e seu conflito com coisas saudáveis; costuma provocar risadas; o final geralmente é feliz}
  \seealsoref{悲剧}{bei1 ju4}
\end{EntryWithPhonetic}

\begin{EntryWithPhonetic}{戏}{xi4}{6}{⼽}[HSK 5]
  \definition*{s.}{Sobrenome Xi}
  \definition[场,部,出,台]{s.}{drama; peça; espetáculo; \emph{show}}
  \definition{v.}{brincar; praticar esportes; jogar | zombar; brincar; provocar}
\end{EntryWithPhonetic}

\begin{EntryWithPhonetic}{戏法}{xi4fa3}{6,8}{⼽、⽔}
  \definition{s.}{truque de mágica | prestidigitação}
\end{EntryWithPhonetic}

\begin{EntryWithPhonetic}{戏剧}{xi4ju4}{6,10}{⼽、⼑}[HSK 5]
  \definition[出,部]{s.}{drama; peça; teatro | roteiro; peça; cenário}
\end{EntryWithPhonetic}

\begin{EntryWithPhonetic}{戏剧般}{xi4ju4ban1}{6,10,10}{⼽、⼑、⾈}
  \definition{adj.}{melodramático}
\end{EntryWithPhonetic}

\begin{EntryWithPhonetic}{戏剧编剧}{xi4ju4bian1ju4}{6,10,12,10}{⼽、⼑、⽷、⼑}
  \definition{s.}{dramaturgo}
\end{EntryWithPhonetic}

\begin{EntryWithPhonetic}{戏剧化地}{xi4ju4hua4di4}{6,10,4,6}{⼽、⼑、⼔、⼟}
  \definition{adv.}{dramaticamente | teatralmente}
\end{EntryWithPhonetic}

\begin{EntryWithPhonetic}{戏剧家}{xi4ju4jia1}{6,10,10}{⼽、⼑、⼧}
  \definition{s.}{dramaturgo}
\end{EntryWithPhonetic}

\begin{EntryWithPhonetic}{戏剧效果}{xi4ju4xiao4guo3}{6,10,10,8}{⼽、⼑、⽁、⽊}
  \definition{s.}{efeito dramático}
\end{EntryWithPhonetic}

\begin{EntryWithPhonetic}{戏剧性}{xi4ju4xing4}{6,10,8}{⼽、⼑、⼼}
  \definition{adj.}{dramático}
\end{EntryWithPhonetic}

\begin{EntryWithPhonetic}{戏剧演出}{xi4ju4yan3chu1}{6,10,14,5}{⼽、⼑、⽔、⼐}
  \definition{s.}{performance dramática}
\end{EntryWithPhonetic}

\begin{EntryWithPhonetic}{戏弄}{xi4nong4}{6,7}{⼽、⼶}
  \definition{v.}{zombar de | pregar peças | provocar}
\end{EntryWithPhonetic}

\begin{EntryWithPhonetic}{戏曲}{xi4 qu3}{6,6}{⼽、⽈}[HSK 6]
  \definition{s.}{drama; ópera chinesa; ópera tradicional; forma teatral tradicional | partes cantadas em 传奇 e zaju 杂剧}
  \seealsoref{传奇}{chuan2qi2}
  \seealsoref{杂剧}{za2ju4}
\end{EntryWithPhonetic}

\begin{EntryWithPhonetic}{戏耍}{xi4shua3}{6,9}{⼽、⽽}
  \definition{v.}{divertir-me | brincar com | provocar}
\end{EntryWithPhonetic}

\begin{EntryWithPhonetic}{戏谑}{xi4xue4}{6,11}{⼽、⾔}
  \definition{v.}{brincar | fazer piadas | ridicularizar}
\end{EntryWithPhonetic}

\begin{EntryWithPhonetic}{戏院}{xi4yuan4}{6,9}{⼽、⾩}
  \definition{s.}{teatro}
\end{EntryWithPhonetic}

\begin{EntryWithPhonetic}{系}{xi4}{7}{⽷}[HSK 3,4]
  \definition*{s.}{Sobrenome Xi}
  \definition{s.}{sistema; série | departamento; faculdade; unidades administrativas de ensino divididas por disciplina nas instituições de ensino superior}
  \definition{v.}{relacionar-se com; suportar; depender de | sentir-se ansioso; estar preocupado | amarrar; prender | ser; expressa julgamento, equivalente a 是}
  \seeref{ji4}
  \seealsoref{是}{shi4}
\end{EntryWithPhonetic}

\begin{EntryWithPhonetic}{系列}{xi4lie4}{7,6}{⽷、⼑}[HSK 4]
  \definition{s.}{série; conjunto; conjunto de coisas relacionadas (matemática)}
\end{EntryWithPhonetic}

\begin{EntryWithPhonetic}{系囚}{xi4qiu2}{7,5}{⽷、⼞}
  \definition{s.}{prisioneiro}
\end{EntryWithPhonetic}

\begin{EntryWithPhonetic}{系统}{xi4tong3}{7,9}{⽷、⽷}[HSK 4]
  \definition{adj.}{sistemático; organizado}
  \definition[套,个]{s.}{sistema; relação de tipos semelhantes (ou seja, grupo de coisas semelhantes)}
\end{EntryWithPhonetic}

\begin{EntryWithPhonetic}{细}{xi4}{8}{⽷}[HSK 4]
  \definition{adj.}{fino; delgado; esguio; esbelto; em oposição a 粗 | fino; em partículas pequenas; grãos pequenos | fino e macio;  um sussuro | fino; requintado; delicado | cuidadoso; detalhado; meticuloso | ínfimo; minúsculo; insignificante; diminuto | jovem; pequeno}
  \seealsoref{粗}{cu1}
\end{EntryWithPhonetic}

\begin{EntryWithPhonetic}{细胞}{xi4bao1}{8,9}{⽷、⾁}[HSK 6]
  \definition[个]{s.}{célula; unidade estrutural e funcional básica de um organismo, com uma variedade de formas, composta principalmente pelo núcleo, citoplasma e membrana celular; as plantas também possuem paredes celulares fora da membrana celular}
\end{EntryWithPhonetic}

\begin{EntryWithPhonetic}{细节}{xi4jie2}{8,5}{⽷、⾋}[HSK 4]
  \definition[处]{s.}{detalhe; particularidade; aspectos secundários ou partes sutis de um enredo ou episódios secundários usados em uma obra literária para expressar o caráter de uma pessoa ou as características essenciais de uma coisa}
\end{EntryWithPhonetic}

\begin{EntryWithPhonetic}{细菌}{xi4jun1}{8,11}{⽷、⾋}[HSK 6]
  \definition[个]{s.}{germe; bactéria; um organismo muito pequeno, invisível aos olhos humanos}
\end{EntryWithPhonetic}

\begin{EntryWithPhonetic}{细菌战}{xi4jun1zhan4}{8,11,9}{⽷、⾋、⼽}
  \definition{s.}{guerra biológica}
\end{EntryWithPhonetic}

\begin{EntryWithPhonetic}{细致}{xi4zhi4}{8,10}{⽷、⾄}[HSK 4]
  \definition{adj.}{meticuloso; cuidadoso; minucioso | intrincado; delicado}
\end{EntryWithPhonetic}

\begin{EntryWithPhonetic}{虾}{xia1}{9}{⾍}
  \definition{s.}{camarão}
\end{EntryWithPhonetic}

\begin{EntryWithPhonetic}{狭}{xia2}{9}{⽝}
  \definition{adj.}{estreito (oposto a 广)}
  \seealsoref{广}{guang3}
\end{EntryWithPhonetic}

\begin{EntryWithPhonetic}{下}{xia4}{3}{⼀}[HSK 1,2]
  \definition{clas.}{número de vezes usado para a ação | volume de um contêiner; quantidade de objetos que cabem em um utensílio | usado depois de 两 e 几 para expressar habilidade, capacidade, destreza}
  \definition{s.}{abaixo | próximo; último; segundo; referindo-se ao que está por vir ou ao que vem em seguida | mais baixo; inferior; de baixo nível ou grau | próximo; último; segundo; em ordem ou em ordem cronológica | indica pertencer a uma determinada faixa, situação, condição, etc. | indica uma determinada época ou estação | usado após um número para indicar posição ou direção | para baixo (após uma preposição) | sob (depois de um substantivo) | para baixo (antes de um verbo)}
  \definition{v.}{desembarcar; descer; sair | cair (chuva, neve, etc.) | enviar; emitir; entregar | ir para | sair; partir; retirar-se | lançar; colocar | descarregar; desmontar; tirar (fora) | formar (uma opinião, ideia, etc.); tomar decisões, fazer julgamentos, etc. | usar; aplicar | dar à luz (animais) | tomar; capturar; conquistar | ceder | terminar; deixar de lado; terminar o trabalho ou os estudos diários na hora prevista | para negação; ser inferior a; ser menor que}
  \seealsoref{几}{ji3}
  \seealsoref{两}{liang3}
\end{EntryWithPhonetic}

\begin{EntryWithPhonetic}{下巴}{xia4ba5}{3,4}{⼀、⼰}
  \definition[个]{s.}{queixo}
\end{EntryWithPhonetic}

\begin{EntryWithPhonetic}{下班}{xia4/ban1}{3,10}{⼀、⽟}[HSK 1]
  \definition{v.+compl.}{sair do trabalho; bater ponto; terminar o trabalho na hora prevista e sair do local de trabalho}
\end{EntryWithPhonetic}

\begin{EntryWithPhonetic}{下边}{xia4 bian5}{3,5}{⼀、⾡}[HSK 1]
  \definition{s.}{abaixo; sob; por baixo | próximo em ordem; seguinte | nível inferior; subordinado | a parte inferior}
\end{EntryWithPhonetic}

\begin{EntryWithPhonetic}{下车}{xia4 che1}{3,4}{⼀、⾞}[HSK 1]
  \definition{v.}{descer ou sair de (um ônibus, trem, carro etc.)}
\end{EntryWithPhonetic}

\begin{EntryWithPhonetic}{下次}{xia4 ci4}{3,6}{⼀、⽋}[HSK 1]
  \definition{s.}{na próxima vez; na próxima oportunidade ou no próximo evento}
\end{EntryWithPhonetic}

\begin{EntryWithPhonetic}{下蛋}{xia4dan4}{3,11}{⼀、⾍}
  \definition{v.}{botar ovos}
\end{EntryWithPhonetic}

\begin{EntryWithPhonetic}{下个月}{xia4 ge4 yue4}{3,3,4}{⼀、⼈、⽉}[HSK 4]
  \definition{s.}{próximo mês; mês que vem; refere-se ao próximo mês do mês atual}
\end{EntryWithPhonetic}

\begin{EntryWithPhonetic}{下海}{xia4/hai3}{3,10}{⼀、⽔}
  \definition{v.+compl.}{ir para o mar; (barco) deixar o porto e iniciar uma jornada | ir pescar no mar | tornar-se ator profissional}
\end{EntryWithPhonetic}

\begin{EntryWithPhonetic}{下降}{xia4 jiang4}{3,8}{⼀、⾩}[HSK 4]
  \definition{v.}{cair; despencar; declinar; descer; diminuir; ir para baixo}
\end{EntryWithPhonetic}

\begin{EntryWithPhonetic}{下课}{xia4/ke4}{3,10}{⼀、⾔}[HSK 1]
  \definition{v.+compl.}{terminar a aula; sair da aula}
\end{EntryWithPhonetic}

\begin{EntryWithPhonetic}{下来}{xia4 lai5}{3,7}{⼀、⽊}[HSK 3]
  \definition{part.}{usado após o verbo, indica que a ação ou o comportamento se dirige para a posição do falante ou que a ação é contínua ou concluída | usado após um adjetivo, indica que uma determinada situação começou a ocorrer e continuará a se desenvolver}
  \definition{v.}{descer (para a minha localização) | (colheitas/frutas/vegetais, etc.) ser colhido; estar maduro o suficiente para ser colhido | (período de tempo) acabar; passar; chegar ao fim; indicar o fim de um período de tempo}
\end{EntryWithPhonetic}

\begin{EntryWithPhonetic}{下楼}{xia4 lou2}{3,13}{⼀、⽊}[HSK 4]
  \definition{v.}{descer as escadas}
\end{EntryWithPhonetic}

\begin{EntryWithPhonetic}{下面}{xia4 mian4}{3,9}{⼀、⾯}[HSK 3]
  \definition{s.}{em baixo; abaixo; parte de baixo | próximo; seguinte; a parte posterior; a parte posterior de um artigo ou discurso em relação ao que está sendo narrado no momento | subordinado; o nível inferior; homens nos níveis inferiores | por baixo}
\end{EntryWithPhonetic}

\begin{EntryWithPhonetic}{下去}{xia4 qu4}{3,5}{⼀、⼛}[HSK 3]
  \definition{part.}{usado depois de verbos para indicar de alto a baixo | usado depois de um verbo para indicar continuação}
  \definition{v.}{descer; baixar (a partir da minha localização) | (após um verbo) continuar (fazendo algo); prosseguir | usado após o verbo, indica uma descida de um ponto alto para um ponto baixo | usado após o verbo, indica continuidade | usado após um adjetivo, indica que o grau continua aumentando}
\end{EntryWithPhonetic}

\begin{EntryWithPhonetic}{下水道}{xia4shui3dao4}{3,4,12}{⼀、⽔、⾡}
  \definition{s.}{esgoto}
\end{EntryWithPhonetic}

\begin{EntryWithPhonetic}{下午}{xia4wu3}{3,4}{⼀、⼗}[HSK 1]
  \definition[个]{s.}{tarde; \emph{post meridiem} (p.m.); refere-se ao período entre o meio-dia e o pôr do sol}
\end{EntryWithPhonetic}

\begin{EntryWithPhonetic}{下午茶}{xia4wu3cha2}{3,4,9}{⼀、⼗、⾋}
  \definition{s.}{chá da tarde (normalmente chás com doces)}
\end{EntryWithPhonetic}

\begin{EntryWithPhonetic}{下线}{xia4xian4}{3,8}{⼀、⽷}
  \definition{v.}{ficar \emph{offline} | (um produto) sair da linha de montagem | pessoa abaixo de si em um esquema de pirâmide}
\end{EntryWithPhonetic}

\begin{EntryWithPhonetic}{下雪}{xia4/xue3}{3,11}{⼀、⾬}[HSK 2]
  \definition{v.+compl.}{nevar}
\end{EntryWithPhonetic}

\begin{EntryWithPhonetic}{下旬}{xia4xun2}{3,6}{⼀、⽇}
  \definition{adv.}{última dezena do mês}
\end{EntryWithPhonetic}

\begin{EntryWithPhonetic}{下雨}{xia4/yu3}{3,8}{⼀、⾬}[HSK 1]
  \definition{v.+compl.}{chover}
\end{EntryWithPhonetic}

\begin{EntryWithPhonetic}{下载}{xia4zai3}{3,10}{⼀、⾞}[HSK 4]
  \definition{v.}{\emph{download}; baixar; salvar informações da \emph{Web} em um dispositivo, como um computador}
\end{EntryWithPhonetic}

\begin{EntryWithPhonetic}{下崽}{xia4zai3}{3,12}{⼀、⼭}
  \definition{v.}{dar à luz (animais) | parir}
\end{EntryWithPhonetic}

\begin{EntryWithPhonetic}{下周}{xia4 zhou1}{3,8}{⼀、⼝}[HSK 2]
  \definition{s.}{próxima semana}
\end{EntryWithPhonetic}

\begin{EntryWithPhonetic}{吓}{xia4}{6}{⼝}[HSK 5]
  \definition{interj.}{interjeição que demonstra espanto; Interjeição que expressa insatisfação}
  \definition{v.}{ameaçar; intimidar; usar ameaças ou meios coercitivos para intimidar ou assustar}
\end{EntryWithPhonetic}

\begin{EntryWithPhonetic}{吓人}{xia4/ren2}{6,2}{⼝、⼈}
  \definition{adj.}{apavorante | assustador}
  \definition{v.+compl.}{assustar-se | tomar um susto}
\end{EntryWithPhonetic}

\begin{EntryWithPhonetic}{夏}{xia4}{10}{⼢}
  \definition*{s.}{Dinastia Xia (2070-1600 a.C.) | China; refere-se à China | Sobrenome Xia}
  \definition{s.}{verão}
\end{EntryWithPhonetic}

\begin{EntryWithPhonetic}{夏季}{xia4 ji4}{10,8}{⼢、⼦}[HSK 4]
  \definition[个]{s.}{verão; segundo trimestre do ano, habitualmente chamado na China de período de três meses, do início do verão ao início do outono, também chamado de ``quarto, quinto e sexto'' meses do calendário lunar}
\end{EntryWithPhonetic}

\begin{EntryWithPhonetic}{夏日}{xia4ri4}{10,4}{⼢、⽇}
  \definition{s.}{horário de verão}
\end{EntryWithPhonetic}

\begin{EntryWithPhonetic}{夏天}{xia4 tian1}{10,4}{⼢、⼤}[HSK 2]
  \definition[个]{s.}{verão}
\end{EntryWithPhonetic}

\begin{EntryWithPhonetic}{仙}{xian1}{5}{⼈}
  \definition{s.}{imortal}
\end{EntryWithPhonetic}

\begin{EntryWithPhonetic}{先}{xian1}{6}{⼉}[HSK 1]
  \definition*{s.}{Sobrenome Xian}
  \definition{adv.}{primeiro; antes; mais cedo; com antecedência | no momento; por enquanto; em um curto espaço de tempo; temporariamente}
  \definition{s.}{início; começo; em ordem cronológica ou de precedência | ancestral; geração mais velha; antepassado | tardio; falecido; morto (honrar os mortos)}
\end{EntryWithPhonetic}

\begin{EntryWithPhonetic}{先不先}{xian1bu4xian1}{6,4,6}{⼉、⼀、⼉}
  \definition{adv.}{(dialeto) antes de tudo | em primeiro lugar}
\end{EntryWithPhonetic}

\begin{EntryWithPhonetic}{先到先得}{xian1dao4xian1de2}{6,8,6,11}{⼉、⼑、⼉、⼻}
  \definition{expr.}{primeiro a chegar | primeiro a ser servido}
\end{EntryWithPhonetic}

\begin{EntryWithPhonetic}{先锋}{xian1 feng1}{6,12}{⼉、⾦}[HSK 6]
  \definition{s.}{pioneiro; vanguarda; a vanguarda de uma batalha ou marcha; geralmente se refere a uma pessoa ou grupo que desempenha um papel de vanguarda}
\end{EntryWithPhonetic}

\begin{EntryWithPhonetic}{先后}{xian1 hou4}{6,6}{⼉、⼝}[HSK 5]
  \definition{adv.}{sucessivamente; um após o outro}
  \definition{s.}{prioridade; ordem; cedo ou tarde; primeiro e último}
\end{EntryWithPhonetic}

\begin{EntryWithPhonetic}{先进}{xian1jin4}{6,7}{⼉、⾡}[HSK 3]
  \definition{adj.}{avançado; progressos rápidos e nível elevado, podendo servir de exemplo a seguir}
  \definition{s.}{indivíduos ou grupos avançados}
\end{EntryWithPhonetic}

\begin{EntryWithPhonetic}{先烈}{xian1lie4}{6,10}{⼉、⽕}
  \definition{s.}{mártir}
\end{EntryWithPhonetic}

\begin{EntryWithPhonetic}{先期}{xian1qi1}{6,12}{⼉、⽉}
  \definition{adv.}{antecipadamente}
  \definition{s.}{prematuro | \emph{front-end}}
\end{EntryWithPhonetic}

\begin{EntryWithPhonetic}{先前}{xian1qian2}{6,9}{⼉、⼑}[HSK 5]
  \definition[出]{s.}{antes; anteriormente; refere-se ao passado ou a um certo tempo anterior}
\end{EntryWithPhonetic}

\begin{EntryWithPhonetic}{先生}{xian1sheng5}{6,5}{⼉、⽣}[HSK 1]
  \definition[个,位]{s.}{professor; títulos honoríficos para professores, médicos, etc. | marido; antigamente, referia-se ao marido de outra pessoa ou ao próprio marido (ambos com pronomes pessoais como determinantes) | médico; títulos usados para se referir aos médicos no passado | refere-se a pessoas cuja profissão envolve contar histórias, adivinhação, etc.; antigamente, era chamado de contador | senhor; \emph{sir}; título dado aos intelectuais}
\end{EntryWithPhonetic}

\begin{EntryWithPhonetic}{先天}{xian1tian1}{6,4}{⼉、⼤}
  \definition{adj.}{congênito | inato | natural}
  \definition{s.}{período embrionário}
\end{EntryWithPhonetic}

\begin{EntryWithPhonetic}{先验}{xian1yan4}{6,10}{⼉、⾺}
  \definition{adj.}{(filosofia) a priori}
\end{EntryWithPhonetic}

\begin{EntryWithPhonetic}{先有}{xian1you3}{6,6}{⼉、⽉}
  \definition{adj.}{preexistente | anterior}
\end{EntryWithPhonetic}

\begin{EntryWithPhonetic}{鲜}{xian1}{14}{⿂}[HSK 4]
  \definition*{s.}{Sobrenome Xian}
  \definition{adj.}{fresco; novo; fresco (experiência, comida etc.) |brilhante; de cores vivas | saboroso; delicioso | exuberante; luxuriante}
  \definition{s.}{aves e animais recém-abatidos; vegetais recém-colhidos; frutas, etc. | alimentos aquáticos; geralmente, peixes vivos, camarões, etc., para alimentação}
  \seeref{xian3}
\end{EntryWithPhonetic}

\begin{EntryWithPhonetic}{鲜花}{xian1 hua1}{14,7}{⿂、⾋}[HSK 4]
  \definition[朵,束,支]{s.}{flor; flores frescas; flores bonitas e frescas}
\end{EntryWithPhonetic}

\begin{EntryWithPhonetic}{鲜明}{xian1ming2}{14,8}{⿂、⽇}[HSK 4]
  \definition{adj.}{brilhante (cor) | distinto; bem definido; nítido; claro; característico}
\end{EntryWithPhonetic}

\begin{EntryWithPhonetic}{鲜艳}{xian1yan4}{14,10}{⿂、⾊}[HSK 5]
  \definition{adj.}{de cores alegres; de cores brilhantes}
\end{EntryWithPhonetic}

\begin{EntryWithPhonetic}{闲}{xian2}{7}{⾨}[HSK 5]
  \definition{adj.}{ocioso; não ocupado; desocupado; sem coisas para fazer; sem atividades; tempo livre | desocupado; (casa, objeto, etc.) não em uso; ocioso | não oficial; não sério; não relacionado ao negócio}
  \definition{s.}{lazer; tempo livre}
\end{EntryWithPhonetic}

\begin{EntryWithPhonetic}{咸}{xian2}{9}{⼝}[HSK 4]
  \definition*{s.}{Sobrenome Xian}
  \definition{adj.}{salgado; em conserva; sabor salgado}
  \definition{adv.}{todos; indica a totalidade de um intervalo, equivalente a 全 e 都}
  \seealsoref{都}{dou1}
  \seealsoref{全}{quan2}
\end{EntryWithPhonetic}

\begin{EntryWithPhonetic}{咸菜}{xian2cai4}{9,11}{⼝、⾋}
  \definition{s.}{legumes salgados | \emph{pickles}}
\end{EntryWithPhonetic}

\begin{EntryWithPhonetic}{咸淡}{xian2dan4}{9,11}{⼝、⽔}
  \definition{s.}{água salobra | grau de salinidade | salgado e sem sal (sabores)}
\end{EntryWithPhonetic}

\begin{EntryWithPhonetic}{咸肉}{xian2rou4}{9,6}{⼝、⾁}
  \definition{s.}{\emph{bacon} | carne curada com sal}
\end{EntryWithPhonetic}

\begin{EntryWithPhonetic}{咸涩}{xian2se4}{9,10}{⼝、⽔}
  \definition{s.}{ácido | salgado e amargo}
\end{EntryWithPhonetic}

\begin{EntryWithPhonetic}{咸水}{xian2shui3}{9,4}{⼝、⽔}
  \definition{s.}{salmora | água salgada}
\end{EntryWithPhonetic}

\begin{EntryWithPhonetic}{咸盐}{xian2yan2}{9,10}{⼝、⽫}
  \definition{s.}{(coloquial) sal | sal de mesa}
\end{EntryWithPhonetic}

\begin{EntryWithPhonetic}{咸鱼}{xian2yu2}{9,8}{⼝、⿂}
  \definition{s.}{peixe salgado}
\end{EntryWithPhonetic}

\begin{EntryWithPhonetic}{嫌}{xian2}{13}{⼥}[HSK 6]
  \definition{s.}{suspeita; suspeição; cisma | inimizade; rancor; má vontade; ressentimento}
  \definition{v.}{importar-se com; não gostar e evitar; reclamar de}
\end{EntryWithPhonetic}

\begin{EntryWithPhonetic}{显}{xian3}{9}{⽇}[HSK 5]
  \definition*{s.}{Sobrenome Xian}
  \definition{adj.}{aparente; óbvio; perceptível | ilustre e influente | evidente; óbvio}
  \definition{v.}{mostrar; exibir; manifestar | aparecer; mostrar; revelar}
\end{EntryWithPhonetic}

\begin{EntryWithPhonetic}{显出}{xian3 chu1}{9,5}{⽇、⼐}[HSK 6]
  \definition{v.}{mostrar; revelar | dar provas; expressar; exibir}
\end{EntryWithPhonetic}

\begin{EntryWithPhonetic}{显得}{xian3de5}{9,11}{⽇、⼻}[HSK 3]
  \definition{v.}{parecer; aparecer; manifestar (alguma situação)}
\end{EntryWithPhonetic}

\begin{EntryWithPhonetic}{显然}{xian3ran2}{9,12}{⽇、⽕}[HSK 3]
  \definition{adj.}{claro; evidente; óbvio; fatos, verdades e outras coisas que são fáceis de descobrir, perceber ou sentir claramente}
\end{EntryWithPhonetic}

\begin{EntryWithPhonetic}{显示}{xian3shi4}{9,5}{⽇、⽰}[HSK 3]
  \definition{v.}{mostrar; manifestar-se claramente| exibir; ostentar}
\end{EntryWithPhonetic}

\begin{EntryWithPhonetic}{显著}{xian3zhu4}{9,11}{⽇、⽬}[HSK 4]
  \definition{adj.}{notável; significativo; notável; extraordinário; muito óbvio; muito claramente demonstrado; muito facilmente visto ou sentido}
\end{EntryWithPhonetic}

\begin{EntryWithPhonetic}{险}{xian3}{9}{⾩}[HSK 6]
  \definition{adj.}{perigoso; arriscado | sinistro; cruel; venenoso}
  \definition{adv.}{por um fio de cabelo; por centímetros; quase}
  \definition{s.}{lugar de difícil acesso; lugar perigoso e difícil de atravessar; passagem estreita; desfiladeiro | abreviação de seguro, 保险 | perigo; risco}
  \seealsoref{保险}{bao3xian3}
\end{EntryWithPhonetic}

\begin{EntryWithPhonetic}{猃}{xian3}{10}{⽝}
  \definition{s.}{(arcaico) um tipo de cão com focinho longo}
\end{EntryWithPhonetic}

\begin{EntryWithPhonetic}{猃狁}{xian3yun3}{10,7}{⽝、⽝}
  \definition*{s.}{Termo da dinastia Zhou para uma tribo nômade do norte mais tarde chamou o Xiongnu (匈奴) nas dinastias Qin e Han}
  \seealsoref{匈奴}{xiong1nu2}
\end{EntryWithPhonetic}

\begin{EntryWithPhonetic}{鲜}{xian3}{14}{⿂}
  \definition{adj.}{raro; pouco; pequeno}
  \definition{adv.}{raramente}
  \seeref{xian1}
\end{EntryWithPhonetic}

\begin{EntryWithPhonetic}{见}{xian4}{4}{⾒}[Kangxi 147]
  \definition{v.}{aparecer; também escrito como 现}
  \seeref{jian4}
  \seealsoref{现}{xian4}
\end{EntryWithPhonetic}

\begin{EntryWithPhonetic}{县}{xian4}{7}{⼛}[HSK 4]
  \definition[个]{s.}{condado; unidade de divisão administrativa}
\end{EntryWithPhonetic}

\begin{EntryWithPhonetic}{现}{xian4}{8}{⾒}
  \definition{adj.}{(dinheiro, etc.) em mãos}
  \definition{adv.}{recente; de improviso; naquela época; temporariamente}
  \definition{s.}{presente; atual; existente | dinheiro; dinheiro de pronto}
  \definition{v.}{mostrar; revelar; aparecer; tornar-se visível}
  \seealsoref{见}{xian4}
\end{EntryWithPhonetic}

\begin{EntryWithPhonetic}{现场}{xian4chang3}{8,6}{⾒、⼟}[HSK 3]
  \definition[个,处]{s.}{local onde ocorreu o acidente, incidente ou desastre| local; ponto; local onde se realizam diretamente atividades como produção, apresentações e competições}
\end{EntryWithPhonetic}

\begin{EntryWithPhonetic}{现代}{xian4dai4}{8,5}{⾒、⼈}[HSK 3]
  \definition*{s.}{Hyundai, empresa sul-coreana}
  \definition{adj.}{moderno; contemporâneo; com características, estilo e conceitos modernos, refletindo a vanguarda, a moda e a inovação da atualidade}
  \definition{s.}{tempos modernos; era contemporânea; atualmente, na divisão cronológica da história da China, refere-se principalmente ao período desde o Movimento 4 de Maio até os dias atuais}
\end{EntryWithPhonetic}

\begin{EntryWithPhonetic}{现货}{xian4huo4}{8,8}{⾒、⾙}
  \definition{s.}{produtos à vista}
\end{EntryWithPhonetic}

\begin{EntryWithPhonetic}{现货的}{xian4huo4 de5}{8,8,8}{⾒、⾙、⽩}
  \definition{s.}{produtos em estoque}
\end{EntryWithPhonetic}

\begin{EntryWithPhonetic}{现金}{xian4jin1}{8,8}{⾒、⾦}[HSK 3]
  \definition[笔]{s.}{dinheiro; dinheiro vivo; moeda que pode ser usada diretamente | reserva de dinheiro em um banco; o dinheiro guardado no cofre do banco}
\end{EntryWithPhonetic}

\begin{EntryWithPhonetic}{现实}{xian4shi2}{8,8}{⾒、⼧}[HSK 3]
  \definition{adj.}{real; efetivo; verdadeiro; de acordo com circunstâncias objetivas}
  \definition[个]{s.}{realidade; factualidade; coisas que existem objetivamente}
\end{EntryWithPhonetic}

\begin{EntryWithPhonetic}{现象}{xian4xiang4}{8,11}{⾒、⾗}[HSK 3]
  \definition[个,种]{s.}{aparência (das coisas); fenômeno; a forma externa e as relações manifestadas pelas coisas em seu desenvolvimento e mudança}
\end{EntryWithPhonetic}

\begin{EntryWithPhonetic}{现有}{xian4 you3}{8,6}{⾒、⽉}[HSK 5]
  \definition{adj.}{agora disponível; existente}
  \definition{v.}{estar disponível agora; existir | Literário: ter em mãos; ter em posse}
\end{EntryWithPhonetic}

\begin{EntryWithPhonetic}{现在}{xian4zai4}{8,6}{⾒、⼟}[HSK 1]
  \definition{adv.}{agora; no momento; atualmente; neste momento, quando se fala, às vezes inclui um período de tempo mais ou menos longo antes ou depois da fala (diferente de 过去 ou 将来)}
  \seealsoref{过去}{guo4 qu4}
  \seealsoref{将来}{jiang1lai2}
\end{EntryWithPhonetic}

\begin{EntryWithPhonetic}{现抓}{xian4zhua1}{8,7}{⾒、⼿}
  \definition{v.}{improvisar}
\end{EntryWithPhonetic}

\begin{EntryWithPhonetic}{现状}{xian4zhuang4}{8,7}{⾒、⽝}[HSK 5]
  \definition{s.}{situação atual}
\end{EntryWithPhonetic}

\begin{EntryWithPhonetic}{现做}{xian4zuo4}{8,11}{⾒、⼈}
  \definition{adj.}{fresco}
  \definition{v.}{fazer (comida) no local}
\end{EntryWithPhonetic}

\begin{EntryWithPhonetic}{线}{xian4}{8}{⽷}[HSK 3]
  \definition{clas.}{usado para coisas abstratas, o número é limitado a ``一''}
  \definition[根,个]{s.}{fio; corda; arame; objetos finos e longos feitos de seda, algodão, metal, etc. | linha; figura formada pelo movimento arbitrário de um ponto| feito de fio de algodão | algo em forma de linha, fio, etc. | rota de transporte; linha | linha de demarcação; limite; zona de fronteira; zona de transição | beira; borda | linha ideológica e política | pista; fio}
\end{EntryWithPhonetic}

\begin{EntryWithPhonetic}{线路}{xian4 lu4}{8,13}{⽷、⾜}[HSK 6]
  \definition[条]{s.}{linha; rota; as rotas que os veículos de transporte percorrem, etc., que as pessoas podem usar para chegar aos seus destinos | Eletricidade: linha; circuito; a rota da corrente elétrica}
\end{EntryWithPhonetic}

\begin{EntryWithPhonetic}{线索}{xian4suo3}{8,10}{⽷、⽷}[HSK 5]
  \definition[条,个]{s.}{pista; fio; metáfora para o desenvolvimento das coisas ou a maneira de explorar um problema | fio; linha; refere-se ao contexto de desenvolvimento do enredo em obras literárias}
\end{EntryWithPhonetic}

\begin{EntryWithPhonetic}{线香}{xian4xiang1}{8,9}{⽷、⾹}
  \definition{s.}{bastão ou vareta de incenso}
\end{EntryWithPhonetic}

\begin{EntryWithPhonetic}{限}{xian4}{8}{⾩}
  \definition{s.}{limite | limiar}
  \definition{v.}{definir um limite; limitar; restringir}
\end{EntryWithPhonetic}

\begin{EntryWithPhonetic}{限制}{xian4zhi4}{8,8}{⾩、⼑}[HSK 4]
  \definition[些]{s.}{limite; restrição; confinamento}
  \definition{v.}{limitar; adstringir; restringir; confinar; fechar em (sobre)}
\end{EntryWithPhonetic}

\begin{EntryWithPhonetic}{宪}{xian4}{9}{⼧}
  \definition*{s.}{Sobrenome Xian}
  \definition{s.}{estatuto; decreto | constituição}
\end{EntryWithPhonetic}

\begin{EntryWithPhonetic}{宪法法院}{xian4fa3fa3yuan4}{9,8,8,9}{⼧、⽔、⽔、⾩}
  \definition{s.}{tribunal constitucional}
\end{EntryWithPhonetic}

\begin{EntryWithPhonetic}{宪政}{xian4zheng4}{9,9}{⼧、⽁}
  \definition{s.}{governo constitucional}
\end{EntryWithPhonetic}

\begin{EntryWithPhonetic}{宪制}{xian4zhi4}{9,8}{⼧、⼑}
  \definition{adj.}{constitucional}
  \definition{s.}{sistema de governo constitucional}
\end{EntryWithPhonetic}

\begin{EntryWithPhonetic}{陷}{xian4}{10}{⾩}
  \definition[个]{s.}{armadilha; cilada | defeito | deficiência; desvantagem}
  \definition{v.}{ficar preso (ou atolado); enredar | afundar; desabar | acusar falsamente; incriminar; armar | (de uma cidade, etc.) ser capturado; cair | ser enquadrado; ser capturado}
\end{EntryWithPhonetic}

\begin{EntryWithPhonetic}{陷入}{xian4ru4}{10,2}{⾩、⼊}[HSK 6]
  \definition{v.}{afundar em; cair em; cair em uma situação desfavorável | estar perdido em; estar profundamente em; estar imerso em; metaforicamente, estar profundamente imerso em (uma situação ou pensamento) | estar atolado (lama fofa, areia, etc.)}
\end{EntryWithPhonetic}

\begin{EntryWithPhonetic}{羡}{xian4}{12}{⽺}
  \definition{v.}{admirar; invejar}
\end{EntryWithPhonetic}

\begin{EntryWithPhonetic}{羡慕}{xian4mu4}{12,14}{⽺、⼼}
  \definition{v.}{invejar | admirar}
\end{EntryWithPhonetic}

\begin{EntryWithPhonetic}{献}{xian4}{13}{⽝}[HSK 5]
  \definition{v.}{oferecer; apresentar; dedicar; doar | mostrar; apresentar; exibir | exibir-se; mostrar-se para que os outros vejam}
\end{EntryWithPhonetic}

\begin{EntryWithPhonetic}{乡}{xiang1}{3}{⼄}[HSK 5]
  \definition[个,座,片]{s.}{país; campo; vilarejo; área rural | local de origem; vila ou cidade natal | município (uma unidade administrativa rural subordinada ao condado) | vila natal; cidade natal | terra ou local famoso por produzir algo}
\end{EntryWithPhonetic}

\begin{EntryWithPhonetic}{乡巴佬}{xiang1ba1lao3}{3,4,8}{⼄、⼰、⼈}
  \definition{s.}{aldeão | caipira}
\end{EntryWithPhonetic}

\begin{EntryWithPhonetic}{乡村}{xiang1 cun1}{3,7}{⼄、⽊}[HSK 5]
  \definition{adj.}{rural | rústico}
  \definition[个]{s.}{vila; campo; área rural; principalmente envolvido na agricultura; áreas com distribuição populacional mais dispersa em relação às cidades}
\end{EntryWithPhonetic}

\begin{EntryWithPhonetic}{相}{xiang1}{9}{⽬}
  \definition*{s.}{Sobrenome Xiang}
  \definition{adv.}{uns aos outros; mutuamente | (para uma ação realizada por uma pessoa em relação a outra) | indica a ação de uma parte em relação à outra parte}
  \definition{s.}{qualidade; substância}
  \definition{v.}{ver por si mesmo (se algo ou algo é do seu agrado)}
  \seeref{xiang4}
\end{EntryWithPhonetic}

\begin{EntryWithPhonetic}{相比}{xiang1 bi3}{9,4}{⽬、⽐}[HSK 3]
  \definition{v.}{combinar; comparar com | comparar mutuamente, usar uma coisa como padrão, perceber as características de outra coisa ou obter uma opinião}
\end{EntryWithPhonetic}

\begin{EntryWithPhonetic}{相处}{xiang1chu3}{9,5}{⽬、⼡}[HSK 4]
  \definition{v.}{dar-se bem; viver juntos; dar-se bem (uns com os outros); viver uns com os outros; entrar em contato uns com os outros, tratar uns aos outros}
\end{EntryWithPhonetic}

\begin{EntryWithPhonetic}{相当}{xiang1dang1}{9,6}{⽬、⼹}[HSK 3]
  \definition{adj.}{adequado; apropriado}
  \definition{adv.}{bastante; razoavelmente; consideravelmente; indica um grau relativamente alto e profundo}
  \definition{v.}{combinar; equilibrar; corresponder a; ser aproximadamente igual a; ser proporcional a}
\end{EntryWithPhonetic}

\begin{EntryWithPhonetic}{相等}{xiang1deng3}{9,12}{⽬、⽵}[HSK 5]
  \definition{v.}{ser igual a; possuir a mesma quantidade, peso, tamanho e grau}
\end{EntryWithPhonetic}

\begin{EntryWithPhonetic}{相反}{xiang1fan3}{9,4}{⽬、⼜}[HSK 4]
  \definition{adj.}{oposto; contrário; dois aspectos das coisas são contraditórios e mutuamente exclusivos}
  \definition{conj.}{pelo contrário; usado no início ou no meio de uma frase para indicar uma contradição de significado com o que foi dito anteriormente.}
\end{EntryWithPhonetic}

\begin{EntryWithPhonetic}{相关}{xiang1guan1}{9,6}{⽬、⼋}[HSK 3]
  \definition{v.}{estar mutuamente relacionado; estar intimamente relacionado; estar inter-relacionado}
\end{EntryWithPhonetic}

\begin{EntryWithPhonetic}{相互}{xiang1 hu4}{9,4}{⽬、⼆}[HSK 3]
  \definition{adj.}{mútuo; recíproco; entre duas pessoas ou coisas}
  \definition{adv.}{mutuamente; um ao outro; tratamento recíproco}
\end{EntryWithPhonetic}

\begin{EntryWithPhonetic}{相聚}{xiang1ju4}{9,14}{⽬、⽿}
  \definition{v.}{reunir-se | montar}
\end{EntryWithPhonetic}

\begin{EntryWithPhonetic}{相亲}{xiang1qin1}{9,9}{⽬、⼇}
  \definition{s.}{encontro às cegas | entrevista arranjada para avaliar a proposta de um parceiro de casamento | apegar-se profundamente um ao outro}
\end{EntryWithPhonetic}

\begin{EntryWithPhonetic}{相思病}{xiang1si1bing4}{9,9,10}{⽬、⼼、⽧}
  \definition{s.}{saudade de amor}
\end{EntryWithPhonetic}

\begin{EntryWithPhonetic}{相似}{xiang1si4}{9,6}{⽬、⼈}[HSK 3]
  \definition{v.}{assemelhar-se; ser semelhante; ser parecido}
\end{EntryWithPhonetic}

\begin{EntryWithPhonetic}{相同}{xiang1tong2}{9,6}{⽬、⼝}[HSK 2]
  \definition{adj.}{semelhante; similar; igual; idêntico; o mesmo; consistentes entre si, sem diferença}
\end{EntryWithPhonetic}

\begin{EntryWithPhonetic}{相信}{xiang1xin4}{9,9}{⽬、⼈}[HSK 2]
  \definition{v.}{acreditar em; estar convencido de; ter fé em; acreditar que algo é certo ou verdadeiro sem dúvida}
\end{EntryWithPhonetic}

\begin{EntryWithPhonetic}{相宜}{xiang1yi2}{9,8}{⽬、⼧}
  \definition{adj.}{adequado | apropriado}
  \definition{v.}{ser adequado ou apropriado}
\end{EntryWithPhonetic}

\begin{EntryWithPhonetic}{相应}{xiang1ying4}{9,7}{⽬、⼴}[HSK 5]
  \definition{adj.}{Dialeto: barato}
  \definition{v.}{corresponder}
\end{EntryWithPhonetic}

\begin{EntryWithPhonetic}{相遇}{xiang1yu4}{9,12}{⽬、⾡}
  \definition{v.}{encontrar (reunião, encontro, etc.)}
\end{EntryWithPhonetic}

\begin{EntryWithPhonetic}{香}{xiang1}{9}{⾹}[HSK 3][Kangxi 186]
  \definition*{s.}{Sobrenome Xiang}
  \definition{adj.}{aromático; perfumado; fragrante; cheiroso; oposto a 臭 | saboroso; saboroso; delicioso; apetitoso | com gosto; com bom apetite | (sono) profundo; dormir confortavelmente e tranquilamente | popular; valorizado; apreciado}
  \definition[根,炷]{s.}{especiaria; perfume; fragrância; aromatizante; substância com aroma intenso | incenso; bastão de incenso; tiras finas feitas de serragem e especiarias, queimadas em rituais para honrar os antepassados ou deuses e budas, e também usadas para afastar odores desagradáveis ou mosquitos| antigamente, referia-se a coisas relacionadas com mulheres ou mulheres}
  \seealsoref{臭}{chou4}
\end{EntryWithPhonetic}

\begin{EntryWithPhonetic}{香槟酒}{xiang1bin1jiu3}{9,14,10}{⾹、⽊、⾣}
  \definition[杯]{s.}{(empréstimo linguístico) \emph{champagne}}
\end{EntryWithPhonetic}

\begin{EntryWithPhonetic}{香波}{xiang1bo1}{9,8}{⾹、⽔}
  \definition{s.}{xampu}
\end{EntryWithPhonetic}

\begin{EntryWithPhonetic}{香肠}{xiang1chang2}{9,7}{⾹、⾁}[HSK 5]
  \definition[根]{s.}{salsicha; linguiça; alimento feito com intestino de porco, recheado com carne picada e temperos}
\end{EntryWithPhonetic}

\begin{EntryWithPhonetic}{香港}{xiang1gang3}{9,12}{⾹、⽔}
  \definition*{s.}{Hong Kong}
  \seealsoref{香港岛}{xiang1gang3 dao3}
\end{EntryWithPhonetic}

\begin{EntryWithPhonetic}{香港岛}{xiang1gang3 dao3}{9,12,7}{⾹、⽔、⼭}
  \definition*{s.}{Ilha de Hong Kong}
  \seealsoref{香港}{xiang1gang3}
\end{EntryWithPhonetic}

\begin{EntryWithPhonetic}{香蕉}{xiang1jiao1}{9,15}{⾹、⾋}[HSK 3]
  \definition[枝,根,个,把,串,束,弓]{s.}{banana}
\end{EntryWithPhonetic}

\begin{EntryWithPhonetic}{香炉}{xiang1lu2}{9,8}{⾹、⽕}
  \definition{s.}{incensário (para queimar incenso) | queimador de incenso | insensório, turíbulo}
\end{EntryWithPhonetic}

\begin{EntryWithPhonetic}{香气}{xiang1qi4}{9,4}{⾹、⽓}
  \definition{s.}{fragrância | aroma | incenso}
\end{EntryWithPhonetic}

\begin{EntryWithPhonetic}{香味}{xiang1wei4}{9,8}{⾹、⼝}
  \definition[股]{s.}{fragrância | cheiro doce}
\end{EntryWithPhonetic}

\begin{EntryWithPhonetic}{香蕈}{xiang1xun4}{9,15}{⾹、⾋}
  \definition{s.}{\emph{shiitake}, cogumelo comestível}
\end{EntryWithPhonetic}

\begin{EntryWithPhonetic}{香烟}{xiang1yan1}{9,10}{⾹、⽕}
  \definition[支,条]{s.}{cigarro | fumaça de incenso queimado}
\end{EntryWithPhonetic}

\begin{EntryWithPhonetic}{香艳}{xiang1yan4}{9,10}{⾹、⾊}
  \definition{adj.}{atraente | erótico | romântico}
\end{EntryWithPhonetic}

\begin{EntryWithPhonetic}{香皂}{xiang1zao4}{9,7}{⾹、⽩}
  \definition{s.}{sabonete | sabonete perfumado}
\end{EntryWithPhonetic}

\begin{EntryWithPhonetic}{箱}{xiang1}{15}{⾋}[HSK 4]
  \definition{s.}{caixa; estojo; baú | qualquer coisa no formato de caixa}
\end{EntryWithPhonetic}

\begin{EntryWithPhonetic}{箱子}{xiang1 zi5}{15,3}{⾋、⼦}[HSK 4]
  \definition[个,只]{s.}{baú; caixa; estojo; maleta; pasta executiva}
\end{EntryWithPhonetic}

\begin{EntryWithPhonetic}{详}{xiang2}{8}{⾔}
  \definition{adj.}{conhecido; reconhecido; saber claramente | detalhado; minucioso; pormenorizado (oposto a 略)}
  \definition{s.}{detalhes; particularidades}
  \definition{v.}{contar; explicar; elaborar | saber claramente}
  \seealsoref{略}{lve4}
\end{EntryWithPhonetic}

\begin{EntryWithPhonetic}{详细}{xiang2xi4}{8,8}{⾔、⽷}[HSK 5]
  \definition{adj.}{explícito; detalhado; minucioso; circunstancial; meticuloso}
\end{EntryWithPhonetic}

\begin{EntryWithPhonetic}{享}{xiang3}{8}{⼇}
  \definition{v.}{aproveitar}
\end{EntryWithPhonetic}

\begin{EntryWithPhonetic}{享受}{xiang3shou4}{8,8}{⼇、⼜}[HSK 5]
  \definition{v.}{aproveitar; desfrutar; estar satisfeito material ou espiritualmente}
\end{EntryWithPhonetic}

\begin{EntryWithPhonetic}{响}{xiang3}{9}{⼝}[HSK 2]
  \definition{adj.}{barulhento; ressonante}
  \definition[声,阵]{s.}{som; ruído; barulho | eco}
  \definition{v.}{tocar; soar; ressoar; fazer um som | soar; fazer algo emitir um som}
\end{EntryWithPhonetic}

\begin{EntryWithPhonetic}{响声}{xiang3 sheng1}{9,7}{⼝、⼠}[HSK 6]
  \definition{s.}{som; ruído}
\end{EntryWithPhonetic}

\begin{EntryWithPhonetic}{想}{xiang3}{13}{⼼}[HSK 1]
  \definition{v.}{pensar; ponderar; refletir | supor; contar; considerar; pensar; estimar | querer; gostaria de; sentir vontade (de fazer algo) | lembrar com saudade; sentir falta}
\end{EntryWithPhonetic}

\begin{EntryWithPhonetic}{想不到}{xiang3 bu2 dao4}{13,4,8}{⼼、⼀、⼑}[HSK 6]
  \definition{adj.}{inesperado; imprevisto}
\end{EntryWithPhonetic}

\begin{EntryWithPhonetic}{想到}{xiang3 dao4}{13,8}{⼼、⼑}[HSK 2]
  \definition{v.}{pensar em; trazer à mente; ter no coração; ter uma ideia (na mente); ter uma ideia (no coração)}
\end{EntryWithPhonetic}

\begin{EntryWithPhonetic}{想法}{xiang3 fa3}{13,8}{⼼、⽔}[HSK 2]
  \definition[种]{s.}{ideia; opinião; pensamento; noção; o que alguém tem em mente; visões e opiniões sobre alguém ou algo obtidas através do pensamento}
  \definition{s.}{maneira de pensar | opinião | noção}
  \definition{v.}{tentar; pensar em uma maneira (de fazer algo); fazer o que puder; encontrar um jeito}
\end{EntryWithPhonetic}

\begin{EntryWithPhonetic}{想念}{xiang3nian4}{13,8}{⼼、⼼}[HSK 4]
  \definition{v.}{sentir falta; pensar em; lembrar com carinho; ficar doente por; desejar ver novamente; lembrar com saudade}
\end{EntryWithPhonetic}

\begin{EntryWithPhonetic}{想起}{xiang3 qi3}{13,10}{⼼、⾛}[HSK 2]
  \definition{v.}{recordar; lembrar; pensar em; trazer à mente; cruzar pelos pensamentos de alguém; passar pelo pensamento de alguém}
\end{EntryWithPhonetic}

\begin{EntryWithPhonetic}{想想看}{xiang3xiang3kan4}{13,13,9}{⼼、⼼、⽬}
  \definition{v.}{pensar sobre isso}
\end{EntryWithPhonetic}

\begin{EntryWithPhonetic}{想象}{xiang3xiang4}{13,11}{⼼、⾗}[HSK 4]
  \definition[个,种,面]{s.}{imaginação; refere-se ao processo mental de processamento e transformação de representações armazenadas na mente para formar novas imagens}
  \definition{v.}{imaginar; vislumbrar; visualizar; refere-se a ter uma imagem concreta de algo que não está na frente dos olhos}
\end{EntryWithPhonetic}

\begin{EntryWithPhonetic}{向}{xiang4}{6}{⼝}[HSK 2]
  \definition*{s.}{Sobrenome Xiang}
  \definition{adv.}{sempre; o tempo todo}
  \definition{prep.}{em direção a; para}
  \definition{s.}{direção | a janela voltada para o norte}
  \definition{v.}{encarar; virar-se para | estar do lado de; ser parcial com; tomar o partido de alguém}
\end{EntryWithPhonetic}

\begin{EntryWithPhonetic}{向导}{xiang4dao3}{6,6}{⼝、⼨}[HSK 5]
  \definition[位]{s.}{guia; a pessoa que lidera todos e lhes indica a direção ao caminhar}
  \definition{v.}{agir como um guia; mostrar a alguém o caminho; levar alguém a algum lugar}
\end{EntryWithPhonetic}

\begin{EntryWithPhonetic}{向前}{xiang4 qian2}{6,9}{⼝、⼑}[HSK 5]
  \definition{adv.}{para frente; adiante}
  \definition{v.}{avançar; ir em direção à frente; mover-se para frente; avançar um pouco mais}
\end{EntryWithPhonetic}

\begin{EntryWithPhonetic}{向上}{xiang4 shang4}{6,3}{⼝、⼀}[HSK 5]
  \definition{adv.}{o superior; acima}
  \definition{v.}{mover-se; subir; ir para um lugar mais alto; ir para um lugar mais alto em relação a um determinado ponto; ir para um desenvolvimento mais alto que o atual | avançar; continuar se aperfeiçoar; subir na vida; desenvolver-se em direção ao progresso}
\end{EntryWithPhonetic}

\begin{EntryWithPhonetic}{向汪}{xiang4wang1}{6,7}{⼝、⽔}
  \definition{v.}{esperar que}
\end{EntryWithPhonetic}

\begin{EntryWithPhonetic}{向往}{xiang4wang3}{6,8}{⼝、⼻}
  \definition{v.}{ansiar por | esperar ansiosamente por}
\end{EntryWithPhonetic}

\begin{EntryWithPhonetic}{相}{xiang4}{9}{⽬}
  \definition*{s.}{Sobrenome Xiang}
  \definition{s.}{aparência | postura; porte; postura sentada, em pé, etc. | (física) fase; refere-se a uma parte homogênea de uma substância com a mesma composição e as mesmas propriedades físicas e químicas | fotografia | primeiro-ministro (na China antiga) | ministro; títulos oficiais de certos países | fácies marinha (carvão) | elefante, uma das peças do xadrez chinês | recepcionista (pessoa que ajuda o anfitrião a receber o hóspede); antigamente, referia-se a alguém que ajudava o anfitrião a receber convidados}
  \definition{v.}{olhar e avaliar; observe a aparência das coisas; julgar sua qualidade | assistir; ajudar; auxiliar}
  \seeref{xiang1}
\end{EntryWithPhonetic}

\begin{EntryWithPhonetic}{相机}{xiang4 ji1}{9,6}{⽬、⽊}[HSK 2]
  \definition[台,部,架,个]{s.}{câmera; máquina fotográfica}
  \definition{v.}{ficar atento a uma oportunidade; procurar oportunidades}
\end{EntryWithPhonetic}

\begin{EntryWithPhonetic}{相片}{xiang4 pian4}{9,4}{⽬、⽚}[HSK 4]
  \definition[张]{s.}{foto; fotografia; uma imagem de uma pessoa ou objeto feita pela exposição de papel fotográfico a um negativo fotográfico e, em seguida, revelando e fixando a imagem.}
\end{EntryWithPhonetic}

\begin{EntryWithPhonetic}{相声}{xiang4sheng5}{9,7}{⽬、⼠}[HSK 5]
  \definition[个,段]{s.}{conversa cruzada; diálogo cômico; forma de performance humorística, em que os atores usam piadas, canções e imitações para satirizar e elogiar}
\end{EntryWithPhonetic}

\begin{EntryWithPhonetic}{项}{xiang4}{9}{⾴}[HSK 4]
  \definition*{s.}{Sobrenome Xiang}
  \definition{clas.}{usado para itens discriminados; taxonomia}
  \definition{s.}{nuca (do pescoço); a parte de trás do pescoço | soma (de dinheiro); fundos para fins especiais | termo; em álgebra, significa uma única fórmula que não é unida por um sinal de mais ou de menos | item}
\end{EntryWithPhonetic}

\begin{EntryWithPhonetic}{项目}{xiang4mu4}{9,5}{⾴、⽬}[HSK 4]
  \definition{s.}{evento; categorias em que as coisas são divididas | item; projeto; trabalhos de engenharia, acadêmicos, etc., de conteúdo específico}
\end{EntryWithPhonetic}

\begin{EntryWithPhonetic}{象}{xiang4}{11}{⾗}
  \definition*{s.}{Sobrenome Xiang}
  \definition[头,群,个]{s.}{elefante | elefante, uma das peças do xadrez chinês | aparência; forma; imagem}
  \definition{v.}{imitar | latir}
\end{EntryWithPhonetic}

\begin{EntryWithPhonetic}{象棋}{xiang4qi2}{11,12}{⾗、⽊}
  \definition[副]{s.}{xadrez chinês; um tipo de jogo de xadrez em que dois jogadores têm dezesseis peças cada: um general, dois soldados, dois elefantes, duas carruagens, dois cavalos, dois canhões e cinco soldados ; cada jogador joga de acordo com as regras e o vencedor é aquele que der o xeque no general do adversário}
\end{EntryWithPhonetic}

\begin{EntryWithPhonetic}{象征}{xiang4zheng1}{11,8}{⾗、⼻}[HSK 5]
  \definition[种]{s.}{símbolo; emblema; insígnia; \emph{token}; objeto concreto que simboliza um significado especial}
  \definition{v.}{simbolizar; significar; representar; expressar um significado especial através de algo concreto}
\end{EntryWithPhonetic}

\begin{EntryWithPhonetic}{像}{xiang4}{13}{⼈}[HSK 2]
  \definition{adv.}{parecer; parecer como se}
  \definition{s.}{imagem; retrato; semelhança a alguém | imagem}
  \definition{v.}{assemelhar-se; ser como; parecer-se com | ser como; ser tal como}
\end{EntryWithPhonetic}

\begin{EntryWithPhonetic}{消}{xiao1}{10}{⽔}
  \definition{v.}{desaparecer | dissipar; remover; eliminar; fazer desaparecer | passar o tempo de forma descontraída (recreação) | precisar; tomar (necessidade, geralmente precedido por 不, 几, 何)}
  \seealsoref{不}{bu4}
  \seealsoref{何}{he2}
  \seealsoref{几}{ji3}
\end{EntryWithPhonetic}

\begin{EntryWithPhonetic}{消除}{xiao1chu2}{10,9}{⽔、⾩}[HSK 5]
  \definition{v.}{dissipar; eliminar; limpar; tornar algo inexistente; remover (algo desfavorável)}
\end{EntryWithPhonetic}

\begin{EntryWithPhonetic}{消毒}{xiao1du2}{10,9}{⽔、⽏}[HSK 5]
  \definition{v.}{desinfetar; esterilizar; matar os microrganismos causadores de doenças por meios físicos ou químicos}
\end{EntryWithPhonetic}

\begin{EntryWithPhonetic}{消防}{xiao1fang2}{10,6}{⽔、⾩}[HSK 5]
  \definition{s.}{combate a incêncios; controle de incêndios}
\end{EntryWithPhonetic}

\begin{EntryWithPhonetic}{消防员}{xiao1fang2yuan2}{10,6,7}{⽔、⾩、⼝}
  \definition{s.}{bombeiro}
\end{EntryWithPhonetic}

\begin{EntryWithPhonetic}{消费}{xiao1fei4}{10,9}{⽔、⾙}[HSK 3]
  \definition{v.}{gastar; consumir; consumir materiais para satisfazer as necessidades de produção ou de vida (geralmente refere-se ao consumo doméstico) | consumir (recursos naturais)}
\end{EntryWithPhonetic}

\begin{EntryWithPhonetic}{消费者}{xiao1 fei4 zhe3}{10,9,8}{⽔、⾙、⽼}[HSK 5]
  \definition{s.}{consumidor; cliente; consumo; indivíduos membros da sociedade que compram e utilizam bens e serviços para consumo pessoal}
\end{EntryWithPhonetic}

\begin{EntryWithPhonetic}{消耗}{xiao1hao4}{10,10}{⽔、⽾}[HSK 6]
  \definition{v.}{gastar; esgotar; consumir; usar; (espírito, força, coisas, etc.) diminuir gradualmente devido ao uso ou perda}
\end{EntryWithPhonetic}

\begin{EntryWithPhonetic}{消化}{xiao1hua4}{10,4}{⽔、⼔}[HSK 4]
  \definition{v.}{digerir (alimentos) | digerir (conhecimento); pensar e absorver; uma metáfora para a compreensão total de novos conhecimentos ou informações e a capacidade de transformá-los em algo que possa ser usado}
\end{EntryWithPhonetic}

\begin{EntryWithPhonetic}{消极}{xiao1ji2}{10,7}{⽔、⽊}[HSK 5]
  \definition{adj.}{negativo; oposto; adverso | passivo; inativo; sem ambição; sem iniciativa; desanimado; apático}
\end{EntryWithPhonetic}

\begin{EntryWithPhonetic}{消灭}{xiao1mie4}{10,5}{⽔、⽕}[HSK 6]
  \definition{v.}{perecer; morrer; falecer; desaparecer | abolir; erradicar; eliminar; aniquilar; exterminar; acabar com; fazer com que não exista}
\end{EntryWithPhonetic}

\begin{EntryWithPhonetic}{消失}{xiao1shi1}{10,5}{⽔、⼤}[HSK 3]
  \definition{v.}{desaparecer; desvanecer; dissolver; dissipar; evaporar; sumir}
\end{EntryWithPhonetic}

\begin{EntryWithPhonetic}{消息}{xiao1xi5}{10,10}{⽔、⼼}[HSK 3]
  \definition[个,条,篇,些]{s.}{notícias; informação; reportagem sobre pessoas ou situações | notícias; novidades;}
\end{EntryWithPhonetic}

\begin{EntryWithPhonetic}{销}{xiao1}{12}{⾦}
  \definition*{s.}{Sobrenome Xiao}
  \definition{s.}{gasto; despesa | pino}
  \definition{v.}{derreter (metal) | cancelar; anular | vender; comercializar | aferrolhar; fixar; prender; pregar | fixar com um parafuso; parafusar | gastar (consumo) | inserir um pino}
\end{EntryWithPhonetic}

\begin{EntryWithPhonetic}{销售}{xiao1shou4}{12,11}{⾦、⼝}[HSK 4]
  \definition{v.}{vender; comercializar}
\end{EntryWithPhonetic}

\begin{EntryWithPhonetic}{嚣}{xiao1}{18}{⼝}
  \definition*{s.}{Sobrenome Xiao}
  \definition{adj.}{lazer}
  \definition{v.}{clamar; fazer barulho}
\end{EntryWithPhonetic}

\begin{EntryWithPhonetic}{嚣张}{xiao1zhang1}{18,7}{⼝、⼸}
  \definition{adj.}{desenfreado | arrogante | agressivo}
\end{EntryWithPhonetic}

\begin{EntryWithPhonetic}{小}{xiao3}{3}{⼩}[HSK 1,2][Kangxi 42]
  \definition*{s.}{Sobrenome Xiao}
  \definition{adj.}{menor; pequeno; insignificante; pouco; volume, área, quantidade, intensidade, etc. não são grandes | jovem | expressões humildes, referindo-se a si mesmo ou a pessoas ou coisas relacionadas a si mesmo | por um tempo; por um curto período; por um curto período de tempo | o mais novo; o último na ordem de antiguidade; em último lugar na classificação}
  \definition{pref.}{usado antes do sobrenome, nome, posição na família, etc.}
  \definition{s.}{os jovens; pessoas mais jovens | concubina}
\end{EntryWithPhonetic}

\begin{EntryWithPhonetic}{小白菜}{xiao3bai2cai4}{3,5,11}{⼩、⽩、⾋}
  \definition[棵]{s.}{\emph{bok choy} | couve chinesa}
\end{EntryWithPhonetic}

\begin{EntryWithPhonetic}{小吃}{xiao3chi1}{3,6}{⼩、⼝}[HSK 4]
  \definition[家]{s.}{lanche; petiscos; comida com especialidades locais, não muito para uma porção | prato frio; prato feito; cortes de frios na culinária ocidental | pratos pequenos e baratos; pratos simples em restaurantes com porções pequenas e preços baixos}
\end{EntryWithPhonetic}

\begin{EntryWithPhonetic}{小费}{xiao3 fei4}{3,9}{⼩、⾙}[HSK 6]
  \definition[笔]{s.}{gorjeta; gratificação; dinheiro extra pago por clientes e viajantes a funcionários de serviços em setores de serviços, como hotéis e pousadas}
\end{EntryWithPhonetic}

\begin{EntryWithPhonetic}{小狗}{xiao3 gou3}{3,8}{⼩、⽝}
  \definition{s.}{filhote de cachorro}
\end{EntryWithPhonetic}

\begin{EntryWithPhonetic}{小孩儿}{xiao3hai2r5}{3,9,2}{⼩、⼦、⼉}[HSK 1]
  \definition[个]{s.}{criança; bebê}
\end{EntryWithPhonetic}

\begin{EntryWithPhonetic}{小伙子}{xiao3huo3zi5}{3,6,3}{⼩、⼈、⼦}[HSK 4]
  \definition[位]{s.}{rapaz jovem; jovem colega}
\end{EntryWithPhonetic}

\begin{EntryWithPhonetic}{小姐}{xiao3jie5}{3,8}{⼩、⼥}[HSK 1]
  \definition[个,位]{s.}{jovem senhora; anteriormente, era assim que se referiam às filhas de famílias ricas. | senhorita; título honorífico para mulheres jovens | (gíria) prostituta}
\end{EntryWithPhonetic}

\begin{EntryWithPhonetic}{小麦}{xiao3mai4}{3,7}{⼩、⿆}[HSK 6]
  \definition[粒,公斤,吨,棵]{s.}{trigo}
\end{EntryWithPhonetic}

\begin{EntryWithPhonetic}{小朋友}{xiao3 peng2 you3}{3,8,4}{⼩、⽉、⼜}[HSK 1]
  \definition[个]{s.}{criança; crianças; refere-se a crianças e adolescentes | (termo usado por um adulto para se dirigir a uma criança) amiguinho; menino (ou menina); termo carinhoso para se referir a crianças e adolescentes}
\end{EntryWithPhonetic}

\begin{EntryWithPhonetic}{小气鬼}{xiao3qi4gui3}{3,4,9}{⼩、⽓、⿁}
  \definition{adj.}{avarento | mão-de-vaca | miserável | pão-duro}
\end{EntryWithPhonetic}

\begin{EntryWithPhonetic}{小区}{xiao3qu1}{3,4}{⼩、⼖}
  \definition{s.}{conjunto habitacional, comunidade, bairro | célula (telecomunicações)}
\end{EntryWithPhonetic}

\begin{EntryWithPhonetic}{小声}{xiao3 sheng1}{3,7}{⼩、⼠}[HSK 2]
  \definition{v.}{falar em voz baixa; falar baixinho; sussurar}
\end{EntryWithPhonetic}

\begin{EntryWithPhonetic}{小时}{xiao3shi2}{3,7}{⼩、⽇}[HSK 1]
  \definition{clas.}{hora; unidade de medida legal do tempo, 1 hora equivale a 60 minutos, é 1/24 de um dia}
  \definition[个]{s.}{hora; refere-se a um período de uma hora}
\end{EntryWithPhonetic}

\begin{EntryWithPhonetic}{小时候}{xiao3 shi2 hou5}{3,7,10}{⼩、⽇、⼈}[HSK 2]
  \definition{s.}{na infância; quando alguém era jovem; refere-se à infância}
\end{EntryWithPhonetic}

\begin{EntryWithPhonetic}{小树}{xiao3shu4}{3,9}{⼩、⽊}
  \definition[棵]{s.}{muda | arbusto | árvore pequena}
\end{EntryWithPhonetic}

\begin{EntryWithPhonetic}{小说}{xiao3shuo1}{3,9}{⼩、⾔}[HSK 2]
  \definition[本,部,篇,章]{s.}{história; romance; ficção; uma forma literária que reflete a vida social por meio da descrição de personagens, ambiente e enredo}
\end{EntryWithPhonetic}

\begin{EntryWithPhonetic}{小偷儿}{xiao3 tou1er5}{3,11,2}{⼩、⼈、⼉}[HSK 5]
  \definition{s.}{ladrão insignificante (ou furtivo); ladrãozinho | ladrão}
\end{EntryWithPhonetic}

\begin{EntryWithPhonetic}{小腿}{xiao3tui3}{3,13}{⼩、⾁}
  \definition{s.}{perna (do joelho ao calcanhar) | haste}
\end{EntryWithPhonetic}

\begin{EntryWithPhonetic}{小屋}{xiao3wu1}{3,9}{⼩、⼫}
  \definition{s.}{cabana | chalé | cabine}
\end{EntryWithPhonetic}

\begin{EntryWithPhonetic}{小小}{xiao3xiao3}{3,3}{⼩、⼩}
  \definition{adj.}{muito pequeno}
\end{EntryWithPhonetic}

\begin{EntryWithPhonetic}{小心}{xiao3xin1}{3,4}{⼩、⼼}[HSK 2]
  \definition{adj.}{cuidadoso; atento; com cautela}
  \definition{v.}{ter cuidado; ser cauteloso; estar atento; tomar cuidado; prestar atenção}
\end{EntryWithPhonetic}

\begin{EntryWithPhonetic}{小型}{xiao3 xing2}{3,9}{⼩、⼟}[HSK 4]
  \definition{adj.}{de tamanho pequeno; em pequena escala; miniatura; tipo pequeno; tamanho de bolso; tipo compacto}
  \definition{s.}{Mediterrâneo: escunas, pequenos veleiros de pesca ou turismo | pequeno \emph{rover} lunar (duas pessoas)}
\end{EntryWithPhonetic}

\begin{EntryWithPhonetic}{小学}{xiao3 xue2}{3,8}{⼩、⼦}[HSK 1]
  \definition[个]{s.}{escola primária (ou fundamental); escolas que oferecem ensino fundamental básico | estudos filológicos; antigamente, referia-se ao estudo da escrita, da fonética e da exegese}
\end{EntryWithPhonetic}

\begin{EntryWithPhonetic}{小学生}{xiao3 xue2 sheng1}{3,8,5}{⼩、⼦、⽣}[HSK 1]
  \definition{s.}{aluno; estudante; estudante do sexo masculino (男); estudante do sexo feminino (女) | um aluno mais novo (do que os outros da sua turma) | (dialeto) um menino pequeno}
  \seealsoref{男}{nan2}
  \seealsoref{女}{nv3}
\end{EntryWithPhonetic}

\begin{EntryWithPhonetic}{小洋白菜}{xiao3 yang2bai2cai4}{3,9,5,11}{⼩、⽔、⽩、⾋}
  \definition{s.}{couve de bruxelas}
\end{EntryWithPhonetic}

\begin{EntryWithPhonetic}{小于}{xiao3 yu2}{3,3}{⼩、⼆}[HSK 6]
  \definition{prep.}{menor que; menos que; indica que um número ou quantidade é menor que outro}
\end{EntryWithPhonetic}

\begin{EntryWithPhonetic}{小众}{xiao3zhong4}{3,6}{⼩、⼈}
  \definition{s.}{minoria da população | nicho (mercado, etc.)}
\end{EntryWithPhonetic}

\begin{EntryWithPhonetic}{小组}{xiao3 zu3}{3,8}{⼩、⽷}[HSK 2]
  \definition[个,名,位]{s.}{grupo; um pequeno grupo de pessoas}
\end{EntryWithPhonetic}

\begin{EntryWithPhonetic}{晓}{xiao3}{10}{⽇}
  \definition{s.}{amanhecer; alvorada}
  \definition{v.}{(um dia) amanhecer; romper | saber; deixar alguém saber; dizer}
\end{EntryWithPhonetic}

\begin{EntryWithPhonetic}{晓得}{xiao3 de2}{10,11}{⽇、⼻}[HSK 6]
  \definition{v.}{saber; entender}[我不晓得他在哪里。===Não sei onde ele está.]
\end{EntryWithPhonetic}

\begin{EntryWithPhonetic}{哮}{xiao4}{10}{⼝}
  \definition{s.}{respiração pesada; chiado}
  \definition{v.}{rugir; uivar}
\end{EntryWithPhonetic}

\begin{EntryWithPhonetic}{哮喘}{xiao4chuan3}{10,12}{⼝、⼝}
  \definition{s.}{asma; sintomas de dispneia: os pacientes sentem que a respiração está muito difícil; pneumonia, insuficiência cardíaca, bronquite crônica e outras doenças causadas por espasmo da musculatura lisa respiratória frequentemente apresentam esse sintoma}
  \definition{v.}{sofrer de asma}
\end{EntryWithPhonetic}

\begin{EntryWithPhonetic}{效}{xiao4}{10}{⽁}
  \definition{s.}{efeito; função | eficiência; resultado}
  \definition{v.}{imitar; seguir o exemplo de | dedicar (a energia ou a vida de alguém) a; prestar (um serviço)}
\end{EntryWithPhonetic}

\begin{EntryWithPhonetic}{效果}{xiao4guo3}{10,8}{⽁、⽊}[HSK 3]
  \definition[种,个]{s.}{efeito; resultado | efeitos sonoros; vários sons ou fenômenos naturais criados para combinar com o enredo em dramas e filmes, como vento e chuva, tiros, fogo, neve, etc.}
\end{EntryWithPhonetic}

\begin{EntryWithPhonetic}{效率}{xiao4lv4}{10,11}{⽁、⽞}[HSK 4]
  \definition[种]{s.}{eficiência; produtividade; a quantidade de trabalho concluído por unidade de tempo}
\end{EntryWithPhonetic}

\begin{EntryWithPhonetic}{校}{xiao4}{10}{⽊}
  \definition[所]{s.}{oficial militar | escola}
  \seeref{jiao4}
\end{EntryWithPhonetic}

\begin{EntryWithPhonetic}{校服}{xiao4fu2}{10,8}{⽊、⽉}
  \definition{s.}{uniforme escolar}
\end{EntryWithPhonetic}

\begin{EntryWithPhonetic}{校规}{xiao4gui1}{10,8}{⽊、⾒}
  \definition{s.}{regras e regulamentos escolares}
\end{EntryWithPhonetic}

\begin{EntryWithPhonetic}{校监}{xiao4jian1}{10,10}{⽊、⽫}
  \definition{s.}{diretor | supervisor (de escola)}
\end{EntryWithPhonetic}

\begin{EntryWithPhonetic}{校园}{xiao4 yuan2}{10,7}{⽊、⼞}[HSK 2]
  \definition[个]{s.}{campus; pátio da escola; refere-se a todos os terrenos e edifícios dentro da área escolar}
\end{EntryWithPhonetic}

\begin{EntryWithPhonetic}{校长}{xiao4zhang3}{10,4}{⽊、⾧}[HSK 2]
  \definition[个,位,名]{s.}{diretor; presidente; reitor; o mais alto líder administrativo e empresarial de uma escola}
\end{EntryWithPhonetic}

\begin{EntryWithPhonetic}{笑}{xiao4}{10}{⽵}[HSK 1]
  \definition{adj.}{ridículo; engraçado; risível; hilário}
  \definition{v.}{sorrir; rir; mostrar expressão de alegria; emitir sons de alegria | ridicularizar; rir de; zombar}
\end{EntryWithPhonetic}

\begin{EntryWithPhonetic}{笑话儿}{xiao4 hua4r5}{10,8,2}{⽵、⾔、⼉}[HSK 2]
  \definition{s.}{piada; brincadeira; gracejo}
\end{EntryWithPhonetic}

\begin{EntryWithPhonetic}{笑话}{xiao4hua5}{10,8}{⽵、⾔}[HSK 2]
  \definition[个]{s.}{piada; brincadeira; uma conversa ou história que faz as pessoas rirem; algo que as pessoas usam como piada}
  \definition{v.}{ridicularizar; zombar; rir de;}
\end{EntryWithPhonetic}

\begin{EntryWithPhonetic}{笑脸}{xiao4 lian3}{10,11}{⽵、⾁}[HSK 6]
  \definition{s.}{\emph{smiley}; rosto sorridente (emoji)}
\end{EntryWithPhonetic}

\begin{EntryWithPhonetic}{笑容}{xiao4 rong2}{10,10}{⽵、⼧}[HSK 6]
  \definition[丝,抹,个]{s.}{sorriso; expressão sorridente; o olhar no rosto de alguém ao sorrir}
\end{EntryWithPhonetic}

\begin{EntryWithPhonetic}{笑声}{xiao4 sheng1}{10,7}{⽵、⼠}[HSK 6]
  \definition{s.}{riso; risada}
\end{EntryWithPhonetic}

\begin{EntryWithPhonetic}{些}{xie1}{8}{⼆}[HSK 4]
  \definition{adv.}{um pouco; um pouco mais; usado após um adjetivo ou parte de um verbo para indicar uma pequena quantidade, equivalente a 一点儿}
  \definition{clas.}{alguns; um pouco; denota uma quantidade indefinida}
  \seealsoref{一点儿}{yi4dian3r5}
\end{EntryWithPhonetic}

\begin{EntryWithPhonetic}{些许}{xie1xu3}{8,6}{⼆、⾔}
  \definition{num.}{um pouco}
\end{EntryWithPhonetic}

\begin{EntryWithPhonetic}{楔}{xie1}{13}{⽊}
  \definition[个]{s.}{cunha | pino; pregos de madeira; pregos de bambu}
  \definition{v.}{cunhar}
\end{EntryWithPhonetic}

\begin{EntryWithPhonetic}{楔子}{xie1zi5}{13,3}{⽊、⼦}
  \definition{s.}{cunha | pino | prólogo ou interlúdio no drama da Dinastia Yuan | prólogo em alguns romances modernos; introduções a óperas e romances | calço; chuteira; lascas de madeira inseridas nas juntas de encaixe e espiga, etc. | estaca de madeira; estaca de bambu; pregos de madeira; pregos de bambu}
\end{EntryWithPhonetic}

\begin{EntryWithPhonetic}{歇}{xie1}{13}{⽋}[HSK 5]
  \definition*{s.}{Sobrenome Xie}
  \definition{s.}{um pouco de tempo}
  \definition{v.}{descansar; fazer uma pausa | parar (o trabalho); encerrar o expediente | dormir; ir para a cama}
\end{EntryWithPhonetic}

\begin{EntryWithPhonetic}{协}{xie2}{6}{⼗}
  \definition*{s.}{Sobrenome Xie}
  \definition{adv.}{conjuntamente; coordenadamente; juntos}
  \definition{s.}{harmonioso}
  \definition{v.}{auxiliar; assistir; ajudar}
\end{EntryWithPhonetic}

\begin{EntryWithPhonetic}{协会}{xie2hui4}{6,6}{⼗、⼈}[HSK 6]
  \definition[个]{s.}{sociedade; instituto; associação; uma organização de massa formada para promover uma causa comum}
\end{EntryWithPhonetic}

\begin{EntryWithPhonetic}{协商}{xie2shang1}{6,11}{⼗、⼝}[HSK 6]
  \definition{v.}{discutir; consultar; negociar; várias partes discutiram e decidiram em conjunto para chegar à mesma visão}
\end{EntryWithPhonetic}

\begin{EntryWithPhonetic}{协调}{xie2tiao2}{6,10}{⼗、⾔}[HSK 6]
  \definition{adj.}{coordenado; harmonioso; em sintonia}
  \definition{v.}{coordenar; concertar; integrar; harmonizar; fazer a harmonia apropriada}
\end{EntryWithPhonetic}

\begin{EntryWithPhonetic}{协议}{xie2yi4}{6,5}{⼗、⾔}[HSK 5]
  \definition[份,项]{s.}{acordo; tratado; decisão conjunta alcançada através de negociação e consulta}
  \definition{v.}{concordar em}
\end{EntryWithPhonetic}

\begin{EntryWithPhonetic}{协议书}{xie2 yi4 shu1}{6,5,4}{⼗、⾔、⼄}[HSK 5]
  \definition{s.}{contrato | protocolo}
\end{EntryWithPhonetic}

\begin{EntryWithPhonetic}{协助}{xie2zhu4}{6,7}{⼗、⼒}[HSK 6]
  \definition{v.}{ajudar; auxiliar; dar assistência; fornecer ajuda}
\end{EntryWithPhonetic}

\begin{EntryWithPhonetic}{斜}{xie2}{11}{⽃}[HSK 5]
  \definition{adj.}{oblíquo; inclinado | enviesado; chanfrado; diagonal; torto; nem paralelo nem perpendicular a um plano ou linha}
  \definition{v.}{virar de lado; inclinar}
\end{EntryWithPhonetic}

\begin{EntryWithPhonetic}{斜阳}{xie2yang2}{11,6}{⽃、⾩}
  \definition{s.}{sol poente}
\end{EntryWithPhonetic}

\begin{EntryWithPhonetic}{谐}{xie2}{11}{⾔}
  \definition{adj.}{harmonioso | humorístico}
\end{EntryWithPhonetic}

\begin{EntryWithPhonetic}{鞋}{xie2}{15}{⾰}[HSK 2]
  \definition[双,只]{s.}{sapatos; usado nos pés; algo que toca o chão ao caminhar; sem cano alto}
\end{EntryWithPhonetic}

\begin{EntryWithPhonetic}{写}{xie3}{5}{⼍}[HSK 1]
  \definition{v.}{escrever | compor; escrever (como autor, repórter, etc.) | descrever; retratar | pintar; desenhar | expressar a imagem das coisas através da linguagem e da escrita | desenhar (pintura)}
\end{EntryWithPhonetic}

\begin{EntryWithPhonetic}{写意}{xie3yi4}{5,13}{⼍、⼼}
  \definition{s.}{estilo de pintura chinesa à mão livre, caracterizado por traços ousados em vez de detalhes precisos}
  \definition{v.}{sugerir (em vez de descrever em detalhes)}
  \seeref{xie4yi4}
\end{EntryWithPhonetic}

\begin{EntryWithPhonetic}{写照}{xie3zhao4}{5,13}{⼍、⽕}
  \definition{s.}{retrato}
\end{EntryWithPhonetic}

\begin{EntryWithPhonetic}{写真}{xie3zhen1}{5,10}{⼍、⼗}
  \definition{s.}{retrato}
  \definition{v.}{descrever algo com precisão}
\end{EntryWithPhonetic}

\begin{EntryWithPhonetic}{写字}{xie3zi4}{5,6}{⼍、⼦}
  \definition{v.}{escrever (à mão) | praticar caligrafia}
\end{EntryWithPhonetic}

\begin{EntryWithPhonetic}{写字匠}{xie3zi4 jiang4}{5,6,6}{⼍、⼦、⼕}
  \definition{s.}{calígrafo}
\end{EntryWithPhonetic}

\begin{EntryWithPhonetic}{写字楼}{xie3 zi4 lou2}{5,6,13}{⼍、⼦、⽊}[HSK 6]
  \definition{s.}{prédio de escritórios}
\end{EntryWithPhonetic}

\begin{EntryWithPhonetic}{写字台}{xie3 zi4 tai2}{5,6,5}{⼍、⼦、⼝}[HSK 6]
  \definition[个,张]{s.}{escrivaninha; secretária; escrivaninha de escrever; uma mesa retangular usada para escrever e trabalhar, com gavetas e algumas com pequenos armários}
\end{EntryWithPhonetic}

\begin{EntryWithPhonetic}{写作}{xie3zuo4}{5,7}{⼍、⼈}[HSK 3]
  \definition{v.}{escrever artigos; escrever livros, etc.; também se refere especificamente à criação de obras literárias}
\end{EntryWithPhonetic}

\begin{EntryWithPhonetic}{血}{xie3}{6}{⾎}[Kangxi 143]
  \seeref{xue4}
\end{EntryWithPhonetic}

\begin{EntryWithPhonetic}{写意}{xie4yi4}{5,13}{⼍、⼼}
  \definition{adj.}{confortável | agradável | relaxado}
  \seeref{xie3yi4}
\end{EntryWithPhonetic}

\begin{EntryWithPhonetic}{泄}{xie4}{8}{⽔}
  \definition*{s.}{Sobrenome Xie}
  \definition{v.}{deixar sair (um fluido ou gás); descarregar; liberar | revelar (um segredo); vazar (notícias, segredos, etc.) | dar vazão a; desabafar}
\end{EntryWithPhonetic}

\begin{EntryWithPhonetic}{泄底}{xie4di3}{8,8}{⽔、⼴}
  \definition{v.}{revelar ou expor o que está no fundo de algo | divulgar a história interna; vazar segredos}
\end{EntryWithPhonetic}

\begin{EntryWithPhonetic}{泄愤}{xie4fen4}{8,12}{⽔、⼼}
  \definition{v.}{dar vazão à raiva}
\end{EntryWithPhonetic}

\begin{EntryWithPhonetic}{泄洪}{xie4hong2}{8,9}{⽔、⽔}
  \definition{v.}{liberar água da enchente (descarga de inundação)}
\end{EntryWithPhonetic}

\begin{EntryWithPhonetic}{泄露}{xie4lou4}{8,21}{⽔、⾬}
  \definition{v.}{vazar; deixar escapar; divulgar; revelar (um segredo, etc.) | vazar; escapar; descarregar (um fluido ou gás)}
\end{EntryWithPhonetic}

\begin{EntryWithPhonetic}{泄气}{xie4/qi4}{8,4}{⽔、⽓}
  \definition{adj.}{decepcionante | frustrante | patético}
  \definition{v.+compl.}{perder o coração | sentir-se desencorajado | ficar desanimado}
\end{EntryWithPhonetic}

\begin{EntryWithPhonetic}{谢}{xie4}{12}{⾔}
  \definition*{s.}{Sobrenome Xie}
  \definition{v.}{agradecer | desculpar-se; pedir desculpas; admitir a própria culpa | recusar; declinar; renunciar | murchar; perder de flores ou folhas}
\end{EntryWithPhonetic}

\begin{EntryWithPhonetic}{谢病}{xie4bing4}{12,10}{⾔、⽧}
  \definition{v.}{desculpar-se por causa de doença}
\end{EntryWithPhonetic}

\begin{EntryWithPhonetic}{谢恩}{xie4'en1}{12,10}{⾔、⼼}
  \definition{v.}{agradecer a alguém pelo favor (especialmente imperador ou oficial superior)}
\end{EntryWithPhonetic}

\begin{EntryWithPhonetic}{谢媒}{xie4mei2}{12,12}{⾔、⼥}
  \definition{v.}{agradecer ao casamenteiro}
\end{EntryWithPhonetic}

\begin{EntryWithPhonetic}{谢世}{xie4shi4}{12,5}{⾔、⼀}
  \definition{v.}{morrer | falecer}
\end{EntryWithPhonetic}

\begin{EntryWithPhonetic}{谢天谢地}{xie4tian1xie4di4}{12,4,12,6}{⾔、⼤、⾔、⼟}
  \definition{expr.}{agradecer a Deus | agradecer aos céus}
\end{EntryWithPhonetic}

\begin{EntryWithPhonetic}{谢谢}{xie4xie5}{12,12}{⾔、⾔}[HSK 1]
  \definition{interj.}{Obrigado!}
  \definition{v.}{agradecer; agradecer a gentileza dos outros}
\end{EntryWithPhonetic}

\begin{EntryWithPhonetic}{谢意}{xie4yi4}{12,13}{⾔、⼼}
  \definition{s.}{gratidão}
\end{EntryWithPhonetic}

\begin{EntryWithPhonetic}{心}{xin1}{4}{⼼}[HSK 3][Kangxi 61]
  \definition*{s.}{Xin, uma das mansões lunares; uma das vinte e oito constelações}
  \definition[颗,个]{s.}{o coraçã; órgão que impulsiona a circulação sanguínea no corpo humano e nos vertebrados| coração; mente; sentimento; intenção; refere-se aos órgãos do pensamento e ao pensamento, sentimentos, etc. | centro; núcleo; parte central}
\end{EntryWithPhonetic}

\begin{EntryWithPhonetic}{心机}{xin1ji1}{4,6}{⼼、⽊}
  \definition{s.}{pensamento | esquema}
\end{EntryWithPhonetic}

\begin{EntryWithPhonetic}{心里}{xin1 li3}{4,7}{⼼、⾥}[HSK 2]
  \definition[个]{s.}{no coração; no coração de alguém | no coração; na mente; na cabeça e no peito}
\end{EntryWithPhonetic}

\begin{EntryWithPhonetic}{心理}{xin1li3}{4,11}{⼼、⽟}[HSK 4]
  \definition[个]{s.}{mentalidade; refere-se à reflexão da mente humana sobre coisas objetivas, incluindo sensação, percepção, memória, pensamento e emoções | psicologia}
\end{EntryWithPhonetic}

\begin{EntryWithPhonetic}{心灵}{xin1ling2}{4,7}{⼼、⽕}[HSK 6]
  \definition[个,颗]{s.}{alma; coração; espírito; refere-se ao coração, espírito, pensamentos, etc.}
\end{EntryWithPhonetic}

\begin{EntryWithPhonetic}{心情}{xin1qing2}{4,11}{⼼、⼼}[HSK 2]
  \definition{s.}{humor; tom de sentimento; estado de espírito; estado emocional interior}
\end{EntryWithPhonetic}

\begin{EntryWithPhonetic}{心声}{xin1sheng1}{4,7}{⼼、⼠}
  \definition{s.}{desejo sincero | voz interior | aspiração}
\end{EntryWithPhonetic}

\begin{EntryWithPhonetic}{心态}{xin1tai4}{4,8}{⼼、⼼}[HSK 5]
  \definition[种,个]{s.}{mentalidade; psicologia; estado mental}
\end{EntryWithPhonetic}

\begin{EntryWithPhonetic}{心疼}{xin1teng2}{4,10}{⼼、⽧}[HSK 5]
  \definition{v.}{amar profundamente; sentir pena porque coisas valiosas foram destruídas ou perdidas; não querer se separar delas | sentir pena; ficar angustiado; preocupar-se e sofrer pelo sofrimento dos outros; estar disposto a cuidar mais por causa da preocupação}
\end{EntryWithPhonetic}

\begin{EntryWithPhonetic}{心愿}{xin1 yuan4}{4,14}{⼼、⽕}[HSK 6]
  \definition[桩]{s.}{desejo acalentado; aspiração; desejo; sonho | o desejo do coração}
\end{EntryWithPhonetic}

\begin{EntryWithPhonetic}{心脏}{xin1zang4}{4,10}{⼼、⾁}[HSK 6]
  \definition[颗,个]{s.}{coração; um órgão importante no corpo de humanos ou animais superiores que faz o sangue circular | coração; o centro ou a parte mais importante de uma metáfora}
\end{EntryWithPhonetic}

\begin{EntryWithPhonetic}{心脏病}{xin1 zang4 bing4}{4,10,10}{⼼、⾁、⽧}[HSK 6]
  \definition{s.}{doença cardíaca; cardiopatia um termo geral para anormalidades ou doenças na estrutura e função do coração humano}
\end{EntryWithPhonetic}

\begin{EntryWithPhonetic}{心中}{xin1zhong1}{4,4}{⼼、⼁}[HSK 2]
  \definition{s.}{no coração; na mente}
\end{EntryWithPhonetic}

\begin{EntryWithPhonetic}{芯}{xin1}{7}{⾋}
  \definition{s.}{medula de junco | pavio}
  \seeref{xin4}
\end{EntryWithPhonetic}

\begin{EntryWithPhonetic}{芯片}{xin1pian4}{7,4}{⾋、⽚}
  \definition{s.}{\emph{chip} de computador; \emph{microchip}; um substrato (geralmente uma pastilha de silício) que contém um circuito integrado completo}
\end{EntryWithPhonetic}

\begin{EntryWithPhonetic}{辛}{xin1}{7}{⾟}[Kangxi 160]
  \definition*{s.}{Sobrenome Xin}
  \definition{adj.}{quente (no sabor, etc.); pungente | difícil; trabalhoso | ponto da bússola chinesa antiga: 285° | oitavo na ordem}
  \definition{pref.}{octa-}
  \definition{s.}{sofrimento}
  \definition{s.}{oitavo dos dez caules celestiais}
\end{EntryWithPhonetic}

\begin{EntryWithPhonetic}{辛苦}{xin1ku3}{7,8}{⾟、⾋}[HSK 5]
  \definition{adj.}{difícil; trabalhoso; árduo; descreve muito trabalho, alta intensidade e pouco descanso}
  \definition{s.}{dificuldades}
  \definition{v.}{trabalhar duro; passar por grandes dificuldades; passar por dificuldades}
\end{EntryWithPhonetic}

\begin{EntryWithPhonetic}{欣}{xin1}{8}{⽋}
  \definition*{s.}{Sobrenome Xin}
  \definition{adj.}{alegre; feliz; contente}
\end{EntryWithPhonetic}

\begin{EntryWithPhonetic}{欣赏}{xin1shang3}{8,12}{⽋、⾙}[HSK 5]
  \definition{v.}{apreciar; admirar; valorizar; apreciar as coisas boas e descubrir o prazer que elas proporcionam | apreciar; gostar; considerar bom}
\end{EntryWithPhonetic}

\begin{EntryWithPhonetic}{新}{xin1}{13}{⽄}[HSK 1]
  \definition*{s.}{Xinjiang, abreviação de 新疆 | Singapura, abreviação de 新加坡 | Sobrenome Xin}
  \definition{adj.}{novo; fresco; inovador; atualizado; aparecer ou ser experimentado pela primeira vez | nunca utilizado; novo; não foi usado ou foi usado por pouco tempo | recém-casado}
  \definition{adv.}{recém; recentemente; há pouco tempo}
  \definition{pref.}{(química) meso-}
  \definition{v.}{atualizar; renovar}
  \seealsoref{新加坡}{xin1jia1po1}
  \seealsoref{新疆}{xin1jiang1}
\end{EntryWithPhonetic}

\begin{EntryWithPhonetic}{新加坡}{xin1jia1po1}{13,5,8}{⽄、⼒、⼟}
  \definition*{s.}{Singapura}
\end{EntryWithPhonetic}

\begin{EntryWithPhonetic}{新疆}{xin1jiang1}{13,19}{⽄、⼸}
  \definition*{s.}{Região Autônoma Uigur de Xinjiang}
\end{EntryWithPhonetic}

\begin{EntryWithPhonetic}{新疆维吾尔自治区}{xin1jiang1 wei2wu2'er3 zi4zhi4qu1}{13,19,11,7,5,6,8,4}{⽄、⼸、⽷、⼝、⼩、⾃、⽔、⼖}
  \definition*{s.}{Região Autônoma Uigur de Xinjiang}
\end{EntryWithPhonetic}

\begin{EntryWithPhonetic}{新郎}{xin1lang2}{13,8}{⽄、⾢}[HSK 4]
  \definition[位,名,个,些]{s.}{noivo; homens no momento do casamento}
\end{EntryWithPhonetic}

\begin{EntryWithPhonetic}{新年}{xin1 nian2}{13,6}{⽄、⼲}[HSK 1]
  \definition*[个]{s.}{Ano Novo}
\end{EntryWithPhonetic}

\begin{EntryWithPhonetic}{新娘}{xin1niang2}{13,10}{⽄、⼥}[HSK 4]
  \definition[位,个]{s.}{noiva; a mulher no momento do casamento}
  \seealsoref{新娘子}{xin1niang2zi5}
\end{EntryWithPhonetic}

\begin{EntryWithPhonetic}{新娘服装}{xin1niang2 fu2zhuang1}{13,10,8,12}{⽄、⼥、⽉、⾐}
  \definition{s.}{roupas de noiva}
\end{EntryWithPhonetic}

\begin{EntryWithPhonetic}{新娘子}{xin1niang2zi5}{13,10,3}{⽄、⼥、⼦}
  \definition{s.}{noiva}
  \seealsoref{新娘}{xin1niang2}
\end{EntryWithPhonetic}

\begin{EntryWithPhonetic}{新人}{xin1 ren2}{13,2}{⽄、⼈}[HSK 6]
  \definition[位]{s.}{pessoas de um novo tipo; nova pessoa;  pessoa que virou uma nova página | nova personalidade; novo talento | recém-chegado; novo membro | noiva ou noivo; recém-casado | \emph{neoanthropus}; \emph{homo sapiens}}
\end{EntryWithPhonetic}

\begin{EntryWithPhonetic}{新闻}{xin1wen2}{13,9}{⽄、⾨}[HSK 2]
  \definition[个,条,则,版]{s.}{notícias; notícias nacionais e internacionais reportadas em jornais, estações de rádio, etc. | notícias; refere-se a coisas importantes ou novas que aconteceram recentemente na sociedade}
\end{EntryWithPhonetic}

\begin{EntryWithPhonetic}{新鲜}{xin1xian1}{13,14}{⽄、⿂}
  \definition{adj.}{fresco (experiência, alimento, etc.)}
  \definition{s.}{frescor}
\end{EntryWithPhonetic}

\begin{EntryWithPhonetic}{新兴}{xin1 xing1}{13,6}{⽄、⼋}[HSK 6]
  \definition[个]{adj.}{recém-desenvolvido; crescente; florescente; emergente; descreve algo que está apenas começando a se tornar popular ou se desenvolver}
\end{EntryWithPhonetic}

\begin{EntryWithPhonetic}{新型}{xin1 xing2}{13,9}{⽄、⼟}[HSK 4]
  \definition[种]{s.}{ultimo modelo; novo tipo; novo padrão; novo estilo}
\end{EntryWithPhonetic}

\begin{EntryWithPhonetic}{薪}{xin1}{16}{⾋}
  \definition{s.}{lenha; combustível | salário; ordenado; pagamento}
\end{EntryWithPhonetic}

\begin{EntryWithPhonetic}{薪水}{xin1shui3}{16,4}{⾋、⽔}[HSK 6]
  \definition[份,笔]{s.}{pagamento; salário; ordenados; dinheiro ou bens pagos regularmente aos trabalhadores como compensação pelo seu trabalho}
\end{EntryWithPhonetic}

\begin{EntryWithPhonetic}{芯}{xin4}{7}{⾋}
  \definition{s.}{núcleo; a parte central de um objeto | língua de cobra}
  \seeref{xin1}
\end{EntryWithPhonetic}

\begin{EntryWithPhonetic}{信}{xin4}{9}{⼈}[HSK 2,3]
  \definition*{s.}{Sobrenome Xin}
  \definition{adj.}{verdade}
  \definition[封,个,张]{s.}{carta; correio | mensagem; notícia; informação | sinal; evidência | confiança; fé; crédito | detonador (de bombas, etc.) | arsênico}
  \definition{v.}{acreditar; fazer um balanço; dar crédito | deixar à vontade; deixar à mercê; deixar ao acaso | professar fé em; acreditar em}
\end{EntryWithPhonetic}

\begin{EntryWithPhonetic}{信访}{xin4fang3}{9,6}{⼈、⾔}
  \definition{s.}{carta de reclamação | carta de petição}
  \seealsoref{上访}{shang4fang3}
\end{EntryWithPhonetic}

\begin{EntryWithPhonetic}{信封}{xin4feng1}{9,9}{⼈、⼨}[HSK 3]
  \definition[个,封]{s.}{envelope para cartas}
\end{EntryWithPhonetic}

\begin{EntryWithPhonetic}{信号}{xin4hao4}{9,5}{⼈、⼝}[HSK 2]
  \definition[个,道]{s.}{sinal; luz, ondas de rádio, som, movimento, etc. usados para transmitir mensagens ou comandos | ponte de sinalização; marcação para chamar a atenção, ajudar na identificação e na memória}
\end{EntryWithPhonetic}

\begin{EntryWithPhonetic}{信经}{xin4jing1}{9,8}{⼈、⽷}
  \definition[个]{s.}{crença | credo (seção da missa católica)}
\end{EntryWithPhonetic}

\begin{EntryWithPhonetic}{信念}{xin4nian4}{9,8}{⼈、⼼}[HSK 5]
  \definition[个,种]{s.}{fé; crença; convicção; concepções consideradas corretas e acreditadas com convicção}
\end{EntryWithPhonetic}

\begin{EntryWithPhonetic}{信任}{xin4ren4}{9,6}{⼈、⼈}[HSK 3]
  \definition{s.}{confiança; um estado mental positivo e conexão emocional}
  \definition{v.}{confiar; ter confiança em; acreditar e ousar confiar}
\end{EntryWithPhonetic}

\begin{EntryWithPhonetic}{信息}{xin4xi1}{9,10}{⼈、⼼}[HSK 2]
  \definition[个,条,段,些]{s.}{notícias; informações; as últimas notícias sobre alguém ou alguma coisa | mensagem; informação; na teoria da informação, uma mensagem transmitida usando símbolos, cujo conteúdo é desconhecido pelo receptor}
\end{EntryWithPhonetic}

\begin{EntryWithPhonetic}{信箱}{xin4 xiang1}{9,15}{⼈、⾋}[HSK 5]
  \definition{s.}{caixa de correio; caixa postal instalada pelos correios para que as pessoas possam depositar cartas | caixa postal; caixas com números, localizadas nos correios, que podem ser alugadas para receber correspondência; chamadas de caixas postais exclusivas}
\end{EntryWithPhonetic}

\begin{EntryWithPhonetic}{信心}{xin4xin1}{9,4}{⼈、⼼}[HSK 2]
  \definition[个]{s.}{confiança; fé (em alguém ou algo) ; a crença de que os desejos se tornarão realidade}
\end{EntryWithPhonetic}

\begin{EntryWithPhonetic}{信仰}{xin4yang3}{9,6}{⼈、⼈}[HSK 6]
  \definition[种]{s.}{crença; religião; refere-se à ideia de acreditar, adorar e tomar algo como padrão e guia para palavras e ações}
  \definition{v.}{acreditar; crer em; acreditar e adorar uma determinada religião ou doutrina e tomá-la como guia para palavras e ações}
\end{EntryWithPhonetic}

\begin{EntryWithPhonetic}{信用}{xin4 yong4}{9,5}{⼈、⽤}[HSK 6]
  \definition{adj.}{crédito; não é necessária nenhuma garantia material e o dinheiro pode ser reembolsado no prazo}
  \definition[些]{s.}{crédito; confiabilidade; a confiança que você ganha ao fazer o que prometeu | crédito; uma relação de empréstimo-pagamento ou situação em que o empréstimo é condicionado ao pagamento; uma situação em que um banco empresta dinheiro temporariamente a um cliente e este posteriormente devolve o dinheiro ao banco}
\end{EntryWithPhonetic}

\begin{EntryWithPhonetic}{信用卡}{xin4yong4ka3}{9,5,5}{⼈、⽤、⼘}[HSK 2]
  \definition[张]{s.}{cartão de crédito; moeda eletrônica emitida por um banco ou outra instituição especializada para consumidores; os titulares do cartão podem usá-lo para sacar dinheiro ou fazer compras de acordo com os regulamentos}
\end{EntryWithPhonetic}

\begin{EntryWithPhonetic}{兴}{xing1}{6}{⼋}
  \definition*{s.}{Sobrenome Xing}
  \definition{adj.}{próspero; florescente}
  \definition{adv.}{Dialeto: talvez}
  \definition{v.}{ascender; prosperar; prevalecer; tornar-se popular | promover; encorajar; fazer prevalecer | começar; iniciar; lançar; mobilizar | erguer-se; levantar-se | (usualmente no negativo) permitir; deixar}
  \seeref{xing4}
\end{EntryWithPhonetic}

\begin{EntryWithPhonetic}{兴奋}{xing1fen4}{6,8}{⼋、⼤}[HSK 4]
  \definition{adj.}{animado; excitante; empolgante;}
  \definition{s.}{excitação; empolgação}
  \definition{v.}{excitar; intoxicar}
\end{EntryWithPhonetic}

\begin{EntryWithPhonetic}{兴旺}{xing1wang4}{6,8}{⼋、⽇}[HSK 6]
  \definition{adj.}{próspero; propício; favorável; auspicioso}
\end{EntryWithPhonetic}

\begin{EntryWithPhonetic}{星}{xing1}{9}{⽇}
  \definition*{s.}{Xing, a vigésima quinta das vinte e oito constelações em que a esfera celeste era dividida na antiga astronomia chinesa, consistindo em sete estrelas em Hydra}
  \definition[颗]{s.}{estrela | (astronomia) corpo celeste | partícula | pequenas marcas no braço de uma balança romana indicando jin e suas frações | artista famoso (estrela de cinema, estrela de jogos de bola, etc.) | satélite (artificial) | pequena quantidade}
\end{EntryWithPhonetic}

\begin{EntryWithPhonetic}{星表}{xing1biao3}{9,8}{⽇、⾐}
  \definition{s.}{catálogo de estrelas}
\end{EntryWithPhonetic}

\begin{EntryWithPhonetic}{星辰}{xing1chen2}{9,7}{⽇、⾠}
  \definition{s.}{estrelas}
\end{EntryWithPhonetic}

\begin{EntryWithPhonetic}{星火}{xing1huo3}{9,4}{⽇、⽕}
  \definition{s.}{trilha de meteoro (usada principalmente em expressões como 急如星火) | faísca}
\end{EntryWithPhonetic}

\begin{EntryWithPhonetic}{星期}{xing1qi1}{9,12}{⽇、⽉}[HSK 1]
  \definition[个]{s.}{semana | dias da semana; usado em conjunto com 日, 一, 二, 三, 四, 五, 六, 天, indica um determinado dia da semana | abreviação de domingo}
  \seealsoref{星期二}{xing1 qi1 er4}
  \seealsoref{星期六}{xing1 qi1 liu4}
  \seealsoref{星期日}{xing1 qi1 ri4}
  \seealsoref{星期三}{xing1 qi1 san1}
  \seealsoref{星期四}{xing1 qi1 si4}
  \seealsoref{星期天}{xing1 qi1 tian1}
  \seealsoref{星期五}{xing1 qi1 wu3}
  \seealsoref{星期一}{xing1 qi1 yi1}
\end{EntryWithPhonetic}

\begin{EntryWithPhonetic}{星期二}{xing1 qi1 er4}{9,12,2}{⽇、⽉、⼆}[HSK 1]
  \definition{s.}{terça-feira}
\end{EntryWithPhonetic}

\begin{EntryWithPhonetic}{星期六}{xing1 qi1 liu4}{9,12,4}{⽇、⽉、⼋}[HSK 1]
  \definition{s.}{sábado}
\end{EntryWithPhonetic}

\begin{EntryWithPhonetic}{星期日}{xing1 qi1 ri4}{9,12,4}{⽇、⽉、⽇}[HSK 1]
  \definition{s.}{domingo}
  \seealsoref{星期天}{xing1 qi1 tian1}
\end{EntryWithPhonetic}

\begin{EntryWithPhonetic}{星期三}{xing1 qi1 san1}{9,12,3}{⽇、⽉、⼀}[HSK 1]
  \definition{s.}{quarta-feira}
\end{EntryWithPhonetic}

\begin{EntryWithPhonetic}{星期四}{xing1 qi1 si4}{9,12,5}{⽇、⽉、⼞}[HSK 1]
  \definition{s.}{quinta-feira}
\end{EntryWithPhonetic}

\begin{EntryWithPhonetic}{星期天}{xing1 qi1 tian1}{9,12,4}{⽇、⽉、⼤}[HSK 1]
  \definition{s.}{domingo}
  \seealsoref{星期日}{xing1 qi1 ri4}
\end{EntryWithPhonetic}

\begin{EntryWithPhonetic}{星期五}{xing1 qi1 wu3}{9,12,4}{⽇、⽉、⼆}[HSK 1]
  \definition{s.}{sexta-feira}
\end{EntryWithPhonetic}

\begin{EntryWithPhonetic}{星期一}{xing1 qi1 yi1}{9,12,1}{⽇、⽉、⼀}[HSK 1]
  \definition{s.}{segunda-feira}
\end{EntryWithPhonetic}

\begin{EntryWithPhonetic}{星星}{xing1 xing5}{9,9}{⽇、⽇}[HSK 2]
  \definition[颗,群,片]{s.}{estrela; em astronomia, refere-se aos corpos celestes luminosos no universo, como as estrelas que brilham no céu noturno | estrela; uma metáfora para alguém ou algo que se destaca em um determinado campo e atrai atenção | objetos em forma de estrela}
\end{EntryWithPhonetic}

\begin{EntryWithPhonetic}{星座}{xing1zuo4}{9,10}{⽇、⼴}
  \definition[张]{s.}{signo astrológico | constelação}
\end{EntryWithPhonetic}

\begin{EntryWithPhonetic}{猩}{xing1}{12}{⽝}
  \definition[只]{s.}{orangotango}
\end{EntryWithPhonetic}

\begin{EntryWithPhonetic}{猩猩}{xing1xing5}{12,12}{⽝、⽝}
  \definition{s.}{orangotango}
\end{EntryWithPhonetic}

\begin{EntryWithPhonetic}{行}{xing2}{6}{⾏}[HSK 1][Kangxi 144]
  \definition*{s.}{Sobrenome Xing}
  \definition{adj.}{de viajar; relacionado a viagens | temporário; improvisado; provisório | capaz; competente}
  \definition{adv.}{em breve}
  \definition{s.}{comportamento; conduta | caligrafia cursiva (na caligrafia chinesa); escrita cursiva}
  \definition{v.}{ir | fazer uma viagem | estar em voga; prevalecer; circular | fazer; executar; realizar; envolver-se em | estar tudo bem; O.K. | indica a realização de uma determinada atividade (usado principalmente antes de verbos dissilábicos) | (em medicina) fazer efeito}
  \seeref{hang2}
  \seeref{heng2}
\end{EntryWithPhonetic}

\begin{EntryWithPhonetic}{行程}{xing2 cheng2}{6,12}{⾏、⽲}[HSK 6]
  \definition{s.}{rota ou distância de viagem; distância; jornada | curso; progresso; processo | curso; deslocamento; viagem}[活塞行程有点不对劲。===Há algo errado com o curso do pistão.]
\end{EntryWithPhonetic}

\begin{EntryWithPhonetic}{行动}{xing2dong4}{6,6}{⾏、⼒}[HSK 2]
  \definition[次,场,项]{s.}{ação; operação; comportamento;}
  \definition{v.}{circular; mover-se; andar | agir; tomar medidas; atividades para atingir um determinado propósito}
\end{EntryWithPhonetic}

\begin{EntryWithPhonetic}{行进}{xing2jin4}{6,7}{⾏、⾡}
  \definition{s.}{avançar | movimentar-se para frente}
\end{EntryWithPhonetic}

\begin{EntryWithPhonetic}{行礼}{xing2li3}{6,5}{⾏、⽰}
  \definition{v.}{saudar | fazer saudação}
\end{EntryWithPhonetic}

\begin{EntryWithPhonetic}{行李}{xing2li5}{6,7}{⾏、⽊}[HSK 3]
  \definition[点,个]{s.}{bagagem, malas, cestas de vime, etc. que você leva quando sai de casa}
\end{EntryWithPhonetic}

\begin{EntryWithPhonetic}{行人}{xing2ren2}{6,2}{⾏、⼈}[HSK 2]
  \definition[个]{s.}{pedestre; transeunte; viajante à pé; pessoas caminhando na estrada}
\end{EntryWithPhonetic}

\begin{EntryWithPhonetic}{行驶}{xing2 shi3}{6,8}{⾏、⾺}[HSK 5]
  \definition{v.}{ir; navegar; viajar (utilizando um veículo, navio, etc.)}
\end{EntryWithPhonetic}

\begin{EntryWithPhonetic}{行为}{xing2wei2}{6,4}{⾏、⼂}[HSK 2]
  \definition[个,种,类]{s.}{ação; comportamento; conduta; atividades que são controladas por pensamentos e manifestadas externamente}
\end{EntryWithPhonetic}

\begin{EntryWithPhonetic}{行星}{xing2xing1}{6,9}{⾏、⽇}
  \definition[颗]{s.}{planeta}
  \seealsoref{惑星}{huo4xing1}
\end{EntryWithPhonetic}

\begin{EntryWithPhonetic}{行凶}{xing2/xiong1}{6,4}{⾏、⼐}
  \definition{v.+compl.}{cometer agressão física ou assassinato | fazer algo violento}
\end{EntryWithPhonetic}

\begin{EntryWithPhonetic}{形}{xing2}{7}{⼺}[HSK 6]
  \definition{s.}{forma; formato | corpo; entidade}
  \definition{v.}{aparecer; revelar; mostrar | comparar; contrastar}
\end{EntryWithPhonetic}

\begin{EntryWithPhonetic}{形成}{xing2cheng2}{7,6}{⼺、⼽}[HSK 3]
  \definition{v.}{moldar; formar; tomar forma; tornar-se algo ou surgir uma situação após mudanças e desenvolvimentos}
\end{EntryWithPhonetic}

\begin{EntryWithPhonetic}{形而上学}{xing2'er2shang4xue2}{7,6,3,8}{⼺、⽽、⼀、⼦}
  \definition{s.}{metafísica}
\end{EntryWithPhonetic}

\begin{EntryWithPhonetic}{形容}{xing2rong2}{7,10}{⼺、⼧}[HSK 4]
  \definition{s.}{aparência; semblante}
  \definition{v.}{descrever}
\end{EntryWithPhonetic}

\begin{EntryWithPhonetic}{形式}{xing2shi4}{7,6}{⼺、⼷}[HSK 3]
  \definition[种,个]{s.}{forma; formato; modalidade; a aparência, estrutura ou estado das coisas, etc.}
\end{EntryWithPhonetic}

\begin{EntryWithPhonetic}{形势}{xing2shi4}{7,8}{⼺、⼒}[HSK 4]
  \definition[个,种]{s.}{terreno; características topográficas; situação geográfica, principalmente de uma perspectiva militar | situação; circunstâncias; a situação geral, a tendência de como as coisas estão se desenvolvendo e mudando | geralmente não é usado em situações pessoais}
\end{EntryWithPhonetic}

\begin{EntryWithPhonetic}{形态}{xing2tai4}{7,8}{⼺、⼼}[HSK 5]
  \definition[种]{s.}{forma; forma como as coisas se apresentam | forma; padrão; postura | morfologia; forma; Gramática: refere-se às formas internas de mudança das palavras, incluindo a formação de palavras e as mudanças morfológicas}
\end{EntryWithPhonetic}

\begin{EntryWithPhonetic}{形象}{xing2xiang4}{7,11}{⼺、⾗}[HSK 3]
  \definition{adj.}{vívido; expressão concreta e vívida}
  \definition[个,种]{s.}{imagem; forma; figura; formas ou posturas específicas que podem despertar pensamentos ou emoções nas pessoas | imagem literária; imagem artística; pessoas ou coisas com características diferentes criadas na literatura, no cinema e em outras artes}
\end{EntryWithPhonetic}

\begin{EntryWithPhonetic}{形状}{xing2zhuang4}{7,7}{⼺、⽝}[HSK 3]
  \definition[个,种]{s.}{forma; aparência ; aspecto; a aparência de um objeto ou figura, representada pela combinação de superfícies ou linhas externas}
\end{EntryWithPhonetic}

\begin{EntryWithPhonetic}{型}{xing2}{9}{⼟}[HSK 4]
  \definition{s.}{molde; modelo | modelo; tipo; padrão}
\end{EntryWithPhonetic}

\begin{EntryWithPhonetic}{型号}{xing2 hao4}{9,5}{⼟、⼝}[HSK 4]
  \definition[个,种]{s.}{modelo; tipo; refere-se ao desempenho, às especificações e ao tamanho de aeronaves, máquinas, implementos agrícolas, etc.}
\end{EntryWithPhonetic}

\begin{EntryWithPhonetic}{省}{xing3}{9}{⽬}
  \definition{v.}{examinar-se criticamente; verificar (os próprios pensamentos, palavras e ações) | visitar (especialmente os pais ou pessoas mais velhas) | estar ciente; tornar-se consciente; compreender; tomar consciência | examinar minuciosamente; inspecionar; escrutinar}
  \seeref{sheng3}
\end{EntryWithPhonetic}

\begin{EntryWithPhonetic}{省悟}{xing3wu4}{9,10}{⽬、⼼}
  \definition{v.}{voltar a si | constatar | ver a verdade | acordar para a realidade}
\end{EntryWithPhonetic}

\begin{EntryWithPhonetic}{醒}{xing3}{16}{⾣}[HSK 4]
  \definition{adj.}{impressionante; notável; admirável; atraente; chamativo}
  \definition{v.}{ficar sóbrio; voltar a si; recuperar a consciência; retornar à normalidade após intoxicação, anestesia ou coma | despertar; estar acordado | ter a mente clara; mover a consciência da confusão para a compreensão | vir a entender; tornar-se ciente de; tomar consciência de}
\end{EntryWithPhonetic}

\begin{EntryWithPhonetic}{兴}{xing4}{6}{⼋}
  \definition{s.}{sentimento ou desejo de fazer algo | interesse em algo | excitação}
  \seeref{xing1}
\end{EntryWithPhonetic}

\begin{EntryWithPhonetic}{兴趣}{xing4 qu4}{6,15}{⼋、⾛}[HSK 4]
  \definition[个,种,点,股,份]{s.}{interesse (desejo de conhecer sobre alguma coisa ou coisa no qual está interessado) | \emph{hobby}}
\end{EntryWithPhonetic}

\begin{EntryWithPhonetic}{姓}{xing4}{8}{⼥}[HSK 2]
  \definition[个]{s.}{sobrenome; nome de família; um caractere que representa um sistema familiar, os chineses colocam o sobrenome em primeiro lugar e o nome em segundo}
  \definition{v.}{ter como sobrenome; tratar um ou mais caracteres como sobrenome}
\end{EntryWithPhonetic}

\begin{EntryWithPhonetic}{姓名}{xing4ming2}{8,6}{⼥、⼝}[HSK 2]
  \definition{s.}{nome; nome completo; sobrenome e nome próprio}
\end{EntryWithPhonetic}

\begin{EntryWithPhonetic}{姓氏}{xing4shi4}{8,4}{⼥、⽒}
  \definition{s.}{sobrenome}
\end{EntryWithPhonetic}

\begin{EntryWithPhonetic}{幸}{xing4}{8}{⼲}
  \definition*{s.}{Sobrenome Xing}
  \definition{adj.}{feliz}
  \definition{adv.}{afortunadamente; felizmente}
  \definition{s.}{felicidade}
  \definition{v.}{alegrar-se; sentir-se feliz e contente | favorecer; patrocinar | vir; chegar; antigamente, referia-se à chegada de um monarca a um determinado lugar}
\end{EntryWithPhonetic}

\begin{EntryWithPhonetic}{幸福}{xing4fu2}{8,13}{⼲、⽰}[HSK 3]
  \definition{adj.}{feliz; a vida, a família e outras circunstâncias deixam as pessoas satisfeitas e felizes}
  \definition{s.}{felicidade; bem estar; sensação ou experiência satisfatória e feliz, etc.}
\end{EntryWithPhonetic}

\begin{EntryWithPhonetic}{幸亏}{xing4kui1}{8,3}{⼲、⼆}
  \definition{adv.}{felizmente}
\end{EntryWithPhonetic}

\begin{EntryWithPhonetic}{幸运}{xing4yun4}{8,7}{⼲、⾡}[HSK 3]
  \definition{adj.}{sortudo; feliz; afortunado}
  \definition[个,点,丝]{s.}{boa sorte; boa fortuna}
\end{EntryWithPhonetic}

\begin{EntryWithPhonetic}{幸运抽奖}{xing4yun4chou1jiang3}{8,7,8,9}{⼲、⾡、⼿、⼤}
  \definition{s.}{loteria | sorteio}
\end{EntryWithPhonetic}

\begin{EntryWithPhonetic}{幸运儿}{xing4yun4'er2}{8,7,2}{⼲、⾡、⼉}
  \definition{s.}{pessoa de sorte}
\end{EntryWithPhonetic}

\begin{EntryWithPhonetic}{性}{xing4}{8}{⼼}[HSK 3]
  \definition[个]{s.}{natureza; caráter; personalidade | propriedade; qualidade; natureza e características das coisas | sexo; gênero | sexualidade; relacionado com a reprodução e a sexualidade | caráter; temperamento}
  \definition{suf.}{indica uma determinada propriedade ou característica de algo; segue um substantivo, verbo ou adjetivo, formando um substantivo abstrato ou um adjetivo que expressa uma propriedade}
\end{EntryWithPhonetic}

\begin{EntryWithPhonetic}{性别}{xing4bie2}{8,7}{⼼、⼑}[HSK 3]
  \definition[种]{s.}{sexo; gênero}
\end{EntryWithPhonetic}

\begin{EntryWithPhonetic}{性格}{xing4ge2}{8,10}{⼼、⽊}[HSK 3]
  \definition[种,个]{s.}{caráter; temperamento; as características psicológicas manifestadas na atitude e no comportamento em relação às pessoas e às coisas}
\end{EntryWithPhonetic}

\begin{EntryWithPhonetic}{性能}{xing4neng2}{8,10}{⼼、⾁}[HSK 5]
  \definition{s.}{natureza; propriedade; desempenho; função (de uma máquina, etc.); grau de conformidade dos produtos mecânicos ou outros produtos industriais com os requisitos de projeto}
\end{EntryWithPhonetic}

\begin{EntryWithPhonetic}{性侵}{xing4qin1}{8,9}{⼼、⼈}
  \definition{s.}{agressão sexual}
  \definition{v.}{agredir sexualmente}
\end{EntryWithPhonetic}

\begin{EntryWithPhonetic}{性生活}{xing4sheng1huo2}{8,5,9}{⼼、⽣、⽔}
  \definition{s.}{vida sexual}
\end{EntryWithPhonetic}

\begin{EntryWithPhonetic}{性质}{xing4zhi4}{8,8}{⼼、⾙}[HSK 4]
  \definition[个,种,类]{s.}{natureza; qualidade; caráter; propriedade; propriedade fundamental que distingue uma coisa de outra}
\end{EntryWithPhonetic}

\begin{EntryWithPhonetic}{凶}{xiong1}{4}{⼐}[HSK 6]
  \definition{adj.}{sinistro; desfavorável; azarado (oposto de "吉") | ruim para as colheitas; improdutivo; ameaçado pela fome | feroz; vicioso; cruel | medroso; terrível}
  \definition[个]{s.}{mal; assassinato; ato de violência; atos de matar ou ferir pessoas | assassino; malfeitor; criminoso; pessoa má; pessoa violenta}
  \definition{v.}{ser feroz; tratar cruelmente}
  \seealsoref{吉}{ji2}
\end{EntryWithPhonetic}

\begin{EntryWithPhonetic}{凶手}{xiong1shou3}{4,4}{⼐、⼿}[HSK 6]
  \definition[名]{s.}{assassino; homicida | agressor (que causou ferimentos a alguém)}
\end{EntryWithPhonetic}

\begin{EntryWithPhonetic}{兄}{xiong1}{5}{⼉}
  \definition{s.}{irmão mais velho | parente mais velho do sexo masculino da mesma geração | uma forma cortês de tratamento entre amigos homens; um título respeitoso para amigos homens}
\end{EntryWithPhonetic}

\begin{EntryWithPhonetic}{兄弟}{xiong1di4}{5,7}{⼉、⼸}[HSK 4]
  \definition{adj.}{fraternal}
  \definition{pron.}{eu, me (termo de uso humilde por homens em discurso público)}
  \definition[个,位]{s.}{irmãos; irmão}
\end{EntryWithPhonetic}

\begin{EntryWithPhonetic}{匈}{xiong1}{6}{⼓}
  \definition*{s.}{Hungria, abreviação de 匈牙利}
  \definition{s.}{peito; seio; tórax}
  \seealsoref{匈牙利}{xiong1ya2li4}
\end{EntryWithPhonetic}

\begin{EntryWithPhonetic}{匈奴}{xiong1nu2}{6,5}{⼓、⼥}
  \definition*{s.}{Xiongnu, um povo da estepe oriental que criou um império que floresceu na época das dinastias Qin e Han}
\end{EntryWithPhonetic}

\begin{EntryWithPhonetic}{匈牙利}{xiong1ya2li4}{6,4,7}{⼓、⽛、⼑}
  \definition*{s.}{Hungria}
\end{EntryWithPhonetic}

\begin{EntryWithPhonetic}{汹}{xiong1}{7}{⽔}
  \definition{adj.}{turbulento; tempestuoso | rugindo; estrondoso | tumultuado}
\end{EntryWithPhonetic}

\begin{EntryWithPhonetic}{汹涌}{xiong1yong3}{7,10}{⽔、⽔}
  \definition{adj.}{turbulento}
  \definition{v.}{aumentar ou emergir violentamente (oceano, rio, lago, etc.)}
\end{EntryWithPhonetic}

\begin{EntryWithPhonetic}{胸}{xiong1}{10}{⾁}
  \definition{s.}{peito | tórax}
\end{EntryWithPhonetic}

\begin{EntryWithPhonetic}{胸部}{xiong1 bu4}{10,10}{⾁、⾢}[HSK 4]
  \definition{s.}{peito; tórax; seios}
\end{EntryWithPhonetic}

\begin{EntryWithPhonetic}{雄}{xiong2}{12}{⾫}
  \definition*{s.}{Sobrenome Xiong}
  \definition{adj.}{masculino | grandioso; imponente; audacioso | poderoso}
  \definition{s.}{uma pessoa ou país com grande poder e influência}
\end{EntryWithPhonetic}

\begin{EntryWithPhonetic}{雄伟}{xiong2wei3}{12,6}{⾫、⼈}[HSK 5]
  \definition{adj.}{magnífico; magnificente | imponente; magnífico}
\end{EntryWithPhonetic}

\begin{EntryWithPhonetic}{熊}{xiong2}{14}{⽕}[HSK 5]
  \definition*{s.}{Sobrenome Xiong}
  \definition[头,只]{s.}{urso}
  \definition{v.}{repreender; censurar}
\end{EntryWithPhonetic}

\begin{EntryWithPhonetic}{熊猫}{xiong2mao1}{14,11}{⽕、⽝}
  \definition[把,只]{s.}{panda gigante}
  \seealsoref{猫熊}{mao1xiong2}
\end{EntryWithPhonetic}

\begin{EntryWithPhonetic}{休}{xiu1}{6}{⼈}
  \definition{adj.}{feliz; alegre; festivo}
  \definition{adv.}{não; indica proibição ou dissuasão, equivalente a 别 ou 不要}
  \definition{s.}{fortuna e infortúnio; bom e mau}
  \definition{v.}{parar; cessar | descansar | abandonar a esposa e mandá-la para casa; antigamente, o marido mandava a esposa de volta para a casa dos pais e rompia o relacionamento conjugal}
  \seealsoref{别}{bie2}
  \seealsoref{不要}{bu2 yao4}
\end{EntryWithPhonetic}

\begin{EntryWithPhonetic}{休兵}{xiu1bing1}{6,7}{⼈、⼋}
  \definition{s.}{armistício; cessar fogo}
  \definition{v.}{cessar fogo}
\end{EntryWithPhonetic}

\begin{EntryWithPhonetic}{休假}{xiu1/jia4}{6,11}{⼈、⼈}[HSK 2]
  \definition{v.+compl.}{ter um feriado; tirar férias; sair de férias}
\end{EntryWithPhonetic}

\begin{EntryWithPhonetic}{休憩}{xiu1qi4}{6,16}{⼈、⼼}
  \definition{v.}{relaxar | descansar | dar um tempo}
\end{EntryWithPhonetic}

\begin{EntryWithPhonetic}{休息室}{xiu1xi1shi4}{6,10,9}{⼈、⼼、⼧}
  \definition{s.}{saguão | salão}
\end{EntryWithPhonetic}

\begin{EntryWithPhonetic}{休息}{xiu1xi5}{6,10}{⼈、⼼}[HSK 1]
  \definition{s.}{descanço}
  \definition{v.}{descansar; descansar um pouco; fazer uma pausa; interromper o trabalho, os estudos ou as atividades para recuperar as energias | dormir}
\end{EntryWithPhonetic}

\begin{EntryWithPhonetic}{休闲}{xiu1xian2}{6,7}{⼈、⾨}[HSK 5]
  \definition{s.}{ócio; lazer; tempo livre}
  \definition{v.}{desfrutar do lazer; sair de férias; aproveitar o tempo livre; parar de trabalhar ou estudar, estar em um estado de lazer e descontração | ficar ocioso}
\end{EntryWithPhonetic}

\begin{EntryWithPhonetic}{休整}{xiu1zheng3}{6,16}{⼈、⽁}
  \definition{v.}{(militar) descansar e reorganizar}
\end{EntryWithPhonetic}

\begin{EntryWithPhonetic}{修}{xiu1}{9}{⼈}[HSK 3]
  \definition*{s.}{Sobrenome Xiu}
  \definition{adj.}{comprido; alto e esbelto}
  \definition{s.}{revisionismo}
  \definition{v.}{embelezar; decorar | consertar; reparar; reformar | escrever; redigir; compilar | estudar; cultivar; aprender e praticar para aperfeiçoar ou melhorar (o caráter e o conhecimento) | construir; edificar | cortar ou aparar, para deixar bonito e arrumado | dedicar-se à prática da religião}
\end{EntryWithPhonetic}

\begin{EntryWithPhonetic}{修车}{xiu1 che1}{9,4}{⼈、⾞}[HSK 6]
  \definition{v.}{consertar uma bicicleta (carro etc.)}[我打算明天去修车。===Pretendo consertar meu carro amanhã.]
\end{EntryWithPhonetic}

\begin{EntryWithPhonetic}{修复}{xiu1fu4}{9,9}{⼈、⼢}[HSK 5]
  \definition{v.}{reparar; restaurar; renovar | reparar; melhorar e restaurar (o relacionamento)}
\end{EntryWithPhonetic}

\begin{EntryWithPhonetic}{修改}{xiu1gai3}{9,7}{⼈、⽁}[HSK 3]
  \definition{v.}{revisar; retocar; corrigir erros e falhas em artigos, planos, etc.}
\end{EntryWithPhonetic}

\begin{EntryWithPhonetic}{修规}{xiu1gui1}{9,8}{⼈、⾒}
  \definition{s.}{plano de construção}
\end{EntryWithPhonetic}

\begin{EntryWithPhonetic}{修建}{xiu1jian4}{9,8}{⼈、⼵}[HSK 5]
  \definition{v.}{construir; erguer; animar; edificar; construir com tijolos, telhas, madeira, cimento, areia, etc.}
\end{EntryWithPhonetic}

\begin{EntryWithPhonetic}{修理}{xiu1li3}{9,11}{⼈、⽟}[HSK 4]
  \definition{v.}{consertar; reparar; restaurar algo danificado à sua forma ou função original | aparar; podar; cortar com tesouras e outras ferramentas para deixar árvores, flores, cabelos, etc. arrumados | culpar; punir; criticar ou punir uma pessoa para mostrar que ela está errada}
\end{EntryWithPhonetic}

\begin{EntryWithPhonetic}{修养}{xiu1yang3}{9,9}{⼈、⼋}[HSK 5]
  \definition[种]{s.}{treinamento; domínio; realização; refere-se a um determinado nível em termos de teoria, conhecimento, arte, pensamento, etc. | auto-cultivo; refere-se à atitude e ao comportamento cultivados ao longo do tempo, em conformidade com as exigências sociais}
\end{EntryWithPhonetic}

\begin{EntryWithPhonetic}{宿}{xiu3}{11}{⼧}
  \definition{s.}{usado para calcular a noite}[谈了半宿。===Conversamos por metade da noite.]
  \seeref{su4}
  \seeref{xiu4}
\end{EntryWithPhonetic}

\begin{EntryWithPhonetic}{绣}{xiu4}{10}{⽷}
  \definition{s.}{bordado}
  \definition{v.}{bordar}
\end{EntryWithPhonetic}

\begin{EntryWithPhonetic}{臭}{xiu4}{10}{⾃}
  \definition{s.}{odor; cheiro}
  \definition{v.}{cheirar; farejar; o mesmo que 嗅}
  \seeref{chou4}
  \seealsoref{嗅}{xiu4}
\end{EntryWithPhonetic}

\begin{EntryWithPhonetic}{袖}{xiu4}{10}{⾐}
  \definition{s.}{manga (de camisa, de camiseta, etc.)}
\end{EntryWithPhonetic}

\begin{EntryWithPhonetic}{袖珍}{xiu4 zhen1}{10,9}{⾐、⽟}[HSK 6]
  \definition{adj.}{do tamanho do bolso; de bolso (livro, agenda, etc.)}
\end{EntryWithPhonetic}

\begin{EntryWithPhonetic}{宿}{xiu4}{11}{⼧}
  \definition{s.}{(astronomia) um termo antigo para constelação}
  \seeref{su4}
  \seeref{xiu3}
\end{EntryWithPhonetic}

\begin{EntryWithPhonetic}{嗅}{xiu4}{13}{⼝}
  \definition{v.}{cheirar; farejar; identificar odores pelo nariz}
\end{EntryWithPhonetic}

\begin{EntryWithPhonetic}{虚}{xu1}{11}{⾌}
  \definition*{s.}{Xu, a décima primeira das vinte e oito constelações em que a esfera celeste foi dividida, consistindo de duas estrelas em linha reta, uma em Aquário e a outra em Equuleus | Xu, uma das mansões lunares | Sobrenome Xu}
  \definition{adj.}{vazio; oco; desocupado | desconfiado; tímido | falso; nominal (oposto a 实) | humilde; modesto | fraco; com saúde debilitada | (física) virtual}
  \definition{adv.}{em vão}
  \definition{s.}{vazio; nulidade; anulação | resumo; teoria; princípios orientadores; ideologia política e outros aspectos}
  \definition{v.}{reservar espaço}
  \seealsoref{实}{shi2}
\end{EntryWithPhonetic}

\begin{EntryWithPhonetic}{虚伪}{xu1wei3}{11,6}{⾌、⼈}
  \definition{adj.}{falso | hipócrita | artificial}
\end{EntryWithPhonetic}

\begin{EntryWithPhonetic}{虚心}{xu1xin1}{11,4}{⾌、⼼}[HSK 5]
  \definition{adj.}{modesto; humilde; de mente aberta; não ser presunçoso, ser capaz de aceitar as opiniões dos outros}
\end{EntryWithPhonetic}

\begin{EntryWithPhonetic}{需}{xu1}{14}{⾬}
  \definition*{s.}{Sobrenome Xu}
  \definition{s.}{necessidades; bens de primeira necessidade}
  \definition{v.}{precisar; querer; exigir}
\end{EntryWithPhonetic}

\begin{EntryWithPhonetic}{需求}{xu1qiu2}{14,7}{⾬、⽔}[HSK 3]
  \definition[种]{s.}{necessidades; demanda; exigência; solicitações decorrentes de necessidades}
\end{EntryWithPhonetic}

\begin{EntryWithPhonetic}{需要}{xu1yao4}{14,9}{⾬、⾑}[HSK 3]
  \definition[种]{s.}{necessidade; desejo ou exigência em relação a algo}
  \definition{v.}{precisar; querer; exigir; demandar; solicitar}
\end{EntryWithPhonetic}

\begin{EntryWithPhonetic}{许}{xu3}{6}{⾔}
  \definition*{s.}{Xu, um estado da Dinastia Zhou | Sobrenome Xu}
  \definition{adv.}{um pouco;  talvez; expressa especulação ou estimativa, equivalente a 或者 ou 可能}
  \definition{part.}{cerca de; aproximadamente; usado depois de certos numerais, frases de quantidade ou 些 ou 少 para indicar um número próximo a um certo número}
  \definition{pron.}{muitos; um monte de}
  \definition{v.}{elogiar; aprovar | prometer; prometer dar antecipadamente; dedicar | permitir; concordar; aprovar | (uma menina) estar prometida a; refere-se especificamente ao noivado}
  \seealsoref{或者}{huo4zhe3}
  \seealsoref{可能}{ke3neng2}
  \seealsoref{少}{shao3}
  \seealsoref{些}{xie1}
\end{EntryWithPhonetic}

\begin{EntryWithPhonetic}{许多}{xu3duo1}{6,6}{⾔、⼣}[HSK 2]
  \definition{num.}{muitos; muito; numerosos; uma grande quantidade de}
\end{EntryWithPhonetic}

\begin{EntryWithPhonetic}{许可}{xu3ke3}{6,5}{⾔、⼝}[HSK 5]
  \definition{v.}{permitir; autorizar}
\end{EntryWithPhonetic}

\begin{EntryWithPhonetic}{恤}{xu4}{9}{⼼}
  \definition{v.}{ter pena; simpatizar | dar alívio; compensar}
\end{EntryWithPhonetic}

\begin{EntryWithPhonetic}{畜}{xu4}{10}{⽥}
  \definition{v.}{criar (animais domésticos)}
  \seeref{chu4}
\end{EntryWithPhonetic}

\begin{EntryWithPhonetic}{宣}{xuan1}{9}{⼧}
  \definition*{s.}{Sobrenome Xuan}
  \definition{v.}{declarar; proclamar; anunciar; falar publicamente | drenar (líquidos)}
\end{EntryWithPhonetic}

\begin{EntryWithPhonetic}{宣布}{xuan1bu4}{9,5}{⼧、⼱}[HSK 3]
  \definition{v.}{declarar; proclamar; pronunciar; anunciar; informar oficialmente a todos sobre as últimas decisões e situações}
\end{EntryWithPhonetic}

\begin{EntryWithPhonetic}{宣传}{xuan1chuan2}{9,6}{⼧、⼈}[HSK 3]
  \definition[个]{v.}{propagar; divulgar; fazer propaganda; explicar e esclarecer às pessoas, para que elas acreditem e sigam as ações}
\end{EntryWithPhonetic}

\begin{EntryWithPhonetic}{宣扬}{xuan1yang2}{9,6}{⼧、⼿}
  \definition{v.}{divulgar | anunciar | espalhar por toda parte}
\end{EntryWithPhonetic}

\begin{EntryWithPhonetic}{玄}{xuan2}{5}{⽞}[Kangxi 95]
  \definition*{s.}{Sobrenome Xuan}
  \definition{adj.}{preto; escuro | profundo; abstruso; escondido | não confiável; irrealista; não confiável}
\end{EntryWithPhonetic}

\begin{EntryWithPhonetic}{玄学}{xuan2xue2}{5,8}{⽞、⼦}
  \definition{s.}{Escola Philosófica Wei e Jin amalgamando os ideais daoísta e confucionistas | tradução da metafísica (形而上学) | Datado: metafísica}
  \seealsoref{形而上学}{xing2'er2shang4xue2}
\end{EntryWithPhonetic}

\begin{EntryWithPhonetic}{悬}{xuan2}{11}{⼼}[HSK 6]
  \definition{adj.}{pendente; não resolvido; sem nenhum resultado | distante; a distância é grande; a diferença é grande | (dialeto) perigoso}
  \definition{v.}{pendurar; suspender | levantar; elevar | sentir-se ansioso; ser solícito | imaginar}
\end{EntryWithPhonetic}

\begin{EntryWithPhonetic}{悬挂}{xuan2gua4}{11,9}{⼼、⼿}
  \definition{v.}{pendurar; pender; suspender; prender um objeto em um ou mais pontos em algum lugar com a ajuda de uma corda, gancho, prego, etc.}
\end{EntryWithPhonetic}

\begin{EntryWithPhonetic}{悬崖}{xuan2ya2}{11,11}{⼼、⼭}
  \definition{s.}{precipício | penhasco}
\end{EntryWithPhonetic}

\begin{EntryWithPhonetic}{旋}{xuan2}{11}{⽅}
  \definition*{s.}{Sobrenome Xuan}
  \definition{adv.}{em breve; rapidamente}
  \definition{s.}{redemoinho; turbilhão; vórtice}
  \definition{v.}{girar; circular; rodar | retornar; voltar}
\end{EntryWithPhonetic}

\begin{EntryWithPhonetic}{旋转}{xuan2zhuan3}{11,8}{⽅、⾞}[HSK 6]
  \definition{v.}{girar; rodar; revolver; rodopiar; o movimento circular de um objeto em torno de um ponto ou eixo}
\end{EntryWithPhonetic}

\begin{EntryWithPhonetic}{选}{xuan3}{9}{⾡}[HSK 2]
  \definition{s.}{pessoa ou coisa selecionada | seleções; antologia; trabalhos selecionados e compilados}
  \definition{v.}{selecionar; escolher | eleger}
\end{EntryWithPhonetic}

\begin{EntryWithPhonetic}{选拔}{xuan3ba2}{9,8}{⾡、⼿}[HSK 6]
  \definition{v.}{selecionar; escolher}
\end{EntryWithPhonetic}

\begin{EntryWithPhonetic}{选举}{xuan3ju3}{9,9}{⾡、⼂}[HSK 6]
  \definition[次,个]{s.}{eleição; as eleições são o processo pelo qual os cidadãos escolhem os seus representantes ou líderes através do voto}
  \definition{v.}{votar; eleger; eleger representantes ou responsáveis ​​votando ou levantando as mãos}
\end{EntryWithPhonetic}

\begin{EntryWithPhonetic}{选手}{xuan3shou3}{9,4}{⾡、⼿}[HSK 3]
  \definition[位,名,个,些]{s.}{jogador; (selecionado) competidor; atleta selecionado para uma competição esportiva; participantes selecionados entre um grande número de candidatos}
\end{EntryWithPhonetic}

\begin{EntryWithPhonetic}{选修}{xuan3 xiu1}{9,9}{⾡、⼈}[HSK 5]
  \definition{v.}{fazer como disciplina eletiva; escolher entre uma seleção de cursos disponíveis}
\end{EntryWithPhonetic}

\begin{EntryWithPhonetic}{选择}{xuan3ze2}{9,8}{⾡、⼿}[HSK 4]
  \definition[个,种,次]{s.}{escolha; opção; resultado da escolha; possibilidade de escolha}
  \definition{v.}{selecionar; escolher}
\end{EntryWithPhonetic}

\begin{EntryWithPhonetic}{薛}{xue1}{16}{⾋}
  \definition*{s.}{Estado vassalo durante a Dinastia Zhou (1046-256 a.C.) | Sobrenome Xue}
  \definition{s.}{erva semelhante ao absinto (clássico)}
\end{EntryWithPhonetic}

\begin{EntryWithPhonetic}{薛稷}{xue1 ji4}{16,15}{⾋、⽲}
  \definition*{s.}{Xue Ji (649-713), um dos quatro grandes calígrafos do início da dinastia Tang, 唐初四大家}
  \seealsoref{唐初四大家}{tang2 chu1 si4 da4jia1}
\end{EntryWithPhonetic}

\begin{EntryWithPhonetic}{学}{xue2}{8}{⼦}[HSK 1]
  \definition[所]{s.}{aprendizagem; conhecimento; sabedoria; erudição | objeto de estudo; ramo do conhecimento | escola; faculdade | teoria; doutrina}
  \definition{v.}{estudar; aprender | imitar; copiar}
\end{EntryWithPhonetic}

\begin{EntryWithPhonetic}{学费}{xue2 fei4}{8,9}{⼦、⾙}[HSK 3]
  \definition[笔]{s.}{mensalidade (taxa); prêmio; taxas que os alunos devem pagar para estudar na escola, conforme estabelecido pela escola | preço pelo que se aprendeu ao custo do próprio bolso; a metáfora do preço a pagar para obter uma determinada experiência | custo; preço; todas as despesas necessárias durante o período de estudos do aluno}
\end{EntryWithPhonetic}

\begin{EntryWithPhonetic}{学分}{xue2fen1}{8,4}{⼦、⼑}[HSK 4]
  \definition{s.}{créditos de um curso; uma unidade de medida do peso e do tempo do curso no ensino superior; cada curso vale um crédito para uma aula por semana durante um semestre; alunos devem concluir o número necessário de créditos para se formar}
\end{EntryWithPhonetic}

\begin{EntryWithPhonetic}{学好}{xue2hao3}{8,6}{⼦、⼥}
  \definition{v.}{seguir bons exemplos | aprender bem}
\end{EntryWithPhonetic}

\begin{EntryWithPhonetic}{学会}{xue2 hui4}{8,6}{⼦、⼈}[HSK 6]
  \definition[个]{s.}{sociedade; instituto; sociedade científica; um grupo acadêmico composto por pessoas que estudam um determinado assunto, como a Sociedade de Física, a Sociedade de Biologia, etc.}
  \definition{v.}{aprender; dominar; aprender e aplicar}
\end{EntryWithPhonetic}

\begin{EntryWithPhonetic}{学科}{xue2 ke1}{8,9}{⼦、⽲}[HSK 5]
  \definition[门,级]{s.}{ramo do aprendizado; disciplina | disciplina escolar; curso de estudo | cursos teóricos oferecidos em treinamento militar ou físico (oposto a 术科)  | disciplina acadêmica | curso | assunto; tema}
  \seealsoref{术科}{shu4ke1}
\end{EntryWithPhonetic}

\begin{EntryWithPhonetic}{学年}{xue2 nian2}{8,6}{⼦、⼲}[HSK 4]
  \definition{s.}{ano letivo; ano acadêmico}
\end{EntryWithPhonetic}

\begin{EntryWithPhonetic}{学期}{xue2qi1}{8,12}{⼦、⽉}[HSK 2]
  \definition[个,段]{s.}{semestre; período escolar; um ano acadêmico é dividido em dois semestres, um semestre do início do outono até as férias de inverno e um semestre do início da primavera até as férias de verão}
\end{EntryWithPhonetic}

\begin{EntryWithPhonetic}{学生}{xue2sheng5}{8,5}{⼦、⽣}[HSK 1]
  \definition{s.}{aluno; estudante; pupilo}
\end{EntryWithPhonetic}

\begin{EntryWithPhonetic}{学生证}{xue2sheng5zheng4}{8,5,7}{⼦、⽣、⾔}
  \definition{s.}{cartão de identidade de estudante}
\end{EntryWithPhonetic}

\begin{EntryWithPhonetic}{学时}{xue2 shi2}{8,7}{⼦、⽇}[HSK 4]
  \definition{s.}{hora-aula; hora de aula; período}
\end{EntryWithPhonetic}

\begin{EntryWithPhonetic}{学术}{xue2shu4}{8,5}{⼦、⽊}[HSK 4]
  \definition[种]{s.}{aprendizagem; aprendizado; ciências; aprendizado sistemático e especializado}
\end{EntryWithPhonetic}

\begin{EntryWithPhonetic}{学位}{xue2wei4}{8,7}{⼦、⼈}[HSK 5]
  \definition[个]{s.}{grau; grau acadêmico; título concedido com base no nível acadêmico profissional, como doutorado, mestrado, etc.}
\end{EntryWithPhonetic}

\begin{EntryWithPhonetic}{学问}{xue2wen4}{8,6}{⼦、⾨}[HSK 4]
  \definition[门,种,个,项]{s.}{aprendizado, conhecimento, erudição; a compreensão correta do mundo objetivo que alguém tem | conhecimento; aprendizado sistemático; conhecimento sistemático sobre algo ou uma ciência que pode ser aprendido em um livro ou em uma experiência prática}
\end{EntryWithPhonetic}

\begin{EntryWithPhonetic}{学习}{xue2xi2}{8,3}{⼦、⼄}[HSK 1]
  \definition{s.}{estudo}
  \definition{v.}{estudar; aprender; adquirir conhecimentos ou habilidades através da leitura, da audição, da pesquisa e da prática}
\end{EntryWithPhonetic}

\begin{EntryWithPhonetic}{学校}{xue2xiao4}{8,10}{⼦、⽊}[HSK 1]
  \definition[所,个]{s.}{escola; instituição de ensino}
\end{EntryWithPhonetic}

\begin{EntryWithPhonetic}{学员}{xue2 yuan2}{8,7}{⼦、⼝}[HSK 6]
  \definition[位,名,批,个]{s.}{estudante; estagiário; geralmente se refere a pessoas que estudam em escolas ou cursos de treinamento diferentes de faculdades, escolas de ensino médio e escolas primárias}
\end{EntryWithPhonetic}

\begin{EntryWithPhonetic}{学院}{xue2yuan4}{8,9}{⼦、⾩}[HSK 1]
  \definition[个,所]{s.}{academia; instituto; um tipo de instituição de ensino superior que se concentra em uma determinada área de especialização, como faculdades de engenharia, faculdades de música, faculdades de educação, etc.}
\end{EntryWithPhonetic}

\begin{EntryWithPhonetic}{学者}{xue2 zhe3}{8,8}{⼦、⽼}[HSK 5]
  \definition[位]{s.}{erudito; homem culto; pessoas que fazem pesquisas acadêmicas geralmente se referem àquelas que alcançaram certo sucesso acadêmico}
\end{EntryWithPhonetic}

\begin{EntryWithPhonetic}{雪}{xue3}{11}{⾬}[HSK 2]
  \definition*{s.}{Sobrenome Xue}
  \definition[场,层]{s.}{neve | algo parecido com neve}
  \definition{v.}{limpar; enxugar; remover}
\end{EntryWithPhonetic}

\begin{EntryWithPhonetic}{雪板}{xue3ban3}{11,8}{⾬、⽊}
  \definition{s.}{prancha de \emph{snowboard}}
  \definition{v.}{praticar \textit{snowboard}}
\end{EntryWithPhonetic}

\begin{EntryWithPhonetic}{雪糕}{xue3gao1}{11,16}{⾬、⽶}
  \definition{s.}{picolé}
\end{EntryWithPhonetic}

\begin{EntryWithPhonetic}{雪花}{xue3hua1}{11,7}{⾬、⾋}
  \definition{s.}{floco de neve}
\end{EntryWithPhonetic}

\begin{EntryWithPhonetic}{雪葩}{xue3pa1}{11,12}{⾬、⾋}
  \definition{s.}{sorvete}
\end{EntryWithPhonetic}

\begin{EntryWithPhonetic}{雪人}{xue3ren2}{11,2}{⾬、⼈}
  \definition{s.}{boneco de neve | \emph{Yeti}}
\end{EntryWithPhonetic}

\begin{EntryWithPhonetic}{雪山}{xue3shan1}{11,3}{⾬、⼭}
  \definition{s.}{montanha coberta de neve}
\end{EntryWithPhonetic}

\begin{EntryWithPhonetic}{雪鞋}{xue3xie2}{11,15}{⾬、⾰}
  \definition[双]{s.}{sapatos de neve}
\end{EntryWithPhonetic}

\begin{EntryWithPhonetic}{血}{xue4}{6}{⾎}[HSK 3][Kangxi 143]
  \definition[滴,袋,口,毫升]{s.}{sangue | parente consanguíneo; com laços de parentesco | pessoa ativa e animada; metáfora para uma personalidade ou espírito forte e sincero | medicina tradicional chinesa refere-se à menstruação}
  \seeref{xie3}
\end{EntryWithPhonetic}

\begin{EntryWithPhonetic}{血管}{xue4 guan3}{6,14}{⾎、⽵}[HSK 6]
  \definition[根,条,种]{s.}{vaso; vaso sanguíneo; os canais tubulares pelos quais o sangue circula são divididos em três tipos: artérias, veias e capilares}
\end{EntryWithPhonetic}

\begin{EntryWithPhonetic}{血汗}{xue4han4}{6,6}{⾎、⽔}
  \definition{s.}{(fig.) suor e labuta, trabalho duro}
\end{EntryWithPhonetic}

\begin{EntryWithPhonetic}{血液}{xue4 ye4}{6,11}{⾎、⽔}[HSK 6]
  \definition[毫升]{s.}{sangue | linha de vida; sangue vital; uma metáfora para o importante componente ou força que mantém a vitalidade coletiva}
\end{EntryWithPhonetic}

\begin{EntryWithPhonetic}{熏}{xun1}{14}{⽕}
  \definition{v.}{expor à fumaça ou vapores; fumigar | tratar (carne, peixe, etc.) com fumaça; defumar | tornar perfumado com incenso, etc. | sufocar (asfixia e envenenamento por gás)}
\end{EntryWithPhonetic}

\begin{EntryWithPhonetic}{熏香}{xun1xiang1}{14,9}{⽕、⾹}
  \definition{s.}{incenso}
\end{EntryWithPhonetic}

\begin{EntryWithPhonetic}{寻}{xun2}{6}{⼨}
  \definition*{s.}{Sobrenome Xun}
  \definition{clas.}{uma unidade antiga de comprimento, igual a 8尺}
  \definition{v.}{procurar; pesquisar; buscar}
  \seealsoref{尺}{chi3}
\end{EntryWithPhonetic}

\begin{EntryWithPhonetic}{寻求}{xun2 qiu2}{6,7}{⼨、⽔}[HSK 5]
  \definition{v.}{procurar; perseguir; explorar; ir em busca de}
\end{EntryWithPhonetic}

\begin{EntryWithPhonetic}{寻找}{xun2zhao3}{6,7}{⼨、⼿}[HSK 4]
  \definition{v.}{buscar; procurar; pesquisar; encontrar, que pode ser usado tanto para coisas concretas quanto para coisas abstratas}
\end{EntryWithPhonetic}

\begin{EntryWithPhonetic}{巡}{xun2}{6}{⾡}
  \definition{clas.}{rodada de bebidas | usado para servir vinho a todos}
  \definition{v.}{patrulhar; fazer rondas; fazer uma excursão de inspeção}
\end{EntryWithPhonetic}

\begin{EntryWithPhonetic}{巡逻}{xun2luo2}{6,11}{⾡、⾡}
  \definition{s.}{patrulha}
  \definition{v.}{patrulhar (polícia, exército ou marinha)}
\end{EntryWithPhonetic}

\begin{EntryWithPhonetic}{询}{xun2}{8}{⾔}
  \definition{v.}{perguntar; indagar; reunir informações | consultar; buscar conselho}
\end{EntryWithPhonetic}

\begin{EntryWithPhonetic}{询问}{xun2wen4}{8,6}{⾔、⾨}[HSK 5]
  \definition{v.}{indagar; perguntar sobre; pedir conselho}
\end{EntryWithPhonetic}

\begin{EntryWithPhonetic}{循}{xun2}{12}{⼻}
  \definition{v.}{seguir; cumprir; cumprir com}
\end{EntryWithPhonetic}

\begin{EntryWithPhonetic}{循环}{xun2huan2}{12,8}{⼻、⽟}[HSK 6]
  \definition{s.}{ciclo; circulação}
  \definition{v.}{circular; as coisas se movem ou mudam em um ciclo}
\end{EntryWithPhonetic}

\begin{EntryWithPhonetic}{训}{xun4}{5}{⾔}
  \definition{s.}{instrução; ensinamento; ensino | padrão; modelo; exemplo; regra; diretriz | explicação ou interpretação crítica de um texto | treinamento; exercício}
  \definition{v.}{instruir; admoestar; dar uma palestra a alguém; ensinar | explicar; instruir; explicação do significado da palavra | treinar}
\end{EntryWithPhonetic}

\begin{EntryWithPhonetic}{训诂}{xun4gu3}{5,7}{⾔、⾔}
  \definition{s.}{estudos exegéticos (de textos antigos); exegese}
  \definition{v.}{explicação de palavras e frases em livros antigos | interpretar e elaborar glossários e comentários sobre textos clássicos}
\end{EntryWithPhonetic}

\begin{EntryWithPhonetic}{训练}{xun4lian4}{5,8}{⾔、⽷}[HSK 3]
  \definition{v.}{treinar; exercitar; planejar e executar de forma sistemática o desenvolvimento de habilidades ou competências específicas}
\end{EntryWithPhonetic}

\begin{EntryWithPhonetic}{迅}{xun4}{6}{⾡}
  \definition{adj.}{rápido; veloz}
  \definition{adv.}{rapidamente; velozmente}
\end{EntryWithPhonetic}

\begin{EntryWithPhonetic}{迅速}{xun4su4}{6,10}{⾡、⾡}[HSK 4]
  \definition{adv.}{rapidamente; velozmente; prontamente}
\end{EntryWithPhonetic}

%%%%% EOF %%%%%


%%%
%%% Y
%%%
\section*{Y}
\addcontentsline{toc}{section}{Y}
\begin{multicols*}{2}

\begin{verbete}[鸭]{ya1}
\significado{ya1}{n.}{
    pato
}
\end{verbete}

\begin{verbete}[压岁钱]{ya1sui4qian2}
\significado{ya1sui4qian2}{n.}{
    dinheiro da sorte|
    dinheiro dado às crianças como presente no Ano Novo Chinês
}
\end{verbete}

\begin{verbete}[牙]{ya2}
\significado{ya2}{n.}{
    dente
}
\end{verbete}

\begin{verbete}[牙齿]{ya2chi3}
\significado{ya2chi3}{n.}{
    dente
}
\end{verbete}

\begin{verbete}[亚洲]{Ya4zhou1}
\significado{Ya4zhou1}{n.}{
    Ásia
}
\end{verbete}

\begin{verbete}[颜色]{yan2se4}
\significado{yan2se4}{n.}{
    cor
}
\end{verbete}

\begin{verbete}[眼镜]{yan3jing4}
\significado{yan3jing4}{n.}{
    óculos|
    \pc{副}
}
\end{verbete}

\begin{verbete}[眼睛]{yan3jing0}
\significado{yan3jing0}{n.}{
    olho(s)
}
\end{verbete}

\begin{verbete}[养]{yang3}
\significado{yang3}{v.}{
    criar (animais), plantar (flores), etc
}
\end{verbete}

\begin{verbete}[样子]{yang4zi0}
\significado{yang4zi0}{n.}{
    aparência;forma
}
\end{verbete}

\begin{verbete}[腰]{yao1}
\significado{yao1}{n.}{
    cintura
}
\end{verbete}

\begin{verbete}[药]{yao4}
\significado{yao4}{n.}{
    medicamento; remédio
}
\end{verbete}

\begin{verbete}[要]{yao4}
\significado{yao4}{v./v.o.}{
    querer; precisar
}
\end{verbete}

\begin{verbete}[要是]{yao4shi0}
\significado{yao4shi0}{conj.}{
    se
}
\end{verbete}

\begin{verbete}[要是······的话]{yao4shi0 ...\  de0hua0}
\significado{yao4shi0 ...\  de0hua0}{conj.}{
    se ... no caso de
}
\end{verbete}

\begin{verbete}[爷爷]{ye2ye0}
\significado{ye2ye0}{n.}{
    avô (paterno)
}
\end{verbete}

\begin{verbete}[也]{ye3}
\significado{ye3}{adv.}{
    também
}
\end{verbete}

\begin{verbete}[夜里]{ye4li0}
\significado{ye4li0}{p.t.}{
    noite
}
\end{verbete}

\begin{verbete}[一]{yi1}
\significado{yi1}{num.}{
    1|
    um, uma (quando usado sozinho)
}
\significado{yi2}{num.}{
    1|
    um, uma (antes de quarto tom)
}
\significado{yi4}{num.}{
    1|
    um, uma
}
\end{verbete}

\begin{verbete}[一]{yi2}
\significado{yi2}{num.}{
    1|
    um, uma (antes de quarto tom)
}
\significado{yi1}{num.}{
    1|
    um, uma (quando usado sozinho)
}
\significado{yi4}{num.}{
    1|
    um, uma
}
\end{verbete}

\begin{verbete}[一半]{yi2ban4}
\significado{yi2ban4}{adj.}{
    metade
}
\end{verbete}

\begin{verbete}[一定]{yi2ding4}
\significado{yi2ding4}{adv.}{
    certamente; definitivamente
}
\end{verbete}

\begin{verbete}[一共]{yi2gong4}
\significado{yi2gong4}{adv.}{
    tudo; no local
}
\end{verbete}

\begin{verbete}[一下]{yi2xia4}
\significado{yi2xia4}{adv.}{
    em um curto tempo; rapidamente
}
\end{verbete}

\begin{verbete}[一样]{yi2yang4}
\significado{yi2yang4}{adj.}{
    igual; mesmo, mesma
}
\end{verbete}

\begin{verbete}[一]{yi4}
\significado{yi4}{num.}{
    1|
    um, uma
}
\significado{yi2}{num.}{
    1|
    um, uma (antes de quarto tom)
}
\significado{yi1}{num.}{
    1|
    um, uma (quando usado sozinho)
}
\end{verbete}

\begin{verbete}[一般]{yi4ban1}
\significado{yi4ban1}{adj.}{
    geral; comum; normal
}
\significado{yi4ban1}{adv.}{
    normalmente
}
\end{verbete}

\begin{verbete}[一点儿]{yi4dianr3}
\significado{yi4dianr3}{adv.}{
    um pouco
}
\end{verbete}

\begin{verbete}[一会儿]{yi4huir4}
\significado{yi4huir4}{adv.}{
    daqui a pouco tempo; pouco tempo
}
\end{verbete}

\begin{verbete}[一起]{yi4qi3}
\significado{yi4qi3}{adv.}{
    juntamente; em conjunto
}
\end{verbete}

\begin{verbete}[一直]{yi4zhi2}
\significado{yi4zhi2}{adv.}{
    diretamente; sempre
}
\end{verbete}

\begin{verbete}[一些]{yi4xie1}
\significado{yi4xie1}{pron.}{
    uns, umas|
    alguns, algumas
}
\end{verbete}

\begin{verbete}[衣服]{yi1fu0}
\significado{yi1fu0}{n.}{
    roupa, vestuário|
    \pc{件}
}
\end{verbete}

\begin{verbete}[医生]{yi1sheng1}
\significado{yi1sheng1}{n.}{
    médico; clínico
}
\end{verbete}

\begin{verbete}[医院]{yi1yuan0}
\significado{yi1yuan0}{n.}{
    hospital
}
\end{verbete}

\begin{verbete}[颐和园]{yi2he2yuan2}
\significado{yi2he2yuan2}{n.}{
    Palácio de Verão
}
\end{verbete}

\begin{verbete}[遗憾]{yi2han4}
\significado{yi2han4}{v.}{
    ter pena de
}
\end{verbete}

\begin{verbete}[以后]{yi3hou4}
\significado{yi3hou4}{n.}{
    depois de; depois; após
}
\end{verbete}

\begin{verbete}[以前]{yi3qian2}
\significado{yi3qian2}{p.t.}{
    antes de; antes
}
\end{verbete}

\begin{verbete}[已经]{yi3jing1}
\significado{yi3jing1}{adv.}{
    já
}
\end{verbete}

\begin{verbete}[亿]{yi4}
\significado{yi4}{num.}{
    100.000.000|
    cem milhões
}
\end{verbete}

\begin{verbete}[意思]{yi4si0}
\significado{yi4si0}{n.}{
    interesse
}
\end{verbete}

\begin{verbete}[阴天]{yin1tian1}
\significado{yin1tian1}{adj.}{
    céu muito nublado; céu cinzento
}
\end{verbete}

\begin{verbete}[因为]{yin1wei4}
\significado{yin1wei4}{conj.}{
    porque
}
\end{verbete}

\begin{verbete}[音乐]{yin1yue4}
\significado{yin1yue4}{n.}{
    música
}
\end{verbete}

\begin{verbete}[银行]{yin2hang2}
\significado{yin2hang2}{n.}{
    banco; agência bancária
}
\end{verbete}

\begin{verbete}[饮料]{yin3liao4}
\significado{yin3liao4}{n.}{
    bebida
}
\end{verbete}

\begin{verbete}[应该]{ying1gai1}
\significado{ying1gai1}{v.}{
    dever; ter de
}
\end{verbete}

\begin{verbete}[英国]{Ying1guo2}
\significado{Ying1guo2}{n.}{
    Reino Unido
}
\end{verbete}

\begin{verbete}[英语]{ying1yu3}
\significado{ying1yu3}{n.}{
    inglês, língua inglesa
}
\end{verbete}

\begin{verbete}[英文]{ying1wen2}
\significado{ying1wen2}{n.}{
    inglês, língua inglesa
}
\end{verbete}

\begin{verbete}[优美]{you1mei3}
\significado{you2jian4}{n.}{
    correspondência
}
\end{verbete}

\begin{verbete}[邮件]{you2jian4}
\significado{you2jian4}{n.}{
    correspondência
}
\end{verbete}

\begin{verbete}[邮局]{you2ju4}
\significado{you2ju4}{n.}{
    correio; agência dos correios
}
\end{verbete}

\begin{verbete}[游]{you2}
\significado{you2}{v.}{
    nadar
}
\end{verbete}

\begin{verbete}[游泳]{you2yong3}
\significado{you2yong3}{v.+compl.}{
    nadar
}
\end{verbete}

\begin{verbete}[游泳池]{you2yong3chi2}
\significado{you2yong3chi2}{n.}{
    piscina
}
\end{verbete}

\begin{verbete}[有]{you3}
\significado{you3}{v.}{
    ter; haver
}
\end{verbete}

\begin{verbete}[有的]{you3de0}
\significado{you3de0}{pron.}{
    algum, alguma, alguns, algumas
}
\end{verbete}

\begin{verbete}[有的时候]{you3de0\ shi2hou0}
\significado{you3de0\ shi2hou0}{}{
    às vezes;
    de vez em quando;
    de quando em quando
}
\end{verbete}

\begin{verbete}[有点儿]{you3dianr3}
\significado{you3dianr3}{adv.}{
    um pouco
}
\end{verbete}

\begin{verbete}[有名]{you3ming2}
\significado{you3ming2}{adj.}{
    famoso, famosa
}
\end{verbete}

\begin{verbete}[有时]{you3shi2}
\significado{you3shi2}{}{
    às vezes;
    de vez em quando;
    de quando em quando
}
\end{verbete}

\begin{verbete}[有时候]{you3shi2hou0}
\significado{you3shi2hou0}{}{
    às vezes;
    de vez em quando;
    de quando em quando
}
\end{verbete}

\begin{verbete}[有意思]{you3yi2si0}
\significado{you3yi2si0}{adj.}{
    interessante
}
\end{verbete}

\begin{verbete}[有用]{you3yong4}
\significado{you3yong4}{adj.}{
    útil
}
\end{verbete}

\begin{verbete}[右]{you4}
\significado{you4}{p.l.}{
    direita
}
\end{verbete}

\begin{verbete}[右边]{you4bian0}
\significado{you4bian0}{p.l.}{
    à direita; ao lado direito
}
\end{verbete}

\begin{verbete}[右面]{you4mian0}
\significado{you4mian0}{p.l.}{
    à direita; ao lado direito
}
\end{verbete}

\begin{verbete}[用]{yong4}
\significado{yong4}{v.}{
    usar
}
\end{verbete}

\begin{verbete}[鱼]{yu2}
\significado{yu2}{n.}{
    peixe|
    \pc{条}
}
\end{verbete}

\begin{verbete}[鱼片]{yu2pian4}
\significado{yu2pian4}{n.}{
    fatia de peixe
}
\end{verbete}

\begin{verbete}[鱼香肉丝]{yu2xiang1rou4si1}
\significado{yu2xiang1rou4si1}{n.}{
    tiras de carne de porco salteadas com molho picante
}
\end{verbete}

\begin{verbete}[玉]{yu3}
\significado{yu3}{n.}{
    jade|
    \pc{块}
}
\end{verbete}

\begin{verbete}[雨]{yu3}
\significado{yu3}{n.}{
    chuva
}
\end{verbete}

\begin{verbete}[雨伞]{yu3san3}
\significado{yu3san3}{n.}{
    guarda-chuva
}
\end{verbete}

\begin{verbete}[雨衣]{yu3yi1}
\significado{yu3yi1}{n.}{
    impermeável
}
\end{verbete}

\begin{verbete}[羽毛球]{yu3mao2qiu2}
\significado{yu3mao2qiu2}{n.}{
    badminton
}
\end{verbete}

\begin{verbete}[语法]{yu3fa3}
\significado{yu3fa3}{n.}{
    gramática
}
\end{verbete}

\begin{verbete}[语言实验室]{yu3yan2shi2yan4shi4}
\significado{yu3yan2shi2yan4shi4}{n.}{
    laboratório de línguas
}
\end{verbete}

\begin{verbete}[预报]{yu4bao4}
\significado{yu4bao4}{n.}{
    previsão (meteorológica); boletim meteorológico
}
\significado{yu4bao4}{v.}{
    prever (o tempo)
}
\end{verbete}

\begin{verbete}[元]{yuan2}
\significado{yuan2}{p.c.}{
    unidade monetária da China
}
\end{verbete}

\begin{verbete}[远]{yuan3}
\significado{yuan3}{adj.}{
    longe; longo, longa
}
\end{verbete}

\begin{verbete}[院子]{yuan4zi0}
\significado{yuan4zi0}{n.}{
    pátio; jardim
}
\end{verbete}

\begin{verbete}[约会]{yue1hui4}
\significado{yue1hui4}{n.}{
    compromisso; encontro marcado
}
\end{verbete}

\begin{verbete}[月]{yue4}
\significado{yue4}{n.}{
    mês
}
\end{verbete}

\begin{verbete}[月亮]{yue4liang0}
\significado{yue4liang0}{n.}{
    lua
}
\end{verbete}

\begin{verbete}[阅读]{yue4du2}
\significado{yue4du2}{n.}{
    leitura
}
\significado{yue4du2}{v.}{
    ler
}
\end{verbete}

\begin{verbete}[越······越······]{yue4...\ yue4...}
\significado{yue4...\ yue4...}{}{
    quanto mais... tanto mais...
}
\end{verbete}

\begin{verbete}[越来越······]{yue4lai2yue4...}
\significado{yue4lai2yue4...}{}{
    cada vez mais...
}
\end{verbete}

\begin{verbete}[阅览室]{yue4lan3shi4}
\significado{yue4lan3shi4}{n.}{
    sala de leitura
}
\end{verbete}

\begin{verbete}[云南]{Yun2nan2}
\significado{Yun2nan2}{n.}{
    Yunnan
}
\end{verbete}

\begin{verbete}[运动]{yun4dong4}
\significado{yun4dong4}{n.}{
    esporte; desporto
}
\end{verbete}

\begin{verbete}[运动场]{yun4dong4chang3}
\significado{yun4dong4chang3}{n.}{
    campo desportivo; campo de jogos
}
\end{verbete}

\begin{verbete}[运动会]{yun4dong4hui4}
\significado{yun4dong4hui4}{n.}{
    jogos desportivos
}
\end{verbete}

\begin{verbete}[运动员]{yun4dong4yuan2}
\significado{yun4dong4yuan2}{n.}{
    jogador, jogadora; atleta
}
\end{verbete}

\end{multicols*}

%%%
%%% Z
%%%
%\section*{Z}
\addcontentsline{toc}{section}{Z}

\begin{verbete}{杂技}{za2ji4}{6,7}
  \significado[场]{s.}{acrobacia}
\end{verbete}

\begin{verbete}{杂志}{za2zhi4}{6,7}
  \significado[本,份,期]{s.}{revista}
\end{verbete}

\begin{verbete}{杂志社}{za2zhi4she4}{6,7,7}
  \significado{s.}{editora de revista}
\end{verbete}

\begin{verbete}{砸}{za2}{10}[Radical 石]
  \significado{v.}{esmagar; bater; falhar; estragar}
\end{verbete}

\begin{verbete}{栽}{zai1}{10}[Radical 木]
  \significado{v.}{cultivar; plantar}
\end{verbete}

\begin{verbete}{栽倒}{zai1dao3}{10,10}
  \significado{v.}{cair; sofrer uma queda}
\end{verbete}

\begin{verbete}{栽培}{zai1pei2}{10,11}
  \significado{v.}{cultivar; educar; patrocinar; treinar}
\end{verbete}

\begin{verbete}{栽培种}{zai1pei2 zhong3}{10,11,9}
  \significado{s.}{espécies cultivadas}
\end{verbete}

\begin{verbete}{栽赃}{zai1zang1}{10,10}
  \significado{v.}{enquadrar alguém (plantar provas nele)}
\end{verbete}

\begin{verbete}{栽植}{zai1zhi2}{10,12}
  \significado{v.}{plantar; transplantar}
\end{verbete}

\begin{verbete}{栽种}{zai1zhong4}{10,9}
  \significado{v.}{plantar}
\end{verbete}

\begin{verbete}{再}{zai4}{6}[Radical 冂]
  \significado{adv.}{de novo; outra vez; uma segunda vez; não importa como\dots (seguido por um adjetivo ou verbo, e então (normalmente) 也 ou 都 para dar ênfase)}
\end{verbete}

\begin{verbete}{再不}{zai4bu4}{6,4}
  \significado{adv.}{nunca mais}
\end{verbete}

\begin{verbete}{再读}{zai4du2}{6,10}
  \significado{v.}{ler novamente; rever (uma lição, etc.)}
\end{verbete}

\begin{verbete}{再度}{zai4du4}{6,9}
  \significado{adv.}{outra vez; mais uma vez}
\end{verbete}

\begin{verbete}{再发}{zai4fa1}{6,5}
  \significado{v.}{reenviar}
\end{verbete}

\begin{verbete}{再见}{zai4jian4}{6,4}
  \significado{v.}{adeus; até à vista; até à próxima; até logo}
\end{verbete}

\begin{verbete}{再临}{zai4lin2}{6,9}
  \significado{v.}{vir de novo}
\end{verbete}

\begin{verbete}{再三}{zai4san1}{6,3}
  \significado{adv.}{de novo e de novo; repetidamente}
\end{verbete}

\begin{verbete}{再审}{zai4shen3}{6,8}
  \significado{s.}{novo julgamento; revisão}
  \significado{v.}{ouvir um caso novamente}
\end{verbete}

\begin{verbete}{再生}{zai4sheng1}{6,5}
  \significado{s.}{reciclagem; regeneração}
  \significado{v.}{reciclar; renascer; regenerar}
\end{verbete}

\begin{verbete}{再说}{zai4shuo1}{6,9}
  \significado{conj.}{além do mais; além disso; o que mais}
  \significado{v.}{adiar uma discussão para mais tarde; dizer novamente}
\end{verbete}

\begin{verbete}{再育}{zai4yu4}{6,8}
  \significado{v.}{aumentar; multiplicar; proliferar}
\end{verbete}

\begin{verbete}{再者}{zai4zhe3}{6,8}
  \significado{conj.}{além do mais; além disso}
\end{verbete}

\begin{verbete}{在}{zai4}{6}[Radical 土]
  \significado{adv.}{para designar ações que estão passando; durante}
  \significado{prep.}{em}
  \significado{v.}{estar; ficar}
\end{verbete}

\begin{verbete}{在此}{zai4ci3}{6,6}
  \significado{adv.}{aqui}
\end{verbete}

\begin{verbete}{在地}{zai4di4}{6,6}
  \significado{s.}{local}
\end{verbete}

\begin{verbete}{在行}{zai4hang2}{6,6}
  \significado{v.}{ser adepto de algo; ser um especialista em um comércio ou profissão}
\end{verbete}

\begin{verbete}{在乎}{zai4hu5}{6,5}
  \significado{v.}{preocupar-se com}
\end{verbete}

\begin{verbete}{在教}{zai4jiao4}{6,11}
  \significado{v.}{ser um crente (em uma religião)}
\end{verbete}

\begin{verbete}{在下}{zai4xia4}{6,3}
  \significado{pron.}{eu mesmo (humildemente)}
\end{verbete}

\begin{verbete}{在线}{zai4xian4}{6,8}
  \significado{s.}{\emph{online}}
\end{verbete}

\begin{verbete}{在意}{zai4yi4}{6,13}
  \significado{v.+compl.}{preocupar-se; importar-se; levar a sério}
\end{verbete}

\begin{verbete}{在于}{zai4yu2}{6,3}
  \significado{v.}{descansar; deitar; ser devido a (um determinado atributo)/(de um assunto) a ser determinado; estar à altura de alguém}
\end{verbete}

\begin{verbete}{咱家}{zan2jia1}{9,10}
  \significado{pron.}{eu (frequentemente usado na literatura vernácula antiga); me; mim, comigo}
\end{verbete}

\begin{verbete}{咱俩}{zan2lia3}{9,9}
  \significado{pron.}{nós dois}
\end{verbete}

\begin{verbete}{咱们}{zan2men5}{9,5}
  \significado{pron.}{nós (incluindo o orador e a(s) pessoa(s) com quem se fala)}
\end{verbete}

\begin{verbete}{赞}{zan4}{16}[Radical 貝]
  \significado{v.}{patrocinar; apoiar; elogiar; (gíria na Internet) para curtir (uma postagem on-line)}
\end{verbete}

\begin{verbete}{赞扬}{zan4yang2}{16,6}
  \significado{v.}{elogiar; aprovar; demonstrar aprovação}
\end{verbete}

\begin{verbete}{赞助}{zan4zhu4}{16,7}
  \significado{s.}{patrocinador}
  \significado{v.}{apoiar; auxiliar; patrocinar}
\end{verbete}

\begin{verbete}{脏}{zang1}{10}[Radical 肉]
  \significado{adj.}{sujo; imundo}
  \veja{脏}{zang4}
\end{verbete}

\begin{verbete}{脏辫}{zang1bian4}{10,17}
  \significado{s.}{\emph{dreadlocks}}
\end{verbete}

\begin{verbete}{脏病}{zang1bing4}{10,10}
  \significado{s.}{doença venérea}
\end{verbete}

\begin{verbete}{脏煤}{zang1mei2}{10,13}
  \significado{s.}{carvão sujo; sujeira (de uma mina de carvão)}
\end{verbete}

\begin{verbete}{脏土}{zang1tu3}{10,3}
  \significado{s.}{solo sujo; lama; lixo}
\end{verbete}

\begin{verbete}{脏脏}{zang1zang1}{10,10}
  \significado{adj.}{sujo}
\end{verbete}

\begin{verbete}{脏字}{zang1zi4}{10,6}
  \significado{s.}{obscenidade}
\end{verbete}

\begin{verbete}{脏}{zang4}{10}[Radical 肉]
  \significado{s.}{órgão (anatomia); víscera}
  \veja{脏}{zang1}
\end{verbete}

\begin{verbete}{脏器}{zang4qi4}{10,16}
  \significado{s.}{órgãos internos}
\end{verbete}

\begin{verbete}{葬}{zang4}{12}[Radical 艸]
  \significado{v.}{enterrar (os mortos); sepultar}
\end{verbete}

\begin{verbete}{遭到}{zao1dao4}{14,8}
  \significado{v.}{sofrer; encontrar-se com (algo infeliz)}
\end{verbete}

\begin{verbete}{遭受}{zao1shou4}{14,8}
  \significado{v.}{sofrer, suportar (perda, infornúnio)}
\end{verbete}

\begin{verbete}{遭遇}{zao1yu4}{14,12}
  \significado{s.}{experiência (amarga)}
  \significado{v.}{encontrar-se com;}
\end{verbete}

\begin{verbete}{糟糕}{zao1gao1}{17,16}
  \significado{adj.}{muito mau; péssimo}
\end{verbete}

\begin{verbete}{早}{zao3}{6}[Radical 日]
  \significado{adj.}{prematuramente}
  \significado{adv.}{cedo; antecipadamante; breve}
  \significado{s.}{manhã}
\end{verbete}

\begin{verbete}{早安}{zao3'an1}{6,6}
  \significado{interj.}{Bom dia!}
\end{verbete}

\begin{verbete}{早餐}{zao3can1}{6,16}
  \significado[份,顿,次]{s.}{café da manhã}
\end{verbete}

\begin{verbete}{早车}{zao3che1}{6,4}
  \significado{s.}{trem matutino; ônibus matutino}
\end{verbete}

\begin{verbete}{早晨}{zao3chen2}{6,11}
  \significado{adv.}{manhã cedo; manhãzinha}
  \significado[个]{s.}{manhã}
\end{verbete}

\begin{verbete}{早饭}{zao3fan4}{6,7}
  \significado[份,顿,次,餐]{s.}{café da manhã}
\end{verbete}

\begin{verbete}{早就}{zao3jiu4}{6,12}
  \significado{adv.}{já em um momento anterior}
\end{verbete}

\begin{verbete}{早前}{zao3qian2}{6,9}
  \significado{adv.}{previamente}
\end{verbete}

\begin{verbete}{早上}{zao3shang5}{6,3}
  \significado{adv.}{manhã cedo; manhãzinha}
  \significado[个]{s.}{manhã}
\end{verbete}

\begin{verbete}{早亡}{zao3wang2}{6,3}
  \significado[个]{s.}{morte prematura}
  \significado{v.}{morrer prematuramente}
\end{verbete}

\begin{verbete}{早早儿}{zao3zao3r5}{6,6,2}
  \significado{adv.}{o mais cedo possível; o mais breve possível}
\end{verbete}

\begin{verbete}{早知}{zao3zhi1}{6,8}
  \significado{v.}{prever; se alguém soubesse antes, \dots}
\end{verbete}

\begin{verbete}{灶台}{zao4tai2}{7,5}
  \significado{s.}{fogão}
\end{verbete}

\begin{verbete}{造}{zao4}{10}[Radical 辵]
  \significado{clas.}{para colheitas, cultivos}
  \significado{v.}{criar; construir; fabricar; inventar}
\end{verbete}

\begin{verbete}{艁}{zao4}{13}
  \variante{造}
\end{verbete}

\begin{verbete}{责怪}{ze2guai4}{8,8}
  \significado{v.}{repreender; censurar}
\end{verbete}

\begin{verbete}{怎}{zen3}{9}[Radical 心]
  \significado{adv.}{como}
\end{verbete}

\begin{verbete}{怎么}{zen3me5}{9,3}
  \significado{pron.}{como?; o que?}
\end{verbete}

\begin{verbete}{怎么办}{zen3me5ban4}{9,3,4}
  \significado{adv.}{o que fazer?}
\end{verbete}

\begin{verbete}{怎么得了}{zen3me5de2liao3}{9,3,11,2}
  \significado{expr.}{Como isso pode ser?; Que bagunça horrível!; O que deve ser feito?}
\end{verbete}

\begin{verbete}{怎么搞的}{zen3me5gao3de5}{9,3,13,8}
  \significado{expr.}{Como isso aconteceu?; O que deu errado?;E aí?; O que está errado?}
\end{verbete}

\begin{verbete}{怎么回事}{zen3me5hui2shi4}{9,3,6,8}
  \significado{expr.}{O que aconteceu?; O que se passou?}
\end{verbete}

\begin{verbete}{怎么了}{zen3me5le5}{9,3,2}
  \significado{expr.}{O que aconteceu?; O que está acontecendo?; E aí?}
\end{verbete}

\begin{verbete}{怎么样}{zen3me5yang4}{9,3,10}
  \significado{adv.}{como?; que tal?}
\end{verbete}

\begin{verbete}{增速}{zeng1su4}{15,10}
  \significado{s.}{(economia) taxa de crescimento}
  \significado{v.}{acelerar;}
\end{verbete}

\begin{verbete}{闸门}{zha2men2}{8,3}
  \significado{s.}{eclusa; comporta}
\end{verbete}

\begin{verbete}{寨}{zhai4}{14}[Radical 宀]
  \significado{s.}{fortaleza; paliçada; acampamento; vila (paliçada)}
\end{verbete}

\begin{verbete}{斩获}{zhan3huo4}{8,10}
  \significado{v.}{matar ou capturar (em batalha); (fig.) (esportes) marcar (um gol), ganhar (uma medalha);(fig.) colher recompensas, obter ganhos}
\end{verbete}

\begin{verbete}{展示}{zhan3shi4}{10,5}
  \significado{v.}{revelar, mostrar, exibir}
\end{verbete}

\begin{verbete}{盏}{zhan3}{10}[Radical 皿]
  \significado{clas.}{para lâmpadas}
  \significado{s.}{copo pequeno}
\end{verbete}

\begin{verbete}{战}{zhan4}{9}[Radical 戈]
  \significado{s.}{luta; guerra; batalha}
  \significado{v.}{lutar}
\end{verbete}

\begin{verbete}{战士}{zhan4shi4}{9,3}
  \significado[个]{s.}{lutador; soldado; guerreiro}
\end{verbete}

\begin{verbete}{战争}{zhan4zheng1}{9,6}
  \significado[場,次]{s.}{guerra; conflito}
\end{verbete}

\begin{verbete}{站}{zhan4}{10}[Radical 立]
  \significado{s.}{estação; ponto; parada}
\end{verbete}

\begin{verbete}{站点}{zhan4dian3}{10,9}
  \significado{s.}{\emph{website}}
\end{verbete}

\begin{verbete}{站台}{zhan4tai2}{10,5}
  \significado{s.}{plataforma (em uma estação ferroviária)}
\end{verbete}

\begin{verbete}{站长}{zhan4zhang3}{10,4}
  \significado{s.}{pessoa responsável pela estação de trem; chefe da estação; \emph{webmaster}; gerente de centro de voluntariado}
\end{verbete}

\begin{verbete}{站姿}{zhan4zi1}{10,9}
  \significado{s.}{postura}
\end{verbete}

\begin{verbete}{张}{zhang1}{7}[Radical 弓]
  \significado*{s.}{sobrenome Zhang}
  \significado{clas.}{para folha de papéis, mapas, etc.; para votos}
  \significado{s.}{folha de papel}
  \significado{v.}{abrir; espalhar}
\end{verbete}

\begin{verbete}{张狂}{zhang1kuang2}{7,7}
  \significado{adj.}{impetuoso; frenético; insolente}
\end{verbete}

\begin{verbete}{张三}{zhang1san1}{7,3}
  \significado*{s.}{Zhang San; Zé Ninguém; nome para uma pessoa não especificada, 1 de 3}
  \veja{李四}{li3si4}
  \veja{王五}{wang2wu3}
\end{verbete}

\begin{verbete}{章}{zhang1}{11}[Radical 音]
  \significado*{s.}{sobrenome Zhang}
  \significado{s.}{capítulo; seção; cláusula;  movimento (de sinfonia); selo; crachá; regulamento}
\end{verbete}

\begin{verbete}{章鱼}{zhang1yu2}{11,8}
  \significado{s.}{polvo; octópode}
\end{verbete}

\begin{verbete}{长}{zhang3}{4}[Radical 長]
  \significado{s.}{chefe; ancião}
  \significado{v.}{crescer; desenvolver; aumentar; melhorar}
  \veja{长}{chang2}
\end{verbete}

\begin{verbete}{涨价}{zhang3jia4}{10,6}
  \significado{s.}{aumento de preços}
  \significado{v.+compl.}{avaliar (em valor); dar preço | aumentar o preço}
\end{verbete}

\begin{verbete}{掌}{zhang3}{12}[Radical 手]
  \significado{s.}{palma da mão; sola do pé; pata; ferradura}
  \significado{v.}{dar um tapa; segurar na mão; empunhar}
\end{verbete}

\begin{verbete}{招}{zhao1}{8}[Radical 手]
  \significado{adj.}{contagioso}
  \significado{s.}{um movimento (xadrez); uma manobra; dispositivo; truque}
  \significado{v.}{recrutar; provocar; acenar; incorrer; infectar; confessar}
\end{verbete}

\begin{verbete}{招手}{zhao1shou3}{8,4}
  \significado{v.+compl.}{acenar}
\end{verbete}

\begin{verbete}{招数}{zhao1shu4}{8,13}
  \significado{s.}{estratégia; movimento (no xadrez, no palco, nas artes marciais); esquema; truque}
\end{verbete}

\begin{verbete}{着}{zhao1}{11}[Radical 目]
  \significado{interj.}{Tudo bem!}
  \significado{s.}{movimento (xadrez); truque}
  \veja{着}{zhao2}
  \veja{着}{zhe5}
  \veja{着}{zhuo2}
\end{verbete}

\begin{verbete}{着数}{zhao1shu4}{11,13}
  \significado{s.}{estratégia; movimento (no xadrez, no palco, nas artes marciais); esquema; truque}
\end{verbete}

\begin{verbete}{朝}{zhao1}{12}[Radical 月]
  \significado{s.}{manhã}
  \veja{朝}{chao2}
\end{verbete}

\begin{verbete}{着}{zhao2}{11}[Radical 目]
  \significado{v.}{ser afetado por; queimar; pegar fogo; entrar em contato com; sentir; tocar}
  \veja{着}{zhao1}
  \veja{着}{zhe5}
  \veja{着}{zhuo2}
\end{verbete}

\begin{verbete}{着地}{zhao2di4}{11,6}
  \significado{v.}{pousar; tocar o chão}
\end{verbete}

\begin{verbete}{着花}{zhao2hua1}{11,7}
  \significado{v.}{florescer}
  \veja{着花}{zhuo2hua1}
\end{verbete}

\begin{verbete}{着急}{zhao2ji2}{11,9}
  \significado{adj.}{inquieto; ansioso}
  \significado{s.}{preocupação; ansiedade}
  \significado{v.+compl.}{preocupar-se; sentir-se ansioso | sentir uma sensação de urgência}
\end{verbete}

\begin{verbete}{着凉}{zhao2liang2}{11,10}
  \significado{v.}{pegar um resfriado}
\end{verbete}

\begin{verbete}{找}{zhao3}{7}[Radical 手]
  \significado{v.}{andar à procura de; procurar; tentar procurar; dar troco; retornar algo}
\end{verbete}

\begin{verbete}{找遍}{zhao3bian4}{7,12}
  \significado{v.}{pentear; pesquisar em todos os lugares}
\end{verbete}

\begin{verbete}{找到}{zhao3dao4}{7,8}
  \significado{v.}{encontrar}
\end{verbete}

\begin{verbete}{找回}{zhao3hui2}{7,6}
  \significado{v.}{recuperar algo}
\end{verbete}

\begin{verbete}{找见}{zhao3jian4}{7,4}
  \significado{v.}{encontrar (algo que está procurando)}
\end{verbete}

\begin{verbete}{找零}{zhao3ling2}{7,13}
  \significado{v.}{trocar dinheiro; dar troco}
\end{verbete}

\begin{verbete}{找钱}{zhao3qian2}{7,10}
  \significado{v.}{dar troco}
\end{verbete}

\begin{verbete}{找事}{zhao3shi4}{7,8}
  \significado{v.}{procurar emprego; começar uma briga}
\end{verbete}

\begin{verbete}{找寻}{zhao3xun2}{7,6}
  \significado{v.}{encontrar falhas; procurar; buscar}
\end{verbete}

\begin{verbete}{找着}{zhao3zhao2}{7,11}
  \significado{v.}{encontrar}
\end{verbete}

\begin{verbete}{找辙}{zhao3zhe2}{7,16}
  \significado{v.}{procurar um pretexto}
\end{verbete}

\begin{verbete}{兆}{zhao4}{6}[Radical 儿]
  \significado{num.}{trilhão, 1.000.000.000.000}
\end{verbete}

\begin{verbete}{照}{zhao4}{13}[Radical 火]
  \significado{adv.}{de acordo com; como antes; como pedido; conforme}
  \significado{s.}{foto}
  \significado{v.}{iluminar; olhar (o reflexo de alguém); refletir; brilhar; tirar uma foto}
\end{verbete}

\begin{verbete}{照亮}{zhao4liang4}{13,9}
  \significado{s.}{iluminação}
  \significado{v.}{iluminar}
\end{verbete}

\begin{verbete}{照片}{zhao4pian4}{13,4}
  \significado[张,套,幅]{s.}{fotografia; foto}
\end{verbete}

\begin{verbete}{照片底版}{zhao4pian4di3ban3}{13,4,8,8}
  \significado{s.}{placa fotográfica}
\end{verbete}

\begin{verbete}{照片子}{zhao4pian4zi5}{13,4,3}
  \significado{v.}{tirar um raio X}
\end{verbete}

\begin{verbete}{照骗}{zhao4pian4}{13,12}
  \significado{s.}{imagem ``photoshopada''}
\end{verbete}

\begin{verbete}{照相}{zhao4xiang4}{13,9}
  \significado{v.+compl.}{tirar fotografia}
\end{verbete}

\begin{verbete}{照相机}{zhao4xiang4ji1}{13,9,6}
  \significado[个,架,部,台,只]{s.}{câmera/máquina fotográfica}
\end{verbete}

\begin{verbete}{照像}{zhao4xiang4}{13,13}
  \variante{照相}
\end{verbete}

\begin{verbete}{照像机}{zhao4xiang4ji1}{13,13,6}
  \variante{照相机}
\end{verbete}

\begin{verbete}{照准}{zhao4zhun3}{13,10}
  \significado{s.}{solicitação concedida (uso formal em documento antigo)}
  \significado{v.}{mirar (arma)}
\end{verbete}

\begin{verbete}{折转}{zhe2zhuan3}{7,8}
  \significado{s.}{reflexo (ângulo)}
  \significado{v.}{voltar atrás}
\end{verbete}

\begin{verbete}{哲理}{zhe2li3}{10,11}
  \significado{s.}{filosofia; teoria filosófica}
\end{verbete}

\begin{verbete}{这}{zhe4}{7}[Radical 辵]
  \significado{pron.}{este, isto}
  \veja{这}{zhei4}
\end{verbete}

\begin{verbete}{这里}{zhe4li3}{7,7}
  \significado{pron.}{aqui}
\end{verbete}

\begin{verbete}{这么}{zhe4me5}{7,3}
  \significado{adv.}{como este; desta maneira}
\end{verbete}

\begin{verbete}{这末}{zhe4me5}{7,5}
  \variante{这么}
\end{verbete}

\begin{verbete}{这麽}{zhe4me5}{7,14}
  \variante{这么}
\end{verbete}

\begin{verbete}{这儿}{zhe4r5}{7,2}
  \significado{pron.}{aqui}
\end{verbete}

\begin{verbete}{这时}{zhe4shi2}{7,7}
  \significado{adv.}{neste momento}
\end{verbete}

\begin{verbete}{这些}{zhe4xie1}{7,8}
  \significado{pron.}{estes}
\end{verbete}

\begin{verbete}{这样}{zhe4yang4}{7,10}
  \significado{adv.}{assim; dessa maneira; deste modo}
\end{verbete}

\begin{verbete}{浙江}{zhe4jiang1}{10,6}
  \significado*{s.}{Zhejiang}
\end{verbete}

\begin{verbete}{着}{zhe5}{11}[Radical 目]
  \significado{part.}{indicando ação em andamento ou estado em andamento}
  \veja{着}{zhao1}
  \veja{着}{zhao2}
  \veja{着}{zhuo2}
\end{verbete}

\begin{verbete}{这}{zhei4}{7}[Radical 辵]
  \significado{pron.}{(coloquial) este}
  \veja{这}{zhe4}
\end{verbete}

\begin{verbete}{珍贵}{zhen1gui4}{9,9}
  \significado{adj.}{precioso}
\end{verbete}

\begin{verbete}{珍珠}{zhen1zhu1}{9,10}
  \significado[颗]{s.}{pérola}
\end{verbete}

\begin{verbete}{眞}{zhen1}{10}
  \variante{真}
\end{verbete}

\begin{verbete}{真}{zhen1}{10}[Radical 目]
  \significado{adj.}{genuíno}
  \significado{adv.}{que\dots tão\dots!; realmente}
\end{verbete}

\begin{verbete}{真理}{zhen1li3}{10,11}
  \significado[个]{s.}{verdade}
\end{verbete}

\begin{verbete}{真牛}{zhen1niu2}{10,4}
  \significado{adj.}{gíria:~muito legal, incrível}
\end{verbete}

\begin{verbete}{真切}{zhen1qie4}{10,4}
  \significado{adj.}{claro; distinto; honesto; sincero; vívido}
\end{verbete}

\begin{verbete}{真声}{zhen1sheng1}{10,7}
  \significado{s.}{voz natural; voz verdadeira}
  \veja{假声}{jia3sheng1}
\end{verbete}

\begin{verbete}{真释}{zhen1shi4}{10,12}
  \significado{s.}{razão genuína; explicação verdadeira}
\end{verbete}

\begin{verbete}{真心}{zhen1xin1}{10,4}
  \significado{adj.}{sincero}
  \significado[片]{s.}{sinceridade}
\end{verbete}

\begin{verbete}{真真}{zhen1zhen1}{10,10}
  \significado{adv.}{genuinamente; realmente; escrupulosamente}
\end{verbete}

\begin{verbete}{真珠}{zhen1zhu1}{10,10}
  \variante{珍珠}
\end{verbete}

\begin{verbete}{枕}{zhen3}{8}[Radical 木]
  \significado{s.}{travesseiro; almofada}
\end{verbete}

\begin{verbete}{阵地}{zhen4di4}{6,6}
  \significado{s.}{posição (militar); frente de batalha; \emph{front}}
\end{verbete}

\begin{verbete}{震撼}{zhen4han4}{15,16}
  \significado{v.}{sacudir; chocar; atordoar}
\end{verbete}

\begin{verbete}{正}{zheng1}{5}[Radical 止]
  \significado{s.}{primeiro mês do ano lunar}
  \veja{正}{zheng4}
\end{verbete}

\begin{verbete}{争霸}{zheng1ba4}{6,21}
  \significado{s.}{hegemonia; uma luta de poder}
  \significado{v.}{disputar a hegemonia}
\end{verbete}

\begin{verbete}{争风吃醋}{zheng1feng1chi1cu4}{6,4,6,15}
  \significado{v.}{rivalizar com alguém pelo carinho de um homem ou mulher;  estar com ciúmes de um rival em um caso de amor}
\end{verbete}

\begin{verbete}{争先}{zheng1xian1}{6,6}
  \significado{v.}{competir para ser o primeiro; contestar o primeiro lugar}
\end{verbete}

\begin{verbete}{挣扎}{zheng1zha2}{9,4}
  \significado{v.}{lutar}
\end{verbete}

\begin{verbete}{整天}{zheng3tian1}{16,4}
  \significado{adv.}{dia todo; o dia inteiro}
\end{verbete}

\begin{verbete}{正}{zheng4}{5}[Radical 止]
  \significado{adj.}{reto; vertical; adequado; principal; positivo (matemática)}
  \significado{adv.}{agora mesmo; no processo de}
  \significado{v.}{corrigir; retificar}
  \veja{正}{zheng1}
\end{verbete}

\begin{verbete}{正常}{zheng4chang2}{5,11}
  \significado{adj.}{regular; normal; ordinário}
\end{verbete}

\begin{verbete}{正在}{zheng4zai4}{5,6}
  \significado{adv.}{no processo de; atualmente; em andamento}
  \significado{v.}{estar a~+~v.inf.; estar~+~v.ger.}
\end{verbete}

\begin{verbete}{正正}{zheng4zheng4}{5,5}
  \significado{adv.}{na hora certa; ordenadamente}
\end{verbete}

\begin{verbete}{正宗}{zheng4zong1}{5,8}
  \significado{adj.}{autêntico; genuíno; \emph{old school}; (fig.) tradicional}
\end{verbete}

\begin{verbete}{证件}{zheng4jian4}{7,6}
  \significado{s.}{documento de identificação; credencial; certificado; comprovante}
\end{verbete}

\begin{verbete}{证据}{zheng4ju4}{7,11}
  \significado{s.}{evidência; prova; testemunho}
\end{verbete}

\begin{verbete}{证实}{zheng4shi2}{7,8}
  \significado{v.}{confirmar (algo como verdadeiro); verificar}
\end{verbete}

\begin{verbete}{挣}{zheng4}{9}[Radical 手]
  \significado{v.}{ganhar dinheiro; esforçar-se para adquirir; lutar para se libertar}
\end{verbete}

\begin{verbete}{挣得}{zheng4de2}{9,11}
  \significado{v.}{ganhar renda ou dinheiro}
\end{verbete}

\begin{verbete}{挣钱}{zheng4qian2}{9,10}
  \significado{v.+compl.}{ganhar dinheiro}
\end{verbete}

\begin{verbete}{政府}{zheng4fu3}{9,8}
  \significado[个]{s.}{governo}
\end{verbete}

\begin{verbete}{政纲}{zheng4gang1}{9,7}
  \significado{s.}{programa ou plataforma política}
\end{verbete}

\begin{verbete}{政治局}{zheng4zhi4ju2}{9,8,7}
  \significado{s.}{o principal comitê de políticas de um partido comunista}
\end{verbete}

\begin{verbete}{之外}{zhi1wai4}{3,5}
  \significado{adv.}{lado de fora}
\end{verbete}

\begin{verbete}{支}{zhi1}{4}[Radical 支][Kangxi 65]
  \significado*{s.}{sobrenome Zhi}
  \significado{clas.}{para varetas como canetas e armas, para divisões do exército e para canções ou composições}
  \significado{v.}{sacar dinheiro; erguer; criar; suportar; sustentar}
\end{verbete}

\begin{verbete}{支承}{zhi1cheng2}{4,8}
  \significado{v.}{suportar o peso de (um edifício); suportar}
\end{verbete}

\begin{verbete}{支持}{zhi1chi2}{4,9}
  \significado[个]{s.}{apoio; suporte}
  \significado{v.}{apoiar; ser a favor de; suportar}
\end{verbete}

\begin{verbete}{支根}{zhi1gen1}{4,10}
  \significado{s.}{raiz ramificada; raízes de apoio; radícula}
\end{verbete}

\begin{verbete}{支票}{zhi1piao4}{4,11}
  \significado[本]{s.}{cheque (banco)}
\end{verbete}

\begin{verbete}{支应}{zhi1ying4}{4,7}
  \significado{v.}{lidar com; fornecer}
\end{verbete}

\begin{verbete}{支支吾吾}{zhi1zhi1wu2wu2}{4,4,7,7}
  \significado{v.}{falhar; murmurar; paralisar; gaguejar}
\end{verbete}

\begin{verbete}{只}{zhi1}{5}[Radical 口]
  \significado{clas.}{para pássaros, gatos, cãezinhos, etc.}
  \veja{只}{zhi3}
\end{verbete}

\begin{verbete}{只身}{zhi1shen1}{5,7}
  \significado{adv.}{sozinho; por si só}
\end{verbete}

\begin{verbete}{芝麻}{zhi1ma5}{6,11}
  \significado{s.}{semente de gergelim}
\end{verbete}

\begin{verbete}{知道}{zhi1dao4}{8,12}
  \significado{v.}{conhecer, saber}
\end{verbete}

\begin{verbete}{知道了}{zhi1dao4le5}{8,12,2}
  \significado{interj.}{Entendi!; OK!}
\end{verbete}

\begin{verbete}{知识}{zhi1shi5}{8,7}
  \significado[门]{s.}{conhecimento}
  \significado{s.}{intelectual}
\end{verbete}

\begin{verbete}{织}{zhi1}{8}[Radical 糸]
  \significado{v.}{tecer; tricotar}
\end{verbete}

\begin{verbete}{脂麻}{zhi1ma5}{10,11}
  \variante{芝麻}
\end{verbete}

\begin{verbete}{蜘蛛}{zhi1zhu1}{14,12}
  \significado{s.}{aranha}
\end{verbete}

\begin{verbete}{蜘蛛网}{zhi1zhu1wang3}{14,12,6}
  \significado{s.}{teia de aranha}
\end{verbete}

\begin{verbete}{执着}{zhi2zhuo2}{6,11}
  \significado{s.}{(budismo) apego}
  \significado{v.}{estar fortemente apegado a; ser dedicado; apegar-se a}
\end{verbete}

\begin{verbete}{直播}{zhi2bo1}{8,15}
  \significado{s.}{transmissão ao vivo; (agricultura) semeadura direta}
  \significado{v.}{(TV, rádio, Internet) transmitir ao vivo}
\end{verbete}

\begin{verbete}{直接}{zhi2jie1}{8,11}
  \significado{adj.}{direto (oposto: indireto 间接); imediato}
  \veja{间接}{jian4jie1}
\end{verbete}

\begin{verbete}{直译}{zhi2yi4}{8,7}
  \significado{s.}{tradução literal}
  \veja{意译}{yi4yi4}
\end{verbete}

\begin{verbete}{直译器}{zhi2yi4qi4}{8,7,16}
  \significado{s.}{interpretador (computação)}
\end{verbete}

\begin{verbete}{职业}{zhi2ye4}{11,5}
  \significado{adj.}{profissional}
  \significado{s.}{ocupação, profissão, vocação}
\end{verbete}

\begin{verbete}{职员}{zhi2yuan2}{11,7}
  \significado[个,位]{s.}{empregado; trabalhador de escritório; membro da equipe}
\end{verbete}

\begin{verbete}{殖}{zhi2}{12}[Radical 歹]
  \significado{v.}{crescer; reproduzir}
\end{verbete}

\begin{verbete}{只}{zhi3}{5}[Radical 口]
  \significado{adv.}{apenas; só}
  \veja{只}{zhi1}
\end{verbete}

\begin{verbete}{只得}{zhi3de5}{5,11}
  \significado{v.}{ser obrigado a; não ter outra alternativa senão}
\end{verbete}

\begin{verbete}{只读}{zhi3du2}{5,10}
  \significado{s.}{somente leitura (computação); \emph{read-only}}
\end{verbete}

\begin{verbete}{只顾}{zhi3gu4}{5,10}
  \significado{adv.}{exclusivamente preocupado (com uma coisa)}
  \significado{v.}{cuidar de apenas um aspecto}
\end{verbete}

\begin{verbete}{只好}{zhi3hao3}{5,6}
  \significado{adv.}{ser forçado a; ter que; sem nenhuma opção melhor; não ter outro remédio senão}
\end{verbete}

\begin{verbete}{只怕}{zhi3pa4}{5,8}
  \significado{adv.}{receio que\dots; talvez; muito provavelmente}
\end{verbete}

\begin{verbete}{只消}{zhi3xiao1}{5,10}
  \significado{conj.}{desde que}
\end{verbete}

\begin{verbete}{只要}{zhi3yao4}{5,9}
  \significado{conj.}{se apenas; contanto que}
\end{verbete}

\begin{verbete}{只要……就……}{zhi3yao4 jiu4}{5,9,12}
  \significado{conj.}{contanto que/desde que/se somente\dots, então\dots}
\end{verbete}

\begin{verbete}{只有……才……}{zhi3you3 cai2}{5,6,3}
  \significado{conj.}{só se\dots então\dots}
\end{verbete}

\begin{verbete}{纸}{zhi3}{7}[Radical 糸]
  \significado{clas.}{para documentos, cartas, etc.}
  \significado[张,沓]{s.}{papel}
\end{verbete}

\begin{verbete}{纸币}{zhi3bi4}{7,4}
  \significado[张]{s.}{nota (dinheiro); cédula}
\end{verbete}

\begin{verbete}{纸巾}{zhi3jin1}{7,3}
  \significado[张,包]{s.}{lenço; guardanapo; papel toalha}
\end{verbete}

\begin{verbete}{纸尿裤}{zhi3niao4ku4}{7,7,12}
  \significado{s.}{fralda descartável}
\end{verbete}

\begin{verbete}{纸烟}{zhi3yan1}{7,10}
  \significado{s.}{cigarro}
\end{verbete}

\begin{verbete}{纸张}{zhi3zhang1}{7,7}
  \significado{s.}{papel}
\end{verbete}

\begin{verbete}{指挥}{zhi3hui1}{9,9}
  \significado[个]{s.}{condutor (de uma orquestra)}
  \significado{v.}{conduzir; comandar; direcionar}
\end{verbete}

\begin{verbete}{指甲}{zhi3jia5}{9,5}
  \significado{s.}{unha da mão}
\end{verbete}

\begin{verbete}{指南针}{zhi3nan2zhen1}{9,9,7}
  \significado{s.}{bússola}
\end{verbete}

\begin{verbete}{至于}{zhi4yu2}{6,3}
  \significado{conj.}{para; quanto a; a respeiro de}
\end{verbete}

\begin{verbete}{志愿}{zhi4yuan4}{7,14}
  \significado{s.}{aspiração; ambição}
  \significado{v.}{ser voluntário}
\end{verbete}

\begin{verbete}{制裁}{zhi4cai2}{8,12}
  \significado{s.}{punição; sanção (inclusive econômica)}
  \significado{v.}{punir}
\end{verbete}

\begin{verbete}{治理}{zhi4li3}{8,11}
  \significado{s.}{governança; governo}
  \significado{v.}{gerir para melhor; administrar; por em ordem}
\end{verbete}

\begin{verbete}{治愈}{zhi4yu4}{8,13}
  \significado{v.}{curar; restaurar a saúde}
\end{verbete}

\begin{verbete}{致敬}{zhi4jing4}{10,12}
  \significado{v.}{saudar; prestar respeitos a; prestar homenagem a}
\end{verbete}

\begin{verbete}{智慧}{zhi4hui4}{12,15}
  \significado{s.}{sabedoria; inteligência}
\end{verbete}

\begin{verbete}{智商}{zhi4shang1}{12,11}
  \significado{s.}{quociente de inteligência, QI}
\end{verbete}

\begin{verbete}{智障}{zhi4zhang4}{12,13}
  \significado{adj./s.}{retardado}
\end{verbete}

\begin{verbete}{置疑}{zhi4yi2}{13,14}
  \significado{v.}{duvidar}
\end{verbete}

\begin{verbete}{中东}{zhong1dong1}{4,5}
  \significado*{s.}{Oriente Médio}
\end{verbete}

\begin{verbete}{中国}{zhong1guo2}{4,8}
  \significado*{s.}{China}
\end{verbete}

\begin{verbete}{中国城}{zhong1guo2cheng2}{4,8,9}
  \significado*{s.}{Bairro Chinês; \emph{Chinatown}}
  \veja{唐人街}{tang2ren2 jie1}
\end{verbete}

\begin{verbete}{中国科学院}{zhong1guo2 ke1xue2yuan4}{4,8,9,8,9}
  \significado*{s.}{Academia Chinesa de Ciências}
\end{verbete}

\begin{verbete}{中国人}{zhong1guo2ren2}{4,8,2}
  \significado{s.}{chinês; nascido na China}
\end{verbete}

\begin{verbete}{中国通}{zhong1guo2tong1}{4,8,10}
  \significado*{s.}{Conhecedor da China; especialista em tudo sobre a China}
\end{verbete}

\begin{verbete}{中间}{zhong1jian1}{4,7}
  \significado{adv.}{central; centro; no meio}
\end{verbete}

\begin{verbete}{中情局}{zhong1qing2ju2}{4,11,7}
  \significado*{s.}{Agência Central de Inteligência dos EUA, CIA (abreviação de 中央情报局)}
  \veja{中央情报局}{zhong1yang1 qing2bao4ju2}
\end{verbete}

\begin{verbete}{中秋节}{zhong1qiu1jie2}{4,9,5}
  \significado*{s.}{Festival do Meio-Outono, Festival do Bolo Lunar (15º dia do oitavo mês lunar)}
\end{verbete}

\begin{verbete}{中文}{zhong1wen2}{4,4}
  \significado{s.}{chinês, língua chinesa}
\end{verbete}

\begin{verbete}{中午}{zhong1wu3}{4,4}
  \significado[个]{s.}{meio-dia}
\end{verbete}

\begin{verbete}{中性}{zhong1xing4}{4,8}
  \significado{adj.}{neutro}
\end{verbete}

\begin{verbete}{中学}{zhong1xue2}{4,8}
  \significado[个]{s.}{escola ensino médio}
\end{verbete}

\begin{verbete}{中学生}{zhong1xue2sheng1}{4,8,5}
  \significado{s.}{estudante da escola ensino médio}
\end{verbete}

\begin{verbete}{中询}{zhong1 xun2}{4,8}
  \significado{adv.}{segunda dezena do mês; meio do mês; em meados do mês}
\end{verbete}

\begin{verbete}{中央情报局}{zhong1yang1 qing2bao4ju2}{4,5,11,7,7}
  \significado*{s.}{Agência Central de Inteligência dos EUA, CIA}
\end{verbete}

\begin{verbete}{中药}{zhong1yao4}{4,9}
  \significado[服,种]{s.}{medicina tradicional chinesa}
\end{verbete}

\begin{verbete}{钟}{zhong1}{9}[Radical 金]
  \significado*{s.}{sobrenome Zhong}
  \significado{clas.}{hora}
\end{verbete}

\begin{verbete}{钟室}{zhong1shi4}{9,9}
  \significado{s.}{campanário; sala do relógio}
\end{verbete}

\begin{verbete}{钟罩}{zhong1zhao4}{9,13}
  \significado{s.}{redoma; dossel de sino}
\end{verbete}

\begin{verbete}{锺}{zhong1}{14}[Radical 金]
  \variante{钟}
\end{verbete}

\begin{verbete}{种}{zhong3}{9}[Radical 禾]
  \significado{clas.}{para tipos, espécies e gêneros}
  \significado{s.}{tipo; espécie}
\end{verbete}

\begin{verbete}{种麻}{zhong3ma2}{9,11}
  \significado{s.}{planta de cânhamo (feminina)}
\end{verbete}

\begin{verbete}{种薯}{zhong3shu3}{9,16}
  \significado{s.}{tubérculo semente}
\end{verbete}

\begin{verbete}{种种}{zhong3zhong3}{9,9}
  \significado{adj.}{todos os tipos de}
\end{verbete}

\begin{verbete}{种子}{zhong3zi5}{9,3}
  \significado[颗,粒]{s.}{semente}
\end{verbete}

\begin{verbete}{种族灭绝}{zhong3zu2mie4jue2}{9,11,5,9}
  \significado{s.}{genocídio; extinção étnica}
\end{verbete}

\begin{verbete}{中意}{zhong4yi4}{4,13}
  \significado{s.}{ser do seu agrado, começar a gostar muito de algo ou de alguém}
\end{verbete}

\begin{verbete}{众}{zhong4}{6}[Radical 人]
  \significado*{s.}{abreviatura de 众议院, Câmara dos Deputados}
  \significado{adj.}{numeroso}
  \significado{adv.}{muitos}
  \significado{s.}{multidão}
  \veja{众议院}{zhong4yi4yuan4}
\end{verbete}

\begin{verbete}{众议院}{zhong4yi4yuan4}{6,5,9}
  \significado*{s.}{Casa baixa da Assembléia Bicameral; Câmara dos Deputados}
\end{verbete}

\begin{verbete}{种地}{zhong4di4}{9,6}
  \significado{v.}{cultivar; trabalhar a terra}
\end{verbete}

\begin{verbete}{重}{zhong4}{9}[Radical ⾥]
  \significado{adj.}{pesado}
  \veja{重}{chong2}
\end{verbete}

\begin{verbete}{重量}{zhong4liang4}{9,12}
  \significado[个]{s.}{peso}
\end{verbete}

\begin{verbete}{重要}{zhong4yao4}{9,9}
  \significado{adj.}{importante, significativo, principal}
\end{verbete}

\begin{verbete}{重重}{zhong4zhong4}{9,9}
  \significado{adv.}{fortemente; severamente}
  \veja{重重}{chong2chong2}
\end{verbete}

\begin{verbete}{周}{zhou1}{8}[Radical 口]
  \significado*{s.}{sobrenome Zhou; Dinastia Zhou (1046-256 BC)}
  \significado{adv.}{semanalmente}
  \significado{s.}{círculo; circunferência; ciclo; uma volta (em um circuito); semana}
  \significado{v.}{fazer um circuito; circular; ajudar financeiramente}
\end{verbete}

\begin{verbete}{周末}{zhou1mo4}{8,5}
  \significado{s.}{final-de-semana}
\end{verbete}

\begin{verbete}{洲}{zhou1}{9}[Radical 水]
  \significado{s.}{continente; ilha em um rio}
\end{verbete}

\begin{verbete}{轴承}{zhou2cheng2}{9,8}
  \significado{s.}{(mecânico) rolamento}
\end{verbete}

\begin{verbete}{咒骂}{zhou4ma4}{8,9}
  \significado{v.}{xingar; amaldiçoar; execrar}
\end{verbete}

\begin{verbete}{珠子}{zhu1zi5}{10,3}
  \significado[粒,颗]{s.}{pérola; contas}
\end{verbete}

\begin{verbete}{猪}{zhu1}{11}[Radical 犬]
  \significado[口,头]{s.}{porco; suíno}
\end{verbete}

\begin{verbete}{猪窠}{zhu1ke1}{11,13}
  \significado{s.}{chiqueiro}
\end{verbete}

\begin{verbete}{猪柳}{zhu1liu3}{11,9}
  \significado{s.}{filé de porco}
\end{verbete}

\begin{verbete}{猪笼}{zhu1long2}{11,11}
  \significado{s.}{estrutura cilíndrica de bambu ou arame usada para restringir um porco durante o transporte}
\end{verbete}

\begin{verbete}{猪头}{zhu1tou2}{11,5}
  \significado{s.}{tolo; idiota}
\end{verbete}

\begin{verbete}{竹编}{zhu2bian1}{6,12}
  \significado{s.}{vime; tecelagem de bambu}
\end{verbete}

\begin{verbete}{竹马}{zhu2ma3}{6,3}
  \significado{s.}{cavalo de bambu; vara de bambu usada como cavalo de brinquedo}
\end{verbete}

\begin{verbete}{竹排}{zhu2pai2}{6,11}
  \significado{s.}{jangada de bambu}
\end{verbete}

\begin{verbete}{竹子}{zhu2zi5}{6,3}
  \significado[棵,支,根]{s.}{bambu}
\end{verbete}

\begin{verbete}{逐步}{zhu2bu4}{10,7}
  \significado{adv.}{pouco a pouco; passo a passo; progressivamente}
\end{verbete}

\begin{verbete}{逐渐}{zhu2jian4}{10,11}
  \significado{adv.}{pouco a pouco; passo a passo; progressivamente}
\end{verbete}

\begin{verbete}{主席}{zhu3xi2}{5,10}
  \significado*[个,位]{s.}{Presidente (da China); Primeiro-Ministro}
\end{verbete}

\begin{verbete}{主席台}{zhu3xi2tai2}{5,10,5}
  \significado[个]{s.}{plataforma; tribuna}
\end{verbete}

\begin{verbete}{主席团}{zhu3xi2tuan2}{5,10,6}
  \significado{s.}{presídio}
\end{verbete}

\begin{verbete}{主义}{zhu3yi4}{5,3}
  \significado{s.}{ideologia; sufixo "ismo"}
\end{verbete}

\begin{verbete}{属}{zhu3}{12}[Radical 尸]
  \significado{v.}{juntar-se; fixar a atenção em; concentrar-se em}
  \veja{属}{shu3}
\end{verbete}

\begin{verbete}{嘱}{zhu3}{15}[Radical 口]
  \significado{v.}{juntar-se; implorar; incitar}
\end{verbete}

\begin{verbete}{嘱咐}{zhu3fu5}{15,8}
  \significado{v.}{ordenar; dizer; exortar}
\end{verbete}

\begin{verbete}{嘱托}{zhu3tuo1}{15,6}
  \significado{v.}{confiar uma tarefa a alguém}
\end{verbete}

\begin{verbete}{住}{zhu4}{7}[Radical 人]
  \significado{v.}{habitar; residir; morar; alojar-se}
\end{verbete}

\begin{verbete}{住处}{zhu4chu4}{7,5}
  \significado{s.}{morada; habitação; residência}
\end{verbete}

\begin{verbete}{住房}{zhu4fang2}{7,8}
  \significado{s.}{habitação}
\end{verbete}

\begin{verbete}{住所}{zhu4suo3}{7,8}
  \significado[处]{s.}{morada; habitação; residência}
\end{verbete}

\begin{verbete}{住宅}{zhu4zhai2}{7,6}
  \significado{s.}{residência}
\end{verbete}

\begin{verbete}{住嘴}{zhu4zui3}{7,16}
  \significado{interj.}{Cale-se!}
  \significado{v.}{calar; calar-se}
\end{verbete}

\begin{verbete}{助兴}{zhu4xing4}{7,6}
  \significado{v.+compl.}{animar as coisas; juntar-se à diversão}
\end{verbete}

\begin{verbete}{注册}{zhu4ce4}{8,5}
  \significado{v.}{inscrever-se; matricular-se; registrar-se}
\end{verbete}

\begin{verbete}{注册表}{zhu4ce4biao3}{8,5,8}
  \significado*{s.}{Registro do Windows}
\end{verbete}

\begin{verbete}{注册人}{zhu4ce4ren2}{8,5,2}
  \significado{s.}{registrante}
\end{verbete}

\begin{verbete}{注册商标}{zhu4ce4shang1biao1}{8,5,11,9}
  \significado{s.}{marca registrada}
\end{verbete}

\begin{verbete}{注意}{zhu4yi4}{8,13}
  \significado{v.}{tomar nota de, prestar atenção em}
\end{verbete}

\begin{verbete}{注意地}{zhu4yi4di4}{8,13,6}
  \significado{s.}{área de cuidado, de observação}
\end{verbete}

\begin{verbete}{注意力}{zhu4yi4li4}{8,13,2}
  \significado{s.}{atenção}
\end{verbete}

\begin{verbete}{注意力缺失症}{zhu4yi4li4que1shi1zheng4}{8,13,2,10,5,10}
  \significado{s.}{transtorno de déficit de atenção}
\end{verbete}

\begin{verbete}{驻军}{zhu4jun1}{8,6}
  \significado{s.}{guarnição}
  \significado{v.}{guarcener ou prover uma tropa}
\end{verbete}

\begin{verbete}{祝}{zhu4}{9}[Radical 示]
  \significado*{s.}{sobrenome Zhu}
  \significado{v.}{desejar (exprimir um bom desejo); congratular; rezar}
\end{verbete}

\begin{verbete}{祝祷}{zhu4dao3}{9,11}
  \significado{v.}{rezar; orar}
\end{verbete}

\begin{verbete}{祝福}{zhu4fu2}{9,13}
  \significado{s.}{bênçãos}
  \significado{v.}{desejar boa sorte a alguém}
\end{verbete}

\begin{verbete}{祝好}{zhu4hao3}{9,6}
  \significado{expr.}{desejo-lhe tudo de melhor! (ao encerrar uma correspondência)}
\end{verbete}

\begin{verbete}{祝贺}{zhu4he4}{9,9}
  \significado[个]{s.}{congratulações}
  \significado{v.}{congratular}
\end{verbete}

\begin{verbete}{祝酒}{zhu4jiu3}{9,10}
  \significado{v.}{parabenizar e fazer um brinde; brindar}
\end{verbete}

\begin{verbete}{祝寿}{zhu4shou4}{9,7}
  \significado{v.}{dar parabéns pelo aniversário (a uma pessoa idosa)}
\end{verbete}

\begin{verbete}{祝颂}{zhu4song4}{9,10}
  \significado{v.}{expressar bons desejos}
\end{verbete}

\begin{verbete}{祝谢}{zhu4xie4}{9,12}
  \significado{v.}{agradecer; dar parabéns}
\end{verbete}

\begin{verbete}{祝愿}{zhu4yuan4}{9,14}
  \significado{v.}{desejar}
\end{verbete}

\begin{verbete}{专业}{zhuan1ye4}{4,5}
  \significado[门,个]{s.}{área de atuação; especialidade}
\end{verbete}

\begin{verbete}{专业户}{zhuan1ye4hu4}{4,5,4}
  \significado{s.}{indústria caseira; empresa familiar produzindo um produto especial}
\end{verbete}

\begin{verbete}{专业化}{zhuan1ye4hua4}{4,5,4}
  \significado{s.}{especialização}
\end{verbete}

\begin{verbete}{专业教育}{zhuan1ye4jiao4yu4}{4,5,11,8}
  \significado{s.}{educação especializada; escola técnica}
\end{verbete}

\begin{verbete}{专业人才}{zhuan1ye4ren2cai2}{4,5,2,3}
  \significado{s.}{especialista (em uma área)}
\end{verbete}

\begin{verbete}{专业人士}{zhuan1ye4ren2shi4}{4,5,2,3}
  \significado{s.}{profissional}
\end{verbete}

\begin{verbete}{专业性}{zhuan1ye4xing4}{4,5,8}
  \significado{s.}{profissionalismo; expertise}
\end{verbete}

\begin{verbete}{砖}{zhuan1}{9}[Radical 石]
  \significado[块]{s.}{tijolo}
\end{verbete}

\begin{verbete}{转}{zhuan3}{8}[Radical 車]
  \significado{v.}{mudar de direção; transferir; encaminhar (correio); virar}
  \veja{转}{zhuan4}
\end{verbete}

\begin{verbete}{转产}{zhuan3chan3}{8,6}
  \significado{v.}{mudar a produção; mudar para novos produtos}
\end{verbete}

\begin{verbete}{转递}{zhuan3di4}{8,10}
  \significado{v.}{passar; retransmitir}
\end{verbete}

\begin{verbete}{转告}{zhuan3gao4}{8,7}
  \significado{v.}{comunicar; transmitir}
\end{verbete}

\begin{verbete}{转念}{zhuan3nian4}{8,8}
  \significado{v.}{ter dúvidas sobre algo; pensar melhor}
\end{verbete}

\begin{verbete}{转账}{zhuan3zhang4}{8,8}
  \significado{v.+compl.}{transferir entre contas; trazer à frente; incluir uma soma de dinheiro do balanço anterior no seguinte}
\end{verbete}

\begin{verbete}{传}{zhuan4}{6}[Radical 人]
  \significado{s.}{biografia; narrativa histórica; comentários; estação de retransmissão}
  \veja{传}{chuan2}
\end{verbete}

\begin{verbete}{转}{zhuan4}{8}[Radical 車]
  \significado{clas.}{para ações repetidas; para rotações (por minuto, etc.): RPM}
  \significado{v.}{circular sobre; dar voltas; andar por aí}
  \veja{转}{zhuan3}
\end{verbete}

\begin{verbete}{转悠}{zhuan4you5}{8,11}
  \significado{v.}{aparecer repetidamente; rolar; passear por aí}
\end{verbete}

\begin{verbete}{转游}{zhuan4you5}{8,12}
  \variante{转悠}
\end{verbete}

\begin{verbete}{妆}{zhuang1}{6}[Radical 女]
  \significado{s.}{maquiagem; adorno; enxoval; maquiagem e figurino de palco}
  \significado{v.}{maquiar-se, enfeitar-se}
\end{verbete}

\begin{verbete}{妆扮}{zhuang1ban4}{6,7}
  \variante{装扮}
\end{verbete}

\begin{verbete}{桩}{zhuang1}{10}[Radical 木]
  \significado{clas.}{para eventos, casos, transações, assuntos, etc.}
  \significado{s.}{toco, estaca, pilha}
\end{verbete}

\begin{verbete}{装}{zhuang1}{12}[Radical 衣]
  \significado{s.}{adorno; roupa; traje (de um ator em uma peça)}
  \significado{v.}{adornar; vestir; desepenhar um papel; fingir; instalar; consertar; embrulhar (algo em um saco); empacotar}
\end{verbete}

\begin{verbete}{装扮}{zhuang1ban4}{12,7}
  \significado{v.}{enfeitar; decorar; disfarçar-me; vestir-se}
\end{verbete}

\begin{verbete}{装备}{zhuang1bei4}{12,8}
  \significado{s.}{equipamento}
  \significado{v.}{equipar}
\end{verbete}

\begin{verbete}{装配}{zhuang1pei4}{12,10}
  \significado{v.}{montar; encaixar}
\end{verbete}

\begin{verbete}{撞车}{zhuang4che1}{15,4}
  \significado{v.+compl.}{(figurativo) colidir (opiniões, cronogramas, etc.) | ser o mesmo (assunto) | colidir (com outro veículo)}
\end{verbete}

\begin{verbete}{撞运气}{zhuang4yun4qi5}{15,7,4}
  \significado{v.}{confiar no destino; tentar a sorte}
\end{verbete}

\begin{verbete}{追赶}{zhui1gan3}{9,10}
  \significado{v.}{perseguir; acelerar; alcançar; ultrapassar}
\end{verbete}

\begin{verbete}{坠}{zhui4}{7}[Radical 土]
  \significado{v.}{cair; pesar; fazer vergar com o peso}
\end{verbete}

\begin{verbete}{坠落}{zhui4luo4}{7,12}
  \significado{v.}{cair}
\end{verbete}

\begin{verbete}{准}{zhun3}{10}[Radical 冫]
  \significado{adv.}{certamente; de acordo com; à luz de}
  \significado{v.}{permitir; conceder}
\end{verbete}

\begin{verbete}{桌}{zhuo1}{10}[Radical 木]
  \significado{clas.}{para mesas de convidados em um banquete etc.}
  \significado{s.}{mesa}
\end{verbete}

\begin{verbete}{桌布}{zhuo1bu4}{10,5}
  \significado[条,块,张]{s.}{computação:~plano de fundo da área de trabalho; toalha de mesa; papel de parede}
\end{verbete}

\begin{verbete}{桌灯}{zhuo1deng1}{10,6}
  \significado{s.}{luminária; lâmpada de mesa}
\end{verbete}

\begin{verbete}{桌机}{zhuo1ji1}{10,6}
  \significado{s.}{computador \emph{desktop}}
\end{verbete}

\begin{verbete}{桌面}{zhuo1mian4}{10,9}
  \significado{s.}{área de trabalho; mesa}
\end{verbete}

\begin{verbete}{桌球}{zhuo1qiu2}{10,11}
  \significado{s.}{bilhar; sinuca; mesa de ping-pong}
\end{verbete}

\begin{verbete}{桌游}{zhuo1you2}{10,12}
  \significado{s.}{jogo de tabuleiro}
\end{verbete}

\begin{verbete}{桌子}{zhuo1zi5}{10,3}
  \significado[张,套]{s.}{mesa}
\end{verbete}

\begin{verbete}{棹}{zhuo1}{12}[Radical 木]
  \variante{桌}
\end{verbete}

\begin{verbete}{着}{zhuo2}{11}[Radical 目]
  \significado{v.}{aplicar; contactar; usar; vestir (roupas)}
  \veja{着}{zhao1}
  \veja{着}{zhao2}
  \veja{着}{zhe5}
\end{verbete}

\begin{verbete}{着花}{zhuo2hua1}{11,7}
  \significado{s.}{floração}
  \significado{v.}{florescer}
  \veja{着花}{zhao2hua1}
\end{verbete}

\begin{verbete}{着手}{zhuo2shou3}{11,4}
  \significado{v.}{colocar a mão nisso; estabelecer; começar uma tarefa}
\end{verbete}

\begin{verbete}{着想}{zhuo2xiang3}{11,13}
  \significado{v.}{considerar (as necessidades de outras pessoas); pensar (para os outros)}
\end{verbete}

\begin{verbete}{着眼}{zhuo2yan3}{11,11}
  \significado{v.}{ter seus olhos em (um objetivo); ter algo em mente; concentrar-se}
\end{verbete}

\begin{verbete}{着装}{zhuo2zhuang1}{11,12}
  \significado{s.}{roupa; vestimenta}
  \significado{v.}{vestir}
\end{verbete}

\begin{verbete}{资}{zi1}{10}[Radical 貝]
  \significado{s.}{recursos; capital; dinheiro; despesa}
  \significado{v.}{fornecer; suprir}
\end{verbete}

\begin{verbete}{资助}{zi1zhu4}{10,7}
  \significado{s.}{subsídio}
  \significado{v.}{subsidiar; fornecer ajuda financeira}
\end{verbete}

\begin{verbete}{子弹}{zi3dan4}{3,11}
  \significado[粒,颗,发]{s.}{bala (de revólver)}
\end{verbete}

\begin{verbete}{紫}{zi3}{12}[Radical 糸]
  \significado{adj.}{púrpura; violeta}
\end{verbete}

\begin{verbete}{紫色}{zi3se4}{12,6}
  \significado{s.}{cor púrpura; cor violeta}
\end{verbete}

\begin{verbete}{字}{zi4}{6}[Radical 子]
  \significado[个]{s.}{carácter; letra; símbolo; palavra}
\end{verbete}

\begin{verbete}{字典}{zi4dian3}{6,8}
  \significado[本]{s.}{dicionário de caracteres chineses (contendo verbetes de caracteres únicos, em contraste com 词典 que contém verbetes para palavras de um ou mais caracteres)}
\end{verbete}

\begin{verbete}{字脚}{zi4jiao3}{6,11}
  \significado[典]{s.}{gancho no final da pincelada; serifa}
\end{verbete}

\begin{verbete}{字母}{zi4mu3}{6,5}
  \significado[个]{s.}{letra (do alfabeto)}
\end{verbete}

\begin{verbete}{字眼}{zi4yan3}{6,11}
  \significado[个]{s.}{palavras; redação}
\end{verbete}

\begin{verbete}{字字珠玉}{zi4zi4zhu1yu4}{6,6,10,5}
  \significado{expr.}{cada palavra é uma jóia}
  \significado{s.}{escrita magnífica}
\end{verbete}

\begin{verbete}{自动化}{zi4dong4hua4}{6,6,4}
  \significado{s.}{automação}
\end{verbete}

\begin{verbete}{自个儿}{zi4ge3r5}{6,3,2}
  \significado{pron.}{(dialeto) a si mesmo, por si mesmo}
\end{verbete}

\begin{verbete}{自己}{zi4ji3}{6,3}
  \significado{pron.}{a si próprio; próprio}
\end{verbete}

\begin{verbete}{自己动手}{zi4ji3dong4shou3}{6,3,6,4}
  \significado{v.}{fazer (algo) sozinho; ajudar-se a}
\end{verbete}

\begin{verbete}{自救}{zi4jiu4}{6,11}
  \significado{v.}{sair a si mesmo de problemas}
\end{verbete}

\begin{verbete}{自来水}{zi4lai2shui3}{6,7,4}
  \significado{s.}{água corrente, água da torneira}
\end{verbete}

\begin{verbete}{自然}{zi4ran2}{6,12}
  \significado{adj.}{natural}
  \significado{adv.}{naturalmente; de maneira natural}
  \significado{s.}{natureza}
\end{verbete}

\begin{verbete}{自燃}{zi4ran2}{6,16}
  \significado{s.}{combustão espontânea}
\end{verbete}

\begin{verbete}{自我}{zi4wo3}{6,7}
  \significado{pron.}{a si mesmo; eu próprio; auto\dots; ego (psicologia)}
\end{verbete}

\begin{verbete}{自我安慰}{zi4wo3'an1wei4}{6,7,6,15}
  \significado{v.}{confortar-se; consolar-se; tranquilizar-se}
\end{verbete}

\begin{verbete}{自我保存}{zi4wo3 bao3cun2}{6,7,9,6}
  \significado{v.}{autopreservação}
\end{verbete}

\begin{verbete}{自我吹嘘}{zi4wo3chui1xu1}{6,7,7,14}
  \significado{expr.}{tocar a própria buzina}
\end{verbete}

\begin{verbete}{自我催眠}{zi4wo3cui1mian2}{6,7,13,10}
  \significado{v.}{consolar-me; tranquilizar-me}
\end{verbete}

\begin{verbete}{自我的人}{zi4wo3de5ren2}{6,7,8,2}
  \significado{s.}{(minha, sua) própria pessoa; (afirmar) a própria personalidade}
\end{verbete}

\begin{verbete}{自我防卫}{zi4wo3fang2wei4}{6,7,6,3}
  \significado{s.}{defesa pessoal; auto-defesa}
\end{verbete}

\begin{verbete}{自我解嘲}{zi4wo3jie3chao2}{6,7,13,15}
  \significado{s.}{referir-se às próprias fraquezas ou falhas com humor autodepreciativo}
\end{verbete}

\begin{verbete}{自我介绍}{zi4wo3jie4shao4}{6,7,4,8}
  \significado{s.}{defesa pessoal; auto-defesa}
\end{verbete}

\begin{verbete}{自我批评}{zi4wo3pi1ping2}{6,7,7,7}
  \significado{s.}{autocrítica}
\end{verbete}

\begin{verbete}{自我实现}{zi4wo3shi2xian4}{6,7,8,8}
  \significado{s.}{psicologia:~auto-atualização; auto-realização}
\end{verbete}

\begin{verbete}{自我陶醉}{zi4wo3tao2zui4}{6,7,10,15}
  \significado{s.}{narcisista; auto-imbuído; satisfeito consigo mesmo}
\end{verbete}

\begin{verbete}{自我意识}{zi4wo3yi4shi2}{6,7,13,7}
  \significado{s.}{autoapresentação}
  \significado{v.}{apresentar-se}
\end{verbete}

\begin{verbete}{自行车}{zi4xing2che1}{6,6,4}
  \significado[辆]{s.}{bicicleta}
\end{verbete}

\begin{verbete}{自行车馆}{zi4xing2che1guan3}{6,6,4,11}
  \significado{s.}{estádio de ciclismo; velódromo}
\end{verbete}

\begin{verbete}{自行车架}{zi4xing2che1jia4}{6,6,4,9}
  \significado{s.}{suporte para bicicleta; bicicletário}
\end{verbete}

\begin{verbete}{自行车赛}{zi4xing2che1sai4}{6,6,4,14}
  \significado{s.}{corrida de bicicleta}
\end{verbete}

\begin{verbete}{自由}{zi4you2}{6,5}
  \significado{adj.}{livre, irrestrito}
  \significado[种]{s.}{liberdade}
\end{verbete}

\begin{verbete}{自由泳}{zi4you2yong3}{6,5,8}
  \significado{s.}{natação de estilo livre}
\end{verbete}

\begin{verbete}{自责}{zi4ze2}{6,8}
  \significado{v.}{culpar-se}
\end{verbete}

\begin{verbete}{棕褐色}{zong1he4se4}{12,14,6}
  \significado{s.}{cor sépia; bronzeado}
\end{verbete}

\begin{verbete}{总}{zong3}{9}[Radical 心]
  \significado{adv.}{em geral; completamente}
\end{verbete}

\begin{verbete}{总长}{zong3chang2}{9,4}
  \significado{s.}{comprimento total}
\end{verbete}

\begin{verbete}{总得}{zong3dei3}{9,11}
  \significado{adv.}{prestes a}
  \significado{v.}{dever; precisar}
\end{verbete}

\begin{verbete}{总督}{zong3du1}{9,13}
  \significado*{s.}{Governador-Geral; Governador; Vice-Rei}
\end{verbete}

\begin{verbete}{总价}{zong3jia4}{9,6}
  \significado{s.}{preço total}
\end{verbete}

\begin{verbete}{总结}{zong3jie2}{9,9}
  \significado[个]{s.}{currículo; resumo}
  \significado{v.}{concluir; resumir}
\end{verbete}

\begin{verbete}{总理}{zong3li3}{9,11}
  \significado*{s.}{Primeiro-Ministro}
\end{verbete}

\begin{verbete}{总台}{zong3tai2}{9,5}
  \significado{s.}{recepção; balcão de recepção}
\end{verbete}

\begin{verbete}{总统}{zong3tong3}{9,9}
  \significado*[个,位,名,届]{s.}{Presidente (de um país)}
\end{verbete}

\begin{verbete}{总务}{zong3wu4}{9,5}
  \significado{s.}{divisão de assuntos gerais; assuntos gerais; pessoa responsável geral}
\end{verbete}

\begin{verbete}{总线}{zong3xian4}{9,8}
  \significado{s.}{barramento (computador); \emph{computer bus}}
\end{verbete}

\begin{verbete}{总站}{zong3zhan4}{9,10}
  \significado{s.}{terminal}
\end{verbete}

\begin{verbete}{总值}{zong3zhi2}{9,10}
  \significado{s.}{valor total}
\end{verbete}

\begin{verbete}{赱}{zou3}{6}
  \variante{走}
\end{verbete}

\begin{verbete}{走}{zou3}{7}[Radical 走][Kangxi 156]
  \significado{v.}{andar; caminhar}
\end{verbete}

\begin{verbete}{走鬼}{zou3gui3}{7,9}
  \significado{s.}{vendedor ambulante sem licença}
\end{verbete}

\begin{verbete}{走过}{zou3guo4}{7,6}
  \significado{v.}{passar}
\end{verbete}

\begin{verbete}{走去}{zou3qu4}{7,5}
  \significado{v.}{caminhar até (para)}
\end{verbete}

\begin{verbete}{走绳}{zou3sheng2}{7,11}
  \significado{v.}{andar na corda bamba}
  \veja{走索}{zou3suo3}
\end{verbete}

\begin{verbete}{走势}{zou3shi4}{7,8}
  \significado{s.}{caminho; tendência}
\end{verbete}

\begin{verbete}{走索}{zou3suo3}{7,10}
  \significado{v.}{andar na corda bamba}
  \veja{走绳}{zou3sheng2}
\end{verbete}

\begin{verbete}{走秀}{zou3xiu4}{7,7}
  \significado{s.}{desfile de moda}
  \significado{v.}{andar na passarela (em um desfile de moda)}
\end{verbete}

\begin{verbete}{走卒}{zou3zu2}{7,8}
  \significado{s.}{lacaio (masculino); peão (isto é, soldado de infantaria); servo}
\end{verbete}

\begin{verbete}{奏效}{zou4xiao4}{9,10}
  \significado{v.}{mostrar resultados, ser eficaz}
\end{verbete}

\begin{verbete}{租}{zu1}{10}[Radical 禾]
  \significado{s.}{imposto sobre propriedade urbana ou rural}
  \significado{v.}{alugar; tomar de aluguel}
\end{verbete}

\begin{verbete}{租船}{zu1chuan2}{10,11}
  \significado{v.}{fretar um navio; alugar um navio}
\end{verbete}

\begin{verbete}{租房}{zu1fang2}{10,8}
  \significado{v.}{alugar um apartamento}
\end{verbete}

\begin{verbete}{租金}{zu1jin1}{10,8}
  \significado{s.}{aluguel}
  \veja{租钱}{zu1qian5}
\end{verbete}

\begin{verbete}{租赁}{zu1lin4}{10,10}
  \significado{v.}{contratar; alugar}
\end{verbete}

\begin{verbete}{租钱}{zu1qian5}{10,10}
  \significado{s.}{aluguel}
  \veja{租金}{zu1jin1}
\end{verbete}

\begin{verbete}{租让}{zu1rang4}{10,5}
  \significado{v.}{alugar; alugar (a propriedade de alguém para outra pessoa)}
\end{verbete}

\begin{verbete}{租用}{zu1yong4}{10,5}
  \significado{v.}{contratar; alugar; alugar (algo de alguém)}
\end{verbete}

\begin{verbete}{租约}{zu1yue1}{10,6}
  \significado{s.}{aluguel}
\end{verbete}

\begin{verbete}{足}{zu2}{7}[Radical 足][Kangxi 157]
  \significado{adj.}{amplo}
  \significado{s.}{pé}
  \significado{v.}{ser suficiente}
  \veja{足}{ju4}
\end{verbete}

\begin{verbete}{足球}{zu2qiu2}{7,11}
  \significado[个]{s.}{futebol; bola de futebol}
\end{verbete}

\begin{verbete}{足球场}{zu2qiu2chang3}{7,11,6}
  \significado{s.}{campo de futebol}
\end{verbete}

\begin{verbete}{足球队}{zu2qiu2dui4}{7,11,4}
  \significado{s.}{time de futebol}
\end{verbete}

\begin{verbete}{足球迷}{zu2qiu2mi2}{7,11,9}
  \significado{s.}{fã de futebol}
\end{verbete}

\begin{verbete}{足球赛}{zu2qiu2sai4}{7,11,14}
  \significado{s.}{competição de futebol; partida de futebol}
\end{verbete}

\begin{verbete}{足球协会}{zu2qiu2xie2hui4}{7,11,6,6}
  \significado*{s.}{Associação de Futebol}
\end{verbete}

\begin{verbete}{足月}{zu2yue4}{7,4}
  \significado{s.}{gestação completa}
\end{verbete}

\begin{verbete}{足足}{zu2zu2}{7,7}
  \significado{adv.}{tanto quanto; extremamente; completamente; não menos que}
\end{verbete}

\begin{verbete}{族}{zu2}{11}[Radical 方]
  \significado{s.}{raça; nacionalidade; etnia; clã; por extensão, grupo social}
\end{verbete}

\begin{verbete}{诅咒}{zu3zhou4}{7,8}
  \significado{v.}{amaldiçoar}
\end{verbete}

\begin{verbete}{阻击}{zu3ji1}{7,5}
  \significado{v.}{verificar; parar}
\end{verbete}

\begin{verbete}{祖国}{zu3guo2}{9,8}
  \significado{s.}{pátria, terra natal}
\end{verbete}

\begin{verbete}{钻戒}{zuan4jie4}{10,7}
  \significado[只]{s.}{anel de diamante}
\end{verbete}

\begin{verbete}{钻石}{zuan4shi2}{10,5}
  \significado[颗]{s.}{diamante}
\end{verbete}

\begin{verbete}{嘴巴}{zui3ba5}{16,4}
  \significado[张]{s.}{boca}
  \significado[个]{s.}{bofetada na cara}
\end{verbete}

\begin{verbete}{嘴巴子}{zui3ba5zi5}{16,4,3}
  \significado{s.}{tapa; bofetada}
\end{verbete}

\begin{verbete}{最}{zui4}{12}[Radical 冂]
  \significado{adv.}{o mais; o melhor; a coisa mais\dots; grau superlativo relativo de superioridade}
\end{verbete}

\begin{verbete}{最初}{zui4chu1}{12,7}
  \significado{adj.}{inicial; original; primário}
  \significado{adv.}{inicialmente; originalmente}
\end{verbete}

\begin{verbete}{最多}{zui4duo1}{12,6}
  \significado{adv.}{no máximo; máximo}
\end{verbete}

\begin{verbete}{最高}{zui4gao1}{12,10}
  \significado{adj.}{altíssimo; supremo; mais alto}
\end{verbete}

\begin{verbete}{最好}{zui4hao3}{12,6}
  \significado{adv.}{ser melhor que}
  \significado{v.}{(você) estar melhor (faça o que sugerimos); querer ser o melhor}
\end{verbete}

\begin{verbete}{最后}{zui4hou4}{12,6}
  \significado{adj.}{final; último}
  \significado{adv.}{finalmente}
\end{verbete}

\begin{verbete}{最佳}{zui4jia1}{12,8}
  \significado{adj.}{melhor (atleta, filme etc); ótimo}
\end{verbete}

\begin{verbete}{最近}{zui4jin4}{12,7}
  \significado{adv.}{ultimamente; recentemente}
\end{verbete}

\begin{verbete}{最善}{zui4shan4}{12,12}
  \significado{adj.}{ótimo; o melhor}
\end{verbete}

\begin{verbete}{最少}{zui4shao3}{12,4}
  \significado{adv.}{finalmente}
\end{verbete}

\begin{verbete}{最先}{zui4xian1}{12,6}
  \significado{adv.}{o primeiro}
\end{verbete}

\begin{verbete}{最新}{zui4xin1}{12,13}
  \significado{adv.}{mais recente; mais novo}
\end{verbete}

\begin{verbete}{最优}{zui4you1}{12,6}
  \significado{adj.}{ótimo}
\end{verbete}

\begin{verbete}{最远}{zui4yuan3}{12,7}
  \significado{adv.}{mais distante; mais longe}
\end{verbete}

\begin{verbete}{最终}{zui4zhong1}{12,8}
  \significado{adv.}{pelo menos; finalmente;}
  \significado{s.}{final; ultimato}
\end{verbete}

\begin{verbete}{罪犯}{zui4fan4}{13,5}
  \significado{s.}{criminoso}
\end{verbete}

\begin{verbete}{罪行}{zui4xing2}{13,6}
  \significado{s.}{crime; ofensa}
\end{verbete}

\begin{verbete}{醉}{zui4}{15}[Radical 酉]
  \significado{v.}{embriagar-se; ficar bêbado}
\end{verbete}

\begin{verbete}{昨}{zuo2}{9}[Radical 日]
  \significado{s.}{ontem}
\end{verbete}

\begin{verbete}{昨日}{zuo2ri4}{9,4}
  \significado{adv.}{ontem}
\end{verbete}

\begin{verbete}{昨天}{zuo2tian1}{9,4}
  \significado{adv.}{ontem}
\end{verbete}

\begin{verbete}{昨晚}{zuo2wan3}{9,11}
  \significado{adv.}{noite passada; ontem à noite}
\end{verbete}

\begin{verbete}{昨夜}{zuo2ye4}{9,8}
  \significado{adv.}{noite passada}
\end{verbete}

\begin{verbete}{左}{zuo3}{5}[Radical 工]
  \significado*{s.}{sobrenome Zuo}
  \significado{p.l.}{esquerda}
\end{verbete}

\begin{verbete}{左边}{zuo3bian5}{5,5}
  \significado{s.}{esquerda; lado esquerdo}
\end{verbete}

\begin{verbete}{左面}{zuo3mian4}{5,9}
  \significado{s.}{esquerda; lado esquerdo}
\end{verbete}

\begin{verbete}{左派}{zuo3pai4}{5,9}
  \significado{s.}{esquerda (política); esquerdista}
\end{verbete}

\begin{verbete}{左倾}{zuo3qing1}{5,10}
  \significado{s.}{esquerdista; progressivo}
\end{verbete}

\begin{verbete}{左袒}{zuo3tan3}{5,10}
  \significado{v.}{ser tendencioso; ser parcial para; favorecer um lado; tomar partido com}
\end{verbete}

\begin{verbete}{左舷}{zuo3xian2}{5,11}
  \significado{s.}{porto (lado de um navio)}
\end{verbete}

\begin{verbete}{左翼}{zuo3yi4}{5,17}
  \significado{s.}{esquerda (política)}
\end{verbete}

\begin{verbete}{左右}{zuo3you4}{5,5}
  \significado{adv.}{cerca de; aproximadamente}
\end{verbete}

\begin{verbete}{坐}{zuo4}{7}[Radical 土]
  \significado*{s.}{sobrenome Zuo}
  \significado{v.}{sentar-se; andar de carro, ônibus, trem, avião, etc.}
\end{verbete}

\begin{verbete}{坐标}{zuo4biao1}{7,9}
  \significado{s.}{coordenada (geometria)}
\end{verbete}

\begin{verbete}{坐车}{zuo4che1}{7,4}
  \significado{v.}{andar de carro, ônibus, trem, etc.}
\end{verbete}

\begin{verbete}{坐垫}{zuo4dian4}{7,9}
  \significado[块]{s.}{assento (motocicleta); almofada}
\end{verbete}

\begin{verbete}{坐好}{zuo4hao3}{7,6}
  \significado{v.}{sentar-se corretamente; sentar direito}
\end{verbete}

\begin{verbete}{坐享}{zuo4xiang3}{7,8}
  \significado{v.}{curtir algo sem levantar um dedo}
\end{verbete}

\begin{verbete}{座标}{zuo4biao1}{10,9}
  \variante{坐标}
\end{verbete}

\begin{verbete}{座位}{zuo4wei4}{10,7}
  \significado[个]{s.}{assento; lugar}
\end{verbete}

\begin{verbete}{座子}{zuo4zi5}{10,3}
  \significado{s.}{soquete; pedestal; sela}
\end{verbete}

\begin{verbete}{做}{zuo4}{11}[Radical 人]
  \significado{v.}{fazer}
\end{verbete}

\begin{verbete}{做法}{zuo4fa3}{11,8}
  \significado[个]{s.}{método para fazer; prática; receita; maneira de lidar com algo; método de trabalho}
\end{verbete}

\begin{verbete}{做饭}{zuo4fan4}{11,7}
  \significado{v.}{preparar uma refeição; cozinhar}
\end{verbete}

\begin{verbete}{做活}{zuo4huo2}{11,9}
  \significado{v.}{trabalhar para ganhar a vida (especialmente de mulher costureira)}
\end{verbete}

\begin{verbete}{做生活}{zuo4sheng1huo2}{11,5,9}
  \significado{v.}{fazer tabalhos manuais}
\end{verbete}

\begin{verbete}{做戏}{zuo4xi4}{11,6}
  \significado{v.}{atuar em uma peça; fazer uma peça}
\end{verbete}

\begin{verbete}{做眼}{zuo4yan3}{11,11}
  \significado{v.}{agir como um guia; trabalhar como espião}
\end{verbete}

\begin{verbete}{做作}{zuo4zuo5}{11,7}
  \significado{adj.}{afetado; artificial}
\end{verbete}

\begin{verbete}{酢}{zuo4}{12}[Radical 酉]
  \significado{v.}{brindar o anfitrião com vinho}
\end{verbete}

%%%%% EOF %%%%%


%\printindex

\end{document}
