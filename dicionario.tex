%%%%%%%%%%%%%%%%%%%%%%%%%%%%%%%%%%%%%%%%%
% XeTeX
%
% Dicionário Chinês -> Português
% Autor: Luiz Eduardo Roncato Cordeiro
%
% Licença:
% CC BY-NC-SA 3.0 (http://creativecommons.org/licenses/by-nc-sa/3.0/)
%%%%%%%%%%%%%%%%%%%%%%%%%%%%%%%%%%%%%%%%%

%\documentclass[a4paper,12pt,twoside,openany,draft]{memoir}
\documentclass[a4paper,12pt,twoside,openany]{memoir}

\usepackage[usenames,dvipsnames]{color}
\usepackage[utf8]{inputenc}
\usepackage[brazil]{babel}
\usepackage{fontspec}
\usepackage{xltxtra}
\usepackage{xeCJK}
\usepackage{xpinyin}
\usepackage{xunicode}
\usepackage{xltxtra}
\usepackage{multicol}
\usepackage{fancyhdr}
\usepackage{imakeidx}
\usepackage{ifthen}
\usepackage{tocloft}
\usepackage{xparse}
\usepackage[inline]{enumitem}
\usepackage{zhnumber}
\usepackage{wasysym}
\usepackage{ragged2e}
\usepackage[explicit]{titlesec}
\usepackage{tikz}

\setCJKmainfont{Noto Serif CJK SC}
\setCJKsansfont{Noto Sans Mono CJK SC}

%\setCJKmainfont{AR PL UKai CN}
%\setCJKsansfont{AR PL UMing CN}

\makeindex[columns=3, title=Índice, intoc]

\setlength{\parindent}{0em}
\setlength{\parskip}{0.5em}
\setlength{\columnsep}{1em}
\setlength{\columnseprule}{0.2mm}

\xpinyinsetup{ratio={.6},vsep={1.2em}}

% Headers & footers
\fancyhead[L]{\rightmark} % Top left header
\fancyhead[R]{\leftmark}  % Top right header
\renewcommand{\headrulewidth}{1.4pt} % Rule under the header
\fancyfoot[C]{\thepage} % Bottom center footer
\renewcommand{\footrulewidth}{1.4pt} % Rule under the header
\pagestyle{fancy} % Use the custom headers and footers throughout the document

\setlength{\headheight}{16pt}
\addtolength{\topmargin}{-0.5pt}

\NewDocumentCommand\mylist{>{\SplitList{;}}m}{
  \begin{enumerate*}[left=0pt,mode=unboxed,itemjoin={{; }}]
  \ProcessList{#1}{\insertitem}
  \end{enumerate*}
}
\newcommand\insertitem[1]{\item #1}

\DeclareDocumentEnvironment{verbete}{ o m o m o }{%
  \leavevmode%
  \markboth{#2{\small«\pinyin{#4}»}}{#2{\small«\pinyin{#4}»}}
  \index{#2«\pinyin{#4}»}
  \begin{minipage}[t][][t]{\linewidth}%
  {\Large$\bullet$ \textbf{#2}}%
  \label{\mdfivesum{#2::#4}}%
  \IfValueTF{#3}{%
    \IfValueT{#1}{\hfill\textsuperscript{\tiny(#1画)}}%
    \IfValueTF{#5}{%
      {#3}{«\pinyin{#4}»}{#5}%
    }{%
      {#3}{«\pinyin{#4}»}%
    }
  }{%
    \IfValueTF{#5}{%
      \IfValueT{#1}{\hfill\textsuperscript{\tiny(#1画)}}%
      { «\pinyin{#4}»}{#5}%
    }{%
      { «\pinyin{#4}»}\IfValueT{#1}{\hfill\textsuperscript{\tiny(#1画)}}%
    }
  }%
  \newline
}{%
  \end{minipage}
}

\DeclareDocumentCommand{\significado}{ o m m }{%
  $\hookrightarrow$\ (\textit{#2})%
  \IfValueT{#1}{{ [p.c.:#1]}}%
  {\mylist{#3}}%
  \newline
}

\DeclareDocumentCommand{\veja}{ m m }{%
  $\hookrightarrow$\ Veja: #1«\pinyin{#2}» na pág. \pageref{\mdfivesum{#1::#2}}\newline%
}

\newcommand{\p}[1]{«\pinyin{#1}»}
\newcommand{\e}[1]{\textcolor{OliveGreen}{#1}}

\makeatletter
\let\old@makechapterhead\@makechapterhead
% Taken from http://mirrors.ctan.org/macros/latex/unpacked/report.cls
\def\fake@makechapterhead#1{%
  \vspace*{50\p@}%
  {\parindent \z@ \raggedright \normalfont
    \ifnum \c@secnumdepth >\m@ne
        \huge\bfseries \strut%\@chapapp\space \thechapter
        \par\nobreak
        \vskip 20\p@
    \fi
    \interlinepenalty\@M
    \Huge \bfseries #1\par\nobreak
    \vskip 40\p@
  }
  \markboth{#1}{\thechapter}
}
\newcommand{\newchapterhead}{\let\@makechapterhead\fake@makechapterhead}
\newcommand{\restorechapterhead}{\let\@makechapterhead\old@makechapterhead}
\makeatother

\DeclareRobustCommand{\&}{%
    \ifdim\fontdimen1\font>0pt
        \textsl{\symbol{`\&}}%
    \else
        \symbol{`\&}%
    \fi
}

\titlespacing{\section}{0pc}{0ex}{0pc}
\titleformat{\section}[hang]{\Large\bfseries}{}{0pt}{%
  \tikz{%
    \node[draw,rounded corners=1mm,text depth=0.2ex,line width=1pt,    anchor=west]{\Large\bfseries#1};
  }%
}

%%%
%%% Documento começa aqui!
%%%

\begin{document}

\newchapterhead

\begin{titlingpage}
	
	\raggedleft
	
	\rule{1pt}{\textheight}
	\hspace{0.05\textwidth}
	\parbox[b]{0.75\textwidth}{
		
		{\Huge\bfseries 汉葡词典}\\[2\baselineskip] % Title
		{\large\textsc{Dicionário Chinês-Português para o\\%
     Curso de Chinês do Instituto Confúcio}}\\%
     [4\baselineskip]
    {\Large\textsc{罗学凯}\\%
     \tiny Luiz Eduardo Roncato Cordeiro\\%
     Aluno do Instituto Confúcio da UNESP}
		
		\vspace{0.5\textheight}
		
		{\noindent \zhtoday}\\[\baselineskip]
	}

\end{titlingpage}

\tableofcontents

\newpage

\chapter{Termos Gramaticais Chineses}

\begin{tabular}{lll}
substantivo/nome       & \textbf{s.}        & 名词 \\
palavra de lugar       & \textbf{p.d.l.}    & 处所词 \\
palavra de localização & \textbf{p.l.}      & 方位词 \\
palavra de tempo       & \textbf{p.t.}      & 时间词 \\
verbo                  & \textbf{v.}        & 动词 \\
verbo direcional       & \textbf{v.d.}      & 趣向\hspace{1em}动词 \\
verbo optativo         & \textbf{v.o.}      & 能缘\hspace{1em}动词 \\
adjetivo               & \textbf{adj.}      & 形容词 \\
numeral                & \textbf{num.}      & 数词 \\
palavra classificadora  & \textbf{p.c.}      & 两量词 \\
pronome                & \textbf{pron.}     & 代词 \\
interrogativo          & \textbf{interr.}   & 疑问词 \\
advérbio               & \textbf{adv.}      & 副词 \\
preposição             & \textbf{prep.}     & 介词 \\
conjunção              & \textbf{conj.}     & 连词 \\
partícula              & \textbf{part.}     & 助词 \\
sujeito                & \textbf{suj.}      & 主语 \\
objeto                 & \textbf{obj.}      & 宾语 \\
atributo               & \textbf{atrib.}    & 定语 \\
adjunto adverbial      & \textbf{a.adv.}    & 状语 \\
complemento            & \textbf{compl.}    & 补语 \\
verbo+complemento      & \textbf{v.+compl.} & 动宾式\hspace{1em}离合词 \\
expressão idiomática   & \textbf{expr.}     & \\
interjeição            & \textbf{interj.}   & \\
\end{tabular}

\newpage

\chapter{汉葡词典}

%%%
%%% Estou ordenando as palavras em ordem alfabética por pinyin.
%%% Obs: Para as palavras diferentes com o mesmo pinyin, a que tem o menor 
%%%      número de traços vem antes.
%%%

\clearpage
\begin{multicols}{2}
%%%
%%% A
%%%
\section*{A}
\addcontentsline{toc}{section}{A}

\begin{verbete}[a0]{啊}[10]
\begin{pronuncia}{a0}
\significado{part.}{ partícula modal usada no final de sentença mostrando afirmação, aprovação ou
  consentimento}
\end{pronuncia}
\begin{pronuncia}{a1}
\significado*{}{ 啊\p{a1} }
\end{pronuncia}
\begin{pronuncia}{a2}
\significado*{}{ 啊\p{a2} }
\end{pronuncia}
\begin{pronuncia}{a3}
\significado*{}{ 啊\p{a3} }
\end{pronuncia}
\begin{pronuncia}{a4}
\significado*{}{ 啊\p{a4} }
\end{pronuncia}
\end{verbete}

\begin{verbete}[a1]{啊}[10]
\begin{pronuncia}{a1}
\significado{interj.}{ Ah!; Oh!; interjeição de surpresa }
\end{pronuncia}
\begin{pronuncia}{a0}
\significado*{}{ 啊\p{a0} }
\end{pronuncia}
\begin{pronuncia}{a2}
\significado*{}{ 啊\p{a2} }
\end{pronuncia}
\begin{pronuncia}{a3}
\significado*{}{ 啊\p{a3} }
\end{pronuncia}
\begin{pronuncia}{a4}
\significado*{}{ 啊\p{a4} }
\end{pronuncia}
\end{verbete}

\begin{verbete}[a1ya1]{啊呀}[10;7]
\begin{pronuncia}{a1ya1}
\significado{interj.}{ Oh meu Deus!; interjeição de surpresa }
\end{pronuncia}
\end{verbete}

\begin{verbete}[a1yo0]{啊哟}[10;9]
\begin{pronuncia}{a1yo0}
\significado{interj.}{ Meu Deus!; Oh!; Ai! ; interjeição de surpresa ou dor }
\end{pronuncia}
\end{verbete}

\begin{verbete}[a2]{啊}[10]
\begin{pronuncia}{a2}
\significado{interj.}{ Eh?; Que?; interjeição expressando dúvida ou exigindo resposta }
\end{pronuncia}
\begin{pronuncia}{a0}
\significado*{}{ 啊\p{a0} }
\end{pronuncia}
\begin{pronuncia}{a1}
\significado*{}{ 啊\p{a1} }
\end{pronuncia}
\begin{pronuncia}{a3}
\significado*{}{ 啊\p{a3} }
\end{pronuncia}
\begin{pronuncia}{a4}
\significado*{}{ 啊\p{a4} }
\end{pronuncia}
\end{verbete}

\begin{verbete}[a3]{啊}[10]
\begin{pronuncia}{a3}
\significado{interj.}{ Eh?; Meu!; E aí?; interjeição de surpresa ou dúvida }
\end{pronuncia}
\begin{pronuncia}{a0}
\significado*{}{ 啊\p{a0} }
\end{pronuncia}
\begin{pronuncia}{a1}
\significado*{}{ 啊\p{a1} }
\end{pronuncia}
\begin{pronuncia}{a2}
\significado*{}{ 啊\p{a2} }
\end{pronuncia}
\begin{pronuncia}{a4}
\significado*{}{ 啊\p{a4} }
\end{pronuncia}
\end{verbete}

\begin{verbete}[a4]{啊}[10]
\begin{pronuncia}{a4}
\significado{interj.}{ Ah!, OK!; Oh, é você!; Hum!; expressão de reconhecimento; 
  interjeição de acordo}
\begin{pronuncia}{a0}
\significado*{}{ 啊\p{a0} }
\end{pronuncia}
\begin{pronuncia}{a1}
\significado*{}{ 啊\p{a1} }
\end{pronuncia}
\begin{pronuncia}{a2}
\significado*{}{ 啊\p{a2} }
\end{pronuncia}
\begin{pronuncia}{a3}
\significado*{}{ 啊\p{a3} }
\end{pronuncia}
\end{pronuncia}
\end{verbete}

\begin{verbete}[ai3]{矮}[13]
\begin{pronuncia}{ai3}
\significado{adj.}{
baixo em estatura, dimensão, grau ou ranque; curto (em comprimento) }
\end{pronuncia}
\end{verbete}

\begin{verbete}[ai3deng4]{矮凳}[13;14]
\begin{pronuncia}{ai3deng4}
\significado{s.}{ banquinho baixo }
\end{pronuncia}
\end{verbete}

\begin{verbete}[ai3lin2]{矮林}[13;8]
\begin{pronuncia}{ai3lin2}
\significado{adj.}{ mato }
\end{pronuncia}
\end{verbete}

\begin{verbete}[ai3pang4]{矮胖}[13;9]
\begin{pronuncia}{ai3pang4}
\significado{adj.}{ atarracado; rechonchudo; curto e robusto }
\end{pronuncia}
\end{verbete}

\begin{verbete}[ai3ren2]{矮人}[13;2]
\begin{pronuncia}{ai3ren2}
\significado{adj.}{ anão }
\end{pronuncia}
\end{verbete}

\begin{verbete}[ai3shu4]{矮树}[13;9]
\begin{pronuncia}{ai3shu4}
\significado{s.}{ arbusto; árvore pequena }
\end{pronuncia}
\end{verbete}

\begin{verbete}[ai3xiao3]{矮小}[13;3]
\begin{pronuncia}{ai3xiao3}
\significado{adj.}{ baixo e pequeno; curto e pequeno; subdimensionado }
\end{pronuncia}
\end{verbete}

\begin{verbete}[ai3xing1]{矮星}[13;9]
\begin{pronuncia}{ai3xing1}
\significado{s.}{ estrela anã }
\end{pronuncia}
\end{verbete}

\begin{verbete}[ai3zi0]{矮子}[13;3]
\begin{pronuncia}{ai3zi0}
\significado{s.}{ pessoa baixa; anão }
\end{pronuncia}
\end{verbete}

\begin{verbete}[ai4]{爱}[10]
\begin{pronuncia}{ai4}
\significado[个]{s.}{ amor; afeição }
\significado{v.}{amar; ter afeição a; gostar de; 
  inclinado a (fazer alguma coisa); 
  tender (a acontecer)
}
\end{pronuncia}
\end{verbete}

\begin{verbete}[ai4ai0]{爱爱}[10;10]
\begin{pronuncia}{ai4ai0}
\significado{v.}{coloquial: fazer amor}
\end{pronuncia}
\end{verbete}

\begin{verbete}[ai4fu3]{爱抚}[10;7]
\begin{pronuncia}{ai4fu3}
\significado{v.}{acariciar;cuidar (com ternura)}
\end{pronuncia}
\end{verbete}

\begin{verbete}[ai4hao4]{爱好}[10;6]
\begin{pronuncia}{ai4hao4}
\significado[个]{s.}{ passatempo; interesse }
\significado{v.}{ ter prazer em; gostar de; ter algo como hobby ; apetite por }
\end{pronuncia}
\end{verbete}

\begin{verbete}[ai4hao4zhe3]{爱好者}[10;6;8]
\begin{pronuncia}{ai4hao4zhe3}
\significado{s.}{ amador; entusiasta; fã; amante de arte, esportes, etc }
\end{pronuncia}
\end{verbete}

\begin{verbete}[ai4ren0]{爱人}[10;2]
\begin{pronuncia}{ai4ren0}
\significado[个]{s.}{ cônjuge; amante }
\end{pronuncia}
\end{verbete}

\begin{verbete}[an1jing4]{安静}[6;14]
\begin{pronuncia}{an1jing4}
\significado{adj.}{ tranquilo; pacífico; calmo }
\end{pronuncia}
\end{verbete}

\begin{verbete}[an1pai2]{安排}[6;11]
\begin{pronuncia}{an1pai2}
\significado{s.}{ arranjos; planos }
\significado{v.}{ organizar; programar; fazer planos }
\end{pronuncia}
\end{verbete}

%%%%% EOF %%%%%

%%%
%%% B
%%%
\section*{B}
\addcontentsline{toc}{section}{B}

\begin{verbete}[7]{吧}{ba0}
  \significado{part.}{partícula modal indicando sugestão ou suposição; ...eu presumo.; ...OK?; ...certo?}
  \veja{吧}{ba1}
  \veja{吧}{bia1}
\end{verbete}

\begin{verbete}[2]{八}{ba1}
  \significado{num.}{8, oito}
\end{verbete}

\begin{verbete}[2;2;4]{八八六}{ba1ba1liu4}
  \significado{expr.}{Bye bye! (em salas de bate-papo e mensagens de texto)}
\end{verbete}

\begin{verbete}[7]{吧}{ba1}
  \significado{s.}{bar (servindo bebidas ou fornecendo acesso à \emph{Internet}); onomatopéia: Bang!}
  \significado{v.}{soprar (em um cachimbo, etc.)}
  \veja{吧}{ba0}
  \veja{吧}{bia1}
\end{verbete}

\begin{verbete}[4;6]{巴西}{Ba1xi1}
  \significado{s.}{Brasil}
\end{verbete}

\begin{verbete}[4;6;2]{巴西人}{ba1xi1ren2}
  \significado[个,位]{s.}{brasileiro; nascido no Brasil}
\end{verbete}

\begin{verbete}[4;6;9;14]{巴西战舞}[\\]{ba1xi1zhan4wu3}
  \significado{s.}{capoeira}
\end{verbete}

\begin{verbete}[8]{爸}{ba4}
  \significado[个,位]{s.}{pai}
\end{verbete}

\begin{verbete}[8;8]{爸爸}{ba4ba0}
  \significado[个,位]{s.}{papai, pai (informal)}
\end{verbete}

\begin{verbete}[8;6]{爸妈}{ba4ma1}
  \significado{s.}{pai e mãe}
\end{verbete}

\begin{verbete}[5]{白}{bai2}
  \significado{adj.}{branco; claro; puro; límpido; simples; em branco; grátis}
  \significado{adv.}{em vão; sem propósito; por nada}
  \significado{s.}{parte falada na ópera; diálogo; dialeto}
\end{verbete}

\begin{verbete}[5;11]{白菜}{bai2cai4}
  \significado[棵,个]{s.}{acelga; repolho chinês}
\end{verbete}

\begin{verbete}[5;11;5]{白蛋白}{bai2dan4bai2}
  \significado{s.}{albumina}
\end{verbete}

\begin{verbete}[5;12]{白鹄}{bai2hu2}
  \significado{s.}{cisne branco}
\end{verbete}

\begin{verbete}[5;8]{白拣}{bai2jian3}
  \significado{s.}{uma escolha barata}
  \significado{v.}{escolher algo que não custa nada}
\end{verbete}

\begin{verbete}[5;11;2]{白萝卜}{bai2luo2bo5}
  \significado{s.}{rabanete branco}
\end{verbete}

\begin{verbete}[5;6]{白色}{bai2se4}
  \significado{s.}{cor branca}
\end{verbete}

\begin{verbete}[5;4]{白天}{bai2tian1}
  \significado{p.t.}{dia; de dia}
  \significado[个]{s.}{dia}
\end{verbete}

\begin{verbete}[5;4]{白苋}{bai2xian4}
  \significado{s.}{amaranto branco; brotos e folhas tenras de espinafre chinês usados como alimento}
\end{verbete}

\begin{verbete}[6]{百}{bai3}
  \significado{num.}{100, cem; centena; cento}
\end{verbete}

\begin{verbete}[6;4]{百分}{bai3fen1}
  \significado{num.}{por cento}
  \significado{s.}{porcentagem}
\end{verbete}

\begin{verbete}[13]{搬}{ban1}
  \significado{v.}{copiar indiscriminadamente; mover-se (ou seja, mudar-se); mover-se (algo relativamente pesado ou volumoso); mudar}
\end{verbete}

\begin{verbete}[13;6]{搬动}{ban1dong4}
  \significado{v.}{mover-se (alguma coisa); se mudar}
\end{verbete}

\begin{verbete}[13;10]{搬家}{ban1jia1}
  \significado{s.}{mudança}
  \significado{v.+compl.}{mudar-se de casa}
\end{verbete}

\begin{verbete}[13;3]{搬口}{ban1kou3}
  \significado{v.}{tagarelar; transmitir histórias (idioma); semear dissensão; contar histórias}
\end{verbete}

\begin{verbete}[13;7]{搬弄}{ban1nong4}
  \significado{v.}{causar problemas; mexer com alguém; mostrar (o que se pode fazer)}
\end{verbete}

\begin{verbete}[13;7]{搬运}{ban1yun4}
  \significado{s.}{frete; transporte}
  \significado{v.}{carregar; transportar}
\end{verbete}

\begin{verbete}[13;7]{搬走}{ban1zou3}
  \significado{v.}{carregar}
\end{verbete}

\begin{verbete}[4]{办}{ban4}
  \significado{v.}{lidar com; lidar; gerenciar; congigurar}
\end{verbete}

\begin{verbete}[4;8]{办法}{ban4fa3}
  \significado[条,个]{s.}{meio (de se fazer alguma coisa); método}
\end{verbete}

\begin{verbete}[4;4;9]{办公室}{ban4gong1shi4}
  \significado[间]{s.}{gabinete; escritório}
\end{verbete}

\begin{verbete}[5]{半}{ban4}
  \significado{adj.}{incompleto}
  \significado{adv.}{semi-}
  \significado{num.}{(depois de um número) ``e meio''}
  \significado{s.}{metade}
\end{verbete}

\begin{verbete}[5;11]{半球}{ban4qiu2}
  \significado{s.}{hemisfério}
\end{verbete}

\begin{verbete}[5;9]{半音}{ban4yin1}
  \significado{s.}{semitom}
\end{verbete}

\begin{verbete}[9]{帮}{bang1}
  \significado{p.c.}{para alguém (como uma ajuda)}
  \significado{s.}{gangue; grupo; contratado (como trabalhador); camada externa; festa; sociedade secreta}
  \significado{v.}{ajudar; apoiar}
\end{verbete}

\begin{verbete}[9;11]{帮教}{bang1jiao4}
  \significado{v.}{orientar}
\end{verbete}

\begin{verbete}[9;7]{帮佣}{bang1yong1}
  \significado{s.}{ajudante doméstico; servo}
\end{verbete}

\begin{verbete}[9;7]{帮助}{bang1zhu4}
  \significado[种]{s.}{ajuda; assistência}
  \significado{v.}{ajudar; dar assistência}
\end{verbete}

\begin{verbete}[5]{包}{bao1}
  \significado{p.c.}{pacotes, sacos, sacolas, embrulhos}
  \significado[个,只]{s.}{bolsa; pacote; recipiente; embrulho}
  \significado{v.}{contratar (para ou para); cobrir; segurar ou abraçar; incluir; para assumir o comando; embrulhar}
\end{verbete}

\begin{verbete}[5;4]{包办}{bao1ban4}
  \significado{v.}{para comandar todo o show; comprometer-se a fazer tudo sozinho}
\end{verbete}

\begin{verbete}[5;3]{包干}{bao1gan1}
  \significado{s.}{tarefa alocada}
  \significado{v.}{ter a responsabilidade total sobre um trabalho}
\end{verbete}

\begin{verbete}[5;9]{包括}{bao1kuo4}
  \significado{v.}{compreender; consistir em; incluir; incorporar; envolver}
\end{verbete}

\begin{verbete}[5;3]{包子}{bao1zi0}
  \significado[个]{s.}{pão recheado cozido no vapor}
\end{verbete}

\begin{verbete}[5;10]{包租}{bao1zu1}
  \significado{s.}{aluguel fixo para terras agrícolas}
  \significado{v.}{fretar; alugar; alugar um terreno ou uma casa para subarrendar}
\end{verbete}

\begin{verbete}[7]{报}{bao4}
  \significado[份,张]{s.}{jornal; recompensa; relatório; vingança}
  \significado{v.}{anunciar; informar}
\end{verbete}

\begin{verbete}[7;13]{报酬}{bao4chou0}
  \significado{s.}{recompensa; remuneração}
\end{verbete}

\begin{verbete}[8]{杯}{bei1}
  \significado{p.c.}{para certos recipientes de líquidos: copo, xícara, etc.}
  \significado{s.}{copo; taça; xícara; copa troféu}
\end{verbete}

\begin{verbete}[8;8]{杯具}{bei1ju4}
  \significado{s.}{parachoque; fiasco; gíria: tragédia}
\end{verbete}

\begin{verbete}[8;3]{杯子}{bei1zi0}
  \significado[个,只]{s.}{copo; caneca; xícara; taça}
\end{verbete}

\begin{verbete}[5]{北}{bei3}
  \significado{p.d.l.}{norte}
  \significado{v.}{ser derrotado (clássico)}
\end{verbete}

\begin{verbete}[5;5]{北边}{bei3bian0}
  \significado{p.l.}{lado norte; ao norte de}
\end{verbete}

\begin{verbete}[5;4]{北方}{bei3fang1}
  \significado{p.l.}{norte; a parte norte de um país}
\end{verbete}

\begin{verbete}[5;8]{北京}{Bei3jing1}
  \significado{s.}{Beijing (Pequim); Capital da China}
\end{verbete}

\begin{verbete}[5;9]{北面}{bei3mian4}
  \significado{p.l.}{lado norte}
\end{verbete}

\begin{verbete}[9]{背}{bei4}
  \significado{p.l.}{a parte de trás de um corpo ou objeto}
  \significado{s.}{costas; gíria: azarado}
  \significado{v.}{esconder algo de; decorar; recitar de memória; virar as costas}
\end{verbete}

\begin{verbete}[10;3]{被子}{bei4zi0}
  \significado[床]{s.}{colcha}
\end{verbete}

\begin{verbete}[5]{本}{ben3}
  \significado{adv.}{inerente; originalmente}
  \significado{p.c.}{para livros, dicionários, periódicos, arquivos, etc.}
  \significado{s.}{origem; fonte; raiz}
\end{verbete}

\begin{verbete}[5;3]{本子}{ben3zi0}
  \significado[本]{s.}{caderno}
\end{verbete}

\begin{verbete}[14;3]{鼻子}{bi2zi0}
  \significado[个,只]{s.}{nariz}
\end{verbete}

\begin{verbete}[4]{比}{bi3}
  \significado{part.}{partícula usada para comparação (superioridade)}
  \significado{prep.}{que; do que}
  \significado{s.}{razão (taxa)}
  \significado{v.}{comparar; contrastar; gesticular (com as mãos)}
\end{verbete}

\begin{verbete}[4;10]{比较}{bi3jiao4}
  \significado{adv.}{comparativamente; relativamente}
  \significado{s.}{comparação; relativamente}
  \significado{v.}{comparar; contrastar}
\end{verbete}

\begin{verbete}[4;11;9]{比萨饼}{bi3sa4bing3}
  \significado[张]{s.}{pizza}
\end{verbete}

\begin{verbete}[4;14]{比赛}{bi3sai4}
  \significado[场,次]{s.}{competição; concurso}
  \significado{v.}{competir}
\end{verbete}

\begin{verbete}[10]{笔}{bi3}
  \significado{p.c.}{para somas de dinheiro, negócios}
  \significado[支,枝]{s.}{caneta; lápis}
\end{verbete}

\begin{verbete}[7]{吧}{bia1}
  \significado{s.}{bar (servindo bebidas ou fornecendo acesso à \emph{Internet}); onomatopéia: Smack! (para beijo)}
  \significado{v.}{soprar (em um cachimbo, etc.)}
  \veja{吧}{ba0}
  \veja{吧}{ba1}
\end{verbete}

\begin{verbete}[5]{边}{bian0}
  \significado{s.}{sufixo de uma palavra de localidade}
\end{verbete}

\begin{verbete}[5]{边}{bian1}
  \significado{adv.}{simultaneamente}
  \significado[个]{s.}{fronteira; limite; borda; margem; lado}
\end{verbete}

\begin{verbete}[9;10]{标准}{biao1zhun3}
  \significado{adj.}{criterioso; padronizado; normatizado}
  \significado[个]{s.}{critério; padrão (oficial); norma}
\end{verbete}

\begin{verbete}[12]{遍}{bian4}
  \significado{p.l.}{em todos os lugares; por toda parte}
  \significado{p.c.}{para a repetição de ações de leitura, fala ou escrita}
\end{verbete}

\begin{verbete}[7]{别}{bie2}
  \significado{adv.}{nada de (pedir a alguém para não fazer); não}
  \significado{pron.}{outro}
  \significado{v.}{classificar; separar; distinguir; partir; deixar; fixar; colar alguma coisa em}
\end{verbete}

\begin{verbete}[7;8]{别的}{bie2de0}
  \significado{pron.}{outro}
\end{verbete}

\begin{verbete}[7;2]{别人}{bie2ren0}
  \significado{pron.}{outra pessoa; outro povo; outros}
\end{verbete}

\begin{verbete}[6]{冰}{bing1}
  \significado{adj.}{hostil; gelado}
  \significado[块]{s.}{gelo; gíria: metanfetamina}
  \significado{v.}{sentir frio; para relaxar alguma coisa}
\end{verbete}

\begin{verbete}[6;4;11;6]{冰天雪地}[\\]{bing1tian1-\ xue3di4}
  \significado{expr.}{um mundo de gelo e neve}
\end{verbete}

\begin{verbete}[6;11]{冰球}{bing1qiu2}
  \significado{s.}{hóquei no gelo}
\end{verbete}

\begin{verbete}[10]{病}{bing4}
  \significado[场]{s.}{doença}
  \significado{v.}{adoecer; estar doente}
\end{verbete}

\begin{verbete}[11;12]{菠菜}{bo1cai4}
  \significado[棵]{s.}{espinafre}
\end{verbete}

\begin{verbete}[12;8;11]{博物馆}[\\]{bo2wu4guan3}
  \significado{s.}{museu}
\end{verbete}

\begin{verbete}[11;3]{脖子}{bo2zi0}
  \significado[个]{s.}{pescoço}
\end{verbete}

\begin{verbete}[4]{不}{bu0}
  \significado{adv.}{não (em expressões ``v.$+$不$+$v.'')}
  \veja{不}{bu2}
  \veja{不}{bu4}
\end{verbete}

\begin{verbete}[4]{不}[\\]{bu2}[ (antes de quarto tom)]
  \significado{adv.}{não}
  \veja{不}{bu0}
  \veja{不}{bu4}
\end{verbete}

\begin{verbete}[4;13]{不错}{bu2cuo4}
  \significado{adj.}{correto; não (é) mau; bastante bom; certo}
\end{verbete}

\begin{verbete}[4;6]{不过}{bu2guo4}
  \significado{conj.}{mas; contudo; no entanto}
\end{verbete}

\begin{verbete}[4;9;4]{不客气}{bu2ke4qi0}
  \significado{expr.}{de nada; não há de que}
\end{verbete}

\begin{verbete}[4;13;8]{不像话}[\\]{bu2xiang4hua4}
  \significado{expr.}{sem razão; demasiado irracionável}
\end{verbete}

\begin{verbete}[4;9]{不要}{bu2yao4}
  \significado{adv.}{nada de (pedir a alguém para não fazer); não}
\end{verbete}

\begin{verbete}[4;5]{不用}{bu2yong4}
  \significado{v.o.}{não precisar}
\end{verbete}

\begin{verbete}[4]{不}{bu4}
  \significado{adv.}{não}
  \veja{不}{bu0}
  \veja{不}{bu2}
\end{verbete}

\begin{verbete}[4;6]{不同}{bu4tong2}
  \significado{adj.}{diferente; distinto}
\end{verbete}

%%%%% EOF %%%%%

%%%
%%% C
%%%
\section*{C}
\addcontentsline{toc}{section}{C}
\begin{multicols*}{2}

\begin{verbete}[菜]{cai4}
\significado{cai4}{n.}{
    hortaliça; verdura|
    prato
}
\end{verbete}

\begin{verbete}[菜单]{cai4dan1}
\significado{cai4dan1}{n.}{
    menu; ementa; cardápio
}
\end{verbete}

\begin{verbete}[草]{cao3}
\significado{cao3}{n.}{
    erva
}
\end{verbete}

\begin{verbete}[草地]{cao3di4}
\significado{cao3di4}{n.}{
    relva; pastagem
}
\end{verbete}

\begin{verbete}[参观]{can1guan3}
\significado{can1guan3}{v.}{
    visitar
}
\end{verbete}

\begin{verbete}[参加]{can1jia1}
\significado{can1jia1}{v.}{
    juntar; participar
}
\end{verbete}

\begin{verbete}[餐厅]{can1ting1}
\significado{can1ting1}{n.}{
    cantina; sala de jantar
}
\end{verbete}

\begin{verbete}[厕所]{ce4suo3}
\significado{ce4suo3}{n.}{
    sanitário; toilette
}
\end{verbete}

\begin{verbete}[层]{ceng2}
\significado{ceng2}{p.c.}{
    para andar, piso
}
\end{verbete}

\begin{verbete}[磁带]{ci2dai4}
\significado[盘]{ci2dai4}{n.}{
    cassete
}
\end{verbete}

\begin{verbete}[磁盘]{ci2pan2}
\significado{ci2pan2}{n.}{
    disquete
}
\end{verbete}

\begin{verbete}[词典]{ci2dian3}
\significado[本]{ci2dian3}{n.}{
    dicionário
}
\end{verbete}

\begin{verbete}[次]{ci4}
\significado{ci4}{p.c.}{
    para frequência (número de vezes)
}
\end{verbete}

\begin{verbete}[茶]{cha2}
\significado{cha2}{n.}{
    chá
}
\end{verbete}

\begin{verbete}[差不多]{cha4bu0duo1}
\significado{cha4bu0duo1}{adj.}{
    mais ou menos
}
\end{verbete}

\begin{verbete}[长]{chang2}
\significado{chang2}{adj.}{
    comprido; longo
}
\end{verbete}

\begin{verbete}[长成]{chang2cheng2}
\significado{chang2cheng2}{n.}{
    Grande Muralha
}
\end{verbete}

\begin{verbete}[常常]{chang2chang2}
\significado{chang2chang2}{adv.}{
    frequentemente; com frequência
}
\end{verbete}

\begin{verbete}[唱]{chang4}
\significado{chang4}{v.}{
    cantar
}
\end{verbete}

\begin{verbete}[唱歌]{chang4ge1}
\significado{chang4ge1}{v.+compl.}{
    cantar
}
\end{verbete}

\begin{verbete}[超市]{chao1shi4}
\significado{chao1shi4}{n.}{
    supermercado
}
\end{verbete}

\begin{verbete}[吵]{chao3}
\significado{chao3}{adj.}{
    barulhento, barulhenta; ruidoso, ruidosa;
}
\end{verbete}

\begin{verbete}[炒]{chao3}
\significado{chao3}{v.}{
    saltear
}
\end{verbete}

\begin{verbete}[车]{che1}
\significado{che1}{n.}{
    veículo; viatura
}
\end{verbete}

\begin{verbete}[车库]{che1ku4}
\significado{che1ku4}{n.}{
    garagem
}
\end{verbete}

\begin{verbete}[车牌]{che1pai2}
\significado{che1pai2}{n.}{
    matrícula; placa de carro
}
\end{verbete}

\begin{verbete}[车水马龙]{che1shui3-ma3long2}
\significado{che1shui3-ma3long2}{}{
    tráfego engarrafado; engarrafamento
}
\end{verbete}

\begin{verbete}[车站]{che1zhan4}
\significado{che1zhan4}{n.}{
    estação; paragem
}
\end{verbete}

\begin{verbete}[衬衫]{chen4shan1}
\significado[件]{chen4shan1}{n.}{
    camisa
}
\end{verbete}

\begin{verbete}[成都]{cheng2du1}
\significado{cheng2du1}{n.}{
    Chengdu
}
\end{verbete}

\begin{verbete}[成绩]{cheng2ji4}
\significado{cheng2ji4}{n.}{
    nota; classificação
}
\end{verbete}

\begin{verbete}[城市]{cheng2shi4}
\significado{cheng2shi4}{n.}{
    cidade
}
\end{verbete}

\begin{verbete}[橙色]{cheng2se4}
\significado{cheng2se4}{n.}{
    cor de laranja
}
\end{verbete}

\begin{verbete}[橙汁]{cheng2zhi1}
\significado{cheng2zhi1}{n.}{
    suco de laranja
}
\end{verbete}

\begin{verbete}[惩罚]{cheng2fa2}
\significado{cheng2fa2}{v.}{
    punir; penalizar
}
\end{verbete}

\begin{verbete}[惩处]{cheng2chu3}
\significado{cheng2chu3}{v.}{
    punir; penalizar
}
\end{verbete}

\begin{verbete}[吃]{chi1}
\significado{chi1}{v.}{
    comer
}
\end{verbete}

\begin{verbete}[迟到]{chi1dao4}
\significado{chi1dao4}{v.}{
    chegar atrasado; tardar
}
\end{verbete}

\begin{verbete}[憧憬]{chong1jing3}
\significado{chong1jing3}{v.}{
    ansiar por; esperar por
}
\end{verbete}

\begin{verbete}[宠物]{chong3wu4}
\significado{chong3wu4}{n.}{
    animal de estimação
}
\end{verbete}

\begin{verbete}[酬劳]{chou2lao2}
\significado{chou2lao2}{n.}{}{
    recompensa
}
\end{verbete}

\begin{verbete}[出]{chu1}
\significado{chu1}{v.d.}{
    sair
}
\end{verbete}

\begin{verbete}[出版]{chu1ban3}
\significado{chu1ban3}{v.}{
    publicar; editar
}
\end{verbete}

\begin{verbete}[出版社]{chu1ban3she4}
\significado{chu1ban3she4}{n.}{
    editora
}
\end{verbete}

\begin{verbete}[出口]{chu1kou3}
\significado{chu1kou3}{n.}{
    exportação
}
\significado{chu1kou3}{v.}{
    exportar
}
\end{verbete}

\begin{verbete}[出去]{chu1qu0}
\significado{chu1qu0}{v.d.}{
    sair; ir para fora
}
\end{verbete}

\begin{verbete}[出站]{chu1zhan4}
\significado{chu1zhan4}{n.}{
    saída da estação
}
\end{verbete}

\begin{verbete}[出租汽车]{chu1zu1qi4che1}
\significado[辆]{chu1zu1qi4che1}{n.}{
    táxi
}
\end{verbete}

\begin{verbete}[穿]{chuan1}
\significado{chuan1}{v.}{
    vestir
}
\end{verbete}

\begin{verbete}[船]{chuan2}
\significado{chuan2}{v.}{
    barco; navio
}
\end{verbete}

\begin{verbete}[传真]{chuan2zhen1}
\significado{chuan2zhen1}{n.}{
    fax; facsímile
}
\end{verbete}

\begin{verbete}[床]{chuang2}
\significado[张]{chuang2}{n.}{
    cama
}
\end{verbete}

\begin{verbete}[春天]{chun1tian1}
\significado{chun1tian1}{n.}{
    primavera
}
\significado{chun1tian1}{p.t.}{
    primavera
}
\end{verbete}

\begin{verbete}[绰号]{chuo4hao4}
\significado{chuo4hao4}{n.}{
    apelido
}
\end{verbete}

\begin{verbete}[聪明]{cong1ming2}
\significado{cong1ming2}{adj.}{
    inteligente; brilhante; esperto
}
\end{verbete}

\begin{verbete}[聪慧]{cong1hui4}
\significado{cong1hui4}{adj.}{
    inteligente; brilhante
}
\end{verbete}

\begin{verbete}[从]{cong2}
\significado{cong2}{prep.}{
    de; desde; a partir de
}
\end{verbete}

\begin{verbete}[醋]{cu4}
\significado{cu4}{n.}{
    vinagre
}
\end{verbete}

\begin{verbete}[错]{cuo4}
\significado{cuo4}{adj.}{
    errado; enganado
}
\end{verbete}

\end{multicols*}

%%%
%%% D
%%%
\section*{D}
\addcontentsline{toc}{section}{D}
%\begin{multicols*}{2}

\begin{verbete}[da3]{打}
\begin{pronuncia}{da3}
\significado{v.}{
jogar
}
\end{pronuncia}
\end{verbete}

\begin{verbete}[da3ban0]{打扮}
\begin{pronuncia}{da3ban0}
\significado{v.}{
arranjar-se; enfeitar-se
}
\end{pronuncia}
\end{verbete}

\begin{verbete}[da3\ dian4hua4]{打电话}
\begin{pronuncia}{da3\ dian4hua4}
\significado{v.}{
ligar; dar um telefonema
}
\end{pronuncia}
\begin{pronuncia}{gei3\ ...\ da3\ dian4hua4}
\significado{expr.}{
telefonar para alguém|veja: 给\ ······\ 打\ 电话
}
\end{pronuncia}
\end{verbete}

\begin{verbete}[da3gong1]{打工}
\begin{pronuncia}{da3gong1}
\significado{v.}{
trabalhar temporariamente para alguém; trabalhar por conta de alguém
}
\end{pronuncia}
\end{verbete}

\begin{verbete}[da3jiao3]{打搅}
\begin{pronuncia}{da3jiao3}
\significado{v.}{
perturbar; incomodar
}
\end{pronuncia}
\end{verbete}

\begin{verbete}[da3qiu2]{打球}
\begin{pronuncia}{da3qiu2}
\significado{v.}{
jogar bola; jogar (futebol, basquetebol, handbol, etc)
}
\end{pronuncia}
\end{verbete}

\begin{verbete}[da3rao3]{打扰}
\begin{pronuncia}{da3rao3}
\significado{v.}{
perturbar; incomodar
}
\end{pronuncia}
\end{verbete}

\begin{verbete}[da3suan4]{打算}
\begin{pronuncia}{da3suan4}
\significado[个]{n.}{
plano
}
\significado{v.}{
pensar; planejar; pretender
}
\end{pronuncia}
\end{verbete}

\begin{verbete}[da3zhen1]{打针}
\begin{pronuncia}{da3zhen1}
\significado{v.+compl.}{
dar injeção; levar injeção
}
\end{pronuncia}
\end{verbete}

\begin{verbete}[da4]{大}
\begin{pronuncia}{da4}
\significado{adj.}{
grande
}
\end{pronuncia}
\end{verbete}

\begin{verbete}[da4fu0]{大夫}
\begin{pronuncia}{da4fu0}
\significado{n.}{
médico; doutor
}
\end{pronuncia}
\end{verbete}

\begin{verbete}[da4gai4]{大概}
\begin{pronuncia}{da4gai4}
\significado{adv.}{
aproximadamente; por volta de
}
\end{pronuncia}
\end{verbete}

\begin{verbete}[da4hai3]{大海}
\begin{pronuncia}{da4hai3}
\significado{n.}{
mar; oceano
}
\end{pronuncia}
\end{verbete}

\begin{verbete}[da4jia1]{大海}
\begin{pronuncia}{da4jia1}
\significado{pron.}{
todos; todas
}
\end{pronuncia}
\end{verbete}

\begin{verbete}[da4suan4]{大蒜}
\begin{pronuncia}{da4suan4}
\significado[瓣,头]{n.}{
alho
}
\end{pronuncia}
\end{verbete}

\begin{verbete}[da4tui3]{大腿}
\begin{pronuncia}{da4tui3}
\significado{n.}{
coxa
}
\end{pronuncia}
\end{verbete}

\begin{verbete}[da4xue2]{大学}
\begin{pronuncia}{da4xue2}
\significado[所]{n.}{
universidade
}
\end{pronuncia}
\end{verbete}

\begin{verbete}[Da4yang2zhou1]{大洋洲}
\begin{pronuncia}{Da4yang2zhou1}
\significado{n.}{
Oceania
}
\end{pronuncia}
\end{verbete}

\begin{verbete}[dai4]{带}
\begin{pronuncia}{dai4}
\significado{v.}{
levar; trazer
}
\end{pronuncia}
\end{verbete}

\begin{verbete}[dai4]{戴}
\begin{pronuncia}{dai4}
\significado{v.}{
usar/vestir (óculos, gravata, relógio de pulso, luvas)|
trazer
}
\end{pronuncia}
\end{verbete}

\begin{verbete}[dan1xin1]{担心}
\begin{pronuncia}{dan1xin1}
\significado{v.}{
preocupar-se; estar preocupado
}
\end{pronuncia}
\end{verbete}

\begin{verbete}[dan4gao1]{蛋糕}
\begin{pronuncia}{dan4gao1}
\significado[块,个]{n.}{
bolo
}
\end{pronuncia}
\end{verbete}

\begin{verbete}[dan4shi4]{但是}
\begin{pronuncia}{dan4shi4}
\significado{conj.}{
mas; contudo
}
\end{pronuncia}
\end{verbete}

\begin{verbete}[dang1ran2]{当然}
\begin{pronuncia}{dang1ran2}
\significado{adv.}{
claro; certamente; com certeza
}
\end{pronuncia}
\end{verbete}

\begin{verbete}[dao3]{倒}
\begin{pronuncia}{dao3}
\significado{v.}{
cair no chão; deitar-se no chão
}
\end{pronuncia}
\end{verbete}

\begin{verbete}[dao4]{到}
\begin{pronuncia}{dao4}
\significado{prep.}{
a; até; para
}
\significado{v.}{
chegar
}
\end{pronuncia}
\end{verbete}

\begin{verbete}[de0]{的}
\begin{pronuncia}{de0}
\significado{part.}{
partícula utilizada em possessivos|
partícula utilizada entre adjetivos e substantivos|
(opcional se substantivo possui apenas um carácter)
}
\end{pronuncia}
\end{verbete}

\begin{verbete}[de0]{得}
\begin{pronuncia}{de0}
\significado{part.}{
partícula estrutural: ligando um verbo à frase seguinte indicando efeito,
grau, possibilidade, etc
}
\end{pronuncia}
\begin{pronuncia}{de2}
\significado{v.}{
obter; ganhar; pegar (uma doença)
}
\end{pronuncia}
\begin{pronuncia}{dei3}
\significado{v.}{
haver de; ter de
}
\end{pronuncia}
\end{verbete}

\begin{verbete}[De2guo2]{德国}
\begin{pronuncia}{De2guo2}
\significado{n.}{
Alemanha
}
\end{pronuncia}
\end{verbete}

\begin{verbete}[de2]{得}
\begin{pronuncia}{de2}
\significado{v.}{
obter; ganhar; pegar (uma doença)
}
\end{pronuncia}
\begin{pronuncia}{de0}
\significado{part.}{
partícula estrutural: ligando um verbo à frase seguinte indicando efeito,
grau, possibilidade, etc
}
\end{pronuncia}
\begin{pronuncia}{dei3}
\significado{v.}{
haver de; ter de
}
\end{pronuncia}
\end{verbete}

\begin{verbete}[de2dao4]{得到}
\begin{pronuncia}{de2dao4}
\significado{v.}{
obter
}
\end{pronuncia}
\end{verbete}

\begin{verbete}[dei3]{得}
\begin{pronuncia}{dei3}
\significado{v.}{
haver de; ter de
}
\end{pronuncia}
\begin{pronuncia}{de0}
\significado{part.}{
partícula estrutural: ligando um verbo à frase seguinte indicando efeito,
grau, possibilidade, etc
}
\end{pronuncia}
\begin{pronuncia}{de2}
\significado{v.}{
obter; ganhar; pegar (uma doença)
}
\end{pronuncia}
\end{verbete}

\begin{verbete}[deng1]{登}
\begin{pronuncia}{deng1}
\significado{v.}{
subir (montanha, cume)
}
\end{pronuncia}
\end{verbete}

\begin{verbete}[deng3]{等}
\begin{pronuncia}{deng3}
\significado{v.}{
esperar
}
\end{pronuncia}
\end{verbete}

\begin{verbete}[di1]{低}
\begin{pronuncia}{di1}
\significado{adj.}{
baixo, baixa
}
\end{pronuncia}
\end{verbete}

\begin{verbete}[di4fang1]{地方}
\begin{pronuncia}{di4fang1}
\significado[处,个,块]{n.}{
lugar; local; sítio
}
\end{pronuncia}
\end{verbete}

\begin{verbete}[di4zhi3]{地址}
\begin{pronuncia}{di4zhi3}
\significado[个]{n.}{
endereço
}
\end{pronuncia}
\end{verbete}

\begin{verbete}[di4tie3]{地铁}
\begin{pronuncia}{di4tie3}
\significado{n.}{
Metrô; metropolitano
}
\end{pronuncia}
\end{verbete}

\begin{verbete}[di4tu2]{地图}
\begin{pronuncia}{di4tu2}
\significado[张,本]{n.}{
mapa
}
\end{pronuncia}
\end{verbete}

\begin{verbete}[di4xia4shi4]{地下室}
\begin{pronuncia}{di4tu2}
\significado{n.}{
subterrâneo; porão
}
\end{pronuncia}
\end{verbete}

\begin{verbete}[di4di0]{弟弟}
\begin{pronuncia}{di4di0}
\significado[个,位]{n.}{
irmão mais novo
}
\end{pronuncia}
\end{verbete}

\begin{verbete}[di4mei4]{弟妹}
\begin{pronuncia}{di4mei4}
\significado{n.}{
esposa do irmão mais novo
}
\end{pronuncia}
\end{verbete}

\begin{verbete}[di4]{第}
\begin{pronuncia}{di4}
\significado{num.}{
prefixo para expressar números ordinais
}
\end{pronuncia}
\end{verbete}

\begin{verbete}[dian3]{点}
\begin{pronuncia}{dian3}
\significado{p.c.}{
hora
}
\end{pronuncia}
\end{verbete}

\begin{verbete}[dian4hua4]{电话}
\begin{pronuncia}{dian4hua4}
\significado[部]{n.}{
telefone
}
\significado[通]{n.}{
chamada telefônica
}
\end{pronuncia}
\end{verbete}

\begin{verbete}[dian4nao3]{电脑}
\begin{pronuncia}{dian4nao3}
\significado[台]{n.}{
computador
}
\end{pronuncia}
\end{verbete}

\begin{verbete}[dian4shi4]{电视}
\begin{pronuncia}{dian4shi4}
\significado[台,个]{n.}{
televisão; TV; televisor
}
\end{pronuncia}
\end{verbete}

\begin{verbete}[dian4ti1]{电梯}
\begin{pronuncia}{dian4ti1}
\significado[台,部]{n.}{
elevador
}
\end{pronuncia}
\end{verbete}

\begin{verbete}[dian4ying3]{电影}
\begin{pronuncia}{dian4ying3}
\significado[部,片,幕,场]{n.}{
cinema; filme
}
\end{pronuncia}
\end{verbete}

\begin{verbete}[dian4zi3]{电子}
\begin{pronuncia}{dian4zi3}
\significado{n.}{
eletrônico, eletrônica|
elétron
}
\end{pronuncia}
\end{verbete}

\begin{verbete}[dian4zi3you2jian4]{电子邮件}
\begin{pronuncia}{dian4zi3you2jian4}
\significado[封,份]{n.}{
correio eletrônico; \textit{e-mail}
}
\end{pronuncia}
\end{verbete}

\begin{verbete}[ding1zhu3]{叮嘱}
\begin{pronuncia}{ding1zhu3}
\significado{v.}{
exortar; avisar; insistir de novo e de novo
}
\end{pronuncia}
\end{verbete}

\begin{verbete}[dong1]{东}
\begin{pronuncia}{dong1}
\significado{n.}{
leste
}
\end{pronuncia}
\end{verbete}

\begin{verbete}[dong1ban4qiu2]{东半球}
\begin{pronuncia}{dong1ban4qiu2}
\significado{n.}{
hemisfério leste
}
\end{pronuncia}
\end{verbete}

\begin{verbete}[dong1bei3]{东北}
\begin{pronuncia}{dong1bei3}
\significado{p.l.}{
nordeste
}
\end{pronuncia}
\end{verbete}

\begin{verbete}[dong1bian0]{东边}
\begin{pronuncia}{dong1bian0}
\significado{p.l.}{
este; leste; lado leste; oriente
}
\end{pronuncia}
\end{verbete}

\begin{verbete}[dong1bu4]{东部}
\begin{pronuncia}{dong1bu4}
\significado{p.l.}{
leste; oriente
}
\end{pronuncia}
\end{verbete}

\begin{verbete}[dong1fang1]{东方}
\begin{pronuncia}{dong1fang1}
\significado{p.l.}{
leste; oriente
}
\end{pronuncia}
\end{verbete}

\begin{verbete}[Dong1fang1\ Xue2yuan4]{东方学院}
\begin{pronuncia}[\\]{Dong1fang1\ Xue2yuan4}
\significado{n.}{
Instituto Oriental
}
\end{pronuncia}
\end{verbete}

\begin{verbete}[dong1tian1]{东天}
\begin{pronuncia}{dong1tian1}
\significado{n.}{
inverno
}
\significado{p.t.}{
inverno
}
\end{pronuncia}
\end{verbete}

\begin{verbete}[dong1mia4]{东面}
\begin{pronuncia}{dong1mian4}
\significado{}{p.l.}{
este; leste; lado leste; oriente
}
\end{pronuncia}
\end{verbete}

\begin{verbete}[dong1xi0]{东西}
\begin{pronuncia}{dong1xi0}
\significado[个,件]{n.}{
coisa
}
\end{pronuncia}
\end{verbete}

\begin{verbete}[dong4wu4]{动物}
\begin{pronuncia}{dong4wu4}
\significado[只,群,个]{n.}{
animal
}
\end{pronuncia}
\end{verbete}

\begin{verbete}[dong4wu4yuan2]{动物园}
\begin{pronuncia}{dong4wu4yuan2}
\significado[个]{n.}{
jardim zoológico
}
\end{pronuncia}
\end{verbete}

\begin{verbete}[dou1]{都}
\begin{pronuncia}{dou1}
\significado{adv.}{
todo, toda, todos, todas
}
\end{pronuncia}
\end{verbete}

\begin{verbete}[du2]{读}
\begin{pronuncia}{du2}
\significado{v.}{
ler
}
\end{pronuncia}
\end{verbete}

\begin{verbete}[du4]{度}
\begin{pronuncia}{du4}
\significado{p.c.}{
para temperatura, etc
}
\significado{v.}{
grau
}
\end{pronuncia}
\end{verbete}

\begin{verbete}[du4zi0]{肚子}
\begin{pronuncia}{du4zi0}
\significado[个]{n.}{
abdómen; barriga
}
\end{pronuncia}
\end{verbete}

\begin{verbete}[duan3]{短}
\begin{pronuncia}{duan3}
\significado{adj.}{
curto, curta
}
\end{pronuncia}
\end{verbete}

\begin{verbete}[duan3ku4]{短裤}
\begin{pronuncia}{duan3ku4}
\significado{n.}{
calção, shorts
}
\end{pronuncia}
\end{verbete}

\begin{verbete}[duan4lian4]{锻炼}
\begin{pronuncia}{duan4lian4}
\significado{v.}{
fazer exercício físico
}
\end{pronuncia}
\end{verbete}

\begin{verbete}[dui4]{对}
\begin{pronuncia}{dui4}
\significado{adj.}{
correto; sim
}
\significado{prep.}{
com; para; para com
}
\end{pronuncia}
\end{verbete}

\begin{verbete}[dui4 ...\  gan3shing4qu4]{对······感兴趣}
\begin{pronuncia}[\\]{dui4 ...\  gan3shing4qu4}
\significado{expr.}{
estar interessado em...; ter interesse em...; interessar-se por...
}
\end{pronuncia}
\end{verbete}

\begin{verbete}[dui4 ...\  shu2xi1]{对······熟悉}
\begin{pronuncia}{dui4 ...\  shu2xi1}
\significado{expr.}{
estar familiarizado com...
}
\end{pronuncia}
\end{verbete}

\begin{verbete}[dui4 ...\  you3xing4qu4]{对······有兴趣}
\begin{pronuncia}[\\]{dui4 ...\  you3shing4qu4}
\significado{expr.}{
estar interessado em...; ter interesse em...; interessar-se por...
}
\end{pronuncia}
\end{verbete}

\begin{verbete}[dui4bu0qi3]{对不起}
\begin{pronuncia}{dui4bu0qi3}
\significado{v.}{
desculpar; pedir desculpa; perdoar
}
\end{pronuncia}
\end{verbete}

\begin{verbete}[dui4hua4]{对话}
\begin{pronuncia}{dui4hua4}
\significado{n.}{
diálogo; conversa
}
\significado{v.}{
dialogar; conversar
}
\end{pronuncia}
\end{verbete}

\begin{verbete}[dui4mian4]{对面}
\begin{pronuncia}{dui4mian4}
\significado{p.l.}{
lado oposto
}
\end{pronuncia}
\end{verbete}

\begin{verbete}[duo1]{多}
\begin{pronuncia}{duo1}
\significado{adj.}{
muito, muita, muitos, muitas
}
\end{pronuncia}
\end{verbete}

\begin{verbete}[duo1da4]{多大}
\begin{pronuncia}{duo1da4}
\significado{interr.}{
quantos anos?; que idade?
}
\end{pronuncia}
\end{verbete}

\begin{verbete}[duo1(me0)]{多(么)}
\begin{pronuncia}{duo1(me0)}
\significado{adv.}{
como
}
\end{pronuncia}
\end{verbete}

\begin{verbete}[duo1shao0]{多少}
\begin{pronuncia}{duo1shao0}
\significado{interr.}{
quanto?, quanta?, quantos?, quantas?; (para mais de 10 itens)
}
\end{pronuncia}
\end{verbete}

\begin{verbete}[duo1yun2]{多云}
\begin{pronuncia}{duo1yun2}
\significado{adj.}{
céu nublado
}
\end{pronuncia}
\end{verbete}

%\end{multicols*}

%%%
%%% E
%%%
\section*{E}
\addcontentsline{toc}{section}{E}
\begin{multicols*}{2}

\begin{verbete}[É\ luo2si1]{俄罗斯}
\begin{pronuncia}{É\ luo2si1}
\significado{}{n.}{
Rússia
}
\end{pronuncia}
\end{verbete}

\begin{verbete}[en1ci4]{恩赐}
\begin{pronuncia}{en1ci4}
\significado{n.}{
favor; caridade
}
\significado{v.}{
conceder (favor, caridade);
}
\end{pronuncia}
\end{verbete}

\begin{verbete}[er2xi2]{儿媳}
\begin{pronuncia}{er2xi2}
\significado{n.}{
esposa do filho
}
\end{pronuncia}
\end{verbete}

\begin{verbete}[er2zi0]{儿子}
\begin{pronuncia}{er2zi0}
\significado{n.}{
filho
}
\end{pronuncia}
\end{verbete}

\begin{verbete}[er3duo0]{耳朵}
\begin{pronuncia}{er3duo0}
\significado[只,个,对]{n.}{
orelha
}
\end{pronuncia}
\end{verbete}

\begin{verbete}[er4]{二}
\begin{pronuncia}{er4}
\significado{num.}{
dois; 2
}
\end{pronuncia}
\end{verbete}

\end{multicols*}

%%%
%%% F
%%%
\section*{F}
\addcontentsline{toc}{section}{F}
\begin{multicols}{2}

\begin{hanzi}[发]{fa1}
\entry{fa1}{v.}{
    enviar; mandar
}
\end{hanzi}

\begin{hanzi}[发烧]{fa1shao1}
\entry{fa1shao1}{v.}{
    ter febre
}
\end{hanzi}

\begin{hanzi}[罚]{fa2}
\entry{fa2}{v.}{
    castigar; punir
}
\end{hanzi}

\begin{hanzi}[罚款]{fa2kuan3}
\entry{fa2kuan3}{v.+compl.}{
    aplicar uma multa; multar
}
\end{hanzi}

\begin{hanzi}[法国]{Fa3guo2}
\entry{Fa3guo2}{n.}{
    França
}
\end{hanzi}

\begin{hanzi}[法语]{fa3yu3}
\entry{fa3yu3}{n.}{
    françês, língua francesa
}
\end{hanzi}

\begin{hanzi}[法文]{fa3wen2}
\entry{fa3wen2}{n.}{
    françês, língua francesa
}
\end{hanzi}

\begin{hanzi}[番茄]{fan4qie2}
\entry{fan4qie2}{n.}{
    tomate
}
\end{hanzi}

\begin{hanzi}[饭店]{fan4dian4}
\entry{fan4dian4}{n.}{
    restaurante
}
\end{hanzi}

\begin{hanzi}[方便]{fang1bian4}
\entry{fang1bian4}{adj.}{
    conveniente
}
\end{hanzi}

\begin{hanzi}[访问]{fang3wen4}
\entry{fang3wen4}{v.}{
    visitar
}
\end{hanzi}

\begin{hanzi}[放假]{fang4jia4}
\entry{fang4jia4}{v.}{
    ter férias
}
\end{hanzi}

\begin{hanzi}[房间]{fang4jian1}
\entry{fang4jian1}{n.}{
    quarto
}
\end{hanzi}

\begin{hanzi}[放心]{fang4xin1}
\entry{fang4xin1}{adj.}{
    descansado; despreocupado
}
\end{hanzi}

\begin{hanzi}[非]{fei1}
\entry{fei1}{adv.}{
    não; nem
}
\end{hanzi}

\begin{hanzi}[非常]{fei1chang2}
\entry{fei1chang2}{adv.}{
    muito
}
\end{hanzi}

\begin{hanzi}[非洲]{Fei1zhou1}
\entry{Fei1zhou1}{n.}{
    África
}
\end{hanzi}

\begin{hanzi}[飞机]{fei1ji1}
\entry{fei1ji1}{n.}{
    avião
}
\end{hanzi}

\begin{hanzi}[(飞)机票]{(fei1)\ ji1piao4}
\entry{(fei1)\ ji1piao4}{n.}{
    bilhete de avião
}
\end{hanzi}

\begin{hanzi}[分]{fen1}
\entry{fen1}{p.c.}{fen1}{
    minuto|
    centavo
}
\end{hanzi}

\begin{hanzi}[分公司]{fen1gong1si1}
\entry{fen1gong1si1}{n.}{fen1gong1si1}{
    sucursal; filial de companhia
}
\end{hanzi}

\begin{hanzi}[分钟]{fen1zhong1}
\entry{fen1zhong1}{n.}{fen1zhong1}{
    minuto
}
\end{hanzi}

\begin{hanzi}[份]{fen4}
\entry{fen4}{p.c.}{fen4}{
    dose
}
\end{hanzi}

\begin{hanzi}[分量]{fen4liang4}
\entry{fen4liang4}{p.}{fen4liang4}{
    peso; componente vetorial; física
}
\end{hanzi}

\begin{hanzi}[风]{feng1}
\entry{feng1}{n.}{
    vento
}
\end{hanzi}

\begin{hanzi}[枫叶]{feng1ye4}
\entry{feng1ye4}{n.}{
    folha de bordo (tipo de árvore/arbusto)
}
\end{hanzi}

\begin{hanzi}[副]{fu4}
\entry{fu4}{p.c.}{
    par; para óculos, luvas, etc
}
\end{hanzi}

\begin{hanzi}[父亲]{fu4qin0}
\entry{fu4quin0}{n.}{
    pai
}
\end{hanzi}

\begin{hanzi}[父母亲]{fu4mu3qin0}
\entry{fu4mu3quin0}{n.}{
    pais
}
\end{hanzi}

\begin{hanzi}[附近]{fu4jin4}
\entry{fu4jin4}{p.l.}{
    aqui perto; perto daqui
}
\end{hanzi}

\end{multicols}

%%%
%%% G
%%%
\section*{G}
\addcontentsline{toc}{section}{G}
\begin{multicols}{2}

\begin{verbete}[干杯]{gan1bei1}
\significado{gan1bei1}{v.+compl.}{
    brindar até a última gota; ``saúde!''
}
\end{verbete}

\begin{verbete}[干净]{gan1jing4}
\significado{gan1jing4}{adj.}{
    limpo
}
\end{verbete}

\begin{verbete}[赶快]{gan3kuai4}
\significado{gan3kuai4}{adv.}{
    rapidamente, imediatamente
}
\end{verbete}

\begin{verbete}[橄榄球]{gan3lan3qiu2}
\significado{gan3lan3qiu2}{n.}{
    rúgbi
}
\end{verbete}

\begin{verbete}[感冒]{gan3mao4}
\significado{gan3mao4}{v.}{
    ficar resfriado; estar com resfriado
}
\end{verbete}

\begin{verbete}[干]{gan4}
\significado{gan4}{v.}{
    fazer
}
\end{verbete}

\begin{verbete}[刚]{gang1}
\significado{gang1}{adv.}{
    acabar de
}
\end{verbete}

\begin{verbete}[高]{gao1}
\significado{gao1}{adj.}{
    alto
}
\end{verbete}

\begin{verbete}[高兴]{gao1xing4}
\significado{gao1xing4}{adj.}{
    feliz; alegre; contente
}
\end{verbete}

\begin{verbete}[告诉]{gao4su0}
\significado{gao4su0}{v.}{
    contar; dar a conhecer; dizer
}
\end{verbete}

\begin{verbete}[歌]{ge1}
\significado{ge1}{n.}{
    canção; canto
}
\end{verbete}

\begin{verbete}[哥哥]{ge1ge0}
\significado{ge1ge0}{n.}{
    irmão mais velho
}
\end{verbete}

\begin{verbete}[个]{ge4}
\significado{ge4}{p.c.}{
    de uso geral
}
\end{verbete}

\begin{verbete}[给]{gei3}
\significado{gei3}{pre.}{
    a; para
}
\significado{gei3}{v.}{
    dar
}
\end{verbete}

\begin{verbete}[给\ ······\ 打\ 电话]{gei3\ ...\ da3\ dian4hua4}
\significado{gei3\ ...\ da3\ dian4hua4}{}{
    telefonar para alguém
}
\end{verbete}

\begin{verbete}[跟]{gen1}
\significado{gen1}{prep.}{
    com
}
\end{verbete}

\begin{verbete}[根据]{gen1ju4}
\significado{gen1ju4}{prep.}{
    de acordo com
}
\end{verbete}

\begin{verbete}[更]{geng4}
\significado{geng4}{adv.}{
    mais
}
\end{verbete}

\begin{verbete}[工作]{gong1zuo4}
\significado{gong1zuo4}{n.}{
    trabalho
}
\significado{gong1zuo4}{v.}{
    trabalhar
}
\end{verbete}

\begin{verbete}[公共汽车]{gong1gong4qi4che1}
\significado{gong1gong4qi4che1}{n.}{
    ônibus
}
\end{verbete}

\begin{verbete}[公克]{gong1ke4}
\significado{gong1ke4}{n.}{
    trabalho escolar; trabalho de casa
}
\end{verbete}

\begin{verbete}[公司]{gong1si1}
\significado{gong1si1}{n.}{
    empresa; companhia
}
\end{verbete}

\begin{verbete}[公园]{gong1yuan2}
\significado{gong1yuan2}{n.}{
    parque
}
\end{verbete}

\begin{verbete}[狗]{gou3}
\significado{gou3}{n.}{
    cão; cachorro|
    \pc{条/只}
}
\end{verbete}

\begin{verbete}[故宫]{Gu4gong1}
\significado{Gu4gong1}{n.}{
    Palácio Imperial
}
\end{verbete}

\begin{verbete}[刮]{gua1}
\significado{gua1}{v.}{
    ventar, soprar (vento)
}
\end{verbete}

\begin{verbete}[刮风]{gua1feng1}
\significado{gua1feng1}{v.+compl.}{
    ventanejar; fazer vento
}
\end{verbete}

\begin{verbete}[拐]{guai3}
\significado{guai3}{v.}{
    virar; cortar
}
\end{verbete}

\begin{verbete}[光盘]{guang1pan2}
\significado{guang1pan2}{n.}{
    CD; disco compacto
}
\end{verbete}

\begin{verbete}[广东]{guang3dong1}
\significado{guang3dong1}{n.}{
    Guangdong
}
\end{verbete}

\begin{verbete}[规定]{gui1ding4}
\significado{gui1ding4}{n.}{
    regulamento
}
\significado{gui1ding4}{v.}{
    estipular
}
\end{verbete}

\begin{verbete}[贵]{gui4}
\significado{gui4}{adj.}{
    caro
}
\end{verbete}

\begin{verbete}[贵姓]{gui4xing4}
\significado{gui4xing4}{interr.}{
    seu sobrenome
}
\end{verbete}

\begin{verbete}[国]{guo2}
\significado{guo2}{n.}{
    país
}
\end{verbete}

\begin{verbete}[国家]{guo2jia1}
\significado{guo2jia1}{n.}{
    país
}
\end{verbete}

\begin{verbete}[果酱]{guo3jiang4}
\significado{guo3jiang4}{n.}{
    geléia; compota ou doce (de frutas)
}
\end{verbete}

\begin{verbete}[过]{guo4}
\significado{guo4}{v.}{
    passar
}
\significado{guo4}{part.}{
    passado
}
\end{verbete}

\begin{verbete}[过年]{guo4nian2}
\significado{guo4nian2}{v.}{
    festejar o Ano Novo Chinês
}
\end{verbete}

\begin{verbete}[过期]{guo4qi1}
\significado{guo4qi1}{v.+compl.}{
    exceder a data; passar a data
}
\end{verbete}

\end{multicols}

%%%
%%% H
%%%
\section*{H}
\addcontentsline{toc}{section}{H}

\begin{verbete}[7]{还}{hai2}
  \significado{adv.}{ainda; também}
  \veja{还}{huan4}
\end{verbete}

\begin{verbete}[7;9]{还是}{hai2shi0}
  \significado{conj.}{ou (somente para frases interrogativas)}
\end{verbete}

\begin{verbete}[9;3]{孩子}{hai2zi0}
  \significado{s.}{criança; filho}
\end{verbete}

\begin{verbete}[10]{海}{hai3}
  \significado[个,片]{s.}{mar; oceano}
\end{verbete}

\begin{verbete}[10;5]{海边}{hai3bian1}
  \significado{p.d.l.}{costa marítima; litoral}
\end{verbete}

\begin{verbete}[10;8]{害怕}{hai4pa4}
  \significado{v.}{ter medo; ficar com medo; temer}
\end{verbete}

\begin{verbete}[12;8]{韩国}{Han2guo2}
  \significado{s.}{Coréia do Sul}
\end{verbete}

\begin{verbete}[12;8;2]{韩国人}[\\]{han2guo2ren2}
  \significado{s.}{coreano; nascido na Coréia}
\end{verbete}

\begin{verbete}[5;12;7;8]{汉葡词典}[\\]{Han4pu2ci2dian3}
  \significado[部,本]{s.}{Dicionário Chinês-Português}
\end{verbete}

\begin{verbete}[5;9]{汉语}{han4yu3}
  \significado[门]{s.}{chinês, língua chinesa, mandarim}
\end{verbete}

\begin{verbete}[6]{行}{hang2}
  \significado{s.}{firma comercial; linha de negócio; profissão; linha (de um tema); linha (em tabela de dados)}
  \veja{行}{xing2}
\end{verbete}

\begin{verbete}[10;10]{航班}{hang2ban1}
  \significado{s.}{voo; número de voo}
\end{verbete}

\begin{verbete}[6]{好}{hao3}
  \significado{adj.}{bom, bem}
  \veja{好}{hao4}
\end{verbete}

\begin{verbete}[6;6]{好吃}{hao3chi1}
  \significado{adj.}{delicioso; saboroso}
\end{verbete}

\begin{verbete}[6;5]{好汉}{hao3han4}
  \significado[条]{s.}{herói; pessoa forte e corajosa}
\end{verbete}

\begin{verbete}[6;9]{好看}{hao3kan4}
  \significado{adj.}{boa aparência; bom (um filme, livro, programa de TV, etc.)}
\end{verbete}

\begin{verbete}[6;7]{好听}{hao3ting1}
  \significado{adj.}{agradável de ouvir}
\end{verbete}

\begin{verbete}[6;8;2]{好玩儿}{hao3wanr2}
  \significado{adj.}{divertido; perazeiroso; interessante}
\end{verbete}

\begin{verbete}[6;8]{好学}{hao3xue2}
  \significado{adj.}{fácil de aprender; estudioso; erudito}
\end{verbete}

\begin{verbete}[5]{号}{hao4}
  \significado{p.c.}{dia do mês; usado para indicar o número de pessoas}
  \significado[个]{s.}{dia do mês; número}
\end{verbete}

\begin{verbete}[5;8]{号码}{hao4ma3}
  \significado[堆,个]{s.}{número}
\end{verbete}

\begin{verbete}[6]{好}{hao4}
  \significado{v.}{gostar de; estar propenso a; ter tendência a}
  \veja{好}{hao3}
\end{verbete}

\begin{verbete}[12]{喝}{he1}
  \significado{interj.}{Meu Deus!}
  \significado{v.}{beber}
  \veja{喝}{he4}
\end{verbete}

\begin{verbete}[6;6]{合同}{he2tong0}
  \significado[个]{s.}{contrato (negócio)}
\end{verbete}

\begin{verbete}[6;10]{合资}{he2zi1}
  \significado{s.}{joint-venture com capitais mistos}
\end{verbete}

\begin{verbete}[6;7]{合作}{he2zuo4}
  \significado[个]{s.}{cooperação}
  \significado{v.}{cooperar; colaborar}
\end{verbete}

\begin{verbete}[8]{和}{he2}
  \significado{conj.}{e (somente para palavras)}
  \veja{和}{he4}
  \veja{和}{hu2}
  \veja{和}{huo4}
\end{verbete}

\begin{verbete}[9]{河}{he2}
  \significado[条,道]{s.}{rio}
\end{verbete}

\begin{verbete}[11]{盒}{he2}
  \significado{p.c.}{caixa pequena}
  \significado{s.}{caixa pequena; estojo}
\end{verbete}

\begin{verbete}[8]{和}{he4}
  \significado{v.}{conversar com os outros; compor um poema em resposta (ao poema de alguém) usando a mesma sequência de rimas; juntar-se ao canto (canção)}
  \veja{和}{he2}
  \veja{和}{hu2}
  \veja{和}{huo4}
\end{verbete}

\begin{verbete}[12]{喝}{he4}
  \significado{v.}{gritar bem alto}
  \veja{喝}{he1}
\end{verbete}

\begin{verbete}[12]{黑}{hei1}
  \significado{adj.}{preto; escuro; ilegal; secreto; sombrio; sinistro}
  \significado{v.}{esconder (algo); difamar; hackear (computador)}
\end{verbete}

\begin{verbete}[12;8]{黑板}{hei1ban3}
  \significado[块,个]{s.}{quadro negro}
\end{verbete}

\begin{verbete}[12;6]{黑色}{hei1se4}
  \significado{s.}{cor preta}
\end{verbete}

\begin{verbete}[9]{很}{hen3}
  \significado{adv.}{muito; mui; advérbio de grau}
\end{verbete}

\begin{verbete}[6]{红}{hong2}
  \significado{adj.}{vermelho; popular; revolucionário}
  \significado{s.}{bônus}
\end{verbete}

\begin{verbete}[6;6]{红色}{hong2se4}
  \significado{s.}{cor vermelha}
\end{verbete}

\begin{verbete}[6;10]{红烧}{hong2shao1}
  \significado{s.}{guisado em molho de soja (prato)}
\end{verbete}

\begin{verbete}[6;5]{后边}{hou4bian0}
  \significado{p.l.}{atrás; detrás}
\end{verbete}

\begin{verbete}[6;9]{后面}{hou4mian0}
  \significado{p.l.}{atrás; detrás}
\end{verbete}

\begin{verbete}[6;6]{后年}{hou4nian2}
  \significado{p.t.}{daqui a dois anos}
\end{verbete}

\begin{verbete}[6;4]{后天}{hou4tian1}
  \significado{p.t.}{depois de amanhã}
\end{verbete}

\begin{verbete}[8]{和}{hu2}
  \significado{v.}{completar um conjunto de Mahjong ou cartas de baralho}
  \veja{和}{he2}
  \veja{和}{he4}
  \veja{和}{huo4}
\end{verbete}

\begin{verbete}[9;11;2]{胡萝卜}{hu2luo2bo0}
  \significado{s.}{cenoura}
\end{verbete}

\begin{verbete}[12]{湖}{hu2}
  \significado[个,片]{s.}{lago}
\end{verbete}

\begin{verbete}[12;9]{湖南}{Hu2nan2}
  \significado{s.}{Hunan}
\end{verbete}

\begin{verbete}[15;7;15;10]{糊里糊涂}[\\]{hu2li0hu2tu0}
  \significado{adj.}{desnorteado; perturbado}
\end{verbete}

\begin{verbete}[4;9]{互相}{hu4xiang1}
  \significado{adv.}{mutuamente}
\end{verbete}

\begin{verbete}[7]{花}{hua1}
  \significado[朵,支,束,把,盆,簇]{s.}{flor}
\end{verbete}

\begin{verbete}[7;2]{花儿}{huar1}
  \significado[朵,支,束,把,盆,簇]{s.}{flor}
\end{verbete}

\begin{verbete}[7;5]{花生}{hua1sheng1}
  \significado[粒]{s.}{amendoim}
\end{verbete}

\begin{verbete}[7;12;11]{花椰菜}{hua1ye1cai4}
  \significado{s.}{couve-flor}
\end{verbete}

\begin{verbete}[6;11;10]{华盛顿}[\\]{Hua2sheng4dun4}
  \significado{s.}{Washington}
\end{verbete}

\begin{verbete}[6;4]{华氏}{hua2shi4}
  \significado{s.}{Fahrenheit}
\end{verbete}

\begin{verbete}[6;13]{华裔}{hua2yi4}
  \significado{s.}{descendente de chinês}
\end{verbete}

\begin{verbete}[12]{滑}{hua2}
  \significado{adj.}{deslizado}
  \significado{v.}{deslizar}
\end{verbete}

\begin{verbete}[12;11]{滑雪}{hua2xue3}
  \significado{v.+compl.}{esquiar; fazer esqui}
\end{verbete}

\begin{verbete}[8]{话}{hua4}
  \significado[种,席,句,口,番]{s.}{palavras; fala; linguagem; dialeto}
\end{verbete}

\begin{verbete}[7]{坏}{huai4}
  \significado{adj.}{avariado; mau}
  \significado{v.}{perder o controle}
\end{verbete}

\begin{verbete}[6;7]{欢迎}{huan1ying2}
  \significado{v.}{dar as boas-vindas; ser bem-vindo}
\end{verbete}

\begin{verbete}[8;14]{环境}{huan1jing4}
  \significado[个]{s.}{ambiente; arredores; circunstâncias}
\end{verbete}

\begin{verbete}[7]{还}{huan4}
  \significado{v.}{devolver; restituir; pagar de volta}
  \veja{还}{hai2}
\end{verbete}

\begin{verbete}[10]{换}{huan4}
  \significado{v.}{mudar; trocar; substituir; converter (moedas)}
\end{verbete}

\begin{verbete}[11]{黄}{huang2}
  \significado{adj.}{amarelo; pornográfico}
\end{verbete}

\begin{verbete}[11;5]{黄瓜}{huang2gua1}
  \significado[条]{s.}{pepino}
\end{verbete}

\begin{verbete}[11;6]{黄色}{huang2se4}
  \significado{s.}{cor amarela}
\end{verbete}

\begin{verbete}[11;8]{黄油}{huang2you2}
  \significado[盒]{s.}{manteiga}
\end{verbete}

\begin{verbete}[6]{回}{hui2}
  \significado{v.d.}{regressar}
\end{verbete}

\begin{verbete}[6;12]{回答}{hui2da2}
  \significado{v.}{responder}
\end{verbete}

\begin{verbete}[6;7]{回来}{hui2lai0}
  \significado{v.d.}{regressar; voltar; estar de volta; (para a minha localização)}
\end{verbete}

\begin{verbete}[6;5]{回去}{hui2qu0}
  \significado{v.d.}{regressar; voltar; estar de volta; (a partir da minha localização)}
\end{verbete}

\begin{verbete}[6]{会}{hui4}
  \significado{v.}{saber}
\end{verbete}

\begin{verbete}[9;6]{活动}{huo2dong4}
  \significado[项,个]{s.}{atividade; evento; campanha}
  \significado{v.}{exercer; operar}
\end{verbete}

\begin{verbete}[4;4]{火车}{huo3che1}
  \significado[列,节,班,趟]{s.}{trem; comboio}
\end{verbete}

\begin{verbete}[8]{和}{huo4}
  \significado{p.c.}{para fervuras de ervas medicinais; para enxágue de roupas}
  \significado{v.}{misturar; misturar (ingredientes) juntos}
  \veja{和}{he2}
  \veja{和}{he4}
  \veja{和}{hu2}
\end{verbete}

\begin{verbete}[8;8]{或者}{huo4zhe3}
  \significado{conj.}{ou (usado em expressões afirmativas)}
\end{verbete}

%%%%% EOF %%%%%

%%%%% Não existem palavras com pinyin iniciado em "I"
%%%
%%% J
%%%
\section*{J}
\addcontentsline{toc}{section}{J}

\begin{verbete}[2]{几}{ji1}
\significado{adv.}{ quase }
\veja{几}{ji3}
\end{verbete}

\begin{verbete}[6;16]{机器}{ji1qi4}
\significado[台,部,个]{s.}{ máquina }
\end{verbete}

\begin{verbete}[7]{鸡}{ji1}
\significado[只]{s.}{ galo, galinha; gíria: prostituta }
\end{verbete}

\begin{verbete}[7;11]{鸡蛋}{ji1dan4}
\significado[个,打]{s.}{ ovo de galinha }
\end{verbete}

\begin{verbete}[6;6]{机场}{ji1chang3}
\significado[家,处]{s.}{ aeroporto; aeródromo }
\end{verbete}

\begin{verbete}[6;11]{机票}{ji1piao4}
\significado[张]{v.}{ bilhete de avião }
\veja{飞机票}{fei1ji1piao4}
\end{verbete}

\begin{verbete}[7;2]{······极了}{...ji2le0}
\significado{expr.}{ muito; extremamente }
\end{verbete}

\begin{verbete}[3;10]{及格}{ji2ge2}
\significado{v.}{ atender a um padrão mínimo; passar em um exame ou teste }
\end{verbete}

\begin{verbete}[2]{几}{ji3}
\significado{interr.}{ quantos?, (até 10 itens); alguns? }
\veja{几}{ji1}
\end{verbete}

\begin{verbete}[8;5]{季节}{ji4jie2}
\significado[个]{s.}{ estação (clima) }
\end{verbete}

\begin{verbete}[10]{家}{jia1}
\significado{p.c.}{ para famílias ou empresas }
\significado[个]{s.}{ família; casa; sufixo de substantivos para designar um especialista em alguma atividade  }
\end{verbete}

\begin{verbete}[10;8]{家具}{jia1ju4}
\significado[件,套]{s.}{ móveis; mobiliário }
\end{verbete}

\begin{verbete}[10;7]{家里}{jia1li0}
\significado{p.d.l.}{ em casa }
\end{verbete}

\begin{verbete}[10;3]{家乡}{jia1xiang1}
\significado[个]{s.}{ terra natal }
\end{verbete}

\begin{verbete}[5;10;3]{加拿大}{Jia1na2da4}
\significado{s.}{ Canadá }
\end{verbete}

\begin{verbete}[5;10;3;2]{加拿大人}[\\]{jia1na2da4ren2}
\significado{s.}{ canadense; pessoa nascida no Canadá }
\end{verbete}

\begin{verbete}[8;14]{肩膀}{jian1bang3}
\significado{s.}{ ombro }
\end{verbete}

\begin{verbete}[11;9]{检查}{jian3cha2}
\significado[次]{s.}{ inspeção }
\significado{v.}{ examinar; inspecionar }
\end{verbete}

\begin{verbete}[13;8]{简单}{jian3dan1}
\significado{adj.}{ simples; sem complicações }
\end{verbete}

\begin{verbete}[4]{见}{jian4}
\significado{v.}{ ver; entrevistar; encontrar alguém; parecer (ser alguma coisa) }
\end{verbete}

\begin{verbete}[4;9]{见面}{jian4mian4}
\significado{v.}{ encontrar-se com alguém }
\end{verbete}

\begin{verbete}[6]{件}{jian4}
\significado{p.c.}{ para eventos, coisas, roupas etc. }
\end{verbete}

\begin{verbete}[8;5]{建议}{jian4yi4}
\significado[个,点]{s.}{ proposta; recomendação; sugestão }
\significado{v.}{ propor; recomendar; sugerir }
\end{verbete}

\begin{verbete}[6;6]{江西}{Jiang1xi1}
\significado{s.}{ Jiangxi }
\end{verbete}

\begin{verbete}[12]{强}{jiang4}
\significado{adj.}{ teimoso; inflexível }
\veja{强}{qiang2}
\veja{强}{qiang3}
\end{verbete}

\begin{verbete}[6;10]{交通}{jiao1tong1}
\significado{s.}{ transporte; tráfego; trânsito; comunicações; conexão }
\significado{v.}{ estar conectado }
\end{verbete}

\begin{verbete}[8;4]{郊区}{jiao1qu1}
\significado[个]{s.}{ subúrbio; distrito suburbano; arredores }
\end{verbete}

\begin{verbete}[10;8]{胶卷}{jiao1juan3}
\significado{s.}{ filme; película; rolo de filme }
\end{verbete}

\begin{verbete}[11]{教}{jiao1}
\significado{v.}{ ensinar }
\veja{教}{jiao4}
\end{verbete}

\begin{verbete}[11]{脚}{jiao3}
\significado{p.c.}{ para chutes }
\significado[双,只]{s.}{ pé; base (de um objeto); perna (de um animal ou objeto) }
\end{verbete}

\begin{verbete}[7]{角}{jiao3}
\significado{p.c.}{ 1 jiao = 10 centavos }
\significado[个]{s.}{ ângulo; esquina; chifre; em forma de chifre }
\end{verbete}

\begin{verbete}[9;3]{饺子}{jiao3zi0}
\significado[个,只]{s.}{ jiaozi; bolinhos chineses; bolinho de massa }
\end{verbete}

\begin{verbete}[5]{叫}{jiao4}
\significado{v.}{ chamar-se; chamar; gritar; pedir (comida em um restaurante) }
\end{verbete}

\begin{verbete}[11]{教}{jiao4}
\significado{s.}{ religião; ensinamento }
\significado{v.}{ causar; como fazer algo; contar (explicar como fazer algo) }
\veja{教}{jiao1}
\end{verbete}

\begin{verbete}[11;8]{教练}{jiao4lian4}
\significado[个,位,名]{s.}{ instrutor; treinador }
\end{verbete}

\begin{verbete}[11;11]{教授}{jiao4shou4}
\significado[个,位]{s.}{ professor (universitário) }
\significado{v.}{ instruir; palestrar sobre }
\end{verbete}

\begin{verbete}[11;6]{教师}{jiao4shi1}
\significado[个]{s.}{ professor; mestre }
\end{verbete}

\begin{verbete}[11;9]{教室}{jiao4shi4}
\significado[间]{s.}{ sala de aula }
\end{verbete}

\begin{verbete}[11;8;13]{教学楼}[\\]{jiao4xue2lou2}
\significado{s.}{ edifício de salas de aula }
\end{verbete}

\begin{verbete}[11]{接}{jie1}
\significado{v.}{ ir buscar (alguém); ir ao encontro de (alguém); receber }
\end{verbete}

\begin{verbete}[11;5;8]{接(电话)}[\\]{jie1(\ dian4hua4)}
\significado{v.}{ atender (o telefone) }
\end{verbete}

\begin{verbete}[11;9]{接待}{jie1dai4}
\significado{v.}{ receber (alguém); acolher; recepcionar }
\end{verbete}

\begin{verbete}[12]{街}{jie1}
\significado[条]{s.}{ rua }
\end{verbete}

\begin{verbete}[5;4]{节日}{jie2ri4}
\significado[个]{s.}{ festival; feriado }
\end{verbete}

\begin{verbete}[9;8]{结果}{jie2guo3}
\significado{s.}{ como resultado; conclusão; resultado }
\significado{v.}{ despachar; matar }
\end{verbete}

\begin{verbete}[8;8]{姐姐}{jie3jie0}
\significado[个]{s.}{ irmã mais velha }
\end{verbete}

\begin{verbete}[8;4]{姐夫}{jie3fu0}
\significado{s.}{ marido da irmã mais velha }
\end{verbete}

\begin{verbete}[4;8]{介绍}{jie4shao4}
\significado{s.}{ introdução; apresentação }
\significado{v.}{ fazer uma apresentação; apresentar (alguém para alguém); apresentar (alguém para um emprego, etc) }
\end{verbete}

\begin{verbete}[7;5]{芥兰}{jie4lan2}
\significado{s.}{ couve }
\end{verbete}

\begin{verbete}[10]{借}{jie4}
\significado{adv.}{ por meio de }
\significado{v.}{ pedir emprestado; emprestar; aproveitar (uma oportunidade) }
\end{verbete}

\begin{verbete}[10;4;7]{借书证}[\\]{jie4shu1zheng4}
\significado{s.}{ cartão de biblioteca; literalmente: cartão para pedir emprestado livros }
\end{verbete}

\begin{verbete}[4;6]{今年}{jin1nian2}
\significado{p.t.}{ este ano }
\end{verbete}

\begin{verbete}[4;4]{今天}{jin1tian1}
\significado{p.t.}{ hoje }
\end{verbete}

\begin{verbete}[8;16]{金融}{jin1rong2}
\significado{s.}{ finança }
\end{verbete}

\begin{verbete}[7]{近}{jin4}
\significado{adj.}{ perto; próximo }
\end{verbete}

\begin{verbete}[7]{进}{jin4}
\significado{p.c.}{ para seções em um edifício ou complexo residencial }
\significado{s.}{ matemática: base de um sistema numérico }
\significado{v.d.}{ entrar }
\end{verbete}

\begin{verbete}[7;5;3]{进出口}{jin4chu1kou3}
\significado{s.}{ importação e exportação }
\significado{v.}{ importar e exportar }
\end{verbete}

\begin{verbete}[7;3]{进口}{jin4kou3}
\significado{adj.}{ importado }
\significado{s.}{ importação; entrada; entrada (para entrada de ar, água, etc.) }
\significado{v.}{ importar }
\end{verbete}

\begin{verbete}[7;7]{进来}{jin4lai2}
\significado{v.d.}{ entrar (para a minha localização) }
\end{verbete}

\begin{verbete}[7;5]{进去}{jin4qu4}
\significado{v.d.}{ entrar (a partir da minha localização) }
\end{verbete}

\begin{verbete}[8;11]{经常}{jing1chang2}
\significado{adv.}{ constantemente; diariamente; dia-a-dia; todo dia; frequentemente; sempre; regularmente }
\end{verbete}

\begin{verbete}[8;9]{经济}{jing1ji4}
\significado{s.}{ economia }
\end{verbete}

\begin{verbete}[8;11]{经理}{jing1li3}
\significado[个,位,名]{s.}{ diretor; gerente }
\end{verbete}

\begin{verbete}[19;14]{警察}{jing3cha2}
\significado[个]{s.}{ polícia; oficial de polícia }
\end{verbete}

\begin{verbete}[2]{九}{jiu3}
\significado{num.}{ 9, nove }
\end{verbete}

\begin{verbete}[9;11]{韭菜}{jiu3cai4}
\significado{s.}{ cebolinha chinesa; figurativo: investidores de varejo que perdem seu dinheiro para operadores mais experientes (ou seja, são ``colhidos'' como cebolinhas) }
\end{verbete}

\begin{verbete}[10]{酒}{jiu3}
\significado[杯,瓶,罐,桶,缸]{s.}{ bebida alcoólica; vinho (especialmente vinho de arroz); aguardente; licor; espíritos }
\end{verbete}

\begin{verbete}[10;11]{酒馆}{jiu3guan3}
\significado{s.}{ bar; taverna; adega }
\end{verbete}

\begin{verbete}[5]{旧}{jiu4}
\significado{adj.}{ velho; antigo; desgastado (com a idade) }
\end{verbete}

\begin{verbete}[12]{就}{jiu4}
\significado{adv.}{ exatamente; justamente }
\significado{v.}{ realizar; se envolver em; para acompanhar (de alimentos); aproveitar; avançar; empreender }
\end{verbete}

\begin{verbete}[5]{句}{ju4}
\significado{p.c.}{ para orações, frases ou linhas de versos}
\significado{s.}{ sentença; cláusula; frase }
\end{verbete}

\begin{verbete}[5;3]{句子}{ju4zi0}
\significado[个]{n}{ sentença; frase; oração }
\end{verbete}

\begin{verbete}[9;6]{举行}{ju3xing2}
\significado{v.}{ realizar (uma reunião, cerimônia etc.) }
\end{verbete}

\begin{verbete}[9;11]{觉得}{jue2de2}
\significado{v.}{ achar; sentir; pensar }
\end{verbete}

%%%%% EOF %%%%

%%%
%%% K
%%%
\section*{K}
\addcontentsline{toc}{section}{K}

\begin{verbete}[ka1fei1]{咖啡}[8;11]
\begin{pronuncia}{ka1fei1}
\significado[杯]{s.}{ café }
\end{pronuncia}
\end{verbete}

\begin{verbete}[ka1fei1guan3]{咖啡馆}[8;11;11]
\begin{pronuncia}{ka1fei1guan3}
\significado[家]{s.}{ cafeteria }
\end{pronuncia}
\end{verbete}

\begin{verbete}[kai1]{开}[4]
\begin{pronuncia}{kai1}
\significado{v.}{ abrir; ligar; dirigir }
\end{pronuncia}
\end{verbete}

\begin{verbete}[kai1che1]{开车}[4;4]
\begin{pronuncia}{kai1che1}
\significado{v.}{ conduzir; dirigir }
\end{pronuncia}
\end{verbete}

\begin{verbete}[kai1fa1qu1]{开发区}[4;5;4]
\begin{pronuncia}{kai1fa1qu1}
\significado{s.}{ zona de desenvolvimento }
\end{pronuncia}
\end{verbete}

\begin{verbete}[kai1shi3]{开始}[4;8]
\begin{pronuncia}{kai1shi3}
\significado{v.}{ começar; iniciar }
\end{pronuncia}
\end{verbete}

\begin{verbete}[kan4]{看}[9]
\begin{pronuncia}{kan4}
\significado{v.}{ olhar; ver; assistir }
\end{pronuncia}
\end{verbete}

\begin{verbete}[kan4jian4]{看见}[9;4]
\begin{pronuncia}{kan4jian4}
\significado{v.}{ encontrar; enxergar; ver }
\end{pronuncia}
\end{verbete}

\begin{verbete}[kao3shi4]{考试}[6;8]
\begin{pronuncia}{kao3shi4}
\significado[次]{s.}{ teste; prova; exame }
\significado{v.+compl.}{ submeter-se a uma prova; fazer um teste }
\end{pronuncia}
\end{verbete}

\begin{verbete}[kao3]{烤}[10]
\begin{pronuncia}{kao3}
\significado{v.}{ assar }
\end{pronuncia}
\end{verbete}

\begin{verbete}[Ke1ji4]{科技}[9;7]
\begin{pronuncia}{Ke1ji4}
\significado{s.}{ Ciência e Tecnologia }
\end{pronuncia}
\end{verbete}

\begin{verbete}[ke3]{颗}[14]
\begin{pronuncia}{ke3}
\significado{p.c.}{ para grãos, pérolas, dentes, corações, satelites, pequenas esferas, etc }
\end{pronuncia}
\end{verbete}

\begin{verbete}[ke3sou0]{咳嗽}[9;14]
\begin{pronuncia}{ke3sou0}
\significado{v.}{ ter tosse; tussir }
\end{pronuncia}
\end{verbete}

\begin{verbete}[ke3'ai4]{可爱}[5;10]
\begin{pronuncia}{ke3'ai4}
\significado{adj.}{ querido; fofo }
\end{pronuncia}
\end{verbete}

\begin{verbete}[ke3kou3ke3le3]{可口可乐}[5;3;5;5]
\begin{pronuncia}[\\]{ke3kou3ke3le3}
\significado{s.}{ Coca-Cola }
\end{pronuncia}
\end{verbete}

\begin{verbete}[ke3neng2]{可能}[5;10]
\begin{pronuncia}{ke3neng2}
\significado{adj.}{ possível }
\significado{adv.}{ possivelmente; provavelmente }
\end{pronuncia}
\end{verbete}

\begin{verbete}[ke3shi4]{可是}[5;9]
\begin{pronuncia}{ke3shi4}
\significado{conj.}{ porém; contudo; mas }
\end{pronuncia}
\end{verbete}

\begin{verbete}[ke3xi1]{可惜}[5;11]
\begin{pronuncia}{ke3xi1}
\significado{adj.}{ é pena }
\end{pronuncia}
\end{verbete}

\begin{verbete}[ke3yi3]{可以}[5;4]
\begin{pronuncia}{ke3yi3}
\significado{v.o.}{ poder }
\end{pronuncia}
\end{verbete}

\begin{verbete}[ke4]{刻}[8]
\begin{pronuncia}{ke4}
\significado{s.}{ quarto (de hora) }
\significado{p.c.}{ para curtos intervalos de tempo }
\end{pronuncia}
\end{verbete}

\begin{verbete}[ke4zhong1]{刻钟}[8;9]
\begin{pronuncia}{ke4zhong1}
\significado{p.c.}{ um quarto de hora }
\end{pronuncia}
\end{verbete}

\begin{verbete}[ke4qi0]{客气}[9;4]
\begin{pronuncia}{ke4qi0}
\significado{adj.}{ cortês; delicado; educado }
\significado{v.}{ fazer cerimônia }
\end{pronuncia}
\end{verbete}

\begin{verbete}[ke4ting1]{客厅}[9;4]
\begin{pronuncia}{ke4ting1}
\significado[间]{s.}{ sala de estar; sala de visitas }
\end{pronuncia}
\end{verbete}

\begin{verbete}[ke4ben3]{课本}[10;5]
\begin{pronuncia}{ke4ben3}
\significado[本]{s.}{ livro do aluno; manual }
\end{pronuncia}
\end{verbete}

\begin{verbete}[ken3ding4]{肯定}[8;8]
\begin{pronuncia}{ken3ding4}
\significado{adv.}{ com certeza; certamente }
\end{pronuncia}
\end{verbete}

\begin{verbete}[kong1qi4]{空气}[8;4]
\begin{pronuncia}{kong1qi4}
\significado{s.}{ ar }
\end{pronuncia}
\end{verbete}

\begin{verbete}[kong1tiao2]{空调}[8;10]
\begin{pronuncia}{kong1tiao2}
\significado[台]{s.}{ ar-condicionado; condicionador de ar }
\end{pronuncia}
\end{verbete}

\begin{verbete}[kong3pa4]{恐怕}[10;8]
\begin{pronuncia}{kong3pa4}
\significado{adv.}{ talvez; possivelmente; provavelmente; (em sentido não tão bom) }
\end{pronuncia}
\end{verbete}

\begin{verbete}[kongr4]{空儿}[8;2]
\begin{pronuncia}{kongr4}
\significado{s.}{ tempo livre }
\end{pronuncia}
\end{verbete}

\begin{verbete}[kou3]{口}[3]
\begin{pronuncia}{kou3}
\significado{p.c.}{ para coisas com bocas (pessoas, animais domésticos, canhões, etc); 
  para mordidas ou bocados }
\end{pronuncia}
\end{verbete}

\begin{verbete}[kou3xiang1tang2]{口香糖}[3;9;16]
\begin{pronuncia}[\\]{kou3xiang1tang2}
\significado{s.}{ goma de mascar; chiclete }
\end{pronuncia}
\end{verbete}

\begin{verbete}[kou3yin1]{口音}[3;9]
\begin{pronuncia}{kou3yin1}
\significado{s.}{ sotaque }
\end{pronuncia}
\end{verbete}

\begin{verbete}[kou3yu3]{口语}[3;9]
\begin{pronuncia}{kou3yu3}
\significado[门]{s.}{ linguagem oral; linguagem falada }
\end{pronuncia}
\end{verbete}

\begin{verbete}[ku3gua1]{苦瓜}[8;5]
\begin{pronuncia}{ku3gua1}
\significado{s.}{ melão amargo (cabaça amarga, pêra bálsamo, maçã bálsamo, pepino amargo) }
\end{pronuncia}
\end{verbete}

\begin{verbete}[ku4zi0]{裤子}[12;3]
\begin{pronuncia}{ku4zi0}
\significado[条]{s.}{ calças }
\end{pronuncia}
\end{verbete}

\begin{verbete}[kuai4]{块}[7]
\begin{pronuncia}{kuai4}
\significado{p.c.}{ para unidades de Reminbi, dinheiro; para peças ou pedaços de roupa, bolos, sabão, etc }
\end{pronuncia}
\end{verbete}

\begin{verbete}[kuai4]{快}[7]
\begin{pronuncia}{kuai4}
\significado{adj.}{ quase; rápido; depressa }
\end{pronuncia}
\end{verbete}

\begin{verbete}[kuai4le4]{快乐}[7;5]
\begin{pronuncia}{kuai4le3}
\significado{s.}{ felicidade }
\significado{adj.}{ feliz }
\end{pronuncia}
\end{verbete}

\begin{verbete}[kuan3]{款}[12]
\begin{pronuncia}{kuan3}
\significado[笔,个]{s.}{ montante de dinheiro }
\end{pronuncia}
\end{verbete}                                                                     

%%%%% EOF %%%%%

%%%
%%% L
%%%
\section*{L}
\addcontentsline{toc}{section}{L}

\begin{verbete}[la1la1dui4]{拉拉队}[8;8;4]
\begin{pronuncia}{la1la1dui4}
\significado{s.}{ claque; torcida }
\end{pronuncia}
\end{verbete}

\begin{verbete}[la4]{辣}[14]
\begin{pronuncia}{la4}
\significado{adj.}{ picante }
\end{pronuncia}
\end{verbete}

\begin{verbete}[lai2]{来}[7]
\begin{pronuncia}{lai2}
\significado{v.}{ vir; trazer }
\end{pronuncia}
\end{verbete}

\begin{verbete}[lan2]{蓝}[13]
\begin{pronuncia}{lan2}
\significado{adj.}{ azul }
\end{pronuncia}
\end{verbete}

\begin{verbete}[lan2se4]{蓝色}[13;6]
\begin{pronuncia}{lan2se4}
\significado{s.}{ cor azul }
\end{pronuncia}
\end{verbete}

\begin{verbete}[lan2qiu2]{篮球}[16;11]
\begin{pronuncia}{lan2qiu2}
\significado[个,只]{s.}{ basquetebol }
\end{pronuncia}
\end{verbete}

\begin{verbete}[lao3ban3]{老板}[6;8]
\begin{pronuncia}{lao3ban3}
\significado[个]{s.}{ patrão, patroa }
\end{pronuncia}
\end{verbete}

\begin{verbete}[lao3jia1]{老家}[6;10]
\begin{pronuncia}{lao3jia1}
\significado{s.}{ terra natal }
\end{pronuncia}
\end{verbete}

\begin{verbete}[lao3ren2jia0]{老人家}[6;2;10]
\begin{pronuncia}{lao3ren2jia0}
\significado{s.}{ senhor ancião; madame; senhora }
\end{pronuncia}
\end{verbete}

\begin{verbete}[lao3shi1]{老师}[6;6]
\begin{pronuncia}{lao3shi1}
\significado[个,位]{s.}{ professor }
\end{pronuncia}
\end{verbete}

\begin{verbete}[lei4]{累}[11]
\begin{pronuncia}{lei4}
\significado{adj.}{ cansado; fatigado }
\end{pronuncia}
\end{verbete}

\begin{verbete}[leng3]{冷}[7]
\begin{pronuncia}{leng3}
\significado{adj.}{ frio }
\end{pronuncia}
\end{verbete}

\begin{verbete}[lei4]{累}[11]
\begin{pronuncia}{lei4}
\significado{adj.}{ cansado }
\end{pronuncia}
\end{verbete}

\begin{verbete}[li2]{离}[10]
\begin{pronuncia}{li2}
\significado{prep.}{ (ser longe) de ... até... }
\end{pronuncia}
\end{verbete}

\begin{verbete}[li3]{里}[7]
\begin{pronuncia}{li3}
\significado{p.l.}{ em; dentro; interior }
\end{pronuncia}
\end{verbete}

\begin{verbete}[Li3si1ben3]{里斯本}[7;12;5]
\begin{pronuncia}{Li3si1ben3}
\significado{s.}{ Lisboa }
\end{pronuncia}
\end{verbete}

\begin{verbete}[Li3si1ben3\ Da4xue2]{里斯本大学}[7;12;5;3;8]
\begin{pronuncia}[\\]{Li3si1ben3\ Da4xue2}
\significado{s.}{ Universidade de Lisboa }
\end{pronuncia}
\end{verbete}

\begin{verbete}[li3jie2]{礼节}[5;5]
\begin{pronuncia}{li3jie2}
\significado{s.}{ cortesia; protocolo; cerimônia; etiqueta }
\end{pronuncia}
\end{verbete}

\begin{verbete}[li3wu4]{礼物}[5;8]
\begin{pronuncia}{li3wu4}
\significado[件,个,份]{s.}{ prenda; lembrança; presente }
\end{pronuncia}
\end{verbete}

\begin{verbete}[li4hai0]{厉害}[5;10]
\begin{pronuncia}{li4hai0}
\significado{adj.}{ severo; rigoroso; exigente }
\end{pronuncia}
\end{verbete}

\begin{verbete}[li4shi3]{历史}[4;5]
\begin{pronuncia}{li4shi3}
\significado[门,段]{s.}{ história }
\end{pronuncia}
\end{verbete}

\begin{verbete}[lian2ou3]{莲藕}[10;18]
\begin{pronuncia}{lian2ou3}
\significado{s.}{ raiz de Lotus }
\end{pronuncia}
\end{verbete}

\begin{verbete}[lian3]{脸}[11]
\begin{pronuncia}{lian3}
\significado[张,个]{s.}{ cara; rosto; face }
\end{pronuncia}
\end{verbete}

\begin{verbete}[lian4xi2]{练习}[8;3]
\begin{pronuncia}{lian4xi2}
\significado[个]{s.}{ exercício }
\significado{v.}{ praticar; exercitar }
\end{pronuncia}
\end{verbete}

\begin{verbete}[liang2kuai0]{凉快}[10;7]
\begin{pronuncia}{liang2kuai0}
\significado{adj.}{ agradável; fresco }
\end{pronuncia}
\end{verbete}

\begin{verbete}[liang3]{两}[7]
\begin{pronuncia}{liang3}
\significado{num.}{ dois; 2; (sempre usado antes de p.c.) }
\end{pronuncia}
\end{verbete}

\begin{verbete}[liang4]{辆}[11]
\begin{pronuncia}{liang4}
\significado{p.c.}{ palavra classificadora para automóveis, veículos, etc }
\end{pronuncia}
\end{verbete}

\begin{verbete}[lin2ju1]{邻居}[7;8]
\begin{pronuncia}{lin2ju1}
\significado[个]{s.}{ vizinho }
\end{pronuncia}
\end{verbete}

\begin{verbete}[ling2jiao3]{菱角}[11;7]]
\begin{pronuncia}{ling2jiao3}
\significado{s.}{ castanha d'água }
\end{pronuncia}
\end{verbete}

\begin{verbete}[ling2]{零/\Circle}[13]
\begin{pronuncia}{ling2}
\significado{num.}{ zero; 0 }
\end{pronuncia}
\end{verbete}

\begin{verbete}[ling3dao3]{领导}[11;6]
\begin{pronuncia}{ling3dao3}
\significado[位,个]{s.}{ chefe; dirigente }
\end{pronuncia}
\end{verbete}

\begin{verbete}[liu2li4]{流利}[10;7]
\begin{pronuncia}{liu2li4}
\significado{adj.}{ fluente }
\end{pronuncia}
\end{verbete}

\begin{verbete}[liu4]{六}[4]
\begin{pronuncia}{liu4}
\significado{num.}{seis; 6}
\end{pronuncia}
\end{verbete}

\begin{verbete}[liu4gou3]{遛狗}[13;8]
\begin{pronuncia}{liu4gou3}
\significado{v.+compl.}{ passear com o cachorro }
\end{pronuncia}
\end{verbete}

\begin{verbete}[long2]{龙}[5]
\begin{pronuncia}{long2}
\significado[条]{s.}{ dragão }
\end{pronuncia}
\end{verbete}

\begin{verbete}[Long2shan1]{龙山}[5;3]
\begin{pronuncia}{Long2shan1}
\significado{s.}{ Longshan }
\end{pronuncia}
\end{verbete}

\begin{verbete}[lou2]{楼}[13]
\begin{pronuncia}{lou2}
\significado{p.c.}{ andar; piso }
\significado[层,座,栋]{s.}{ edifício; prédio }
\end{pronuncia}
\end{verbete}

\begin{verbete}[lu2sun3]{芦笋}[7;10]
\begin{pronuncia}{lu2sun3}
\significado{s.}{ aspargos }
\end{pronuncia}
\end{verbete}

\begin{verbete}[lu4yin1]{录音}[8;9]
\begin{pronuncia}{lu4yin1}
\significado[个]{s.}{ gravação }
\significado{v.+compl.}{ gravar }
\end{pronuncia}
\end{verbete}

\begin{verbete}[lu4]{路}[13]
\begin{pronuncia}{lu4}
\significado[条]{s.}{ caminho; via }
\end{pronuncia}
\end{verbete}

\begin{verbete}[lu4kou3]{路口}[13;3]
\begin{pronuncia}{lu4kou3}
\significado{s.}{ cruzamento; interseção }
\end{pronuncia}
\end{verbete}

\begin{verbete}[lu4xiang4dai4]{录像带}[8;13;9]
\begin{pronuncia}{lu4xiang4dai4}
\significado[盘]{s.}{ video-cassete }
\end{pronuncia}
\end{verbete}

\begin{verbete}[lu4xiang4ji1]{录像机}[8;13;6]
\begin{pronuncia}{lu4xiang4ji1}
\significado[台]{s.}{ gravador de vídeo }
\end{pronuncia}
\end{verbete}

\begin{verbete}[lu4yin1ji1]{录音机}[8;9;6]
\begin{pronuncia}{lu4yin1ji1}
\significado[台]{s.}{ gravador de áudio }
\end{pronuncia}
\end{verbete}

\begin{verbete}[lun2dun1]{伦敦}[6;12]
\begin{pronuncia}{lun2dun1}
\significado{s.}{ Londres }
\end{pronuncia}
\end{verbete}

\begin{verbete}[lv3xing2]{旅行}[10;6]
\begin{pronuncia}{lv3xing2}
\significado{v.}{ viajar }
\end{pronuncia}
\end{verbete}

\begin{verbete}[lv3you2]{旅游}[10;12]
\begin{pronuncia}{lv3you2}
\significado{v.}{ viajar }
\end{pronuncia}
\end{verbete}

\begin{verbete}[lv4]{绿}[11]
\begin{pronuncia}{lv4}
\significado{adj.}{ verde }
\end{pronuncia}
\end{verbete}

\begin{verbete}[lv4dou4]{绿豆}[11;7]
\begin{pronuncia}{lv4dou4}
\significado{s.}{ vagens }
\end{pronuncia}
\end{verbete}

\begin{verbete}[lv4dou4ya2]{绿豆芽}[11;7;7]
\begin{pronuncia}{lv4dou4ya2}
\significado{s.}{ broto de feijão verde }
\end{pronuncia}
\end{verbete}

\begin{verbete}[lv4se4]{绿色}[11;6]
\begin{pronuncia}{lv4se4}
\significado{s.}{ cor verde }
\end{pronuncia}
\end{verbete}

%%%%% EOF %%%%%

%%%
%%% M
%%%
\section*{M}
\addcontentsline{toc}{section}{M}
\begin{multicols}{2}

\begin{hanzi}[吗]{ma0}
\entry{ma0}{part.}{ma0}{
    partícula interrogativa|
    usado em perguntas ``sim-não''
}
\end{hanzi}

\begin{hanzi}[妈妈]{ma1ma0}
\entry{ma1ma0}{n.}{ma1ma0}{
    mamãe, mãe
}
\end{hanzi}

\begin{hanzi}[麻烦]{ma2fan0}
\entry{ma2fan0}{adj.}{
    enfastidioso; maçante
}
\entry{ma2fan0}{n.}{
    incômodo
}
\entry{ma2fan0}{v.}{
    incomodar
}
\end{hanzi}

\begin{hanzi}[麻辣豆腐]{ma2la4dou4fu0}
\entry{ma2la4dou4fu0}{n.}{
    tofú guisado em molho picante
}
\end{hanzi}

\begin{hanzi}[马路]{ma3lu4}
\entry{ma3lu4}{n.}{
    rua|
    \pc{条}
}
\end{hanzi}

\begin{hanzi}[马上]{ma3shang4}
\entry{ma3shang4}{adv.}{
    já; imediatamente
}
\end{hanzi}

\begin{hanzi}[买]{mai3}
\entry{mai3}{v.}{
    comprar
}
\end{hanzi}

\begin{hanzi}[买东西]{mai3dong1xi0}
\entry{mai3dong1xi0}{v.}{
    fazer compras
}
\end{hanzi}

\begin{hanzi}[卖]{mai4}
\entry{mai4}{v.}{
    vender
} 
\end{hanzi}

\begin{hanzi}[猫]{mao1}
\entry{mao1}{n.}{
    gato|
    \pc{只}
}
\end{hanzi}

\begin{hanzi}[毛]{mao2}
\entry{mao2}{p.c.}{
    1 mao = 10 centavos
}
\end{hanzi}

\begin{hanzi}[忙]{mang1}
\entry{mang1}{adj.}{
    ocupado; ocupada
}
\end{hanzi}

\begin{hanzi}[忙]{mei2mao0}
\entry{mei2mao0}{n.}{
    sobrancelha
}
\end{hanzi}

\begin{hanzi}[没关系]{mei2guan1xi0}
\entry{mei2guan1xi0}{v.}{
    não ter problema; não ter importância; não fazer mal
}
\end{hanzi}

\begin{hanzi}[没有]{mei2you3}
\entry{mei2you3}{v.}{
    não há; não tem
}
\end{hanzi}

\begin{hanzi}[美国]{Mei2guo1}
\entry{Mei2guo1}{n.}{
    Estados Unidos da América
}
\end{hanzi}

\begin{hanzi}[每]{mei3}
\entry{mei3}{pron.}{
    cada
}
\end{hanzi}

\begin{hanzi}[每次]{mei3ci4}
\entry{mei3ci4}{adv.}{
    toda vez; cada vez
}
\end{hanzi}

\begin{hanzi}[每天]{mei3tian1}
\entry{mei3tian1}{adv.}{
    todo dia; cada dia
}
\end{hanzi}

\begin{hanzi}[美丽]{mei3li4}
\entry{mei3li4}{adj.}{
    bonito, bonita; lindo, linda
}
\end{hanzi}

\begin{hanzi}[美洲]{Mei3zhou1}
\entry{Mei3zhou1}{n.}{
    América
}
\end{hanzi}

\begin{hanzi}[妹夫]{mei4fu0}
\entry{mei4fu0}{n.}{
    marido da irmã mais nova
}
\end{hanzi}

\begin{hanzi}[妹妹]{mei4mei0}
\entry{mei4mei0}{n.}{
    irmã mais nova
}
\end{hanzi}

\begin{hanzi}[门口]{men2kou3}
\entry{men2kou3}{p.l.}{
    entrada; porta
}
\end{hanzi}

\begin{hanzi}[们]{men0}
\entry{men0}{sufixo}{
    sufixo para plural (pessoas e pronomes)
}
\end{hanzi}

\begin{hanzi}[面]{mian4}
\entry{mian4}{n.}{
    farinha; massa
}
\end{hanzi}

\begin{hanzi}[面包]{mian4bao1}
\entry{mian4bao1}{n.}{
    pão|
    \pc{个/块/片}
}
\end{hanzi}

\begin{hanzi}[面积]{mian4ji1}
\entry{mian4ji1}{n.}{
    área; superfície
}
\end{hanzi}

\begin{hanzi}[面条]{mian4tiao2}
\entry{mian4tiao2}{n.}{
    massa; espaguete
}
\end{hanzi}

\begin{hanzi}[名片]{ming2pian4}
\entry{ming2pian4}{n.}{
    cartão de visita
}
\end{hanzi}

\begin{hanzi}[名字]{ming2zi0}
\entry{ming2zi0}{n.}{
    nome
}
\end{hanzi}

\begin{hanzi}[明白]{ming2bai0}
\entry{ming2bai0}{adj.}{
    compreendido; percebido
}
\entry{ming2bai0}{v.}{
    compreender; perceber
}
\end{hanzi}

\begin{hanzi}[明天]{ming2tian1}
\entry{ming2tian1}{p.t.}{
    amanhã
}
\end{hanzi}

\begin{hanzi}[明年]{ming2nian2}
\entry{ming2nian2}{n.}{
    próximo ano
}
\end{hanzi}

\begin{hanzi}[墨镜]{mo4jing4}
\entry{mo4jing4}{n.}{
    óculos escuros
}
\end{hanzi}

\begin{hanzi}[母亲]{mu3qin0}                                               
\entry{mu3quin0}{n.}{
    mãe
}                                                 
\end{hanzi}                                                                     

\begin{hanzi}[米饭]{mv3fan4}
\entry{mv3fan4}{n.}{
    arroz cozido
}
\end{hanzi}

\end{multicols}

%%%
%%% N
%%%
\section*{N}
\addcontentsline{toc}{section}{N}
%\begin{multicols*}{2}

\begin{verbete}[na2]{拿}
\begin{pronuncia}{na2}
\significado{v.}{
segurar; tomar; pegar em
}
\end{pronuncia}
\end{verbete}

\begin{verbete}[na3]{哪}
\begin{pronuncia}{na3}
\significado{interr.}{
que?; qual?
}
\end{pronuncia}
\end{verbete}

\begin{verbete}[nar3]{哪儿}
\begin{pronuncia}{nar3}
\significado{interr.}{
onde?
}
\end{pronuncia}
\end{verbete}

\begin{verbete}[na3guo2ren2]{哪国人}
\begin{pronuncia}{na3guo2ren2}
\significado{interr.}{
de qual país?
}
\end{pronuncia}
\end{verbete}

\begin{verbete}[na3li0]{哪里}
\begin{pronuncia}{na3li0}
\significado{interr.}{
onde?
}
\end{pronuncia}
\end{verbete}

\begin{verbete}[na3xie1]{哪些}
\begin{pronuncia}{na3xie1}
\significado{interr.}{
quais?
}
\end{pronuncia}
\end{verbete}

\begin{verbete}[na4]{那}
\begin{pronuncia}{na4}
\significado{conj.}{
nessa situação; nesse caso
}
\significado{pron.}{
aquele; aquilo; aquela
}
\end{pronuncia}
\end{verbete}

\begin{verbete}[na4li0]{那里}
\begin{pronuncia}{na4li0}
\significado{pron.}{
lá; ali
}
\end{pronuncia}
\end{verbete}

\begin{verbete}[na4me0]{那么}
\begin{pronuncia}{na4me0}
\significado{adv.}{
então; como aquele; dessa maneira
}
\end{pronuncia}
\end{verbete}

\begin{verbete}[na4xie1]{那些}
\begin{pronuncia}{na4xie1}
\significado{pron.}{
aqueles, aquelas
}
\end{pronuncia}
\end{verbete}

\begin{verbete}[nar4]{那儿}
\begin{pronuncia}{nar4}
\significado{pron.}{
lá; ali
}
\end{pronuncia}
\end{verbete}

\begin{verbete}[nai3nai0]{奶奶}
\begin{pronuncia}{nai3nai0}
\significado[位]{n.}{
avó(paterna)|
dona da casa
}
\end{pronuncia}
\end{verbete}

\begin{verbete}[nan2]{男}
\begin{pronuncia}{nan2}
\significado{adj.}{
masculino
}
\end{pronuncia}
\end{verbete}

\begin{verbete}[nan2peng2you0]{男朋友}
\begin{pronuncia}{nan2peng2you0}
\significado{n.}{
namorado
}
\end{pronuncia}
\end{verbete}

\begin{verbete}[nan2hair2]{男孩儿}
\begin{pronuncia}{nan2hair2}
\significado{n.}{
menino; rapaz
}
\end{pronuncia}
\end{verbete}

\begin{verbete}[nan2bian0]{南边}
\begin{pronuncia}{nan2bian0}
\significado{p.l.}{
sul
}
\end{pronuncia}
\end{verbete}

\begin{verbete}[nan2fang1]{南方}
\begin{pronuncia}{nan2fang1}
\significado{p.l.}{
sul
}
\end{pronuncia}
\end{verbete}

\begin{verbete}[nan2mian0]{南面}
\begin{pronuncia}{nan2mian0}
\significado{p.l.}{
sul
}
\end{pronuncia}
\end{verbete}

\begin{verbete}[nan2]{难}
\begin{pronuncia}{nan2}
\significado{adj.}{
difícil
}
\end{pronuncia}
\end{verbete}

\begin{verbete}[ne0]{呢}
\begin{pronuncia}{ne0}
\significado{interr.}{
partícula interrogativa enfática
}
\end{pronuncia}
\end{verbete}

\begin{verbete}[neng2]{能}
\begin{pronuncia}{neng2}
\significado{v.}{
poder
}
\end{pronuncia}
\end{verbete}

\begin{verbete}[ni3]{你}
\begin{pronuncia}{ni3}
\significado{pron.}{
você (informal); tu
}
\end{pronuncia}
\end{verbete}

\begin{verbete}[ni3de0]{你的}
\begin{pronuncia}{ni3de0}
\significado{pron.}{
seu, sua
}
\end{pronuncia}
\end{verbete}

\begin{verbete}[ni3men0]{你们}
\begin{pronuncia}{ni3men0}
\significado{pron.}{
vocês (informal); vós
}
\end{pronuncia}
\end{verbete}

\begin{verbete}[ni3men0de0]{你们的}
\begin{pronuncia}{ni3men0de0}
\significado{pron.}{
vossos, vossas
}
\end{pronuncia}
\end{verbete}

\begin{verbete}[nian2]{年}
\begin{pronuncia}{nian2}
\significado{p.c.}{
ano
}
\significado{p.t.}{
ano
}
\end{pronuncia}
\end{verbete}

\begin{verbete}[nian2ji2]{年级}
\begin{pronuncia}{nian2ji2}
\significado[个]{n.}{
classe; ano (escola)
}
\end{pronuncia}
\end{verbete}

\begin{verbete}[nian2ji4]{年纪}
\begin{pronuncia}{nian2ji4}
\significado[把,个]{n.}{
idade
}
\end{pronuncia}
\end{verbete}

\begin{verbete}[nian2huo4]{年货}
\begin{pronuncia}{nian2huo4}
\significado{n.}{
mercadorias de Ano Novo Chinês
}
\end{pronuncia}
\end{verbete}

\begin{verbete}[nian2qing1]{年轻}
\begin{pronuncia}{nian2qing1}
\significado{adj.}{
jovem
}
\end{pronuncia}
\end{verbete}

\begin{verbete}[niaor3]{鸟儿}
\begin{pronuncia}{niaor3}
\significado[只]{n.}{
pássaro; ave
}
\end{pronuncia}
\end{verbete}

\begin{verbete}[nin2]{您}
\begin{pronuncia}{nin2}
\significado{pron.}{
você (formal); tu
}
\end{pronuncia}
\end{verbete}

\begin{verbete}[niu2]{牛}
\begin{pronuncia}{niu2}
\significado[条,头]{n.}{
boi, vaca|
gíria: incrível
}
\end{pronuncia}
\end{verbete}

\begin{verbete}[niu2nai3]{牛奶}
\begin{pronuncia}{niu2nai3}
\significado[瓶,杯]{n.}{
leite
}
\end{pronuncia}
\end{verbete}

\begin{verbete}[niu2rou4]{牛肉}
\begin{pronuncia}{niu2rou4}
\significado{n.}{
carne de vaca
}
\end{pronuncia}
\end{verbete}

\begin{verbete}[niu2zai3ku4]{牛仔裤}
\begin{pronuncia}{niu2zai3ku4}
\significado[条]{n.}{
calça de ganga, jeans
}
\end{pronuncia}
\end{verbete}

\begin{verbete}[nong2cun1]{农村}
\begin{pronuncia}{nong2cun1}
\significado[个]{n.}{
campo rural; aldeia; povoação rústica
}
\end{pronuncia}
\end{verbete}

\begin{verbete}[nu3li4]{努力}
\begin{pronuncia}{nu3li4}
\significado{adj.}{
diligente; aplicado
}
\end{pronuncia}
\end{verbete}

\begin{verbete}[nuan3huo0]{暖和}
\begin{pronuncia}{nuan3huo0}
\significado{adj.}{
morno, morna; quente
}
\end{pronuncia}
\end{verbete}

\begin{verbete}[nuan3qi4]{暖气}
\begin{pronuncia}{nuan3qi4}
\significado{n.}{
aquecimento
}
\end{pronuncia}
\end{verbete}

\begin{verbete}[nv3]{女}
\begin{pronuncia}{nv3}
\significado{adj.}{
feminino
}
\end{pronuncia}
\end{verbete}

\begin{verbete}[nv3'er2]{女儿}
\begin{pronuncia}{nv3'er2}
\significado{n.}{
filha
}
\end{pronuncia}
\end{verbete}

\begin{verbete}[nv3hair2]{女孩儿}
\begin{pronuncia}{nv3hair2}
\significado{n.}{
menina; garota
}
\end{pronuncia}
\end{verbete}

\begin{verbete}[nv3peng2you0]{女朋友}
\begin{pronuncia}{nv3peng2you0}
\significado{n.}{
namorada
}
\end{pronuncia}
\end{verbete}

\begin{verbete}[nv3wang2]{女王}
\begin{pronuncia}{nv3wang2}
\significado{n.}{
rainha
}
\end{pronuncia}
\end{verbete}

\begin{verbete}[nv3xu4]{女婿}
\begin{pronuncia}{nv3xu4}
\significado{n.}{
marido da filha
}
\end{pronuncia}
\end{verbete}

%\end{multicols*}

%%%
%%% O
%%%
\section*{O}
\addcontentsline{toc}{section}{O}

\begin{verbete}[Ou1zhou1]{欧洲}[8;9]
\begin{pronuncia}{Ou1zhou1}
\significado{s.}{ Europa }
\end{pronuncia}
\end{verbete}

\begin{verbete}[ou1zhou1ren2]{欧洲人}[8;9;2]
\begin{pronuncia}{ou1zhou1ren2}
\significado{s.}{ europeu; nascido na Europa }
\end{pronuncia}
\end{verbete}

%%%%% EOF %%%%%

%%%
%%% P
%%%
\section*{P}
\addcontentsline{toc}{section}{P}
\begin{multicols}{2}

\begin{hanzi}[爬]{pa2}
\entry{pa2}{v.}{
    escalar; trepar
}
\end{hanzi}

\begin{hanzi}[怕]{pa4}
\entry{pa4}{v.}{
    ter medo de
}
\end{hanzi}

\begin{hanzi}[拍照]{pai1zhao4}
\entry{pai1zhao4}{v.+compl.}{
    tirar fotografia
}
\end{hanzi}

\begin{hanzi}[排球]{pai2qiu2}
\entry{pai2qiu2}{n.}{
    voleibol
}
\end{hanzi}

\begin{hanzi}[盘]{pan2}
\entry{pan2}{p.c.}{
    para cassete; video-cassete
}
\end{hanzi}

\begin{hanzi}[旁边]{pang2bian1}
\entry{pang2bian1}{p.l.}{
    junto a; próximo de; ao lado
}
\end{hanzi}

\begin{hanzi}[胖]{pang4}
\entry{pang4}{adj.}{
    gordo, gorda
}
\end{hanzi}

\begin{hanzi}[跑步]{pao3bu4}
\entry{pao3bu4}{v.}{
    correr
}
\end{hanzi}

\begin{hanzi}[陪]{pei2}
\entry{pei2}{v.}{
    acompanhar
}
\end{hanzi}

\begin{hanzi}[配]{pei3}
\entry{pei3}{v.}{
    combinar
}
\end{hanzi}

\begin{hanzi}[朋友]{peng2you0}
\entry{peng2you0}{n.}{
    namorado, namorada|
    amigo, amiga
}
\end{hanzi}

\begin{hanzi}[啤酒]{pi2jiu3}
\entry{pi2jiu3}{n.}{
    cerveja
}
\end{hanzi}

\begin{hanzi}[啤酒馆]{pi2jiu3guan3}
\entry{pi2jiu3guan3}{n.}{
    cervejaria
}
\end{hanzi}

\begin{hanzi}[屁股]{pi4gu}
\entry{pi4gu}{n.}{
    nádega; quadris
}
\end{hanzi}

\begin{hanzi}[票]{piao4}
\entry{piao4}{n.}{
    bilhete|
    \pc{张}
}
\end{hanzi}

\begin{hanzi}[漂亮]{piao4liang0}
\entry{piao4liang0}{adj.}{
    bonita, linda|
    bonito, lindo (para objetos inanimados)
}
\end{hanzi}

\begin{hanzi}[片]{pian4}
\entry{pian4}{p.c.}{
    palavra classificadora, para algumas coisas finas, com pouca espessura; 
    fatia, rodela
}
\end{hanzi}

\begin{hanzi}[瓶]{ping2}
\entry{ping2}{n.}{garrafa}
\entry{ping2}{p.c.}{
    palavra classificadora, garrafa
}
\end{hanzi}

\begin{hanzi}[平时]{ping2shi2}
\entry{ping2shi2}{p.t.}{
    normalmente; numa época normal
}
\end{hanzi}

\begin{hanzi}[苹果]{ping2guo3}
\entry{ping2guo3}{n.}{
    maçã
}
\end{hanzi}

\begin{hanzi}[葡汉词典]{pu2han4ci2dian3}
\entry{pu2han4ci2dian3}{n.}{
    dicionário português-chinês
}
\end{hanzi}

\begin{hanzi}[葡萄牙]{Pu2tao2ya2}
\entry{Pu2tao2ya2}{n.}{
    Portugal
}
\end{hanzi}

\begin{hanzi}[葡萄牙语]{pu2tao2ya2yu3}
\entry{pu2tao2ya2yu3}{n.}{
    português, língua portuguesa
}
\end{hanzi}

\begin{hanzi}[葡语]{pu2yu3}
\entry{pu2yu3}{n.}{
    português, língua portuguesa
}
\end{hanzi}

\begin{hanzi}[葡文]{pu2wen2}
\entry{pu2wen2}{n.}{
    português, língua portuguesa
}
\end{hanzi}

\begin{hanzi}[普通话]{pu3tong1hua4}
\entry{pu3tong1hua4}{n.}{
    mandarim (lit. ``linguagem comum'')
}
\end{hanzi}

\begin{hanzi}[便宜]{pian2yi0}
\entry{pian2yi0}{adj.}{
    barato
}
\end{hanzi}

\begin{hanzi}[乒乓球]{ping1pang1qiu2}
\entry{ping1pang1qiu2}{n.}{
    tênis de mesa; ping-pong
}
\end{hanzi}

\end{multicols}

%%%
%%% Q
%%%
\section*{Q}
\addcontentsline{toc}{section}{Q}

\begin{verbete}[2]{七}{qi1}
  \significado{num.}{7, sete}
\end{verbete}

\begin{verbete}[11]{骑}{qi2}
  \significado{p.c.}{para cavalos de sela}
  \significado{v.}{andar (cavalo, bicicleta, etc.); sentar-se montado}
\end{verbete}

\begin{verbete}[11;4]{骑车}{qi2che1}
  \significado{v.}{andar de bicicleta; pedalar}
\end{verbete}

\begin{verbete}[6;5]{企业}{qi3ye4}
  \significado[家]{s.}{empresa; corporação; empreendimento; firma}
\end{verbete}

\begin{verbete}[6;6;6;11]{岂有此理}[\\]{qi3you3ci3li3}
  \significado{interj.}{Absurdo!; Como isso pode ser assim?; Ridículo!}
\end{verbete}

\begin{verbete}[10;7]{起床}{qi3chuang2}
  \significado{v.+compl.}{sair da cama; levantar-se}
\end{verbete}

\begin{verbete}[10;7]{起来}{qi3lai0}
  \significado{v.+compl.}{levantar-se}
\end{verbete}

\begin{verbete}[7;4]{汽车}{qi4che1}
  \significado[辆]{s.}{automóvel; carro; veículo motorizado}
\end{verbete}

\begin{verbete}[4;12]{气温}{qi4wen1}
  \significado[个]{s.}{temperatura do ar}
\end{verbete}

\begin{verbete}[3]{千}{qian1}
  \significado{num.}{1.000, mil}
\end{verbete}

\begin{verbete}[3;3;3;3]{千千万万}[\\]{qian1qian1wan4wan4}
  \significado{num.}{inumerável; números incontáveis; milhares e milhares}
\end{verbete}

\begin{verbete}[13]{签}{qian1}
  \significado{s.}{vara de bambu com inscrição (usada em adivinhação, jogos de azar, sorteios, etc.); rótulo; pequena lasca de madeira; etiqueta}
\end{verbete}

\begin{verbete}[13;6]{签}{qian1ming2}
  \significado{s.}{assinatura}
  \significado{v.}{autografar; assinar (o nome com uma caneta, etc.)}
\end{verbete}

\begin{verbete}[9]{前}{qian2}
  \significado{p.l.}{frente; em frente de; AC (por exemplo, 前293年)}
  \veja{公元}{gong1yuan2}
\end{verbete}

\begin{verbete}[9;6]{前年}{qian2nian2}
  \significado{p.t.}{há dois anos}
\end{verbete}

\begin{verbete}[9;9]{前面}{qian2bian0}
  \significado{p.l.}{à frente; da frente}
\end{verbete}

\begin{verbete}[9;9]{前面}{qian2mian0}
  \significado{p.l.}{à frente; da frente}
\end{verbete}

\begin{verbete}[9;4]{前天}{qian2tian1}
  \significado{p.t.}{anteontem}
\end{verbete}

\begin{verbete}[10]{钱}{qian2}
  \significado[笔]{s.}{moeda; dinheiro}
\end{verbete}

\begin{verbete}[10;5]{钱包}{qian2bao1}
  \significado{s.}{carteira; bolsa}
\end{verbete}

\begin{verbete}[12]{强}{qiang2}
  \significado{adj.}{melhor em sua categoria; melhor; poderoso; forte; vigoroso; violento}
  \veja{强}{jiang4}
  \veja{强}{qiang3}
\end{verbete}

\begin{verbete}[12]{强}{qiang3}
  \significado{v.}{obrigar; forçar; fazer um esforço; esforçar-se}
  \veja{强}{jiang4}
  \veja{强}{qiang2}
\end{verbete}

\begin{verbete}[5;7;2]{巧克力}{qiao3ke4li4}
  \significado[块]{s.}{chocolate}
\end{verbete}

\begin{verbete}[8;3]{茄子}{qie2zi0}
  \significado{s.}{berinjela chinesa; ``xis'' fonético (ao ser fotografado), equivale ao ``diga xis''}
\end{verbete}

\begin{verbete}[7;11]{芹菜}{qin2cai4}
  \significado{s.}{salsão}
\end{verbete}

\begin{verbete}[8;11]{青菜}{qing1cai4}
  \significado{s.}{verduras}
\end{verbete}

\begin{verbete}[8;11]{青椒}{qing1jiao1}
  \significado{s.}{pimenta verde}
\end{verbete}

\begin{verbete}[8;4]{青天}{qing1tian1}
  \significado{s.}{céu claro; céu limpo; céu azul}
\end{verbete}

\begin{verbete}[8;5;6]{青玉米}{qing1yu4mi3}
  \significado{s.}{milho verde}
\end{verbete}

\begin{verbete}[11;13]{清楚}{qing1chu0}
  \significado{adj.}{claro; límpido}
  \significado{v.}{ser claro sobre; entender completamente}
\end{verbete}

\begin{verbete}[10]{请}{qing3}
  \significado{v.}{por favor (fazer alguma coisa); perguntar; convidar; solicitar}
\end{verbete}

\begin{verbete}[10;9]{请客}{qing3ke4}
  \significado{v.+compl.}{entreter os convidados; dar um jantar; convidar para jantar}
\end{verbete}

\begin{verbete}[10;6]{请问}{qing3wen4}
  \significado{expr.}{Com licença, posso perguntar...? (para perguntar por qualquer coisa)}
\end{verbete}

\begin{verbete}[10;11;7]{请假条}[\\]{qing3jia4tiao2}
  \significado{s.}{pedido de licença de ausência (do trabalho ou da escola)}
\end{verbete}

\begin{verbete}[9;4]{秋天}{qiu1tian1}
  \significado[个]{s.}{outono}
  \significado{p.t.}{outono}
\end{verbete}

\begin{verbete}[11]{球}{qiu2}
  \significado[个]{s.}{bola; esfera; globo}
  \significado[场]{s.}{jogo; partida de bola}
\end{verbete}

\begin{verbete}[6;12;11]{曲棍球}[\\]{qu1gun4qiu2}
  \significado{s.}{hóquei em campo}
\end{verbete}

\begin{verbete}[5]{去}{qu4}
  \significado{v.}{ir; eufenismo: morrer}
\end{verbete}

\begin{verbete}[5;6]{去年}{qu4nian2}
  \significado{s.}{ano passado}
\end{verbete}

\begin{verbete}[12;3]{裙子}{qun2zi0}
  \significado[条]{s.}{saia; vestido}
\end{verbete}

%%%%% EOF %%%%%

%%%
%%% R
%%%
\section*{R}
\addcontentsline{toc}{section}{R}
\begin{multicols*}{2}

\begin{verbete}[ran2hou4]{然后}
\begin{pronuncia}{ran2hou4}
\significado{conj.}{
depois; logo; portanto
}
\end{pronuncia}
\end{verbete}

\begin{verbete}[rang4]{让}
\begin{pronuncia}{rang4}
\significado{v.}{
deixar; permitir
}
\end{pronuncia}
\end{verbete}

\begin{verbete}[re4]{热}
\begin{pronuncia}{re4}
\significado{adj.}{
quente
}
\end{pronuncia}
\end{verbete}

\begin{verbete}[re4nao0]{热闹}
\begin{pronuncia}{re4nao0}
\significado{adj.}{
animado; movimentado
}
\end{pronuncia}
\end{verbete}

\begin{verbete}[ren2]{人}
\begin{pronuncia}{ren2}
\significado[个,位]{n.}{
pessoa; gente
}
\end{pronuncia}
\end{verbete}

\begin{verbete}[ren2kou3]{人口}
\begin{pronuncia}{ren2kou3}
\significado{n.}{
população
}
\end{pronuncia}
\end{verbete}

\begin{verbete}[Ren2min2bi4]{人民币}
\begin{pronuncia}{Ren2min2bi4}
\significado{n.}{
RMB; CYN|
nome da moeda chinesa
}
\end{pronuncia}
\end{verbete}

\begin{verbete}[ren4shi0]{认识}
\begin{pronuncia}{ren4shi0}
\significado{v.}{
conhecer
}
\end{pronuncia}
\end{verbete}

\begin{verbete}[ri4]{日}
\begin{pronuncia}{ri4}
\significado{p.c.}{
dia (mais usado em escrita)
}
\end{pronuncia}
\end{verbete}

\begin{verbete}[Ri4ben3]{日本}
\begin{pronuncia}{Ri4ben3}
\significado{n.}{
Japão
}
\end{pronuncia}
\end{verbete}

\begin{verbete}[rong2yi4]{容易}
\begin{pronuncia}{rong2yi4}
\significado{adj.}{
fácil
}
\end{pronuncia}
\end{verbete}

\begin{verbete}[rou4]{肉}
\begin{pronuncia}{rou4}
\significado{n.}{
carne; polpa de uma fruta
}
\end{pronuncia}
\end{verbete}

\begin{verbete}[ru2guo3]{如果}
\begin{pronuncia}{ru2guo3}
\significado{conj.}{
se; caso; no caso de
}
\end{pronuncia}
\end{verbete}

\begin{verbete}[ru3fang2]{乳房}
\begin{pronuncia}{ru3fang2}
\significado{n.}{
seio; mama
}
\end{pronuncia}
\end{verbete}

\end{multicols*}

%%%
%%% S
%%%
\section*{S}
\addcontentsline{toc}{section}{S}

\begin{verbete}[san1]{三}[3]
\begin{pronuncia}{san1}
\significado{num.}{ três; 3 }
\end{pronuncia}
\end{verbete}

\begin{verbete}[san4bu4]{散步}[12;7]
\begin{pronuncia}{san4bu4}
\significado{v.+compl.}{ dar um passeio; passear }
\end{pronuncia}
\end{verbete}

\begin{verbete}[sao3zi0]{嫂子}[12;3]
\begin{pronuncia}{sao3zi0}
\significado{s.}{ esposa do irmão mais velho }
\end{pronuncia}
\end{verbete}

\begin{verbete}[sen1lin2]{森林}[12;8]
\begin{pronuncia}{sen1lin2}
\significado{s.}{ floresta }
\end{pronuncia}
\end{verbete}

\begin{verbete}[sha1mo4]{沙漠}[7;13]
\begin{pronuncia}{sha1mo4}
\significado[个]{s.}{ deserto }
\end{pronuncia}
\end{verbete}

\begin{verbete}[shan1]{山}[3]
\begin{pronuncia}{shan1}
\significado[座]{s.}{ montanha; monte }
\end{pronuncia}
\end{verbete}

\begin{verbete}[Shan4dong3]{山东}[3;5]
\begin{pronuncia}{Shan4dong3}
\significado{s.}{ Shandong }
\end{pronuncia}
\end{verbete}

\begin{verbete}[shan1qu1]{山区}[3;4]
\begin{pronuncia}{shan1qu1}
\significado[个]{s.}{ área montanhosa; montanhas }
\end{pronuncia}
\end{verbete}

\begin{verbete}[shang1]{伤}[6]
\begin{pronuncia}{shang1}
\significado{s.}{ ferida }
\significado[家,个]{v.}{ ferir; ferir-se }
\end{pronuncia}
\end{verbete}

\begin{verbete}[shang1dian4]{商店}[11;8]
\begin{pronuncia}{shang1dian4}
\significado[家,个]{s.}{ loja }
\end{pronuncia}
\end{verbete}

\begin{verbete}[shang1mao4]{商贸}[11;9]
\begin{pronuncia}{shang1mao4}
\significado{s.}{ comércio }
\end{pronuncia}
\end{verbete}

\begin{verbete}[shang3ci4]{赏赐}[12;12]
\begin{pronuncia}{shang3ci4}
\significado{s.}{ recompensa; prêmio }
\significado{v.}{ recompensar; premiar }
\end{pronuncia}
\end{verbete}

\begin{verbete}[shang4]{上}[3]
\begin{pronuncia}{shang4}
\significado{p.l.}{ acima; em cima de }
\significado{v.d.}{ subir }
\end{pronuncia}
\end{verbete}

\begin{verbete}[shang4ban1]{上班}[3;10]
\begin{pronuncia}{shang4ban1}
\significado{v.+compl.}{ ir para o trabalho; ir para o emprego }
\end{pronuncia}
\end{verbete}

\begin{verbete}[shang4bian0]{上边}[3;5]
\begin{pronuncia}{shang4bian0}
\significado{p.l.}{ acima de; parte de cima; por cima }
\end{pronuncia}
\end{verbete}

\begin{verbete}[shang4che0]{上车}[3;4]
\begin{pronuncia}{shang4che0}
\significado{v.}{ entrar (em ônibus) }
\end{pronuncia}
\end{verbete}

\begin{verbete}[Shang4hai3]{上海}[3;10]
\begin{pronuncia}{Shang4hai3}
\significado{s.}{ Shangai (Xangai) }
\end{pronuncia}
\end{verbete}

\begin{verbete}[shang4ke4]{上课}[3;10]
\begin{pronuncia}{shang4ke4}
\significado{v.}{ ter aulas }
\end{pronuncia}
\end{verbete}

\begin{verbete}[shang4lai0]{上来}[3;7]
\begin{pronuncia}{shang4lai0}
\significado{v.}{ subir (para a minha localização) }
\end{pronuncia}
\end{verbete}

\begin{verbete}[shang4mian0]{上面}[3;9]
\begin{pronuncia}{shang4mian0}
\significado{p.l.}{ acima de; parte de cima; por cima }
\end{pronuncia}
\end{verbete}

\begin{verbete}[shang4qu0]{上去}[3;5]
\begin{pronuncia}{shang4qu0}
\significado{v.}{ subir (a partir da minha localização) }
\end{pronuncia}
\end{verbete}

\begin{verbete}[shang4wang3]{上网}[3;6]
\begin{pronuncia}{shang4wang3}
\significado{v.}{ acessar a Internet }
\end{pronuncia}
\end{verbete}

\begin{verbete}[shang4wu3]{上午}[3;4]
\begin{pronuncia}{shang4wu3}
\significado{p.t.}{ manhã; de manhã; período antes do meio-dia }
\end{pronuncia}
\end{verbete}

\begin{verbete}[shang4xun2]{上询}[3;8]
\begin{pronuncia}{shang4xun2}
\significado{p.t.}{ primeira dezena do mês }
\end{pronuncia}
\end{verbete}

\begin{verbete}[shao3]{少}[4]
\begin{pronuncia}{shao3}
\significado{adj.}{ pouco, poucos }
\end{pronuncia}
\end{verbete}

\begin{verbete}[she2tou0]{舌头}[6;5]
\begin{pronuncia}{she2tou0}
\significado[个]{s.}{ língua }
\end{pronuncia}
\end{verbete}

\begin{verbete}[she2shi4]{摄氏}[13;4]
\begin{pronuncia}{she2shi4}
\significado{s.}{ Celsius, centígrado }
\end{pronuncia}
\end{verbete}

\begin{verbete}[shei2]{谁}[10]
\begin{pronuncia}{shei2}
\significado{interr.}{ quem? }
\end{pronuncia}
\begin{pronuncia}{shui2}
\significado*{}{ 谁\p{shui2} }
\end{pronuncia}
\end{verbete}

\begin{verbete}[shen1ti3]{身体}[7;7]
\begin{pronuncia}{shen1ti3}
\significado[具,个]{s.}{ corpo; saúde }
\end{pronuncia}
\end{verbete}

\begin{verbete}[shen2me0]{什么}[4;3]
\begin{pronuncia}{shen2me0}
\significado{interr.}{ que?; o que? }
\end{pronuncia}
\end{verbete}

\begin{verbete}[shen2me0shi2hou0]{什么时候}[4;3;7;10]
\begin{pronuncia}{shen2me0shi2hou0}
\significado{interr.}{ quando?; a que horas? }
\end{pronuncia}
\end{verbete}

\begin{verbete}[sheng1]{生}[5]
\begin{pronuncia}{sheng1}
\significado{adj.}{ cru; não cozido }
\end{pronuncia}
\end{verbete}

\begin{verbete}[sheng1ri0]{生日}[5;4]
\begin{pronuncia}{sheng1ri0}
\significado[个]{s.}{ aniversário; dia de anos }
\end{pronuncia}
\end{verbete}

\begin{verbete}[sheng1yi0]{生意}[5;13]
\begin{pronuncia}{sheng1yi0}
\significado[笔]{s.}{ negócio }
\end{pronuncia}
\end{verbete}

\begin{verbete}[sheng1yu2pian4]{生鱼片}[5;8;4]
\begin{pronuncia}{sheng1yu2pian4}
\significado{s.}{ fatias de peixe cru, \textit{sashimi} }
\end{pronuncia}
\end{verbete}

\begin{verbete}[Sheng4zhang3]{省长}[9;4]
\begin{pronuncia}{Sheng4zhang3}
\significado{s.}{ Governador; governador de uma província }
\end{pronuncia}
\end{verbete}

\begin{verbete}[Sheng4dan4jie2]{圣诞节}[5;8;5]
\begin{pronuncia}{Sheng4dan4jie2}
\significado{s.}{ Natal }
\end{pronuncia}
\end{verbete}

\begin{verbete}[shi2]{十}[2]
\begin{pronuncia}{shi2}
\significado{num.}{ dez; dezena; 10 }
\end{pronuncia}
\end{verbete}

\begin{verbete}[shi2hou0]{时候}[7;10]
\begin{pronuncia}{shi2hou0}
\significado{s.}{ horas; tempo }
\significado{interr.}{ quando? }
\end{pronuncia}
\end{verbete}

\begin{verbete}[shi2jian1]{时间}[7;7]
\begin{pronuncia}{shi2jian1}
\significado{s.}{ (conceito de, duração de, um ponto no) tempo }
\end{pronuncia}
\end{verbete}

\begin{verbete}[shi2pin3]{食品}[9;9]
\begin{pronuncia}{shi2pin3}
\significado[种]{s.}{ comida; alimento; produtos alimentícios }
\end{pronuncia}
\end{verbete}

\begin{verbete}[shi2tang2]{食堂}[9;11]
\begin{pronuncia}{shi2tang2}
\significado[个,间]{s.}{ cantina; sala de jantar }
\end{pronuncia}
\end{verbete}

\begin{verbete}[shi4chang3]{市场}[5;6]
\begin{pronuncia}{shi4chang3}
\significado{s.}{ mercado }
\end{pronuncia}
\end{verbete}

\begin{verbete}[shi4qu1]{市区}[5;4]
\begin{pronuncia}{shi4qu1}
\significado{s.}{ cidade própria; distrito urbano }
\end{pronuncia}
\end{verbete}

\begin{verbete}[shi4zhong1xin1]{市中心}[5;4;4]
\begin{pronuncia}{shi4zhong1xin1}
\significado{s.}{ centro da cidade }
\end{pronuncia}
\end{verbete}

\begin{verbete}[shi4]{事}[8]
\begin{pronuncia}{shi4}
\significado[件,桩,回]{s.}{ coisa; assunto; item; trabalho; caso }
\end{pronuncia}
\end{verbete}

\begin{verbete}[shir4]{事儿}[8;2]
\begin{pronuncia}{shir4}
\significado[件,桩]{s.}{ o emprego; negócio; afazeres; assunto que precisa ser resolvido; matéria }
\end{pronuncia}
\end{verbete}

\begin{verbete}[shi4gu4]{事故}[8;9]
\begin{pronuncia}{shi4gu4}
\significado[桩,起,次]{s.}{ acidente }
\end{pronuncia}
\end{verbete}

\begin{verbete}[shi4]{试}[8]
\begin{pronuncia}{shi4}
\significado{v.}{ experimentar; provar }
\end{pronuncia}
\end{verbete}

\begin{verbete}[shi4]{室}[9]
\begin{pronuncia}{shi4}
\significado{s.}{ quarto }
\end{pronuncia}
\end{verbete}

\begin{verbete}[shi4]{是}[9]
\begin{pronuncia}{shi4}
\significado{v.}{ ser }
\end{pronuncia}
\end{verbete}

\begin{verbete}[shi4de]{是的}[9;8]
\begin{pronuncia}{shi4de}
\significado{adv.}{shi4de}{ sim }
\end{pronuncia}
\end{verbete}

\begin{verbete}[shou1dao4]{收到}[6;8]
\begin{pronuncia}{shou1dao4}
\significado{v.}{ receber }
\end{pronuncia}
\end{verbete}

\begin{verbete}[shou3]{手}[4]
\begin{pronuncia}{shou3}
\significado[双,只]{s.}{ mão }
\end{pronuncia}
\end{verbete}

\begin{verbete}[shou3bi4]{手臂}[4;17]
\begin{pronuncia}{shou3bi4}
\significado{s.}{ braço }
\end{pronuncia}
\end{verbete}

\begin{verbete}[shou3xiang4]{首相}[9;9]
\begin{pronuncia}{shou3xiang4}
\significado{s.}{ Primeiro-Ministro (Japão, UK, etc) }
\end{pronuncia}
\end{verbete}

\begin{verbete}[shou4]{瘦}[14]
\begin{pronuncia}{shou4}
\significado{adj.}{ magro; emagrecido }
\end{pronuncia}
\end{verbete}

\begin{verbete}[shu1]{书}[4]
\begin{pronuncia}{shu1}
\significado[本,册,部]{s.}{ livro; carta; documento }
\end{pronuncia}
\end{verbete}

\begin{verbete}[shu1fu2]{舒服}[12;8]
\begin{pronuncia}{shu1fu2}
\significado{adj.}{ estar confortável; bem disposto; (sentir-se) bem }
\end{pronuncia}
\end{verbete}

\begin{verbete}[shu2xi1]{熟悉}[15;11]
\begin{pronuncia}{shu2xi1}
\significado{v.}{ conhecer bem }
\end{pronuncia}
\end{verbete}

\begin{verbete}[shu3]{属}[12]
\begin{pronuncia}{shu3}
\significado{v.}{ nascer no ano do signo de (um dos doze animais zodiacais) }
\end{pronuncia}
\end{verbete}

\begin{verbete}[shu3jia4]{暑假}[12;11]
\begin{pronuncia}{shu3jia4}
\significado[个]{s.}{ férias de verão }
\end{pronuncia}
\end{verbete}

\begin{verbete}[shu4]{树}[9]
\begin{pronuncia}{shu4}
\significado[棵]{s.}{ árvore }
\end{pronuncia}
\end{verbete}

\begin{verbete}[shu4mu4]{树木}[9;4]
\begin{pronuncia}{shu4mu4}
\significado{s.}{árvore}
\end{pronuncia}
\end{verbete}

\begin{verbete}[shui2]{谁}[10]
\begin{pronuncia}{shui2}
\significado{interr.}{ quem? }
\end{pronuncia}
\begin{pronuncia}{shei2}
\significado*{}{ 谁\p{shei2} }
\end{pronuncia}
\end{verbete}

\begin{verbete}[shui3]{水}[4]
\begin{pronuncia}{shui3}
\significado{s.}{ água }
\significado{p.c.}{ para número de lavagens }
\end{pronuncia}
\end{verbete}

\begin{verbete}[shui3guo3]{水果}[4;8]
\begin{pronuncia}{shui3guo3}
\significado[个]{s.}{ fruta }
\end{pronuncia}
\end{verbete}

\begin{verbete}[shui3jiao3]{水饺}[4;9]
\begin{pronuncia}{shui3jiao3}
\significado{s.}{ dumplings; raviólis chineses }
\end{pronuncia}
\end{verbete}

\begin{verbete}[shui3ping2]{水平}[4;5]
\begin{pronuncia}{shui3ping2}
\significado{s.}{ nível; padrão }
\end{pronuncia}
\end{verbete}

\begin{verbete}[shui4jiao4]{睡觉}[13;9]
\begin{pronuncia}{shui4jiao4}
\significado{v.}{ ir para a cama; dormir; deitar-se }
\end{pronuncia}
\end{verbete}

\begin{verbete}[shuo1]{说}[9]
\begin{pronuncia}{shuo1}
\significado{v.}{ falar; dizer }
\end{pronuncia}
\end{verbete}

\begin{verbete}[shuo1-wan2]{说完}[9;7]
\begin{pronuncia}{shuo1-wan2}
\significado{expr.}{ acabar/terminar palavras }
\end{pronuncia}
\end{verbete}

\begin{verbete}[shuai1]{摔}[14]
\begin{pronuncia}{shuai1}
\significado{v.}{ cair; cair e quebrar; partir }
\end{pronuncia}
\end{verbete}

\begin{verbete}[shuai4]{帅}[5]
\begin{pronuncia}{shuai4}
\significado{adj.}{ elegante; agradável à vista }
\end{pronuncia}
\end{verbete}

\begin{verbete}[si3]{死}[6]
\begin{pronuncia}{si3}
\significado{v.}{ morrer; falecer }
\end{pronuncia}
\end{verbete}

\begin{verbete}[si4]{四}[5]
\begin{pronuncia}{si4}
\significado{num.}{ quatro; 4 }
\end{pronuncia}
\end{verbete}

\begin{verbete}[Si4chuan1]{四川}[5;3]
\begin{pronuncia}{Si4chuan1}
\significado{s.}{ Sichuan }
\end{pronuncia}
\end{verbete}

\begin{verbete}[si4ji4-ru2chun1]{四季如春}[5;8;6;9]
\begin{pronuncia}[\\]{si4ji4-ru2chun1}
\significado{expr.}{é primavera todo o ano}
\end{pronuncia}
\end{verbete}

\begin{verbete}[si4ji4-fen1ming2]{四季分明}[5;8;4;8]
\begin{pronuncia}[\\]{si4ji4-fen1ming2}
\significado{expr.}{as quatro estações são muito distintas}
\end{pronuncia}
\end{verbete}

\begin{verbete}[song4]{送}[9]
\begin{pronuncia}{song4}
\significado{v.}{ distribuir; entregar; dar; oferecer (alguma coisa como presente) }
\end{pronuncia}
\end{verbete}

\begin{verbete}[su4she4]{宿舍}[11;8]
\begin{pronuncia}{su4she4}
\significado[间]{s.}{ dormitório; quarto de dormir; hostel }
\end{pronuncia}
\end{verbete}

\begin{verbete}[suan1]{酸}[14]
\begin{pronuncia}{suan1}
\significado{adj.}{ ácido; avinagrado }
\end{pronuncia}
\end{verbete}

\begin{verbete}[suan1la4tang1]{酸辣汤}[14;14;6]
\begin{pronuncia}{suan1la4tang1}
\significado{s.}{ sopa avinagrada e picante }
\end{pronuncia}
\end{verbete}

\begin{verbete}[suan4le0]{算了}[14;2]
\begin{pronuncia}{suan4le0}
\significado{v.}{ deixar }
\end{pronuncia}
\end{verbete}

\begin{verbete}[sui2bian4]{随便}[11;9]
\begin{pronuncia}{sui2bian4}
\significado{adj.}{ à vontade; como queira }
\end{pronuncia}
\end{verbete}

\begin{verbete}[sui4]{岁}[6]
\begin{pronuncia}{sui4}
\significado{s.}{ ano (idade ou colheita) }
\significado{p.c}{ para anos de idade }
\end{pronuncia}
\end{verbete}

\begin{verbete}[sun1nur3]{孙女}[6;3]
\begin{pronuncia}{sun1nur3}
\significado{s.}{ filha do filho }
\end{pronuncia}
\end{verbete}

\begin{verbete}[sun1zi0]{孙子}[6;3]
\begin{pronuncia}{sun1zi0}
\significado{s.}{ filho do filho }
\end{pronuncia}
\end{verbete}

\begin{verbete}[sun3]{笋}[10]
\begin{pronuncia}{sun3}
\significado{s.}{ broto de bambu }
\end{pronuncia}
\end{verbete}

\begin{verbete}[suo3yi3]{所以}[8;4]
\begin{pronuncia}{suo3yi3}
\significado{conj.}{por isso}
\end{pronuncia}
\end{verbete}

%%%%% EOF %%%%%

%%%
%%% T
%%%
\section*{T}
\addcontentsline{toc}{section}{T}

\begin{verbete}[9]{T-恤}{T-xu4}
  \significado{s.}{camiseta; pulôver; suéter}
\end{verbete}

\begin{verbete}[5]{它}{ta1}
  \significado{pron.}{ele (para objetos inanimados); se, o, lhe; si, consigo, eles}
\end{verbete}

\begin{verbete}[5;5]{它们}{ta1men0}
  \significado{pron.}{eles (para objetos inanimados); se, os, lhes; si, consigo, eles}
\end{verbete}

\begin{verbete}[5]{他}{ta1}
  \significado{pron.}{ele; se, o, lhe; si, consigo, ele}
\end{verbete}

\begin{verbete}[5;8]{他的}{ta1de0}
  \significado{pron.}{dele}
\end{verbete}

\begin{verbete}[5;5]{他们}{ta1men0}
  \significado{pron.}{eles; se, os, lhes; si, consigo, eles}
\end{verbete}

\begin{verbete}[5;5;8]{他们的}{ta1men0de0}
  \significado{pron.}{deles}
\end{verbete}

\begin{verbete}[6]{她}{ta1}
  \significado{pron.}{ela; se, a, lhe; si, consigo, ela}
\end{verbete}

\begin{verbete}[6;8]{她的}{ta1de0}
  \significado{pron.}{dela}
\end{verbete}

\begin{verbete}[6;5]{她们}{ta1men0}
  \significado{pron.}{elas; se, as, lhes; si, consigo, elas}
\end{verbete}

\begin{verbete}[6;5;8]{她们的}{ta1men0de0}
  \significado{pron.}{delas}
\end{verbete}

\begin{verbete}[5]{台}{tai2}
  \significado{p.c.}{para veículos ou máquinas}
  \significado{s.}{Estação de transmissão; contador; help desk; suporte técnico; plataforma; terraço; tufão}
\end{verbete}

\begin{verbete}[4]{太}{tai4}
  \significado{adv.}{excessivamente; demais; muito}
\end{verbete}

\begin{verbete}[4;4]{太太}{tai4tai0}
  \significado[个,位]{s.}{esposa; madame; mulher casada}
\end{verbete}

\begin{verbete}[4;6]{太阳}{tai4yang0}
  \significado[个]{s.}{sol}
\end{verbete}

\begin{verbete}[4;6;12]{太阳窗}[\\]{tai4yang2chuang1}
  \significado{s.}{teto solar (de veículos)}
\end{verbete}

\begin{verbete}[4;6;6]{太阳灯}[\\]{tai4yang2deng1}
  \significado{s.}{lâmpada solar (com células fotovoltaicas)}
\end{verbete}

\begin{verbete}[4;6;4]{太阳风}[\\]{tai4yang2feng1}
  \significado{s.}{vento solar}
\end{verbete}

\begin{verbete}[4;6;16]{太阳镜}[\\]{tai4yang2jing4}
  \significado{s.}{óculos de sol}
\end{verbete}

\begin{verbete}[4;6;4]{太阳日}{tai4yang2ri4}
  \significado{s.}{dia solar}
\end{verbete}

\begin{verbete}[4;6;17]{太阳翼}{tai4yang2yi4}
  \significado{s.}{painel solar}
\end{verbete}

\begin{verbete}[4;6;8]{太阳雨}{tai4yang2yu3}
  \significado{s.}{banho de sol}
\end{verbete}

\begin{verbete}[10;8]{谈话}{tan2hua4}
  \significado[次]{s.}{conversa; fala}
  \significado{v.+compl.}{conversar; falar}
\end{verbete}

\begin{verbete}[6]{汤}{tang1}
  \significado{s.}{sopa; caldo; decocção de ervas medicinais; água quente ou fervente; água em que algo foi fervido}
\end{verbete}

\begin{verbete}[16]{糖}{tang2}
  \significado[颗,块]{s.}{açúcar; doces}
\end{verbete}

\begin{verbete}[16;15;8]{糖醋鱼}{tang2cu4yu2}
  \significado{s.}{peixe guisado em molho agridoce (prato)}
\end{verbete}

\begin{verbete}[10]{套}{tao4}
  \significado{p.c.}{para conjuntos, coleções}
  \significado{s.}{cobertura; fórmula; laço de corda}
  \significado{v.}{cobrir; envolver; intercalar; sobrepor}
\end{verbete}

\begin{verbete}[10;6]{套问}{tao4wen4}
  \significado{s.}{retórica}
  \significado{v.}{descobrir por meio de questionamento indireto diplomático}
\end{verbete}

\begin{verbete}[10;7]{特别}{te4bie2}
  \significado{adv.}{especialmente}
  \significado{adj.}{especial; paricular; incomum}
\end{verbete}

\begin{verbete}[10]{疼}{teng2}
  \significado{adj.}{dolorido; doído}
  \significado{v.}{doer; amar ternamente}
\end{verbete}

\begin{verbete}[15]{踢}{ti1}
  \significado{v.}{chutar; jogar (por exemplo, futebol); dar pontapés em}
\end{verbete}

\begin{verbete}[15;19]{踢爆}{ti1bao4}
  \significado{v.}{expor; revelar}
\end{verbete}

\begin{verbete}[15;17;14]{踢蹋舞}{ti1ta4wu3}
  \significado{s.}{sapateado; passo de dança}
\end{verbete}

\begin{verbete}[12;10]{提高}{ti2gao1}
  \significado{v.}{melhorar; aumentar; elevar}
\end{verbete}

\begin{verbete}[4]{天}{tian1}
  \significado{s.}{dia; céu; paraíso}
\end{verbete}

\begin{verbete}[4;12]{天鹅}{tian1e2}
  \significado{s.}{cisne}
\end{verbete}

\begin{verbete}[4;4]{天气}{tian1qi4}
  \significado{s.}{clima; tempo}
\end{verbete}

\begin{verbete}[4;8]{天使}{tian1shi3}
  \significado{s.}{anjo}
\end{verbete}

\begin{verbete}[4;4]{天天}{tian1tian1}
  \significado{adv.}{todo dia}
\end{verbete}

\begin{verbete}[4;8]{天择}{tian1ze2}
  \significado{s.}{seleção natural}
\end{verbete}

\begin{verbete}[11]{甜}{tian2}
  \significado{adj.}{doce}
\end{verbete}

\begin{verbete}[11;10]{甜酒}{tian2jiu3}
  \significado{s.}{licor doce}
\end{verbete}

\begin{verbete}[11;11]{甜菊}{tian2ju2}
  \significado{s.}{estévia, arbusto cujas folhas produzem substituto do açúcar}
\end{verbete}

\begin{verbete}[11;9]{甜品}{tian2pin3}
  \significado{s.}{sobremesa}
\end{verbete}

\begin{verbete}[11;11;11]{甜甜圈}[\\]{tian2tian2quan1}
  \significado{s.}{rosquinha; \textit{doughnut}}
\end{verbete}

\begin{verbete}[11;12]{甜筒}{tian2tong3}
  \significado{s.}{sorvete de casquinha}
\end{verbete}

\begin{verbete}[11;5]{甜头}{tian2tou0}
  \significado{s.}{benefício; sabor doce (de poder, sucesso, etc.)}
\end{verbete}

\begin{verbete}[11;9]{甜食}{tian2shi2}
  \significado{s.}{doces; sobremesa}
\end{verbete}

\begin{verbete}[11;14]{甜酸}{tian2suan1}
  \significado{adj.}{agridoce}
\end{verbete}

\begin{verbete}[11;4]{甜心}{tian2xin1}
  \significado{s.}{querido}
\end{verbete}

\begin{verbete}[11;7]{甜言}{tian2yan2}
  \significado{s.}{boa conversa; palavras amáveis}
\end{verbete}

\begin{verbete}[11;13]{甜稚}{tian2zhi4}
  \significado{s.}{doce e inocente}
\end{verbete}

\begin{verbete}[11;5;6]{甜玉米}{tian2yu4mi3}
  \significado{s.}{milho doce}
\end{verbete}

\begin{verbete}[7]{条}{tiao2}
  \significado{p.c.}{para coisas longas e finas (fita, rio, estrada, calças, etc.)}
  \significado{s.}{artigo; cláusula (de lei ou tratado); item; faixa}
\end{verbete}

\begin{verbete}[7;12]{条幅}{tiao2fu2}
  \significado{s.}{faixa; banner; pergaminho de parede (para pintura ou caligrafia)}
\end{verbete}

\begin{verbete}[7;9]{条贯}{tiao2guan4}
  \significado{s.}{ordem; procedimentos; sequência; sistema}
\end{verbete}

\begin{verbete}[7;6]{条件}{tiao2jian4}
  \significado[个]{s.}{circunstâncias; condição; fator; prerequisito; qualificação; requisito}
\end{verbete}

\begin{verbete}[7;8]{条例}{tiao2li4}
  \significado{s.}{código de conduta; ordenanças; regulamentos; regras; estatutos}
\end{verbete}

\begin{verbete}[7;5]{条目}{tiao2mu4}
  \significado{s.}{cláusulas e subcláusulas (em documento formal); verbete (em um dicionário, enciclopédia, etc.)}
\end{verbete}

\begin{verbete}[13]{跳}{tiao4}
  \significado{v.}{pular; saltar}
\end{verbete}

\begin{verbete}[13;9]{跳挡}{tiao4dang3}
  \significado{v.}{pular marcha (de um carro); perder a marcha}
\end{verbete}

\begin{verbete}[13;5]{跳电}{tiao4dian4}
  \significado{v.}{desarmar (um disjuntor ou interruptor)}
\end{verbete}

\begin{verbete}[13;13]{跳频}{tiao4pin2}
  \significado{s.}{FHSS, \textit{Frequency-Hopping Spread Spectrum}, método de transmissão de sinais de rádio}
\end{verbete}

\begin{verbete}[13;13;16]{跳跳糖}[\\]{tiao4tiao4tang2}
  \significado{s.}{Pop Rocks}
\end{verbete}

\begin{verbete}[13;14]{跳舞}{tiao4wu3}
  \significado{v.+compl.}{dançar}
\end{verbete}

\begin{verbete}[13;7]{跳远}{tiao4yuan3}
  \significado{v.+compl.}{salto em distância (atletismo)}
\end{verbete}

\begin{verbete}[13;9]{跳蚤}{tiao4zao0}
  \significado{s.}{pulga}
\end{verbete}

\begin{verbete}[7]{听}{ting1}
  \significado{p.c.}{para bebidas enlatadas }
  \significado{s.}{lata de bebida (do inglês ``tin'') }
  \significado{v.}{ouvir; escutar; obedecer}
\end{verbete}

\begin{verbete}[7;11]{听断}{ting1duan4}
  \significado{v.}{ouvir e decidir; julgar (ou seja, ouvir e julgar em um tribunal)}
\end{verbete}

\begin{verbete}[7;9]{听骨}{ting1gu3}
  \significado{v.}{ossículos (do ouvido médio)}
  \veja*{听小骨}{ting1xiao3gu3}
\end{verbete}

\begin{verbete}[7;6]{听会}{ting1hui4}
  \significado{v.}{participar de uma reunião (e ouvir o que é discutido)}
\end{verbete}

\begin{verbete}[7;7]{听来}{ting1lai2}
  \significado{v.}{ouvir de algum lugar; soar (antigo, estrangeiro, excitante, certo, etc.); soar como se (ou seja, dar uma impressão ao ouvinte)}
\end{verbete}

\begin{verbete}[7;2]{听力}{ting1li4}
  \significado{s.}{audição; capacidade de compreensão oral}
\end{verbete}

\begin{verbete}[7;2;11;13]{听力理解}[\\]{ting1li4li3jie3}
  \significado{s.}{compreensão auditiva}
\end{verbete}

\begin{verbete}[7;8]{听命}{ting1ming4}
  \significado{v.}{obedecer ordens; receber ordens}
\end{verbete}

\begin{verbete}[7;8]{听凭}{ting1ping2}
  \significado{v.}{permitir (alguém a fazer o que desejar)}
\end{verbete}

\begin{verbete}[7;9]{听说}{ting1shuo1}
  \significado{v.}{ouvir dizer}
\end{verbete}

\begin{verbete}[7;11]{听随}{ting1sui2}
  \significado{v.}{permitir; obedecer}
\end{verbete}

\begin{verbete}[7;6]{听戏}{ting1xi4}
  \significado{v.}{assistir a uma ópera; ver uma ópera}
\end{verbete}

\begin{verbete}[7;3;9]{听小骨}{ting1xiao3gu3}
  \significado{v.}{ossículos (do ouvido médio)}
  \veja{听骨}{ting1gu3}
\end{verbete}

\begin{verbete}[7;5]{听写}{ting1xie3}
  \significado{v.}{transcrever música de ouvido; escrever (em um exercício de ditado)}
  \significado{s.}{ditado}
\end{verbete}

\begin{verbete}[11;4;6]{停车场}[\\]{ting2che1chang3}
  \significado{s.}{parque de estacionamento}
\end{verbete}

\begin{verbete}[10]{通}{tong1}
  \significado{p.c.}{ para cartas, telegramas, telefonemas, etc.}
  \significado{s.}{suffixo: especialista}
  \significado{v.}{ligar para; conseguir a ligação}
\end{verbete}

\begin{verbete}[10;13]{通牒}{tong1die2}
  \significado{s.}{nota diplomática}
\end{verbete}

\begin{verbete}[10;6]{通观}{tong1guan1}
  \significado{v.}{ter uma visão geral de algo}
\end{verbete}

\begin{verbete}[10;7]{通识}{tong1shi2}
  \significado{s.}{conhecimento comum; erudição; conhecimento geral; amplamente conhecido}
\end{verbete}

\begin{verbete}[6]{同}{tong2}
  \significado{adj.}{junto}
  \significado{adv.}{junto com}
\end{verbete}

\begin{verbete}[6;6]{同伙}{tong2huo3}
  \significado[个]{s.}{cúmplice; colega}
\end{verbete}

\begin{verbete}[6;9]{同屋}{tong2wu1}
  \significado[个]{s.}{companheiro de quarto; colega de quarto}
\end{verbete}

\begin{verbete}[6;9]{同砚}{tong2yuan4}
  \significado[位,个]{s.}{colega de classe; colega estudante}
\end{verbete}

\begin{verbete}[6;8]{同学}{tong2xue2}
  \significado[位,个]{s.}{colega de classe; colega estudante}
\end{verbete}

\begin{verbete}[5]{头}{tou2}
  \significado[个]{s.}{cabeça}
  \significado{p.c.}{para suínos ou gado}
\end{verbete}

\begin{verbete}[5;5]{头发}{tou2fa0}
  \significado{s.}{cabelo}
\end{verbete}

\begin{verbete}[5;5]{头号}{tou2hao4}
  \significado{adj.}{primeira classe; número um; \textit{top rank}}
\end{verbete}

\begin{verbete}[5;5]{头头}{tou2tou2}
  \significado{s.}{chefe; o cabeça}
\end{verbete}

\begin{verbete}[7;10]{投资}{tou2zi1}
  \significado{s.}{investimento}
  \significado{v.}{investir}
\end{verbete}

\begin{verbete}[7;10;6;7;11]{投资回报率}[\\]{tou2zi1hui2bao4lv4}
  \significado{s.}{retorno sobre o investimento (ROI)}
\end{verbete}

\begin{verbete}[7;10;10]{投资家}{tou2zi1jia1}
  \significado{s.}{investidor}
  \veja{投资人}{tou2zi1ren2}
  \veja{投资者}{tou2zi1zhe3}
\end{verbete}

\begin{verbete}[7;10;2]{投资人}{tou2zi1ren2}
  \significado{s.}{investidor}
  \veja{投资家}{tou2zi1jia1}
  \veja{投资者}{tou2zi1zhe3}
\end{verbete}

\begin{verbete}[7;10;8]{投资者}{tou2zi1zhe3}
  \significado{s.}{investidor}
  \veja{投资家}{tou2zi1jia1}
  \veja{投资人}{tou2zi1ren2}
\end{verbete}

\begin{verbete}[7;10;4;9]{投资风险}[\\]{tou2zi1feng1xian3}
  \significado{s.}{risco de investimento}
\end{verbete}

\begin{verbete}[8;4;11]{图书馆}[\\]{tu2shu1guan3}
  \significado[家,个]{s.}{biblioteca}
\end{verbete}

\begin{verbete}[3;7]{土豆}{tu3dou4}
  \significado[个,颗]{s.}{batata}
\end{verbete}

\begin{verbete}[3;7;8]{土豆泥}{tu3dou4ni2}
  \significado{s.}{purê de batatas}
\end{verbete}

\begin{verbete}[11;7]{推迟}{tui1chi2}
  \significado{v.}{adiar; deixar para mais tarde; tardar}
\end{verbete}

\begin{verbete}[13]{腿}{tui3}
  \significado[条]{s.}{perna; osso do quadril}
\end{verbete}

\begin{verbete}[13;5]{腿号}{tui3hao4}
  \significado{s.}{anilha numerada (por exemplo, usada para identificar pássaros}
\end{verbete}

%%%%% EOF %%%%%

%%%%% Não existem palavras com pinyin iniciado em "U"
%%%%% Não existem palavras com pinyin iniciado em "V"
%%%
%%% W
%%%
\section*{W}
\addcontentsline{toc}{section}{W}

\begin{verbete}[5]{外}{wai4}
\significado{p.l.}{ fora; por fora; exterior; estrangeiro }
\end{verbete}

\begin{verbete}[5;5]{外边}{wai4bian0}
\significado{p.l.}{ fora do país; superfície externa; fora; lugar diferente de sua casa }
\end{verbete}

\begin{verbete}[5;12]{外插}{wai4cha1}
\significado{s.}{ extrapolar; computação: conectar (um dispositivo periférico, etc) }
\end{verbete}

\begin{verbete}[5;4]{外公}{wai4gong1}
\significado{s.}{ avô materno }
\end{verbete}

\begin{verbete}[5;8]{外国}{wai4guo2}
\significado[个]{s.}{ país estrangeiro }
\end{verbete}

\begin{verbete}[5;8:2]{外国人}{wai4guo2ren2}
\significado{s.}{ estrangeiro; nascido fora do país }
\end{verbete}

\begin{verbete}[5;10]{外海}{wai4hai3}
\significado{s.}{ mar aberto }
\end{verbete}

\begin{verbete}[5;5]{外号}{wai4hao4}
\significado{s.}{ apelido }
\end{verbete}

\begin{verbete}[5;10]{外积}{wai4ji1}
\significado{s.}{ produto exterior; matemática: o produto vetorial de dois vetores }
\end{verbete}

\begin{verbete}[5;6]{外交}{wai4jiao1}
\significado{adj.}{ diplomático }
\significado[个]{s.}{ diplomacia; relações exteriores }
\end{verbete}

\begin{verbete}[5;9]{外贸}{wai4mao4}
\significado{s.}{ comércio exterior }
\end{verbete}

\begin{verbete}[5;9]{外貌协会}[\\]{wai4mao4xie2hui4}
\significado{s.}{ o ``clube da boa aparência'': pessoas que dão grande importância à aparência de uma pessoa }
\veja{外协}{wai4xie2}
\end{verbete}

\begin{verbete}[5;9]{外面}{wai4mian4}
\significado{p.l.}{ fora; por fora; exterior; superfície }
\end{verbete}

\begin{verbete}[5;11]{外婆}{wai4po2}
\significado{s.}{ avó materna }
\end{verbete}

\begin{verbete}[5;8]{外事}{wai4shi4}
\significado{s.}{ assuntos ou relações exteriores }
\end{verbete}

\begin{verbete}[5;4]{外水}{wai4shui3}
\significado{s.}{ renda extra }
\end{verbete}

\begin{verbete}[5;6]{外孙}{wai4sun1}
\significado{s.}{ filho da filha }
\end{verbete}

\begin{verbete}[5;6;3]{外孙女}{wai4sun1nv3}
\significado{s.}{ filha da filha }
\end{verbete}

\begin{verbete}[5;7]{外围}{wai4wei2}
\significado{p.l.}{ arredores }
\end{verbete}

\begin{verbete}[5;6]{外协}{wai4xie2}
\significado{s.}{ terceirização; pessoas que julgam os outros pela aparência }
\veja{外貌协会}{wai4mao4xie2hui4}
\end{verbete}

\begin{verbete}[5;6]{外衣}{wai4yi1}
\significado{s.}{ aparência; roupa de cima }
\end{verbete}

\begin{verbete}[5;9]{外语}{wai4yu3}
\significado[门]{s.}{ língua estrangeira }
\end{verbete}

\begin{verbete}[15;7]{豌豆}{wan1dou4}
\significado{s.}{ ervilha }
\end{verbete}

\begin{verbete}[7]{完}{wan2}
\significado{v.}{ acabar; completar; terminar }
\end{verbete}

\begin{verbete}[7;8]{完备}{wan2bei4}
\significado{adj.}{ completo; impecável; perfeito }
\significado{v.}{ não deixar nada a desejar }
\end{verbete}

\begin{verbete}[7;6]{完毕}{wan2bi4}
\significado{v.}{ completar; terminar; acabar }
\end{verbete}

\begin{verbete}[7;6]{完成}{wan2cheng2}
\significado{v.}{ realizar; completar }
\end{verbete}

\begin{verbete}[7;13]{完满}{wan2man3}
\significado{adj.}{ satisfatório; bem-sucedido }
\end{verbete}

\begin{verbete}[7;9]{完美}{wan2mei3}
\significado{adj.}{ perfeito }
\significado{adv.}{ perfeitamente }
\significado{s.}{ perfeição }
\end{verbete}

\begin{verbete}[7;6]{完全}{wan2quan2}
\significado{adj.}{ completo; todo }
\significado{adv.}{ inteiramente; totalmente }
\end{verbete}

\begin{verbete}[7;2]{完人}{wan2ren2}
\significado{s.}{ pessoa perfeita }
\end{verbete}

\begin{verbete}[7;7;6;6]{完完全全}[\\]{wan2wan2quan2quan2}
\significado{adv.}{ completamente }
\end{verbete}

\begin{verbete}[7;12]{完税}{wan2shui4}
\significado{v.}{ pagar imposto }
\end{verbete}

\begin{verbete}[8]{玩}{wan2}
\significado{s.}{ brinquedo; algo usado para diversão }
\significado{v.}{ divertir-se; manter algo para entretenimento; brincar com  }
\end{verbete}

\begin{verbete}[8;7]{玩伴}{wan2ban4}
  \significado{s.}{ parceiro de brincadeira }
\end{verbete}

\begin{verbete}[8;12]{玩遍}{wan2bian4}
\significado{v.}{ passear (todo o país, toda a cidade, etc); visitar (um grande número de lugares) }
\end{verbete}

\begin{verbete}[8;2]{玩儿}{wanr2}
\significado{v.}{ divertir-se }
\end{verbete}

\begin{verbete}[7;10]{玩家}{wan2jia1}
\significado{s.}{ entusiasta (áudio, modelos de aviões, etc); jogador (de um jogo) }
\end{verbete}

\begin{verbete}[7;8]{玩耍}{wan2shua3}
\significado{v.}{ divertir-me; brincar (como as crianças fazem) }
\end{verbete}

\begin{verbete}[8;8]{玩味}{wan2wei4}
\significado{v.}{ ponderar sutilezas; ruminar (pensamentos) }
\end{verbete}

\begin{verbete}[8;13]{玩味}{wan2yi4}
\significado{s.}{ ato; brinquedo; coisa; truque (em uma performance, show de palco, acrobacias, etc) }
\end{verbete}

\begin{verbete}[8;8]{玩者}{wan2zhe3}
\significado{s.}{ jogador }
\end{verbete}

\begin{verbete}[11]{晚}{wan3}
\significado{adj.}{ tarde; noite }
\end{verbete}

\begin{verbete}[11;7]{晚报}{wan3bao4}
\significado{s.}{ jornal da noite  }
\end{verbete}

\begin{verbete}[11;9]{晚点}{wan3dian3}
\significado{adj.}{ atrasado }
\significado{s.}{ jantar leve }
\end{verbete}

\begin{verbete}[11;16]{晚餐}{wan3can1}
\significado[份,顿,次]{s.}{ jantar; refeição noturna }
\end{verbete}

\begin{verbete}[11;7]{晚饭}{wan3fan4}
\significado[份,顿,次,餐]{s.}{ jantar }
\end{verbete}

\begin{verbete}[11;6]{晚会}{wan3hui4}
\significado[个]{s.}{ festa noturna }
\end{verbete}

\begin{verbete}[11;7]{晚近}{wan3jin4}
\significado{adv.}{ ultimamente; recentemente }
\significado{adj.}{ recente; mais recente no passado }
\end{verbete}

\begin{verbete}[11;12]{晚景}{wan3jing3}
\significado{s.}{ circunstâncias dos anos de declínio de alguém; cena noturna }
\end{verbete}

\begin{verbete}[11;3]{晚上}{wan3shang0}
\significado{p.t.}{ noite; à noite }
\end{verbete}

\begin{verbete}[11;8]{晚育}{wan3yu4}
\significado{n.}{ parto tardio }
\significado{v.}{ ter um filho mais tarde }
\end{verbete}

\begin{verbete}[13]{碗}{wan3}
\significado[只,个]{n}{ tigela }
\significado{p.c.}{ tigelas }
\end{verbete}

\begin{verbete}[13;8]{碗柜}{wan3gui4}
\significado{n}{ armário }
\end{verbete}

\begin{verbete}[13;3]{碗子}{wan3zi0}
\significado{n}{ tigela }
\end{verbete}

\begin{verbete}[3]{万}{wan4}
\significado{adj.}{ um grande número }
\significado{num.}{ 10.000, dez mil }
\end{verbete}

\begin{verbete}[3;3]{万万}{wan4wan4}
\significado{adv.}{ absolutamente; totalmente }
\end{verbete}

\begin{verbete}[8]{往}{wang3}
\significado{prep.}{ para; em direção a }
\end{verbete}

\begin{verbete}[8;12]{往程}{wang3cheng2}
\significado{s.}{ saída (de uma viagem de ônibus ou trem, etc) }
\end{verbete}

\begin{verbete}[8;7]{往返}{wang3fan3}
\significado{s.}{ ida e volta }
\significado{v.}{ ir e voltar; ir e vir }
\end{verbete}

\begin{verbete}[8;9]{往复}{wang3fu4}
\significado{s.}{ para trás e para frente (por exemplo, da ação do pistão ou da bomba); }
\significado{v.}{ ir e voltar; fazer uma viagem de volta }
\end{verbete}

\begin{verbete}[8;9]{往迹}{wang3ji4}
\significado{s.}{ eventos passados }
\end{verbete}

\begin{verbete}[8;7]{往来}{wang3lai2}
\significado{s.}{ contatos; negociações }
\end{verbete}

\begin{verbete}[8;8]{往例}{wang3li4}
\significado{s.}{ prática (habitual) do passado; precedente }
\end{verbete}

\begin{verbete}[8;4]{往日}{wang3ri4}
\significado{p.t.}{ dias passados }
\significado{s.}{ o passado }
\end{verbete}

\begin{verbete}[8;5]{往生}{wang3sheng1}
\significado{v.}{ renascer; morrer; Budismo: viver no paraíso }
\end{verbete}

\begin{verbete}[8;8]{往事}{wang3shi4}
\significado{s.}{ acontecimentos anteriores; eventos passados }
\end{verbete}

\begin{verbete}[8;8]{往往}{wang3wang3}
\significado{adv.}{ em muitos casos; mais frequentes do que não; geralmente }
\end{verbete}

\begin{verbete}[8;8]{往昔}{wang3xi1}
\significado{s.}{ o passado }
\end{verbete}

\begin{verbete}[6]{网}{wang3}
\significado{s.}{ rede }
\end{verbete}

\begin{verbete}[6;10]{网罟}{wang3gu3}
\significado{s.}{ figurativo: a rede da justiça; rede usada para capturar peixes (ou outros animais, como pássaros) }
\veja{法网}{fa3wang3}
\end{verbete}

\begin{verbete}[6;7;6;9]{网际网路}[\\]{wang3ji4wang3lu4}
\significado{s.}{ Internet }
\veja{网际网络}{wang3ji4wang3luo4}
\veja{网路}{wang3lu4}
\end{verbete}

\begin{verbete}[6;7;6;9]{网际网络}[\\]{wang3ji4wang3luo4}
\significado{s.}{ Internet }
\veja{网际网路}{wang3ji4wang3lu4}
\veja{网路}{wang3lu4}
\end{verbete}

\begin{verbete}[6;13]{网路}{wang3lu4}
\significado{s.}{ Internet }
\veja{网际网路}{wang3ji4wang3lu4}
\veja{网际网络}{wang3ji4wang3luo4}
\end{verbete}

\begin{verbete}[6;11]{网球}{wang3qiu2}
\significado[个]{s.}{ tênis (esporte); bola de tênis }
\end{verbete}

\begin{verbete}[6;11]{网上银行}[\\]{wang3shang4yin2hang2}
\significado[个]{s.}{ banco online; acesso a operações bancárias via Internet }
\veja{网银}{wang3yin2}
\end{verbete}

\begin{verbete}[6;11]{网银}{wang3yin2}
\significado[个]{s.}{ banco online; acesso a operações bancárias via Internet }
\veja{网上银行}{wang3shang4yin2hang2}
\end{verbete}

\begin{verbete}[7]{忘}{wang4}
\significado{v.}{ esquecer; neglicenciar; ignorar }
\end{verbete}

\begin{verbete}[7;5]{忘本}{wang4ben3}
\significado{v.}{ esquecer as próprias raízes }
\end{verbete}

\begin{verbete}[7;16]{忘餐}{wang4can1}
\significado{v.}{ esquecer as refeições }
\end{verbete}

\begin{verbete}[7;11]{忘掉}{wang4diao4}
\significado{v.}{ esquecer }
\end{verbete}

\begin{verbete}[7;10]{忘恩}{wang4en1}
\significado{v.}{ ser ingrato }
\end{verbete}

\begin{verbete}[7;7]{忘怀}{wang4huai2}
\significado{v.}{ esquecer }
\end{verbete}

\begin{verbete}[7;5]{忘记}{wang4ji4}
\significado{v.}{ esquecer }
\end{verbete}

\begin{verbete}[7;7]{忘却}{wang4que4}
\significado{v.}{ esquecer }
\end{verbete}

\begin{verbete}[12;9]{温度}{wen1du4}
\significado[个]{s.}{ temperatura }
\end{verbete}

\begin{verbete}[12;9;8]{温度表}[\\]{wen1du4biao3}
\significado{s.}{ termômetro }
\end{verbete}

\begin{verbete}[12;9;4]{温度计}{wen1du4ji4}
\significado{s.}{ termógrafo; termômetro }
\end{verbete}

\begin{verbete}[12;9;11;9]{温度梯度}[\\]{wen1du4ti1du4}
\significado{s.}{ gradiente de temperatura }
\end{verbete}

\begin{verbete}[12]{喂}{wei2}
\significado{interj.}{ Alô!; Olá! (quando respondendo a um telefonema) }
\veja{喂}{wei4}
\end{verbete}

\begin{verbete}[3;5;6;9]{卫生防疫}[\\]{wei4sheng1fang2yi4}
\significado{s.}{ prevenção contra a epidemia }
\end{verbete}

\begin{verbete}[3;5]{卫生}{wei4sheng1}
\significado{s.}{ saúde; higiene; saneamento }
\end{verbete}

\begin{verbete}[3;5;10]{卫生部}[\\]{wei4sheng1bu4}
\significado{s.}{ Ministério da Saúde }
\end{verbete}

\begin{verbete}[3;5;7]{卫生间}[\\]{wei4sheng1jian1}
\significado[间]{s.}{ banheiro; toilette }
\end{verbete}

\begin{verbete}[3;5;3]{卫生巾}[\\]{wei4sheng1jin1}
\significado{s.}{ absorvente higiênico }
\end{verbete}

\begin{verbete}[3;5;7]{卫生局}{wei4sheng1ju2}
\significado{s.}{ Departamento de Saúde; Escritório de Saúde }
\end{verbete}

\begin{verbete}[3;5;12]{卫生棉}[\\]{wei4sheng1mian2}
\significado{s.}{ absorvente; algodão absorvente esterilizado (usado para curativos ou limpeza de feridas); absorvente tampão }
\end{verbete}

\begin{verbete}[3;5;11]{卫生球}[\\]{wei4sheng1qiu2}
\significado{s.}{ naftalina }
\end{verbete}

\begin{verbete}[3;5;13]{卫生署}[\\]{wei4sheng1shu3}
\significado{s.}{ Agência de Saúde (ou Escritório, ou Departamento) }
\end{verbete}

\begin{verbete}[3;5;10]{卫生套}[\\]{wei4sheng1tao4}
\significado[只]{s.}{ camisinha; preservativo }
\end{verbete}

\begin{verbete}[3;5;4]{卫生厅}[\\]{wei4sheng1ting1}
\significado{s.}{ Departamento de Saúde (da província) }
\end{verbete}

\begin{verbete}[3;5;7]{卫生纸}[\\]{wei4sheng1zhi3}
\significado{s.}{ papel higiênico }
\end{verbete}

\begin{verbete}[4]{为}{wei4}
\significado{prep.}{ para }
\end{verbete}

\begin{verbete}[4;4;3]{为什么}[\\]{wei4shen2me0}
\significado{interr.}{ por que? }
\end{verbete}

\begin{verbete}[7]{位}{wei4}
\significado{p.c.}{ para pessoas (com cortesia); classificador para bits binários (por exemplo, 十六位 16-bits ou 2 bytes);  }
\significado{s.}{ física: potencial; localização; lugar; posição; assento }
\end{verbete}

\begin{verbete}[7;8]{位居}{wei4ju1}
\significado{v.}{ estar localizado em }
\end{verbete}

\begin{verbete}[7;13]{位置}{wei4zhi0}
\significado[个]{s.}{ lugar; posição; assento }
\end{verbete}

\begin{verbete}[8]{味}{wei4}
\significado{p.c.}{ para medicamentos }
\significado{s.}{ cheiro; gosto }
\end{verbete}

\begin{verbete}[8;12]{味道}{wei4dao0}
\significado{s.}{ sabor; odor }
\end{verbete}

\begin{verbete}[8;2]{味儿}{weir4}
\significado{s.}{ sabor }
\end{verbete}

\begin{verbete}[12]{喂}{wei4}
\significado{interj.}{ Ei!; chamar atenção }
\significado{v.}{ alimentar; alimentar (um animal, bebê, inválido, etc) }
\veja{喂}{wei2}
\end{verbete}

\begin{verbete}[12;10]{喂哺}{wei4bu3}
\significado{v.}{ alimentar (um bebê) }
\end{verbete}

\begin{verbete}[12;10]{喂料}{wei4liao4}
\significado{v.}{ alimentar (também no sentido figurativo) }
\end{verbete}

\begin{verbete}[12;5;8]{喂母乳}{wei4mu3ru3}
\significado{s.}{ amamentação }
\end{verbete}

\begin{verbete}[12;5]{喂奶}{wei4nai3}
\significado{v.}{ amamentar }
\end{verbete}

\begin{verbete}[12;9]{喂食}{wei4shi2}
\significado{v.}{ alimentar }
\end{verbete}

\begin{verbete}[12;9]{喂养}{wei4yang3}
\significado{v.}{ alimentar (uma criança, animal doméstico, etc); manter; criar (um animal) }
\end{verbete}

\begin{verbete}[4;4]{文化}{wen2hua4}
\significado[个,种]{s.}{ cultura; civilização }
\end{verbete}

\begin{verbete}[4;4;7]{文化层}[\\]{wen2hua4ceng2}
\significado{s.}{ nível de cultura (em sítio arqueológico) }
\end{verbete}

\begin{verbete}[4;4;9]{文化宫}[\\]{wen2hua4gong1}
\significado{s.}{ palácio cultural }
\end{verbete}

\begin{verbete}[4;4;11]{文化圈}[\\]{wen2hua4quan1}
\significado{s.}{ esfera de influência cultural }
\end{verbete}

\begin{verbete}[4;4;10]{文化热}{wen2hua4re4}
\significado{s.}{ mania cultural; febre cultural }
\end{verbete}

\begin{verbete}[4;4;5]{文化史}{wen2hua4shi3}
\significado{s.}{ História Cultural }
\end{verbete}

\begin{verbete}[4;8;7]{文学系}{Wen2xue2xi4}
\significado{s.}{ Faculdade de Letras }
\end{verbete}

\begin{verbete}[4;8;13;13]{文化障碍}[\\]{wen2xue2zhang4ai4}
\significado{s.}{ barreira cultural }
\end{verbete}

\begin{verbete}[6]{问}{wen4}
\significado{v.}{ perguntar }
\end{verbete}

\begin{verbete}[6;6]{问安}{wen4'an1}
\significado{s.}{ saudações }
\significado{v.}{ dar cumprimentos a; prestar homenagem }
\end{verbete}

\begin{verbete}[6;12]{问鼎}{wen4ding3}
\significado{v.}{ visar (o primeiro lugar, etc.); aspirar ao trono }
\end{verbete}

\begin{verbete}[6;8]{问卷}{wen4juan3}
\significado[份]{s.}{ questionário }
\end{verbete}

\begin{verbete}[6;15]{问题}{wen4ti2}
\significado[个]{s.}{ pergunta; questão; problema }
\end{verbete}

\begin{verbete}[6;5]{问市}{wen4shi4}
\significado{v.}{ chegar ao marcado; bater o mercado; atingir o mercado }
\end{verbete}

\begin{verbete}[7]{我}{wo3}
\significado{pron.}{ eu }
\end{verbete}

\begin{verbete}[7;8]{我的}{wo3de0}
\significado{pron.}{ meu, meus }
\end{verbete}

\begin{verbete}[7;5]{我们}{wo3men0}
\significado{pron.}{ nós }
\end{verbete}

\begin{verbete}[7;5;8]{我们的}{wo3men0de0}
\significado{pron.}{ nosso, nossos }
\end{verbete}

\begin{verbete}[8]{卧}{wo4}
\significado{v.}{ agachar; deitar }
\end{verbete}

\begin{verbete}[8;10]{卧病}{wo4bing4}
\significado{s.}{ acamado; doente na cama }
\end{verbete}

\begin{verbete}[8;10]{卧舱}{wo4cang1}
\significado{s.}{ cabine de dormir em um barco ou trem }
\end{verbete}

\begin{verbete}[8;4]{卧车}{wo4che1}
\significado{s.}{ um carro-leito; vagão-leito }
\end{verbete}

\begin{verbete}[8;7]{卧床}{wo4chuang2}
\significado{adj.}{ acamado }
\significado{s.}{ cama }
\significado{v.}{ deitar na cama }
\end{verbete}

\begin{verbete}[8;10]{卧倒}{wo4dao3}
\significado{v.}{ cair no chão; deitar-se }
\end{verbete}

\begin{verbete}[8;11]{卧推}{wo4tui1}
\significado{s.}{ supino }
\end{verbete}

\begin{verbete}[8;14]{卧榻}{wo4ta4}
\significado{s.}{ um sofá; uma cama estreita }
\end{verbete}

\begin{verbete}[8;6]{卧式}{wo4shi4}
\significado{adj.}{ horizontal }
\end{verbete}

\begin{verbete}[8;9]{卧室}{wo4shi4}
\significado[间]{s.}{ quarto de dormir }
\end{verbete}

\begin{verbete}[4]{午}{wu3}
\significado{p.t.}{ 11h00-13h00; meio-dia }
\end{verbete}

\begin{verbete}[4;16]{午餐}{wu3can1}
\significado[份,顿,次]{s.}{ almoço }
\end{verbete}

\begin{verbete}[4;7]{午饭}{wu3fan4}
\significado[份,顿,次,餐]{s.}{ almoço }
\end{verbete}

\begin{verbete}[4;6]{午后}{wu3hou4}
\significado{p.t.}{ tarde; período da tarde }
\end{verbete}

\begin{verbete}[4;9]{午前}{wu3qian2}
\significado{p.t.}{ A.M.; manhã; período da manhã }
\end{verbete}

\begin{verbete}[4;14]{午睡}{wu3shui4}
\significado{s.}{ siesta }
\significado{v.}{ tirar uma soneca }
\end{verbete}

\begin{verbete}[4;6]{午休}{wu3xiu1}
\significado{s.}{ pausa para almoço; cochilo na hora do almoço; intervalo do meio-dia }
\end{verbete}

\begin{verbete}[4;10]{午宴}{wu3yan4}
\significado{s.}{ banquete de almoço }
\end{verbete}

\begin{verbete}[4;8]{午夜}{wu3ye4}
\significado{p.t.}{ meia-noite }
\end{verbete}

\begin{verbete}[4]{五}{wu3}
\significado{num.}{ 5, cinco }
\end{verbete}

\begin{verbete}[4;4]{五五}{wu3wu3}
\significado{num.}{ 50-50 }
\significado{s.}{ igual (partilha, parceria, etc) }
\end{verbete}

\begin{verbete}[14]{舞}{wu3}
\significado{s.}{ dança }
\end{verbete}

\begin{verbete}[14;7]{舞抃}{wu3bian4}
\significado{s.}{ dançar por prazer }
\end{verbete}

\begin{verbete}[14;6]{舞会}{wu3hui4}
\significado{s.}{ baile }
\end{verbete}

\begin{verbete}[14;6;14]{舞会舞}{wu3hui4wu3}
\significado{s.}{ baile }
\end{verbete}

\begin{verbete}[14;4]{舞厅}{wu3ting1}
\significado[间]{s.}{ salão de dança; salão de baile }
\end{verbete}

\begin{verbete}[14;4;14]{舞厅舞}{wu3ting1wu3}
\significado{s.}{ dança de salão }
\end{verbete}

%%%%% EOF %%%%%

%%%
%%% X
%%%
\section*{X}
\addcontentsline{toc}{section}{X}

\begin{verbete}[6]{西}{xi1}
  \significado{p.l.}{oeste}
\end{verbete}

\begin{verbete}[6;6]{西安}{xi1'an1}
  \significado{s.}{Xi'an}
\end{verbete}

\begin{verbete}[6;10;4;4]{西班牙文}[\\]{xi1ban1ya2wen2}
  \significado{s.}{espanhol, língua espanhola}
  \veja{西文}{xi1wen2}
\end{verbete}

\begin{verbete}[6;10;4;9]{西班牙语}[\\]{xi1ban1ya2yu3}
  \significado{s.}{espanhol, língua espanhola}
  \veja{西语}{xi1yu3}
\end{verbete}

\begin{verbete}[6;5;11]{西半球}{xi1ban4qiu2}
  \significado{s.}{hemisfério oeste}
\end{verbete}

\begin{verbete}[6;5]{西边}{xi1bian0}
  \significado{p.l.}{ao oeste de; oeste; lado oeste; parte ocidental}
\end{verbete}

\begin{verbete}[6;10]{西部}{xi1bu4}
  \significado{p.l.}{parte ocidental}
\end{verbete}

\begin{verbete}[6;4]{西方}{xi1fang1}
  \significado{p.l.}{países ocidentais; o Ocidente; o Oeste}
\end{verbete}

\begin{verbete}[6;5;7]{西兰花}{xi1lan2hua1}
  \significado{s.}{brócolis}
\end{verbete}

\begin{verbete}[6;9]{西面}{xi1mian4}
  \significado{p.l.}{oeste; lado oeste}
\end{verbete}

\begin{verbete}[6;4]{西文}{xi1wen2}
  \significado{s.}{espanhol, língua espanhola}
  \veja{西班牙文}{xi1ban1ya2wen2}
\end{verbete}

\begin{verbete}[6;9]{西语}{xi1yu3}
  \significado{s.}{espanhol, língua espanhola}
  \veja{西班牙语}{xi1ban1ya2yu3}
\end{verbete}

\begin{verbete}[6;6]{西西}{xi1xi1}
  \significado{n.}{centímetro cúbico}
\end{verbete}

\begin{verbete}[7;11]{希望}{xi1wang4}
  \significado[个]{s.}{desejo}
  \significado{v.}{desejar}
\end{verbete}

\begin{verbete}[11;13]{悉数}{xi1shu3}
  \significado{adv.}{enumerar em detalhes; explicar claramente}
  \veja{悉数}{xi1shu4}
\end{verbete}

\begin{verbete}[11;13]{悉数}{xi1shu4}
  \significado{adv.}{todos; cada um; toda a soma}
  \veja{悉数}{xi1shu3}
\end{verbete}

\begin{verbete}[11;4]{悉心}{xi1xin1}
  \significado{adv.}{colocar o coração (e a alma) em algo; com muito cuidado}
\end{verbete}

\begin{verbete}[11;5]{悉尼}{Xi1ni2}
  \significado{s.}{Sidney}
\end{verbete}

\begin{verbete}[9]{洗}{xi3}
  \significado{v.}{lavar; revelar (fotos); tomar banho}
\end{verbete}

\begin{verbete}[9;10]{洗涤}{xi3di2}
  \significado{s.}{enxágue; lava}
  \significado{v.}{enxaguar; lavar}
\end{verbete}

\begin{verbete}[9;10;7]{洗涤间}{xi3di2jian1}
  \significado{s.}{lavanderia}
\end{verbete}

\begin{verbete}[9;7]{洗劫}{xi3jie2}
  \significado{v.}{saquear; pilhar; roubar}
\end{verbete}

\begin{verbete}[9;8]{洗净}{xi3jing4}
  \significado{v.}{lavar (limpeza)}
\end{verbete}

\begin{verbete}[9;5]{洗礼}{xi3li3}
  \significado{s.}{batismo}
  \significado{v.}{batizar}
\end{verbete}

\begin{verbete}[9;4]{洗手}{xi3shou3}
  \significado{v.}{ir ao banheiro; lavar as mãos}
\end{verbete}

\begin{verbete}[9;4;4;3]{洗手不干}[\\]{xi3shou3bu2gan4}
  \significado{v.}{parar totalmente de fazer algo}
\end{verbete}

\begin{verbete}[9;4;6]{洗手池}{xi3shou3chi2}
  \significado{s.}{pia de banheiro; lavatório}
  \veja{洗手盆}{xi3shou3pen2}
\end{verbete}

\begin{verbete}[9;4;7]{洗手间}{xi3shou3jian1}
  \significado{s.}{sanitário; toilette; banheiro}
\end{verbete}

\begin{verbete}[9;4;9]{洗手盆}{xi3shou3pen2}
  \significado{s.}{pia de banheiro; lavatório}
  \veja{洗手池}{xi3shou3chi2}
\end{verbete}

\begin{verbete}[9;4;8]{洗手乳}{xi3shou3ru3}
  \significado{s.}{sabonete líquido para lavar as mãos}
  \veja{洗手液}{xi3shou3ye4}
\end{verbete}

\begin{verbete}[9;4;11]{洗手液}{xi3shou3ye4}
  \significado{s.}{sabonete líquido para lavar as mãos}
  \veja{洗手乳}{xi3shou3ru3}
\end{verbete}

\begin{verbete}[9;11]{洗脱}{xi3tuo1}
  \significado{v.}{limpar; purgar; lavar}
\end{verbete}

\begin{verbete}[9;9]{洗胃}{xi3wei4}
  \significado{s.}{medicina: lavagem gástrica}
  \significado{v.}{ter o estômago lavado}
\end{verbete}

\begin{verbete}[9;13]{洗碗}{xi3wan3}
  \significado{v.}{lavar pratos}
\end{verbete}

\begin{verbete}[9;6;6]{洗衣机}{xi3yi3ji1}
  \significado[台]{s.}{máquina de lavar roupa}
\end{verbete}

\begin{verbete}[9;16;7]{洗澡间}{xi3zao3jian1}
  \significado[间]{s.}{banheiro}
\end{verbete}

\begin{verbete}[12;6]{喜欢}{xi3huan0}
  \significado{v.}{gostar}
\end{verbete}

\begin{verbete}[7]{系}{xi4}
  \significado{s.}{faculdade (da universidade); departamento}
  \significado{v.}{prender; vincular; conectar; relacionar com; amarrar; se preocupar}
\end{verbete}

\begin{verbete}[7;6]{系列}{xi4lie4}
    \significado{s.}{série; conjunto}
\end{verbete}

\begin{verbete}[7;5]{系囚}{xi4qiu2}
  \significado{s.}{prisioneiro}
\end{verbete}

\begin{verbete}[7;9]{系统}{xi4tong3}
  \significado{s.}{sistema}
\end{verbete}

\begin{verbete}[3]{下}{xia4}
  \significado{p.l.}{abaixo; em baixo de}
  \significado{v.d.}{descer; chegar a (uma decisão, conclusão, etc.); recusar}
\end{verbete}

\begin{verbete}[3;4]{下巴}{xia4ba0}
  \significado[个]{s.}{queixo}
\end{verbete}

\begin{verbete}[3;5]{下边}{xia4bian0}
  \significado{p.l.}{em baixo; abaixo; parte de baixo}
\end{verbete}

\begin{verbete}[3;4]{下车}{xia4che1}
  \significado{v.}{descer; sair (de ônibus, carro, etc.)}
\end{verbete}

\begin{verbete}[3;10]{下课}{xia4ke4}
  \significado{v.+compl.}{acabar a aula; terminar a aula}
\end{verbete}

\begin{verbete}[3;7]{下来}{xia4lai0}
  \significado{v.+compl.}{descer (para a minha localização)}
\end{verbete}

\begin{verbete}[3;9]{下面}{xia4mian0}
  \significado{p.l.}{em baixo; abaixo; parte de baixo}
\end{verbete}

\begin{verbete}[3;5]{下去}{xia4qu0}
  \significado{v.+compl.}{descer (a partir da minha localização)}
\end{verbete}

\begin{verbete}[3;4]{下午}{xia4wu3}
  \significado{p.t.}{tarde; à tarde; período logo após o meio-dia}
\end{verbete}

\begin{verbete}[3;6]{下旬}{xia4xun2}
  \significado{p.t.}{última dezena do mês}
\end{verbete}

\begin{verbete}[3;8]{下雨}{xia4yu3}
  \significado{v.+compl.}{chover}
\end{verbete}

\begin{verbete}[3;10]{下载}{xia4zai3}
  \significado{v.}{baixar; \textit{download}}
\end{verbete}

\begin{verbete}[3;12]{下崽}{xia4zai3}
  \significado{v.}{dar à luz (animais); parir}
\end{verbete}

\begin{verbete}[10;8]{夏天}{xia4tian1}
  \significado[个]{s.}{verão}
  \significado{p.t.}{verão}
\end{verbete}

\begin{verbete}[10;4]{夏日}{xia4ri4}
  \significado{s.}{horário de verão}
\end{verbete}

\begin{verbete}[6]{先}{xian1}
  \significado{adv.}{em primeiro lugar; primeiramente}
\end{verbete}

\begin{verbete}[6;4;6]{先不先}{xian1bu4xian1}
  \significado{adv.}{dialeto: antes de tudo; em primeiro lugar}
\end{verbete}

\begin{verbete}[6;8;6;11]{先到先得}[\\]{xian1dao4xian1de2}
  \significado{s.}{primeiro a chegar, primeiro a ser servido}
\end{verbete}

\begin{verbete}[6;10]{先烈}{xian1lie4}
  \significado{s.}{mártir}
\end{verbete}

\begin{verbete}[6;12]{先期}{xian1qi1}
  \significado{adv.}{antecipadamente}
  \significado{s.}{prematuro; \textit{front-end}}
\end{verbete}

\begin{verbete}[6;5]{先生}{xian1sheng0}
  \significado[位]{s.}{senhor; marido; professor; dialeto: doutor}
\end{verbete}

\begin{verbete}[6;10]{先验}{xian1yan4}
  \significado{adj.}{filosofia: a priori}
\end{verbete}

\begin{verbete}[6;6]{先有}{xian1you3}
  \significado{adj.}{preexistente; anterior}
\end{verbete}

\begin{verbete}[9]{咸}{xian2}
  \significado{adj.}{salgado}
\end{verbete}

\begin{verbete}[9;12]{咸菜}{xian2cai4}
  \significado{s.}{legumes salgados; \textit{pickles}}
\end{verbete}

\begin{verbete}[9;11]{咸淡}{xian2dan4}
  \significado{s.}{água salobra; grau de salinidade; salgado e sem sal (sabores)}
\end{verbete}

\begin{verbete}[9;6]{咸肉}{xian2rou4}
  \significado{s.}{\textit{bacon}; carne curada com sal}
\end{verbete}

\begin{verbete}[9;10]{咸涩}{xian2se4}
  \significado{s.}{ácido; salgado e amargo}
\end{verbete}

\begin{verbete}[9;4]{咸水}{xian2shui3}
  \significado{s.}{salmora; água salgada}
\end{verbete}

\begin{verbete}[9;10]{咸盐}{xian2yan2}
  \significado{s.}{coloquial: sal; sal de mesa}
\end{verbete}

\begin{verbete}[9;8]{咸鱼}{xian2yu2}
  \significado{s.}{peixe salgado}
\end{verbete}

\begin{verbete}[8;8]{现货}{xian4huo4}
  \significado{s.}{produtos à vista}
\end{verbete}

\begin{verbete}[8;8;8]{现货的}{xian4huo4de0}
  \significado{s.}{produtos em estoque}
\end{verbete}

\begin{verbete}[9;8]{现实}{xian2shi2}
  \significado{adj.}{real; realístico}
  \significado{s.}{realidade}
\end{verbete}

\begin{verbete}[8;11]{现象}{xian4xiang4}
  \significado[个,种]{s.}{fenômeno}
\end{verbete}

\begin{verbete}[8;6]{现有}{xian4you3}
  \significado{adj.}{disponível atualmente; atualmente existente}
\end{verbete}

\begin{verbete}[8;6]{现在}{xian4zai4}
  \significado{p.t.}{agora; neste momento}
\end{verbete}

\begin{verbete}[8;7]{现抓}{xian4zhua1}
  \significado{v.}{improvisar}
\end{verbete}

\begin{verbete}[8;11]{现做}{xian4zuo4}
  \significado{adj.}{fresco}
  \significado{v.}{fazer (comida) no local}
\end{verbete}

\begin{verbete}[9;8]{香波}{xiang1bo1}
  \significado{s.}{xampu}
\end{verbete}

\begin{verbete}[9;7]{香肠}{xiang1chang2}
  \significado[根]{s.}{salsicha}
\end{verbete}

\begin{verbete}[9;12]{香港}{xiang1gang3}
  \significado{s.}{Hong Kong}
\end{verbete}

\begin{verbete}[9;12]{香港岛}[\\]{xiang1gang3dao3}
  \significado{s.}{Ilha de Hong Kong}
\end{verbete}

\begin{verbete}[9;15]{香蕉}{xiang1jiao1}
  \significado[枝,根,个,把]{s.}{banana}
\end{verbete}

\begin{verbete}[9;8]{香味}{xiang1wei4}
  \significado[股]{s.}{fragrância; cheiro doce}
\end{verbete}

\begin{verbete}[9;15]{香蕈}{xiang1xun4}
  \significado{s.}{\textit{shiitake}, cogumelo comestível}
\end{verbete}

\begin{verbete}[9;7]{香皂}{xiang1yan1}
  \significado[支,条]{s.}{cigarro; fumaça da queima de incenso}
\end{verbete}

\begin{verbete}[9;7]{香皂}{xiang1zao4}
  \significado{s.}{sabonete; sabonete perfumado}
\end{verbete}

\begin{verbete}[13]{想}{xiang3}
  \significado{v./v.o.}{acreditar; sentir falta (sentir-se melancólico com a ausência de alguém ou algo); supor; pensar; querer; desejar}
\end{verbete}

\begin{verbete}[13;8]{想念}{xiang3nian4}
  \significado{v.}{perder; sentir falta; lembrar com saudade}
\end{verbete}

\begin{verbete}[13;8]{想法}{xiang3xiang3kan4}
  \significado[个]{s.}{noção; opinião; jeito de pensar}
  \significado{v.}{pensar em uma maneira (de fazer algo)}
\end{verbete}

\begin{verbete}[13;13;9]{想想看}[\\]{xiang3xiang3kan4}
  \significado{v.}{pensar sobre isso}
\end{verbete}

\begin{verbete}[13;11]{想象}{xiang3xiang4}
  \significado{v.}{imaginar}
\end{verbete}

\begin{verbete}[6]{向}{xiang4}
  \significado{prep.}{para}
\end{verbete}

\begin{verbete}[6;7]{向汪}{xiang4wang3}
  \significado{v.}{esperar que}
\end{verbete}

\begin{verbete}[3]{小}{xiao3}
  \significado{adj.}{pequeno; jovem}
\end{verbete}

\begin{verbete}[3;5;11]{小白菜}[\\]{xiao3bai2cai3}
  \significado[棵]{s.}{bok choy; couve chinesa}
\end{verbete}

\begin{verbete}[3;8]{小姐}{xiao3jie0}
  \significado[个,位]{s.}{senhorita; jovem senhora; gíria: prostituta}
\end{verbete}

\begin{verbete}[3;9]{小说}{xiao3shuo1}
  \significado[本,部]{s.}{romance; ficção}
\end{verbete}

\begin{verbete}[3;7]{小时}{xiao3shi2}
  \significado{p.c.}{hora; para horas}
  \significado[个]{s.}{hora}
\end{verbete}

\begin{verbete}[3;9]{小树}{xiao3shu4}
  \significado[棵]{s.}{muda; arbusto; árvore pequena}
\end{verbete}

\begin{verbete}[3;13]{小腿}{xiao3tui3}
  \significado{s.}{perna (do joelho ao calcanhar); haste}
\end{verbete}

\begin{verbete}[3]{小小}{xiao3xiao3}
  \significado{adj.}{muito pequeno}
\end{verbete}

\begin{verbete}[3;4]{小心}{xiao3xin1}
  \significado{adj.}{cuidado}
\end{verbete}

\begin{verbete}[3;8]{小学}{xiao3xue2}
  \significado{s.}{escola ensino fundamental}
\end{verbete}

\begin{verbete}[3;9;5;12]{小洋白菜}[\\]{xiao3yang2bai2cai1}
  \significado{s.}{couve de bruxelas}
\end{verbete}

\begin{verbete}[10]{校}{xiao4}
  \significado[所]{s.}{oficial militar; escola}
  \veja{校}{jiao4}
\end{verbete}

\begin{verbete}[10;8]{校服}{xiao4fu2}
  \significado{s.}{uniforme escolar}
\end{verbete}

\begin{verbete}[10;8]{校规}{xiao4gui1}
  \significado{s.}{regras e regulamentos escolares}
\end{verbete}

\begin{verbete}[10;10]{校监}{xiao4jian4}
  \significado{s.}{diretor; supervisor (de escola)}
\end{verbete}

\begin{verbete}[10;4]{校长}{xiao4zhang3}
  \significado[个,位,名]{s.}{diretor de escola; reitor (universidade)}
\end{verbete}

\begin{verbete}[8]{些}{xie1}
  \significado{adv.}{uns; alguns; vários}
  \significado{p.c.}{que indica uma pequena quantidade ou pequeno número maior que 1}
\end{verbete}

\begin{verbete}[8;6]{些许}{xie1xu3}
  \significado{num.}{um pouco}
\end{verbete}

\begin{verbete}[5]{写}{xie3}
  \significado{v.}{escrever}
\end{verbete}

\begin{verbete}[5;13]{写意}{xie3yi4}
  \significado{s.}{estilo de pintura chinesa à mão livre, caracterizado por traços ousados em vez de detalhes precisos}
  \significado{v.}{sugerir (em vez de descrever em detalhes)}
\end{verbete}

\begin{verbete}[5;13]{写照}{xie3zhao4}
  \significado{s.}{retrato}
\end{verbete}

\begin{verbete}[5;10]{写真}{xie3zhen1}
  \significado{s.}{retrato}
  \significado{v.}{descrever algo com precisão}
\end{verbete}

\begin{verbete}[5;7]{写作}{xie3zuo4}
  \significado{s.}{escrita; redação; composição}
  \significado{v.}{escrever}
\end{verbete}

\begin{verbete}[12;10]{谢病}{xie4bing4}
  \significado{v.}{desculpar-se por causa de doença}
\end{verbete}

\begin{verbete}[12;10]{谢恩}{xie4en1}
  \significado{v.}{agradecer a alguém pelo favor (especialmente imperador ou oficial superior)}
\end{verbete}

\begin{verbete}[12;12]{谢媒}{xie4mei2}
  \significado{v.}{para agradecer ao casamenteiro}
\end{verbete}

\begin{verbete}[12;5]{谢世}{xie4shi4}
  \significado{v.}{morrer; falecer}
\end{verbete}

\begin{verbete}[12;4;12;6]{谢天谢地}[\\]{xie4tian1xie4di4}
  \significado{expr.}{agradecer a Deus; agradecer aos céus}
\end{verbete}

\begin{verbete}[12;12]{谢谢}{xie4xie0}
  \significado{interj.}{Obrigado!}
  \significado{v.}{agradecer}
\end{verbete}

\begin{verbete}[12;13]{谢意}{xie4yi4}
  \significado{s.}{gratidão}
\end{verbete}

\begin{verbete}[13]{新}{xin1}
  \significado{adj.}{novo}
  \significado{adv.}{novo}
\end{verbete}

\begin{verbete}[13;6]{新年}{xin1nian2}
  \significado[个]{s.}{Ano Novo}
\end{verbete}

\begin{verbete}[13;9]{新闻}{xin1wen2}
  \significado[条,个]{s.}{notícia}
\end{verbete}

\begin{verbete}[13;14]{新鲜}{xin1xian1}
  \significado{adj.}{fresco (experiência, alimento, etc.)}
  \significado{s.}{frescor}
\end{verbete}

\begin{verbete}[9]{信}{xin4}
  \significado[封]{s.}{carta; correspondência}
\end{verbete}

\begin{verbete}[9;6]{信访}{xin4fan3}
  \significado{s.}{carta de reclamação; carta de petição}
  \veja{上访}{shang4fang3}
\end{verbete}

\begin{verbete}[9;9]{信封}{xin4feng1}
  \significado[个]{s.}{envelope}
\end{verbete}

\begin{verbete}[9;8]{信经}{xin4jing1}
  \significado[个]{s.}{Crença; Credo (seção da missa católica)}
\end{verbete}

\begin{verbete}[9;4]{信心}{xin4xin1}
  \significado[个]{s.}{confiança; fé (em alguém ou algo)}
\end{verbete}

\begin{verbete}[9;8]{星表}{xing1biao3}
  \significado{s.}{catálogo de estrelas}
\end{verbete}

\begin{verbete}[9;4]{星火}{xing1huo3}
  \significado{s.}{trilha de meteoro (usada principalmente em expressões como 急如星火); faísca}
\end{verbete}

\begin{verbete}[9;12]{星期}{xing1qi1}
  \significado[个]{s.}{semana}
\end{verbete}

\begin{verbete}[9;12;1]{星期一}{xing1qi1yi4}
  \significado{p.t.}{segunda-feira}
\end{verbete}

\begin{verbete}[9;12;2]{星期二}{xing1qi1er4}
  \significado{p.t.}{terça-feira}
\end{verbete}

\begin{verbete}[9;12;3]{星期三}{xing1qi1san1}
  \significado{p.t.}{quarta-feira}
\end{verbete}

\begin{verbete}[9;12;5]{星期四}{xing1qi1si4}
  \significado{p.t.}{quinta-feira}
\end{verbete}

\begin{verbete}[9;12;4]{星期五}{xing1qi1wu3}
  \significado{p.t.}{sexta-feira}
\end{verbete}

\begin{verbete}[9;12;3]{星期六}{xing1qi1liu4}
  \significado{p.t.}{sábado}
\end{verbete}

\begin{verbete}[9;12;4]{星期天}[\\]{xing1qi1tian1}
  \significado{p.t.}{domingo}
  \veja{星期日}{xing1qi1ri4}
\end{verbete}

\begin{verbete}[9;12;4]{星期日}{xing1qi1ri4}
  \significado{p.t.}{domingo}
  \veja{星期天}{xing1qi1tian1}
\end{verbete}

\begin{verbete}[9;9]{星星}{xing1xing0}
  \significado{p.t.}{estrela}
\end{verbete}

\begin{verbete}[9;10]{星座}{xing1zuo4}
  \significado[张]{s.}{signo astrológico; constelação}
\end{verbete}

\begin{verbete}[6]{行}{xing2}
  \significado{interj.}{OK!}
  \significado{v.}{claro que sim; de acordo; está bem}
  \veja{行}{hang2}
\end{verbete}

\begin{verbete}[6;6]{行动}{xing2dong4}
  \significado[个]{s.}{ação; operação}
  \significado{v.}{mover}
\end{verbete}

\begin{verbete}[6;7]{行进}{xing2jin4}
  \significado{s.}{avançar; movimentar-se para frente}
\end{verbete}

\begin{verbete}[6;7]{行李}{xing2li0}
  \significado[件]{s.}{bagagem}
\end{verbete}

\begin{verbete}[6;2]{行人}{xing2ren2}
  \significado{s.}{transeunte; viajante à pé}
\end{verbete}

\begin{verbete}[6;9]{行星}{xing2xing1}
  \significado[颗]{s.}{planeta}
\end{verbete}

\begin{verbete}[8]{姓}{xing4}
  \significado[个]{s.}{sobrenome}
  \significado{v.}{ter o sobrenome}
\end{verbete}

\begin{verbete}[8;6]{姓名}{xing4ming2}
  \significado{s.}{nome completo}
\end{verbete}

\begin{verbete}[8;4]{姓氏}{xing4shi4}
  \significado{s.}{sobrenome}
\end{verbete}

\begin{verbete}[6;15]{兴趣}{xing4qu4}
  \significado[个]{s.}{interesse (desejo de conhecer sobre alguma coisa ou coisa no qual está interessado); hobby}
\end{verbete}

\begin{verbete}[10]{胸}{xiong1}
  \significado{s.}{peito; tórax}
\end{verbete}

\begin{verbete}[9]{修}{xiu1}
  \significado{v.}{reparar; consertar; construir}
\end{verbete}

\begin{verbete}[6;7]{休兵}{xiu1bing1}
  \significado{s.}{armistício}
  \significado{v.}{cessar fogo}
\end{verbete}

\begin{verbete}[9;7]{修改}{xiu1gai3}
  \significado{v.}{alterar; modificar; complementar}
\end{verbete}

\begin{verbete}[9;8]{修规}{xiu1gui1}
  \significado{s.}{plano de construção}
\end{verbete}

\begin{verbete}[6;16]{休憩}{xiu1qi4}
  \significado{v.}{relaxar; descansar; dar um tempo}
\end{verbete}

\begin{verbete}[6;10]{休息}{xiu1xi0}
  \significado{s.}{descanço}
  \significado{v.}{descansar}
\end{verbete}

\begin{verbete}[6;10;9]{休息室}{xiu1xi1shi4}
  \significado{s.}{saguão; salão; }
\end{verbete}

\begin{verbete}[6;7]{休闲}{xiu1xian2}
  \significado{s.}{ócio; lazer}
  \significado{v.}{desfrutar do lazer}
\end{verbete}

\begin{verbete}[6;16]{休整}{xiu1xheng3}
  \significado{v.}{militar: descansar e reorganizar}
\end{verbete}

\begin{verbete}[14;9]{需要}{xu1yao4}
  \significado{s.}{necessidade}
  \significado{v.}{precisar; necessitar}
\end{verbete}

\begin{verbete}[8]{学}{xue2}
  \significado{v.}{aprender; estudar}
\end{verbete}

\begin{verbete}[8;9]{学费}{xue2fei4}
  \significado[个]{s.}{mensalidade}
\end{verbete}

\begin{verbete}[8;4]{学分}{xue2fen1}
  \significado{s.}{créditos de um curso}
\end{verbete}

\begin{verbete}[8;12]{学问}{xue2qi1}
  \significado[个]{s.}{semestre}
\end{verbete}

\begin{verbete}[8;5]{学生}{xue2sheng0}
  \significado{s.}{estudante; aluno}
\end{verbete}

\begin{verbete}[8;5;7]{学生证}[\\]{xue2sheng0zheng4}
  \significado{s.}{cartão de identidade de estudante}
\end{verbete}

\begin{verbete}[8;5]{学术}{xue2shu4}
  \significado{s.}{aprendizagem; ciência}
\end{verbete}

\begin{verbete}[8;6]{学问}{xue2wen4}
  \significado{s.}{conhecimento; aprendizagem}
\end{verbete}

\begin{verbete}[8;3]{学习}{xue2xi2}
  \significado{v.}{estudar; aprender}
\end{verbete}

\begin{verbete}[8;10]{学校}{xue2xiao4}
  \significado{s.}{escola; instituição de ensino}
\end{verbete}

\begin{verbete}[8;9]{学院}{xue2yuan4}
  \significado[所]{s.}{instituto}
\end{verbete}

\begin{verbete}[11]{雪}{xue3}
  \significado[场]{s.}{neve}
\end{verbete}

\begin{verbete}[11;8]{雪板}{xue3ban3}
  \significado{s.}{prancha de \textit{snowboard}}
  \significado{v.}{praticar \textit{snowboard}}
\end{verbete}

\begin{verbete}[11;7]{雪花}{xue3hua1}
  \significado{s.}{floco de neve}
\end{verbete}

\begin{verbete}[11;12]{雪葩}{xue3pa1}
  \significado{s.}{sorvete}
\end{verbete}

\begin{verbete}[11;2]{雪人}{xue3ren2}
  \significado{s.}{boneco de neve; \textit{Yeti}}
\end{verbete}

\begin{verbete}[11;15]{雪鞋}{xue3xie2}
  \significado[双]{s.}{sapatos de neve}
\end{verbete}

%%%%% EOF %%%%%

%%%
%%% Y
%%%
\section*{Y}
\addcontentsline{toc}{section}{Y}
\begin{multicols*}{2}

\begin{verbete}[鸭]{ya1}
\significado{ya1}{n.}{
    pato
}
\end{verbete}

\begin{verbete}[压岁钱]{ya1sui4qian2}
\significado{ya1sui4qian2}{n.}{
    dinheiro da sorte|
    dinheiro dado às crianças como presente no Ano Novo Chinês
}
\end{verbete}

\begin{verbete}[牙]{ya2}
\significado{ya2}{n.}{
    dente
}
\end{verbete}

\begin{verbete}[牙齿]{ya2chi3}
\significado{ya2chi3}{n.}{
    dente
}
\end{verbete}

\begin{verbete}[亚洲]{Ya4zhou1}
\significado{Ya4zhou1}{n.}{
    Ásia
}
\end{verbete}

\begin{verbete}[颜色]{yan2se4}
\significado{yan2se4}{n.}{
    cor
}
\end{verbete}

\begin{verbete}[眼镜]{yan3jing4}
\significado{yan3jing4}{n.}{
    óculos|
    \pc{副}
}
\end{verbete}

\begin{verbete}[眼睛]{yan3jing0}
\significado{yan3jing0}{n.}{
    olho(s)
}
\end{verbete}

\begin{verbete}[养]{yang3}
\significado{yang3}{v.}{
    criar (animais), plantar (flores), etc
}
\end{verbete}

\begin{verbete}[样子]{yang4zi0}
\significado{yang4zi0}{n.}{
    aparência;forma
}
\end{verbete}

\begin{verbete}[腰]{yao1}
\significado{yao1}{n.}{
    cintura
}
\end{verbete}

\begin{verbete}[药]{yao4}
\significado{yao4}{n.}{
    medicamento; remédio
}
\end{verbete}

\begin{verbete}[要]{yao4}
\significado{yao4}{v./v.o.}{
    querer; precisar
}
\end{verbete}

\begin{verbete}[要是]{yao4shi0}
\significado{yao4shi0}{conj.}{
    se
}
\end{verbete}

\begin{verbete}[要是······的话]{yao4shi0 ...\  de0hua0}
\significado{yao4shi0 ...\  de0hua0}{conj.}{
    se ... no caso de
}
\end{verbete}

\begin{verbete}[爷爷]{ye2ye0}
\significado{ye2ye0}{n.}{
    avô (paterno)
}
\end{verbete}

\begin{verbete}[也]{ye3}
\significado{ye3}{adv.}{
    também
}
\end{verbete}

\begin{verbete}[夜里]{ye4li0}
\significado{ye4li0}{p.t.}{
    noite
}
\end{verbete}

\begin{verbete}[一]{yi1}
\significado{yi1}{num.}{
    1|
    um, uma (quando usado sozinho)
}
\significado{yi2}{num.}{
    1|
    um, uma (antes de quarto tom)
}
\significado{yi4}{num.}{
    1|
    um, uma
}
\end{verbete}

\begin{verbete}[一]{yi2}
\significado{yi2}{num.}{
    1|
    um, uma (antes de quarto tom)
}
\significado{yi1}{num.}{
    1|
    um, uma (quando usado sozinho)
}
\significado{yi4}{num.}{
    1|
    um, uma
}
\end{verbete}

\begin{verbete}[一半]{yi2ban4}
\significado{yi2ban4}{adj.}{
    metade
}
\end{verbete}

\begin{verbete}[一定]{yi2ding4}
\significado{yi2ding4}{adv.}{
    certamente; definitivamente
}
\end{verbete}

\begin{verbete}[一共]{yi2gong4}
\significado{yi2gong4}{adv.}{
    tudo; no local
}
\end{verbete}

\begin{verbete}[一下]{yi2xia4}
\significado{yi2xia4}{adv.}{
    em um curto tempo; rapidamente
}
\end{verbete}

\begin{verbete}[一样]{yi2yang4}
\significado{yi2yang4}{adj.}{
    igual; mesmo, mesma
}
\end{verbete}

\begin{verbete}[一]{yi4}
\significado{yi4}{num.}{
    1|
    um, uma
}
\significado{yi2}{num.}{
    1|
    um, uma (antes de quarto tom)
}
\significado{yi1}{num.}{
    1|
    um, uma (quando usado sozinho)
}
\end{verbete}

\begin{verbete}[一般]{yi4ban1}
\significado{yi4ban1}{adj.}{
    geral; comum; normal
}
\significado{yi4ban1}{adv.}{
    normalmente
}
\end{verbete}

\begin{verbete}[一点儿]{yi4dianr3}
\significado{yi4dianr3}{adv.}{
    um pouco
}
\end{verbete}

\begin{verbete}[一会儿]{yi4huir4}
\significado{yi4huir4}{adv.}{
    daqui a pouco tempo; pouco tempo
}
\end{verbete}

\begin{verbete}[一起]{yi4qi3}
\significado{yi4qi3}{adv.}{
    juntamente; em conjunto
}
\end{verbete}

\begin{verbete}[一直]{yi4zhi2}
\significado{yi4zhi2}{adv.}{
    diretamente; sempre
}
\end{verbete}

\begin{verbete}[一些]{yi4xie1}
\significado{yi4xie1}{pron.}{
    uns, umas|
    alguns, algumas
}
\end{verbete}

\begin{verbete}[衣服]{yi1fu0}
\significado{yi1fu0}{n.}{
    roupa, vestuário|
    \pc{件}
}
\end{verbete}

\begin{verbete}[医生]{yi1sheng1}
\significado{yi1sheng1}{n.}{
    médico; clínico
}
\end{verbete}

\begin{verbete}[医院]{yi1yuan0}
\significado{yi1yuan0}{n.}{
    hospital
}
\end{verbete}

\begin{verbete}[颐和园]{yi2he2yuan2}
\significado{yi2he2yuan2}{n.}{
    Palácio de Verão
}
\end{verbete}

\begin{verbete}[遗憾]{yi2han4}
\significado{yi2han4}{v.}{
    ter pena de
}
\end{verbete}

\begin{verbete}[以后]{yi3hou4}
\significado{yi3hou4}{n.}{
    depois de; depois; após
}
\end{verbete}

\begin{verbete}[以前]{yi3qian2}
\significado{yi3qian2}{p.t.}{
    antes de; antes
}
\end{verbete}

\begin{verbete}[已经]{yi3jing1}
\significado{yi3jing1}{adv.}{
    já
}
\end{verbete}

\begin{verbete}[亿]{yi4}
\significado{yi4}{num.}{
    100.000.000|
    cem milhões
}
\end{verbete}

\begin{verbete}[意思]{yi4si0}
\significado{yi4si0}{n.}{
    interesse
}
\end{verbete}

\begin{verbete}[阴天]{yin1tian1}
\significado{yin1tian1}{adj.}{
    céu muito nublado; céu cinzento
}
\end{verbete}

\begin{verbete}[因为]{yin1wei4}
\significado{yin1wei4}{conj.}{
    porque
}
\end{verbete}

\begin{verbete}[音乐]{yin1yue4}
\significado{yin1yue4}{n.}{
    música
}
\end{verbete}

\begin{verbete}[银行]{yin2hang2}
\significado{yin2hang2}{n.}{
    banco; agência bancária
}
\end{verbete}

\begin{verbete}[饮料]{yin3liao4}
\significado{yin3liao4}{n.}{
    bebida
}
\end{verbete}

\begin{verbete}[应该]{ying1gai1}
\significado{ying1gai1}{v.}{
    dever; ter de
}
\end{verbete}

\begin{verbete}[英国]{Ying1guo2}
\significado{Ying1guo2}{n.}{
    Reino Unido
}
\end{verbete}

\begin{verbete}[英语]{ying1yu3}
\significado{ying1yu3}{n.}{
    inglês, língua inglesa
}
\end{verbete}

\begin{verbete}[英文]{ying1wen2}
\significado{ying1wen2}{n.}{
    inglês, língua inglesa
}
\end{verbete}

\begin{verbete}[优美]{you1mei3}
\significado{you2jian4}{n.}{
    correspondência
}
\end{verbete}

\begin{verbete}[邮件]{you2jian4}
\significado{you2jian4}{n.}{
    correspondência
}
\end{verbete}

\begin{verbete}[邮局]{you2ju4}
\significado{you2ju4}{n.}{
    correio; agência dos correios
}
\end{verbete}

\begin{verbete}[游]{you2}
\significado{you2}{v.}{
    nadar
}
\end{verbete}

\begin{verbete}[游泳]{you2yong3}
\significado{you2yong3}{v.+compl.}{
    nadar
}
\end{verbete}

\begin{verbete}[游泳池]{you2yong3chi2}
\significado{you2yong3chi2}{n.}{
    piscina
}
\end{verbete}

\begin{verbete}[有]{you3}
\significado{you3}{v.}{
    ter; haver
}
\end{verbete}

\begin{verbete}[有的]{you3de0}
\significado{you3de0}{pron.}{
    algum, alguma, alguns, algumas
}
\end{verbete}

\begin{verbete}[有的时候]{you3de0\ shi2hou0}
\significado{you3de0\ shi2hou0}{}{
    às vezes;
    de vez em quando;
    de quando em quando
}
\end{verbete}

\begin{verbete}[有点儿]{you3dianr3}
\significado{you3dianr3}{adv.}{
    um pouco
}
\end{verbete}

\begin{verbete}[有名]{you3ming2}
\significado{you3ming2}{adj.}{
    famoso, famosa
}
\end{verbete}

\begin{verbete}[有时]{you3shi2}
\significado{you3shi2}{}{
    às vezes;
    de vez em quando;
    de quando em quando
}
\end{verbete}

\begin{verbete}[有时候]{you3shi2hou0}
\significado{you3shi2hou0}{}{
    às vezes;
    de vez em quando;
    de quando em quando
}
\end{verbete}

\begin{verbete}[有意思]{you3yi2si0}
\significado{you3yi2si0}{adj.}{
    interessante
}
\end{verbete}

\begin{verbete}[有用]{you3yong4}
\significado{you3yong4}{adj.}{
    útil
}
\end{verbete}

\begin{verbete}[右]{you4}
\significado{you4}{p.l.}{
    direita
}
\end{verbete}

\begin{verbete}[右边]{you4bian0}
\significado{you4bian0}{p.l.}{
    à direita; ao lado direito
}
\end{verbete}

\begin{verbete}[右面]{you4mian0}
\significado{you4mian0}{p.l.}{
    à direita; ao lado direito
}
\end{verbete}

\begin{verbete}[用]{yong4}
\significado{yong4}{v.}{
    usar
}
\end{verbete}

\begin{verbete}[鱼]{yu2}
\significado{yu2}{n.}{
    peixe|
    \pc{条}
}
\end{verbete}

\begin{verbete}[鱼片]{yu2pian4}
\significado{yu2pian4}{n.}{
    fatia de peixe
}
\end{verbete}

\begin{verbete}[鱼香肉丝]{yu2xiang1rou4si1}
\significado{yu2xiang1rou4si1}{n.}{
    tiras de carne de porco salteadas com molho picante
}
\end{verbete}

\begin{verbete}[玉]{yu3}
\significado{yu3}{n.}{
    jade|
    \pc{块}
}
\end{verbete}

\begin{verbete}[雨]{yu3}
\significado{yu3}{n.}{
    chuva
}
\end{verbete}

\begin{verbete}[雨伞]{yu3san3}
\significado{yu3san3}{n.}{
    guarda-chuva
}
\end{verbete}

\begin{verbete}[雨衣]{yu3yi1}
\significado{yu3yi1}{n.}{
    impermeável
}
\end{verbete}

\begin{verbete}[羽毛球]{yu3mao2qiu2}
\significado{yu3mao2qiu2}{n.}{
    badminton
}
\end{verbete}

\begin{verbete}[语法]{yu3fa3}
\significado{yu3fa3}{n.}{
    gramática
}
\end{verbete}

\begin{verbete}[语言实验室]{yu3yan2shi2yan4shi4}
\significado{yu3yan2shi2yan4shi4}{n.}{
    laboratório de línguas
}
\end{verbete}

\begin{verbete}[预报]{yu4bao4}
\significado{yu4bao4}{n.}{
    previsão (meteorológica); boletim meteorológico
}
\significado{yu4bao4}{v.}{
    prever (o tempo)
}
\end{verbete}

\begin{verbete}[元]{yuan2}
\significado{yuan2}{p.c.}{
    unidade monetária da China
}
\end{verbete}

\begin{verbete}[远]{yuan3}
\significado{yuan3}{adj.}{
    longe; longo, longa
}
\end{verbete}

\begin{verbete}[院子]{yuan4zi0}
\significado{yuan4zi0}{n.}{
    pátio; jardim
}
\end{verbete}

\begin{verbete}[约会]{yue1hui4}
\significado{yue1hui4}{n.}{
    compromisso; encontro marcado
}
\end{verbete}

\begin{verbete}[月]{yue4}
\significado{yue4}{n.}{
    mês
}
\end{verbete}

\begin{verbete}[月亮]{yue4liang0}
\significado{yue4liang0}{n.}{
    lua
}
\end{verbete}

\begin{verbete}[阅读]{yue4du2}
\significado{yue4du2}{n.}{
    leitura
}
\significado{yue4du2}{v.}{
    ler
}
\end{verbete}

\begin{verbete}[越······越······]{yue4...\ yue4...}
\significado{yue4...\ yue4...}{}{
    quanto mais... tanto mais...
}
\end{verbete}

\begin{verbete}[越来越······]{yue4lai2yue4...}
\significado{yue4lai2yue4...}{}{
    cada vez mais...
}
\end{verbete}

\begin{verbete}[阅览室]{yue4lan3shi4}
\significado{yue4lan3shi4}{n.}{
    sala de leitura
}
\end{verbete}

\begin{verbete}[云南]{Yun2nan2}
\significado{Yun2nan2}{n.}{
    Yunnan
}
\end{verbete}

\begin{verbete}[运动]{yun4dong4}
\significado{yun4dong4}{n.}{
    esporte; desporto
}
\end{verbete}

\begin{verbete}[运动场]{yun4dong4chang3}
\significado{yun4dong4chang3}{n.}{
    campo desportivo; campo de jogos
}
\end{verbete}

\begin{verbete}[运动会]{yun4dong4hui4}
\significado{yun4dong4hui4}{n.}{
    jogos desportivos
}
\end{verbete}

\begin{verbete}[运动员]{yun4dong4yuan2}
\significado{yun4dong4yuan2}{n.}{
    jogador, jogadora; atleta
}
\end{verbete}

\end{multicols*}

%%%
%%% Z
%%%
\section*{Z}
\addcontentsline{toc}{section}{Z}

\begin{verbete}[6]{在}{zai4}
\significado{adv.}{ para designar ações que estão passando }
\significado{prep.}{ em }
\significado{v.}{ estar; ficar }
\end{verbete}

\begin{verbete}[6]{再}{zai4}
\significado{adv.}{ de novo; outra vez }
\end{verbete}

\begin{verbete}[6;4]{再见}{zai4jian4}
\significado{v.}{ adeus; até à vista; até à próxima; até logo }
\end{verbete}

\begin{verbete}[9;5]{咱们}{zan2men0}
\significado{pron.}{ nós (eu e você) }
\end{verbete}

\begin{verbete}[10]{脏}{zang1}
\significado{adj.}{ sujo }
\end{verbete}

\begin{verbete}[6]{早}{zao3}
\significado{adj.}{ cedo }
\end{verbete}

\begin{verbete}[6;7]{早饭}{zao3fan4}
\significado{s.}{ café da manhã }
\end{verbete}

\begin{verbete}[6;3]{早上}{zao3shang0}
\significado{p.t.}{ manhã cedo; manhãzinha }
\end{verbete}

\begin{verbete}[9;3]{怎么}{zen3me0}
\significado{interr.}{ como? }
\end{verbete}

\begin{verbete}[9;3;6;8]{怎么回事}[\\]{zen3me0hui2shi4}
\significado{expr.}{ O que aconteceu?; O que se passou? }
\end{verbete}

\begin{verbete}[9;3;10]{怎么样}[\\]{zen3me0yang4}
\significado{interr.}{ como?; que tal? }
\end{verbete}

\begin{verbete}[10]{站}{zhan4}
\significado{s.}{ estação; ponto; parada }
\end{verbete}

\begin{verbete}[7]{张}{zhang1}
\significado{p.c.}{ para folha de papéis, mapas, etc }
\end{verbete}

\begin{verbete}[11;9]{着急}{zhao2ji2}
\significado{adj.}{ inquieto; ansioso }
\end{verbete}

\begin{verbete}[7]{找}{zhao3}
\significado{v.}{ andar à procura de; procurar; dar troco }
\end{verbete}

\begin{verbete}[13;4]{照片}{zhao4pian4}
\significado[张,套,幅]{s.}{ fotografia; foto }
\end{verbete}

\begin{verbete}[9;4]{照相}{zhao4xiang4}
\significado{v.+compl.}{ tirar fotografia }
\end{verbete}

\begin{verbete}[13;9;6]{照相机}[\\]{zhao4xiang4ji1}
\significado[个,架,部,台,只]{s.}{ câmera/máquina fotográfica }
\end{verbete}

\begin{verbete}[11]{着}{zhe0}
\significado{part.}{ partícula indicando ação em andamento ou estado em andamento }
\end{verbete}

\begin{verbete}[7]{这}{zhe4}
\significado{pron.}{ este, isto }
\end{verbete}

\begin{verbete}[7;7]{这里}{zhe4li0}
\significado{pron.}{ aqui }
\end{verbete}

\begin{verbete}[7;3]{这么}{zhe4me0}
\significado{adv.}{ como este; desta maneira }
\end{verbete}

\begin{verbete}[7;8]{这些}{zhe4xie1}
\significado{pron.}{ estes }
\end{verbete}

\begin{verbete}[7;10]{这样}{zhe4yang4}
\significado{adv.}{ assim; dessa maneira; deste modo }
\end{verbete}

\begin{verbete}[7;2]{这儿}{zher4}
\significado{pron.}{ aqui }
\end{verbete}

\begin{verbete}[10;6]{浙江}{Zhe4jiang1}
\significado{s.}{ Zhejiang }
\end{verbete}

\begin{verbete}[10]{真}{zhen1}
\significado{adv.}{ que...tão...!; realmente }
\end{verbete}

\begin{verbete}[9;10]{挣钱}{zheng4qian2}
\significado{v.+compl.}{ ganhar dinheiro }
\end{verbete}

\begin{verbete}[5;6]{正在}{zheng4zai4}
\significado{adv.}{ estar a + inf.; estar + ger. }
\end{verbete}

\begin{verbete}[4]{支}{zhi1}
\significado{p.c.}{ para caneta, lápis, etc }
\end{verbete}

\begin{verbete}[5]{只}{zhi1} 
\significado{p.c.}{ para pássaros, gatos, cãezinhos, etc }
\veja{只}{zhi3}
\end{verbete}

\begin{verbete}[8;12]{知道}{zhi1dao0}
\significado{v.}{ conhecer; saber }
\end{verbete}

\begin{verbete}[11;7]{职员}{zhi2yuan2}
\significado{s.}{ empregado }
\end{verbete}

\begin{verbete}[5]{只}{zhi3}
\significado{adv.}{ apenas; só }
\veja{只}{zhi1}
\end{verbete}

\begin{verbete}[5;6]{只好}{zhi3hao3}
\significado{adv.}{ não ter outro remédio senão }
\end{verbete}

\begin{verbete}[4;8]{中国}{Zhong1guo2}
\significado{s.}{ China }
\end{verbete}

\begin{verbete}[4;8;2]{中国人}[\\]{zhong1guo2ren2}
\significado{s.}{ chinês; nascido na China }
\end{verbete}

\begin{verbete}[4;8;10]{中国通}[\\]{Zhong1guo2tong1}
\significado{s.}{ conhecedor da China; especialista em tudo sobre a China }
\end{verbete}

\begin{verbete}[4;7]{中间}{zhong1jian1}
\significado{p.l.}{ central; centro; no meio }
\end{verbete}

\begin{verbete}[4;4]{中文}{zhong1wen2}
\significado{s.}{ chinês, língua chinesa }
\end{verbete}

\begin{verbete}[4;8]{中学}{zhong1xue2}
\significado[个]{s.}{ escola ensino médio }
\end{verbete}

\begin{verbete}[4;8;5]{中学生}[\\]{zhong1xue2sheng1}
\significado{s.}{ estudante da escola ensino médio }
\end{verbete}

\begin{verbete}[4;8]{中询}{zhong1xun2}
\significado{p.t.}{ segunda dezena do mês; meio do mês; em meados do mês }
\end{verbete}

\begin{verbete}[9]{钟}{zhong1}
\significado{p.c.}{ hora }
\end{verbete}

\begin{verbete}[9]{种}{zhong3}
\significado{p.c.}{ para tipos, espécies e gêneros }
\end{verbete}

\begin{verbete}[9]{重}{zhong4}
\significado{adj.}{ pesado }
\end{verbete}

\begin{verbete}[8;12]{重量}{zhong4liang4}
\significado[个]{s.}{ peso }
\end{verbete}

\begin{verbete}[8;5]{周末}{zhou1mo4}
\significado{s.}{ final-de-semana }
\end{verbete}

\begin{verbete}[11]{猪}{zhu1}
\significado[口,头]{s.}{ porco; suíno }
\end{verbete}

\begin{verbete}[5;10]{主席}{Zhu3xi2}
\significado[个,位]{s.}{ Presidente (da China); Primeiro-Ministro }
\end{verbete}

\begin{verbete}[7]{住}{zhu4}
\significado{v.}{ morar; viver; alojar-se }
\end{verbete}

\begin{verbete}[7;6]{住宅}{zhu4zhai2}
\significado{s.}{ residência }
\end{verbete}

\begin{verbete}[8;5]{注册}{zhu4ce4}
\significado{v.}{ inscrever-se; matricular-se }
\end{verbete}

\begin{verbete}[9]{祝}{zhu4}
\significado{v.}{ desejar (exprimir um bom desejo); congratular }
\end{verbete}

\begin{verbete}[15;8]{嘱咐}{zhu4fu4}
\significado{v.}{ ordenar; dizer; exortar }
\end{verbete}

\begin{verbete}[4;5]{专业}{zhuan1ye4}
\significado[门,个]{s.}{ área de atuação; especialidade }
\end{verbete}

\begin{verbete}[12]{装}{zhuang1}
\significado{v.}{ instalar; montar }
\end{verbete}

\begin{verbete}[10;3]{桌子}{zhuo1zi0}
\significado[张,套]{s.}{ mesa }
\end{verbete}

\begin{verbete}[12;6]{紫色}{zi3se4}
\significado{s.}{ cor roxa }
\end{verbete}

\begin{verbete}[6]{字}{zi4}
\significado[个]{s.}{ carácter; letra; símbolo; palavra }
\end{verbete}

\begin{verbete}[6;3]{自己}{zi4ji3}
\significado{pron.}{ a si próprio; próprio }
\end{verbete}

\begin{verbete}[6;6;4]{自行车}{zi4xing2che1}
\significado[辆]{s.}{ bicicleta }
\end{verbete}

\begin{verbete}[6;7]{自我}{zi4wo3}
\significado{pron.}{ a si mesmo; eu próprio; auto-... }
\end{verbete}

\begin{verbete}[9;13]{总督}{Zong3du1}
\significado{s.}{ Governador; Governador-Geral; Vice-Rei }
\end{verbete}

\begin{verbete}[9;11]{总理}{Zong3li3}
\significado[个,位,名]{s.}{ Primeiro-Ministro }
\end{verbete}

\begin{verbete}[9;9]{总统}{Zong3tong3}
\significado[个,位,名,届]{s.}{ Presidente (de um país) }
\end{verbete}

\begin{verbete}[7]{走}{zou3}
\significado{v.}{ andar; caminhar }
\end{verbete}

\begin{verbete}[10]{租}{zu1}
\significado{s.}{ imposto sobre propriedade urbana ou rural }
\significado{v.}{ alugar; tomar de aluguel }
\end{verbete}

\begin{verbete}[7;11]{足球}{zu2qiu2}
\significado[个]{s.}{ futebol; bola de futebol }
\end{verbete}

\begin{verbete}[16;4]{嘴巴}{zui3ba0}
\significado[张]{s.}{ boca }
\significado[个]{s.}{ bofetada na cara }
\end{verbete}

\begin{verbete}[12]{最}{zui4}
\significado{adv.}{ o mais; grau superlativo relativo de superioridade }
\end{verbete}

\begin{verbete}[12;6]{最后}{zui4hou4}
\significado{adj.}{ final; último }
\end{verbete}

\begin{verbete}[12;7]{最近}{zui4jin4}
\significado{adv.}{ ultimamente; recentemente }
\end{verbete}

\begin{verbete}[9;4]{昨天}{zuo2tian1}
\significado{p.t.}{ ontem }
\end{verbete}

\begin{verbete}[5]{左}{zuo3}
\significado{p.l.}{ esquerda }
\end{verbete}

\begin{verbete}[5;5]{左边}{zuo3bian0}
\significado{p.l.}{ esquerda; lado esquerdo }
\end{verbete}

\begin{verbete}[5;9]{左面}{zuo3mian0}
\significado{p.l.}{ esquerda; lado esquerdo }
\end{verbete}

\begin{verbete}[5;5]{左右}{zuo3you4}
\significado{part.}{ cerca de; aproximadamente }
\end{verbete}

\begin{verbete}[7]{坐}{zuo4}
\significado{v.}{ sentar-se; andar de carro, ônibus, trem, avião, etc }
\end{verbete}

\begin{verbete}[11]{做}{zuo4}
\significado{v.}{ fazer }
\end{verbete}

%%%%% EOF %%%%%

\end{multicols}

\printindex

\end{document}
