%%%
%%% 2画
%%%

\section*{2画}\addcontentsline{toc}{section}{2画}

\begin{verbete}{八}{ba1}{2}[Radical 八][Kangxi 12]
  \significado{num.}{oito; 8}
\end{verbete}

\begin{verbete}{八八六}{ba1ba1liu4}{2,2,4}
  \significado{expr.}{\emph{Bye bye!} (em salas de bate-papo e mensagens de texto)}
\end{verbete}

\begin{verbete}{刀}{dao1}{2}[Radical ⼑][Kangxi 18]
  \significado*{s.}{sobrenome Dao}
  \significado{clas.}{para cortes de faca ou facadas}
  \significado[把]{s.}{faca; lâmina;  espada de fio único; cutelo; (gíria) dólar (empréstimo linguístico)}
\end{verbete}

\begin{verbete}{儿}{er2}{2}[Radical ⼉][Kangxi 10]
  \significado{s.}{criança; filho}
  \veja{儿}{r5}
  \veja{儿}{ren2}
\end{verbete}

\begin{verbete}{儿子}{er2zi5}{2,3}
  \significado{s.}{filho}
\end{verbete}

\begin{verbete}{儿媳}{er2xi2}{2,13}
  \significado{s.}{esposa do filho}
\end{verbete}

\begin{verbete}{二}{er4}{2}[Radical ⼆][Kangxi 7]
  \significado{num.}{dois, 2}
\end{verbete}

\begin{verbete}{二战}{er4zhan4}{2,9}
  \significado*{s.}{Segunda Guerra Mundial}
\end{verbete}

\begin{verbete}{几}{ji1}{2}[Radical 几][Kangxi 16]
  \significado{adv.}{quase}
  \significado{s.}{mesa pequena}
  \veja{几}{ji3}
\end{verbete}

\begin{verbete}{几乎}{ji1hu1}{2,5}
  \significado{adv.}{quase}
\end{verbete}

\begin{verbete}{几}{ji3}{2}[Radical 几]
  \significado{interr.}{quantos?, (até 10 itens); alguns?}
  \veja{几}{ji1}
\end{verbete}

\begin{verbete}{几何}{ji3he2}{2,7}
  \significado{s.}{geometria}
\end{verbete}

\begin{verbete}{九}{jiu3}{2}[Radical 乙]
  \significado{num.}{nove, 9}
\end{verbete}

\begin{verbete}{了}{le5}{2}[Radical 亅]
  \significado{part.}{marcador de ação concluída; partícula modal indicando mudança de estado, situação; partícula modal intensificando a cláusula anterior}
  \veja{了}{liao3}
  \veja{了}{liao4}
\end{verbete}

\begin{verbete}{力量}{li4liang5}{2,12}
  \significado{s.}{poder, vigor, força}
\end{verbete}

\begin{verbete}{了}{liao3}{2}[Radical 亅]
  \significado{v.}{terminar; alcançar; entender claramente}
  \veja{了}{le5}
  \veja{了}{liao4}
\end{verbete}

\begin{verbete}{了解}{liao3jie3}{2,13}
  \significado{v.}{entender; perceber; descobrir}
\end{verbete}

\begin{verbete}{了}{liao4}{2}[Radical 亅]
  \significado{adj.}{brilhantes (olhos)}
  \significado{v.}{observar; olhar para fora; olhar para baixo de um lugar mais alto; compreender claramente}
  \veja{了}{le5}
  \veja{了}{liao3}
\end{verbete}

\begin{verbete}{七}{qi1}{2}[Radical 一]
  \significado{num.}{sete, 7}
\end{verbete}

\begin{verbete}{七夕}{qi1xi1}{2,3}
  \significado*{s.}{Dia dos Namorados Chinês, quando o vaqueiro e a tecelã (牛郎织女) têm permissão para se reunirem anualmente; Festival das Meninas; Festival Duplo Sete, noite do sétimo mês lunar}
  \veja{牛郎织女}{niu2lang2zhi1nv3}
\end{verbete}

\begin{verbete}{儿}{r5}{2}[Radical 儿]
  \significado{conj.}{sufixo diminutivo não silábico; final retroflexo}
  \veja{儿}{er2}
  \veja{儿}{ren2}
\end{verbete}

\begin{verbete}{人}{ren2}{2}[Radical 人][Kangxi 9]
  \significado[个,位]{s.}{pessoa; gente}
\end{verbete}

\begin{verbete}{人才}{ren2cai2}{2,3}
  \significado{s.}{talento; pessoa talentosa}
\end{verbete}

\begin{verbete}{人口}{ren2kou3}{2,3}
  \significado{s.}{pessoas; população}
\end{verbete}

\begin{verbete}{人民}{ren2min2}{2,5}
  \significado[个]{s.}{povo; população}
\end{verbete}

\begin{verbete}{人民币}{ren2min2bi4}{2,5,4}
  \significado*{s.}{Renminbi (RMB); Yuan Chinês (CYN); nome da moeda chinesa}
\end{verbete}

\begin{verbete}{人生}{ren2sheng1}{2,5}
  \significado{s.}{vida (tempo de alguém na Terra)}
\end{verbete}

\begin{verbete}{人权}{ren2quan2}{2,6}
  \significado*{s.}{Direitos Humanos}
  \veja{人权法}{ren2quan2fa3}
\end{verbete}

\begin{verbete}{人权法}{ren2quan2fa3}{2,6,8}
  \significado*{s.}{Direitos Humanos}
  \veja{人权}{ren2quan2}
\end{verbete}

\begin{verbete}{人行道}{ren2xing2dao4}{2,6,12}
  \significado{s.}{calçada}
\end{verbete}

\begin{verbete}{人材}{ren2cai2}{2,7}
  \variante{人才}
\end{verbete}

\begin{verbete}{人间}{ren2jian1}{2,7}
  \significado{s.}{o mundo humano; a Terra}
\end{verbete}

\begin{verbete}{人鱼}{ren2yu2}{2,8}
  \significado{s.}{sereia; peixe-boi; salamandra gigante}
\end{verbete}

\begin{verbete}{人类}{ren2lei4}{2,9}
  \significado{s.}{humanidade; raça humana}
\end{verbete}

\begin{verbete}{人海}{ren2hai3}{2,10}
  \significado{s.}{uma multidão; um mar de pessoas}
\end{verbete}

\begin{verbete}{人道}{ren2dao4}{2,12}
  \significado{s.}{solidariedade humana; humanitarismo; humano; a ``maneira humana'', um dos estágios do ciclo de reencarnação (budismo); relação sexual}
\end{verbete}

\begin{verbete}{人像}{ren2xiang4}{2,13}
  \significado{s.}{``retrato'' de uma pessoa (esboço, foto, escultura, etc.)}
\end{verbete}

\begin{verbete}{儿}{ren2}{2}[Radical 儿]
  \significado{s.}{pessoa, radical em caracteres chineses}
  \variante{人}
  \veja{儿}{er2}
  \veja{儿}{r5}
\end{verbete}

\begin{verbete}{入门}{ru4men2}{2,3}
  \significado{s.}{curso elementar; ABC; guia}
  \significado{v.+compl.}{atravessar o limiar; aprender o ABC de; introduzir um assunto; aprender os rudimentos de um assunto |}
\end{verbete}

\begin{verbete}{入乡随俗}{ru4xiang1-sui2su2}{2,3,11,9}
  \significado{expr.}{Em roma, faça como os romanos!}
\end{verbete}

\begin{verbete}{入党}{ru4dang3}{2,10}
  \significado{v.}{ingressar em um partido político (especialmente o partido comunista)}
\end{verbete}

\begin{verbete}{入境}{ru4jing4}{2,14}
  \significado{s.}{imigração}
  \significado{v.+compl.}{entrar em um país; imigrar}
\end{verbete}

\begin{verbete}{十}{shi2}{2}[Radical 十][Kangxi 24]
  \significado{num.}{dez, 10; dezena}
\end{verbete}

\begin{verbete}{十分}{shi2fen1}{2,4}
  \significado{adv.}{muito; extremamente; totalmente; absolutamente}
\end{verbete}

\begin{verbete}{十足}{shi2zu2}{2,7}
  \significado{adj.}{amplo; completo; cento por cento; tom puro (de alguma cor)}
\end{verbete}

\begin{verbete}{又}{you4}{2}[Radical 又][Kangxi 29]
  \significado{adv.}{mais uma vez; (usado para dar ênfase) de qualquer maneira; e ainda; e também}
\end{verbete}

\begin{verbete}{又一次}{you4yi2ci4}{2,1,6}
  \significado{adv.}{outra vez; mais uma vez; de novo}
\end{verbete}

\begin{verbete}{又及}{you4ji2}{2,3}
  \significado{s.}{P.S.; \emph{postscript}}
\end{verbete}

\begin{verbete}{又名}{you4ming2}{2,6}
  \significado{s.}{também conhecido como; nome alternativo}
\end{verbete}

\begin{verbete}{又称}{you4cheng1}{2,10}
  \significado{s.}{também conhecido como}
\end{verbete}

%%%%% EOF %%%%%

