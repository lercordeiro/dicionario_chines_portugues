\begin{entry}{一定}{1,8}{⼀、⼧}
  \begin{phonetics}{一定}{yi2ding4}[][HSK 2]
    \definition{adv.}{certamente | definitivamente}
  \end{phonetics}
\end{entry}
