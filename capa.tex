\begin{titlingpage}
  \raggedleft
  \rule{1pt}{\textheight}
  \hspace{0.1\textwidth}
  \parbox[b]{0.75\textwidth}{
    \vspace{0.05\textheight}
    {\HUGE\bfseries 汉葡词典}\\[2\baselineskip] % Title
    {\Large\textsc{Dicionário Chinês-Português}\\%
     \large\textsc{\zhtoday}}\\% Subtitle
    [4\baselineskip]
    {\Large\textsc{罗学凯}\\%
     \small Luiz Eduardo Roncato Cordeiro\\%
            Aluno do Instituto Confúcio da UNESP}\\% Author
    \vspace{0.5\textheight}\\%
    {Instituto Confúcio, Curso de Chinês}\\[\baselineskip] % Date
  }
  \newpage
  \raggedright
  \setlength{\parindent}{0pt}
  \setlength{\parskip}{\baselineskip}
  \mbox{}
  \vfill
%   \begin{tabular}{@{} l l}
%     \textcopyright{} 2024 & Luiz Eduardo Roncato Cordeiro
%   \end{tabular}
  %\scriptsize
  \footnotesize
  \textcopyright{} 2024 por Luiz Eduardo Roncato Cordeiro está licenciado sob CC BY-NC-SA 4.0\\
  \begin{itemize}
    \item Para visualizar uma cópia desta licença, visite:\\ \url{http://creativecommons.org/licenses/by-nc-sa/4.0/}
    \item Este trabalho ainda está em andamento e o ``código fonte'' está localizado em:\\ \url{https://github.com/lercordeiro/dicionario_chines_portugues}
    \item A última versão compilada também pode ser encontrada em:\\ \url{https://ler.cordeiro.nom.br/}
  \end{itemize}
%   \begin{tabular}{ll}
%   First edition: & T.B.D. \\
%   \end{tabular}
\end{titlingpage}
