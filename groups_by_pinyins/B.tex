%%%
%%% B
%%%

\section*{B}\addcontentsline{toc}{section}{B}

\begin{EntryWithPhonetic}{八}{ba1}{2}{⼋}[HSK 1][Kangxi 12]
  \definition{num.}{oito; 8}
\end{EntryWithPhonetic}

\begin{EntryWithPhonetic}{八八六}{ba1 ba1 liu4}{2,2,4}{⼋、⼋、⼋}
  \definition{expr.}{\emph{Bye bye!}, em salas de bate-papo e mensagens de texto}
\end{EntryWithPhonetic}

\begin{EntryWithPhonetic}{八卦}{ba1gua4}{2,8}{⼋、⼘}[HSK 7-9]
  \definition*{s.}{Oito Trigramas; na China antiga, havia um conjunto de símbolos com significados simbólicos. "—" representa Yang, "--" representa Yin, e oito grupos desses símbolos são chamados de Bagua; cada hexagrama representa uma coisa específica e, mais tarde, foi usado para prever sucesso ou fracasso, sorte ou infortúnio, etc.}
  \definition[个,条]{adj.}{fofoca}
  \definition{adj.}{fofoqueiro}
\end{EntryWithPhonetic}

\begin{EntryWithPhonetic}{巴}{ba1}{4}{⼰}
  \definition*{s.}{Ba, um estado da Dinastia Zhou | Nome antigo de Sichuan Oriental | Sobrenome Ba}
  \definition{s.}{píton (cobra) | crosta; formação semelhante a uma crosta | parte de órgãos biológicos que denotam ponta, extremidade, cauda | bar, unidade física de pressão, 1 atm = 1,01325 bar}
  \definition{v.}{esperar sinceramente; esperar ansiosamente por; aguardar ansiosamente | agarrar-se a; grudar em; manter-se fiel a | escalar física e socialmente | estar perto de; estar ao lado de | abrir; dividir; rachar; quebrar}
\end{EntryWithPhonetic}

\begin{EntryWithPhonetic}{巴不得}{ba1bu5de2}{4,4,11}{⼰、⼀、⼻}[HSK 7-9]
  \definition{v.}{desejar sinceramente; ansiar ansiosamente por; estar muito ansioso (para fazer algo); ansiosamente aguardar; usado na linguagem falada}
\end{EntryWithPhonetic}

\begin{EntryWithPhonetic}{巴勒斯坦}{ba1le4si1tan3}{4,11,12,8}{⼰、⼒、⽄、⼟}
  \definition*{s.}{Palestina}
\end{EntryWithPhonetic}

\begin{EntryWithPhonetic}{巴士}{ba1 shi4}{4,3}{⼰、⼠}[HSK 4]
  \definition[辆]{s.}{ônibus, transliteração da palavra inglesa \emph{bus}}
\end{EntryWithPhonetic}

\begin{EntryWithPhonetic}{巴西}{ba1xi1}{4,6}{⼰、⾑}
  \definition*{s.}{Brasil}
\end{EntryWithPhonetic}

\begin{EntryWithPhonetic}{巴西人}{ba1xi1 ren2}{4,6,2}{⼰、⾑、⼈}
  \definition[个,位]{s.}{brasileiro | pessoa ou povo do Brasil}[他是巴西人。===Ele é brasileiro.]
\end{EntryWithPhonetic}

\begin{EntryWithPhonetic}{巴西战舞}{ba1xi1zhan4wu3}{4,6,9,14}{⼰、⾑、⼽、⾇}
  \definition{s.}{capoeira}
\end{EntryWithPhonetic}

\begin{EntryWithPhonetic}{扒}{ba1}{5}{⼿}[HSK 7-9]
  \definition{v.}{segurar; agarrar-se a | cavar; varrer; puxar para baixo | empurrar para o lado | despir-se; tirar}
  \seeref{pa2}
\end{EntryWithPhonetic}

\begin{EntryWithPhonetic}{吧}{ba1}{7}{⼝}
  \definition{s.}{som de estalo, som crepitante |  abreviação de bar, 酒吧 | cibercafé; um local público que fornece computadores e serviços de \emph{Internet} onde as pessoas podem navegar, jogar, etc.}
  \definition{v.}{fumar; dar uma tragada (puxar) no cachimbo}
  \seeref{ba5}
  \seealsoref{酒吧}{jiu3ba1}
\end{EntryWithPhonetic}

\begin{EntryWithPhonetic}{芭}{ba1}{7}{⾋}
  \definition{s.}{Arcaico: uma planta perfumada | junco; um tipo de erva | flor}
\end{EntryWithPhonetic}

\begin{EntryWithPhonetic}{芭蕾}{ba1lei3}{7,16}{⾋、⾋}[HSK 7-9]
  \definition[场]{s.}{Empréstimo linguístico: balé; um tipo de dança popular na Europa, em que as dançarinas costumam tocar o chão com os dedos dos pés}
  \seealsoref{芭蕾舞}{ba1lei3wu3}
\end{EntryWithPhonetic}

\begin{EntryWithPhonetic}{芭蕾舞}{ba1lei3wu3}{7,16,14}{⾋、⾋、⾇}
  \definition{s.}{Empréstimo linguístico: balé; um tipo de dança clássica europeia; também conhecido como balé clássico europeu}
\end{EntryWithPhonetic}

\begin{EntryWithPhonetic}{拔}{ba2}{8}{⼿}[HSK 5]
  \definition{v.aux.}{puxar para cima; puxar para fora; arrastar para fora | extrair; sugar | escolher; selecionar | superar; destacar-se entre | apreender; capturar | esfriar na água; mergulhar algo em água fria para que esfrie}
\end{EntryWithPhonetic}

\begin{EntryWithPhonetic}{拔尖}{ba2/jian1}{8,6}{⼿、⼩}
  \definition{adj.}{topo de linha | fora do comum | o melhor}
  \definition{v.+compl.}{empurrar-se para a frente | sentir que é superior aos outros}
\end{EntryWithPhonetic}

\begin{EntryWithPhonetic}{把}{ba3}{7}{⼿}[HSK 3]
  \definition{adj.}{referindo-se à relação de irmandade}
  \definition{clas.}{usado antes de objetos com alças ou coisas para segurar | um punhado de; a quantidade que se pode pegar com uma mão | usado antes de coisas abstratas | usado em coisas feitas com as mãos | número de ações, coisas}
  \definition{part.}{adicionado após quantificadores como 百, 千, 万 e 里, 斤, 个, indica que a quantidade é próxima dessa unidade (não pode ser adicionado outro quantificador antes)}
  \definition{prep.}{fazer uma determinada alteração em um objeto; causar uma determinada mudança em um objeto | fazer com que os outros façam/sintam algo}
  \definition{s.}{alça; punho; a parte que se segura | feixe; molho; algo que se segura com as mãos ou se amarra em pequenos feixes}
  \definition{v.}{agarrar; segurar | segurar (um bebê enquanto ele urina) | controlar; dominar; monopolizar | encostar-se; apoiar-se | vigiar (locais importantes); observar; guardar | dar | usar algo como; considerar como; tratar como; conter o significado de 拿 | acorrentar; trancar}
  \seeref{ba4}
  \seealsoref{百}{bai3}
  \seealsoref{个}{ge4}
  \seealsoref{斤}{jin1}
  \seealsoref{里}{li3}
  \seealsoref{拿}{na2}
  \seealsoref{千}{qian1}
  \seealsoref{万}{wan4}
\end{EntryWithPhonetic}

\begin{EntryWithPhonetic}{把柄}{ba3bing3}{7,9}{⼿、⽊}[HSK 7-9]
  \definition[个]{s.}{alça; a parte de um objeto que é fácil de segurar com as mãos | evidências que podem ser obtidas em ações judiciais ou argumentos; uma metáfora para um erro ou falha que pode ser usada para chantagear alguém}
\end{EntryWithPhonetic}

\begin{EntryWithPhonetic}{把持}{ba3chi2}{7,9}{⼿、⼿}
  \definition{v.}{(frequentemente pejorativo) dominar; monopolizar | controlar (os próprios sentimentos, etc.) | manter sob controle}
\end{EntryWithPhonetic}

\begin{EntryWithPhonetic}{把风}{ba3feng1}{7,4}{⼿、⾵}
  \definition{v.}{vigiar (em uma atividade clandestina) | estar atento}
\end{EntryWithPhonetic}

\begin{EntryWithPhonetic}{把关}{ba3/guan1}{7,6}{⼿、⼋}[HSK 7-9]
  \definition{v.+compl.}{verificar rigorosamente; examinar cuidadosamente para ver se algo está sendo feito de acordo com o padrão fixo; fazer a verificação final | proteger uma fronteira, passagem, portões, etc.}
\end{EntryWithPhonetic}

\begin{EntryWithPhonetic}{把脉}{ba3mai4}{7,9}{⼿、⾁}
  \definition{v.}{resolver problemas por meio de investigação e estudo | sentir o pulso | para tomar o pulso de alguém}
\end{EntryWithPhonetic}

\begin{EntryWithPhonetic}{把式}{ba3shi4}{7,6}{⼿、⼷}
  \definition{s.}{pessoa qualificada em um comércio}
\end{EntryWithPhonetic}

\begin{EntryWithPhonetic}{把守}{ba3shou3}{7,6}{⼿、⼧}
  \definition{v.}{vigiar | guardar}
\end{EntryWithPhonetic}

\begin{EntryWithPhonetic}{把手}{ba3shou5}{7,4}{⼿、⼿}[HSK 7-9]
  \definition[个]{s.}{pega; botão; alça; as partes de portas, janelas, móveis, etc. que são fáceis de segurar com as mãos}
\end{EntryWithPhonetic}

\begin{EntryWithPhonetic}{把玩}{ba3wan2}{7,8}{⼿、⽟}
  \definition{v.}{brincar com | mexer com | girar nas mãos}
\end{EntryWithPhonetic}

\begin{EntryWithPhonetic}{把稳}{ba3wen3}{7,14}{⼿、⽲}
  \definition{adj.}{confiável}
  \definition{v.}{ter certeza de; ser firme; manter-se firme}
\end{EntryWithPhonetic}

\begin{EntryWithPhonetic}{把握}{ba3wo4}{7,12}{⼿、⼿}[HSK 3]
  \definition[的]{s.}{seguro; garantia; certeza; confiabilidade do sucesso}
  \definition{v.}{agarrar; segurar; apreender |  (algo abstrato) agarrar; segurar}
\end{EntryWithPhonetic}

\begin{EntryWithPhonetic}{把戏}{ba3xi4}{7,6}{⼿、⼽}
  \definition{s.}{acrobacia | malabarismo | truque barato}
\end{EntryWithPhonetic}

\begin{EntryWithPhonetic}{靶}{ba3}{13}{⾰}
  \definition{s.}{alvo; um alvo para prática de tiro}
\end{EntryWithPhonetic}

\begin{EntryWithPhonetic}{靶子}{ba3zi5}{13,3}{⾰、⼦}[HSK 7-9]
  \definition[个]{s.}{alvo; alvos para prática de tiro ou arco e flecha}
\end{EntryWithPhonetic}

\begin{EntryWithPhonetic}{坝}{ba4}{7}{⼟}[HSK 7-9]
  \definition[座,道,个,条]{s.}{barragem | dique; aterro | Dialeto: banco de areia | (usualmente em nomes de lugares) planície}
\end{EntryWithPhonetic}

\begin{EntryWithPhonetic}{把}{ba4}{7}{⼿}
  \definition{s.}{punho; alça; empunhadura; parte do utensílio que é fácil de segurar com a mão |haste (de uma folha, flor ou fruto) | motivo de ridículo; alvo; comportamentos e declarações que servem de assunto para piadas}
  \seeref{ba3}
\end{EntryWithPhonetic}

\begin{EntryWithPhonetic}{爸}{ba4}{8}{⽗}[HSK 1]
  \definition[个,位]{s.}{(informal) pai}
  \seealsoref{爸爸}{ba4ba5}
\end{EntryWithPhonetic}

\begin{EntryWithPhonetic}{爸爸}{ba4ba5}{8,8}{⽗、⽗}[HSK 1]
  \definition[个,位,名,群]{s.}{(informal) pai; papai; papa}
  \seealsoref{爸}{ba4}
\end{EntryWithPhonetic}

\begin{EntryWithPhonetic}{爸妈}{ba4ma1}{8,6}{⽗、⼥}
  \definition{s.}{pai e mãe}
\end{EntryWithPhonetic}

\begin{EntryWithPhonetic}{罢}{ba4}{10}{⽹}
  \definition{v.}{parar; cessar | revogar; destituir; encerrar | terminar | abandonar uma ideia; esqueçer sobre algo; deixar estar (passar)}
  \seeref{ba5}
\end{EntryWithPhonetic}

\begin{EntryWithPhonetic}{罢工}{ba4gong1}{10,3}{⽹、⼯}[HSK 6]
  \definition{v.}{parar de trabalhar; entrar em greve; abandonar o emprego}
\end{EntryWithPhonetic}

\begin{EntryWithPhonetic}{罢了}{ba4 le5}{10,2}{⽹、⼅}[HSK 6]
  \definition{part.}{usado no final de uma frase, significa 仅此而已, geralmente seguido de 无非, 不过, 只是}
  \seeref{ba4 liao3}
  \seealsoref{不过}{bu2guo4}
  \seealsoref{仅此而已}{jin3ci3'er2yi3}
  \seealsoref{无非}{wu2fei1}
  \seealsoref{只是}{zhi3 shi4}
\end{EntryWithPhonetic}

\begin{EntryWithPhonetic}{罢了}{ba4 liao3}{10,2}{⽹、⼅}
  \definition{part.}{uma partícula modal indicando (não se preocupe, ok)}
  \seeref{ba4 le5}
\end{EntryWithPhonetic}

\begin{EntryWithPhonetic}{罢免}{ba4mian3}{10,7}{⽹、⼉}[HSK 7-9]
  \definition{v.}{destituir; remover do cargo; demitir alguém do seu posto | destituir do cargo uma pessoa eleita pelo eleitorado ou por um órgão representativo}
\end{EntryWithPhonetic}

\begin{EntryWithPhonetic}{罢休}{ba4xiu1}{10,6}{⽹、⼈}[HSK 7-9]
  \definition{v.}{parar; desistir; deixar o assunto de lado; parar de fazer algo, enfatizando a determinação de parar de fazê-lo}
\end{EntryWithPhonetic}

\begin{EntryWithPhonetic}{霸}{ba4}{21}{⾬}
  \definition*{s.}{Sobrenome Ba}
  \definition{adj.}{arrogante; dominador; tirânico}
  \definition{s.}{líder dos senhores feudais; suserano | tirano; déspota; valentão; \emph{bully} | poder hegemônico; hegemonismo; hegemonia | chefe dos príncipes feudais; líder da antiga aliança feudal}
  \definition{v.}{dominar; tiranizar; governar (ocupar) pela força}
\end{EntryWithPhonetic}

\begin{EntryWithPhonetic}{霸权}{ba4quan2}{21,6}{⾬、⽊}
  \definition{s.}{hegemonia; supremacia}
\end{EntryWithPhonetic}

\begin{EntryWithPhonetic}{霸占}{ba4zhan4}{21,5}{⾬、⼘}[HSK 7-9]
  \definition{v.}{ocupar à força; apreender ilegalmente | apreender; usurpar}
\end{EntryWithPhonetic}

\begin{EntryWithPhonetic}{吧}{ba5}{7}{⼝}[HSK 1]
  \definition{part.}{indica discussão, sugestão, solicitação ou comando no final de uma frase | indica concordância ou aprovação no final de uma frase | indica uma pergunta ou especulação no final de uma frase | indica incerteza no final de uma frase | em uma frase, indica uma pausa, carrega um tom hipotético, frequentemente apresenta um contraste e implica um dilema}
  \seeref{ba1}
\end{EntryWithPhonetic}

\begin{EntryWithPhonetic}{罢}{ba5}{10}{⽹}
  \definition{part.}{partícula final, a mesma que 吧}
  \seeref{ba4}
  \seealsoref{吧}{ba5}
\end{EntryWithPhonetic}

\begin{EntryWithPhonetic}{掰}{bai1}{12}{⼿}[HSK 7-9]
  \definition{v.}{separar ou quebrar coisas com as mãos | Dialeto: romper (relacionamento); cortar | Dialeto: analisar; estudar; examinar}
\end{EntryWithPhonetic}

\begin{EntryWithPhonetic}{白}{bai2}{5}{⽩}[HSK 1,3][Kangxi 106]
  \definition*{s.}{Sobrenome Bai}
  \definition{adj.}{branco | claro; entendível; compreendível | puro; claro; simples; sem mistura; em branco | branco (como símbolo de reação) | escrito incorretamente ou pronunciado incorretamente}
  \definition{adv.}{em vão; sem propósito; sem resultados | gratuito; sem custos}
  \definition{s.}{parte falada em ópera, etc.; frases de peças de teatro, etc. | dialeto local | funeral}
  \definition{v.}{explicar; apresentar; esclarecer; declarar | branquear | olhar para as pessoas com o branco dos olhos (olhar vazio, de desaprovação); olhar para alguém com desdém}
\end{EntryWithPhonetic}

\begin{EntryWithPhonetic}{白白}{bai2bai2}{5,5}{⽩、⽩}[HSK 7-9]
  \definition{adv.}{em vão; sem propósito; sem nenhum efeito | sem custo; gratuito; nenhum preço a pagar; pago mas não recebido}
\end{EntryWithPhonetic}

\begin{EntryWithPhonetic}{白菜}{bai2 cai4}{5,11}{⽩、⾋}[HSK 3]
  \definition[棵,种]{s.}{couve chinesa | \emph{pak choi}, um tipo de couve}
\end{EntryWithPhonetic}

\begin{EntryWithPhonetic}{白痴}{bai2chi1}{5,13}{⽩、⽧}
  \definition{adj.}{idiota; uma pessoa que sofre de idiotice; frequentemente usado para menosprezar alguém que é incompetente ou incapaz de fazer as coisas}
  \definition{s.}{idiotice; uma doença caracterizada por retardo mental, demência, fala arrastada, movimentos lentos e até mesmo incapacidade de cuidar de si mesmo}
\end{EntryWithPhonetic}

\begin{EntryWithPhonetic}{白蛋白}{bai2dan4bai2}{5,11,5}{⽩、⾍、⽩}
  \definition{s.}{albumina}
\end{EntryWithPhonetic}

\begin{EntryWithPhonetic}{白鹄}{bai2hu2}{5,12}{⽩、⿃}
  \definition{s.}{cisne branco}
\end{EntryWithPhonetic}

\begin{EntryWithPhonetic}{白拣}{bai2jian3}{5,8}{⽩、⼿}
  \definition{s.}{uma escolha barata}
  \definition{v.}{escolher algo que não custa nada}
\end{EntryWithPhonetic}

\begin{EntryWithPhonetic}{白酒}{bai2 jiu3}{5,10}{⽩、⾣}[HSK 5]
  \definition[瓶,杯,壶]{s.}{aguardente branca; aguardente (geralmente destilada de sorgo ou milho); bebidas destiladas tradicionais chinesas, feitas de sorgo, milho, etc., transparentes e incolores, com alto teor alcoólico}
\end{EntryWithPhonetic}

\begin{EntryWithPhonetic}{白领}{bai2 ling3}{5,11}{⽩、⾴}[HSK 6]
  \definition[个,名,位,些]{s.}{colarinho branco; trabalhador de colarinho branco; refere-se a funcionários cujo trabalho principal envolve trabalho intelectual, são conhecidos por suas roupas elegantes, colarinhos e camisas brancas; atualmente é frequentemente usado para se referir àqueles que trabalham em cargos de gestão ou técnicos em empresas e ganham salários relativamente altos}
\end{EntryWithPhonetic}

\begin{EntryWithPhonetic}{白萝卜}{bai2luo2bo5}{5,11,2}{⽩、⾋、⼘}
  \definition{s.}{rabanete branco | \emph{daikon}}
\end{EntryWithPhonetic}

\begin{EntryWithPhonetic}{白色}{bai2 se4}{5,6}{⽩、⾊}[HSK 2]
  \definition{s.}{a cor branca}
\end{EntryWithPhonetic}

\begin{EntryWithPhonetic}{白天}{bai2 tian1}{5,4}{⽩、⼤}[HSK 1]
  \definition{adv.}{dia;  de dia}
  \definition[个]{s.}{dia; horário diurno; durante o dia}
\end{EntryWithPhonetic}

\begin{EntryWithPhonetic}{白苋}{bai2xian4}{5,7}{⽩、⾋}
  \definition{s.}{amaranto branco | brotos e folhas tenras de espinafre chinês usados como alimento}
\end{EntryWithPhonetic}

\begin{EntryWithPhonetic}{百}{bai3}{6}{⽩}[HSK 1]
  \definition{adj.}{todos; todos os tipos de; multifacetados; numerosos}
  \definition{adv.}{muito; sempre}
  \definition{num.}{cem; 100}
\end{EntryWithPhonetic}

\begin{EntryWithPhonetic}{百般}{bai3ban1}{6,10}{⽩、⾈}
  \definition{adv.}{de todas as maneiras possíveis | por todos os meios}
\end{EntryWithPhonetic}

\begin{EntryWithPhonetic}{百分}{bai3fen1}{6,4}{⽩、⼑}
  \definition{s.}{por cento | nota máxima; pontuação máxima; 100 pontos (em um sistema de classificação de cem pontos) | um jogo específico; um jogo de pôquer}
\end{EntryWithPhonetic}

\begin{EntryWithPhonetic}{百分比}{bai3fen1bi3}{6,4,4}{⽩、⼑、⽐}[HSK 7-9]
  \definition{s.}{porcentagem}[按百分比计算。===Calculado como uma porcentagem.]
\end{EntryWithPhonetic}

\begin{EntryWithPhonetic}{百分点}{bai3 fen1 dian3}{6,4,9}{⽩、⼑、⽕}[HSK 6]
  \definition[个]{s.}{ponto percentual; em estatística, um por cento é chamado de ponto percentual}
\end{EntryWithPhonetic}

\begin{EntryWithPhonetic}{百合}{bai3he2}{6,6}{⽩、⼝}[HSK 7-9]
  \definition[朵,束,株,枝,个,片]{s.}{lírio | bulbo de lírio}
\end{EntryWithPhonetic}

\begin{EntryWithPhonetic}{百货}{bai3 huo4}{6,8}{⽩、⾙}[HSK 4]
  \definition{s.}{mercadorias em geral; loja de departamentos; um termo geral para bens que incluem principalmente roupas, utensílios e necessidades diárias gerais}
\end{EntryWithPhonetic}

\begin{EntryWithPhonetic}{百科全书}{bai3ke1 quan2shu1}{6,9,6,4}{⽩、⽲、⼊、⼄}[HSK 7-9]
  \definition[本,部,套,集]{s.}{enciclopédia; tesauro; \emph{thesaurus}}
\end{EntryWithPhonetic}

\begin{EntryWithPhonetic}{伯}{bai3}{7}{⼈}
  \seeref{bo2}
  \seealsoref{大伯子}{da4 bai3zi5}
\end{EntryWithPhonetic}

\begin{EntryWithPhonetic}{柏}{bai3}{9}{⽊}
  \seeref{bo2}
  \seeref{bo4}
  \seealsoref{柏树}{bai3shu4}
\end{EntryWithPhonetic}

\begin{EntryWithPhonetic}{柏树}{bai3shu4}{9,9}{⽊、⽊}[HSK 7-9]
  \definition[棵,株]{s.}{cipreste}
\end{EntryWithPhonetic}

\begin{EntryWithPhonetic}{摆}{bai3}{13}{⼿}[HSK 4]
  \definition*{s.}{Festival de Ganbai; uma reunião realizada nas áreas Dai durante festivais religiosos, para celebrar uma boa colheita ou para trocar materiais; geralmente se refere a uma reunião em massa | Sobrenome Bai}
  \definition{s.}{pêndulo; dispositivo mecânico que controla a frequência de oscilação em relógios e instrumentos |  a bainha inferior de um vestido, jaqueta ou saia}
  \definition{v.}{colocar; posicionar; organizar | assumir; mostrar intencionalmente | balançar; ondular; balançar para frente e para trás | revelar; listar; afirmar claramente | dizer; falar; declarar | libertar-se; livrar-se}
\end{EntryWithPhonetic}

\begin{EntryWithPhonetic}{摆动}{bai3 dong4}{13,6}{⼿、⼒}[HSK 4]
  \definition{v.}{balançar; balançar para frente e para trás; oscilar; vibrar}
\end{EntryWithPhonetic}

\begin{EntryWithPhonetic}{摆放}{bai3fang4}{13,8}{⼿、⽅}[HSK 7-9]
  \definition{v.}{colocar; posicionar; arranjar; organizar}
\end{EntryWithPhonetic}

\begin{EntryWithPhonetic}{摆烂}{bai3lan4}{13,9}{⼿、⽕}
  \definition{v.}{(neologismo, gíria) parar de lutar (especialmente quando se sabe que não pode ter sucesso) | deixar tudo ir para o inferno}
\end{EntryWithPhonetic}

\begin{EntryWithPhonetic}{摆平}{bai3/ping2}{13,5}{⼿、⼲}[HSK 7-9]
  \definition{v.+compl.}{ser justo com; ser imparcial com; tratar com justiça | Dialeto: punir}
\end{EntryWithPhonetic}

\begin{EntryWithPhonetic}{摆设}{bai3she4}{13,6}{⼿、⾔}[HSK 7-9]
  \definition{v.}{mobiliar e decorar (um cômodo)}
\end{EntryWithPhonetic}

\begin{EntryWithPhonetic}{摆手}{bai3/shou3}{13,4}{⼿、⼿}
  \definition{v.+compl.}{gesticular com a mão (acenando, acenando adeus, etc.) | balançar os braços | acenar com as mãos}
\end{EntryWithPhonetic}

\begin{EntryWithPhonetic}{摆脱}{bai3tuo1}{13,11}{⼿、⾁}[HSK 4]
  \definition{v.}{sacudir; rejeitar; romper com; libertar-se (ou desembaraçar-se) de; livrar-se de dificuldades, escravidão, controle, etc.}
\end{EntryWithPhonetic}

\begin{EntryWithPhonetic}{败}{bai4}{8}{⾒}[HSK 4]
  \definition{adj.}{ruim; deteriorado; murcho; dilapidado; decadente}
  \definition{v.}{ser derrotado; perder (oposto a 胜) | derrotar; bater | falha (oposto a 成) | estragar; arruinar | decair; murchar | quebrar; neutralizar; dissipar}
  \seealsoref{成}{cheng2}
  \seealsoref{胜}{sheng4}
\end{EntryWithPhonetic}

\begin{EntryWithPhonetic}{拜}{bai4}{9}{⼿}
  \definition*{s.}{Sobrenome Bai}
  \definition{adv.}{respeitosamente (usado na comunicação interpessoal);}
  \definition{v.}{fazer uma visita de cortesia | adorar; prestar homenagem | fazer uma chamada cerimonial | ligar; fazer uma visita | intitular alguém com cerimônia; conceder uma posição oficial ou um determinado título com certa etiqueta | estabelecer ou jurar formalmente relacionamentos}
\end{EntryWithPhonetic}

\begin{EntryWithPhonetic}{拜访}{bai4fang3}{9,6}{⼿、⾔}[HSK 5]
  \definition{v.}{visitar; fazer uma visita (respeitosamente)}
\end{EntryWithPhonetic}

\begin{EntryWithPhonetic}{拜会}{bai4hui4}{9,6}{⼿、⼈}[HSK 7-9]
  \definition{v.}{fazer uma visita oficial; fazer uma visita de cortesia; visitar; visitar e conhecer (agora usado principalmente para visitas diplomáticas oficiais)}
\end{EntryWithPhonetic}

\begin{EntryWithPhonetic}{拜见}{bai4jian4}{9,4}{⼿、⾒}[HSK 7-9]
  \definition{v.}{fazer uma visita formal; ligar para prestar homenagens | encontrar-se com alguém superior ou senior}
\end{EntryWithPhonetic}

\begin{EntryWithPhonetic}{拜年}{bai4/nian2}{9,6}{⼿、⼲}[HSK 7-9]
  \definition{v.+compl.}{fazer uma visita de Ano Novo; desejar a alguém um Feliz Ano Novo; fazer uma visita cerimonial no Ano Novo}
\end{EntryWithPhonetic}

\begin{EntryWithPhonetic}{拜托}{bai4tuo1}{9,6}{⼿、⼿}[HSK 7-9]
  \definition{v.}{pedir a alguém para fazer algo; pedir para outra pessoa fazer coisas para você}
\end{EntryWithPhonetic}

\begin{EntryWithPhonetic}{扳}{ban1}{7}{⼿}[HSK 7-9]
  \definition{v.}{puxar; virar | reconquistar; compensar | reconquistar; virar o jogo; virar-se (uma situação de perda)}
  \seeref{pan1}
\end{EntryWithPhonetic}

\begin{EntryWithPhonetic}{班}{ban1}{10}{⽟}[HSK 1]
  \definition*{s.}{Sobrenome Ban}
  \definition{adj.}{regular; programado; executado regularmente; com horários fixos (meios de transporte)}
  \definition{clas.}{um grupo de; uma classe de; usado para pessoas | meios de transporte com horários fixos}
  \definition[个]{s.}{equipe; turma; organização estruturada | dever; turno; período de trabalho dentro de um dia | equipe; esquadrão; unidade básica das forças armadas | nome usado antigamente para designar uma companhia teatral}
  \definition{v.}{mover; implantar; implementar}
\end{EntryWithPhonetic}

\begin{EntryWithPhonetic}{班级}{ban1 ji2}{10,6}{⽟、⽷}[HSK 3]
  \definition[个]{s.}{classe; série (na escola); o nome geral para as séries e turmas da escola}
\end{EntryWithPhonetic}

\begin{EntryWithPhonetic}{班长}{ban1 zhang3}{10,4}{⽟、⾧}[HSK 2]
  \definition[个,位,名]{s.}{monitor de turma; líder de equipe; alunos responsáveis nas turmas da escola | líder de esquadrão; responsável por uma turma de soldados, geralmente com patente de sargento}
\end{EntryWithPhonetic}

\begin{EntryWithPhonetic}{般}{ban1}{10}{⾈}
  \definition{clas.}{tipo; classe; gênero; amostra}
  \definition{part.}{(o mesmo) que; como; semelhante}
  \seeref{bo1}
  \seeref{pan2}
\end{EntryWithPhonetic}

\begin{EntryWithPhonetic}{颁}{ban1}{10}{⾴}
  \definition{v.}{promulgar; emitir; enviar | conceder ou conferir}
\end{EntryWithPhonetic}

\begin{EntryWithPhonetic}{颁布}{ban1bu4}{10,5}{⾴、⼱}[HSK 7-9]
  \definition{v.}{promulgar; emitir; publicar; anunciar (leis, regulamentos, etc.), com um escopo de uso mais restrito do que 公布}
  \seealsoref{公布}{gong1bu4}
\end{EntryWithPhonetic}

\begin{EntryWithPhonetic}{颁发}{ban1fa1}{10,5}{⾴、⼜}[HSK 7-9]
  \definition{v.}{promulgar; emitir (comandos, instruções, regulamentos, etc.) a um superior | premiar (prêmio, medalha, certificado, etc.)}
\end{EntryWithPhonetic}

\begin{EntryWithPhonetic}{颁奖}{ban1/jiang3}{10,9}{⾴、⼤}[HSK 7-9]
  \definition{v.+compl.}{conceder prêmios, bônus, certificados, etc.; distribuir prêmios, bônus, certificados, etc.}
\end{EntryWithPhonetic}

\begin{EntryWithPhonetic}{斑}{ban1}{12}{⽂}
  \definition{adj.}{manchado; listrado; de cor variegada}
  \definition[块,片,个]{s.}{mancha; pinta; salpico; listra | nódoa; estria; mácula; imperfeição}
\end{EntryWithPhonetic}

\begin{EntryWithPhonetic}{斑点}{ban1dian3}{12,9}{⽂、⽕}[HSK 7-9]
  \definition[块,片,个]{s.}{salpico; cisco; ponto; pinta; sardas | mancha; marca}
\end{EntryWithPhonetic}

\begin{EntryWithPhonetic}{搬}{ban1}{13}{⼿}[HSK 3]
  \definition{v.}{tirar; mover; remover | mudar-se (de casa) | aplicar indiscriminadamente; copiar mecanicamente}
\end{EntryWithPhonetic}

\begin{EntryWithPhonetic}{搬动}{ban1dong4}{13,6}{⼿、⼒}
  \definition{v.}{mover (algo ao redor) | mudar de casa}
\end{EntryWithPhonetic}

\begin{EntryWithPhonetic}{搬家}{ban1/jia1}{13,10}{⼿、⼧}[HSK 3]
  \definition{v.+compl.}{mudar de casa; mudar-se para outro lugar}
\end{EntryWithPhonetic}

\begin{EntryWithPhonetic}{搬口}{ban1kou3}{13,3}{⼿、⼝}
  \definition{v.}{tagarelar | (idioma) transmitir histórias;  semear dissensão | contar histórias}
\end{EntryWithPhonetic}

\begin{EntryWithPhonetic}{搬弄}{ban1nong4}{13,7}{⼿、⼶}
  \definition{v.}{causar problemas | mexer com alguém | mostrar (o que se pode fazer)}
\end{EntryWithPhonetic}

\begin{EntryWithPhonetic}{搬迁}{ban1qian1}{13,6}{⼿、⾡}[HSK 7-9]
  \definition{v.}{mover; transferir; realocar}
\end{EntryWithPhonetic}

\begin{EntryWithPhonetic}{搬运}{ban1yun4}{13,7}{⼿、⾡}
  \definition{v.}{carregar; transportar}
\end{EntryWithPhonetic}

\begin{EntryWithPhonetic}{搬走}{ban1zou3}{13,7}{⼿、⾛}
  \definition{v.}{carregar}
\end{EntryWithPhonetic}

\begin{EntryWithPhonetic}{板}{ban3}{8}{⽊}[HSK 3]
  \definition{adj.}{rígido; não natural; inflexível}
  \definition[块,个]{s.}{tábua; placa; prato; objeto rígido em forma de placa | veneziana; persiana; refere-se especificamente aos painéis de portas de lojas | badalos (instrumento musical que marca o ritmo) | uma batida acentuada (ritmo) na música e na ópera tradicional | chefe}
  \definition{v.}{parecer sério | corrigir maus hábitos ou defeitos | ser rígido como uma tábua}
\end{EntryWithPhonetic}

\begin{EntryWithPhonetic}{板块}{ban3kuai4}{8,7}{⽊、⼟}[HSK 7-9]
  \definition[个]{s.}{placa tectônica; segmentos móveis da crosta terrestre | seção; uma metáfora para uma combinação de partes que têm algo em comum ou conectado}
\end{EntryWithPhonetic}

\begin{EntryWithPhonetic}{版}{ban3}{8}{⽚}[HSK 5]
  \definition{clas.}{usado como uma palavra de medida para materiais impressos (por exemplo, livros, jornais, edições)}
  \definition{s.}{chapa, placa ou bloco de impressão | edição (livros impressos) | página (de um jornal) | moldes ou fromas de construção}
\end{EntryWithPhonetic}

\begin{EntryWithPhonetic}{办}{ban4}{4}{⼒}[HSK 2]
  \definition{v.}{fazer; lidar com; gerenciar; cuidar de | executar; configurar | preparar algo; comprar uma quantidade razoável de | punir; levar à justiça; punir com medidas}
\end{EntryWithPhonetic}

\begin{EntryWithPhonetic}{办不到}{ban4 bu5 dao4}{4,4,8}{⼒、⼀、⼑}[HSK 7-9]
  \definition{v.}{ser incapaz de fazer algo; ser incapaz de realizar; não poder ser feito; não poder fazer}
\end{EntryWithPhonetic}

\begin{EntryWithPhonetic}{办法}{ban4fa3}{4,8}{⼒、⽔}[HSK 2]
  \definition[个,种]{s.}{método; meio; medida; caminho; maneira; método de lidar com situações ou resolver problemas}
\end{EntryWithPhonetic}

\begin{EntryWithPhonetic}{办公}{ban4 gong1}{4,4}{⼒、⼋}[HSK 6]
  \definition{v.}{trabalhar; fazer trabalho de escritório; lidar com negócios oficiais; tratar de assuntos oficiais}
\end{EntryWithPhonetic}

\begin{EntryWithPhonetic}{办公室}{ban4gong1shi4}{4,4,9}{⼒、⼋、⼧}[HSK 2]
  \definition[个,间]{s.}{órgãos, escolas, grupos, empresas e outras entidades que lidam com assuntos administrativos cotidianos | escritório; sala de escritório}
\end{EntryWithPhonetic}

\begin{EntryWithPhonetic}{办理}{ban4li3}{4,11}{⼒、⽟}[HSK 3]
  \definition{v.}{conduzir; lidar; transacionar; negociar; solicitar um documento ou realizar um procedimento específico}
\end{EntryWithPhonetic}

\begin{EntryWithPhonetic}{办事}{ban4 shi4}{4,8}{⼒、⼅}[HSK 4]
  \definition{v.}{trabalhar | lidar com assuntos; manipular transações}
\end{EntryWithPhonetic}

\begin{EntryWithPhonetic}{办事处}{ban4 shi4 chu4}{4,8,5}{⼒、⼅、⼡}[HSK 6]
  \definition[个,家]{s.}{escritório; agência; agências enviadas pelo governo, militares, grupos, etc.}
\end{EntryWithPhonetic}

\begin{EntryWithPhonetic}{办学}{ban4/xue2}{4,8}{⼒、⼦}[HSK 6]
  \definition{v.+compl.}{administrar uma escola}
\end{EntryWithPhonetic}

\begin{EntryWithPhonetic}{半}{ban4}{5}{⼗}[HSK 1]
  \definition{adv.}{parcialmente; usado antes de verbos ou adjetivos para indicar incompletude}
  \definition{num.}{(depois de um número) ``e meio'' | meio; metade | na metade; no meio | muito pouco; o mínimo}
\end{EntryWithPhonetic}

\begin{EntryWithPhonetic}{半边天}{ban4bian1tian1}{5,5,4}{⼗、⾡、⼤}[HSK 7-9]
  \definition{s.}{metade do céu; parte do céu | mulheres modernas; mulheres (do ditado de Mao Zedong ``As mulheres podem sustentar metade do céu.'')}
\end{EntryWithPhonetic}

\begin{EntryWithPhonetic}{半场}{ban4chang3}{5,6}{⼗、⼟}[HSK 7-9]
  \definition{s.}{metade de um jogo ou competição (tempo) | meia quadra (no basquete)}
\end{EntryWithPhonetic}

\begin{EntryWithPhonetic}{半岛}{ban4dao3}{5,7}{⼗、⼭}[HSK 7-9]
  \definition[个]{s.}{península; terra que se estende até o mar ou lago, cercada por água em três lados e conectada à terra em um lado}
\end{EntryWithPhonetic}

\begin{EntryWithPhonetic}{半决赛}{ban4 jue2 sai4}{5,6,14}{⼗、⼎、⾙}[HSK 6]
  \definition{s.}{semifinais}
\end{EntryWithPhonetic}

\begin{EntryWithPhonetic}{半路}{ban4lu4}{5,13}{⼗、⾜}[HSK 7-9]
  \definition{adv.}{a caminho | em andamento}
  \definition{s.}{na metade do caminho; no meio do caminho}
\end{EntryWithPhonetic}

\begin{EntryWithPhonetic}{半年}{ban4 nian2}{5,6}{⼗、⼲}[HSK 1]
  \definition{s.}{meio ano}
\end{EntryWithPhonetic}

\begin{EntryWithPhonetic}{半球}{ban4qiu2}{5,11}{⼗、⽟}
  \definition{s.}{hemisfério}
\end{EntryWithPhonetic}

\begin{EntryWithPhonetic}{半数}{ban4shu4}{5,13}{⼗、⽁}[HSK 7-9]
  \definition{s.}{metade do total; metade}
\end{EntryWithPhonetic}

\begin{EntryWithPhonetic}{半天}{ban4 tian1}{5,4}{⼗、⼤}[HSK 1]
  \definition{s.}{metade do dia; metade do dia dividida pelo meio-dia | um longo tempo; bastante tempo; refere-se a um período de tempo relativamente longo (com um tom exagerado)}
\end{EntryWithPhonetic}

\begin{EntryWithPhonetic}{半途而废}{ban4tu2'er2fei4}{5,10,6,8}{⼗、⾡、⽽、⼴}[HSK 7-9]
  \definition{expr.}{desistir no meio do caminho; deixar inacabado; fazer algo pela metade; parar no meio do caminho; metaforicamente, parar antes de concluir uma tarefa; não terminar o que foi iniciado}
\end{EntryWithPhonetic}

\begin{EntryWithPhonetic}{半信半疑}{ban4xin4-ban4yi2}{5,9,5,14}{⼗、⼈、⼗、⽦}[HSK 7-9]
  \definition{expr.}{meio acreditar e meio duvidar; ser bastante duvidoso sobre (acerca de) uma coisa; estar incerto quanto ao que acreditar; meio seriamente e meio cético; não totalmente convencido; bastante desconfiado | meio acreditando, meio duvidando}
\end{EntryWithPhonetic}

\begin{EntryWithPhonetic}{半夜}{ban4 ye4}{5,8}{⼗、⼣}[HSK 2]
  \definition{s.}{no meio da noite; metade da noite | por volta da meia-noite, também se refere à madrugada}
\end{EntryWithPhonetic}

\begin{EntryWithPhonetic}{半音}{ban4yin1}{5,9}{⼗、⾳}
  \definition{s.}{semitom; na música, uma oitava é dividida em doze notas e o intervalo entre duas notas adjacentes é chamado de semitom}
\end{EntryWithPhonetic}

\begin{EntryWithPhonetic}{半真半假}{ban4zhen1-ban4jia3}{5,10,5,11}{⼗、⼗、⼗、⼈}[HSK 7-9]
  \definition{expr.}{meio verdadeiro e meio falso | meio genuíno, meio falso; parcialmente verdadeiro, parcialmente falso | meio de brincadeira, meio a sério; meio brincando}
\end{EntryWithPhonetic}

\begin{EntryWithPhonetic}{伴}{ban4}{7}{⼈}[HSK 7-9]
  \definition[个,位]{s.}{companheiro; parceiro}
  \definition{v.}{acompanhar; estar perto}[伴君如伴虎。===Acompanhar o rei é como acompanhar um tigre.]
\end{EntryWithPhonetic}

\begin{EntryWithPhonetic}{伴侣}{ban4lv3}{7,8}{⼈、⼈}[HSK 7-9]
  \definition[个,对]{s.}{companheiro; parceiro | parceiro; companheiro; refere-se a um casal ou a um dos casais}
\end{EntryWithPhonetic}

\begin{EntryWithPhonetic}{伴随}{ban4sui2}{7,11}{⼈、⾩}[HSK 7-9]
  \definition{v.}{seguir; acompanhar}
\end{EntryWithPhonetic}

\begin{EntryWithPhonetic}{伴奏}{ban4zou4}{7,9}{⼈、⼤}[HSK 7-9]
  \definition{v.}{acompanhar (com instrumentos musicais); tocar (um instrumento musical) em conjunto com canto, dança ou apresentação solo, etc.}
\end{EntryWithPhonetic}

\begin{EntryWithPhonetic}{扮}{ban4}{7}{⼿}[HSK 7-9]
  \definition{v.}{vestir-se como; desempenhar o papel de | maquiar-se; disfarçar-se como | (expressão facial) fazer cara de}
\end{EntryWithPhonetic}

\begin{EntryWithPhonetic}{扮演}{ban4yan3}{7,14}{⼿、⽔}[HSK 5]
  \definition{v.}{desempenhar o papel de; ter um papel (em uma peça, etc.); atuar}
\end{EntryWithPhonetic}

\begin{EntryWithPhonetic}{拌}{ban4}{8}{⼿}[HSK 7-9]
  \definition{v.}{misturar | mexer e misturar | discutir; brigar; ter uma discussão}
\end{EntryWithPhonetic}

\begin{EntryWithPhonetic}{帮}{bang1}{9}{⼱}[HSK 1]
  \definition*{s.}{Sobrenome Bang}
  \definition{clas.}{um grupo de; um bando de; uma gangue de; um grupo de pessoas}
  \definition{s.}{lateral; superior; partes ao lado ou ao redor do objeto | folha externa; parte mais grossa das folhas externas dos vegetais | gangue; banda; grupo; conglomerado}
  \definition{v.}{ajudar; assistir; auxiliar | trabalho; refere-se ao envolvimento em trabalho assalariado}
\end{EntryWithPhonetic}

\begin{EntryWithPhonetic}{帮教}{bang1jiao4}{9,11}{⼱、⽁}
  \definition{v.}{orientar}
\end{EntryWithPhonetic}

\begin{EntryWithPhonetic}{帮忙}{bang1/mang2}{9,6}{⼱、⼼}[HSK 1]
  \definition{v.+compl.}{ajudar; dar uma mão; dar uma mãozinha; fazer um favor; fazer uma boa ação; ajudar os outros a fazer algo, referindo-se, de maneira geral, a oferecer ajuda quando alguém está com dificuldades}
\end{EntryWithPhonetic}

\begin{EntryWithPhonetic}{帮佣}{bang1yong1}{9,7}{⼱、⼈}
  \definition{s.}{trabalhador doméstico; empregada doméstica; servo; servente}
  \definition{v.}{trabalhar ou ser contratado como trabalhador doméstico, servo, etc.}
\end{EntryWithPhonetic}

\begin{EntryWithPhonetic}{帮助}{bang1zhu4}{9,7}{⼱、⼒}[HSK 2]
  \definition[个,次,回,份,种]{s.}{ajuda; auxílio; socorro; função de promoção ou auxílio}
  \definition{v.}{ajudar; assistir; apoiar; quando alguém está passando por dificuldades, oferecer apoio financeiro ou material, ou ainda apoio moral, dar conselhos, pensar em soluções, fazer coisas por essa pessoa, etc.}
\end{EntryWithPhonetic}

\begin{EntryWithPhonetic}{绑}{bang3}{9}{⽷}[HSK 7-9]
  \definition{v.}{amarrar; atar | enrolar ou amarrar com corda}
\end{EntryWithPhonetic}

\begin{EntryWithPhonetic}{绑架}{bang3jia4}{9,9}{⽷、⽊}[HSK 7-9]
  \definition{v.}{sequestrar; abduzir | amarrar; atar}
\end{EntryWithPhonetic}

\begin{EntryWithPhonetic}{榜}{bang3}{14}{⽊}
  \definition[块]{s.}{lista publicada de nomes | Literário: placa horizontal inscrita | aviso; anúncio; proclamação antiga}
\end{EntryWithPhonetic}

\begin{EntryWithPhonetic}{榜首}{bang3shou3}{14,9}{⽊、⾸}[HSK 7-9]
  \definition{s.}{cabeça da lista de candidatos aprovados; primeiro lugar em um concurso, etc. | topo da lista}
\end{EntryWithPhonetic}

\begin{EntryWithPhonetic}{榜样}{bang3yang4}{14,10}{⽊、⽊}[HSK 7-9]
  \definition[个,位]{s.}{exemplo; modelo; padrão; pessoas ou coisas boas que valem a pena aprender, usado principalmente na linguagem falada}
\end{EntryWithPhonetic}

\begin{EntryWithPhonetic}{傍}{bang4}{12}{⼈}
  \definition*{s.}{Sobrenome Bang}
  \definition{v.}{estar perto de (à distância); aproximar-se | estar perto de (no tempo) | depender de; confiar em}
\end{EntryWithPhonetic}

\begin{EntryWithPhonetic}{傍晚}{bang4wan3}{12,11}{⼈、⽇}[HSK 6]
  \definition[个]{s.}{ao entardecer; ao cair da noite; (tarde) refere-se ao momento em que se aproxima o anoitecer, frequentemente usado na linguagem escrita}
\end{EntryWithPhonetic}

\begin{EntryWithPhonetic}{棒}{bang4}{12}{⽊}[HSK 5]
  \definition{adj.}{bom; forte; excelente}
  \definition[根]{s.}{porrete; bastão; cajado; clava}
\end{EntryWithPhonetic}

\begin{EntryWithPhonetic}{棒棒糖}{bang4bang4tang2}{12,12,16}{⽊、⽊、⽶}
  \definition[根]{s.}{pirulito}
\end{EntryWithPhonetic}

\begin{EntryWithPhonetic}{棒冰}{bang4bing1}{12,6}{⽊、⼎}
  \definition{s.}{picolé}
\end{EntryWithPhonetic}

\begin{EntryWithPhonetic}{棒球}{bang4qiu2}{12,11}{⽊、⽟}[HSK 7-9]
  \definition[个,只]{s.}{beisebol}
\end{EntryWithPhonetic}

\begin{EntryWithPhonetic}{磅}{bang4}{15}{⽯}[HSK 7-9]
  \definition{clas.}{libra | Tipografia: pt, ponto (tamanho de letra, por exemplo: 10pt)}
  \definition{s.}{escalas}
  \definition{v.}{pesar com uma balança}
  \seeref{pang2}
\end{EntryWithPhonetic}

\begin{EntryWithPhonetic}{包}{bao1}{5}{⼓}[HSK 1]
  \definition*{s.}{Sobrenome Bao}
  \definition{clas.}{pacote; embalagem; embrulho; usado para coisas empacotadas}
  \definition[个,只]{s.}{feixe; pacote; encomenda; algo embrulhado | saco; sacola; saco para guardar coisas | caroço; inchaço; protuberância; inchaço ou protuberância no corpo ou em objetos | tenda; tenda com cúpula feita de feltro}
  \definition{v.}{embrulhar; envolver com papel, tecido, etc. | cercar; rodear; envolver; envelopar | incluir; conter | realizar todo o processo; assumir toda a responsabilidade | assegurar; garantir | contratar; reservar; fretar; comprar ou alugar tudo; acordar uso exclusivo}
\end{EntryWithPhonetic}

\begin{EntryWithPhonetic}{包办}{bao1ban4}{5,4}{⼓、⼒}
  \definition{v.}{cuidar de tudo que diz respeito a um trabalho | comandar todo o espetáculo; monopolizar tudo | assumir tudo; manter tudo em suas próprias mãos}
\end{EntryWithPhonetic}

\begin{EntryWithPhonetic}{包袱}{bao1fu5}{5,11}{⼓、⾐}[HSK 7-9]
  \definition[个,堆,身]{s.}{um pacote embrulhado em pano; uma bolsa embrulhada em pano contendo roupas e outras necessidades diárias | carga; peso; fardo; uma metáfora para um fardo que afeta o pensamento ou a ação}
\end{EntryWithPhonetic}

\begin{EntryWithPhonetic}{包干}{bao1gan1}{5,3}{⼓、⼲}
  \definition{s.}{tarefa alocada}
  \definition{v.}{ter a responsabilidade total sobre um trabalho}
\end{EntryWithPhonetic}

\begin{EntryWithPhonetic}{包裹}{bao1guo3}{5,14}{⼓、⾐}[HSK 4]
  \definition[个,件]{s.}{pacote; embrulho}
  \definition{v.}{embrulhar; amarrar; enrolar coisas em pano ou outra coisa}
\end{EntryWithPhonetic}

\begin{EntryWithPhonetic}{包含}{bao1han2}{5,7}{⼓、⼝}[HSK 4]
  \definition{v.}{conter; implicar; incluir; conter dentro, resumir, enfatizar o que está contido dentro, focar em relações internas, muitas vezes coisas abstratas}
\end{EntryWithPhonetic}

\begin{EntryWithPhonetic}{包括}{bao1kuo4}{5,9}{⼓、⼿}[HSK 4]
  \definition{v.}{incluir; compreender; consistir em; conter, conter dentro, resumir junto, enfatizar a listagem de todas as partes, ou a citação de uma parte delas, que podem ser coisas abstratas ou concretas}
\end{EntryWithPhonetic}

\begin{EntryWithPhonetic}{包容}{bao1rong2}{5,10}{⼓、⼧}[HSK 7-9]
  \definition{adj.}{inclusivo}
  \definition{v.}{perdoar; mostrar tolerância; fácil de aceitar ideias diferentes ou ser capaz de perdoar o comportamento ou linguagem hostil dos outros | segurar; conter; quantas pessoas ou coisas podem ser colocadas em um determinado espaço}
\end{EntryWithPhonetic}

\begin{EntryWithPhonetic}{包围}{bao1wei2}{5,7}{⼓、⼞}[HSK 5]
  \definition{v.}{circundar; cercar; rodear}
\end{EntryWithPhonetic}

\begin{EntryWithPhonetic}{包扎}{bao1za1}{5,4}{⼓、⼿}[HSK 7-9]
  \definition{v.}{empacotar; amarrar; embrulhar}
\end{EntryWithPhonetic}

\begin{EntryWithPhonetic}{包装}{bao1zhuang1}{5,12}{⼓、⾐}[HSK 5]
  \definition[个,款]{s.}{embalagem; materiais usados para embalar produtos, como papel, sacolas, garrafas ou caixas}
  \definition{v.}{embalar; embrulhar; empacotar | aumentar a fama e o apelo de alguém ou algo por meio de publicidade | tornar alguém ou algo mais comercialmente viável ou atraente por meio de embelezamento ou publicidade}
\end{EntryWithPhonetic}

\begin{EntryWithPhonetic}{包子}{bao1 zi5}{5,3}{⼓、⼦}[HSK 1]
  \definition[个]{s.}{pão recheado cozido no vapor; alimentos, com recheio de vegetais, carne ou açúcar, etc., com massa levedada como invólucro, embrulhados e cozidos no vapor}
\end{EntryWithPhonetic}

\begin{EntryWithPhonetic}{包租}{bao1zu1}{5,10}{⼓、⽲}
  \definition{s.}{aluguel fixo para terras agrícolas}
  \definition{v.}{fretar | alugar | alugar um terreno ou uma casa para subarrendar}
\end{EntryWithPhonetic}

\begin{EntryWithPhonetic}{炮}{bao1}{9}{⽕}
  \definition{v.}{processar; o método de preparação da medicina chinesa é colocar as ervas cruas em uma panela de ferro em alta temperatura e fritá-las até que fiquem marrons e estourem | secar alimentos pelo calor; refogar}
  \seeref{pao2}
  \seeref{pao4}
\end{EntryWithPhonetic}

\begin{EntryWithPhonetic}{剥}{bao1}{10}{⼑}[HSK 7-9]
  \definition{v.}{descascar; despelar; remover a casca ou pele externa}
  \seeref{bo1}
\end{EntryWithPhonetic}

\begin{EntryWithPhonetic}{煲}{bao1}{13}{⽕}[HSK 7-9]
  \definition{s.}{panela; caldeira}
  \definition{v.}{cozinhar; ensopar; ferver}
\end{EntryWithPhonetic}

\begin{EntryWithPhonetic}{薄}{bao2}{16}{⾋}[HSK 4]
  \definition{adj.}{fino; frágil | frio; indiferente; carente de calor | leve; fraco | pobre; infértil}
  \seeref{bo2}
  \seeref{bo4}
\end{EntryWithPhonetic}

\begin{EntryWithPhonetic}{宝}{bao3}{8}{⼧}[HSK 4]
  \definition*{s.}{Sobrenome Bao}
  \definition{adj.}{antigo; precioso; estimado}
  \definition{pron.}{estimado; um termo educado usado para se referir à família, loja, etc. de alguém}
  \definition[个,件]{s.}{tesouro; objeto estimado; coisa preciosa | dinheiro; moeda; moeda antiga com furo quadrado no centro; moeda de prata}
\end{EntryWithPhonetic}

\begin{EntryWithPhonetic}{宝宝}{bao3 bao5}{8,8}{⼧、⼧}[HSK 4]
  \definition[个,位]{s.}{querida; \emph{darling}; \emph{baby}; apelido para crianças}
\end{EntryWithPhonetic}

\begin{EntryWithPhonetic}{宝贝}{bao3bei4}{8,4}{⼧、⾙}[HSK 4]
  \definition{adj.}{excêntrico; estranho; imprestável; um termo depreciativo para uma pessoa incompetente ou ridícula}
  \definition[个,件]{s.}{tesouro; objeto estimado; coisa preciosa | querida; \emph{darling}; \emph{baby}; apelido para crianças}
\end{EntryWithPhonetic}

\begin{EntryWithPhonetic}{宝贵}{bao3gui4}{8,9}{⼧、⾙}[HSK 4]
  \definition{adj.}{precioso; extremamente valioso, muito raro, pode ser usado para descrever coisas específicas, também pode ser usado para descrever coisas abstratas | valioso; como um tesouro}
\end{EntryWithPhonetic}

\begin{EntryWithPhonetic}{宝库}{bao3ku4}{8,7}{⼧、⼴}[HSK 7-9]
  \definition[座,个]{s.}{tesouro; casa de tesouro; um lugar onde coisas preciosas são armazenadas (frequentemente usado metaforicamente)}[图书馆是知识的宝库。===A biblioteca é um tesouro de conhecimento.]
\end{EntryWithPhonetic}

\begin{EntryWithPhonetic}{宝石}{bao3 shi2}{8,5}{⼧、⽯}[HSK 4]
  \definition[颗,枚,块,粒]{s.}{gema; jóia; pedra preciosa; mineral precioso que tem um brilho lindo e uma dureza de mais de sete graus, não é afetado pela atmosfera ou por produtos químicos e pode ser usado como decoração, suporte de instrumentos ou abrasivos}
\end{EntryWithPhonetic}

\begin{EntryWithPhonetic}{宝藏}{bao3zang4}{8,17}{⼧、⾋}[HSK 7-9]
  \definition[座,个]{s.}{depósitos preciosos (minerais); tesouros ou riquezas armazenadas, principalmente minerais}
\end{EntryWithPhonetic}

\begin{EntryWithPhonetic}{饱}{bao3}{8}{⾷}[HSK 2]
  \definition{adj.}{cheio; comer até ficar satisfeito | cheio; rechonchudo}
  \definition{adv.}{totalmente; completamente; plenamente}
  \definition{v.}{satisfazer}
\end{EntryWithPhonetic}

\begin{EntryWithPhonetic}{饱和}{bao3he2}{8,8}{⾷、⼝}[HSK 7-9]
  \definition{v.}{estar saturado; a uma certa temperatura ou pressão, a quantidade de soluto contida na solução atinge seu limite máximo e não consegue mais se dissolver | estar saturado; metaforicamente, a quantidade de algo atinge um máximo dentro de um certo intervalo}
\end{EntryWithPhonetic}

\begin{EntryWithPhonetic}{饱满}{bao3man3}{8,13}{⾷、⽔}[HSK 7-9]
  \definition{adj.}{cheio; rechonchudo; bem empilhado; preenchido | robusto; abundante; pleno; vigoroso}
\end{EntryWithPhonetic}

\begin{EntryWithPhonetic}{保}{bao3}{9}{⼈}[HSK 3]
  \definition*{s.}{Sobrenome Bao}
  \definition{s.}{fiador; babá ou responsável pela guarda de crianças | oficial responsável; sistema administrativo; unidade administrativa do antigo registro civil}
  \definition{v.}{defender; proteger | manter; preservar; conservar em boas condições | assegurar; garantir | ser fiador de alguém}
\end{EntryWithPhonetic}

\begin{EntryWithPhonetic}{保安}{bao3 an1}{9,6}{⼈、⼧}[HSK 3]
  \definition[个,位,名]{s.}{guarda de segurança; segurança}
  \definition{v.}{proteger; manter em segurança; defender a segurança social | garantir a segurança; proteger a segurança dos trabalhadores e prevenir acidentes durante o processo de produção}
\end{EntryWithPhonetic}

\begin{EntryWithPhonetic}{保持}{bao3chi2}{9,9}{⼈、⼿}[HSK 3]
  \definition{v.}{manter; conservar; reter; preservar; manter um determinado estado, para que não desapareça ou não se altere}
\end{EntryWithPhonetic}

\begin{EntryWithPhonetic}{保存}{bao3cun2}{9,6}{⼈、⼦}[HSK 3]
  \definition{v.}{salvar; preservar; conservar; manter a existência com ênfase em que as coisas, as propriedades, os significados, os estilos, etc. não sofram perdas ou mudanças | (computação) salvar (um arquivo, etc.)}
\end{EntryWithPhonetic}

\begin{EntryWithPhonetic}{保管}{bao3guan3}{9,14}{⼈、⽵}[HSK 7-9]
  \definition{adv.}{certamente; expressa confiança}
  \definition{s.}{gerente; almoxarife; lojista; pessoas que realizam trabalho de custódia}
  \definition{v.}{cuidar de; ser responsável por}
\end{EntryWithPhonetic}

\begin{EntryWithPhonetic}{保护}{bao3hu4}{9,7}{⼈、⼿}[HSK 3]
  \definition{v.}{proteger, guardar, cuidar; salvaguardar; cuidar ao máximo, para que não seja danificado, referindo-se principalmente a coisas concretas}
\end{EntryWithPhonetic}

\begin{EntryWithPhonetic}{保护国}{bao3hu4guo2}{9,7,8}{⼈、⼿、⼞}
  \definition{s.}{protetorado}
\end{EntryWithPhonetic}

\begin{EntryWithPhonetic}{保护剂}{bao3hu4ji4}{9,7,8}{⼈、⼿、⼑}
  \definition{s.}{agente protetor; protetor}
\end{EntryWithPhonetic}

\begin{EntryWithPhonetic}{保护区}{bao3hu4qu1}{9,7,4}{⼈、⼿、⼖}
  \definition[个,片]{s.}{zona de proteção | área de preservação; reserva natural}
\end{EntryWithPhonetic}

\begin{EntryWithPhonetic}{保护色}{bao3hu4se4}{9,7,6}{⼈、⼿、⾊}
  \definition{s.}{camuflagem | coloração protetora}
\end{EntryWithPhonetic}

\begin{EntryWithPhonetic}{保护神}{bao3hu4shen2}{9,7,9}{⼈、⼿、⽰}
  \definition{s.}{anjo da guarda | santo patrono}
\end{EntryWithPhonetic}

\begin{EntryWithPhonetic}{保护物}{bao3hu4 wu4}{9,7,8}{⼈、⼿、⽜}
  \definition{s.}{protetor}
\end{EntryWithPhonetic}

\begin{EntryWithPhonetic}{保护性}{bao3hu4xing4}{9,7,8}{⼈、⼿、⼼}
  \definition{s.}{proteção; protetor}
\end{EntryWithPhonetic}

\begin{EntryWithPhonetic}{保护者}{bao3hu4zhe3}{9,7,8}{⼈、⼿、⽼}
  \definition{s.}{protetor | segurador}
\end{EntryWithPhonetic}

\begin{EntryWithPhonetic}{保护主义}{bao3hu4zhu3yi4}{9,7,5,3}{⼈、⼿、⼂、⼂}
  \definition{s.}{protecionismo}
\end{EntryWithPhonetic}

\begin{EntryWithPhonetic}{保健}{bao3 jian4}{9,10}{⼈、⼈}[HSK 6]
  \definition{s.}{cuidados de saúde; proteção da saúde}
  \definition{v.}{cuidar da sua saúde; proteger sua saúde}
\end{EntryWithPhonetic}

\begin{EntryWithPhonetic}{保留}{bao3liu2}{9,10}{⼈、⽥}[HSK 3]
  \definition{v.}{manter; continuar a ter; manter o estado original inalterado | conter; reter; deixar ficar; não tirar | reservar; colocar os direitos, opiniões, etc. de lado, não exercê-los ou expressá-los por enquanto}
\end{EntryWithPhonetic}

\begin{EntryWithPhonetic}{保密}{bao3mi4}{9,11}{⼈、⼧}[HSK 4]
  \definition{v.}{manter segredo; manter algo em segredo; manter a confidencialidade}
\end{EntryWithPhonetic}

\begin{EntryWithPhonetic}{保姆}{bao3mu3}{9,8}{⼈、⼥}[HSK 7-9]
  \definition[位,名,个]{s.}{babá; mulheres empregadas como cuidadoras de crianças ou empregadas domésticas}
\end{EntryWithPhonetic}

\begin{EntryWithPhonetic}{保暖}{bao3/nuan3}{9,13}{⼈、⽇}[HSK 7-9]
  \definition{v.+compl.}{manter-se aquecido; ficar aquecido}[保暖能防感冒。===Manter-se aquecido pode prevenir resfriados.]
\end{EntryWithPhonetic}

\begin{EntryWithPhonetic}{保守}{bao3shou3}{9,6}{⼈、⼧}[HSK 4]
  \definition{adj.}{retrógrado; conservador; pensamentos e conceitos que são retrógrados e não conseguem acompanhar o desenvolvimento da situação}
  \definition{v.}{manter; guardar; evitar perder}
\end{EntryWithPhonetic}

\begin{EntryWithPhonetic}{保亭}{bao3ting2}{9,9}{⼈、⼇}[HSK 7-9]
  \definition*{s.}{Condado autônomo de Li e Miao, Hainan}
\end{EntryWithPhonetic}

\begin{EntryWithPhonetic}{保卫}{bao3wei4}{9,3}{⼈、⼙}[HSK 5]
  \definition{v.}{defender; proteger; salvaguardar; proteger-se de ser violado}
\end{EntryWithPhonetic}

\begin{EntryWithPhonetic}{保鲜}{bao3xian1}{9,14}{⼈、⿂}[HSK 7-9]
  \definition{v.}{manter fresco; preservar o frescor}
\end{EntryWithPhonetic}

\begin{EntryWithPhonetic}{保险}{bao3xian3}{9,9}{⼈、⾩}[HSK 3]
  \definition{adj.}{seguro; pode ficar tranquilo}
  \definition[个,份,种]{s.}{seguro; um tipo de seguro comercial que garante que o segurado receba uma indenização em caso de prejuízo}
  \definition{v.}{ter certeza; estar obrigado a; garantir que algo aconteça (o que as pessoas desejam)}
\end{EntryWithPhonetic}

\begin{EntryWithPhonetic}{保修}{bao3xiu1}{9,9}{⼈、⼈}[HSK 7-9]
  \definition{v.}{garantir que a loja ou fábrica forneça serviço de reparo gratuito dentro de um determinado período de tempo; refere-se ao reparo gratuito de mercadorias pela unidade de vendas ou pelo fabricante, de acordo com os regulamentos, após a venda das mercadorias | manter; manter em bom estado}
\end{EntryWithPhonetic}

\begin{EntryWithPhonetic}{保养}{bao3yang3}{9,9}{⼈、⼋}[HSK 5]
  \definition{v.}{preservar; cuidar bem (ou conservar) da saúde |  fazer manutenção; conservar; manter; manter em bom estado de conservação}
\end{EntryWithPhonetic}

\begin{EntryWithPhonetic}{保佑}{bao3you4}{9,7}{⼈、⼈}[HSK 7-9]
  \definition{s.}{benção}[他希望神明保佑。===Ele espera a bênção de Deus.]
  \definition{v.}{abençoar e proteger; superstição, refere-se à proteção e ajuda dos deuses}[神保佑我今天顺利。===Deus me abençoe hoje.]
\end{EntryWithPhonetic}

\begin{EntryWithPhonetic}{保障}{bao3zhang4}{9,13}{⼈、⾩}[HSK 7-9]
  \definition{s.}{segurança; garantia; coisas que podem fornecer proteção}
  \definition{v.}{proteger; salvaguardar; proteger contra violação e destruição | assegurar; garantir; prometer; garantir que não ocorrerá acidentes}
\end{EntryWithPhonetic}

\begin{EntryWithPhonetic}{保证}{bao3zheng4}{9,7}{⼈、⾔}[HSK 3]
  \definition[种,份]{s.}{compromisso; garantia; caução; aval; condições ou coisas que garantem a realização de algo}
  \definition{v.}{prometer; garantir; assegurar; certamente concluir algo; garantir que determinados padrões e requisitos sejam alcançados}
\end{EntryWithPhonetic}

\begin{EntryWithPhonetic}{保质期}{bao3zhi4qi1}{9,8,12}{⼈、⾙、⽉}[HSK 7-9]
  \definition{s.}{data de validade; data de uso (em alimentos); período durante o qual os alimentos pré-embalados mantêm a sua qualidade nas condições de armazenamento especificadas no rótulo}
\end{EntryWithPhonetic}

\begin{EntryWithPhonetic}{保重}{bao3zhong4}{9,9}{⼈、⾥}[HSK 7-9]
  \definition{v.}{cuidar de si mesmo}
\end{EntryWithPhonetic}

\begin{EntryWithPhonetic}{堡}{bao3}{12}{⼟}
  \definition{s.}{forte; fortaleza | uma terraplenagem | castelo | posição de defesa | usado em nomes de lugares}
  \seeref{bu3}
  \seeref{pu4}
\end{EntryWithPhonetic}

\begin{EntryWithPhonetic}{堡垒}{bao3lei3}{12,9}{⼟、⼟}[HSK 7-9]
  \definition[处,座,个]{s.}{forte; fortaleza; casamata | bastião | fortificação}
\end{EntryWithPhonetic}

\begin{EntryWithPhonetic}{报}{bao4}{7}{⼿}[HSK 3,7-9]
  \definition[份,张]{s.}{jornal | revista; periódico; referência a uma publicação específica | relatório; boletim; algo que transmite alguma informação | telegrama | julgamento; retribuição}
  \definition{v.}{relatar; declarar; anunciar; informar; comunicar | responder; retribuir; revidar | retribuir; recompensar | vingar-se; retaliar | relatar; condenar de acordo com a lei e reportar às autoridades superiores | enviar; submeter; especificamente, relatar ao superior}
\end{EntryWithPhonetic}

\begin{EntryWithPhonetic}{报仇}{bao4/chou2}{7,4}{⼿、⼈}[HSK 7-9]
  \definition{v.+compl.}{vingar; vingar-se}
\end{EntryWithPhonetic}

\begin{EntryWithPhonetic}{报酬}{bao4chou5}{7,13}{⼿、⾣}[HSK 7-9]
  \definition[笔,个]{s.}{pagamento; recompensa; remuneração; dinheiro ou bens pagos a outros pelo uso de seu trabalho, objetos, etc.}
\end{EntryWithPhonetic}

\begin{EntryWithPhonetic}{报答}{bao4da2}{7,12}{⼿、⽵}[HSK 5]
  \definition{v.}{reembolsar; devolver; retribuir; pagar de volta; mostrar seu apreço de forma tangível}
\end{EntryWithPhonetic}

\begin{EntryWithPhonetic}{报到}{bao4/dao4}{7,8}{⼿、⼑}[HSK 3]
  \definition{v.+compl.}{apresentar-se ao serviço; fazer o check-in; registrar-se; assinar o livro de presença; informar à organização que você já chegou}
\end{EntryWithPhonetic}

\begin{EntryWithPhonetic}{报道}{bao4dao4}{7,12}{⼿、⾡}[HSK 3]
  \definition[个,篇,分]{s.}{história; reportagem; comunicado de imprensa publicado por escrito ou transmitido pela rádio}
  \definition{v.}{cobrir; reportar (notícias); divulgar notícias ao público através de jornais, rádio, etc.}
\end{EntryWithPhonetic}

\begin{EntryWithPhonetic}{报废}{bao4/fei4}{7,8}{⼿、⼴}[HSK 7-9]
  \definition{v.+compl.}{sucatear; rejeitar; descartar como inútil}
\end{EntryWithPhonetic}

\begin{EntryWithPhonetic}{报复}{bao4fu4}{7,9}{⼿、⼢}[HSK 7-9]
  \definition{v.}{retaliar; fazer represálias; agir de forma muito cruel com alguém que criticou ou prejudicou seus interesses}
\end{EntryWithPhonetic}

\begin{EntryWithPhonetic}{报告}{bao4gao4}{7,7}{⼿、⼝}[HSK 3]
  \definition[份,篇]{s.}{relatório; discurso; palestra; consultivo; declaração formal feita a superiores ou ao público}
  \definition{v.}{relatar; divulgar; informar; informar formalmente sobre um assunto ou opinião aos superiores ou ao público em geral}
\end{EntryWithPhonetic}

\begin{EntryWithPhonetic}{报警}{bao4jing3}{7,19}{⼿、⾔}[HSK 5]
  \definition{v.}{relatar (um incidente) à polícia; relatar uma situação crítica ou sinalizar uma emergência às autoridades competentes}
\end{EntryWithPhonetic}

\begin{EntryWithPhonetic}{报刊}{bao4 kan1}{7,5}{⼿、⼑}[HSK 6]
  \definition{s.}{a imprensa; jornais e periódicos}
\end{EntryWithPhonetic}

\begin{EntryWithPhonetic}{报考}{bao4 kao3}{7,6}{⼿、⽼}[HSK 6]
  \definition{v.}{inscrever-se para um exame}
\end{EntryWithPhonetic}

\begin{EntryWithPhonetic}{报名}{bao4/ming2}{7,6}{⼿、⼝}[HSK 2]
  \definition{v.+compl.}{inscrever-se; alistar-se; registrar seu nome; cadastrar-se; matricular-se; informar seu nome à pessoa responsável, órgão, grupo etc., indicando que você deseja participar de alguma atividade ou organização}
\end{EntryWithPhonetic}

\begin{EntryWithPhonetic}{报社}{bao4she4}{7,7}{⼿、⽰}[HSK 7-9]
  \definition[家,个]{s.}{escritório de jornal; escritório geral de um jornal; uma organização que edita e publica jornais}
\end{EntryWithPhonetic}

\begin{EntryWithPhonetic}{报销}{bao4xiao1}{7,12}{⼿、⾦}[HSK 7-9]
  \definition{v.}{reembolsar; enviar uma fatura; elaborar a relação dos valores recebidos ou das contas de receitas e despesas e comunicar ao superior para verificação | anular; acabar; consumir; matar alguém; fazer algo perder sua utilidade; comer toda a comida}
\end{EntryWithPhonetic}

\begin{EntryWithPhonetic}{报纸}{bao4zhi3}{7,7}{⼿、⽷}[HSK 2]
  \definition[分,期,张]{s.}{jornal; publicações periódicas cujo conteúdo principal é notícias, geralmente referem-se a jornais diários | papel jornal; um tipo de papel usado para imprimir jornais ou publicações em geral}
\end{EntryWithPhonetic}

\begin{EntryWithPhonetic}{抱}{bao4}{8}{⼿}[HSK 4]
  \definition*{s.}{Sobrenome Bao}
  \definition{clas.}{braçada; medida dos dois braços}
  \definition{v.}{carregar no peito; segurar com ambos os braços; abraçar | ter o primeiro filho ou neto | adotar um bebê ou criança | ficar juntos, unidos | encaixar ou servir perfeitamente (roupas e sapatos do tamanho certo) | estimar; nutrir; abrigar; ter em mente | continuar; sobrecarregar com | chocar ovos}
\end{EntryWithPhonetic}

\begin{EntryWithPhonetic}{抱负}{bao4fu4}{8,6}{⼿、⾙}[HSK 7-9]
  \definition{s.}{aspiração; ambição; objetivo elevado; grandes intenções e determinação, frequentemente usados na linguagem escrita}
\end{EntryWithPhonetic}

\begin{EntryWithPhonetic}{抱歉}{bao4qian4}{8,14}{⼿、⽋}[HSK 6]
  \definition{adj.}{pesaroso; arrependido; sentir pena de alguém porque você causou perda, inconveniência ou não atendeu às suas necessidades}
\end{EntryWithPhonetic}

\begin{EntryWithPhonetic}{抱怨}{bao4yuan4}{8,9}{⼿、⼼}[HSK 5]
  \definition{v.}{reclamar ou expressar descontentamento ou insatisfação; falar com os outros sobre pessoas ou coisas com as quais você não está satisfeito}
\end{EntryWithPhonetic}

\begin{EntryWithPhonetic}{豹}{bao4}{10}{⾘}[HSK 7-9]
  \definition*{s.}{Sobrenome Bao}
  \definition[只]{s.}{leopardo; pantera | espécies de gato da montanha}
\end{EntryWithPhonetic}

\begin{EntryWithPhonetic}{豹子}{bao4zi5}{10,3}{⾘、⼦}
  \definition[头]{s.}{leopardo}
\end{EntryWithPhonetic}

\begin{EntryWithPhonetic}{暴}{bao4}{15}{⽇}
  \definition*{s.}{Sobrenome Bao}
  \definition{adj.}{repentino e violento | cruel; selvagem; feroz | temperamental | severo e tirânico; brutal | irritável; irascível; impaciente}
  \definition{adv.}{de repente e ferozmente}
  \definition{s.}{violência; ferocidade}
  \definition{v.}{sobressair; destacar-se; inchar | expor; transmitir | desperdiçar; arruinar; estragar}
\end{EntryWithPhonetic}

\begin{EntryWithPhonetic}{暴风雨}{bao4 feng1 yu3}{15,4,8}{⽇、⾵、⾬}[HSK 6]
  \definition{s.}{tempestade; tormenta; temporal; borrasca; vento e chuva fortes e violentos}
\end{EntryWithPhonetic}

\begin{EntryWithPhonetic}{暴风骤雨}{bao4feng1-zhou4yu3}{15,4,17,8}{⽇、⾵、⾺、⾬}[HSK 7-9]
  \definition{expr.}{tempestade violenta; furacão; tempestade | vento violento e tempestade de chuva}
\end{EntryWithPhonetic}

\begin{EntryWithPhonetic}{暴力}{bao4li4}{15,2}{⽇、⼒}[HSK 6]
  \definition{s.}{violência; força (usada em tempos de conflito); poder de coerção}
\end{EntryWithPhonetic}

\begin{EntryWithPhonetic}{暴利}{bao4li4}{15,7}{⽇、⼑}[HSK 7-9]
  \definition{s.}{lucros enormes repentinos | lucros exorbitantes; lucros extravagantes; lucros excessivos}
\end{EntryWithPhonetic}

\begin{EntryWithPhonetic}{暴露}{bao4lu4}{15,21}{⽇、⾬}[HSK 6]
  \definition{adj.}{reveladoras (roupas inadequadas que expõem muito o corpo)}
  \definition{v.}{expor; desnudar; revelar; tornar público algo oculto}
\end{EntryWithPhonetic}

\begin{EntryWithPhonetic}{暴乱}{bao4luan4}{15,7}{⽇、⼄}
  \definition{s.}{rebelião | revolta | tumulto}
\end{EntryWithPhonetic}

\begin{EntryWithPhonetic}{暴行}{bao4xing2}{15,6}{⽇、⾏}
  \definition{s.}{ato selvagem | atrocidade | indignação}
\end{EntryWithPhonetic}

\begin{EntryWithPhonetic}{暴雨}{bao4yu3}{15,8}{⽇、⾬}[HSK 6]
  \definition[场,次,阵]{s.}{tempestade; chuva torrencial; chuva forte com precipitação intensa; em meteorologia, refere-se a chuvas de 16 mm ou mais em uma hora ou 50 mm ou mais em 24 horas}
\end{EntryWithPhonetic}

\begin{EntryWithPhonetic}{暴躁}{bao4zao4}{15,20}{⽇、⾜}[HSK 7-9]
  \definition{adj.}{irascível; febril; irritável; temperamental; descreve uma pessoa que é impaciente, não consegue controlar suas emoções e fica com raiva facilmente}
\end{EntryWithPhonetic}

\begin{EntryWithPhonetic}{瀑}{bao4}{18}{⽔}
  \definition{s.}{chuva torrencial; tempestade}
  \seeref{pu4}
\end{EntryWithPhonetic}

\begin{EntryWithPhonetic}{曝}{bao4}{19}{⽇}
  \definition{v.}{usado em  曝光}
  \seeref{pu4}
  \seealsoref{曝光}{bao4/guang1}
\end{EntryWithPhonetic}

\begin{EntryWithPhonetic}{曝光}{bao4/guang1}{19,6}{⽇、⼉}[HSK 7-9]
  \definition{v.+compl.}{expor; sensibilizar filme fotográfico ou papel fotossensível para formar uma imagem latente | expor; tornar (algo ruim) público; metáfora para revelar coisas secretas (geralmente vergonhosas) ao mundo}
\end{EntryWithPhonetic}

\begin{EntryWithPhonetic}{爆}{bao4}{19}{⽕}[HSK 6]
  \definition{v.}{explodir; estourar | fritar rapidamente; ferver rapidamente | aparecer (ou ocorrer) inesperadamente}
\end{EntryWithPhonetic}

\begin{EntryWithPhonetic}{爆发}{bao4fa1}{19,5}{⽕、⼜}[HSK 6]
  \definition{v.}{entrar em erupção; explodir | estourar; irromper; ocorrer de forma repentina e violenta}
\end{EntryWithPhonetic}

\begin{EntryWithPhonetic}{爆冷门}{bao4 leng3men2}{19,7,3}{⽕、⼎、⾨}[HSK 7-9]
  \definition{s.}{um avanço | uma reviravolta (especialmente nos esportes) | reviravolta inesperada dos acontecimentos}
  \definition{v.}{dar um golpe}
\end{EntryWithPhonetic}

\begin{EntryWithPhonetic}{爆满}{bao4man3}{19,13}{⽕、⽔}[HSK 7-9]
  \definition{v.}{(teatro, cinema, estádio, etc.) lotar; ter casa cheia | estar lotado}
\end{EntryWithPhonetic}

\begin{EntryWithPhonetic}{爆米花}{bao4mi3hua1}{19,6,7}{⽕、⽶、⾋}
  \definition{s.}{pipoca (de milho) | pipoca de arroz}
\end{EntryWithPhonetic}

\begin{EntryWithPhonetic}{爆炸}{bao4zha4}{19,9}{⽕、⽕}[HSK 6]
  \definition{s.}{explosão}
  \definition{v.}{explodir; explodir; detonar | aumentar bruscamente em um curto espaço de tempo (de quantidade)}
\end{EntryWithPhonetic}

\begin{EntryWithPhonetic}{爆竹}{bao4 zhu2}{19,6}{⽕、⽵}[HSK 7-9]
  \definition[串,个]{s.}{fogo de artifício}
\end{EntryWithPhonetic}

\begin{EntryWithPhonetic}{卑}{bei1}{8}{⼗}
  \definition{adj.}{Literário: baixo | inferior; médio | Literário: modesto; humilde}
\end{EntryWithPhonetic}

\begin{EntryWithPhonetic}{卑鄙}{bei1bi3}{8,13}{⼗、⾢}[HSK 7-9]
  \definition{adj.}{ruim; vulgar; vil; desprezível}
\end{EntryWithPhonetic}

\begin{EntryWithPhonetic}{杯}{bei1}{8}{⽊}[HSK 1]
  \definition{clas.}{para certos recipientes de líquidos: copo, xícara, etc.}
  \definition[只,个]{s.}{copo; caneca; xícara | taça; troféu; prêmio em forma de taça}
\end{EntryWithPhonetic}

\begin{EntryWithPhonetic}{杯具}{bei1ju4}{8,8}{⽊、⼋}
  \definition{s.}{parachoque | fiasco | (gíria) tragédia}
\end{EntryWithPhonetic}

\begin{EntryWithPhonetic}{杯子}{bei1 zi5}{8,3}{⽊、⼦}[HSK 1]
  \definition[个,只,种]{s.}{xícara; copo; recipiente para bebidas ou outros líquidos, geralmente cilíndrico ou com a parte inferior ligeiramente mais estreita, com capacidade geralmente pequena}
\end{EntryWithPhonetic}

\begin{EntryWithPhonetic}{背}{bei1}{9}{⾁}[HSK 2]
  \definition{clas.}{carga; pacote; para transportar coisas nas costas}
  \definition{v.}{carregar nas costas | suportar; carregar}
  \seeref{bei4}
\end{EntryWithPhonetic}

\begin{EntryWithPhonetic}{背包}{bei1 bao1}{9,5}{⾁、⼓}[HSK 5]
  \definition[个,只,款]{s.}{mochila; mochila de ataque; mochila de infantaria; pacotes de roupas carregados nas costas quando marcham}
\end{EntryWithPhonetic}

\begin{EntryWithPhonetic}{悲}{bei1}{12}{⽕}
  \definition{adj.}{triste; pesaroso; melancólico | compassivo; misericordioso}
\end{EntryWithPhonetic}

\begin{EntryWithPhonetic}{悲哀}{bei1'ai1}{12,9}{⽕、⼝}[HSK 7-9]
  \definition{adj.}{triste}
  \definition{s.}{tristeza; refere-se a coisas tristes e dolorosas}
\end{EntryWithPhonetic}

\begin{EntryWithPhonetic}{悲惨}{bei1can3}{12,11}{⽕、⽕}[HSK 6]
  \definition{adj.}{trágico; miserável; extremamente doloroso e triste}
\end{EntryWithPhonetic}

\begin{EntryWithPhonetic}{悲观}{bei1guan1}{12,6}{⽕、⾒}[HSK 7-9]
  \definition{adj.}{pessimista; negativismo, falta de confiança no futuro (oposto a 乐观)}
  \seealsoref{乐观}{le4guan1}
\end{EntryWithPhonetic}

\begin{EntryWithPhonetic}{悲欢离合}{bei1huan1-li2he2}{12,6,10,6}{⽕、⽋、⼇、⼝}[HSK 7-9]
  \definition{expr.}{alegrias e tristezas | separações e reencontros | as vicissitudes da vida}
\end{EntryWithPhonetic}

\begin{EntryWithPhonetic}{悲剧}{bei1 ju4}{12,10}{⽕、⼑}[HSK 5]
  \definition[出,部]{s.}{tragédia; drama trágico; uma das principais categorias de teatro, caracterizada basicamente pela representação do conflito irreconciliável entre o protagonista e a realidade e seu final trágico | tragédia; evento triste; metáfora para encontro infeliz}
\end{EntryWithPhonetic}

\begin{EntryWithPhonetic}{悲伤}{bei1 shang1}{12,6}{⽕、⼈}[HSK 5]
  \definition{adj.}{triste; pesaroso}
\end{EntryWithPhonetic}

\begin{EntryWithPhonetic}{悲痛}{bei1tong4}{12,12}{⽕、⽧}[HSK 7-9]
  \definition{adj.}{triste; aflito}
  \definition{s.}{tristeza; sofrimento}
\end{EntryWithPhonetic}

\begin{EntryWithPhonetic}{碑}{bei1}{13}{⽯}[HSK 7-9]
  \definition[块,座,个,面]{s.}{estela; uma tábua de pedra; uma pedra gravada com palavras e erguida como um memorial ou marca}
\end{EntryWithPhonetic}

\begin{EntryWithPhonetic}{北}{bei3}{5}{⼔}[HSK 1]
  \definition*{s.}{Norte (os países desenvolvidos) | Sobrenome Bei}
  \definition{s.}{norte; uma das quatro direções básicas, a esquerda quando se está de frente para o sol pela manhã (oposta ao 南)}
  \definition{v.}{ser derrotado}
  \seealsoref{南}{nan2}
\end{EntryWithPhonetic}

\begin{EntryWithPhonetic}{北边}{bei3 bian1}{5,5}{⼔、⾡}[HSK 1]
  \definition{s.}{norte; o lado norte}
\end{EntryWithPhonetic}

\begin{EntryWithPhonetic}{北部}{bei3 bu4}{5,10}{⼔、⾢}[HSK 3]
  \definition{s.}{parte norte de uma região ou país}
\end{EntryWithPhonetic}

\begin{EntryWithPhonetic}{北大西洋公约组织}{bei3 da4xi1 yang2 gong1 yue1 zu3zhi1}{5,3,6,9,4,6,8,8}{⼔、⼤、⾑、⽔、⼋、⽷、⽷、⽷}
  \definition*{s.}{Organização do Tratado do Atlântico Norte, OTAN}
\end{EntryWithPhonetic}

\begin{EntryWithPhonetic}{北方}{bei3fang1}{5,4}{⼔、⽅}[HSK 2]
  \definition{s.}{norte; indicando a direção norte | o Norte; a parte norte da China, especialmente a área ao norte do rio Huang He}
\end{EntryWithPhonetic}

\begin{EntryWithPhonetic}{北极}{bei3ji2}{5,7}{⼔、⽊}[HSK 5]
  \definition*{s.}{Polo Norte; Polo Ártico}
  \definition{s.}{polo norte magnético; o ponto mais setentrional da Terra, também se refere à região mais setentrional da Terra}
\end{EntryWithPhonetic}

\begin{EntryWithPhonetic}{北京}{bei3 jing1}{5,8}{⼔、⼇}[HSK 1]
  \definition*{s.}{Pequim (Beijing), Capital da República Popular da China | Capital da China, localizada no nordeste do país, fundada em 700 a.C., a cidade é um importante centro comercial, industrial e cultural}
\end{EntryWithPhonetic}

\begin{EntryWithPhonetic}{北面}{bei3mian4}{5,9}{⼔、⾯}
  \definition{s.}{norte; o lado norte}
\end{EntryWithPhonetic}

\begin{EntryWithPhonetic}{北约}{bei3yue1}{5,6}{⼔、⽷}
  \definition*{s.}{OTAN, Organização do Tratado do Atlântico Norte; Abreviação de 北大西洋公约组织}
  \seealsoref{北大西洋公约组织}{bei3 da4xi1 yang2 gong1 yue1 zu3zhi1}
\end{EntryWithPhonetic}

\begin{EntryWithPhonetic}{贝}{bei4}{4}{⾙}[Kangxi 154]
  \definition*{s.}{Sobrenome Bei}
  \definition{clas.}{bel (= dez decibéis)}
  \definition{s.}{marisco; crustáceo; um termo geral para moluscos com concha | moeda antiga feita de conchas}
\end{EntryWithPhonetic}

\begin{EntryWithPhonetic}{贝壳}{bei4ke2}{4,7}{⾙、⼠}[HSK 7-9]
  \definition[个,种]{s.}{concha; concha do mar; concha de animais de concha}
\end{EntryWithPhonetic}

\begin{EntryWithPhonetic}{备}{bei4}{8}{⼡}
  \definition*{s.}{Sobrenome Bei}
  \definition{adv.}{totalmente; de todas as maneiras possíveis | todos; tudo}
  \definition{s.}{equipamento}
  \definition{v.}{estar equipar com; ter; possuir | preparar; ficar pronto | providenciar (ou preparar) contra; tomar precauções contra}
\end{EntryWithPhonetic}

\begin{EntryWithPhonetic}{备份}{bei4fen4}{8,6}{⼡、⼈}
  \definition{s.}{cópia de segurança | \emph{backup}}
\end{EntryWithPhonetic}

\begin{EntryWithPhonetic}{备课}{bei4/ke4}{8,10}{⼡、⾔}[HSK 7-9]
  \definition{v.+compl.}{(professor) preparar aulas}
\end{EntryWithPhonetic}

\begin{EntryWithPhonetic}{备受}{bei4shou4}{8,8}{⼡、⼜}[HSK 7-9]
  \definition{v.}{experimentar plenamente (o bem ou o mal)}
\end{EntryWithPhonetic}

\begin{EntryWithPhonetic}{备胎}{bei4tai1}{8,9}{⼡、⾁}
  \definition{s.}{pneu sobressalente | (gíria) substituto}
\end{EntryWithPhonetic}

\begin{EntryWithPhonetic}{备用}{bei4yong4}{8,5}{⼡、⽤}[HSK 7-9]
  \definition{v.}{reservar; guardar algo para uso futuro}
\end{EntryWithPhonetic}

\begin{EntryWithPhonetic}{背}{bei4}{9}{⾁}[HSK 3]
  \definition{adj.}{azarado | fora do caminho; um lugar muito distante do centro movimentado, onde poucas pessoas aparecem | deficiente auditivo}
  \definition{s.}{parte posterior do corpo; costas; coluna vertebral; parte do tronco entre os ombros e a região lombar | parte de trás de um objeto}
  \definition{v.}{afastar-se; virar as costas | decorar; memorizar; recitar de memória | esconder algo de; fazer algo em segredo | sair, ir embora; partir; abandonar | quebrar; violar; agir de forma contrária a}
  \seeref{bei1}
\end{EntryWithPhonetic}

\begin{EntryWithPhonetic}{背后}{bei4 hou4}{9,6}{⾁、⼝}[HSK 3]
  \definition{s.}{parte posterior; parte de trás; traseira | pelas costas de alguém}
\end{EntryWithPhonetic}

\begin{EntryWithPhonetic}{背景}{bei4jing3}{9,12}{⾁、⽇}[HSK 4]
  \definition[种]{s.}{pano de fundo; fundo; cenário de teatro, filme ou drama de TV | fundo; cenário que permeia a imagem principal na tela | condições sociais; ambientes históricos (significativamente influentes para algo ou alguém) | poder que dá suporte a alguém}
\end{EntryWithPhonetic}

\begin{EntryWithPhonetic}{背面}{bei4mian4}{9,9}{⾁、⾯}[HSK 7-9]
  \definition{s.}{costas; lado reverso; lado avesso (oposto a 正面) | o verso; o reverso; o avesso}
  \seealsoref{正面}{zheng4mian4}
\end{EntryWithPhonetic}

\begin{EntryWithPhonetic}{背叛}{bei4pan4}{9,9}{⾁、⼜}[HSK 7-9]
  \definition{s.}{traição; deslealdade; refere-se ao ato ou evento de traição}
  \definition{v.}{trair; refere-se à traição, rebelião e atos que violam a moralidade e traem a confiança}
\end{EntryWithPhonetic}

\begin{EntryWithPhonetic}{背诵}{bei4song4}{9,9}{⾁、⾔}[HSK 7-9]
  \definition{v.}{recitar; repetir de memória; ler de memória o texto ou as frases que você leu}
\end{EntryWithPhonetic}

\begin{EntryWithPhonetic}{背心}{bei4 xin1}{9,4}{⾁、⼼}[HSK 6]
  \definition[件]{s.}{colete; vestimenta sem mangas; \emph{tops} sem gola e sem mangas}
\end{EntryWithPhonetic}

\begin{EntryWithPhonetic}{背着}{bei4 zhe5}{9,11}{⾁、⽬}[HSK 6]
  \definition{adv.}{pelas costas; atrás de alguém}
  \definition{v.}{carregar nas costas}
\end{EntryWithPhonetic}

\begin{EntryWithPhonetic}{倍}{bei4}{10}{⼈}[HSK 4]
  \definition{adv.}{ainda mais; especialmente | (antes de certos adjetivos) muito; particularmente; é pronunciado como um som erhua e é usado antes de certos adjetivos para expressar um alto grau de profundidade, equivalente a 非常 ou 特别}
  \definition{clas.}{vezes; usado após um numeral, significa que o valor anterior é multiplicado por este número}[增长了五倍。===Aumentou cinco vezes. | 二的三倍是六。===Três vezes dois é seis.]
  \seealsoref{非常}{fei1chang2}
  \seealsoref{特别}{te4bie2}
\end{EntryWithPhonetic}

\begin{EntryWithPhonetic}{被}{bei4}{10}{⾐}[HSK 3]
  \definition{part.}{usada antes de verbos para formar frases verbais passivas}
  \definition{prep.}{usado em uma estrutura passiva para introduzir o executor da ação ou apenas a ação | usado em frases para expressar passividade, com o sujeito sendo o objeto}
  \definition{s.}{colcha}
  \definition{v.}{cobrir; espalhar | sofrer}
\end{EntryWithPhonetic}

\begin{EntryWithPhonetic}{被捕}{bei4bu3}{10,10}{⾐、⼿}[HSK 7-9]
  \definition{v.}{ser preso; estar sob prisão}
\end{EntryWithPhonetic}

\begin{EntryWithPhonetic}{被单}{bei4dan1}{10,8}{⾐、⼗}
  \definition[床]{s.}{lençol (de cama) | envelope para uma colcha acolchoada}
\end{EntryWithPhonetic}

\begin{EntryWithPhonetic}{被动}{bei4dong4}{10,6}{⾐、⼒}[HSK 5]
  \definition{adj.}{passivo;  agir com base em um impulso externo (oposto de 主动) | passivo; impossibilidade de prosseguir como pretendido devido a resistência ou interferência}
  \seealsoref{主动}{zhu3dong4}
\end{EntryWithPhonetic}

\begin{EntryWithPhonetic}{被告}{bei4gao4}{10,7}{⾐、⼝}[HSK 6]
  \definition{s.}{réu; indiciado; acusado (oposto a 原告)}
  \seealsoref{原告}{yuan2gao4}
\end{EntryWithPhonetic}

\begin{EntryWithPhonetic}{被迫}{bei4 po4}{10,8}{⾐、⾡}[HSK 4]
  \definition{v.}{ser forçado; ser coagido; ser compelido; ser constrangido; ser forçado a fazer algo por força externa}
\end{EntryWithPhonetic}

\begin{EntryWithPhonetic}{被套}{bei4tao4}{10,10}{⾐、⼤}
  \definition{s.}{capa de \emph{edredon}}
  \definition{v.}{ter dinheiro preso (em ações, imóveis, etc.)}
\end{EntryWithPhonetic}

\begin{EntryWithPhonetic}{被窝}{bei4wo1}{10,12}{⾐、⽳}
  \definition{s.}{colcha}
\end{EntryWithPhonetic}

\begin{EntryWithPhonetic}{被子}{bei4zi5}{10,3}{⾐、⼦}[HSK 3]
  \definition[条,床]{s.}{colcha; cobertor; algo com que você se cobre quando dorme, geralmente feito de pano ou seda, com forro de pano e preenchido com algodão ou fio de seda}
\end{EntryWithPhonetic}

\begin{EntryWithPhonetic}{辈}{bei4}{12}{⾞}[HSK 5]
  \definition*{s.}{Sobrenome Bei}
  \definition{s.}{pessoas de um certo tipo; semelhantes | geração; geração na família | duração da vida | círculo familiar}
\end{EntryWithPhonetic}

\begin{EntryWithPhonetic}{奔}{ben1}{8}{⼤}
  \definition{v.}{correr rápido; correr com pressa | apressar | fugir; escapar | galopar | fugir; termo antigo para uma mulher que foge com um homem}
  \seeref{ben4}
\end{EntryWithPhonetic}

\begin{EntryWithPhonetic}{奔波}{ben1bo1}{8,8}{⼤、⽔}[HSK 7-9]
  \definition{v.}{correr; estar ocupado correndo; correr para frente e para trás, com dificuldade e ocupado}
\end{EntryWithPhonetic}

\begin{EntryWithPhonetic}{奔驰}{ben1chi2}{8,6}{⼤、⾺}
  \definition*{s.}{Benz de Mercedes-Benz}
  \definition{v.}{acelerar; galopar; (carro, cavalo, etc.) mover-se ou correr rapidamente}
  \seealsoref{梅赛德斯-奔驰}{mei2sai4de2si1-ben1chi2}
\end{EntryWithPhonetic}

\begin{EntryWithPhonetic}{奔赴}{ben1fu4}{8,9}{⼤、⾛}[HSK 7-9]
  \definition{v.}{correr para; apressar-se para; correr em direção a (um certo destino)}
\end{EntryWithPhonetic}

\begin{EntryWithPhonetic}{奔跑}{ben1 pao3}{8,12}{⼤、⾜}[HSK 6]
  \definition{v.}{correr; correr muito rápido, com uma gama de aplicações mais ampla do que 奔驰, usado principalmente na linguagem falada}
  \seealsoref{奔驰}{ben1chi2}
\end{EntryWithPhonetic}

\begin{EntryWithPhonetic}{本}{ben3}{5}{⽊}[HSK 1,6]
  \definition*{s.}{Sobrenome Ben}
  \definition{adj.}{original; inerente | principal; central}
  \definition{adv.}{originalmente}
  \definition{clas.}{para livros, dicionários, periódicos, arquivos, etc. | para vídeos de uma determinada duração | para peças de teatro, ópera}
  \definition{prep.}{de acordo com; em consonância com; em conformidade com; equivalentes a 依照 e 按照}
  \definition{pron.}{nativo; próprio; refere-se ao próprio interlocutor ou ao grupo, instituição, empresa, local, etc. ao qual o interlocutor pertence | isto; atual; presente}
  \definition[个]{s.}{caule ou raiz de plantas | base; origem; fundamento; fundação;  alicerce | capital; capital social | livro; caderno; livreto | edição; versão | cópia; roteiro; manuscrito | memorial do trono; na era feudal, referia-se a um documento oficial}
  \definition{v.}{seguir; basear-se em; estar de acordo com}
  \seealsoref{按照}{an4zhao4}
  \seealsoref{依照}{yi1 zhao4}
\end{EntryWithPhonetic}

\begin{EntryWithPhonetic}{本地}{ben3 di4}{5,6}{⽊、⼟}[HSK 6]
  \definition{s.}{local; nativo; localidade; a área onde as pessoas e as coisas estão localizadas; uma área específica referida em uma narrativa}
\end{EntryWithPhonetic}

\begin{EntryWithPhonetic}{本分}{ben3fen4}{5,4}{⽊、⼑}[HSK 7-9]
  \definition{adj.}{honesto; decente}
  \definition{s.}{o trabalho de alguém; o dever de alguém; as próprias responsabilidades e obrigações | estar satisfeito com sua posição e ambiente atuais}
\end{EntryWithPhonetic}

\begin{EntryWithPhonetic}{本金}{ben3 jin1}{5,8}{⽊、⾦}
  \definition{s.}{capital; capital para a operação do comércio e da indústria; capital para a operação de negócios | valor principal; dinheiro retirado ao depositar ou tomar emprestado (diferente de 利息)}
  \seealsoref{利息}{li4xi1}
\end{EntryWithPhonetic}

\begin{EntryWithPhonetic}{本科}{ben3ke1}{5,9}{⽊、⽲}[HSK 4]
  \definition{s.}{graduação; bacharelado; o curso básico de uma universidade ou faculdade}
\end{EntryWithPhonetic}

\begin{EntryWithPhonetic}{本来}{ben3lai2}{5,7}{⽊、⽊}[HSK 3]
  \definition{adj.}{original}
  \definition{adv.}{anteriormente; originalmente; indica que antes disso | claro; em primeiro lugar; como deveria ser; indica que algo é natural ou óbvio}
\end{EntryWithPhonetic}

\begin{EntryWithPhonetic}{本领}{ben3 ling3}{5,11}{⽊、⾴}[HSK 3]
  \definition[项,个,种]{s.}{habilidade; capacidade; faculdade; poder; destreza; talento}
\end{EntryWithPhonetic}

\begin{EntryWithPhonetic}{本能}{ben3neng2}{5,10}{⽊、⾁}[HSK 7-9]
  \definition{adv.}{instintivamente; pela luz da natureza; inconscientemente, subconscientemente}
  \definition{s.}{instinto; as habilidades que humanos e animais desenvolvem durante o processo de evolução são fixadas pela hereditariedade e não precisam ser ensinadas}
\end{EntryWithPhonetic}

\begin{EntryWithPhonetic}{本期}{ben3 qi1}{5,12}{⽊、⽉}[HSK 6]
  \definition{adv.}{o período atual | este prazo (geralmente em finanças)}
\end{EntryWithPhonetic}

\begin{EntryWithPhonetic}{本钱}{ben3qian2}{5,10}{⽊、⾦}[HSK 7-9]
  \definition{s.}{capital; dinheiro usado para gerar lucros, juros ou para participar de atividades como jogos de azar | habilidades, qualificações ou condições que podem ser usadas para fazer algo; metaforicamente falando, qualificações, habilidades e outras condições em que se pode confiar}
\end{EntryWithPhonetic}

\begin{EntryWithPhonetic}{本人}{ben3ren2}{5,2}{⽊、⼈}[HSK 5]
  \definition{pron.}{eu (mim, mim mesmo); o orador refere-se a si mesmo | a si mesmo; em pessoa; refere-se à própria pessoa ou à pessoa mencionada anteriormente}
\end{EntryWithPhonetic}

\begin{EntryWithPhonetic}{本色}{ben3se4}{5,6}{⽊、⾊}[HSK 7-9]
  \definition{s.}{(algo, geralmente tecidos não tingidos) cor natural; a cor original de algo (geralmente se refere a tecidos não tingidos)}
\end{EntryWithPhonetic}

\begin{EntryWithPhonetic}{本身}{ben3shen1}{5,7}{⽊、⾝}[HSK 6]
  \definition{pron.}{próprio; em si mesmo; refere-se à pessoa, unidade ou coisa em si}
\end{EntryWithPhonetic}

\begin{EntryWithPhonetic}{本事}{ben3shi4}{5,8}{⽊、⼅}
  \definition{s.}{habilidade; aptidão; capacidade; competência; refere-se às habilidades, capacidades ou talentos que uma pessoa possui em determinada área | habilidade; aptidão; capacidade; competência; a capacidade e os meios necessários para atingir um determinado objetivo ou concluir uma determinada tarefa | status; poder; posição; autoridade; refere-se à identidade, posição ou poder de uma pessoa.}
  \seeref{ben3shi5}
\end{EntryWithPhonetic}

\begin{EntryWithPhonetic}{本事}{ben3shi5}{5,8}{⽊、⼅}[HSK 3]
  \definition{s.}{habilidade; capacidade; talento; aptidão}
  \seeref{ben3shi4}
\end{EntryWithPhonetic}

\begin{EntryWithPhonetic}{本土}{ben3 tu3}{5,3}{⽊、⼟}[HSK 6]
  \definition{s.}{território metropolitano; pátria-mãe; refere-se ao território do país | país (ou terra) natal de alguém; nativo; cidade natal; local original de crescimento}
\end{EntryWithPhonetic}

\begin{EntryWithPhonetic}{本性}{ben3xing4}{5,8}{⽊、⼼}[HSK 7-9]
  \definition{s.}{qualidade inerente | instintos naturais | natureza}
\end{EntryWithPhonetic}

\begin{EntryWithPhonetic}{本意}{ben3yi4}{5,13}{⽊、⼼}[HSK 7-9]
  \definition{s.}{ideia original; intenção real (original)}
\end{EntryWithPhonetic}

\begin{EntryWithPhonetic}{本着}{ben3zhe5}{5,11}{⽊、⽬}[HSK 7-9]
  \definition{prep.}{com base em; de acordo com; em conformidade com; à luz de}
\end{EntryWithPhonetic}

\begin{EntryWithPhonetic}{本质}{ben3zhi4}{5,8}{⽊、⾙}[HSK 6]
  \definition{s.}{essência; natureza; caráter inato; qualidade intrínseca; refere-se aos atributos fundamentais inerentes às próprias coisas, que desempenham um papel decisivo na natureza, condição e desenvolvimento das coisas (distinguido de 现象)}
  \seealsoref{现象}{xian4xiang4}
\end{EntryWithPhonetic}

\begin{EntryWithPhonetic}{本子}{ben3 zi5}{5,3}{⽊、⼦}[HSK 1]
  \definition[个,本]{s.}{livro; caderno | edição | impressão | licença; certificado de competência emitido por uma instituição especializada, obtido após aprovação no exame | \emph{script}; roteiro}
\end{EntryWithPhonetic}

\begin{EntryWithPhonetic}{奔}{ben4}{8}{⼤}[HSK 7-9]
  \definition{prep.}{em direção a}
  \definition{v.}{ir direto em direção a; seguir em direção a; ir direto para o seu destino | aproximar-se; estar prestes a | estar ocupado correndo por aí; correr por algo}
  \seeref{ben1}
\end{EntryWithPhonetic}

\begin{EntryWithPhonetic}{笨}{ben4}{11}{⽵}[HSK 4]
  \definition{adj.}{estúpido; sem graça; tolo; de pouca habilidade; sem inteligência | desajeitado; tosco; inflexível | incômodo; pesado; desajeitado; difícil de manejar; trabalhoso}
\end{EntryWithPhonetic}

\begin{EntryWithPhonetic}{笨蛋}{ben4dan4}{11,11}{⽵、⾍}[HSK 7-9]
  \definition[个]{s.}{tolo; idiota; (depreciativo) refere-se a uma pessoa muito estúpida ou sem cérebro; geralmente usado para insultar pessoas}
\end{EntryWithPhonetic}

\begin{EntryWithPhonetic}{笨重}{ben4zhong4}{11,9}{⽵、⾥}[HSK 7-9]
  \definition{adj.}{pesado; desajeitado; incômodo; grande e pesado, inconveniente de usar | pesado; difícil de manejar; pesado e trabalhoso}
\end{EntryWithPhonetic}

\begin{EntryWithPhonetic}{崩}{beng1}{11}{⼭}
  \definition{v.}{colapsar |  estourar; quebrar | atingir por explosão | matar atirando; atirar; executar | (de um imperador) morrer | rachar; romper | atingir | executar atirando}
\end{EntryWithPhonetic}

\begin{EntryWithPhonetic}{崩溃}{beng1kui4}{11,12}{⼭、⽔}[HSK 7-9]
  \definition{v.}{colapsar; desmoronar; cair aos pedaços; as coisas estão destruídas; as emoções das pessoas estão fora de controle}
\end{EntryWithPhonetic}

\begin{EntryWithPhonetic}{绷}{beng1}{11}{⽷}[HSK 7-9]
  \definition{s.}{estrutura de cama amarrada com cordas, tiras de vime, etc.}
  \definition{v.}{esticar (ou puxar) com força | saltar; quicar | alinhavar; fixar | Dialeto: conseguir fazer algo com dificuldade | (roupas) apertar | costurar ou alfinetar com parcimônia |Dialeto: fraudar; roubar dinheiro}
  \seeref{beng3}
  \seeref{beng4}
\end{EntryWithPhonetic}

\begin{EntryWithPhonetic}{绷带}{beng1dai4}{11,9}{⽷、⼱}[HSK 7-9]
  \definition[条,卷]{s.}{curativo | Empréstimo linguístico: \emph{bandage}; a atadura de gaze usada para enfaixar feridas ou áreas afetadas}
\end{EntryWithPhonetic}

\begin{EntryWithPhonetic}{甭}{beng2}{9}{⽤}
  \definition{adv.}{não; não precisa; não tem que; contração de 不用}
  \seealsoref{不用}{bu2 yong4}
\end{EntryWithPhonetic}

\begin{EntryWithPhonetic}{绷}{beng3}{11}{⽷}
  \definition{v.}{mostrar uma cara sombria, tensa; parecer descontente | conter o próprio temperamento}
  \seeref{beng1}
  \seeref{beng4}
\end{EntryWithPhonetic}

\begin{EntryWithPhonetic}{绷}{beng4}{11}{⽷}
  \definition{adv.}{muito; extremamente; altamente; usado antes de certos adjetivos para indicar um alto grau de severidade}
  \definition{v.}{rachar; dividir; rasgar; fissurar}
  \seeref{beng1}
  \seeref{beng3}
\end{EntryWithPhonetic}

\begin{EntryWithPhonetic}{蹦}{beng4}{18}{⾜}[HSK 7-9]
  \definition{v.}{pular; saltar; quicar}
\end{EntryWithPhonetic}

\begin{EntryWithPhonetic}{蹦极}{beng4ji2}{18,7}{⾜、⽊}
  \definition{s.}{\emph{bungee jumping}}
\end{EntryWithPhonetic}

\begin{EntryWithPhonetic}{逼}{bi1}{12}{⾡}[HSK 6]
  \definition{adj.}{estreito}
  \definition{v.}{forçar; pressionar; compelir | extorquir; pressionar por | fechar em; pressionar em direção a; aproximar-se}
\end{EntryWithPhonetic}

\begin{EntryWithPhonetic}{逼近}{bi1jin4}{12,7}{⾡、⾡}[HSK 7-9]
  \definition{adj.}{aproximado (valor númerico)}
  \definition{s.}{aproximação (função matemática mais simples)}
  \definition{v.}{avançar em direção a; aproximar-se de; aproximar-se; aproximar-se | ganhar em (sobre); aglomerar-se em}
\end{EntryWithPhonetic}

\begin{EntryWithPhonetic}{逼迫}{bi1po4}{12,8}{⾡、⾡}[HSK 7-9]
  \definition{v.}{forçar; compelir; coagir | restringir; exercer pressão para induzir; forçar}
\end{EntryWithPhonetic}

\begin{EntryWithPhonetic}{逼真}{bi1zhen1}{12,10}{⾡、⼗}[HSK 7-9]
  \definition{adj.}{fiel à realidade; realista; muito semelhante à coisa real | claro; distinto; verdadeiro}
\end{EntryWithPhonetic}

\begin{EntryWithPhonetic}{鼻}{bi2}{14}{⿐}[Kangxi 209]
  \definition{s.}{nariz}
\end{EntryWithPhonetic}

\begin{EntryWithPhonetic}{鼻涕}{bi2ti4}{14,10}{⿐、⽔}[HSK 7-9]
  \definition[些,点]{s.}{ranho; muco nasal; secreção nasal; fluido secretado pela mucosa nasal}
\end{EntryWithPhonetic}

\begin{EntryWithPhonetic}{鼻子}{bi2zi5}{14,3}{⿐、⼦}[HSK 5]
  \definition[个,只]{s.}{nariz; órgão da face, responsável pela respiração e pelo olfato}
\end{EntryWithPhonetic}

\begin{EntryWithPhonetic}{比}{bi3}{4}{⽐}[HSK 1][Kangxi 81]
  \definition*{s.}{Abreviação de Bélgica, 比利时}
  \definition{adj.}{específico}
  \definition{adv.}{recentemente}
  \definition{part.}{partícula usada para comparação (superioridade)}
  \definition{prep.}{que; do que | (seguido por um substantivo e adjetivo) mais \{adj.\} do que \{s.\}}
  \definition{s.}{razão; proporção | contraste; comparação | metáfora em poesia; técnica de composição poética}
  \definition{v.}{estar ao lado de; estar próximo a | igualar; comparar; competir; contrastar; emular; comparar superioridade, inferioridade, comprimento, distância, qualidade, etc. | assemelhar-se a; comparar com; fazer uma analogia | gesticular; fazer gestos | ser treinado em; ser direcionado a | copiar; imitar | poder ser comparado | apegar-se a; depender}
  \seealsoref{比利时}{bi3li4shi2}
\end{EntryWithPhonetic}

\begin{EntryWithPhonetic}{比比皆是}{bi3bi3-jie1shi4}{4,4,9,9}{⽐、⽐、⽩、⽇}[HSK 7-9]
  \definition{suf.}{pode ser encontrado em todos os lugares; ao redor | encontrar o olhar em todos os lugares; ser grande em número; ser visto em todos os lugares -- em abundância; um após o outro}
\end{EntryWithPhonetic}

\begin{EntryWithPhonetic}{比不上}{bi3 bu5 shang4}{4,4,3}{⽐、⼀、⼀}[HSK 7-9]
  \definition{expr.}{não se compara a; uma coisa não pode ser comparada com outra, não pode atingir o mesmo nível}
\end{EntryWithPhonetic}

\begin{EntryWithPhonetic}{比方}{bi3fang1}{4,4}{⽐、⽅}[HSK 5]
  \definition{conj.}{se; suponha que; expressa uma hipótese, equivalente a 如果 (com eufemismos)}
  \definition{s.}{analogia; exemplo; instância; expressão que usa uma coisa para descrever outra (expressão idiomática); (figurativo) usar uma coisa para descrever outra}
  \definition{v.}{ilustrar; exemplificar; fazer uma analogia; usar uma coisa para descrever outra (expressão idiomática)}
  \seealsoref{如果}{ru2guo3}
\end{EntryWithPhonetic}

\begin{EntryWithPhonetic}{比分}{bi3 fen1}{4,4}{⽐、⼑}[HSK 4]
  \definition{s.}{pontuação; comparação de pontuações entre as duas equipes em uma partida}
\end{EntryWithPhonetic}

\begin{EntryWithPhonetic}{比较}{bi3jiao4}{4,10}{⽐、⾞}[HSK 3]
  \definition{adv.}{razoavelmente; relativamente; bastante; um pouco; comparativamente; indica um certo grau, com o significado de 相当}
  \definition{prep.}{usado para comparar uma diferença de grau. para distinguir as diferenças ou superioridades entre duas ou mais coisas semelhantes}
  \definition{v.}{comparar; contrastar; usado para comparar diferenças em propriedades e graus; para distinguir semelhanças, diferenças ou superioridade entre duas ou mais coisas semelhantes}
  \seealsoref{相当}{xiang1dang1}
\end{EntryWithPhonetic}

\begin{EntryWithPhonetic}{比利时}{bi3li4shi2}{4,7,7}{⽐、⼑、⽇}
  \definition*{s.}{Bélgica}
\end{EntryWithPhonetic}

\begin{EntryWithPhonetic}{比例}{bi3li4}{4,8}{⽐、⼈}[HSK 3]
  \definition{s.}{escala; razão; relação de múltiplos entre dois números | porporção; a quantidade que uma parte representa no todo | proporção; na descrição do grau de coordenação das coisas}
\end{EntryWithPhonetic}

\begin{EntryWithPhonetic}{比拼}{bi3pin1}{4,9}{⽐、⼿}
  \definition{s.}{concurso}
  \definition{v.}{competir ferozmente}
\end{EntryWithPhonetic}

\begin{EntryWithPhonetic}{比起}{bi3qi3}{4,10}{⽐、⾛}[HSK 7-9]
  \definition{adv./prep.}{comparado com}
\end{EntryWithPhonetic}

\begin{EntryWithPhonetic}{比如}{bi3ru2}{4,6}{⽐、⼥}[HSK 2]
  \definition{conj.}{por exemplo; tal como; suponha; digamos; a seguir, apresentamos alguns exemplos; na linguagem coloquial, também se pode dizer 比如说}
  \seealsoref{比如说}{bi3 ru2 shuo1}
\end{EntryWithPhonetic}

\begin{EntryWithPhonetic}{比如说}{bi3 ru2 shuo1}{4,6,9}{⽐、⼥、⾔}[HSK 2]
  \definition{adv.}{por exemplo}
  \seealsoref{比如}{bi3ru2}
\end{EntryWithPhonetic}

\begin{EntryWithPhonetic}{比萨饼}{bi3sa4bing3}{4,11,9}{⽐、⾋、⾷}
  \definition[张]{s.}{pizza}
\end{EntryWithPhonetic}

\begin{EntryWithPhonetic}{比赛}{bi3sai4}{4,14}{⽐、⾙}[HSK 3]
  \definition[场,次,轮,站,个]{s.}{competição; atividades da competição}
  \definition{v.}{competir; disputar; comparar o nível e a qualidade das habilidades e competências}
\end{EntryWithPhonetic}

\begin{EntryWithPhonetic}{比试}{bi3shi4}{4,8}{⽐、⾔}[HSK 7-9]
  \definition{v.}{competir; ter uma competição; contestar | medir com a mão ou o braço; fazer um gesto de medição}
\end{EntryWithPhonetic}

\begin{EntryWithPhonetic}{比亚迪}{bi3ya4di2}{4,6,8}{⽐、⼆、⾡}
  \definition*{s.}{Montadora BYD}
\end{EntryWithPhonetic}

\begin{EntryWithPhonetic}{比喻}{bi3yu4}{4,12}{⽐、⼝}[HSK 7-9]
  \definition[个,种]{s.}{analogia; metáfora; um método ou exemplo de uso de uma imagem ou coisa concreta para ilustrar outra coisa}
  \definition{v.}{fazer uma analogia; usar uma metáfora; usar algo semelhante para comparar com algo que você quer dizer}
\end{EntryWithPhonetic}

\begin{EntryWithPhonetic}{比重}{bi3zhong4}{4,9}{⽐、⾥}[HSK 5]
  \definition{s.}{proporção; o peso da parte em relação ao todo | Física: densidade específica; a relação entre o peso de um objeto e seu volume}
\end{EntryWithPhonetic}

\begin{EntryWithPhonetic}{彼}{bi3}{8}{⼻}
  \definition{s.}{aquele; aquilo (oposto a 此) ; outro | a outra parte}
  \seealsoref{此}{ci3}
\end{EntryWithPhonetic}

\begin{EntryWithPhonetic}{彼此}{bi3ci3}{8,6}{⼻、⽌}[HSK 5]
  \definition{pron.}{um ao outro; uns com os outros; este e aquele têm algum tipo de relacionamento; ambas as partes}
\end{EntryWithPhonetic}

\begin{EntryWithPhonetic}{笔}{bi3}{10}{⽵}[HSK 2]
  \definition{clas.}{usado para grandes quantias de dinheiro, compras, negócios, propriedades, etc. | usado em caligrafia e pintura, etc.}
  \definition[支,枝]{s.}{caneta; lápis; pincel para escrever; ferramentas para escrever ou desenhar | técnica de escrita; caligrafia ou desenho | traço}
  \definition{v.}{escrever à mão}
\end{EntryWithPhonetic}

\begin{EntryWithPhonetic}{笔记}{bi3 ji4}{10,5}{⽵、⾔}[HSK 2]
  \definition[篇,本,个]{s.}{notas; anotações feitas durante aulas, palestras e leituras | ensaios; esboços}
  \definition{v.}{tomar nota (por escrito)}
\end{EntryWithPhonetic}

\begin{EntryWithPhonetic}{笔记本}{bi3ji4ben3}{10,5,5}{⽵、⾔、⽊}[HSK 2]
  \definition[个,本]{s.}{caderno para anotações | \emph{laptop}; refere-se a um computador portátil}
  \definition{s.}{\emph{laptop}}
\end{EntryWithPhonetic}

\begin{EntryWithPhonetic}{笔试}{bi3 shi4}{10,8}{⽵、⾔}[HSK 6]
  \definition{s.}{exame escrito; um tipo de exame que exige respostas escritas; diferente de 口试}
  \seealsoref{口试}{kou3 shi4}
\end{EntryWithPhonetic}

\begin{EntryWithPhonetic}{鄙}{bi3}{13}{⾢}
  \definition*{s.}{Sobrenome Bi}
  \definition{adj.}{baixo; mesquinho; vulgar | rústico; básico; desprezível}
  \definition{pron.}{(auto-depreciativo) meu}
  \definition{s.}{Literário: um lugar remoto; cidade fronteiriça; cidade pequena}
  \definition{v.}{Literário: desprezar; desdenhar; menosprezar; olhar de cima para baixo}
\end{EntryWithPhonetic}

\begin{EntryWithPhonetic}{鄙视}{bi3shi4}{13,8}{⾢、⾒}[HSK 7-9]
  \definition{v.}{desprezar; desdenhar; menosprezar}
\end{EntryWithPhonetic}

\begin{EntryWithPhonetic}{必}{bi4}{5}{⼼}[HSK 5]
  \definition{adv.}{certamente; necessariamente; indica que algo é certo ou que alguém acredita que esteja correto | deve; tem que}
\end{EntryWithPhonetic}

\begin{EntryWithPhonetic}{必不可少}{bi4bu4ke3shao3}{5,4,5,4}{⼼、⼀、⼝、⼩}[HSK 7-9]
  \definition{suf.}{indispensável; essencial; absolutamente necessário; insubstituível; inevitável}
\end{EntryWithPhonetic}

\begin{EntryWithPhonetic}{必定}{bi4ding4}{5,8}{⼼、⼧}[HSK 7-9]
  \definition{adv.}{certamente; estar vinculado a; ter certeza de; para expressar a certeza de um julgamento ou inferência | definitivamente; para expressar determinação de vontade; ter certeza de fazê-lo}
\end{EntryWithPhonetic}

\begin{EntryWithPhonetic}{必将}{bi4 jiang1}{5,9}{⼼、⼨}[HSK 6]
  \definition{adv.}{certamente; certamente irá; usado para expressar inevitabilidade (ou necessidade)}
\end{EntryWithPhonetic}

\begin{EntryWithPhonetic}{必然}{bi4ran2}{5,12}{⼼、⽕}[HSK 3]
  \definition{adj.}{certo; inevitável; necessário; definido e inalterável; imutável}
  \definition{adv.}{inevitavelmente}
  \definition{s.}{necessidade; em filosofia, refere-se às leis objetivas do desenvolvimento que não são influenciadas pela vontade humana}
\end{EntryWithPhonetic}

\begin{EntryWithPhonetic}{必修}{bi4 xiu1}{5,9}{⼼、⼈}[HSK 6]
  \definition{adj.}{(de um curso acadêmico) obrigatório; compulsório; mandatório; obrigatório estudar de acordo com os regulamentos (em oposição a 选修)}
  \seealsoref{选修}{xuan3 xiu1}
\end{EntryWithPhonetic}

\begin{EntryWithPhonetic}{必须}{bi4xu1}{5,9}{⼼、⾴}[HSK 2]
  \definition{adv.}{necessariamente; obrigatoriamente; indica a necessidade lógica e emocional | deve; tem que; é obrigado a}
\end{EntryWithPhonetic}

\begin{EntryWithPhonetic}{必需}{bi4 xu1}{5,14}{⼼、⾬}[HSK 5]
  \definition{adj.}{essencial; indispensável}
  \definition{v.}{ser essencial; ser indispensável}
\end{EntryWithPhonetic}

\begin{EntryWithPhonetic}{必要}{bi4yao4}{5,9}{⼼、⾑}[HSK 3]
  \definition{adj.}{necessário; essencial; indispensável}
  \definition[个,些]{s.}{necessidade; características indispensáveis}
\end{EntryWithPhonetic}

\begin{EntryWithPhonetic}{毕}{bi4}{6}{⽐}
  \definition*{s.}{Bi, uma das mansões lunares; a décima nona das vinte e oito constelações em que a esfera celeste foi dividida, consistindo de oito estrelas, seis em Híades e duas em Touro | Sobrenome Bi}
  \definition{adv.}{tudo; completamente; totalmente}
  \definition{v.}{terminar; realizar; concluir  | completar; terminar}
\end{EntryWithPhonetic}

\begin{EntryWithPhonetic}{毕竟}{bi4jing4}{6,11}{⽐、⾳}[HSK 5]
  \definition{adv.}{afinal de contas; quando tudo estiver dito e feito; em última análise; indica um resultado que não pode ser alterado, enfatizando que se trata de uma causa ou fato que precisa ser enfocado para referência | significa 到底, 究竟, 终究, indicando a conclusão final alcançada}
  \seealsoref{到底}{dao4di3}
  \seealsoref{究竟}{jiu1jing4}
  \seealsoref{终究}{zhong1jiu1}
\end{EntryWithPhonetic}

\begin{EntryWithPhonetic}{毕业}{bi4/ye4}{6,5}{⽐、⼀}[HSK 4]
  \definition{v.+compl.}{formar-se}
\end{EntryWithPhonetic}

\begin{EntryWithPhonetic}{毕业生}{bi4 ye4 sheng1}{6,5,5}{⽐、⼀、⽣}[HSK 4]
  \definition[个,名,位,些]{s.}{diplomado; graduado; bacharel; pessoa que recebeu um diploma, grau ou certificado}
\end{EntryWithPhonetic}

\begin{EntryWithPhonetic}{闭}{bi4}{6}{⾨}[HSK 6]
  \definition*{s.}{Sobrenome Bi}
  \definition{v.}{fechar; encerrar | bloquear; obstruir; parar}
\end{EntryWithPhonetic}

\begin{EntryWithPhonetic}{闭幕}{bi4/mu4}{6,13}{⾨、⼱}[HSK 5]
  \definition{v.+compl.}{fechar; concluir; (conferência, exposição, etc.) terminar | cair a cortina; abaixar a cortina; terminar a apresentação e a cortina se fechar em frente ao palco}
\end{EntryWithPhonetic}

\begin{EntryWithPhonetic}{闭幕式}{bi4 mu4 shi4}{6,13,6}{⾨、⼱、⼷}[HSK 5]
  \definition{s.}{cerimônia de encerramento; cerimônia formal realizada no final de uma conferência ou exposição}
\end{EntryWithPhonetic}

\begin{EntryWithPhonetic}{闭嘴}{bi4zui3}{6,16}{⾨、⼝}
  \definition{expr.}{Cale-se!; Pare de falar!}
\end{EntryWithPhonetic}

\begin{EntryWithPhonetic}{秘}{bi4}{10}{⽲}
  \definition*{s.}{Abreviação de Peru, 秘鲁 | Sobrenome Bi}
  \seeref{mi4}
  \seealsoref{秘鲁}{bi4lu3}
\end{EntryWithPhonetic}

\begin{EntryWithPhonetic}{秘鲁}{bi4lu3}{10,12}{⽲、⿂}
  \definition*{s.}{Peru}
\end{EntryWithPhonetic}

\begin{EntryWithPhonetic}{弊}{bi4}{14}{⼶}
  \definition{s.}{fraude; abuso; negligência médica | desvantagem; falha; defeito; dano (oposto a 利) | trapaça; fraude, engano e falsificação}
  \seealsoref{利}{li4}
\end{EntryWithPhonetic}

\begin{EntryWithPhonetic}{弊病}{bi4bing4}{14,10}{⼶、⽧}[HSK 7-9]
  \definition{s.}{doença; mal; negligência | incinveniente; desvantagem; problemas com coisas}
\end{EntryWithPhonetic}

\begin{EntryWithPhonetic}{弊端}{bi4duan1}{14,14}{⼶、⽴}[HSK 7-9]
  \definition{s.}{abuso; negligência; prática corrupta; danos ao interesse público devido a uma lacuna no trabalho}
\end{EntryWithPhonetic}

\begin{EntryWithPhonetic}{碧}{bi4}{14}{⽯}
  \definition*{s.}{Sobrenome Bi}
  \definition{adj.}{verde claro | azul claro | azul; verde-azulado; esverdeado; azul-celeste. turquesa}
  \definition{s.}{Literário: jade verde | safira}
\end{EntryWithPhonetic}

\begin{EntryWithPhonetic}{碧绿}{bi4lv4}{14,11}{⽯、⽷}[HSK 7-9]
  \definition{adj.}{verde jade; verde esmeralda; descreve um verde muito brilhante e profundo}
\end{EntryWithPhonetic}

\begin{EntryWithPhonetic}{壁}{bi4}{16}{⼟}
  \definition*{s.}{Bi, a décima quarta das vinte e oito constelações em que a esfera celeste foi dividida, consistindo em duas estrelas em linha reta, uma em Pégaso e a outra em Andrômeda | A Estrela Bìxìu, uma das Vinte e Oito Mansões da astronomia tradicional chinesa}
  \definition[道]{s.}{parede | superfície plana como uma parede | penhasco | muralha; parapeito | barreira}
\end{EntryWithPhonetic}

\begin{EntryWithPhonetic}{壁虎}{bi4hu3}{16,8}{⼟、⾌}
  \definition{s.}{lagartixa}
\end{EntryWithPhonetic}

\begin{EntryWithPhonetic}{壁画}{bi4hua4}{16,8}{⼟、⽥}[HSK 7-9]
  \definition[幅]{s.}{mural; afresco; desenhos nas paredes ou tetos de edifícios}
\end{EntryWithPhonetic}

\begin{EntryWithPhonetic}{壁纸}{bi4zhi3}{16,7}{⼟、⽷}
  \definition{s.}{papel de parede; papel colado em paredes internas para decoração ou proteção, com diversos tipos e cores}
\end{EntryWithPhonetic}

\begin{EntryWithPhonetic}{避}{bi4}{16}{⾌}[HSK 4]
  \definition{v.}{evitar; evadir; esquivar-se; buscar abrigo; fugir | impedir; manter afastado; repelir; previnir}
\end{EntryWithPhonetic}

\begin{EntryWithPhonetic}{避免}{bi4mian3}{16,7}{⾌、⼉}[HSK 4]
  \definition{v.}{evitar; desviar; abster-se de; tentar não fazer com que algo aconteça; prevenir; tentar impedir (que algo ruim aconteça) com antecedência}
\end{EntryWithPhonetic}

\begin{EntryWithPhonetic}{避难}{bi4/nan4}{16,10}{⾌、⾫}[HSK 7-9]
  \definition{s.}{refúgio}
  \definition{v.+compl.}{refugiar-se; buscar asilo (político etc.)}
\end{EntryWithPhonetic}

\begin{EntryWithPhonetic}{避暑}{bi4/shu3}{16,12}{⾌、⽇}[HSK 7-9]
  \definition{v.+compl.}{ir de férias em um resort de verão; ir para um lugar fresco para evitar o calor do verão | prevenir insolação}
\end{EntryWithPhonetic}

\begin{EntryWithPhonetic}{边}{bian1}{5}{⾡}[HSK 2]
  \definition*{s.}{Sobrenome Bian}
  \definition{adv.}{dois ou mais 边 são usados separadamente antes de diferentes verbos, indicando que diferentes ações ocorrem simultaneamente}
  \definition[条,个]{s.}{lado (de uma figura geométrica) | borda; lado; margem; aba; rebordo | fronteira; limite | ao lado de; lugar próximo a; perto de um objeto; lateral | aro; aba; borda; decoração em forma de faixa incrustada ou pintada na borda de um objeto}
  \definition{suf.}{lado; anexado a palavras de localização monossilábicas, formando palavras de localização dissílabas}
  \seeref{bian5}
\end{EntryWithPhonetic}

\begin{EntryWithPhonetic}{边防}{bian1fang2}{5,6}{⾡、⾩}
  \definition{s.}{defesa da fronteira}
\end{EntryWithPhonetic}

\begin{EntryWithPhonetic}{边关}{bian1guan1}{5,6}{⾡、⼋}
  \definition{s.}{posto de fronteira | posição defensiva estratégica na fronteira}
\end{EntryWithPhonetic}

\begin{EntryWithPhonetic}{边疆}{bian1jiang1}{5,19}{⾡、⼸}[HSK 7-9]
  \definition{s.}{fronteira; área de fronteira; região de fronteira; território próximo à fronteira}
\end{EntryWithPhonetic}

\begin{EntryWithPhonetic}{边界}{bian1jie4}{5,9}{⾡、⽥}[HSK 7-9]
  \definition{s.}{limite; linha de fronteira; fronteiras entre países ou regiões}
\end{EntryWithPhonetic}

\begin{EntryWithPhonetic}{边境}{bian1jing4}{5,14}{⾡、⼟}[HSK 5]
  \definition{s.}{fronteira; borda; perto da fronteira}
\end{EntryWithPhonetic}

\begin{EntryWithPhonetic}{边缘}{bian1yuan2}{5,12}{⾡、⽷}[HSK 6]
  \definition{s.}{borda; beira; franja; uma área ou objeto próximo ao extremo |  borda; beira; algo está muito próximo de uma situação perigosa | interdisciplinar; relacionado a muitas coisas}
\end{EntryWithPhonetic}

\begin{EntryWithPhonetic}{边远}{bian1yuan3}{5,7}{⾡、⾡}[HSK 7-9]
  \definition{adj.}{longe do centro; remoto; periférico; perto da fronteira; longe da área central}
\end{EntryWithPhonetic}

\begin{EntryWithPhonetic}{编}{bian1}{12}{⽷}[HSK 4]
  \definition*{s.}{Sobrenome Bian}
  \definition{s.}{livro; volume; parte de um livro | organização e pessoal; estabelecimento}
  \definition{v.}{tecer; trançar; entrançar | fazer uma lista; organizar em uma lista; organizar; agrupar | editar; compilar | compor; escrever | fabricar; inventar; fazer; preparar}
\end{EntryWithPhonetic}

\begin{EntryWithPhonetic}{编程}{bian1cheng2}{12,12}{⽷、⽲}
  \definition{v.}{programar computador}
\end{EntryWithPhonetic}

\begin{EntryWithPhonetic}{编号}{bian1hao4}{12,5}{⽷、⼝}[HSK 7-9]
  \definition{s.}{número de série; números dados em sequência}
  \definition{v.}{numerar; dar números às pessoas ou coisas em ordem}
\end{EntryWithPhonetic}

\begin{EntryWithPhonetic}{编辑}{bian1ji2}{12,13}{⽷、⾞}[HSK 5]
  \definition[名,位,个]{s.}{editor; compilador; uma pessoa que organiza e processa dados ou trabalhos existentes}
  \definition{v.}{editar; compilar; organizar e processar dados ou trabalhos existentes}
  \seeref{bian1ji5}
\end{EntryWithPhonetic}

\begin{EntryWithPhonetic}{编辑}{bian1ji5}{12,13}{⽷、⾞}[HSK 5]
  \definition{s.}{editor; compilador; pessoa que organiza e processa dados ou trabalhos existentes}
  \seeref{bian1ji2}
\end{EntryWithPhonetic}

\begin{EntryWithPhonetic}{编剧}{bian1ju4}{12,10}{⽷、⼑}[HSK 7-9]
  \definition[个,位,名]{s.}{dramaturgo | roteirista}
  \definition{v.}{escrever uma peça, um roteiro, etc.}
\end{EntryWithPhonetic}

\begin{EntryWithPhonetic}{编排}{bian1pai2}{12,11}{⽷、⼿}[HSK 7-9]
  \definition{v.}{dispor; organizar em uma determinada ordem | escrever uma peça e ensaiá-la}
\end{EntryWithPhonetic}

\begin{EntryWithPhonetic}{编写}{bian1xie3}{12,5}{⽷、⼍}[HSK 7-9]
  \definition{v.}{compilar; organizar materiais existentes em um livro ou artigo | escrever; compor; criar | oletar informações e organizá-las ou criar algo}
\end{EntryWithPhonetic}

\begin{EntryWithPhonetic}{编造}{bian1zao4}{12,10}{⽷、⾡}[HSK 7-9]
  \definition{v.}{compilar; compor; preparar | fabricar; inventar; criar; cozinhar | criar a partir da imaginação}
\end{EntryWithPhonetic}

\begin{EntryWithPhonetic}{编制}{bian1 zhi4}{12,8}{⽷、⼑}[HSK 6]
  \definition{s.}{estabelecimento; organização e pessoal; refere-se à estrutura organizacional de uma unidade, cotas de pessoal, alocação de tarefas, etc.}
  \definition{v.}{tecer; trançar; entrelaçar tiras de vime, salgueiro, bambu, etc. para fazer objetos | resolver; realizar; elaborar; fazer de acordo com os dados (procedimentos, planos, etc.)}
\end{EntryWithPhonetic}

\begin{EntryWithPhonetic}{邉}{bian1}{17}{⾡}
  \variantof{边}
\end{EntryWithPhonetic}

\begin{EntryWithPhonetic}{鞭}{bian1}{18}{⾰}
  \definition[条]{s.}{chicote; oçoite; chibata | um bastão de ferro usado como arma na China antiga | algo parecido com um chicote | uma série de pequenos fogos de artifício | pênis de animal; refere-se ao pênis de certos mamíferos usado para fins medicinais ou comestíveis}
  \definition{v.}{açoitar; chicotear; flagelar}
\end{EntryWithPhonetic}

\begin{EntryWithPhonetic}{鞭策}{bian1ce4}{18,12}{⾰、⽵}[HSK 7-9]
  \definition{v.}{estimular; incitar; incentivar}
\end{EntryWithPhonetic}

\begin{EntryWithPhonetic}{鞭炮}{bian1pao4}{18,9}{⾰、⽕}[HSK 7-9]
  \definition[串,挂,盒,捆,箱,个]{s.}{\emph{maroon}, um tipo de foguete usado como alarme ou aviso; fogos de artifício; um termo geral para fogos de artifício grandes e pequenos}
\end{EntryWithPhonetic}

\begin{EntryWithPhonetic}{贬}{bian3}{8}{⾙}
  \definition{adj.}{depreciativo; derrogativo; rebaixante}
  \definition{v.}{(nos tempos antigos) rebaixar de posição; (nos tempos modernos) diminuir de valor | reduzir valor; desvalorizar | censurar; menosprezar; depreciar; dar uma avaliação ruim | degradar; rebaixar; relegar}
\end{EntryWithPhonetic}

\begin{EntryWithPhonetic}{贬值}{bian3zhi2}{8,10}{⾙、⼈}[HSK 7-9]
  \definition{v.}{depreciar; tornar-se desvalorizado; refere-se à diminuição do poder de compra do dinheiro | depreciar; geralmente se refere à diminuição do valor de algo | desvalorizar; reduzir o teor de ouro da moeda de um país ou reduzir a taxa de câmbio da moeda de um país em relação às moedas estrangeiras}
\end{EntryWithPhonetic}

\begin{EntryWithPhonetic}{扁}{bian3}{9}{⼾}[HSK 6]
  \definition{adj.}{plano}
  \definition{v.}{(coloquial)  bater em alguém}
  \seeref{pian1}
\end{EntryWithPhonetic}

\begin{EntryWithPhonetic}{变}{bian4}{8}{⼜}[HSK 2]
  \definition{adj.}{alterado; mutável; que pode mudar; que está mudando ou já mudou}
  \definition{s.}{uma reviravolta inesperada nos acontecimentos; mudanças significativas repentinas}
  \definition{v.}{mudar; tornar-se diferente; fazer mudanças | tornar-se; transformar-se; natureza, estado ou situação diferentes dos originais | alterar; mudar; transformar}
\end{EntryWithPhonetic}

\begin{EntryWithPhonetic}{变成}{bian4 cheng2}{8,6}{⼜、⼽}[HSK 2]
  \definition{v.}{crescer; tornar-se; fazer; desenvolver-se; revelar-se; resultar; acontecer; passar a ser; passar para; acumular-se; converter-se; transformar-se; transformar-se em; mudar-se em; transformação da situação ou condição anterior para a situação ou condição atual}
\end{EntryWithPhonetic}

\begin{EntryWithPhonetic}{变动}{bian4 dong4}{8,6}{⼜、⼒}[HSK 5]
  \definition{s.}{mudança; alteração; oscilação; modificação; variação}
  \definition{v.}{mudar; alterar; oscilar; modificar; variar}
\end{EntryWithPhonetic}

\begin{EntryWithPhonetic}{变革}{bian4ge2}{8,9}{⼜、⾰}[HSK 7-9]
  \definition{s.}{mudança; transformação; a natureza das coisas foi reformada}
  \definition{v.}{transformar; mudar (de sistemas sociais, políticas, etc.); mudar o antigo e inovar; mudar a essência das coisas (referindo-se principalmente aos sistemas sociais)}
\end{EntryWithPhonetic}

\begin{EntryWithPhonetic}{变更}{bian4 geng1}{8,7}{⼜、⽈}[HSK 6]
  \definition{v.}{alterar; mudar; modificar}
\end{EntryWithPhonetic}

\begin{EntryWithPhonetic}{变化}{bian4hua4}{8,4}{⼜、⼔}[HSK 3]
  \definition[个]{s.}{mudança; variação; a nova situação após uma mudança em pessoas ou coisas}
  \definition{v.}{mudar;  variar}
\end{EntryWithPhonetic}

\begin{EntryWithPhonetic}{变幻莫测}{bian4huan4-mo4ce4}{8,4,10,9}{⼜、⼳、⾋、⽔}[HSK 7-9]
  \definition{expr.}{mutável; imprevisível | errático | mudar imprevisivelmente | traiçoeiro}
\end{EntryWithPhonetic}

\begin{EntryWithPhonetic}{变换}{bian4 huan4}{8,10}{⼜、⼿}[HSK 6]
  \definition{v.}{variar; alternar; mudar a forma ou o conteúdo de algo de uma coisa para outra}
\end{EntryWithPhonetic}

\begin{EntryWithPhonetic}{变节}{bian4jie2}{8,5}{⼜、⾋}
  \definition{s.}{traição | deserção | vira-casaca}
  \definition{v.}{retratar-se e submeter-se; renunciar e render-se | mudar de lado politicamente}
\end{EntryWithPhonetic}

\begin{EntryWithPhonetic}{变迁}{bian4qian1}{8,6}{⼜、⾡}[HSK 7-9]
  \definition{s.}{mudanças; transição; vicissitudes; mudança em tendências ou condições; mudança de situação ou estágio}
\end{EntryWithPhonetic}

\begin{EntryWithPhonetic}{变数}{bian4shu4}{8,13}{⼜、⽁}
  \definition{s.}{(matemática) variável | fatores variáveis}
\end{EntryWithPhonetic}

\begin{EntryWithPhonetic}{变为}{bian4 wei2}{8,4}{⼜、⼂}[HSK 3]
  \definition{v.}{transformar-se em; tornar-se | mudar para}
\end{EntryWithPhonetic}

\begin{EntryWithPhonetic}{变心}{bian4/xin1}{8,4}{⼜、⼼}
  \definition{v.+compl.}{deixar de ser fiel}
\end{EntryWithPhonetic}

\begin{EntryWithPhonetic}{变形}{bian4/xing2}{8,7}{⼜、⼺}[HSK 6]
  \definition{v.+compl.}{deformar; ficar fora de forma | transformar; transformar-se em outras formas}
\end{EntryWithPhonetic}

\begin{EntryWithPhonetic}{变性}{bian4xing4}{8,8}{⼜、⼼}
  \definition{s.}{desnaturação | transexual}
  \definition{v.}{desnaturar | mudar de sexo}
\end{EntryWithPhonetic}

\begin{EntryWithPhonetic}{变异}{bian4yi4}{8,6}{⼜、⼶}[HSK 7-9]
  \definition{s.}{variação; mutação; muta; diferenças nas características morfológicas e fisiológicas entre gerações da mesma espécie ou entre indivíduos da mesma geração}
  \definition{v.}{variar; mudar}
\end{EntryWithPhonetic}

\begin{EntryWithPhonetic}{变质}{bian4/zhi4}{8,8}{⼜、⾙}[HSK 7-9]
  \definition{v.+compl.}{deteriorar-se; estragar-se | tornar-se moralmente degenerado}
\end{EntryWithPhonetic}

\begin{EntryWithPhonetic}{变装}{bian4zhuang1}{8,12}{⼜、⾐}
  \definition{v.}{trocar de roupa | vestir-se | vestir uma fantasia | disfarçar-se ou fantasiar-se de personagem real ou ficcional, \emph{cosplay} | travestir-se}
\end{EntryWithPhonetic}

\begin{EntryWithPhonetic}{便}{bian4}{9}{⼈}[HSK 6]
  \definition{adj.}{prático; conveniente | simples; comum; informal}
  \definition{adv.}{então; apenas no caso de; mesmo significado e uso de 就}
  \definition{conj.}{mesmo que; expressa uma concessão hipotética}
  \definition{s.}{facilidade; conveniência; o momento certo; a oportunidade | fezes ou urina}
  \definition{v.}{aliviar-se; excretar fezes e urina}
  \seeref{pian2}
  \seealsoref{就}{jiu4}
\end{EntryWithPhonetic}

\begin{EntryWithPhonetic}{便道}{bian4dao4}{9,12}{⼈、⾡}[HSK 7-9]
  \definition[条]{s.}{atalho | pavimento; calçada | estrada improvisada; estrada temporária}
\end{EntryWithPhonetic}

\begin{EntryWithPhonetic}{便饭}{bian4fan4}{9,7}{⼈、⾷}[HSK 7-9]
  \definition[顿]{s.}{refeição comum; refeição simples}
  \definition{v.}{fazer uma refeição leve}
\end{EntryWithPhonetic}

\begin{EntryWithPhonetic}{便函}{bian4han2}{9,8}{⼈、⼐}
  \definition{s.}{carta informal enviada por uma organização (oposto a 公函)}
  \seealsoref{公函}{gong1han2}
\end{EntryWithPhonetic}

\begin{EntryWithPhonetic}{便捷}{bian4jie2}{9,11}{⼈、⼿}[HSK 7-9]
  \definition{adj.}{fácil; conveniente; direto e simples; direto e conveniente | ligeiro; ágil}
\end{EntryWithPhonetic}

\begin{EntryWithPhonetic}{便利}{bian4li4}{9,7}{⼈、⼑}[HSK 5]
  \definition{adj.}{fácil; conveniente}
  \definition{s.}{facilidade; conveniência; coisas ou condições convenientes}
  \definition{v.}{facilitar; fornecer ajuda para que os outros se sintam confortáveis}
\end{EntryWithPhonetic}

\begin{EntryWithPhonetic}{便利店}{bian4li4dian4}{9,7,8}{⼈、⼑、⼴}[HSK 7-9]
  \definition[个,家]{s.}{loja de conveniência; pequenas lojas de varejo localizadas em áreas residenciais para a conveniência dos moradores}
\end{EntryWithPhonetic}

\begin{EntryWithPhonetic}{便是}{bian4 shi4}{9,9}{⼈、⽇}[HSK 6]
  \definition{adv.}{exatamente; precisamente; para expressar afirmação ou ênfase}
  \definition{conj.}{mesmo; mesmo que; usado para introduzir um caso extremo hipotético, enfatizando que o mesmo resultado ocorreria em circunstâncias tão extremas, sem mencionar circunstâncias normais; você também pode usar 即便是}
  \definition{part.}{usada no final de uma frase para expressar afirmação}
  \seealsoref{即便是}{ji2bian4 shi4}
\end{EntryWithPhonetic}

\begin{EntryWithPhonetic}{便条}{bian4tiao2}{9,7}{⼈、⽊}[HSK 5]
  \definition[张,个]{s.}{nota ou mensagem informal; geralmente uma mensagem ou notificação}
\end{EntryWithPhonetic}

\begin{EntryWithPhonetic}{便宜}{bian4yi2}{9,8}{⼈、⼧}
  \definition{adj.}{prático; conveniente; adequado}
  \seeref{pian2yi5}
\end{EntryWithPhonetic}

\begin{EntryWithPhonetic}{便于}{bian4yu2}{9,3}{⼈、⼆}[HSK 5]
  \definition{v.}{ser fácil para; ser conveniente para (algo ou fazer algo)}
\end{EntryWithPhonetic}

\begin{EntryWithPhonetic}{遍}{bian4}{12}{⾡}[HSK 2]
  \definition{adv.}{por toda parte; em toda parte; em todos os lugares}
  \definition{clas.}{usado para a repetição de ações de leitura, fala ou escrita}
\end{EntryWithPhonetic}

\begin{EntryWithPhonetic}{遍布}{bian4bu4}{12,5}{⾡、⼱}[HSK 7-9]
  \definition{v.}{encontrar em todos os lugares; espalhar por toda parte; distribuir em todos os lugares}
\end{EntryWithPhonetic}

\begin{EntryWithPhonetic}{遍地}{bian4 di4}{12,6}{⾡、⼟}[HSK 6]
  \definition{adv.}{em todos os lugares; em toda parte; por toda parte}
\end{EntryWithPhonetic}

\begin{EntryWithPhonetic}{辨}{bian4}{16}{⾟}[HSK 7-9]
  \definition{v.}{diferenciar; distinguir; discriminar | reconhecer; distinguir; identificar; discernir}
\end{EntryWithPhonetic}

\begin{EntryWithPhonetic}{辨别}{bian4bie2}{16,7}{⾟、⼑}[HSK 7-9]
  \definition{v.}{diferenciar; distinguir; discriminar; encontrar características de diferentes coisas e diferenciá-las}
\end{EntryWithPhonetic}

\begin{EntryWithPhonetic}{辨认}{bian4ren4}{16,4}{⾟、⾔}[HSK 7-9]
  \definition{v.}{identificar; reconhecer; identificar e julgar com base em características para encontrar ou identificar um objeto}
\end{EntryWithPhonetic}

\begin{EntryWithPhonetic}{辩}{bian4}{16}{⾟}
  \definition{v.}{argumentar; disputar; debater}
\end{EntryWithPhonetic}

\begin{EntryWithPhonetic}{辩护}{bian4hu4}{16,7}{⾟、⼿}[HSK 7-9]
  \definition{v.}{pleitear; defender | defender; argumentar em favor de}
\end{EntryWithPhonetic}

\begin{EntryWithPhonetic}{辩解}{bian4jie3}{16,13}{⾟、⾓}[HSK 7-9]
  \definition{v.}{fornecer uma explicação; tentar se defender; explicar uma visão ou comportamento criticado; eliminar a crítica ou reduzir sua gravidade}
\end{EntryWithPhonetic}

\begin{EntryWithPhonetic}{辩论}{bian4lun4}{16,6}{⾟、⾔}[HSK 4]
  \definition{v.}{debater; obter um entendimento unificado ou correto, ambos os lados usam linguagem, palavras etc. para explicar seus pontos de vista, apontar os erros ou as contradições do outro lado}
\end{EntryWithPhonetic}

\begin{EntryWithPhonetic}{辫}{bian4}{17}{⾟}
  \definition{s.}{trança; rabo de cavalo | para coisas como uma trança}
\end{EntryWithPhonetic}

\begin{EntryWithPhonetic}{辫子}{bian4zi5}{17,3}{⾟、⼦}[HSK 7-9]
  \definition[条,根,种]{s.}{trança; rabo de cavalo; uma mecha de cabelo presa reta ou trançada em seções | Metafórico: erro; deficiência; fraqueza | coisas parecidas com tranças}
\end{EntryWithPhonetic}

\begin{EntryWithPhonetic}{边}{bian5}{5}{⾡}
  \definition{suf.}{sufixo de uma palavra de localidade (lado); indica posição e direção, usado após palavras que indicam direção, como 上, 下, 前, 后, 左, 右}
  \seeref{bian1}
\end{EntryWithPhonetic}

\begin{EntryWithPhonetic}{标}{biao1}{9}{⽊}[HSK 7-9]
  \definition{clas.}{usado para equipes (o numeral é limitado a um, 一, o que é comum no chinês moderno)}
  \definition[个]{s.}{copa da árvore (significado original) | marca; sinal | padrão; cota | sinal externo; sintoma | prêmio; troféu | oferta; licitação comercial pública | a ponta de uma árvore | aparência externa; ramos ou superfícies | partes aéreas das plantas | rótulo; etiqueta; identificação; sinal | regimento na Dinastia Qing; uma das organizações militares no final da Dinastia Qing}
  \definition{v.}{colocar uma marca, etiqueta ou rótulo em; rotular | agrupar; formar equipe | marcar; expressar com palavras ou outras coisas |}
\end{EntryWithPhonetic}

\begin{EntryWithPhonetic}{标榜}{biao1bang3}{9,14}{⽊、⽊}[HSK 7-9]
  \definition{v.}{ostentar; anunciar; desfilar | elogiar; elogiar excessivamente | dar publicidade favorável a; fazer uma exibição de; gabar-se | impulsionar; elogiar excessivamente; gabar-se de}
\end{EntryWithPhonetic}

\begin{EntryWithPhonetic}{标本}{biao1ben3}{9,5}{⽊、⽊}[HSK 7-9]
  \definition{s.}{espécime; amostra | Medicina chinesa: causa raiz e sintomas de uma doença}
\end{EntryWithPhonetic}

\begin{EntryWithPhonetic}{标签}{biao1qian1}{9,13}{⽊、⽵}[HSK 7-9]
  \definition[个,张,枚,套]{s.}{rótulo; etiqueta; um pedaço de papel anexado ou amarrado a um item para indicar o nome do produto, finalidade, preço, etc.}
\end{EntryWithPhonetic}

\begin{EntryWithPhonetic}{标示}{biao1shi4}{9,5}{⽊、⽰}[HSK 7-9]
  \definition{v.}{marcar; indicar | indicar; exibir como texto ou gráfico}
\end{EntryWithPhonetic}

\begin{EntryWithPhonetic}{标题}{biao1ti2}{9,15}{⽊、⾴}[HSK 3]
  \definition[个,条,篇]{s.}{título; manchete; cabeçalho; resumo conciso do conteúdo da obra}
\end{EntryWithPhonetic}

\begin{EntryWithPhonetic}{标语}{biao1yu3}{9,9}{⽊、⾔}[HSK 7-9]
  \definition[幅,张,条,个]{s.}{\emph{slogan}; cartaz; \emph{slogans} curtos de propaganda afixados ou pendurados em locais públicos}
\end{EntryWithPhonetic}

\begin{EntryWithPhonetic}{标志}{biao1zhi4}{9,7}{⽊、⼼}[HSK 4]
  \definition[个,种]{s.}{sinal; marca; logotipo; símbolo; emblema; marcações que caracterizam um objeto para facilitar a identificação}
  \definition{v.}{marcar; indicar; simbolizar; identificar}
\end{EntryWithPhonetic}

\begin{EntryWithPhonetic}{标致}{biao1zhi4}{9,10}{⽊、⾄}
  \definition*{s.}{Peugeot, montadora de automóveis}
  \definition{adj.}{bela aparência e postura (principalmente para mulheres)}
  \seeref{biao1zhi5}
\end{EntryWithPhonetic}

\begin{EntryWithPhonetic}{标致}{biao1zhi5}{9,10}{⽊、⾄}[HSK 7-9]
  \definition{adj.}{bonita (mulher)}
\end{EntryWithPhonetic}

\begin{EntryWithPhonetic}{标准}{biao1zhun3}{9,10}{⽊、⼎}[HSK 3]
  \definition{adj.}{padrão (que serve como ou está em conformidade com um padrão); em conformidade com os documentos e princípios regulamentares}
  \definition[个,条,项,种]{s.}{padrão; critério; critérios de avaliação das coisas}
\end{EntryWithPhonetic}

\begin{EntryWithPhonetic}{髟}{biao1}{10}{⾽}[Kangxi 190]
  \definition{adj.}{(de cabelo) solto, caído}
\end{EntryWithPhonetic}

\begin{EntryWithPhonetic}{彪}{biao1}{11}{⾌}
  \definition*{s.}{Sobrenome Biao}
  \definition{adj.}{semelhante a um tigre (metáfora para estatura alta)}
  \definition{s.}{tigre jovem}
\end{EntryWithPhonetic}

\begin{EntryWithPhonetic}{镖}{biao1}{16}{⾦}
  \definition{s.}{dardo | arma de arremesso | mercadorias enviadas sob a proteção de uma escolta armada}
\end{EntryWithPhonetic}

\begin{EntryWithPhonetic}{飙}{biao1}{16}{⾵}
  \definition{s.}{tempestade; furacão; redemoinho | Literário: vento violento; redemoinho}
\end{EntryWithPhonetic}

\begin{EntryWithPhonetic}{飙升}{biao1sheng1}{16,4}{⾵、⼗}[HSK 7-9]
  \definition{v.}{disparar; subir rapidamente; (preço, quantidade, etc.) aumentam rapidamente}
\end{EntryWithPhonetic}

\begin{EntryWithPhonetic}{表}{biao3}{8}{⾐}[HSK 2]
  \definition*{s.}{Sobrenome Biao}
  \definition{s.}{exterior; superfície; externo | a relação entre os filhos ou netos de um irmão e uma irmã ou de irmãs | modelo; exemplo; padrão | memorial a um imperador; um tipo de petição da antiguidade, frequentemente usado para expressar intenções; mais tarde, também usado para expressar opiniões sobre eventos importantes | formulário; lista; gráfico; tabela | medidor; instrumento para medir uma determinada quantidade | relógio; um dispositivo para medir o tempo, menor que um relógio, que geralmente pode ser carregado no bolso | medidor de luz solar; antiga vara de madeira para medir o tempo através da sombra do sol | coluna usada antigamente para marcação}
  \definition{v.}{mostrar; expressar; expressar ideias, pensamentos, sentimentos, etc. | administrar medicamentos para aliviar o resfriado; na medicina tradicional chinesa refere-se ao uso de medicamentos para dissipar o frio e o vento que afetam o corpo}
\end{EntryWithPhonetic}

\begin{EntryWithPhonetic}{表白}{biao3bai2}{8,5}{⾐、⽩}[HSK 7-9]
  \definition{v.}{justificar; explicar-se; expressar ou declarar claramente; explicar (as próprias intenções) aos outros}
\end{EntryWithPhonetic}

\begin{EntryWithPhonetic}{表达}{biao3da2}{8,6}{⾐、⾡}[HSK 3]
  \definition{v.}{entregar; expressar; mostrar; manifestar; transmitir; comunicar; refere-se ao processo de transmitir pensamentos, sentimentos ou opiniões pessoais a outras pessoas por meio de linguagem, texto, ações, etc.}
\end{EntryWithPhonetic}

\begin{EntryWithPhonetic}{表格}{biao3ge2}{8,10}{⾐、⽊}[HSK 3]
  \definition[张,份,个]{s.}{tabela; formulário}
\end{EntryWithPhonetic}

\begin{EntryWithPhonetic}{表决}{biao3jue2}{8,6}{⾐、⼎}[HSK 7-9]
  \definition{v.}{votar; colocar em votação; decidir por votação}
\end{EntryWithPhonetic}

\begin{EntryWithPhonetic}{表面}{biao3mian4}{8,9}{⾐、⾯}[HSK 3]
  \definition{s.}{superfície; face; exterior; aparência | aparência; superficialidade | mostrador (placa); mostrador do relógio | aparência; a aparência externa das coisas ou a parte não essencial delas}
\end{EntryWithPhonetic}

\begin{EntryWithPhonetic}{表面上}{biao3 mian4 shang4}{8,9,3}{⾐、⾯、⼀}[HSK 6]
  \definition{adj.}{superficial; ostensivo; aparente}
\end{EntryWithPhonetic}

\begin{EntryWithPhonetic}{表明}{biao3ming2}{8,8}{⾐、⽇}[HSK 3]
  \definition{v.}{indicar; demonstrar; expressar; marcar; expressar claramente; expressar de forma clara}
\end{EntryWithPhonetic}

\begin{EntryWithPhonetic}{表情}{biao3qing2}{8,11}{⾐、⼼}[HSK 4]
  \definition[个,种,幅]{s.}{expressão; expressão facial; expressão de pensamentos e sentimentos internos por meio de mudanças faciais ou de gestos}
  \definition{v.}{expressar pensamentos e sentimentos internos por meio de mudanças faciais ou de gestos}
\end{EntryWithPhonetic}

\begin{EntryWithPhonetic}{表示}{biao3shi4}{8,5}{⾐、⽰}[HSK 2]
  \definition{s.}{expressão; indicação}
  \definition{v.}{mostrar; expressar; indicar | significar | expressar pensamentos e sentimentos através de palavras, ações ou expressões faciais}
\end{EntryWithPhonetic}

\begin{EntryWithPhonetic}{表述}{biao3shu4}{8,8}{⾐、⾡}[HSK 7-9]
  \definition{s.}{formulação; expressão}
  \definition{v.}{declarar; explicar; descrever em palavras ou texto}
\end{EntryWithPhonetic}

\begin{EntryWithPhonetic}{表率}{biao3shuai4}{8,11}{⾐、⽞}[HSK 7-9]
  \definition{s.}{modelo; exemplo}
\end{EntryWithPhonetic}

\begin{EntryWithPhonetic}{表态}{biao3/tai4}{8,8}{⾐、⼼}[HSK 7-9]
  \definition{v.+compl.}{tornar conhecida a sua posição; declarar onde se posiciona; comprometer-se; expressar claramente a atitude de alguém em relação a algo}
\end{EntryWithPhonetic}

\begin{EntryWithPhonetic}{表现}{biao3xian4}{8,8}{⾐、⾒}[HSK 3]
  \definition[个,种,份]{s.}{desempenho; expressão; manifestação; comportamento; as ideias, o estilo, as qualidades, o nível ou as capacidades demonstrados em ação.}
  \definition{v.}{mostrar; expressar; exibir; manifestar; descrever; demonstrar algum tipo de pensamento, espírito, qualidade, sentimento ou habilidade, etc. | exibir-se; demonstrar de forma inadequada e intencional alguma habilidade, ponto forte ou vantagem.}
\end{EntryWithPhonetic}

\begin{EntryWithPhonetic}{表演}{biao3yan3}{8,14}{⾐、⽔}[HSK 3]
  \definition[场]{s.}{performance; exposição; refere-se às atividades expressas pelos atores por meio da linguagem, voz, expressões faciais, instrumentos musicais ou movimentos}
  \definition{v.}{atuar; representar; interpretar | demonstrar; fazer demonstrações | fingir; agir de forma afetada; metáfora para fingir deliberadamente uma determinada atitude para enganar alguém}
\end{EntryWithPhonetic}

\begin{EntryWithPhonetic}{表演赛}{biao3yan3sai4}{8,14,14}{⾐、⽔、⾙}
  \definition{s.}{partida de exibição; jogo de exibição; uma competição realizada para celebração, comemoração, demonstração, publicidade, etc.}
\end{EntryWithPhonetic}

\begin{EntryWithPhonetic}{表演特技}{biao3yan3 te4ji4}{8,14,10,7}{⾐、⽔、⽜、⼿}
  \definition{s.}{acrobacia | pirueta | façanha}
\end{EntryWithPhonetic}

\begin{EntryWithPhonetic}{表演艺术}{biao3yan3 yi4shu4}{8,14,4,5}{⾐、⽔、⾋、⽊}
  \definition{s.}{arte performática}
\end{EntryWithPhonetic}

\begin{EntryWithPhonetic}{表演游戏}{biao3yan3 you2xi4}{8,14,12,6}{⾐、⽔、⽔、⼽}
  \definition{s.}{exibição dramática}
\end{EntryWithPhonetic}

\begin{EntryWithPhonetic}{表演者}{biao3yan3 zhe3}{8,14,8}{⾐、⽔、⽼}
  \definition{s.}{artista; intérprete}
\end{EntryWithPhonetic}

\begin{EntryWithPhonetic}{表扬}{biao3yang2}{8,6}{⾐、⼿}[HSK 4]
  \definition{v.}{elogiar; louvar; elogiar publicamente as pessoas boas e as boas ações}
\end{EntryWithPhonetic}

\begin{EntryWithPhonetic}{表扬信}{biao3yang2 xin4}{8,6,9}{⾐、⼿、⼈}
  \definition{s.}{carta de elogio; depoimento}
\end{EntryWithPhonetic}

\begin{EntryWithPhonetic}{表彰}{biao3zhang1}{8,14}{⾐、⼺}[HSK 7-9]
  \definition{v.}{citar; honrar; elogiar}
\end{EntryWithPhonetic}

\begin{EntryWithPhonetic}{憋}{bie1}{15}{⼼}[HSK 7-9]
  \definition{adj.}{sufocado; oprimido}
  \definition{v.}{suprimir; conter | Dialeto: obrigar | Dialeto: ponderar; contemplar | Dialeto: ficar de olho em | Dialeto: destruir (por pressão interna) | calar a boca; inibir; bloquear | sufocar; abafar}
\end{EntryWithPhonetic}

\begin{EntryWithPhonetic}{别}{bie2}{7}{⼑}[HSK 1,4]
  \definition*{s.}{Sobrenome Bie}
  \definition{adv.}{não; nada de (pedir a alguém para não fazer); é melhor não | talvez, usado em conjunto com a palavra 是 para indicar especulação}
  \definition{pron.}{outro; algum outro}
  \definition{s.}{distinção; diferença | classificação}
  \definition{v.}{deixar; partir; separar | diferenciar; distinguir; encontrar aspectos diferentes | fixar objetos com pinos | girar; virar | aderir; colar; preder}
  \seeref{bie4}
  \seealsoref{是}{shi4}
\end{EntryWithPhonetic}

\begin{EntryWithPhonetic}{别不过}{bie2 bu2guo4}{7,4,6}{⼑、⼀、⾡}
  \definition{expr.}{Não, mas}
\end{EntryWithPhonetic}

\begin{EntryWithPhonetic}{别的}{bie2 de5}{7,8}{⼑、⽩}[HSK 1]
  \definition{pron.}{outro; o resto}
\end{EntryWithPhonetic}

\begin{EntryWithPhonetic}{别具匠心}{bie2ju4-jiang4xin1}{7,8,6,4}{⼑、⼋、⼕、⼼}[HSK 7-9]
  \definition{expr.}{único e engenhoso | mostrar engenhosidade; ter originalidade}
\end{EntryWithPhonetic}

\begin{EntryWithPhonetic}{别看}{bie2kan4}{7,9}{⼑、⽬}[HSK 7-9]
  \definition{conj.}{apesar de}
\end{EntryWithPhonetic}

\begin{EntryWithPhonetic}{别人}{bie2 ren2}{7,2}{⼑、⼈}[HSK 1]
  \definition{pron.}{outros; outras pessoas}
  \definition{s.}{outros; pessoas; outras pessoas; refere-se a alguém diferente de si mesmo}
\end{EntryWithPhonetic}

\begin{EntryWithPhonetic}{别墅}{bie2shu4}{7,14}{⼑、⼟}[HSK 7-9]
  \definition[栋,幢,座,套,个]{s.}{vila; mansão; residência de campo}
\end{EntryWithPhonetic}

\begin{EntryWithPhonetic}{别说}{bie2shuo1}{7,9}{⼑、⾔}[HSK 7-9]
  \definition{v.}{Coloquial: não dizer nada de; não mencionar; deixar sozinho, usado no início de uma frase para reconhecer a declaração seguinte}[别说在下雨, 你现在出去也太晚了。===Não se preocupe com a chuva, já é tarde demais para você sair.]
\end{EntryWithPhonetic}

\begin{EntryWithPhonetic}{别提了}{bie2ti2 le5}{7,12,2}{⼑、⼿、⼅}[HSK 7-9]
  \definition{v.}{não mencionar mais isso, não falar mais sobre isso}
\end{EntryWithPhonetic}

\begin{EntryWithPhonetic}{别致}{bie2zhi4}{7,10}{⼑、⾄}[HSK 7-9]
  \definition{adj.}{único; não convencional; novo, diferente do comum}
\end{EntryWithPhonetic}

\begin{EntryWithPhonetic}{别}{bie4}{7}{⼑}
  \definition{v.}{fazer com que alguém mude seus hábitos, opiniões, etc. | mudar a opinião de alguém (usado principalmente em 别不过)}
  \seeref{bie2}
  \seealsoref{别不过}{bie2 bu2guo4}
\end{EntryWithPhonetic}

\begin{EntryWithPhonetic}{别扭}{bie4niu5}{7,7}{⼑、⼿}[HSK 7-9]
  \definition{adj.}{estranho; desconfortável | desajeitado (ao falar ou escrever); não suave; não fluente | nervoso e desconfortável; muito nervoso; muito antinatural}
  \definition{v.}{ser difícil com alguém; relacionamentos ruins, opiniões diferentes, conflitos frequentes; falta de coordenação}
\end{EntryWithPhonetic}

\begin{EntryWithPhonetic}{宾}{bin1}{10}{⼧}
  \definition*{s.}{Sobrenome Bin}
  \definition[个,位,名,些]{s.}{convidado}
\end{EntryWithPhonetic}

\begin{EntryWithPhonetic}{宾馆}{bin1guan3}{10,11}{⼧、⾷}[HSK 5]
  \definition[家,个,座]{s.}{hotel; acomodações públicas para hóspedes}
\end{EntryWithPhonetic}

\begin{EntryWithPhonetic}{彬}{bin1}{11}{⼺}
  \definition*{s.}{Sobrenome Bin}
  \definition{adj.}{Literário: fino; elegante}
\end{EntryWithPhonetic}

\begin{EntryWithPhonetic}{彬彬有礼}{bin1bin1-you3li3}{11,11,6,5}{⼺、⼺、⽉、⽰}[HSK 7-9]
  \definition{expr.}{refinado e cortês; urbano}
\end{EntryWithPhonetic}

\begin{EntryWithPhonetic}{滨}{bin1}{13}{⽔}
  \definition[个,片]{s.}{banco; beira; costa | praia; margem (beira) do rio; beira da água; perto da água}
  \definition{v.}{estar perto (do mar, de um rio, etc.)}
\end{EntryWithPhonetic}

\begin{EntryWithPhonetic}{滨海}{bin1 hai3}{13,10}{⽔、⽔}[HSK 7-9]
  \definition*{s.}{Condado de Binhai em Yancheng, Jiangsu | A cidade fictícia de Binhai na sátira política}
  \definition{adj.}{costeiro}
  \definition{v.}{estar situado (ou localizado) perto do mar}
\end{EntryWithPhonetic}

\begin{EntryWithPhonetic}{缤}{bin1}{13}{⽷}
  \definition{adj.}{Arcaico: vários; numerosos; profusos | Arcaico: em confusão | Arcaico: cores misturadas}
\end{EntryWithPhonetic}

\begin{EntryWithPhonetic}{缤纷}{bin1fen1}{13,7}{⽷、⽷}[HSK 7-9]
  \definition{adj.}{em profusão desenfreada; numerosos e confusos}
\end{EntryWithPhonetic}

\begin{EntryWithPhonetic}{冰}{bing1}{6}{⼎}[HSK 4]
  \definition[块,层,些]{s.}{gelo; água em estado sólido |  algo parecido com gelo | (gíria) metanfetamina}
  \definition{v.}{colocar gelo; colocar gelo ao redor; colocar no gelo; resfriar objetos com gelo ou água fria | sentir frio}
\end{EntryWithPhonetic}

\begin{EntryWithPhonetic}{冰糕}{bing1gao1}{6,16}{⼎、⽶}
  \definition{s.}{sorvete | picolé}
\end{EntryWithPhonetic}

\begin{EntryWithPhonetic}{冰棍}{bing1gun4}{6,12}{⼎、⽊}
  \definition[根,种,支]{s.}{picolé}
\end{EntryWithPhonetic}

\begin{EntryWithPhonetic}{冰棍儿}{bing1gun4r5}{6,12,2}{⼎、⽊、⼉}[HSK 7-9]
  \definition[根,种,支]{s.}{picolé; pirulito congelado}
\end{EntryWithPhonetic}

\begin{EntryWithPhonetic}{冰激凌}{bing1ji1ling2}{6,16,10}{⼎、⽔、⼎}
  \definition{s.}{sorvete}
\end{EntryWithPhonetic}

\begin{EntryWithPhonetic}{冰球}{bing1qiu2}{6,11}{⼎、⽟}
  \definition[个]{s.}{hóquei no gelo | disco; a ``bola'' usada no hóquei no gelo}
\end{EntryWithPhonetic}

\begin{EntryWithPhonetic}{冰山}{bing1shan1}{6,3}{⼎、⼭}[HSK 7-9]
  \definition[座]{s.}{montanha gelada; montanha coberta de gelo | \emph{iceberg}; enormes blocos de gelo flutuando no mar | Figurativo: indivíduo ou grupo em que não se pode confiar por muito tempo; uma metáfora para um poder em que não se pode confiar por muito tempo}
\end{EntryWithPhonetic}

\begin{EntryWithPhonetic}{冰天雪地}{bing1tian1-xue3di4}{6,4,11,6}{⼎、⼤、⾬、⼟}
  \definition{expr.}{um mundo de gelo e neve}
\end{EntryWithPhonetic}

\begin{EntryWithPhonetic}{冰箱}{bing1xiang1}{6,15}{⼎、⾋}[HSK 4]
  \definition[台,个]{s.}{geladeira; freezer; refrigerador; aparelhos para congelar alimentos ou medicamentos com gelo para mantê-los frios}
\end{EntryWithPhonetic}

\begin{EntryWithPhonetic}{冰雪}{bing1 xue3}{6,11}{⼎、⾬}[HSK 4]
  \definition{adj.}{puro como gelo e neve; descreve uma pessoa como pura}
  \definition[片,场]{s.}{gelo e neve}
\end{EntryWithPhonetic}

\begin{EntryWithPhonetic}{兵}{bing1}{7}{⼋}[HSK 4]
  \definition[个,种]{s.}{armas; armamentos | soldado; pessoal militar | exército; tropas | soldado raso; membro mais jovem do exército | assuntos militares (estratégia) | peão, uma das peças do xadrez chinês}
\end{EntryWithPhonetic}

\begin{EntryWithPhonetic}{兵器}{bing1qi4}{7,16}{⼋、⼝}
  \definition{s.}{armas | armamento}
\end{EntryWithPhonetic}

\begin{EntryWithPhonetic}{屏}{bing1}{9}{⼫}
  \definition{s.}{antigamente, referia-se à pequena parede de tela em frente ao portão de um antigo palácio; no chinês moderno, também é usado como uma palavra humilde para expressar o significado de 惶恐}
  \seeref{bing3}
  \seeref{ping2}
  \seealsoref{惶恐}{huang2kong3}
\end{EntryWithPhonetic}

\begin{EntryWithPhonetic}{丙}{bing3}{5}{⼀}[HSK 7-9]
  \definition*{s.}{o terceiro dos Dez Troncos Celestiais}
  \definition{s.}{terceiro | Literário: fogo}
\end{EntryWithPhonetic}

\begin{EntryWithPhonetic}{秉}{bing3}{8}{⽲}
  \definition{clas.}{unidade antiga de volume; 16 hu}
  \definition{s.}{Sobrenome Bing}
  \definition{v.}{Literário: segurar; agarrar | Literário: controlar; presidir; assumir o comando de}
  \seealsoref{斛}{hu2}
\end{EntryWithPhonetic}

\begin{EntryWithPhonetic}{秉承}{bing3cheng2}{8,8}{⽲、⼿}[HSK 7-9]
  \definition{v.}{receber (ordens); receber (comandos); aceitar e seguir (uma ordem ou instrução)}
\end{EntryWithPhonetic}

\begin{EntryWithPhonetic}{屏}{bing3}{9}{⼫}
  \definition*{s.}{Sobrenome Bing}
  \definition{v.}{prender (a respiração); conter a respiração | rejeitar; livrar-se de; remover; pôr (colocar) de lado; abandonar; descartar}
  \seeref{bing1}
  \seeref{ping2}
\end{EntryWithPhonetic}

\begin{EntryWithPhonetic}{饼}{bing3}{9}{⾷}[HSK 5]
  \definition[张]{s.}{um bolo redondo e plano; massa assada ou cozida no vapor | algo que tem o formato de um bolo; semelhante a uma torta}
\end{EntryWithPhonetic}

\begin{EntryWithPhonetic}{饼干}{bing3gan1}{9,3}{⾷、⼲}[HSK 5]
  \definition[块,片,包,盒,袋]{s.}{biscoito; bolacha; \emph{cookie}; alimentos, pedaços pequenos e finos cozidos em farinha com açúcar, ovos, leite, etc.}
\end{EntryWithPhonetic}

\begin{EntryWithPhonetic}{并}{bing4}{6}{⼲}[HSK 3,4]
  \definition{adv.}{lado a lado; igualmente; simultaneamente | (usado para reforçar uma negação) na verdade; definitivamente | mesmo assim | (usado para reforçar uma negação) na verdade; de forma alguma | todos; indica o conjunto completo, equivalente a 全部}
  \definition{conj.}{e; além disso}
  \definition{v.}{combinar; fundir; incorporar | ficar (ou colocar) lado a lado | estar paralelo a | anexar; juntar}
  \seealsoref{全部}{quan2bu4}
\end{EntryWithPhonetic}

\begin{EntryWithPhonetic}{并非}{bing4fei1}{6,8}{⼲、⾮}[HSK 7-9]
  \definition{adv.}{realmente não é; na verdade}
\end{EntryWithPhonetic}

\begin{EntryWithPhonetic}{并购}{bing4gou4}{6,8}{⼲、⾙}[HSK 7-9]
  \definition{s.}{aquisição; fusão e aquisição}
  \definition{v.}{fundir; adquirir | assumir}
\end{EntryWithPhonetic}

\begin{EntryWithPhonetic}{并列}{bing4lie4}{6,6}{⼲、⼑}[HSK 7-9]
  \definition{v.}{ficar lado a lado; ser justaposto; ter a mesma importância; organizar lado a lado}
\end{EntryWithPhonetic}

\begin{EntryWithPhonetic}{并排}{bing4pai2}{6,11}{⼲、⼿}
  \definition{adv.}{lado a lado}
\end{EntryWithPhonetic}

\begin{EntryWithPhonetic}{并且}{bing4qie3}{6,5}{⼲、⼀}[HSK 3]
  \definition{conj.}{e; bem como; usado entre verbos, adjetivos ou frases paralelas para indicar que várias ações são realizadas ao mesmo tempo ou que propriedades existem ao mesmo tempo | além disso; além do mais; ademais; usado na segunda metade de uma frase complexa para expressar um significado adicional}
\end{EntryWithPhonetic}

\begin{EntryWithPhonetic}{并行}{bing4xing2}{6,6}{⼲、⾏}[HSK 7-9]
  \definition{adj.}{simultâneo | Computação: paralelo | lado a lado (de dois processos, desenvolvimentos, pensamentos etc.)}
  \definition{v.}{caminhar lado a lado; correr lado a lado | fazer duas coisas ao mesmo tempo | prosseguir em paralelo}
\end{EntryWithPhonetic}

\begin{EntryWithPhonetic}{幷}{bing4}{8}{⼲}
  \variantof{并}
\end{EntryWithPhonetic}

\begin{EntryWithPhonetic}{倂}{bing4}{10}{⼈}
  \variantof{并}
\end{EntryWithPhonetic}

\begin{EntryWithPhonetic}{病}{bing4}{10}{⽧}[HSK 1]
  \definition[种]{s.}{doença; enfermidade | doença; males | falha; defeito; desvantagem; erro}
  \definition{v.}{adoecer; ficar doente | ferir; causar danos a | angustiar; desaprovar}
\end{EntryWithPhonetic}

\begin{EntryWithPhonetic}{病床}{bing4chuang2}{10,7}{⽧、⼴}[HSK 7-9]
  \definition[号,张]{s.}{cama de hospital | leito de doente}
\end{EntryWithPhonetic}

\begin{EntryWithPhonetic}{病毒}{bing4du2}{10,9}{⽧、⽏}[HSK 5]
  \definition[种,株,类]{s.}{vírus; patógenos que são menores que os germes e visíveis somente com um microscópio eletrônico | Computação: vírus de computador}
\end{EntryWithPhonetic}

\begin{EntryWithPhonetic}{病房}{bing4 fang2}{10,8}{⽧、⼾}[HSK 6]
  \definition[个,间]{s.}{enfermaria de um hospital; quartos onde ficam os pacientes em hospitais e onde vivem em casas de repouso}
\end{EntryWithPhonetic}

\begin{EntryWithPhonetic}{病情}{bing4 qing2}{10,11}{⽧、⼼}[HSK 6]
  \definition{s.}{estado de uma doença; condição do paciente; mudanças na doença}
\end{EntryWithPhonetic}

\begin{EntryWithPhonetic}{病人}{bing4 ren2}{10,2}{⽧、⼈}[HSK 1]
  \definition[个,位]{s.}{doente; paciente; pessoas doentes; pessoas em tratamento}
\end{EntryWithPhonetic}

\begin{EntryWithPhonetic}{病症}{bing4zheng4}{10,10}{⽧、⽧}[HSK 7-9]
  \definition[种]{s.}{doença; enfermidade}
\end{EntryWithPhonetic}

\begin{EntryWithPhonetic}{拨}{bo1}{8}{⼿}[HSK 7-9]
  \definition{clas.}{usado para agrupar pessoas; grupo; lote}
  \definition{v.}{mover (mexer) com a mão, o pé, o bastão, etc.; usar as mãos, os pés ou os bastões para mover objetos | atribuir; alocar; reservar | virar-se; inverter a marcha | dedilhar (uma corda de violão) com os dedos ou com um instrumento | chamar (alguém)}
\end{EntryWithPhonetic}

\begin{EntryWithPhonetic}{拨打}{bo1 da3}{8,5}{⼿、⼿}[HSK 6]
  \definition{v.}{ligar; discar; de acordo com o número da chamada, discar o número no telefone ou pressionar as teclas numéricas para fazer uma chamada}
\end{EntryWithPhonetic}

\begin{EntryWithPhonetic}{拨及}{bo1ji2}{8,3}{⼿、⼃}[HSK 7-9]
  \definition{v.}{espalhar para; envolver; afetar}
\end{EntryWithPhonetic}

\begin{EntryWithPhonetic}{拨款}{bo1kuan3}{8,12}{⼿、⽋}[HSK 7-9]
  \definition[项,笔]{s.}{dinheiro apropriado; apropriação; subsídio financeiro do estado; alocação de fundos; financiamento alocado}
  \definition{v.}{apropriar-se de dinheiro; alocar fundos}
\end{EntryWithPhonetic}

\begin{EntryWithPhonetic}{拨通}{bo1/tong1}{8,10}{⼿、⾡}[HSK 7-9]
  \definition{v.+compl.}{discar (os números de um telefone, etc.)}
\end{EntryWithPhonetic}

\begin{EntryWithPhonetic}{拨转}{bo1zhuan3}{8,8}{⼿、⾞}
  \definition{v.}{transferir (fundos, etc.) | virar | dar a volta}
\end{EntryWithPhonetic}

\begin{EntryWithPhonetic}{波}{bo1}{8}{⽔}
  \definition*{s.}{Polônia, abreviação de 波兰 | Sobrenome Bo}
  \definition{s.}{ondas, a superfície irregular da água em rios, lagos e oceanos | onda, o processo de propagação da vibração | mudanças inesperadas; uma reviravolta inesperada nos acontecimentos; metáfora para mudanças inesperadas nas coisas | olhos; metáfora do olhar errante}
  \seealsoref{波兰}{bo1lan2}
\end{EntryWithPhonetic}

\begin{EntryWithPhonetic}{波动}{bo1 dong4}{8,6}{⽔、⼒}[HSK 6]
  \definition{s.}{ondulação; flutuação; movimento de onda}
  \definition{v.}{ondular; flutuar}
\end{EntryWithPhonetic}

\begin{EntryWithPhonetic}{波兰}{bo1lan2}{8,5}{⽔、⼋}
  \definition*{s.}{Polônia}
\end{EntryWithPhonetic}

\begin{EntryWithPhonetic}{波澜}{bo1lan2}{8,15}{⽔、⽔}[HSK 7-9]
  \definition[个,场,阵]{s.}{ondas grandes}
\end{EntryWithPhonetic}

\begin{EntryWithPhonetic}{波浪}{bo1lang4}{8,10}{⽔、⽔}[HSK 6]
  \definition{s.}{onda; a superfície irregular da água nos rios, lagos e oceanos, geralmente se refere a águas menores e mais bonitas, frequentemente usada na linguagem falada}
\end{EntryWithPhonetic}

\begin{EntryWithPhonetic}{波涛}{bo1tao1}{8,10}{⽔、⽔}[HSK 7-9]
  \definition{s.}{grandes (enormes) ondas; ondas de maré; ondas grandes costumam se referir a paisagens espetaculares ou emocionantes; são usadas tanto na linguagem falada quanto na escrita.}
\end{EntryWithPhonetic}

\begin{EntryWithPhonetic}{波音}{bo1yin1}{8,9}{⽔、⾳}
  \definition*{s.}{Boeing (empresa aeroespacial)}
  \definition{s.}{mordente (música)}
\end{EntryWithPhonetic}

\begin{EntryWithPhonetic}{波折}{bo1zhe2}{8,7}{⽔、⼿}[HSK 7-9]
  \definition{s.}{reviravoltas; contratempo; as reviravoltas que ocorrem durante o curso das coisas, o que significa que você sofre dificuldades ou contratempos}
\end{EntryWithPhonetic}

\begin{EntryWithPhonetic}{玻}{bo1}{9}{⽟}
  \definition{s.}{vidro}
\end{EntryWithPhonetic}

\begin{EntryWithPhonetic}{玻璃}{bo1li5}{9,14}{⽟、⽟}[HSK 5]
  \definition[张,块]{s.}{vidro; corpo duro, quebradiço e transparente, geralmente feito de areia, calcário, carbonato de sódio, etc. | \emph{nylon}; plástico; refere-se a determinados plásticos que se assemelham ao vidro.}
\end{EntryWithPhonetic}

\begin{EntryWithPhonetic}{剥}{bo1}{10}{⼑}
  \definition{v.}{Dialeto: descascar; despelar; remover a casca ou pele externa | (pele, tinta, etc.) sair; descascar | privar; explorar}
  \seeref{bao1}
\end{EntryWithPhonetic}

\begin{EntryWithPhonetic}{剥夺}{bo1duo2}{10,6}{⼑、⼤}[HSK 7-9]
  \definition{v.}{roubar algo de alguém; tirar algo de alguém à força | privar; privar por lei; cancelar de acordo com a lei}
\end{EntryWithPhonetic}

\begin{EntryWithPhonetic}{剥削}{bo1xue1}{10,9}{⼑、⼑}[HSK 7-9]
  \definition{v.}{explorar; apropriar-se do trabalho ou dos frutos do trabalho de outrem sem remuneração}
\end{EntryWithPhonetic}

\begin{EntryWithPhonetic}{般}{bo1}{10}{⾈}
  \definition{s.}{utilizado em 般若}
  \seeref{ban1}
  \seeref{pan2}
  \seealsoref{般若}{bo1re3}
\end{EntryWithPhonetic}

\begin{EntryWithPhonetic}{般若}{bo1re3}{10,8}{⾈、⾋}
  \definition*{s.}{Prajña (sânscrito), \emph{insight} sobre a verdadeira natureza da realidade}
  \definition{s.}{budismo: sabedoria}
\end{EntryWithPhonetic}

\begin{EntryWithPhonetic}{啵}{bo1}{11}{⼝}
  \definition{part.}{denotando pedido, comando, etc.; o uso é semelhante ao de 吧, que é mais comum no vernáculo antigo}
  \definition{v.aux.}{indicando uma sugestão, pedido ou comando suave | indicando consentimento ou aprovação | em uma pergunta tendenciosa que pede a confirmação de uma suposição | indicando alguma dúvida na mente do falante | marcando uma pausa após suposições como alternativas}
  \seeref{bo5}
  \seealsoref{吧}{ba5}
\end{EntryWithPhonetic}

\begin{EntryWithPhonetic}{菠}{bo1}{11}{⾋}
  \definition{s.}{espinafre}
\end{EntryWithPhonetic}

\begin{EntryWithPhonetic}{菠菜}{bo1cai4}{11,11}{⾋、⾋}
  \definition[棵]{s.}{espinafre}
\end{EntryWithPhonetic}

\begin{EntryWithPhonetic}{播}{bo1}{15}{⼿}[HSK 6]
  \definition{v.}{espalhar; transmitir | semear | mover-se; migrar; ir para o exílio}
\end{EntryWithPhonetic}

\begin{EntryWithPhonetic}{播出}{bo1 chu1}{15,5}{⼿、⼐}[HSK 3]
  \definition{v.}{radiodifundir; transmitir; estar no ar; transmitir via rádio e televisão}
\end{EntryWithPhonetic}

\begin{EntryWithPhonetic}{播放}{bo1fang4}{15,8}{⼿、⽅}[HSK 3]
  \definition{v.}{ir ao ar; transmitir por rádio | mostrar; exibir; transmitir (um programa de TV)}
\end{EntryWithPhonetic}

\begin{EntryWithPhonetic}{播音}{bo1/yin1}{15,9}{⼿、⾳}
  \definition{s.}{transmissão}
  \definition{v.+compl.}{transmitir}
\end{EntryWithPhonetic}

\begin{EntryWithPhonetic}{蕃}{bo1}{15}{⾋}
  \definition[种]{s.}{estrangeiros}
  \seeref{fan1}
  \seeref{fan2}
\end{EntryWithPhonetic}

\begin{EntryWithPhonetic}{伯}{bo2}{7}{⼈}
  \definition*{s.}{Conde; o terceiro dos cinco graus de nobreza | Sobrenome Bo}
  \definition{s.}{tio; o primeiro (mais velho) dos irmãos; o mais velho entre irmãos}
\end{EntryWithPhonetic}

\begin{EntryWithPhonetic}{伯伯}{bo2bo5}{7,7}{⼈、⼈}[HSK 7-9]
  \definition[个,位]{s.}{tio; irmão mais velho do pai | termo de tratamento para um homem da mesma geração do seu pai, mas mais velho que ele}
\end{EntryWithPhonetic}

\begin{EntryWithPhonetic}{伯父}{bo2fu4}{7,4}{⼈、⽗}[HSK 7-9]
  \definition[个,位]{s.}{tio; irmão mais velho do pai | tio; termo usado para se referir a um homem um pouco mais velho que o pai; um título respeitoso para o pai de um colega ou colega de classe}
\end{EntryWithPhonetic}

\begin{EntryWithPhonetic}{伯母}{bo2mu3}{7,5}{⼈、⽏}[HSK 7-9]
  \definition[个,位]{s.}{tia; esposa do irmão mais velho do pai; esposa do tio | forma educada de se dirigir a uma mulher que tem aproximadamente a mesma idade da mãe; um título respeitoso para a mãe de um colega ou colega de classe}
\end{EntryWithPhonetic}

\begin{EntryWithPhonetic}{驳}{bo2}{7}{⾺}
  \definition{adj.}{Literário: misturado; heterogêneo; de cores diferentes; originalmente se refere à cor impura do cabelo do cavalo, estendida aos ingredientes impuros; cores misturadas}
  \definition{s.}{barcaça; fragata}
  \definition{v.}{refutar; contradizer; contestar; argumentar; distinguir o certo do errado; dar razões para refutar as opiniões erradas dos outros | transporte por barcaça | Dialeto: estender ou alargar (um banco, um dique ou um aterro)}
\end{EntryWithPhonetic}

\begin{EntryWithPhonetic}{驳回}{bo2hui2}{7,6}{⾺、⼞}[HSK 7-9]
  \definition{v.}{rejeitar; refutar; anular; recusar; não concordar (solicitação)}
\end{EntryWithPhonetic}

\begin{EntryWithPhonetic}{柏}{bo2}{9}{⽊}
  \definition{s.}{cipreste | usado para transcrever nomes}[柏林,德国城市名。===Berlim, uma cidade alemã.]
  \seeref{bai3}
  \seeref{bo4}
\end{EntryWithPhonetic}

\begin{EntryWithPhonetic}{柏林}{bo2lin2}{9,8}{⽊、⽊}
  \definition*{s.}{Berlim, capital da Alemanha}
\end{EntryWithPhonetic}

\begin{EntryWithPhonetic}{脖}{bo2}{11}{⾁}
  \definition[个]{s.}{pescoço | em forma de pescoço | parte semelhante ao pescoço}
\end{EntryWithPhonetic}

\begin{EntryWithPhonetic}{脖子}{bo2zi5}{11,3}{⾁、⼦}[HSK 7-9]
  \definition[条,个]{s.}{pescoço; a parte onde a cabeça e o tronco se conectam}
\end{EntryWithPhonetic}

\begin{EntryWithPhonetic}{博}{bo2}{12}{⼗}
  \definition*{s.}{Sobrenome Bo}
  \definition{adj.}{rico; abundante | erudito; bem informado | solto; grande | grande}
  \definition{s.}{doutor em filosofia; doutorado}
  \definition{v.}{ter um amplo conhecimento de; ser bem lido | ganhar; vencer | jogar}
\end{EntryWithPhonetic}

\begin{EntryWithPhonetic}{博客}{bo2 ke4}{12,9}{⼗、⼧}[HSK 5]
  \definition[个]{s.}{\emph{blog}; página da Web ou site gerenciado por um indivíduo, geralmente composto por postagens organizadas da mais recente para a mais antiga | blogueiro; \emph{blogger}; pessoas que possuem ou escrevem \emph{blogs}}
\end{EntryWithPhonetic}

\begin{EntryWithPhonetic}{博览会}{bo2lan3hui4}{12,9,6}{⼗、⾒、⼈}[HSK 5]
  \definition[次,届]{s.}{exposição; feira internacional; exposições de produtos em grande escala}
\end{EntryWithPhonetic}

\begin{EntryWithPhonetic}{博士}{bo2shi4}{12,3}{⼗、⼠}[HSK 5]
  \definition[位,名,个,些]{s.}{doutorado; grau de doutor; nível mais alto de um diploma; também, uma pessoa que obteve esse diploma | doutor; antigo título honorífico para uma pessoa que é habilidosa em um determinado ofício ou especializada em uma determinada ocupação | doutor; autoridades que ensinavam as escrituras na China nos tempos antigos}
\end{EntryWithPhonetic}

\begin{EntryWithPhonetic}{博文}{bo2wen2}{12,4}{⼗、⽂}
  \definition{s.}{artigo em um blog}
  \definition{v.}{escrever um artigo em um blog}
\end{EntryWithPhonetic}

\begin{EntryWithPhonetic}{博物馆}{bo2wu4guan3}{12,8,11}{⼗、⽜、⾷}[HSK 5]
  \definition[座,个]{s.}{museu; locais para coleta, armazenamento, pesquisa, exibição e exposição de relíquias culturais ou espécimes relacionados à história, cultura, arte, ciências naturais, ciência e tecnologia, etc.}
\end{EntryWithPhonetic}

\begin{EntryWithPhonetic}{博主}{bo2zhu3}{12,5}{⼗、⼂}
  \definition{s.}{blogueiro}
\end{EntryWithPhonetic}

\begin{EntryWithPhonetic}{搏}{bo2}{13}{⼿}
  \definition{v.}{brigar; lutar; combater | atacar | bater; pulsar (coração)}
\end{EntryWithPhonetic}

\begin{EntryWithPhonetic}{搏斗}{bo2dou4}{13,4}{⼿、⽃}[HSK 7-9]
  \definition{v.}{brigar; lutar; combater | envolver-se em combate corpo a corpo}
\end{EntryWithPhonetic}

\begin{EntryWithPhonetic}{薄}{bo2}{16}{⾋}
  \definition*{s.}{Sobrenome Bo}
  \definition{adj.}{pequeno; leve; magro | mau; cruel; mesquinho | frívolo; fútil; não solene | fraco; frágil}
  \definition{v.}{desprezar; tratar com desprezo; menosprezar | aproximar-se}
  \seeref{bao2}
  \seeref{bo4}
\end{EntryWithPhonetic}

\begin{EntryWithPhonetic}{薄弱}{bo2ruo4}{16,10}{⾋、⼸}[HSK 5]
  \definition{adj.}{fraco; frágil; não é firme; não é sólido}
\end{EntryWithPhonetic}

\begin{EntryWithPhonetic}{柏}{bo4}{9}{⽊}
  \definition{s.}{cedro; cipreste amarelo}
  \seeref{bai3}
  \seeref{bo2}
\end{EntryWithPhonetic}

\begin{EntryWithPhonetic}{薄}{bo4}{16}{⾋}
  \definition{s.}{menta; uma erva perene com aroma refrescante nos caules e folhas}
  \seeref{bao2}
  \seeref{bo2}
\end{EntryWithPhonetic}

\begin{EntryWithPhonetic}{啵}{bo5}{11}{⼝}
  \definition{part.}{partícula gramaticalmente equivalente a 吧}
  \seeref{bo1}
  \seealsoref{吧}{ba5}
\end{EntryWithPhonetic}

\begin{EntryWithPhonetic}{不}{bu2}[(antes de quarto tom)]{4}{⼀}[HSK 1]
  \seeref{bu4}
  \seeref{bu5}
\end{EntryWithPhonetic}

\begin{EntryWithPhonetic}{不必}{bu2 bi4}{4,5}{⼀、⼼}[HSK 3]
  \definition{adv.}{não precisa; não tem que; indica que não é necessário em termos de razão ou emoção}
\end{EntryWithPhonetic}

\begin{EntryWithPhonetic}{不便}{bu2 bian4}{4,9}{⼀、⼈}[HSK 6]
  \definition{adj.}{inconveniente; inapropriado | não ter dinheiro em mãos; estar com pouco dinheiro}
  \definition{v.}{inadequado fazer algo; indica que fazer algo é inapropriado ou inconveniente}
\end{EntryWithPhonetic}

\begin{EntryWithPhonetic}{不错}{bu2 cuo4}{4,13}{⼀、⾦}[HSK 2]
  \definition{adj.}{certo; correto | nada mal; muito bom}
\end{EntryWithPhonetic}

\begin{EntryWithPhonetic}{不大}{bu2 da4}{4,3}{⼀、⼤}[HSK 1]
  \definition{adv.}{não muito (indicando um grau baixo); não demasiado | não com frequência; raramente; dificilmente}
\end{EntryWithPhonetic}

\begin{EntryWithPhonetic}{不大离}{bu2da4li2}{4,3,10}{⼀、⼤、⼇}
  \definition{adj.}{bem perto | quase certo | nada mal}
\end{EntryWithPhonetic}

\begin{EntryWithPhonetic}{不但}{bu2 dan4}{4,7}{⼀、⼈}[HSK 2]
  \definition{conj.}{não só\dots mas também; usado na primeira parte de uma frase composta que expressa progressão, a segunda parte geralmente contém conjunções como 而且,  并且 ou advérbios como 也, 还 que correspondem à primeira parte}
  \seealsoref{并且}{bing4qie3}
  \seealsoref{而且}{er2 qie3}
  \seealsoref{还}{hai2}
  \seealsoref{也}{ye3}
\end{EntryWithPhonetic}

\begin{EntryWithPhonetic}{不但……而且……}{bu2 dan4 er2qie3}{4,7,6,5}{⼀、⼈、⽽、⼀}[HSK 2]
  \definition{conj.}{não só\dots mas também\dots}
\end{EntryWithPhonetic}

\begin{EntryWithPhonetic}{不到}{bu2dao4}{4,8}{⼀、⼑}
  \definition{adj.}{insuficiente}
  \definition{adv.}{menos que}
  \definition{v.}{não chegar}
\end{EntryWithPhonetic}

\begin{EntryWithPhonetic}{不定}{bu2ding4}{4,8}{⼀、⼧}[HSK 7-9]
  \definition{adj.}{indeterminado; indica pensamentos ou ações instáveis; incerteza; às vezes de uma maneira, às vezes de outra}
  \definition{adv.}{não é certo; não necessariamente; não sei, não tenho certeza do que vai acontecer}
\end{EntryWithPhonetic}

\begin{EntryWithPhonetic}{不断}{bu2duan4}{4,11}{⼀、⽄}[HSK 3]
  \definition{adv.}{incessantemente; ininterruptamente; continuamente; constantemente}
  \definition{v.}{continuar; enfatiza a continuação da ação}
\end{EntryWithPhonetic}

\begin{EntryWithPhonetic}{不对}{bu2 dui4}{4,5}{⼀、⼨}[HSK 1]
  \definition{adj.}{incorreto; errado | anormal; anômalo; estranho | desarmonia; incompatibilidade; discórdia}
\end{EntryWithPhonetic}

\begin{EntryWithPhonetic}{不够}{bu2 gou4}{4,11}{⼀、⼣}[HSK 2]
  \definition{adv.}{insuficiente; indica que não atingiu o nível esperado}
  \definition{v.}{não ser suficiente; indica que é inferior ao exigido em quantidade ou grau}
\end{EntryWithPhonetic}

\begin{EntryWithPhonetic}{不顾}{bu2gu4}{4,10}{⼀、⾴}[HSK 5]
  \definition{v.}{não considerar; desconsiderar | desconsiderar; não levar em consideração; ignorar; não se preocupar com}
\end{EntryWithPhonetic}

\begin{EntryWithPhonetic}{不过}{bu2guo4}{4,6}{⼀、⾡}[HSK 2]
  \definition{adv.}{apenas; meramente; nada mais do que; indica que não excede um determinado limite, equivalente a 仅 ou 只 | como intensificador após certos adjetivos}
  \definition{conj.}{mas; no entanto; apenas; usado no início da segunda parte da frase, indica o contrário do sentido anterior e modifica ou complementa o significado anterior}
\end{EntryWithPhonetic}

\begin{EntryWithPhonetic}{不计其数}{bu2 ji4 qi2 shu4}{4,4,8,13}{⼀、⾔、⼋、⽁}
  \definition{expr.}{seu número não pode ser contado; incontáveis; inumeráveis}
\end{EntryWithPhonetic}

\begin{EntryWithPhonetic}{不见}{bu2 jian4}{4,4}{⼀、⾒}[HSK 6]
  \definition{v.}{não ver; não conhecer; não encontrar | estar desaparecido; desaparecer; não consiguir encontrar algo}
\end{EntryWithPhonetic}

\begin{EntryWithPhonetic}{不见得}{bu2jian4de2}{4,4,11}{⼀、⾒、⼻}[HSK 7-9]
  \definition{adv.}{não pode; não é provável; não necessariamente; é improvável que}
\end{EntryWithPhonetic}

\begin{EntryWithPhonetic}{不客气}{bu2 ke4 qi5}{4,9,4}{⼀、⼧、⽓}[HSK 1]
  \definition{adj.}{rude; indelicado; duro | franco; sincero; direto}
  \definition{expr.}{de modo algum; não mencione isso; de nada}
  \definition{v.}{dizer palavras ou fazer gestos indelicados}
\end{EntryWithPhonetic}

\begin{EntryWithPhonetic}{不利}{bu2 li4}{4,7}{⼀、⼑}[HSK 5]
  \definition{adj.}{desfavorável; desvantajoso; nocivo; prejudicial | malsucedido}
\end{EntryWithPhonetic}

\begin{EntryWithPhonetic}{不利于}{bu2li4 yu2}{4,7,3}{⼀、⼑、⼆}[HSK 7-9]
  \definition{v.}{ser prejudicial a}
\end{EntryWithPhonetic}

\begin{EntryWithPhonetic}{不料}{bu2liao4}{4,10}{⼀、⽃}[HSK 6]
  \definition{conj.}{inesperadamente; para surpresa de alguém}
\end{EntryWithPhonetic}

\begin{EntryWithPhonetic}{不论}{bu2 lun4}{4,6}{⼀、⾔}[HSK 3]
  \definition{conj.}{não importa (o que, quem, como, etc.); se \dots ou \dots; significa que as condições ou situações são diferentes, mas os resultados permanecem os mesmos; geralmente é seguido por palavras paralelas ou pronomes interrogativos; geralmente é seguido por advérbios como 都 e 总}
  \definition{v.}{não discutir nem argumentar; não discutir; não debater}
  \seealsoref{都}{dou1}
  \seealsoref{总}{zong3}
\end{EntryWithPhonetic}

\begin{EntryWithPhonetic}{不论……都……}{bu2lun4 dou1}{4,6,10}{⼀、⾔、⾢}
  \definition{conj.}{não apenas\dots, (o que, quem, como, etc.), \dots}
\end{EntryWithPhonetic}

\begin{EntryWithPhonetic}{不论……也……}{bu2lun4 ye3}{4,6,3}{⼀、⾔、⼄}
  \definition{conj.}{não apenas\dots, (o que, quem, como, etc.), \dots}
\end{EntryWithPhonetic}

\begin{EntryWithPhonetic}{不耐烦}{bu2nai4fan2}{4,9,10}{⼀、⽽、⽕}[HSK 5]
  \definition{adj.}{impaciente; significa não ser capaz de suportar coisas tediosas ou que causam distração}
\end{EntryWithPhonetic}

\begin{EntryWithPhonetic}{不日}{bu2ri4}{4,4}{⼀、⽇}
  \definition{adv.}{em alguns dias}
\end{EntryWithPhonetic}

\begin{EntryWithPhonetic}{不慎}{bu2shen4}{4,13}{⼀、⼼}[HSK 7-9]
  \definition{adj.}{descuidado; desavisado}
\end{EntryWithPhonetic}

\begin{EntryWithPhonetic}{不是话}{bu2shi4hua4}{4,9,8}{⼀、⽇、⾔}
  \definition{expr.}{inacreditável; além das palavras; (palavras) não fazem sentido}
  \seealsoref{不像话}{bu2xiang4hua4}
  \seealsoref{不成话}{bu4cheng2hua4}
\end{EntryWithPhonetic}

\begin{EntryWithPhonetic}{不适}{bu2shi4}{4,9}{⼀、⾡}[HSK 7-9]
  \definition{adj.}{Literário: desconfortável; indisposto; mal-estar; sentir-se mal | Literário: não adequado; inadequado}
\end{EntryWithPhonetic}

\begin{EntryWithPhonetic}{不算}{bu2 suan4}{4,14}{⼀、⽵}[HSK 7-9]
  \definition{adv.}{não realmente; não em vão; não é grande coisa}
\end{EntryWithPhonetic}

\begin{EntryWithPhonetic}{不太}{bu2 tai4}{4,4}{⼀、⼤}[HSK 2]
  \definition{adv.}{não exatamente | não muito bom}
\end{EntryWithPhonetic}

\begin{EntryWithPhonetic}{不像话}{bu2xiang4hua4}{4,13,8}{⼀、⼈、⾔}[HSK 7-9]
  \definition{expr.}{absurdo; sem sentido; irracional; uma determinada prática ou afirmação não está de acordo com o senso comum ou a razão e parece irracional | chocante; ultrajante; uma ação, palavra ou situação tão extrema que não pode ser aceita ou tolerada}
  \seealsoref{不是话}{bu2shi4hua4}
  \seealsoref{不成话}{bu4cheng2hua4}
\end{EntryWithPhonetic}

\begin{EntryWithPhonetic}{不屑}{bu2xie4}{4,10}{⼀、⼫}[HSK 7-9]
  \definition{adj.}{desdenhoso; zombador}
  \definition{v.}{pensar que algo não vale a pena fazer; sentir que está abaixo da dignidade de alguém fazer algo; ignorar}
\end{EntryWithPhonetic}

\begin{EntryWithPhonetic}{不懈}{bu2xie4}{4,16}{⼀、⼼}[HSK 7-9]
  \definition{adj.}{incansável; incessante; infatigável}
\end{EntryWithPhonetic}

\begin{EntryWithPhonetic}{不幸}{bu2 xing4}{4,8}{⼀、⼲}[HSK 5]
  \definition{adj.}{triste; infeliz; lamentável; azarado | infeliz; indica o mais indesejável (que aconteceu)}
  \definition[些]{s.}{morte; desastre; infortúnio; adversidade; calamidade}
\end{EntryWithPhonetic}

\begin{EntryWithPhonetic}{不亚于}{bu2ya4yu2}{4,6,3}{⼀、⼆、⼆}[HSK 7-9]
  \definition{v.}{ser tão bom quanto; não ser inferior a}
\end{EntryWithPhonetic}

\begin{EntryWithPhonetic}{不要}{bu2 yao4}{4,9}{⼀、⾑}[HSK 2]
  \definition{adv.}{nada de (pedir a alguém para não fazer) | não; expressa proibição e dissuasão}
\end{EntryWithPhonetic}

\begin{EntryWithPhonetic}{不要紧}{bu2yao4jin3}{4,9,10}{⼀、⾑、⽷}[HSK 4]
  \definition{adj.}{não é sério; não importa; não importa; não é um problema; nenhum obstáculo; nenhum problema; parece estar tudo bem; à primeira vista, não parece haver nenhum obstáculo}
\end{EntryWithPhonetic}

\begin{EntryWithPhonetic}{不亦乐乎}{bu2yi4le4hu1}{4,6,5,5}{⼀、⼇、⼃、⼃}[HSK 7-9]
  \definition{expr.}{terrivelmente; extremamente; Não é um grande prazer?; Que delícia seria se\dots; o significado original é "Não é também muito feliz?" (visto em "Os Analectos de Confúcio·Aprendizado"), e agora é frequentemente usado para expressar atingir o extremo | (como um complemento após 忙得) extremamente; terrivelmente | não é um grande prazer}
  \seealsoref{忙得}{mang2de2}
\end{EntryWithPhonetic}

\begin{EntryWithPhonetic}{不易}{bu2 yi4}{4,8}{⼀、⽇}[HSK 5]
  \definition{adj.}{não é fácil; difícil | imutável}
  \definition{v.}{não é fácil fazer algo | não pode ser alterado}
\end{EntryWithPhonetic}

\begin{EntryWithPhonetic}{不翼而飞}{bu2yi4'er2fei1}{4,17,6,3}{⼀、⽻、⽽、⾶}[HSK 7-9]
  \definition{expr.}{(um objeto) desaparecer sem deixar vestígios; desaparecer no ar; “voar sem asas” -- desaparecer sem deixar rastros; ganhar asas; desaparecer de repente; desaparecer repentinamente | espalhar-se rapidamente como se tivesse asas; espalhar-se como fogo}
\end{EntryWithPhonetic}

\begin{EntryWithPhonetic}{不用}{bu2 yong4}{4,5}{⼀、⽤}[HSK 1]
  \definition{v.}{não precisar; não ter necessidade; indicar que, na verdade, não é necessário}
\end{EntryWithPhonetic}

\begin{EntryWithPhonetic}{不用说}{bu2yong4shuo1}{4,5,9}{⼀、⽤、⾔}[HSK 7-9]
  \definition{v.}{desnecessário dizer; ser evidente}
\end{EntryWithPhonetic}

\begin{EntryWithPhonetic}{不再}{bu2 zai4}{4,6}{⼀、⼌}[HSK 6]
  \definition{adv.}{não mais; não repita uma segunda vez}
  \definition{v.}{ter ido embora; não retornar; não aparecer; não existir mais}
\end{EntryWithPhonetic}

\begin{EntryWithPhonetic}{不在乎}{bu2 zai4 hu1}{4,6,5}{⼀、⼟、⼃}[HSK 4]
  \definition{v.}{não se importar; não dar a mínima; não dar atenção}
\end{EntryWithPhonetic}

\begin{EntryWithPhonetic}{不正之风}{bu2zheng4zhi1feng1}{4,5,3,4}{⼀、⽌、⼂、⾵}[HSK 7-9]
  \definition[种]{expr.}{tendências prejudiciais; tendência doentia; práticas inadequadas; maus estilos de trabalho; ventos malignos; tendências doentias | tendência doentia; negligência médica}
\end{EntryWithPhonetic}

\begin{EntryWithPhonetic}{不至于}{bu2 zhi4 yu2}{4,6,3}{⼀、⾄、⼆}[HSK 6]
  \definition{adv.}{não pode ir tão longe a ponto de; não tanto\dots. a ponto de\dots; não a ponto de}
\end{EntryWithPhonetic}

\begin{EntryWithPhonetic}{不注意}{bu2zhu4yi4}{4,8,13}{⼀、⽔、⼼}
  \definition{adj.}{impensado | distraído}
  \definition{s.}{descuido | distração}
\end{EntryWithPhonetic}

\begin{EntryWithPhonetic}{补}{bu3}{7}{⾐}[HSK 3]
  \definition*{s.}{Sobrenome Bu}
  \definition{s.}{ajuda; uso; benefício; utilidade}
  \definition{v.}{reparar; consertar; remendar; adicionar materiais, consertar coisas quebradas | abastecer; encher; repor; adicionar suplemento; complementar; completar; preencher | nutrir}
\end{EntryWithPhonetic}

\begin{EntryWithPhonetic}{补偿}{bu3chang2}{7,11}{⾐、⼈}[HSK 5]
  \definition{v.}{compensar (perda, consumo); compensar (deficiências, diferenças)}
\end{EntryWithPhonetic}

\begin{EntryWithPhonetic}{补充}{bu3chong1}{7,6}{⾐、⼉}[HSK 3]
  \definition{adj.}{adicional | suplementar}
  \definition{v.}{reabastecer; suplementar; complementar; aumentar uma parte quando houver insuficiência ou perda}
\end{EntryWithPhonetic}

\begin{EntryWithPhonetic}{补给}{bu3ji3}{7,9}{⾐、⽷}[HSK 7-9]
  \definition{v.}{fornecer; prover; equipar; reabastecer; alimentar; recarregar}
\end{EntryWithPhonetic}

\begin{EntryWithPhonetic}{补救}{bu3jiu4}{7,11}{⾐、⽁}[HSK 7-9]
  \definition{v.}{remediar; depois que algo dá errado, tomar medidas para compensar e salvar a situação}
\end{EntryWithPhonetic}

\begin{EntryWithPhonetic}{补考}{bu3 kao3}{7,6}{⾐、⽼}[HSK 6]
  \definition{v.}{repetir ou refazer um exame}
\end{EntryWithPhonetic}

\begin{EntryWithPhonetic}{补课}{bu3/ke4}{7,10}{⾐、⾔}[HSK 6]
  \definition{v.+compl.}{compensar uma aula perdida; compensar cursos perdidos | refazer; fazer algo de novo; metáfora para refazer algo que não foi bem feito}
\end{EntryWithPhonetic}

\begin{EntryWithPhonetic}{补贴}{bu3tie1}{7,9}{⾐、⾙}[HSK 5]
  \definition[笔,项,种,份]{s.}{subsídio; ajuda de custo; custos de indenização ou assistência concedida a empresas ou indivíduos pelo estado ou governo}
  \definition{v.}{subsidiar; compensar a falta de dinheiro ou coisas; refere-se principalmente à compensação financeira ou ajuda dada pelo estado ou governo a empresas ou indivíduos}
\end{EntryWithPhonetic}

\begin{EntryWithPhonetic}{补习}{bu3 xi2}{7,3}{⾐、⼄}[HSK 6]
  \definition{clas.}{treinar; ter aulas depois da escola ou do trabalho; estudar depois da aula ou no seu tempo livre para compensar a falta de conhecimento}
\end{EntryWithPhonetic}

\begin{EntryWithPhonetic}{补助}{bu3 zhu4}{7,7}{⾐、⼒}[HSK 6]
  \definition{s.}{subsídio; mesada}
  \definition{v.}{ajudar financeiramente; subsidiar}
\end{EntryWithPhonetic}

\begin{EntryWithPhonetic}{哺}{bu3}{10}{⼝}
  \definition{s.}{comida na boca; chorando por comida | alimentos de mastigação; mastigando comida}
  \definition{v.}{alimentar (um bebê); amamentar}
\end{EntryWithPhonetic}

\begin{EntryWithPhonetic}{哺育}{bu3yu4}{10,8}{⼝、⾁}[HSK 7-9]
  \definition{v.}{alimentar | Figurativo: nutrir; fomentar | desenvolver}
\end{EntryWithPhonetic}

\begin{EntryWithPhonetic}{捕}{bu3}{10}{⼿}[HSK 6]
  \definition{v.}{pegar; apreender; prender}
\end{EntryWithPhonetic}

\begin{EntryWithPhonetic}{捕捉}{bu3zhuo1}{10,10}{⼿、⼿}[HSK 7-9]
  \definition{v.}{caçar; perseguir; pegar; capturar; apreender; pegar; fazer uma pessoa ou animal cair nas mãos; pode ser usado tanto para pessoas quanto para coisas; tem uma ampla gama de aplicações; usado tanto na linguagem falada quanto na escrita}
\end{EntryWithPhonetic}

\begin{EntryWithPhonetic}{堡}{bu3}{12}{⼟}
  \definition{s.}{forte; vila; cidade; assentamento fortificado (uma vila ou cidade cercada por muros de terra, frequentemente usada em nomes de lugares) | cidade; frequentemente usado em nomes de lugares}
  \seeref{bao3}
  \seeref{pu4}
\end{EntryWithPhonetic}

\begin{EntryWithPhonetic}{不}{bu4}{4}{⼀}[HSK 1]
  \definition{adv.}{(antes de verbos, adjetivos e outros advérbios; nunca antes do verbo 有) não; não vai; não quer | em algumas expressões educadas, significa que não é necessário fazer isso, o que equivale a 不用 ou 不要 | | (entre um verbo e seu complemento) não pode | usado com 就 para indicar escolha}
  \definition{part.}{no final da frase para indicar uma pergunta; (usar sozinho ou com uma partícula nas respostas) não}
  \definition{pref.}{(antes de certos substantivos para formar um adjetivo) un-; in-}
  \seeref{bu2}
  \seeref{bu5}
  \seealsoref{不要}{bu2 yao4}
  \seealsoref{不用}{bu2 yong4}
  \seealsoref{就}{jiu4}
  \seealsoref{有}{you3}
\end{EntryWithPhonetic}

\begin{EntryWithPhonetic}{不安}{bu4'an1}{4,6}{⼀、⼧}[HSK 3]
  \definition{adj.}{(humor) inquieto; (ambiente, etc.)  instável; intranquilo; perturbado; sem paz | desculpe; frases de cortesia, expressões de desculpas, equivalentes a 不好意思}
  \seealsoref{不好意思}{bu4 hao3 yi4 si5}
\end{EntryWithPhonetic}

\begin{EntryWithPhonetic}{不曾}{bu4 ceng2}{4,12}{⼀、⽈}[HSK 5]
  \definition{adv.}{nunca (ter feito algo); indica que não aconteceu (negação de 曾经)}
  \seealsoref{曾经}{ceng2jing1}
\end{EntryWithPhonetic}

\begin{EntryWithPhonetic}{不成}{bu4 cheng2}{4,6}{⼀、⼽}[HSK 6]
  \definition{adj.}{não é bom; não funciona; impraticável}
  \definition{part.}{usada no final de uma frase para expressar especulação ou tom contraintuitivo, geralmente precedido por palavras como 嘛 ou 莫非}
  \definition{v.}{não ser permitido; não ser permissível; ser impossível}
  \seealsoref{莫非}{mo4fei1}
  \seealsoref{难道}{nan2dao4}
\end{EntryWithPhonetic}

\begin{EntryWithPhonetic}{不成话}{bu4cheng2hua4}{4,6,8}{⼀、⼽、⾔}
  \definition{expr.}{irracional | chocante; ultrajante; inapropriado}
  \seealsoref{不是话}{bu2shi4hua4}
  \seealsoref{不像话}{bu2xiang4hua4}
\end{EntryWithPhonetic}

\begin{EntryWithPhonetic}{不耻下问}{bu4chi3-xia4wen4}{4,10,3,6}{⼀、⽿、⼀、⾨}[HSK 7-9]
  \definition{expr.}{não ter vergonha de perguntar e aprender com os subordinados; Os Analectos de Confúcio: Gongye Chang: "Seja rápido para aprender e não tenha vergonha de fazer perguntas" significa não ter vergonha de pedir conselhos a pessoas de status inferior ou menos informadas do que você}
\end{EntryWithPhonetic}

\begin{EntryWithPhonetic}{不辞而别}{bu4ci2'er2bie2}{4,13,6,7}{⼀、⾟、⽽、⼑}[HSK 7-9]
  \definition{expr.}{ir embora sem se despedir; sair sem se despedir}
  \definition{v.}{ir embora sem dizer adeus}
\end{EntryWithPhonetic}

\begin{EntryWithPhonetic}{不得不}{bu4de2bu4}{4,11,4}{⼀、⼻、⼀}[HSK 3]
  \definition{adv.}{ter que; não ter outra escolha a não ser; como obrigação ou necessidade}
\end{EntryWithPhonetic}

\begin{EntryWithPhonetic}{不得而知}{bu4de2'er2zhi1}{4,11,6,8}{⼀、⼻、⽽、⽮}[HSK 7-9]
  \definition{expr.}{desconhecido; incapaz de descobrir}
\end{EntryWithPhonetic}

\begin{EntryWithPhonetic}{不得了}{bu4de2liao3}{4,11,2}{⼀、⼻、⼅}[HSK 5]
  \definition{adj.}{terrível; horrível; extremamente sério; indica uma situação grave}
  \definition{adv.}{muito; extremamente; excessivamente; indica um grau profundo}
\end{EntryWithPhonetic}

\begin{EntryWithPhonetic}{不得已}{bu4de2yi3}{4,11,3}{⼀、⼻、⼰}[HSK 7-9]
  \definition{adj.}{agir contra a própria vontade; não ter alternativa senão; não há alternativa; tem que ser assim}
\end{EntryWithPhonetic}

\begin{EntryWithPhonetic}{不妨}{bu4fang2}{4,7}{⼀、⼥}[HSK 7-9]
  \definition{adv.}{pode muito bem; não há mal nenhum em; significa que você pode fazer isso, não há problema}
\end{EntryWithPhonetic}

\begin{EntryWithPhonetic}{不服}{bu4fu2}{4,8}{⼀、⽉}[HSK 7-9]
  \definition{v.}{recusar-se a obedecer (ou cumprir); recusar-se a aceitar como final; permanecer não convencido por; não ceder a | não estar acostumado a | recusar-se a obedecer; recalcitrar}
\end{EntryWithPhonetic}

\begin{EntryWithPhonetic}{不服气}{bu4 fu2qi4}{4,8,4}{⼀、⽉、⽓}[HSK 7-9]
  \definition{v.}{recusar-se a ser convencido; não confiar}
\end{EntryWithPhonetic}

\begin{EntryWithPhonetic}{不敢当}{bu4gan3dang1}{4,11,6}{⼀、⽁、⼹}[HSK 5]
  \definition{expr.}{Eu realmente não mereço isso.; Eu não sou digno de tais elogios.; Não estou à altura da honra.; Você me lisonjeia.; palavra de humildade, para mostrar que você não pode pagar (hospitalidade, elogios, etc.)}
\end{EntryWithPhonetic}

\begin{EntryWithPhonetic}{不公}{bu4gong1}{4,4}{⼀、⼋}
  \definition{adj.}{injusto}
\end{EntryWithPhonetic}

\begin{EntryWithPhonetic}{不管}{bu4guan3}{4,14}{⼀、⽵}[HSK 4]
  \definition{conj.}{não importa (o que, como, etc.); independentemente de; indica que, embora as condições ou circunstâncias tenham mudado, o resultado permanece o mesmo; 不管 deve ser seguido por algo incerto}
  \seealsoref{不管……都……}{bu4guan3 dou1}
  \seealsoref{不管……也……}{bu4guan3 ye3}
\end{EntryWithPhonetic}

\begin{EntryWithPhonetic}{不管……都……}{bu4guan3 dou1}{4,14,10}{⼀、⽵、⾢}
  \definition{conj.}{não apenas\dots, (o que, quem, como, etc.), \dots}
\end{EntryWithPhonetic}

\begin{EntryWithPhonetic}{不管……也……}{bu4guan3 ye3}{4,14,3}{⼀、⽵、⼄}
  \definition{conj.}{não apenas\dots, (o que, quem, como, etc.), \dots}
\end{EntryWithPhonetic}

\begin{EntryWithPhonetic}{不光}{bu4 guang1}{4,6}{⼀、⼉}[HSK 3]
  \definition{adv.}{não é o único; não apenas; não só; indica que excede uma determinada quantidade ou faixa}
  \definition{conj.}{não somente; não só}
\end{EntryWithPhonetic}

\begin{EntryWithPhonetic}{不好意思}{bu4 hao3 yi4 si5}{4,6,13,9}{⼀、⼥、⼼、⼼}[HSK 2]
  \definition{adj.}{envergonhado; desconfortável; constrangido; sem jeito}
  \definition{interj.}{com licença; peço desculpas; desculpe-me}
  \definition{v.}{achar constrangedor (fazer algo) | pedir desculpas (por incomodar alguém) | sentir-se envergonhado | achar algo embaraçoso}
\end{EntryWithPhonetic}

\begin{EntryWithPhonetic}{不假思索}{bu4jia3-si1suo3}{4,11,9,10}{⼀、⼈、⼼、⽷}[HSK 7-9]
  \definition{expr.}{(agir, responder, etc.) sem pensar; sem hesitação; prontamente; de ​​improviso; reagir instantaneamente}
\end{EntryWithPhonetic}

\begin{EntryWithPhonetic}{不解}{bu4jie3}{4,13}{⼀、⾓}[HSK 7-9]
  \definition{adj.}{inexplicável; inseparável}
  \definition{v.}{não entender; deixar de compreender}
\end{EntryWithPhonetic}

\begin{EntryWithPhonetic}{不禁}{bu4jin1}{4,13}{⼀、⽰}[HSK 6]
  \definition{adv.}{não pode evitar (fazer algo); não pode se abster de; incapaz de conter (produzir certas emoções, realizar certas ações)}
\end{EntryWithPhonetic}

\begin{EntryWithPhonetic}{不仅}{bu4jin3}{4,4}{⼀、⼈}[HSK 3]
  \definition{adv.}{não apenas (em número, quantidade ou extensão); costuma-se dizer 不仅仅}
  \definition{conj.}{não somente}
  \seealsoref{不仅仅}{bu4 jin3 jin3}
\end{EntryWithPhonetic}

\begin{EntryWithPhonetic}{不仅仅}{bu4 jin3 jin3}{4,4,4}{⼀、⼈、⼈}[HSK 6]
  \definition{adv.}{não só; não apenas}
\end{EntryWithPhonetic}

\begin{EntryWithPhonetic}{不经意}{bu4jing1yi4}{4,8,13}{⼀、⽷、⼼}[HSK 7-9]
  \definition{v.}{não tomar cuidado; ser descuidado; ser desatento}
\end{EntryWithPhonetic}

\begin{EntryWithPhonetic}{不景气}{bu4jing3qi4}{4,12,4}{⼀、⽇、⽓}[HSK 7-9]
  \definition{adj.}{frouxo; lento; em estado de depressão | Economia: em depressão; em recessão; em crise; estagnada | estado depressivo; não próspero}
\end{EntryWithPhonetic}

\begin{EntryWithPhonetic}{不久}{bu4 jiu3}{4,3}{⼀、⼃}[HSK 2]
  \definition{adv.}{em breve; dentro em breve; num futuro próximo | logo depois; pouco tempo depois | não muito tempo (antes ou depois de algo)}
\end{EntryWithPhonetic}

\begin{EntryWithPhonetic}{不堪}{bu4kan1}{4,12}{⼀、⼟}[HSK 7-9]
  \definition{adj.}{extremamente indesejável; usado após adjetivos negativos | (após palavras com conotação negativa) insuportável;muito ruim}
  \definition{v.}{não poder suportar; não poder ficar de pé; ser incapaz de suportar | (geralmente referindo-se a algo indesejável) não poder; ser incapaz de}
\end{EntryWithPhonetic}

\begin{EntryWithPhonetic}{不可避免}{bu4ke3-bi4mian3}{4,5,16,7}{⼀、⼝、⾌、⼉}[HSK 7-9]
  \definition{expr.}{inevitavelmente; inescapabilidade | inevitável}
\end{EntryWithPhonetic}

\begin{EntryWithPhonetic}{不可思议}{bu4ke3-si1yi4}{4,5,9,5}{⼀、⼝、⼼、⾔}[HSK 7-9]
  \definition{expr.}{incrível; inacreditável; inimaginável; inconcebível; algo que é difícil de entender ou imaginar.}
\end{EntryWithPhonetic}

\begin{EntryWithPhonetic}{不肯}{bu4 ken3}{4,8}{⼀、⾁}[HSK 7-9]
  \definition{v.aux.}{não irá; não iria; usado como verbo auxiliar negativo para expressar recusa}
\end{EntryWithPhonetic}

\begin{EntryWithPhonetic}{不理}{bu4 li3}{4,11}{⼀、⽟}[HSK 7-9]
  \definition{v.}{ignorar; desconsiderar; recusar-se a reconhecer; não prestar atenção a; não tomar conhecimento de; não responder | não fazer; não lidar com}
\end{EntryWithPhonetic}

\begin{EntryWithPhonetic}{不良}{bu4 liang2}{4,7}{⼀、⾉}[HSK 5]
  \definition{adj.}{ruim; prejudicial; nocivo; insalubre}
\end{EntryWithPhonetic}

\begin{EntryWithPhonetic}{不了了之}{bu4liao3-liao3zhi1}{4,2,2,3}{⼀、⼅、⼅、⼂}[HSK 7-9]
  \definition{expr.}{deixar um assunto sem solução; acabar sem nada definido; pontas soltas | resolver um assunto deixando-o sem solução; abafar; deixando-o sem solução; concluir sem uma conclusão; acabar sem nada definido; concluir sem resultado concreto (decisão); deixá-lo sem solução; deixar (um assunto) seguir seu próprio curso | deixe-o inquieto}
\end{EntryWithPhonetic}

\begin{EntryWithPhonetic}{不满}{bu4 man3}{4,13}{⼀、⽔}[HSK 2]
  \definition{adj.}{ressentido; insatisfeito; descontente}
  \definition{v.}{estar descontente com; insatisfação ou descontentamento com alguém ou alguma coisa |ser menor que; quantidade ou tempo insuficientes ou inadequados}
\end{EntryWithPhonetic}

\begin{EntryWithPhonetic}{不免}{bu4mian3}{4,7}{⼀、⼉}[HSK 5]
  \definition{adv.}{inevitavelmente; inexoravelmente}
\end{EntryWithPhonetic}

\begin{EntryWithPhonetic}{不难}{bu4 nan2}{4,10}{⼀、⾫}[HSK 7-9]
  \definition{v.}{não ser difícil}[这台电脑修理起来不难。===Este computador não é difícil de consertar.]
\end{EntryWithPhonetic}

\begin{EntryWithPhonetic}{不能不}{bu4 neng2 bu4}{4,10,4}{⼀、⾁、⼀}[HSK 5]
  \definition{adv.}{tem que; não pode, mas; necessariamente; definitivamente}
\end{EntryWithPhonetic}

\begin{EntryWithPhonetic}{不平}{bu4ping2}{4,5}{⼀、⼲}[HSK 7-9]
  \definition{adj.}{irregular; não nivelado | injusto}
  \definition[些,点]{s.}{injustiça; deslealdade | ressentimento; queixa}
  \definition{v.}{estar indignado; estar ressentido | estar irregular; não estar nivelado; não estar liso | ser injusto}
\end{EntryWithPhonetic}

\begin{EntryWithPhonetic}{不起眼}{bu4qi3yan3}{4,10,11}{⼀、⾛、⽬}[HSK 7-9]
  \definition{adj.}{imperceptível; discreto; não chamativo; despercebido; não apreciado}
\end{EntryWithPhonetic}

\begin{EntryWithPhonetic}{不然}{bu4ran2}{4,12}{⼀、⽕}[HSK 4]
  \definition{adj.}{não é assim; não é o caso}
  \definition{conj.}{se não; caso contrário; indica outra consequência ou circunstância que teria ocorrido se não fosse}
\end{EntryWithPhonetic}

\begin{EntryWithPhonetic}{不容}{bu4rong2}{4,10}{⼀、⼧}[HSK 7-9]
  \definition{s.}{pontos de acupuntura}
  \definition{v.}{não tolerar; não permitir; não poder deixar de}
\end{EntryWithPhonetic}

\begin{EntryWithPhonetic}{不如}{bu4ru2}{4,6}{⼀、⼥}[HSK 2]
  \definition{conj.}{em vez de; melhor do que; seria melhor; preferiria; seria melhor; usado no início da segunda parte da frase, indica uma escolha feita após comparação (geralmente em correspondência com o termo 与其 no texto anterior)}
  \definition{v.}{ser inferior a; não ser igual a; não ser tão bom quanto;  não poder fazer melhor que}
  \seealsoref{与其}{yu3qi2}
\end{EntryWithPhonetic}

\begin{EntryWithPhonetic}{不如说}{bu4ru2 shuo1}{4,6,9}{⼀、⼥、⾔}[HSK 7-9]
  \definition{expr.}{é melhor dizer\dots;  usada para indicar que uma afirmação é mais apropriada, precisa ou preferível a outra; é frequentemente usada em situações de comparação e escolha para enfatizar uma afirmação ou ação superior}
\end{EntryWithPhonetic}

\begin{EntryWithPhonetic}{不少}{bu4 shao3}{4,4}{⼀、⼩}[HSK 2]
  \definition{adj.}{muitos; bastante; não poucos; indica uma quantidade considerável, equivalente a muitos ou bastante}
\end{EntryWithPhonetic}

\begin{EntryWithPhonetic}{不时}{bu4shi2}{4,7}{⼀、⽇}[HSK 5]
  \definition{adv.}{frequentemente; de tempos em tempos | a qualquer momento}
\end{EntryWithPhonetic}

\begin{EntryWithPhonetic}{不是……而是}{bu4shi4 er2 shi4}{4,9,6,9}{⼀、⽇、⽽、⽇}
  \definition{conj.}{não somente\dots mas também\dots, expressam um relacionamento mais profundo e avançado em significado, mas as orações antes e depois são consistentes em expressar significados negativos e afirmativos, entretanto, a primeira metade da frase expressa negação, e a segunda metade expressa afirmação, e o significado das orações anteriores e seguintes não pode ser de um nível mais alto}
\end{EntryWithPhonetic}

\begin{EntryWithPhonetic}{不停}{bu4 ting2}{4,11}{⼀、⼈}[HSK 5]
  \definition{adv.}{sem parar; sem interrupção; continuamente}
\end{EntryWithPhonetic}

\begin{EntryWithPhonetic}{不通}{bu4 tong1}{4,10}{⼀、⾡}[HSK 6]
  \definition{adj.}{sem sentido; ilógico; agramatical | usado para se referir a coisas abstratas}
  \definition{v.}{obstruir; bloquear; estar obstruído; estar bloqueado; ser intransitável | não saber; não entender; não poder aceitar}
\end{EntryWithPhonetic}

\begin{EntryWithPhonetic}{不同}{bu4 tong2}{4,6}{⼀、⼝}[HSK 2]
  \definition{adj.}{diferente; distinto; não semelhante;}
\end{EntryWithPhonetic}

\begin{EntryWithPhonetic}{不同寻常}{bu4tong2-xun2chang2}{4,6,6,11}{⼀、⼝、⼨、⼱}[HSK 7-9]
  \definition{adj.}{extraordinário; incomum}
\end{EntryWithPhonetic}

\begin{EntryWithPhonetic}{不为人知}{bu4wei2ren2zhi1}{4,4,2,8}{⼀、⼂、⼈、⽮}[HSK 7-9]
  \definition{expr.}{não conhecido por ninguém | segredo | desconhecido}
\end{EntryWithPhonetic}

\begin{EntryWithPhonetic}{不惜}{bu4xi1}{4,11}{⼀、⼼}[HSK 7-9]
  \definition{v.}{não hesitar (em fazer algo)}
\end{EntryWithPhonetic}

\begin{EntryWithPhonetic}{不相上下}{bu4xiang1-shang4xia4}{4,9,3,3}{⼀、⽬、⼀、⼀}[HSK 7-9]
  \definition{expr.}{igualmente semelhante; quase igual; quase no mesmo nível | ser aproximadamente o mesmo; quase igual; ser igual a\dots; estar no mesmo nível; estar um pouco no mesmo nível de\dots; mais ou menos de força igual; aumentar o tamanho para; há pouca (não há muito) para escolher entre os dois; sem muita diferença}
\end{EntryWithPhonetic}

\begin{EntryWithPhonetic}{不行}{bu4 xing2}{4,6}{⼀、⾏}[HSK 2]
  \definition{adj.}{não funciona; não é bom; falta de capacidade e habilidade; nível baixo}
  \definition{adv.}{profundamente; terrivelmente; extremamente; expressa um grau muito profundo; incrível (usado após o caractere 得 como complemento)}
  \definition{v.}{não servir; não ser permitido; estar fora de questão | estar à beira da morte}
  \seealsoref{得}{de5}
\end{EntryWithPhonetic}

\begin{EntryWithPhonetic}{不许}{bu4 xu3}{4,6}{⼀、⾔}[HSK 5]
  \definition{v.}{não permitir; ser proibido; proibir firmemente | não pode (usado em perguntas retóricas)}
\end{EntryWithPhonetic}

\begin{EntryWithPhonetic}{不一定}{bu4 yi2 ding4}{4,1,8}{⼀、⼀、⼧}[HSK 2]
  \definition{adv.}{talvez; incerto; não tenho certeza; não necessariamente assim; refere-se a algo que não pode ser determinado}
\end{EntryWithPhonetic}

\begin{EntryWithPhonetic}{不一会儿}{bu4 yi2 hui4r5}{4,1,6,2}{⼀、⼀、⼈、⼉}[HSK 2]
  \definition{expr.}{em um momento; em pouco tempo; em breve; depois de algum tempo}
\end{EntryWithPhonetic}

\begin{EntryWithPhonetic}{不宜}{bu4yi2}{4,8}{⼀、⼧}[HSK 7-9]
  \definition{adv.}{não adequado; desaconselhável; inapropriado}
\end{EntryWithPhonetic}

\begin{EntryWithPhonetic}{不已}{bu4yi3}{4,3}{⼀、⼰}[HSK 7-9]
  \definition{v.aux.}{ser infinito; usado depois de um verbo para indicar que ele continua sem parar}
\end{EntryWithPhonetic}

\begin{EntryWithPhonetic}{不以为然}{bu4yi3wei2ran2}{4,4,4,12}{⼀、⼈、⼂、⽕}[HSK 7-9]
  \definition{expr.}{não aceitar como correto; objetar | desaprovar | ter exceção a}
\end{EntryWithPhonetic}

\begin{EntryWithPhonetic}{不由得}{bu4you2de5}{4,5,11}{⼀、⽥、⼻}[HSK 7-9]
  \definition{adv.}{involuntariamente; como uma consequência necessária}
  \definition{v.}{não pode deixar de; não pode se abster de; não pode evitar (fazer algo); não permitir que certos resultados ocorram em determinadas circunstâncias}
\end{EntryWithPhonetic}

\begin{EntryWithPhonetic}{不由自主}{bu4you2zi4zhu3}{4,5,6,5}{⼀、⽥、⾃、⼂}[HSK 7-9]
  \definition{expr.}{não pode ajudar; involuntariamente | além do controle de alguém; apesar de si mesmo; incapaz de se controlar | não posso evitar}
\end{EntryWithPhonetic}

\begin{EntryWithPhonetic}{不予}{bu4yu3}{4,4}{⼀、⼅}[HSK 7-9]
  \definition{v.}{Literário: não dar; negar; recusar; não conceder}
\end{EntryWithPhonetic}

\begin{EntryWithPhonetic}{不约而同}{bu4yue1'er2tong2}{4,6,6,6}{⼀、⽷、⽽、⼝}[HSK 7-9]
  \definition{expr.}{por coincidência; involuntariamente o mesmo; sem acordo prévio, mas de acordo; acontecer de coincidir; opiniões ou ações unânimes sem consulta prévia | fazer ou pensar a mesma coisa sem consulta prévia; acontecer de coincidir}
\end{EntryWithPhonetic}

\begin{EntryWithPhonetic}{不怎么}{bu4 zen3 me5}{4,9,3}{⼀、⼼、⼃}[HSK 6]
  \definition{adv.}{não muito; não particularmente; não exatamente}
\end{EntryWithPhonetic}

\begin{EntryWithPhonetic}{不怎么样}{bu4 zen3 me5 yang4}{4,9,3,10}{⼀、⼼、⼃、⽊}[HSK 6]
  \definition{adj.}{não muito bom; não particularmente bom | muito indiferente; mais ou menos}
\end{EntryWithPhonetic}

\begin{EntryWithPhonetic}{不知}{bu4zhi1}{4,8}{⼀、⽮}[HSK 7-9]
  \definition{v.}{não saber; ser ignorante de; não saber nada sobre; não ter ideia de; não estar ciente de; não ouvir falar de; não ter a mínima ideia}
\end{EntryWithPhonetic}

\begin{EntryWithPhonetic}{不知不觉}{bu4zhi1-bu4jue2}{4,8,4,9}{⼀、⽮、⼀、⾒}[HSK 7-9]
  \definition{expr.}{imperceptivelmente; inconscientemente; involuntariamente; sem saber; sem querer}
\end{EntryWithPhonetic}

\begin{EntryWithPhonetic}{不值}{bu4 zhi2}{4,10}{⼀、⼈}[HSK 6]
  \definition{v.}{não valer a pena}
\end{EntryWithPhonetic}

\begin{EntryWithPhonetic}{不止}{bu4zhi3}{4,4}{⼀、⽌}[HSK 5]
  \definition{adv.}{mais do que; não limitado a; indica mais do que esse valor ou intervalo}
  \definition{v.}{exceder; superar; não ser possível interromper a ação}
\end{EntryWithPhonetic}

\begin{EntryWithPhonetic}{不准}{bu4 zhun3}{4,10}{⼀、⼎}[HSK 7-9]
  \definition{v.}{proibir; impedir; vedar; não permitir}
\end{EntryWithPhonetic}

\begin{EntryWithPhonetic}{不足}{bu4zu2}{4,7}{⼀、⾜}[HSK 5]
  \definition{adj.}{não o bastante; inadequado; insuficiente}
  \definition{s.}{deficiência; inadequação; desvantagens, não é bom o suficiente}
  \definition{v.}{não exceder um determinado número | não valer a pena; ser inferior; não merecer | não pode; não deveria}
\end{EntryWithPhonetic}

\begin{EntryWithPhonetic}{布}{bu4}{5}{⼱}[HSK 3]
  \definition*{s.}{Sobrenome Bu}
  \definition[块,幅,匹]{s.}{tecido; tecido de algodão; algodão, linho ou fibras sintéticas tecidas, que podem ser utilizadas como material para confecção de roupas ou outros objetos | uma moeda antiga | algo parecido com um pano}
  \definition{v.}{declarar; anunciar; publicar; proclamar | divulgar; espalhar por toda parte; difundir amplamente | implantar; dispor; organizar}
\end{EntryWithPhonetic}

\begin{EntryWithPhonetic}{布谷鸟}{bu4gu3niao3}{5,7,5}{⼱、⾕、⿃}
  \definition{s.}{cuco (pássaro)}
  \seealsoref{杜鹃}{du4juan1}
  \seealsoref{杜鹃鸟}{du4juan1niao3}
  \seealsoref{杜宇}{du4yu3}
\end{EntryWithPhonetic}

\begin{EntryWithPhonetic}{布局}{bu4ju2}{5,7}{⼱、⼫}[HSK 7-9]
  \definition{s.}{\emph{layout}; distribuição; arranjo geral; arranjo abrangente: planejamento e arranjo da estrutura geral das coisas; especialmente o arranjo de materiais e tramas na criação artística}
  \definition{v.}{planejar; compor uma imagem, ensaio, etc. (geralmente se refere a escrever, pintar, jogar xadrez, etc.) | posicionar as peças em um tabuleiro de xadrez}
\end{EntryWithPhonetic}

\begin{EntryWithPhonetic}{布满}{bu4 man3}{5,13}{⼱、⽔}[HSK 6]
  \definition{v.}{abundar em; estar cheio de; espalhar-se e preencher um certo espaço}
\end{EntryWithPhonetic}

\begin{EntryWithPhonetic}{布署}{bu4shu3}{5,13}{⼱、⽹}
  \variantof{部署}
\end{EntryWithPhonetic}

\begin{EntryWithPhonetic}{布置}{bu4zhi4}{5,13}{⼱、⽹}[HSK 4]
  \definition{v.}{arrumar; organizar; decorar; colocar adequadamente objetos ou paisagismo, conforme necessário | designar; tomar providências para; dar instruções sobre; organizar trabalho, atividades, etc.}
\end{EntryWithPhonetic}

\begin{EntryWithPhonetic}{步}{bu4}{7}{⽌}[HSK 3]
  \definition*{s.}{Geralmente em nomes de lugares | Sobrenome Bu}[盐步===Yanbu, na província de Guangdong]
  \definition{clas.}{uma unidade antiga para medida de comprimento, equivalente a cinco 尺}
  \definition{s.}{passo; ritmo | etapa; passo | condição; situação; estado | cais; píer | porto; cidade portuária | (geralmente em nomes de lugares)}
  \definition{v.}{caminhar; ir a pé | seguir os passos de alguém | (dialeto) medir com passos | seguir; acompanhar | medir a distância com os passos}
  \seealsoref{尺}{chi3}
\end{EntryWithPhonetic}

\begin{EntryWithPhonetic}{步伐}{bu4fa2}{7,6}{⽌、⼈}[HSK 7-9]
  \definition{s.}{passo; ritmo; os passos de uma pessoa ao caminhar; o ritmo | passo; ritmo; metáfora para a velocidade com que as coisas acontecem}
\end{EntryWithPhonetic}

\begin{EntryWithPhonetic}{步入}{bu4ru4}{7,2}{⽌、⼊}[HSK 7-9]
  \definition{v.}{entrar (em)}[所有人梳妆打扮以步入上流社会。===Todo mundo se veste bem para se encaixar na alta sociedade.]
\end{EntryWithPhonetic}

\begin{EntryWithPhonetic}{步行}{bu4 xing2}{7,6}{⽌、⾏}[HSK 4]
  \definition{v.}{caminhar; ir a pé; andar a pé (diferente de andar de carro, a cavalo, etc.)}
\end{EntryWithPhonetic}

\begin{EntryWithPhonetic}{步骤}{bu4zhou4}{7,17}{⽌、⾺}[HSK 7-9]
  \definition[个]{s.}{passo; movimento; procedimento; medida; o procedimento para que as coisas aconteçam}
\end{EntryWithPhonetic}

\begin{EntryWithPhonetic}{部}{bu4}{10}{⾢}[HSK 3]
  \definition*{s.}{Sobrenome Bu}
  \definition{clas.}{usado para obras de literatura, livros, filmes, etc.}
  \definition[根]{s.}{parte; seção | unidade; ministério; departamento; conselho | sede; matriz; quartel general | tropas; forças | divisão; região}
  \definition{v.}{comandar; liderar}
\end{EntryWithPhonetic}

\begin{EntryWithPhonetic}{部队}{bu4 dui4}{10,4}{⾢、⾩}[HSK 6]
  \definition[支,个]{s.}{militar; exército; forças armadas | tropas; refere-se a uma parte do exército}
\end{EntryWithPhonetic}

\begin{EntryWithPhonetic}{部分}{bu4fen5}{10,4}{⾢、⼑}[HSK 2]
  \definition[个,些,快,份]{s.}{parte; seção; porção; parte do todo; alguns indivíduos dentro do todo | ramo; parte separada de um sistema ou entidade}
\end{EntryWithPhonetic}

\begin{EntryWithPhonetic}{部件}{bu4jian4}{10,6}{⾢、⼈}[HSK 7-9]
  \definition[个]{s.}{peças; partes; componentes; um componente de uma máquina, montado a partir de várias partes | partes; componentes (para caracteres chineses); uma unidade de caracteres chineses composta por traços, por exemplo, 氵, 礻, 口 são todos componentes de caracteres chineses}
\end{EntryWithPhonetic}

\begin{EntryWithPhonetic}{部门}{bu4men2}{10,3}{⾢、⾨}[HSK 3]
  \definition[个]{s.}{departamento; ramo; classe; seção; partes ou unidades que compõem um todo}
\end{EntryWithPhonetic}

\begin{EntryWithPhonetic}{部属}{bu4shu3}{10,12}{⾢、⼫}
  \definition{s.}{afiliado a um ministério | subordinado | tropas sob comando de alguém}
\end{EntryWithPhonetic}

\begin{EntryWithPhonetic}{部署}{bu4shu3}{10,13}{⾢、⽹}[HSK 7-9]
  \definition{v.}{organizar; implantar; dispor; organizar ou dispor de maneira planejada (usado principalmente em grandes aspectos)}
\end{EntryWithPhonetic}

\begin{EntryWithPhonetic}{部位}{bu4wei4}{10,7}{⾢、⼈}[HSK 5]
  \definition{s.}{lugar; posição (usado principalmente para o corpo humano)}
\end{EntryWithPhonetic}

\begin{EntryWithPhonetic}{部下}{bu4xia4}{10,3}{⾢、⼀}
  \definition{s.}{subordinado | tropas sob comando de alguém}
\end{EntryWithPhonetic}

\begin{EntryWithPhonetic}{部长}{bu4 zhang3}{10,4}{⾢、⾧}[HSK 3]
  \definition[个,位,名]{s.}{ministro; chefe de departamento; um alto funcionário do estado encarregado pelo chefe de estado ou chefe executivo do governo da gestão das atividades governamentais de um departamento | chefe de seção; líder tribal}
\end{EntryWithPhonetic}

\begin{EntryWithPhonetic}{部族}{bu4zu2}{10,11}{⾢、⽅}
  \definition{adj.}{tribal}
  \definition{s.}{tribo}
\end{EntryWithPhonetic}

\begin{EntryWithPhonetic}{不}{bu5}{4}{⼀}[HSK 1]
  \definition{adv.}{não (em expressões \{v.\} + 不 + \{v.\})}
  \seeref{bu2}
  \seeref{bu4}
\end{EntryWithPhonetic}

%%%%% EOF %%%%%

