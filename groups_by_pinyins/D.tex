%%%
%%% D
%%%

\section*{D}\addcontentsline{toc}{section}{D}

\begin{EntryWithPhonetic}{搭}{da1}{12}{⼿}[HSK 6]
  \definition{v.}{colocar em prática; construir | ficar pendurado; colocar para cima | entrar em contato; juntar-se | adicionar (mais pessoas, dinheiro, etc.) | levantar algo junto |
pegar (um navio, avião, etc.); viajar (ou ir) por}
  \variantof{褡}
\end{EntryWithPhonetic}

\begin{EntryWithPhonetic}{搭档}{da1dang4}{12,10}{⼿、⽊}[HSK 6]
  \definition[个,名,位]{s.}{parceiro; colega de trabalho}
  \definition{v.}{cooperar; trabalhar em conjunto; formar pares; colaborar; formar uma parceria}
\end{EntryWithPhonetic}

\begin{EntryWithPhonetic}{搭配}{da1pei4}{12,10}{⼿、⾣}[HSK 6]
  \definition{v.}{emparelhar; organizar em pares ou grupos; organizar a distribuição de acordo com certos requisitos | encaixar; combinar}
\end{EntryWithPhonetic}

\begin{EntryWithPhonetic}{搭讪}{da1shan4}{12,5}{⼿、⾔}
  \definition{v.}{bater em alguém | incitar uma conversa | começar a conversar para acabar com um silêncio constrangedor ou uma situação embaraçosa}
\end{EntryWithPhonetic}

\begin{EntryWithPhonetic}{答}{da1}{12}{⽵}[HSK 5]
  \definition{v.}{concordar; responder | responder; prestar atenção}
  \seeref{答}{da2}
\end{EntryWithPhonetic}

\begin{EntryWithPhonetic}{答应}{da1ying5}{12,7}{⽵、⼴}[HSK 2]
  \definition{v.}{responder; retribuir; reagir; retrucar | concordar; prometer; cumprir}
\end{EntryWithPhonetic}

\begin{EntryWithPhonetic}{褡}{da1}{14}{⾐}
  \definition{s.}{bolsa; malote; algibeira | jaqueta sem mangas}
\end{EntryWithPhonetic}

\begin{EntryWithPhonetic}{打}{da2}{5}{⼿}
  \definition{clas./s.}{(empréstimo linguístico) dúzia}
  \seeref{打}{da3}
\end{EntryWithPhonetic}

\begin{EntryWithPhonetic}{达}{da2}{6}{⾡}
  \definition*{s.}{Sobrenome Da}
  \definition{adj.}{eminente; distinto; refere-se a um funcionário distinto; \emph{status} elevado | otimista; de mente aberta}
  \definition{v.}{prolongar | alcançar; atingir; equivaler a | entender completamente; compreender (assuntos) | expressar; comunicar}
\end{EntryWithPhonetic}

\begin{EntryWithPhonetic}{达成}{da2cheng2}{6,6}{⾡、⼽}[HSK 5]
  \definition{v.}{concluir; chegar (a um acordo); conseguir; obter (principalmente como resultado de uma negociação)}
\end{EntryWithPhonetic}

\begin{EntryWithPhonetic}{达到}{da2dao4}{6,8}{⾡、⼑}[HSK 3]
  \definition{v.}{alcançar; atender o padrão; atingir (refere-se principalmente a coisas abstratas ou graus); chegar a um determinado ponto ou grau}
\end{EntryWithPhonetic}

\begin{EntryWithPhonetic}{答}{da2}{12}{⽵}[HSK 5]
  \definition{v.}{responder; dar resposta a; responder a | retribuir; devolver (uma visita, etc.); retribuir um favor feito a alguém por outro; fazer o bem}
  \seeref{答}{da1}
\end{EntryWithPhonetic}

\begin{EntryWithPhonetic}{答案}{da2'an4}{12,10}{⽵、⽊}[HSK 4]
  \definition[个,条,种,些]{s.}{chave; resposta; solução}
\end{EntryWithPhonetic}

\begin{EntryWithPhonetic}{答复}{da2fu4}{12,9}{⽵、⼢}[HSK 5]
  \definition[个]{s.}{resposta; respostas a perguntas ou solicitações}
  \definition{v.}{responder; dar uma resposta}
\end{EntryWithPhonetic}

\begin{EntryWithPhonetic}{打}{da3}{5}{⼿}[HSK 1,4,5]
  \definition{prep.}{de; desde; ponto de partida que indica lugar, tempo ou extensão; indica rotas e locais percorridos | devido a; origem da introdução de coisas novas}
  \definition{v.}{golpear; acertar; bater | quebrar; esmagar | lutar; atacar; espancar | entrar com uma ação judicial; negociar; fazer representações | construir; edificar | fabricar (em uma ferraria); forjar | misturar; mexer; bater | amarrar; embalar | tricotar; tecer | desenhar; pintar; deixar uma marca; imprimir | abrir; perfurar; cavar | içar; levantar | enviar; despachar; projetar | emitir ou receber (um certificado, etc.) | remover; livrar-se de | colher; tirar; retirar | comprar | capturar; caçar | reunir; coletar; colher; recolher através de ações como cortar e podar | estimar; calcular; contar; determinar | fazer; envolver-se em | jogar algum tipo de jogo | expressar certos movimentos corporais | adotar; usar; adotar uma determinada abordagem | pegar (um táxi) | indicar a melhora de seu estado mental; melhorar o estado mental}
  \seeref{打}{da2}
\end{EntryWithPhonetic}

\begin{EntryWithPhonetic}{打败}{da3 bai4}{5,8}{⼿、⾒}[HSK 4]
  \definition{v.}{derrotar; vencer; piorar | sofrer uma derrota; ser derrotado}
\end{EntryWithPhonetic}

\begin{EntryWithPhonetic}{打扮}{da3ban5}{5,7}{⼿、⼿}[HSK 5]
  \definition{s.}{estilo de se vestir; o modo de se vestir; as roupas que se usa}
  \definition{v.}{vestir-se bem; maquiar-se; dar uma boa aparência e vestir-se bem; adornar}
\end{EntryWithPhonetic}

\begin{EntryWithPhonetic}{打包}{da3bao1}{5,5}{⼿、⼓}[HSK 5]
  \definition{v.}{levar a comida embora; levar para viagem; refere-se especificamente a comer em um restaurante e levar as sobras em uma caixa, sacola ou outro recipiente | embalar; empacotar | desembalar; desempacotar}
\end{EntryWithPhonetic}

\begin{EntryWithPhonetic}{打车}{da3 che1}{5,4}{⼿、⾞}[HSK 1]
  \definition{v.}{pegar um táxi; chamar um táxi; dar sinal para um táxi}
\end{EntryWithPhonetic}

\begin{EntryWithPhonetic}{打的}{da3di1}{5,8}{⼿、⽩}
  \definition{v.+compl.}{(coloquial) pegar um táxi | ir de táxi}
\end{EntryWithPhonetic}

\begin{EntryWithPhonetic}{打电话}{da3 dian4 hua4}{5,5,8}{⼿、⽥、⾔}[HSK 1]
  \definition{v.}{telefonar; fazer uma chamada telefônica; dar um telefonema}
  \seealsoref{给……打电话}{gei3 da3 dian4 hua4}
\end{EntryWithPhonetic}

\begin{EntryWithPhonetic}{打动}{da3 dong4}{5,6}{⼿、⼒}[HSK 6]
  \definition{v.}{mover; tocar}[这番话打动了她的心。===Essas palavras tocaram seu coração.]
\end{EntryWithPhonetic}

\begin{EntryWithPhonetic}{打断}{da3 duan4}{5,11}{⼿、⽄}[HSK 6]
  \definition{v.}{interromper uma atividade (fala; pensamento ou ação) | fraturar (osso do corpo)  com força | arrombar; bater com força para quebrar}
\end{EntryWithPhonetic}

\begin{EntryWithPhonetic}{打发}{da3 fa5}{5,5}{⼿、⼜}[HSK 6]
  \definition{v.}{enviar; despachar | dispensar; mandar embora | passar o tempo; matar o tempo}
\end{EntryWithPhonetic}

\begin{EntryWithPhonetic}{打工}{da3gong1}{5,3}{⼿、⼯}[HSK 2]
  \definition{v.}{contratar para trabalhar; trabalhar em tempo parcial; realizar trabalho manual (para alguém, geralmente temporariamente)}
\end{EntryWithPhonetic}

\begin{EntryWithPhonetic}{打工人}{da3gong1ren2}{5,3,2}{⼿、⼯、⼈}
  \definition{s.}{trabalhador}
\end{EntryWithPhonetic}

\begin{EntryWithPhonetic}{打官司}{da3guan1si5}{5,8,5}{⼿、⼧、⼝}[HSK 6]
  \definition{v.+compl.}{ir ao tribunal (ou à lei); envolver-se em um processo judicial}
\end{EntryWithPhonetic}

\begin{EntryWithPhonetic}{打击}{da3ji1}{5,5}{⼿、⼐}[HSK 5]
  \definition{v.}{golpear; atacar; reprimir; atacar para frustrar; machucar | bater; bater (em um tambor, etc.); golpear ou bater em algo}
\end{EntryWithPhonetic}

\begin{EntryWithPhonetic}{打架}{da3jia4}{5,9}{⼿、⽊}[HSK 5]
  \definition{v.+compl.}{brigar; discutir; entrar em conflito | contradizer; conflitar; ser inconsistente}
\end{EntryWithPhonetic}

\begin{EntryWithPhonetic}{打搅}{da3jiao3}{5,12}{⼿、⼿}
  \definition{v.}{perturbar | incomodar}
\end{EntryWithPhonetic}

\begin{EntryWithPhonetic}{打结}{da3jie2}{5,9}{⼿、⽷}
  \definition{v.}{dar um nó | amarrar}
\end{EntryWithPhonetic}

\begin{EntryWithPhonetic}{打开}{da3 kai1}{5,4}{⼿、⼶}[HSK 1]
  \definition{v.}{abrir; desdobrar; desenrolar | descobrir; revelar; desvendar | ativar; ligar; ligar o circuito | romper | abrir-se; espalhar-se; expandir; ampliar | abrir; iniciar o funcionamento do software, etc.}
\end{EntryWithPhonetic}

\begin{EntryWithPhonetic}{打瞌睡}{da3ke1shui4}{5,15,13}{⼿、⽬、⽬}
  \definition{v.}{cochilar}
\end{EntryWithPhonetic}

\begin{EntryWithPhonetic}{打雷}{da3 lei2}{5,13}{⼿、⾬}[HSK 4]
  \definition{v.}{trovejar; produzir ruídos altos quando as nuvens descarregam eletricidade}
\end{EntryWithPhonetic}

\begin{EntryWithPhonetic}{打猎}{da3lie4}{5,11}{⼿、⽝}
  \definition{v.}{ir caçar}
\end{EntryWithPhonetic}

\begin{EntryWithPhonetic}{打骂}{da3ma4}{5,9}{⼿、⾺}
  \definition{v.}{bater e repreender}
\end{EntryWithPhonetic}

\begin{EntryWithPhonetic}{打磨}{da3mo2}{5,16}{⼿、⽯}
  \definition{v.}{polir | fazer brilhar}
\end{EntryWithPhonetic}

\begin{EntryWithPhonetic}{打牌}{da3 pai2}{5,12}{⼿、⽚}[HSK 6]
  \definition{v.}{jogar cartas, usar cartas para entretenimento ou jogos de azar}
\end{EntryWithPhonetic}

\begin{EntryWithPhonetic}{打屁股}{da3pi4gu5}{5,7,8}{⼿、⼫、⾁}
  \definition{v.}{dar um tapa no bumbum de alguém}
\end{EntryWithPhonetic}

\begin{EntryWithPhonetic}{打破}{da3 po4}{5,10}{⼿、⽯}[HSK 3]
  \definition{v.}{quebrar; esmagar; quebrar recordes, regras ou restrições existentes, etc.}
\end{EntryWithPhonetic}

\begin{EntryWithPhonetic}{打球}{da3 qiu2}{5,11}{⼿、⽟}[HSK 1]
  \definition{v.}{jogar bola (com as mãos) | jogar (basquetebol, handbol, etc.) | jogar um jogo de bola}
\end{EntryWithPhonetic}

\begin{EntryWithPhonetic}{打扰}{da3rao3}{5,7}{⼿、⼿}[HSK 5]
  \definition{v.}{perturbar; incomodar; interferir no trabalho normal, na vida ou no que as outras pessoas estão fazendo, etc. | usado para expressar um pedido de desculpas por ajuda; gratidão por ajuda; hospitalidade recebida}
\end{EntryWithPhonetic}

\begin{EntryWithPhonetic}{打扫}{da3sao3}{5,6}{⼿、⼿}[HSK 4]
  \definition{v.}{varrer; limpar; varrer para limpar}
\end{EntryWithPhonetic}

\begin{EntryWithPhonetic}{打算}{da3suan4}{5,14}{⼿、⽵}[HSK 2]
  \definition[个,项]{s.}{plano; intenção; consideração; cálculo; ideias sobre a direção e os métodos da ação; pensamentos}
  \definition{v.}{pretender; planejar; calcular; considerar com antecedência}
\end{EntryWithPhonetic}

\begin{EntryWithPhonetic}{打听}{da3ting5}{5,7}{⼿、⼝}[HSK 3]
  \definition{v.}{perguntar sobre; indagar sobre; obter uma linha sobre}
\end{EntryWithPhonetic}

\begin{EntryWithPhonetic}{打压}{da3ya1}{5,6}{⼿、⼚}
  \definition{v.}{reprimir | derrotar}
\end{EntryWithPhonetic}

\begin{EntryWithPhonetic}{打印}{da3yin4}{5,5}{⼿、⼙}[HSK 2]
  \definition{v.}{imprimir; imprimir em papel ou outro suporte de gravação, como uma impressora}
\end{EntryWithPhonetic}

\begin{EntryWithPhonetic}{打印机}{da3 yin4 ji1}{5,5,6}{⼿、⼙、⽊}[HSK 6]
  \definition[个,部,台]{s.}{impressora; uma máquina de escrever controlada por um microcomputador, sem teclado, que converte códigos de caracteres em caracteres e os imprime}
\end{EntryWithPhonetic}

\begin{EntryWithPhonetic}{打造}{da3 zao4}{5,10}{⼿、⾡}[HSK 6]
  \definition{v.}{forjar (trabalhar em metal); fabricar (principalmente objetos de metal) | fazer; criar; construir; desenvolver}
\end{EntryWithPhonetic}

\begin{EntryWithPhonetic}{打折}{da3zhe2}{5,7}{⼿、⼿}[HSK 4]
  \definition{v.+compl.}{dar desconto; dar um desconto; vender produtos a um preço reduzido em uma determinada porcentagem do preço original; metáfora para não cumprir 100\% do que foi originalmente padronizado ou prometido}
\end{EntryWithPhonetic}

\begin{EntryWithPhonetic}{打针}{da3zhen1}{5,7}{⼿、⾦}[HSK 4]
  \definition{v.+compl.}{dar ou receber uma injeção; injetar um medicamento líquido em um organismo com uma seringa}
\end{EntryWithPhonetic}

\begin{EntryWithPhonetic}{大}{da4}{3}{⼤}[HSK 1][Kangxi 37]
  \definition*{s.}{Sobrenome Da}
  \definition{adj.}{grande; amplo; grande em volume, área, etc. | mais velho; em primeiro lugar no ranking | tamanho; descreve o grau de grandeza | usado em certas épocas do ano, condições climáticas, feriados ou antes de um determinado momento, para enfatizar | o tempo mais distante; há muito tempo}
  \definition{adv.}{grandemente; totalmente; expressa um grau muito profundo | não muito; não frequentemente; usado após 不, indica um grau baixo ou poucas vezes}
  \definition{s.}{adulto; crescido; pessoas idosas | pai | irmão do pai de alguém; tio}
  \seeref{大}{dai4}
  \seealsoref{不}{bu4}
\end{EntryWithPhonetic}

\begin{EntryWithPhonetic}{大巴}{da4 ba1}{3,4}{⼤、⼰}[HSK 4]
  \definition{s.}{ônibus}
\end{EntryWithPhonetic}

\begin{EntryWithPhonetic}{大部分}{da4 bu4 fen4}{3,10,4}{⼤、⾢、⼑}[HSK 2]
  \definition[把]{s.}{a maioria; a maior parte; em grande parte; refere-se a uma quantidade superior a metade do total}
\end{EntryWithPhonetic}

\begin{EntryWithPhonetic}{大城}{da4cheng2}{3,9}{⼤、⼟}
  \definition*[个,座]{s.}{Condado de Dacheng em Langfang 廊坊, Hebei | Município de Tacheng no condado de Changhua | Condado de Changhua, Taiwan}
  \seealsoref{廊坊}{lang2fang2}
\end{EntryWithPhonetic}

\begin{EntryWithPhonetic}{大大}{da4 da4}{3,3}{⼤、⼤}[HSK 2]
  \definition{adv.}{grandemente; enormemente; enfatizar grande quantidade ou grau profundo}
\end{EntryWithPhonetic}

\begin{EntryWithPhonetic}{大胆}{da4 dan3}{3,9}{⼤、⾁}[HSK 5]
  \definition{adj.}{ousado; atrevido; audacioso; corajoso; destemido}
\end{EntryWithPhonetic}

\begin{EntryWithPhonetic}{大道}{da4 dao4}{3,12}{⼤、⾡}[HSK 6]
  \definition*{s.}{O Grande Tao; O Grande Caminho}
  \definition[条]{s.}{estrada principal | o caminho da justiça | avenida | rua principal}
\end{EntryWithPhonetic}

\begin{EntryWithPhonetic}{大抵}{da4di3}{3,8}{⼤、⼿}
  \definition{adv.}{no geral; de um modo geral; provavelmente; principalmente}
\end{EntryWithPhonetic}

\begin{EntryWithPhonetic}{大都}{da4 dou1}{3,10}{⼤、⾢}[HSK 5]
  \seeref{大都}{da4 du1}
\end{EntryWithPhonetic}

\begin{EntryWithPhonetic}{大豆}{da4dou4}{3,7}{⼤、⾖}
  \definition{s.}{soja}
\end{EntryWithPhonetic}

\begin{EntryWithPhonetic}{大都}{da4 du1}{3,10}{⼤、⾢}
  \definition*{s.}{Dadu, capital da China durante a Dinastia Yuan (1280-1368), atual Pequim}
  \definition{adv.}{em sua maior parte; na maior parte; indica que a maioria das pessoas ou coisas em um determinado intervalo tem a mesma natureza e características; também pronunciado como \dpy{da4dou1} na língua falada}
  \seeref{大都}{da4 dou1}
\end{EntryWithPhonetic}

\begin{EntryWithPhonetic}{大多}{da4 duo1}{3,6}{⼤、⼣}[HSK 4]
  \definition{adv.}{majoritariamente; em sua maior parte; em sua maioria; em grande parte}
\end{EntryWithPhonetic}

\begin{EntryWithPhonetic}{大多数}{da4 duo1 shu4}{3,6,13}{⼤、⼣、⽁}[HSK 2]
  \definition{s.}{grande maioria; vasta maioria; a maior parte; mais da metade, um número significativo}
\end{EntryWithPhonetic}

\begin{EntryWithPhonetic}{大方}{da4fang1}{3,4}{⼤、⽅}
  \definition{s.}{generosidades; liberalidades | estudioso; pessoas com conhecimento especializado | um tipo de chá verde, produzido principalmente no Condado de Shexian, Província de Anhui, Condado de Chun'an, Província de Zhejiang, etc.}
\end{EntryWithPhonetic}

\begin{EntryWithPhonetic}{大方}{da4fang5}{3,4}{⼤、⽅}[HSK 4]
  \definition{adj.}{generoso | não afetado; natural e equilibrado |  de bom gosto}
  \seeref{大方}{da4fang1}
\end{EntryWithPhonetic}

\begin{EntryWithPhonetic}{大夫}{da4fu1}{3,4}{⼤、⼤}
  \definition[个,位,名]{s.}{oficial sênior (na China Imperial)}
  \seeref{大夫}{dai4fu5}
\end{EntryWithPhonetic}

\begin{EntryWithPhonetic}{大概}{da4gai4}{3,13}{⼤、⽊}[HSK 3]
  \definition{adj.}{geral; grosseiro; aproximado; não é muito preciso ou muito detalhado}
  \definition{adv.}{sobre; provavelmente; estimativas ou suposições imprecisas sobre eventos, quantidades, tempo, localização, etc.| geralmente; brevemente; não muito seriamente, casualmente; não muito cuidadosamente}
  \definition{s.}{ideia geral; esboço geral; conteúdo geral ou situação}
\end{EntryWithPhonetic}

\begin{EntryWithPhonetic}{大纲}{da4 gang1}{3,7}{⼤、⽷}[HSK 5]
  \definition{s.}{esboço; compêndio; programa de estudos; resumo; fundamentos da organização sistemática de conteúdos (livros, discursos, programas, etc.)}
\end{EntryWithPhonetic}

\begin{EntryWithPhonetic}{大哥}{da4 ge1}{3,10}{⼤、⼝}[HSK 4]
  \definition{s.}{irmão mais velho | \emph{big brother}; tratamento educado para um homem da mesma idade que você | líder de gangue; pessoa mais poderosa em uma organização que realiza atividades ilegais na sociedade}
\end{EntryWithPhonetic}

\begin{EntryWithPhonetic}{大规模}{da4 gui1 mo2}{3,8,14}{⼤、⾒、⽊}[HSK 4]
  \definition{adj.}{em larga escala; extensivo; maciço; massivo}
  \definition{adv.}{em larga escala; extensivo; maciço; massa}
\end{EntryWithPhonetic}

\begin{EntryWithPhonetic}{大海}{da4 hai3}{3,10}{⼤、⽔}[HSK 2]
  \definition{s.}{o mar; o oceano; o mar aberto, ou seja, a parte do oceano que não está fechada entre cabos nem incluída em estreitos}
\end{EntryWithPhonetic}

\begin{EntryWithPhonetic}{大后天}{da4 hou4 tian1}{3,6,4}{⼤、⼝、⼤}
  \definition{s.}{daqui a três dias}
\end{EntryWithPhonetic}

\begin{EntryWithPhonetic}{大黄}{da4huang2}{3,11}{⼤、⿈}
  \definition{s.}{ruibarbo chinês}
\end{EntryWithPhonetic}

\begin{EntryWithPhonetic}{大会}{da4 hui4}{3,6}{⼤、⼈}[HSK 4]
  \definition[场,次,个,届]{s.}{sessão plenária; reunião geral de membros; reuniões convocadas por partidos políticos socialistas | reunião de massa; comício de massa}
\end{EntryWithPhonetic}

\begin{EntryWithPhonetic}{大伙儿}{da4huo3r5}{3,6,2}{⼤、⼈、⼉}[HSK 5]
  \definition{pron.}{todos nós; todos vocês; todo mundo; todos; equivalente a 大家}
  \seealsoref{大家}{da4jia1}
\end{EntryWithPhonetic}

\begin{EntryWithPhonetic}{大家}{da4jia1}{3,10}{⼤、⼧}[HSK 2]
  \definition{pron.}{todos; toda a gente; refere-se a todas as pessoas dentro de um determinado âmbito}
  \definition{s.}{grande mestre; autoridade; especialista renomado | família nobre; família rica e influente; família tradicional}
\end{EntryWithPhonetic}

\begin{EntryWithPhonetic}{大奖赛}{da4 jiang3 sai4}{3,9,14}{⼤、⼤、⾙}[HSK 5]
  \definition{s.}{grande competição; grande prêmio; \emph{grand prix}}
\end{EntryWithPhonetic}

\begin{EntryWithPhonetic}{大街}{da4 jie1}{3,12}{⼤、⾏}[HSK 6]
  \definition[条,个]{s.}{avenida; rua; rua principal}
\end{EntryWithPhonetic}

\begin{EntryWithPhonetic}{大姐}{da4 jie3}{3,8}{⼤、⼥}[HSK 4]
  \definition[个,位]{s.}{irmã mais velha (também um termo educado para se dirigir a uma garota ou mulher um pouco mais velha do que a pessoa que fala)}
\end{EntryWithPhonetic}

\begin{EntryWithPhonetic}{大口}{da4kou3}{3,3}{⼤、⼝}
  \definition{s.}{grande bocado (de comida, bebida, fumo, etc.)}
\end{EntryWithPhonetic}

\begin{EntryWithPhonetic}{大力}{da4 li4}{3,2}{⼤、⼒}[HSK 6]
  \definition{adv.}{energicamente; vigorosamente; indica uso de grande força}
  \definition{s.}{grande força, poder}
\end{EntryWithPhonetic}

\begin{EntryWithPhonetic}{大量}{da4 liang4}{3,12}{⼤、⾥}[HSK 2]
  \definition{adj.}{numeroso; em grande quantidade; grande em número ou quantidade | generoso; magnânimo; descreve uma pessoa que não fica zangada quando os outros cometem erros e que costuma perdoar os outros}
\end{EntryWithPhonetic}

\begin{EntryWithPhonetic}{大楼}{da4 lou2}{3,13}{⼤、⽊}[HSK 4]
  \definition[座,幢]{s.}{edifício; mansão; edifício de vários andares disponível para uso residencial e comercial}
\end{EntryWithPhonetic}

\begin{EntryWithPhonetic}{大陆}{da4 lu4}{3,7}{⼤、⾩}[HSK 4]
  \definition*{s.}{China continental; refere-se especificamente à vasta porção terrestre do território da China}
  \definition[个,块]{s.}{terra firme; continente; vasta extensão de terra}
\end{EntryWithPhonetic}

\begin{EntryWithPhonetic}{大妈}{da4 ma1}{3,6}{⼤、⼥}[HSK 4]
  \definition[个,位]{s.}{tia; esposa do irmão mais velho do pai | tratamento respeitoso às mulheres idosas}
\end{EntryWithPhonetic}

\begin{EntryWithPhonetic}{大马}{da4ma3}{3,3}{⼤、⾺}
  \definition*{s.}{Malásia}
\end{EntryWithPhonetic}

\begin{EntryWithPhonetic}{大门}{da4 men2}{3,3}{⼤、⾨}[HSK 2]
  \definition{s.}{portão; entrada; portão grande, referindo-se especificamente ao portão principal de um edifício (como uma casa, pátio ou parque) que dá para a rua (em contraste com o segundo portão e as portas das várias divisões)}
\end{EntryWithPhonetic}

\begin{EntryWithPhonetic}{大米}{da4 mi3}{3,6}{⼤、⽶}[HSK 6]
  \definition[颗,粒,斤,包,袋]{s.}{arroz; arroz descascado; arroz bom}
\end{EntryWithPhonetic}

\begin{EntryWithPhonetic}{大脑}{da4 nao3}{3,10}{⼤、⾁}[HSK 5]
  \definition{s.}{cérebro; encéfalo}
\end{EntryWithPhonetic}

\begin{EntryWithPhonetic}{大批}{da4 pi1}{3,7}{⼤、⼿}[HSK 6]
  \definition{num.}{grandes quantidades de; exércitos; inundações}[大批书籍被印刷出来。===Grandes quantidades de livros foram impressas.]
\end{EntryWithPhonetic}

\begin{EntryWithPhonetic}{大前天}{da4qian2tian1}{3,9,4}{⼤、⼑、⼤}
  \definition{adv.}{três dias atrás}
\end{EntryWithPhonetic}

\begin{EntryWithPhonetic}{大全}{da4quan2}{3,6}{⼤、⼊}
  \definition{s.}{coleção abrangente}
\end{EntryWithPhonetic}

\begin{EntryWithPhonetic}{大人}{da4 ren2}{3,2}{⼤、⼈}[HSK 2]
  \definition[个,位]{s.}{senhor; ilustre; sua excelência; antigo título honorífico para funcionários públicos | adulto; crescido; maduro;}
\end{EntryWithPhonetic}

\begin{EntryWithPhonetic}{大赛}{da4 sai4}{3,14}{⼤、⾙}[HSK 6]
  \definition{s.}{grande torneio; competição importante; um evento de grande porte e alto nível; um grande evento}
\end{EntryWithPhonetic}

\begin{EntryWithPhonetic}{大神}{da4shen2}{3,9}{⼤、⽰}
  \definition{s.}{deidade | (gíria da Internet) guru | \emph{expert} | gênio}
\end{EntryWithPhonetic}

\begin{EntryWithPhonetic}{大声}{da4 sheng1}{3,7}{⼤、⼠}[HSK 2]
  \definition{adj.}{alto; volume alto; em voz alta}
\end{EntryWithPhonetic}

\begin{EntryWithPhonetic}{大师}{da4 shi1}{3,6}{⼤、⼱}[HSK 6]
  \definition*{s.}{Grande Mestre, título de cortesia usado para se dirigir a um monge budista}
  \definition{s.}{grande mestre; mestre; maestro; uma pessoa com realizações profundas}
\end{EntryWithPhonetic}

\begin{EntryWithPhonetic}{大使}{da4 shi3}{3,8}{⼤、⼈}[HSK 6]
  \definition[位,任]{s.}{embaixador; o representante diplomático de mais alto nível enviado por um país a outro país}
  \seealsoref{全称特命全权大使}{quan2cheng1 te4ming4 quan2quan2 da4shi3}
\end{EntryWithPhonetic}

\begin{EntryWithPhonetic}{大使馆}{da4shi3guan3}{3,8,11}{⼤、⼈、⾷}[HSK 3]
  \definition[座,个]{s.}{embaixada; uma representação diplomática de um país em outro país, chefiada por um embaixador}
\end{EntryWithPhonetic}

\begin{EntryWithPhonetic}{大事}{da4 shi4}{3,8}{⼤、⼅}[HSK 5]
  \definition{adv.}{em grande escala; em grande estilo; em grande parte}
  \definition[件,桩]{s.}{grande evento; grande acontecimento; assunto importante; grande questão; algo importante | situação geral}
\end{EntryWithPhonetic}

\begin{EntryWithPhonetic}{大蒜}{da4suan4}{3,13}{⼤、⾋}
  \definition[瓣,头]{s.}{alho}
\end{EntryWithPhonetic}

\begin{EntryWithPhonetic}{大厅}{da4 ting1}{3,4}{⼤、⼚}[HSK 5]
  \definition{s.}{\emph{hall}; saguão, uma sala grande para reuniões ou atividades em um edifício de grande porte}
\end{EntryWithPhonetic}

\begin{EntryWithPhonetic}{大腿}{da4tui3}{3,13}{⼤、⾁}
  \definition{s.}{coxa}
\end{EntryWithPhonetic}

\begin{EntryWithPhonetic}{大王}{da4wang2}{3,4}{⼤、⽟}
  \definition{s.}{rei; magnata | pessoa da mais alta classe ou habilidade em algo; ás | barões | pessoa com habilidade especializada em algo}
  \seeref{大王}{dai4wang5}
\end{EntryWithPhonetic}

\begin{EntryWithPhonetic}{大戏}{da4xi4}{3,6}{⼤、⼽}
  \definition*{s.}{Drama, Ópera Chinesa}
\end{EntryWithPhonetic}

\begin{EntryWithPhonetic}{大象}{da4xiang4}{3,11}{⼤、⾗}[HSK 5]
  \definition[只,头,群,个]{s.}{elefante}
\end{EntryWithPhonetic}

\begin{EntryWithPhonetic}{大小}{da4 xiao3}{3,3}{⼤、⼩}[HSK 2]
  \definition{adv.}{no mínimo; grande ou pequeno (geralmente pequeno), significa que ainda pode ser considerado}
  \definition[家]{s.}{tamanho; o grau de tamanho | ordem de senioridade; hierarquia | adultos e crianças | grande ou pequeno}
\end{EntryWithPhonetic}

\begin{EntryWithPhonetic}{大猩猩}{da4xing1xing5}{3,12,12}{⼤、⽝、⽝}
  \definition{s.}{gorila}
\end{EntryWithPhonetic}

\begin{EntryWithPhonetic}{大型}{da4xing2}{3,9}{⼤、⼟}[HSK 4]
  \definition{adj.}{grande; em larga escala; tamanho e volume grandes | larga escala (importante e influente)}
\end{EntryWithPhonetic}

\begin{EntryWithPhonetic}{大熊猫}{da4 xiong2 mao1}{3,14,11}{⼤、⽕、⽝}[HSK 5]
  \definition{s.}{panda gigante}
\end{EntryWithPhonetic}

\begin{EntryWithPhonetic}{大学}{da4 xue2}{3,8}{⼤、⼦}[HSK 1]
  \definition[所,座]{s.}{universidade; faculdade; tipo de instituição de ensino superior que, na China, geralmente se refere a uma universidade abrangente}
\end{EntryWithPhonetic}

\begin{EntryWithPhonetic}{大学生}{da4 xue2 sheng1}{3,8,5}{⼤、⼦、⽣}[HSK 1]
  \definition[名,个]{s.}{estudante universitário; estudante de faculdade; estudantes de graduação ou cursos técnicos em instituições de ensino superior}
\end{EntryWithPhonetic}

\begin{EntryWithPhonetic}{大洋洲}{da4yang2zhou1}{3,9,9}{⼤、⽔、⽔}
  \definition*{s.}{Oceania}
\end{EntryWithPhonetic}

\begin{EntryWithPhonetic}{大爷}{da4 ye2}{3,6}{⼤、⽗}
  \definition[个,位]{s.}{Coloquial: irmão mais velho do pai; tio | tratamento respeitoso para um homem mais velho}
  \seeref{大爷}{da4 ye5}
\end{EntryWithPhonetic}

\begin{EntryWithPhonetic}{大爷}{da4 ye5}{3,6}{⼤、⽗}[HSK 4]
  \definition[个,位]{s.}{irmão mais velho do pai; tio | tio (homenagem aos homens mais velhos)}
  \seeref{大爷}{da4 ye2}
\end{EntryWithPhonetic}

\begin{EntryWithPhonetic}{大衣}{da4 yi1}{3,6}{⼤、⾐}[HSK 2]
  \definition[件,个]{s.}{sobretudo; casaco; casaco ocidental mais comprido}
\end{EntryWithPhonetic}

\begin{EntryWithPhonetic}{大于}{da4 yu2}{3,3}{⼤、⼆}[HSK 5]
  \definition{v.}{ser maior, mais numeroso, mais importante, etc. do que}
\end{EntryWithPhonetic}

\begin{EntryWithPhonetic}{大雨}{da4yu3}{3,8}{⼤、⾬}
  \definition[场]{s.}{chuva pesada, forte}
\end{EntryWithPhonetic}

\begin{EntryWithPhonetic}{大约}{da4yue1}{3,6}{⼤、⽷}[HSK 3]
  \definition{adv.}{aproximadamente; sobre; estimativa não muito precisa| provavelmente; expressar suposições sobre a situação}
\end{EntryWithPhonetic}

\begin{EntryWithPhonetic}{大战}{da4zhan4}{3,9}{⼤、⼽}
  \definition{s.}{guerra}
  \definition{v.}{guerrear | lutar em uma guerra}
\end{EntryWithPhonetic}

\begin{EntryWithPhonetic}{大致}{da4zhi4}{3,10}{⼤、⾄}[HSK 5]
  \definition{adj.}{geral; no todo}
  \definition{adv.}{grosso modo; aproximadamente; mais ou menos; indica uma estimativa aproximada da situação}
\end{EntryWithPhonetic}

\begin{EntryWithPhonetic}{大众}{da4 zhong4}{3,6}{⼤、⼈}[HSK 4]
  \definition{s.}{massas; população; pessoas comuns; público em geral}
\end{EntryWithPhonetic}

\begin{EntryWithPhonetic}{大自然}{da4 zi4 ran2}{3,6,12}{⼤、⾃、⽕}[HSK 2]
  \definition{s.}{natureza}
\end{EntryWithPhonetic}

\begin{EntryWithPhonetic}{呆}{dai1}{7}{⼝}[HSK 5]
  \definition*{s.}{Sobrenome Dai}
  \definition{adj.}{maçante; de raciocínio lento | em branco; de madeira; rígido; inflexível}
  \definition{v.}{ficar; permanecer}
\end{EntryWithPhonetic}

\begin{EntryWithPhonetic}{待}{dai1}{9}{⼻}[HSK 5]
  \definition{v.}{ficar; permanecer | ir além (de um período de tempo)}
  \seeref{待}{dai4}
\end{EntryWithPhonetic}

\begin{EntryWithPhonetic}{待会儿}{dai1 hui4r5}{9,6,2}{⼻、⼈、⼉}[HSK 6]
  \definition{adv.}{em um momento; depois de um tempo | mais tarde; depois}
\end{EntryWithPhonetic}

\begin{EntryWithPhonetic}{歹}{dai3}{4}{⽍}[Kangxi 78]
  \definition{adj.}{maligno; cruel; ruim, refere-se principalmente a pessoas e coisas}
\end{EntryWithPhonetic}

\begin{EntryWithPhonetic}{歹徒}{dai3tu2}{4,10}{⽍、⼻}
  \definition{s.}{malfeitor | gangster | bandido}
\end{EntryWithPhonetic}

\begin{EntryWithPhonetic}{逮}{dai3}{11}{⾡}
  \definition{v.}{(coloquial) pegar, aproveitar, capturar}
  \seeref{逮}{dai4}
\end{EntryWithPhonetic}

\begin{EntryWithPhonetic}{大}{dai4}{3}{⼤}[Kangxi 37]
  \definition{s.}{usado em 大夫: médico, doutor | usado em 大王: grande rei}
  \seeref{大}{da4}
  \seealsoref{大夫}{dai4fu5}
  \seealsoref{大王}{dai4wang5}
\end{EntryWithPhonetic}

\begin{EntryWithPhonetic}{大夫}{dai4fu5}{3,4}{⼤、⼤}[HSK 3]
  \definition[个,位,名]{s.}{médico, doutor}
  \seeref{大夫}{da4fu1}
\end{EntryWithPhonetic}

\begin{EntryWithPhonetic}{大王}{dai4wang5}{3,4}{⼤、⽟}
  \definition{s.}{magnata; barões | barão ladrão (em ópera, histórias antigas)}
  \seeref{大王}{da4wang2}
\end{EntryWithPhonetic}

\begin{EntryWithPhonetic}{代}{dai4}{5}{⼈}[HSK 3]
  \definition*{s.}{Sobrenome Dai}
  \definition{s.}{dinastia | geração; hierarquia familiar | era; o segundo nível da divisão geológica é o período, acima do qual está a era e abaixo do qual está o período, por exemplo, o Paleozóico, o Mesozóico e o Cenozóico pertencem à era Phanerozoico | período histórico; época}
  \definition{v.}{tomar o lugar de; estar no lugar de | agir em nome de; exercer}
\end{EntryWithPhonetic}

\begin{EntryWithPhonetic}{代表}{dai4biao3}{5,8}{⼈、⾐}[HSK 3]
  \definition[位,名,个,些]{s.}{deputado; delegado; representante; pessoas eleitas para representar eleitores ou expressar opiniões, ou pessoas encarregadas ou designadas para representar indivíduos, grupos ou governos ou expressar opiniões | representante oficial; pessoas ou coisas que refletem as características comuns de um grupo específico}
  \definition{v.}{representar; defender | usar pessoas ou coisas para expressar um significado ou conceito específico}
\end{EntryWithPhonetic}

\begin{EntryWithPhonetic}{代表团}{dai4 biao3 tuan2}{5,8,6}{⼈、⾐、⼞}[HSK 3]
  \definition[个]{s.}{delegação; contingente; um grupo temporário de grande dimensão formado para participar de uma determinada atividade em nome de um país, governo ou outra organização social}
\end{EntryWithPhonetic}

\begin{EntryWithPhonetic}{代称}{dai4cheng1}{5,10}{⼈、⽲}
  \definition{s.}{nome alternativo | antonomásia}
  \definition{v.}{referir-se a algo ou alguém por outro nome}
\end{EntryWithPhonetic}

\begin{EntryWithPhonetic}{代价}{dai4jia4}{5,6}{⼈、⼈}[HSK 5]
  \definition[种,个]{s.}{preço; material, energia gasta ou sacrifícios feitos para atingir um objetivo | custo; preço; dinheiro pago para obter algo}
\end{EntryWithPhonetic}

\begin{EntryWithPhonetic}{代理}{dai4li3}{5,11}{⼈、⽟}[HSK 5]
  \definition{v.}{agir em nome de alguém em uma posição de responsabilidade; substituir alguém | agir como procurador; agir como agente; ser encarregado pelas partes de realizar atividades e conduzir assuntos em seu nome dentro do escopo de sua autorização}
\end{EntryWithPhonetic}

\begin{EntryWithPhonetic}{代替}{dai4ti4}{5,12}{⼈、⽈}[HSK 4]
  \definition{v.}{substituir; substituir por; tomar o lugar de}
\end{EntryWithPhonetic}

\begin{EntryWithPhonetic}{代言}{dai4yan2}{5,7}{⼈、⾔}
  \definition{v.}{ser um porta-voz | ser um embaixador (para uma marca) | endossar}
\end{EntryWithPhonetic}

\begin{EntryWithPhonetic}{带}{dai4}{9}{⼱}[HSK 2]
  \definition*{s.}{Sobrenome Dai}
  \definition[根]{s.}{cinto; faixa; banda; fita; fita adesiva; algo parecido com uma fita | pneu | zona; área; faixa; cinturão; região; uma determinada área geográfica com determinadas características | leucorreia; corrimento branco; corrimento vaginal}
  \definition{v.}{levar; trazer; transportar | liderar; dirigir; conduzir; assumir | cuidar de crianças; criar filhos; educar | fazer uma coisa e, ao mesmo tempo, fazer outra coisa |suportar; conter | ter algo anexado, simultâneo | trazer consigo | carregar consigo | demonstrar; parecer | incluir; acrescentar}
\end{EntryWithPhonetic}

\begin{EntryWithPhonetic}{带动}{dai4 dong4}{9,6}{⼱、⼒}[HSK 3]
  \definition{v.}{dirigir; ativar; fazer algo funcionar; acionar | liderar; trazer; estimular; motivar; atrair; liderar o avanço; dar o exemplo e fazer com que os outros sigam o exemplo}
\end{EntryWithPhonetic}

\begin{EntryWithPhonetic}{带来}{dai4 lai2}{9,7}{⼱、⽊}[HSK 2]
  \definition{v.}{provocar; produzir; causar}
\end{EntryWithPhonetic}

\begin{EntryWithPhonetic}{带领}{dai4ling3}{9,11}{⼱、⾴}[HSK 3]
  \definition{v.}{guiar, na frente, liderando | liderar e comandar}
\end{EntryWithPhonetic}

\begin{EntryWithPhonetic}{带有}{dai4 you3}{9,6}{⼱、⽉}[HSK 5]
  \definition{v.}{ter; envolver; carregar; implicar}
\end{EntryWithPhonetic}

\begin{EntryWithPhonetic}{待}{dai4}{9}{⼻}
  \definition*{s.}{Sobrenome Dai}
  \definition{v.}{tratar; lidar com | entreter; receber (convidados) | aguardar; esperar por | precisar; necessitar | desejar; pretender; querer}
  \seeref{待}{dai1}
\end{EntryWithPhonetic}

\begin{EntryWithPhonetic}{待遇}{dai4yu4}{9,12}{⼻、⾡}[HSK 4]
  \definition[种,项,份]{s.}{tratamento; refere-se a direitos, status social, etc. | salário; ordenado; remuneração}
\end{EntryWithPhonetic}

\begin{EntryWithPhonetic}{贷}{dai4}{9}{⾙}
  \definition[笔]{s.}{empréstimo; valor do empréstimo}
  \definition{v.}{pedir dinheiro emprestado ou emprestar dinheiro | fugir da responsabilidade | perdoar}
\end{EntryWithPhonetic}

\begin{EntryWithPhonetic}{贷款}{dai4kuan3}{9,12}{⾙、⽋}[HSK 5]
  \definition[个,笔]{s.}{empréstimo; crédito}
  \definition{v.}{fornecer um empréstimo; conceder um empréstimo; conceder crédito a; emprestar dinheiro para quem precisa}
\end{EntryWithPhonetic}

\begin{EntryWithPhonetic}{袋}{dai4}{11}{⾐}[HSK 4]
  \definition{clas.}{usado para coisas que podem ser colocadas nos bolsos | usado para cigarros, narguilé ou tabaco seco}
  \definition[口]{s.}{saco; sacola; bolso; bolsa}
\end{EntryWithPhonetic}

\begin{EntryWithPhonetic}{逮}{dai4}{11}{⾡}
  \definition*{s.}{Sobrenome Dai}
  \definition{v.}{alcançar | prender, usado em 逮捕}
  \seeref{逮}{dai3}
  \seealsoref{逮捕}{dai4bu3}
\end{EntryWithPhonetic}

\begin{EntryWithPhonetic}{逮捕}{dai4bu3}{11,10}{⾡、⼿}
  \definition{v.}{prender | apreender | levar sob custódia}
\end{EntryWithPhonetic}

\begin{EntryWithPhonetic}{戴}{dai4}{17}{⼽}[HSK 4]
  \definition*{s.}{Sobrenome Dai}
  \definition{v.}{usar/vestir (óculos, gravata, relógio de pulso, luvas); colocar objetos em sua cabeça, rosto, pescoço, peito, braços etc. | honrar; respeitar;}
\end{EntryWithPhonetic}

\begin{EntryWithPhonetic}{单}{dan1}{8}{⼗}[HSK 4]
  \definition*{s.}{Sobrenome Dan}
  \definition{adj.}{sozinho; único | ímpar; número ímpar (oposto a 双) | simples; poucos projetos e tipos; estrutura e ideias simples | fino; fraco; frágil}
  \definition{adv.}{isoladamente; sozinho; indica que uma ação ou coisa está dentro de um escopo limitado e não é combinada com outras; equivale a 只 ou 仅}
  \definition[个]{s.}{lençol; um único pedaço grande de pano usado para cobrir | conta; lista; pedaços de papel para anotações detalhadas (geralmente folhas soltas)}
  \seeref{单}{chan2}
  \seeref{单}{shan4}
  \seealsoref{仅}{jin3}
  \seealsoref{双}{shuang1}
  \seealsoref{只}{zhi3}
\end{EntryWithPhonetic}

\begin{EntryWithPhonetic}{单纯}{dan1chun2}{8,7}{⼗、⽷}[HSK 4]
  \definition{adj.}{puro; simples; descomplicado}
  \definition{adv.}{sozinho; puramente; meramente}
\end{EntryWithPhonetic}

\begin{EntryWithPhonetic}{单打}{dan1 da3}{8,5}{⼗、⼿}[HSK 6]
  \definition[场,局,次]{s.}{Esporte: simples; competição um contra um}
\end{EntryWithPhonetic}

\begin{EntryWithPhonetic}{单调}{dan1diao4}{8,10}{⼗、⾔}[HSK 4]
  \definition{adj.}{maçante; monótono}
\end{EntryWithPhonetic}

\begin{EntryWithPhonetic}{单独}{dan1du2}{8,9}{⼗、⽝}[HSK 4]
  \definition{adv.}{solo; sozinho; por si mesmo; por conta própria}
\end{EntryWithPhonetic}

\begin{EntryWithPhonetic}{单脚滑行车}{dan1jiao3hua2xing2che1}{8,11,12,6,4}{⼗、⾁、⽔、⾏、⾞}
  \definition{s.}{\emph{scooter}}
\end{EntryWithPhonetic}

\begin{EntryWithPhonetic}{单位}{dan1wei4}{8,7}{⼗、⼈}[HSK 2]
  \definition[个,家]{s.}{unidade (como padrão de medida) | unidade (como uma organização, departamento, divisão, seção, etc.) | unidade (grupo de pessoas como um todo) | unidade de trabalho (local de trabalho, especialmente na República Popular da China antes da reforma econômica)}
\end{EntryWithPhonetic}

\begin{EntryWithPhonetic}{单一}{dan1 yi1}{8,1}{⼗、⼀}[HSK 5]
  \definition{adj.}{único; unitário; exclusivo}
\end{EntryWithPhonetic}

\begin{EntryWithPhonetic}{单元}{dan1yuan2}{8,4}{⼗、⼉}[HSK 3]
  \definition[个,组,套]{s.}{unidade (de algo); um conjunto completo, com parágrafos e sistemas próprios, que forma uma unidade independente}
\end{EntryWithPhonetic}

\begin{EntryWithPhonetic}{单质}{dan1zhi4}{8,8}{⼗、⾙}
  \definition{s.}{substância simples (consistindo puramente de um elemento, como diamante, grafite, etc.)}
\end{EntryWithPhonetic}

\begin{EntryWithPhonetic}{担}{dan1}{8}{⼿}
  \definition{v.}{carregar em uma vara de ombro e baldes; carregar nos ombros | assumir; empreender; não ter medo de correr riscos}
  \seeref{担}{dan4}
\end{EntryWithPhonetic}

\begin{EntryWithPhonetic}{担保}{dan1bao3}{8,9}{⼿、⼈}[HSK 4]
  \definition{v.}{garantir; atestar; expressar responsabilidade e garantir que não haverá problemas ou que eles serão resolvidos}
\end{EntryWithPhonetic}

\begin{EntryWithPhonetic}{担任}{dan1ren4}{8,6}{⼿、⼈}[HSK 4]
  \definition{v.}{servir como; assumir o cargo de; ocupar o posto de; ocupar um determinado cargo ou emprego}
\end{EntryWithPhonetic}

\begin{EntryWithPhonetic}{担心}{dan1xin1}{8,4}{⼿、⼼}[HSK 4]
  \definition{v.}{preocupar-se; ficar ansioso; sentir-se desconfortável com algo}
\end{EntryWithPhonetic}

\begin{EntryWithPhonetic}{担忧}{dan1 you1}{8,7}{⼿、⼼}[HSK 6]
  \definition[项,条,套,种]{v.}{preocupar-se; estar ansioso}
\end{EntryWithPhonetic}

\begin{EntryWithPhonetic}{耽}{dan1}{10}{⽿}
  \definition*{s.}{Sobrenome Dan}
  \definition{v.}{atrasar | (literário) abandonar-se a; entregar-se a}
\end{EntryWithPhonetic}

\begin{EntryWithPhonetic}{耽心}{dan1xin1}{10,4}{⽿、⼼}
  \variantof{担心}
\end{EntryWithPhonetic}

\begin{EntryWithPhonetic}{胆}{dan3}{9}{⾁}[HSK 5]
  \definition[个,颗]{s.}{vesícula biliar | coragem; bravura | um recipiente interno semelhante a uma bexiga; algo que se encaixa dentro de um objeto e pode conter água, ar, etc.}
\end{EntryWithPhonetic}

\begin{EntryWithPhonetic}{胆小}{dan3 xiao3}{9,3}{⾁、⼩}[HSK 5]
  \definition{adj.}{tímido; covarde}
\end{EntryWithPhonetic}

\begin{EntryWithPhonetic}{胆小鬼}{dan3xiao3gui3}{9,3,9}{⾁、⼩、⿁}
  \definition{adj.}{covarde | medroso}
\end{EntryWithPhonetic}

\begin{EntryWithPhonetic}{石}{dan4}{5}{⽯}[Kangxi 112]
  \definition{clas.}{dan, uma unidade de medida seca para grãos; unidade de capacidade, 10 斗 é igual a 1 石}
  \seeref{石}{shi2}
  \seealsoref{斗}{dou4}
\end{EntryWithPhonetic}

\begin{EntryWithPhonetic}{但}{dan4}{7}{⼈}[HSK 2]
  \definition*{s.}{Sobrenome Dan}
  \definition{adv.}{apenas; meramente; indica uma restrição ao âmbito da ação, equivalente a 只 ou 仅}
  \definition{conj.}{mas; ainda assim; mesmo assim; no entanto; contudo; usado na última oração, conecta duas orações, expressando uma relação de transição, equivalente a 可是 ou 不过}
  \seealsoref{不过}{bu2guo4}
  \seealsoref{仅}{jin3}
  \seealsoref{可是}{ke3shi4}
  \seealsoref{只}{zhi3}
\end{EntryWithPhonetic}

\begin{EntryWithPhonetic}{但是}{dan4 shi4}{7,9}{⼈、⽇}[HSK 2]
  \definition{conj.}{mas; contudo; no entanto; mesmo assim; usado na segunda parte da frase para indicar uma mudança, geralmente acompanhada de expressões como 虽然 ou 尽管}
  \seealsoref{尽管}{jin3guan3}
  \seealsoref{虽然}{sui1 ran2}
\end{EntryWithPhonetic}

\begin{EntryWithPhonetic}{担}{dan4}{8}{⼿}
  \definition{clas.}{dan, uma unidade de peso (=50 quilogramas) ; 100 jin = 1 dan | usado em coisas usadas para transportar cargas}
  \definition{s.}{carga; fardo; cargas de mercadorias transportadas em uma vara de ombro por um mascate itinerante}
  \seeref{担}{dan1}
\end{EntryWithPhonetic}

\begin{EntryWithPhonetic}{诞}{dan4}{8}{⾔}
  \definition{adj.}{absurdo; fantástico; irreal; irracional}
  \definition{adv.}{absurdamente; fantasticamente}
  \definition{s.}{aniversário de nascimento | nascimento}
  \definition{v.}{nascer | dar à luz}
\end{EntryWithPhonetic}

\begin{EntryWithPhonetic}{诞生}{dan4sheng1}{8,5}{⾔、⽣}[HSK 6]
  \definition{v.}{nascer; vir a existir; uma pessoa nasce; também significa que algo novo surgiu e tem um impacto positivo na sociedade}
\end{EntryWithPhonetic}

\begin{EntryWithPhonetic}{弹}{dan4}{11}{⼸}
  \definition{s.}{bola; pelota; pequenas bolas disparadas com um estilingue | bomba; bala; explosivos que podem ser lançados ou arremessados, com poder destrutivo e letal}
  \seeref{弹}{tan2}
\end{EntryWithPhonetic}

\begin{EntryWithPhonetic}{淡}{dan4}{11}{⽔}[HSK 4]
  \definition*{s.}{Sobrenome Dan}
  \definition{adj.}{sem gosto; fraco; não tem sabor forte; não é salgado | leve; fraco; pálido | indiferente; frio; sem entusiasmo | frouxo; sem brilho | sem sentido; trivial | fino; leve}
\end{EntryWithPhonetic}

\begin{EntryWithPhonetic}{蛋}{dan4}{11}{⾍}[HSK 2]
  \definition[个,只]{s.}{ovo; ovos produzidos por aves, tartarugas, cobras, etc. | algo em forma de ovo | tolo; idiota; metáfora para pessoas com determinadas características (com conotação pejorativa) | se perder; colocado após certos verbos, forma um verbo com conotação pejorativa | testículos; em algumas regiões, refere-se aos testículos de certos animais ou pessoas}
\end{EntryWithPhonetic}

\begin{EntryWithPhonetic}{蛋糕}{dan4gao1}{11,16}{⾍、⽶}[HSK 5]
  \definition[个,块,盒]{s.}{bolo; bolo fofo feito de ovos e farinha com açúcar e óleo}
\end{EntryWithPhonetic}

\begin{EntryWithPhonetic}{当}{dang1}{6}{⼹}[HSK 2,6]
  \definition*{s.}{Sobrenome Dang}
  \definition{adj.}{igual; adequado; compatível}
  \definition{prep.}{na presença de alguém; na cara de alguém | exatamente em (um momento ou lugar); em algum momento, em algum lugar | na frente de alguém}
  \definition{s.}{topo; cume |uma lacuna no espaço ou no tempo; refere-se a um espaço ou intervalo de tempo}
  \definition{s.}{Onomatopéia: barulho metálico, som de um gongo ou sino}
  \definition{v.}{dever; ter que; dever ser | trabalhar como; servir como; ser; assumir; desempenhar a função de | suportar; aceitar; merecer | dirigir; gerenciar; estar no comando; ser responsável por;  presidir | conter; bloquear; segurar; reter; resistir}
  \seeref{当}{dang4}
\end{EntryWithPhonetic}

\begin{EntryWithPhonetic}{当场}{dang1chang3}{6,6}{⼹、⼟}[HSK 5]
  \definition{adv.}{na hora; de imediato; na mesma hora}
\end{EntryWithPhonetic}

\begin{EntryWithPhonetic}{当初}{dang1chu1}{6,7}{⼹、⾐}[HSK 3]
  \definition{s.}{no começo; originalmente; no início; em primeiro lugar; refere-se a algo que aconteceu no passado, seja em geral ou especificamente}
\end{EntryWithPhonetic}

\begin{EntryWithPhonetic}{当代}{dang1dai4}{6,5}{⼹、⼈}[HSK 5]
  \definition{s.}{a era atual; a era contemporânea}
\end{EntryWithPhonetic}

\begin{EntryWithPhonetic}{当地}{dang1di4}{6,6}{⼹、⼟}
  \definition{s.}{local; o lugar onde as pessoas e as coisas estão ou onde as coisas acontecem}
\end{EntryWithPhonetic}

\begin{EntryWithPhonetic}{当年}{dang1 nian2}{6,6}{⼹、⼲}[HSK 5]
  \definition{s.}{aqueles anos (ou dias) | naqueles anos (ou dias) | durante esse tempo}
  \definition{v.}{estar no auge da vida}
  \seeref{当年}{dang4 nian2}
\end{EntryWithPhonetic}

\begin{EntryWithPhonetic}{当前}{dang1qian2}{6,9}{⼹、⼑}[HSK 5]
  \definition{s.}{presente; atual}
  \definition{v.}{estar diante de alguém; estar frente a frente com alguém; na frente de, geralmente refere-se a uma situação perigosa}
\end{EntryWithPhonetic}

\begin{EntryWithPhonetic}{当然}{dang1ran2}{6,12}{⼹、⽕}[HSK 3]
  \definition{adj.}{natural; verdadeiro; espontâneo}
  \definition{adv.}{sem dúvida; certamente; claro}
\end{EntryWithPhonetic}

\begin{EntryWithPhonetic}{当时}{dang1shi2}{6,7}{⼹、⽇}[HSK 2]
  \definition{s.}{naquela época; aquela ocasião; aquela vez; refere-se a algo que aconteceu no passado}
  \definition{v.}{ser o momento adequado; acontecer no momento certo}
  \seeref{当时}{dang4shi2}
\end{EntryWithPhonetic}

\begin{EntryWithPhonetic}{当天}{dang1 tian1}{6,4}{⼹、⼤}[HSK 6]
  \definition{s.}{no mesmo dia; naquele mesmo dia; refere-se ao dia em que algo aconteceu no passado}
\end{EntryWithPhonetic}

\begin{EntryWithPhonetic}{当选}{dang1xuan3}{6,9}{⼹、⾡}[HSK 5]
  \definition{v.}{ser eleito}
\end{EntryWithPhonetic}

\begin{EntryWithPhonetic}{当中}{dang1 zhong1}{6,4}{⼹、⼁}[HSK 3]
  \definition{prep.}{no meio; no centro | entre; dentro}
\end{EntryWithPhonetic}

\begin{EntryWithPhonetic}{挡}{dang3}{9}{⼿}[HSK 5]
  \definition{s.}{persiana; veneziana; paralama; coisas para cobrir ou bloquear | caixa de câmbio (automóvel)}
  \definition{v.}{bloquear; resistir; manter afastado; afastar | cobrir; bloquear; atrapalhar}
  \seeref{挡}{dang4}
\end{EntryWithPhonetic}

\begin{EntryWithPhonetic}{挡风玻璃}{dang3feng1bo1li5}{9,4,9,14}{⼿、⾵、⽟、⽟}
  \definition{s.}{parabrisa}
\end{EntryWithPhonetic}

\begin{EntryWithPhonetic}{党}{dang3}{10}{⼉}[HSK 6]
  \definition*{s.}{O Partido (Partido Comunista da China) | Sobrenome Dang}
  \definition{s.}{partido político; partido | camarilha; facção; gangue | Datado: parentes}
  \definition{v.}{ser parcial; tomar partido de}
\end{EntryWithPhonetic}

\begin{EntryWithPhonetic}{当}{dang4}{6}{⼹}
  \definition{adj.}{adequado; correto; apropriado | igual; o mesmo}
  \definition{pron.}{naquele mesmo (dia, etc.); refere-se ao momento em que algo aconteceu}
  \definition{s.}{algo penhorado; penhor; garantia; objetos físicos penhorados em casas de penhores}
  \definition{v.}{corresponder; ser igual a; combinar | tratar como; considerar como; tomar como | pensar que; achar que | penhorar; empréstimo com garantia real em uma loja de penhores}
  \seeref{当}{dang1}
\end{EntryWithPhonetic}

\begin{EntryWithPhonetic}{当成}{dang4 cheng2}{6,6}{⼹、⼽}[HSK 6]
  \definition{v.}{considerar como; tratar como; tomar por}
\end{EntryWithPhonetic}

\begin{EntryWithPhonetic}{当年}{dang4 nian2}{6,6}{⼹、⼲}
  \definition{s.}{no mesmo ano; naquele mesmo ano}
  \seeref{当年}{dang1 nian2}
\end{EntryWithPhonetic}

\begin{EntryWithPhonetic}{当时}{dang4shi2}{6,7}{⼹、⽇}
  \definition{adv.}{(depois de fazer algo ou algo acontecer) imediatamente; de imediato; agora mesmo}
  \seeref{当时}{dang1shi2}
\end{EntryWithPhonetic}

\begin{EntryWithPhonetic}{当作}{dang4 zuo4}{6,7}{⼹、⼈}[HSK 6]
  \definition{v.}{tratar como; considerar como}
\end{EntryWithPhonetic}

\begin{EntryWithPhonetic}{挡}{dang4}{9}{⼿}
  \definition{v.}{organizar}
  \seeref{挡}{dang3}
\end{EntryWithPhonetic}

\begin{EntryWithPhonetic}{档}{dang4}{10}{⽊}[HSK 6]
  \definition{clas.}{festa; usado para eventos, shows}
  \definition{s.}{prateleiras (para arquivos); compartimentos para documentos | arquivos; arquivos | travessa (de uma mesa, etc.) | qualidade; nota}
\end{EntryWithPhonetic}

\begin{EntryWithPhonetic}{档案}{dang4'an4}{10,10}{⽊、⽊}[HSK 6]
  \definition[份,个]{s.}{arquivos; registro; dossiê; arquivos e materiais armazenados de forma classificada para referência futura}
\end{EntryWithPhonetic}

\begin{EntryWithPhonetic}{刀}{dao1}{2}{⼑}[HSK 3][Kangxi 18]
  \definition*{s.}{Sobrenome Dao}
  \definition{clas.}{unidade de medida para papel, geralmente cem folhas por pacote}
  \definition[把,口]{s.}{faca; espada; armas antigas, referindo-se a ferramentas para cortar, retalhar, raspar, golpear e fatiar, geralmente feitas de ferro e aço | ferramenta; ferramenta de corte; lâminas para tornos; fresas (ferramentas; ferramentas de ferro para máquinas) | algo com a forma de uma faca}
\end{EntryWithPhonetic}

\begin{EntryWithPhonetic}{导}{dao3}{6}{⼨}
  \definition[个,位,名,些]{s.}{guia turístico | diretor}
  \definition{v.}{liderar; guiar | conduzir; transmitir | ensinar; instruir; dar orientação a}
\end{EntryWithPhonetic}

\begin{EntryWithPhonetic}{导弹}{dao3dan4}{6,11}{⼨、⼸}
  \definition[枚]{s.}{míssil (guiado)}
\end{EntryWithPhonetic}

\begin{EntryWithPhonetic}{导演}{dao3yan3}{6,14}{⼨、⽔}[HSK 3]
  \definition[位,名,个]{s.}{diretor; pessoa que exerce a função de diretor}
  \definition{v.}{dirigir (um filme, peça, etc.); ensaio de peças teatrais ou filmagem de filmes e séries de TV; organização e orientação do trabalho de produção}
\end{EntryWithPhonetic}

\begin{EntryWithPhonetic}{导游}{dao3you2}{6,12}{⼨、⽔}[HSK 4]
  \definition[个,位,名]{s.}{guia turístico; pessoas que trabalham como guias turísticos}
  \definition{v.}{guiar; conduzir um passeio turístico}
\end{EntryWithPhonetic}

\begin{EntryWithPhonetic}{导致}{dao3zhi4}{6,10}{⼨、⾄}[HSK 4]
  \definition{v.}{causar; levar a; dar origem a (um resultado ruim)}
\end{EntryWithPhonetic}

\begin{EntryWithPhonetic}{岛}{dao3}{7}{⼭}[HSK 6]
  \definition[个,座]{s.}{ilha; uma massa de terra menor que um continente cercada por água}
\end{EntryWithPhonetic}

\begin{EntryWithPhonetic}{倒}{dao3}{10}{⼈}[HSK 2]
  \definition{v.}{cair; tombar | falhar; entrar em colapso | ficar rouco | mudar; trocar; transferir; converter | movimentar-se; manobrar | oferecer (casa, loja) para venda; vender mercadorias ou lojas a terceiros a um preço fixo | derrubar; derrubar com}
  \seeref{倒}{dao4}
\end{EntryWithPhonetic}

\begin{EntryWithPhonetic}{倒闭}{dao3bi4}{10,6}{⼈、⾨}[HSK 4]
  \definition{v.}{fechar; ir à falência; entrar em liquidação; sair do negócio; (empresa, loja ou banco) deixar de operar devido ao baixo desempenho}
\end{EntryWithPhonetic}

\begin{EntryWithPhonetic}{倒车}{dao3che1}{10,4}{⼈、⾞}[HSK 4]
  \definition{v.}{trocar de trem ou ônibus (no meio do caminho)}
  \seeref{倒车}{dao4che1}
\end{EntryWithPhonetic}

\begin{EntryWithPhonetic}{倒地}{dao3di4}{10,6}{⼈、⼟}
  \definition{v.}{cair no chão}
\end{EntryWithPhonetic}

\begin{EntryWithPhonetic}{倒楣}{dao3mei2}{10,13}{⼈、⽊}
  \variantof{倒霉}
\end{EntryWithPhonetic}

\begin{EntryWithPhonetic}{倒霉}{dao3mei2}{10,15}{⼈、⾬}
  \definition{adj.}{azarado}
  \definition{s.}{azar | má sorte}
  \definition{v.}{estar sem sorte | ter azar}
  \seealsoref{倒血霉}{dao3xue4mei2}
\end{EntryWithPhonetic}

\begin{EntryWithPhonetic}{倒血霉}{dao3xue4mei2}{10,6,15}{⼈、⾎、⾬}
  \definition{v.}{ter muito azar (versão mais forte de 倒霉)}
  \seealsoref{倒霉}{dao3mei2}
\end{EntryWithPhonetic}

\begin{EntryWithPhonetic}{到}{dao4}{8}{⼑}[HSK 1]
  \definition*{s.}{Sobrenome Dao}
  \definition{adj.}{atencioso}
  \definition{prep.}{a; até; para; indica o tempo em que a ação ou comportamento foi alcançado}
  \definition{v.}{ir para; partir para | chegar; alcançar; chegar a | como complemento de um verbo para mostrar o resultado de uma ação}
\end{EntryWithPhonetic}

\begin{EntryWithPhonetic}{到处}{dao4chu4}{8,5}{⼑、⼡}[HSK 2]
  \definition{adv.}{em todos os lugares; em todos os locais; por toda parte}
\end{EntryWithPhonetic}

\begin{EntryWithPhonetic}{到达}{dao4da2}{8,6}{⼑、⾡}[HSK 3]
  \definition{v.}{chegar (a um determinado local, a uma determinada fase); alcançar}
\end{EntryWithPhonetic}

\begin{EntryWithPhonetic}{到底}{dao4di3}{8,8}{⼑、⼴}[HSK 3]
  \definition{adv.}{na terra (usado em frases interrogativas para expressar a determinação de alguém em encontrar uma resposta definitiva) | afinal | finalmente; por fim; no fim; indica uma situação que finalmente se concretizou após várias mudanças ou reviravoltas}
\end{EntryWithPhonetic}

\begin{EntryWithPhonetic}{到来}{dao4 lai2}{8,7}{⼑、⽊}[HSK 5]
  \definition{v.}{chegar; chegar aqui de outro lugar}
\end{EntryWithPhonetic}

\begin{EntryWithPhonetic}{到期}{dao4 qi1}{8,12}{⼑、⽉}[HSK 6]
  \definition{v.+compl.}{expirar; amadurecer; tornar-se devido; tornar-se devido}
\end{EntryWithPhonetic}

\begin{EntryWithPhonetic}{倒}{dao4}{10}{⼈}[HSK 2]
  \definition{adj.}{inverso; invertido; de cabeça para baixo}
  \definition{adv.}{mas; pelo contrário; expressa o contrário do esperado, equivalente a 反倒 | indicando que algo não é o que se pensa; indica que as coisas não são assim | usado para indicar uma transição ou concessão | transmitindo a sensação de ``urgência''; expressa pressa ou insistência, com um tom impaciente}
  \definition{v.}{ser inverso; estar invertido; estar de cabeça para baixo; inverter a posição original para cima e para baixo ou para a frente e para trás | recuar; virar de cabeça para baixo; fazer mover na direção oposta ou inverter | inclinar ou virar o recipiente para retirar o conteúdo; inclinar; derramar}
  \seeref{倒}{dao3}
  \seealsoref{反倒}{fan3dao4}
\end{EntryWithPhonetic}

\begin{EntryWithPhonetic}{倒车}{dao4che1}{10,4}{⼈、⾞}[HSK 4]
  \definition{v.}{dar marcha à ré (em um veículo)}
  \seeref{倒车}{dao3che1}
\end{EntryWithPhonetic}

\begin{EntryWithPhonetic}{倒是}{dao4 shi4}{10,9}{⼈、⽇}[HSK 5]
  \definition{adv.}{usado para indicar o oposto do que geralmente é verdade; ao contrário do senso comum; pelo contrário | usado para indicar o que é contrário aos fatos, com um toque de crítica; indica que as coisas não são assim (com um sentimento de culpa) | usado de algo inesperado; expressando surpresa | usado para indicar concessão | usado para indicar uma mudança de significado; indica um ponto de virada | usado para modificar ou suavizar uma declaração anterior; para suavizar o tom | usado para pressionar ou questionar alguém; para instar ou perguntar}
\end{EntryWithPhonetic}

\begin{EntryWithPhonetic}{盗}{dao4}{11}{⽫}
  \definition[个,伙,帮,窝]{s.}{ladrão; assaltante}
  \definition{v.}{roubar; saquear | usurpar; buscar ganho pessoal ou ganho por meios impróprios}
\end{EntryWithPhonetic}

\begin{EntryWithPhonetic}{盗版}{dao4 ban3}{11,8}{⽫、⽚}[HSK 6]
  \definition{s.}{cópia ilegal; cópia pirata; refere-se a livros, periódicos e produtos audiovisuais pirateados (diferentes dos 正版)}
  \definition{v.}{piratear; copiar ou vender ilegalmente; para obter lucros enormes, reimprimir ou copiar livros, periódicos ou produtos audiovisuais em grandes quantidades sem o consentimento do detentor dos direitos autorais}
  \seealsoref{正版}{zheng4 ban3}
\end{EntryWithPhonetic}

\begin{EntryWithPhonetic}{道}{dao4}{12}{⾡}[HSK 2]
  \definition*{s.}{Taoismo;  Taoista | Sobrenome Dao}
  \definition{clas.}{usado para pratos em refeições, etapas em um procedimento, etc. | usado para certos objetos longos e estreitos; tira | usado para portas, paredes, etc.; pesado | usado para comandos, títulos, etc.}
  \definition[条]{s.}{estrada; caminho; trilha | curso; canal; o caminho percorrido pelo fluxo da água | maneira; método; princípio; raciocínio | moral; moralidade | habilidade; técnica | doutrina; princípio; sistema de pensamento acadêmico ou religioso; origem de todas as coisas no universo | taoísta; taoísmo; pertencente ao taoísmo | seita supersticiosa; certas organizações reacionárias e supersticiosas | linha; traços finos e alongados | trato; os canais dentro do corpo}
  \definition{v.}{dizer; falar; expressar-se | pensar; supor; considerar; acreditar que}
\end{EntryWithPhonetic}

\begin{EntryWithPhonetic}{道德}{dao4de2}{12,15}{⾡、⼻}[HSK 5]
  \definition{adj.}{moral; descreve uma pessoa ou comportamento que atende aos requisitos morais; mais usado em situações negativas}
  \definition[种]{s.}{moral; ética; moralidade; regras e normas para que as pessoas vivam juntas e se comportem em comum}
\end{EntryWithPhonetic}

\begin{EntryWithPhonetic}{道行}{dao4 heng2}{12,6}{⾡、⾏}
  \definition{s.}{realizações de um monge budista ou sacerdote taoísta | habilidades; capacidades; aptidões | (figurativo) habilidade | habilidades adquiridas através da prática religiosa}
\end{EntryWithPhonetic}

\begin{EntryWithPhonetic}{道教}{dao4 jiao4}{12,11}{⾡、⽁}[HSK 6]
  \definition*{s.}{Taoísmo (sistema de crenças chinês)}
  \definition{s.}{a religião taoísta; taoísmo}
\end{EntryWithPhonetic}

\begin{EntryWithPhonetic}{道理}{dao4li5}{12,11}{⾡、⽟}[HSK 2]
  \definition[个,种]{s.}{verdade; princípio; a lei das coisas | sentido; razão}
\end{EntryWithPhonetic}

\begin{EntryWithPhonetic}{道路}{dao4 lu4}{12,13}{⾡、⾜}[HSK 2]
  \definition[条,段]{s.}{estrada; caminho; os canais de comunicação entre os dois lugares, incluindo terrestres e aquáticos | caminho; processo; refere-se à vida, à existência (significado abstrato)}
\end{EntryWithPhonetic}

\begin{EntryWithPhonetic}{道歉}{dao4qian4}{12,14}{⾡、⽋}[HSK 6]
  \definition{v.+compl.}{pedir desculpas; fazer um pedido de desculpas; dizer aos outros que você estava errado e pedir perdão}
\end{EntryWithPhonetic}

\begin{EntryWithPhonetic}{得}{de2}{11}{⼻}[HSK 2]
  \definition{adj.}{adequado; apropriado | satisfeito; complacente; orgulhoso de si mesmo}
  \definition{interj.}{usado para encerrar uma conversa para indicar concordância ou proibição | usado quando a situação não é a esperada, para expressar impotência}
  \definition{v.}{obter (em oposição a 失); conseguir; ganhar |  (de um cálculo) igual; resultar em | estar pronto; estar acabado | pegar; apanhar; contrair uma doença}
  \definition{v.aux.}{usado antes de outros verbos para expressar permissão | usado antes de outros verbos para indicar que é possível (usado principalmente na forma negativa) | usado em conversas para indicar que não há necessidade de dizer mais nada}
  \seeref{得}{de5}
  \seeref{得}{dei3}
  \seealsoref{失}{shi1}
\end{EntryWithPhonetic}

\begin{EntryWithPhonetic}{得出}{de2 chu1}{11,5}{⼻、⼐}[HSK 2]
  \definition{v.}{chegar (a uma conclusão); obter (a um resultado); deduzir ou calcular (conclusão ou resultado)}
\end{EntryWithPhonetic}

\begin{EntryWithPhonetic}{得到}{de2 dao4}{11,8}{⼻、⼑}[HSK 1]
  \definition{v.}{obter; conseguir; ganhar; receber; possuir algo; adquirir}
\end{EntryWithPhonetic}

\begin{EntryWithPhonetic}{得分}{de2 fen1}{11,4}{⼻、⼑}[HSK 3]
  \definition{s.}{pontuação; classificação; nota; pontuação obtida em jogos ou competições}
  \definition{v.}{fazer pontos; pontuar}
\end{EntryWithPhonetic}

\begin{EntryWithPhonetic}{得了}{de2le5}{11,2}{⼻、⼅}[HSK 5]
  \definition{expr.}{Tudo bem!; É o bastante!}
  \seeref{得了}{de2liao3}
\end{EntryWithPhonetic}

\begin{EntryWithPhonetic}{得了}{de2liao3}{11,2}{⼻、⼅}
  \definition{adj.}{(enfaticamente, em perguntas retóricas) possível; indica que a situação é séria (usado principalmente em perguntas retóricas ou formas negativas)}
  \seeref{得了}{de2le5}
\end{EntryWithPhonetic}

\begin{EntryWithPhonetic}{得以}{de2 yi3}{11,4}{⼻、⼈}[HSK 5]
  \definition{v.}{ser capaz de; para que\dots possa (ou possa)\dots}
\end{EntryWithPhonetic}

\begin{EntryWithPhonetic}{得意}{de2yi4}{11,13}{⼻、⼼}[HSK 4]
  \definition{adj.}{complacente; orgulhoso de si mesmo; satisfeito consigo mesmo}
\end{EntryWithPhonetic}

\begin{EntryWithPhonetic}{德}{de2}{15}{⼻}
  \definition*{s.}{Alemanha, abreviação de 德国 | Sobrenome De}
  \definition{s.}{virtude; moral; caráter moral; moralidade; conduta; qualidades políticas | coração; mente; pensamentos | bondade; favor; graça}
  \seealsoref{德国}{de2guo2}
\end{EntryWithPhonetic}

\begin{EntryWithPhonetic}{德国}{de2guo2}{15,8}{⼻、⼞}
  \definition*{s.}{Alemanha}
\end{EntryWithPhonetic}

\begin{EntryWithPhonetic}{德国人}{de2guo2ren2}{15,8,2}{⼻、⼞、⼈}
  \definition{s.}{alemão | pessoa ou povo da Alemanha}
\end{EntryWithPhonetic}

\begin{EntryWithPhonetic}{地}{de5}{6}{⼟}[HSK 1]
  \definition{part.}{(estrutural) utilizada antes de um verbo ou adjetivo, ligando-o ao adjunto adverbial modificador precedente}
  \seeref{地}{di4}
\end{EntryWithPhonetic}

\begin{EntryWithPhonetic}{底}{de5}{8}{⼴}
  \definition{part.}{usada após uma palavra ou frase que é usada como determinante para indicar subordinação à palavra central}
  \seeref{底}{di3}
\end{EntryWithPhonetic}

\begin{EntryWithPhonetic}{的}{de5}{8}{⽩}
  \definition{part.}{usado para indicar posse | formar uma frase nominal ou expressão nominal | substituir a pessoa ou coisa mencionada anteriormente | no final de uma frase declarativa, para dar ênfase; usado após o verbo predicativo, enfatiza o agente da ação, o tempo, o local, etc. | usado no final de uma frase declarativa, expressa afirmação, ênfase, certeza, etc. | indica que alguém obteve uma determinada posição ou status | usado com 是 para indicar predicado ou ênfase; indica que alguém é o objeto da ação | e assim por diante; e assim por diante; e similares; usado após palavras paralelas, significa 等等, 之类 | indica uma ação (o pronome é o objeto da ação); combinado com o verbo anterior, expressa uma ação, e o pronome é o objeto dessa ação}
  \seeref{的}{di1}
  \seeref{的}{di2}
  \seeref{的}{di4}
  \seealsoref{等等}{deng3 deng3}
  \seealsoref{是}{shi4}
  \seealsoref{之类}{zhi1 lei4}
\end{EntryWithPhonetic}

\begin{EntryWithPhonetic}{的话}{de5 hua4}{8,8}{⽩、⾔}[HSK 2]
  \definition{part.}{se; caso; suponha que; partícula usada após uma frase hipotética para introduzir o texto seguinte}
\end{EntryWithPhonetic}

\begin{EntryWithPhonetic}{的时候}{de5 shi2hou4}{8,7,10}{⽩、⽇、⼈}
  \definition{part.}{naquele momento; quando; em; descreve o momento específico em que um evento ocorreu}
\end{EntryWithPhonetic}

\begin{EntryWithPhonetic}{得}{de5}{11}{⼻}[HSK 2]
  \definition{part.}{depois de um verbo ou adjetivo para expressar possibilidade ou capacidade | entre um verbo e seu complemento para expressar possibilidade | ligando um verbo ou um adjetivo a um complemento que descreve a maneira ou o grau}
  \seeref{得}{de2}
  \seeref{得}{dei3}
\end{EntryWithPhonetic}

\begin{EntryWithPhonetic}{得}{dei3}{11}{⼻}[HSK 4]
  \definition{v.}{precisar; expressa uma necessidade lógica, factual ou subjetiva; deve; é necessário | ter de; ser obrigado a; indica uma necessidade de vontade ou de fato | certamente irá; expressa a inevitabilidade da especulação}
  \seeref{得}{de2}
  \seeref{得}{de5}
\end{EntryWithPhonetic}

\begin{EntryWithPhonetic}{灯}{deng1}{6}{⽕}[HSK 2]
  \definition*{s.}{Sobrenome Deng}
  \definition[盏,个]{s.}{lâmpada; luz; lanterna; dispositivo luminoso, usado principalmente para iluminação | queimador; um aparelho que brilha e aquece como uma lâmpada e pode ser usado para aquecer | tubo; válvula; o nome popular dado aos tubos eletrônicos com formato semelhante a lâmpadas encontrados em aparelhos antigos, como rádios}
\end{EntryWithPhonetic}

\begin{EntryWithPhonetic}{灯标}{deng1biao1}{6,9}{⽕、⽊}
  \definition{s.}{farol | luz de farol}
\end{EntryWithPhonetic}

\begin{EntryWithPhonetic}{灯光}{deng1 guang1}{6,6}{⽕、⼉}[HSK 4]
  \definition[束,盏,点,打]{s.}{iluminação; luminosidade da lâmpada | luminação (palco); equipamento de iluminação para palco ou estúdio}
\end{EntryWithPhonetic}

\begin{EntryWithPhonetic}{灯号}{deng1hao4}{6,5}{⽕、⼝}
  \definition{s.}{sinal luminoso | luz indicadora}
\end{EntryWithPhonetic}

\begin{EntryWithPhonetic}{灯泡}{deng1pao4}{6,8}{⽕、⽔}
  \definition[个]{s.}{lâmpada | (gíria) terceiro indesejado estragando encontro de casal}
  \seealsoref{电灯泡}{dian4deng1pao4}
\end{EntryWithPhonetic}

\begin{EntryWithPhonetic}{灯丝}{deng1si1}{6,5}{⽕、⼀}
  \definition{s.}{filamento (de uma lâmpada)}
\end{EntryWithPhonetic}

\begin{EntryWithPhonetic}{登}{deng1}{12}{⽨}[HSK 4]
  \definition{v.}{subir; montar; escalar (uma altura) | publicar; registrar; inserir | recolher e levar para a eira | pisar em; pisar | calçar (calçados ou calças) | partir; começar uma jornada; embarcar em uma jornada}
\end{EntryWithPhonetic}

\begin{EntryWithPhonetic}{登记}{deng1ji4}{12,5}{⽨、⾔}[HSK 4]
  \definition{v.+compl.}{registrar-se; fazer o \emph{check-in} | registrar; reportar; informar; relatar por escrito a um superior ou autoridade relevante (usado principalmente para documentos legais)}
\end{EntryWithPhonetic}

\begin{EntryWithPhonetic}{登录}{deng1lu4}{12,8}{⽨、⼹}[HSK 4]
  \definition{v.}{fazer \emph{logon}; fazer \emph{login} | gravar; registrar; computadores eletrônicos e sua terminologia de rede, referindo-se ao acesso ao sistema operacional ou ao site a ser visitado}
\end{EntryWithPhonetic}

\begin{EntryWithPhonetic}{登山}{deng1 shan1}{12,3}{⽨、⼭}[HSK 4]
  \definition{s.}{escalar; fazer alpinismo; subir uma montanha}
\end{EntryWithPhonetic}

\begin{EntryWithPhonetic}{等}{deng3}{12}{⽵}[HSK 1,2]
  \definition*{s.}{Sobrenome Deng}
  \definition{adj.}{igual; na mesma medida ou quantidade}
  \definition{clas.}{usado para classe, grau, classificação | usado para tipo}
  \definition{part.}{e assim por diante; etc.; indica que a enumeração não está completa (pode ser usada repetidamente) | indica o fim de uma enumeração; após a enumeração, é usado para encerrar; geralmente é seguido pelo total dos itens anteriores}
  \definition{pron.}{usado após pronomes pessoais ou substantivos que se referem a pessoas; indica plural}
  \definition{s.}{classe; série; posição | equilíbrio; balança para pesar pequenas quantidades de objetos valiosos e ervas medicinais; atualmente, geralmente escrita como 戥}
  \definition{v.}{esperar; aguardar | esperar até}
\end{EntryWithPhonetic}

\begin{EntryWithPhonetic}{等待}{deng3dai4}{12,9}{⽵、⼻}[HSK 3]
  \definition{v.}{esperar; aguardar; não agir até que a pessoa, coisa ou situação desejada apareça}
\end{EntryWithPhonetic}

\begin{EntryWithPhonetic}{等到}{deng3 dao4}{12,8}{⽵、⼑}[HSK 2]
  \definition{prep.}{na hora; quando; expressão de condições temporais | esperar até; aguardar até}
\end{EntryWithPhonetic}

\begin{EntryWithPhonetic}{等等}{deng3 deng3}{12,12}{⽵、⽵}
  \definition{part.}{etc.; e assim por diante; usada depois de duas ou mais palavras paralelas para indicar que a lista não está completa}
\end{EntryWithPhonetic}

\begin{EntryWithPhonetic}{等候}{deng3hou4}{12,10}{⽵、⼈}[HSK 5]
  \definition{v.}{esperar; aguardar; expectar; usado principalmente para objetos específicos}
\end{EntryWithPhonetic}

\begin{EntryWithPhonetic}{等级}{deng3ji2}{12,6}{⽵、⽷}[HSK 5]
  \definition[个]{s.}{grau; classificação; posição; distinções por qualidade, grau, status, etc. | estado social; estrato social; ordem e grau; grupos sociais desiguais em termos de status social e legal}
\end{EntryWithPhonetic}

\begin{EntryWithPhonetic}{等于}{deng3yu2}{12,3}{⽵、⼆}[HSK 2]
  \definition{adv.}{igual a | equivalente a}
  \definition{v.}{equivaler a; ser equivalente a; ser quase igual a; não ter diferença}
\end{EntryWithPhonetic}

\begin{EntryWithPhonetic}{低}{di1}{7}{⼈}[HSK 2]
  \definition*{s.}{Sobrenome Di}
  \definition{adj.}{baixo; distância pequena de baixo para cima; próximo ao solo | abaixo da média; abaixo do padrão geral | inferior (em grau); de nível inferior}
  \definition{v.}{deixar cair; pendurar; abaixar (a cabeça)}
\end{EntryWithPhonetic}

\begin{EntryWithPhonetic}{低潮}{di1chao2}{7,15}{⼈、⽔}
  \definition{s.}{maré baixa/vazante; o nível mais baixo da maré durante um ciclo de maré (distinto da 高潮) | vazante baixa; o ponto mais baixo; uma metáfora para o baixo estágio de desenvolvimento das coisas}
  \seealsoref{高潮}{gao1chao2}
\end{EntryWithPhonetic}

\begin{EntryWithPhonetic}{低等}{di1 deng3}{7,12}{⼈、⽵}
  \definition{adj.}{inferior; classe baixa (oposto a 高等) | inferior}
  \seealsoref{高等}{gao1 deng3}
\end{EntryWithPhonetic}

\begin{EntryWithPhonetic}{低头}{di1 tou2}{7,5}{⼈、⼤}[HSK 6]
  \definition{v.}{abaixar a cabeça; curvar a cabeça; pendurar a cabeça | ceder; submeter-se; refere-se à rendição e à admissão da derrota}
\end{EntryWithPhonetic}

\begin{EntryWithPhonetic}{低温}{di1 wen1}{7,12}{⼈、⽔}[HSK 6]
  \definition{s.}{baixa temperatura | Meteorologia: microtermia  | Medicina: hipotermia}
\end{EntryWithPhonetic}

\begin{EntryWithPhonetic}{低于}{di1 yu2}{7,3}{⼈、⼆}[HSK 5]
  \definition{v.}{ser inferior a; algo ou fenômeno é, de alguma forma, inferior ou pior do que outra coisa}
\end{EntryWithPhonetic}

\begin{EntryWithPhonetic}{的}{di1}{8}{⽩}
  \definition{s.}{abreviação de 的士: um táxi}
  \seeref{的}{de5}
  \seeref{的}{di2}
  \seeref{的}{di4}
  \seealsoref{的士}{di1shi4}
\end{EntryWithPhonetic}

\begin{EntryWithPhonetic}{的士}{di1shi4}{8,3}{⽩、⼠}
  \definition{s.}{(empréstimo linguístico) táxi}
\end{EntryWithPhonetic}

\begin{EntryWithPhonetic}{堤}{di1}{12}{⼟}
  \definition[道,条]{s.}{dique; aterro}
\end{EntryWithPhonetic}

\begin{EntryWithPhonetic}{堤坝}{di1ba4}{12,7}{⼟、⼟}
  \definition{s.}{represa | dique | barragem}
\end{EntryWithPhonetic}

\begin{EntryWithPhonetic}{滴}{di1}{14}{⽔}[HSK 6]
  \definition{clas.}{gota; quantificador para "gotejamento"}
  \definition{s.}{uma gota}
  \definition{v.}{pingar}
\end{EntryWithPhonetic}

\begin{EntryWithPhonetic}{的}{di2}{8}{⽩}
  \definition{adv.}{verdadeiramente; exatamente; realmente}
  \seeref{的}{de5}
  \seeref{的}{di1}
  \seeref{的}{di4}
\end{EntryWithPhonetic}

\begin{EntryWithPhonetic}{的确}{di2que4}{8,12}{⽩、⽯}[HSK 4]
  \definition{adv.}{realmente; de fato, ao expressar certeza sobre a situação}
\end{EntryWithPhonetic}

\begin{EntryWithPhonetic*}{敌}{di2}{10}{⾆}
  \definition[个,名,位,种]{s.}{inimigo; adversário}
  \definition{v.}{opor-se; lutar; resistir; suportar | combinar; igualar}
\end{EntryWithPhonetic*}

\begin{EntryWithPhonetic}{敌人}{di2ren2}{10,2}{⾆、⼈}[HSK 4]
  \definition[群,伙,帮,个,队]{s.}{inimigo; pessoa hostil; parte hostil}
\end{EntryWithPhonetic}

\begin{EntryWithPhonetic}{笛}{di2}{11}{⽵}
  \definition[只]{s.}{flauta de bambu | sirene; apito; buzina}
\end{EntryWithPhonetic}

\begin{EntryWithPhonetic}{底}{di3}{8}{⼴}[HSK 4]
  \definition*{s.}{Sobrenome Di}
  \definition{pron.}{o que? |  isto; isso; aqui | assim; tal}
  \definition{s.}{base; fundo; parte inferior de um objeto | detalhes; o cerne da questão; base, fonte ou contexto de uma coisa | rascunho; cópia mantida como registro; rascunho que pode ser usado como base | final de um ano ou mês | chão; fundo; fundação | a última parte de algo}
  \seeref{底}{de5}
\end{EntryWithPhonetic}

\begin{EntryWithPhonetic}{底气}{di3qi4}{8,4}{⼴、⽓}
  \definition{s.}{capacidade pulmonar | ousadia | confiança | autoconfiança | vigor}
\end{EntryWithPhonetic}

\begin{EntryWithPhonetic}{底下}{di3 xia4}{8,3}{⼴、⼀}[HSK 3]
  \definition{adv.}{em baixo; abaixo; sob | próximo; mais tarde; depois; daqui para a frente}
\end{EntryWithPhonetic}

\begin{EntryWithPhonetic}{抵}{di3}{8}{⼿}
  \definition{v.}{apoiar; sustentar | resistir; suportar | compensar; fazer o bem | hipotecar; dar como garantia; garantir | equilibrar; cancelar; compensar | ser igual a; corresponder | alcançar; chegar a | colidir; dar cabeçada (por animais com chifres)}
\end{EntryWithPhonetic}

\begin{EntryWithPhonetic}{抵达}{di3da2}{8,6}{⼿、⾡}[HSK 6]
  \definition{v.}{chegar; alcançar}
\end{EntryWithPhonetic}

\begin{EntryWithPhonetic}{抵抗}{di3kang4}{8,7}{⼿、⼿}[HSK 6]
  \definition{s.}{resistência}
  \definition{v.}{resistir; usar ação para resistir ou parar o ataque da outra parte}
\end{EntryWithPhonetic}

\begin{EntryWithPhonetic}{地}{di4}{6}{⼟}[HSK 1]
  \definition*{s.}{A Terra | Sobrenome Di}
  \definition[块,片]{s.}{terra; solo | campos | chão; piso | posição; situação | contexto; base | distância percorrida (medida em 里 ou paradas 站) | indicando estado de espírito | território | lugar; local | parte do espaço | distância}
  \seeref{地}{de5}
\end{EntryWithPhonetic}

\begin{EntryWithPhonetic}{地板}{di4 ban3}{6,8}{⼟、⽊}[HSK 6]
  \definition[块]{s.}{piso de madeira; tábuas de madeira especiais para pavimentação do piso | piso; piso interno pavimentado com tábuas de madeira; geralmente se refere ao piso de um edifício}
\end{EntryWithPhonetic}

\begin{EntryWithPhonetic}{地带}{di4 dai4}{6,9}{⼟、⼱}[HSK 5]
  \definition[个,条]{s.}{distrito; região; zona; área de uma determinada natureza ou extensão}
\end{EntryWithPhonetic}

\begin{EntryWithPhonetic}{地点}{di4dian3}{6,9}{⼟、⽕}[HSK 1]
  \definition[个]{s.}{lugar; local; região; localização}
\end{EntryWithPhonetic}

\begin{EntryWithPhonetic}{地方}{di4fang1}{6,4}{⼟、⽅}
  \definition[个]{s.}{distrito; localidade;  em oposição a 中央, o número total de unidades administrativas em todos os níveis abaixo do centro | governo local e população; refere-se a outros setores que não o militar}
  \seeref{地方}{di4fang5}
  \seealsoref{中央}{zhong1yang1}
\end{EntryWithPhonetic}

\begin{EntryWithPhonetic}{地方}{di4fang5}{6,4}{⼟、⽅}[HSK 1,4]
  \definition[个,处,块]{s.}{lugar; cômodo; área; refere-se a um espaço específico | parte}
  \seeref{地方}{di4fang1}
\end{EntryWithPhonetic}

\begin{EntryWithPhonetic}{地核}{di4he2}{6,10}{⼟、⽊}
  \definition{s.}{(geologia) núcleo da Terra}
\end{EntryWithPhonetic}

\begin{EntryWithPhonetic}{地理}{di4li3}{6,11}{⼟、⽟}
  \definition{s.}{geografia}
\end{EntryWithPhonetic}

\begin{EntryWithPhonetic}{地面}{di4 mian4}{6,9}{⼟、⾯}[HSK 4]
  \definition{s.}{a superfície da Terra | térreo; piso; camada de material colocada no chão dentro e ao redor dos edifícios | localidade; chão | região; território; principalmente áreas administrativas}
\end{EntryWithPhonetic}

\begin{EntryWithPhonetic}{地名}{di4 ming2}{6,6}{⼟、⼝}[HSK 6]
  \definition{s.}{nome de um lugar | nome de lugar | topônimo}
\end{EntryWithPhonetic}

\begin{EntryWithPhonetic}{地球}{di4qiu2}{6,11}{⼟、⽟}[HSK 2]
  \definition[个]{s.}{o planeta Terra}
\end{EntryWithPhonetic}

\begin{EntryWithPhonetic}{地区}{di4qu1}{6,4}{⼟、⼖}[HSK 3]
  \definition[个,片]{s.}{área; distrito; região; um lugar maior | prefeitura; unidade administrativa | latitudes; localidade; lado | em determinadas circunstâncias, algumas regiões administrativas locais da China, como Hong Kong e Macau, participam individualmente em algumas atividades internacionais}
  \definition{suf.}{como sufixo do nome da cidade, significa prefeitura ou condado}
\end{EntryWithPhonetic}

\begin{EntryWithPhonetic}{地上}{di4 shang5}{6,3}{⼟、⼀}[HSK 1]
  \definition{adv.}{no chão; no solo; em terra}
\end{EntryWithPhonetic}

\begin{EntryWithPhonetic}{地铁}{di4tie3}{6,10}{⼟、⾦}[HSK 2]
  \definition[条,班,列,趟]{s.}{metrô; trem subterrâneo; também se refere ao vagão do metrô}
\end{EntryWithPhonetic}

\begin{EntryWithPhonetic}{地铁站}{di4 tie3 zhan4}{6,10,10}{⼟、⾦、⽴}[HSK 2]
  \definition[个,座]{s.}{estação de metrô}
\end{EntryWithPhonetic}

\begin{EntryWithPhonetic}{地图}{di4tu2}{6,8}{⼟、⼞}[HSK 1]
  \definition[张,本]{s.}{mapa; mapa que mostra a distribuição de coisas e fenômenos na superfície da Terra, com símbolos e textos, e às vezes também com cores}
\end{EntryWithPhonetic}

\begin{EntryWithPhonetic}{地位}{di4wei4}{6,7}{⼟、⼈}[HSK 4]
  \definition[个]{s.}{lugar; status; posição; posição da pessoa ou do grupo nas relações sociais | lugar; posição (ocupada por uma pessoa ou coisa); espaço ocupado por uma pessoa ou coisa}
\end{EntryWithPhonetic}

\begin{EntryWithPhonetic}{地下}{di4 xia4}{6,3}{⼟、⼀}[HSK 4]
  \definition{s.}{subterrâneo | secreta (atividade) | recursos ocultos}
\end{EntryWithPhonetic}

\begin{EntryWithPhonetic}{地下室}{di4 xia4 shi4}{6,3,9}{⼟、⼀、⼧}[HSK 6]
  \definition{s.}{subterrâneo; porão; adega | abóbadas; cripta}
\end{EntryWithPhonetic}

\begin{EntryWithPhonetic}{地形}{di4 xing2}{6,7}{⼟、⼺}[HSK 5]
  \definition{s.}{topografia; forma do terreno; relevo; disposição do terreno; característica do relevo; característica da superfície; terreno}
\end{EntryWithPhonetic}

\begin{EntryWithPhonetic}{地狱}{di4yu4}{6,9}{⼟、⽝}
  \definition*{s.}{Budismo: Naraka}
  \definition{adj.}{infernal}
  \definition[个,层,重,处]{s.}{inferno; submundo; algumas religiões se referem ao lugar onde a alma sofre após a morte | inferno na terra; lugar de tormento como o inferno; uma metáfora para um ambiente de vida sombrio e miserável}
\end{EntryWithPhonetic}

\begin{EntryWithPhonetic}{地震}{di4zhen4}{6,15}{⼟、⾬}[HSK 5]
  \definition[场,次,级]{s.}{sismo; terremoto; tremor de terra; vibrações na crosta terrestre}
  \definition{v.}{sacudir com vibrações sísmicas}
\end{EntryWithPhonetic}

\begin{EntryWithPhonetic}{地址}{di4zhi3}{6,7}{⼟、⼟}[HSK 4]
  \definition[个,条]{s.}{endereço; local de residência ou correspondência}
\end{EntryWithPhonetic}

\begin{EntryWithPhonetic}{地砖}{di4zhuan1}{6,9}{⼟、⽯}
  \definition{s.}{ladrilho de piso}
\end{EntryWithPhonetic}

\begin{EntryWithPhonetic}{弟}{di4}{7}{⼸}[HSK 1]
  \definition*{s.}{Sobrenome Di}
  \definition[个]{s.}{irmão mais novo | (entre amigos homens) eu | geralmente se refere a colegas do sexo masculino mais jovens na família ou entre parentes | forma humilde que os amigos usam para se referir uns aos outros, usada principalmente em correspondência}
\end{EntryWithPhonetic}

\begin{EntryWithPhonetic}{弟弟}{di4 di5}{7,7}{⼸、⼸}[HSK 1]
  \definition[个,位]{s.}{irmão mais novo | primo}
\end{EntryWithPhonetic}

\begin{EntryWithPhonetic}{弟妹}{di4mei4}{7,8}{⼸、⼥}
  \definition{s.}{esposa do irmão mais novo}
\end{EntryWithPhonetic}

\begin{EntryWithPhonetic}{的}{di4}{8}{⽩}
  \definition{adj.}{alvo; centro do alvo}
  \seeref{的}{de5}
  \seeref{的}{di1}
  \seeref{的}{di2}
\end{EntryWithPhonetic}

\begin{EntryWithPhonetic}{帝}{di4}{9}{⼱}
  \definition*{s.}{Ser Supremo; Deus}
  \definition[位,名,个]{s.}{imperador | (abreviação) imperialismo}
\end{EntryWithPhonetic}

\begin{EntryWithPhonetic}{帝国}{di4guo2}{9,8}{⼱、⼞}
  \definition{adj.}{imperial}
  \definition{s.}{império}
\end{EntryWithPhonetic}

\begin{EntryWithPhonetic}{递}{di4}{10}{⾡}[HSK 5]
  \definition{adv.}{na ordem correta; sucessivamente}
  \definition{v.}{entregar; passar; dar; transmitir}
\end{EntryWithPhonetic}

\begin{EntryWithPhonetic}{递给}{di4 gei3}{10,9}{⾡、⽷}[HSK 5]
  \definition{v.}{entregar algo a alguém; passar itens ou coisas para outras pessoas}
\end{EntryWithPhonetic}

\begin{EntryWithPhonetic}{第}{di4}{11}{⽵}[HSK 1]
  \definition*{s.}{Sobrenome Di}
  \definition{adv.}{mas, apenas, somente; Indica que a ação não está sujeita a restrições ou condições; equivalente a 只管}
  \definition{conj.}{mas; contudo; orações de conexão; indicando uma relação de transição; equivalente a 但是}
  \definition{pref.}{palavra auxiliar para números ordinais; usado antes de números inteiros, indica ordem}
  \definition{s.}{diferentes notas dos candidatos aprovados nos exames imperiais | a residência de um alto funcionário; grandes residências dos burocratas da era feudal}
  \seealsoref{但是}{dan4 shi4}
  \seealsoref{只管}{zhi3 guan3}
\end{EntryWithPhonetic}

\begin{EntryWithPhonetic}{墬}{di4}{14}{⼟}
  \variantof{地}
\end{EntryWithPhonetic}

\begin{EntryWithPhonetic}{典}{dian3}{8}{⼋}
  \definition{s.}{lei; cânone; padrão; sistema; regulamentos | trabalho padrão de bolsa de estudos; livros que podem servir como padrões ou especificações | alusão; citação literária | cerimônia; uma grande cerimônia (nos tempos antigos, a etiqueta era um dos sistemas importantes do estado) | modelo; normas; regras}
  \definition{v.}{estar no comando de | hipotecar; usar imóveis ou casas como garantia ao pedir dinheiro emprestado}
\end{EntryWithPhonetic}

\begin{EntryWithPhonetic}{典礼}{dian3li3}{8,5}{⼋、⽰}[HSK 5]
  \definition[个,次,场]{s.}{cerimônia; celebração; comemoração}
\end{EntryWithPhonetic}

\begin{EntryWithPhonetic}{典型}{dian3xing2}{8,9}{⼋、⼟}[HSK 4]
  \definition{adj.}{típico; representativo}
  \definition[个,种]{s.}{modelo; caso típico; indivíduo ou evento representativo | personagens típicos; personalidades modelo (em obras literárias); personagens na literatura e na arte que refletem a natureza de uma determinada sociedade e têm uma personalidade distinta}
\end{EntryWithPhonetic}

\begin{EntryWithPhonetic}{点}{dian3}{9}{⽕}[HSK 1]
  \definition{clas.}{hora cheia | ponto, uma unidade de medida para tipos; antigamente, a contagem do tempo durante a noite era feita por turnos, sendo cada turno dividido em cinco pontos | quantidade ínfima; um pouco; um pouquinho; alguma coisa; indica uma pequena quantidade | usado para itens}
  \definition{s.}{gota (de líquido); (ponto) pequena gota de líquido | mancha; ponto; salpico; (um pouco) Um pequeno vestígio | (ponto) Traço de um caractere chinês, cuja forma é ``、''  | ponto; (matemática) refere-se a uma figura geométrica que não tem comprimento, largura ou altura, mas apenas uma posição | gongo, instrumento musical de metal | ponto decimal; refere-se ao ponto decimal, símbolo matemático que representa os números decimais | lugar específico | lanche leve; petisco | lugar; grau; sinalização de um determinado local ou grau | hora marcada; hora regulamentar | aspecto; característica; partes ou aspectos específicos de algo | ritmo; batida}
  \definition{v.}{andar na ponta dos pés | dar uma dica, sugestão | tocar levemente com o dedo, pincel ou vara; tocar muito brevemente; passar rapidamente | acenar; baixar ligeiramente a cabeça e levantar rapidamente | gotejar; fazer cair líquido | semear em buracos; plantar com um plantador | verificar um por um | colocar um ponto; usar caneta e outras ferramentas para adicionar ideias | sugerir; indicar; dar uma dica | decorar; realçar | selecionar; escolher; especificar o que é exigido | acender; queimar; inflamar | (pedido) comer uma pequena quantidade de comida para saciar a fome}
\end{EntryWithPhonetic}

\begin{EntryWithPhonetic}{点火}{dian3huo3}{9,4}{⽕、⽕}
  \definition{s.}{ignição}
  \definition{v.}{inflamar | acender um fogo | agitar | dar partida em um motor | (figurativo) provocar problemas}
\end{EntryWithPhonetic}

\begin{EntryWithPhonetic}{点名}{dian3 ming2}{9,6}{⽕、⼝}[HSK 4]
  \definition{v.}{fazer a lista de chamada; manter o controle da presença de alguém; chamar nomes para controle de presença | mencionar alguém pelo nome}
\end{EntryWithPhonetic}

\begin{EntryWithPhonetic}{点燃}{dian3 ran2}{9,16}{⽕、⽕}[HSK 5]
  \definition{v.}{acender; inflamar; acender uma fogueira, para iluminar}
\end{EntryWithPhonetic}

\begin{EntryWithPhonetic}{点头}{dian3 tou2}{9,5}{⽕、⼤}[HSK 2]
  \definition{v.}{acenar com a cabeça; balançar a cabeça; mover ligeiramente a cabeça para baixo; indicar permissão, aprovação, compreensão ou saudação}
\end{EntryWithPhonetic}

\begin{EntryWithPhonetic}{电}{dian4}{5}{⽥}[HSK 1]
  \definition*{s.}{Sobrenome Dian}
  \definition{s.}{eletricidade; energia elétrica | telegrama | relâmpago}
  \definition{v.}{dar ou receber um choque elétrico | enviar telegrama, telefonar ou enviar fax}
\end{EntryWithPhonetic}

\begin{EntryWithPhonetic}{电冰箱}{dian4bing1xiang1}{5,6,15}{⽥、⼎、⾋}
  \definition[台]{s.}{frigorífico | refrigerador}
\end{EntryWithPhonetic}

\begin{EntryWithPhonetic}{电车}{dian4 che1}{5,4}{⽥、⾞}[HSK 6]
  \definition[辆,班,趟,路]{s.}{bonde; veículos de transporte público urbano movidos por linhas aéreas e acionados por motores de tração}
\end{EntryWithPhonetic}

\begin{EntryWithPhonetic}{电车司机}{dian4che1 si1ji1}{5,4,5,6}{⽥、⾞、⼝、⽊}
  \definition{s.}{motorista de bonde}
\end{EntryWithPhonetic}

\begin{EntryWithPhonetic}{电池}{dian4chi2}{5,6}{⽥、⽔}[HSK 5]
  \definition[节,块,组,个]{s.}{célula; bateria}
\end{EntryWithPhonetic}

\begin{EntryWithPhonetic}{电灯}{dian4 deng1}{5,6}{⽥、⽕}[HSK 4]
  \definition[盏,个]{s.}{luz elétrica; lâmpada elétrica; lâmpadas que usam eletricidade como fonte de energia}
\end{EntryWithPhonetic}

\begin{EntryWithPhonetic}{电灯泡}{dian4deng1pao4}{5,6,8}{⽥、⽕、⽔}
  \definition{s.}{lâmpada elétrica | (gíria) terceiro convidado indesejado}
\end{EntryWithPhonetic}

\begin{EntryWithPhonetic}{电动}{dian4 dong4}{5,6}{⽥、⼒}[HSK 6]
  \definition{adj.}{motorizado; acionado por energia elétrica; operado por energia elétrica elétrico}
\end{EntryWithPhonetic}

\begin{EntryWithPhonetic}{电动车}{dian4 dong4 che1}{5,6,4}{⽥、⼒、⾞}[HSK 4]
  \definition{s.}{veículo elétrico (\emph{scooter}, bicicleta, carro, etc.)}
\end{EntryWithPhonetic}

\begin{EntryWithPhonetic}{电饭锅}{dian4 fan4 guo1}{5,7,12}{⽥、⾷、⾦}[HSK 5]
  \definition[台,个]{s.}{panela elétrica de arroz}
\end{EntryWithPhonetic}

\begin{EntryWithPhonetic}{电话}{dian4 hua4}{5,8}{⽥、⾔}[HSK 1]
  \definition[部]{s.}{telefone; aparelho telefônico; telefonia}
  \definition[通]{s.}{chamada telefônica; telefonema}
\end{EntryWithPhonetic}

\begin{EntryWithPhonetic}{电力}{dian4 li4}{5,2}{⽥、⼒}[HSK 6]
  \definition{s.}{energia elétrica; fornecimento de energia elétrica | energia elétrica | eletricidade}
\end{EntryWithPhonetic}

\begin{EntryWithPhonetic}{电脑}{dian4nao3}{5,10}{⽥、⾁}[HSK 1]
  \definition[个,台]{s.}{computador eletrônico}
\end{EntryWithPhonetic}

\begin{EntryWithPhonetic}{电脑语言}{dian4nao3yu3yan2}{5,10,9,7}{⽥、⾁、⾔、⾔}
  \definition{s.}{linguagem de programação | linguagem de computador}
\end{EntryWithPhonetic}

\begin{EntryWithPhonetic}{电器}{dian4 qi4}{5,16}{⽥、⼝}[HSK 6]
  \definition[件,种]{s.}{dispositivo elétrico; cargas em circuitos e dispositivos usados ​​para controlar, regular ou proteger circuitos, motores, etc.; como alto-falantes; interruptores; resistores; fusíveis, etc. | eletrodomésticos ou aparelhos elétricos domésticos; refere-se a eletrodomésticos, como televisores, gravadores, geladeiras, máquinas de lavar, etc.}
\end{EntryWithPhonetic}

\begin{EntryWithPhonetic}{电视}{dian4shi4}{5,8}{⽥、⾒}[HSK 1]
  \definition[部,台,个]{s.}{televisão; TV; televisor}
\end{EntryWithPhonetic}

\begin{EntryWithPhonetic}{电视机}{dian4 shi4 ji1}{5,8,6}{⽥、⾒、⽊}[HSK 1]
  \definition[个,台]{s.}{aparelho de TV; receptor de televisão; receptor de imagem; televisor; aparelho de televisão}
\end{EntryWithPhonetic}

\begin{EntryWithPhonetic}{电视剧}{dian4 shi4 ju4}{5,8,10}{⽥、⾒、⼑}[HSK 3]
  \definition[部,集,个]{s.}{série de TV; drama de TV; novela; drama escrito e gravado para transmissão pela televisão}
\end{EntryWithPhonetic}

\begin{EntryWithPhonetic}{电视台}{dian4 shi4 tai2}{5,8,5}{⽥、⾒、⼝}[HSK 3]
  \definition[家,座,个]{s.}{canal de TV; estação de televisão; locais e instituições que transmitem programas de televisão}
\end{EntryWithPhonetic}

\begin{EntryWithPhonetic}{电台}{dian4 tai2}{5,5}{⽥、⼝}[HSK 3]
  \definition[个,家]{s.}{transceptor; transmissor-receptor | aparelho de rádio; estação de rádio; estação de transmissão}
\end{EntryWithPhonetic}

\begin{EntryWithPhonetic}{电梯}{dian4ti1}{5,11}{⽥、⽊}[HSK 4]
  \definition[部,台,架]{s.}{elevador}
\end{EntryWithPhonetic}

\begin{EntryWithPhonetic}{电梯司机}{dian4ti1 si1ji1}{5,11,5,6}{⽥、⽊、⼝、⽊}
  \definition{s.}{ascensorista}
\end{EntryWithPhonetic}

\begin{EntryWithPhonetic}{电影}{dian4ying3}{5,15}{⽥、⼺}[HSK 1]
  \definition[部,片,幕,场]{s.}{filme; longa-metragem; cinema}
\end{EntryWithPhonetic}

\begin{EntryWithPhonetic}{电影奖}{dian4ying3jiang3}{5,15,9}{⽥、⼺、⼤}
  \definition{s.}{premiações de cinema}
\end{EntryWithPhonetic}

\begin{EntryWithPhonetic}{电影节}{dian4ying3jie2}{5,15,5}{⽥、⼺、⾋}
  \definition{s.}{festival de cinema}
\end{EntryWithPhonetic}

\begin{EntryWithPhonetic}{电影界}{dian4ying3jie4}{5,15,9}{⽥、⼺、⽥}
  \definition{s.}{indústria cinematográfica}
\end{EntryWithPhonetic}

\begin{EntryWithPhonetic}{电影票}{dian4ying3piao4}{5,15,11}{⽥、⼺、⽰}
  \definition{s.}{ingresso de filme}
\end{EntryWithPhonetic}

\begin{EntryWithPhonetic}{电影术}{dian4ying3 shu4}{5,15,5}{⽥、⼺、⽊}
  \definition{s.}{cinematografia}
\end{EntryWithPhonetic}

\begin{EntryWithPhonetic}{电影艺术}{dian4ying3 yi4shu4}{5,15,4,5}{⽥、⼺、⾋、⽊}
  \definition{s.}{arte cinematográfica}
\end{EntryWithPhonetic}

\begin{EntryWithPhonetic}{电影音乐}{dian4ying3 yin1yue4}{5,15,9,5}{⽥、⼺、⾳、⼃}
  \definition{s.}{música cinematográfica}
\end{EntryWithPhonetic}

\begin{EntryWithPhonetic}{电影院}{dian4 ying3 yuan4}{5,15,9}{⽥、⼺、⾩}[HSK 1]
  \definition[家,座,个]{s.}{cinema; sala de cinema; teatro; salão de cinema; local comercial dedicado à exibição de filmes}
\end{EntryWithPhonetic}

\begin{EntryWithPhonetic}{电邮}{dian4you2}{5,7}{⽥、⾢}
  \definition{s.}{correio eletrônico, \emph{e-mail} | abreviação de~电子邮件}
  \seealsoref{电子邮件}{dian4zi3you2jian4}
\end{EntryWithPhonetic}

\begin{EntryWithPhonetic}{电源}{dian4yuan2}{5,13}{⽥、⽔}[HSK 4]
  \definition[台,个,套]{s.}{fonte de alimentação; fonte de energia; fonte de energia elétrica; dispositivo que fornece energia elétrica a um aparelho, como uma bateria, um gerador, etc.}
\end{EntryWithPhonetic}

\begin{EntryWithPhonetic}{电子}{dian4zi3}{5,3}{⽥、⼦}
  \definition{s.}{eletrônico | elétron}
\end{EntryWithPhonetic}

\begin{EntryWithPhonetic}{电子版}{dian4 zi3 ban3}{5,3,8}{⽥、⼦、⽚}[HSK 5]
  \definition[个]{s.}{edição eletrônica}
\end{EntryWithPhonetic}

\begin{EntryWithPhonetic}{电子名片}{dian4zi3 ming2pian4}{5,3,6,4}{⽥、⼦、⼝、⽚}
  \definition{s.}{cartão de visita eletrônico}
\end{EntryWithPhonetic}

\begin{EntryWithPhonetic}{电子邮件}{dian4zi3you2jian4}{5,3,7,6}{⽥、⼦、⾢、⼈}[HSK 3]
  \definition[封,份,个,条]{s.}{correio eletrônico; \emph{e-mail}}
  \seealsoref{电邮}{dian4you2}
\end{EntryWithPhonetic}

\begin{EntryWithPhonetic}{店}{dian4}{8}{⼴}[HSK 2]
  \definition[家,间,个]{s.}{loja; armazém; loja de venda de mercadorias | pousada; pequena pousada com instalações simples | usado para nomes de lugares}
\end{EntryWithPhonetic}

\begin{EntryWithPhonetic}{店员}{dian4yuan2}{8,7}{⼴、⼝}
  \definition{s.}{assistente de loja | balconista | vendedor}
\end{EntryWithPhonetic}

\begin{EntryWithPhonetic}{店主}{dian4zhu3}{8,5}{⼴、⼂}
  \definition{s.}{lojista | dono de loja}
\end{EntryWithPhonetic}

\begin{EntryWithPhonetic}{垫}{dian4}{9}{⼟}
  \definition[个]{s.}{almofada}
  \definition{v.}{colocar algo sob; elevar ou nivelar; encher; preencher | pagar por alguém e esperar ser reembolsado mais tarde | colocar algo sob algo para elevá-lo ou nivelá-lo; usar algo para apoiar, espalhar ou forrar algo para torná-lo mais alto, mais grosso ou mais plano | preencher uma vaga; preencher uma lacuna}
\end{EntryWithPhonetic}

\begin{EntryWithPhonetic}{垫子}{dian4zi5}{9,3}{⼟、⼦}
  \definition{s.}{colchão | esteira | almofada}
\end{EntryWithPhonetic}

\begin{EntryWithPhonetic}{钿}{dian4}{10}{⾦}
  \definition{s.}{ornamento incrustado antigo em forma de flor | enfeite de cabelo feminino com flores douradas | incrustação de madrepérola; um padrão incrustado com conchas de caracóis em madeira e laca}
  \definition{v.}{incrustar com ouro, prata, etc.}
  \seeref{钿}{tian2}
\end{EntryWithPhonetic}

\begin{EntryWithPhonetic}{淀}{dian4}{11}{⽔}
  \definition{s.}{lago raso, frequentemente usado em nomes de lugares}
  \definition{v.}{formar sedimentos | sedimentar; precipitar}
\end{EntryWithPhonetic}

\begin{EntryWithPhonetic}{貂}{diao1}{12}{⾘}
  \definition*{s.}{Sobrenome Diao}
  \definition[只]{s.}{marta; fuinha; arminho}
\end{EntryWithPhonetic}

\begin{EntryWithPhonetic}{雕}{diao1}{16}{⾫}
  \definition*{s.}{Sobrenome Diao}
  \definition{s.}{abutre; águia | escultura ou obras esculpidas}
  \definition{v.}{esculpir; gravar}
\end{EntryWithPhonetic}

\begin{EntryWithPhonetic}{雕刻}{diao1ke4}{16,8}{⾫、⼑}
  \definition{s.}{escultura}
  \definition{v.}{esculpir | gravar}
\end{EntryWithPhonetic}

\begin{EntryWithPhonetic}{鸟}{diao3}{5}{⿃}[Kangxi 196]
  \definition{s.}{(em romances tradicionais, como termo pejorativo) maldito; condenado; fudido; o mesmo que 屌}
  \seeref{鸟}{niao3}
  \seealsoref{屌}{diao3}
\end{EntryWithPhonetic}

\begin{EntryWithPhonetic}{屌}{diao3}{9}{⼫}
  \definition{adj.}{(gíria) legal ou extraordinário}
  \definition{s.}{órgão genital masculino; pênis}
  \definition{v.}{(cantonês) foder}
\end{EntryWithPhonetic}

\begin{EntryWithPhonetic}{屌丝}{diao3si1}{9,5}{⼫、⼀}
  \definition{adj.}{panaca | zé-ninguém | (gíria de \emph{Internet}) \emph{looser}}
\end{EntryWithPhonetic}

\begin{EntryWithPhonetic}{吊}{diao4}{6}{⼝}[HSK 6]
  \definition{clas.}{uma sequência de 1.000 em dinheiro; antigamente, uma unidade monetária geralmente era composta por mil pequenas moedas de cobre}
  \definition{s.}{guindaste}
  \definition{v.}{pendurar; suspender | levantar ou abaixar com uma corda, etc. | colocar um forro de pele; adicionar revestimentos ou forros aos barris de couro para fazer roupas | revogar; retirar; recuperar documentos emitidos | lamentar; prestar homenagem os mortos ou oferecer condolências às famílias ou grupos que sofreram uma perda}
\end{EntryWithPhonetic}

\begin{EntryWithPhonetic}{钓}{diao4}{8}{⾦}
  \definition{v.}{pescar com anzol e linha | buscar (fama e ganho pessoal) | fisgar; defraudar por meios desleais}
\end{EntryWithPhonetic}

\begin{EntryWithPhonetic}{钓鱼}{diao4yu2}{8,8}{⾦、⿂}
  \definition{v.}{pescar (com linha e anzol) | (figurativo) aprisionar}
\end{EntryWithPhonetic}

\begin{EntryWithPhonetic}{调}{diao4}{10}{⾔}[HSK 3]
  \definition{s.}{sotaque; pronúncia | nota (musical) | melodia; música | tom; refere-se ao tom da fala, ou seja, a elevação e descida do tom das palavras | estilo; ambiente; estilo metafórico, talento, etc. | argumento; discurso}
  \definition{v.}{deslocar; mover; transferir; mover (pessoas, objetos, etc.) de um lugar para outro | examinar; investigar}
  \seeref{调}{tiao2}
\end{EntryWithPhonetic}

\begin{EntryWithPhonetic}{调查}{diao4cha2}{10,9}{⾔、⽊}[HSK 3]
  \definition[项,个,份]{s.}{pesquisa; investigação; informações obtidas após perguntar a outras pessoas ou investigar}
  \definition{v.}{investigar; indagar; inquerir; examinar; realizar uma investigação (geralmente no local) para entender a situação}
\end{EntryWithPhonetic}

\begin{EntryWithPhonetic}{调动}{diao4dong4}{10,6}{⾔、⼒}[HSK 5]
  \definition{v.}{mudar; transferir; pessoal, trabalho | mobilizar; despertar; pôr em jogo; melhorar (motivação, entusiasmo, etc.) por meio de alguns meios | reunir; manobrar; mover (tropas); mobilizar forças militares}
\end{EntryWithPhonetic}

\begin{EntryWithPhonetic}{调研}{diao4 yan2}{10,9}{⾔、⽯}[HSK 6]
  \definition{v.}{pesquisar e estudar; investigar e pesquisar; pesquisar}
\end{EntryWithPhonetic}

\begin{EntryWithPhonetic}{掉}{diao4}{11}{⼿}[HSK 2]
  \definition{v.}{cair; soltar-se; desprender-se | ficar para trás | perder; desaparecer; omitir | diminuir; reduzir | balançar; abanar; oscilar | virar; voltar; retornar | alterar; trocar; intercambiar}
  \definition{v.aux.}{usado após certos verbos para indicar a conclusão de uma ação}
\end{EntryWithPhonetic}

\begin{EntryWithPhonetic}{掉包}{diao4bao1}{11,5}{⼿、⼓}
  \definition{v.}{vender uma falsificação pelo artigo genuíno | roubar o item valioso de alguém e substituí-lo por um item de aparência semelhante, mas sem valor}
\end{EntryWithPhonetic}

\begin{EntryWithPhonetic}{掉膘}{diao4biao1}{11,15}{⼿、⾁}
  \definition{v.}{perder peso (gado)}
\end{EntryWithPhonetic}

\begin{EntryWithPhonetic}{掉队}{diao4dui4}{11,4}{⼿、⾩}
  \definition{v.}{abandonar | ficar para trás}
\end{EntryWithPhonetic}

\begin{EntryWithPhonetic}{掉落}{diao4luo4}{11,12}{⼿、⾋}
  \definition{v.}{derrubar}
\end{EntryWithPhonetic}

\begin{EntryWithPhonetic}{掉线}{diao4xian4}{11,8}{⼿、⽷}
  \definition{v.}{desconectar-se (da \emph{Internet})}
\end{EntryWithPhonetic}

\begin{EntryWithPhonetic}{掉转}{diao4zhuan3}{11,8}{⼿、⾞}
  \definition{v.}{dar a volta}
\end{EntryWithPhonetic}

\begin{EntryWithPhonetic}{跌}{die1}{12}{⾜}[HSK 6]
  \definition{s.}{(de um objeto, etc.) queda; tombo | (de preços, etc.) queda}
  \definition{v.}{cair; tombar; perder o equilíbrio e cair | cair (objetos caindo); descer | cair (queda de preços)}
\end{EntryWithPhonetic}

\begin{EntryWithPhonetic}{叮}{ding1}{5}{⼝}
  \definition{v.}{picar; ferroar | dizer ou perguntar novamente para ter certeza; verificar; insistir; certificar-se | sondar; perseguir}
\end{EntryWithPhonetic}

\begin{EntryWithPhonetic}{叮嘱}{ding1zhu3}{5,15}{⼝、⼝}
  \definition{v.}{exortar | avisar | insistir de novo e de novo}
\end{EntryWithPhonetic}

\begin{EntryWithPhonetic}{顶}{ding3}{8}{⾴}[HSK 4]
  \definition{adv.}{muito (linguagem falada); a maioria; extremamente; expressa o grau mais alto, equivalente a 最 e 极}
  \definition{clas.}{usado para coisas que têm um topo}
  \definition{prep.}{até}
  \definition{s.}{coroa da cabeça; parte mais alta do corpo ou objeto | topo; limite superior; ponto mais alto}
  \definition{v.}{carregar na cabeça; carregar em sua cabeça | empurrar (ou apoiar) para cima; empurrar por baixo (ou por trás) | dar cabeçadas; dar uma coronhada | sustentar; apoiar; suportar | resistir; ir contra; enfrentar | rebater; retorquir; responder de volta | cooperar; enfrentar; apoiar; dar suporte | igualar; ser equivalente a | substituir; tomar o lugar de | assumir o controle; transferir ou adquirir o direito de administrar um negócio ou alugar uma casa ou terreno}
  \seealsoref{极}{ji2}
  \seealsoref{最}{zui4}
\end{EntryWithPhonetic}

\begin{EntryWithPhonetic}{鼎}{ding3}{12}{⿍}[Kangxi 206]
  \definition{adj.}{grande; generoso | importante; grandioso}
  \definition{adv.}{exatamente quando; exatamente o momento para}
  \definition[尊]{s.}{um antigo recipiente de cozinha com duas alças e três ou quatro pernas | pote; caldeirão | poder do estado; o trono | como símbolo de dinastia; nos tempos antigos, era considerada uma ferramenta importante para estabelecer um país}
\end{EntryWithPhonetic}

\begin{EntryWithPhonetic}{订}{ding4}{4}{⾔}[HSK 3]
  \definition{v.}{concluir; elaborar; concordar com |assinar (um jornal, etc.); reservar (assentos, ingressos, etc.); encomendar (mercadorias, etc.) | fazer correções; revisar | grampear junto; unir; usar linha ou arame para encadernar páginas soltas ou folhas de papel | julgar; determinar}
\end{EntryWithPhonetic}

\begin{EntryWithPhonetic}{定}{ding4}{8}{⼧}[HSK 4]
  \definition{adj.}{calmo; estável}
  \definition{adv.}{certamente; com certeza; definitivamente; espressa certeza ou necessidade}
  \definition{v.}{decidir; fixar; definir; determinar; ter certeza | acalmar; estabilizar; tornar estável | assinar (um jornal, etc.); reservar (assentos, ingressos, etc.); encomendar (mercadorias, etc.)}
\end{EntryWithPhonetic}

\begin{EntryWithPhonetic}{定价}{ding4 jia4}{8,6}{⼧、⼈}[HSK 6]
  \definition{s.}{fixação de preços; preço especificado}
  \definition{v.}{fixar um preço | fazer um preço; definir um preço}
\end{EntryWithPhonetic}

\begin{EntryWithPhonetic}{定期}{ding4qi1}{8,12}{⼧、⽉}[HSK 3]
  \definition{adj.}{regular; periódico; em intervalos regulares; com prazo determinado; por tempo limitado}
  \definition{v.}{fixar (definir) uma data; determinar a data; confirmar a data}
\end{EntryWithPhonetic}

\begin{EntryWithPhonetic}{定时}{ding4 shi2}{8,7}{⼧、⽇}[HSK 6]
  \definition{s.}{em um horário fixo; em intervalos regulares}
  \definition{v.}{cronometrar; fixar um tempo (para fazer algo)}
\end{EntryWithPhonetic}

\begin{EntryWithPhonetic}{定位}{ding4 wei4}{8,7}{⼧、⼈}[HSK 6]
  \definition{s.}{posição; localização; posição medida ou definida}
  \definition{v.}{localizar; posicionar; orientar; avaliar algo; usar instrumentos para determinar a localização de objetos; definir o \emph{status} das coisas}
\end{EntryWithPhonetic}

\begin{EntryWithPhonetic}{丢}{diu1}{6}{⼛}[HSK 5]
  \definition{v.}{perder; extraviar; estar ausente | lançar; atirar | colocar (deixar) de lado | deixar (para trás)}
\end{EntryWithPhonetic}

\begin{EntryWithPhonetic}{丢掉}{diu1diao4}{6,11}{⼛、⼿}
  \definition{v.}{jogar fora | descartar |perder}
\end{EntryWithPhonetic}

\begin{EntryWithPhonetic}{丢官}{diu1guan1}{6,8}{⼛、⼧}
  \definition{v.}{perder um cargo oficial}
\end{EntryWithPhonetic}

\begin{EntryWithPhonetic}{丢开}{diu1kai1}{6,4}{⼛、⼶}
  \definition{v.}{jogar fora ou deixar de lado | esquecer por um tempo}
\end{EntryWithPhonetic}

\begin{EntryWithPhonetic}{丢脸}{diu1lian3}{6,11}{⼛、⾁}
  \definition{adj.}{vergonhoso}
\end{EntryWithPhonetic}

\begin{EntryWithPhonetic}{丢弃}{diu1qi4}{6,7}{⼛、⼶}
  \definition{v.}{jogar fora | descartar}
\end{EntryWithPhonetic}

\begin{EntryWithPhonetic}{丢失}{diu1shi1}{6,5}{⼛、⼤}
  \definition{v.}{perder}
\end{EntryWithPhonetic}

\begin{EntryWithPhonetic}{丢下}{diu1xia4}{6,3}{⼛、⼀}
  \definition{v.}{abandonar}
\end{EntryWithPhonetic}

\begin{EntryWithPhonetic}{东}{dong1}{5}{⼀}[HSK 1]
  \definition*{s.}{Sobrenome Dong}
  \definition{s.}{leste; uma das quatro direções básicas, o lado onde o sol nasce | proprietário; dono | anfitrião (antigamente, o anfitrião ficava a leste e os convidados a oeste)}
\end{EntryWithPhonetic}

\begin{EntryWithPhonetic}{东半球}{dong1ban4qiu2}{5,5,11}{⼀、⼗、⽟}
  \definition*{s.}{Hemisfério Oriental}
\end{EntryWithPhonetic}

\begin{EntryWithPhonetic}{东北}{dong1 bei3}{5,5}{⼀、⼔}[HSK 2]
  \definition*{s.}{Nordeste da China; O Nordeste | Manchúria}
  \definition{s.}{nordeste}
\end{EntryWithPhonetic}

\begin{EntryWithPhonetic}{东边}{dong1 bian5}{5,5}{⼀、⾡}[HSK 1]
  \definition{s.}{leste; o lado leste; refere-se à fronteira oriental}
\end{EntryWithPhonetic}

\begin{EntryWithPhonetic}{东部}{dong1 bu4}{5,10}{⼀、⾢}[HSK 3]
  \definition{s.}{o leste; parte oriental; a parte oriental de uma determinada região}
\end{EntryWithPhonetic}

\begin{EntryWithPhonetic}{东方}{dong1 fang1}{5,4}{⼀、⽅}[HSK 2]
  \definition*{s.}{Sobrenome Dongfang}
  \definition{s.}{leste | oriente; o leste; o Oriente}
\end{EntryWithPhonetic}

\begin{EntryWithPhonetic}{东方学院}{dong1fang1 xue2yuan4}{5,4,8,9}{⼀、⽅、⼦、⾩}
  \definition*{s.}{Instituto Oriental}
\end{EntryWithPhonetic}

\begin{EntryWithPhonetic}{东面}{dong1mian4}{5,9}{⼀、⾯}
  \definition{s.}{lado leste (de algo)}
\end{EntryWithPhonetic}

\begin{EntryWithPhonetic}{东南}{dong1 nan2}{5,9}{⼀、⼗}[HSK 2]
  \definition*{s.}{Sudeste da China; O Sudeste; refere-se à região costeira sudeste da China, incluindo as províncias e cidades de Xangai, Jiangsu, Zhejiang, Fujian, Taiwan, etc.}
  \definition{s.}{sudeste}
\end{EntryWithPhonetic}

\begin{EntryWithPhonetic}{东西}{dong1xi1}{5,6}{⼀、⾑}
  \definition{s.}{leste e oeste | de leste a oeste; a distância de um lugar de leste a oeste}
  \seeref{东西}{dong1xi5}
\end{EntryWithPhonetic}

\begin{EntryWithPhonetic}{东西}{dong1xi5}{5,6}{⼀、⾑}[HSK 1]
  \definition[个,件]{s.}{coisa; refere-se a todos os tipos de coisas concretas ou abstratas | coisa; criatura; refere-se especificamente a pessoas ou coisas que causam repulsa ou simpatia}
  \seeref{东西}{dong1xi1}
\end{EntryWithPhonetic}

\begin{EntryWithPhonetic}{冬}{dong1}{5}{⼎}
  \definition*{s.}{Sobrenome Dong}
  \definition{s.}{inverno}
  \definition{s.}{onomatopéia: som de um tambor batendo, batendo na porta, etc.}
\end{EntryWithPhonetic}

\begin{EntryWithPhonetic}{冬瓜}{dong1gua1}{5,5}{⼎、⽠}
  \definition{s.}{melão de inverno}
\end{EntryWithPhonetic}

\begin{EntryWithPhonetic}{冬季}{dong1 ji4}{5,8}{⼎、⼦}[HSK 4]
  \definition[个,次,种]{s.}{inverno; o quarto trimestre do ano, habitualmente referido na China como o período de três meses entre o início do inverno e o início da primavera, e também referido como ``décimo, décimo primeiro e décimo segundo'' meses do calendário lunar}
\end{EntryWithPhonetic}

\begin{EntryWithPhonetic}{冬天}{dong1 tian1}{5,4}{⼎、⼤}[HSK 2]
  \definition[个]{s.}{inverno; a quarta estação do ano, na China, geralmente se refere aos três meses entre outubro e dezembro do calendário lunar}
\end{EntryWithPhonetic}

\begin{EntryWithPhonetic}{懂}{dong3}{15}{⼼}[HSK 2]
  \definition*{s.}{Sobrenome Dong}
  \definition{v.}{compreender; entender}
\end{EntryWithPhonetic}

\begin{EntryWithPhonetic}{懂得}{dong3 de5}{15,11}{⼼、⼻}[HSK 2]
  \definition{v.}{saber (significado, prática, etc.); compreender; entender}
\end{EntryWithPhonetic}

\begin{EntryWithPhonetic}{动}{dong4}{6}{⼒}[HSK 1]
  \definition{adj.}{não estacionário; móvel; variável; mutável}
  \definition{adv.}{facilmente; frequentemente}
  \definition{s.}{ação; movimento}
  \definition{v.}{mover; mexer; (pessoas ou coisas) mudar a posição ou o estado original | agir; começar a agir; entrar em ação | alterar; mudar; alterar a posição ou o estado original | usar; utilizar; tornar ativo | despertar; tocar (o coração de alguém); provocar mudanças emocionais, reações | [geralmente na forma negativa] comer ou beber | emocionar; deixar emocionado}
\end{EntryWithPhonetic}

\begin{EntryWithPhonetic}{动感}{dong4gan3}{6,13}{⼒、⼼}
  \definition{adj.}{dinâmica | vívida}
  \definition{adv.}{dinamicamente}
  \definition{s.}{senso de movimento (geralmente em uma obra de arte estática)}
\end{EntryWithPhonetic}

\begin{EntryWithPhonetic}{动画}{dong4 hua4}{6,8}{⼒、⽥}[HSK 6]
  \definition[部]{s.}{desenho animado; animação; a imagem em movimento formada pela fotografia contínua das imagens desenhadas}
\end{EntryWithPhonetic}

\begin{EntryWithPhonetic}{动画片}{dong4 hua4 pian4}{6,8,4}{⼒、⽥、⽚}[HSK 4]
  \definition[部,集,个]{s.}{desenho animado; animações; filme de animação}
\end{EntryWithPhonetic}

\begin{EntryWithPhonetic}{动机}{dong4ji1}{6,6}{⼒、⽊}[HSK 5]
  \definition[个]{s.}{motivo; razão; intenção; ideias que motivam as pessoas a se envolverem em determinados comportamentos}
\end{EntryWithPhonetic}

\begin{EntryWithPhonetic}{动力}{dong4li4}{6,2}{⼒、⼒}
  \definition[种,个]{s.}{poder; a força que faz com que as máquinas funcionem, por exemplo, energia elétrica, eólica, hidráulica, etc. | ímpeto; força motriz (ou propulsora); refere-se, de maneira geral, à força que impulsiona o desenvolvimento das coisas}
\end{EntryWithPhonetic}

\begin{EntryWithPhonetic}{动漫}{dong4man4}{6,14}{⼒、⽔}
  \definition{s.}{desenhos animados | quadrinhos | anime | mangá}
\end{EntryWithPhonetic}

\begin{EntryWithPhonetic}{动人}{dong4 ren2}{6,2}{⼒、⼈}[HSK 3]
  \definition{adj.}{comovente; emocionante; tocante}
\end{EntryWithPhonetic}

\begin{EntryWithPhonetic}{动身}{dong4shen1}{6,7}{⼒、⾝}
  \definition{v.+compl.}{fazer uma jornada | começar uma jornada | partir | partir em uma jornada | sair (para um lugar distante)}
\end{EntryWithPhonetic}

\begin{EntryWithPhonetic}{动手}{dong4shou3}{6,4}{⼒、⼿}[HSK 5]
  \definition{v.+compl.}{iniciar o trabalho; começar a trabalhar | tocar; manusear; manipular | bater; levantar a mão (para bater); espancar}
\end{EntryWithPhonetic}

\begin{EntryWithPhonetic}{动态}{dong4tai4}{6,8}{⼒、⼼}[HSK 5]
  \definition{s.}{tendências; desenvolvimentos; tendência geral dos assuntos; causa provável de ação; curso dos acontecimentos | expressão; comportamento ativo | estado dinâmico; condição dinâmica; de ou em relação a um estado de movimento}
\end{EntryWithPhonetic}

\begin{EntryWithPhonetic}{动物}{dong4wu4}{6,8}{⼒、⽜}[HSK 2]
  \definition[个,只,群,种]{s.}{animal; uma grande classe de seres vivos, que se alimentam principalmente de matéria orgânica, possuem sistema nervoso, são sensíveis e capazes de se mover; refere-se a todos os tipos de coisas concretas ou abstratas}
\end{EntryWithPhonetic}

\begin{EntryWithPhonetic}{动物园}{dong4 wu4 yuan2}{6,8,7}{⼒、⽜、⼞}[HSK 2]
  \definition[个,座,家]{s.}{jardim zoológico; zoo; parque que cria muitos tipos de animais (especialmente animais com valor científico ou raros na região) para exibição ao público}
\end{EntryWithPhonetic}

\begin{EntryWithPhonetic}{动摇}{dong4 yao2}{6,13}{⼒、⼿}[HSK 4]
  \definition{adj.}{instável}
  \definition{v.}{ondular; pairar; agitar; balançar; sacudir | hesitar; vacilar; esmorecer; abalar}
\end{EntryWithPhonetic}

\begin{EntryWithPhonetic}{动员}{dong4yuan2}{6,7}{⼒、⼝}[HSK 5]
  \definition{v.}{despertar; mobilizar; iniciar (para fazer algo ou participar de uma atividade) | mobilizar toda a nação; transferir dos setores militar, político e econômico para uma situação de guerra}
\end{EntryWithPhonetic}

\begin{EntryWithPhonetic}{动作}{dong4zuo4}{6,7}{⼒、⼈}[HSK 1]
  \definition[个]{s.}{movimento; ação; atividade de todo o corpo ou parte do corpo}
  \definition{v.}{agir; começar a se mover; entrar em ação}
\end{EntryWithPhonetic}

\begin{EntryWithPhonetic}{冻}{dong4}{7}{⼎}[HSK 5]
  \definition*{s.}{Sobrenome Dong}
  \definition{s.}{geleia; gelatina}
  \definition{v.}{congelar; ser congelado | ficar com frio ou sentir frio}
\end{EntryWithPhonetic}

\begin{EntryWithPhonetic}{洞}{dong4}{9}{⽔}[HSK 5]
  \definition{adj.}{profundo; minucioso; claro; completo; abrangente}
  \definition{s.}{buraco; cavidade; orifício; furo; parte penetrante ou profundamente recuada de um objeto; uma caverna}
\end{EntryWithPhonetic}

\begin{EntryWithPhonetic}{洞穴}{dong4xue2}{9,5}{⽔、⽳}
  \definition{s.}{caverna}
\end{EntryWithPhonetic}

\begin{EntryWithPhonetic}{都}{dou1}{10}{⾢}[HSK 1]
  \definition{adv.}{todos; representa a soma total | apenas por causa de; usado em conjunto com a palavra 是, explica o motivo | mesmo; até; indicativo de ênfase | já; significa 已经}
  \seeref{都}{du1}
  \seealsoref{是}{shi4}
  \seealsoref{已经}{yi3jing1}
\end{EntryWithPhonetic}

\begin{EntryWithPhonetic}{斗}{dou3}{4}{⽃}[Kangxi 68]
  \definition*{s.}{Ursa Maior; Abreviação de Estrela Polar | Dou, uma das mansões lunares; uma das 28 constelações | Sobrenome Dou}
  \definition{clas.}{dou, unidade de medida seca para grãos (=1 decalitro)}
  \definition{s.}{dou, medida para grãos; instrumento para medir grãos | um objeto com a forma de um copo ou concha; algo que se parece com um balde | espiral (de uma impressão digital)}
  \seeref{斗}{dou4}
\end{EntryWithPhonetic}

\begin{EntryWithPhonetic}{斗}{dou4}{4}{⽃}[Kangxi 68]
  \definition{v.}{brigar; lutar | lutar contra; denunciar | competir com; disputar com | fazer animais ou insetos lutarem (como um jogo) | encaixar-se | brincar com; provocar | provocar (risos, etc.); divertir | ser remendado; juntar; unir; combinar}
  \seeref{斗}{dou3}
\end{EntryWithPhonetic}

\begin{EntryWithPhonetic}{斗争}{dou4zheng1}{4,6}{⽃、⼑}[HSK 6]
  \definition[场,番]{s.}{luta; conflito; batalha; os dois lados entram em conflito entre si}
  \definition{v.}{lutar; combater; batalhar; os dois lados estão em conflito, lutando para derrotar um ao outro | esforçar-se por; lutar por; trabalhar duro por; expor e criticar; atacar | lutar; batalhar}
\end{EntryWithPhonetic}

\begin{EntryWithPhonetic}{豆}{dou4}{7}{⾖}[Kangxi 151]
  \definition*{s.}{Sobrenome Dou}
  \definition{s.}{planta que produz vagens ou suas sementes | coisa em forma de feijão | leguminosas ou sementes de leguminosas; feijões; ervilhas | uma xícara ou tigela antiga com haste}
\end{EntryWithPhonetic}

\begin{EntryWithPhonetic}{豆腐}{dou4fu5}{7,14}{⾖、⾁}[HSK 4]
  \definition[块,盒,斤,盘]{s.}{\emph{tofu}}
\end{EntryWithPhonetic}

\begin{EntryWithPhonetic}{豆荚}{dou4jia2}{7,9}{⾖、⾋}
  \definition{s.}{vagem (de legumes)}
\end{EntryWithPhonetic}

\begin{EntryWithPhonetic}{豆角}{dou4jiao3}{7,7}{⾖、⾓}
  \definition{s.}{feijão verde}
\end{EntryWithPhonetic}

\begin{EntryWithPhonetic}{豆制品}{dou4 zhi4 pin3}{7,8,9}{⾖、⼑、⼝}[HSK 5]
  \definition{s.}{produtos de soja}
\end{EntryWithPhonetic}

\begin{EntryWithPhonetic}{读}{dou4}{10}{⾔}
  \definition{s.}{vírgula; uma breve pausa na leitura}
  \seeref{读}{du2}
\end{EntryWithPhonetic}

\begin{EntryWithPhonetic}{都}{du1}{10}{⾢}
  \definition*{s.}{Sobrenome Du}
  \definition[座]{s.}{capital | cidade grande; metrópole}
  \seeref{都}{dou1}
\end{EntryWithPhonetic}

\begin{EntryWithPhonetic}{都市}{du1 shi4}{10,5}{⾢、⼱}[HSK 6]
  \definition[个]{s.}{cidade grande; grandes cidades}
\end{EntryWithPhonetic}

\begin{EntryWithPhonetic}{嘟}{du1}{13}{⼝}
  \definition{part.}{(onomatopéia) buzina}
  \definition{v.}{fazer beicinho}
\end{EntryWithPhonetic}

\begin{EntryWithPhonetic}{毒}{du2}{9}{⽏}[HSK 5]
  \definition*{s.}{Sobrenome Du}
  \definition{adj.}{veneno; toxina; propriedade ou substância prejudicial aos organismos vivos | droga; narcóticos | vírus; vírus de computador | influência venenosa}
  \definition{adj.}{venenoso; tóxico; envenenado | malicioso; cruel; feroz}
  \definition{v.}{matar com veneno; envenenar | envenenar (a mente de alguém)}
\end{EntryWithPhonetic}

\begin{EntryWithPhonetic}{毒害}{du2hai4}{9,10}{⽏、⼧}
  \definition{s.}{envenenamento}
  \definition{v.}{envenenar (prejudicar com uma substância tóxica) | envenenar (as mentes das pessoas)}
\end{EntryWithPhonetic}

\begin{EntryWithPhonetic}{毒品}{du2pin3}{9,9}{⽏、⼝}[HSK 6]
  \definition[种,点]{s.}{drogas; veneno; narcóticos; refere-se ao ópio, morfina, heroína, etc. usados ​​como vício}
\end{EntryWithPhonetic}

\begin{EntryWithPhonetic}{毒杀}{du2sha1}{9,6}{⽏、⽊}
  \definition{v.}{matar por envenenamento}
\end{EntryWithPhonetic}

\begin{EntryWithPhonetic}{毒蛇}{du2she2}{9,11}{⽏、⾍}
  \definition{s.}{víbora | cobra venenosa}
\end{EntryWithPhonetic}

\begin{EntryWithPhonetic}{毒物}{du2wu4}{9,8}{⽏、⽜}
  \definition{s.}{substância venenosa | toxina}
\end{EntryWithPhonetic}

\begin{EntryWithPhonetic}{独}{du2}{9}{⽝}
  \definition*{s.}{Sobrenome Du}
  \definition{adj.}{só; solteiro | (coloquial) distante | único; só}
  \definition{adv.}{unicamente; somente | sozinho; por si mesmo; em solidão}
  \definition{s.}{idosos sem descendência; os sem filhos}
\end{EntryWithPhonetic}

\begin{EntryWithPhonetic}{独立}{du2li4}{9,5}{⽝、⽴}[HSK 4]
  \definition{adj.}{independente; por conta própria | separado; respectivo; descreve algo que é separado e não está em contato com outra coisa}
  \definition{v.}{ficar sozinho | alcançar a independência; tornar-se um país independente; liberdade de um Estado, regime ou organização contra interferência, controle e dominação por forças externas}
\end{EntryWithPhonetic}

\begin{EntryWithPhonetic}{独特}{du2te4}{9,10}{⽝、⽜}[HSK 4]
  \definition{adj.}{único; distinto; original; especial}
\end{EntryWithPhonetic}

\begin{EntryWithPhonetic}{独自}{du2 zi4}{9,6}{⽝、⾃}[HSK 4]
  \definition{adv.}{sozinho; por si mesmo; por conta própria}
\end{EntryWithPhonetic}

\begin{EntryWithPhonetic}{读}{du2}{10}{⾔}[HSK 1]
  \definition*{s.}{Sobrenome Du}
  \definition{v.}{ler em voz alta | ler; ler o texto e compreendera seu significado | frequentar a escola; refere-se a ir à escola ou estudar | (computação) ler dados}
  \seeref{读}{dou4}
\end{EntryWithPhonetic}

\begin{EntryWithPhonetic}{读书}{du2 shu1}{10,4}{⾔、⼄}[HSK 1]
  \definition{v.+compl.}{ler; estudar | frequentar a escola}
\end{EntryWithPhonetic}

\begin{EntryWithPhonetic}{读音}{du2 yin1}{10,9}{⾔、⾳}[HSK 2]
  \definition[种]{s.}{pronúncia}
\end{EntryWithPhonetic}

\begin{EntryWithPhonetic}{读者}{du2 zhe3}{10,8}{⾔、⽼}[HSK 3]
  \definition[个,位,名,些,群]{s.}{leitor; (para obras, autores, revistas, etc.) Pessoas que compram ou leem livros, revistas, artigos, jornais, etc.}
\end{EntryWithPhonetic}

\begin{EntryWithPhonetic}{肚}{du3}{7}{⾁}
  \definition{s.}{tripas; entranhas}
  \seeref{肚}{du4}
\end{EntryWithPhonetic}

\begin{EntryWithPhonetic}{堵}{du3}{11}{⼟}[HSK 4]
  \definition*{s.}{Sobrenome Du}
  \definition{adj.}{asfixiado; abafado; sufocado; oprimido}
  \definition{clas.}{usado para paredes}
  \definition{s.}{parede}
  \definition{v.}{impedir; bloquear}
\end{EntryWithPhonetic}

\begin{EntryWithPhonetic}{堵车}{du3che1}{11,4}{⼟、⾞}[HSK 4]
  \definition{v.}{congestionar (trânsito)}
  \definition{v.+compl.}{congestionamento; tráfego intenso; ficar congestionado (no tráfego); bloqueio de vias devido ao excesso de tráfego, etc.}
\end{EntryWithPhonetic}

\begin{EntryWithPhonetic}{赌}{du3}{12}{⾙}[HSK 6]
  \definition{v.}{jogar | apostar; geralmente se refere à luta pela vitória ou derrota}
\end{EntryWithPhonetic}

\begin{EntryWithPhonetic}{赌博}{du3bo2}{12,12}{⾙、⼗}[HSK 6]
  \definition{v.}{apostar; jogar; usar jogos de cartas, rolagem de dados, etc., para apostar dinheiro}
\end{EntryWithPhonetic}

\begin{EntryWithPhonetic}{杜}{du4}{7}{⽊}
  \definition*{s.}{Sobrenome Du}
  \definition{s.}{pêra de folha de bétula}
  \definition{v.}{excluir; parar; impedir; bloquear}
\end{EntryWithPhonetic}

\begin{EntryWithPhonetic}{杜鹃}{du4juan1}{7,12}{⽊、⿃}
  \definition{s.}{cuco (pássaro)}
  \seealsoref{布谷鸟}{bu4gu3niao3}
  \seealsoref{杜鹃鸟}{du4juan1niao3}
  \seealsoref{杜宇}{du4yu3}
\end{EntryWithPhonetic}

\begin{EntryWithPhonetic}{杜鹃鸟}{du4juan1niao3}{7,12,5}{⽊、⿃、⿃}
  \definition{s.}{cuco (pássaro)}
  \seealsoref{布谷鸟}{bu4gu3niao3}
  \seealsoref{杜鹃}{du4juan1}
  \seealsoref{杜宇}{du4yu3}
\end{EntryWithPhonetic}

\begin{EntryWithPhonetic}{杜宇}{du4yu3}{7,6}{⽊、⼧}
  \definition{s.}{cuco (pássaro)}
  \seealsoref{布谷鸟}{bu4gu3niao3}
  \seealsoref{杜鹃}{du4juan1}
  \seealsoref{杜鹃鸟}{du4juan1niao3}
\end{EntryWithPhonetic}

\begin{EntryWithPhonetic}{肚}{du4}{7}{⾁}
  \definition{s.}{barriga; abdômen; estômago | tolerância}
  \seeref{肚}{du3}
\end{EntryWithPhonetic}

\begin{EntryWithPhonetic}{肚子}{du4zi5}{7,3}{⾁、⼦}[HSK 4]
  \definition[个,只]{s.}{abdômen; barriguinha; ventre; barriga}
\end{EntryWithPhonetic}

\begin{EntryWithPhonetic}{度}{du4}{9}{⼴}[HSK 2]
  \definition*{s.}{Sobrenome Du}
  \definition{clas.}{grau; unidade de medida para ângulos, temperatura, etc. | quilowatt-hora (kWh) | usado para indicar a quantidade de álcool presente no vinho | usado para arcos e ângulos | usado para indicar o grau de curvatura da lente dos óculos ou o grau de miopia | tempo; número de vezes | usado para longitude e latitude, localização geográfica}
  \definition{s.}{medida linear; padrões e instrumentos para medir comprimentos | grau de intensidade; refere-se especificamente ao grau alcançado por uma determinada propriedade de uma coisa | limite; extensão; grau; quota | regras; código de conduta; diretrizes | tolerância; magnanimidade; refere-se especificamente ao grau de tolerância | maneira; temperamento; disposição; a personalidade ou aparência de uma pessoa | indicador de grau, nível alcançado por algo | tempo ou espaço limitado; um determinado período de tempo ou espaço}
  \definition{v.}{passar; atravessar; passar por cima | (em termos de tempo) passar; passar por | (de monges ou monjas budistas, ou sacerdotes taoístas) pregar; converter; proselitar}
  \seeref{度}{duo2}
\end{EntryWithPhonetic}

\begin{EntryWithPhonetic}{度过}{du4guo4}{9,6}{⼴、⾡}[HSK 4]
  \definition{s.}{passar o tempo; fazer o tempo desaparecer no trabalho, na vida, no lazer e no descanso}
\end{EntryWithPhonetic}

\begin{EntryWithPhonetic}{渡}{du4}{12}{⽔}[HSK 6]
  \definition{s.}{(usualmente em nomes de lugares) travessia de balsa}
  \definition{v.}{atravessar (um rio, o mar, etc.) | superar; sobreviver | transportar (pessoas, mercadorias, etc.) através}
\end{EntryWithPhonetic}

\begin{EntryWithPhonetic}{渡过}{du4guo4}{12,6}{⽔、⾡}
  \definition{v.}{atravessar | passar por}
\end{EntryWithPhonetic}

\begin{EntryWithPhonetic}{镀}{du4}{14}{⾦}
  \definition{v.}{cobrir ou revestir (com um metal)}
\end{EntryWithPhonetic}

\begin{EntryWithPhonetic}{镀金}{du4jin1}{14,8}{⾦、⾦}
  \definition{v.}{banhar a ouro | dourar | (figurativo) fazer algo muito comum parecer especial}
\end{EntryWithPhonetic}

\begin{EntryWithPhonetic}{端}{duan1}{14}{⽴}[HSK 6]
  \definition*{s.}{Sobrenome Duan}
  \definition{adj.}{adequado; próprio | reto; correto}
  \definition{s.}{fim; extremidade | começo | item; ponto; pista, projeto ou aspecto | causa; razão | problema; incidente; coisas (geralmente se refere a coisas ruins, como acidentes, disputas, etc.)}
  \definition{v.}{carregar; segurar algo nivelado com ambas as mãos; segurar algo horizontalmente | erradicar; eliminar; acabar com; remover completamente; varrer | dar ares de superioridade | revelar}
\end{EntryWithPhonetic}

\begin{EntryWithPhonetic}{端午节}{duan1wu3jie2}{14,4,5}{⽴、⼗、⾋}[HSK 6]
  \definition*[个]{s.}{Festa do Duplo Cinco, Festival dos Barcos-Dragão (5º~dia do quinto mês lunar)}
\end{EntryWithPhonetic}

\begin{EntryWithPhonetic}{短}{duan3}{12}{⽮}[HSK 2]
  \definition{adj.}{curto; comprimento pequeno de uma extremidade à outra (em oposição a 长) | curto; breve; a distância entre o ponto inicial e o ponto final de um determinado período é pequena | raso; superficial}
  \definition{s.}{falha; defeito; ponto fraco; desvantagens | tonelada curta (EUA)}
  \definition{v.}{dever; carecer}
  \seealsoref{长}{zhang3}
\end{EntryWithPhonetic}

\begin{EntryWithPhonetic}{短处}{duan3 chu4}{12,5}{⽮、⼡}[HSK 3]
  \definition[个]{s.}{deficiência; ponto fraco; defeito; fraqueza}
\end{EntryWithPhonetic}

\begin{EntryWithPhonetic}{短促}{duan3cu4}{12,9}{⽮、⼈}
  \definition{adj.}{curto (tom de voz) | fugaz | ofegante (respiração) | curto no tempo}
\end{EntryWithPhonetic}

\begin{EntryWithPhonetic}{短裤}{duan3 ku4}{12,12}{⽮、⾐}[HSK 3]
  \definition[条]{s.}{calças curtas; calção; \emph{shorts}; calças com bainha acima do joelho}
\end{EntryWithPhonetic}

\begin{EntryWithPhonetic}{短跑}{duan3 pao3}{12,12}{⽮、⾜}
  \definition{s.}{corrida de curta distância; corrida rápida (oposto a 长跑)}
  \seealsoref{长跑}{chang2 pao3}
\end{EntryWithPhonetic}

\begin{EntryWithPhonetic}{短片}{duan3 pian4}{12,4}{⽮、⽚}[HSK 6]
  \definition{s.}{curta-metragem; curtas-metragens documentais ou educativos exibidos individualmente ou em série}
\end{EntryWithPhonetic}

\begin{EntryWithPhonetic}{短期}{duan3 qi1}{12,12}{⽮、⽉}[HSK 3]
  \definition{adj.}{de curta duração; de prazo curto}
  \definition[个]{s.}{curto prazo}
\end{EntryWithPhonetic}

\begin{EntryWithPhonetic}{短缺}{duan3que1}{12,10}{⽮、⽸}
  \definition{s.}{escassez}
\end{EntryWithPhonetic}

\begin{EntryWithPhonetic}{短少}{duan3shao3}{12,4}{⽮、⼩}
  \definition{v.}{estar aquém do valor total}
\end{EntryWithPhonetic}

\begin{EntryWithPhonetic}{短视}{duan3shi4}{12,8}{⽮、⾒}
  \definition{adj.}{míope}
\end{EntryWithPhonetic}

\begin{EntryWithPhonetic}{短信}{duan3xin4}{12,9}{⽮、⼈}[HSK 2]
  \definition[条,个,封]{s.}{mensagem de texto; refere-se especificamente a mensagens de texto curtas, imagens, etc., enviadas ou recebidas por celular}
\end{EntryWithPhonetic}

\begin{EntryWithPhonetic}{短暂}{duan3zan4}{12,12}{⽮、⽇}
  \definition{adj.}{momentâneo | de curta duração}
\end{EntryWithPhonetic}

\begin{EntryWithPhonetic}{段}{duan4}{9}{⽎}[HSK 2]
  \definition*{s.}{Sobrenome Duan}
  \definition{clas.}{parte; seção; segmento; usado para dividir objetos em várias partes | passagem; parágrafo; parte de algo que tem características de continuidade | seção; período; usado para uma certa distância no tempo ou no espaço}
  \definition{s.}{nível; dan (no judô, weiqi, etc.) | seção (como nível administrativo em uma mina ou fábrica) | parte; etapa; estágio}
  \definition{v.}{cortar; separar}
\end{EntryWithPhonetic}

\begin{EntryWithPhonetic}{断}{duan4}{11}{⽄}[HSK 3]
  \definition*{s.}{Sobrenome Duan}
  \definition{adv.}{(geralmente na forma negativa) absolutamente; decididamente}
  \definition{v.}{quebrar; partir; (objetos longos) dividir em segmentos não conectados | parar; interromper; romper; isolar; fazer com que não se sucedam mais | desistir; abster-se de; parar de fumar, beber, etc. | julgar; decidir | interceptar}
\end{EntryWithPhonetic}

\begin{EntryWithPhonetic}{断交}{duan4jiao1}{11,6}{⽄、⼇}
  \definition{v.+compl.}{terminar uma amizade | romper relações diplomáticas}
\end{EntryWithPhonetic}

\begin{EntryWithPhonetic}{锻}{duan4}{14}{⾦}
  \definition{v.}{forjar; moldar}
\end{EntryWithPhonetic}

\begin{EntryWithPhonetic}{锻炼}{duan4lian4}{14,9}{⾦、⽕}[HSK 4]
  \definition{v.}{exercitar-se; fazer (ou fazer) exercícios; submeter-se a treinamento físico; fortalecer o corpo por meio do esporte | fortalecer; endurecer; aprimorar as habilidades de trabalho e de vida por meio de trabalho e outras atividades | forjar ou moldar metal para torná-lo mais refinado; refere-se à transformação de materiais metálicos em objetos de determinada forma e tamanho por meio de aquecimento, batimento, prensagem etc.}
\end{EntryWithPhonetic}

\begin{EntryWithPhonetic}{堆}{dui1}{11}{⼟}[HSK 5]
  \definition{clas.}{amontoado; pilha; multidão; usado para pilhas de coisas}
  \definition{s.}{amontoado; pilha; empilhamento | (em nomes de lugares)  colina; monte| multidão de pessoas ou coisas}
  \definition{v.}{empilhar; amontoar; acumular; juntar; reunir}
\end{EntryWithPhonetic}

\begin{EntryWithPhonetic}{队}{dui4}{4}{⾩}[HSK 2]
  \definition*{s.}{Jovens Pioneiros, refere-se especificamente à Patrulha de Jovens Pioneiros}
  \definition[条,个,支]{s.}{fila de pessoas | equipe; grupo;}
\end{EntryWithPhonetic}

\begin{EntryWithPhonetic}{队伍}{dui4wu3}{4,6}{⾩、⼈}[HSK 6]
  \definition[条,行,列,支,个]{s.}{tropas; exército | fileiras; contingente | desfile}
\end{EntryWithPhonetic}

\begin{EntryWithPhonetic}{队友}{dui4you3}{4,4}{⾩、⼜}
  \definition{s.}{companheiro de equipe}
\end{EntryWithPhonetic}

\begin{EntryWithPhonetic}{队员}{dui4 yuan2}{4,7}{⾩、⼝}[HSK 3]
  \definition[名,位,个]{s.}{membro da equipe; a composição de uma equipe}
\end{EntryWithPhonetic}

\begin{EntryWithPhonetic}{队长}{dui4 zhang3}{4,4}{⾩、⾧}[HSK 2]
  \definition[个,位,名]{s.}{capitão (de equipe); capitães | líder de equipe}
\end{EntryWithPhonetic}

\begin{EntryWithPhonetic}{对}{dui4}{5}{⼨}[HSK 1,2]
  \definition{adj.}{certo; correto; em conformidade com determinados padrões | oposto; contrário}
  \definition{adv.}{mutuamente; cara a cara}
  \definition{clas.}{usado para pessoas ou coisas que formam pares; casais}
  \definition{prep.}{o que diz respeito a; relativo a; com relação a; introduz o objeto da ação}
  \definition[幅]{s.}{dístico; refere-se a um par de versos | par; parceiro; pessoas ou coisas que se complementam}
  \definition{v.}{responder; dar uma resposta | tratar; lidar com; combater | ser treinado para; ser direcionado para; enfrentar | colocar (duas coisas) em contato; encaixar uma na outra; combinar ou cooperar entre si | comparar; verificar; identificar; comparar e verificar se estão de acordo | definir; ajustar; ajustar para atender a determinados requisitos | misturar (refere-se principalmente a líquidos); adicionar | dividir ao meio; dividir em duas partes iguais | combinar; concordar; dar-se bem; harmonizar-se}
\end{EntryWithPhonetic}

\begin{EntryWithPhonetic}{对比}{dui4bi3}{5,4}{⼨、⽐}[HSK 4]
  \definition{s.}{razão; proporção | contraste; comparação; diferenças ou lacunas encontradas após comparação}
  \definition{v.}{contrastar; comparar}
\end{EntryWithPhonetic}

\begin{EntryWithPhonetic}{对不起}{dui4bu5qi3}{5,4,10}{⼨、⼀、⾛}[HSK 1]
  \definition{interj.}{Desculpe! | Desculpe-me! | Perdoe-me! | Com licença?}
  \definition{v.}{desculpar; pedir desculpas; perdoar}
\end{EntryWithPhonetic}

\begin{EntryWithPhonetic}{对待}{dui4dai4}{5,9}{⼨、⼻}[HSK 3]
  \definition{v.}{tratar; abordar; manusear; estar em uma posição relacionada ou comparada a outra; expressar uma certa atitude ou agir de determinada maneira em relação a pessoas ou coisas}
\end{EntryWithPhonetic}

\begin{EntryWithPhonetic}{对得起}{dui4de5qi3}{5,11,10}{⼨、⼻、⾛}
  \definition{v.}{não decepcionar alguém | tratar alguém de maneira justa | ser digno de}
\end{EntryWithPhonetic}

\begin{EntryWithPhonetic}{对方}{dui4fang1}{5,4}{⼨、⽅}[HSK 3]
  \definition{s.}{outro lado; lado oposto; outra parte; a parte contrária ao sujeito da ação ou outras pessoas envolvidas em um determinado evento ou situação}
\end{EntryWithPhonetic}

\begin{EntryWithPhonetic}{对付}{dui4fu5}{5,5}{⼨、⼈}[HSK 4]
  \definition{adj.}{em bons termos; estar em termos agradáveis ​​(frequentemente usado em negativas); dialeto usado para descrever duas pessoas que têm um bom relacionamento e se dão bem, frequentemente usado para negar}
  \definition{v.}{enfrentar; tratar; lidar com | fazer acontecer; (informal) fazer algo que você não quer fazer; aceitar algo que você não gosta}
\end{EntryWithPhonetic}

\begin{EntryWithPhonetic}{对……感兴趣}{dui4 gan3xing4qu4}{5,13,6,15}{⼨、⼼、⼋、⾛}
  \definition{expr.}{estar interessado em\dots; ter interesse em\dots; interessar-se por\dots}
  \seealsoref{对……有兴趣}{dui4 you3xing4qu4}
\end{EntryWithPhonetic}

\begin{EntryWithPhonetic}{对话}{dui4hua4}{5,8}{⼨、⾔}[HSK 2]
  \definition[段,番,个]{s.}{diálogo; conversa; refere-se especificamente a diálogos entre personagens em obras literárias, como peças de teatro e romances}
  \definition{v.}{conversar com; comunicar-se com | manter um diálogo; conversar uns com os outros}
\end{EntryWithPhonetic}

\begin{EntryWithPhonetic}{对抗}{dui4kang4}{5,7}{⼨、⼿}[HSK 6]
  \definition{v.}{antagonizar; confrontar | resistir; opor-se; contra-atacar}
\end{EntryWithPhonetic}

\begin{EntryWithPhonetic}{对立}{dui4li4}{5,5}{⼨、⽴}[HSK 5]
  \definition{v.}{opor-se; contrastar; filosoficamente, refere-se a duas coisas ou dois aspectos da mesma coisa que se contradizem, se excluem ou entram em conflito entre si | opor-se; ser antagônico a}
\end{EntryWithPhonetic}

\begin{EntryWithPhonetic}{对面}{dui4mian4}{5,9}{⼨、⾯}[HSK 2]
  \definition{adv.}{cara a cara}
  \definition[面]{s.}{lado oposto; o outro lado; os nomes dados às duas margens opostas de ruas, rios, etc. | bem na frente; diretamente à frente}
\end{EntryWithPhonetic}

\begin{EntryWithPhonetic}{对手}{dui4shou3}{5,4}{⼨、⼿}[HSK 3]
  \definition[个,名,位,对]{s.}{oponente; adversário na competição | igual; correspondente; refere-se especificamente ao adversário em uma competição em que as habilidades e o nível são praticamente iguais}
\end{EntryWithPhonetic}

\begin{EntryWithPhonetic}{对……熟悉}{dui4 shu2xi1}{5,15,11}{⼨、⽕、⼼}
  \definition{expr.}{estar familiarizado com\dots}
\end{EntryWithPhonetic}

\begin{EntryWithPhonetic}{对……说}{dui4 shuo5}{5,9}{⼨、⾔}
  \definition{v.}{dizer a alguém}
\end{EntryWithPhonetic}

\begin{EntryWithPhonetic}{对外}{dui4 wai4}{5,5}{⼨、⼣}[HSK 6]
  \definition{adj.}{externo; para fora | estrangeiro; no exterior}
\end{EntryWithPhonetic}

\begin{EntryWithPhonetic}{对象}{dui4xiang4}{5,11}{⼨、⾗}[HSK 3]
  \definition[个,位]{s.}{alvo; objeto; a pessoa ou coisa que serve como objetivo ao agir ou pensar | parceiro; namorado; namorada; refere-se especificamente à pessoa amada}
\end{EntryWithPhonetic}

\begin{EntryWithPhonetic}{对应}{dui4ying4}{5,7}{⼨、⼴}[HSK 5]
  \definition{adj.}{homólogo; correspondente}
  \definition{v.}{corresponder; ser equivalente a}
\end{EntryWithPhonetic}

\begin{EntryWithPhonetic}{对……有兴趣}{dui4 you3xing4qu4}{5,6,6,15}{⼨、⽉、⼋、⾛}
  \definition{expr.}{estar interessado em\dots; ter interesse em\dots; interessar-se por\dots}
  \seealsoref{对……感兴趣}{dui4 gan3xing4qu4}
\end{EntryWithPhonetic}

\begin{EntryWithPhonetic}{对于}{dui4yu2}{5,3}{⼨、⼆}[HSK 4]
  \definition{prep.}{para; relativo a; no que diz respeito a; a respeito de}
\end{EntryWithPhonetic}

\begin{EntryWithPhonetic}{吨}{dun1}{7}{⼝}[HSK 5]
  \definition{clas.}{tonelada}
\end{EntryWithPhonetic}

\begin{EntryWithPhonetic}{蹲}{dun1}{19}{⾜}[HSK 6]
  \definition{v.}{agachamento sobre os calcanhares; dobrar as pernas o máximo possível, como se estivesse sentado, mas não deixar as nádegas tocarem o chão | ficar; metáfora para ficar ocioso em casa}
\end{EntryWithPhonetic}

\begin{EntryWithPhonetic}{蹲下}{dun1xia4}{19,3}{⾜、⼀}
  \definition{v.}{agachar | agachar-se}
\end{EntryWithPhonetic}

\begin{EntryWithPhonetic}{钝}{dun4}{9}{⾦}
  \definition{adj.}{sem corte; opaco (oposto a 快, 利, 锐) | estúpido; sem noção | maçante}
  \seealsoref{快}{kuai4}
  \seealsoref{利}{li4}
  \seealsoref{锐}{rui4}
\end{EntryWithPhonetic}

\begin{EntryWithPhonetic}{顿}{dun4}{10}{⾴}[HSK 3]
  \definition*{s.}{Sobrenome Dun}
  \definition{adj.}{cansado; fatigado}
  \definition{adv.}{de repente; imediatamente; indica que o tempo é curto, equivalente a 立刻}
  \definition{clas.}{usado para refeições | usado para surras, repreensões, castigos físicos, etc.}
  \definition{s.}{um lugar para ficar; acomodação e alimentação}
  \definition{v.}{pausar; parar; fazer uma pausa | pausar na escrita para reforçar o início ou o fim de um traço; ao escrever com pincel, pressione o pincel com força e pare um pouco sobre o papel | tocar o chão (com a cabeça) | bater o pé); chutar o chão ou bater no chão com um objeto | resolver; arranjar | montar acampamento; ficar temporariamente; parar para se hospedar; acampar}
  \seealsoref{立刻}{li4ke4}
\end{EntryWithPhonetic}

\begin{EntryWithPhonetic}{多}{duo1}{6}{⼣}[HSK 1,2]
  \definition*{s.}{Sobrenome Duo}
  \definition{adj.}{grande quantidade (oposto de 少, 寡) | excessivo; desnecessário | excessivo; em demasia; indica um grande grau de diferença | mais do que o número correto ou necessário; em excesso}
  \definition{adv.}{acima de um valor especificado; e mais | em que medida; usado em frases interrogativas para indagar sobre grau ou quantidade, equivalente a 多么 | uma extensão não especificada; usado em frases exclamativas para expressar um alto grau, equivalente a 多么 | quase; significa que a maior parte do intervalo é assim | mais;  sobre; ímpar; usado depois de um quantificador para indicar uma fração}
  \definition{num.}{(após um número) ímpar}
  \definition{pref.}{multi- | poli-}
  \definition{v.}{ter (uma quantidade específica) a mais ou a mais (oposto a 少) | ter algo em abundância  | (em perguntas) até que ponto | (em exclamações) até que ponto | ter mais}
  \seealsoref{多么}{duo1me5}
  \seealsoref{寡}{gua3}
  \seealsoref{少}{shao3}
\end{EntryWithPhonetic}

\begin{EntryWithPhonetic}{多半}{duo1 ban4}{6,5}{⼣、⼗}[HSK 6]
  \definition{adv.}{geralmente; mais frequentemente do que não}
  \definition{num.}{a maioria; a maior parte; mais da metade}
\end{EntryWithPhonetic}

\begin{EntryWithPhonetic}{多重}{duo1chong2}{6,9}{⼣、⾥}
  \definition{pref.}{multi (facetado, cultural, étnico, etc.)}
\end{EntryWithPhonetic}

\begin{EntryWithPhonetic}{多次}{duo1 ci4}{6,6}{⼣、⽋}[HSK 4]
  \definition{adv.}{muitas vezes; de vez em quando; repetidamente; em muitas ocasiões}
\end{EntryWithPhonetic}

\begin{EntryWithPhonetic}{多大}{duo1da4}{6,3}{⼣、⼤}
  \definition{adj.}{quantos anos? | que idade? | quão grande?}
\end{EntryWithPhonetic}

\begin{EntryWithPhonetic}{多方面}{duo1 fang1 mian4}{6,4,9}{⼣、⽅、⾯}[HSK 6]
  \definition{adj.}{de muitas maneiras; todos os aspectos}
  \definition{s.}{multifacetado; multiaspecto}
\end{EntryWithPhonetic}

\begin{EntryWithPhonetic}{多久}{duo1 jiu3}{6,3}{⼣、⼃}[HSK 2]
  \definition{pron.}{quanto tempo?; quanto tempo; perguntar quanto tempo leva}
\end{EntryWithPhonetic}

\begin{EntryWithPhonetic}{多么}{duo1me5}{6,3}{⼣、⼃}[HSK 2]
  \definition{adv.}{(em exclamações) como; o quê; em que medida; usado em frases exclamativas, indica um grau muito alto | em grau indeterminado; usado em frases declarativas, indica um grau mais profundo | como (usado em uma frase interrogativa para perguntar sobre grau ou número)}
\end{EntryWithPhonetic}

\begin{EntryWithPhonetic}{多媒体}{duo1 mei2 ti3}{6,12,7}{⼣、⼥、⼈}[HSK 6]
  \definition{s.}{multimídia; uma combinação de múltiplas mídias}
\end{EntryWithPhonetic}

\begin{EntryWithPhonetic}{多年}{duo1 nian2}{6,6}{⼣、⼲}[HSK 4]
  \definition{adv.}{por muitos anos; durante muitos anos}
\end{EntryWithPhonetic}

\begin{EntryWithPhonetic}{多少}{duo1shao3}{6,4}{⼣、⼩}
  \definition{adv.}{um pouco; mais ou menos; até certo ponto}
  \definition{s.}{número; quantidade; volume}
  \seeref{多少}{duo1shao5}
\end{EntryWithPhonetic}

\begin{EntryWithPhonetic}{多少}{duo1shao5}{6,4}{⼣、⼩}[HSK 1]
  \definition{adv.}{quantos?; quanto?; usado em perguntas para perguntar sobre quantidade | expressar uma quantidade ou número não especificado; quantidade indefinida}
  \seeref{多少}{duo1shao3}
\end{EntryWithPhonetic}

\begin{EntryWithPhonetic}{多数}{duo1 shu4}{6,13}{⼣、⽁}[HSK 2]
  \definition{adj.}{maioria; a maioria; plural}
  \definition{pref.}{pluri-}
\end{EntryWithPhonetic}

\begin{EntryWithPhonetic}{多样}{duo1 yang4}{6,10}{⼣、⽊}[HSK 4]
  \definition{adj.}{diversos; variados; diversificado}
  \definition{s.}{diversidade}
\end{EntryWithPhonetic}

\begin{EntryWithPhonetic}{多云}{duo1 yun2}{6,4}{⼣、⼆}[HSK 2]
  \definition{adj.}{céu nublado; em meteorologia, refere-se a condições atmosféricas em que a cobertura de nuvens médias e baixas ocupa entre 40\% e 70\% da área do céu, ou a cobertura de nuvens altas ocupa entre 60\% e 100\% da área do céu}
\end{EntryWithPhonetic}

\begin{EntryWithPhonetic}{多咱}{duo1 zan5}{6,9}{⼣、⼝}
  \definition{adv.}{que horas?; quando?}
\end{EntryWithPhonetic}

\begin{EntryWithPhonetic}{多种}{duo1 zhong3}{6,9}{⼣、⽲}[HSK 4]
  \definition{adj.}{diverso; vários tipos de; múltiplo; diversificado}
  \definition{pref.}{multi-}
\end{EntryWithPhonetic}

\begin{EntryWithPhonetic}{夺}{duo2}{6}{⼤}[HSK 6]
  \definition{v.}{tomar à força; apreender; arrancar; roubar | forçar a passagem; empurrar para abrir | lutar por; competir por; esforçar-se por; obter primeiro | privar; perder | perder; tirar | decidir; tomar uma decisão | omitir (palavra em um texto)}
\end{EntryWithPhonetic}

\begin{EntryWithPhonetic}{夺冠}{duo2guan4}{6,9}{⼤、⼍}
  \definition{v.}{apoderar-se da coroa | (fig.) ganhar um campeonato | ganhar a medalha de ouro}
\end{EntryWithPhonetic}

\begin{EntryWithPhonetic}{夺取}{duo2 qu3}{6,8}{⼤、⼜}[HSK 6]
  \definition{v.}{capturar; apreender; arrancar; tomar à força | esforçar-se para; alcançar}
\end{EntryWithPhonetic}

\begin{EntryWithPhonetic}{度}{duo2}{9}{⼴}
  \definition{v.}{supor; estimar; especular}
  \seeref{度}{du4}
\end{EntryWithPhonetic}

\begin{EntryWithPhonetic}{朵}{duo3}{6}{⽊}[HSK 5]
  \definition*{s.}{Sobrenome Duo}
  \definition{clas.}{usado para flores, nuvens ou coisas que se assemelham a flores e nuvens}
\end{EntryWithPhonetic}

\begin{EntryWithPhonetic}{躲}{duo3}{13}{⾝}[HSK 5]
  \definition{v.}{esconder (a si mesmo); ocultar (a si mesmo); esconder-se | evitar; esquivar-se}
\end{EntryWithPhonetic}

\begin{EntryWithPhonetic}{躲闪}{duo3shan3}{13,5}{⾝、⾨}
  \definition{v.}{desviar | evadir | esquivar (para fora do caminho)}
\end{EntryWithPhonetic}

%%%%% EOF %%%%%

