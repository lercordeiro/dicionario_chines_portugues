%%%
%%% E
%%%

\section*{E}\addcontentsline{toc}{section}{E}

\begin{EntryWithPhonetic}{阿}{e1}{7}{⾩}
  \definition*{s.}{Dong'e, um condado na província de Shandong | Sobrenome E}
  \definition{s.}{grande monte (ou colina) | um lugar sinuoso (montanha, água, etc.)}
  \definition{v.}{bajular; satisfazer}
  \seeref{a1}
\end{EntryWithPhonetic}

\begin{EntryWithPhonetic}{讹}{e2}{6}{⾔}
  \definition{adj.}{errôneo; equivocado}
  \definition{v.}{extorquir sob falsos pretextos; chantagear; enganar}
\end{EntryWithPhonetic}

\begin{EntryWithPhonetic}{讹诈}{e2zha4}{6,7}{⾔、⾔}[HSK 7-9]
  \definition{v.}{extorquir sob falsos pretextos; chantagear; intimidar}
\end{EntryWithPhonetic}

\begin{EntryWithPhonetic}{俄}{e2}{9}{⼈}
  \definition*{s.}{Rússia, abreviação de 俄罗斯}
  \definition{adv.}{muito em breve; em breve; de repente}
  \seealsoref{俄罗斯}{e2luo2si1}
\end{EntryWithPhonetic}

\begin{EntryWithPhonetic}{俄罗斯}{e2luo2si1}{9,8,12}{⼈、⽹、⽄}
  \definition*{s.}{Rússia}
\end{EntryWithPhonetic}

\begin{EntryWithPhonetic}{俄罗斯人}{e2luo2si1ren2}{9,8,12,2}{⼈、⽹、⽄、⼈}
  \definition{s.}{russo | pessoa ou povo da Rússia}
\end{EntryWithPhonetic}

\begin{EntryWithPhonetic}{俄语}{e2yu3}{9,9}{⼈、⾔}[HSK 7-9]
  \definition{s.}{russo; língua russa}
\end{EntryWithPhonetic}

\begin{EntryWithPhonetic}{哦}{e2}{10}{⼝}
  \definition{v.}{cantar suavemente (um poema)}
  \seeref{o2}
  \seeref{o4}
  \seeref{o5}
\end{EntryWithPhonetic}

\begin{EntryWithPhonetic}{鹅}{e2}{12}{⿃}[HSK 7-9]
  \definition[只,群]{s.}{ganso}
\end{EntryWithPhonetic}

\begin{EntryWithPhonetic}{额}{e2}{15}{⾴}
  \definition*{s.}{Sobrenome E}
  \definition[块]{s.}{testa; a área abaixo do cabelo e acima das sobrancelhas em humanos; a área aproximadamente equivalente na cabeça de alguns animais | uma tábua horizontal; placa horizontal inscrita; uma placa pendurada no lintel de uma porta ou na parede | um número específico (ou quantidade); limite superior de número; número limitado | a parte superior de algo}
\end{EntryWithPhonetic}

\begin{EntryWithPhonetic}{额外}{e2wai4}{15,5}{⾴、⼣}[HSK 7-9]
  \definition{adj.}{extra; adicional; excede a quantidade ou intervalo prescrito}
\end{EntryWithPhonetic}

\begin{EntryWithPhonetic}{恶}{e3}{10}{⼼}
  \definition{part.}{elementos formadores de palavras}
  \seeref{e4}
  \seeref{wu1}
  \seeref{wu4}
\end{EntryWithPhonetic}

\begin{EntryWithPhonetic}{恶心}{e3xin5}{10,4}{⼼、⼼}[HSK 4]
  \definition{adj.}{nauseante; repugnante}
  \definition{s.}{náusea; repugnância}
  \definition{v.}{repugnar; ser nauseante; sentir-se mal | envergonhar (deliberadamente)}
  \seeref{e4xin1}
\end{EntryWithPhonetic}

\begin{EntryWithPhonetic}{厄}{e4}{4}{⼚}
  \definition[个]{s.}{ponto estratégico; lugares perigosos | adversidade; desastre; dificuldade}
  \definition{v.}{estar em perigo; estar abandonado; estar; encurralado}
\end{EntryWithPhonetic}

\begin{EntryWithPhonetic}{厄运}{e4yun4}{4,7}{⼚、⾡}[HSK 7-9]
  \definition{s.}{adversidade; infortúnio; experiência infeliz}
\end{EntryWithPhonetic}

\begin{EntryWithPhonetic}{恶}{e4}{10}{⼼}[HSK 7-9]
  \definition{adj.}{feroz | ruim; maligno; perverso | vicioso | feio | grosseiro}
  \definition{s.}{mal; vício; crime (oposto a 善) | maldade; comportamento muito ruim; coisas criminosas}
  \seeref{e3}
  \seeref{wu1}
  \seeref{wu4}
  \seealsoref{善}{shan4}
\end{EntryWithPhonetic}

\begin{EntryWithPhonetic}{恶化}{e4hua4}{10,4}{⼼、⼔}[HSK 7-9]
  \definition{v.}{piorar; deteriorar; exacerbar | piorar a situação}
\end{EntryWithPhonetic}

\begin{EntryWithPhonetic}{恶劣}{e4lie4}{10,6}{⼼、⼒}[HSK 7-9]
  \definition{adj.}{mau; odioso; abominável; repugnante; desprezível; muito mau; muito ruim}
\end{EntryWithPhonetic}

\begin{EntryWithPhonetic}{恶心}{e4xin1}{10,4}{⼼、⼼}
  \definition{s.}{mau hábito; hábito vicioso; vício}
  \seeref{e3xin5}
\end{EntryWithPhonetic}

\begin{EntryWithPhonetic}{恶性}{e4xing4}{10,8}{⼼、⼼}[HSK 7-9]
  \definition{adj.}{maligno; pernicioso; vicioso (oposto a 良性) | produzindo o mal | rápido (declínio) | descontrolada (inflação) | vicioso (círculo) | perverso}
  \seealsoref{良性}{liang2xing4}
\end{EntryWithPhonetic}

\begin{EntryWithPhonetic}{恶意}{e4yi4}{10,13}{⼼、⼼}[HSK 7-9]
  \definition[丝]{s.}{malícia; má vontade; más intenções}
\end{EntryWithPhonetic}

\begin{EntryWithPhonetic}{饿}{e4}{10}{⾷}[HSK 1]
  \definition{adj.}{faminto}
  \definition{v.}{passar fome; causar fome}
\end{EntryWithPhonetic}

\begin{EntryWithPhonetic}{遏}{e4}{12}{⾡}
  \definition{v.}{reprimir; restringir; reter; impedir; proibir}
\end{EntryWithPhonetic}

\begin{EntryWithPhonetic}{遏制}{e4zhi4}{12,8}{⾡、⼑}[HSK 7-9]
  \definition{v.}{conter; restringir; controlar e prevenir ativamente o desenvolvimento de coisas que possam trazer perigo; usado principalmente para discutir tópicos formais}
\end{EntryWithPhonetic}

\begin{EntryWithPhonetic}{鳄}{e4}{17}{⿂}
  \definition{s.}{crocodilo;  jacaré}
\end{EntryWithPhonetic}

\begin{EntryWithPhonetic}{鳄鱼}{e4yu2}{17,8}{⿂、⿂}[HSK 7-9]
  \definition[只,群,些]{s.}{crocodilo; jacaré}
\end{EntryWithPhonetic}

\begin{EntryWithPhonetic}{恩}{en1}{10}{⼼}
  \definition*{s.}{Sobrenome En}
  \definition{s.}{bondade; favor; graça; gentileza}
\end{EntryWithPhonetic}

\begin{EntryWithPhonetic}{恩赐}{en1ci4}{10,12}{⼼、⾙}[HSK 7-9]
  \definition{s.}{favor; caridade; esmola}
  \definition{v.}{conceder (favores, caridade, etc.); recompensar}
\end{EntryWithPhonetic}

\begin{EntryWithPhonetic}{恩惠}{en1hui4}{10,12}{⼼、⼼}[HSK 7-9]
  \definition[份]{s.}{favor; generosidade | bondade; graça; benefícios concedidos ou recebidos}
\end{EntryWithPhonetic}

\begin{EntryWithPhonetic}{恩情}{en1qing2}{10,11}{⼼、⼼}[HSK 7-9]
  \definition{s.}{amor; bondade; afeição profunda}
\end{EntryWithPhonetic}

\begin{EntryWithPhonetic}{恩人}{en1 ren2}{10,2}{⼼、⼈}[HSK 6]
  \definition{s.}{benfeitor; uma pessoa que ajudou significativamente alguém}
\end{EntryWithPhonetic}

\begin{EntryWithPhonetic}{恩怨}{en1yuan4}{10,9}{⼼、⼼}[HSK 7-9]
  \definition{s.}{sentimento de gratidão ou ressentimento (inimizade) | ressentimento; queixa; velhas contas}
\end{EntryWithPhonetic}

\begin{EntryWithPhonetic}{儿}{er2}{2}{⼉}[Kangxi 10]
  \definition{adj.}{macho}
  \definition{s.}{criança | jovem; juventude | filho}
  \definition{suf.}{adicionado a substantivos para expressar pequenez  | adicionado a verbos, adjetivos e classificadores para formar substantivos | adicionado a substantivos para formar substantivos com significados diferentes | sufixos de alguns verbos | anexado após adjetivos duplicados}
  \seeref{r5}
\end{EntryWithPhonetic}

\begin{EntryWithPhonetic}{儿科}{er2 ke1}{2,9}{⼉、⽲}[HSK 6]
  \definition{s.}{(departamento de) pediatria | pediatria; o ramo da medicina que trata do desenvolvimento, cuidado e doença das crianças}
\end{EntryWithPhonetic}

\begin{EntryWithPhonetic}{儿女}{er2 nv3}{2,3}{⼉、⼥}[HSK 5]
  \definition{s.}{crianças; filhos e filhas | homem e mulher jovens (apaixonados)}
\end{EntryWithPhonetic}

\begin{EntryWithPhonetic}{儿童}{er2tong2}{2,12}{⼉、⽴}[HSK 4]
  \definition[个,群]{s.}{criança; menor de idade (mais jovem do que 少年)}
  \seealsoref{少年}{shao4 nian2}
\end{EntryWithPhonetic}

\begin{EntryWithPhonetic}{儿媳}{er2xi2}{2,13}{⼉、⼥}
  \definition{s.}{esposa do filho}
\end{EntryWithPhonetic}

\begin{EntryWithPhonetic}{儿子}{er2zi5}{2,3}{⼉、⼦}[HSK 1]
  \definition[个]{s.}{filho}
  \seealsoref{女儿}{nv3'er2}
\end{EntryWithPhonetic}

\begin{EntryWithPhonetic}{而}{er2}{6}{⽽}[HSK 4][Kangxi 126]
  \definition{conj.}{e (coordenação) | e ainda (restrição) | conexão de componentes com continuidade semântica | conecxão de componentes afirmativos e negativos que se complementam | conexão de componentes com significados opostos para indicar um contraste |  conexão de componentes de causa e efeito no raciocínio | significa “chegar” ou “alcançar” | conexão de componentes que indicam tempo ou modo ao verbo | inserido entre o sujeito e o predicado, significa 如果}
  \seealsoref{如果}{ru2guo3}
\end{EntryWithPhonetic}

\begin{EntryWithPhonetic}{而况}{er2kuang4}{6,7}{⽽、⼎}
  \definition{conj.}{além disso | além do mais}
\end{EntryWithPhonetic}

\begin{EntryWithPhonetic}{而且}{er2 qie3}{6,5}{⽽、⼀}[HSK 2]
  \definition{conj.}{e também; indica igualdade | e isso; não só\dots mas (também); indica um passo adiante}
\end{EntryWithPhonetic}

\begin{EntryWithPhonetic}{而是}{er2 shi4}{6,9}{⽽、⽇}[HSK 4]
  \definition{conj.}{mas; em vez disso; geralmente usada em conjunto com 不是 para formar o correlativo 不是……而是, indicando uma relação paralela}
  \seealsoref{不是……而是}{bu4shi4 er2 shi4}
\end{EntryWithPhonetic}

\begin{EntryWithPhonetic}{而已}{er2yi3}{6,3}{⽽、⼰}[HSK 7-9]
  \definition{part.}{isso é tudo; nada mais; usado no final de uma frase declarativa, geralmente é precedido por 不过 ou 只 para expressar que é exatamente assim (罢了)}
  \seealsoref{罢了}{ba4 le5}
  \seealsoref{不过}{bu2guo4}
  \seealsoref{只}{zhi3}
\end{EntryWithPhonetic}

\begin{EntryWithPhonetic}{耳}{er3}{6}{⽿}[Kangxi 128]
  \definition*{s.}{Sobrenome Er}
  \definition{part.}{(clássico) somente; apenas}
  \definition{s.}{orelha | coisa parecida com uma orelha | em ambos os lados; lado | orelha de um utensílio}
\end{EntryWithPhonetic}

\begin{EntryWithPhonetic}{耳朵}{er3duo5}{6,6}{⽿、⽊}[HSK 5]
  \definition[双,只,个,对]{s.}{orelha; ouvido; órgão da audição e do equilíbrio}
\end{EntryWithPhonetic}

\begin{EntryWithPhonetic}{耳光}{er3guang1}{6,6}{⽿、⼉}[HSK 7-9]
  \definition[个,记]{s.}{uma bofetada na orelha; um tapa na cara; (bater) no rosto em frente à orelha; a ação de bater no rosto}
\end{EntryWithPhonetic}

\begin{EntryWithPhonetic}{耳机}{er3 ji1}{6,6}{⽿、⽊}[HSK 4]
  \definition[副,个,对]{s.}{fone de ouvido; receptor (de telefone); dispositivos que permitem que uma pessoa ouça sons sozinha, como ouvir música, histórias, chamadas telefônicas etc., usados na cabeça ou inseridos nos ouvidos}
\end{EntryWithPhonetic}

\begin{EntryWithPhonetic}{耳目一新}{er3mu4-yi4xin1}{6,5,1,13}{⽿、⽬、⼀、⽄}[HSK 7-9]
  \definition{expr.}{encontrar tudo fresco e novo; encontrar-se em um mundo inteiramente novo; apresentar uma nova aparência (de um lugar); uma mudança agradável de atmosfera; ``Tudo o que ouço e vejo mudou e parece novo.''}
\end{EntryWithPhonetic}

\begin{EntryWithPhonetic}{耳熟能详}{er3shu2-neng2xiang2}{6,15,10,8}{⽿、⽕、⾁、⾔}[HSK 7-9]
  \definition{expr.}{o que é ouvido com frequência pode ser repetido em detalhes; já ouvi isso muitas vezes e estou familiarizado o suficiente para falar sobre isso em detalhes}
\end{EntryWithPhonetic}

\begin{EntryWithPhonetic}{耳闻目睹}{er3wen2-mu4du3}{6,9,5,13}{⽿、⾨、⽬、⽬}[HSK 7-9]
  \definition{expr.}{testemunhar pessoalmente; ver e ouvir pessoalmente; o que se vê e se ouve}
\end{EntryWithPhonetic}

\begin{EntryWithPhonetic}{二}{er4}{2}{⼆}[HSK 1][Kangxi 7]
  \definition{adj.}{diferente; refere-se a duas coisas ou coisas diferentes | bobo; pateta; tolo; sem inteligência | desleal; infiel; indiferente; sem determinação}
  \definition{num.}{dois; 2}
\end{EntryWithPhonetic}

\begin{EntryWithPhonetic}{二胡}{er4hu2}{2,9}{⼆、⾁}
  \definition{s.}{erhu; um instrumento de arco de duas cordas com um registro mais baixo que o 京胡; um tipo de 胡琴, a caixa de som é feita de bambu, madeira, etc., coberta com pele de cobra, etc., tem duas cordas e o tom é baixo e suave}
  \seealsoref{胡琴}{hu2qin2}
  \seealsoref{京胡}{jing1hu2}
\end{EntryWithPhonetic}

\begin{EntryWithPhonetic}{二手}{er4 shou3}{2,4}{⼆、⼿}[HSK 4]
  \definition{adj.}{usado; de segunda mão; refere-se especificamente a usados e revendidos}
\end{EntryWithPhonetic}

\begin{EntryWithPhonetic}{二手车}{er4shou3che1}{2,4,4}{⼆、⼿、⾞}[HSK 7-9]
  \definition{s.}{carro usado; carro de segunda mão}
\end{EntryWithPhonetic}

\begin{EntryWithPhonetic}{二维码}{er4 wei2 ma3}{2,11,8}{⼆、⽷、⽯}[HSK 5]
  \definition[个]{s.}{\emph{QR code}; um identificador gráfico que distribui formas geométricas específicas em um plano ou direção bidimensional de acordo com certas regras para expressar um conjunto de informações}
\end{EntryWithPhonetic}

\begin{EntryWithPhonetic}{二氧化碳}{er4yang3hua4tan4}{2,10,4,14}{⼆、⽓、⼔、⽯}[HSK 7-9]
  \definition{s.}{CO$_2$; dióxido de carbono; gás carbônico}
\end{EntryWithPhonetic}

\begin{EntryWithPhonetic}{二战}{er4zhan4}{2,9}{⼆、⼽}
  \definition*{s.}{Segunda Guerra Mundial}
\end{EntryWithPhonetic}

%%%%% EOF %%%%%

