%%%
%%% M
%%%

\section*{M}\addcontentsline{toc}{section}{M}

\begin{EntryWithPhonetic}{妈}{ma1}{6}{⼥}[HSK 1]
  \definition[个,位]{s.}{mãe; mamãe | uma forma de tratamento para uma mulher casada uma geração mais velha | (antigo) uma forma de tratamento para uma empregada doméstica de meia-idade ou velha}
  \seealsoref{妈妈}{ma1 ma5}
\end{EntryWithPhonetic}

\begin{EntryWithPhonetic}{妈妈}{ma1 ma5}{6,6}{⼥、⼥}[HSK 1]
  \definition[个,位]{s.}{mamãe; mãe | uma forma de chamar uma mulher de meia-idade; títulos de respeito para mulheres mais velhas}
\end{EntryWithPhonetic}

\begin{EntryWithPhonetic}{抹}{ma1}{8}{⼿}
  \definition{v.}{esfregar; limpar | deslizar algo para fora; tirar}
  \seeref{抹}{mo3}
  \seeref{抹}{mo4}
\end{EntryWithPhonetic}

\begin{EntryWithPhonetic}{蚂}{ma1}{9}{⾍}
  \definition{part.}{caracter formador de palavras}
  \definition[只]{s.}{libélula}
  \seeref{蚂}{ma3}
  \seeref{蚂}{ma4}
\end{EntryWithPhonetic}

\begin{EntryWithPhonetic}{吗}{ma2}{6}{⼝}
  \definition{adv.}{(coloquial) que?}
  \seeref{吗}{ma3}
  \seeref{吗}{ma5}
\end{EntryWithPhonetic}

\begin{EntryWithPhonetic}{麻}{ma2}{11}{⿇}[Kangxi 200]
  \definition*{s.}{Sobrenome Ma}
  \definition{adj.}{áspero; grosseiro | marcado; manchado | espinhas; manchas ásperas; cicatrizes deixadas após a varíola}
  \definition[棵,株]{s.}{nome geral para cânhamo, linho, etc. | fibra de cânhamo, linho, etc. para têxteis | sésamo; gergelim | marcas de varíola; um rosto com marcas de varíola}
  \definition{v.}{anestesiar | corromper (a mente de alguém); envenenar}
\end{EntryWithPhonetic}

\begin{EntryWithPhonetic}{麻烦}{ma2fan5}{11,10}{⿇、⽕}[HSK 3]
  \definition{adj.}{incômodo; inconveniente; complicado; trabalhoso; burocrático | incômodo; inconveniente; (a situação) é confusa e complicada}
  \definition[个,些,点,堆]{s.}{problema; inconveniência; assuntos complicados e difíceis de resolver}
  \definition{v.}{incomodar; perturbar; incomodar alguém; irritar; aborrecer; causar incômodo ou sobrecarregar outras pessoas}
\end{EntryWithPhonetic}

\begin{EntryWithPhonetic}{麻将}{ma2jiang4}{11,9}{⿇、⼨}
  \definition*[副]{s.}{Mahjong}
\end{EntryWithPhonetic}

\begin{EntryWithPhonetic}{麻辣豆腐}{ma2la4 dou4fu5}{11,14,7,14}{⿇、⾟、⾖、⾁}
  \definition{s.}{tofú guisado em molho picante (prato)}
\end{EntryWithPhonetic}

\begin{EntryWithPhonetic}{马}{ma3}{3}{⾺}[HSK 3][Kangxi 187]
  \definition*{s.}{Sobrenome Ma}
  \definition{adj.}{grande; extenso; amplo}
  \definition[匹,头,只,群]{s.}{cavalo | a peça do cavalo no xadrez chinês}
\end{EntryWithPhonetic}

\begin{EntryWithPhonetic}{马车}{ma3 che1}{3,4}{⾺、⾞}[HSK 6]
  \definition[辆]{s.}{carruagem (puxada por cavalo); carroça; charrete}
\end{EntryWithPhonetic}

\begin{EntryWithPhonetic}{马耳他}{ma3'er3ta1}{3,6,5}{⾺、⽿、⼈}
  \definition*{s.}{Malta}
\end{EntryWithPhonetic}

\begin{EntryWithPhonetic}{马克思列宁主义}{ma3ke4si1 lie4ning2 zhu3yi4}{3,7,9,6,5,5,3}{⾺、⼗、⼼、⼑、⼧、⼂、⼂}
  \definition*{s.}{Marxismo-Leninismo}
\end{EntryWithPhonetic}

\begin{EntryWithPhonetic}{马路}{ma3lu4}{3,13}{⾺、⾜}[HSK 1]
  \definition[条]{s.}{estrada; rua; avenida; estradas largas e planas para o tráfego de carros e cavalos nas cidades ou nos subúrbios}
\end{EntryWithPhonetic}

\begin{EntryWithPhonetic}{马马虎虎}{ma3ma3hu3hu3}{3,3,8,8}{⾺、⾺、⾌、⾌}
  \definition{adj.}{descuidado | casual | tolerável | vago | mais ou menos}
\end{EntryWithPhonetic}

\begin{EntryWithPhonetic}{马上}{ma3shang4}{3,3}{⾺、⼀}[HSK 1]
  \definition{adv.}{imediatamente; de uma só vez; em um piscar de olhos | em breve; em um futuro próximo; em um curto espaço de tempo}
\end{EntryWithPhonetic}

\begin{EntryWithPhonetic}{马尾}{ma3wei3}{3,7}{⾺、⼫}
  \definition{s.}{(penteado) rabo de cavalo | cauda de cavalo}
\end{EntryWithPhonetic}

\begin{EntryWithPhonetic}{吗}{ma3}{6}{⼝}
  \definition{s.}{usada em 吗啡, morfina}
  \seealsoref{吗啡}{ma3fei1}
\end{EntryWithPhonetic}

\begin{EntryWithPhonetic}{吗啡}{ma3fei1}{6,11}{⼝、⼝}
  \definition{s.}{morfina (empréstimo linguístico)}
\end{EntryWithPhonetic}

\begin{EntryWithPhonetic}{码}{ma3}{8}{⽯}
  \definition{clas.}{refere-se a um assunto específico ou a uma categoria de assuntos; refere-se a uma coisa ou a uma classe de coisas | jarda; unidade de comprimento britânica e americana}
  \definition{s.}{um sinal ou objeto que indica número; código; símbolos ou ferramentas que indicam números}
  \definition{v.}{empilhar; acumular}
\end{EntryWithPhonetic}

\begin{EntryWithPhonetic}{码头}{ma3tou2}{8,5}{⽯、⼤}[HSK 5]
  \definition[个,座]{s.}{doca; cais; píer; molhe; edifícios à beira-mar ou à beira do rio destinados exclusivamente à atracação de embarcações, embarque e desembarque de passageiros e carga e descarga de mercadorias | cidade portuária; centro comercial e de transportes; refere-se a uma cidade comercial com transporte terrestre e marítimo bem desenvolvido.}
\end{EntryWithPhonetic}

\begin{EntryWithPhonetic}{蚂}{ma3}{9}{⾍}
  \definition{part.}{caracter formador de palavras}
  \seeref{蚂}{ma1}
  \seeref{蚂}{ma4}
\end{EntryWithPhonetic}

\begin{EntryWithPhonetic}{蚂蚁}{ma3yi3}{9,9}{⾍、⾍}
  \definition{s.}{formiga}
\end{EntryWithPhonetic}

\begin{EntryWithPhonetic}{蚂}{ma4}{9}{⾍}
  \definition{part.}{caracter formador de palavras}
  \seeref{蚂}{ma1}
  \seeref{蚂}{ma3}
\end{EntryWithPhonetic}

\begin{EntryWithPhonetic}{骂}{ma4}{9}{⾺}[HSK 5]
  \definition{v.}{abusar; xingar; insultar; insultar alguém com palavras grosseiras ou maliciosas | repreender; censurar; condenar}
\end{EntryWithPhonetic}

\begin{EntryWithPhonetic}{骂街}{ma4jie1}{9,12}{⾺、⾏}
  \definition{v.}{gritar abusos na rua}
\end{EntryWithPhonetic}

\begin{EntryWithPhonetic}{骂名}{ma4ming2}{9,6}{⾺、⼝}
  \definition{s.}{infâmia}
\end{EntryWithPhonetic}

\begin{EntryWithPhonetic}{吗}{ma5}{6}{⼝}[HSK 1]
  \definition{part.}{usado no final de uma pergunta | como uma pausa em uma frase antes de introduzir o ponto principal | usado no final de uma pergunta retórica}
  \seeref{吗}{ma2}
  \seeref{吗}{ma3}
\end{EntryWithPhonetic}

\begin{EntryWithPhonetic}{嘛}{ma5}{14}{⼝}[HSK 6]
  \definition{part.}{usado no final de uma declaração para expressar que é claro que é verdade que é óbvio | usado no final de uma frase imperativa para expressar expectativa ou dissuasão | usado em uma frase para indicar uma pausa e chamar a atenção da outra pessoa}
\end{EntryWithPhonetic}

\begin{EntryWithPhonetic}{埋}{mai2}{10}{⼟}[HSK 6]
  \definition{v.}{cobrir (com terra, neve, etc.); enterrar | esconder | enterrar (uma pessoa morta)}
  \seeref{埋}{man2}
\end{EntryWithPhonetic}

\begin{EntryWithPhonetic}{埋伏}{mai2fu2}{10,6}{⼟、⼈}
  \definition{s.}{emboscada}
  \definition{v.}{emboscar}
\end{EntryWithPhonetic}

\begin{EntryWithPhonetic}{买}{mai3}{6}{⼄}[HSK 1]
  \definition*{s.}{Sobrenome Mai}
  \definition{v.}{comprar; adquirir | comprar; subornar; usar dinheiro ou outros meios para angariar apoio| pedir; obter; trocar dinheiro por coisas}
\end{EntryWithPhonetic}

\begin{EntryWithPhonetic}{买东西}{mai3 dong1xi5}{6,5,6}{⼄、⼀、⾑}
  \definition{v.}{fazer compras; comprar bens ou serviços}
\end{EntryWithPhonetic}

\begin{EntryWithPhonetic}{买卖}{mai3 mai4}{6,8}{⼄、⼗}[HSK 5]
  \definition[笔,桩,宗,家]{s.}{negócio; compra e venda; transação | Privado: loja; armazém}
\end{EntryWithPhonetic}

\begin{EntryWithPhonetic}{麦}{mai4}{7}{⿆}[Kangxi 199]
  \definition*{s.}{Sobrenome Mai}
  \definition[袋,筐,车]{s.}{um termo geral para trigo, cevada, etc.}
\end{EntryWithPhonetic}

\begin{EntryWithPhonetic}{麦当劳}{mai4dang1lao2}{7,6,7}{⿆、⼹、⼒}
  \definition*{s.}{McDonald's, restaurante de \emph{fast-food}}
  \seealsoref{麦当劳叔叔}{mai4dang1lao2 shu1shu5}
\end{EntryWithPhonetic}

\begin{EntryWithPhonetic}{麦当劳叔叔}{mai4dang1lao2 shu1shu5}{7,6,7,8,8}{⿆、⼹、⼒、⼜、⼜}
  \definition*{s.}{Ronald McDonald}
  \seealsoref{麦当劳}{mai4dang1lao2}
\end{EntryWithPhonetic}

\begin{EntryWithPhonetic}{麦淇淋}{mai4qi2lin2}{7,11,11}{⿆、⽔、⽔}
  \definition{s.}{Empréstimo linguístico: margarina}
\end{EntryWithPhonetic}

\begin{EntryWithPhonetic}{卖}{mai4}{8}{⼗}[HSK 2]
  \definition*{s.}{Sobrenome Mai}
  \definition{clas.}{um prato (nos tempos antigos); antigamente, os restaurantes chamavam cada prato vendido de 一卖 (uma porção)}
  \definition{v.}{vender (oposto de 买) | trair (o próprio país ou amigos); alcançar objetivos pessoais à custa dos interesses do país, da nação e dos outros | não poupar esforços; esforçar-se ao máximo; tentar fazer o máximo possível | mostrar-se intencionalmente; exibir-se | vender o próprio trabalho; trabalhar em troca de dinheiro}
  \seealsoref{买}{mai3}
\end{EntryWithPhonetic}

\begin{EntryWithPhonetic}{埋}{man2}{10}{⼟}
  \definition{part.}{caracter formador de palavras}
  \seeref{埋}{mai2}
\end{EntryWithPhonetic}

\begin{EntryWithPhonetic}{谩}{man2}{13}{⾔}
  \definition{v.}{enganar; ludibriar; iludir}
  \seeref{谩}{man4}
\end{EntryWithPhonetic}

\begin{EntryWithPhonetic}{蔓}{man2}{14}{⾋}
  \definition{s.}{couve-chinesa | nabo}
  \seeref{蔓}{man4}
  \seeref{蔓}{wan4}
\end{EntryWithPhonetic}

\begin{EntryWithPhonetic}{馒}{man2}{14}{⾷}
  \definition{s.}{pão cozido no vapor}
\end{EntryWithPhonetic}

\begin{EntryWithPhonetic}{馒头}{man2tou5}{14,5}{⾷、⼤}[HSK 6]
  \definition[个,锅,屉,筐]{s.}{pão cozido no vapor; um alimento cozido no vapor feito de farinha fermentada, geralmente redondo na parte superior e plano na parte inferior, sem recheio}
\end{EntryWithPhonetic}

\begin{EntryWithPhonetic}{满}{man3}{13}{⽔}[HSK 2]
  \definition*{s.}{Etnia Manchu | Sobrenome Man}
  \definition{adj.}{cheio; repleto; lotado; totalmente cheio; atingindo o limite da capacidade | tudo; inteiro; completo | presunçoso; complacente; orgulhoso}
  \definition{adv.}{muito; um tanto; bastante | completamente; inteiramente; perfeitamente}
  \definition{v.}{encher | sentir-se satisfeito; sentir que já é o suficiente | expirar; atingir o limite; atingir um determinado prazo ou limite}
\end{EntryWithPhonetic}

\begin{EntryWithPhonetic}{满分}{man3fen1}{13,4}{⽔、⼑}
  \definition{s.}{pontuação completa}
\end{EntryWithPhonetic}

\begin{EntryWithPhonetic}{满满}{man3man3}{13,13}{⽔、⽔}
  \definition{adj.}{completo | densamente empacotado}
\end{EntryWithPhonetic}

\begin{EntryWithPhonetic}{满意}{man3yi4}{13,13}{⽔、⼼}[HSK 2]
  \definition{adj.}{satisfeito; contente; gratificado}
  \definition{v.}{estar satisfeito; sentir-se contente; satisfazer os seus desejos; estar de acordo com os seus desejos}
\end{EntryWithPhonetic}

\begin{EntryWithPhonetic}{满足}{man3zu2}{13,7}{⽔、⾜}[HSK 3]
  \definition{v.}{estar satisfeito; contentar-se; sentir-se satisfeito | satisfazer}
\end{EntryWithPhonetic}

\begin{EntryWithPhonetic}{谩}{man4}{13}{⾔}
  \definition{v.}{ser desrespeitoso | caluniar | desconsiderar}
  \seeref{谩}{man2}
\end{EntryWithPhonetic}

\begin{EntryWithPhonetic}{谩骂}{man4ma4}{13,9}{⾔、⾺}
  \definition{v.}{ridicularizar | abusar}
\end{EntryWithPhonetic}

\begin{EntryWithPhonetic}{慢}{man4}{14}{⼼}[HSK 1]
  \definition*{s.}{Sobrenome Man}
  \definition{adj.}{lento; devagar; baixa velocidade; longa duração (em oposição a 快) | rude; arrogante; sem educação com as pessoas | frouxo; lento}
  \definition{adv.}{lentamente}
  \seealsoref{快}{kuai4}
\end{EntryWithPhonetic}

\begin{EntryWithPhonetic}{慢车}{man4 che1}{14,4}{⼼、⾞}[HSK 6]
  \definition{s.}{trem lento com muitas paradas (oposto a 快车) | ônibus ou trem local; parada do trem}
  \seealsoref{快车}{kuai4 che1}
\end{EntryWithPhonetic}

\begin{EntryWithPhonetic}{慢动作}{man4dong4zuo4}{14,6,7}{⼼、⼒、⼈}
  \definition{s.}{(cinema) câmera lenta}
\end{EntryWithPhonetic}

\begin{EntryWithPhonetic}{慢慢}{man4 man4}{14,14}{⼼、⼼}[HSK 3]
  \definition{adv.}{lentamente; vagarosamente; gradualmente | lentamente; vagarosamente; gradualmente; depois de um longo período de tempo}
\end{EntryWithPhonetic}

\begin{EntryWithPhonetic}{漫}{man4}{14}{⽔}
  \definition{adj.}{livre; irrestrito; casual | longo; extenso | em todos os lugares; por toda parte | aleatório; irrestrito; livre}
  \definition{adv.}{não}
  \definition{v.}{transbordar; inundar | estar em todo lugar}
\end{EntryWithPhonetic}

\begin{EntryWithPhonetic}{漫长}{man4chang2}{14,4}{⽔、⾧}[HSK 5]
  \definition{adj.}{muito longo; interminável; (tempo, espaço) dura muito tempo}
\end{EntryWithPhonetic}

\begin{EntryWithPhonetic}{漫画}{man4hua4}{14,8}{⽔、⽥}[HSK 5]
  \definition[幅,本,张,套]{s.}{desenho animado; caricatura; \emph{cartoon}}
\end{EntryWithPhonetic}

\begin{EntryWithPhonetic}{漫骂}{man4ma4}{14,9}{⽔、⾺}
  \variantof{谩骂}
\end{EntryWithPhonetic}

\begin{EntryWithPhonetic}{蔓}{man4}{14}{⾋}
  \definition{s.}{uma videira com gavinhas; caule fino que não consegue ficar em pé}
  \definition{v.}{rastejar; espalhar; estender}
  \seeref{蔓}{man2}
  \seeref{蔓}{wan4}
\end{EntryWithPhonetic}

\begin{EntryWithPhonetic}{蔓草}{man4cao3}{14,9}{⾋、⾋}
  \definition{s.}{videira | trepadeira}
\end{EntryWithPhonetic}

\begin{EntryWithPhonetic}{忙}{mang2}{6}{⼼}[HSK 1]
  \definition*{s.}{Sobrenome Mang}
  \definition{adj.}{ocupado; movimentado; totalmente ocupado; muitas coisas para fazer, sem tempo livre (oposto de 闲) | imperativo; ansioso; urgente}
  \definition{v.}{apressar-se; agitar-se; fazer algo com urgência e constantemente | trabalhar; fazer}
  \seealsoref{闲}{xian2}
\end{EntryWithPhonetic}

\begin{EntryWithPhonetic}{盲}{mang2}{8}{⽬}
  \definition{adj.}{cego | incapaz de distinguir coisas | totalmente incompetente}
  \definition{adv.}{cegamente}
\end{EntryWithPhonetic}

\begin{EntryWithPhonetic}{盲目}{mang2mu4}{8,5}{⽬、⽬}
  \definition{adj.}{ignorante | sem compreensão}
  \definition{adv.}{cegamente}
  \definition{s.}{cego}
\end{EntryWithPhonetic}

\begin{EntryWithPhonetic}{盲人}{mang2 ren2}{8,2}{⽬、⼈}[HSK 6]
  \definition[个,位,名]{s.}{cego; pessoa cega; pessoas com deficiência visual}
\end{EntryWithPhonetic}

\begin{EntryWithPhonetic}{猫}{mao1}{11}{⽝}[HSK 2]
  \definition*[只,种,群,窝,个]{s.}{gato |  Empréstimo linguístico: MODEM}
  \definition{v.}{esconder-se; entrar em esconderijo | inclinar-se para a frente; curvar-se}
  \seeref{猫}{mao2}
\end{EntryWithPhonetic}

\begin{EntryWithPhonetic}{猫熊}{mao1xiong2}{11,14}{⽝、⽕}
  \definition[把,只]{s.}{panda gigante}
  \seealsoref{熊猫}{xiong2mao1}
\end{EntryWithPhonetic}

\begin{EntryWithPhonetic}{毛}{mao2}{4}{⽑}[HSK 1,3][Kangxi 82]
  \definition*{s.}{Sobrenome Mao}
  \definition{adj.}{bruto; semiacabado | grosseiro | pequeno | fino | descuidado; rude; precipitado | assustado; nervoso; em pânico | impetuoso | rústico; sem acabamento | impuro | (de moeda) que não vale mais seu valor nominal; depreciado}
  \definition{clas.}{mao, uma unidade fracionária de dinheiro na China; dez centavos; uma peça de dez centavos}
  \definition[根]{s.}{(de um animal, planta, etc.) cabelo; pena; penugem | (de humanos) cabelo; barba | planta; colheita | lã | mofo; bolor}
  \definition{v.}{depreciar; desvalorizar; refere-se à desvalorização da moeda | (de cavalos, gado, etc.) assustar-se; sentir medo}
\end{EntryWithPhonetic}

\begin{EntryWithPhonetic}{毛笔}{mao2 bi3}{4,10}{⽑、⽵}[HSK 5]
  \definition[支,枝,根,管]{s.}{pincel para escrever; pincel chinês; canetas feitas com pelos de coelho, carneiro, doninha, etc., são materiais tradicionais utilizados para escrever caracteres chineses e pintar pinturas tradicionais chinesas}
\end{EntryWithPhonetic}

\begin{EntryWithPhonetic}{毛病}{mao2bing4}{4,10}{⽑、⽧}[HSK 3]
  \definition[个,点,种,些]{s.}{doença ou deficiência; condição de saúde precária ou deficiência física | problema; fracasso; onde o produto está com defeito ou não funciona corretamente | mau hábito; deficiência; falhas no comportamento humano}
\end{EntryWithPhonetic}

\begin{EntryWithPhonetic}{毛巾}{mao2jin1}{4,3}{⽑、⼱}[HSK 4]
  \definition[条,块]{s.}{toalha; toalha de banho}
\end{EntryWithPhonetic}

\begin{EntryWithPhonetic}{毛衣}{mao2 yi1}{4,6}{⽑、⾐}[HSK 4]
  \definition[件,个]{s.}{suéter; blusa feita de lã}
\end{EntryWithPhonetic}

\begin{EntryWithPhonetic}{矛}{mao2}{5}{⽭}[Kangxi 110]
  \definition{s.}{Arcaico: lança; lanceta}
\end{EntryWithPhonetic}

\begin{EntryWithPhonetic}{矛盾}{mao2dun4}{5,9}{⽭、⽬}[HSK 5]
  \definition{adj.}{contraditório; descreve pessoas ou coisas que se opõem ou se repelem mutuamente}
  \definition[对,个,种]{s.}{problema; contradição; discrepância; inconsistência | disputas e conflitos; relacionamento de oposição entre as duas partes devido a diferenças de opinião ou abordagem}
  \definition{v.}{opor-se; entrar em conflito; contradizer; nesta situação, apenas uma das opções está correta ou é verdadeira; não é possível que ambas estejam corretas ao mesmo tempo}
\end{EntryWithPhonetic}

\begin{EntryWithPhonetic}{牦}{mao2}{8}{⽜}
  \definition[头]{s.}{iaque; boi da Tartária}
\end{EntryWithPhonetic}

\begin{EntryWithPhonetic}{牦牛}{mao2niu2}{8,4}{⽜、⽜}
  \definition{s.}{iaque}
\end{EntryWithPhonetic}

\begin{EntryWithPhonetic}{茅}{mao2}{8}{⾋}
  \definition*{s.}{Sobrenome Mao}
  \definition[座]{s.}{capim-cogon | planta semelhante ao capim-cogon (como palha)}
\end{EntryWithPhonetic}

\begin{EntryWithPhonetic}{茅厕}{mao2ce4}{8,8}{⾋、⼚}
  \definition{s.}{(dialeto) latrina}
\end{EntryWithPhonetic}

\begin{EntryWithPhonetic}{猫}{mao2}{11}{⽝}
  \definition{v.}{utilizado em 猫腰 \dpy{mao2yao1}}
  \seeref{猫}{mao1}
  \seealsoref{猫腰}{mao2yao1}
\end{EntryWithPhonetic}

\begin{EntryWithPhonetic}{猫腰}{mao2yao1}{11,13}{⽝、⾁}
  \definition{v.}{curvar-se}
\end{EntryWithPhonetic}

\begin{EntryWithPhonetic}{冒}{mao4}{9}{⽇}[HSK 5]
  \definition*{s.}{Sobrenome Mao}
  \definition{adv.}{com ousadia; precipitadamente | fingidamente; falsamente; fraudulentamente}
  \definition{v.}{emitir; liberar; enviar (para cima) | arriscar; ser corajoso}
\end{EntryWithPhonetic}

\begin{EntryWithPhonetic}{冒险}{mao4xian3}{9,9}{⽇、⾩}
  \definition{adj.}{corajoso}
  \definition{s.}{risco | aventura}
  \definition{v.+compl.}{correr risco | arriscar-se | aventurar-se em}
\end{EntryWithPhonetic}

\begin{EntryWithPhonetic}{贸}{mao4}{9}{⾙}
  \definition*{s.}{Sobrenome Mao}
  \definition{s.}{comércio; negociação}
\end{EntryWithPhonetic}

\begin{EntryWithPhonetic}{贸易}{mao4yi4}{9,8}{⾙、⽇}[HSK 5]
  \definition[笔,宗,项,个]{s.}{comércio; troca; negócios; refere-se a atividades comerciais, como a troca de mercadorias}
  \definition{v.}{fazer uma transação comercial}
\end{EntryWithPhonetic}

\begin{EntryWithPhonetic}{帽}{mao4}{12}{⼱}
  \definition[个,顶]{s.}{chapéu; boné | capa; uma coisa que cobre um objeto e tem a função ou formato de um chapéu | elmo; capacete}
\end{EntryWithPhonetic}

\begin{EntryWithPhonetic}{帽子}{mao4zi5}{12,3}{⼱、⼦}[HSK 4]
  \definition[顶,个,种]{s.}{boné; chapéu; capacete | etiqueta; rótulo; marca}
\end{EntryWithPhonetic}

\begin{EntryWithPhonetic}{没}{mei2}{7}{⽔}[HSK 1]
  \definition{adv.}{não; nunca; negar que uma ação ou situação tenha ocorrido, com o significado de 不曾}
  \definition{pref.}{não (prefixo negativo para verbos, traduzido para outras línguas com verbos no pretérito)}
  \definition{v.}{não possuir; não ter | não existe; não há | ninguém; usado antes de 谁, 什么, 哪个, significa 全都不 | não ser tão bom quanto; ser inferior a; não chega a; não é tão bom quanto | menor que; insuficiente}
  \seeref{没}{mo4}
  \seealsoref{不曾}{bu4 ceng2}
  \seealsoref{哪个}{na3ge5}
  \seealsoref{全都不}{quan2dou1 bu4}
  \seealsoref{谁}{shei2}
  \seealsoref{什么}{shen2me5}
\end{EntryWithPhonetic}

\begin{EntryWithPhonetic}{没错}{mei2 cuo4}{7,13}{⽔、⾦}[HSK 4]
  \definition{adv.}{está certo; é isso mesmo; não há como errar}
\end{EntryWithPhonetic}

\begin{EntryWithPhonetic}{没法儿}{mei2 fa3r5}{7,8,2}{⽔、⽔、⼉}[HSK 4]
  \definition{adv.}{não pode; sem chance}
\end{EntryWithPhonetic}

\begin{EntryWithPhonetic}{没关系}{mei2guan1xi5}{7,6,7}{⽔、⼋、⽷}[HSK 1]
  \definition{v.}{está tudo bem; não é nada; não importa; não se preocupe}
  \seealsoref{没有关系}{mei2you3guan1xi5}
\end{EntryWithPhonetic}

\begin{EntryWithPhonetic}{没了}{mei2le5}{7,2}{⽔、⼅}
  \definition{v.}{estar morto | deixar de existir}
\end{EntryWithPhonetic}

\begin{EntryWithPhonetic}{没什么}{mei2 shen2 me5}{7,4,3}{⽔、⼈、⼃}[HSK 1]
  \definition{expr.}{não é nada; está tudo bem; não importa}
\end{EntryWithPhonetic}

\begin{EntryWithPhonetic}{没事儿}{mei2 shi4r5}{7,8,2}{⽔、⼅、⼉}[HSK 1]
  \definition{expr.}{fora de perigo; nada sério | não importa; não é nada; está tudo bem; não importa | está tudo bem; sem problemas; não se preocupe com isso; não é grande coisa; não há nada errado}
  \definition{v.}{não ter nada para fazer; ser livre; estar perdido | estar desempregado; estar sem trabalho | não ter responsabilidade}
\end{EntryWithPhonetic}

\begin{EntryWithPhonetic}{没想到}{mei2 xiang3 dao4}{7,13,8}{⽔、⼼、⼑}[HSK 4]
  \definition{expr.}{não esperava; inesperado}
\end{EntryWithPhonetic}

\begin{EntryWithPhonetic}{没用}{mei2 yong4}{7,5}{⽔、⽤}[HSK 3]
  \definition{adj.}{inútil; imprestável; sem valor; sem préstimo; vão; que não serve para nada}
\end{EntryWithPhonetic}

\begin{EntryWithPhonetic}{没有}{mei2 you3}{7,6}{⽔、⽉}[HSK 1]
  \definition{adv.}{ainda não; (usado com o pretérito) não; ação ou estado negativo ocorreu}
  \definition{v.}{não há; não tem; não existe}
\end{EntryWithPhonetic}

\begin{EntryWithPhonetic}{没有次序}{mei2you3 ci4xu4}{7,6,6,7}{⽔、⽉、⽋、⼴}
  \definition{adj.}{sem ordem; nenhuma ordem}
\end{EntryWithPhonetic}

\begin{EntryWithPhonetic}{没有关系}{mei2you3guan1xi5}{7,6,6,7}{⽔、⽉、⼋、⽷}
  \definition{expr.}{Está tudo bem; sem problemas}
  \seealsoref{没关系}{mei2guan1xi5}
\end{EntryWithPhonetic}

\begin{EntryWithPhonetic}{没有哪一种东西}{mei2you3 na3 yi4 zhong3 dong1xi1}{7,6,9,1,9,5,6}{⽔、⽉、⼝、⼀、⽲、⼀、⾑}
  \definition{pron.}{nada; não existe tal coisa}
\end{EntryWithPhonetic}

\begin{EntryWithPhonetic}{没有谁}{mei2you3 shei2}{7,6,10}{⽔、⽉、⾔}
  \definition{pron.}{ninguém}
\end{EntryWithPhonetic}

\begin{EntryWithPhonetic}{没有意思}{mei2you3yi4si5}{7,6,13,9}{⽔、⽉、⼼、⼼}
  \definition{adj.}{tedioso | chato | sem interesse}
\end{EntryWithPhonetic}

\begin{EntryWithPhonetic}{眉}{mei2}{9}{⽬}
  \definition{s.}{sobrancelha | margem superior}
\end{EntryWithPhonetic}

\begin{EntryWithPhonetic}{眉毛}{mei2mao5}{9,4}{⽬、⽑}
  \definition[根]{s.}{sobrancelha}
\end{EntryWithPhonetic}

\begin{EntryWithPhonetic}{眉头}{mei2tou2}{9,5}{⽬、⼤}
  \definition{s.}{testa}
\end{EntryWithPhonetic}

\begin{EntryWithPhonetic}{梅}{mei2}{11}{⽊}
  \definition*{s.}{Sobrenome Mei}
  \definition{s.}{ameixa | flor de ameixa | ameixeira | estação chuvosa}
\end{EntryWithPhonetic}

\begin{EntryWithPhonetic}{梅花}{mei2 hua1}{11,7}{⽊、⾋}[HSK 6]
  \definition[朵,枝,片,瓣,束,株]{s.}{paus ♣ (um naipe em jogos de cartas) | flor de ameixa | doçura-de-inverno; refere-se especificamente à flor-de-inverno ; também se refere a algo que se parece com esta flor}
  \seealsoref{方片}{fang1 pian4}
  \seealsoref{黑桃}{hei1 tao2}
  \seealsoref{红心}{hong2 xin1}
\end{EntryWithPhonetic}

\begin{EntryWithPhonetic}{梅赛德斯-奔驰}{mei2sai4de2si1-ben1chi2}{11,14,15,12,8,6}{⽊、⾙、⼻、⽄、⼤、⾺}
  \definition*{s.}{Mercedes-Benz}
\end{EntryWithPhonetic}

\begin{EntryWithPhonetic}{媒}{mei2}{12}{⼥}
  \definition{s.}{casamenteiro; intermediário | intermediário; médio}
  \definition{v.}{fazer uma combinação}
\end{EntryWithPhonetic}

\begin{EntryWithPhonetic}{媒体}{mei2ti3}{12,7}{⼥、⼈}[HSK 3]
  \definition[家,个,种]{s.}{mídia; mídia de massa; vários meios de comunicação e transmissão de informações, como televisão, rádio, jornais, etc.}
\end{EntryWithPhonetic}

\begin{EntryWithPhonetic}{煤}{mei2}{13}{⽕}[HSK 5]
  \definition[块,吨,斤,堆]{s.}{carvão; carvão vegetal; minério sólido preto}
\end{EntryWithPhonetic}

\begin{EntryWithPhonetic}{煤气}{mei2 qi4}{13,4}{⽕、⽓}[HSK 5]
  \definition[罐,瓶]{s.}{gás; gás de carvão; gás obtido a partir do processamento do carvão não tem cor nem odor, é tóxico e pode ser queimado ou utilizado como matéria-prima na indústria química | envenenamento por monóxido de carbono}
\end{EntryWithPhonetic}

\begin{EntryWithPhonetic}{每}{mei3}{7}{⽏}[HSK 3]
  \definition{adv.}{cada um; cada qual; indica qualquer uma das repetições ou um conjunto de repetições de um movimento}
  \definition{pron.}{cada; cada um; cada qual; refere-se a qualquer indivíduo do grupo, enfatizando as semelhanças entre os indivíduos}
\end{EntryWithPhonetic}

\begin{EntryWithPhonetic}{每次}{mei3ci4}{7,6}{⽏、⽋}
  \definition{adv.}{toda vez | cada vez}
\end{EntryWithPhonetic}

\begin{EntryWithPhonetic}{每个}{mei3ge4}{7,3}{⽏、⼈}
  \definition{pron.}{cada; cada um}
\end{EntryWithPhonetic}

\begin{EntryWithPhonetic}{每个人}{mei3ge5ren2}{7,3,2}{⽏、⼈、⼈}
  \definition{pron.}{todo mundo | todos}
\end{EntryWithPhonetic}

\begin{EntryWithPhonetic}{每天}{mei3tian1}{7,4}{⽏、⼤}
  \definition{adv.}{todo dia | cada dia}
\end{EntryWithPhonetic}

\begin{EntryWithPhonetic}{美}{mei3}{9}{⽺}[HSK 3]
  \definition*{s.}{Abreviatura de América, 美洲 | Abreviatura de Estados Unidos da América, 美国 | As Américas, 美洲}
  \definition{adj.}{belo; bonito (oposto de 丑) | muito satisfatório; bom; agradável}
  \definition{s.}{beleza (oposto de 丑)}
  \definition{v.}{embelezar; tornar mais bonito | estar satisfeito consigo mesmo; orgulhar-se; sentir-se presunçoso}
  \seealsoref{丑}{chou3}
  \seealsoref{美国}{mei3guo2}
  \seealsoref{美洲}{mei3zhou1}
\end{EntryWithPhonetic}

\begin{EntryWithPhonetic}{美国}{mei3guo2}{9,8}{⽺、⼞}
  \definition*{s.}{Estados Unidos da América}
\end{EntryWithPhonetic}

\begin{EntryWithPhonetic}{美国人}{mei3guo2ren2}{9,8,2}{⽺、⼞、⼈}
  \definition{s.}{americano | pessoa ou povo dos Estados Unidos da América}
\end{EntryWithPhonetic}

\begin{EntryWithPhonetic}{美好}{mei3 hao3}{9,6}{⽺、⼥}[HSK 3]
  \definition{adj.}{bem; feliz; glorioso; descreve a vida, os desejos, etc. como sendo muito bons e satisfatórios}
\end{EntryWithPhonetic}

\begin{EntryWithPhonetic}{美甲}{mei3jia3}{9,5}{⽺、⽥}
  \definition{s.}{manicure e/ou pedicure}
\end{EntryWithPhonetic}

\begin{EntryWithPhonetic}{美金}{mei3 jin1}{9,8}{⽺、⾦}[HSK 4]
  \definition{s.}{USD; dólar americano: a moeda local dos Estados Unidos}
\end{EntryWithPhonetic}

\begin{EntryWithPhonetic}{美丽}{mei3li4}{9,7}{⽺、⼀}[HSK 3]
  \definition{adj.}{bonito; lindo; capaz de proporcionar uma sensação de beleza}
\end{EntryWithPhonetic}

\begin{EntryWithPhonetic}{美女}{mei3 nv3}{9,3}{⽺、⼥}[HSK 4]
  \definition[个,位,名,些]{s.}{beldade; mulher bonita; uma jovem linda}
\end{EntryWithPhonetic}

\begin{EntryWithPhonetic}{美容}{mei3 rong2}{9,10}{⽺、⼧}[HSK 6]
  \definition{v.}{embelezar; melhorar a aparência de alguém; deixar seu rosto bonito retocando, cuidando, etc.}
\end{EntryWithPhonetic}

\begin{EntryWithPhonetic}{美食}{mei3 shi2}{9,9}{⽺、⾷}[HSK 3]
  \definition[种,道,桌]{s.}{iguaria; (gastronomia) comida saborosa}
\end{EntryWithPhonetic}

\begin{EntryWithPhonetic}{美术}{mei3shu4}{9,5}{⽺、⽊}[HSK 3]
  \definition[种]{s.}{arte; artes plásticas: arte que ocupa um determinado espaço, compõe imagens estéticas e permite que as pessoas apreciem visualmente, incluindo pintura, escultura, arquitetura, etc. | pintura; pintura tradicional chinesa}
\end{EntryWithPhonetic}

\begin{EntryWithPhonetic}{美味}{mei3wei4}{9,8}{⽺、⼝}
  \definition{adj.}{delicioso}
  \definition{s.}{comida deliciosa | delicadeza (\emph{delicacy})}
\end{EntryWithPhonetic}

\begin{EntryWithPhonetic}{美学}{mei3xue2}{9,8}{⽺、⼦}
  \definition{s.}{estética; a ciência que estuda as leis e os princípios gerais da beleza na natureza, na sociedade e na arte explora principalmente a natureza da beleza, a relação entre arte e realidade e as leis gerais da criação artística}
\end{EntryWithPhonetic}

\begin{EntryWithPhonetic}{美元}{mei3yuan2}{9,4}{⽺、⼉}[HSK 3]
  \definition*[元,笔,沓]{s.}{Dólar Americano; a moeda dos Estados Unidos}
\end{EntryWithPhonetic}

\begin{EntryWithPhonetic}{美洲}{mei3zhou1}{9,9}{⽺、⽔}
  \definition*{s.}{América (incluindo Norte, Central e Sul)}
\end{EntryWithPhonetic}

\begin{EntryWithPhonetic}{美洲人}{mei3zhou1ren2}{9,9,2}{⽺、⽔、⼈}
  \definition{s.}{americano | pessoa ou povo do continente Americano}
\end{EntryWithPhonetic}

\begin{EntryWithPhonetic}{妹}{mei4}{8}{⼥}[HSK 1]
  \definition*{s.}{Sobrenome Mei}
  \definition[个]{s.}{irmã mais nova | parente do sexo feminino da mesma geração | jovem garota; jovem mulher ou menina}
  \seealsoref{妹妹}{mei4 mei5}
\end{EntryWithPhonetic}

\begin{EntryWithPhonetic}{妹夫}{mei4fu5}{8,4}{⼥、⼤}
  \definition{s.}{marido da irmã mais nova}
\end{EntryWithPhonetic}

\begin{EntryWithPhonetic}{妹妹}{mei4 mei5}{8,8}{⼥、⼥}[HSK 1]
  \definition[个]{s.}{irmã mais nova}
\end{EntryWithPhonetic}

\begin{EntryWithPhonetic}{魅}{mei4}{14}{⿁}
  \definition{s.}{espírito maligno; demônio | \emph{goblin}; trasgo; gnomo; duende maléfico}
  \definition{v.}{atormentar; cativar}
\end{EntryWithPhonetic}

\begin{EntryWithPhonetic}{魅力}{mei4li4}{14,2}{⿁、⼒}
  \definition{s.}{charme | fascínio | glamour | carisma}
\end{EntryWithPhonetic}

\begin{EntryWithPhonetic}{闷}{men1}{7}{⾨}
  \definition{adj.}{abafado; fechado; sufocante; baixa pressão de ar ou má circulação de ar | abafado; som baixo ou opaco}
  \definition{v.}{cubrir bem; fazer algo hermético | ficar sem fala; parar de falar | fechar a si mesmo ou alguém dentro de casa; ficar em casa e não sair}
  \seeref{闷}{men4}
\end{EntryWithPhonetic}

\begin{EntryWithPhonetic}{闷热}{men1re4}{7,10}{⾨、⽕}
  \definition{adj.}{abafado | quente e abafado | sufocantemente quente | quente e sensual}
\end{EntryWithPhonetic}

\begin{EntryWithPhonetic}{门}{men2}{3}{⾨}[HSK 1][Kangxi 169]
  \definition*{s.}{Sobrenome Men}
  \definition{clas.}{para equipamentos de artilharia (por exemplo: canhões) | para trabalhos escolares, ciência e tecnologia, etc. | para idiomas | para casamentos | para parentes}
  \definition[个,把,道,扇]{s.}{entradas e saídas de edifícios, veículos, navios, aviões, etc. | válvula; interruptor; algo que funciona como um interruptor ou como uma porta | habilidade; método; acesso; maneira de fazer algo | família; ramo de uma família ou clã | seita (religiosa); escola (de pensamento); faculdades acadêmicas, ideológicas ou religiosas | classe; categoria; ramo de estudo; refere-se à categoria geral de coisas | filo; segundo nível da classificação biológica | (computador) \emph{gate}; porta (lógica) | porta; portão; entrada; refere-se a uma porta que pode ser aberta e fechada, instalada na entrada e saída | qualquer abertura; partes de objetos que podem ser abertas e fechadas | orifício no corpo humano; refere-se especificamente aos orifícios do corpo humano | estudar com o mesmo professor; refere-se especificamente ao professor ou mestre | posição em um jogo de apostas (em relação ao local onde se senta ou onde se faz uma aposta)}
\end{EntryWithPhonetic}

\begin{EntryWithPhonetic}{门口}{men2 kou3}{3,3}{⾨、⼝}[HSK 1]
  \definition[个]{s.}{porta; portão; entrada; porta de entrada}
\end{EntryWithPhonetic}

\begin{EntryWithPhonetic}{门票}{men2 piao4}{3,11}{⾨、⽰}[HSK 1]
  \definition{s.}{bilhete de entrada; bilhete de admissão; ingressos para locais de turismo, entretenimento, etc.}
\end{EntryWithPhonetic}

\begin{EntryWithPhonetic}{门诊}{men2 zhen3}{3,7}{⾨、⾔}[HSK 5]
  \definition{s.}{(no hospital) clínica ambulatorial; seção para pacientes ambulatoriais; local onde os médicos atendem pacientes que não estão internados no hospital}
\end{EntryWithPhonetic}

\begin{EntryWithPhonetic}{闷}{men4}{7}{⾨}
  \definition{adj.}{entediado; deprimido; irritado; desanimado | hermeticamente fechado; selado | triste e silencioso; chateado | hermético}
  \definition{s.}{desânimo}
  \seeref{闷}{men1}
\end{EntryWithPhonetic}

\begin{EntryWithPhonetic}{们}{men5}{5}{⼈}[HSK 1]
  \definition{suf.}{usado após pronomes ou substantivos que se referem a pessoas para indicar pluralidade}
\end{EntryWithPhonetic}

\begin{EntryWithPhonetic}{蒙}{meng1}{13}{⾋}[HSK 6]
  \definition{adj.}{inconsciente; sem sentido;  em coma | confuso; perplexo}
  \definition{v.}{enganar; enganar; trapacear; iludir; trair | fazer um palpite ousado; dar um palpite ousado; arriscar-se}
  \seeref{蒙}{meng2}
  \seeref{蒙}{meng3}
\end{EntryWithPhonetic}

\begin{EntryWithPhonetic}{蒙}{meng2}{13}{⾋}[HSK 6]
  \definition*{s.}{Sobrenome Meng}
  \definition{adj.}{ignorância; analfabetismo; falta de instrução | nebuloso; aparência pequena e pouco clara, como chuva ou neblina}
  \definition{s.}{aberto; inicial}
  \definition{v.}{cobrir; espalhar | receber apoio | receber; encontrar-se com; encontrar-se; palavras respeitosas; expressam os benefícios recebidos de outros | sofrer; incorrer}
  \seeref{蒙}{meng1}
  \seeref{蒙}{meng3}
\end{EntryWithPhonetic}

\begin{EntryWithPhonetic}{蒙面}{meng2mian4}{13,9}{⾋、⾯}
  \definition{adj.}{descarado | desavergonhado | mascarado}
  \definition{v.}{cobrir o rosto | usar uma máscara}
\end{EntryWithPhonetic}

\begin{EntryWithPhonetic}{猛}{meng3}{11}{⽝}[HSK 6]
  \definition*{s.}{Sobrenome Meng}
  \definition{adj.}{feroz; violento | enérgico; vigoroso | valente}
  \definition{adv.}{de repente; abruptamente | vigorosamente; com força repentina | (coloquial) ao contentamento do coração; de todo o coração | ferozmente; violentamente}
\end{EntryWithPhonetic}

\begin{EntryWithPhonetic}{猛然}{meng3ran2}{11,12}{⽝、⽕}
  \definition{adv.}{de repente; abruptamente; indica ação repentina e rápida}
\end{EntryWithPhonetic}

\begin{EntryWithPhonetic}{蒙}{meng3}{13}{⾋}
  \definition{s.}{grupo étnico mongol; mongol}
  \seeref{蒙}{meng1}
  \seeref{蒙}{meng2}
\end{EntryWithPhonetic}

\begin{EntryWithPhonetic}{懵}{meng3}{18}{⼼}
  \definition{adj.}{confuso; ignorante; irracional | inconsciente; entorpecido}
\end{EntryWithPhonetic}

\begin{EntryWithPhonetic}{懵懂}{meng3dong3}{18,15}{⼼、⼼}
  \definition{adj.}{confuso | ignorante}
\end{EntryWithPhonetic}

\begin{EntryWithPhonetic}{梦}{meng4}{11}{⼣}[HSK 4]
  \definition*{s.}{Sobrenome Meng}
  \definition[个,场]{s.}{sonho; atividade de representação no cérebro durante o sono}
  \definition{v.}{sonhar; ter um sonho}
\end{EntryWithPhonetic}

\begin{EntryWithPhonetic}{梦见}{meng4 jian4}{11,4}{⼣、⾒}[HSK 4]
  \definition{v.}{sonhar; sonhar com; ver em um sonho}
\end{EntryWithPhonetic}

\begin{EntryWithPhonetic}{梦想}{meng4xiang3}{11,13}{⼣、⼼}[HSK 4]
  \definition[个,种,些,番]{s.}{sonho; esperança vã; sonho irreal; divagação; um desejo ou ideia que você espera particularmente realizar}
  \definition{v.}{sonhar; desejar sinceramente; ansiar}
\end{EntryWithPhonetic}

\begin{EntryWithPhonetic}{眯}{mi1}{11}{⽬}
  \definition{v.}{estreitar os olhos | esmagar | (dialeto) tirar uma soneca}
  \seeref{眯}{mi2}
\end{EntryWithPhonetic}

\begin{EntryWithPhonetic}{迷}{mi2}{9}{⾡}[HSK 3]
  \definition[个]{s.}{fã; entusiasta; aficionado; pessoa que gosta excessivamente de algo}
  \definition{v.}{estar confuso; perder o rumo; se perder-se; perda da capacidade de discernimento e julgamento | ficar fascinado por; entregar-se a; ficar encantado com (por); ser louco por | confundir; desorientar; fascinar; encantar; tornar indistinto; deixar encantado e fascinado}
\end{EntryWithPhonetic}

\begin{EntryWithPhonetic}{迷宫}{mi2gong1}{9,9}{⾡、⼧}
  \definition{s.}{labirinto}
\end{EntryWithPhonetic}

\begin{EntryWithPhonetic}{迷恋}{mi2lian4}{9,10}{⾡、⼼}
  \definition{adj.}{obcecado}
  \definition{v.}{estar/ser apaixonado por | ficar encantado por | estar/ser obcecado por}
\end{EntryWithPhonetic}

\begin{EntryWithPhonetic}{迷路}{mi2lu4}{9,13}{⾡、⾜}
  \definition{s.}{labirinto | ouvido interno}
  \definition{v.+compl.}{perder o caminho | perder-se | seguir pelo caminho errado | não conseguir encontrar o caminho}
\end{EntryWithPhonetic}

\begin{EntryWithPhonetic}{迷你}{mi2ni3}{9,7}{⾡、⼈}
  \definition{adj.}{(empréstimo linguístico) mini, como em minissaia ou \emph{Mini Cooper}}
\end{EntryWithPhonetic}

\begin{EntryWithPhonetic}{迷人}{mi2ren2}{9,2}{⾡、⼈}[HSK 5]
  \definition{adj.}{encantador; fascinante; sedutor; hipnotizante}
  \definition{v.}{confundir; intrigar; enganar}
\end{EntryWithPhonetic}

\begin{EntryWithPhonetic}{迷信}{mi2xin4}{9,9}{⾡、⼈}[HSK 5]
  \definition{s.}{superstição; crença supersticiosa | fé cega; adoração cega}
  \definition{v.}{ter fé cega em; ter um fetiche de}
\end{EntryWithPhonetic}

\begin{EntryWithPhonetic}{眯}{mi2}{11}{⽬}
  \definition{v.}{cegar (como com poeira)}
  \seeref{眯}{mi1}
\end{EntryWithPhonetic}

\begin{EntryWithPhonetic}{米}{mi3}{6}{⽶}[HSK 2,3][Kangxi 119]
  \definition*{s.}{Sobrenome Mi}
  \definition{clas.}{m, metro; unidade principal de comprimento do sistema métrico}
  \definition[粒,斤]{s.}{arroz | sementes descascadas; refere-se a sementes comestíveis descascadas ou sem casca | qualquer coisa que se assemelhe a um grão de arroz}
\end{EntryWithPhonetic}

\begin{EntryWithPhonetic}{米饭}{mi3fan4}{6,7}{⽶、⾷}[HSK 1]
  \definition{s.}{arroz (cozido)}
\end{EntryWithPhonetic}

\begin{EntryWithPhonetic}{秘}{mi4}{10}{⽲}
  \definition{adj.}{secreto; misterioso | raro; raramente visto; estranho}
  \definition{adv.}{secretamente; privadamente}
  \definition{s.}{secretário}
  \definition{v.}{manter algo em segredo; esconder algo; guardar segredos | bloquear; obstruir; ter dificuldade para defecar}
  \seeref{秘}{bi4}
\end{EntryWithPhonetic}

\begin{EntryWithPhonetic}{秘密}{mi4mi4}{10,11}{⽲、⼧}[HSK 4]
  \definition{adj.}{secreto}
  \definition[个,条,些]{s.}{segredo; algo secreto; coisas que você não quer que as pessoas saibam}
\end{EntryWithPhonetic}

\begin{EntryWithPhonetic}{秘书}{mi4shu1}{10,4}{⽲、⼄}[HSK 4]
  \definition[个,位,名]{s.}{o cargo de secretário; funções de secretariado | secretário; pessoas encarregadas da correspondência e que auxiliam o chefe do órgão ou departamento na condução diária de seu trabalho}
\end{EntryWithPhonetic}

\begin{EntryWithPhonetic}{秘书长}{mi4 shu1 zhang3}{10,4,4}{⽲、⼄、⾧}[HSK 6]
  \definition{s.}{secretário-geral}
\end{EntryWithPhonetic}

\begin{EntryWithPhonetic}{密}{mi4}{11}{⼧}[HSK 4]
  \definition*{s.}{Sobrenome Mi}
  \definition{adj.}{fechado; denso; espesso | íntimo; próximo; afetuoso | delicado; fino; cuidadoso; meticuloso}
  \definition{adv.}{secretamente}
  \definition{s.}{segredo | densidade | senha; \emph{password}}
\end{EntryWithPhonetic}

\begin{EntryWithPhonetic}{密码}{mi4ma3}{11,8}{⼧、⽯}[HSK 4]
  \definition[个,种]{s.}{código; senha; um código secreto especialmente formulado usado entre as partes acordadas (diferente do 明码)}
  \seealsoref{明码}{ming2ma3}
\end{EntryWithPhonetic}

\begin{EntryWithPhonetic}{密切}{mi4qie4}{11,4}{⼧、⼑}[HSK 4]
  \definition{adj.}{próximo; íntimo; relacionamento próximo}
  \definition{adv.}{cuidadosamente; atentamente; intimamente}
  \definition{v.}{tornar-se próximo; tornar-se íntimo; conectar-se}
\end{EntryWithPhonetic}

\begin{EntryWithPhonetic}{蜜}{mi4}{14}{⾍}
  \definition{adj.}{melado; doce}
  \definition{s.}{mel | semelhante ao mel | coisas parecidas com mel; melaço}
\end{EntryWithPhonetic}

\begin{EntryWithPhonetic}{蜜桃}{mi4tao2}{14,10}{⾍、⽊}
  \definition{s.}{pêssego suculento}
\end{EntryWithPhonetic}

\begin{EntryWithPhonetic}{棉}{mian2}{12}{⽊}
  \definition{adj.}{almofadado com algodão; acolchoado}
  \definition[些,种,类]{s.}{termo genérico para algodão ou paina | algodão | material semelhante ao algodão | acolchoado ou estofado de algodão}
\end{EntryWithPhonetic}

\begin{EntryWithPhonetic}{免}{mian3}{7}{⼉}
  \definition*{s.}{Sobrenome Mian}
  \definition{v.}{desculpar alguém de algo; isentar; dispensar; renunciar | remover do cargo; demitir | evitar; desviar; escapar | não deveria ser permitido; não precisar fazer algo | remover; livrar-se de | isentar; dispensar | não permitir}
\end{EntryWithPhonetic}

\begin{EntryWithPhonetic}{免得}{mian3de5}{7,11}{⼉、⼻}[HSK 6]
  \definition{conj.}{de modo a não; para evitar; para que não; indica evitar uma situação que não é desejável e é frequentemente usado no início da oração seguinte}
\end{EntryWithPhonetic}

\begin{EntryWithPhonetic}{免费}{mian3fei4}{7,9}{⼉、⾙}[HSK 4]
  \definition{v.+compl.}{isentar de taxas; tonar grátis}
\end{EntryWithPhonetic}

\begin{EntryWithPhonetic}{免税}{mian3shui4}{7,12}{⼉、⽲}
  \definition{adj.}{isento de impostos (tributação)}
  \definition{s.}{livre de impostos | isenção de impostos}
  \definition{v.+compl.}{isentar impostos}
\end{EntryWithPhonetic}

\begin{EntryWithPhonetic}{靣}{mian4}{8}{⼀}[Kangxi 176]
  \variantof{面}
\end{EntryWithPhonetic}

\begin{EntryWithPhonetic}{面}{mian4}{9}{⾯}[HSK 2][Kangxi 176]
  \definition*{s.}{Sobrenome Mian}
  \definition{adj.}{macio e farinhento; descreve algo que é muito macio ao comer | superficial}
  \definition{adv.}{diretamente; pessoalmente; na frente de alguém; cara a cara}
  \definition{clas.}{usado para objetos planos | usado para indicar o número de vezes que as pessoas se encontram}
  \definition[斤,两,碗]{s.}{face; parte frontal da cabeça; rosto | topo; superfície | capa; exterior; a parte externa de um objeto ou a face frontal de um tecido (em oposição à 里) | (matemática) superfície | cara; sentimento; emoção | geral; área total; abrangente; toda a região | lado; aspecto | escopo; escala; extensão; alcance; âmbito | farinha; farinha de trigo | pó; algo em pó | macarrão; \emph{noodle}}
  \definition{suf.}{sufixo para localização ou direção; anexado ao final de palavras que indicam localização, equivalente a 边}
  \definition{v.}{encarar algo | encontrar; revelar-se}
  \seealsoref{边}{bian1}
  \seealsoref{里}{li3}
\end{EntryWithPhonetic}

\begin{EntryWithPhonetic}{面包}{mian4bao1}{9,5}{⾯、⼓}[HSK 1]
  \definition[个,片,袋,块]{s.}{pão}[我买八个面包了。===Comprei oito pães. | 他在吃两片面包。===Ele está comendo duas fatias de pão. | 我在家里带了一袋面包。===Trouxe um saco de pão para casa. | 我拿了一块面包。===Peguei um pedaço de pão.]
\end{EntryWithPhonetic}

\begin{EntryWithPhonetic}{面对}{mian4dui4}{9,5}{⾯、⼨}[HSK 3]
  \definition{v.}{enfrentar; defrontar; olhar para (uma pessoa ou um objeto específico) | confrontar (problema); problemas, dificuldades e outras questões que precisam ser resolvidas e que merecem atenção}
\end{EntryWithPhonetic}

\begin{EntryWithPhonetic}{面对面}{mian4 dui4 mian4}{9,5,9}{⾯、⼨、⾯}[HSK 6]
  \definition*{expr.}{frente a frente; cara a cara; vis-à-vis}
\end{EntryWithPhonetic}

\begin{EntryWithPhonetic}{面对面吃面}{mian4dui4mian4 chi1 mian4}{9,5,9,6,9}{⾯、⼨、⾯、⼝、⾯}
  \definition{expr.}{Comer macarrão cara a cara; indica que o seu estado atual, ou algumas das posições em que você está, ou algumas das coisas que você fez são muito claras}
\end{EntryWithPhonetic}

\begin{EntryWithPhonetic}{面积}{mian4ji1}{9,10}{⾯、⽲}[HSK 3]
  \definition{s.}{área (de um andar, pedaço de terreno, etc.); área de uma superfície; o tamanho de uma superfície plana ou da superfície de um objeto}
\end{EntryWithPhonetic}

\begin{EntryWithPhonetic}{面临}{mian4lin2}{9,9}{⾯、⼁}[HSK 4]
  \definition{v.}{ser confrontado com; encontrar (uma situação) na frente de}
\end{EntryWithPhonetic}

\begin{EntryWithPhonetic}{面貌}{mian4mao4}{9,14}{⾯、⾘}[HSK 5]
  \definition[种,个]{s.}{rosto; traços faciais; formato do rosto; aparência | aparência; aspecto; aparência (das coisas)}
\end{EntryWithPhonetic}

\begin{EntryWithPhonetic}{面前}{mian4 qian2}{9,9}{⾯、⼑}[HSK 2]
  \definition{s.}{antes; na frente de; diante de}
\end{EntryWithPhonetic}

\begin{EntryWithPhonetic}{面试}{mian4 shi4}{9,8}{⾯、⾔}[HSK 4]
  \definition{v.}{entrevistar (é realizado na forma de perguntas e respostas orais presenciais)}
\end{EntryWithPhonetic}

\begin{EntryWithPhonetic}{面条}{mian4tiao2}{9,7}{⾯、⽊}
  \definition{s.}{macarrão | espaguete}
\end{EntryWithPhonetic}

\begin{EntryWithPhonetic}{面条儿}{mian4 tiao2r5}{9,7,2}{⾯、⽊、⼉}[HSK 1]
  \definition{s.}{macarrão; \emph{noodles}}
\end{EntryWithPhonetic}

\begin{EntryWithPhonetic}{面团}{mian4tuan2}{9,6}{⾯、⼞}
  \definition{s.}{massa | pasta}
\end{EntryWithPhonetic}

\begin{EntryWithPhonetic}{面向}{mian4 xiang4}{9,6}{⾯、⼝}[HSK 6]
  \definition{v.}{virar o rosto para; virar na direção de; defrontar; voltado para algum lugar | estar orientado para as necessidades de; atender a; principalmente para um certo tipo de pessoas}
\end{EntryWithPhonetic}

\begin{EntryWithPhonetic}{面子}{mian4zi5}{9,3}{⾯、⼦}[HSK 5]
  \definition{s.}{face; exterior; parte externa; superfície do objeto | imagem; reputação; prestígio; decência; vaidade superficial | sentimentos; sensibilidades | pó}
\end{EntryWithPhonetic}

\begin{EntryWithPhonetic}{糆}{mian4}{15}{⽶}
  \variantof{面}
\end{EntryWithPhonetic}

\begin{EntryWithPhonetic}{麫}{mian4}{15}{⿆}
  \variantof{面}
\end{EntryWithPhonetic}

\begin{EntryWithPhonetic}{描}{miao2}{11}{⼿}
  \definition{v.}{traçar; copiar | retocar; retocar | traçar um desenho | retratar | esboçar}
\end{EntryWithPhonetic}

\begin{EntryWithPhonetic}{描述}{miao2 shu4}{11,8}{⼿、⾡}[HSK 4]
  \definition[段,种]{s.}{descrição; trecho que descreve um evento ou uma cena}
  \definition{v.}{descrever; representar}
\end{EntryWithPhonetic}

\begin{EntryWithPhonetic}{描写}{miao2xie3}{11,5}{⼿、⼍}[HSK 4]
  \definition{v.}{representar; retratar; descrever; usar a linguagem e as palavras para transmitir uma imagem concreta de uma pessoa, evento ou situação}
\end{EntryWithPhonetic}

\begin{EntryWithPhonetic}{秒}{miao3}{9}{⽲}[HSK 5]
  \definition{adv.}{instantaneamente}
  \definition{s.}{segundo (unidade de tempo) | segundo (unidade de medida angular)}
\end{EntryWithPhonetic}

\begin{EntryWithPhonetic}{妙}{miao4}{7}{⼥}[HSK 6]
  \definition*{s.}{Sobrenome Miao}
  \definition{adj.}{maravilhoso; excelente; bom | engenhoso; esperto; sutil | extraordinário | requintado; mágico; engenhoso; misterioso}
\end{EntryWithPhonetic}

\begin{EntryWithPhonetic}{妙招}{miao4zhao1}{7,8}{⼥、⼿}
  \definition{adj.}{escorregadio}
  \definition{s.}{movimento inteligente | maneira inteligente de fazer algo}
\end{EntryWithPhonetic}

\begin{EntryWithPhonetic}{灭}{mie4}{5}{⽕}[HSK 6]
  \definition{v.}{extinguir-se | extinguir; apagar; desligar | afogar; inundar; submergir | perecer; destruir | exterminar; apagar; acabar com; tornar inexistente}
\end{EntryWithPhonetic}

\begin{EntryWithPhonetic}{灭火}{mie4huo3}{5,4}{⽕、⽕}
  \definition{s.}{combate a incêndios}
  \definition{v.}{extinguir um incêndio}
\end{EntryWithPhonetic}

\begin{EntryWithPhonetic}{民}{min2}{5}{⽒}
  \definition*{s.}{Sobrenome Min}
  \definition{adj.}{folclórico ; civil (não militar)}
  \definition{s.}{pessoa | membro de um grupo étnico | uma pessoa de uma determinada ocupação | do povo; folclore | civil; cidadão | o povo | um membro de uma nacionalidade}
\end{EntryWithPhonetic}

\begin{EntryWithPhonetic}{民歌}{min2 ge1}{5,14}{⽒、⽋}[HSK 6]
  \definition[支,首]{s.}{canção folclórica; os nomes dos autores das canções transmitidas oralmente são muitas vezes desconhecidos}
\end{EntryWithPhonetic}

\begin{EntryWithPhonetic}{民工}{min2 gong1}{5,3}{⽒、⼯}[HSK 6]
  \definition{s.}{trabalhador trabalhando em um projeto público | trabalhador temporário alistado em um projeto público | agricultor que trabalha em empregos temporários na cidade | trabalhador migrante}
\end{EntryWithPhonetic}

\begin{EntryWithPhonetic}{民间}{min2jian1}{5,7}{⽒、⾨}[HSK 3]
  \definition{s.}{entre o povo | não governamental; de pessoa para pessoa}
\end{EntryWithPhonetic}

\begin{EntryWithPhonetic}{民警}{min2 jing3}{5,19}{⽒、⾔}[HSK 6]
  \definition{s.}{polícia; policial}
\end{EntryWithPhonetic}

\begin{EntryWithPhonetic}{民意}{min2 yi4}{5,13}{⽒、⼼}[HSK 6]
  \definition{s.}{vontade do povo; vontade popular | opinião pública}
\end{EntryWithPhonetic}

\begin{EntryWithPhonetic}{民众}{min2zhong4}{5,6}{⽒、⼈}
  \definition{s.}{a população | as massas | as pessoas comuns}
\end{EntryWithPhonetic}

\begin{EntryWithPhonetic}{民主}{min2zhu3}{5,5}{⽒、⼂}[HSK 6]
  \definition{adj.}{democrático; em consonância com os princípios democráticos}
  \definition[个]{s.}{democracia; direitos democráticos; refere-se ao direito do povo de participar da vida política e dos assuntos do Estado e de expressar livremente suas opiniões}
\end{EntryWithPhonetic}

\begin{EntryWithPhonetic}{民族}{min2zu2}{5,11}{⽒、⽅}[HSK 3]
  \definition[个]{s.}{nação; uma comunidade estável formada ao longo da história pela humanidade, com uma língua comum, uma região comum, uma vida econômica comum e uma mentalidade comum expressa em uma cultura comum | grupo étnico; refere-se, de maneira geral, às comunidades formadas ao longo da história por pessoas em diferentes estágios de desenvolvimento social}
\end{EntryWithPhonetic}

\begin{EntryWithPhonetic}{敏}{min3}{11}{⽁}
  \definition*{s.}{Sobrenome Min}
  \definition{adj.}{rápido; ágil | perspicaz; inteligente; rápido | inteligente; esperto}
\end{EntryWithPhonetic}

\begin{EntryWithPhonetic}{敏感}{min3gan3}{11,13}{⽁、⼼}[HSK 5]
  \definition{adj.}{sensível; descreve pessoas ou animais que rapidamente percebem mudanças ou estímulos externos | reativo; sensível; fácil de causar reações intensas}
\end{EntryWithPhonetic}

\begin{EntryWithPhonetic}{名}{ming2}{6}{⼝}[HSK 2]
  \definition*{s.}{Sobrenome Ming}
  \definition{adj.}{notável; famoso; conhecido; renomado}
  \definition{clas.}{usado para pessoas | usado para classificação por ordem}
  \definition{s.}{nome; denominação | desculpa; pretexto | fama; reputação}
  \definition{v.}{nome próprio (é) | expressar; descrever | possuir; tomar; ter}
\end{EntryWithPhonetic}

\begin{EntryWithPhonetic}{名称}{ming2 cheng1}{6,10}{⼝、⽲}[HSK 2]
  \definition[个,种]{s.}{nomes, apelidos e formas de se referir a pessoas ou coisas}
\end{EntryWithPhonetic}

\begin{EntryWithPhonetic}{名单}{ming2 dan1}{6,8}{⼝、⼗}[HSK 2]
  \definition[个,份]{s.}{lista com nomes de pessoas ou nomes de organizações}
\end{EntryWithPhonetic}

\begin{EntryWithPhonetic}{名额}{ming2'e2}{6,15}{⼝、⾴}[HSK 6]
  \definition[个]{s.}{cota de pessoas; número de pessoas designadas ou permitidas; número necessário de pessoal}
\end{EntryWithPhonetic}

\begin{EntryWithPhonetic}{名牌儿}{ming2 pai2r5}{6,12,2}{⼝、⽚、⼉}[HSK 4]
  \definition{s.}{marca famosa}
\end{EntryWithPhonetic}

\begin{EntryWithPhonetic}{名片}{ming2pian4}{6,4}{⼝、⽚}[HSK 4]
  \definition[张,盒,叠]{s.}{cartão de visita; um pedaço de papel retangular com o nome, o cargo, o endereço etc. impressos}
\end{EntryWithPhonetic}

\begin{EntryWithPhonetic}{名人}{ming2 ren2}{6,2}{⼝、⼈}[HSK 4]
  \definition[位,个]{s.}{celebridade; pessoa famosa}
\end{EntryWithPhonetic}

\begin{EntryWithPhonetic}{名胜}{ming2 sheng4}{6,9}{⼝、⾁}[HSK 6]
  \definition[处,个]{s.}{pontos turísticos; atrações famosas; lugares famosos com locais históricos ou belas paisagens}
\end{EntryWithPhonetic}

\begin{EntryWithPhonetic}{名义}{ming2 yi4}{6,3}{⼝、⼂}[HSK 6]
  \definition{s.}{nominal; em nome (geralmente seguido por 上); um nome ou título usado como base para fazer algo}[有人盗用我名义申请贷款。===Alguém solicitou um empréstimo em meu nome. | 他们只是名义上的夫妻。===Eles são marido e mulher apenas no nome.]
  \seealsoref{上}{shang4}
\end{EntryWithPhonetic}

\begin{EntryWithPhonetic}{名誉}{ming2yu4}{6,13}{⼝、⾔}[HSK 6]
  \definition{adj.}{1. honorário; nominal (geralmente se refere ao nome de um presente, com um sentido de respeito)}[他是学校的名誉教授。===Ele é professor honorário da escola.]
  \definition{s.}{fama; reputação; honra}[名誉才是最神圣的。===Reputação é a coisa mais sagrada. | 我用自己的名誉发誓。===Juro pela minha honra.]
\end{EntryWithPhonetic}

\begin{EntryWithPhonetic}{名字}{ming2zi5}{6,6}{⼝、⼦}[HSK 1]
  \definition[个]{s.}{nome; nome próprio | nome (de uma coisa)}
\end{EntryWithPhonetic}

\begin{EntryWithPhonetic}{明}{ming2}{8}{⽇}
  \definition*{s.}{Dinastia Ming (1368-1644) | Sobrenome Ming}
  \definition{adj.}{claro; brilhante; brilhante | claro; distinto; de fácil entendimento | aberto; evidente; explícito; exposto | de ​​olhos aguçados; boa visão; visão nítida | honesto}
  \definition{adv.}{claramente; definitivamente; aparentemente; de fato}
  \definition{s.}{imediatamente a seguir no tempo; ao lado deste ano e hoje; visão}
  \definition{v.}{mostrar; revelar; tornar conhecido; deixar claro | entender; compreender}
\end{EntryWithPhonetic}

\begin{EntryWithPhonetic}{明白}{ming2bai5}{8,5}{⽇、⽩}[HSK 1]
  \definition{adj.}{claro; óbvio; evidente; inequívoco | sensato; razoável | aberto; franco; inequívoco; explícito}
  \definition{v.}{entender; compreender; saber}
\end{EntryWithPhonetic}

\begin{EntryWithPhonetic}{明亮}{ming2 liang4}{8,9}{⽇、⼇}[HSK 5]
  \definition{adj.}{claro; bem iluminado | brilhante; resplandecente | claro; simples; compreensível}
\end{EntryWithPhonetic}

\begin{EntryWithPhonetic}{明码}{ming2ma3}{8,8}{⽇、⽯}
  \definition{s.}{código simples, em claro (oposto a 密码) | preço claramente marcado}
  \seealsoref{密码}{mi4ma3}
\end{EntryWithPhonetic}

\begin{EntryWithPhonetic}{明明}{ming2ming2}{8,8}{⽇、⽇}[HSK 5]
  \definition{adv.}{obviamente; claramente; sem dúvida; indica que o fenômeno ou princípio é evidente}
\end{EntryWithPhonetic}

\begin{EntryWithPhonetic}{明年}{ming2 nian2}{8,6}{⽇、⼲}[HSK 1]
  \definition{s.}{próximo ano}
\end{EntryWithPhonetic}

\begin{EntryWithPhonetic}{明确}{ming2que4}{8,12}{⽇、⽯}[HSK 3]
  \definition{adj.}{claro; definido; específico}
  \definition{v.}{deixar claro; tornar definitivo; tornar um ponto de vista, uma tarefa, etc. claro, compreensível e definitivo}
\end{EntryWithPhonetic}

\begin{EntryWithPhonetic}{明日}{ming2 ri4}{8,4}{⽇、⽇}[HSK 6]
  \definition{s.}{amanhã}
  \seealsoref{明天}{ming2tian1}
\end{EntryWithPhonetic}

\begin{EntryWithPhonetic}{明天}{ming2tian1}{8,4}{⽇、⼤}[HSK 1]
  \definition{s.}{amanhã | futuro próximo}
\end{EntryWithPhonetic}

\begin{EntryWithPhonetic}{明显}{ming2xian3}{8,9}{⽇、⽇}[HSK 3]
  \definition{adj.}{claro; óbvio; distinto; claramente visível}
\end{EntryWithPhonetic}

\begin{EntryWithPhonetic}{明星}{ming2xing1}{8,9}{⽇、⽇}[HSK 2]
  \definition[个,位,颗,名]{s.}{estrela; ator, atleta, cantor famosos, etc. | talento de ponta; profissional de destaque; também é usado como metáfora para pessoas ou grupos que se destacam pelo seu bom desempenho ou excelência | estrela brilhante; estrela resplandecente; referindo-se a estrelas muito brilhantes}
\end{EntryWithPhonetic}

\begin{EntryWithPhonetic}{明珠}{ming2zhu1}{8,10}{⽇、⽟}
  \definition{s.}{pérola | jóia (de grande valor)}
\end{EntryWithPhonetic}

\begin{EntryWithPhonetic}{鸣}{ming2}{8}{⿃}
  \definition{v.}{chorar (pássaros, animais e insetos) | fazer um som | dar voz (gratidão, queixas, etc.)}
\end{EntryWithPhonetic}

\begin{EntryWithPhonetic}{命}{ming4}{8}{⼝}[HSK 6]
  \definition[条]{s.}{vida | sorte; destino; fado | ordem; comando; instrução | atribuição de um nome, título etc.}
  \definition{v.}{ordenar; nomear | atribuir (um nome etc.)}
\end{EntryWithPhonetic}

\begin{EntryWithPhonetic}{命令}{ming4ling4}{8,5}{⼝、⼈}[HSK 5]
  \definition[条,项,道,个]{s.}{ordem; comando; instruções emitidas pelos superiores aos subordinados}
  \definition{v.}{ordenar; comandar}
\end{EntryWithPhonetic}

\begin{EntryWithPhonetic}{命运}{ming4yun4}{8,7}{⼝、⾡}[HSK 3]
  \definition[个]{s.}{tendência de desenvolvimento; tendência de futuro; metáfora para a direção e tendência do desenvolvimento e das mudanças | destino; sina; sorte; refere-se à vida e à morte, à riqueza e à pobreza e a todas as experiências da vida}
\end{EntryWithPhonetic}

\begin{EntryWithPhonetic}{摸}{mo1}{13}{⼿}[HSK 4]
  \definition{v.}{sentir; acariciar; tocar; tocar (um objeto) levemente com a mão e depois removê-lo ou mover a mão suavemente sobre a superfície do objeto | sentir para; tatear para; sentir algo com as mãos | descobrir; sentir; sondar; explorar; tentar fazer ou entender | sentir o caminho; tatear no escuro; andar por estradas que você não consegue reconhecer | furtar; roubar}
\end{EntryWithPhonetic}

\begin{EntryWithPhonetic}{模}{mo2}{14}{⽊}
  \definition{s.}{padrão | modelo; exemplo | modelo (pessoa) | exame simulado | módulo}
  \definition{v.}{imitar | copiar; emular}
  \seeref{模}{mu2}
\end{EntryWithPhonetic}

\begin{EntryWithPhonetic}{模范}{mo2fan4}{14,9}{⽊、⾋}[HSK 5]
  \definition{adj.}{exemplar}
  \definition{s.}{modelo; exemplo excelente; pessoa exemplar; coisa exemplar; pessoas ou coisas exemplares que servem de modelo}
\end{EntryWithPhonetic}

\begin{EntryWithPhonetic}{模仿}{mo2fang3}{14,6}{⽊、⼈}[HSK 5]
  \definition{v.}{copiar; imitar; aprender a fazer algo seguindo um modelo pronto}
\end{EntryWithPhonetic}

\begin{EntryWithPhonetic}{模糊}{mo2hu5}{14,15}{⽊、⽶}[HSK 5]
  \definition{adj.}{vago; confuso; indistinto}
  \definition{v.}{confundir; desorientar}
\end{EntryWithPhonetic}

\begin{EntryWithPhonetic}{模式}{mo2shi4}{14,6}{⽊、⼷}[HSK 5]
  \definition{s.}{modelo; modo; padrão; a forma padrão de algo ou o modelo padrão que as pessoas podem seguir}
\end{EntryWithPhonetic}

\begin{EntryWithPhonetic}{模特儿}{mo2 te4r5}{14,10,2}{⽊、⽜、⼉}[HSK 4]
  \definition[个,名,位]{s.}{modelo (pessoa que posa para um fotógrafo ou pintor ou escultor); objeto de representação ou referência usado por artistas para esboços e esculturas, como o corpo humano, objetos, modelos etc.; também se refere aos arquétipos que os estudiosos da literatura usam para retratar seus personagens | modelo (uma pessoa que usa roupas para exibir modas); pessoa ou manequim usado para exibir estilos de roupas}
\end{EntryWithPhonetic}

\begin{EntryWithPhonetic}{模型}{mo2xing2}{14,9}{⽊、⼟}[HSK 4]
  \definition[个]{s.}{modelo; padrão; itens feitos em escala com base em objetos ou desenhos | molde; padrão; molde para fundir máquinas, objetos, etc.}
\end{EntryWithPhonetic}

\begin{EntryWithPhonetic}{膜}{mo2}{14}{⾁}[HSK 6]
  \definition[张]{s.}{membrana | filme; revestimento fino}
\end{EntryWithPhonetic}

\begin{EntryWithPhonetic}{膜拜}{mo2bai4}{14,9}{⾁、⼿}
  \definition{v.}{ajoelhar-se e se curvar com as mãos unidas no nível da testa | ter ou mostrar sentimentos fortes de respeito e admiração por um deus}
\end{EntryWithPhonetic}

\begin{EntryWithPhonetic}{摩}{mo2}{15}{⼿}
  \definition{v.}{esfregar; raspar; tocar | refletir; estudar | afagar}
\end{EntryWithPhonetic}

\begin{EntryWithPhonetic}{摩擦}{mo2ca1}{15,17}{⼿、⼿}[HSK 5]
  \definition{s.}{atrito; desacordo; conflito (entre duas partes); a ação de impedir o movimento relativo entre dois objetos em contato, produzida na superfície de contato | atrito; metáfora para o conflito entre as duas partes}
  \definition{v.}{esfregar}
\end{EntryWithPhonetic}

\begin{EntryWithPhonetic}{摩托}{mo2 tuo1}{15,6}{⼿、⼿}[HSK 5]
  \definition[辆]{s.}{Empréstimo linguístico: motor; motor de combustão interna | Empréstimo linguístico: motocicleta, abreviação de 摩托车}
  \seealsoref{摩托车}{mo2tuo1che1}
\end{EntryWithPhonetic}

\begin{EntryWithPhonetic}{摩托车}{mo2tuo1che1}{15,6,4}{⼿、⼿、⾞}
  \definition[辆,部]{s.}{(empréstimo linguístico) motocicleta}
\end{EntryWithPhonetic}

\begin{EntryWithPhonetic}{磨}{mo2}{16}{⽯}[HSK 6]
  \definition{v.}{esfregar; desgastar | moer; refletir; polir | desgastar; esgotar; cansar; exaurir | incomodar; causar problemas | destruir; obliterar; extinguir-se | ficar ocioso; perder tempo; perder tempo; procrastinar}
  \seeref{磨}{mo4}
\end{EntryWithPhonetic}

\begin{EntryWithPhonetic}{磨菇}{mo2gu5}{16,11}{⽯、⾋}
  \variantof{蘑菇}
\end{EntryWithPhonetic}

\begin{EntryWithPhonetic}{蘑}{mo2}{19}{⾋}
  \definition{s.}{cogumelo}
\end{EntryWithPhonetic}

\begin{EntryWithPhonetic}{蘑菇}{mo2gu5}{19,11}{⾋、⾋}
  \definition{s.}{cogumelo}
  \definition{v.}{mandriar | embromar | amofinar | incomodar alguém com solicitações ou interrupções frequentes ou persistentes}
\end{EntryWithPhonetic}

\begin{EntryWithPhonetic}{魔}{mo2}{20}{⿁}
  \definition{adj.}{místico; misterioso; mágico}
  \definition{s.}{espírito maligno; demônio; diabo; monstro | mágico; místico}
\end{EntryWithPhonetic}

\begin{EntryWithPhonetic}{魔术}{mo2shu4}{20,5}{⿁、⽊}
  \definition{s.}{magia}
\end{EntryWithPhonetic}

\begin{EntryWithPhonetic}{魔头}{mo2tou2}{20,5}{⿁、⼤}
  \definition{s.}{monstro | diabo}
\end{EntryWithPhonetic}

\begin{EntryWithPhonetic}{抹}{mo3}{8}{⼿}
  \definition{v.}{colocar; aplicar; untar; engessar | limpar | anular; apagar | (para nuvem, etc.) irradiar; raiar; riscar; traçar | riscar; cancelar; marcar; remover; excluir}
  \seeref{抹}{ma1}
  \seeref{抹}{mo3}
\end{EntryWithPhonetic}

\begin{EntryWithPhonetic}{抹泪}{mo3lei4}{8,8}{⼿、⽔}
  \definition{v.}{limpar as lágrimas | (figurativo) derramar lágrimas}
\end{EntryWithPhonetic}

\begin{EntryWithPhonetic}{末}{mo4}{5}{⽊}[HSK 4]
  \definition{adj.}{último; final}
  \definition{s.}{ponta; terminal; extremidade; o final de algo | não essenciais; detalhes secundários | fim; final | pó; poeira | um papel na ópera tradicional}
\end{EntryWithPhonetic}

\begin{EntryWithPhonetic}{没}{mo4}{7}{⽔}
  \definition{adj.}{último; final}
  \definition{v.}{afundar na água; submergir | transbordar; subir além; exceder ou ultrapassar | esconder-se; desaparecer; sumir; ocultar-se | confiscar; expropriar | morrer}
  \variantof{没}
\end{EntryWithPhonetic}

\begin{EntryWithPhonetic}{没收}{mo4 shou1}{7,6}{⽔、⽁}[HSK 6]
  \definition{v.}{confiscar; expropriar; os bens e pertences de pessoas ou grupos que violem leis ou proibições serão tornados propriedade pública, de acordo com a lei}
\end{EntryWithPhonetic}

\begin{EntryWithPhonetic}{抹}{mo4}{8}{⼿}
  \definition{v.}{rebocar; engessar; alisar a massa ou o gesso com uma espátula | virar; contornar; dar uma volta de perto}
  \seeref{抹}{ma1}
  \seeref{抹}{mo3}
\end{EntryWithPhonetic}

\begin{EntryWithPhonetic}{莫}{mo4}{10}{⾋}
  \definition*{s.}{Sobrenome Mo}
  \definition{adv.}{não, frequentemente usado em frases imperativas | não; não pode | pode ser que; não pode ser que; é possível que}
  \definition{pron.}{nenhum; nada; ninguém; significa 没有谁 ou 没有哪一种东西}
  \seealsoref{没有哪一种东西}{mei2you3 na3 yi4 zhong3 dong1xi1}
  \seealsoref{没有谁}{mei2you3 shei2}
\end{EntryWithPhonetic}

\begin{EntryWithPhonetic}{莫非}{mo4fei1}{10,8}{⾋、⾮}
  \definition{expr.}{Não é mesmo?; é frequentemente usado com 不成}
  \definition{v.}{pode ser que; é possível que}
  \seealsoref{不成}{bu4 cheng2}
\end{EntryWithPhonetic}

\begin{EntryWithPhonetic}{莫名其妙}{mo4ming2qi2miao4}{10,6,8,7}{⾋、⼝、⼋、⼥}
  \definition{adj.}{desconcertante | bizzaro | inexplicável | perplexo}
\end{EntryWithPhonetic}

\begin{EntryWithPhonetic}{墨}{mo4}{15}{⿊}
  \definition*{s.}{Escola Moísta; Moísmo | México, abreviação de 墨西哥}
  \definition{adj.}{preto; escuro como breu | corrupto | escuro}
  \definition{s.}{tinta chinesa; bastão de tinta | pigmento; tinta | caligrafia ou pintura | aprendizagem; alfabetização | marcador de linha de carpinteiro; marcador de tinta | tatuar o rosto (um castigo); uma punição na China antiga | corrupção; peculato; fraude}
  \seealsoref{墨西哥}{mo4xi1ge1}
\end{EntryWithPhonetic}

\begin{EntryWithPhonetic}{墨镜}{mo4jing4}{15,16}{⿊、⾦}
  \definition[只,双,副]{s.}{óculos escuros}
\end{EntryWithPhonetic}

\begin{EntryWithPhonetic}{墨水}{mo4 shui3}{15,4}{⿊、⽔}[HSK 6]
  \definition[瓶]{s.}{tinta chinesa preparada; tinta (para caneta-tinteiro) | aprendizagem; alfabetização; uma metáfora para o conhecimento ou a capacidade de ler e escrever}
\end{EntryWithPhonetic}

\begin{EntryWithPhonetic}{墨西哥}{mo4xi1ge1}{15,6,10}{⿊、⾑、⼝}
  \definition*{s.}{México; Planalto no México}
\end{EntryWithPhonetic}

\begin{EntryWithPhonetic}{磨}{mo4}{16}{⽯}
  \definition[盘]{s.}{mó (pedra pesada e redonda para moinho)}
  \definition{v.}{moer; esfarelar; triturar | virar; inverter a marcha}
\end{EntryWithPhonetic}

\begin{EntryWithPhonetic}{默}{mo4}{16}{⿊}
  \definition*{s.}{Sobrenome Mo}
  \definition{adj.}{taciturno; reservado | silencioso}
  \definition{v.}{escrever de memória}
\end{EntryWithPhonetic}

\begin{EntryWithPhonetic}{默默}{mo4mo4}{16,16}{⿊、⿊}[HSK 4]
  \definition{adj.}{mudo; quieto; silencioso}
  \definition{adv.}{silenciosamente}
\end{EntryWithPhonetic}

\begin{EntryWithPhonetic}{默契}{mo4qi4}{16,9}{⿊、⼤}
  \definition{adj.}{(de membros da equipe) bem coordenados}
  \definition{s.}{entendimento tácito | entendimento mútuo | conectado em um nível mútuo profundo | (de membros da equipe) bem coordenados}
\end{EntryWithPhonetic}

\begin{EntryWithPhonetic}{某}{mou3}{9}{⽊}[HSK 3]
  \definition{pron.}{alguém ou algo indefinido; refere-se a pessoas ou coisas incertas | referindo-se a si mesmo; em vez do seu próprio nome | alguns; certos; refere-se a uma pessoa ou coisa específica cujo nome não se sabe ou não se pode revelar | tal e tal; substituir o nome de outra pessoa (geralmente com um tom rude)}
\end{EntryWithPhonetic}

\begin{EntryWithPhonetic}{模}{mu2}{14}{⽊}
  \definition*{s.}{Sobrenome Mu}
  \definition{s.}{molde; padrão; matriz}
  \seeref{模}{mo2}
\end{EntryWithPhonetic}

\begin{EntryWithPhonetic}{模具}{mu2ju4}{14,8}{⽊、⼋}
  \definition{s.}{molde | matriz | padrão}
\end{EntryWithPhonetic}

\begin{EntryWithPhonetic}{模样}{mu2yang4}{14,10}{⽊、⽊}[HSK 5]
  \definition[副,种]{s.}{aparência; a aparência ou o estilo de vestir de uma pessoa | indicando uma estimativa aproximada de tempo ou idade; expressão de estimativas relativas a tempo, idade, etc. | tendência; situação; inclinação}
\end{EntryWithPhonetic}

\begin{EntryWithPhonetic}{母}{mu3}{5}{⽏}[HSK 6][Kangxi 80]
  \definition*{s.}{Sobrenome Mu}
  \definition{adj.}{fêmea}
  \definition[位,名,个,些]{s.}{mãe | fêmea (animal) (oposto a 公) | origem; pais | parentes idosas; geralmente se refere a mulheres idosas | côncavo | fonte; algo que tem a capacidade ou função de produzir outras coisas}
  \seealsoref{公}{gong1}
\end{EntryWithPhonetic}

\begin{EntryWithPhonetic}{母鸡}{mu3ji1}{5,7}{⽏、⿃}[HSK 6]
  \definition{s.}{galinha}
\end{EntryWithPhonetic}

\begin{EntryWithPhonetic}{母女}{mu3 nv3}{5,3}{⽏、⼥}[HSK 6]
  \definition{s.}{mãe e filha}
\end{EntryWithPhonetic}

\begin{EntryWithPhonetic}{母亲}{mu3qin1}{5,9}{⽏、⼇}[HSK 3]
  \definition[位,名,个,些]{s.}{mãe}
\end{EntryWithPhonetic}

\begin{EntryWithPhonetic}{母语}{mu3yu3}{5,9}{⽏、⾔}
  \definition{s.}{língua materna | língua nativa}
\end{EntryWithPhonetic}

\begin{EntryWithPhonetic}{母子}{mu3 zi3}{5,3}{⽏、⼦}[HSK 6]
  \definition{s.}{mãe e filho}
\end{EntryWithPhonetic}

\begin{EntryWithPhonetic}{亩}{mu3}{7}{⼇}
  \definition{clas.}{usado para campos | unidade de área igual a um décimo quinto de um hectare}
\end{EntryWithPhonetic}

\begin{EntryWithPhonetic}{木}{mu4}{4}{⽊}[Kangxi 75]
  \definition{adj.}{de madeira; feito de madeira | estúpido; de raciocínio lento; atordoado; lento para reagir | simplório; chato | 3. entorpecido; de madeira; dormência localizada ou perda de sensibilidade}
  \definition{s.}{árvore | madeira; madeiramento | caixão}
\end{EntryWithPhonetic}

\begin{EntryWithPhonetic}{木偶}{mu4'ou3}{4,11}{⽊、⼈}
  \definition{s.}{fantoche, marionete}
\end{EntryWithPhonetic}

\begin{EntryWithPhonetic}{木头}{mu4tou5}{4,5}{⽊、⼤}[HSK 3]
  \definition[根,块,堆,截]{s.}{tronco; madeira; lenha; denominação genérica para madeira e materiais de madeira}
\end{EntryWithPhonetic}

\begin{EntryWithPhonetic}{目}{mu4}{5}{⽬}[Kangxi 109]
  \definition*{s.}{Sobrenome Mu}
  \definition{s.}{olho | item | (biologia) ordem | lista de coisas; catálogo; sumário | buraco em uma rede; malha (abertura)  | (de documentos, teses, etc.) nome; título | ponto; ponto de território, um termo do Go; refere-se à intersecção das linhas verticais e horizontais no tabuleiro, uma intersecção é chamada de 一目, \dpy{yi2 mu4}}
  \definition{v.}{(literário) olhar; considerar}
\end{EntryWithPhonetic}

\begin{EntryWithPhonetic}{目标}{mu4biao1}{5,9}{⽬、⽊}[HSK 3]
  \definition[个,项]{s.}{alvo; objetivo; objeto de tiro, ataque ou busca| objetivo; meta; destino; a situação ou padrão que se deseja alcançar}
\end{EntryWithPhonetic}

\begin{EntryWithPhonetic}{目的}{mu4di4}{5,8}{⽬、⽩}[HSK 2]
  \definition[个,些,种]{s.}{objetivo; meta; alvo; finalidade; propósito; o lugar ou situação que se deseja alcançar; o resultado que se deseja obter; o centro do alvo}
\end{EntryWithPhonetic}

\begin{EntryWithPhonetic}{目光}{mu4guang1}{5,6}{⽬、⼉}[HSK 5]
  \definition[道,束,种]{s.}{olhar fixo; a expressão e atitude reveladas pelos olhos | visão; vista; percepção visual; a linha imaginária formada entre os olhos e o objeto quando se olha para ele | perspicácia (capacidade de observar e reconhecer coisas); conhecimento adquirido através do contato com as coisas, capacidade de observar as coisas}
\end{EntryWithPhonetic}

\begin{EntryWithPhonetic}{目前}{mu4qian2}{5,9}{⽬、⼑}[HSK 3]
  \definition{adv.}{agora; recentemente; no momento; no presente}
\end{EntryWithPhonetic}

\begin{EntryWithPhonetic}{墓}{mu4}{13}{⼟}
  \definition[座,个,号]{s.}{sepultura; túmulo; mausoléu}
\end{EntryWithPhonetic}

\begin{EntryWithPhonetic}{幕}{mu4}{13}{⼱}
  \definition{s.}{cortina ou tela | dossel ou tenda | quartel de um general | ato (de uma peça)}
\end{EntryWithPhonetic}

%%%%% EOF %%%%%

