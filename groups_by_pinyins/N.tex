%%%
%%% N
%%%

\section*{N}\addcontentsline{toc}{section}{N}

\begin{EntryWithPhonetic}{那}{na1}{6}{⾢}
  \definition*{s.}{Sobrenome Na}
  \seeref{na3}
  \seeref{na4}
  \seeref{ne4}
  \seeref{nei4}
  \seeref{nuo2}
\end{EntryWithPhonetic}

\begin{EntryWithPhonetic}{拿}{na2}{10}{⼿}[HSK 1]
  \definition{part.}{usado da mesma forma que 把: para marcar o seguinte substantivo seguinte como objeto direto}
  \definition{prep.}{ferramentas, materiais, métodos, etc. utilizados para a introdução | os objetos que estão sendo manipulados para introdução}
  \definition{v.}{segurar; pegar; pegar ou mover objetos com as mãos ou de outra forma | apreender; capturar; prender; usar força bruta para capturar | ter certeza de; ser capaz de fazer; ter uma compreensão firme de | tornar as coisas difíceis para alguém; colocar alguém em uma situação difícil; obstruir; chantagear; coagir; causar dificuldades intencionalmente | fingir ou fazer (algum tipo de postura ou aparência) | ter certeza de; tomar uma decisão | obter; ganhar; receber}
\end{EntryWithPhonetic}

\begin{EntryWithPhonetic}{拿出}{na2 chu1}{10,5}{⼿、⼐}[HSK 2]
  \definition{v.}{apresentar (evidências) | fornecer | apresentar (uma proposta) | oferecer; servir | retirar; tirar}
\end{EntryWithPhonetic}

\begin{EntryWithPhonetic}{拿到}{na2 dao4}{10,8}{⼿、⼑}[HSK 2]
  \definition{v.}{pegar; obter, conseguir}
\end{EntryWithPhonetic}

\begin{EntryWithPhonetic}{拿走}{na2 zou3}{10,7}{⼿、⾛}[HSK 6]
  \definition{v.}{tirar; remover}
\end{EntryWithPhonetic}

\begin{EntryWithPhonetic}{那}{na3}{6}{⾢}
  \definition{adv.}{expressa negação em perguntas retóricas}
  \definition{pron.}{qual? | qualquer que seja; qualquer que; para expressar incerteza em uma declaração | variante de 哪}
  \seeref{na1}
  \seeref{na4}
  \seeref{ne4}
  \seeref{nei4}
  \seeref{nuo2}
  \seealsoref{哪}{na3}
\end{EntryWithPhonetic}

\begin{EntryWithPhonetic}{哪}{na3}{9}{⼝}[HSK 1,4]
  \definition{adv.}{para expressar uma pergunta retórica, indicando que é impossível}
  \definition{pron.}{qual?; o que?; expressa a necessidade de determinar um entre várias pessoas ou coisas | qualquer; ser usado em um sentido geral | qual?; o que?; (usado sozinho, o mesmo que 什么, frequentemente usado de forma intercambiável com 什么) | qualquer; qualquer que seja; refere-se a qualquer um, geralmente seguido por 都 ou 也, ou usando dois 哪 antes e depois | qual (indica algo incerto)}
  \seeref{na5}
  \seeref{nei3}
  \seealsoref{都}{dou1}
  \seealsoref{什么}{shen2me5}
  \seealsoref{也}{ye3}
\end{EntryWithPhonetic}

\begin{EntryWithPhonetic}{哪个}{na3ge5}{9,3}{⼝、⼈}
  \definition{pron.}{qual deles (pergunta sobre o objeto) | quem (perguntar a alguém ou indicar qualquer pessoa)}
\end{EntryWithPhonetic}

\begin{EntryWithPhonetic}{哪国人}{na3 guo2ren2}{9,8,2}{⼝、⼞、⼈}
  \definition{expr.}{de qual país?}
\end{EntryWithPhonetic}

\begin{EntryWithPhonetic}{哪里}{na3 li3}{9,7}{⼝、⾥}[HSK 1]
  \definition{adv.}{usado em perguntas retóricas para expressar um significado negativo}
  \definition{pron.}{onde?; em que lugar? | onde quer que seja; em qualquer lugar | usado como uma resposta educada a um elogio}
\end{EntryWithPhonetic}

\begin{EntryWithPhonetic}{哪怕}{na3pa4}{9,8}{⼝、⼼}[HSK 4]
  \definition{conj.}{mesmo; mesmo se; mesmo que; não importa o quão}
\end{EntryWithPhonetic}

\begin{EntryWithPhonetic}{哪儿}{na3r5}{9,2}{⼝、⼉}[HSK 1]
  \definition{adv.}{usado para perguntas retóricas, indicando negação}
  \definition{pron.}{onde? | onde quer que seja; em qualquer lugar | usado como uma resposta educada a um elogio}
\end{EntryWithPhonetic}

\begin{EntryWithPhonetic}{哪些}{na3xie1}{9,8}{⼝、⼆}[HSK 1]
  \definition{pron.}{quais?}
\end{EntryWithPhonetic}

\begin{EntryWithPhonetic}{那}{na4}{6}{⾢}[HSK 1,2]
  \definition{conj.}{então; nessa situação; nesse caso; o mesmo que 那么}
  \definition{pron.}{aquele; aquilo; indica pessoas ou coisas distantes | aquele; aquilo; expressa muitas coisas, sem se referir especificamente a uma pessoa ou coisa, e é frequentemente usado em conjunto com 这}
  \seeref{na1}
  \seeref{na3}
  \seeref{ne4}
  \seeref{nei4}
  \seeref{nuo2}
  \seealsoref{那么}{na4 me5}
  \seealsoref{这}{zhe4}
\end{EntryWithPhonetic}

\begin{EntryWithPhonetic}{那边}{na4 bian5}{6,5}{⾢、⾡}[HSK 1]
  \definition{pron.}{ali; acolá; aquele lado}
\end{EntryWithPhonetic}

\begin{EntryWithPhonetic}{那个}{na4ge5}{6,3}{⾢、⼈}
  \definition{pron.}{aquele | usado antes de verbos e adjetivos para indicar exagero | para substituir o discurso direto inconveniente}
\end{EntryWithPhonetic}

\begin{EntryWithPhonetic}{那会儿}{na4 hui4r5}{6,6,2}{⾢、⼈、⼉}[HSK 2]
  \definition{pron.}{então; naquela época; refere-se ao passado ou ao futuro}
\end{EntryWithPhonetic}

\begin{EntryWithPhonetic}{那里}{na4 li3}{6,7}{⾢、⾥}[HSK 1]
  \definition{pron./s.}{lá; ali; aquele lugar; indica um lugar distante}
\end{EntryWithPhonetic}

\begin{EntryWithPhonetic}{那么}{na4 me5}{6,3}{⾢、⼃}[HSK 2]
  \definition{conj.}{então; nesse caso; afirmar o resultado esperado ou fazer um julgamento}
  \definition{pron.}{assim; dessa maneira; indica a natureza, o estado, a forma, o grau, etc. | assim; sobre; colocado antes do numeral, indica uma estimativa}
\end{EntryWithPhonetic}

\begin{EntryWithPhonetic}{那麽}{na4 me5}{6,14}{⾢、⿇}
  \variantof{那么}
\end{EntryWithPhonetic}

\begin{EntryWithPhonetic}{那儿}{na4r5}{6,2}{⾢、⼉}[HSK 1]
  \definition{pron.}{lá; ali; naquele lugar | então; naquela época (usado após 打, 从 e 由)}
  \seealsoref{从}{cong2}
  \seealsoref{打}{da3}
  \seealsoref{由}{you2}
\end{EntryWithPhonetic}

\begin{EntryWithPhonetic}{那时}{na4 shi2}{6,7}{⾢、⽇}[HSK 2]
  \definition{pron.}{então; naquela época; naqueles dias; geralmente se refere a um período de tempo distante do presente}
  \seealsoref{那时候}{na4 shi2 hou5}
\end{EntryWithPhonetic}

\begin{EntryWithPhonetic}{那时候}{na4 shi2 hou5}{6,7,10}{⾢、⽇、⼈}[HSK 2]
  \definition{adv.}{naquela hora; em algum momento no passado}
  \seealsoref{那时}{na4 shi2}
\end{EntryWithPhonetic}

\begin{EntryWithPhonetic}{那些}{na4 xie1}{6,8}{⾢、⼆}[HSK 1]
  \definition{pron.}{aqueles; indica duas ou mais pessoas ou coisas}
\end{EntryWithPhonetic}

\begin{EntryWithPhonetic}{那样}{na4 yang4}{6,10}{⾢、⽊}[HSK 2]
  \definition{pron.}{assim; tal; desse tipo; desse gênero; dessa natureza; desse tipo; indica a natureza, o estado, a maneira, o grau ou refere-se a uma ação ou situação específica}
\end{EntryWithPhonetic}

\begin{EntryWithPhonetic}{那咱}{na4 zan5}{6,9}{⾢、⼝}
  \definition{s.}{(informal) naquela época; então | (antigo) naquela época}
\end{EntryWithPhonetic}

\begin{EntryWithPhonetic}{哪}{na5}{9}{⼝}
  \definition{part.}{usado depois de uma palavra com a terminação -n, é equivalente a 啊}
  \seeref{na3}
  \seeref{nei3}
  \seealsoref{啊}{a5}
\end{EntryWithPhonetic}

\begin{EntryWithPhonetic}{奶}{nai3}{5}{⼥}[HSK 1]
  \definition{adj.}{bebê; infância; infantil}
  \definition[杯,滴,瓶,只,桶]{s.}{seios; mama | leite; produtos lácteos}
  \definition{v.}{amamentar; mamar}
\end{EntryWithPhonetic}

\begin{EntryWithPhonetic}{奶茶}{nai3 cha2}{5,9}{⼥、⾋}[HSK 3]
  \definition[杯]{s.}{chá com leite; chá com leite de vaca ou de ovelha}
\end{EntryWithPhonetic}

\begin{EntryWithPhonetic}{奶粉}{nai3 fen3}{5,10}{⼥、⽶}[HSK 6]
  \definition[袋,桶,罐,勺]{s.}{leite em pó}
\end{EntryWithPhonetic}

\begin{EntryWithPhonetic}{奶奶}{nai3nai5}{5,5}{⼥、⼥}[HSK 1]
  \definition[位]{s.}{avó (paterna) | vovó; avó; mulheres mais velhas | jovem senhora da casa}
\end{EntryWithPhonetic}

\begin{EntryWithPhonetic}{奶牛}{nai3 niu2}{5,4}{⼥、⽜}[HSK 6]
  \definition{s.}{vaca leiteira (ou leiteira); vaca}
\end{EntryWithPhonetic}

\begin{EntryWithPhonetic}{耐}{nai4}{9}{⽽}
  \definition{v.}{ser capaz de suportar; aguentar}
\end{EntryWithPhonetic}

\begin{EntryWithPhonetic}{耐心}{nai4xin1}{9,4}{⽽、⼼}[HSK 5]
  \definition{adj.}{paciente}
  \definition[些]{s.}{paciência; uma pessoa que não se importa com problemas e é paciente}
\end{EntryWithPhonetic}

\begin{EntryWithPhonetic}{男}{nan2}{7}{⽥}[HSK 1]
  \definition{adj.}{homem; macho; masculino (em oposição a 女)}
  \definition[个,位]{s.}{filho; menino | homem | barão (o mais baixo de cinco ordens de nobreza)}
  \seealsoref{女}{nv3}
\end{EntryWithPhonetic}

\begin{EntryWithPhonetic}{男孩儿}{nan2hai2r5}{7,9,2}{⽥、⼦、⼉}[HSK 1]
  \definition{s.}{menino; rapaz}
\end{EntryWithPhonetic}

\begin{EntryWithPhonetic}{男女}{nan2 nv3}{7,3}{⽥、⼥}[HSK 4]
  \definition{s.}{homens e mulheres; masculino e feminino}
\end{EntryWithPhonetic}

\begin{EntryWithPhonetic}{男朋友}{nan2 peng2 you5}{7,8,4}{⽥、⽉、⼜}[HSK 1]
  \definition{s.}{namorado}
\end{EntryWithPhonetic}

\begin{EntryWithPhonetic}{男人}{nan2 ren2}{7,2}{⽥、⼈}[HSK 1]
  \definition[个]{s.}{homem adulto; macho; cavalheiro | marido}
\end{EntryWithPhonetic}

\begin{EntryWithPhonetic}{男生}{nan2 sheng1}{7,5}{⽥、⽣}[HSK 1]
  \definition[个]{s.}{menino; estudante; estudante do sexo masculino; aluno do sexo masculino}
\end{EntryWithPhonetic}

\begin{EntryWithPhonetic}{男士}{nan2 shi4}{7,3}{⽥、⼠}[HSK 4]
  \definition{s.}{cavalheiro; \emph{gentleman}}
\end{EntryWithPhonetic}

\begin{EntryWithPhonetic}{男性}{nan2 xing4}{7,8}{⽥、⼼}[HSK 5]
  \definition{s.}{masculino; homem; masculinidade; em oposição a 女性}
  \seealsoref{女性}{nv3 xing4}
\end{EntryWithPhonetic}

\begin{EntryWithPhonetic}{男子}{nan2zi3}{7,3}{⽥、⼦}[HSK 3]
  \definition[个,位]{s.}{uma pessoa do sexo masculino; um homem}
\end{EntryWithPhonetic}

\begin{EntryWithPhonetic}{南}{nan2}{9}{⼗}[HSK 1]
  \definition*{s.}{Sobrenome Nan}
  \definition{s.}{sul; uma das quatro direções básicas, o lado direito quando se está de frente para o sol pela manhã (oposto ao 北) | especificamente no sul da China}
  \seealsoref{北}{bei3}
\end{EntryWithPhonetic}

\begin{EntryWithPhonetic}{南北}{nan2 bei3}{9,5}{⼗、⼔}[HSK 5]
  \definition{s.}{(território) norte e sul | (distância) de norte a sul}
\end{EntryWithPhonetic}

\begin{EntryWithPhonetic}{南边}{nan2 bian5}{9,5}{⼗、⾡}[HSK 1]
  \definition{s.}{sul; lado sul}
\end{EntryWithPhonetic}

\begin{EntryWithPhonetic}{南部}{nan2 bu4}{9,10}{⼗、⾢}[HSK 3]
  \definition{s.}{parte sul; sul | a parte sul}
\end{EntryWithPhonetic}

\begin{EntryWithPhonetic}{南方}{nan2 fang1}{9,4}{⼗、⽅}[HSK 2]
  \definition{s.}{sul; indica a direção sul | o sul; a região sul}
\end{EntryWithPhonetic}

\begin{EntryWithPhonetic}{南极}{nan2ji2}{9,7}{⼗、⽊}[HSK 5]
  \definition*{s.}{Polo Sul; Polo Antártico | Polo sul magnético}
  \definition{s.}{polo sul magnético}
\end{EntryWithPhonetic}

\begin{EntryWithPhonetic}{南京}{nan2jing1}{9,8}{⼗、⼇}
  \definition*{s.}{Nanquim, capital da província de Jiangsu, 江苏}
  \seealsoref{江苏}{jiang1su1}
\end{EntryWithPhonetic}

\begin{EntryWithPhonetic}{南面}{nan2mian4}{9,9}{⼗、⾯}
  \definition{s.}{sul | lado sul}
\end{EntryWithPhonetic}

\begin{EntryWithPhonetic}{难}{nan2}{10}{⾫}[HSK 1]
  \definition{adj.}{difícil; duro; problemático (oposto a 易) | dificilmente possível; inevitável | ruim; desagradável | problemático; improvável}
  \definition{s.}{dificuldade}
  \definition{v.}{colocar alguém em uma situação difícil}
  \seeref{nan4}
  \seealsoref{易}{yi4}
\end{EntryWithPhonetic}

\begin{EntryWithPhonetic}{难道}{nan2dao4}{10,12}{⾫、⾡}[HSK 3]
  \definition{adv.}{certamente não significa que\dots?; é possível que\dots?; não me diga\dots; poderia ser que\dots?; usado em frases interrogativas para reforçar o tom interrogativo; frequentemente usado com palavras como "吗" e "不成".}
  \seealsoref{不成}{bu4 cheng2}
  \seealsoref{吗}{ma5}
\end{EntryWithPhonetic}

\begin{EntryWithPhonetic}{难得}{nan2de2}{10,11}{⾫、⼻}[HSK 5]
  \definition{adj.}{raro; difícil de encontrar; difícil de obter ou realizar, indicando que é valioso}
  \definition{adv.}{raramente; com pouca frequência}
\end{EntryWithPhonetic}

\begin{EntryWithPhonetic}{难度}{nan2 du4}{10,9}{⾫、⼴}[HSK 3]
  \definition{s.}{dificuldade; grau de dificuldade}
\end{EntryWithPhonetic}

\begin{EntryWithPhonetic}{难过}{nan2guo4}{10,6}{⾫、⾡}[HSK 2]
  \definition{adj.}{triste; ruim; psicologicamente desconfortável | difícil; árduo}
\end{EntryWithPhonetic}

\begin{EntryWithPhonetic}{难看}{nan2 kan4}{10,9}{⾫、⽬}[HSK 2]
  \definition{adj.}{feio; desagradável à vista | vergonhoso; embaraçoso; desonroso; sem glória; sem dignidade}
\end{EntryWithPhonetic}

\begin{EntryWithPhonetic}{难受}{nan2shou4}{10,8}{⾫、⼜}[HSK 2]
  \definition{adj.}{sentir dor; sentir-se mal; sentir-se desconfortável | sentir-se mal; sentir-se infeliz; de mau humor; triste}
\end{EntryWithPhonetic}

\begin{EntryWithPhonetic}{难题}{nan2 ti2}{10,15}{⾫、⾴}[HSK 2]
  \definition[个,道]{s.}{desafio; problema difícil; questão difícil; questões difíceis de responder ou resolver}
\end{EntryWithPhonetic}

\begin{EntryWithPhonetic}{难听}{nan2 ting1}{10,7}{⾫、⼝}[HSK 2]
  \definition{adj.}{desagradável de ouvir | ofensivo; grosseiro; vulgar e desagradável | escandaloso; indigno}
\end{EntryWithPhonetic}

\begin{EntryWithPhonetic}{难忘}{nan2 wang4}{10,7}{⾫、⼼}[HSK 6]
  \definition{adj.}{memorável; inesquecível}
\end{EntryWithPhonetic}

\begin{EntryWithPhonetic}{难以}{nan2 yi3}{10,4}{⾫、⼈}[HSK 5]
  \definition{adj.}{difícil; complicado}
\end{EntryWithPhonetic}

\begin{EntryWithPhonetic}{难}{nan4}{10}{⾫}
  \definition{s.}{catástrofe; calamidade; desastre; adversidade; grande infortúnio}
  \definition{v.}{acusar; culpar}
  \seeref{nan2}
\end{EntryWithPhonetic}

\begin{EntryWithPhonetic}{难免}{nan4mian3}{10,7}{⾫、⼉}[HSK 4]
  \definition{adj.}{inevitável; difícil de evitar}
\end{EntryWithPhonetic}

\begin{EntryWithPhonetic}{孬}{nao1}{10}{⼥}
  \definition{adj.}{ruim | covarde | Dialeto: não (é) bom (contração de 不 + 好)}
  \seealsoref{不}{bu4}
  \seealsoref{好}{hao3}
\end{EntryWithPhonetic}

\begin{EntryWithPhonetic}{呶}{nao2}{8}{⼝}
  \definition{interj.}{(onomatopéia) ruído alto e contínuo}
  \definition{v.}{(literário) gritar; clamar; falar ruidosamente}
  \seealsoref{努}{nu3}
\end{EntryWithPhonetic}

\begin{EntryWithPhonetic}{脑}{nao3}{10}{⾁}
  \definition{s.}{(fisiologia) cérebro | tofu;  substância branca semelhante ao cérebro ou à medula espinhal cerebral | cabeça | a essência de um objeto}
\end{EntryWithPhonetic}

\begin{EntryWithPhonetic}{脑袋}{nao3dai5}{10,11}{⾁、⾐}[HSK 4]
  \definition[颗,个]{s.}{cabeça; a parte mais alta do corpo humano ou a parte mais alta de um animal que contém órgãos como a boca, o nariz, os olhos etc. | mente; cérebro; capacidade de pensar, lembrar, etc.}
\end{EntryWithPhonetic}

\begin{EntryWithPhonetic}{脑瓜}{nao3gua1}{10,5}{⾁、⽠}
  \definition{s.}{crânio | cérebro | cabeça | mente | mentalidade | ideia}
  \seealsoref{脑瓜子}{nao3gua1zi5}
\end{EntryWithPhonetic}

\begin{EntryWithPhonetic}{脑瓜子}{nao3gua1zi5}{10,5,3}{⾁、⽠、⼦}
  \definition{s.}{crânio | cérebro | cabeça | mente | mentalidade | ideia}
  \seealsoref{脑瓜}{nao3gua1}
\end{EntryWithPhonetic}

\begin{EntryWithPhonetic}{脑子}{nao3 zi5}{10,3}{⾁、⼦}[HSK 5]
  \definition[个]{s.}{cérebro | mente; cabeça; cérebro; inteligência; poder mental; refere-se à capacidade de pensar, memorizar, raciocinar, etc.; inteligência}
\end{EntryWithPhonetic}

\begin{EntryWithPhonetic}{闹}{nao4}{8}{⾾}[HSK 4]
  \definition{adj.}{barulhento}
  \definition{v.}{fazer barulho; provocar problemas | dar vazão (à sua raiva, ressentimento, etc.) | sofrer de; ser incomodado por; ocorrer (um desastre ou coisa ruim) | fazer;  entrar em ação | agitar; perturbar | brincar; fazer bagunça}
\end{EntryWithPhonetic}

\begin{EntryWithPhonetic}{闹钟}{nao4 zhong1}{8,9}{⾾、⾦}[HSK 4]
  \definition[个,台,只,款]{s.}{despertador; relógios capazes de tocar alarmes em horários predeterminados}
\end{EntryWithPhonetic}

\begin{EntryWithPhonetic}{那}{ne4}{6}{⾢}
  \definition{conj.}{então; nesse caso; o mesmo que 那么}
  \definition{pron.}{aquele; aquilo; pronúncia coloquial de 那 (\dpy{na4})}
  \seeref{na1}
  \seeref{na3}
  \seeref{na4}
  \seeref{nei4}
  \seeref{nuo2}
  \seealsoref{那么}{na4 me5}
\end{EntryWithPhonetic}

\begin{EntryWithPhonetic}{呢}{ne5}{8}{⼝}[HSK 1]
  \definition{part.}{usada no final de frases interrogativas (especificamente perguntas, perguntas de escolha e perguntas retóricas) para indicar um tom interrogativo | usada no final de uma frase declarativa, indica que uma ação ou situação está em andamento | usada em frases para indicar uma pausa (muitas vezes em pares) | usada no final de uma frase declarativa para confirmar um fato e convencer o interlocutor (com um tom de indicação e exagero)}
  \seeref{ni2}
\end{EntryWithPhonetic}

\begin{EntryWithPhonetic}{哪}{nei3}{9}{⼝}
  \definition{part.}{qual? (interrogativo, seguido de classificador ou numeral-classificador)}
  \seeref{na3}
  \seeref{na5}
\end{EntryWithPhonetic}

\begin{EntryWithPhonetic}{内}{nei4}{4}{⼌}[HSK 3]
  \definition*{s.}{Sobrenome Nei}
  \definition{s.}{dentro; interior; parte interna ou lateral (oposto de 外) |  (antes de um substantivo ou verbo na formação de uma palavra composta) interno | (depois de um substantivo para indicar lugar, tempo, escopo ou limites) dentro; em | coração; mente | esposa ou parentes da esposa}
  \seealsoref{外}{wai4}
\end{EntryWithPhonetic}

\begin{EntryWithPhonetic}{内部}{nei4bu4}{4,10}{⼌、⾢}[HSK 4]
  \definition{s.}{interior; dentro; interno; dentro de um determinado intervalo}
\end{EntryWithPhonetic}

\begin{EntryWithPhonetic}{内存}{nei4cun2}{4,6}{⼌、⼦}
  \definition{s.}{armazenamento interno | memória do computador | RAM (\emph{random access memory})}
  \seealsoref{随机存取存储器}{sui2ji1cun2qu3cun2chu3qi4}
  \seealsoref{随机存取记忆体}{sui2ji1cun2qu3ji4yi4ti3}
\end{EntryWithPhonetic}

\begin{EntryWithPhonetic}{内地}{nei4 di4}{4,6}{⼌、⼟}[HSK 6]
  \definition{s.}{interior; sertão | China continental (RPC excluindo Hong Kong e Macau, mas incluindo ilhas como Hainan) | Japão (usado em Taiwan durante a colonização japonesa)}
\end{EntryWithPhonetic}

\begin{EntryWithPhonetic}{内科}{nei4ke1}{4,9}{⼌、⽲}[HSK 4]
  \definition{s.}{medicina geral; clínica geral; clínica médica}
\end{EntryWithPhonetic}

\begin{EntryWithPhonetic}{内燃机}{nei4ran2ji1}{4,16,6}{⼌、⽕、⽊}
  \definition{s.}{motor de combustão interna}
\end{EntryWithPhonetic}

\begin{EntryWithPhonetic}{内容}{nei4rong2}{4,10}{⼌、⼧}[HSK 3]
  \definition[份,个,项]{s.}{conteúdo; substância; a substância ou significado contido em algo}
\end{EntryWithPhonetic}

\begin{EntryWithPhonetic}{内外}{nei4 wai4}{4,5}{⼌、⼣}[HSK 6]
  \definition{s.}{dentro e fora; nacional e estrangeiro; interno e externo | ao redor; aproximadamente; número aproximado de exibição}
\end{EntryWithPhonetic}

\begin{EntryWithPhonetic}{内心}{nei4 xin1}{4,4}{⼌、⼼}[HSK 3]
  \definition{s.}{coração; interior; íntimo do ser}
\end{EntryWithPhonetic}

\begin{EntryWithPhonetic}{内省}{nei4xing3}{4,9}{⼌、⽬}
  \definition{s.}{introspecção}
  \definition{v.}{refletir sobre si mesmo}
\end{EntryWithPhonetic}

\begin{EntryWithPhonetic}{内衣}{nei4 yi1}{4,6}{⼌、⾐}[HSK 6]
  \definition[件,个]{s.}{roupa íntima}
\end{EntryWithPhonetic}

\begin{EntryWithPhonetic}{内在}{nei4zai4}{4,6}{⼌、⼟}[HSK 5]
  \definition{adj.}{intrínseco; algo que existe em si mesmo, mas que não pode ser descoberto através da observação direta | interno; imanente; difícil de perceber}
\end{EntryWithPhonetic}

\begin{EntryWithPhonetic}{内资}{nei4 zi1}{4,10}{⼌、⾙}
  \definition{s.}{capital nacional; financiamento interno; investimento de fontes nacionais (em oposição a 外资)}
  \seealsoref{外资}{wai4 zi1}
\end{EntryWithPhonetic}

\begin{EntryWithPhonetic}{那}{nei4}{6}{⾢}
  \definition{conj.}{então; o mesmo que 那么}
  \definition{pron.}{aquele; aquilo; A pronúncia coloquial de 那 (\dpy{na4})}
  \seeref{na1}
  \seeref{na3}
  \seeref{na4}
  \seeref{ne4}
  \seeref{nuo2}
  \seealsoref{那么}{na4 me5}
\end{EntryWithPhonetic}

\begin{EntryWithPhonetic}{能}{neng2}{10}{⾁}[HSK 1]
  \definition*{s.}{Sobrenome Neng}
  \definition{adv.}{talvez}
  \definition{s.}{habilidade; capacidade; competência | potência; energia; em física, refere-se à energia}
  \definition{v.}{poder fazer; ser capaz de | ser possível | entre 不 \dots 不 para expressar obrigação, certeza ou grande probabilidade | poder; ter permissão para | ser bom em fazer algo | permitir}
\end{EntryWithPhonetic}

\begin{EntryWithPhonetic}{能不能}{neng2 bu4 neng2}{10,4,10}{⾁、⼀、⾁}[HSK 3]
  \definition{adv.}{pode ou não pode\dots?}
\end{EntryWithPhonetic}

\begin{EntryWithPhonetic}{能否}{neng2 fou3}{10,7}{⾁、⼝}[HSK 6]
  \definition{adv.}{é possível; se ou não; pode ou não pode; Você consegue?; expressa dúvida, frequentemente usado em perguntas de sim ou não}
\end{EntryWithPhonetic}

\begin{EntryWithPhonetic}{能干}{neng2gan4}{10,3}{⾁、⼲}[HSK 4]
  \definition{adj.}{apto; capaz; competente}
\end{EntryWithPhonetic}

\begin{EntryWithPhonetic}{能够}{neng2 gou4}{10,11}{⾁、⼣}[HSK 2]
  \definition{v.}{poder; ser capaz de; indica que possui uma determinada capacidade ou que atingiu um determinado nível de eficiência | poder; ser capaz de; indica que algo é permitido sob certas condições ou por motivos razoáveis}
\end{EntryWithPhonetic}

\begin{EntryWithPhonetic}{能力}{neng2li4}{10,2}{⾁、⼒}[HSK 3]
  \definition[个,种]{s.}{habilidade; capacidade; aptidão; as condições subjetivas para ser competente para uma tarefa}
\end{EntryWithPhonetic}

\begin{EntryWithPhonetic}{能量}{neng2liang4}{10,12}{⾁、⾥}[HSK 5]
  \definition[种]{s.}{energia; quantidade de energia; Uma grandeza física que mede a capacidade da matéria de realizar trabalho | capacidade; competências; capacidade e papel que uma pessoa pode desempenhar}
\end{EntryWithPhonetic}

\begin{EntryWithPhonetic}{能上能下}{neng2shang4neng2xia4}{10,3,10,3}{⾁、⼀、⾁、⼀}
  \definition{s.}{pronto para aceitar qualquer trabalho, alto ou baixo}
\end{EntryWithPhonetic}

\begin{EntryWithPhonetic}{呢}{ni2}{8}{⼝}
  \definition{s.}{(tecido feito de) lã; tecido de lã (para roupas pesadas); tecido de lã pesada; revestimento ou roupa de lã}
  \seeref{ne5}
\end{EntryWithPhonetic}

\begin{EntryWithPhonetic}{泥}{ni2}{8}{⽔}[HSK 6]
  \definition*{s.}{Sobrenome Ni}
  \definition{s.}{lama; atoleiro | pasta ou polpa; amassado | qualquer matéria pastosa; purê de vegetais ou frutas}
  \seeref{ni4}
\end{EntryWithPhonetic}

\begin{EntryWithPhonetic}{泥潭}{ni2tan2}{8,15}{⽔、⽔}
  \definition{s.}{atoleiro | lamaçal | charco | pântano}
\end{EntryWithPhonetic}

\begin{EntryWithPhonetic}{你}{ni3}{7}{⼈}[HSK 1]
  \definition{pron.}{você (segunda pessoa do singular); refere-se à pessoa com quem se está conversando | (referindo-se a qualquer pessoa) você; um; qualquer um | com 我 ou 你 em estruturas paralelas para indicar várias ou muitas pessoas se comportando da mesma maneira}
  \seealsoref{您}{nin2}
  \seealsoref{我}{wo3}
\end{EntryWithPhonetic}

\begin{EntryWithPhonetic}{你的}{ni3 de5}{7,8}{⼈、⽩}
  \definition{pron.}{seu}
\end{EntryWithPhonetic}

\begin{EntryWithPhonetic}{你好}{ni3hao3}{7,6}{⼈、⼥}
  \definition{interj.}{Olá! | Oi!}
\end{EntryWithPhonetic}

\begin{EntryWithPhonetic}{你们}{ni3men5}{7,5}{⼈、⼈}[HSK 1]
  \definition{pron.}{você (segunda pessoa do plural); refere-se a mais de uma pessoa ou a várias pessoas, incluindo a outra parte}
\end{EntryWithPhonetic}

\begin{EntryWithPhonetic}{你们的}{ni3men5 de5}{7,5,8}{⼈、⼈、⽩}
  \definition{pron.}{vossos}
\end{EntryWithPhonetic}

\begin{EntryWithPhonetic}{伲}{ni4}{7}{⼈}
  \definition{pron.}{Dialeto: eu; nós; meu; nosso}
  \seealsoref{你}{ni3}
\end{EntryWithPhonetic}

\begin{EntryWithPhonetic}{泥}{ni4}{8}{⽔}
  \definition{adj.}{fanático; teimoso; obstinado; cabeçudo}
  \definition{v.}{cobrir ou rebocar com gesso, massa de vidraceiro, etc.}
  \seeref{ni2}
\end{EntryWithPhonetic}

\begin{EntryWithPhonetic}{逆}{ni4}{9}{⾡}
  \definition{adj.}{contrário (oposto a 顺) ; contra; oposto; inverso | traidor; rebelde}
  \definition{adv.}{antecipadamente; com antecedência}
  \definition{s.}{traidor; rebelde}
  \definition{v.}{ir contra; opor-se; desobedecer; resistir; desafiar (oposto a 顺) | (literário) saudar; cumprimentar}
  \seealsoref{顺}{shun4}
\end{EntryWithPhonetic}

\begin{EntryWithPhonetic}{逆境}{ni4jing4}{9,14}{⾡、⼟}
  \definition{s.}{adversidade | tribulação}
\end{EntryWithPhonetic}

\begin{EntryWithPhonetic}{年}{nian2}{6}{⼲}[HSK 1]
  \definition*{s.}{Sobrenome Nian}
  \definition{clas.}{ano; usado para calcular o número de anos}
  \definition{s.}{ano | idade | um período (época) da história | colheita anual | Ano Novo | artigos para o dia de Ano Novo | um período da vida de uma pessoa; fases da vida humana divididas por idade}
\end{EntryWithPhonetic}

\begin{EntryWithPhonetic}{年初}{nian2 chu1}{6,7}{⼲、⾐}[HSK 3]
  \definition{s.}{o começo do ano; os primeiros dias do ano}
\end{EntryWithPhonetic}

\begin{EntryWithPhonetic}{年代}{nian2dai4}{6,5}{⼲、⼈}[HSK 3]
  \definition[个]{s.}{idade; anos; tempo; um período de tempo com características distintas na história | uma década de um século; período de dez anos}
\end{EntryWithPhonetic}

\begin{EntryWithPhonetic}{年底}{nian2 di3}{6,8}{⼲、⼴}[HSK 3]
  \definition[个]{s.}{fim de ano; o fim do ano; geralmente os últimos dias de dezembro ou o fim do ano}
\end{EntryWithPhonetic}

\begin{EntryWithPhonetic}{年度}{nian2du4}{6,9}{⼲、⼴}[HSK 5]
  \definition{s.}{ano; de acordo com a natureza e as necessidades de um negócio, há um prazo de doze meses com data de início e término definidas}
\end{EntryWithPhonetic}

\begin{EntryWithPhonetic}{年货}{nian2huo4}{6,8}{⼲、⾙}
  \definition{s.}{mercadorias vendidas no Ano Novo Chinês}
\end{EntryWithPhonetic}

\begin{EntryWithPhonetic}{年级}{nian2ji2}{6,6}{⼲、⽷}[HSK 2]
  \definition[个]{s.}{série; ano; níveis divididos de acordo com o tempo de estudo dos alunos na escola}
\end{EntryWithPhonetic}

\begin{EntryWithPhonetic}{年纪}{nian2ji4}{6,6}{⼲、⽷}[HSK 3]
  \definition[把,个]{s.}{idade (de uma pessoa)}
\end{EntryWithPhonetic}

\begin{EntryWithPhonetic}{年龄}{nian2ling2}{6,13}{⼲、⿒}[HSK 5]
  \definition[个,段]{s.}{idade; animais, plantas e outros seres vivos vivem e crescem no mundo durante um determinado número de anos}
\end{EntryWithPhonetic}

\begin{EntryWithPhonetic}{年前}{nian2 qian2}{6,9}{⼲、⼑}[HSK 5]
  \definition{s.}{(pouco) antes da virada do ano | antes do final do ano | antes do ano novo}
\end{EntryWithPhonetic}

\begin{EntryWithPhonetic}{年轻}{nian2qing1}{6,9}{⼲、⾞}[HSK 2]
  \definition{adj.}{jovem; não muito velho (geralmente se refere a pessoas entre 10 e 20 anos)}
\end{EntryWithPhonetic}

\begin{EntryWithPhonetic}{碾}{nian3}{15}{⽯}
  \definition[台,个]{s.}{rolo e mó; rolo de pedra | rolo compressor}
  \definition{v.}{moer ou descascar com um rolo; esmagar | (literário) cortar e polir (jade, vidro, etc.) | achatar | pisar; pisotear, 轧}
  \seealsoref{辗}{zhan3}
\end{EntryWithPhonetic}

\begin{EntryWithPhonetic}{碾碎}{nian3sui4}{15,13}{⽯、⽯}
  \definition{v.}{pulverizar | esmagar}
\end{EntryWithPhonetic}

\begin{EntryWithPhonetic}{廿}{nian4}{4}{⼶}
  \definition{num.}{(dialeto) vinte; 20}
\end{EntryWithPhonetic}

\begin{EntryWithPhonetic}{念}{nian4}{8}{⼼}[HSK 3]
  \definition*{s.}{Sobrenome Nian}
  \definition{num.}{vinte; 20; capitalização do número 廿}
  \definition{s.}{ideia; pensamento; pensamentos ou intenções internas}
  \definition{v.}{ler em voz alta | estudar; frequentar a escola | considerar; levar em conta | sentir falta; pensar em; pensar sobre; pensar frequentemente sobre}
  \seealsoref{廿}{nian4}
\end{EntryWithPhonetic}

\begin{EntryWithPhonetic}{鸟}{niao3}{5}{⿃}[HSK 2][Kangxi 196]
  \definition*{s.}{Sobrenome Niao}
  \definition[只,群]{s.}{pássaro; ave}
  \seeref{diao3}
\end{EntryWithPhonetic}

\begin{EntryWithPhonetic}{鸟儿}{niao3r5}{5,2}{⿃、⼉}
  \definition[只]{s.}{pássaro | ave}
\end{EntryWithPhonetic}

\begin{EntryWithPhonetic}{尿}{niao4}{7}{⼫}
  \definition[泡]{s.}{urina}
  \definition{v.}{urinar}
  \seeref{sui1}
\end{EntryWithPhonetic}

\begin{EntryWithPhonetic}{您}{nin2}{11}{⼼}[HSK 1]
  \definition{pron.}{você; a forma de tratamento respeitosa da segunda pessoa do singular 你}
  \seealsoref{你}{ni3}
\end{EntryWithPhonetic}

\begin{EntryWithPhonetic}{宁}{ning2}{5}{⼧}
  \definition*{s.}{Região Autônoma de Ningxia Hui, abreviação de 宁夏回族自治区 | outro nome para Nanquim, 南京 | Sobrenome Ning}
  \definition{adj.}{calmo, pacífico, sereno | saudável}
  \definition{v.}{Literário: fazer uma visita (aos pais ou aos mais velhos); | Literário: pacificar; apaziguar}
  \seeref{ning4}
  \seealsoref{南京}{nan2jing1}
  \seealsoref{宁夏回族自治区}{ning2xia4 hui2zu2 zi4zhi4qu1}
\end{EntryWithPhonetic}

\begin{EntryWithPhonetic}{宁静}{ning2 jing4}{5,14}{⼧、⾭}[HSK 4]
  \definition{adj.}{calmo; tranquilo; pacífico}
\end{EntryWithPhonetic}

\begin{EntryWithPhonetic}{宁夏回族自治区}{ning2xia4 hui2zu2 zi4zhi4qu1}{5,10,6,11,6,8,4}{⼧、⼢、⼞、⽅、⾃、⽔、⼖}
  \definition*{s.}{Região Autônoma de Ningxia Hui}
\end{EntryWithPhonetic}

\begin{EntryWithPhonetic}{拧}{ning2}{8}{⼿}
  \definition{v.}{torcer | beliscar; torcer a pele com os dedos e virá-la com força}
  \seeref{ning3}
  \seeref{ning4}
\end{EntryWithPhonetic}

\begin{EntryWithPhonetic}{柠}{ning2}{9}{⽊}
  \definition{s.}{limão}
\end{EntryWithPhonetic}

\begin{EntryWithPhonetic}{柠檬}{ning2meng2}{9,17}{⽊、⽊}
  \definition[个,片,只]{s.}{limão}
\end{EntryWithPhonetic}

\begin{EntryWithPhonetic}{拧}{ning3}{8}{⼿}
  \definition{adj.}{errado; equivocado; de cabeça para baixo; oposto}
  \definition{v.}{torcer; parafusar | divergir; discordar; estar em desacordo}
  \seeref{ning2}
  \seeref{ning4}
\end{EntryWithPhonetic}

\begin{EntryWithPhonetic}{拧开}{ning3kai1}{8,4}{⼿、⼶}
  \definition{v.}{desaparafusar | desatarrachar | torcer (uma tampa) | abrir (uma torneira) | ligar (girando um botão) | girar (maçaneta da porta)}
\end{EntryWithPhonetic}

\begin{EntryWithPhonetic}{宁}{ning4}{5}{⼧}
  \definition{conj.}{mais\dots do que\dots, melhor\dots do que\dots}
  \seeref{ning2}
\end{EntryWithPhonetic}

\begin{EntryWithPhonetic}{宁可}{ning4ke3}{5,5}{⼧、⼝}
  \definition{conj.}{mais\dots do que\dots | melhor\dots do que\dots}
\end{EntryWithPhonetic}

\begin{EntryWithPhonetic}{宁可……也不……}{ning4ke3 ye3bu4}{5,5,3,4}{⼧、⼝、⼄、⼀}
  \definition{conj.}{preferiria\dots do que\dots}
\end{EntryWithPhonetic}

\begin{EntryWithPhonetic}{宁可……也要……}{ning4ke3 ye3yao4}{5,5,3,9}{⼧、⼝、⼄、⾑}
  \definition{conj.}{mesmo que tenhamos que\dots nós iremos\dots}
\end{EntryWithPhonetic}

\begin{EntryWithPhonetic}{宁肯}{ning4ken3}{5,8}{⼧、⾁}
  \definition{conj.}{mais\dots do que\dots, melhor\dots do que\dots}
\end{EntryWithPhonetic}

\begin{EntryWithPhonetic}{宁愿}{ning4yuan4}{5,14}{⼧、⽕}
  \definition{conj.}{mais\dots do que\dots, melhor\dots do que\dots}
\end{EntryWithPhonetic}

\begin{EntryWithPhonetic}{拧}{ning4}{8}{⼿}
  \definition{adj.}{teimoso}
  \seeref{ning2}
  \seeref{ning3}
\end{EntryWithPhonetic}

\begin{EntryWithPhonetic}{牛}{niu2}{4}{⽜}[HSK 3,5][Kangxi 93]
  \definition*{s.}{Sobrenome Niu}
  \definition{adj.}{muito capaz ou bom; descreve pessoas ou coisas como sendo muito capazes, muito competentes | teimoso; arrogante; descreve uma pessoa que é muito orgulhosa ou muito insistente em suas opiniões, difícil de mudar}
  \definition{clas.}{Newton (medida física de força)}
  \definition[头]{s.}{gado; boi | niu (nona das vinte e oito constelações em que a esfera celeste foi dividida, consistindo de seis estrelas, três em Áries e três em Sagitário)}
\end{EntryWithPhonetic}

\begin{EntryWithPhonetic}{牛顿}{niu2dun4}{4,10}{⽜、⾴}
  \definition*{s.}{Newton (nome) | N; Newton, unidade de força do SI}
\end{EntryWithPhonetic}

\begin{EntryWithPhonetic}{牛郎织女}{niu2 lang2 zhi1nv3}{4,8,8,3}{⽜、⾢、⽷、⼥}
  \definition*{s.}{Vaqueiro e Tecelã (personagens de contos folclóricos) | Altair e Vega (estrelas)}[我们这些牛郎织女都恨透了那条无情的“天河”。===Nós, o Vaqueiro e a Tecelã, odiamos a implacável ``Via Láctea''.]
  \definition{s.}{marido e mulher que vivem longe um do outro}
\end{EntryWithPhonetic}

\begin{EntryWithPhonetic}{牛奶}{niu2nai3}{4,5}{⽜、⼥}[HSK 1]
  \definition[杯,袋,瓶,盒,箱,桶]{s.}{leite}
\end{EntryWithPhonetic}

\begin{EntryWithPhonetic}{牛人}{niu2ren2}{4,2}{⽜、⼈}
  \definition{s.}{(coloquial) o cara | verdadeiro especialista | \emph{badass}}
\end{EntryWithPhonetic}

\begin{EntryWithPhonetic}{牛肉}{niu2rou4}{4,6}{⽜、⾁}
  \definition{s.}{carne de vaca | bife}
\end{EntryWithPhonetic}

\begin{EntryWithPhonetic}{牛仔裤}{niu2zai3ku4}{4,5,12}{⽜、⼈、⾐}[HSK 5]
  \definition[条]{s.}{calças jeans; calças geralmente feitas de tecido jeans azul grosso}
\end{EntryWithPhonetic}

\begin{EntryWithPhonetic}{扭}{niu3}{7}{⼿}[HSK 6]
  \definition{v.}{virar-se; girar | torcer; girar | torcer; luxar | rolar; balançar (ao caminhar) | agarrar; pegar;  lutar com}
\end{EntryWithPhonetic}

\begin{EntryWithPhonetic}{农}{nong2}{6}{⼍}
  \definition*{s.}{Sobrenome Nong}
  \definition{s.}{agricultura; criação de animais | camponês; fazendeiro}
\end{EntryWithPhonetic}

\begin{EntryWithPhonetic}{农产品}{nong2 chan3 pin3}{6,6,9}{⼍、⼇、⼝}[HSK 5]
  \definition[批]{s.}{produtos agrícolas}
\end{EntryWithPhonetic}

\begin{EntryWithPhonetic}{农村}{nong2cun1}{6,7}{⼍、⽊}[HSK 3]
  \definition{s.}{aldeia; campo; área rural; locais onde vivem os trabalhadores principalmente dedicados à produção agrícola}
\end{EntryWithPhonetic}

\begin{EntryWithPhonetic}{农民}{nong2min2}{6,5}{⼍、⽒}[HSK 3]
  \definition[个,位,名,些]{s.}{fazendeiro; camponês; campesinato; trabalhadores que participam da produção agrícola há muito tempo}
\end{EntryWithPhonetic}

\begin{EntryWithPhonetic}{农业}{nong2ye4}{6,5}{⼍、⼀}[HSK 3]
  \definition{s.}{agricultura}
\end{EntryWithPhonetic}

\begin{EntryWithPhonetic}{浓}{nong2}{9}{⽔}[HSK 4]
  \definition{adj.}{denso; espesso; concentrado; um líquido ou gás que contém mais de um determinado ingrediente | grande; forte; profundo (de grau ou extensão) | profundo; (algumas cores) escuro}
\end{EntryWithPhonetic}

\begin{EntryWithPhonetic}{弄}{nong4}{7}{⼶}[HSK 2]
  \definition{v.}{fazer, realizar; tratar; organizar | obter; buscar; tentar conseguir; encontrar uma maneira de conseguir | brincar com; enganar | pregar uma peça; brincar; manipular | mexer com; perturbar}
  \seeref{long4}
\end{EntryWithPhonetic}

\begin{EntryWithPhonetic}{努}{nu3}{7}{⼒}
  \definition{v.}{(coloquial) aplicar (a força de alguém); exercer (o esforço de alguém) | (dialeto) machucar-se por esforço excessivo | projetar-se; inchar | aplicar (força); exercer (esforço); usar}
  \seealsoref{呶}{nao2}
\end{EntryWithPhonetic}

\begin{EntryWithPhonetic}{努力}{nu3li4}{7,2}{⼒、⼒}[HSK 2]
  \definition{adj.}{extenuante; árduo | diligente; trabalhador; quem faz as coisas com o máximo de capacidade ou esforço possível}
  \definition{s.}{esforço; tentativa; fazer o melhor possível}
  \definition{v.}{fazer grandes esforços; esforçar-se; empenhar-se | esforçar-se; usar toda a força possível}
\end{EntryWithPhonetic}

\begin{EntryWithPhonetic}{怒}{nu4}{9}{⼼}
  \definition{adj.}{zangado; furioso | feroz; forte; descreve um forte impulso}
  \definition{adv.}{com força; vigorosamente; dinamicamente | com raiva}
  \definition{s.}{raiva; fúria}
  \definition{v.}{enfurecer-se; ficar com raiva}
\end{EntryWithPhonetic}

\begin{EntryWithPhonetic}{怒放}{nu4fang4}{9,8}{⼼、⽅}
  \definition{v.}{florescer em plena floração}
\end{EntryWithPhonetic}

\begin{EntryWithPhonetic}{怒骂}{nu4ma4}{9,9}{⼼、⾺}
  \definition{v.}{praguejar de raiva}
\end{EntryWithPhonetic}

\begin{EntryWithPhonetic}{暖}{nuan3}{13}{⽇}[HSK 5]
  \definition{adj.}{caloroso; cordial}
  \definition{v.}{aquecer; esquentar; aquecer algo ou aquecer o corpo}
\end{EntryWithPhonetic}

\begin{EntryWithPhonetic}{暖和}{nuan3huo5}{13,8}{⽇、⼝}[HSK 3]
  \definition{adj.}{morno; nem frio nem quente}
  \definition{v.}{aquecer; esquentar}
\end{EntryWithPhonetic}

\begin{EntryWithPhonetic}{暖气}{nuan3qi4}{13,4}{⽇、⽓}[HSK 4]
  \definition[个,种]{s.}{aquecedor; aquecimento; aquecimento central}
\end{EntryWithPhonetic}

\begin{EntryWithPhonetic}{那}{nuo2}{6}{⾢}
  \definition*{s.}{Sobrenome Nuo}
  \seeref{na1}
  \seeref{na3}
  \seeref{na4}
  \seeref{ne4}
  \seeref{nei4}
\end{EntryWithPhonetic}

\begin{EntryWithPhonetic}{诺}{nuo4}{10}{⾔}
  \definition*{s.}{Sobrenome Nuo}
  \definition{interj.}{Sim!}
  \definition{v.}{prometer}
\end{EntryWithPhonetic}

\begin{EntryWithPhonetic}{诺贝尔奖}{nuo4bei4'er3 jiang3}{10,4,5,9}{⾔、⾙、⼩、⼤}
  \definition*{s.}{Prêmio Nobel}
\end{EntryWithPhonetic}

\begin{EntryWithPhonetic}{诺奖}{nuo4jiang3}{10,9}{⾔、⼤}
  \definition*{s.}{Prêmio Nobel, abreviação de 诺贝尔奖}
  \seealsoref{诺贝尔奖}{nuo4bei4'er3 jiang3}
\end{EntryWithPhonetic}

\begin{EntryWithPhonetic}{女}{nv3}{3}{⼥}[HSK 1][Kangxi 38]
  \definition{adj.}{mulher; feminino (em oposição a 男) | fêmea (de certos animais)}
  \definition{s.}{menina; filha | nü, uma das mansões lunares | mulher}
  \seealsoref{男}{nan2}
\end{EntryWithPhonetic}

\begin{EntryWithPhonetic}{女儿}{nv3'er2}{3,2}{⼥、⼉}[HSK 1]
  \definition[个]{s.}{menina; filha}
  \seealsoref{儿子}{er2zi5}
\end{EntryWithPhonetic}

\begin{EntryWithPhonetic}{女孩}{nv3hai2}{3,9}{⼥、⼦}
  \definition{s.}{menina | garota}
\end{EntryWithPhonetic}

\begin{EntryWithPhonetic}{女孩儿}{nv3 hai2r5}{3,9,2}{⼥、⼦、⼉}[HSK 1]
  \definition{s.}{garota; menina; atualmente também se refere a mulher adolescente | filha}
\end{EntryWithPhonetic}

\begin{EntryWithPhonetic}{女朋友}{nv3 peng2 you5}{3,8,4}{⼥、⽉、⼜}[HSK 1]
  \definition{s.}{namorada}
\end{EntryWithPhonetic}

\begin{EntryWithPhonetic}{女人}{nv3 ren2}{3,2}{⼥、⼈}[HSK 1]
  \definition[个,位]{s.}{mulher adulta}
\end{EntryWithPhonetic}

\begin{EntryWithPhonetic}{女生}{nv3 sheng1}{3,5}{⼥、⽣}[HSK 1]
  \definition[个]{s.}{estudante; aluna; estudante do sexo feminino | menina; jovem mulher}
\end{EntryWithPhonetic}

\begin{EntryWithPhonetic}{女士}{nv3shi4}{3,3}{⼥、⼠}[HSK 4]
  \definition{pron.}{Sra.; Senhorita; Senhora; título honorífico para mulheres (agora usado em contextos diplomáticos)}
  \definition[位,名,个,些]{s.}{senhora; madame}
\end{EntryWithPhonetic}

\begin{EntryWithPhonetic}{女王}{nv3wang2}{3,4}{⼥、⽟}
  \definition{s.}{rainha}
\end{EntryWithPhonetic}

\begin{EntryWithPhonetic}{女性}{nv3 xing4}{3,8}{⼥、⼼}[HSK 5]
  \definition[个,位,名]{s.}{mulher; feminino; feminilidade; em oposição a 男性}
  \seealsoref{男性}{nan2 xing4}
\end{EntryWithPhonetic}

\begin{EntryWithPhonetic}{女婿}{nv3xu5}{3,12}{⼥、⼥}
  \definition{s.}{marido da filha}
\end{EntryWithPhonetic}

\begin{EntryWithPhonetic}{女子}{nv3 zi3}{3,3}{⼥、⼦}[HSK 3]
  \definition[位,名,个]{s.}{mulher; feminino; pessoa do sexo feminino}
\end{EntryWithPhonetic}

%%%%% EOF %%%%%

