%%%
%%% H
%%%

\section*{H}\addcontentsline{toc}{section}{H}

\begin{EntryWithPhonetic}{哈}{ha1}{9}{⼝}
  \definition{interj.}{Onomatopéia: ha; descreve o riso, usado principalmente em duplicata | indica orgulho ou satisfação, frequentemente usado de forma duplicada}
  \definition{v.}{soprar; expirar (com a boca aberta) | dobrar}
  \seeref{ha3}
  \seealsoref{哈哈}{ha1 ha1}
\end{EntryWithPhonetic}

\begin{EntryWithPhonetic}{哈哈}{ha1 ha1}{9,9}{⼝、⼝}[HSK 3]
  \definition{expr.}{(onomatopéia)  ha ha; o som de uma gargalhada}
\end{EntryWithPhonetic}

\begin{EntryWithPhonetic}{哈马斯}{ha1ma3si1}{9,3,12}{⼝、⾺、⽄}
  \definition*{s.}{Hamas (Grupo Palestino)}
\end{EntryWithPhonetic}

\begin{EntryWithPhonetic}{哈}{ha3}{9}{⼝}
  \definition*{s.}{Sobrenome Ha}
  \definition{v.}{repreender}
  \seeref{ha1}
\end{EntryWithPhonetic}

\begin{EntryWithPhonetic}{咳}{hai1}{9}{⼝}
  \definition{interj.}{expressa tristeza, arrependimento ou espanto}
  \seeref{ke2}
\end{EntryWithPhonetic}

\begin{EntryWithPhonetic}{还}{hai2}{7}{⾡}[HSK 1]
  \definition{adv.}{ainda; indica que a ação ou estado permanece inalterado, equivalente a 仍然 | também; além disso; em adição; indica que há um aumento ou suplemento além do escopo já indicado | ainda mais; usado com 比 para indicar que as características e o grau das coisas comparadas aumentaram, o que é equivalente a 更加 razoavelmente; medianamente; usado antes de um adjetivo, indica que algo atinge apenas o nível mínimo exigido | mesmo; usado na primeira parte da frase como complemento, e na segunda parte como conclusão, equivalente a 尚且 | que expressa realização ou descoberta; expressa surpresa por algo que não se esperava, mas que acabou acontecendo | tão cedo quanto; por um curto período de tempo; indica que já era assim há muito tempo | para dar ênfase; para reforçar o tom}
  \seeref{huan2}
  \seealsoref{比}{bi3}
  \seealsoref{更加}{geng4 jia1}
  \seealsoref{仍然}{reng2ran2}
  \seealsoref{尚且}{shang4 qie3}
\end{EntryWithPhonetic}

\begin{EntryWithPhonetic}{还是}{hai2shi5}{7,9}{⾡、⽇}[HSK 1]
  \definition{adv.}{ainda; ainda assim; não é a continuação de um determinado estado, fenômeno ou ação; o resultado é o mesmo de antes, sem mudanças  |que expressa uma preferência por uma alternativa; expressa comparação ou escolha feita após consideração cuidadosa, frequentemente usado para fazer sugestões | que expressa realização ou descoberta; indica que o resultado final foi inesperado}
  \definition{conj.}{ou (somente para frases interrogativas); indica várias opções, geralmente usado em perguntas | tudo; se; não importa; independentemente de; significa que, independentemente das mudanças que ocorram, o resultado permanecerá o mesmo}
\end{EntryWithPhonetic}

\begin{EntryWithPhonetic}{还有}{hai2 you3}{7,6}{⾡、⽉}[HSK 1]
  \definition{adv.}{também; ainda; além disso; então novamente; enfatizar as partes complementares, excedentes ou não mencionadas além do que já é conhecido}
\end{EntryWithPhonetic}

\begin{EntryWithPhonetic}{孩}{hai2}{9}{⼦}
  \definition[个]{s.}{criança}
\end{EntryWithPhonetic}

\begin{EntryWithPhonetic}{孩子}{hai2 zi5}{9,3}{⼦、⼦}[HSK 1]
  \definition[个]{s.}{criança; crianças; pessoas com idade entre alguns anos ou na adolescência, geralmente com menos de 14 anos | crianças; filho ou filha}
\end{EntryWithPhonetic}

\begin{EntryWithPhonetic}{海}{hai3}{10}{⽔}[HSK 2]
  \definition*{s.}{Sobrenome Hai}
  \definition{adj.}{extragrande; de grande capacidade; descreve capacidade, tom de voz, etc.}
  \definition{adv.}{aleatoriamente; sem rumo; sem limites; sem restrições}
  \definition[片]{s.}{mar; grande lago; a parte do oceano próxima à costa, alguns grandes lagos também são chamados de mar | grande número de pessoas ou coisas reunidas; metáfora para muitas coisas semelhantes que formam um grande conjunto}
\end{EntryWithPhonetic}

\begin{EntryWithPhonetic}{海岸}{hai3'an4}{10,8}{⽔、⼭}[HSK 7-9]
  \definition[段]{s.}{litoral; costa; praia}
\end{EntryWithPhonetic}

\begin{EntryWithPhonetic}{海拔}{hai3ba2}{10,8}{⽔、⼿}[HSK 7-9]
  \definition{s.}{altitude; altura em relação ao nível médio do mar}
\end{EntryWithPhonetic}

\begin{EntryWithPhonetic}{海报}{hai3 bao4}{10,7}{⽔、⼿}[HSK 6]
  \definition[张,份,幅]{s.}{pôster; cartaz; cartazes anunciando apresentações culturais, exibições de filmes ou competições esportivas, etc.}
\end{EntryWithPhonetic}

\begin{EntryWithPhonetic}{海边}{hai3 bian1}{10,5}{⽔、⾡}[HSK 2]
  \definition{s.}{praia; costa; litoral; orla marítima; a parte marginal do oceano e as grandes áreas de água salgada cercadas por terra firme, onde a terra e a água se encontram, formam a costa}
\end{EntryWithPhonetic}

\begin{EntryWithPhonetic}{海滨}{hai3bin1}{10,13}{⽔、⽔}[HSK 7-9]
  \definition{s.}{praia; beira-mar; litoral; um lugar perto do mar}
\end{EntryWithPhonetic}

\begin{EntryWithPhonetic}{海盗}{hai3dao4}{10,11}{⽔、⽫}[HSK 7-9]
  \definition{s.}{pirata; viajante do mar | bucaneiro; viking; pirata}
\end{EntryWithPhonetic}

\begin{EntryWithPhonetic}{海底}{hai3 di3}{10,8}{⽔、⼴}[HSK 6]
  \definition{s.}{fundo do mar; fundo do oceano; solo oceânico}
\end{EntryWithPhonetic}

\begin{EntryWithPhonetic}{海风}{hai3feng1}{10,4}{⽔、⾵}
  \definition{s.}{brisa do mar | vento que vem do mar}
\end{EntryWithPhonetic}

\begin{EntryWithPhonetic}{海关}{hai3guan1}{10,6}{⽔、⼋}[HSK 3]
  \definition[个]{s.}{alfândega; órgão administrativo nacional, sua principal função é supervisionar e inspecionar os bens e meios de transporte que entram e saem do país, cobrar impostos alfandegários e reprimir o contrabando}
\end{EntryWithPhonetic}

\begin{EntryWithPhonetic}{海军}{hai3 jun1}{10,6}{⽔、⼍}[HSK 6]
  \definition[支,名,位,个]{s.}{marinha; o exército que luta no mar geralmente é composto por navios de superfície, submarinos, aviação naval, fuzileiros navais e outros ramos, além de diversas forças profissionais}
\end{EntryWithPhonetic}

\begin{EntryWithPhonetic}{海浪}{hai3 lang4}{10,10}{⽔、⽔}[HSK 6]
  \definition{s.}{ondas do mar}
\end{EntryWithPhonetic}

\begin{EntryWithPhonetic}{海里}{hai3li3}{10,7}{⽔、⾥}
  \definition{s.}{milha náutica}
\end{EntryWithPhonetic}

\begin{EntryWithPhonetic}{海量}{hai3liang4}{10,12}{⽔、⾥}[HSK 7-9]
  \definition{s.}{magnanimidade | alta tolerância ao álcool | cargas de; uma grande quantidade de}
\end{EntryWithPhonetic}

\begin{EntryWithPhonetic}{海绵}{hai3mian2}{10,11}{⽔、⽷}[HSK 7-9]
  \definition{s.}{espuma de borracha; espuma de plástico; esponja; um material poroso feito de borracha ou plástico que é elástico, como uma esponja | esponja marinha seca; poríferos; osso esponjoso; refere-se especificamente ao esqueleto queratinoso das esponjas}
\end{EntryWithPhonetic}

\begin{EntryWithPhonetic}{海面}{hai3mian4}{10,9}{⽔、⾯}[HSK 7-9]
  \definition{s.}{nível do mar | superfície do mar}
\end{EntryWithPhonetic}

\begin{EntryWithPhonetic}{海内外}{hai3 nei4wai4}{10,4,5}{⽔、⼌、⼣}[HSK 7-9]
  \definition{s.}{em casa e no exterior | nacional e internacional}
\end{EntryWithPhonetic}

\begin{EntryWithPhonetic}{海鸥}{hai3'ou1}{10,9}{⽔、⿃}
  \definition{s.}{gaivota}
\end{EntryWithPhonetic}

\begin{EntryWithPhonetic}{海水}{hai3 shui3}{10,4}{⽔、⽔}[HSK 4]
  \definition[把]{s.}{água do mar; salmoura}
\end{EntryWithPhonetic}

\begin{EntryWithPhonetic}{海滩}{hai3tan1}{10,13}{⽔、⽔}[HSK 7-9]
  \definition[个,片]{s.}{praia; praia com declive suave em direção ao mar}
\end{EntryWithPhonetic}

\begin{EntryWithPhonetic}{海棠}{hai3tang2}{10,12}{⽔、⽊}
  \definition{s.}{begônia}
\end{EntryWithPhonetic}

\begin{EntryWithPhonetic}{海外}{hai3 wai4}{10,5}{⽔、⼣}[HSK 6]
  \definition[次]{s.}{fora das fronteiras nacionais; no exterior}
\end{EntryWithPhonetic}

\begin{EntryWithPhonetic}{海湾}{hai3 wan1}{10,12}{⽔、⽔}[HSK 6]
  \definition{s.}{baía; golfo | lago}
\end{EntryWithPhonetic}

\begin{EntryWithPhonetic}{海峡}{hai3xia2}{10,9}{⽔、⼭}[HSK 7-9]
  \definition*{s.}{Estreito de Taiwan}
  \definition[个]{s.}{estreito; canal; um canal estreito que conecta dois oceanos entre duas massas de terra}
\end{EntryWithPhonetic}

\begin{EntryWithPhonetic}{海鲜}{hai3xian1}{10,14}{⽔、⿂}[HSK 4]
  \definition[种,份,桌,批,些]{s.}{frutos do mar; mariscos; peixes marinhos frescos, camarões, etc., para consumo |}
\end{EntryWithPhonetic}

\begin{EntryWithPhonetic}{海啸}{hai3xiao4}{10,11}{⽔、⼝}[HSK 7-9]
  \definition{s.}{\emph{tsunami}; maremoto}
\end{EntryWithPhonetic}

\begin{EntryWithPhonetic}{海洋}{hai3yang2}{10,9}{⽔、⽔}[HSK 6]
  \definition[片,个]{s.}{mar; oceano; um termo geral para os mares e oceanos que formam uma entidade contínua na superfície da Terra; também pode ser usado para descrever um grande número de coisas semelhantes}
\end{EntryWithPhonetic}

\begin{EntryWithPhonetic}{海域}{hai3yu4}{10,11}{⽔、⼟}[HSK 7-9]
  \definition{s.}{área marítima; espaço marítimo; refere-se a uma determinada área do oceano (tanto acima quanto abaixo da água)}
\end{EntryWithPhonetic}

\begin{EntryWithPhonetic}{海运}{hai3yun4}{10,7}{⽔、⾡}[HSK 7-9]
  \definition{s.}{transporte marítimo; transporte oceânico}
  \definition{v.}{transportar pelo mar}
\end{EntryWithPhonetic}

\begin{EntryWithPhonetic}{海藻}{hai3zao3}{10,19}{⽔、⾋}[HSK 7-9]
  \definition{s.}{alga marinha; planta marítima}
\end{EntryWithPhonetic}

\begin{EntryWithPhonetic}{骇}{hai4}{9}{⾺}
  \definition{adj.}{assustado; chocado}
  \definition{v.}{ficar surpreso; ficar chocado}
\end{EntryWithPhonetic}

\begin{EntryWithPhonetic}{骇人听闻}{hai4ren2ting1wen2}{9,2,7,9}{⾺、⼈、⼝、⾨}[HSK 7-9]
  \definition{expr.}{chocante; terrível; assustador para os ouvidos; espantoso; fabuloso; chocante (notícia); aterrorizante; horripilante}
\end{EntryWithPhonetic}

\begin{EntryWithPhonetic}{害}{hai4}{10}{⼧}[HSK 5]
  \definition{adj.}{prejudicial; destrutivo; injurioso; nocivo}
  \definition{s.}{mal; maldade; dano; calamidade}
  \definition{v.}{prejudicar; fazer mal a; causar problemas a | matar; assassinar | sofrer de; contrair (uma doença) | sentir-se (envergonhado, com medo, etc.); despertar (um sentimento ou uma emoção)}
\end{EntryWithPhonetic}

\begin{EntryWithPhonetic}{害虫}{hai4chong2}{10,6}{⼧、⾍}[HSK 7-9]
  \definition[种,只,个]{s.}{verme; bicho; inseto nocivo (ou destrutivo); praga (oposto a 益虫)}
  \seealsoref{益虫}{yi4chong2}
\end{EntryWithPhonetic}

\begin{EntryWithPhonetic}{害怕}{hai4pa4}{10,8}{⼧、⼼}[HSK 3]
  \definition{v.}{estar assustado; ter medo; encontrar dificuldades, perigos, etc., e sentir-se inquieto ou nervoso}
\end{EntryWithPhonetic}

\begin{EntryWithPhonetic}{害臊}{hai4/sao4}{10,17}{⼧、⾁}[HSK 7-9]
  \definition{v.+compl.}{sentir vergonha; ser tímido}
\end{EntryWithPhonetic}

\begin{EntryWithPhonetic}{害羞}{hai4/xiu1}{10,10}{⼧、⽺}[HSK 7-9]
  \definition{v.+compl.}{ser tímido; parecer tímido; tornar-se tímido}
\end{EntryWithPhonetic}

\begin{EntryWithPhonetic}{酣}{han1}{12}{⾣}
  \definition{adj.}{intoxicado}
\end{EntryWithPhonetic}

\begin{EntryWithPhonetic}{酣畅}{han1chang4}{12,8}{⾣、⽥}[HSK 7-9]
  \definition{adj.}{alegre e animado (com bebida) | profundo (sono profundo)}
  \definition{adv.}{com facilidade e entusiasmo; totalmente; refere-se a obras literárias e artísticas}
\end{EntryWithPhonetic}

\begin{EntryWithPhonetic}{酣睡}{han1shui4}{12,13}{⾣、⽬}[HSK 7-9]
  \definition{v.}{dormir profundamente; estar em sono profundo | estar profundamente adormecido; cair em sono profundo}
\end{EntryWithPhonetic}

\begin{EntryWithPhonetic}{汗}{han2}{6}{⽔}
  \definition*{s.}{Abreviação de Khan}[他是成吉思汗。===Ele é Genghis Khan.]
  \seeref{han4}
\end{EntryWithPhonetic}

\begin{EntryWithPhonetic}{含}{han2}{7}{⼝}[HSK 4]
  \definition{v.}{manter na boca (sem engolir ou cuspir) | conter; incluir | cuidar; acalentar; abrigar}
\end{EntryWithPhonetic}

\begin{EntryWithPhonetic}{含糊}{han2hu5}{7,15}{⼝、⽶}[HSK 7-9]
  \definition{adj.}{(atitude, palavras, etc.) vago; ambíguo; pouco claro | (falar, fazer coisas, etc.) descuidado; desleixado (usado principalmente em termos negativos) | covarde; demonstrando fraqueza (usado principalmente em sentido negativo)}
\end{EntryWithPhonetic}

\begin{EntryWithPhonetic}{含金量}{han2jin1liang4}{7,8,12}{⼝、⾦、⾥}
  \definition{adj.}{conteúdo de ouro | (fig.) valioso}
\end{EntryWithPhonetic}

\begin{EntryWithPhonetic}{含量}{han2 liang4}{7,12}{⼝、⾥}[HSK 4]
  \definition{s.}{conteúdo; a quantidade de um componente contido em uma substância}
\end{EntryWithPhonetic}

\begin{EntryWithPhonetic}{含蓄}{han2xu4}{7,13}{⼝、⾋}[HSK 7-9]
  \definition{v.}{conter; incorporar}
\end{EntryWithPhonetic}

\begin{EntryWithPhonetic}{含义}{han2yi4}{7,3}{⼝、⼂}[HSK 4]
  \definition[个,种,层]{s.}{sentido; mensagem; significado; implicação; o significado contido em (palavras, frases, sentenças e discursos)}
\end{EntryWithPhonetic}

\begin{EntryWithPhonetic}{含有}{han2 you3}{7,6}{⼝、⽉}[HSK 4]
  \definition{v.}{conter; ter; incluir}
\end{EntryWithPhonetic}

\begin{EntryWithPhonetic}{函}{han2}{8}{⼐}
  \definition*{s.}{Sobrenome Han}
  \definition[封]{s.}{caixa; envelope; capa | carta}
\end{EntryWithPhonetic}

\begin{EntryWithPhonetic}{函授}{han2shou4}{8,11}{⼐、⼿}[HSK 7-9]
  \definition{v.}{ensinar por correspondência; utilizar principalmente tutoria por correspondência para ministrar cursos}
\end{EntryWithPhonetic}

\begin{EntryWithPhonetic}{函数}{han2shu4}{8,13}{⼐、⽁}
  \definition[个]{s.}{Matemática: função; em um determinado processo, duas variáveis $​​x$ e $y$ têm um certo valor de $y$ correspondente a cada valor de $x$ dentro de um determinado intervalo, $y$ é uma função de $x$; essa relação geralmente é expressa como $y = f(x)$}
\end{EntryWithPhonetic}

\begin{EntryWithPhonetic}{涵}{han2}{11}{⽔}
  \definition{s.}{bueiro; galeria}
  \definition{v.}{conter; incorporar}
\end{EntryWithPhonetic}

\begin{EntryWithPhonetic}{涵盖}{han2gai4}{11,11}{⽔、⽫}[HSK 7-9]
  \definition{v.}{cobrir; incluir; conter; conter completamente}
\end{EntryWithPhonetic}

\begin{EntryWithPhonetic}{涵义}{han2yi4}{11,3}{⽔、⼂}[HSK 7-9]
  \definition[层,种]{s.}{significado; implicação | conotação | conteúdo}
\end{EntryWithPhonetic}

\begin{EntryWithPhonetic}{寒}{han2}{12}{⼧}
  \definition*{s.}{Sobrenome Han}
  \definition{adj.}{frio | pobre; necessitado | (autodepreciativo) meu/minha humilde\dots | assustado; medroso | com medo; tremendo (de medo) | humilde}
  \definition{s.}{estação fria; inverno (oposto a 暑) | (medicina chinesa) sintomas causados por fatores frios}
  \seealsoref{暑}{shu3}
\end{EntryWithPhonetic}

\begin{EntryWithPhonetic}{寒假}{han2jia4}{12,11}{⼧、⼈}[HSK 4]
  \definition[个,段]{s.}{férias de inverno (feriados); férias escolares no meio do inverno, em janeiro e fevereiro (na China)}
\end{EntryWithPhonetic}

\begin{EntryWithPhonetic}{寒冷}{han2 leng3}{12,7}{⼧、⼎}[HSK 4]
  \definition[度,阵,股]{adj.}{frio; frígido; gélido; gelado}
\end{EntryWithPhonetic}

\begin{EntryWithPhonetic}{韩}{han2}{12}{⾱}
  \definition*{s.}{Um estado durante o Período dos Estados Combatentes nas atuais províncias centrais de Henan e sudeste de Shanxi | O nome de um estado feudal durante a dinastia Zhou, localizado no que hoje é o nordeste de Hejin, província de Shanxi | Coreia do Sul, abreviação de 韩国; República da Coreia (RC) | Sobrenome Han}
  \seealsoref{韩国}{han2guo2}
\end{EntryWithPhonetic}

\begin{EntryWithPhonetic}{韩国}{han2guo2}{12,8}{⾱、⼞}
  \definition*{s.}{Coréia do Sul; República da Coreia}
\end{EntryWithPhonetic}

\begin{EntryWithPhonetic}{韩国人}{han2guo2ren2}{12,8,2}{⾱、⼞、⼈}
  \definition{s.}{coreano | pessoa ou povo da Coréia}
\end{EntryWithPhonetic}

\begin{EntryWithPhonetic}{厂}{han3}{2}{⼚}[Kangxi 27]
  \definition[家,间]{s.}{radical ``penhasco'' em caracteres chineses (radical Kangxi 27)}
  \seeref{an1}
  \seeref{chang3}
\end{EntryWithPhonetic}

\begin{EntryWithPhonetic}{罕}{han3}{7}{⽹}
  \definition*{s.}{Sobrenome Han}
  \definition{adj.}{raro; escasso | raro; incomum}
\end{EntryWithPhonetic}

\begin{EntryWithPhonetic}{罕见}{han3jian4}{7,4}{⽹、⾒}[HSK 7-9]
  \definition{adj.}{raro; raramente visto}
\end{EntryWithPhonetic}

\begin{EntryWithPhonetic}{喊}{han3}{12}{⼝}[HSK 2]
  \definition{v.}{gritar; clamar; berrar | chamar (uma pessoa) | chamar; dirigir-se a}
\end{EntryWithPhonetic}

\begin{EntryWithPhonetic}{汉}{han4}{5}{⽔}
  \definition*{s.}{Dinastia Han (206 a.C.-220 d.C.)  | Astronomia: A Via Láctea | Sobrenome Han}
  \definition{s.}{grupo étnico Han | chinês (língua) | homem}
\end{EntryWithPhonetic}

\begin{EntryWithPhonetic}{汉堡包}{han4bao3bao1}{5,12,5}{⽔、⼟、⼓}
  \definition[个]{s.}{hambúrguer}
\end{EntryWithPhonetic}

\begin{EntryWithPhonetic}{汉堡王}{han4bao3wang2}{5,12,4}{⽔、⼟、⽟}
  \definition*{s.}{Burguer King, restaurante de \emph{fast-food}}
\end{EntryWithPhonetic}

\begin{EntryWithPhonetic}{汉服}{han4fu2}{5,8}{⽔、⽉}
  \definition{s.}{vestido chinês tradicional Han}
\end{EntryWithPhonetic}

\begin{EntryWithPhonetic}{汉葡词典}{han4-pu2 ci2dian3}{5,12,7,8}{⽔、⾋、⾔、⼋}
  \definition[部,本]{s.}{dicionário chinês-português}
  \seealsoref{葡汉词典}{pu2-han4 ci2dian3}
\end{EntryWithPhonetic}

\begin{EntryWithPhonetic}{汉语}{han4yu3}{5,9}{⽔、⾔}[HSK 1]
  \definition[门]{s.}{língua chinesa, mandarim}
\end{EntryWithPhonetic}

\begin{EntryWithPhonetic}{汉字}{han4 zi4}{5,6}{⽔、⼦}[HSK 1]
  \definition[个]{s.}{caractere chinês; ideograma chinês; sinograma; com pouquíssimas exceções, os caracteres chineses representam uma sílaba cada um}
\end{EntryWithPhonetic}

\begin{EntryWithPhonetic}{汗}{han4}{6}{⽔}[HSK 5]
  \definition{s.}{suor; transpiração; perspiração}
  \seeref{han2}
\end{EntryWithPhonetic}

\begin{EntryWithPhonetic}{汗水}{han4shui3}{6,4}{⽔、⽔}[HSK 7-9]
  \definition{s.}{transpiração; suor (em grandes quantidades)}
\end{EntryWithPhonetic}

\begin{EntryWithPhonetic}{汗腺}{han4xian4}{6,13}{⽔、⾁}
  \definition{s.}{glândula sudorípara}
\end{EntryWithPhonetic}

\begin{EntryWithPhonetic}{汗液}{han4ye4}{6,11}{⽔、⽔}
  \definition{s.}{suor}
\end{EntryWithPhonetic}

\begin{EntryWithPhonetic}{旱}{han4}{7}{⽇}[HSK 7-9]
  \definition{adj.}{atingido pela seca (em oposição a 涝) | seco; árido | em terra; terrestre}
  \definition{s.}{período de seca; nenhuma precipitação ou precipitação muito baixa; seca (oposto a 涝) | rota terrestre (ou comunicação) | terra firme}
  \seealsoref{涝}{lao4}
\end{EntryWithPhonetic}

\begin{EntryWithPhonetic}{旱灾}{han4zai1}{7,7}{⽇、⽕}[HSK 7-9]
  \definition[场]{s.}{seca (oposto a 水灾); desastres causados ​​por secas prolongadas e escassez de água que resultam na morte de colheitas ou redução significativa da produção}
  \seealsoref{水灾}{shui3zai1}
\end{EntryWithPhonetic}

\begin{EntryWithPhonetic}{捍}{han4}{10}{⼿}
  \definition{v.}{defender; guardar | defender-se | afastar (um golpe) | resistir}
\end{EntryWithPhonetic}

\begin{EntryWithPhonetic}{捍卫}{han4wei4}{10,3}{⼿、⼙}[HSK 7-9]
  \definition{v.}{defender; guardar; proteger; defender-se pela força ou outros meios de ser violado ou prejudicado}
\end{EntryWithPhonetic}

\begin{EntryWithPhonetic}{焊}{han4}{11}{⽕}[HSK 7-9]
  \definition{v.}{soldar; usar metal fundido para reparar objetos de metal ou conectar peças de metal}
\end{EntryWithPhonetic}

\begin{EntryWithPhonetic}{撼}{han4}{16}{⼿}
  \definition{v.}{agitar; sacudir}
\end{EntryWithPhonetic}

\begin{EntryWithPhonetic}{行}{hang2}{6}{⾏}[HSK 3][Kangxi 144]
  \definition{adj.}{temporário; improvisado | capaz; competente}
  \definition{adv.}{logo; em breve}
  \definition{clas.}{linha; fileira; coisas usadas para formar filas, linhas}
  \definition{s.}{comportamento; conduta | linha; fileira | empresa comercial; certas instituições comerciais | comércio; profissão; ramo de atividade | especialista; conhecedor; refere-se ao conhecimento e experiência em um determinado setor}
  \definition{v.}{ir; caminhar; viajar | estar atualizado; circular | fazer; executar; realizar | (antes de um verbo dissílabo, indicando a realização de alguma ação) | ficar bem; vai dar certo | (remédio) fazer efeito | classificar (entre irmãos e irmãs por ordem de idade)}
  \seeref{heng2}
  \seeref{xing2}
\end{EntryWithPhonetic}

\begin{EntryWithPhonetic}{行家}{hang2jia5}{6,10}{⾏、⼧}[HSK 7-9]
  \definition[位,名,个,些]{s.}{especialista; \emph{expert}; conhecedor; \emph{connoisseur}}
\end{EntryWithPhonetic}

\begin{EntryWithPhonetic}{行列}{hang2lie4}{6,6}{⾏、⼑}[HSK 7-9]
  \definition{s.}{fileiras}
\end{EntryWithPhonetic}

\begin{EntryWithPhonetic}{行情}{hang2qing2}{6,11}{⾏、⼼}[HSK 7-9]
  \definition{s.}{preço; cotações de mercado; o preço geral dos bens no mercado também se refere à situação geral das taxas de juros, taxas de câmbio, preços de títulos, etc. no mercado financeiro}
\end{EntryWithPhonetic}

\begin{EntryWithPhonetic}{行业}{hang2ye4}{6,5}{⾏、⼀}[HSK 4]
  \definition[种,个]{s.}{comércio; indústria; setor; profissão; categorias em negócios e indústria referem-se a ocupações em geral}
\end{EntryWithPhonetic}

\begin{EntryWithPhonetic}{航}{hang2}{10}{⾈}
  \definition*{s.}{Sobrenome Hang}
  \definition[趟]{s.}{barco; navio}
  \definition{v.}{navegar (por água ou ar) | velejar}
\end{EntryWithPhonetic}

\begin{EntryWithPhonetic}{航班}{hang2ban1}{10,10}{⾈、⽟}[HSK 4]
  \definition[个,次]{s.}{número do voo; voo programado; o horário de um navio ou avião de passageiros}
\end{EntryWithPhonetic}

\begin{EntryWithPhonetic}{航海}{hang2hai3}{10,10}{⾈、⽔}[HSK 7-9]
  \definition{v.}{velejar; navegar}
\end{EntryWithPhonetic}

\begin{EntryWithPhonetic}{航空}{hang2kong1}{10,8}{⾈、⽳}[HSK 4]
  \definition{s.}{viagem; aviação; refere-se ao voo de uma aeronave no ar}
\end{EntryWithPhonetic}

\begin{EntryWithPhonetic}{航天}{hang2tian1}{10,4}{⾈、⼤}[HSK 7-9]
  \definition{s.}{voo espacial; astronáutica}
  \definition{v.}{voar ou viajar no espaço}
\end{EntryWithPhonetic}

\begin{EntryWithPhonetic}{航天员}{hang2tian1yuan2}{10,4,7}{⾈、⼤、⼝}[HSK 7-9]
  \definition[名,位,个]{s.}{astronauta}
\end{EntryWithPhonetic}

\begin{EntryWithPhonetic}{航行}{hang2xing2}{10,6}{⾈、⾏}[HSK 7-9]
  \definition{v.}{velejar; voar; navegar pela água, pelo ar}
\end{EntryWithPhonetic}

\begin{EntryWithPhonetic}{航运}{hang2yun4}{10,7}{⾈、⾡}[HSK 7-9]
  \definition{s.}{transporte hidroviário; transporte marítimo}
\end{EntryWithPhonetic}

\begin{EntryWithPhonetic}{号}{hao2}{5}{⼝}
  \definition{v.}{uivar; gritar; gritar em voz alta e prolongada | lamentar; chorar alto | uivar; (vento) assobiar, assoviar}
  \seeref{hao4}
\end{EntryWithPhonetic}

\begin{EntryWithPhonetic}{蚝}{hao2}{10}{⾍}
  \definition[只]{s.}{ostra}
\end{EntryWithPhonetic}

\begin{EntryWithPhonetic}{毫}{hao2}{11}{⽊}
  \definition{adv.}{nem um pouco; absolutamente nenhum; completamente sem}
  \definition{clas.}{hao, uma unidade de comprimento igual a um milésimo de polegada ou 1/30 de milímetro | hao, uma unidade de peso igual a um milésimo de um centavo ou 0,005 grama |
uma fração minúscula; uma parte muito pequena}
  \definition{pref.}{mili-, usado com a unidade de uma quantidade física para representar um milésimo dessa quantidade}
  \definition{s.}{cabelo longo e fino | pincel de escrita | uma das duas ou três alças de uma balança para pendurar na mão do usuário | cerda; uma corda de mão em uma balança ou equilíbrio | fio de cabelo}
\end{EntryWithPhonetic}

\begin{EntryWithPhonetic}{毫不费力}{hao2bu2fei4li4}{11,4,9,2}{⽊、⼀、⾙、⼒}
  \definition{expr.}{sem esforço; não despender o menor esforço}
\end{EntryWithPhonetic}

\begin{EntryWithPhonetic}{毫不}{hao2 bu4}{11,4}{⽊、⼀}[HSK 7-9]
  \definition{adv.}{dificilmente; de ​​jeito nenhum; nem um pouco}
\end{EntryWithPhonetic}

\begin{EntryWithPhonetic}{毫不犹豫}{hao2 bu4 you2yu4}{11,4,7,15}{⽊、⼀、⽝、⾗}[HSK 7-9]
  \definition{expr.}{sem hesitação; sem a menor hesitação}
\end{EntryWithPhonetic}

\begin{EntryWithPhonetic}{毫米}{hao2mi3}{11,6}{⽊、⽶}[HSK 4]
  \definition{clas.}{milímetro; unidade legal de medida de comprimento, 1 mm equivale a 0,1 cm}
\end{EntryWithPhonetic}

\begin{EntryWithPhonetic}{毫升}{hao2 sheng1}{11,4}{⽊、⼗}[HSK 4]
  \definition{clas.}{mililitro; unidade de volume, milésimo de um litro (ml)}
\end{EntryWithPhonetic}

\begin{EntryWithPhonetic}{毫无}{hao2wu2}{11,4}{⽊、⽆}[HSK 7-9]
  \definition{adv.}{não; nada; de jeito nenhum}
\end{EntryWithPhonetic}

\begin{EntryWithPhonetic}{豪}{hao2}{14}{⾗}
  \definition*{s.}{Sobrenome Hao}
  \definition{adj.}{direto; irrestrito; ousado | despótico; intimidador | rico e poderoso}
  \definition{s.}{pessoa com poderes ou dons extraordinários}
\end{EntryWithPhonetic}

\begin{EntryWithPhonetic}{豪华}{hao2hua2}{14,6}{⾗、⼗}[HSK 7-9]
  \definition{adj.}{luxo; luxuoso; (edifício, equipamento ou decoração) magnífico; muito lindo}
\end{EntryWithPhonetic}

\begin{EntryWithPhonetic}{好}{hao3}{6}{⼥}[HSK 1,2,4]
  \definition{adj.}{bom; ótimo; agradável; vantajoso; satisfatório | amigável; gentil; amistoso; amável | saudável; bem | pronto; concluído; usado após um verbo para indicar conclusão ou perfeição | fácil (de fazer); conveniente; responsável (por)}
  \definition{adv.}{muito; bastante; tão; usado na frente de uma palavra de quantidade ou uma palavra de tempo para indicar muito ou por muito tempo | em que medida; como; usado antes de adjetivos e verbos para indicar profundidade e com exclamação}
  \definition{interj.}{O.K.; tudo bem; aprovação, acordo ou encerramento | (no início de uma frase ou oração) expressa concordância (ou desaprovação, surpresa, etc.)}
  \definition{prep.}{de modo a; para que}
  \definition{s.}{referindo-se a palavras de elogio ou aplauso | saudações; cumprimentos}
  \definition{suf.}{sufixo que indica conclusão ou prontidão | depois de um pronome significa ``olá''}
  \definition{v.}{deve; precisa; tem que; deveria | apaixonar-se}
  \seeref{hao4}
\end{EntryWithPhonetic}

\begin{EntryWithPhonetic}{好比}{hao3bi3}{6,4}{⼥、⽐}[HSK 7-9]
  \definition{v.}{pode ser comparado a; ser exatamente como}
\end{EntryWithPhonetic}

\begin{EntryWithPhonetic}{好(不)容易}{hao3 bu4 rong2 yi4}{6,4,10,8}{⼥、⼀、⼧、⽇}[HSK 6]
  \definition{adv.}{com grande dificuldade; muito difícil}
  \definition{v.}{ter dificuldade (em fazer algo)}
\end{EntryWithPhonetic}

\begin{EntryWithPhonetic}{好吃}{hao3chi1}{6,6}{⼥、⼝}[HSK 1]
  \definition{adj.}{bom; saboroso; delicioso; descreve o sabor agradável de algo, que as pessoas gostam de comer}
  \seeref{hao4chi1}
\end{EntryWithPhonetic}

\begin{EntryWithPhonetic}{好处}{hao3chu4}{6,5}{⼥、⼡}[HSK 2]
  \definition[个]{s.}{bom; benefício; vantagem; fatores favoráveis a pessoas ou coisas | ganho; lucro; algo que não se deveria receber, dado por outra pessoa ou obtido através de uma oportunidade; geralmente tem conotação pejorativa}
\end{EntryWithPhonetic}

\begin{EntryWithPhonetic}{好歹}{hao3dai3}{6,4}{⼥、⽍}[HSK 7-9]
  \definition{adv.}{de qualquer forma; em qualquer caso | de alguma forma; não importa de que maneira; não importa o que}
  \definition{s.}{bom e mau; o que é bom e o que é mau | acidente; desastre; refere-se a situações de risco de vida}
\end{EntryWithPhonetic}

\begin{EntryWithPhonetic}{好多}{hao3 duo1}{6,6}{⼥、⼣}[HSK 2]
  \definition{adj.}{muitos; uma boa quantidade; uma grande quantidade; uma quantidade enorme}
  \definition{pron.}{quantos?; quanto?; frequentemente usado para perguntar sobre quantidade}
\end{EntryWithPhonetic}

\begin{EntryWithPhonetic}{好感}{hao3gan3}{6,13}{⼥、⼼}[HSK 7-9]
  \definition{s.}{boa opinião; impressão favorável; sentimentos de satisfação ou simpatia por pessoas ou coisas}
\end{EntryWithPhonetic}

\begin{EntryWithPhonetic}{好汉}{hao3han4}{6,5}{⼥、⽔}
  \definition[条]{s.}{herói | pessoa forte e corajosa}
\end{EntryWithPhonetic}

\begin{EntryWithPhonetic}{好好}{hao3 hao3}{6,6}{⼥、⼥}[HSK 3]
  \definition{adj.}{realmente bom/bem; em perfeitas condições; quando tudo está bem}
  \definition{adv.}{diretamente; seriamente; cuidadosamente; com todo o empenho; ao máximo}
\end{EntryWithPhonetic}

\begin{EntryWithPhonetic}{好坏}{hao3huai4}{6,7}{⼥、⼟}[HSK 7-9]
  \definition{s.}{bom e mau; o que é bom e o que é mau | bom ou ruim | qualidade |padrão}
\end{EntryWithPhonetic}

\begin{EntryWithPhonetic}{好家伙}{hao3jia1huo5}{6,10,6}{⼥、⼧、⼈}[HSK 7-9]
  \definition[个]{interj.}{Bom Deus!; Céus!; Bom Senhor!; expressa surpresa ou admiração}
\end{EntryWithPhonetic}

\begin{EntryWithPhonetic}{好久}{hao3jiu3}{6,3}{⼥、⼃}[HSK 2]
  \definition{adv.}{por muito tempo | por eras (no passado)}
\end{EntryWithPhonetic}

\begin{EntryWithPhonetic}{好看}{hao3 kan4}{6,9}{⼥、⽬}[HSK 1]
  \definition{adj.}{de boa aparência; agradável; bonito | interessante; descreve o enredo ou conteúdo de filmes, romances, performances, etc., como sendo cativante, agradável ou apreciável}
\end{EntryWithPhonetic}

\begin{EntryWithPhonetic}{好评}{hao3ping2}{6,7}{⼥、⾔}[HSK 7-9]
  \definition{s.}{comentário favorável; opinião elevada; boas críticas; altas críticas}
\end{EntryWithPhonetic}

\begin{EntryWithPhonetic}{好人}{hao3 ren2}{6,2}{⼥、⼈}[HSK 2]
  \definition[个,位,名]{s.}{pessoa boa (ou excelente) (oposto de 坏人) | pessoa saudável | pessoa gentil que tenta se dar bem com todos (muitas vezes em detrimento dos princípios)}
  \seealsoref{坏人}{huai4 ren2}
\end{EntryWithPhonetic}

\begin{EntryWithPhonetic}{好生}{hao3sheng1}{6,5}{⼥、⽣}
  \definition{adv.}{bastante; extremamente | cuidadosamente; apropriadamente}
\end{EntryWithPhonetic}

\begin{EntryWithPhonetic}{好事}{hao3 shi4}{6,8}{⼥、⼅}[HSK 2]
  \definition[个,件]{s.}{boa ação; gentileza | (antigo) obra de caridade | acontecimento feliz; evento festivo}
  \seeref{hao4 shi4}
\end{EntryWithPhonetic}

\begin{EntryWithPhonetic}{好说}{hao3shuo1}{6,9}{⼥、⾔}[HSK 7-9]
  \definition{adj.}{palavras elogiosas; usadas quando alguém agradece ou elogia você; usadas para expressar que você não é digno do elogio | sem problemas; expressa concordância ou vontade de negociar}
\end{EntryWithPhonetic}

\begin{EntryWithPhonetic}{好似}{hao3 si4}{6,6}{⼥、⼈}[HSK 6]
  \definition{v.}{parecer; ser como}
\end{EntryWithPhonetic}

\begin{EntryWithPhonetic}{好听}{hao3 ting1}{6,7}{⼥、⼝}[HSK 1]
  \definition{adj.}{agradável de ouvir (de som ou voz) | bom; palatável; satisfatório (de palavras)  | decente; honrado (de ações, etc.); descreve uma coisa que parece prestigiosa | interessante; descreve palavras, histórias e outras coisas interessantes}
\end{EntryWithPhonetic}

\begin{EntryWithPhonetic}{好玩儿}{hao3 wan2r5}{6,8,2}{⼥、⽟、⼉}[HSK 1]
  \definition{adj.}{divertido; interessante; capaz de despertar interesse}
\end{EntryWithPhonetic}

\begin{EntryWithPhonetic}{好象}{hao3xiang4}{6,11}{⼥、⾗}
  \variantof{好像}
\end{EntryWithPhonetic}

\begin{EntryWithPhonetic}{好像}{hao3xiang4}{6,13}{⼥、⼈}[HSK 2]
  \definition{adv.}{como se; um pouco parecido; como se fosse}
  \definition{v.}{parecer; ser como; parecer-se com}
\end{EntryWithPhonetic}

\begin{EntryWithPhonetic}{好笑}{hao3xiao4}{6,10}{⼥、⽵}[HSK 7-9]
  \definition{adj.}{engraçado; divertido; ridículo}
\end{EntryWithPhonetic}

\begin{EntryWithPhonetic}{好心}{hao3xin1}{6,4}{⼥、⼼}[HSK 7-9]
  \definition{adj./s.}{bondade; boas intenções}
\end{EntryWithPhonetic}

\begin{EntryWithPhonetic}{好心人}{hao3xin1ren2}{6,4,2}{⼥、⼼、⼈}[HSK 7-9]
  \definition{s.}{boa alma; pessoa de bom coração | pessoa gentil}
\end{EntryWithPhonetic}

\begin{EntryWithPhonetic}{好学}{hao3xue2}{6,8}{⼥、⼦}
  \definition{adj.}{fácil de aprender}
  \seeref{hao4xue2}
\end{EntryWithPhonetic}

\begin{EntryWithPhonetic}{好意}{hao3yi4}{6,13}{⼥、⼼}[HSK 7-9]
  \definition{s.}{boas intenções; gentileza}
\end{EntryWithPhonetic}

\begin{EntryWithPhonetic}{好用}{hao3yong4}{6,5}{⼥、⽤}
  \definition{adj.}{fácil de usar | adequado ao uso}
\end{EntryWithPhonetic}

\begin{EntryWithPhonetic}{好友}{hao3you3}{6,4}{⼥、⼜}[HSK 4]
  \definition[位,名,个,些]{s.}{bom amigo; amigo próximo}
\end{EntryWithPhonetic}

\begin{EntryWithPhonetic}{好运}{hao3 yun4}{6,7}{⼥、⾡}[HSK 5]
  \definition{s.}{boa sorte, fortuna ou oportunidade}
\end{EntryWithPhonetic}

\begin{EntryWithPhonetic}{好在}{hao3zai4}{6,6}{⼥、⼟}[HSK 7-9]
  \definition{adv.}{felizmente; afortunadamente; indica que existem fatores favoráveis ​​em condições difíceis ou desfavoráveis}
\end{EntryWithPhonetic}

\begin{EntryWithPhonetic}{好转}{hao3 zhuan3}{6,8}{⼥、⾞}[HSK 6]
  \definition{v.}{melhorar; dar uma guinada para melhor; tomar um rumo favorável}
\end{EntryWithPhonetic}

\begin{EntryWithPhonetic}{号}{hao4}{5}{⼝}[HSK 1]
  \definition{clas.}{usado para o número de pessoas |  tipo; espécie; classificação | usado para pessoas ou negócios; número de vezes utilizado para transações}
  \definition[把]{s.}{nome | nome presumido; nome alternativo; pseudônimo; apelido | casa de negócios; loja | marca; sinal; sinalização | número | data | ordem; no exército, as ordens são transmitidas verbalmente ou por meio de clarins | qualquer instrumento de sopro e latão; trombeta usada no exército ou em bandas | qualquer coisa usada como buzina | chamada de corneta; qualquer chamada feita em uma corneta; usar um apito para emitir um som com um significado específico | pessoa em uma condição especial; pessoas que se encontram em uma situação especial}
  \definition{suf.}{sufixo de navio}
  \definition{v.}{marcar; fazer uma marca | sentir; colocar a mão no pulso do paciente e avaliar a situação através do fluxo sanguíneo}
  \seeref{hao2}
\end{EntryWithPhonetic}

\begin{EntryWithPhonetic}{号称}{hao4cheng1}{5,10}{⼝、⽲}[HSK 7-9]
  \definition{v.}{ser conhecido como; ser conhecido por um certo nome | afirmar ser; alegar}
\end{EntryWithPhonetic}

\begin{EntryWithPhonetic}{号角}{hao4jiao3}{5,7}{⼝、⾓}
  \definition{s.}{corneta | trombeta}
\end{EntryWithPhonetic}

\begin{EntryWithPhonetic}{号码}{hao4ma3}{5,8}{⼝、⽯}[HSK 4]
  \definition[个,组,串]{s.}{número}
\end{EntryWithPhonetic}

\begin{EntryWithPhonetic}{号召}{hao4zhao4}{5,5}{⼝、⼝}[HSK 5]
  \definition{s.}{chamado; apelo; desejo ou pedido solene (de um governo, partido político, organização etc.) para que as massas façam algo}
  \definition{v.}{chamar;  (governo, partido político, organização, etc.) fazer um pedido solene às massas para que façam algo, na esperança de que todos se esforcem para alcançá-lo}
\end{EntryWithPhonetic}

\begin{EntryWithPhonetic}{好}{hao4}{6}{⼥}
  \definition*{s.}{Sobrenome Hao}
  \definition{adv.}{algo que acontece com frequência, que é fácil de acontecer}
  \definition{v.}{gostar; amar; ter afeição por}
  \seeref{hao3}
\end{EntryWithPhonetic}

\begin{EntryWithPhonetic}{好吃}{hao4chi1}{6,6}{⼥、⼝}
  \definition{v.}{ser guloso; gostar de comer boa comida}
  \seeref{hao3chi1}
\end{EntryWithPhonetic}

\begin{EntryWithPhonetic}{好客}{hao4ke4}{6,9}{⼥、⼧}[HSK 7-9]
  \definition{adj.}{hospitaleiro; refere-se a estar disposto a receber convidados e ser afetuoso com eles}
\end{EntryWithPhonetic}

\begin{EntryWithPhonetic}{好奇}{hao4qi2}{6,8}{⼥、⼤}[HSK 3]
  \definition{adj.}{curioso; curiosidade e interesse por coisas não conhecidas}
  \definition{s.}{curiosidade}
  \definition{v.}{ser ou estar curioso}
\end{EntryWithPhonetic}

\begin{EntryWithPhonetic}{好奇心}{hao4qi2xin1}{6,8,4}{⼥、⼤、⼼}[HSK 7-9]
  \definition{s.}{curiosidade; uma emoção que expressa atenção especial a algo}
\end{EntryWithPhonetic}

\begin{EntryWithPhonetic}{好事}{hao4 shi4}{6,8}{⼥、⼅}
  \definition[个,件]{s.}{intrometido; gostar de se meter na vida dos outros}
  \seeref{hao3 shi4}
\end{EntryWithPhonetic}

\begin{EntryWithPhonetic}{好学}{hao4xue2}{6,8}{⼥、⼦}[HSK 6]
  \definition[个]{s.}{apaixonado para aprender; estudioso; erudito}
  \seeref{hao3xue2}
\end{EntryWithPhonetic}

\begin{EntryWithPhonetic}{浩}{hao4}{10}{⽔}
  \definition*{s.}{Sobrenome Hao}
  \definition{adj.}{grande; vasto; grandioso; sem limites | um grande número; infinito}
\end{EntryWithPhonetic}

\begin{EntryWithPhonetic}{浩劫}{hao4jie2}{10,7}{⽔、⼒}[HSK 7-9]
  \definition[场,次]{s.}{grande calamidade; catástrofe | devastação; holocausto; flagelo}
\end{EntryWithPhonetic}

\begin{EntryWithPhonetic}{耗}{hao4}{10}{⽾}[HSK 7-9]
  \definition{s.}{más notícias}[听到噩耗,他碎心裂胆。===Ele ficou arrasado ao ouvir as más notícias.]
  \definition{v.}{consumir; custar | perder tempo; procrastinar}
\end{EntryWithPhonetic}

\begin{EntryWithPhonetic}{耗费}{hao4fei4}{10,9}{⽾、⾙}[HSK 7-9]
  \definition{v.}{gastar; consumir; esgotar}
\end{EntryWithPhonetic}

\begin{EntryWithPhonetic}{耗时}{hao4shi2}{10,7}{⽾、⽇}[HSK 7-9]
  \definition{adj.}{demorado; levar um período de ($x$ quantidade de tempo)}
\end{EntryWithPhonetic}

\begin{EntryWithPhonetic}{呵}{he1}{8}{⼝}
  \definition{interj.}{Meu Deus!| Ah!; Oh!}
  \definition{v.}{expirar (com a boca aberta) | repreender}
  \seeref{a1}
\end{EntryWithPhonetic}

\begin{EntryWithPhonetic}{呵护}{he1hu4}{8,7}{⼝、⼿}[HSK 7-9]
  \definition{v.}{proteger; cuidar bem de}
\end{EntryWithPhonetic}

\begin{EntryWithPhonetic}{欱}{he1}{10}{⽋}
  \definition{v.}{beber | beber bebida alcoólica}
  \variantof{喝}
\end{EntryWithPhonetic}

\begin{EntryWithPhonetic}{喝}{he1}{12}{⼝}[HSK 1]
  \definition{interj.}{Meu Deus!; Oh!; Ah!; Uau!}
  \definition{s.}{bebida; especificamente, vinho}
  \definition{v.}{beber; engolir líquidos ou alimentos líquidos | beber bebida alcoólica; referência específica ao consumo de álcool}
  \seeref{he4}
\end{EntryWithPhonetic}

\begin{EntryWithPhonetic}{喝醉}{he1zui4}{12,15}{⼝、⾣}
  \definition{v.}{ficar bêbado}
\end{EntryWithPhonetic}

\begin{EntryWithPhonetic}{禾}{he2}{5}{⽲}[Kangxi 115]
  \definition[棵]{s.}{mudas (especialmente de arroz) | painço}
\end{EntryWithPhonetic}

\begin{EntryWithPhonetic}{禾苗}{he2miao2}{5,8}{⽲、⾋}[HSK 7-9]
  \definition[棵,片]{s.}{mudas de cereais | muda (de arroz ou outro grão)}
\end{EntryWithPhonetic}

\begin{EntryWithPhonetic}{合}{he2}{6}{⼝}[HSK 3]
  \definition{adj.}{todo; completo; inteiro}
  \definition{clas.}{usado para rodadas | 100ml | medida para grãos secos igual a um décimo de 升, ou um centésimo de 斗}
  \definition{s.}{casamento; união matrimonial | (astronomia) conjunção | nota da escala em Gongchepu (工尺谱), correspondente ao 5 na notação musical numerada}
  \definition{v.}{fechar | juntar; combinar (oposto de 分) | adequar-se; concordar; conformar-se a | ser igual a; somar | ser adequado}
  \seealsoref{斗}{dou4}
  \seealsoref{分}{fen1}
  \seealsoref{工尺谱}{gong1 che3 pu3}
  \seealsoref{升}{sheng1}
\end{EntryWithPhonetic}

\begin{EntryWithPhonetic}{合并}{he2bing4}{6,6}{⼝、⼲}[HSK 5]
  \definition{v.}{fundir; amalgamar; combinar várias coisas em uma coisa só | (doença) ser complicada por outra doença; uma doença levar a outra, ataques simultâneos (de várias doenças)}
\end{EntryWithPhonetic}

\begin{EntryWithPhonetic}{合唱}{he2chang4}{6,11}{⼝、⼝}[HSK 7-9]
  \definition{v.}{cantar em coro; cantar ou apresentar-se junto}[他们一起合唱一台戏。===Eles cantam uma ópera juntos.]
\end{EntryWithPhonetic}

\begin{EntryWithPhonetic}{合成}{he2cheng2}{6,6}{⼝、⼽}[HSK 5]
  \definition{s.}{compor; integrar; combinar; misturar | Química: sintetizar, reação química para transformar uma substância com uma composição simples em uma substância com uma composição complexa}
\end{EntryWithPhonetic}

\begin{EntryWithPhonetic}{合法}{he2fa3}{6,8}{⼝、⽔}[HSK 3]
  \definition{adj.}{legal; legítimo; lícito;  justo; válido; em conformidade com as disposições legais}
\end{EntryWithPhonetic}

\begin{EntryWithPhonetic}{合格}{he2ge2}{6,10}{⼝、⽊}[HSK 3]
  \definition{adj.}{qualificado; dentro dos padrões; em conformidade com os requisitos ou normas}
\end{EntryWithPhonetic}

\begin{EntryWithPhonetic}{合乎}{he2hu1}{6,5}{⼝、⼃}[HSK 7-9]
  \definition{v.}{conformar-se com (ou a); corresponder a; concordar com; coincidir com; ser consistente com}
\end{EntryWithPhonetic}

\begin{EntryWithPhonetic}{合伙}{he2huo3}{6,6}{⼝、⼈}[HSK 7-9]
  \definition{v.}{formar uma parceria; formar uma parceria relativamente fixa (para se envolver em atividades comerciais ou fazer coisas ruins)}
\end{EntryWithPhonetic}

\begin{EntryWithPhonetic}{合计}{he2ji4}{6,4}{⼝、⾔}[HSK 7-9]
  \definition{v.}{pensar sobre; descobrir | consultar | somar; totalizar}
  \seeref{he2ji5}
\end{EntryWithPhonetic}

\begin{EntryWithPhonetic}{合计}{he2ji5}{6,4}{⼝、⾔}
  \definition{v.}{totalizar; somar; estimar | discutir; negociar; deliberar}
  \seeref{he2ji4}
\end{EntryWithPhonetic}

\begin{EntryWithPhonetic}{合理}{he2li3}{6,11}{⼝、⽟}[HSK 3]
  \definition{adj.}{racional; razoável; equitativo; razoável ou lógico}
\end{EntryWithPhonetic}

\begin{EntryWithPhonetic}{合情合理}{he2qing2-he2li3}{6,11,6,11}{⼝、⼼、⼝、⽟}[HSK 7-9]
  \definition{expr.}{razoável; razoável e lógico; justo e racional; justificável e sensato; justo e razoável; justo e sensato}
\end{EntryWithPhonetic}

\begin{EntryWithPhonetic}{合适}{he2shi4}{6,9}{⼝、⾡}[HSK 2]
  \definition{adj.}{correto; adequado; apropriado; conveniente; em conformidade com a realidade ou com os requisitos objetivos}
\end{EntryWithPhonetic}

\begin{EntryWithPhonetic}{合同}{he2tong5}{6,6}{⼝、⼝}[HSK 4]
  \definition[个,份]{s.}{contrato; acordo; uma disposição para observância mútua por duas ou mais partes na condução de um assunto com o objetivo de determinar seus respectivos direitos e obrigações.}
\end{EntryWithPhonetic}

\begin{EntryWithPhonetic}{合宪性}{he2xian4xing4}{6,9,8}{⼝、⼧、⼼}
  \definition{s.}{constitucionalismo}
\end{EntryWithPhonetic}

\begin{EntryWithPhonetic}{合影}{he2/ying3}{6,15}{⼝、⼺}[HSK 7-9]
  \definition[张,个]{s.}{foto de grupo; imagem de grupo}
  \definition{v.+compl.}{tirar uma foto em grupo; tirar uma foto}
\end{EntryWithPhonetic}

\begin{EntryWithPhonetic}{合约}{he2 yue1}{6,6}{⼝、⽷}[HSK 6]
  \definition[份]{s.}{contrato; geralmente se refere a contratos com cláusulas mais simples}
\end{EntryWithPhonetic}

\begin{EntryWithPhonetic}{合资}{he2zi1}{6,10}{⼝、⾙}[HSK 7-9]
  \definition{s.}{consórcio; \emph{joint-venture} com capitais mistos; investimento conjunto por duas ou mais partes (diferente de 独资)}
  \definition{v.}{investir conjuntamente em}
  \seealsoref{独资}{du2zi1}
\end{EntryWithPhonetic}

\begin{EntryWithPhonetic}{合作}{he2zuo4}{6,7}{⼝、⼈}[HSK 3]
  \definition{v.}{cooperar; colaborar; trabalhar em conjunto; trabalhar em conjunto para realizar algo ou concluir uma tarefa}
\end{EntryWithPhonetic}

\begin{EntryWithPhonetic}{合作社}{he2zuo4she4}{6,7,7}{⼝、⼈、⽰}[HSK 7-9]
  \definition{s.}{cooperativa | cooperativa de trabalhadores ou produtores agrícolas, etc.}
\end{EntryWithPhonetic}

\begin{EntryWithPhonetic}{何}{he2}{7}{⼈}
  \definition*{s.}{Sobrenome He}
  \definition{adv.}{enfatiza um alto grau de intensidade, equivalente a 多么}
  \definition{pron.}{O que?; Qual?; em nome de pessoas ou coisas, equivalente a 什么 | Onde?; em nome do lugar, equivalente a 哪里 | Por que?; Como?; a razão, é equivalente a 为什么 e 怎么}
  \seealsoref{多么}{duo1me5}
  \seealsoref{哪里}{na3 li3}
  \seealsoref{岂}{qi3}
  \seealsoref{什么}{shen2me5}
  \seealsoref{为什么}{wei4shen2me5}
  \seealsoref{怎}{zen3}
  \seealsoref{怎么}{zen3me5}
\end{EntryWithPhonetic}

\begin{EntryWithPhonetic}{何必}{he2bi4}{7,5}{⼈、⼼}[HSK 7-9]
  \definition{adv.}{por que?; não há necessidade; use um tom interrogativo para expressar que não é necessário}[你何必如此匆忙?===Por que você está com tanta pressa?]
\end{EntryWithPhonetic}

\begin{EntryWithPhonetic}{何不}{he2bu4}{7,4}{⼈、⼀}
  \definition{adv.}{por que não?; use o tom interrogativo para expressar "deveria" ou "pode"}
\end{EntryWithPhonetic}

\begin{EntryWithPhonetic}{何处}{he2chu4}{7,5}{⼈、⼡}[HSK 7-9]
  \definition{pron.}{onde; que lugar}[你要去何处?===Onde você está indo?]
\end{EntryWithPhonetic}

\begin{EntryWithPhonetic}{何典}{he2 dian3}{7,8}{⼈、⼋}
  \definition*{s.}{He Dian; este é um romance clássico chinês único, com uma arte inesquecível e um estilo humorístico único; o romance satiriza o submundo}
\end{EntryWithPhonetic}

\begin{EntryWithPhonetic}{何故}{he2gu4}{7,9}{⼈、⽁}
  \definition{adv.}{qual razão?; por que? | para quê? | qual é o motivo?}
\end{EntryWithPhonetic}

\begin{EntryWithPhonetic}{何苦}{he2ku3}{7,8}{⼈、⾋}[HSK 7-9]
  \definition{adv.}{por que se preocupar?; vale a pena o esforço?; use uma pergunta retórica para expressar que não vale a pena e por que se preocupar}[为了这点事,何苦生气呢?===Por que ficar bravo com uma coisa dessas?]
\end{EntryWithPhonetic}

\begin{EntryWithPhonetic}{何况}{he2kuang4}{7,7}{⼈、⼎}[HSK 7-9]
  \definition{conj.}{muito menos; use um tom interrogativo para expressar que é mais óbvio e razoável em comparação | além disso; além do mais; indique outras razões ou razões adicionais}
\end{EntryWithPhonetic}

\begin{EntryWithPhonetic}{何时}{he2shi2}{7,7}{⼈、⽇}[HSK 7-9]
  \definition{adv.}{quando?; que horas?}[我何时能再见到你?===Quando poderei vê-lo novamente?]
\end{EntryWithPhonetic}

\begin{EntryWithPhonetic}{和}{he2}{8}{⼝}[HSK 1]
  \definition*{s.}{Sobrenome He}
  \definition{adj.}{gentil; suave; amável | harmonioso; em boas condições}
  \definition{conj.}{e (somente para palavras); unidos com}
  \definition{prep.}{relacionado com | para; com; indica correlação; comparação, etc.}
  \definition{s.}{soma; soma total | japonês; refere-se ao Japão}
  \definition{v.}{disputar; reconciliar; acabar com a guerra ou a disputa | empatar; (próxima edição ou torneio) sem vencedor}
  \seeref{he4}
  \seeref{hu2}
  \seeref{huo2}
  \seeref{huo4}
\end{EntryWithPhonetic}

\begin{EntryWithPhonetic}{和蔼}{he2'ai3}{8,14}{⼝、⾋}[HSK 7-9]
  \definition{adj.}{gentil; afável; amável}
\end{EntryWithPhonetic}

\begin{EntryWithPhonetic}{和解}{he2jie3}{8,13}{⼝、⾓}[HSK 7-9]
  \definition{v.}{reconciliar}
\end{EntryWithPhonetic}

\begin{EntryWithPhonetic}{和睦}{he2mu4}{8,13}{⼝、⽬}[HSK 7-9]
  \definition{adj.}{harmonioso; amigável; amistoso}
  \definition{s.}{harmonia; concórdia}
\end{EntryWithPhonetic}

\begin{EntryWithPhonetic}{和平}{he2ping2}{8,5}{⼝、⼲}[HSK 3]
  \definition{adj.}{pacífico; moderado; não violento | pacífico; tranquilo; sereno}
  \definition{s.}{paz ;ausência de guerra}
\end{EntryWithPhonetic}

\begin{EntryWithPhonetic}{和平共处}{he2ping2 gong4chu3}{8,5,6,5}{⼝、⼲、⼋、⼡}[HSK 7-9]
  \definition{expr.}{coexistência pacífica de nações, sociedades etc.; refere-se a países com diferentes sistemas sociais que resolvem disputas pacificamente e desenvolvem laços econômicos e culturais com base na igualdade e no benefício mútuo}
\end{EntryWithPhonetic}

\begin{EntryWithPhonetic}{和气}{he2qi5}{8,4}{⼝、⽓}[HSK 7-9]
  \definition{adj.}{gentil; bondoso; educado | amigável; amável; harmonioso}
  \definition{s.}{amizade; relações harmoniosas; atmosfera harmoniosa; sentimentos harmoniosos}
\end{EntryWithPhonetic}

\begin{EntryWithPhonetic}{和尚}{he2shang5}{8,8}{⼝、⼩}[HSK 7-9]
  \definition[个,名,位]{s.}{monge budista; refere-se aos monges budistas do sexo masculino que praticam o budismo}
\end{EntryWithPhonetic}

\begin{EntryWithPhonetic}{和谐}{he2xie2}{8,11}{⼝、⾔}[HSK 6]
  \definition{adj.}{harmonioso; sem conflitos | em perfeita harmonia; ajuste adequado e simétrico}
  \definition{v.}{(eufemismo) censurar}
\end{EntryWithPhonetic}

\begin{EntryWithPhonetic}{河}{he2}{8}{⽔}[HSK 2]
  \definition*{s.}{Astronomia: o sistema da Via Láctea | O Rio Amarelo; O Rio Huanghe | Sobrenome He}
  \definition[条,道]{s.}{rio; refere-se a grandes cursos de água}
\end{EntryWithPhonetic}

\begin{EntryWithPhonetic}{河蚌}{he2bang4}{8,10}{⽔、⾍}
  \definition{s.}{mexilhões | bivalves cultivados em rios e lagos}
\end{EntryWithPhonetic}

\begin{EntryWithPhonetic}{河流}{he2liu2}{8,10}{⽔、⽔}[HSK 7-9]
  \definition[条]{s.}{rio; córrego; um termo geral para grandes fluxos naturais de água (como rios, etc.) na superfície da Terra}
\end{EntryWithPhonetic}

\begin{EntryWithPhonetic}{河南}{he2nan2}{8,9}{⽔、⼗}
  \definition*{s.}{Província de Henan}
\end{EntryWithPhonetic}

\begin{EntryWithPhonetic}{河畔}{he2pan4}{8,10}{⽔、⽥}[HSK 7-9]
  \definition{s.}{planície fluvial | beira-rio}
\end{EntryWithPhonetic}

\begin{EntryWithPhonetic}{核}{he2}{10}{⽊}[HSK 7-9]
  \definition{adj.}{Literário: verdadeiro; fiel}
  \definition{s.}{poço; pedra; caroço | núcleo | núcleo atômico}
  \definition{v.}{examinar; verificar}
  \seeref{hu2}
\end{EntryWithPhonetic}

\begin{EntryWithPhonetic}{核电站}{he2dian4zhan4}{10,5,10}{⽊、⽥、⽴}[HSK 7-9]
  \definition{s.}{usina nuclear; usina que utiliza energia nuclear para gerar eletricidade}
\end{EntryWithPhonetic}

\begin{EntryWithPhonetic}{核对}{he2dui4}{10,5}{⽊、⼨}[HSK 7-9]
  \definition{v.}{verificar; checar; verificar cuidadosamente (para ver se corresponde)}
\end{EntryWithPhonetic}

\begin{EntryWithPhonetic}{核能}{he2neng2}{10,10}{⽊、⾁}[HSK 7-9]
  \definition{s.}{energia nuclear}
\end{EntryWithPhonetic}

\begin{EntryWithPhonetic}{核实}{he2shi2}{10,8}{⽊、⼧}[HSK 7-9]
  \definition{v.}{verificar; checar; verificar se é verdade}
\end{EntryWithPhonetic}

\begin{EntryWithPhonetic}{核桃}{he2tao5}{10,10}{⽊、⽊}[HSK 7-9]
  \definition[颗,个,棵,顆]{s.}{noz | nogueira}
\end{EntryWithPhonetic}

\begin{EntryWithPhonetic}{核武器}{he2wu3qi4}{10,8,16}{⽊、⽌、⼝}[HSK 7-9]
  \definition[个]{s.}{arma nuclear}
\end{EntryWithPhonetic}

\begin{EntryWithPhonetic}{核心}{he2xin1}{10,4}{⽊、⼼}[HSK 6]
  \definition[个]{s.}{núcleo; elite; coração; centro; parte principal (em termos de relacionamento entre as coisas)}
\end{EntryWithPhonetic}

\begin{EntryWithPhonetic}{荷}{he2}{10}{⾋}
  \definition*{s.}{Países Baixos; Holanda, abreviação de 荷兰 | Sobrenome He}
  \definition{s.}{lótus}
  \seeref{he4}
  \seealsoref{荷兰}{he2lan2}
\end{EntryWithPhonetic}

\begin{EntryWithPhonetic}{荷花}{he2hua1}{10,7}{⾋、⾋}[HSK 7-9]
  \definition[朵,枝,片]{s.}{lótus; flor de lótus}
\end{EntryWithPhonetic}

\begin{EntryWithPhonetic}{荷兰}{he2lan2}{10,5}{⾋、⼋}
  \definition*{s.}{Países Baixos; Holanda}
\end{EntryWithPhonetic}

\begin{EntryWithPhonetic}{盒}{he2}{11}{⽫}[HSK 5]
  \definition{clas.}{caixa (de pequena dimensão)}
  \definition[个]{s.}{caixa; estojo; recipiente; receptáculo}
\end{EntryWithPhonetic}

\begin{EntryWithPhonetic}{盒饭}{he2 fan4}{11,7}{⽫、⾷}[HSK 5]
  \definition[份]{s.}{refeição embalada; marmita; \emph{fast-food} vendida em caixas}
\end{EntryWithPhonetic}

\begin{EntryWithPhonetic}{盒子}{he2zi5}{11,3}{⽫、⼦}[HSK 5]
  \definition[个,只,堆]{s.}{caixa; recipiente que têm tampas na parte superior e podem conter coisas dentro, geralmente é pequeno e plano}
\end{EntryWithPhonetic}

\begin{EntryWithPhonetic}{和}{he4}{8}{⼝}
  \definition{v.}{compor um poema em resposta (ao poema de alguém) usando a mesma sequência de rimas | juntar-se à cantoria; cantar junto com outros em harmonia}
  \seeref{he2}
  \seeref{hu2}
  \seeref{huo2}
  \seeref{huo4}
\end{EntryWithPhonetic}

\begin{EntryWithPhonetic}{贺}{he4}{9}{⾙}
  \definition*{s.}{Sobrenome He}
  \definition{v.}{parabenizar; congratular | celebrar; comemorar}
\end{EntryWithPhonetic}

\begin{EntryWithPhonetic}{贺电}{he4dian4}{9,5}{⾙、⽥}[HSK 7-9]
  \definition[封]{s.}{mensagem de felicitações; telegrama de felicitações}
\end{EntryWithPhonetic}

\begin{EntryWithPhonetic}{贺卡}{he4 ka3}{9,5}{⾙、⼘}[HSK 5]
  \definition[张]{s.}{cartão de felicitações; pedaço de papel para parabenizar amigos e parentes em seu casamento, aniversário ou festivais, geralmente impresso com palavras e desenhos de felicitações}
\end{EntryWithPhonetic}

\begin{EntryWithPhonetic}{贺信}{he4xin4}{9,9}{⾙、⼈}[HSK 7-9]
  \definition{s.}{carta de felicitações; carta de congratulações}
\end{EntryWithPhonetic}

\begin{EntryWithPhonetic}{荷}{he4}{10}{⾋}
  \definition{s.}{fardo; responsabilidade}
  \definition{v.}{carregar no ombro ou nas costas | aceitar um favor, frequentemente usado em cartas para expressar cortesia}
  \seeref{he2}
\end{EntryWithPhonetic}

\begin{EntryWithPhonetic}{喝}{he4}{12}{⼝}
  \definition{v.}{gritar bem alto}
  \seeref{he1}
\end{EntryWithPhonetic}

\begin{EntryWithPhonetic}{喝采}{he4/cai3}{12,8}{⼝、⾤}[HSK 7-9]
  \definition{v.+compl.}{aclamar; aplaudir}
\end{EntryWithPhonetic}

\begin{EntryWithPhonetic}{喝彩}{he4cai3}{12,11}{⼝、⼺}
  \definition{s.}{aclamar | torcer}
\end{EntryWithPhonetic}

\begin{EntryWithPhonetic}{褐}{he4}{14}{⾐}
  \definition{adj.}{marrom; castanho; pardo}
  \definition{s.}{pano de cânhamo grosso}
\end{EntryWithPhonetic}

\begin{EntryWithPhonetic}{褐色}{he4 se4}{14,6}{⾐、⾊}
  \definition{s.}{cor marrom}
\end{EntryWithPhonetic}

\begin{EntryWithPhonetic}{赫}{he4}{14}{⾚}
  \definition*{s.}{Sobrenome He}
  \definition{adj.}{conspícuo; grandioso | vermelho brilhante e flamejante; vermelho como fogo}
  \definition{clas.}{Hz, hertz; abreviação de 赫兹}
  \seealsoref{赫兹}{he4zi1}
\end{EntryWithPhonetic}

\begin{EntryWithPhonetic}{赫然}{he4ran2}{14,12}{⾚、⽕}[HSK 7-9]
  \definition{adj.}{inesperado e chocante/impressionante; descreve algo que é muito marcante ou surpreendente | (raiva, etc.) terrível; violento; descreve o olhar de raiva | grande; eminente; florescente; excepcional; descreve a aparência de ser proeminente}
\end{EntryWithPhonetic}

\begin{EntryWithPhonetic}{赫兹}{he4zi1}{14,9}{⾚、⼋}
  \definition{s.}{hertz (Hz), unidade de frequência}
  \definition{s.}{Heinrich Hertz (1857-1894), físico e meteorologista alemão, pioneiro da radiação eletromagnética}
\end{EntryWithPhonetic}

\begin{EntryWithPhonetic}{鹤}{he4}{15}{⿃}
  \definition*{s.}{Sobrenome He}
  \definition[只]{s.}{grou (ave)}
\end{EntryWithPhonetic}

\begin{EntryWithPhonetic}{鹤立鸡群}{he4li4ji1qun2}{15,5,7,13}{⿃、⽴、⿃、⽺}[HSK 7-9]
  \definition{expr.}{destaque-se da multidão; manifestamente superior; muito acima do comum; como um guindaste em pé entre galinhas --- fique de pé acima dos outros}
\end{EntryWithPhonetic}

\begin{EntryWithPhonetic}{黑}{hei1}{12}{⿊}[HSK 2][Kangxi 203]
  \definition*{s.}{Província de Heilongjiang, abreviação de 黑龙江 | Sobrenome Hei}
  \definition{adj.}{preto; cor semelhante à do carvão | escuro | obscuro; secreto | perverso; sinistro; ruim; cruel | reacionário}
  \definition{s.}{noite}
  \definition{v.}{fazer algo ilegalmente ou de forma desonesta; enganar; desviar dinheiro ilegalmente | invadir (uma rede, sites, computador, etc.)}
  \seealsoref{黑龙江}{hei1long2jiang1}
\end{EntryWithPhonetic}

\begin{EntryWithPhonetic}{黑暗}{hei1 an4}{12,13}{⿊、⽇}[HSK 4]
  \definition{adj.}{escuro; sombrio; sem luz | maligno; corrupto; reacionário}
\end{EntryWithPhonetic}

\begin{EntryWithPhonetic}{黑白}{hei1bai2}{12,5}{⿊、⽩}[HSK 7-9]
  \definition[只]{s.}{preto e branco | certo e errado; metáfora para o certo e o errado, o bem e o mal}
\end{EntryWithPhonetic}

\begin{EntryWithPhonetic}{黑板}{hei1ban3}{12,8}{⿊、⽊}[HSK 2]
  \definition[块,个]{s.}{quadro negro; quadro de giz; uma placa, na qual se pode escrever com giz}
\end{EntryWithPhonetic}

\begin{EntryWithPhonetic}{黑客}{hei1ke4}{12,9}{⿊、⼧}[HSK 7-9]
  \definition[个,些,位,名]{s.}{Empréstimo linguístico: \emph{hacker}; \emph{cracker}; intruso cibernético; gênio da computação; originalmente se refere a pessoas que não são profissionais de informática, mas são muito proficientes em tecnologia de computadores; agora se refere especificamente a pessoas que podem escrever programas de descriptografia para invadir ilegalmente redes de computadores de outras pessoas para interferir ou destruí-las}
\end{EntryWithPhonetic}

\begin{EntryWithPhonetic}{黑龙江}{hei1long2jiang1}{12,5,6}{⿊、⿓、⽔}
  \definition*{s.}{Província de Heilongjiang | Rio Heilong Jiang;  Rio Amur (na Rússia)}
\end{EntryWithPhonetic}

\begin{EntryWithPhonetic}{黑马}{hei1ma3}{12,3}{⿊、⾺}[HSK 7-9]
  \definition[匹,群]{s.}{azarão (cavalo preto) | Figurativo: pessoa pouco conhecida que alcança sucesso inesperado}
\end{EntryWithPhonetic}

\begin{EntryWithPhonetic}{黑色}{hei1 se4}{12,6}{⿊、⾊}[HSK 2]
  \definition{adj.}{metafórico: suspeito, ilegal}
  \definition{s.}{cor preta}
\end{EntryWithPhonetic}

\begin{EntryWithPhonetic}{黑手}{hei1shou3}{12,4}{⿊、⼿}[HSK 7-9]
  \definition{s.}{mão negra; manipulador maligno dos bastidores | uma pessoa cruel manipulando alguém ou algo nos bastidores; uma metáfora para pessoas ou forças que secretamente realizam atividades de conspiração}
\end{EntryWithPhonetic}

\begin{EntryWithPhonetic}{黑桃}{hei1 tao2}{12,10}{⿊、⽊}
  \definition{s.}{espadas ♠ (em jogos de cartas)}
  \seealsoref{方片}{fang1 pian4}
  \seealsoref{红心}{hong2 xin1}
  \seealsoref{梅花}{mei2 hua1}
\end{EntryWithPhonetic}

\begin{EntryWithPhonetic}{黑心}{hei1xin1}{12,4}{⿊、⼼}[HSK 7-9]
  \definition{adj.}{malvado; perverso | ganancioso; avarento | (certos bens) de má qualidade | implacável e sem consciência | de mente viciosa cheia de ódio e ciúme}
  \definition{s.}{coração negro; mente maligna | núcleo preto (falha na cerâmica)}
\end{EntryWithPhonetic}

\begin{EntryWithPhonetic}{黑夜}{hei1 ye4}{12,8}{⿊、⼣}[HSK 6]
  \definition[个]{s.}{noite ; uma noite muito escura sem luz}
\end{EntryWithPhonetic}

\begin{EntryWithPhonetic}{嘿}{hei1}{15}{⼝}[HSK 7-9]
  \definition{interj.}{Ei!; indicando uma saudação ou chamar a atenção | expressando orgulho ou satisfação | expressando espanto, surpresa}
  \seeref{mo4}
\end{EntryWithPhonetic}

\begin{EntryWithPhonetic}{痕}{hen2}{11}{⽧}
  \definition[个]{s.}{marca; traço}
\end{EntryWithPhonetic}

\begin{EntryWithPhonetic}{痕迹}{hen2ji4}{11,9}{⽧、⾡}[HSK 7-9]
  \definition{s.}{marca; traço; a marca deixada por um objeto que entra em contato com outro | traço; rastro; vestígio; coisas ou fenômenos deixados por algo que já existiu}
\end{EntryWithPhonetic}

\begin{EntryWithPhonetic}{很}{hen3}{9}{⼻}[HSK 1]
  \definition{adv.}{muito; bastante; terrivelmente; indica um grau bastante elevado; definitivo; o mais alto}
\end{EntryWithPhonetic}

\begin{EntryWithPhonetic}{很难说}{hen3 nan2 shuo1}{9,10,9}{⼻、⾫、⾔}[HSK 6]
  \definition{adj.}{difícil dizer}
\end{EntryWithPhonetic}

\begin{EntryWithPhonetic}{狠}{hen3}{9}{⽝}[HSK 6]
  \definition{adj.}{impiedoso; implacável; feroz | firme; resoluto; severo; determinado}
  \definition{adv.}{muito; bastante; bastante | também, frequentemente usado antes de um adjetivo sem intensificar seu significado, ou seja, como um elemento sintático sem sentido}
  \definition{v.}{endurecer (o coração); suprimir (os próprios sentimentos)}
  \variantof{很}
  \seealsoref{很}{hen3}
\end{EntryWithPhonetic}

\begin{EntryWithPhonetic}{恨}{hen4}{9}{⼼}[HSK 5]
  \definition{s.}{ódio; resentimento}
  \definition{v.}{odiar; ressentir-se}
\end{EntryWithPhonetic}

\begin{EntryWithPhonetic}{恨不得}{hen4bu5de5}{9,4,11}{⼼、⼀、⼻}[HSK 7-9]
  \definition{v.}{estar muito ansioso para; querer poder (fazer algo); mal poder esperar para; expressa um desejo ansioso de realizar algo, geralmente usado para coisas que não podem realmente ser feitas}
\end{EntryWithPhonetic}

\begin{EntryWithPhonetic}{哼}{heng1}{10}{⼝}[HSK 7-9]
  \definition{v.}{gemer; bufar | cantarolar}
  \seeref{hng5}
\end{EntryWithPhonetic}

\begin{EntryWithPhonetic}{行}{heng2}{6}{⾏}[Kangxi 144]
  \definition{s.}{usado em 道行}
  \seeref{hang2}
  \seeref{xing2}
  \seealsoref{道行}{dao4 heng2}
\end{EntryWithPhonetic}

\begin{EntryWithPhonetic}{恒}{heng2}{9}{⼼}
  \definition*{s.}{Sobrenome Heng}
  \definition{adj.}{permanente; duradouro | usual; comum; constante | usual; frequente; constante}
  \definition{s.}{perseverança; constância}
\end{EntryWithPhonetic}

\begin{EntryWithPhonetic}{恒星系}{heng2xing1xi4}{9,9,7}{⼼、⽇、⽷}
  \definition{s.}{sistema estelar | galáxia}
\end{EntryWithPhonetic}

\begin{EntryWithPhonetic}{横}{heng2}{15}{⽊}[HSK 6]
  \definition{adj.}{horizontal; transversal; paralelo ao plano horizontal (oposto de 竖 e 直) | em ângulo reto com; direção esquerda-direita (em oposição à 竖, 直 ou 纵) | e leste a oeste ou de oeste a leste; direção leste-oeste (oposta a 纵) | desenfreado; turbulento | violento; feroz; irracional}
  \definition{adv.}{de qualquer forma; em qualquer caso | provavelmente; muito provavelmente}
  \definition{s.}{traço horizontal (em caracteres chineses)}
  \definition{v.}{deitar-se transversalmente; estar de lado | colocar algo transversalmente (ou horizontalmente)}
  \seeref{heng4}
  \seealsoref{竖}{shu4}
  \seealsoref{直}{zhi2}
  \seealsoref{纵}{zong4}
\end{EntryWithPhonetic}

\begin{EntryWithPhonetic}{横七竖八}{heng2qi1-shu4ba1}{15,2,9,2}{⽊、⼀、⽴、⼋}[HSK 7-9]
  \definition{expr.}{em desordem; em seis e sete; desorganizado}
\end{EntryWithPhonetic}

\begin{EntryWithPhonetic}{横竖}{heng2shu5}{15,9}{⽊、⽴}
  \definition{adv.}{de qualquer forma; em qualquer maneira; isso significa que não importa o que aconteça, o resultado ou a conclusão não mudará; equivale a 反正}
  \seealsoref{反正}{fan3zheng4}
\end{EntryWithPhonetic}

\begin{EntryWithPhonetic}{横向}{heng2xiang4}{15,6}{⽊、⼝}[HSK 7-9]
  \definition{adj.}{horizontal; transversal (oposto a 竖向,纵向) | lateral | ortogonal | perpendicular}
  \seealsoref{竖向}{shu4xiang4}
  \seealsoref{纵向}{zong4xiang4}
\end{EntryWithPhonetic}

\begin{EntryWithPhonetic}{衡}{heng2}{16}{⾏}
  \definition*{s.}{Sobrenome Heng}
  \definition[个]{s.}{braço graduado de uma balança | balança; aparelho de pesagem}
  \definition{v.}{pesar; medir; julgar}
\end{EntryWithPhonetic}

\begin{EntryWithPhonetic}{衡量}{heng2 liang2}{16,12}{⾏、⾥}[HSK 6]
  \definition{v.}{pesar; medir; comparar; avaliar | considerar; pensar sobre; deliberar}
\end{EntryWithPhonetic}

\begin{EntryWithPhonetic}{横}{heng4}{15}{⽊}[HSK 7-9]
  \definition{adj.}{chocante e irracional; inesperado}
  \seeref{heng2}
\end{EntryWithPhonetic}

\begin{EntryWithPhonetic}{哼}{hng5}{10}{⼝}
  \definition{interj.}{Hmm; Humph; expressa insatisfação, desprezo, desdém ou indignação}
\end{EntryWithPhonetic}

\begin{EntryWithPhonetic}{轰}{hong1}{8}{⾞}[HSK 7-9]
  \definition{interj.}{Onomatopéia: Bum!; Bang!; refere-se aos ruídos altos feitos por trovões, fogo de artilharia, etc.}
  \definition{v.}{retumbar; trovejar; bombardear; explodir | espantar; expulsar}
\end{EntryWithPhonetic}

\begin{EntryWithPhonetic}{轰动}{hong1dong4}{8,6}{⾞、⼒}[HSK 7-9]
  \definition{v.}{causar (criar) uma sensação; fazer um rebuliço; criar um rebuliço}
\end{EntryWithPhonetic}

\begin{EntryWithPhonetic}{轰鸣}{hong1ming2}{8,8}{⾞、⿃}
  \definition{s.}{trovão; rugido}
  \definition{v.}{rosnar; rugir; trovejar}
\end{EntryWithPhonetic}

\begin{EntryWithPhonetic}{轰炸}{hong1zha4}{8,9}{⾞、⽕}[HSK 7-9]
  \definition{v.}{bombardear; lançar bombas de aeronaves sobre vários alvos no solo ou na água}
\end{EntryWithPhonetic}

\begin{EntryWithPhonetic}{轰炸机}{hong1zha4ji1}{8,9,6}{⾞、⽕、⽊}
  \definition{s.}{bombardeiro (aeronave)}
\end{EntryWithPhonetic}

\begin{EntryWithPhonetic}{哄}{hong1}{9}{⼝}[HSK 7-9]
  \definition{interj.}{Onomatopéia: gargalhadas ou alvoroço}
  \definition{s.}{rugido; clamor; comoção}
  \seeref{hong3}
  \seeref{hong4}
\end{EntryWithPhonetic}

\begin{EntryWithPhonetic}{哄堂大笑}{hong1tang2-da4xiao4}{9,11,3,10}{⼝、⼟、⼤、⽵}[HSK 7-9]
  \definition{expr.}{fazer a sala inteira rugir (em alvoroço); (causar) uma explosão geral de risos; ``Todos caíram na gargalhada.''; uma explosão de risos na plateia; ``A plateia caiu na gargalhada.''; ``As pessoas de toda a casa caíram na gargalhada.''; ``A sala inteira riu (balançando).''}
\end{EntryWithPhonetic}

\begin{EntryWithPhonetic}{烘}{hong1}{10}{⽕}
  \definition{v.}{secar; assar; aquecer; usar fogo ou vapor para aquecer o corpo ou para cozinhar, aquecer ou secar algo | destacar}
\end{EntryWithPhonetic}

\begin{EntryWithPhonetic}{烘干}{hong1gan1}{10,3}{⽕、⼲}[HSK 7-9]
  \definition{v.}{secar em fogo alto | secar ao lado ou sobre o fogo | assar; secar no forno}
\end{EntryWithPhonetic}

\begin{EntryWithPhonetic}{烘托}{hong1tuo1}{10,6}{⽕、⼿}[HSK 7-9]
  \definition{v.}{adicionar sombreamento ao redor de um objeto para destacá-lo; um dos métodos de pintura chinesa, que utiliza tinta ou cores claras para pontilhar o contorno do objeto e torná-lo mais claro | destacar por contraste; colocar em nítido relevo; fazer com que se destaque}
\end{EntryWithPhonetic}

\begin{EntryWithPhonetic}{弘}{hong2}{5}{⼸}
  \definition{adj.}{grande; grandioso; magnífico}
  \definition{s.}{Sobrenome Hong}
  \definition{v.}{ampliar; expandir | promover}
\end{EntryWithPhonetic}

\begin{EntryWithPhonetic}{弘扬}{hong2yang2}{5,6}{⼸、⼿}[HSK 7-9]
  \definition{v.}{melhorar; levar adiante; desenvolver e expandir; promover; promover vigorosamente}
\end{EntryWithPhonetic}

\begin{EntryWithPhonetic}{红}{hong2}{6}{⽷}[HSK 2]
  \definition*{s.}{Sobrenome Hong}
  \definition{adj.}{vermelho | popular; bem-sucedido; símbolo de sucesso ou valorização | vermelho; revolucionário; símbolo da revolução | festivo; símbolo de alegria}
  \definition{s.}{tecido vermelho, bandeirinhas, etc. usados em ocasiões festivas | bônus; dividendo}
\end{EntryWithPhonetic}

\begin{EntryWithPhonetic}{红包}{hong2 bao1}{6,5}{⽷、⼓}[HSK 4]
  \definition[个]{s.}{saco de papel vermelho ou envelope contendo dinheiro como presente, gorjeta ou bônus | suborno; propina}
\end{EntryWithPhonetic}

\begin{EntryWithPhonetic}{红宝石}{hong2bao3shi2}{6,8,5}{⽷、⼧、⽯}
  \definition{s.}{rubi}
\end{EntryWithPhonetic}

\begin{EntryWithPhonetic}{红茶}{hong2 cha2}{6,9}{⽷、⾋}[HSK 3]
  \definition[杯,壶,斤,种]{s.}{chá preto; chá acabado produzido através de fermentação completa}
\end{EntryWithPhonetic}

\begin{EntryWithPhonetic}{红灯}{hong2deng1}{6,6}{⽷、⽕}[HSK 7-9]
  \definition[盏]{s.}{semáforo vermelho | Figurativo: barreira; proibição | luz vermelha}
\end{EntryWithPhonetic}

\begin{EntryWithPhonetic}{红火}{hong2huo5}{6,4}{⽷、⽕}[HSK 7-9]
  \definition{adj.}{florescente; próspero; descreve prosperidade, prosperidade e agitação | florescente; próspero (meio de vida, carreira)}
\end{EntryWithPhonetic}

\begin{EntryWithPhonetic}{红酒}{hong2 jiu3}{6,10}{⽷、⾣}[HSK 3]
  \definition[瓶,杯,壶,斤,箱]{s.}{vinho tinto}
\end{EntryWithPhonetic}

\begin{EntryWithPhonetic}{红绿灯}{hong2lv4deng1}{6,11,6}{⽷、⽷、⽕}
  \definition[个]{s.}{semáforo | sinal de trânsito}
\end{EntryWithPhonetic}

\begin{EntryWithPhonetic}{红扑扑}{hong2pu1pu1}{6,5,5}{⽷、⼿、⼿}[HSK 7-9]
  \definition{adj.}{corado | vermelho | rosado}
\end{EntryWithPhonetic}

\begin{EntryWithPhonetic}{红润}{hong2run4}{6,10}{⽷、⽔}[HSK 7-9]
  \definition[张]{adj.}{avermelhado; rosado | suave, macio e rosado (pele, bochechas, etc.)}
\end{EntryWithPhonetic}

\begin{EntryWithPhonetic}{红色}{hong2 se4}{6,6}{⽷、⾊}[HSK 2]
  \definition{adj.}{vermelho; revolucionário; símbolo da revolução ou da consciência política elevada}
  \definition{s.}{cor vermelha}
\end{EntryWithPhonetic}

\begin{EntryWithPhonetic}{红烧}{hong2shao1}{6,10}{⽷、⽕}
  \definition{s.}{guisado em molho de soja (prato)}
\end{EntryWithPhonetic}

\begin{EntryWithPhonetic}{红薯}{hong2shu3}{6,16}{⽷、⾋}[HSK 7-9]
  \definition[个]{s.}{batata doce}
\end{EntryWithPhonetic}

\begin{EntryWithPhonetic}{红线}{hong2xian4}{6,8}{⽷、⽷}
  \definition{s.}{linha vermelha}
\end{EntryWithPhonetic}

\begin{EntryWithPhonetic}{红心}{hong2 xin1}{6,4}{⽷、⼼}
  \definition{s.}{coração vermelho, um coração leal à causa da revolução proletária | alvo | coração ♥ (em jogos de cartas) | red, heart-shaped symbol}
  \seealsoref{方片}{fang1 pian4}
  \seealsoref{黑桃}{hei1 tao2}
  \seealsoref{梅花}{mei2 hua1}
\end{EntryWithPhonetic}

\begin{EntryWithPhonetic}{红眼}{hong2yan3}{6,11}{⽷、⽬}[HSK 7-9]
  \definition{s.}{conjuntivite | Figurativo: inveja; ciúme}
  \definition{v.}{ficar furioso | Dialeto: ter inveja; ter ciúmes de}
\end{EntryWithPhonetic}

\begin{EntryWithPhonetic}{宏}{hong2}{7}{⼧}
  \definition*{s.}{Sobrenome Hong}
  \definition{adj.}{grande; grandioso; magnífico}
  \definition{v.}{divulgar algo; promover algo; atualmente, geralmente é escrito como 弘}
  \seealsoref{弘}{hong2}
\end{EntryWithPhonetic}

\begin{EntryWithPhonetic}{宏大}{hong2 da4}{7,3}{⼧、⼤}[HSK 6]
  \definition{adj.}{grande; ótimo | imenso; vasto}
\end{EntryWithPhonetic}

\begin{EntryWithPhonetic}{宏观}{hong2guan1}{7,6}{⼧、⾒}[HSK 7-9]
  \definition{adj.}{macroscópico; grande norma, relacionada ao todo}
  \definition{pref.}{macro-}
\end{EntryWithPhonetic}

\begin{EntryWithPhonetic}{宏伟}{hong2wei3}{7,6}{⼧、⼈}[HSK 7-9]
  \definition{adj.}{grandioso; magnífico; (escala, planta, etc.) magnífico e grandioso}
\end{EntryWithPhonetic}

\begin{EntryWithPhonetic}{洪}{hong2}{9}{⽔}
  \definition*{s.}{Sobrenome Hong}
  \definition{adj.}{alto; vasto | grande; grandioso}
  \definition[场]{s.}{enchente; inundação}
\end{EntryWithPhonetic}

\begin{EntryWithPhonetic}{洪亮}{hong2liang4}{9,9}{⽔、⼇}[HSK 7-9]
  \definition{adj.}{alto e claro; ressonante; sonoro}
\end{EntryWithPhonetic}

\begin{EntryWithPhonetic}{洪水}{hong2shui3}{9,4}{⽔、⽔}[HSK 6]
  \definition[场]{s.}{dilúvio; inundação; enchente; um aumento repentino em um rio causado por chuva forte ou derretimento de neve}
\end{EntryWithPhonetic}

\begin{EntryWithPhonetic}{哄}{hong3}{9}{⼝}[HSK 7-9]
  \definition{v.}{brincar; enganar; tapear | persuadir; agradar os outros com palavras ou ações, especialmente observando ou cuidando de crianças}
  \seeref{hong1}
  \seeref{hong4}
\end{EntryWithPhonetic}

\begin{EntryWithPhonetic}{哄}{hong4}{9}{⼝}[HSK 7-9]
  \definition{s.}{comoção; tumulto}
  \seeref{hong1}
  \seeref{hong3}
\end{EntryWithPhonetic}

\begin{EntryWithPhonetic}{喉}{hou2}{12}{⼝}
  \definition{s.}{laringe; garganta; a parte do órgão respiratório de humanos e vertebrados terrestres, localizada entre a faringe e a traqueia, tem as funções de ventilação e pronúncia; a faringe e a laringe são geralmente misturadas e chamadas de garganta ou caixa vocal}
\end{EntryWithPhonetic}

\begin{EntryWithPhonetic}{喉咙}{hou2long2}{12,8}{⼝、⼝}[HSK 7-9]
  \definition{s.}{garganta; laringe}
\end{EntryWithPhonetic}

\begin{EntryWithPhonetic}{猴}{hou2}{12}{⽝}[HSK 5]
  \definition{adj.}{esperto; inteligente; perspicaz | travesso (menino)}
  \definition[只,群]{s.}{macaco}
\end{EntryWithPhonetic}

\begin{EntryWithPhonetic}{猴子}{hou2zi5}{12,3}{⽝、⼦}
  \definition[只]{s.}{macaco}
\end{EntryWithPhonetic}

\begin{EntryWithPhonetic}{吼}{hou3}{7}{⼝}[HSK 7-9]
  \definition{v.}{rugido; uivo | rugido; gritar alto quando estiver com raiva}
\end{EntryWithPhonetic}

\begin{EntryWithPhonetic}{后}{hou4}{6}{⼝}[HSK 1]
  \definition*{s.}{Sobrenome Hou}
  \definition{s.}{atrás; traseiro; a direção oposta àquela para a qual a pessoa está voltada; a direção oposta àquela para a qual a parte de trás de uma casa está voltada (o oposto de 前)  | depois; mais tarde no tempo; futuro (em oposição a 先 ou 前) | último | posteridade; descendência | rainha; imperatriz | governante; soberano; monarca antigo}
  \seealsoref{前}{qian2}
  \seealsoref{先}{xian1}
\end{EntryWithPhonetic}

\begin{EntryWithPhonetic}{后备}{hou4bei4}{6,8}{⼝、⼡}[HSK 7-9]
  \definition{s.}{reserva; preparado para reabastecimento (pessoal, suprimentos, etc.)}
\end{EntryWithPhonetic}

\begin{EntryWithPhonetic}{后备箱}{hou4bei4xiang1}{6,8,15}{⼝、⼡、⾋}[HSK 7-9]
  \definition{s.}{porta-malas (de um carro)}
\end{EntryWithPhonetic}

\begin{EntryWithPhonetic}{后边}{hou4 bian5}{6,5}{⼝、⾡}[HSK 1]
  \definition{adv.}{costas; traseira; atrás}
\end{EntryWithPhonetic}

\begin{EntryWithPhonetic}{后代}{hou4dai4}{6,5}{⼝、⼈}[HSK 7-9]
  \definition{s.}{períodos posteriores (na história); eras posteriores; a era após uma certa era | gerações posteriores; posteridade; descendentes; gerações futuras}
\end{EntryWithPhonetic}

\begin{EntryWithPhonetic}{后盾}{hou4dun4}{6,9}{⼝、⽬}[HSK 7-9]
  \definition{s.}{apoio; força de apoio | suporte; assistência; suporte; apoiador}
\end{EntryWithPhonetic}

\begin{EntryWithPhonetic}{后方}{hou4 fang1}{6,4}{⼝、⽅}
  \definition{s.}{traseira; retaguarda (oposto à 前线 e 前方) | na parte de trás; na parte traseira}
  \seealsoref{前方}{qian2 fang1}
  \seealsoref{前线}{qian2 xian4}
\end{EntryWithPhonetic}

\begin{EntryWithPhonetic}{后顾之忧}{hou4gu4zhi1you1}{6,10,3,7}{⼝、⾴、⼂、⼼}[HSK 7-9]
  \definition{expr.}{preocupações com o que ficou para trás; preocupações com problemas futuros; preocupações persistentes; preocupações não resolvidas; preocupações ou problemas potenciais; ``Ansiedade que exige olhar para trás.''; refere-se a preocupações com o lar, a família ou o futuro que surgem ao seguir em frente ou sair}
\end{EntryWithPhonetic}

\begin{EntryWithPhonetic}{后果}{hou4guo3}{6,8}{⼝、⽊}[HSK 3]
  \definition{s.}{consequência; resultado (geralmente negativo)}
\end{EntryWithPhonetic}

\begin{EntryWithPhonetic}{后悔}{hou4hui3}{6,10}{⼝、⼼}[HSK 5]
  \definition{v.}{lamentar; ter remorso; arrepender-se}
\end{EntryWithPhonetic}

\begin{EntryWithPhonetic}{后来}{hou4lai2}{6,7}{⼝、⽊}[HSK 2]
  \definition{adv.}{mais tarde; depois; refere-se a um período posterior a um determinado momento no passado}
\end{EntryWithPhonetic}

\begin{EntryWithPhonetic}{后面}{hou4mian4}{6,9}{⼝、⾯}
  \definition{adv.}{parte de trás; retaguarda; atrás; a parte posterior do espaço ou localização | mais tarde; depois; no futuro; a parte posterior de um artigo ou discurso em relação ao que está sendo narrado no momento}
\end{EntryWithPhonetic}

\begin{EntryWithPhonetic}{后年}{hou4nian2}{6,6}{⼝、⼲}[HSK 3]
  \definition{s.}{daqui a dois anos; no ano seguinte ao próximo ano}
\end{EntryWithPhonetic}

\begin{EntryWithPhonetic}{后期}{hou4qi1}{6,12}{⼝、⽉}[HSK 7-9]
  \definition{s.}{estágio posterior; período posterior; a última fase de um período}
\end{EntryWithPhonetic}

\begin{EntryWithPhonetic}{后勤}{hou4qin2}{6,13}{⼝、⼒}[HSK 7-9]
  \definition{s.}{logística; serviços de retaguarda; todo o trabalho de fornecimento de áreas distantes da linha de frente para as áreas de linha de frente; trabalho administrativo em agências governamentais, empresas, etc., incluindo finanças, reparos, etc.}
\end{EntryWithPhonetic}

\begin{EntryWithPhonetic}{后人}{hou4ren2}{6,2}{⼝、⼈}[HSK 7-9]
  \definition{s.}{gerações posteriores; gerações futuras | posteridade; descendentes; futuridade}
\end{EntryWithPhonetic}

\begin{EntryWithPhonetic}{后台}{hou4tai2}{6,5}{⼝、⼝}[HSK 7-9]
  \definition{s.}{bastidores; plano de fundo | apoiador de bastidores; apoiador dos bastidores; uma metáfora para uma pessoa ou grupo que manipula ou apoia algo nos bastidores}
\end{EntryWithPhonetic}

\begin{EntryWithPhonetic}{后天}{hou4 tian1}{6,4}{⼝、⼤}[HSK 1]
  \definition{s.}{depois de amanhã; período em que uma pessoa ou animal vive e cresce sozinho após deixar o útero materno (em oposição a 先天)}
  \seealsoref{先天}{xian1tian1}
\end{EntryWithPhonetic}

\begin{EntryWithPhonetic}{后头}{hou4 tou5}{6,5}{⼝、⼤}[HSK 4]
  \definition{adv.}{posteriormente; atrás; mais tarde}
  \definition{s.}{a parte de trás; a parte traseira}
\end{EntryWithPhonetic}

\begin{EntryWithPhonetic}{后退}{hou4tui4}{6,9}{⼝、⾡}[HSK 7-9]
  \definition{v.}{recuar; retrocerder; retornar (para um lugar posterior ou para um estágio anterior de desenvolvimento)}
\end{EntryWithPhonetic}

\begin{EntryWithPhonetic}{后续}{hou4xu4}{6,11}{⼝、⽷}[HSK 7-9]
  \definition{adj.}{subsequente; de acompanhamento; decorrente; seguimento}
  \definition{v.}{casar novamente após a morte da esposa}
\end{EntryWithPhonetic}

\begin{EntryWithPhonetic}{后遗症}{hou4yi2zheng4}{6,12,10}{⼝、⾡、⽧}[HSK 7-9]
  \definition{s.}{sequelas; sintomas como defeitos ou disfunções orgânicas que permanecem após a recuperação de certas doenças | ressaca; efeito colateral; consequência; efeito residual}
\end{EntryWithPhonetic}

\begin{EntryWithPhonetic}{后裔}{hou4yi4}{6,13}{⼝、⾐}[HSK 7-9]
  \definition{s.}{descendente (de uma pessoa morta); prole | descendente; posteridade; progênie}
\end{EntryWithPhonetic}

\begin{EntryWithPhonetic}{后者}{hou4zhe3}{6,8}{⼝、⽼}[HSK 7-9]
  \definition{pron.}{o último; a última de duas ou mais pessoas ou coisas mencionadas ou autoevidentes}
  \definition{s.}{o último (oposto a 前者)}
  \seealsoref{前者}{qian2zhe3}
\end{EntryWithPhonetic}

\begin{EntryWithPhonetic}{厚}{hou4}{9}{⼚}[HSK 4]
  \definition*{s.}{Sobrenome Hou}
  \definition{adj.}{espesso; grosso (oposto a 薄) | profundo | gentil; magnânimo | grande; generoso | rico ou forte em sabor}
  \definition[米,厘米]{s.}{espessura | profundidade}
  \definition{v.}{favorecer; enfatizar}
  \seealsoref{薄}{bao2}
\end{EntryWithPhonetic}

\begin{EntryWithPhonetic}{厚道}{hou4dao5}{9,12}{⼚、⾡}[HSK 7-9]
  \definition{adj.}{honesto e gentil; sincero e generoso}
\end{EntryWithPhonetic}

\begin{EntryWithPhonetic}{厚度}{hou4du4}{9,9}{⼚、⼴}[HSK 7-9]
  \definition{s.}{espessura; a distância entre a parte superior e inferior de um objeto plano}
\end{EntryWithPhonetic}

\begin{EntryWithPhonetic}{候}{hou4}{10}{⼈}
  \definition{s.}{tempo; estação | condição; estado | situação meteorológica | uma unidade tradicional de tempo no antigo calendário chinês; antigamente, cinco dias constituíam uma estação, o que ainda é usado na meteorologia hoje em dia}
  \definition{s.}{Sobrenome Hou}
  \definition{v.}{esperar; aguardar | perguntar depois | assistir; observar}
\end{EntryWithPhonetic}

\begin{EntryWithPhonetic}{候选人}{hou4xuan3ren2}{10,9,2}{⼈、⾡、⼈}[HSK 7-9]
  \definition[个,名,位]{s.}{candidato}
\end{EntryWithPhonetic}

\begin{EntryWithPhonetic}{呼}{hu1}{8}{⼝}
  \definition*{s.}{Sobrenome Hu}
  \definition{s.}{Onomatopéia: descreve o som do vento}
  \definition{v.}{expirar | gritar; clamar | chamar; ligar; ligar para alguém}
\end{EntryWithPhonetic}

\begin{EntryWithPhonetic}{呼风唤雨}{hu1feng1-huan4yu3}{8,4,10,8}{⼝、⾵、⼝、⾬}[HSK 7-9]
  \definition{expr.}{``Fazer vento e chover.''; refere-se originalmente ao poder mágico de imortais e taoístas; atualmente, é usado como metáfora para a capacidade de controlar a natureza e, às vezes, como metáfora para a realização de atividades inflamatórias; invocar vento e chuva --- exercer poderes mágicos; causar problemas}
\end{EntryWithPhonetic}

\begin{EntryWithPhonetic}{呼唤}{hu1huan4}{8,10}{⼝、⼝}[HSK 7-9]
  \definition{v.}{chamar; gritar para}
\end{EntryWithPhonetic}

\begin{EntryWithPhonetic}{呼救}{hu1jiu4}{8,11}{⼝、⽁}[HSK 7-9]
  \definition{v.}{pedir ajuda; enviar sinais de SOS}
\end{EntryWithPhonetic}

\begin{EntryWithPhonetic}{呼啦啦}{hu1 la1 la1}{8,11,11}{⼝、⼝、⼝}
  \definition{s.}{Onomatopéia: som de bater asas}
\end{EntryWithPhonetic}

\begin{EntryWithPhonetic}{呼声}{hu1sheng1}{8,7}{⼝、⼠}[HSK 7-9]
  \definition[片]{s.}{choro; voz}[良心的呼声。===A voz da consciência.]
\end{EntryWithPhonetic}

\begin{EntryWithPhonetic}{呼吸}{hu1xi1}{8,6}{⼝、⼝}[HSK 4]
  \definition{s.}{um suspiro; metáfora para um período de tempo muito curto}
  \definition{v.}{respirar}
\end{EntryWithPhonetic}

\begin{EntryWithPhonetic}{呼啸}{hu1xiao4}{8,11}{⼝、⼝}
  \definition{v.}{assobiar}
\end{EntryWithPhonetic}

\begin{EntryWithPhonetic}{呼应}{hu1ying4}{8,7}{⼝、⼴}[HSK 7-9]
  \definition{v.}{ecoar; trabalhar em conjunto (com alguém); entrar em contato ou cuidar um do outro um dia de cada vez}
\end{EntryWithPhonetic}

\begin{EntryWithPhonetic}{呼吁}{hu1yu4}{8,6}{⼝、⼝}[HSK 7-9]
  \definition{v.}{apelar; chamar; apelar a um indivíduo ou sociedade, solicitar assistência ou hospedar um apelo a um indivíduo ou sociedade, na esperança de ganhar simpatia e apoio}
\end{EntryWithPhonetic}

\begin{EntryWithPhonetic}{忽}{hu1}{8}{⼼}
  \definition*{s.}{Sobrenome Hu}
  \definition{adv.}{agora\dots, agora\dots | de repente; subitamente}[天气忽冷忽热。===O clima está frio em um minuto e quente no outro.]
  \definition{v.}{negligenciar; ignorar; não prestar atenção; não levar a sério}
\end{EntryWithPhonetic}

\begin{EntryWithPhonetic}{忽高忽低}{hu1gao1-hu1di1}{8,10,8,7}{⼼、⾼、⼼、⼈}[HSK 7-9]
  \definition{expr.}{``Altos e baixos.''; ora alto, ora baixo}
\end{EntryWithPhonetic}

\begin{EntryWithPhonetic}{忽略}{hu1lve4}{8,11}{⼼、⽥}[HSK 6]
  \definition{v.}{negligenciar; ignorar; não perceber}
\end{EntryWithPhonetic}

\begin{EntryWithPhonetic}{忽然}{hu1ran2}{8,12}{⼼、⽕}[HSK 2]
  \definition{adv.}{repentinamente; de repente; sem aviso prévio; significa que algo aconteceu de forma rápida e inesperada}
\end{EntryWithPhonetic}

\begin{EntryWithPhonetic}{忽视}{hu1shi4}{8,8}{⼼、⾒}[HSK 4]
  \definition{v.}{ignorar; negligenciar; menosprezar; desprezar; dar de ombros}
\end{EntryWithPhonetic}

\begin{EntryWithPhonetic}{忽悠}{hu1you5}{8,11}{⼼、⼼}[HSK 7-9]
  \definition{v.}{balançar; cintilar; sacudir | enganar; enganar alguém}
\end{EntryWithPhonetic}

\begin{EntryWithPhonetic}{糊}{hu1}{15}{⽶}
  \definition{v.}{colar; untar; usar uma pasta mais espessa para revestir costuras, furos ou superfícies planas}
  \seeref{hu2}
  \seeref{hu4}
\end{EntryWithPhonetic}

\begin{EntryWithPhonetic}{和}{hu2}{8}{⼝}
  \definition{v.}{completar um conjunto de Mahjong, 麻将, ou cartas de baralho}
  \seeref{he2}
  \seeref{he4}
  \seeref{huo2}
  \seeref{huo4}
  \seealsoref{麻将}{ma2jiang4}
\end{EntryWithPhonetic}

\begin{EntryWithPhonetic}{胡}{hu2}{9}{⾁}
  \definition*{s.}{Sobrenome Hu}
  \definition{adj.}{introduzidos de nacionalidades do norte e do oeste ou do exterior | nos tempos antigos, o termo "Oriente e Ocidente" se referia às minorias étnicas do norte e do oeste, e também, de modo geral, às pessoas do exterior}
  \definition{adv.}{imprudentemente; desenfreadamente; escandalosamente; sem lei, ordem ou razão}
  \definition{pron.}{Por que?; palavras interrogativas: 为什么, 何故}
  \definition{s.}{nos tempos antigos, geralmente se referia às minorias étnicas do norte e do oeste | violino chinês | barba; bigode}
  \seealsoref{何故}{he2gu4}
  \seealsoref{为什么}{wei4shen2me5}
\end{EntryWithPhonetic}

\begin{EntryWithPhonetic}{胡萝卜}{hu2luo2bo5}{9,11,2}{⾁、⾋、⼘}
  \definition{s.}{cenoura}
\end{EntryWithPhonetic}

\begin{EntryWithPhonetic}{胡闹}{hu2nao4}{9,8}{⾁、⾾}[HSK 7-9]
  \definition{v.}{correr solto; ser travesso; causar problemas; agir de forma irracional | agir intencionalmente; fazer uma cena; agir de forma imprudente; fazer coisas de forma imprudente}
\end{EntryWithPhonetic}

\begin{EntryWithPhonetic}{胡琴}{hu2qin2}{9,12}{⾁、⽟}
  \definition{s.}{huqin, um termo geral para certos instrumentos de arco de duas cordas, como 二胡, 京胡, etc. | família de violinos chineses de duas cordas, com caixa de ressonância de madeira revestida de pele de cobra e arco de bambu com corda de crina de cavalo}
  \seealsoref{二胡}{er4hu2}
  \seealsoref{京胡}{jing1hu2}
\end{EntryWithPhonetic}

\begin{EntryWithPhonetic}{胡说}{hu2shuo1}{9,9}{⾁、⾔}[HSK 7-9]
  \definition{v.}{falar bobagens}
\end{EntryWithPhonetic}

\begin{EntryWithPhonetic}{胡思乱想}{hu2si1-luan4xiang3}{9,9,7,13}{⾁、⼼、⼄、⼼}[HSK 7-9]
  \definition{expr.}{``Tem uma abelha em sua capota.''; deixar-se levar pela fantasia; dar lugar a fantasias tolas; deixar a imaginação correr solta; divagar}
\end{EntryWithPhonetic}

\begin{EntryWithPhonetic}{胡同儿}{hu2 tong4r5}{9,6,2}{⾁、⼝、⼉}[HSK 5]
  \definition{s.}{beco}
\end{EntryWithPhonetic}

\begin{EntryWithPhonetic}{胡同}{hu2tong5}{9,6}{⾁、⼝}
  \definition[条,个]{s.}{beco; rua pequena}
\end{EntryWithPhonetic}

\begin{EntryWithPhonetic}{胡子}{hu2 zi5}{9,3}{⾁、⼦}[HSK 5]
  \definition[团,根,个,撮]{s.}{barba; bigode | bandido; salteador}
\end{EntryWithPhonetic}

\begin{EntryWithPhonetic}{壶}{hu2}{10}{⼠}[HSK 6]
  \definition*{s.}{Sobrenome Hu}
  \definition[个,把]{s.}{chaleira; panela | garrafa; frasco; recipiente para líquidos}
\end{EntryWithPhonetic}

\begin{EntryWithPhonetic}{核}{hu2}{10}{⽊}
  \definition{s.}{semente; o mesmo que 核}
  \seeref{he2}
\end{EntryWithPhonetic}

\begin{EntryWithPhonetic}{斛}{hu2}{11}{⽃}
  \definition*{s.}{Sobrenome Hu}
  \definition{s.}{Arcaico: uma medida seca usada antigamente, originalmente igual a 10 dou (斗), mais tarde 5 dou}
  \seealsoref{斗}{dou4}
\end{EntryWithPhonetic}

\begin{EntryWithPhonetic}{湖}{hu2}{12}{⽔}[HSK 2]
  \definition*{s.}{Huzhou, abreviação de 湖州 | Um nome que se refere às províncias de Hunan, 湖南,  e Hubei, 湖北}
  \definition[个,片]{s.}{lago}
  \seealsoref{湖北}{hu2bei3}
  \seealsoref{湖南}{hu2nan2}
  \seealsoref{湖州}{hu2zhou1}
\end{EntryWithPhonetic}

\begin{EntryWithPhonetic}{湖北}{hu2bei3}{12,5}{⽔、⼔}
  \definition*{s.}{Província de Hubei (Hupeh), na China central}
\end{EntryWithPhonetic}

\begin{EntryWithPhonetic}{湖南}{hu2nan2}{12,9}{⽔、⼗}
  \definition*{s.}{Província de Hunan}
\end{EntryWithPhonetic}

\begin{EntryWithPhonetic}{湖泊}{hu2po1}{12,8}{⽔、⽔}[HSK 7-9]
  \definition[个,片,些]{s.}{lago; nome geral para lagos}[湖泊中有丰富的鱼类。===O lago é abundante em peixes.]
\end{EntryWithPhonetic}

\begin{EntryWithPhonetic}{湖州}{hu2zhou1}{12,6}{⽔、⼮}
  \definition*{s.}{Cidade de Huzhou, em Zhejiang}
\end{EntryWithPhonetic}

\begin{EntryWithPhonetic}{葫}{hu2}{12}{⾋}
  \definition{s.}{cabaça}
\end{EntryWithPhonetic}

\begin{EntryWithPhonetic}{葫芦}{hu2lu5}{12,7}{⾋、⾋}
  \definition{adj.}{confuso}
  \definition{s.}{cabaça | termo genérico para bloco e equipamento (ou partes dele)}
\end{EntryWithPhonetic}

\begin{EntryWithPhonetic}{糊}{hu2}{15}{⽶}[HSK 7-9]
  \definition{adj.}{queimado}
  \definition{s.}{mingau; pasta; papa}
  \definition{v.}{colar com pasta; colar | (comida) ser queimado}
  \seeref{hu1}
  \seeref{hu4}
\end{EntryWithPhonetic}

\begin{EntryWithPhonetic}{糊里糊涂}{hu2 li5 hu2tu5}{15,7,15,10}{⽶、⾥、⽶、⽔}
  \definition{adj.}{desnorteado | perturbado}
\end{EntryWithPhonetic}

\begin{EntryWithPhonetic}{糊涂}{hu2tu5}{15,10}{⽶、⽔}[HSK 7-9]
  \definition{adj.}{confuso; perplexo; desnorteado; com compreensão pouco clara ou confusa das coisas | confuso; com conteúdo confuso}
\end{EntryWithPhonetic}

\begin{EntryWithPhonetic}{蝴}{hu2}{15}{⾍}
  \definition[对]{s.}{borboleta}
\end{EntryWithPhonetic}

\begin{EntryWithPhonetic}{蝴蝶}{hu2die2}{15,15}{⾍、⾍}
  \definition[只]{s.}{borboleta}
\end{EntryWithPhonetic}

\begin{EntryWithPhonetic}{虎}{hu3}{8}{⾌}[HSK 5]
  \definition*{s.}{Sobrenome Hu}
  \definition{adj.}{corajoso; bravo; valente; vigoroso}
  \definition[只]{s.}{tigre}
  \definition{v.}{blefar; o mesmo que 唬 | parecer feroz; mostrar a aparência feroz de alguém}
  \seealsoref{唬}{hu3}
  \seealsoref{老虎}{lao3hu3}
\end{EntryWithPhonetic}

\begin{EntryWithPhonetic}{虎虎}{hu3hu3}{8,8}{⾌、⾌}
  \definition{adj.}{formidável | forte | vigoroso}
\end{EntryWithPhonetic}

\begin{EntryWithPhonetic}{虎口}{hu3kou3}{8,3}{⾌、⼝}
  \definition{s.}{lugar perigoso | cova do tigre}
\end{EntryWithPhonetic}

\begin{EntryWithPhonetic}{虎鼬}{hu3you4}{8,18}{⾌、⿏}
  \definition{s.}{doninha}
\end{EntryWithPhonetic}

\begin{EntryWithPhonetic}{唬}{hu3}{11}{⼝}
  \definition{v.}{blefar, exagerar para assustar ou confundir}
\end{EntryWithPhonetic}

\begin{EntryWithPhonetic}{互}{hu4}{4}{⼆}
  \definition{adv.}{mutuamente; um ao outro}
  \definition{pron.}{um ao outro; mútuo}
\end{EntryWithPhonetic}

\begin{EntryWithPhonetic}{互补}{hu4bu3}{4,7}{⼆、⾐}[HSK 7-9]
  \definition{v.}{complementar; complementar-se}
\end{EntryWithPhonetic}

\begin{EntryWithPhonetic}{互动}{hu4 dong4}{4,6}{⼆、⼒}[HSK 6]
  \definition{v.}{interagir; participar juntos; promover uns aos outros}
\end{EntryWithPhonetic}

\begin{EntryWithPhonetic}{互访}{hu4fang3}{4,6}{⼆、⾔}[HSK 7-9]
  \definition{s.}{visitas mútuas}
  \definition{v.}{trocar visitas}
\end{EntryWithPhonetic}

\begin{EntryWithPhonetic}{互利}{hu4li4}{4,7}{⼆、⼑}
  \definition{s.}{benefício mútuo}
\end{EntryWithPhonetic}

\begin{EntryWithPhonetic}{互联网}{hu4 lian2 wang3}{4,12,6}{⼆、⽿、⽹}[HSK 3]
  \definition{s.}{\emph{Internet}; uma enorme rede conectando computadores e redes de computadores}
  \seealsoref{网际网路}{wang3 ji4 wang3 lu4}
  \seealsoref{网际网络}{wang3 ji4 wang3 luo4}
  \seealsoref{网路}{wang3 lu4}
\end{EntryWithPhonetic}

\begin{EntryWithPhonetic}{互相}{hu4xiang1}{4,9}{⼆、⽬}[HSK 3]
  \definition{adv.}{mutuamente; um ao outro; expressa uma relação de igualdade entre as partes}
\end{EntryWithPhonetic}

\begin{EntryWithPhonetic}{互信}{hu4xin4}{4,9}{⼆、⼈}[HSK 7-9]
  \definition{s.}{confiança mútua}
  \definition{v.}{confiar um no outro}
\end{EntryWithPhonetic}

\begin{EntryWithPhonetic}{互助}{hu4zhu4}{4,7}{⼆、⼒}[HSK 7-9]
  \definition{v.}{ajudar uns aos outros; ajudar-se mutualmente}[我们应该互助合作。===Devemos ajudar e cooperar uns com os outros.]
\end{EntryWithPhonetic}

\begin{EntryWithPhonetic}{户}{hu4}{4}{⼾}[HSK 4][Kangxi 63]
  \definition*{s.}{Sobrenome Hu}
  \definition[个]{s.}{porta com um painel; porta | domicílio; residência; família | status familiar | conta (banco)}
\end{EntryWithPhonetic}

\begin{EntryWithPhonetic}{户外}{hu4 wai4}{4,5}{⼾、⼣}[HSK 6]
  \definition{s.}{ao ar livre; espaço aberto ao ar livre}
\end{EntryWithPhonetic}

\begin{EntryWithPhonetic}{护}{hu4}{7}{⼿}[HSK 6]
  \definition{v.}{proteger; defender | blindar; ser parcial; proteger-se da censura}
\end{EntryWithPhonetic}

\begin{EntryWithPhonetic}{护理}{hu4li3}{7,11}{⼿、⽟}[HSK 7-9]
  \definition{v.}{cuidar; cooperar com médicos para tratar e cuidar de pacientes, idosos e deficientes | cuidar e proteger; proteger e gerenciar para permitir a vida normal ou o crescimento}
\end{EntryWithPhonetic}

\begin{EntryWithPhonetic}{护士}{hu4shi5}{7,3}{⼿、⼠}[HSK 4]
  \definition[名,位]{s.}{enfermeiro; pessoas especializadas em enfermagem em hospitais ou instituições epidemiológicas}
\end{EntryWithPhonetic}

\begin{EntryWithPhonetic}{护照}{hu4zhao4}{7,13}{⼿、⽕}[HSK 2]
  \definition[本,个]{s.}{passaporte; documento emitido pela autoridade competente do país para comprovar a nacionalidade e a identidade dos cidadãos que viajam para o exterior}
\end{EntryWithPhonetic}

\begin{EntryWithPhonetic}{糊}{hu4}{15}{⽶}
  \definition{s.}{pasta; comida que parece mingau}
  \seeref{hu1}
  \seeref{hu2}
\end{EntryWithPhonetic}

\begin{EntryWithPhonetic}{化}{hua1}{4}{⼔}
  \variantof{花}
  \seeref{hua4}
\end{EntryWithPhonetic}

\begin{EntryWithPhonetic}{花}{hua1}{7}{⾋}[HSK 1,2,4]
  \definition*{s.}{Sobrenome Hua}
  \definition{adj.}{multicolorido; colorido | embaçado; obscuro; deslumbrado e confuso | extravagante; florido; vistoso}
  \definition[朵,支,束,把,盆,簇]{s.}{flor; órgãos de reprodução sexual de plantas com sementes | flor; planta ornamental |  qualquer coisa que se assemelhe a uma flor | fogos de artifício | padrão; design; design decorativo | flor; metáfora para a essência de uma causa | prostituta; cortesã; referindo-se a prostitutas ou a assuntos relacionados a prostitutas | algodão | varíola | ferimento; ferida; lesões traumáticas sofridas em combate}
  \definition{v.}{gastar; despender; consumir}
\end{EntryWithPhonetic}

\begin{EntryWithPhonetic}{花瓣}{hua1ban4}{7,19}{⾋、⽠}[HSK 7-9]
  \definition[片,个]{s.}{pétala; um componente da corola, semelhante em estrutura a uma folha, mas com células contendo vários pigmentos, resultando em uma variedade de cores}
\end{EntryWithPhonetic}

\begin{EntryWithPhonetic}{花茶}{hua1cha2}{7,9}{⾋、⾋}
  \definition[杯,壶]{s.}{chá perfumado}
\end{EntryWithPhonetic}

\begin{EntryWithPhonetic}{花店}{hua1dian4}{7,8}{⾋、⼴}
  \definition{s.}{floricultura}
\end{EntryWithPhonetic}

\begin{EntryWithPhonetic}{花费}{hua1 fei4}{7,9}{⾋、⾙}[HSK 6]
  \definition[笔]{s.}{dinheiro gasto; despesas | custo; gastos; desembolso | despesa}
  \definition{v.}{gastar (tempo ou dinheiro)}
\end{EntryWithPhonetic}

\begin{EntryWithPhonetic}{花卉}{hua1hui4}{7,5}{⾋、⼗}[HSK 7-9]
  \definition{s.}{flores e plantas | pintura de flores e plantas no estilo tradicional chinês}
\end{EntryWithPhonetic}

\begin{EntryWithPhonetic}{花脸}{hua1lian3}{7,11}{⾋、⾁}
  \definition*{s.}{Hualian, personagem do rosto florido, um nome popular para 净 (assim chamado devido à elaborada pintura facial)}
  \seealsoref{净}{jing4}
\end{EntryWithPhonetic}

\begin{EntryWithPhonetic}{花瓶}{hua1 ping2}{7,10}{⾋、⽡}[HSK 6]
  \definition[个,对]{s.}{vaso de flores; vaso usado para arranjos florais colocado em ambientes internos como decoração | Figurativo: um ornamento; mulher empregada não por sua habilidade, mas por sua aparência; uma metáfora para uma pessoa ou coisa que é usada apenas para exibição e não tem uso prático}
\end{EntryWithPhonetic}

\begin{EntryWithPhonetic}{花儿}{hua1r5}{7,2}{⾋、⼉}
  \definition[朵,支,束,把,盆,簇]{s.}{flor}
\end{EntryWithPhonetic}

\begin{EntryWithPhonetic}{花生}{hua1sheng1}{7,5}{⾋、⽣}[HSK 6]
  \definition[把,颗,粒,袋]{s.}{amendoim}
  \seealsoref{落花生}{luo4 hua1 sheng1}
\end{EntryWithPhonetic}

\begin{EntryWithPhonetic}{花纹}{hua1wen2}{7,7}{⾋、⽷}[HSK 7-9]
  \definition[条,道,个]{s.}{figura; padrão decorativo; (padrões) várias listras e gráficos}
\end{EntryWithPhonetic}

\begin{EntryWithPhonetic}{花样}{hua1yang4}{7,10}{⾋、⽊}[HSK 7-9]
  \definition[种]{s.}{padrão, geralmente referindo-se a vários padrões ou tipos | truque | variedade; padrão decorativo; padrão de bordado, geralmente cortado ou esculpido em papel}
\end{EntryWithPhonetic}

\begin{EntryWithPhonetic}{花样游泳}{hua1yang4 you2yong3}{7,10,12,8}{⾋、⽊、⽔、⽔}
  \definition{s.}{nado sincronizado}
\end{EntryWithPhonetic}

\begin{EntryWithPhonetic}{花椰菜}{hua1ye1cai4}{7,12,11}{⾋、⽊、⾋}
  \definition{s.}{couve-flor}
\end{EntryWithPhonetic}

\begin{EntryWithPhonetic}{花园}{hua1 yuan2}{7,7}{⾋、⼞}[HSK 2]
  \definition[个,座]{s.}{jardim; um local onde se plantam flores e árvores para passear e descansar}
\end{EntryWithPhonetic}

\begin{EntryWithPhonetic}{哗}{hua1}{9}{⼝}
  \definition{s.}{(onomatopéia) sons de impacto, batida, fluxo de água, etc.}
  \seeref{hua2}
\end{EntryWithPhonetic}

\begin{EntryWithPhonetic}{哗啦啦}{hua1la1 la5}{9,11,11}{⼝、⼝、⼝}
  \definition{s.}{(onomatopéia) som de colisão, batida}
\end{EntryWithPhonetic}

\begin{EntryWithPhonetic}{划}{hua2}{6}{⼑}[HSK 4]
  \definition{adj.}{rentável; vale (o esforço); compensa (fazer alguma coisa)}
  \definition{v.}{remar | ser vantajoso para alguém; ser uma pechincha | arranhar; cortar a superfície de; cortar em outra coisa com um objeto pontiagudo | arranhar; golpear;  esfregar uma coisa ou varrer sobre outra}
  \seeref{hua4}
\end{EntryWithPhonetic}

\begin{EntryWithPhonetic}{划船}{hua2 chuan2}{6,11}{⼑、⾈}[HSK 3]
  \definition[次,回]{s.}{remo (ato de remar); passeios de barco; a atividade ou esporte de “remar um barco com remos”}
  \definition{v.}{remar um barco; a ação ou comportamento de mover um barco na água usando remos}
\end{EntryWithPhonetic}

\begin{EntryWithPhonetic}{划拳}{hua2quan2}{6,10}{⼑、⼿}
  \definition{pron.}{jogo de adivinhação de dedos; ao beber, duas pessoas levantam os dedos e dizem um número, quem disser o número que corresponde ao total de dedos ganha, o perdedor bebe}
  \definition{v.}{jogar o jogo de adivinhação de dedos (jogado em um jantar por duas pessoas)}
\end{EntryWithPhonetic}

\begin{EntryWithPhonetic}{划算}{hua2suan4}{6,14}{⼑、⽵}[HSK 7-9]
  \definition{adj.}{lucrativo; que vale a pena}
  \definition{v.}{calcular; pesar; planejar e esquematizar}
\end{EntryWithPhonetic}

\begin{EntryWithPhonetic}{划艇}{hua2ting3}{6,12}{⼑、⾈}
  \definition{s.}{barco a remo}
\end{EntryWithPhonetic}

\begin{EntryWithPhonetic}{华}{hua2}{6}{⼗}
  \definition*{s.}{China; refere-se à China (anteriormente conhecida como Huaxia, 华夏, mais tarde chamada de Zhonghua, 中华, ou simplesmente Hua, 华)}
  \definition{adj.}{esplêndido; magnífico | próspero; florescente | chamativo; extravagante; vaidoso | grisalho}
  \definition{s.}{corona; um halo colorido ao redor do sol ou da lua causado pela difração da luz através das nuvens | creme; melhor parte; a melhor parte das coisas | chinês; refere-se à nacionalidade Han (língua e escrita) | vezes; anos; refere-se a (bons) momentos | elixir; essência líquida; substâncias formadas pela sedimentação de minerais na água de nascente | Seu, palavra honorífica, usada para se referir a coisas relacionadas à outra pessoa}
  \seeref{hua4}
  \seealsoref{华夏}{hua2xia4}
  \seealsoref{中华}{zhong1 hua2}
\end{EntryWithPhonetic}

\begin{EntryWithPhonetic}{华丽}{hua2li4}{6,7}{⼗、⼀}[HSK 7-9]
  \definition{adj.}{magnífico; resplandecente; deslumbrante; lindo e radiante}
\end{EntryWithPhonetic}

\begin{EntryWithPhonetic}{华侨}{hua2qiao2}{6,8}{⼗、⼈}[HSK 7-9]
  \definition[个,位,名]{s.}{chineses que vivem no exterior}
\end{EntryWithPhonetic}

\begin{EntryWithPhonetic}{华人}{hua2 ren2}{6,2}{⼗、⼈}[HSK 3]
  \definition[名,位,个]{s.}{Chinês; chinês étnico | chineses no exterior; refere-se a cidadãos estrangeiros de ascendência chinesa que obtiveram a nacionalidade do país em que residem}
\end{EntryWithPhonetic}

\begin{EntryWithPhonetic}{华盛顿}{hua2sheng4dun4}{6,11,10}{⼗、⽫、⾴}
  \definition*{s.}{Washington}
\end{EntryWithPhonetic}

\begin{EntryWithPhonetic}{华氏}{hua2shi4}{6,4}{⼗、⽒}
  \definition{s.}{graus Fahrenheit (°F)}
\end{EntryWithPhonetic}

\begin{EntryWithPhonetic}{华夏}{hua2xia4}{6,10}{⼗、⼢}
  \definition*{s.}{Huaxia, nome antigo da China | Catai}
\end{EntryWithPhonetic}

\begin{EntryWithPhonetic}{华裔}{hua2yi4}{6,13}{⼗、⾐}[HSK 7-9]
  \definition[位,名,个]{s.}{etnia chinesa; crianças nascidas de chineses no exterior no país de residência e que adquiriram a nacionalidade do país de residência}
\end{EntryWithPhonetic}

\begin{EntryWithPhonetic}{华语}{hua2 yu3}{6,9}{⼗、⾔}[HSK 5]
  \definition*{s.}{Chinês (idioma)}
\end{EntryWithPhonetic}

\begin{EntryWithPhonetic}{哗}{hua2}{9}{⼝}
  \definition{v.}{ser barulhento; fazer alvoroço}
  \seeref{hua1}
\end{EntryWithPhonetic}

\begin{EntryWithPhonetic}{哗变}{hua2bian4}{9,8}{⼝、⼜}[HSK 7-9]
  \definition{s.}{(um exército) motim | rebelião}
\end{EntryWithPhonetic}

\begin{EntryWithPhonetic}{哗然}{hua2ran2}{9,12}{⼝、⽕}[HSK 7-9]
  \definition{adj.}{Literário: barulhento; em alvoroço; em comoção}[举座哗然。===Todo o público ficou em alvoroço.]
\end{EntryWithPhonetic}

\begin{EntryWithPhonetic}{滑}{hua2}{12}{⽔}[HSK 5]
  \definition*{s.}{Sobrenome Hua}
  \definition{adj.}{escorregadio; liso; objetos com superfícies lisas e baixo atrito | astuto; ardiloso; escorregadio}
  \definition{v.}{escorregar; deslizar | se atrapalhar; se safar de algo}
\end{EntryWithPhonetic}

\begin{EntryWithPhonetic}{滑冰}{hua2bing1}{12,6}{⽔、⼎}[HSK 7-9]
  \definition{s.}{patinação no gelo; um evento esportivo em que os atletas usam patins especiais para patinar no gelo, competindo em velocidade ou realizando manobras}
  \definition{v.}{patinar; patinar no gelo; deslizar no gelo}
\end{EntryWithPhonetic}

\begin{EntryWithPhonetic}{滑稽}{hua2ji1}{12,15}{⽔、⽲}[HSK 7-9]
  \definition{adj.}{engraçado; divertido; cômico; (palavras, ações ou gestos) que fazem as pessoas rirem}
  \definition{s.}{conversa cômica; um tipo de arte popular, popular nas áreas de Xangai, Jiangsu e Zhejiang, semelhante ao \emph{crosstalk}}
\end{EntryWithPhonetic}

\begin{EntryWithPhonetic}{滑梯}{hua2ti1}{12,11}{⽔、⽊}[HSK 7-9]
  \definition{s.}{escorregador infantil}
\end{EntryWithPhonetic}

\begin{EntryWithPhonetic}{滑雪}{hua2/xue3}{12,11}{⽔、⾬}[HSK 7-9]
  \definition{v.+compl.}{esquiar; praticar esqui; usar pranchas especiais nos pés para deslizar na neve}
\end{EntryWithPhonetic}

\begin{EntryWithPhonetic}{豁}{hua2}{17}{⾕}
  \definition{v.}{jogar o jogo de adivinhação de dedos chinês (Huoquan); o mesmo que 划拳}[豁拳规则很简单。===As regras do Huoquan são muito simples.]
  \seeref{huo1}
  \seeref{huo4}
  \seealsoref{划拳}{hua2quan2}
\end{EntryWithPhonetic}

\begin{EntryWithPhonetic}{化}{hua4}{4}{⼔}[HSK 3]
  \definition*{s.}{Sobrenome Hua}
  \definition{s.}{química | cultura; costumes e tradições}
  \definition{suf.}{modernizar; modernização; anexado a componentes nominais ou adjetivos para formar verbos, indicando a transformação em algum estado ou qualidade}
  \definition{v.}{mudar; converter; transformar; causasr mudanças | converter; influenciar; influenciar e induzir as pessoas com palavras e ações, levando-as a mudar | derreter; dissolver; fundir | digerir | queimar; reduzir a cinzas | (monge, taoísta) morrer | (de monges budistas ou sacerdotes taoístas) pedir esmolas; arrecadar bens, alimentos, etc.}
  \seeref{hua1}
\end{EntryWithPhonetic}

\begin{EntryWithPhonetic}{化肥}{hua4fei2}{4,8}{⼔、⾁}[HSK 7-9]
  \definition[袋,吨,种]{s.}{fertilizante químico}
\end{EntryWithPhonetic}

\begin{EntryWithPhonetic}{化合}{hua4he2}{4,6}{⼔、⼝}
  \definition{s.}{combinação química}
  \definition{s.}{Química: combinar; sintentizar}
\end{EntryWithPhonetic}

\begin{EntryWithPhonetic}{化解}{hua4 jie3}{4,13}{⼔、⾓}[HSK 6]
  \definition{v.}{resolver; eliminar; dissolver; neutralizar}
\end{EntryWithPhonetic}

\begin{EntryWithPhonetic}{化身}{hua4shen1}{4,7}{⼔、⾝}[HSK 7-9]
  \definition{s.}{encarnação; corporificação; personificação}
\end{EntryWithPhonetic}

\begin{EntryWithPhonetic}{化石}{hua4shi2}{4,5}{⼔、⽯}[HSK 5]
  \definition{s.}{fóssil; restos, relíquias ou vestígios de organismos antigos enterrados no solo e transformados em objetos semelhantes a pedras}
\end{EntryWithPhonetic}

\begin{EntryWithPhonetic}{化纤}{hua4xian1}{4,6}{⼔、⽷}[HSK 7-9]
  \definition[吨]{s.}{fibra sintética; abreviação de 化学纤维}[这件衣服是化纤材质。===Este vestido é feito de fibra sintética.]
  \seealsoref{化学纤维}{hua4 xue2 xian1 wei2}
\end{EntryWithPhonetic}

\begin{EntryWithPhonetic}{化险为夷}{hua4xian3wei2yi2}{4,9,4,6}{⼔、⾩、⼂、⼤}[HSK 7-9]
  \definition{expr.}{``Evite o perigo.''; transformar perigo em segurança; evitar um desastre}
\end{EntryWithPhonetic}

\begin{EntryWithPhonetic}{化学}{hua4xue2}{4,8}{⼔、⼦}
  \definition[门]{s.}{química; a ciência que estuda a composição, estrutura, propriedades e leis de mudança da matéria | celuloide}
\end{EntryWithPhonetic}

\begin{EntryWithPhonetic}{化学纤维}{hua4 xue2 xian1 wei2}{4,8,6,11}{⼔、⼦、⽷、⽷}
  \definition{s.}{fibra química | fibra sintética}
\end{EntryWithPhonetic}

\begin{EntryWithPhonetic}{化验}{hua4yan4}{4,10}{⼔、⾺}[HSK 7-9]
  \definition{v.}{fazer um teste de laboratório; conduzir um exame químico; usar métodos físicos ou químicos para examinar a composição e as propriedades das substâncias}
\end{EntryWithPhonetic}

\begin{EntryWithPhonetic}{化妆}{hua4/zhuang1}{4,6}{⼔、⼥}[HSK 7-9]
  \definition{v.+compl.}{maquiar-se; colocar maquiagem; usar algo para deixar o rosto mais bonito}
\end{EntryWithPhonetic}

\begin{EntryWithPhonetic}{划}{hua4}{6}{⼑}[HSK 4]
  \definition*{s.}{Sobrenome Hua}
  \definition{s.}{traço de um caracter chinês}
  \definition{v.}{delimitar; diferenciar; delinear | transferir; ceder | planejar; programar | desenhar; marcar; delinear; fazer linhas ou escrever como marcadores com uma caneta ou objeto semelhante a uma caneta}
  \seeref{hua2}
\end{EntryWithPhonetic}

\begin{EntryWithPhonetic}{划分}{hua4fen1}{6,4}{⼑、⼑}[HSK 5]
  \definition{v.}{dividir; particionar; reparticionar | diferenciar; encontrar aspectos diferentes}
\end{EntryWithPhonetic}

\begin{EntryWithPhonetic}{划时代}{hua4shi2dai4}{6,7,5}{⼑、⽇、⼈}[HSK 7-9]
  \definition{adj.}{marcando uma nova época; marcando época}[具有划时代的意义。===É de importância histórica.]
\end{EntryWithPhonetic}

\begin{EntryWithPhonetic}{华}{hua4}{6}{⼗}
  \definition*{s.}{Huashan Mountain (na província de Shaanxi) | Sobrenome Hua}
  \seeref{hua2}
\end{EntryWithPhonetic}

\begin{EntryWithPhonetic}{画}{hua4}{8}{⽥}[HSK 2]
  \definition*{s.}{Sobrenome Hua}
  \definition{clas.}{traços (de um caractere chinês)}
  \definition[张,幅]{s.}{desenho; pintura; imagem; figura desenhada | traço horizontal (em caracteres chineses)}
  \definition{v.}{desenhar; pintar | desenhar; marcar; assinar}
  \seealsoref{划}{hua4}
\end{EntryWithPhonetic}

\begin{EntryWithPhonetic}{画册}{hua4ce4}{8,5}{⽥、⼌}[HSK 7-9]
  \definition[部,本]{s.}{álbum de imagens; álbum de pinturas; pinturas ou imagens encadernadas}
\end{EntryWithPhonetic}

\begin{EntryWithPhonetic}{画地为牢}{hua4di4wei2lao2}{8,6,4,7}{⽥、⼟、⼂、⼧}
  \definition{expr.}{desenhar um círculo no chão para servir como uma prisão; restringir as atividades de alguém a uma área ou esfera designada; limitar; restringir | (literário) ser confinado dentro de um círculo desenhado no chão | (figurativo) limitar-se a uma gama restrita de atividades}
\end{EntryWithPhonetic}

\begin{EntryWithPhonetic}{画家}{hua4 jia1}{8,10}{⽥、⼧}[HSK 2]
  \definition[个,位,名,些]{s.}{pintor; pessoa com talento para pintura}
\end{EntryWithPhonetic}

\begin{EntryWithPhonetic}{画龙点睛}{hua4long2-dian3jing1}{8,5,9,13}{⽥、⿓、⽕、⽬}[HSK 7-9]
  \definition{expr.}{``Toque final.''; dar vida a um dragão pintado colocando as pupilas dos seus olhos -- adicionar o toque que dá vida a uma obra de arte; adicionar o toque final; adicionar uma palavra apropriada para concluir o ponto; adicionar uma ou duas palavras para finalizar o ponto; um toque crucial que reforça um ponto que de outra forma seria difícil de explicar; dar os retoques finais no trabalho de alguém; completar uma imagem}
\end{EntryWithPhonetic}

\begin{EntryWithPhonetic}{画面}{hua4 mian4}{8,9}{⽥、⾯}[HSK 5]
  \definition[个,幅,帧]{s.}{quadro; aparência geral de uma imagem; imagem apresentada no quadro, na tela, etc.}
\end{EntryWithPhonetic}

\begin{EntryWithPhonetic}{画儿}{hua4r5}{8,2}{⽥、⼉}[HSK 2]
  \definition[幅,张]{s.}{imagem; desenho; pintura; obra de arte pintada}
\end{EntryWithPhonetic}

\begin{EntryWithPhonetic}{画蛇添足}{hua4she2-tian1zu2}{8,11,11,7}{⽥、⾍、⽔、⾜}[HSK 7-9]
  \definition{expr.}{arruinar o efeito adicionando algo supérfluo; dourar; fazer coisas desnecessárias pode sair pela culatra e piorar as coisas}
\end{EntryWithPhonetic}

\begin{EntryWithPhonetic}{画展}{hua4zhan3}{8,10}{⽥、⼫}[HSK 7-9]
  \definition{s.}{exposição de pinturas; exposição de arte}
\end{EntryWithPhonetic}

\begin{EntryWithPhonetic}{话}{hua4}{8}{⾔}[HSK 1]
  \definition[句,段,番,种]{s.}{palavra; conversa; a voz que expressa os pensamentos quando falada, ou os caracteres que registram essa voz}
  \definition{v.}{falar sobre; falar a respeito}
\end{EntryWithPhonetic}

\begin{EntryWithPhonetic}{话费}{hua4fei4}{8,9}{⾔、⾙}[HSK 7-9]
  \definition{s.}{uma conta ou taxa telefônica; tarifas de uso do telefone; às vezes se refere simplesmente ao custo de uso do telefone}
\end{EntryWithPhonetic}

\begin{EntryWithPhonetic}{话剧}{hua4 ju4}{8,10}{⾔、⼑}[HSK 3]
  \definition[场,幕,部,出,台]{s.}{drama moderno; peça de teatro; peça teatral representada através de diálogos e ações}
\end{EntryWithPhonetic}

\begin{EntryWithPhonetic}{话题}{hua4ti2}{8,15}{⾔、⾴}[HSK 3]
  \definition[个,种,项]{s.}{assunto de uma palestra; tópico de uma conversa; o foco da conversa}
\end{EntryWithPhonetic}

\begin{EntryWithPhonetic}{话筒}{hua4tong3}{8,12}{⾔、⽵}[HSK 7-9]
  \definition[个,只]{s.}{microfone; um termo geral para microfones | megafone; um tubo em forma de cone usado para falar alto com muitas pessoas próximas também é chamado de megafone | receptor (telefone); bocal}
\end{EntryWithPhonetic}

\begin{EntryWithPhonetic}{话语}{hua4yu3}{8,9}{⾔、⾔}[HSK 7-9]
  \definition[句]{s.}{palavras; fala; enunciado; discurso; palavras ditas}
\end{EntryWithPhonetic}

\begin{EntryWithPhonetic}{怀}{huai2}{7}{⼼}
  \definition*{s.}{Sobrenome Huai}
  \definition{s.}{seio; peito | mente}
  \definition{v.}{manter em mente; estimar; abrigar | sentir falta; pensar em; ansiar por | conceber (uma criança)}
\end{EntryWithPhonetic}

\begin{EntryWithPhonetic}{怀抱}{huai2bao4}{7,8}{⼼、⼿}[HSK 7-9]
  \definition{s.}{seio; peito | ambição; aspiração; intenção}
  \definition{v.}{abraçar; carregar nos braços; segurar nos braços | estimar; ter em mente}
\end{EntryWithPhonetic}

\begin{EntryWithPhonetic}{怀旧}{huai2jiu4}{7,5}{⼼、⽇}[HSK 7-9]
  \definition{s.}{lembrança afetuosa de tempos passados | nostalgia}
  \definition{v.}{sentir ou demonstrar nostalgia; lembrar de tempos passados ​​ou de velhos conhecidos (geralmente com pensamentos gentis)}
\end{EntryWithPhonetic}

\begin{EntryWithPhonetic}{怀里}{huai2 li5}{7,7}{⼼、⾥}[HSK 7-9]
  \definition{s.}{seio; abraço}
\end{EntryWithPhonetic}

\begin{EntryWithPhonetic}{怀念}{huai2nian4}{7,8}{⼼、⼼}[HSK 4]
  \definition{v.}{pensar em; valorizar a memória de}
\end{EntryWithPhonetic}

\begin{EntryWithPhonetic}{怀疑}{huai2yi2}{7,14}{⼼、⽦}[HSK 4]
  \definition{v.}{duvidar; suspeitar | supor}
\end{EntryWithPhonetic}

\begin{EntryWithPhonetic}{怀孕}{huai2/yun4}{7,5}{⼼、⼦}[HSK 7-9]
  \definition{v.+compl.}{estar (ficar) grávida}
\end{EntryWithPhonetic}

\begin{EntryWithPhonetic}{怀着}{huai2zhe5}{7,11}{⼼、⽬}[HSK 7-9]
  \definition{v.}{nutrir; abrigar; ser preenchido com}
\end{EntryWithPhonetic}

\begin{EntryWithPhonetic}{槐}{huai2}{13}{⽊}
  \definition*{s.}{Sobrenome Huai}
  \definition{s.}{sophora japonica; alfarrobeira; acácia}
\end{EntryWithPhonetic}

\begin{EntryWithPhonetic}{槐树}{huai2shu4}{13,9}{⽊、⽊}[HSK 7-9]
  \definition[棵,株]{s.}{acácia; árvore de alfarroba; árvore de pagode}
\end{EntryWithPhonetic}

\begin{EntryWithPhonetic}{坏}{huai4}{7}{⼟}[HSK 1]
  \definition{adj.}{ruim; prejudicial; insatisfatório; péssimo | mal; extremamente; indica um grau profundo, geralmente usado após verbos ou adjetivos que expressam estado psicológico | podre; estragado; impróprio; prejudicial ao uso}
  \definition[种]{s.}{ideia maligna; truque sujo; péssima ideia}
  \definition{v.}{estragar; destruir; corromper}
\end{EntryWithPhonetic}

\begin{EntryWithPhonetic}{坏处}{huai4 chu4}{7,5}{⼟、⼡}[HSK 2]
  \definition[个]{s.}{dano; prejuízo; desvantagem; fatores prejudiciais a pessoas ou coisas}
\end{EntryWithPhonetic}

\begin{EntryWithPhonetic}{坏蛋}{huai4dan4}{7,11}{⼟、⾍}
  \definition{s.}{bastardo | canalha | pessoa má}
\end{EntryWithPhonetic}

\begin{EntryWithPhonetic}{坏人}{huai4 ren2}{7,2}{⼟、⼈}[HSK 2]
  \definition[个,种]{s.}{malfeitor; canalha; pessoa má; pessoa de má qualidade; pessoa que faz coisas ruins}
\end{EntryWithPhonetic}

\begin{EntryWithPhonetic}{坏事}{huai4shi4}{7,8}{⼟、⼅}[HSK 7-9]
  \definition{s.}{coisa ruim; ação má; coisas que são prejudiciais à sociedade}
  \definition{v.}{arruinar algo; piorar as coisas}
\end{EntryWithPhonetic}

\begin{EntryWithPhonetic}{欢}{huan1}{6}{⽋}
  \definition*{s.}{Sobrenome Huan}
  \definition{adj.}{alegre; feliz; jubilante | vigoroso; energético; em pleno andamento; com grande impulso}
  \definition{s.}{amante; querida; um apelido usado por mulheres nos tempos antigos para se referir aos seus amantes; agora, geralmente se refere a alguém de quem você gosta ou com quem tem um relacionamento romântico}
\end{EntryWithPhonetic}

\begin{EntryWithPhonetic}{欢呼}{huan1hu1}{6,8}{⽋、⼝}[HSK 7-9]
  \definition{v.}{saudar; aplaudir; aclamar; dar vivas}
\end{EntryWithPhonetic}

\begin{EntryWithPhonetic}{欢聚}{huan1ju4}{6,14}{⽋、⽿}[HSK 7-9]
  \definition{s.}{celebração | festa}
  \definition{v.}{desfrutar de uma reunião feliz; reunir-se alegremente | celebrar | reunir-se socialmente}
\end{EntryWithPhonetic}

\begin{EntryWithPhonetic}{欢快}{huan1kuai4}{6,7}{⽋、⼼}[HSK 7-9]
  \definition{adj.}{alegre; animado; alegre e despreocupado; feliz e alegre}
\end{EntryWithPhonetic}

\begin{EntryWithPhonetic}{欢乐}{huan1le4}{6,5}{⽋、⼃}[HSK 3]
  \definition{adj.}{feliz; alegre; felicidade (geralmente coletiva)}
\end{EntryWithPhonetic}

\begin{EntryWithPhonetic}{欢声笑语}{huan1sheng1-xiao4yu3}{6,7,10,9}{⽋、⼠、⽵、⾔}[HSK 7-9]
  \definition{expr.}{risos felizes e vozes alegres}
\end{EntryWithPhonetic}

\begin{EntryWithPhonetic}{欢迎}{huan1ying2}{6,7}{⽋、⾡}[HSK 2]
  \definition{adj.}{bem-vindo}
  \definition{v.}{dar as boas-vindas; cumprimentar; receber com alegria | dar as boas-vindas; receber favoravelmente (bem)}
\end{EntryWithPhonetic}

\begin{EntryWithPhonetic}{还}{huan2}{7}{⾡}[HSK 1]
  \definition*{s.}{Sobrenome Huan}
  \definition{v.}{voltar; retornar; voltar ao lugar original ou restaurar o estado original | retribuir; devolver; reembolsar; devolver o dinheiro ou os bens emprestados ao seu proprietário | dar ou fazer algo em troca; retribuir as ações dos outros}
  \seeref{hai2}
\end{EntryWithPhonetic}

\begin{EntryWithPhonetic}{还款}{huan2 kuan3}{7,12}{⾡、⽋}[HSK 7-9]
  \definition{v.}{reembolsar | devolver dinheiro}
\end{EntryWithPhonetic}

\begin{EntryWithPhonetic}{还原}{huan2/yuan2}{7,10}{⾡、⼚}[HSK 7-9]
  \definition{v.+compl.}{restaurar ao estado ou forma original | desoxidar; reduzir; refere-se à privação de oxigênio de substâncias que o contêm; também se refere, de modo geral, ao processo pelo qual uma substância ganha elétrons ou pares de elétrons em uma reação química}
\end{EntryWithPhonetic}

\begin{EntryWithPhonetic}{环}{huan2}{8}{⽟}[HSK 3]
  \definition*{s.}{Sobrenome Huan}
  \definition{clas.}{usado para anéis}
  \definition[个,串]{s.}{anel; arco | elo; \emph{link}; passo; etapa | anel; objeto em forma de círculo | arredores}
  \definition{v.}{cercar; rodear; circular; circundar}
\end{EntryWithPhonetic}

\begin{EntryWithPhonetic}{环保}{huan2 bao3}{8,9}{⽟、⼈}[HSK 3]
  \definition{adj.}{ecológico; benefício para o meio ambiente; não prejudica o meio ambiente}
  \definition{s.}{proteção ambiental}
\end{EntryWithPhonetic}

\begin{EntryWithPhonetic}{环节}{huan2jie2}{8,5}{⽟、⾋}[HSK 5]
  \definition[个]{s.}{\emph{link}; ligação; vínculo; uma das muitas coisas que estão inter-relacionadas | segmento; estrutura anelar de alguns animais inferiores}
\end{EntryWithPhonetic}

\begin{EntryWithPhonetic}{环境}{huan2jing4}{8,14}{⽟、⼟}[HSK 3]
  \definition[个]{s.}{ambiente; os arredores | arredores; circunstâncias; condições políticas, econômicas, culturais, etc., dentro de um determinado âmbito}
\end{EntryWithPhonetic}

\begin{EntryWithPhonetic}{环境卫生}{huan2jing4wei4sheng1}{8,14,3,5}{⽟、⼟、⼙、⽣}
  \definition{s.}{saneamento ambiental}
  \seealsoref{环卫}{huan2wei4}
\end{EntryWithPhonetic}

\begin{EntryWithPhonetic}{环球}{huan2qiu2}{8,11}{⽟、⽟}[HSK 7-9]
  \definition*{s.}{Terra}
  \definition{adj.}{global; mundial}
  \definition{adv.}{ao redor do mundo; circulando a Terra}
  \definition{s.}{mundo inteiro}
\end{EntryWithPhonetic}

\begin{EntryWithPhonetic}{环绕}{huan2rao4}{8,9}{⽟、⽷}[HSK 7-9]
  \definition{v.}{cercar; rodear; envolver}
\end{EntryWithPhonetic}

\begin{EntryWithPhonetic}{环卫}{huan2wei4}{8,3}{⽟、⼙}
  \definition{s.}{limpeza pública; saneamento ambiental; saneamento geral; abreviação de 环境卫生 | Arcaico: guardas imperiais; guardas}
  \seealsoref{环境卫生}{huan2jing4wei4sheng1}
\end{EntryWithPhonetic}

\begin{EntryWithPhonetic}{缓}{huan3}{12}{⽶}[HSK 7-9]
  \definition{adj.}{lento; sem pressa | sem tensão; relaxado}
  \definition{v.}{atrasar; adiar; protelar | recuperar; reviver; voltar a si}
\end{EntryWithPhonetic}

\begin{EntryWithPhonetic}{缓和}{huan3he2}{12,8}{⽶、⼝}[HSK 7-9]
  \definition{adj.}{relaxado; moderado; suave; pacífico e relaxante; não tenso ou intenso}
  \definition{v.}{relaxar; aliviar; atenuar; facilitar}
\end{EntryWithPhonetic}

\begin{EntryWithPhonetic}{缓缓}{huan3huan3}{12,12}{⽶、⽶}[HSK 7-9]
  \definition{adv.}{lentamente; vagarosamente; gradualmente}
\end{EntryWithPhonetic}

\begin{EntryWithPhonetic}{缓解}{huan3jie3}{12,13}{⽶、⾓}[HSK 4]
  \definition{v.}{facilitar; aliviar; atenuar; amenizar; reduzir}
\end{EntryWithPhonetic}

\begin{EntryWithPhonetic}{缓慢}{huan3man4}{12,14}{⽶、⼼}[HSK 7-9]
  \definition{adj.}{lento; vagaroso}
  \definition{adv.}{lentamente; vagarosamente}
\end{EntryWithPhonetic}

\begin{EntryWithPhonetic}{幻}{huan4}{4}{⼳}
  \definition{adj.}{irreal; imaginário; ilusório | mágico; mutável}
  \definition{v.}{mudar magicamente}
\end{EntryWithPhonetic}

\begin{EntryWithPhonetic}{幻觉}{huan4jue2}{4,9}{⼳、⾒}[HSK 7-9]
  \definition{s.}{uma ilusão; uma alucinação; sensações falsas na visão, audição, tato, etc., que ocorrem sem estímulo externo}
\end{EntryWithPhonetic}

\begin{EntryWithPhonetic}{幻想}{huan4xiang3}{4,13}{⼳、⼼}[HSK 6]
  \definition[个,种]{s.}{fantasia; visão; arco-íris; ilusão; fruto da imaginação de alguém; imaginar algo que é difícil ou impossível de alcançar}
  \definition{v.}{imaginar; fantasiar; imaginar coisas que ainda não foram realizadas com base em ideais e desejos sociais ou pessoais}
\end{EntryWithPhonetic}

\begin{EntryWithPhonetic}{幻影}{huan4ying3}{4,15}{⼳、⼺}[HSK 7-9]
  \definition{s.}{fantasma; imagem irreal | miragem}
\end{EntryWithPhonetic}

\begin{EntryWithPhonetic}{唤}{huan4}{10}{⼝}
  \definition{v.}{chamar; fazer um barulho alto para fazer a outra parte acordar, prestar atenção ou vir até você}
\end{EntryWithPhonetic}

\begin{EntryWithPhonetic}{唤起}{huan4qi3}{10,10}{⼝、⾛}[HSK 7-9]
  \definition{v.}{despertar | chamar; evocar}
\end{EntryWithPhonetic}

\begin{EntryWithPhonetic}{换}{huan4}{10}{⼿}[HSK 2]
  \definition{v.}{negociar; trocar; permutar; dar algo a alguém e, ao mesmo tempo, obter algo dele em troca | mudar; transformar; substituir | trocar dinheiro (câmbio) | transfundir (sangue) | transplantar (um órgão)}
\end{EntryWithPhonetic}

\begin{EntryWithPhonetic}{换成}{huan4cheng2}{10,6}{⼿、⼽}[HSK 7-9]
  \definition{v.}{trocar (algo) por (outro); indica a substituição de um objeto, estado ou situação por outro}
\end{EntryWithPhonetic}

\begin{EntryWithPhonetic}{换钱}{huan4/qian2}{10,10}{⼿、⾦}
  \definition{v.+compl.}{trocar dinheiro (em pequenas valores ou em outra moeda) | trocar (mercadorias) por dinheiro | vender}
\end{EntryWithPhonetic}

\begin{EntryWithPhonetic}{换取}{huan4qu3}{10,8}{⼿、⼜}[HSK 7-9]
  \definition{v.}{trocar (ou escambo) algo por; obter em troca | trocar algo por; obter por troca}
\end{EntryWithPhonetic}

\begin{EntryWithPhonetic}{换位}{huan4wei4}{10,7}{⼿、⼈}[HSK 7-9]
  \definition{v.}{trocar posições; transpor | mudar de posição}
\end{EntryWithPhonetic}

\begin{EntryWithPhonetic}{换言之}{huan4yan2zhi1}{10,7,3}{⼿、⾔、⼂}[HSK 7-9]
  \definition{adv.}{em outras palavras}
\end{EntryWithPhonetic}

\begin{EntryWithPhonetic}{患}{huan4}{11}{⼼}[HSK 7-9]
  \definition*{s.}{Sobrenome Huan}
  \definition{s.}{perigo; problema; desastre; flagelo | preocupação; ansiedade}
  \definition{v.}{contrair (doença); sofrer de}
\end{EntryWithPhonetic}

\begin{EntryWithPhonetic}{患病}{huan4bing4}{11,10}{⼼、⽧}[HSK 7-9]
  \definition{v.}{estar doente; ficar doente; adoecer; sofrer de uma doença}
\end{EntryWithPhonetic}

\begin{EntryWithPhonetic}{患有}{huan4you3}{11,6}{⼼、⽉}[HSK 7-9]
  \definition{v.}{sofrer de; refere-se a alguém que sofre de uma doença ou condição específica}
\end{EntryWithPhonetic}

\begin{EntryWithPhonetic}{患者}{huan4zhe3}{11,8}{⼼、⽼}[HSK 6]
  \definition[个,位,名]{s.}{paciente; sofredor; pessoas com certas doenças}
\end{EntryWithPhonetic}

\begin{EntryWithPhonetic}{焕}{huan4}{11}{⽕}
  \definition{adj.}{Literário: brilhante; reluzente; radiante}
\end{EntryWithPhonetic}

\begin{EntryWithPhonetic}{焕发}{huan4fa1}{11,5}{⽕、⼜}[HSK 7-9]
  \definition{v.}{brilhar; revigorar; irradiar}
\end{EntryWithPhonetic}

\begin{EntryWithPhonetic}{荒}{huang1}{9}{⾋}[HSK 7-9]
  \definition*{s.}{Sobrenome Huang}
  \definition{adj.}{(terra) não utilizada; não cultivada | desolado; estéril | irracional; delirante; fantástico; absurdo | incerto; duvidoso | dissoluto; autoindulgente | grosseiramente processado; bruto}
  \definition[片,块]{s.}{terra devastada; terra inculta; deserto | fome; quebra de safra | escassez | lixo; restos | terra selvagem (floresta)}
  \definition{v.}{(coloquial) negligenciar; estar fora de prática}
\end{EntryWithPhonetic}

\begin{EntryWithPhonetic}{荒诞}{huang1dan4}{9,8}{⾋、⾔}[HSK 7-9]
  \definition{adj.}{fantástico; absurdo; incrível; inacreditável}
\end{EntryWithPhonetic}

\begin{EntryWithPhonetic}{荒凉}{huang1liang2}{9,10}{⾋、⼎}[HSK 7-9]
  \definition{adj.}{selvagem; sombrio e desolado; escassamente povoado; deserto}
\end{EntryWithPhonetic}

\begin{EntryWithPhonetic}{荒谬}{huang1miu4}{9,13}{⾋、⾔}[HSK 7-9]
  \definition{adj.}{absurdo; ridículo; extremamente errado; extremamente irracional}
\end{EntryWithPhonetic}

\begin{EntryWithPhonetic}{荒唐}{huang1tang2}{9,10}{⾋、⼝}
  \definition{adj.}{absurdo; fantástico; grosseiramente exagerado; descreve pensamentos, palavras ou comportamentos anormais, fazendo as pessoas se sentirem estranhas ou ridículas | dissipado; dissoluto; descreve pessoas que não controlam seus desejos, não são limitadas por restrições e fazem as coisas casualmente}
\end{EntryWithPhonetic}

\begin{EntryWithPhonetic}{荒芜}{huang1wu2}{9,7}{⾋、⾋}
  \definition{adj.}{estéril}
\end{EntryWithPhonetic}

\begin{EntryWithPhonetic}{慌}{huang1}{12}{⼼}[HSK 5]
  \definition{adj.}{agitado; perturbado; confuso; que inspira terror}
  \definition{v.}{estar em estado de pânico; ficar com medo; ficar nervoso | estar com pressa}
\end{EntryWithPhonetic}

\begin{EntryWithPhonetic}{慌乱}{huang1luan4}{12,7}{⼼、⼄}[HSK 7-9]
  \definition{adj.}{agitado; alarmado e confuso; em pânico e ocupado}
\end{EntryWithPhonetic}

\begin{EntryWithPhonetic}{慌忙}{huang1 mang2}{12,6}{⼼、⼼}[HSK 5]
  \definition{adj.}{apressado; afobado; com muita pressa}
  \definition{adv.}{apressadamente}
\end{EntryWithPhonetic}

\begin{EntryWithPhonetic}{慌张}{huang1zhang1}{12,7}{⼼、⼸}[HSK 7-9]
  \definition{adj.}{em pânico; agitado; perturbado; confuso}
\end{EntryWithPhonetic}

\begin{EntryWithPhonetic}{皇}{huang2}{9}{⽩}
  \definition*{s.}{Sobrenome Huang}
  \definition{adj.}{grandioso; magnífico}
  \definition{s.}{imperador, o governante supremo de uma dinastia feudal após a Dinastia Qin; soberano}
\end{EntryWithPhonetic}

\begin{EntryWithPhonetic}{皇帝}{huang2di4}{9,9}{⽩、⼱}[HSK 6]
  \definition[个,位,任]{s.}{imperador; o título do mais alto governante feudal na China começou com o título de Imperador Qin Shi Huang}
\end{EntryWithPhonetic}

\begin{EntryWithPhonetic}{皇宫}{huang2gong1}{9,9}{⽩、⼧}[HSK 7-9]
  \definition{s.}{palácio (imperial) | palácio imperial; onde o imperador morava}
\end{EntryWithPhonetic}

\begin{EntryWithPhonetic}{皇后}{huang2hou4}{9,6}{⽩、⼝}[HSK 7-9]
  \definition[个,位,任]{s.}{rainha; imperatriz; a esposa do imperador}
\end{EntryWithPhonetic}

\begin{EntryWithPhonetic}{皇上}{huang2shang5}{9,3}{⽩、⼀}[HSK 7-9]
  \definition*{s.}{Sua Majestade; Vossa Majestade | Sua Majestade Imperial | Sua Majestade o Imperador}
  \definition{s.}{imperador; trono; soberano reinante}
\end{EntryWithPhonetic}

\begin{EntryWithPhonetic}{皇室}{huang2shi4}{9,9}{⽩、⼧}[HSK 7-9]
  \definition{s.}{família imperial (ou casa) | governo imperial; corte real | casa imperial | membro da família real}
\end{EntryWithPhonetic}

\begin{EntryWithPhonetic}{凰}{huang2}{11}{⼏}
  \definition[只]{s.}{Mitologia: fênix fêmea}
\end{EntryWithPhonetic}

\begin{EntryWithPhonetic}{黄}{huang2}{11}{⿈}[HSK 2][Kangxi 201]
  \definition*{s.}{Rio Huanghe, abreviação de 黄河 | Refere-se ao Imperador Amarelo, um imperador da mitologia chinesa antiga | Sobrenome Huang ou Hwang}
  \definition{adj.}{amarelo | obsceno; indecente; pornográfico; símbolo de corrupção e decadência, referindo-se especificamente à pornografia}
  \definition{s.}{gema; ovas de caranguejo; refere-se a certas coisas de cor amarela}
  \definition{v.}{fracassar; dar errado}
  \seealsoref{黄河}{huang2he2}
\end{EntryWithPhonetic}

\begin{EntryWithPhonetic}{黄瓜}{huang2 gua1}{11,5}{⿈、⽠}[HSK 4]
  \definition[根,棵,株,条]{s.}{pepino}
\end{EntryWithPhonetic}

\begin{EntryWithPhonetic}{黄河}{huang2he2}{11,8}{⿈、⽔}
  \definition*{s.}{Rio Amarelo | Rio Huang He}
\end{EntryWithPhonetic}

\begin{EntryWithPhonetic}{黄昏}{huang2hun1}{11,8}{⿈、⽇}[HSK 7-9]
  \definition[个]{s.}{crepúsculo; refere-se ao período do pôr do sol ao anoitecer}
\end{EntryWithPhonetic}

\begin{EntryWithPhonetic}{黄金}{huang2jin1}{11,8}{⿈、⾦}[HSK 4]
  \definition{adj.}{de primeira qualidade; dourado;}
  \definition[块,克,两]{s.}{ouro; \emph{aurum}; um tipo de metal, de cor amarela, mais precioso, abreviação de 金, frequentemente falado como 金子}
  \seealsoref{金}{jin1}
  \seealsoref{金子}{jin1zi5}
\end{EntryWithPhonetic}

\begin{EntryWithPhonetic}{黄色}{huang2 se4}{11,6}{⿈、⾊}[HSK 2]
  \definition{adj.}{decadente; obsceno; erótico; pornográfico; símbolo de corrupção e decadência, referindo-se especificamente à pornografia}
  \definition[种]{s.}{cor amarela}
\end{EntryWithPhonetic}

\begin{EntryWithPhonetic}{黄油}{huang2you2}{11,8}{⿈、⽔}
  \definition[盒]{s.}{manteiga}
\end{EntryWithPhonetic}

\begin{EntryWithPhonetic}{惶}{huang2}{12}{⼼}
  \definition{adj.}{cheio de medo; assustado}
  \definition{s.}{medo; pânico}
  \definition{v.}{temer}
\end{EntryWithPhonetic}

\begin{EntryWithPhonetic}{惶恐}{huang2kong3}{12,10}{⼼、⼼}
  \definition{adj.}{aterrorizado; em pânico; petrificado | inquieto; apreensivo}
\end{EntryWithPhonetic}

\begin{EntryWithPhonetic}{恍}{huang3}{9}{⼼}
  \definition{adv.}{(junto com 如, 若, etc.) parecer; como se | de repente}
  \seealsoref{如}{ru2}
  \seealsoref{若}{ruo4}
\end{EntryWithPhonetic}

\begin{EntryWithPhonetic}{恍然大悟}{huang3ran2-da4wu4}{9,12,3,10}{⼼、⽕、⼤、⼼}[HSK 7-9]
  \definition{expr.}{de repente ver a luz; de repente perceber o que aconteceu; perceber de repente}
\end{EntryWithPhonetic}

\begin{EntryWithPhonetic}{晃}{huang3}{10}{⽇}[HSK 7-9]
  \definition*{s.}{Sobrenome Huang}
  \definition{adj.}{deslumbrante}
  \definition{v.}{passar rapidamente | deslumbrar; cegar}
  \seeref{huang4}
\end{EntryWithPhonetic}

\begin{EntryWithPhonetic}{谎}{huang3}{11}{⾔}
  \definition[句]{s.}{mentira; falsidade}
  \definition{v.}{contar uma mentira; mentir}
\end{EntryWithPhonetic}

\begin{EntryWithPhonetic}{谎话}{huang3hua4}{11,8}{⾔、⾔}[HSK 7-9]
  \definition[个]{s.}{mentira; falsidade; palavras falsas e enganosas}
\end{EntryWithPhonetic}

\begin{EntryWithPhonetic}{谎言}{huang3yan2}{11,7}{⾔、⾔}[HSK 7-9]
  \definition[个,派]{s.}{mentira; falsidade; é uma declaração falsa e inverídica, frequentemente usada para enganar os outros}
\end{EntryWithPhonetic}

\begin{EntryWithPhonetic}{晃}{huang4}{10}{⽇}[HSK 7-9]
  \definition{v.}{sacudir; balançar}
  \seeref{huang3}
\end{EntryWithPhonetic}

\begin{EntryWithPhonetic}{晃荡}{huang4dang5}{10,9}{⽇、⾋}[HSK 7-9]
  \definition{v.}{balançar; sacudir | Coloquial: vagar; ficar ocioso; divagar | oscilar}
\end{EntryWithPhonetic}

\begin{EntryWithPhonetic}{灰}{hui1}{6}{⽕}[HSK 7-9]
  \definition{adj.}{cinza (cor) | desanimado; desencorajado; deprimido}
  \definition[把,堆]{s.}{cinzas; pó que sobra após a queima de um objeto | pó; poeira; substância em pó | cal; argamassa (de cal)}
\end{EntryWithPhonetic}

\begin{EntryWithPhonetic}{灰尘}{hui1chen2}{6,6}{⽕、⼩}[HSK 7-9]
  \definition[层,堆]{s.}{cinza; poeira; sujeira; pó}
\end{EntryWithPhonetic}

\begin{EntryWithPhonetic}{灰色}{hui1 se4}{6,6}{⽕、⾊}[HSK 5]
  \definition{adj.}{obscuro; ambíguo | sombrio; pessimista}
  \definition[种]{s.}{cor cinza; acinzentado}
\end{EntryWithPhonetic}

\begin{EntryWithPhonetic}{灰心}{hui1/xin1}{6,4}{⽕、⼼}[HSK 7-9]
  \definition{v.+compl.}{desanimar; ficar desanimado; ficar desapontado; (devido a dificuldades, fracassos) ficar deprimido}
\end{EntryWithPhonetic}

\begin{EntryWithPhonetic}{恢}{hui1}{9}{⼼}
  \definition{adj.}{extenso; vasto | grande; ótimo}
  \definition{v.}{recuperar; restaurar; restabelecer}
\end{EntryWithPhonetic}

\begin{EntryWithPhonetic}{恢复}{hui1fu4}{9,9}{⼼、⼢}[HSK 5]
  \definition{v.}{retomar; renovar; restaurar; voltar a | reviver; recuperar; reencontrar | restaurar; restabelecer; reabilitar; regenerar; ressurgir; restabelecer alguém em; recuperar o que foi perdido}
\end{EntryWithPhonetic}

\begin{EntryWithPhonetic}{挥}{hui1}{9}{⼿}[HSK 7-9]
  \definition{v.}{acenar; empunhar; socar | limpar lágrimas, suor, etc. com as mãos | comandar (um exército) | espalhar; dispersar | afastar-se; livrar-se de}
\end{EntryWithPhonetic}

\begin{EntryWithPhonetic}{挥汗如雨}{hui1han4ru2yu3}{9,6,6,8}{⼿、⽔、⼥、⾬}
  \definition{s.}{suor derramado}
  \definition{v.}{pingar com suor}
\end{EntryWithPhonetic}

\begin{EntryWithPhonetic}{辉}{hui1}{12}{⾞}
  \definition{s.}{brilho; esplendor; fulgor}
  \definition{v.}{brilhar}
\end{EntryWithPhonetic}

\begin{EntryWithPhonetic}{辉煌}{hui1huang2}{12,13}{⾞、⽕}[HSK 7-9]
  \definition{adj.}{brilhante; esplêndido; deslumbrante  | brilhante; glorioso; descreve conquistas notáveis}
\end{EntryWithPhonetic}

\begin{EntryWithPhonetic}{囘}{hui2}{5}{⼞}
  \variantof{回}
\end{EntryWithPhonetic}

\begin{EntryWithPhonetic}{回}{hui2}{6}{⼞}[HSK 1,2]
  \definition*{s.}{Sobrenome Hui}
  \definition*{s.}{Etnia Hui (mulçumanos chineses)}
  \definition{clas.}{usado para coisas, ações, número de vezes |  um trecho de um conto; um capítulo de um romance em capítulos | seção ou capítulo (de um livro clássico)}
  \definition{v.}{circular; enrolar | retornar; voltar; voltar ao lugar de origem | dar meia-volta | responder; contestar | relatar; reportar; responder}
\end{EntryWithPhonetic}

\begin{EntryWithPhonetic}{回报}{hui2bao4}{6,7}{⼞、⼿}[HSK 5]
  \definition{s.}{recompensa; pagamento; benefícios recebidos como resultado de assistência, esforço ou afeto | retornos; benefícios recebidos por meio de investimentos}
  \definition{v.}{pagar de volta; beneficiar pessoas ou organizações que os ajudaram ou cuidaram deles de alguma forma}
\end{EntryWithPhonetic}

\begin{EntryWithPhonetic}{回避}{hui2bi4}{6,16}{⼞、⾌}
  \definition{v.}{fugir (de um problema); em direito, refere-se especificamente à não participação nos procedimentos de um caso de um oficial de justiça, etc., que tenha interesse no caso ou nas partes do caso | esquivar-se; evadir-se; evitar (encontrar alguém)}
\end{EntryWithPhonetic}

\begin{EntryWithPhonetic}{回答}{hui2da2}{6,12}{⼞、⽵}[HSK 1]
  \definition[个]{s.}{resposta}
  \definition{v.}{responder; explicar a questão; expressar opinião sobre a solicitação}
\end{EntryWithPhonetic}

\begin{EntryWithPhonetic}{回到}{hui2 dao4}{6,8}{⼞、⼑}[HSK 1]
  \definition{v.}{retornar para; voltar e chegar (ao lugar onde estava originalmente); (após uma mudança nas circunstâncias) retornar ao estado original}
\end{EntryWithPhonetic}

\begin{EntryWithPhonetic}{回复}{hui2 fu4}{6,9}{⼞、⼢}[HSK 4]
  \definition{v.}{responder (a uma carta) | retornar ao estado normal; restaurar algo ao seu estado original}
\end{EntryWithPhonetic}

\begin{EntryWithPhonetic}{回顾}{hui2gu4}{6,10}{⼞、⾴}[HSK 5]
  \definition{v.}{olhar para trás | revisar; fazer uma retrospectiva; olhar para trás, pensar no passado}
\end{EntryWithPhonetic}

\begin{EntryWithPhonetic}{回归}{hui2gui1}{6,5}{⼞、⼹}[HSK 7-9]
  \definition{v.}{retornar; regredir; retornar para (local original, organização, etc.)}
\end{EntryWithPhonetic}

\begin{EntryWithPhonetic}{回国}{hui2 guo2}{6,8}{⼞、⼞}[HSK 2]
  \definition{v.}{regressar ao seu país (terra natal); referindo-se a voltar do exterior}
\end{EntryWithPhonetic}

\begin{EntryWithPhonetic}{回家}{hui2 jia1}{6,10}{⼞、⼧}[HSK 1]
  \definition{v.}{ir (voltar) para casa; estar em casa; voltar para casa}
\end{EntryWithPhonetic}

\begin{EntryWithPhonetic}{回扣}{hui2kou4}{6,6}{⼞、⼿}[HSK 7-9]
  \definition[笔,的]{s.}{propina; desconto; comissão sobre vendas (para agente)}
\end{EntryWithPhonetic}

\begin{EntryWithPhonetic}{回馈}{hui2kui4}{6,12}{⼞、⾶}[HSK 7-9]
  \definition{v.}{retribuir; recompensar | dar \emph{feedback}; fornecer \emph{feedback}; dar retorno; dar parecer}
\end{EntryWithPhonetic}

\begin{EntryWithPhonetic}{回来}{hui2 lai5}{6,7}{⼞、⽊}[HSK 1]
  \definition{v.}{voltar; regressar (para a minha localização) | retornar; usado após um verbo, significa ``vir ao lugar original''}
\end{EntryWithPhonetic}

\begin{EntryWithPhonetic}{回落}{hui2luo4}{6,12}{⼞、⾋}[HSK 7-9]
  \definition{v.}{(níveis de água, preços, etc.) cair após uma subida; diminuir (oposto a 回升) | recuar | retornar ao nível baixo após uma subida (no nível da água, preço etc.)}
  \seealsoref{回升}{hui2sheng1}
\end{EntryWithPhonetic}

\begin{EntryWithPhonetic}{回去}{hui2 qu4}{6,5}{⼞、⼛}[HSK 1]
  \definition{v.}{retornar; voltar; estar de volta ; (a partir da minha localização)}
\end{EntryWithPhonetic}

\begin{EntryWithPhonetic}{回升}{hui2sheng1}{6,4}{⼞、⼗}[HSK 7-9]
  \definition{v.}{levantar-se novamente (após uma queda); levantar-se (oposto a 回落) | recuperar; cair e depois subir novamente}
  \seealsoref{回落}{hui2luo4}
\end{EntryWithPhonetic}

\begin{EntryWithPhonetic}{回收}{hui2shou1}{6,6}{⼞、⽁}[HSK 5]
  \definition{v.}{reciclar; reciclar itens (geralmente resíduos ou produtos antigos) para reutilização | recuperar; recolher; recuperar o que foi emitido ou demitido}
\end{EntryWithPhonetic}

\begin{EntryWithPhonetic}{回首}{hui2shou3}{6,9}{⼞、⾸}[HSK 7-9]
  \definition{v.}{virar a cabeça; virar-se (em volta) | olhar para trás; lembrar-se; recordar}
\end{EntryWithPhonetic}

\begin{EntryWithPhonetic}{回头}{hui2 tou2}{6,5}{⼞、⼤}[HSK 5]
  \definition{adv.}{mais tarde; depois de um tempo}
  \definition{conj.}{ou então; usado no início da segunda metade de uma frase para indicar o que acontecerá se você não fizer o que fez na primeira metade da frase}
  \definition{v.}{dar a meia-volta; virar a cabeça; virar a cabeça para trás | retornar; voltar | arrepender-se; corrigir seu caminho; reconhecer e corrigir erros}
\end{EntryWithPhonetic}

\begin{EntryWithPhonetic}{回味}{hui2wei4}{6,8}{⼞、⼝}[HSK 7-9]
  \definition{s.}{sabor residual; recordar o sabor agradável de\dots; o gosto residual que fica na boca depois de comer}
  \definition{v.}{recordar e ponderar sobre; reviver coisas que você vivenciou ou com as quais entrou em contato}
\end{EntryWithPhonetic}

\begin{EntryWithPhonetic}{回想}{hui2xiang3}{6,13}{⼞、⼼}[HSK 7-9]
  \definition{v.}{recordar; pensar de novo; pensar (no passado)}
\end{EntryWithPhonetic}

\begin{EntryWithPhonetic}{回信}{hui2/xin4}{6,9}{⼞、⼈}[HSK 5]
  \definition[封]{s.}{uma carta em resposta; uma mensagem verbal em resposta}
  \definition{v.+compl.}{escrever em resposta; escrever de volta; responder uma carta; responder verbalmente uma mensagem}
\end{EntryWithPhonetic}

\begin{EntryWithPhonetic}{回旋}{hui2xuan2}{6,11}{⼞、⽅}
  \definition{v.}{circular | rodar | dar a volta}
\end{EntryWithPhonetic}

\begin{EntryWithPhonetic}{回忆}{hui2yi4}{6,4}{⼞、⼼}[HSK 5]
  \definition[个,段]{s.}{memória; lembrança de eventos ou experiências passadas}
  \definition{v.}{lembrar; recordar}
\end{EntryWithPhonetic}

\begin{EntryWithPhonetic}{回忆录}{hui2yi4lu4}{6,4,8}{⼞、⼼、⼹}[HSK 7-9]
  \definition{s.}{memórias; reminiscências; lembranças; um gênero de escrita que relata experiências pessoais ou eventos históricos familiares}
\end{EntryWithPhonetic}

\begin{EntryWithPhonetic}{回应}{hui2 ying4}{6,7}{⼞、⼴}[HSK 6]
  \definition{v.}{responder}
\end{EntryWithPhonetic}

\begin{EntryWithPhonetic}{廻}{hui2}{8}{⼵}
  \variantof{回}
\end{EntryWithPhonetic}

\begin{EntryWithPhonetic}{悔}{hui3}{10}{⼼}
  \definition{v.}{lamentar; arrepender-se}
\end{EntryWithPhonetic}

\begin{EntryWithPhonetic}{悔恨}{hui3hen4}{10,9}{⼼、⼼}[HSK 7-9]
  \definition{v.}{arrepender-se profundamente; estar amargamente arrependido}
\end{EntryWithPhonetic}

\begin{EntryWithPhonetic}{毁}{hui3}{13}{⽎}[HSK 6]
  \definition{v.}{destruir; arruinar; danificar | (dialeto)  transformar, remodelar um item antigo em outra coisa, geralmente roupas | queimar | difamar; caluniar}
\end{EntryWithPhonetic}

\begin{EntryWithPhonetic}{毁坏}{hui3huai4}{13,7}{⽎、⼟}[HSK 7-9]
  \definition{v.}{danificar; devastar; degradar; destroçar; estilhaçar; vandalizar}
\end{EntryWithPhonetic}

\begin{EntryWithPhonetic}{毁灭}{hui3mie4}{13,5}{⽎、⽕}[HSK 7-9]
  \definition{v.}{arruinar; destruir; exterminar; destruir ou eliminar completamente}
\end{EntryWithPhonetic}

\begin{EntryWithPhonetic}{汇}{hui4}{5}{⽔}[HSK 4]
  \definition{s.}{montagem; coleção; coisas coletadas}
  \definition{v.}{convergir | reunir; coletar | remeter | trocar (câmbio de moedas)}
\end{EntryWithPhonetic}

\begin{EntryWithPhonetic}{汇报}{hui4bao4}{5,7}{⽔、⼿}[HSK 4]
  \definition[份,次]{s.}{relatório; referindo-se ao conteúdo de declarações escritas ou orais feitas a um superior ou pessoa relevante para apresentar uma situação ou refletir um problema}
  \definition{v.}{relatar; fazer um relato de}
\end{EntryWithPhonetic}

\begin{EntryWithPhonetic}{汇合}{hui4he2}{5,6}{⽔、⼝}[HSK 7-9]
  \definition{s.}{confluência; fusão}
  \definition{v.}{convergir; juntar; reunir; encontrar}
\end{EntryWithPhonetic}

\begin{EntryWithPhonetic}{汇集}{hui4ji2}{5,12}{⽔、⾫}[HSK 7-9]
  \definition{v.}{aduzir; coletar; compilar | reunir-se; congestionar; convergir; reunir; juntar}
\end{EntryWithPhonetic}

\begin{EntryWithPhonetic}{汇聚}{hui4ju4}{5,14}{⽔、⽿}[HSK 7-9]
  \definition{v.}{juntar; montar; reunir; ajuntar; reunir-se; convergência e acumulação (usado principalmente para objetos)}
\end{EntryWithPhonetic}

\begin{EntryWithPhonetic}{汇款}{hui4/kuan3}{5,12}{⽔、⽋}[HSK 5]
  \definition[笔,个]{s.}{remessa; dinheiro enviado ou recebido}
  \definition{v.+compl.}{remeter dinheiro; fazer uma remessa; enviar dinheiro}
\end{EntryWithPhonetic}

\begin{EntryWithPhonetic}{汇率}{hui4lv4}{5,11}{⽔、⽞}[HSK 4]
  \definition[个,种]{s.}{taxa de câmbio; relação entre a moeda de um país e a de outro}
\end{EntryWithPhonetic}

\begin{EntryWithPhonetic}{会}{hui4}{6}{⼈}[HSK 1,2]
  \definition{adv.}{um momento}
  \definition{clas.}{momento; um curto período de tempo}
  \definition{s.}{reunião; festa; conferência; reunião com um objetivo específico | reunião; reunião no trabalho | feira do templo; festival religioso | associação; sociedade; sindicato; certas organizações públicas | oportunidade; ocasião; momento oportuno | cidade principal; capital; cidade central}
  \definition{suf.}{união; grupo; associação}
  \definition{v.}{ser provável que; ter certeza de; indica que é possível realizar (é possível responder à pergunta separadamente) |  poder; ser capaz de; significa saber como fazer ou ter a capacidade de fazer (geralmente se refere a coisas que precisam ser aprendidas) | saber; compreender; entender | encontrar; ver | reunir-se; reunir; agregar; juntar | destacar-se em; ser bom em; ser hábil em; indica proficiência | pagar (ou custear) contas}
  \seeref{kuai4}
\end{EntryWithPhonetic}

\begin{EntryWithPhonetic}{会场}{hui4chang3}{6,6}{⼈、⼟}[HSK 7-9]
  \definition[个]{s.}{local de encontro; lugar onde as pessoas se reúnem}
\end{EntryWithPhonetic}

\begin{EntryWithPhonetic}{会见}{hui4 jian4}{6,4}{⼈、⾒}[HSK 6]
  \definition{v.}{entrevistar; encontrar-se com (especialmente um visitante estrangeiro)}
\end{EntryWithPhonetic}

\begin{EntryWithPhonetic}{会面}{hui4/mian4}{6,9}{⼈、⾯}[HSK 7-9]
  \definition{v.+compl.}{reunir-se; encontrar}
\end{EntryWithPhonetic}

\begin{EntryWithPhonetic}{会首}{hui4shou3}{6,9}{⼈、⾸}
  \definition{s.}{chefe de uma sociedade | patrocinador de uma organização}
\end{EntryWithPhonetic}

\begin{EntryWithPhonetic}{会谈}{hui4 tan2}{6,10}{⼈、⾔}[HSK 5]
  \definition{v.}{manter conversações; comumente usado em assuntos internacionais ou atividades diplomáticas}
\end{EntryWithPhonetic}

\begin{EntryWithPhonetic}{会晤}{hui4wu4}{6,11}{⼈、⽇}[HSK 7-9]
  \definition{v.}{reunir-se (com líderes estaduais ou figuras sociais para discutir assuntos importantes)}
\end{EntryWithPhonetic}

\begin{EntryWithPhonetic}{会议}{hui4yi4}{6,5}{⼈、⾔}[HSK 3]
  \definition[次,届,个,场]{s.}{reunião; conferência; reunião organizada pela organização relevante para ouvir opiniões, discutir questões e distribuir tarefas | conselho; congresso; um órgão ou organização permanente que discute e trata frequentemente assuntos importantes}
\end{EntryWithPhonetic}

\begin{EntryWithPhonetic}{会意}{hui4yi4}{6,13}{⼈、⼼}[HSK 7-9]
  \definition{s.}{compreensão; conhecimento | compostos associativos, uma das seis categorias de caracteres chineses, que são formados pela combinação de dois ou mais elementos, cada um com um significado próprio, para criar um novo significado, por exemplo, 信, um caractere composto de 人 (homem) e 言 (palavra), significando uma mensagem ou algo em que se pode acreditar ou confiar}
  \definition{v.}{entender; saber}
\end{EntryWithPhonetic}

\begin{EntryWithPhonetic}{会员}{hui4 yuan2}{6,7}{⼈、⼝}[HSK 3]
  \definition[位,名,个,些]{s.}{membro; associado; membros de certos grupos ou organizações}
\end{EntryWithPhonetic}

\begin{EntryWithPhonetic}{会长}{hui4 zhang3}{6,4}{⼈、⾧}[HSK 6]
  \definition[位,名,个,些]{s.}{presidente de uma associação ou sociedade | presidente de um clube, comitê etc.}
\end{EntryWithPhonetic}

\begin{EntryWithPhonetic}{会诊}{hui4/zhen3}{6,7}{⼈、⾔}[HSK 7-9]
  \definition{s.}{consulta de médicos; consulta (de grupo)}[医生举行会诊,决定是否需要动手术。===Os médicos realizam uma consulta para decidir se a cirurgia é necessária.]
  \definition{v.+compl.}{(médicos) realizar uma consulta médica; consultar}
\end{EntryWithPhonetic}

\begin{EntryWithPhonetic}{绘}{hui4}{9}{⽷}
  \definition{v.}{pintar; desenhar}
\end{EntryWithPhonetic}

\begin{EntryWithPhonetic}{绘画}{hui4 hua4}{9,8}{⽷、⽥}[HSK 6]
  \definition{s.}{desenho; pintura}
  \definition{v.}{desenhar; pintar}
\end{EntryWithPhonetic}

\begin{EntryWithPhonetic}{绘声绘色}{hui4sheng1-hui4se4}{9,7,9,6}{⽷、⼠、⽷、⾊}[HSK 7-9]
  \definition{expr.}{vívido e colorido; uma descrição animada; vívido; animado}
\end{EntryWithPhonetic}

\begin{EntryWithPhonetic}{贿}{hui4}{10}{⾙}
  \definition[行]{s.}{bens; riqueza; objetos de valor; propriedade | suborno | Literário: wealth}
  \definition{v.}{subornar}
\end{EntryWithPhonetic}

\begin{EntryWithPhonetic}{贿赂}{hui4lu4}{10,10}{⾙、⾙}[HSK 7-9]
  \definition[笔]{s.}{suborno}
  \definition{v.}{subornar; subornar outros com dinheiro}
\end{EntryWithPhonetic}

\begin{EntryWithPhonetic}{昏}{hun1}{8}{⽇}
  \definition*{s.}{Sobrenome Hun}
  \definition{adj.}{escuro; fraco; embaçado | confuso; embaraçado; inconsciente}
  \definition{s.}{crepúsculo; tarde}
  \definition{v.}{perder a consciência; desmaiar}
\end{EntryWithPhonetic}

\begin{EntryWithPhonetic}{昏迷}{hun1mi2}{8,9}{⽇、⾡}[HSK 7-9]
  \definition{s.}{coma; um estado em que uma pessoa perde a sensibilidade e o conhecimento}
  \definition{v.}{entrar em coma}
\end{EntryWithPhonetic}

\begin{EntryWithPhonetic}{荤}{hun1}{9}{⾋}[Kangxi 9]
  \definition{adj.}{obsceno; lascivo; vulgar}
  \definition{s.}{carne ou peixe (oposto a 素) | Budismo: vegetais picantes proibidos aos vegetarianos budistas, como cebola, alho-poró, alho, etc. | alimentos não vegetarianos (carne, peixe etc.) | vegetais com cheiro forte (alho etc.)}
  \seealsoref{素}{su4}
\end{EntryWithPhonetic}

\begin{EntryWithPhonetic}{婚}{hun1}{11}{⼥}
  \definition{s.}{casamento}
  \definition{v.}{casar}
\end{EntryWithPhonetic}

\begin{EntryWithPhonetic}{婚礼}{hun1li3}{11,5}{⼥、⽰}[HSK 4]
  \definition[场]{s.}{casamento; núpcias; cerimônia de casamento}
\end{EntryWithPhonetic}

\begin{EntryWithPhonetic}{婚纱}{hun1sha1}{11,7}{⼥、⽷}[HSK 7-9]
  \definition[件,套,个]{s.}{vestido de noiva; um vestido especial usado pela noiva em seu casamento}
\end{EntryWithPhonetic}

\begin{EntryWithPhonetic}{婚姻}{hun1yin1}{11,9}{⼥、⼥}[HSK 7-9]
  \definition[桩,次,段]{s.}{casamento; matrimônio}
\end{EntryWithPhonetic}

\begin{EntryWithPhonetic}{浑}{hun2}{9}{⽔}
  \definition*{s.}{Sobrenome Hun}
  \definition{adj.}{lamacento; turvo | tolo; estúpido | simples e natural; sem sofisticação | inteiro; por toda parte}
  \variantof{混}
\end{EntryWithPhonetic}

\begin{EntryWithPhonetic}{浑身}{hun2shen1}{9,7}{⽔、⾝}[HSK 7-9]
  \definition{s.}{por todo o corpo; da cabeça aos pés; corpo inteiro}
\end{EntryWithPhonetic}

\begin{EntryWithPhonetic}{混}{hun2}{11}{⽔}
  \definition{adj.}{nublado; o mesmo que 浑, turvo | confuso; embaraçado; irracional}
  \variantof{浑}
  \seeref{hun4}
\end{EntryWithPhonetic}

\begin{EntryWithPhonetic}{魂}{hun2}{13}{⿁}[HSK 7-9]
  \definition[个]{s.}{alma | humor; espírito | espírito elevado de uma nação}
\end{EntryWithPhonetic}

\begin{EntryWithPhonetic}{混}{hun4}{11}{⽔}[HSK 6]
  \definition{adj.}{confuso; imundo; turvo; lamacento; impuro}
  \definition{adv.}{de forma imprudente; irresponsável; irrefletidamente}
  \definition{v.}{misturar; confundir; misturar verdadeiro e falso | passar por; esgueirar-se | vagar à deriva; arrastar-se; sobreviver de maneira superficial; contentar-se com | se dar bem com alguém}
  \seeref{hun2}
\end{EntryWithPhonetic}

\begin{EntryWithPhonetic}{混饭}{hun4/fan4}{11,7}{⽔、⾷}
  \definition{v.+compl.}{trabalhar para viver | Coloquial: se envolver em um trabalho apenas para ganhar a vida (sem ter nenhum interesse real nele) | comer às custas de outra pessoa}
\end{EntryWithPhonetic}

\begin{EntryWithPhonetic}{混合}{hun4he2}{11,6}{⽔、⼝}[HSK 6]
  \definition{s.}{híbrido; composto; refere-se a duas ou mais substâncias misturadas sem reação química, mas ainda mantendo suas respectivas propriedades (diferente de 化合)}
  \definition{v.}{misturar; mixar; misturar-se}
  \seealsoref{化合}{hua4he2}
\end{EntryWithPhonetic}

\begin{EntryWithPhonetic}{混乱}{hun4luan4}{11,7}{⽔、⼄}[HSK 6]
  \definition{adj.}{caótico; confuso; desordenado; desorganizado; fora de ordem}
  \definition[片]{s.}{caos; confusão}
\end{EntryWithPhonetic}

\begin{EntryWithPhonetic}{混凝土}{hun4ning2tu3}{11,16,3}{⽔、⼎、⼟}[HSK 7-9]
  \definition{s.}{concreto; material de construção feito pela mistura de cimento, areia, cascalho e água em uma determinada proporção; após o endurecimento, apresenta propriedades como resistência à pressão, resistência à água e resistência ao fogo}
\end{EntryWithPhonetic}

\begin{EntryWithPhonetic}{混淆}{hun4xiao2}{11,11}{⽔、⽔}[HSK 7-9]
  \definition{v.}{misturar; confundir; colocar duas coisas muito parecidas juntas sem conseguir diferenciá-las}
\end{EntryWithPhonetic}

\begin{EntryWithPhonetic}{混浊}{hun4zhuo2}{11,9}{⽔、⽔}[HSK 7-9]
  \definition{adj.}{lamacento; turvo; nublado | impuro; não transparente; não claro; turvo}
  \definition{s.}{nubécula; opacidade na córnea do olho}
\end{EntryWithPhonetic}

\begin{EntryWithPhonetic}{豁}{huo1}{17}{⾕}[HSK 7-9]
  \definition{v.}{cortar; quebrar; rachar; dividir | sacrificar; desistir; pagar o preço cruelmente}
  \seeref{hua2}
  \seeref{huo4}
\end{EntryWithPhonetic}

\begin{EntryWithPhonetic}{豁出去}{huo1/chu5qu4}{17,5,5}{⾕、⼐、⼛}[HSK 7-9]
  \definition{v.+compl.}{seguir em frente independentemente; pronto para arriscar tudo}[她豁出去所有,去追逐梦想。===Ela arriscou tudo para perseguir seu sonho.]
\end{EntryWithPhonetic}

\begin{EntryWithPhonetic}{和}{huo2}{8}{⼝}
  \definition{v.}{combinar uma substância em pó (farinha, gesso, etc.) com água; adicionar líquido ao pó e mexer ou amassar até ficar pegajoso ou espesso}
  \seeref{he2}
  \seeref{he4}
  \seeref{hu2}
  \seeref{huo4}
\end{EntryWithPhonetic}

\begin{EntryWithPhonetic}{活}{huo2}{9}{⽔}[HSK 3]
  \definition{adj.}{vivo; vivendo; indica que (alguma ação) foi realizada enquanto a pessoa ainda estava viva | vívido; animado; ativo | móvel; em movimento; ativo}
  \definition{adv.}{exatamente; simplesmente; expressa um grau elevado, equivalente a 真正 ou 简直}
  \definition{s.}{emprego; meios de subsistência; trabalho (geralmente refere-se a trabalho físico) | produto; algo fabricado}
  \definition{v.}{viver; ter vida; sobreviver (em oposição a 死) | salvar (a vida de uma pessoa); fazer sobreviver; manter a vida}
  \seealsoref{简直}{jian3zhi2}
  \seealsoref{死}{si3}
  \seealsoref{真正}{zhen1zheng4}
\end{EntryWithPhonetic}

\begin{EntryWithPhonetic}{活动}{huo2dong4}{9,6}{⽔、⼒}[HSK 2]
  \definition{adj.}{móvel; flexível para alterações ou mudanças}
  \definition[些,个,种,类,次]{s.}{atividade; ação tomada com o objetivo de alcançar um determinado objetivo}
  \definition{v.}{fazer exercício; movimentar-se | usar influência pessoal; usar meios irregulares | mover-se}
\end{EntryWithPhonetic}

\begin{EntryWithPhonetic}{活该}{huo2gai1}{9,8}{⽔、⾔}[HSK 7-9]
  \definition{v.aux.}{merecer (uma consequência negativa) | deveria (incluindo o significado de destino); ser decretado pelo destino}
\end{EntryWithPhonetic}

\begin{EntryWithPhonetic}{活力}{huo2li4}{9,2}{⽔、⼒}[HSK 5]
  \definition{s.}{vigor; vitalidade; energia; muito forte, geralmente usado para descrever pessoas, cidades, empresas, economias, etc.}
\end{EntryWithPhonetic}

\begin{EntryWithPhonetic}{活路}{huo2lu4}{9,13}{⽔、⾜}
  \definition{s.}{maneira de sobreviver | meio de subsistência}
  \seeref{huo2lu5}
\end{EntryWithPhonetic}

\begin{EntryWithPhonetic}{活路}{huo2lu5}{9,13}{⽔、⾜}
  \definition{s.}{labor | trabalho físico}
  \seeref{huo2lu4}
\end{EntryWithPhonetic}

\begin{EntryWithPhonetic}{活泼}{huo2po1}{9,8}{⽔、⽔}[HSK 5]
  \definition{adj.}{vívido; ativo; animado; brilhante; vivaz; cheio de vida | Química: reativo; significa que a substância é ativa e reage facilmente com outras substâncias}
\end{EntryWithPhonetic}

\begin{EntryWithPhonetic}{活期}{huo2qi1}{9,12}{⽔、⽉}[HSK 7-9]
  \definition{adj.}{atual; corrente; presente}
\end{EntryWithPhonetic}

\begin{EntryWithPhonetic}{活儿}{huo2r5}{9,2}{⽔、⼉}[HSK 7-9]
  \definition[点]{s.}{emprego; trabalho; geralmente trabalho físico | produto; produtos acabados: artesanato, tecnologia}
\end{EntryWithPhonetic}

\begin{EntryWithPhonetic}{活跃}{huo2yue4}{9,11}{⽔、⾜}[HSK 6]
  \definition{adj.}{ativo; dinâmico; pensamentos, ações ou atividades positivas; ocorrências frequentes | rápido; ativo; dinâmico}
  \definition{v.}{animar; tornar ativo | ser ativo}
\end{EntryWithPhonetic}

\begin{EntryWithPhonetic}{活着}{huo2zhe5}{9,11}{⽔、⽬}
  \definition{adj.}{vivo}
\end{EntryWithPhonetic}

\begin{EntryWithPhonetic}{火}{huo3}{4}{⽕}[HSK 3,4][Kangxi 86]
  \definition*{s.}{Sobrenome Huo}
  \definition{adj.}{ardente; flamejante; vermelho como fogo | efervescente; próspero}
  \definition{adv.}{urgentemente}
  \definition{clas.}{usado para unidades militares (antigo)}
  \definition[场,把,团,堆]{s.}{fogo; a luz e as chamas emitidas pela combustão de um objeto | fúria; metáfora para emoções agitadas, irritadas ou raivosas | calor interno (uma das seis causas de doenças) | armas de fogo e munições | a ação de lutar}
  \definition{v.}{ficar com raiva; perder a paciência}
\end{EntryWithPhonetic}

\begin{EntryWithPhonetic}{火暴}{huo3bao4}{4,15}{⽕、⽇}[HSK 7-9]
  \definition{adj.}{impetuoso; impetuoso; irritável; impaciente | próspero; florescente; vigoroso; animado}
\end{EntryWithPhonetic}

\begin{EntryWithPhonetic}{火柴}{huo3chai2}{4,10}{⽕、⽊}[HSK 5]
  \definition[根,盒,包]{s.}{fósforo (palito de fósforo); fósforo de segurança; iniciador de fogo feito de uma tira fina de madeira mergulhada em um composto de fósforo ou enxofre}
\end{EntryWithPhonetic}

\begin{EntryWithPhonetic}{火车}{huo3 che1}{4,4}{⽕、⾞}[HSK 1]
  \definition[个,列,节,班,趟]{s.}{trem; comboio}
\end{EntryWithPhonetic}

\begin{EntryWithPhonetic}{火车司机}{huo3che1 si1ji1}{4,4,5,6}{⽕、⾞、⼝、⽊}
  \definition{s.}{maquinista de trem}
\end{EntryWithPhonetic}

\begin{EntryWithPhonetic}{火锅}{huo3guo1}{4,12}{⽕、⾦}[HSK 7-9]
  \definition[顿]{s.}{\emph{hot pot}; uma panela feita de metal ou outro material, que pode ser usada para aquecer a sopa continuamente com eletricidade, álcool, etc., e depois adicionar carne, vegetais, etc. à sopa e comê-la enquanto cozinha; atualmente, refere-se principalmente a alimentos cozidos dessa maneira}
\end{EntryWithPhonetic}

\begin{EntryWithPhonetic}{火海}{huo3hai3}{4,10}{⽕、⽔}
  \definition{s.}{um mar de chamas}
\end{EntryWithPhonetic}

\begin{EntryWithPhonetic}{火候}{huo3hou5}{4,10}{⽕、⼈}[HSK 7-9]
  \definition{s.}{duração e grau de aquecimento, cozimento, fundição, etc.  |Coloquial: nível de realização | Coloquial: momento crucial, crítico; emergência}
\end{EntryWithPhonetic}

\begin{EntryWithPhonetic}{火花}{huo3hua1}{4,7}{⽕、⾋}[HSK 7-9]
  \definition[簇]{s.}{faísca; explosão de chamas | um padrão brilhante}
\end{EntryWithPhonetic}

\begin{EntryWithPhonetic}{火箭}{huo3jian4}{4,15}{⽕、⽵}[HSK 6]
  \definition[个,艘,发,枚]{s.}{foguete; uma aeronave de alta velocidade que utiliza força de reação para se impulsionar para a frente; é usado para lançar satélites, naves espaciais, etc.; também pode ser equipado com uma ogiva para fabricar um míssil}
\end{EntryWithPhonetic}

\begin{EntryWithPhonetic}{火炬}{huo3ju4}{4,8}{⽕、⽕}[HSK 7-9]
  \definition[把]{s.}{tocha}
\end{EntryWithPhonetic}

\begin{EntryWithPhonetic}{火辣辣}{huo3la4la4}{4,14,14}{⽕、⾟、⾟}[HSK 7-9]
  \definition{adj.}{ardente; escaldante; abrasador}
\end{EntryWithPhonetic}

\begin{EntryWithPhonetic}{火热}{huo3re4}{4,10}{⽕、⽕}[HSK 7-9]
  \definition{adj.}{ardente; fervente; fervoroso; apaixonado; escaldante; abrasador (opp. 冰冷)}
  \seealsoref{冰冷}{bing1leng3}
\end{EntryWithPhonetic}

\begin{EntryWithPhonetic}{火山}{huo3shan1}{4,3}{⽕、⼭}[HSK 7-9]
  \definition[座]{s.}{vulcão}
\end{EntryWithPhonetic}

\begin{EntryWithPhonetic}{火速}{huo3su4}{4,10}{⽕、⾡}[HSK 7-9]
  \definition{adv.}{em alta velocidade; com pressa; usar a velocidade mais rápida (para fazer coisas urgentes)}
\end{EntryWithPhonetic}

\begin{EntryWithPhonetic}{火腿}{huo3 tui3}{4,13}{⽕、⾁}[HSK 5]
  \definition[道,个]{s.}{presunto; as pernas de porco marinadas mais famosas são produzidas em Jinhua, na província de Zhejiang, e em Xuanwei, na província de Yunnan.}
\end{EntryWithPhonetic}

\begin{EntryWithPhonetic}{火焰}{huo3yan4}{4,12}{⽕、⽕}[HSK 7-9]
  \definition[团,缕,股,道]{s.}{chama; labareda; flama}
\end{EntryWithPhonetic}

\begin{EntryWithPhonetic}{火药}{huo3yao4}{4,9}{⽕、⾋}[HSK 7-9]
  \definition[桶,克]{s.}{pólvora; um tipo de explosivo que explode com fumaça, como pólvora preta, ou sem fumaça, como nitrato de celulose}
\end{EntryWithPhonetic}

\begin{EntryWithPhonetic}{火灾}{huo3 zai1}{4,7}{⽕、⽕}[HSK 5]
  \definition[起,场]{s.}{fogo (como um desastre); conflagração; desastres causados por incêndios}
\end{EntryWithPhonetic}

\begin{EntryWithPhonetic}{伙}{huo3}{6}{⼈}[HSK 4]
  \definition{clas.}{grupo; multidão; banda}
  \definition{s.}{iguaria; alimentação; refeições | parceiro; companheiro | coletivo de colegas}
  \definition{v.}{combinar; unir}
\end{EntryWithPhonetic}

\begin{EntryWithPhonetic}{伙伴}{huo3ban4}{6,7}{⼈、⼈}[HSK 4]
  \definition[个,位,群]{s.}{parceiro; companheiro; antigo sistema militar de dez pessoas para uma fogueira, o chefe da fogueira, uma pessoa encarregada de cozinhar, com a fogueira é chamado de parceiro da fogueira, agora se refere à participação comum em uma determinada organização ou engajada em certas atividades}
\end{EntryWithPhonetic}

\begin{EntryWithPhonetic}{伙食}{huo3shi2}{6,9}{⼈、⾷}[HSK 7-9]
  \definition{s.}{angu; rancho; comida; refeições; refere-se às refeições no refeitório coletivo da unidade}
\end{EntryWithPhonetic}

\begin{EntryWithPhonetic}{和}{huo4}{8}{⼝}
  \definition{clas.}{usado para enxágues de roupas | usado para fervuras de ervas medicinais}
  \definition{v.}{misturar (ingredientes); misturar pós ou grãos; misturar com água para obter uma consistência mais líquida}
  \seeref{he2}
  \seeref{he4}
  \seeref{hu2}
  \seeref{huo2}
\end{EntryWithPhonetic}

\begin{EntryWithPhonetic}{或}{huo4}{8}{⼽}[HSK 2]
  \definition{adv.}{talvez; possivelmente; provavelmente | (geralmente na forma negativa) um pouco; ligeiramente}
  \definition{conj.}{ou (indicando escolha); ou\dots ou\dots}
  \definition{pron.}{alguém; algumas pessoas; refere-se a alguém ou algo, equivalente a 有人 ou 有的}
  \seealsoref{有的}{you3 de5}
  \seealsoref{有人}{you3 ren2}
\end{EntryWithPhonetic}

\begin{EntryWithPhonetic}{或多或少}{huo4duo1-huo4shao3}{8,6,8,4}{⼽、⼣、⼽、⼩}[HSK 7-9]
  \definition{expr.}{mais ou menos}
\end{EntryWithPhonetic}

\begin{EntryWithPhonetic}{或是}{huo4 shi4}{8,9}{⼽、⽇}[HSK 5]
  \definition{adv.}{um ou outro; o outro}
  \definition{conj.}{ou; às vezes, é apenas uma de duas coisas}
\end{EntryWithPhonetic}

\begin{EntryWithPhonetic}{或许}{huo4xu3}{8,6}{⼽、⾔}[HSK 4]
  \definition{adv.}{talvez; possivelmente; receio; não tenho certeza}
\end{EntryWithPhonetic}

\begin{EntryWithPhonetic}{或者}{huo4zhe3}{8,8}{⼽、⽼}[HSK 2]
  \definition{adv.}{talvez; possivelmente}
  \definition{conj.}{ou (usado em expressões afirmativas); ou\dots ou\dots; usado em frases narrativas para indicar uma relação de escolha | ou (usado para indicar equação); indica relação de equivalência, indicando que os objetos anterior e posterior são iguais}
\end{EntryWithPhonetic}

\begin{EntryWithPhonetic}{货}{huo4}{8}{⾙}[HSK 4]
  \definition{s.}{dinheiro; moeda | bens; mercadorias; \emph{commodity} | refere-se a uma pessoa com um certo mau caráter (usado como um insulto) | riqueza; fortuna; um termo geral para dinheiro, joias, tecidos, etc.}
  \definition{v.}{vender}
\end{EntryWithPhonetic}

\begin{EntryWithPhonetic}{货币}{huo4bi4}{8,4}{⾙、⼱}[HSK 7-9]
  \definition[种]{s.}{dinheiro; moeda}
\end{EntryWithPhonetic}

\begin{EntryWithPhonetic}{货车}{huo4che1}{8,4}{⾙、⾞}[HSK 7-9]
  \definition[辆]{s.}{trem de mercadorias; trem de carga | vagão de carga; caminhão de carga | caminhão; caminhões e veículos de entrega}
\end{EntryWithPhonetic}

\begin{EntryWithPhonetic}{货物}{huo4wu4}{8,8}{⾙、⽜}[HSK 7-9]
  \definition[件,批,些,吨]{s.}{bens; mercadoria; \emph{commodity}; geralmente se refere a itens para venda}
\end{EntryWithPhonetic}

\begin{EntryWithPhonetic}{货运}{huo4yun4}{8,7}{⾙、⾡}[HSK 7-9]
  \definition{s.}{transporte de carga (oposto a 客运) | carga; frete | mercadorias transportadas}
  \seealsoref{客运}{ke4yun4}
\end{EntryWithPhonetic}

\begin{EntryWithPhonetic}{获}{huo4}{10}{⾋}[HSK 4]
  \definition*{s.}{Sobrenome Huo}
  \definition{v.}{capturar; pegar | obter; ganhar; colher | colher; ceifar}
\end{EntryWithPhonetic}

\begin{EntryWithPhonetic}{获得}{huo4de2}{10,11}{⾋、⼻}[HSK 4]
  \definition{v.}{adquirir; ganhar; obter; alcançar}
\end{EntryWithPhonetic}

\begin{EntryWithPhonetic}{获奖}{huo4 jiang3}{10,9}{⾋、⼤}[HSK 4]
  \definition{v.}{ganhar prêmio; ser recompensado; ganhar um prêmio; receber um prêmio}
\end{EntryWithPhonetic}

\begin{EntryWithPhonetic}{获取}{huo4 qu3}{10,8}{⾋、⼜}[HSK 4]
  \definition{v.}{adquirir; obter; ganhar; colher}
\end{EntryWithPhonetic}

\begin{EntryWithPhonetic}{获胜}{huo4sheng4}{10,9}{⾋、⾁}[HSK 7-9]
  \definition{v.}{vencer; ser vitorioso; triunfar; alcançar a vitória}
\end{EntryWithPhonetic}

\begin{EntryWithPhonetic}{获悉}{huo4xi1}{10,11}{⾋、⼼}[HSK 7-9]
  \definition{v.}{saber (de um evento); receber notícias; ser informado}
\end{EntryWithPhonetic}

\begin{EntryWithPhonetic}{祸}{huo4}{11}{⽰}
  \definition[场]{s.}{infortúnio; desastre; calamidade (oposto de 福) | desgraça; catástrofe}
  \definition{v.}{trazer desastre; arruinar | causar problemas}
  \seealsoref{福}{fu2}
\end{EntryWithPhonetic}

\begin{EntryWithPhonetic}{祸害}{huo4hai5}{11,10}{⽰、⼧}[HSK 7-9]
  \definition[个]{s.}{desastre; calamidade | maldição; flagelo}
  \definition{v.}{causar desastre; danificar; destruir; arruinar}
\end{EntryWithPhonetic}

\begin{EntryWithPhonetic}{惑}{huo4}{12}{⼼}
  \definition{v.}{ficar confuso; ficar perplexo | iludir; enganar; confundir}
\end{EntryWithPhonetic}

\begin{EntryWithPhonetic}{惑星}{huo4xing1}{12,9}{⼼、⽇}
  \definition{s.}{planeta}
  \seealsoref{行星}{xing2xing1}
\end{EntryWithPhonetic}

\begin{EntryWithPhonetic}{霍}{huo4}{16}{⾬}
  \definition*{s.}{Sobrenome Huo}
  \definition{adv.}{Literário: de repente; rapidamente}
\end{EntryWithPhonetic}

\begin{EntryWithPhonetic}{霍乱}{huo4luan4}{16,7}{⾬、⼄}[HSK 7-9]
  \definition{s.}{cólera; uma doença altamente contagiosa causada pelo Vibrio Cholerae | gastroenterite aguda (geralmente se refere a sintomas como vômitos intensos, diarreia, dor abdominal e cólicas)}
\end{EntryWithPhonetic}

\begin{EntryWithPhonetic}{豁}{huo4}{17}{⾕}
  \definition{adj.}{claro; aberto; de mente aberta; generoso}
  \definition{v.}{isentar; remeter}
\end{EntryWithPhonetic}

\begin{EntryWithPhonetic}{豁达}{huo4da2}{17,6}{⾕、⾡}[HSK 7-9]
  \definition{adj.}{otimista; de mente aberta; aberto e claro}
\end{EntryWithPhonetic}

%%%%% EOF %%%%%

