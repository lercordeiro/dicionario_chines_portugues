%%%
%%% F
%%%

\section*{F}\addcontentsline{toc}{section}{F}

\begin{EntryWithPhonetic}{发}{fa1}{5}{⼜}[HSK 2]
  \definition*{s.}{Sobrenome Fa}
  \definition{clas.}{bala, usada para munições e cartuchos}
  \definition{v.}{distribuir; enviar; entregar | emitir; disparar; lançar; descarregar | produzir; gerar; criar (dar origem a) | proferir; emitir; expressar | expandir; desenvolver | prosperar; prosperidade graças à aquisição de bens materiais | crescer ou expandir quando fermentado ou embebido | difundir; dispersar; espalhar | expor; descobrir; revelar | transformar-se; tornar-se; entrar em um determinado estado | demonstrar seus sentimentos; expressar (sentimentos) | sentir; ter um sentimento | começar; estabelecer | fazer com que se faça; iniciar um empreendimento; começar a agir; provocar uma ação}
  \seeref{fa4}
\end{EntryWithPhonetic}

\begin{EntryWithPhonetic}{发表}{fa1biao3}{5,8}{⼜、⾐}[HSK 3]
  \definition{v.}{publicar; entregar; emitir; expressar; anunciar; expressar (opiniões) ou divulgar (assuntos) ao público, verbalmente ou por escrito | publicar em jornais (artigos, etc.)}
\end{EntryWithPhonetic}

\begin{EntryWithPhonetic}{发病}{fa1 bing4}{5,10}{⼜、⽧}[HSK 6]
  \definition{v.}{(de uma doença) avanço | patogênese; morbidade | surto (de uma doença)}
\end{EntryWithPhonetic}

\begin{EntryWithPhonetic}{发布}{fa1bu4}{5,5}{⼜、⼱}[HSK 5]
  \definition{v.}{emitir; publicar; liberar; anunciar; fazer ordens públicas, anúncios, notícias, etc.}
\end{EntryWithPhonetic}

\begin{EntryWithPhonetic}{发财}{fa1/cai2}{5,7}{⼜、⾙}
  \definition{v.+compl.}{ficar rico | fazer fortuna}
\end{EntryWithPhonetic}

\begin{EntryWithPhonetic}{发愁}{fa1/chou2}{5,13}{⼜、⼼}
  \definition{v.+compl.}{preocupar-se | ficar ansioso | ficar triste}
\end{EntryWithPhonetic}

\begin{EntryWithPhonetic}{发出}{fa1 chu1}{5,5}{⼜、⼐}[HSK 3]
  \definition{v.}{fazer; produzir; deixar sair; ocorrer (som, dúvida, etc.) | emitir; anunciar; publicar; divulgar (ordens, instruções) | enviar (mercadorias, cartas, etc.); partir (veículos, etc.) | emitir; exalar (cheiro, calor, etc.)}
\end{EntryWithPhonetic}

\begin{EntryWithPhonetic}{发达}{fa1da2}{5,6}{⼜、⾡}[HSK 3]
  \definition{adj.}{desenvolvido; florescente; (coisas) Já estão bem desenvolvidas; (negócios) prosperam}
  \definition{v.}{desenvolver; promover; florescer; a pessoa tem um bom desempenho profissional e é muito bem-sucedida}
\end{EntryWithPhonetic}

\begin{EntryWithPhonetic}{发电}{fa1 dian4}{5,5}{⼜、⽥}[HSK 6]
  \definition{s.}{geração de energia elétrica; produção de eletricidade; fornecimento de energia}
  \definition{v.}{gerar eletricidade (ou energia elétrica) | enviar um telegrama}
\end{EntryWithPhonetic}

\begin{EntryWithPhonetic}{发动}{fa1dong4}{5,6}{⼜、⼒}[HSK 3]
  \definition{v.}{iniciar; começar; lançar | chamar à ação; mobilizar; despertar | ligar o motor; dar a partida; dar o pontapé inicial (motor de combustão interna) | estimular; colocar em ação}
\end{EntryWithPhonetic}

\begin{EntryWithPhonetic}{发动机}{fa1dong4ji1}{5,6,6}{⼜、⼒、⽊}
  \definition[台]{s.}{motor}
\end{EntryWithPhonetic}

\begin{EntryWithPhonetic}{发抖}{fa1dou3}{5,7}{⼜、⼿}
  \definition{v.}{tremer | sacudir | estremecer}
\end{EntryWithPhonetic}

\begin{EntryWithPhonetic}{发放}{fa1 fang4}{5,8}{⼜、⽅}[HSK 6]
  \definition{v.}{conceder; estender; fornecer; (governo, organização) distribuir dinheiro ou suprimentos para os necessitados | emitir; enviar}
\end{EntryWithPhonetic}

\begin{EntryWithPhonetic}{发挥}{fa1hui1}{5,9}{⼜、⼿}[HSK 4]
  \definition{v.}{colocar em jogo; dar jogo a; dar espaço a; dar rédea solta a; revelar a natureza ou a capacidade interior | expressar; desenvolver (uma ideia, um tema, etc.); elaborar; fazer valer o ponto ou o motivo}
\end{EntryWithPhonetic}

\begin{EntryWithPhonetic}{发觉}{fa1jue2}{5,9}{⼜、⾒}[HSK 5]
  \definition{v.}{vir a saber; estar ciente (de); perceber; tornar-se consciente | encontrar; detectar; perceber; descobrir}
\end{EntryWithPhonetic}

\begin{EntryWithPhonetic}{发明}{fa1ming2}{5,8}{⼜、⽇}[HSK 3]
  \definition[个,项,种]{s.}{invenção; novos produtos ou métodos inventados}
  \definition{v.}{inventar; pesquisa que cria (novos produtos ou novos métodos) | expor; explicar; explicação criativa}
\end{EntryWithPhonetic}

\begin{EntryWithPhonetic}{发明者}{fa1ming2zhe3}{5,8,8}{⼜、⽇、⽼}
  \definition{s.}{inventor}
\end{EntryWithPhonetic}

\begin{EntryWithPhonetic}{发怒}{fa1 nu4}{5,9}{⼜、⼼}[HSK 6]
  \definition{v.}{ficar com raiva; explodir; perder a paciência | entrar em fúria | entrar em fúria (paixão)}
\end{EntryWithPhonetic}

\begin{EntryWithPhonetic}{发票}{fa1piao4}{5,11}{⼜、⽰}[HSK 4]
  \definition[张,种]{s.}{conta; recibo; fatura; recibos emitidos por lojas ou outros escritórios de cobrança}
\end{EntryWithPhonetic}

\begin{EntryWithPhonetic}{发起}{fa1 qi3}{5,10}{⼜、⾛}[HSK 6]
  \definition{s.}{iniciador; patrocinador}
  \definition{v.}{iniciar; patrocinar; começar; lançar}
\end{EntryWithPhonetic}

\begin{EntryWithPhonetic}{发烧}{fa1shao1}{5,10}{⼜、⽕}[HSK 4]
  \definition{v.}{ter febre; a temperatura corporal normal de uma pessoa é de cerca de 37ºC; se exceder 37,5ºC, é febre}
\end{EntryWithPhonetic}

\begin{EntryWithPhonetic}{发射}{fa1she4}{5,10}{⼜、⼨}[HSK 5]
  \definition{v.}{subir; disparar; lançar; irradiar; projetar; descarregar; enviar algo (como uma bala, um projétil, um satélite, etc.) de um dispositivo em uma velocidade muito alta}
\end{EntryWithPhonetic}

\begin{EntryWithPhonetic}{发生}{fa1sheng1}{5,5}{⼜、⽣}[HSK 3]
  \definition{v.}{ocorrer; acontecer; tomar lugar; surgir algo que não existia antes}
\end{EntryWithPhonetic}

\begin{EntryWithPhonetic}{发送}{fa1 song4}{5,9}{⼜、⾡}[HSK 3]
  \definition{v.}{enviar; despachar | transmitir (rádio)}
\end{EntryWithPhonetic}

\begin{EntryWithPhonetic}{发现}{fa1xian4}{5,8}{⼜、⾒}[HSK 2]
  \definition[个,项]{s.}{descoberta; achado}
  \definition{v.}{encontrar; descobrir; detectar; identificar; através de pesquisa, exploração, etc., ver ou encontrar coisas ou leis que os antepassados não viram | descobrir; perceber; perceber; notar; estar ciente de}
\end{EntryWithPhonetic}

\begin{EntryWithPhonetic}{发现者}{fa1xian4 zhe3}{5,8,8}{⼜、⾒、⽼}
  \definition{s.}{descobridor}
\end{EntryWithPhonetic}

\begin{EntryWithPhonetic}{发泄}{fa1xie4}{5,8}{⼜、⽔}
  \definition{v.}{soltar; abreviar; dar vazão a; desabafar emoções ou desejos}
\end{EntryWithPhonetic}

\begin{EntryWithPhonetic}{发行}{fa1xing2}{5,6}{⼜、⾏}[HSK 5]
  \definition{v.}{emitir; liberar; publicar; emitir ou vender de publicações recém-impressas, moeda, selos, etc.}
\end{EntryWithPhonetic}

\begin{EntryWithPhonetic}{发言}{fa1/yan2}{5,7}{⼜、⾔}[HSK 3]
  \definition[个]{s.}{discurso; declaração; palestra; opiniões publicadas}
  \definition{v.+compl.}{falar; fazer uma declaração (discurso); expressar opinião (geralmente em reuniões)}
\end{EntryWithPhonetic}

\begin{EntryWithPhonetic}{发言人}{fa1 yan2 ren2}{5,7,2}{⼜、⾔、⼈}[HSK 6]
  \definition{s.}{porta-voz}
\end{EntryWithPhonetic}

\begin{EntryWithPhonetic}{发炎}{fa1yan2}{5,8}{⼜、⽕}[HSK 6]
  \definition{s.}{inflamação}
  \definition{v.}{irritar; inflamar; reação complexa de organismos a fatores patogênicos, como microrganismos, substâncias químicas e estímulos físicos; os sintomas sistêmicos incluem aumento da temperatura corporal, alterações na composição do sangue, vermelhidão local, inchaço, febre, dor, etc.}
\end{EntryWithPhonetic}

\begin{EntryWithPhonetic}{发音}{fa1yin1}{5,9}{⼜、⾳}
  \definition{s.}{pronúncia}
  \definition{v.}{pronunciar}
\end{EntryWithPhonetic}

\begin{EntryWithPhonetic}{发展}{fa1zhan3}{5,10}{⼜、⼫}[HSK 3]
  \definition{v.}{crescer; expandir; avançar; desenvolver; a mudança das coisas de pequeno para grande, de simples para complexo, de inferior para superior | recrutar; admitir expandir (organização, escala, etc.)}
\end{EntryWithPhonetic}

\begin{EntryWithPhonetic}{罚}{fa2}{9}{⽹}[HSK 5]
  \definition{s.}{punição; penalidade}
  \definition{v.}{punir; penalizar; multar; confiscar}
\end{EntryWithPhonetic}

\begin{EntryWithPhonetic}{罚款}{fa2/kuan3}{9,12}{⽹、⽋}[HSK 5]
  \definition[笔,次,宗]{s.}{multa; penalidade; refere-se ao dinheiro pago por uma pessoa ou entidade de acordo com as disposições de um delito ou violação de contrato ou contrato}
  \definition{v.+compl.}{multar; penalizar; exigir, de acordo com os regulamentos, uma determinada quantia de dinheiro de uma pessoa ou entidade que tenha violado a lei ou descumprido um regulamento ou contrato}
\end{EntryWithPhonetic}

\begin{EntryWithPhonetic}{筏}{fa2}{12}{⽵}
  \definition[条]{s.}{jangada (de troncos, bambus, etc.)}
\end{EntryWithPhonetic}

\begin{EntryWithPhonetic}{法}{fa3}{8}{⽔}[HSK 4]
  \definition*{s.}{Doutrina budista; o dharma | França, abreviação de 法国 | Sobrenome Fa}
  \definition{adj.}{(usado após advérbios negativos) legal; cumpridor da lei}
  \definition{clas.}{F; Farad, medida de capacitância}
  \definition{s.}{lei; termo geral para regras de comportamento estabelecidas ou endossadas pelo Estado | maneira; método; modo; meios | padrão; modelo | artes mágicas; feitiço}
  \definition{v.}{seguir; imitar; aprender (os pontos fortes dos outros) |}
  \seealsoref{法国}{fa3guo2}
\end{EntryWithPhonetic}

\begin{EntryWithPhonetic}{法官}{fa3 guan1}{8,8}{⽔、⼧}[HSK 4]
  \definition[位,名,个,些]{s.}{juiz; justiça; termo genérico para um membro do judiciário em um tribunal de justiça}
\end{EntryWithPhonetic}

\begin{EntryWithPhonetic}{法规}{fa3 gui1}{8,8}{⽔、⾒}[HSK 5]
  \definition[部,项,条,套,个]{s.}{lei e regulamento; estatuto; termo geral para leis, decretos, regulamentos, regras, estatutos, etc.}
\end{EntryWithPhonetic}

\begin{EntryWithPhonetic}{法国}{fa3guo2}{8,8}{⽔、⼞}
  \definition*{s.}{França}
\end{EntryWithPhonetic}

\begin{EntryWithPhonetic}{法国人}{fa3guo2ren2}{8,8,2}{⽔、⼞、⼈}
  \definition{s.}{francês | pessoa ou povo da França}
\end{EntryWithPhonetic}

\begin{EntryWithPhonetic}{法律}{fa3lv4}{8,9}{⽔、⼻}[HSK 4]
  \definition[项,条,套,个]{s.}{lei; estatuto; regras de conduta formuladas pelo legislativo e cuja aplicação é garantida pelo poder estatal}
\end{EntryWithPhonetic}

\begin{EntryWithPhonetic}{法庭}{fa3 ting2}{8,9}{⽔、⼴}[HSK 6]
  \definition{s.}{corte; tribunal | tribunal; um órgão estatal que exerce o poder judicial de forma independente}
\end{EntryWithPhonetic}

\begin{EntryWithPhonetic}{法网}{fa3wang3}{8,6}{⽔、⽹}
  \definition*{s.}{Torneio de Roland Garros (French Open), torneio de tênis}
\end{EntryWithPhonetic}

\begin{EntryWithPhonetic}{法文}{fa3wen2}{8,4}{⽔、⽂}
  \definition[份]{s.}{françês, língua francesa}
\end{EntryWithPhonetic}

\begin{EntryWithPhonetic}{法语}{fa3 yu3}{8,9}{⽔、⾔}[HSK 6]
  \definition[种,门,句,段]{s.}{françês, língua francesa}
\end{EntryWithPhonetic}

\begin{EntryWithPhonetic}{法院}{fa3yuan4}{8,9}{⽔、⾩}[HSK 4]
  \definition[所,座]{s.}{tribunal; corte; órgãos estatais que exercem poder judicial independente}
\end{EntryWithPhonetic}

\begin{EntryWithPhonetic}{法制}{fa3 zhi4}{8,8}{⽔、⼑}[HSK 5]
  \definition{s.}{legalidade; instituições jurídicas; sistema jurídico}
\end{EntryWithPhonetic}

\begin{EntryWithPhonetic}{发}{fa4}{5}{⼜}
  \definition*{s.}{Sobrenome Fa}
  \definition[件]{s.}{cabelo}
  \seeref{fa1}
\end{EntryWithPhonetic}

\begin{EntryWithPhonetic}{发型}{fa4xing2}{5,9}{⼜、⼟}
  \definition{s.}{penteado}
\end{EntryWithPhonetic}

\begin{EntryWithPhonetic}{发簪}{fa4zan1}{5,18}{⼜、⽵}
  \definition{s.}{grampo de cabelo}
\end{EntryWithPhonetic}

\begin{EntryWithPhonetic}{番}{fan1}{12}{⽥}[HSK 6]
  \definition{adj.}{estrangeiro; de tribos estrangeiras; estrangeiro ou alienígena}
  \definition{clas.}{usado para o número de vezes que uma ação é executada, equivalente a 回 ou 次 | usado para o tipo de coisas, equivalente a 种}
  \definition{s.}{estrangeiro; de tribos estrangeiras; (velho) refere-se a países estrangeiros ou raças estrangeiras | tomate; batata-doce | aborígenes; nativos; povos indígenas}
  \definition{v.}{revezar; rotacionar; substituir}
  \seealsoref{次}{ci4}
  \seealsoref{回}{hui2}
  \seealsoref{种}{zhong3}
\end{EntryWithPhonetic}

\begin{EntryWithPhonetic}{番茄}{fan1 qie2}{12,8}{⽥、⾋}[HSK 6]
  \definition[个,斤,磅,公斤]{s.}{tomate | tomateiro}
\end{EntryWithPhonetic}

\begin{EntryWithPhonetic}{蕃}{fan1}{15}{⾋}
  \definition[种]{s.}{estrangeiros; aborígenes}
  \seeref{bo1}
  \seeref{fan2}
\end{EntryWithPhonetic}

\begin{EntryWithPhonetic}{蕃茄}{fan1 qie2}{15,8}{⾋、⾋}
  \variantof{番茄}
\end{EntryWithPhonetic}

\begin{EntryWithPhonetic}{翻}{fan1}{18}{⽻}[HSK 4]
  \definition{v.}{virar; dar a volta; inverter; mudar de posição; torcer; reverter | vasculhar; procurar; pesquisar; mover objetos para localizar algo | reverter; retrair; retirar | passar por cima; ultrapassar; cruzar | multiplicar | traduzir; decodificar | romper-se; cair; desentender-se com alguém}
\end{EntryWithPhonetic}

\begin{EntryWithPhonetic}{翻过}{fan1guo4}{18,6}{⽻、⾡}
  \definition{v.}{virar |  transformar}
\end{EntryWithPhonetic}

\begin{EntryWithPhonetic}{翻脸}{fan1/lian3}{18,11}{⽻、⾁}
  \definition{v.+compl.}{brigar com alguém | tornar-se hostil}
\end{EntryWithPhonetic}

\begin{EntryWithPhonetic}{翻译}{fan1yi4}{18,7}{⽻、⾔}[HSK 4]
  \definition[个,位,名]{s.}{tradutor; intérprete; pessoas que fazem trabalhos de tradução}
  \definition{v.}{traduzir; interpretar; colocar o significado de palavras de um idioma em palavras de outro idioma (expressão idiomática); expressar um significado em outro idioma}
\end{EntryWithPhonetic}

\begin{EntryWithPhonetic}{凡}{fan2}{3}{⼏}
  \definition*{s.}{Sobrenome Fan}
  \definition{adj.}{comum; ordinário}
  \definition{adv.}{qualquer; todos; todo | em tudo; completamente}
  \definition{s.}{este mundo mortal; a terra | o mundo secular; refere-se ao mundo humano | uma nota da escala em Gongchepu (工尺谱), correspondente a 4 na notação musical numerada | ideia geral; esboço}
  \seealsoref{工尺谱}{gong1 che3 pu3}
\end{EntryWithPhonetic}

\begin{EntryWithPhonetic}{凡是}{fan2shi4}{3,9}{⼏、⽇}[HSK 6]
  \definition{adv.}{todos; qualquer; cada; resumir tudo dentro de um determinado âmbito}
\end{EntryWithPhonetic}

\begin{EntryWithPhonetic}{烦}{fan2}{10}{⽕}[HSK 4]
  \definition{adj.}{redundante e confuso | supérfluo e confuso; muito bagunçado}
  \definition{v.}{aborrecer | irritar; incomodar; estar cansado de; ficar irritado | incomodar; solicitar}
\end{EntryWithPhonetic}

\begin{EntryWithPhonetic}{蕃}{fan2}{15}{⾋}
  \definition{adj.}{exuberante; próspero}
  \definition{v.}{multiplicar; proliferar}
  \seeref{bo1}
  \seeref{fan1}
\end{EntryWithPhonetic}

\begin{EntryWithPhonetic}{繁}{fan2}{17}{⽷}
  \definition{adj.}{em grande número; numerosos; múltiplos (oposto a 简) | em grande número; numerosos; complexos; complicado}
  \definition{v.}{propagar; multiplicar}
  \seealsoref{简}{jian3}
\end{EntryWithPhonetic}

\begin{EntryWithPhonetic}{繁荣}{fan2rong2}{17,9}{⽷、⾋}[HSK 5]
  \definition{adj.}{florescente; próspero}
  \definition{v.}{promover; prosperar}
\end{EntryWithPhonetic}

\begin{EntryWithPhonetic}{繁殖}{fan2zhi2}{17,12}{⽷、⽍}[HSK 6]
  \definition{v.}{criar; reproduzir; propagar; multiplicar; os organismos produzem novos indivíduos}
\end{EntryWithPhonetic}

\begin{EntryWithPhonetic}{反}{fan3}{4}{⼜}[HSK 4]
  \definition{adj.}{oposto; contrário; invertido}
  \definition{adv.}{pelo contrário; inversamente}
  \definition{v.}{inverter o lado; de cabeça para baixo; na direção oposta | virar; converter | retornar | opor-se; combater; voltar-se contra | rebelar-se; revoltar-se | inferir; deduzir; raciocinar por analogia}
\end{EntryWithPhonetic}

\begin{EntryWithPhonetic}{反倒}{fan3dao4}{4,10}{⼜、⼈}
  \definition{adv.}{em vez disso; pelo contrário}
  \definition{conj.}{em vez disso; pelo contrário; frequentemente acompanhadas por várias palavras que expressam negação}
\end{EntryWithPhonetic}

\begin{EntryWithPhonetic}{反对}{fan3dui4}{4,5}{⼜、⼨}[HSK 3]
  \definition{v.}{lutar; opor-se; objetar a; ser contra; discordar}
\end{EntryWithPhonetic}

\begin{EntryWithPhonetic}{反对党}{fan3dui4dang3}{4,5,10}{⼜、⼨、⼉}
  \definition{s.}{partido de oposição}
\end{EntryWithPhonetic}

\begin{EntryWithPhonetic}{反对派}{fan3dui4pai4}{4,5,9}{⼜、⼨、⽔}
  \definition{s.}{facção de oposição}
\end{EntryWithPhonetic}

\begin{EntryWithPhonetic}{反对票}{fan3dui4piao4}{4,5,11}{⼜、⼨、⽰}
  \definition{s.}{voto dissidente}
\end{EntryWithPhonetic}

\begin{EntryWithPhonetic}{反而}{fan3'er2}{4,6}{⼜、⽽}[HSK 4]
  \definition{adv.}{em vez disso; ao contrário de; contrário ao significado da frase anterior ou inesperado, desempenha o papel de uma reviravolta em uma frase}
\end{EntryWithPhonetic}

\begin{EntryWithPhonetic}{反复}{fan3fu4}{4,9}{⼜、⼢}[HSK 3]
  \definition{adv.}{repetidamente; de ​​novo e de novo; várias vezes}
  \definition{s.}{reversão; recaída; a situação anterior se repetiu}
  \definition{v.}{recuar; cortar e mudar; virar de cabeça para baixo; arrepender-se; aparecer várias vezes (usado principalmente em situações ruins)}
\end{EntryWithPhonetic}

\begin{EntryWithPhonetic}{反抗}{fan3kang4}{4,7}{⼜、⼿}[HSK 6]
  \definition{s.}{resistência}
  \definition{v.}{revoltar-se; resistir; opor-se com ação}
\end{EntryWithPhonetic}

\begin{EntryWithPhonetic}{反面}{fan3mian4}{4,9}{⼜、⾯}
  \definition{adj.}{oposto; negativo. ruim}
  \definition{s.}{costas; lado reverso; lado avesso; o lado de um objeto oposto à frente | o reverso de um estado de coisas, um problema, etc.; o outro lado de uma questão, problema, etc.}
\end{EntryWithPhonetic}

\begin{EntryWithPhonetic}{反问}{fan3wen4}{4,6}{⼜、⾨}[HSK 6]
  \definition{v.}{fazer uma pergunta em resposta; responder a uma pergunta com outra pergunta | fazer uma pergunta retórica (uma pergunta com significado negativo)}
\end{EntryWithPhonetic}

\begin{EntryWithPhonetic}{反响}{fan3 xiang3}{4,9}{⼜、⼝}[HSK 6]
  \definition{s.}{eco; reverberação; repercusão}
\end{EntryWithPhonetic}

\begin{EntryWithPhonetic}{反省}{fan3xing3}{4,9}{⼜、⽬}
  \definition{v.}{examinar a consciência | questionar-se | refletir sobre si mesmo | sondar a alma}
\end{EntryWithPhonetic}

\begin{EntryWithPhonetic}{反应}{fan3ying4}{4,7}{⼜、⼴}[HSK 3]
  \definition[个]{s.}{reação; resposta; opiniões, atitudes ou ações causadas pelo acontecimento}
  \definition{v.}{reagir; responder; atividade correspondente causada pela estimulação do organismo}
\end{EntryWithPhonetic}

\begin{EntryWithPhonetic}{反映}{fan3ying4}{4,9}{⼜、⽇}[HSK 4]
  \definition{s.}{reflexão; opiniões sobre pessoas ou situações}
  \definition{v.}{refletir; espelhar; figurativamente, trazer à tona a essência de uma questão objetiva (expressão idiomática); expressar a essência de algo objetivamente | relatar; tornar conhecido; informar às autoridades superiores | refletir; espelhar; a imagem de um objeto aparece invertida em outro objeto}
\end{EntryWithPhonetic}

\begin{EntryWithPhonetic}{反正}{fan3zheng4}{4,5}{⼜、⽌}[HSK 3]
  \definition{adv.}{de qualquer forma; de qualquer maneira; embora as circunstâncias sejam diferentes, o resultado é o mesmo | tudo igual; em qualquer caso; tom de voz que expressa afirmação categórica}
\end{EntryWithPhonetic}

\begin{EntryWithPhonetic}{返}{fan3}{7}{⾡}
  \definition{v.}{retornar; vir ou voltar}
\end{EntryWithPhonetic}

\begin{EntryWithPhonetic}{返回}{fan3 hui2}{7,6}{⾡、⼞}[HSK 5]
  \definition{v.}{retornar; ir (voltar); reverter; recorrer; retroceder; voltar para (o lugar original)}
\end{EntryWithPhonetic}

\begin{EntryWithPhonetic}{犯}{fan4}{5}{⽝}[HSK 6]
  \definition{s.}{criminoso}
  \definition{v.}{ofender; violar; ir contra | atacar; violar; trabalhar contra | fazer; ocorrer | voltar a; ter uma recorrência de; recair; retornar a (velhos hábitos)}
\end{EntryWithPhonetic}

\begin{EntryWithPhonetic}{犯法}{fan4fa3}{5,8}{⽝、⽔}
  \definition{v.}{violar (quebrar) a lei}
\end{EntryWithPhonetic}

\begin{EntryWithPhonetic}{犯规}{fan4 gui1}{5,8}{⽝、⾒}[HSK 6]
  \definition{v.}{quebrar as regras; violar regras | Esporte: cometer uma falta contra}
\end{EntryWithPhonetic}

\begin{EntryWithPhonetic}{犯罪}{fan4/zui4}{5,13}{⽝、⽹}[HSK 6]
  \definition{v.+compl.}{cometer  um crime}
\end{EntryWithPhonetic}

\begin{EntryWithPhonetic}{饭}{fan4}{7}{⾷}[HSK 1]
  \definition{s.}{(empréstimo linguístico) fã, devoto}
  \definition[顿,份,碗,口,锅]{s.}{cereais cozidos; grãos cozidos | refeição; alimentos consumidos diariamente em horários regulares | trabalho; meio de subsistência; meio de vida}
\end{EntryWithPhonetic}

\begin{EntryWithPhonetic}{饭店}{fan4dian4}{7,8}{⾷、⼴}[HSK 1]
  \definition[家,个]{s.}{restaurante | hotel; hotel grande e bem equipado}
\end{EntryWithPhonetic}

\begin{EntryWithPhonetic}{饭馆}{fan4 guan3}{7,11}{⾷、⾷}[HSK 2]
  \definition[家,个]{s.}{restaurante; lanchonete}
\end{EntryWithPhonetic}

\begin{EntryWithPhonetic}{范}{fan4}{9}{⾋}
  \definition*{s.}{Sobrenome Fan}
  \definition{s.}{padrão; molde; matriz | modelo; exemplo; modelo a seguir | limites; escopo | restrição; limite}
\end{EntryWithPhonetic}

\begin{EntryWithPhonetic}{范围}{fan4wei2}{9,7}{⾋、⼞}[HSK 3]
  \definition[个]{s.}{escopo; limite; alcance}
  \definition{v.}{estabelecer limites para; limitar o escopo de}
\end{EntryWithPhonetic}

\begin{EntryWithPhonetic}{方}{fang1}{4}{⽅}[HSK 4][Kangxi 70]
  \definition*{s.}{Alquimia, 方术 | Sobrenome Fang}
  \definition{adj.}{reto; honesto; imparcial}
  \definition{adv.}{exatamente quando; no momento em que}
  \definition{clas.}{usado para coisas quadradas | quadrado ou cúbico (geralmente metro quadrado ou cúbico)}
  \definition[个,张]{s.}{quadrado; um quadrado ou sólido com seis faces quadradas | matemática: potência; o número de vezes que uma quantidade deve ser multiplicada por si mesma | direção | lado; festa | lugar; região; localidade | maneira; método; solução | prescrição | lei; regra}
  \seealsoref{方术}{fang1 shu4}
\end{EntryWithPhonetic}

\begin{EntryWithPhonetic}{方案}{fang1'an4}{4,10}{⽅、⽊}[HSK 4]
  \definition[个,些,种]{s.}{plano; esquema; programa; planos específicos para tratar de um determinado problema | o esquema criado pelo governo; medidas ou regulamentações formuladas e implementadas pelo governo ou autoridades relevantes}
\end{EntryWithPhonetic}

\begin{EntryWithPhonetic}{方便}{fang1bian4}{4,9}{⽅、⼈}[HSK 2]
  \definition{adj.}{conveniente; sem complicações; sem dificuldades; muito fácil| adequado; condições ou circunstâncias adequadas}
  \definition{s.}{conveniência}
  \definition{v.}{ir ao banheiro; uma maneira delicada de dizer ``ir ao banheiro'' | facilitar; tornar algo conveniente para alguém; facilitar a realização de tarefas ou o alcance de objetivos | ter dinheiro sobrando}
\end{EntryWithPhonetic}

\begin{EntryWithPhonetic}{方便面}{fang1 bian4 mian4}{4,9,9}{⽅、⼈、⾯}[HSK 2]
  \definition[袋,包,碗,桶]{s.}{macarrão instantâneo}
\end{EntryWithPhonetic}

\begin{EntryWithPhonetic}{方法}{fang1fa3}{4,8}{⽅、⽔}[HSK 2]
  \definition[种,个,套,类]{s.}{método; meio; maneira; sobre os meios e procedimentos para resolver questões relacionadas com o pensamento, a fala e as ações, etc.}
\end{EntryWithPhonetic}

\begin{EntryWithPhonetic}{方面}{fang1mian4}{4,9}{⽅、⾯}[HSK 2]
  \definition[个,种]{s.}{lado; campo; aspecto; respeito}
\end{EntryWithPhonetic}

\begin{EntryWithPhonetic}{方片}{fang1 pian4}{4,4}{⽅、⽚}
  \definition{s.}{ouros ♦ (em jogos de cartas)}
  \seealsoref{黑桃}{hei1 tao2}
  \seealsoref{红心}{hong2 xin1}
  \seealsoref{梅花}{mei2 hua1}
\end{EntryWithPhonetic}

\begin{EntryWithPhonetic}{方式}{fang1shi4}{4,6}{⽅、⼷}[HSK 3]
  \definition[种,个]{s.}{maneira; método}
\end{EntryWithPhonetic}

\begin{EntryWithPhonetic}{方术}{fang1 shu4}{4,5}{⽅、⽊}
  \definition{s.}{artes de cura, adivinhação, horóscopo etc. | Arcaico: artes sobrenaturais}
\end{EntryWithPhonetic}

\begin{EntryWithPhonetic}{方向}{fang1xiang4}{4,6}{⽅、⼝}[HSK 2]
  \definition[个,种]{s.}{direção; orientação; referindo-se a leste, sul, oeste, norte, sudeste, sudoeste, nordeste, noroeste, etc. | objetivo; meta; finalidade}
\end{EntryWithPhonetic}

\begin{EntryWithPhonetic}{方言}{fang1yan2}{4,7}{⽅、⾔}
  \definition*{s.}{O primeiro Dicionário de Dialeto Chinês, editado por Yang Xiong, 扬雄, no século I, contendo mais de 9.000 caracteres}
  \definition[口]{s.}{dialeto; um ramo regional de uma língua, formado durante sua evolução, que difere da língua padrão e é usado apenas em uma determinada área}
  \seealsoref{扬雄}{yang2xiong2}
\end{EntryWithPhonetic}

\begin{EntryWithPhonetic}{方针}{fang1zhen1}{4,7}{⽅、⾦}[HSK 4]
  \definition[个,项]{s.}{política; diretriz; princípio orientador; orientação da direção e das metas de um empreendimento}
\end{EntryWithPhonetic}

\begin{EntryWithPhonetic}{防}{fang2}{6}{⾩}[HSK 3]
  \definition*{s.}{Sobrenome Fang}
  \definition{s.}{defesa | dique; aterro | barragem; represa; estrutura para conter a água}
  \definition{v.}{proteger contra; prevenir contra; tomar precauções contra | defender-se contra}
\end{EntryWithPhonetic}

\begin{EntryWithPhonetic}{防范}{fang2 fan4}{6,9}{⾩、⾋}[HSK 6]
  \definition{v.}{vigiar; estar em guarda; ficar de olho}
\end{EntryWithPhonetic}

\begin{EntryWithPhonetic}{防护}{fang2hu4}{6,7}{⾩、⼿}
  \definition{v.}{defender | proteger}
\end{EntryWithPhonetic}

\begin{EntryWithPhonetic}{防晒}{fang2shai4}{6,10}{⾩、⽇}
  \definition{s.}{protetor solar}
\end{EntryWithPhonetic}

\begin{EntryWithPhonetic}{防守}{fang2shou3}{6,6}{⾩、⼧}[HSK 6]
  \definition{v.}{defender; guardar}
\end{EntryWithPhonetic}

\begin{EntryWithPhonetic}{防止}{fang2zhi3}{6,4}{⾩、⽌}[HSK 3]
  \definition{v.}{evitar; prevenir; prevenir; proteger contra; preparar-se com antecedência para evitar que coisas ruins aconteçam}
\end{EntryWithPhonetic}

\begin{EntryWithPhonetic}{防治}{fang2zhi4}{6,8}{⾩、⽔}[HSK 5]
  \definition{s.}{tratamento preventivo; prevenção e cura; profilaxia e tratamento}
\end{EntryWithPhonetic}

\begin{EntryWithPhonetic}{房}{fang2}{8}{⼾}
  \definition*{s.}{Fang, a quarta das vinte e oito constelações nas quais a esfera celeste foi dividida, consistindo de quatro estrelas quase em linha reta em Escorpião | Sobrenome Fang}
  \definition[幢,个,间]{s.}{casa; edifício | sala; quarto; câmara | estrutura semelhante a uma casa | um ramo de uma família extensa | loja; estoque | local de trabalho do artesão; oficina; moinho}
\end{EntryWithPhonetic}

\begin{EntryWithPhonetic}{房东}{fang2dong1}{8,5}{⼾、⼀}[HSK 3]
  \definition[个,位,名]{s.}{dono;  proprietário; senhorio; pessoas que alugam ou emprestam imóveis (para os 房客 )}
  \seealsoref{房客}{fang2ke4}
\end{EntryWithPhonetic}

\begin{EntryWithPhonetic}{房价}{fang2 jia4}{8,6}{⼾、⼈}[HSK 6]
  \definition{s.}{custo de moradia; tarifa de quarto | preço da casa}
\end{EntryWithPhonetic}

\begin{EntryWithPhonetic}{房间}{fang2jian1}{8,7}{⼾、⾨}[HSK 1]
  \definition[个,间,套]{s.}{sala; câmara; escritório; apartamento; divisões internas da casa}
\end{EntryWithPhonetic}

\begin{EntryWithPhonetic}{房客}{fang2ke4}{8,9}{⼾、⼧}[HSK 3]
  \definition{s.}{inquilino (de um quarto ou casa); hóspede (oposto a 房东) | inquilino; hóspede; pessoas que alugam ou emprestam imóveis para moradia (para o 房东)}
  \seealsoref{房东}{fang2dong1}
\end{EntryWithPhonetic}

\begin{EntryWithPhonetic}{房屋}{fang2 wu1}{8,9}{⼾、⼫}[HSK 3]
  \definition[间,所,套]{s.}{casas; habitação; edifícios}
\end{EntryWithPhonetic}

\begin{EntryWithPhonetic}{房主}{fang2zhu3}{8,5}{⼾、⼂}
  \definition{s.}{proprietário | dono de um imóvel}
\end{EntryWithPhonetic}

\begin{EntryWithPhonetic}{房子}{fang2 zi5}{8,3}{⼾、⼦}[HSK 1]
  \definition[栋,幢,座,套,间]{s.}{casa; edifício; prédio}
\end{EntryWithPhonetic}

\begin{EntryWithPhonetic}{房租}{fang2 zu1}{8,10}{⼾、⽲}[HSK 3]
  \definition[笔]{s.}{aluguel}
\end{EntryWithPhonetic}

\begin{EntryWithPhonetic}{仿}{fang3}{6}{⼈}
  \definition{adv.}{semelhante; como}
  \definition{s.}{caracteres escritos segundo um modelo de caligrafia | cartas modeladas a partir de uma cópia; palavras escritas de acordo com o modelo}
  \definition{v.}{imitar; copiar | assemelhar-se; ser como}
\end{EntryWithPhonetic}

\begin{EntryWithPhonetic}{仿佛}{fang3fu2}{6,7}{⼈、⼈}[HSK 6]
  \definition{adv.}{parece que; como se}
  \definition{v.}{ser como; parecer}
\end{EntryWithPhonetic}

\begin{EntryWithPhonetic}{访}{fang3}{6}{⾔}
  \definition{v.}{visitar; fazer uma visita; ligar para | procurar por meio de investigação ou busca; tentar obter; obter uma entrevista | entrevistar | investigar; procurar por meio de investigação (pesquisar)}
\end{EntryWithPhonetic}

\begin{EntryWithPhonetic}{访问}{fang3wen4}{6,6}{⾔、⾨}[HSK 3]
  \definition{v.}{visitar; ligar; entrevistar; visitar e conversar com um objetivo específico | visitar um \emph{site}}
\end{EntryWithPhonetic}

\begin{EntryWithPhonetic}{放}{fang4}{8}{⽅}[HSK 1]
  \definition{v.}{deixar ir; libertar; soltar | ceder; deixar-se levar | levar para se alimentar; pastar | soltar; liberar (ou expelir) | exibir (um filme, etc.); reproduzir (um disco, etc.) | acender; inflamar | emprestar (dinheiro) com juros | tornar maior ou mais longo; soltar; abaixar | moderar (a atitude ou o comportamento de alguém) | (de flores) florescer; abrir | colocar; posicionar; deitar | fazer com que algo (ou alguém) caia no chão | deixar de lado; guardar (para uso futuro); conservar | (seguido por 着\dots 不\dots) permitir que algo permaneça (por fazer, por pegar, por usar, etc.) | adicionar; colocar | colocar em pastagem; soltar para caçar | deixar de lado; suspender; interromper | remover; aliviar; livrar-se; proteger; libertar | deixar-se levar; sem restrições; libertino | mandar embora; tirar o prisioneiro da prisão e deportá-lo para uma região remota | distribuir; emitir; lançar | atear fogo | expandir; ampliar; prolongar | reajustar-se até certo ponto; controlar suas ações, adotar uma determinada atitude, atingir um certo equilíbrio | derrubar}
\end{EntryWithPhonetic}

\begin{EntryWithPhonetic}{放鞭炮}{fang4bian1pao4}{8,18,9}{⽅、⾰、⽕}
  \definition{s.}{um conjunto de bombinhas ou traques}
\end{EntryWithPhonetic}

\begin{EntryWithPhonetic}{放出}{fang4chu1}{8,5}{⽅、⼐}
  \definition{v.}{liberar | libertar}
\end{EntryWithPhonetic}

\begin{EntryWithPhonetic}{放大}{fang4da4}{8,3}{⽅、⼤}[HSK 5]
  \definition{v.}{amplificar; magnificar; aumentar; ampliar; aumentar o tamanho de imagens, textos, sons, etc.}
\end{EntryWithPhonetic}

\begin{EntryWithPhonetic}{放到}{fang4 dao4}{8,8}{⽅、⼑}[HSK 3]
  \definition{v.}{colocar em; meter}
\end{EntryWithPhonetic}

\begin{EntryWithPhonetic}{放电}{fang4dian4}{8,5}{⽅、⽥}
  \definition{s.}{descarga elétrica}
\end{EntryWithPhonetic}

\begin{EntryWithPhonetic}{放飞}{fang4fei1}{8,3}{⽅、⾶}
  \definition{s.}{deixar voar}
\end{EntryWithPhonetic}

\begin{EntryWithPhonetic}{放过}{fang4guo4}{8,6}{⽅、⾡}
  \definition{v.}{deixar | deixar alguém escapar impune | passar despercebido}
\end{EntryWithPhonetic}

\begin{EntryWithPhonetic}{放假}{fang4/jia4}{8,11}{⽅、⼈}[HSK 1]
  \definition{v.}{tirar férias (ou feriado); ter um dia de folga}
  \definition{v.+compl.}{tirar férias (ou feriado); começar as férias; ter um dia de folga; estar de férias (feriado)}
\end{EntryWithPhonetic}

\begin{EntryWithPhonetic}{放弃}{fang4qi4}{8,7}{⽅、⼶}[HSK 5]
  \definition{v.}{desistir, abandonar; descartar (direitos originais, reivindicações, opiniões, etc.)}
\end{EntryWithPhonetic}

\begin{EntryWithPhonetic}{放弃权利}{fang4qi4 quan2li4}{8,7,6,7}{⽅、⼶、⽊、⼑}
  \definition{s.}{renúncia}
\end{EntryWithPhonetic}

\begin{EntryWithPhonetic}{放弃者}{fang4qi4zhe3}{8,7,8}{⽅、⼶、⽼}
  \definition{s.}{desistente}
\end{EntryWithPhonetic}

\begin{EntryWithPhonetic}{放任}{fang4ren4}{8,6}{⽅、⼈}
  \definition{v.}{ignorar | saciar-se | deixar sozinho}
\end{EntryWithPhonetic}

\begin{EntryWithPhonetic}{放肆}{fang4si4}{8,13}{⽅、⾀}
  \definition{adj.}{atrevido | pesunçoso | devasso}
\end{EntryWithPhonetic}

\begin{EntryWithPhonetic}{放松}{fang4song1}{8,8}{⽅、⽊}[HSK 4]
  \definition{v.}{relaxar; afrouxar; soltar; desprender}
\end{EntryWithPhonetic}

\begin{EntryWithPhonetic}{放下}{fang4 xia4}{8,3}{⽅、⼀}[HSK 2]
  \definition{v.}{deitar-se; colocar no chão| deixar ir; soltar; desistir; largar | colocar; acomodar; depositar}
\end{EntryWithPhonetic}

\begin{EntryWithPhonetic}{放心}{fang4xin1}{8,4}{⽅、⼼}[HSK 2]
  \definition{adj.}{despreocupado}
  \definition{v.}{confiar; ter confiança em alguém; sentir-se aliviado; ficar tranquilo; ficar com a consciência tranquila}
\end{EntryWithPhonetic}

\begin{EntryWithPhonetic}{放学}{fang4/xue2}{8,8}{⽅、⼦}[HSK 1]
  \definition{v.+compl.}{encerrar; sair da escola; as aulas terminaram; a escola acabou (por hoje); voltar para casa depois de um dia ou meio dia de aula}
\end{EntryWithPhonetic}

\begin{EntryWithPhonetic}{放养}{fang4yang3}{8,9}{⽅、⼋}
  \definition{v.}{criar (gado, peixes, culturas, etc.) | crescer | criar}
\end{EntryWithPhonetic}

\begin{EntryWithPhonetic}{放走}{fang4zou3}{8,7}{⽅、⾛}
  \definition{v.}{permitir (uma pessoa ou um animal) ir | liberar | libertar}
\end{EntryWithPhonetic}

\begin{EntryWithPhonetic}{飞}{fei1}{3}{⾶}[HSK 1][Kangxi 183]
  \definition{adj.}{inesperado; acidental; surgido do nada}
  \definition{adv.}{rapidamente; velozmente}
  \definition{s.}{roda livre de uma bicicleta}
  \definition{v.}{voar; esvoaçar; (pássaros, insetos, etc.) voar pelo ar batendo as asas | voar; utilizar máquinas motorizadas para se deslocar no ar | voar; (objetos naturais) flutuar ou esvoaçar no ar | volatilizar; evaporar; um gás se dissipar no ar | ir muito rapidamente; movimentar-se rapidamente, como se estivesse voando}
\end{EntryWithPhonetic}

\begin{EntryWithPhonetic}{飞船}{fei1 chuan2}{3,11}{⾶、⾈}[HSK 6]
  \definition{s.}{nave espacial; espaçonave | dirigível; aerobarco}
\end{EntryWithPhonetic}

\begin{EntryWithPhonetic}{飞碟}{fei1die2}{3,14}{⾶、⽯}
  \definition{s.}{disco-voador, OVNI, \emph{UFO} | \emph{frisbee}}
\end{EntryWithPhonetic}

\begin{EntryWithPhonetic}{飞机}{fei1ji1}{3,6}{⾶、⽊}[HSK 1]
  \definition[架,个]{s.}{avião; aeronave; aroplano}
\end{EntryWithPhonetic}

\begin{EntryWithPhonetic}{飞机票}{fei1ji1 piao4}{3,6,11}{⾶、⽊、⽰}
  \definition[张]{s.}{bilhete de avião; documento emitido mediante pagamento de passagem aérea, que autoriza o titular a viajar}
  \seealsoref{机票}{ji1 piao4}
\end{EntryWithPhonetic}

\begin{EntryWithPhonetic}{飞行}{fei1 xing2}{3,6}{⾶、⾏}[HSK 3]
  \definition{s.}{voo; aviação}
  \definition{v.}{voar; fazer um voo; (aviões, foguetes, etc.) voar no ar}
\end{EntryWithPhonetic}

\begin{EntryWithPhonetic}{飞行员}{fei1 xing2 yuan2}{3,6,7}{⾶、⾏、⼝}[HSK 6]
  \definition[名,班]{s.}{piloto; aviador; pilotos de aeronaves}
\end{EntryWithPhonetic}

\begin{EntryWithPhonetic}{非}{fei1}{8}{⾮}[HSK 4][Kangxi 175]
  \definition*{s.}{África, abreviação de 非洲 | Sobrenome Fei}
  \definition{adv.}{Em resposta a 不, indica necessidade (deve)}
  \definition{pref.}{indicando negatividade ou exclusão}
  \definition{s.}{engano; erro}
  \definition{v.}{opor-se a; culpar; censurar | não estar em conformidade com; ser contrário a | não ser | ter que; simplesmente precisar (fazer algo)}
  \seealsoref{不}{bu4}
  \seealsoref{非洲}{fei1zhou1}
\end{EntryWithPhonetic}

\begin{EntryWithPhonetic}{非常}{fei1chang2}{8,11}{⾮、⼱}[HSK 1]
  \definition{adj.}{extraordinário; incomum; especial}
  \definition{adv.}{muito; extremamente; altamente}
\end{EntryWithPhonetic}

\begin{EntryWithPhonetic}{非洲}{fei1zhou1}{8,9}{⾮、⽔}
  \definition*{s.}{África}
\end{EntryWithPhonetic}

\begin{EntryWithPhonetic}{非洲人}{fei1zhou1ren2}{8,9,2}{⾮、⽔、⼈}
  \definition{s.}{africano | pessoa ou povo da África}
\end{EntryWithPhonetic}

\begin{EntryWithPhonetic}{肥}{fei2}{8}{⾁}[HSK 4]
  \definition{adj.}{gordo; gorduroso; contém muita gordura (o oposto de 瘦, geralmente não usado para descrever pessoas) | fértil; rico | solto; largo; folgado; (roupas, etc.) largas (em oposição a 瘦) | lucrativo; rendendo bons lucros}
  \definition{s.}{fertilizante; esterco}
  \definition{v.}{fertilizar; tornar fértil ou obeso | enriquecer com renda ilegal, ilícita}
  \seealsoref{瘦}{shou4}
\end{EntryWithPhonetic}

\begin{EntryWithPhonetic}{狒}{fei4}{8}{⽝}
  \definition{s.}{babuíno (uma espécie de macaco)}
\end{EntryWithPhonetic}

\begin{EntryWithPhonetic}{狒狒}{fei4fei4}{8,8}{⽝、⽝}
  \definition{s.}{babuíno}
\end{EntryWithPhonetic}

\begin{EntryWithPhonetic}{肺}{fei4}{8}{⾁}[HSK 6]
  \definition[叶]{s.}{pulmão | pulmões; órgãos respiratórios de humanos e animais superiores}
\end{EntryWithPhonetic}

\begin{EntryWithPhonetic}{费}{fei4}{9}{⾙}[HSK 3]
  \definition*{s.}{Sobrenome Fei}
  \definition{s.}{taxa; despesa; encargo}
  \definition{v.}{custar; gastar; despender | ser desperdiçador; consumir em excesso; gastar algo muito rapidamente; consumo excessivo (oposto a 省)}
  \seealsoref{省}{sheng3}
\end{EntryWithPhonetic}

\begin{EntryWithPhonetic}{费用}{fei4 yong4}{9,5}{⾙、⽤}[HSK 3]
  \definition[笔,个]{s.}{custo; despesa; desembolso}
\end{EntryWithPhonetic}

\begin{EntryWithPhonetic}{分}{fen1}{4}{⼑}[HSK 1,2]
  \definition{adj.}{filial (de uma organização)}
  \definition{clas.}{parte ou subdivisão | fração | um décimo (de certas unidades) | unidade de comprimento equivalente a 0,33cm | unidade de área (=66,666 metros quadrados) | unidade de peso (=1/2 grama) | minuto (unidade de tempo) | minuto (unidade de medida angular) | 0,01 yuan (unidade de dinheiro) | taxa de juros | marca; ponto; unidade de contagem para avaliação de notas, etc.}
  \definition{s.}{fração}
  \definition{v.}{separar; dividir; partir; dividir algo inteiro em várias partes ou separar coisas que estão ligadas entre si | atribuir; designar; distribuir | distinguir; diferenciar; diferenciar um do outro}
  \seeref{fen4}
\end{EntryWithPhonetic}

\begin{EntryWithPhonetic}{分别}{fen1bie2}{4,7}{⼑、⼑}[HSK 3]
  \definition{adv.}{diferentemente; de maneiras diferentes; expressar de maneiras diferentes | separadamente; individualmente; respectivamente}
  \definition{s.}{diferença; pontos diferentes}
  \definition{v.}{partir; deixar um ao outro; não estar mais junto | distinguir; diferenciar}
\end{EntryWithPhonetic}

\begin{EntryWithPhonetic}{分布}{fen1bu4}{4,5}{⼑、⼱}[HSK 4]
  \definition{v.}{espalhar; distribuir; dispersar (em uma determinada área)}
\end{EntryWithPhonetic}

\begin{EntryWithPhonetic}{分成}{fen1 cheng2}{4,6}{⼑、⼽}[HSK 5]
  \definition{v.}{dividir em; separar em; dividir dinheiro, bens, etc. de acordo com a porcentagem}
\end{EntryWithPhonetic}

\begin{EntryWithPhonetic}{分工}{fen1 gong1}{4,3}{⼑、⼯}[HSK 6]
  \definition[种]{s.}{divisão do trabalho}
  \definition{v.}{dividir o trabalho; envolver-se em várias tarefas diferentes, mas complementares}
\end{EntryWithPhonetic}

\begin{EntryWithPhonetic}{分公司}{fen1gong1si1}{4,4,5}{⼑、⼋、⼝}
  \definition{s.}{sucursal | filial de companhia}
\end{EntryWithPhonetic}

\begin{EntryWithPhonetic}{分解}{fen1jie3}{4,13}{⼑、⾓}[HSK 5]
  \definition{v.}{quebrar; separar em partes; dividir um todo em seus componentes | resolver; decompor | Química: transformar uma substância em duas ou mais substâncias por meio de uma reação química | desintegrar-se; dividir-se; desunir uma organização | mediar; fazer a paz; resolver conflitos e disputas | explicar; defender-se}
\end{EntryWithPhonetic}

\begin{EntryWithPhonetic}{分开}{fen1/kai1}{4,4}{⼑、⼶}[HSK 2]
  \definition{v.+compl.}{separar; dividir; desacoplar; desembalar; romper; desfolhar; decolar; romper; distribuir; separar de (em); dividir\dots de\dots | separar; fazer com que uma pessoa ou algo deixe de estar junto com outra pessoa ou coisa}
\end{EntryWithPhonetic}

\begin{EntryWithPhonetic}{分类}{fen1/lei4}{4,9}{⼑、⽶}[HSK 5]
  \definition{v.+compl.}{ordenar; classificar; categorizar; classificar as coisas de acordo com sua natureza e características}
\end{EntryWithPhonetic}

\begin{EntryWithPhonetic}{分离}{fen1 li2}{4,10}{⼑、⼇}[HSK 5]
  \definition{v.}{cortar; separar (de coisas) | separar; sair; separar (de pessoas); partir (em uma longa viagem)}
\end{EntryWithPhonetic}

\begin{EntryWithPhonetic}{分量}{fen1liang4}{4,12}{⼑、⾥}
  \definition{s.}{componente vetorial}
  \seeref{fen4liang4}
  \seeref{fen4liang5}
\end{EntryWithPhonetic}

\begin{EntryWithPhonetic}{分裂}{fen1lie4}{4,12}{⼑、⾐}[HSK 6]
  \definition{s.}{fissão; divisão}
  \definition{v.}{dividir; separar; romper}
\end{EntryWithPhonetic}

\begin{EntryWithPhonetic}{分配}{fen1pei4}{4,10}{⼑、⾣}[HSK 3]
  \definition{v.}{atribuir; dispor; organizar o trabalho, as tarefas, os recursos, o tempo, etc. | atribuir; compartilhar; distribuir dinheiro ou bens às pessoas envolvidas de acordo com um determinado plano, padrão ou regulamento}
\end{EntryWithPhonetic}

\begin{EntryWithPhonetic}{分散}{fen1san4}{4,12}{⼑、⽁}[HSK 4]
  \definition{adj.}{espalhado; disperso; desviado; fragmentado; sem foco}
  \definition{v.}{dispersar; espalhar; descentralizar | separar-se; desunir-se}
\end{EntryWithPhonetic}

\begin{EntryWithPhonetic}{分手}{fen1/shou3}{4,4}{⼑、⼿}[HSK 4]
  \definition{v.+compl.}{separar; romper; terminar um relacionamento ou um casal | separar-se (de uma empresa); dizer adeus; despedir-se da família, dos amigos, etc.}
\end{EntryWithPhonetic}

\begin{EntryWithPhonetic}{分数}{fen1 shu4}{4,13}{⼑、⽁}[HSK 2]
  \definition[个]{s.}{fração; número fracionário | nota; classificação; ponto; pontuação registrada ao avaliar o resultado ou a vitória/derrota}
\end{EntryWithPhonetic}

\begin{EntryWithPhonetic}{分为}{fen1 wei2}{4,4}{⼑、⼂}[HSK 4]
  \definition{v.}{subdividir; dividir algo em}
\end{EntryWithPhonetic}

\begin{EntryWithPhonetic}{分析}{fen1xi1}{4,8}{⼑、⽊}[HSK 5]
  \definition{v.}{analisar; dividir uma coisa, um fenômeno, um conceito em componentes mais simples e descobrir as propriedades essenciais desses componentes e a relação entre eles (em oposição à 综合)}
  \seealsoref{综合}{zong1he2}
\end{EntryWithPhonetic}

\begin{EntryWithPhonetic}{分享}{fen1 xiang3}{4,8}{⼑、⼇}[HSK 5]
  \definition{v.}{compartilhar; partilhar}
\end{EntryWithPhonetic}

\begin{EntryWithPhonetic}{分之}{fen1 zhi1}{4,3}{⼑、⼂}[HSK 4]
  \definition{expr.}{indicando uma fração; formatação e leitura de frações, ou seja, partes de um total}[顾客减少了三分之一。===O número de clientes caiu em um terço.]
\end{EntryWithPhonetic}

\begin{EntryWithPhonetic}{分钟}{fen1zhong1}{4,9}{⼑、⾦}[HSK 2]
  \definition{clas.}{minuto (usado em intervalos de tempo); 60 segundos}
\end{EntryWithPhonetic}

\begin{EntryWithPhonetic}{分子}{fen1zi3}{4,3}{⼑、⼦}
  \definition{s.}{molécula | (matemática) numerador de uma fração}
  \seeref{fen4zi3}
\end{EntryWithPhonetic}

\begin{EntryWithPhonetic}{分组}{fen1 zu3}{4,8}{⼑、⽷}[HSK 3]
  \definition{v.}{dividir em grupos}
\end{EntryWithPhonetic}

\begin{EntryWithPhonetic}{纷}{fen1}{7}{⽷}
  \definition[场]{adj.}{confuso; emaranhado; desordenado | muitos e variados; profusos; numerosos}
\end{EntryWithPhonetic}

\begin{EntryWithPhonetic}{纷纷}{fen1fen1}{7,7}{⽷、⽷}[HSK 4]
  \definition{adj.}{numeroso e confuso; muitos e desordenados}
  \definition{adv.}{um após o outro; em sucessão; em rápida sucessão}
\end{EntryWithPhonetic}

\begin{EntryWithPhonetic}{焚}{fen2}{12}{⽕}
  \definition{v.}{queimar}
\end{EntryWithPhonetic}

\begin{EntryWithPhonetic}{焚香}{fen2xiang1}{12,9}{⽕、⾹}
  \definition{v.}{queimar incenso}
\end{EntryWithPhonetic}

\begin{EntryWithPhonetic}{粉}{fen3}{10}{⽶}
  \definition{adj.}{branco | rosa}
  \definition{s.}{pó | cosméticos em pó | farinha de trigo | macarrão ou outro alimento feito de feijão, arroz, batata, amido de batata-doce, etc. | macarrão de arroz}
  \definition{v.}{virar pó | caiar}
\end{EntryWithPhonetic}

\begin{EntryWithPhonetic}{粉色}{fen3 se4}{10,6}{⽶、⾊}
  \definition{s.}{cor-de-rosa}
\end{EntryWithPhonetic}

\begin{EntryWithPhonetic}{粉丝}{fen3si1}{10,5}{⽶、⼀}
  \definition{s.}{(empréstimo linguístico) fã | entusiasta de alguém ou alguma coisa}
  \definition[把]{s.}{aletria de amido de feijão | aletria chinesa | macarrão de celofane ou macarrão de vidro (transparente)}
\end{EntryWithPhonetic}

\begin{EntryWithPhonetic}{分}{fen4}{4}{⼑}[HSK 2]
  \definition{s.}{componente | o que está dentro dos deveres ou direitos de alguém; limites das responsabilidades e direitos | afeto; sentimento de amizade}
  \definition{v.}{pensar; esperar; estimar}
  \seeref{fen1}
\end{EntryWithPhonetic}

\begin{EntryWithPhonetic}{分量}{fen4liang4}{4,12}{⼑、⾥}
  \definition{s.}{tamanho da porção (comida)}
  \seeref{fen1liang4}
  \seeref{fen4liang5}
\end{EntryWithPhonetic}

\begin{EntryWithPhonetic}{分量}{fen4liang5}{4,12}{⼑、⾥}
  \definition{s.}{quantidade | peso | medida}
  \seeref{fen1liang4}
  \seeref{fen4liang4}
\end{EntryWithPhonetic}

\begin{EntryWithPhonetic}{分子}{fen4zi3}{4,3}{⼑、⼦}
  \definition{s.}{membros de uma classe ou grupo | elementos políticos (como intelectuais ou extremistas)}
  \seeref{fen1zi3}
\end{EntryWithPhonetic}

\begin{EntryWithPhonetic}{份}{fen4}{6}{⼈}
  \definition{clas.}{usado para emparelhar itens em grupos | usado para jornais, documentos, etc. | usado para partes de um todo | usado para aparência, estado, etc.}
  \definition{s.}{porção; parte | a unidade de divisão; usado após 省, 县, 年, 月,  indica a unidade de divisão | grau; extensão de algo}
  \seealsoref{年}{nian2}
  \seealsoref{省}{sheng3}
  \seealsoref{县}{xian4}
  \seealsoref{月}{yue4}
\end{EntryWithPhonetic}

\begin{EntryWithPhonetic}{奋}{fen4}{8}{⼤}
  \definition{adv.}{energicamente; com força e espírito}
  \definition{v.}{esforçar-se; agir vigorosamente; preparar-se | levantar | aplicar energia; resolver; animar-se | acenar; sacudir; levantar}
\end{EntryWithPhonetic}

\begin{EntryWithPhonetic}{奋斗}{fen4dou4}{8,4}{⼤、⽃}[HSK 4]
  \definition{v.}{lutar; esforçar-se; batalhar; trabalhar duro para atingir um determinado objetivo}
\end{EntryWithPhonetic}

\begin{EntryWithPhonetic}{奋战}{fen4zhan4}{8,9}{⼤、⼽}
  \definition{v.}{lutar bravamente | trabalhar duro}
\end{EntryWithPhonetic}

\begin{EntryWithPhonetic}{愤}{fen4}{12}{⼼}
  \definition{s.}{raiva; indignação; ressentimento; exasperação}
  \definition{v.}{ressentir-se; ficar indignado; ficar com raiva}
\end{EntryWithPhonetic}

\begin{EntryWithPhonetic}{愤怒}{fen4nu4}{12,9}{⼼、⼼}[HSK 6]
  \definition{adj.}{zangado; enraivecido; iracundo; furioso; emocionalmente agitado por extrema insatisfação}
\end{EntryWithPhonetic}

\begin{EntryWithPhonetic}{愤世嫉俗}{fen4shi4ji2su2}{12,5,13,9}{⼼、⼀、⼥、⼈}
  \definition{v.}{ser cínico | ser amargurado}
\end{EntryWithPhonetic}

\begin{EntryWithPhonetic}{丰}{feng1}{4}{⼁}
  \definition*{s.}{Sobrenome Feng}
  \definition[阵,丝]{adj.}{cheio; rico; abundante | ótimo | bonito; de boa aparência; cheio e redondo}
\end{EntryWithPhonetic}

\begin{EntryWithPhonetic}{丰富}{feng1fu4}{4,12}{⼁、⼧}[HSK 3]
  \definition{adj.}{rico; abundante; pleno; (riqueza material, conhecimento, experiência, etc.) variedade ou quantidade}
  \definition{v.}{enriquecer}
\end{EntryWithPhonetic}

\begin{EntryWithPhonetic}{丰收}{feng1shou1}{4,6}{⼁、⽁}[HSK 5]
  \definition{v.}{ter uma boa colheita; obter uma colheita boa e abundante; obter bons resultados}
\end{EntryWithPhonetic}

\begin{EntryWithPhonetic}{风}{feng1}{4}{⾵}[HSK 1][Kangxi 182]
  \definition*{s.}{Sobrenome Feng}
  \definition{adj.}{lendário; sem fundamento concreto | rápido; veloz | promíscuo; libertino; sedutor}
  \definition[阵,丝]{s.}{vento; fluxo de ar | prática; ambiente; costume | cena; vista | notícias; fofocas; rumores | comportamento; maneira; estilo | canção folclórica | certas doenças}
  \definition{v.}{colocar para secar ou arejar; secar ao vento}
\end{EntryWithPhonetic}

\begin{EntryWithPhonetic}{风暴}{feng1bao4}{4,15}{⾵、⽇}[HSK 6]
  \definition{s.}{tempestade; vendaval; um termo geral para perturbações violentas na atmosfera e mudanças drásticas no clima, como tempestades de areia, tornados, ciclones tropicais, etc. | tempestade; comoção violenta; uma metáfora para um evento tão poderoso que abala toda a sociedade}
\end{EntryWithPhonetic}

\begin{EntryWithPhonetic}{风度}{feng1du4}{4,9}{⾵、⼴}[HSK 5]
  \definition{s.}{postura; comportamento; porte; conduta; atitude}
\end{EntryWithPhonetic}

\begin{EntryWithPhonetic}{风格}{feng1ge2}{4,10}{⾵、⽊}[HSK 4]
  \definition{s.}{modo; estilo; maneira; caráter | características das criações literárias de diferentes épocas, povos, escolas ou indivíduos em termos de conteúdo ideológico e técnicas artísticas}
\end{EntryWithPhonetic}

\begin{EntryWithPhonetic}{风光}{feng1guang1}{4,6}{⾵、⼉}[HSK 5]
  \definition{s.}{cena; vista; paisagens naturais e humanas}
\end{EntryWithPhonetic}

\begin{EntryWithPhonetic}{风景}{feng1jing3}{4,12}{⾵、⽇}[HSK 4]
  \definition[种,处,道]{s.}{cenário; paisagem; cenários e vistas que podem ser apreciados, inclui paisagens, flores, árvores, edifícios e determinados fenômenos naturais}
\end{EntryWithPhonetic}

\begin{EntryWithPhonetic}{风扇}{feng1shan4}{4,10}{⾵、⼾}
  \definition{s.}{ventilador elétrico}
\end{EntryWithPhonetic}

\begin{EntryWithPhonetic}{风俗}{feng1su2}{4,9}{⾵、⼈}[HSK 4]
  \definition[种,个,些]{s.}{costumes; a soma de costumes sociais, maneiras, hábitos, etc., desenvolvidos ao longo do tempo}
\end{EntryWithPhonetic}

\begin{EntryWithPhonetic}{风险}{feng1xian3}{4,9}{⾵、⾩}[HSK 3]
  \definition[个,种]{s.}{risco; perigo; ameaça; riscos possíveis}
\end{EntryWithPhonetic}

\begin{EntryWithPhonetic}{风筝}{feng1zheng5}{4,12}{⾵、⽵}
  \definition{s.}{pipa | papagaio | pandorga}
\end{EntryWithPhonetic}

\begin{EntryWithPhonetic}{枫}{feng1}{8}{⽊}
  \definition[棵]{s.}{goma doce chinesa | bordo; \emph{maple}}
\end{EntryWithPhonetic}

\begin{EntryWithPhonetic}{枫叶}{feng1ye4}{8,5}{⽊、⼝}
  \definition{s.}{folha de bordo (maple, tipo de árvore)}
\end{EntryWithPhonetic}

\begin{EntryWithPhonetic}{封}{feng1}{9}{⼨}[HSK 2,5]
  \definition*{s.}{Sobrenome Feng}
  \definition{clas.}{usado para objetos selados, especialmente cartas}
  \definition{s.}{feudalismo | embalagem; envelope | pacote}
  \definition{v.}{conferir (um título, território, etc.) a | selar | acender uma fogueira | fechar}
\end{EntryWithPhonetic}

\begin{EntryWithPhonetic}{封闭}{feng1bi4}{9,6}{⼨、⾨}[HSK 4]
  \definition{adj.}{fechado; aqueles que não têm contato com o mundo exterior; aqueles que são muito conservadores (em seu pensamento) e não se comunicam com os outros}
  \definition{v.}{selar; fechar; lacrar; vedar; de modo a impedir a passagem, o uso ou a abertura}
\end{EntryWithPhonetic}

\begin{EntryWithPhonetic}{封底}{feng1di3}{9,8}{⼨、⼴}
  \definition{s.}{contracapa de um livro}
\end{EntryWithPhonetic}

\begin{EntryWithPhonetic}{封冻}{feng1dong4}{9,7}{⼨、⼎}
  \definition{v.}{congelar (água ou terra)}
\end{EntryWithPhonetic}

\begin{EntryWithPhonetic}{封盖}{feng1gai4}{9,11}{⼨、⽫}
  \definition{s.}{boné | capa | selo}
  \definition{v.}{cobrir}
\end{EntryWithPhonetic}

\begin{EntryWithPhonetic}{封建}{feng1jian4}{9,8}{⼨、⼵}
  \definition{adj.}{feudal}
  \definition{s.}{feudalismo}
\end{EntryWithPhonetic}

\begin{EntryWithPhonetic}{封口}{feng1kou3}{9,3}{⼨、⼝}
  \definition{v.}{selar | fechar | curar (uma ferida) | manter os lábios selados}
\end{EntryWithPhonetic}

\begin{EntryWithPhonetic}{封面}{feng1mian4}{9,9}{⼨、⾯}
  \definition{s.}{capa (de uma publicação) | sobrecapa}
\end{EntryWithPhonetic}

\begin{EntryWithPhonetic}{封印}{feng1yin4}{9,5}{⼨、⼙}
  \definition{s.}{selo (em envelopes)}
\end{EntryWithPhonetic}

\begin{EntryWithPhonetic}{封斋}{feng1zhai1}{9,10}{⼨、⽂}
  \definition*{s.}{Ramadã (Islã)}
\end{EntryWithPhonetic}

\begin{EntryWithPhonetic}{疯}{feng1}{9}{⽧}[HSK 5]
  \definition{adj.}{louco; insano | tolo; leviano | (de uma planta, safra de grãos, etc.) esguia; refere-se ao crescimento vigoroso das plantações, mas sem frutos | com todas as forças; fazer o máximo possível}
  \definition{v.}{jogar sem restrições}
\end{EntryWithPhonetic}

\begin{EntryWithPhonetic}{疯狂}{feng1kuang2}{9,7}{⽧、⽝}[HSK 5]
  \definition{adj.}{louco; insano; frenético; desenfreado}
\end{EntryWithPhonetic}

\begin{EntryWithPhonetic}{峰}{feng1}{10}{⼭}
  \definition{clas.}{usado para camelos}
  \definition{s.}{pico; cume; o pico proeminente de uma montanha | coisa parecida com um pico; coisas em forma de montanhas}
\end{EntryWithPhonetic}

\begin{EntryWithPhonetic}{峰会}{feng1 hui4}{10,6}{⼭、⼈}[HSK 6]
  \definition{s.}{cúpula; reunião de cúpula}
\end{EntryWithPhonetic}

\begin{EntryWithPhonetic}{缝}{feng2}{13}{⽷}
  \definition{v.}{costurar}
  \seeref{feng4}
\end{EntryWithPhonetic}

\begin{EntryWithPhonetic}{缝纫}{feng2ren4}{13,6}{⽷、⽷}
  \definition{v.}{costurar}
\end{EntryWithPhonetic}

\begin{EntryWithPhonetic}{缝纫机}{feng2ren4ji1}{13,6,6}{⽷、⽷、⽊}
  \definition[架]{s.}{máquina de costura}
\end{EntryWithPhonetic}

\begin{EntryWithPhonetic}{凤}{feng4}{4}{⼏}
  \definition[只]{s.}{fênix}
  \definition{s.}{Sobrenome Feng}
\end{EntryWithPhonetic}

\begin{EntryWithPhonetic}{凤凰}{feng4huang2}{4,11}{⼏、⼏}
  \definition{s.}{fênix}
\end{EntryWithPhonetic}

\begin{EntryWithPhonetic}{奉}{feng4}{8}{⼤}
  \definition*{s.}{Sobrenome Feng}
  \definition{v.}{Literário: dedicar ou presentear com respeito | receber (pedidos, instruções, etc.) | Literário: estimar; reverenciar | Litrário: acreditar em  | esperar; atender; servir}
\end{EntryWithPhonetic}

\begin{EntryWithPhonetic}{奉献}{feng4xian4}{8,13}{⼤、⽝}[HSK 6]
  \definition{v.}{dedicar; oferecer como tributo; apresentar com todo respeito; entregar respeitosamente}
\end{EntryWithPhonetic}

\begin{EntryWithPhonetic}{缝}{feng4}{13}{⽷}
  \definition[道]{s.}{costura | fenda; rachadura; fissura; brecha}
  \seeref{feng2}
\end{EntryWithPhonetic}

\begin{EntryWithPhonetic}{佛}{fo2}{7}{⼈}[HSK 6]
  \definition*{s.}{Buda, abreviação de 佛陀 | Budismo}
  \definition{s.}{imagem de Buda | budista | nome de Buda; escritura budista | uma pessoa que alcançou a perfeição na prática espiritual; budista real | estátua do Buda}
  \seeref{fu2}
  \seealsoref{佛陀}{fo2tuo2}
\end{EntryWithPhonetic}

\begin{EntryWithPhonetic}{佛教}{fo2 jiao4}{7,11}{⼈、⽁}[HSK 6]
  \definition*{s.}{Budismo; uma das principais religiões do mundo, diz-se que foi fundada por Sakyamuni, um príncipe do antigo reino indiano de Kapilavastu (no atual Nepal), no século VI ou V a.C.; foi amplamente difundida em muitos países asiáticos e introduzida na China no final da Dinastia Han Ocidental}
\end{EntryWithPhonetic}

\begin{EntryWithPhonetic}{佛陀}{fo2tuo2}{7,7}{⼈、⾩}
  \definition{s.}{Buda, um título para Sakyamuni ou uma pessoa que atingiu a iluminação | Buda, uma pessoa que atingiu a Budeidade, ou especificamente Siddhartha Gautama}
\end{EntryWithPhonetic}

\begin{EntryWithPhonetic}{否}{fou3}{7}{⼝}
  \definition{adv.}{não; expressa discordância, equivalente à palavra falada 不 | usado no final de uma pergunta para indicar investigação | 是否, 能否 e 可否 que significa respectivamente 是不是, 能不能 e 可不可}
  \definition{v.}{negar}
  \seeref{pi3}
  \seealsoref{不}{bu4}
  \seealsoref{可}{ke3}
  \seealsoref{能}{neng2}
  \seealsoref{是}{shi4}
\end{EntryWithPhonetic}

\begin{EntryWithPhonetic}{否定}{fou3ding4}{7,8}{⼝、⼧}[HSK 3]
  \definition{adj.}{negativo; contrário}
  \definition{v.}{rejeitar; negar a existência ou a autenticidade de algo}
\end{EntryWithPhonetic}

\begin{EntryWithPhonetic}{否认}{fou3ren4}{7,4}{⼝、⾔}[HSK 3]
  \definition{v.}{negar; repudiar; não reconhecer}
\end{EntryWithPhonetic}

\begin{EntryWithPhonetic}{否则}{fou3ze2}{7,6}{⼝、⼑}[HSK 4]
  \definition{conj.}{senão; se não; ou então; se não for isso}
\end{EntryWithPhonetic}

\begin{EntryWithPhonetic}{夫}{fu1}{4}{⼤}
  \definition{s.}{marido | homem | (velho) alguém que faz algum tipo de trabalho manual | (velho) uma pessoa que serviu em trabalho forçado}
  \seeref{fu2}
\end{EntryWithPhonetic}

\begin{EntryWithPhonetic}{夫妇}{fu1fu4}{4,6}{⼤、⼥}[HSK 4]
  \definition[对]{s.}{casal; marido e mulher}
\end{EntryWithPhonetic}

\begin{EntryWithPhonetic}{夫妻}{fu1qi1}{4,8}{⼤、⼥}[HSK 4]
  \definition[对]{s.}{casal; marido e mulher}
\end{EntryWithPhonetic}

\begin{EntryWithPhonetic}{夫人}{fu1ren2}{4,2}{⼤、⼈}[HSK 4]
  \definition[位,名,个]{s.}{senhora; \emph{lady}; madame; na antiguidade, as esposas dos senhores feudais eram chamadas de ``madame'' e, nas dinastias Ming e Qing, as esposas dos oficiais de primeiro e segundo escalão eram chamadas de ``madame'', que mais tarde foi usada para homenagear as esposas das pessoas em geral e agora é usada principalmente em ocasiões diplomáticas}
\end{EntryWithPhonetic}

\begin{EntryWithPhonetic}{夫}{fu2}{4}{⼤}
  \definition{part.}{usado no início de uma frase | usado no final de uma frase ou em uma pausa no meio de uma frase para expressar uma exclamação}
  \definition{pron.}{isto; isso; aqueles; estes | ele}
  \seeref{fu1}
\end{EntryWithPhonetic}

\begin{EntryWithPhonetic}{佛}{fu2}{7}{⼈}
  \definition{adv.}{aparentemente}
  \definition{s.}{ornamento da cabeça (feminino)}
  \seeref{fo2}
\end{EntryWithPhonetic}

\begin{EntryWithPhonetic}{扶}{fu2}{7}{⼿}[HSK 5]
  \definition*{s.}{Sobrenome Fu}
  \definition{v.}{segurar; apoiar com a mão; segurar algo com o apoio das mãos para que ninguém, objeto ou pessoa caia | dar apoio a; ajudar uma pessoa deitada ou caída a se levantar com as mãos; endireitar um objeto caído com as mãos | ajudar; tirar de baixo}
\end{EntryWithPhonetic}

\begin{EntryWithPhonetic}{扶梯}{fu2ti1}{7,11}{⼿、⽊}
  \definition{s.}{escada rolante}
\end{EntryWithPhonetic}

\begin{EntryWithPhonetic}{服}{fu2}{8}{⽉}[HSK 6]
  \definition*{s.}{Sobrenome Fu}
  \definition{s.}{roupas | vestuário de luto; refere-se a roupas de luto}
  \definition{v.}{vestir (roupas) | tomar (remédio) | envolver-se em; servir | obedecer; ser convencido | convencer; persuadir | adaptar-se; acostumar-se a}
  \seeref{fu4}
\end{EntryWithPhonetic}

\begin{EntryWithPhonetic}{服从}{fu2cong2}{8,4}{⽉、⼈}[HSK 5]
  \definition{v.}{obedecer; submeter-se a; estar subordinado a}
\end{EntryWithPhonetic}

\begin{EntryWithPhonetic}{服务}{fu2 wu4}{8,5}{⽉、⼒}[HSK 2]
  \definition{v.}{prestar serviço a; estar a serviço de; servir; trabalhar para o benefício coletivo (ou de outras pessoas) ou para uma causa específica | trabalhar; servir}
\end{EntryWithPhonetic}

\begin{EntryWithPhonetic}{服务员}{fu2wu4yuan2}{8,5,7}{⽉、⼒、⼝}
  \definition{s.}{atendente | garçom | garçonete | pessoal de atendimento ao cliente}
\end{EntryWithPhonetic}

\begin{EntryWithPhonetic}{服装}{fu2zhuang1}{8,12}{⽉、⾐}[HSK 3]
  \definition[套,件,身]{s.}{roupas; vestuário; trajes; termo genérico para roupas, sapatos e chapéus, geralmente referido especificamente a roupas}
\end{EntryWithPhonetic}

\begin{EntryWithPhonetic}{浮}{fu2}{10}{⽔}[HSK 6]
  \definition*{s.}{Sobrenome Fu}
  \definition{adj.}{superficial; na superfície | móvel; removível | temporário; provisório | superficial e frívolo; volátil; impetuoso | oco; vazio; inflado | excessivo; excedente}
  \definition{v.}{flutuar (oposto a 沉) | (dialeto) nadar | flutuar; derivar; flutuar na superfície do líquido}
  \seealsoref{沉}{chen2}
\end{EntryWithPhonetic}

\begin{EntryWithPhonetic}{浮力}{fu2li4}{10,2}{⽔、⼒}
  \definition{s.}{flutuabilidade}
\end{EntryWithPhonetic}

\begin{EntryWithPhonetic}{浮图}{fu2tu2}{10,8}{⽔、⼞}
  \definition*{s.}{Termo alternativo para 佛陀}
  \variantof{浮屠}
  \seealsoref{佛陀}{fo2tuo2}
\end{EntryWithPhonetic}

\begin{EntryWithPhonetic}{浮屠}{fu2tu2}{10,11}{⽔、⼫}
  \definition*{s.}{Buda | Templo (Stupa) Budista (transliteração de Pali Thuo)}
\end{EntryWithPhonetic}

\begin{EntryWithPhonetic}{符}{fu2}{11}{⽵}
  \definition*{s.}{Sobrenome Fu}
  \definition[个]{s.}{registro emitido por um governante para generais, enviados, etc., como credenciais na China antiga | símbolo; emblema | figuras mágicas desenhadas por sacerdotes taoístas para invocar ou expulsar espíritos e trazer boa ou má sorte | marca; sinal}
  \definition{v.}{(usado com 相 xiāng ou 不) coincidir com; concordar com | encaixar bem; combinar com; em conformidade com}
  \seealsoref{不}{bu4}
  \seealsoref{相}{xiang1}
\end{EntryWithPhonetic}

\begin{EntryWithPhonetic}{符号}{fu2hao4}{11,5}{⽵、⼝}[HSK 4]
  \definition[个]{s.}{marca; símbolo; sinais que marcam as coisas | insígnia; emblema; um símbolo usado no corpo para indicar posição, \emph{status}, etc.}
\end{EntryWithPhonetic}

\begin{EntryWithPhonetic}{符合}{fu2he2}{11,6}{⽵、⼝}[HSK 4]
  \definition{v.}{conformar-se com, estar de acordo com, estar em conformidade com}
\end{EntryWithPhonetic}

\begin{EntryWithPhonetic}{幅}{fu2}{12}{⼱}[HSK 5]
  \definition{clas.}{usado para tecidos, telas de lã, pinturas, etc.}
  \definition{s.}{largura do tecido, seda, tweed, etc. | tamanho; largura; geralmente se refere à largura}
\end{EntryWithPhonetic}

\begin{EntryWithPhonetic}{幅度}{fu2du4}{12,9}{⼱、⼴}[HSK 5]
  \definition{s.}{alcance; escopo; extensão; largura; largura da propagação de um objeto que vibra ou balança, uma metáfora para a magnitude de uma mudança em algo}
\end{EntryWithPhonetic}

\begin{EntryWithPhonetic}{福}{fu2}{13}{⽰}[HSK 3]
  \definition*{s.}{Província de Fujian | Sobrenome Fu}
  \definition{s.}{benção; felicidade; boa sorte; boa fortuna; sorte (oposto de 祸)}
  \definition{v.}{(de uma mulher) fazer uma reverência; antigamente, as mulheres faziam a reverência 万福 (colocando as duas mãos na cintura do mesmo lado e dobrando ligeiramente os joelhos)}
  \seealsoref{祸}{huo4}
  \seealsoref{万福}{wan4fu2}
\end{EntryWithPhonetic}

\begin{EntryWithPhonetic}{福克斯}{fu2ke4si1}{13,7,12}{⽰、⼗、⽄}
  \definition*{s.}{Fox (empresa de mídia) | Focus (automóvel fabricado pela Ford)}
\end{EntryWithPhonetic}

\begin{EntryWithPhonetic}{福利}{fu2li4}{13,7}{⽰、⼑}[HSK 5]
  \definition[项,种]{s.}{bem-estar; benefícios materiais}
  \definition{v.}{melhorar suas condições de vida; facilitar a vida}
\end{EntryWithPhonetic}

\begin{EntryWithPhonetic}{福泽}{fu2ze2}{13,8}{⽰、⽔}
  \definition{s.}{boa sorte}
\end{EntryWithPhonetic}

\begin{EntryWithPhonetic}{俯}{fu3}{10}{⼈}
  \definition{v.}{curvar (a cabeça) (oposto a 仰) | inclinar-se | (datado, em documentos ou cartas oficiais) condescender com | curvar-se; fazer uma reverência}
  \seealsoref{仰}{yang3}
\end{EntryWithPhonetic}

\begin{EntryWithPhonetic}{辅}{fu3}{11}{⾞}
  \definition*{s.}{Sobrenome Fu}
  \definition{adj.}{subsidiário}
  \definition{s.}{barras laterais do carrinho atuando como proteção da roda; duas barras retas de madeira são adicionadas na parte externa da roda para prender o cubo | maçã do rosto | assistente oficial; títulos oficiais antigos | (literário) território que circunda a capital}
  \definition{v.}{auxiliar; complementar; suplementar | ajudar}
\end{EntryWithPhonetic}

\begin{EntryWithPhonetic}{辅助}{fu3zhu4}{11,7}{⾞、⼒}[HSK 5]
  \definition{adj.}{auxiliar; suplementar; complementar}
  \definition{v.}{auxiliar; ajudar; colocar os outros em primeiro lugar e dar-lhes alguma ajuda externa}
\end{EntryWithPhonetic}

\begin{EntryWithPhonetic}{父}{fu4}{4}{⽗}[Kangxi 88]
  \definition{s.}{pai | um homem mais velho na família ou parentes | fundador; uma pessoa que inventa ou inicia algo}
\end{EntryWithPhonetic}

\begin{EntryWithPhonetic}{父母}{fu4 mu3}{4,5}{⽗、⽏}[HSK 3]
  \definition{s.}{pai e mãe; pais}
\end{EntryWithPhonetic}

\begin{EntryWithPhonetic}{父母亲}{fu4mu3qin1}{4,5,9}{⽗、⽏、⼇}
  \definition{s.}{os pais; pai e mãe}
\end{EntryWithPhonetic}

\begin{EntryWithPhonetic}{父女}{fu4 nv3}{4,3}{⽗、⼥}[HSK 6]
  \definition{s.}{pai e filha}
\end{EntryWithPhonetic}

\begin{EntryWithPhonetic}{父亲}{fu4qin1}{4,9}{⽗、⼇}[HSK 3]
  \definition[个,位,名]{s.}{pai; homem com filhos; pai dos filhos}
\end{EntryWithPhonetic}

\begin{EntryWithPhonetic}{父子}{fu4 zi3}{4,3}{⽗、⼦}[HSK 6]
  \definition{s.}{pai e filho}
\end{EntryWithPhonetic}

\begin{EntryWithPhonetic}{付}{fu4}{5}{⼈}[HSK 3]
  \definition*{s.}{Sobrenome Fu}
  \definition{clas.}{usado para pares ou conjuntos de coisas | usado para expressões faciais}
  \definition{v.}{comprometer-se com; entregar (ou transferir) para | pagar; refere-se especificamente a dar dinheiro}
\end{EntryWithPhonetic}

\begin{EntryWithPhonetic}{付出}{fu4 chu1}{5,5}{⼈、⼐}[HSK 4]
  \definition{v.}{pagar; gastar; entregar (dinheiro, consideração, etc.)}
\end{EntryWithPhonetic}

\begin{EntryWithPhonetic}{付款}{fu4/kuan3}{5,12}{⼈、⽋}
  \definition{s.}{pagamento}
  \definition{v.+compl.}{pagar uma quantia em dinheiro}
\end{EntryWithPhonetic}

\begin{EntryWithPhonetic}{妇}{fu4}{6}{⼥}
  \definition{s.}{mulher | mulher casada | esposa}
\end{EntryWithPhonetic}

\begin{EntryWithPhonetic}{妇女}{fu4nv3}{6,3}{⼥、⼥}[HSK 6]
  \definition[个,位,群,名,帮]{s.}{mulher; mulheres; um termo geral para mulheres adultas}
\end{EntryWithPhonetic}

\begin{EntryWithPhonetic}{负}{fu4}{6}{⾙}[HSK 6]
  \definition{adj.}{negativo; menor que zero | negativo; referindo-se ao que recebe elétrons (oposto a 正)}
  \definition{v.}{carregar; transportar nas costas ou nos ombros | suportar; assumir; encarar | confiar em; contar com; depender | sofrer | aproveitar; desfrutar | ter dívidas | trair; violar | perder; ser derrotado}
  \seealsoref{正}{zheng4}
\end{EntryWithPhonetic}

\begin{EntryWithPhonetic}{负担}{fu4dan1}{6,8}{⾙、⼿}[HSK 4]
  \definition{s.}{carga; fardo; frete; ônus; pressão ou responsabilidade, despesas, etc.}
  \definition{v.}{carregar; carregar (um fardo); assumir (responsabilidade, trabalho, despesas, etc.)}
\end{EntryWithPhonetic}

\begin{EntryWithPhonetic}{负面}{fu4mian4}{6,9}{⾙、⾯}
  \definition{adj.}{ruim; negativo; prejudicial; desvantajoso}
  \definition{s.}{lado reverso; o negativo; refere-se aos aspectos ou partes ruins, negativas, prejudiciais ou desfavoráveis}
\end{EntryWithPhonetic}

\begin{EntryWithPhonetic}{负责}{fu4ze2}{6,8}{⾙、⾙}[HSK 3]
  \definition{adj.}{consciencioso; ser sério e responsável}
  \definition{v.}{ser responsável por; estar encarregado de; assumir responsabilidades}
\end{EntryWithPhonetic}

\begin{EntryWithPhonetic}{负责人}{fu4 ze2 ren2}{6,8,2}{⾙、⾙、⼈}[HSK 5]
  \definition[位]{s.}{pessoa responsável; pessoa encarregada; pessoas com responsabilidades de liderança}
\end{EntryWithPhonetic}

\begin{EntryWithPhonetic}{附}{fu4}{7}{⾩}
  \definition*{s.}{Sobrenome Fu}
  \definition{v.}{adicionar; anexar; incluir | chegar perto de; estar perto de | depender de; confiar em; cumprir com | concordar com; anexar a; aderir a; cumprir com; depender de}
\end{EntryWithPhonetic}

\begin{EntryWithPhonetic}{附件}{fu4jian4}{7,6}{⾩、⼈}[HSK 5]
  \definition*{s.}{\emph{Adnexa Uteri}, refere-se à genitália interna feminina que não seja o útero, as trompas de falópio e os ovários}
  \definition{s.}{apêndice; documentos que acompanham o documento principal | acessório; anexo; peças ou sobressalentes que não sejam peças principais de máquinas e equipamentos | anexo; documentos ou itens relevantes emitidos com o documento}
\end{EntryWithPhonetic}

\begin{EntryWithPhonetic}{附近}{fu4jin4}{7,7}{⾩、⾡}[HSK 4]
  \definition{adj.}{perto; vizinho}
  \definition{s.}{vizinhança; bairro}
\end{EntryWithPhonetic}

\begin{EntryWithPhonetic}{服}{fu4}{8}{⽉}
  \definition{clas.}{usado para remédio: dose; usado na medicina tradicional chinesa}
  \seeref{fu2}
\end{EntryWithPhonetic}

\begin{EntryWithPhonetic}{复}{fu4}{9}{⼢}
  \definition*{s.}{Sobrenome Fu}
  \definition{adj.}{composto; complexo; nem um único; dois ou mais}
  \definition{adv.}{de novo; novamente; indica o reaparecimento de uma situação, equivalente a 再}
  \definition{s.}{jaqueta; roupas forradas}
  \definition{v.}{virar; virar-se | responder; retornar | recuperar; retornar a; restaurar | vingar | duplicar; repetir}
  \seealsoref{再}{zai4}
\end{EntryWithPhonetic}

\begin{EntryWithPhonetic}{复活节}{fu4huo2jie2}{9,9,5}{⼢、⽔、⾋}
  \definition*{s.}{Páscoa}
\end{EntryWithPhonetic}

\begin{EntryWithPhonetic}{复刻}{fu4ke4}{9,8}{⼢、⼑}
  \definition{v.}{reimprimir (um trabalho que esteve fora do catálogo) | reeditar (um disco de vinil, um CD, etc.) | replicar | recriar | (empréstimo linguístico) (computação) \emph{fork}}
\end{EntryWithPhonetic}

\begin{EntryWithPhonetic}{复苏}{fu4 su1}{9,7}{⼢、⾋}[HSK 6]
  \definition{s.}{recuperação}
  \definition{v.}{reviver; recuperar; ressuscitar; voltar à vida}
\end{EntryWithPhonetic}

\begin{EntryWithPhonetic}{复习}{fu4xi2}{9,3}{⼢、⼄}[HSK 2]
  \definition{s.}{revisão}
  \definition{v.}{revisar; corrigir (lições, etc.); repetir o que já aprendeu para consolidar o conhecimento}
\end{EntryWithPhonetic}

\begin{EntryWithPhonetic}{复印}{fu4yin4}{9,5}{⼢、⼙}[HSK 3]
  \definition{v.}{fotografar; fotocopiar; duplicar; sem passar pelo processo de impressão, obter uma cópia diretamente do original (geralmente referindo-se à cópia feita com uma copiadora)}
\end{EntryWithPhonetic}

\begin{EntryWithPhonetic}{复杂}{fu4za2}{9,6}{⼢、⽊}[HSK 3]
  \definition{adj.}{complexo; complicado; em oposição a 单纯 e 简单}
  \seealsoref{单纯}{dan1chun2}
  \seealsoref{简单}{jian3dan1}
\end{EntryWithPhonetic}

\begin{EntryWithPhonetic}{复制}{fu4zhi4}{9,8}{⼢、⼑}[HSK 4]
  \definition{v.}{copiar; duplicar; reproduzir; fazer uma cópia de; fazer uma cópia do original ou reproduzi-lo, reimprimi-lo ou copiá-lo em sua forma original (geralmente referindo-se a relíquias culturais ou obras de arte)}
\end{EntryWithPhonetic}

\begin{EntryWithPhonetic}{副}{fu4}{11}{⼑}[HSK 6]
  \definition{adj.}{segundo em exercício; deputado; auxiliar | subsidiário; incidental; secundário}
  \definition{clas.}{usado para conjuntos completos de itens; usado para \emph{kits} | usado para expressões faciais | usado para som ou voz}
  \definition{pref.}{vice-}
  \definition{s.}{assistente; ajudante; auxiliar; posição auxiliar; pessoa que ocupa uma posição auxiliar}
  \definition{v.}{ajustar; corresponder a; conformar-se a}
\end{EntryWithPhonetic}

\begin{EntryWithPhonetic}{副研}{fu4yan2}{11,9}{⼑、⽯}
  \definition{s.}{pesquisador adjunto}
\end{EntryWithPhonetic}

\begin{EntryWithPhonetic}{富}{fu4}{12}{⼧}[HSK 3]
  \definition*{s.}{Sobrenome Fu}
  \definition{adj.}{rico; abastado; abundante; refere-se a ter muito dinheiro (oposto de 贫) | rico; abundante}
  \definition{v.}{tornar-se rico; enriquecer}
  \seealsoref{贫}{pin2}
\end{EntryWithPhonetic}

\begin{EntryWithPhonetic}{富人}{fu4 ren2}{12,2}{⼧、⼈}[HSK 6]
  \definition{s.}{os ricos; os abastados}
\end{EntryWithPhonetic}

\begin{EntryWithPhonetic}{富有}{fu4 you3}{12,6}{⼧、⽉}[HSK 6]
  \definition{adj.}{rico; abastado; possuir uma grande quantidade de propriedades | rico em espírito; metáfora para uma vida espiritual rica}
  \definition{v.}{ser rico ou abundante em; principalmente referindo-se a coisas abstratas com significados positivos que são suficientes}
\end{EntryWithPhonetic}

\begin{EntryWithPhonetic}{覆}{fu4}{18}{⾑}
  \definition{v.}{cobrir; encapar | derrubar; perturbar; virar de cabeça para baixo}
\end{EntryWithPhonetic}

\begin{EntryWithPhonetic}{覆盆子}{fu4pen2zi5}{18,9,3}{⾑、⽫、⼦}
  \definition{s.}{framboesa}
\end{EntryWithPhonetic}

%%%%% EOF %%%%%

