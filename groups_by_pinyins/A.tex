%%%
%%% A
%%%

\section*{A}\addcontentsline{toc}{section}{A}

\begin{EntryWithPhonetic}{阿}{a1}{7}{⾩}
  \definition{pref.}{em dialetos do sul para formar termos carinhosos, antes de nomes de animais de estimação, sobrenomes monossilábicos ou números que denotam ordem de antiguidade em uma; anexado a 大, 二, 三,\dots\ para indicar classificação (e, às vezes, intimidade) | antes dos termos de parentesco; na frente de um sobrenome, de um nome próprio ou de um determinado título, com uma conotação de intimidade | em alguns contextos, pode soar infantil ou muito informal (por exemplo, chamar um colega de trabalho por ``阿 + Nome'' sem intimidade)}[阿妈===mamãe | 阿明 ===forma carinhosa de chamar uma pessoa chamada Ming]
  \seeref{e1}
\end{EntryWithPhonetic}

\begin{EntryWithPhonetic}{阿哥}{a1ge1}{7,10}{⾩、⼝}
  \definition{s.}{irmão mais velho (afetivo) | forma afetuosa de tratamento entre homens da mesma idade}[阿哥,帮我拿一下书包!===Irmão, ajude-me com minha mochila escolar!]
\end{EntryWithPhonetic}

\begin{EntryWithPhonetic}{阿拉伯语}{a1la1bo2 yu3}{7,8,7,9}{⾩、⼿、⼈、⾔}[HSK 7-9]
  \definition{s.}{árabe; idioma árabe}
\end{EntryWithPhonetic}

\begin{EntryWithPhonetic}{阿姨}{a1yi2}{7,9}{⾩、⼥}[HSK 4]
  \definition[个,位,名,些]{s.}{tia; uma forma de tratamento para uma mulher da geração dos pais; dirigir-se a uma mulher que tem aproximadamente a mesma idade da sua mãe, geralmente não é parente | babá em uma família; professora em um jardim de infância | tia; irmã da mãe (mais comum no sul da China)}[阿姨,生日快乐!===Tia, feliz aniversário! | 阿姨,这个苹果多少钱一斤?===Tia/Senhora, quanto custa o quilo dessas maçãs? | 阿姨,我想喝水。===Tia/Babá, eu quero beber água.]
\end{EntryWithPhonetic}

\begin{EntryWithPhonetic}{呵}{a1}{8}{⼝}
  \variantof{啊}
  \seeref{he1}
\end{EntryWithPhonetic}

\begin{EntryWithPhonetic}{啊}{a1}{10}{⼝}[HSK 2]
  \definition{interj.}{Ah!; Oh!; expressar surpresa ou admiração}
  \seeref{a2}
  \seeref{a3}
  \seeref{a4}
  \seeref{a5}
\end{EntryWithPhonetic}

\begin{EntryWithPhonetic}{啊呀}{a1ya1}{10,7}{⼝、⼝}
  \definition{interj.}{Oh meu Deus! | interjeição de surpresa}
\end{EntryWithPhonetic}

\begin{EntryWithPhonetic}{啊哟}{a1yo5}{10,9}{⼝、⼝}
  \definition{interj.}{Meu Deus! | Oh! | Ai! | interjeição de surpresa ou dor}
\end{EntryWithPhonetic}

\begin{EntryWithPhonetic}{啊}{a2}{10}{⼝}[HSK 2]
  \definition{interj.}{Eh?; Ei?; Que?; Por que?; expressar questionamento, dúvida ou solicitar opinião}
  \seeref{a1}
  \seeref{a3}
  \seeref{a4}
  \seeref{a5}
\end{EntryWithPhonetic}

\begin{EntryWithPhonetic}{嗄}{a2}{13}{⼝}
  \variantof{啊}
  \seeref{sha4}
\end{EntryWithPhonetic}

\begin{EntryWithPhonetic}{啊}{a3}{10}{⼝}[HSK 2]
  \definition{interj.}{Eh?; Meu!; E aí?; Que?; expressar surpresa e dúvida}
  \seeref{a1}
  \seeref{a2}
  \seeref{a4}
  \seeref{a5}
\end{EntryWithPhonetic}

\begin{EntryWithPhonetic}{啊}{a4}{10}{⼝}[HSK 2]
  \definition{interj.}{Bem!; Sim!; expressa concordância, pronúncia mais curta | Oh!; Ah!; indica que compreendeu, com pronúncia mais longa | Oh!; expressa surpresa ou admiração, com pronúncia mais longa | Desgraça!; expressa tristeza ou pesar}
  \seeref{a1}
  \seeref{a2}
  \seeref{a3}
  \seeref{a5}
\end{EntryWithPhonetic}

\begin{EntryWithPhonetic}{啊}{a5}{10}{⼝}[HSK 2,4]
  \definition{part.}{usado no final da frase para expressar admiração | usado no final da frase para expressar afirmação, justificativa, insistência, recomendação, etc. | usado no final da frase para indicar dúvida | usado para fazer uma pequena pausa na frase, chamando a atenção para o que vem a seguir | usado após os itens enumerados | usado após verbos repetitivos, indica um processo longo}
  \seeref{a1}
  \seeref{a2}
  \seeref{a3}
  \seeref{a4}
\end{EntryWithPhonetic}

\begin{EntryWithPhonetic}{哎}{ai1}{8}{⼝}[HSK 7-9]
  \definition{interj.}{Por que?; Ei!; Ai!; expressar surpresa ou insatisfação | Ei!; Cuidado!}
\end{EntryWithPhonetic}

\begin{EntryWithPhonetic}{哎呀}{ai1ya1}{8,7}{⼝、⼝}[HSK 7-9]
  \definition{interj.}{expressar surpresa ou espanto | expressar reclamação, impaciência, etc.}
\end{EntryWithPhonetic}

\begin{EntryWithPhonetic}{哀}{ai1}{9}{⼝}
  \definition*{s.}{Sobrenome Ai}
  \definition{adj.}{triste; pesaroso}
  \definition{adv.}{tristemente; lamentavelmente}
  \definition{s.}{luto | tristeza; pesar | pena; misericórdia}
  \definition{v.}{lamentar; lamentar-se por | Literário: estar triste}
\end{EntryWithPhonetic}

\begin{EntryWithPhonetic}{哀求}{ai1qiu2}{9,7}{⼝、⽔}[HSK 7-9]
  \definition{v.}{suplicar; implorar | suplicar; implorar piedosamente}
\end{EntryWithPhonetic}

\begin{EntryWithPhonetic}{唉}{ai1}{10}{⼝}
  \definition{interj.}{Sim!; Certo!; Bem! | Ai de mim!; o som dos suspiros}
  \seeref{ai4}
\end{EntryWithPhonetic}

\begin{EntryWithPhonetic}{挨}{ai1}{10}{⼿}
  \definition{prep.}{por turnos; em sequência; indica sequencialmente}
  \definition{v.}{estar próximo de; estar (ou chegar) perto de; abordar}
  \seeref{ai2}
\end{EntryWithPhonetic}

\begin{EntryWithPhonetic}{挨家挨户}{ai1jia1-ai1hu4}{10,10,10,4}{⼿、⼧、⼿、⼾}[HSK 7-9]
  \definition{expr.}{ir de casa em casa, de porta em porta ; um após o outro}
\end{EntryWithPhonetic}

\begin{EntryWithPhonetic}{挨着}{ai1 zhe5}{10,11}{⼿、⽬}[HSK 6]
  \definition{adv.}{ao lado de; perto de; imediatamente depois}
\end{EntryWithPhonetic}

\begin{EntryWithPhonetic}{挨}{ai2}{10}{⼿}[HSK 6]
  \definition{v.}{sofrer; suportar | arrastar-se; lutar para sobreviver (tempos difíceis); passar (tempo) com dificuldade | parar; atrasar; adiar; procrastinar}
  \seeref{ai1}
\end{EntryWithPhonetic}

\begin{EntryWithPhonetic}{挨打}{ai2/da3}{10,5}{⼿、⼿}[HSK 6]
  \definition{v.+compl.}{levar uma surra; ser atacado; ser espancado}
\end{EntryWithPhonetic}

\begin{EntryWithPhonetic}{癌}{ai2}{17}{⽧}[HSK 7-9]
  \definition{s.}{câncer; carcinoma; tumor maligno}
\end{EntryWithPhonetic}

\begin{EntryWithPhonetic}{癌症}{ai2zheng4}{17,10}{⽧、⽧}[HSK 7-9]
  \definition[种]{s.}{câncer; tumores malignos no corpo}
\end{EntryWithPhonetic}

\begin{EntryWithPhonetic}{矮}{ai3}{13}{⽮}[HSK 4]
  \definition{adj.}{baixo; baixa estatura; pequeno em altura | baixo; refere-se a grau, classificação, nível, etc.}[他比我矮。===Ele é mais baixo que eu. | 这栋楼很矮,只有三层。===Esse prédio é baixo, tem só três andares. | 她虽然矮,但是跑得很快!===Ela pode ser baixinha, mas corre muito rápido!]
\end{EntryWithPhonetic}

\begin{EntryWithPhonetic}{矮凳}{ai3deng4}{13,14}{⽮、⼏}
  \definition{s.}{banquinho baixo | banqueta}[这个矮凳是木制的,很结实。===Este banquinho é de madeira e bem resistente.]
\end{EntryWithPhonetic}

\begin{EntryWithPhonetic}{矮林}{ai3lin2}{13,8}{⽮、⽊}
  \definition{s.}{mata rasteira | bosque baixo}[这片矮林里有很多野兔和鸟类。===Neste bosque baixo há muitos coelhos selvagens e pássaros. | 山坡上长满了矮林,远看像绿色的地毯。===A encosta está coberta de mata rasteira, que de longe parece um tapete verde.]
\end{EntryWithPhonetic}

\begin{EntryWithPhonetic}{矮胖}{ai3pang4}{13,9}{⽮、⾁}
  \definition{adj.}{atarracado; gorducho; rechonchudo; roliço; baixo e robusto | chamar alguém diretamente de 矮胖 pode ser ofensivo}[我家猫矮胖矮胖的,像个毛球。===Meu gato é baixinho e gordinho, parece uma bolinha de pelo.]
\end{EntryWithPhonetic}

\begin{EntryWithPhonetic}{矮人}{ai3ren2}{13,2}{⽮、⼈}
  \definition{s.}{anão; pessoa de baixa estatura (indivíduo) | homúnculo; figuras criadas artificialmente pelos alquimistas em frascos de destilação | nanismo}[他虽然是矮人,但很有力气。===Embora ele seja baixo, é muito forte. | 北欧神话中的矮人是技艺高超的工匠。===Na mitologia nórdica, os anões são artesãos habilidosos. | 他因为身高被嘲笑为‘矮人’,这让他很伤心。===Ele foi zombado por ser chamado de ‘anão’ devido à sua altura, o que o magoou.]
\end{EntryWithPhonetic}

\begin{EntryWithPhonetic}{矮树}{ai3shu4}{13,9}{⽮、⽊}
  \definition[棵]{s.}{arbusto | árvore pequena, baixa}[矮树比高树更容易修剪。===Árvores baixas são mais fáceis de podar do que árvores altas. | 我们种了些矮树作为花园的边界。===Plantamos alguns arbustos como cerca natural do jardim.]
\end{EntryWithPhonetic}

\begin{EntryWithPhonetic}{矮小}{ai3 xiao3}{13,3}{⽮、⼩}[HSK 4]
  \definition{adj.}{subdimensionado; curto e pequeno; baixo e pequeno | quando usado para pessoas, pode soar depreciativo se não for em contexto neutro ou afetuoso}[这位矮小的老人是村里的智者。===Este idoso baixinho é o sábio da vila. | 这种矮小的灌木适合盆栽。===Este tipo de arbusto pequeno é ideal para vasos. | 山脚下有一片矮小的房屋,显得格外宁静。===Ao pé da montanha, havia casas baixas que transmitiam uma tranquilidade única.]
\end{EntryWithPhonetic}

\begin{EntryWithPhonetic}{矮星}{ai3xing1}{13,9}{⽮、⽇}
  \definition{s.}{estrela anã}[白矮星是恒星演化的最终阶段之一。===Anãs brancas são um dos estágios finais da evolução estelar.]
\end{EntryWithPhonetic}

\begin{EntryWithPhonetic}{矮子}{ai3zi5}{13,3}{⽮、⼦}
  \definition{s.}{pessoa baixa; anão; baixinho}[白雪公主和七个小矮子住在森林里。===Branca de Neve e os sete anões vivem na floresta. | 用`矮子'称呼他人是不礼貌的。===Chamar alguém de `baixinho' é falta de educação.]
\end{EntryWithPhonetic}

\begin{EntryWithPhonetic}{艾}{ai4}{5}{⾋}
  \definition*{s.}{Botânica: Artemísia chinesa (Artemisia argyi)}
  \definition*{s.}{Sobrenome Ai}
  \definition{adj.}{Literário: gracioso; bonito; lindo}
  \definition{s.}{artemísia; absinto; artemísia chinesa}
  \definition{v.}{Literário: parar; terminar}
  \seeref{yi4}
\end{EntryWithPhonetic}

\begin{EntryWithPhonetic}{艾滋病}{ai4zi1bing4}{5,12,10}{⾋、⽔、⽧}[HSK 7-9]
  \definition*{s.}{Síndrome da Imunodeficiência Adquirida (AIDS)}
\end{EntryWithPhonetic}

\begin{EntryWithPhonetic}{唉}{ai4}{10}{⼝}[HSK 7-9]
  \definition{interj.}{Oh!; Ah!; Bem!; interjeição que expressa tristeza ou arrependimento | Bem!; Argh!; usado para responder ou reconhecer}
  \seeref{ai1}
\end{EntryWithPhonetic}

\begin{EntryWithPhonetic}{爱}{ai4}{10}{⽖}[HSK 1]
  \definition*{s.}{Sobrenome Ai}
  \definition[个]{s.}{amor; afeição profunda; preocupação profunda; especialmente amor entre pessoas}[爱是理解和包容。===O amor é compreensão e tolerância.]
  \definition{v.}{amar; ter sentimentos profundos por pessoas ou coisas | gostar; gostar de; estar interessado em |  cuidar; valorizar; ter em alta estima; cuidar bem de | estar apto a; ter o hábito de}[他们深深爱着对方。===Eles se amam profundamente. | 我爱我的家人。===Eu amo minha família. | 我爱旅行。===Eu adoro viajar.]
\end{EntryWithPhonetic}

\begin{EntryWithPhonetic}{爱爱}{ai4'ai5}{10,10}{⽖、⽖}
  \definition{v.}{coloquial: fazer amor ou relações íntimas | pode ser usado como um apelido entre casais, transmitindo ternura | pode soar vulgar se usado em contextos inadequados}[他们俩刚结婚,天天都想爱爱。===Eles acabaram de se casar e querem fazer amor todo dia. | 爱爱,你今天好漂亮!===Amor, você está linda hoje!]
\end{EntryWithPhonetic}

\begin{EntryWithPhonetic}{爱不释手}{ai4bu2shi4shou3}{10,4,12,4}{⽖、⼀、⾤、⼿}[HSK 7-9]
  \definition{expr.}{``Não consigo parar de ler.''; ``Amo tanto que não consigo deixar passar.''; gostar (amar) algo tanto que não se pode suportar separar-se dele}
\end{EntryWithPhonetic}

\begin{EntryWithPhonetic}{爱抚}{ai4fu3}{10,7}{⽖、⼿}
  \definition{s.}{carinho; carícia}
  \definition{v.}{acariciar; afagar; cuidar (com ternura)}[他轻轻爱抚她的头发。===Ele afagou suavemente o cabelo dela. | 母亲爱抚婴儿的脸颊。===A mãe acaricia a bochecha do bebê. | 她爱抚着小猫的耳朵。===Ela acariciou as orelhas do gatinho.]
\end{EntryWithPhonetic}

\begin{EntryWithPhonetic}{爱国}{ai4 guo2}{10,8}{⽖、⼞}[HSK 4]
  \definition{adj.}{patriótico; patriotismo}[爱国是每个公民的责任。===O patriotismo é o dever de todo cidadão. | 这部电影讲述了英雄的爱国故事。===Este filme conta a história patriótica de um herói.]
  \definition{v.}{ser patriota; amar o seu país}
\end{EntryWithPhonetic}

\begin{EntryWithPhonetic}{爱好}{ai4 hao4}{10,6}{⽖、⼥}[HSK 1]
  \definition[个,种]{s.}{passatempo; interesse; \emph{hobby}; sentimentos de interesse especial ou afeição por algo | 爱好 é mais usado para atividades regulares (esportes, música), enquanto 喜欢 é para preferências gerais}[他的爱好是收集邮票。===Seu hobby era colecionar selos.  | 我的爱好是读书和旅行。===Meus hobbies são ler e viajar.]
  \definition{v.}{estar interessado em; ter prazer em; ter um forte interesse em algo; ter sentimentos profundos por alguém ou algo}
  \seealsoref{喜欢}{xi3huan5}
\end{EntryWithPhonetic}

\begin{EntryWithPhonetic}{爱好者}{ai4 hao4 zhe3}{10,6,8}{⽖、⼥、⽼}
  \definition{s.}{hobbista; amador; entusiasta; fã; amante (de arte, esportes, etc.)}[他是一位摄影爱好者。===Ele é um entusiasta de fotografia. | 她是位潜水爱好者,经常去东南亚潜水。===Ela é uma mergulhadora amadora e frequentemente mergulha no Sudeste Asiático.  | 我们为书法爱好者创建了一个微信群。===Criamos um grupo no WeChat para amantes de caligrafia.]
\end{EntryWithPhonetic}

\begin{EntryWithPhonetic}{爱护}{ai4hu4}{10,7}{⽖、⼿}[HSK 4]
  \definition{v.}{acalentar; valorizar; salvaguardar; cuidar bem de}[全社会都应爱护老年人。===Toda a sociedade deve tratar os idosos com cuidado e respeito. | 请爱护公园里的小动物。===Por favor, tratem os animais do parque com cuidado.]
\end{EntryWithPhonetic}

\begin{EntryWithPhonetic}{爱理不理}{ai4li3-bu4li3}{10,11,4,11}{⽖、⽟、⼀、⽟}[HSK 7-9]
  \definition{expr.}{frio e indiferente; distante}
\end{EntryWithPhonetic}

\begin{EntryWithPhonetic}{爱面子}{ai4/mian4zi5}{10,9,3}{⽖、⾯、⼦}[HSK 7-9]
  \definition{v.+compl.}{estar preocupado em salvar a face; ser vigilante em relação à reputação; ser sensível ao próprio orgulho; valorizar minha própria dignidade e ter medo que os outros me desprezem}[他爱面子,怕别人笑话他。===Ele se importa com sua reputação e tem medo que os outros riam dele.]
\end{EntryWithPhonetic}

\begin{EntryWithPhonetic}{爱情}{ai4qing2}{10,11}{⽖、⼼}[HSK 2]
  \definition{s.}{amor (entre pessoas); afeição}[爱情是盲目的。===O amor é cego. | 爱情如同玫瑰,美丽却带刺。===O amor é como uma rosa, bela mas com espinhos.  | 这首歌讲述了破碎的爱情故事。===Esta música conta uma história de amor fracassado.]
\end{EntryWithPhonetic}

\begin{EntryWithPhonetic}{爱人}{ai4 ren5}{10,2}{⽖、⼈}[HSK 2]
  \definition[个]{s.}{amante; \emph{dollbaby}; namorado(a) | marido ou esposa; mais usado em ocasiões formais}[这是我的爱人。===Este é o meu/minha esposo/companheiro. | 她是我一生的爱人。===Ela é o amor da minha vida. | 请携带爱人出席晚宴。===Por favor, traga seu cônjuge para o jantar.]
\end{EntryWithPhonetic}

\begin{EntryWithPhonetic}{爱上}{ai4shang4}{10,3}{⽖、⼀}
  \definition{v.}{perder o coração por; apaixonar-se por}[他在旅行时爱上了一位法国女孩。===Ele se apaixonou por uma garota francesa durante a viagem.  | 来到杭州后,我爱上了龙井茶。===Depois de chegar em Hangzhou, me apaixonei pelo chá Longjing. | 我从来没想过自己会爱上健身。===Eu nunca imaginei que iria me apaixonar por academia.]
\end{EntryWithPhonetic}

\begin{EntryWithPhonetic}{爱惜}{ai4xi1}{10,11}{⽖、⼼}[HSK 7-9]
  \definition{v.}{valorizar; prezar; estimar; usar com moderação; não desperdiçar}
\end{EntryWithPhonetic}

\begin{EntryWithPhonetic}{爱心}{ai4xin1}{10,4}{⽖、⼼}[HSK 3]
  \definition[片]{s.}{amor; carinho; compaixão; um sentimento de preocupação e carinho por outras pessoas ou animais}
\end{EntryWithPhonetic}

\begin{EntryWithPhonetic}{碍}{ai4}{13}{⽯}
  \definition{v.}{atrapalhar; dificultar; obstruir; estar no caminho de | levar em consideração}
\end{EntryWithPhonetic}

\begin{EntryWithPhonetic}{碍事}{ai4/shi4}{13,8}{⽯、⼅}[HSK 7-9]
  \definition{s.}{importa; tem consequências | (usualmente em frases negativas) sem consequência, não importa}[这决定不碍什么事。===Esta decisão não importa.]
  \definition{v.+compl.}{ser um obstáculo; estar no caminho; manter-se sob os pés de alguém; afetar o trabalho; causar inconveniência}
\end{EntryWithPhonetic}

\begin{EntryWithPhonetic}{厂}{an1}{2}{⼚}[Kangxi 27]
  \definition{s.}{usado principalmente em nomes pessoais}[他名中有个厂字。===O nome dele contém a palavra `An'.]
  \seeref{chang3}
  \seeref{han3}
\end{EntryWithPhonetic}

\begin{EntryWithPhonetic}{广}{an1}{3}{⼴}[Kangxi 53]
  \definition{s.}{mais comum em nomes de pessoas; o mesmo que 庵}[广安是我的朋友。===An'an é meu amigo.]
  \seeref{guang3}
  \seeref{yan3}
  \seealsoref{庵}{an1}
\end{EntryWithPhonetic}

\begin{EntryWithPhonetic}{安}{an1}{6}{⼧}[HSK 4]
  \definition*{s.}{Sobrenome An}
  \definition{adj.}{pacífico; quieto; tranquilo; calmo | seguro; protegido (oposto a 危) | com boa saúde | em paz; bem}
  \definition{adv.}{pacificamente; silenciosamente | com segurança; em segurança | em perguntas retóricas: como?}
  \definition{pron.}{usado como pronome interrogativo, como em 哪里,怎么; 谁,何,如何}
  \definition{s.}{segurança; proteção; paz | ampère; abreviação de ampère, 安培}
  \definition{v.}{tranquilizar (a mente de alguém); acalmar | contentar-se; ficar satisfeito | colocar em uma posição adequada; encontrar um lugar para | instalar; consertar; encaixar; configurar | trazer (uma acusação contra alguém); dar (a alguém um apelido); reivindicar (crédito por algo) | abrigar (uma intenção) | acalmar; estabilizar | sentir-se satisfeito e à vontade}
  \seealsoref{安培}{an1pei2}
  \seealsoref{何}{he2}
  \seealsoref{哪里}{na3 li3}
  \seealsoref{如何}{ru2he2}
  \seealsoref{谁}{shei2}
  \seealsoref{危}{wei1}
  \seealsoref{怎么}{zen3me5}
\end{EntryWithPhonetic}

\begin{EntryWithPhonetic}{安定}{an1ding4}{6,8}{⼧、⼧}[HSK 7-9]
  \definition{adj.}{estável; tranquilo; estabelecido; pacífico e em ordem; estável e normal, sem flutuações}
  \definition{s.}{tranquilizante; medicina ocidental comumente usada, com efeitos sedativos e anticonvulsivantes}
  \definition{v.}{acalmar; estabilizar; manter}
\end{EntryWithPhonetic}

\begin{EntryWithPhonetic}{安抚}{an1fu3}{6,7}{⼧、⼿}[HSK 7-9]
  \definition{v.}{pacificar; consolar; apaziguar; tranquilizar e acalmar}
\end{EntryWithPhonetic}

\begin{EntryWithPhonetic}{安家}{an1/jia1}{6,10}{⼧、⼧}
  \definition{v.+compl.}{montar uma casa | estabelecer-se}
\end{EntryWithPhonetic}

\begin{EntryWithPhonetic}{安检}{an1 jian3}{6,11}{⼧、⽊}[HSK 6]
  \definition{s.}{verificação de segurança}
  \definition{v.}{realizar verificação de segurança}
\end{EntryWithPhonetic}

\begin{EntryWithPhonetic}{安静}{an1jing4}{6,14}{⼧、⾭}[HSK 2]
  \definition{adj.}{silencioso; tranquilo; sem som; sem barulho e sem algazarra}
\end{EntryWithPhonetic}

\begin{EntryWithPhonetic}{安眠药}{an1mian2yao4}{6,10,9}{⼧、⽬、⾋}[HSK 7-9]
  \definition[片,颗,粒,瓶]{s.}{comprimido para dormir; soporífero; pílula para dormir; medicamentos que podem suprimir o córtex cerebral e induzir o sono}
\end{EntryWithPhonetic}

\begin{EntryWithPhonetic}{安宁}{an1ning2}{6,5}{⼧、⼧}[HSK 7-9]
  \definition{adj.}{pacífico; tranquilo; descreve um estado de ordem normal sem fatores externos que causem desordem ou inquietação | calmo; composto; livre de preocupações; sem preocupações, ansiedades ou inquietações}
\end{EntryWithPhonetic}

\begin{EntryWithPhonetic}{安排}{an1pai2}{6,11}{⼧、⼿}[HSK 3]
  \definition{s.}{plano; programação; organização; tabela do plano de atividades ou horários}
  \definition{v.}{organizar (assuntos) de acordo com a sequência ou regras; tratar as coisas de acordo com uma determinada ordem ou regras | atribuir tarefas a alguém; colocar as pessoas nos cargos de trabalho determinados, conforme planejado}
\end{EntryWithPhonetic}

\begin{EntryWithPhonetic}{安培}{an1pei2}{6,11}{⼧、⼟}
  \definition{clas.}{A; empréstimo linguístico: ampere; física: unidade de corrente elétrica}
\end{EntryWithPhonetic}

\begin{EntryWithPhonetic}{安全}{an1quan2}{6,6}{⼧、⼊}[HSK 2]
  \definition{adj.}{seguro; protegido; sem perigo; sem ameaças; sem acidentes}
  \definition{s.}{segurança; proteção; refere-se a um estado ou conceito, geralmente indicando ausência de ameaças ou perigo}
\end{EntryWithPhonetic}

\begin{EntryWithPhonetic}{安神}{an1/shen2}{6,9}{⼧、⽰}
  \definition{v.+compl.}{acalmar os nervos | aliviar a inquietação pela tranquilização da mente e do corpo}
\end{EntryWithPhonetic}

\begin{EntryWithPhonetic}{安慰}{an1wei4}{6,15}{⼧、⼼}[HSK 5]
  \definition{adj.}{confortar; tranquilizar; consolar; apaziguar;}
  \definition[句,通,番,声,个]{s.}{conforto; consolo; comportamento que alivia a dor de alguém e o acalma com palavras ou gestos}
  \definition{v.}{confortar; consolar; acalmar e confortar; deixar a mente tranquila}
\end{EntryWithPhonetic}

\begin{EntryWithPhonetic}{安稳}{an1wen3}{6,14}{⼧、⽲}[HSK 7-9]
  \definition{adj.}{seguro; suave e estável; estável | composto; calmo e equilibrado; (comportamento) estável; calmo}
\end{EntryWithPhonetic}

\begin{EntryWithPhonetic}{安心}{an1xin1}{6,4}{⼧、⼼}[HSK 7-9]
  \definition{adj.}{aliviado; à vontade; tranquilo; seguro}
  \definition{v.}{abrigar (más) intenções; acalentar certas intenções; ter (pensamentos ruins) em mente}
\end{EntryWithPhonetic}

\begin{EntryWithPhonetic}{安逸}{an1yi4}{6,11}{⼧、⾡}[HSK 7-9]
  \definition{adj.}{fácil; fácil e confortável; relaxado e confortável}
  \definition{s.}{conforto e facilidade; conforto}
\end{EntryWithPhonetic}

\begin{EntryWithPhonetic}{安置}{an1zhi4}{6,13}{⼧、⽹}[HSK 4]
  \definition{v.}{providenciar; encontrar um lugar para; ajudar a estabelecer-se; colocar pessoas ou coisas em uma determinada posição ou organizá-las adequadamente}
\end{EntryWithPhonetic}

\begin{EntryWithPhonetic}{安装}{an1zhuang1}{6,12}{⼧、⾐}[HSK 3]
  \definition{v.}{instalar; consertar; configurar; fixar máquinas ou equipamentos (geralmente conjuntos) em um determinado local, de acordo com métodos e especificações específicos}
\end{EntryWithPhonetic}

\begin{EntryWithPhonetic}{庵}{an1}{11}{⼴}
  \definition*{s.}{Sobrenome An}
  \definition[个,座]{s.}{cabana | convento de freiras; templos budistas, principalmente onde vivem as freiras}
\end{EntryWithPhonetic}

\begin{EntryWithPhonetic}{岸}{an4}{8}{⼭}[HSK 5]
  \definition{adj.}{arrogante; orgulhoso; grandioso (de maneira sombria ou condescendente)}
  \definition[条,道,段,面]{s.}{margem; costa; litoral; terreno à beira da água}
\end{EntryWithPhonetic}

\begin{EntryWithPhonetic}{岸上}{an4 shang4}{8,3}{⼭、⼀}[HSK 5]
  \definition{s.}{em terra; costa; margem | na margem do rio; na beira do rio}
\end{EntryWithPhonetic}

\begin{EntryWithPhonetic}{按}{an4}{9}{⼿}[HSK 3]
  \definition{prep.}{de acordo com; à luz de; com base em; em conformidade com}
  \definition{v.}{pressionar; empurrar para baixo; pressionar ou apertar com a mão ou os dedos | pôr de parte; deixar de lado; deixar para mais tarde | restringir; controlar; inibir; parar | verificar; consultar | comentar ou anotar (por um editor ou autor)}
\end{EntryWithPhonetic}

\begin{EntryWithPhonetic}{按键}{an4jian4}{9,13}{⼿、⾦}[HSK 7-9]
  \definition[个]{s.}{tecla; botão; teclas pressionadas manualmente}[键盘上的按键非常灵敏。===As teclas do teclado são muito responsivas.]
\end{EntryWithPhonetic}

\begin{EntryWithPhonetic}{按理}{an4li3}{9,11}{⼿、⽟}
  \definition{adv.}{de acordo com o princípio ou a razão; no curso normal dos eventos; normalmente | de acordo com a razão; de acordo com a prática comum; por (bons) direitos}
\end{EntryWithPhonetic}

\begin{EntryWithPhonetic}{按理说}{an4li3 shuo1}{9,11,9}{⼿、⽟、⾔}[HSK 7-9]
  \definition{adv.}{de acordo com o princípio ou a razão; no curso normal dos eventos; normalmente | é razoável dizer que\dots}
\end{EntryWithPhonetic}

\begin{EntryWithPhonetic}{按摩}{an4mo2}{9,15}{⼿、⼿}[HSK 5]
  \definition{s.}{massagem; empurrar, pressionar, beliscar e amassar o corpo de uma pessoa com as mãos para promover a circulação sanguínea, aumentar a resistência da pele e regular a função dos nervos}
\end{EntryWithPhonetic}

\begin{EntryWithPhonetic}{按时}{an4shi2}{9,7}{⼿、⽇}[HSK 4]
  \definition{adv.}{na hora; no horário; pontualmente; de acordo com o tempo estipulado}
\end{EntryWithPhonetic}

\begin{EntryWithPhonetic}{按说}{an4shuo1}{9,9}{⼿、⾔}[HSK 7-9]
  \definition{adv.}{no curso normal dos eventos; ordinariamente; normalmente | de acordo com o fato (senso comum); refere-se a falar de acordo com fatos ou senso comum; como uma questão de razão; expressões semelhantes incluem 按理 e 按理说}
  \seealsoref{按理}{an4li3}
  \seealsoref{按理说}{an4li3 shuo1}
\end{EntryWithPhonetic}

\begin{EntryWithPhonetic}{按照}{an4zhao4}{9,13}{⼿、⽕}[HSK 3]
  \definition{prep.}{de acordo com; em conformidade com; à luz de; com base em; apresentar os fundamentos ou critérios de julgamento para fazer algo}
\end{EntryWithPhonetic}

\begin{EntryWithPhonetic}{案}{an4}{10}{⽊}
  \definition{s.}{mesa; escrivaninha; mesa longa | caso; caso de direito (legal) | registro; arquivo; arquivo de caso | um plano submetido para consideração; proposta; um documento que propõe planos, sugestões, métodos, etc.}
\end{EntryWithPhonetic}

\begin{EntryWithPhonetic}{案件}{an4jian4}{10,6}{⽊、⼈}[HSK 7-9]
  \definition[个,起,件,类]{s.}{caso; caso de direito; caso legal; contencioso e eventos ilegais}
\end{EntryWithPhonetic}

\begin{EntryWithPhonetic}{暗}{an4}{13}{⽇}[HSK 4]
  \definition{adj.}{escuro; opaco; sem graça; pouca luz | escondido; secreto; não revelado | pouco claro; nebuloso; vago; confuso | subterrâneo}
  \definition{adv.}{secretamente | no escuro}
\end{EntryWithPhonetic}

\begin{EntryWithPhonetic}{暗地里}{an4di4li3}{13,6,7}{⽇、⼟、⾥}[HSK 7-9]
  \definition{adv.}{secretamente; interiormente; às escondidas}
\end{EntryWithPhonetic}

\begin{EntryWithPhonetic}{暗恋}{an4lian4}{13,10}{⽇、⼼}
  \definition{s.}{amor secreto}
  \definition{v.}{estar secretamente apaixonado por}
\end{EntryWithPhonetic}

\begin{EntryWithPhonetic}{暗杀}{an4sha1}{13,6}{⽇、⽊}[HSK 7-9]
  \definition{s.}{assassinato}
  \definition{v.}{assassinar | matar secretamente}
\end{EntryWithPhonetic}

\begin{EntryWithPhonetic}{暗示}{an4shi4}{13,5}{⽇、⽰}[HSK 4]
  \definition[种]{s.}{sugestão; insinuação; intimação; (psicologia) refere-se ao uso de palavras, gestos, expressões, etc. para fazer as pessoas aceitarem involuntariamente uma determinada opinião ou fazerem algo}
  \definition{v.}{dar uma dica; sugerir secretamente; indicar algo a alguém usando outras palavras, expressões faciais ou gestos sem dizer em voz alta}
\end{EntryWithPhonetic}

\begin{EntryWithPhonetic}{暗香}{an4xiang1}{13,9}{⽇、⾹}
  \definition{s.}{fragrância sutil}
\end{EntryWithPhonetic}

\begin{EntryWithPhonetic}{暗中}{an4zhong1}{13,4}{⽇、⼁}[HSK 7-9]
  \definition{adv.}{no escuro; na escuridão | em segredo; às escondidas; sorrateiramente | furtivamente; nos bastidores}
\end{EntryWithPhonetic}

\begin{EntryWithPhonetic}{暗自}{an4zi4}{13,6}{⽇、⾃}
  \definition{adv.}{interiormente; secretamente; para si mesmo}
\end{EntryWithPhonetic}

\begin{EntryWithPhonetic}{昂}{ang2}{8}{⽇}
  \definition{adj.}{alto; subindo}
  \definition{v.}{manter (a cabeça) erguida | elevar; levantar; olhar para cima}
\end{EntryWithPhonetic}

\begin{EntryWithPhonetic}{昂贵}{ang2gui4}{8,9}{⽇、⾙}[HSK 7-9]
  \definition{adj.}{caro; dispendioso; algo é muito caro, o preço é particularmente alto; metaforicamente, o custo de fazer algo é particularmente alto}
\end{EntryWithPhonetic}

\begin{EntryWithPhonetic}{凹}{ao1}{5}{⼐}[HSK 7-9]
  \definition{adj.}{afundado; amassado (oposto a 凸) | côncavo; oco; amassado; mais baixo que a área circundante}
  \definition{v.}{cavar; chanfrar | desabar; afundar}
  \seeref{wa1}
  \seealsoref{凸}{tu1}
\end{EntryWithPhonetic}

\begin{EntryWithPhonetic}{熬}{ao1}{14}{⽕}
  \definition{v.}{ensopar; ferver; cozinhar em água}
  \seeref{ao2}
\end{EntryWithPhonetic}

\begin{EntryWithPhonetic}{熬}{ao2}{14}{⽕}[HSK 7-9]
  \definition{v.}{ferver; ensopar; fazer uma decocção; cozinhar em fogo baixo por muito tempo | preparar; infundir; extrair a essência por fervura longa | resistir; suportar (angústia, tempos difíceis, etc.)}
  \seeref{ao1}
\end{EntryWithPhonetic}

\begin{EntryWithPhonetic}{熬夜}{ao2/ye4}{14,8}{⽕、⼣}[HSK 7-9]
  \definition{v.+compl.}{ficar acordado a noite toda ou até tarde da noite}
\end{EntryWithPhonetic}

\begin{EntryWithPhonetic}{傲}{ao4}{12}{⼈}[HSK 7-9]
  \definition{adj.}{orgulhoso; altivo | arrogante}
  \definition{v.}{recusar-se a ceder; desafiar}
\end{EntryWithPhonetic}

\begin{EntryWithPhonetic}{傲慢}{ao4man4}{12,14}{⼈、⼼}[HSK 7-9]
  \definition{adj.}{altivo; arrogante; autoritário}
\end{EntryWithPhonetic}

\begin{EntryWithPhonetic}{奥}{ao4}{12}{⼤}
  \definition*{s.}{Oersted, a unidade eletromagnética de intensidade do campo magnético; abreviação de 奥斯特 | Sobrenome Ao}
  \definition{adj.}{profundo e difícil de entender; abstruso | significado profundo, não é fácil de entender}
  \definition{s.}{canto secreto da casa; antigamente, referia-se ao canto sudoeste de uma casa e também, de modo geral, à profundidade de uma casa}
  \seealsoref{奥斯特}{ao4 si1 te4}
\end{EntryWithPhonetic}

\begin{EntryWithPhonetic}{奥林匹克运动会}{ao4lin2pi3ke4 yun4dong4hui4}{12,8,4,7,7,6,6}{⼤、⽊、⼖、⼗、⾡、⼒、⼈}
  \definition*{s.}{Jogos Olímpicos, Olimpíadas}
\end{EntryWithPhonetic}

\begin{EntryWithPhonetic}{奥秘}{ao4mi4}{12,10}{⼤、⽲}[HSK 7-9]
  \definition[个]{s.}{enigma; mistério profundo; fenômenos ou princípios profundos e misteriosos}
\end{EntryWithPhonetic}

\begin{EntryWithPhonetic}{奥斯特}{ao4 si1 te4}{12,12,10}{⼤、⽄、⽜}
  \definition{s.}{Oersted}
\end{EntryWithPhonetic}

\begin{EntryWithPhonetic}{奥特曼}{ao4te4man4}{12,10,11}{⼤、⽜、⽈}
  \definition*{s.}{Ultraman,  super-herói de ficção científica japonesa}
\end{EntryWithPhonetic}

\begin{EntryWithPhonetic}{奥运}{ao4yun4}{12,7}{⼤、⾡}
  \definition*{s.}{Jogos Olímpicos, Olimpíadas; Abreviação de 奥林匹克运动会}
  \seealsoref{奥林匹克运动会}{ao4lin2pi3ke4 yun4dong4hui4}
\end{EntryWithPhonetic}

\begin{EntryWithPhonetic}{奥运会}{ao4yun4hui4}{12,7,6}{⼤、⾡、⼈}[HSK 7-9]
  \definition*[届,次]{s.}{Jogos Olímpicos, Olimpíadas; Abreviação de 奥林匹克运动会}
  \seealsoref{奥林匹克运动会}{ao4lin2pi3ke4 yun4dong4hui4}
\end{EntryWithPhonetic}

\begin{EntryWithPhonetic}{澳}{ao4}{15}{⽔}
  \definition*{s.}{Abreviação de Austrália, 澳大利亚 | Sobrenome Ao}
  \definition{s.}{baía; uma entrada do mar; um lugar curvo na costa onde os barcos podem ser atracados, frequentemente usado em nomes de lugares}
  \seealsoref{澳大利亚}{ao4da4li4ya4}
\end{EntryWithPhonetic}

\begin{EntryWithPhonetic}{澳大利亚}{ao4da4li4ya4}{15,3,7,6}{⽔、⼤、⼑、⼆}
  \definition*{s.}{Austrália}
\end{EntryWithPhonetic}

%%%%% EOF %%%%%

