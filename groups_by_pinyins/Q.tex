%%%
%%% Q
%%%

\section*{Q}\addcontentsline{toc}{section}{Q}

\begin{EntryWithPhonetic}{七}{qi1}{2}{⼀}[HSK 1]
  \definition*{s.}{Sobrenome Qi}
  \definition{num.}{sete; 7}
  \definition{s.}{antigamente, os mortos eram homenageados a cada sete dias, chamados de 七, até o quadragésimo nono dia, num total de sete 七}
\end{EntryWithPhonetic}

\begin{EntryWithPhonetic}{七夕}{qi1xi1}{2,3}{⼀、⼣}
  \definition*{s.}{Dia dos Namorados Chinês, quando o vaqueiro e a tecelã (牛郎织女) têm permissão para se reunirem anualmente | Festival das Meninas | Festival Duplo Sete, noite do sétimo mês lunar}
  \seealsoref{牛郎织女}{niu2 lang2 zhi1nv3}
\end{EntryWithPhonetic}

\begin{EntryWithPhonetic}{妻}{qi1}{8}{⼥}
  \definition{s.}{esposa}
  \seeref{qi4}
\end{EntryWithPhonetic}

\begin{EntryWithPhonetic}{妻子}{qi1zi3}{8,3}{⼥、⼦}
  \definition[个]{s.}{esposa e filhos; (chinês antigo) refere-se a esposas, filhos e filhas}
  \seeref{qi1zi5}
\end{EntryWithPhonetic}

\begin{EntryWithPhonetic}{妻子}{qi1zi5}{8,3}{⼥、⼦}[HSK 4]
  \definition[个]{s.}{esposa (não é usado como um termo carinhoso)}
  \seeref{qi1zi3}
\end{EntryWithPhonetic}

\begin{EntryWithPhonetic}{期}{qi1}{12}{⽉}[HSK 3]
  \definition{clas.}{questão; número; termo; coisas usadas para parcelamento}
  \definition{s.}{um período de tempo; fase; estágio | horário agendado; data agendada | tempo designado (programado)}
  \definition{v.}{marcar uma consulta | esperar; aguardar | esperar; ter esperança}
\end{EntryWithPhonetic}

\begin{EntryWithPhonetic}{期待}{qi1dai4}{12,9}{⽉、⼻}[HSK 4]
  \definition{v.}{aguardar; esperar; aguardar ansiosamente; ter em mente a realização de um determinado fim ou a ocorrência de uma determinada situação}
\end{EntryWithPhonetic}

\begin{EntryWithPhonetic}{期间}{qi1jian1}{12,7}{⽉、⾨}[HSK 4]
  \definition{s.}{prazo; tempo; período}
\end{EntryWithPhonetic}

\begin{EntryWithPhonetic}{期末}{qi1 mo4}{12,5}{⽉、⽊}[HSK 4]
  \definition{s.}{terminal; final do prazo; fim do período}
\end{EntryWithPhonetic}

\begin{EntryWithPhonetic}{期望}{qi1wang4}{12,11}{⽉、⽉}[HSK 5]
  \definition{s.}{esperança; expectativa}
  \definition{v.}{esperar; ter esperança}
\end{EntryWithPhonetic}

\begin{EntryWithPhonetic}{期限}{qi1xian4}{12,8}{⽉、⾩}[HSK 4]
  \definition{s.}{prazo; limite de tempo; tempo alocado; período de tempo limitado, também o limite final do limite de tempo; \emph{deadline}}
\end{EntryWithPhonetic}

\begin{EntryWithPhonetic}{期中}{qi1 zhong1}{12,4}{⽉、⼁}[HSK 4]
  \definition{adj.}{provisório; interino; intermediário}
\end{EntryWithPhonetic}

\begin{EntryWithPhonetic}{欺}{qi1}{12}{⽋}
  \definition{v.}{enganar; trapacear | intimidar; tirar vantagem de alguém; tirar vantagem da fraqueza de (alguém, etc.)}
\end{EntryWithPhonetic}

\begin{EntryWithPhonetic}{欺负}{qi1fu5}{12,6}{⽋、⾙}[HSK 6]
  \definition{v.}{violar, oprimir ou insultar com meios irracionais; \emph{bully}}
\end{EntryWithPhonetic}

\begin{EntryWithPhonetic}{齐}{qi2}{6}{⿑}[HSK 3][Kangxi 210]
  \definition*{s.}{Qi, um estado da Dinastia Zhou | Dinastia Qi do Sul (479-502), uma das Dinastias do Sul | Dinastia Qi do Norte (550-577), uma das Dinastias do Norte | Sobrenome Qi}
  \definition{adj.}{arrumado; uniforme; regular; comprimento, tamanho, etc. são praticamente iguais; uniformes | semelhante; similar; da mesma forma; de acordo| tudo pronto; todos presentes; completo; perfeito}
  \definition{adv.}{juntos; simultaneamente; ao mesmo tempo}
  \definition{v.}{estar no mesmo nível que; alcançar o mesmo nível | estar nivelado em um ponto ou ao longo de uma linha; tornar consistente; harmonizar}
\end{EntryWithPhonetic}

\begin{EntryWithPhonetic}{齐国}{qi2 guo2}{6,8}{⿑、⼞}
  \definition*{s.}{Estado Qi de Zhou Ocidental e os Estados Combatentes (1122-265 a.C.), centrado em Shandong}
\end{EntryWithPhonetic}

\begin{EntryWithPhonetic}{齐全}{qi2quan2}{6,6}{⿑、⼊}[HSK 5]
  \definition{adj.}{completo; tudo pronto}
\end{EntryWithPhonetic}

\begin{EntryWithPhonetic}{其}{qi2}{8}{⼋}[HSK 5]
  \definition*{s.}{Sobrenome Qi}
  \definition{adv.}{fazer uma suposição ou uma réplica | expressar comando, ordem}
  \definition{pron.}{dele (dela, deles, delas) | ele, ela, isso, eles; elas | isso; tal | isso (referindo-se a nenhuma pessoa ou coisa específica)}
  \definition{suf.}{sufixo de palavra, anexado ao advérbio}
\end{EntryWithPhonetic}

\begin{EntryWithPhonetic}{其次}{qi2ci4}{8,6}{⼋、⽋}[HSK 3]
  \definition{adj.}{secundário}
  \definition{conj.}{próximo; então; em segundo lugar; mais tarde na ordem}
\end{EntryWithPhonetic}

\begin{EntryWithPhonetic}{其实}{qi2shi2}{8,8}{⼋、⼧}[HSK 3]
  \definition{adv.}{na verdade; na realidade; a primeira parte é a situação aparente, e 其实 é usado para introduzir a situação real}
\end{EntryWithPhonetic}

\begin{EntryWithPhonetic}{其他}{qi2ta1}{8,5}{⼋、⼈}[HSK 2]
  \definition{pron.}{outra pessoa/outra coisa | outras coisas; outras pessoas; em substituição de outras pessoas ou coisas}
\end{EntryWithPhonetic}

\begin{EntryWithPhonetic}{其余}{qi2yu2}{8,7}{⼋、⼈}[HSK 4]
  \definition{pron.}{o resto; os outros; o restante}
\end{EntryWithPhonetic}

\begin{EntryWithPhonetic}{其中}{qi2zhong1}{8,4}{⼋、⼁}[HSK 2]
  \definition{pron.}{dentro; entre (os quais, eles, etc.); em (o qual, ele, etc.); nas pessoas ou coisas mencionadas anteriormente}
\end{EntryWithPhonetic}

\begin{EntryWithPhonetic}{奇}{qi2}{8}{⼤}
  \definition{adj.}{ímpar (número); singular; solteiro; não em pares (ao contrário de 偶)}
  \definition{s.}{lotes ímpares; quantidade fracionária (acima daquela mencionada em um número redondo)}
  \seealsoref{偶}{ou3}
\end{EntryWithPhonetic}

\begin{EntryWithPhonetic}{奇怪}{qi2guai4}{8,8}{⼤、⼼}[HSK 3]
  \definition{adj.}{estranho; diferente do habitual; raramente visto, até um pouco irracional | estranho; esquisito; a descrição é diferente do imaginado e é difícil de entender}
  \definition{v.}{ficar perplexo; maravilhar-se; sentir-se surpreso; sentir-se estranho; sentir-se incompreensível}
\end{EntryWithPhonetic}

\begin{EntryWithPhonetic}{奇迹}{qi2ji4}{8,9}{⼤、⾡}
  \definition[个,种]{s.}{milagre; maravilha; coisas extraordinárias inimagináveis}
\end{EntryWithPhonetic}

\begin{EntryWithPhonetic}{奇妙}{qi2miao4}{8,7}{⼤、⼥}[HSK 6]
  \definition{adj.}{maravilhoso; milagroso; intrigante; muito inteligente e engenhoso (usado principalmente para descrever coisas interessantes e novas)}
\end{EntryWithPhonetic}

\begin{EntryWithPhonetic}{骑}{qi2}{11}{⾺}[HSK 2]
  \definition{s.}{cavalos ou outros animais para montaria | cavalaria; cavaleiro, também se refere genericamente a qualquer pessoa que monta a cavalo}
  \definition{v.}{montar (um animal ou bicicleta); sentar-se na parte de trás de | montar; abranger ambos os lados}
\end{EntryWithPhonetic}

\begin{EntryWithPhonetic}{骑车}{qi2 che1}{11,4}{⾺、⾞}[HSK 2]
  \definition{v.}{andar de bicicleta; pedalar}
\end{EntryWithPhonetic}

\begin{EntryWithPhonetic}{旗}{qi2}{14}{⽅}
  \definition[面]{s.}{bandeira}
\end{EntryWithPhonetic}

\begin{EntryWithPhonetic}{企}{qi3}{6}{⼈}
  \definition{v.}{ficar na ponta dos pés | esperar ansiosamente por algo; ansiar por | planejar um projeto}
\end{EntryWithPhonetic}

\begin{EntryWithPhonetic}{企图}{qi3tu2}{6,8}{⼈、⼞}[HSK 6]
  \definition[种]{s.}{plano; tentativa; intenção (principalmente negativa)}
  \definition{v.}{procurar; tentar; pretender}
\end{EntryWithPhonetic}

\begin{EntryWithPhonetic}{企业}{qi3ye4}{6,5}{⼈、⼀}[HSK 4]
  \definition[家,个]{s.}{empresa; estabelecimento; empreendimento; negócio; setores envolvidos em atividades econômicas como produção, transporte, comércio, etc., como fábricas, minas, ferrovias, empresas comerciais, etc.}
\end{EntryWithPhonetic}

\begin{EntryWithPhonetic}{岂}{qi3}{6}{⼭}
  \definition*{s.}{Sobrenome Qi}
  \definition{adv.}{Litarário: expressa uma pergunta retórica, equivalente a 哪里, 怎么 e 难道}
  \seealsoref{哪里}{na3 li3}
  \seealsoref{难道}{nan2dao4}
  \seealsoref{怎么}{zen3me5}
\end{EntryWithPhonetic}

\begin{EntryWithPhonetic}{岂有此理}{qi3you3ci3li3}{6,6,6,11}{⼭、⽉、⽌、⽟}
  \definition{interj.}{Que exorbitante! | Absurdo! | Como isso pode ser assim? | Ridículo!}
\end{EntryWithPhonetic}

\begin{EntryWithPhonetic}{启}{qi3}{7}{⼝}
  \definition*{s.}{Sobrenome Qi}
  \definition{s.}{nota; carta; um dos antigos estilos literários, uma carta relativamente curta}
  \definition{v.}{abrir | despertar; iluminar | começar; iniciar | declarar; informar}
\end{EntryWithPhonetic}

\begin{EntryWithPhonetic}{启动}{qi3 dong4}{7,6}{⼝、⼒}[HSK 5]
  \definition{v.}{ligar (uma máquina); acionar; ligar máquinas, equipamentos elétricos, etc., para começar a trabalhar | entrar em vigor; começar a vigorar e a ser implementados planos, projetos, documentos jurídicos, etc.}
\end{EntryWithPhonetic}

\begin{EntryWithPhonetic}{启发}{qi3fa1}{7,5}{⼝、⼜}[HSK 5]
  \definition{s.}{iluminação; esclarecimento; fenômenos e princípios que levam as pessoas a refletir e a abrir suas mentes}
  \definition{v.}{despertar; inspirar; esclarecer; orientar, fazer com que compreendam}
\end{EntryWithPhonetic}

\begin{EntryWithPhonetic}{启事}{qi3shi4}{7,8}{⼝、⼅}[HSK 5]
  \definition[个,则,份,张,条]{s.}{aviso; anúncio; texto publicado em jornais ou afixado em paredes com o objetivo de divulgar publicamente algo}
\end{EntryWithPhonetic}

\begin{EntryWithPhonetic}{起}{qi3}{10}{⾛}[HSK 1]
  \definition{clas.}{caso; instância | lote; grupo}
  \definition{prep.}{de; colocado antes de uma palavra de tempo ou lugar, indica um ponto de partida | por; colocado antes de uma palavra de lugar, indica um lugar por onde passou}
  \definition{v.}{levantar-se; ficar de pé| iniciar; lançar; deicar a posição original | subir; ascender | aparecer; levantar; crescer (bolhas, protuberâncias, brotoeja) | puxar para cima; puxar para fora; tirar o que está guardado ou incorporado | crescer; aumentar | esboçar; elaborar | construir; montar; estabelecer | receber (comprovante) | começar; iniciar; combina com 从 e 由; indica quando, onde e quem começou | buscar; pegar; usado após um verbo, indica movimento para cima | indicar se alguém tem força suficiente ou não; usado após um verbo, indica que a força é suficiente ou insuficiente | indicar que a ação envolve alguém ou algo; equivalente a 及 ou 到 | começar; iniciar; usado depois de um verbo, indica o início de uma ação | juntar; implodir; (informal) usado depois de um verbo, para unir coisas ou fechá-las}
  \seealsoref{从}{cong2}
  \seealsoref{到}{dao4}
  \seealsoref{及}{ji2}
  \seealsoref{由}{you2}
\end{EntryWithPhonetic}

\begin{EntryWithPhonetic}{起床}{qi3/chuang2}{10,7}{⾛、⼴}[HSK 1]
  \definition{v.+compl.}{levantar-se; sair da cama; acordar e sair da cama (geralmente pela manhã); levantar-se da posição sentada, deitada ou deitada de bruços, ou sentar-se a partir da posição deitada}
\end{EntryWithPhonetic}

\begin{EntryWithPhonetic}{起到}{qi3 dao4}{10,8}{⾛、⼑}[HSK 5]
  \definition{v.}{ter (um efeito motivador, etc.); desempenhar (um papel estabilizador, etc.)}
\end{EntryWithPhonetic}

\begin{EntryWithPhonetic}{起点}{qi3 dian3}{10,9}{⾛、⽕}[HSK 6]
  \definition[个]{s.}{ponto de partida (para o tempo ou local do início de algo); o lugar ou hora de início | ponto de partida (para o nível ou base de algo feito inicialmente); refere-se especificamente ao ponto de partida designado em um evento de pista}
\end{EntryWithPhonetic}

\begin{EntryWithPhonetic}{起飞}{qi3fei1}{10,3}{⾛、⾶}[HSK 2]
  \definition{v.}{decolar; levantar voo | crescer rapidamente; decolar; disparar; metáfora para o rápido desenvolvimento de negócios, economia, etc.}
\end{EntryWithPhonetic}

\begin{EntryWithPhonetic}{起来}{qi3/lai2}{10,7}{⾛、⽊}[HSK 1]
  \definition{v.+compl.}{levantar-se; passar de posições como deitado, sentado ou ajoelhado para ficar em pé | levantar-se; sair da cama | levantar-se; revoltar-se; rebelar-se; refere-se a ascensão, surgimento, levantamento, etc.}
  \seeref{qi3lai5}
  \seeref{qi5lai2}
\end{EntryWithPhonetic}

\begin{EntryWithPhonetic}{起来}{qi3lai5}{10,7}{⾛、⽊}
  \definition{v.aux.}{usado depois de um verbo para indicar movimento ascendente}
  \seeref{qi3/lai2}
  \seeref{qi5lai2}
\end{EntryWithPhonetic}

\begin{EntryWithPhonetic}{起码}{qi3ma3}{10,8}{⾛、⽯}[HSK 5]
  \definition{adj.}{mínimo; elementar; rudimentar}
  \definition{adv.}{mínimamente; pelo menos;}
\end{EntryWithPhonetic}

\begin{EntryWithPhonetic}{起诉}{qi3 su4}{10,7}{⾛、⾔}[HSK 6]
  \definition{v.}{processar; entrar com uma ação judicial}
\end{EntryWithPhonetic}

\begin{EntryWithPhonetic}{起跳}{qi3tiao4}{10,13}{⾛、⾜}
  \definition{v.}{(atletismo) decolar (no início de um salto) | (de preço, salário, etc.) começar (de um determinado nível)}
\end{EntryWithPhonetic}

\begin{EntryWithPhonetic}{气}{qi4}{4}{⽓}[HSK 2][Kangxi 84]
  \definition*{s.}{Sobrenome Qi}
  \definition[口]{s.}{gás; gás em geral | ar; especificamente, o ar | respiração | clima; refere-se a fenômenos naturais como sol, chuva, frio e calor | cheiro; odor; o cheiro que o nariz sente | ânimo; moral; estado mental | ares; estilo; maneiras; refere-se ao estilo e aos hábitos de uma pessoa | raiva; irritação; aborrecimento; sentimento de irritação | energia vital; energia da vida; na medicina tradicional chinesa refere-se às substâncias sutis que circulam no corpo humano e permitem que os vários órgãos funcionem normalmente | certos sintomas (de doenças); na medicina tradicional chinesa refere-se a um determinado quadro clínico}
  \definition{v.}{ficar com raiva; ficar furioso; ficar irritado | irritar; enfurecer; deixar com raiva | ser intimidado; sofrer injustiça; intimidar}
\end{EntryWithPhonetic}

\begin{EntryWithPhonetic}{气氛}{qi4fen1}{4,8}{⽓、⽓}[HSK 6]
  \definition{s.}{atmosfera; sensação circundante; uma certa emoção ou cena que existe em um determinado ambiente e pode fazer as pessoas sentirem}
\end{EntryWithPhonetic}

\begin{EntryWithPhonetic}{气候}{qi4hou4}{4,10}{⽓、⼈}[HSK 3]
  \definition[种]{s.}{clima; tempo; condições meteorológicas gerais obtidas após muitos anos de observação em uma determinada região, estão relacionadas com correntes de ar, latitude, altitude acima do nível do mar, relevo, etc. | tendência; situação; metáfora do ambiente social, de uma determinada tendência | resultado; influência; conquista; realização; metáfora para algum tipo de resultado, conquista, influência significativa ou potencial de desenvolvimento}
\end{EntryWithPhonetic}

\begin{EntryWithPhonetic}{气球}{qi4qiu2}{4,11}{⽓、⽟}[HSK 4]
  \definition[个,只]{s.}{balão; bolas feitas de borracha, plástico, etc., que podem ser aumentadas soprando ar nelas e podem ser usadas como brinquedos, decorações ou meios de transporte}
\end{EntryWithPhonetic}

\begin{EntryWithPhonetic}{气体}{qi4 ti3}{4,7}{⽓、⼈}[HSK 5]
  \definition[种,瓶,升]{s.}{gás; não têm forma nem volume definidos e podem fluir.; o ar, o oxigênio, o gás metano e outros são gases}
\end{EntryWithPhonetic}

\begin{EntryWithPhonetic}{气温}{qi4 wen1}{4,12}{⽓、⽔}[HSK 2]
  \definition[个]{s.}{temperatura do ar}
\end{EntryWithPhonetic}

\begin{EntryWithPhonetic}{气象}{qi4xiang4}{4,11}{⽓、⾗}[HSK 5]
  \definition[种,派]{s.}{fenômenos meteorológicos; condições e fenômenos atmosféricos, como vento, relâmpagos, trovões, geadas, neve, etc. | meteorologia | situação; atmosfera; cena; circunstância | maneira imponente}
\end{EntryWithPhonetic}

\begin{EntryWithPhonetic}{气质}{qi4zhi4}{4,8}{⽓、⾙}
  \definition{s.}{traços de personalidade, temperamento, disposição | aura, ar, sentimento, \emph{vibe} | refinamento, sofisticação, classe}
\end{EntryWithPhonetic}

\begin{EntryWithPhonetic}{汽}{qi4}{7}{⽔}
  \definition{s.}{vapor | vaporizador}
\end{EntryWithPhonetic}

\begin{EntryWithPhonetic}{汽车}{qi4 che1}{7,4}{⽔、⾞}[HSK 1]
  \definition[辆,种,款]{s.}{automóvel; carro; veículo motorizado; veículo movido a motor de combustão interna, que circula principalmente em rodovias ou ruas, geralmente com quatro ou mais pneus de borracha, usado para transportar pessoas ou mercadorias}
\end{EntryWithPhonetic}

\begin{EntryWithPhonetic}{汽水}{qi4 shui3}{7,4}{⽔、⽔}[HSK 4]
  \definition[罐,杯,瓶,听,口]{s.}{refrigerante; refrigerante gaseificado; bebida refrescante, feita com a pressão de dióxido de carbono para dissolver na água e adicionar açúcar, suco de frutas, especiarias etc.}
\end{EntryWithPhonetic}

\begin{EntryWithPhonetic}{汽油}{qi4you2}{7,8}{⽔、⽔}[HSK 4]
  \definition[桶,升,吨]{s.}{gasolina; mistura líquida de hidrocarbonetos com volatilidade e combustibilidade, que é usada como combustível a partir do fracionamento ou craqueamento do petróleo}
\end{EntryWithPhonetic}

\begin{EntryWithPhonetic}{妻}{qi4}{8}{⼥}
  \definition{v.}{casar uma mulher com (alguém)}
  \seeref{qi1}
\end{EntryWithPhonetic}

\begin{EntryWithPhonetic}{器}{qi4}{16}{⼝}
  \definition[台]{s.}{dispositivo | ferramenta | utensílio}
\end{EntryWithPhonetic}

\begin{EntryWithPhonetic}{器官}{qi4guan1}{16,8}{⼝、⼧}[HSK 4]
  \definition[个,种]{s.}{órgão; aparelho; parte de um organismo que consiste em vários tipos de tecidos celulares que podem desempenhar uma função fisiológica separada}
\end{EntryWithPhonetic}

\begin{EntryWithPhonetic}{起来}{qi5lai2}{10,7}{⾛、⽊}
  \definition{v.}{descrever resultados, retratar comportamentos, transmitir movimento}
  \seeref{qi3/lai2}
  \seeref{qi3lai5}
\end{EntryWithPhonetic}

\begin{EntryWithPhonetic}{卡}{qia3}{5}{⼘}
  \definition*{s.}{Sobrenome Qia}
  \definition[张,片]{s.}{clipe; prendedor; pinça; utensílio para prender objetos | posto de controle; posto de guarda ou posto de controle localizado em vias de comunicação importantes ou em locais com terreno acidentado}
  \definition{v.}{encravar; ficar preso; impedir de se mover | parar; controlar; impedir | pressionar firmemente com a palma da mão}
  \seeref{ka3}
\end{EntryWithPhonetic}

\begin{EntryWithPhonetic}{恰}{qia4}{9}{⼼}
  \definition{adv.}{exatamente | apenas}
\end{EntryWithPhonetic}

\begin{EntryWithPhonetic}{恰当}{qia4dang4}{9,6}{⼼、⼹}[HSK 6]
  \definition{adj.}{adequado; apropriado; conveniente; apropriado; a linguagem ou abordagem é muito apropriada}
\end{EntryWithPhonetic}

\begin{EntryWithPhonetic}{恰到好处}{qia4dao4hao3chu4}{9,8,6,5}{⼼、⼑、⼥、⼡}
  \definition{expr.}{é simplesmente perfeito | é simplesmente correto}
\end{EntryWithPhonetic}

\begin{EntryWithPhonetic}{恰好}{qia4 hao3}{9,6}{⼼、⼥}[HSK 6]
  \definition{adv.}{na medida certa; como a sorte quis}
\end{EntryWithPhonetic}

\begin{EntryWithPhonetic}{恰恰}{qia4 qia4}{9,9}{⼼、⼼}[HSK 6]
  \definition{adv.}{justamente; exatamente; precisamente; bem na hora}
\end{EntryWithPhonetic}

\begin{EntryWithPhonetic}{千}{qian1}{3}{⼗}[HSK 2]
  \definition*{s.}{Sobrenome Qian}
  \definition{num.}{mil; 1.000; 1000 | a grande quantidade de; um grande número de}
\end{EntryWithPhonetic}

\begin{EntryWithPhonetic}{千古}{qian1gu3}{3,5}{⼗、⼝}
  \definition{adv.}{por toda a eternidade | em todas as idades}
  \definition{s.}{eternidade (usada em um dístico elegíaco, coroa de flores, etc., dedicada aos mortos)}
\end{EntryWithPhonetic}

\begin{EntryWithPhonetic}{千克}{qian1 ke4}{3,7}{⼗、⼗}[HSK 2]
  \definition{clas.}{kg; quilo; quilograma; 1 quilograma equivale a 1.000 gramas, ou 2 jin (斤)}
  \seealsoref{斤}{jin1}
\end{EntryWithPhonetic}

\begin{EntryWithPhonetic}{千年}{qian1nian2}{3,6}{⼗、⼲}
  \definition{s.}{milênio}
\end{EntryWithPhonetic}

\begin{EntryWithPhonetic}{千千万万}{qian1qian1wan4wan4}{3,3,3,3}{⼗、⼗、⼀、⼀}
  \definition{num.}{inumerável | números incontáveis | milhares e milhares}
\end{EntryWithPhonetic}

\begin{EntryWithPhonetic}{千万}{qian1wan4}{3,3}{⼗、⼀}[HSK 3]
  \definition{adv.}{(usado para indicar desejos fortes) por todos os meios; sob quaisquer circunstâncias; expressa uma exortação sincera, equivalente a 务必}
  \definition{num.}{dez milhões; 10.000.000; 1000.0000; milhões e milhões; um número aproximado, indicando um grande número}
  \seealsoref{务必}{wu4bi4}
\end{EntryWithPhonetic}

\begin{EntryWithPhonetic}{牵}{qian1}{9}{⽜}[HSK 6]
  \definition{v.}{conduzir (segurando a mão, o cabresto, etc.); puxar | envolver-se | sentir falta; preocupar-se com | controlar; restringir; ser retido; ser constrangido}
\end{EntryWithPhonetic}

\begin{EntryWithPhonetic}{铅}{qian1}{10}{⾦}
  \definition[根,盒]{s.}{chumbo (Pb) | grafite (em um lápis); grafite preta |}
\end{EntryWithPhonetic}

\begin{EntryWithPhonetic}{铅笔}{qian1bi3}{10,10}{⾦、⽵}[HSK 6]
  \definition[支,盒,种,枝,杆]{s.}{lápis; canetas com pontas de grafite ou argila pigmentada}
\end{EntryWithPhonetic}

\begin{EntryWithPhonetic}{谦}{qian1}{12}{⾔}
  \definition*{s.}{Sobrenome Qian}
  \definition{adj.}{modesto}
  \definition{s.}{modéstia}
\end{EntryWithPhonetic}

\begin{EntryWithPhonetic}{谦虚}{qian1xu1}{12,11}{⾔、⾌}[HSK 6]
  \definition{adj.}{modesto; não se orgulhe de suas próprias conquistas e esteja disposto a aceitar críticas e opiniões de outras pessoas}
  \definition{v.}{falar modestamente; quando recebo elogios e cumprimentos de outras pessoas, sinto que não sou tão bom}
\end{EntryWithPhonetic}

\begin{EntryWithPhonetic}{签}{qian1}{13}{⽵}[HSK 5]
  \definition[个,根,支]{s.}{tiras de bambu usadas para adivinhação ou sorteio; pPequenas tiras de bambu ou varas finas com caracteres e símbolos gravados, usadas para adivinhação, jogos de azar ou como fichas para contagem, etc. | etiqueta; adesivo; pequena tira usada como marca | um pedaço fino e pontiagudo de bambu ou madeira; pequeno bastão pontiagudo}
  \definition{v.}{assinar; autografar; escrever o nome, palavras ou fazer marcas em documentos ou recibos | fazer comentários breves em um documento; escrever brevemente (pontos principais ou opiniões) | (em costura) alinhavar; costura grosseira}
\end{EntryWithPhonetic}

\begin{EntryWithPhonetic}{签订}{qian1 ding4}{13,4}{⽵、⾔}[HSK 5]
  \definition{v.}{concluir e assinar (um tratado, etc.)}
\end{EntryWithPhonetic}

\begin{EntryWithPhonetic}{签名}{qian1/ming2}{13,6}{⽵、⼝}[HSK 5]
  \definition[个,次]{s.}{assinatura; autógrafo}
  \definition{v.+compl.}{assinar o próprio nome; autografar; escrever seu nome para indicar concordância, apoio ou homenagem, etc.}
\end{EntryWithPhonetic}

\begin{EntryWithPhonetic}{签约}{qian1 yue1}{13,6}{⽵、⽷}[HSK 5]
  \definition{v.}{assinar um contrato; assinar contratos e tratados, frequentemente utilizado no trabalho e em cooperações comerciais}
\end{EntryWithPhonetic}

\begin{EntryWithPhonetic}{签证}{qian1zheng4}{13,7}{⽵、⾔}[HSK 5]
  \definition[张,个,份]{s.}{visto; visto de entrada em um país}
\end{EntryWithPhonetic}

\begin{EntryWithPhonetic}{签字}{qian1 zi4}{13,6}{⽵、⼦}[HSK 5]
  \definition{v.}{assinar; colocar a assinatura; escrever seu nome à mão em documentos, recibos, etc., para demonstrar responsabilidade}
\end{EntryWithPhonetic}

\begin{EntryWithPhonetic}{前}{qian2}{9}{⼑}[HSK 1]
  \definition*{s.}{Sobrenome Qian}
  \definition{s.}{frente | futuro; perspectiva | atrás; antes; mais cedo do que uma coisa ou um momento | à frente; para a frente; na parte frontal (referindo-se ao espaço, em oposição a 后) | precedente; antes que algo aconteça | antigo; antigamente | topo; primeiro; primeiro na ordem | frente; campo de batalha | A.C. (Antes de~Cristo)}[前293年===293 a.C.]
  \definition{v.}{seguir em frente; ir em frente}
  \seealsoref{公元}{gong1yuan2}
  \seealsoref{后}{hou4}
\end{EntryWithPhonetic}

\begin{EntryWithPhonetic}{前边}{qian2 bian5}{9,5}{⼑、⾡}[HSK 1]
  \definition{adv.}{à frente; na frente}
\end{EntryWithPhonetic}

\begin{EntryWithPhonetic}{前方}{qian2 fang1}{9,4}{⼑、⽅}[HSK 6]
  \definition{s.}{frente; o espaço à frente; a direção voltada para a frente; a frente (em oposição à 后方) | linha de frente; frente de batalha; áreas onde os exércitos de ambos os lados estão se aproximando ou lutando}
  \seealsoref{后方}{hou4 fang1}
\end{EntryWithPhonetic}

\begin{EntryWithPhonetic}{前后}{qian2 hou4}{9,6}{⼑、⼝}[HSK 3]
  \definition{s.}{em volta; sobre; um período de tempo ligeiramente anterior ou posterior a um horário específico| do início ao fim; refere-se ao período de tempo do início ao fim de algo | frente e verso; na frente e atrás de algo}
\end{EntryWithPhonetic}

\begin{EntryWithPhonetic}{前进}{qian2 jin4}{9,7}{⼑、⾡}[HSK 3]
  \definition{v.}{marchar; avançar; para ir em frente; seguir em frente; geralmente se refere ao desenvolvimento futuro}
\end{EntryWithPhonetic}

\begin{EntryWithPhonetic}{前景}{qian2jing3}{9,12}{⼑、⽇}[HSK 5]
  \definition{s.}{primeiro plano (de uma vista, imagem, foto, etc.); as imagens que parecem mais próximas do espectador em pinturas, palcos e telas | vista; perspectiva; prospecto; ponto de vista; situações que podem ocorrer no trabalho, na carreira, etc.}
\end{EntryWithPhonetic}

\begin{EntryWithPhonetic}{前来}{qian2 lai2}{9,7}{⼑、⽊}[HSK 6]
  \definition{v.}{vir; em direção à localização e direção do falante}
\end{EntryWithPhonetic}

\begin{EntryWithPhonetic}{前面}{qian2mian4}{9,9}{⼑、⾯}[HSK 3]
  \definition{s.}{frente; a parte frontal do espaço ou posição | parte anterior; acima; a parte que vem primeiro na ordem; a parte de um artigo ou discurso que precede a narração atual}
\end{EntryWithPhonetic}

\begin{EntryWithPhonetic}{前年}{qian2 nian2}{9,6}{⼑、⼲}[HSK 2]
  \definition{adv.}{há dois anos; dois anos atrás}
\end{EntryWithPhonetic}

\begin{EntryWithPhonetic}{前提}{qian2ti2}{9,12}{⼑、⼿}[HSK 5]
  \definition[个,项]{s.}{premissa; pressuposto | pré-requisito; pressuposição; condições prévias para que algo aconteça ou se desenvolva}
\end{EntryWithPhonetic}

\begin{EntryWithPhonetic}{前天}{qian2 tian1}{9,4}{⼑、⼤}[HSK 1]
  \definition{adv.}{anteontem; dia anterior a ontem}
\end{EntryWithPhonetic}

\begin{EntryWithPhonetic}{前头}{qian2 tou5}{9,5}{⼑、⼤}[HSK 4]
  \definition{s.}{à frente; na frente; adiante}
\end{EntryWithPhonetic}

\begin{EntryWithPhonetic}{前途}{qian2tu2}{9,10}{⼑、⾡}[HSK 4]
  \definition[片,段,种]{s.}{futuro; perspectiva; prospecto; originalmente, refere-se à jornada à frente, mas, metaforicamente, refere-se ao futuro.}
\end{EntryWithPhonetic}

\begin{EntryWithPhonetic}{前往}{qian2 wang3}{9,8}{⼑、⼻}[HSK 3]
  \definition{v.}{ir para; prosseguir para; partir para; ir em frente}
\end{EntryWithPhonetic}

\begin{EntryWithPhonetic}{前线}{qian2 xian4}{9,8}{⼑、⽷}
  \definition{s.}{linha de frente; frente (oposto à 后方) | frente de batalha; a área onde os dois exércitos se aproximam durante uma batalha (em oposição à 后方)}
  \seealsoref{后方}{hou4 fang1}
\end{EntryWithPhonetic}

\begin{EntryWithPhonetic}{钱}{qian2}{10}{⾦}[HSK 1]
  \definition*{s.}{Sobrenome Qian}
  \definition{clas.}{qian, uma unidade de peso (=5 gramas) | qian, uma unidade de peso (um décimo de um tael 两)}
  \definition[笔]{s.}{dinheiro; riqueza; bens | moeda de cobre; dinheiro | objeto em forma de moeda de cobre | fundo; montante | dinheiro guardado ou gasto para algum fim específico (geralmente se refere a quantias significativas de dinheiro que entram e saem de órgãos públicos, organizações, etc.)}
  \seealsoref{两}{liang3}
\end{EntryWithPhonetic}

\begin{EntryWithPhonetic}{钱包}{qian2 bao1}{10,5}{⾦、⼓}[HSK 1]
  \definition[个]{s.}{carteira; bolsa; bolsa de dinheiro}
\end{EntryWithPhonetic}

\begin{EntryWithPhonetic}{潜}{qian2}{15}{⽔}
  \definition*{s.}{Sobrenome Qian}
  \definition{adj.}{latente; oculto}
  \definition{adv.}{furtivamente; secretamente; às escondidas}
  \definition{v.}{ir para debaixo d'água; esconder-se debaixo d'água; mergulhar | esconder | vadear (atravessar) na água | enterrar | fugir de casa}
\end{EntryWithPhonetic}

\begin{EntryWithPhonetic}{潜力}{qian2li4}{15,2}{⽔、⼒}[HSK 6]
  \definition{s.}{potencial; potencialidade; capacidade latente; as habilidades e possibilidades de desenvolvimento que as pessoas e as coisas ainda não demonstraram}
\end{EntryWithPhonetic}

\begin{EntryWithPhonetic}{潜在}{qian2zai4}{15,6}{⽔、⼟}
  \definition{adj.}{oculto | latente}
  \definition{s.}{potencial}
\end{EntryWithPhonetic}

\begin{EntryWithPhonetic}{浅}{qian3}{8}{⽔}[HSK 4]
  \definition{adj.}{raso; superficial;  (em oposição a 深) | fácil; simples; redação, conteúdo, etc. simples e fáceis de entender | superficial; não é profundo em aprendizado, percepção e sabedoria | não próximo; não íntimo; sentimentos não profundos | (cor) claro; pálido;  cor pouco intensa; leve |experiência breve; duração de tempo breve | baixo grau; peso leve; nível baixo}
  \seeref{jian1}
  \seealsoref{深}{shen1}
\end{EntryWithPhonetic}

\begin{EntryWithPhonetic}{欠}{qian4}{4}{⽋}[HSK 5][Kangxi 76]
  \definition{v.}{bocejar | levantar ligeiramente (uma parte do corpo) | estar em dívida; estar atrasado com; não devolver o que pediu emprestado a outra pessoa, ou não dar o que deveria ter dado a outra pessoa | faltar; não ser suficiente}
\end{EntryWithPhonetic}

\begin{EntryWithPhonetic}{抢}{qiang1}{7}{⼿}
  \definition{prep.}{contra; direção relativa inversa}
  \definition{v.}{bater; tocar}
  \seeref{qiang3}
\end{EntryWithPhonetic}

\begin{EntryWithPhonetic}{枪}{qiang1}{8}{⽊}[HSK 5]
  \definition*{s.}{Sobrenome Qiang}
  \definition[把,杆,支,挺]{s.}{lança | arma; rifle; arma de fogo | uma coisa em forma de arma | enxada; ferramenta para cavar a terra}
  \definition{v.}{escrever artigos ou responder perguntas para outras pessoas}
\end{EntryWithPhonetic}

\begin{EntryWithPhonetic}{将}{qiang1}{9}{⼨}
  \definition{v.}{pedir; apelar para}
  \seeref{jiang1}
  \seeref{jiang4}
\end{EntryWithPhonetic}

\begin{EntryWithPhonetic}{强}{qiang2}{12}{⼸}[HSK 3]
  \definition*{s.}{Sobrenome Qiang}
  \definition{adj.}{forte; poderoso  (em oposição a 弱) | melhor; superior | mais; extra; adicional; um pouco mais que; usado após uma fração ou decimal para indicar que é um pouco maior que o número | resoluto; firme | violento | alto padrão}
  \definition{v.}{fortalecer; tornar forte; tornar poderoso}
  \seeref{jiang4}
  \seeref{qiang3}
  \seealsoref{弱}{ruo4}
\end{EntryWithPhonetic}

\begin{EntryWithPhonetic}{强大}{qiang2 da4}{12,3}{⼸、⼤}[HSK 3]
  \definition{adj.}{forte; poderoso; potente; possante; descreve força forte e grande poder}
\end{EntryWithPhonetic}

\begin{EntryWithPhonetic}{强盗}{qiang2 dao4}{12,11}{⼸、⽫}[HSK 6]
  \definition[个,群,伙,帮]{s.}{ladrão; bandido; uma pessoa que usa violência para confiscar a propriedade de outros; também se refere a uma pessoa ou força que se envolve em comportamento semelhante}
\end{EntryWithPhonetic}

\begin{EntryWithPhonetic}{强调}{qiang2diao4}{12,10}{⼸、⾔}[HSK 3]
  \definition{v.}{salientar; sublinhar; enfatizar; dar ênfase a; vincar}
\end{EntryWithPhonetic}

\begin{EntryWithPhonetic}{强度}{qiang2 du4}{12,9}{⼸、⼴}[HSK 5]
  \definition[个,种]{s.}{intensidade; força | magnitude; rigor; avidez}
\end{EntryWithPhonetic}

\begin{EntryWithPhonetic}{强化}{qiang2 hua4}{12,4}{⼸、⼔}[HSK 6]
  \definition{v.}{intensificar; fortalecer; consolidar; tornar mais forte, melhorar sua habilidade e nível}
\end{EntryWithPhonetic}

\begin{EntryWithPhonetic}{强烈}{qiang2lie4}{12,10}{⼸、⽕}[HSK 3]
  \definition{adj.}{muito forte; intenso; poderoso | violento; impetuoso; nível muito alto; atitude muito firme, sem espaço para mudanças | afiado; marcante; mostrado em contraste; muito claro}
\end{EntryWithPhonetic}

\begin{EntryWithPhonetic}{强势}{qiang2 shi4}{12,8}{⼸、⼒}[HSK 6]
  \definition*{adj.}{forte; poderoso; dominante}
  \definition{s.}{momento; ímpeto; grande impulso; forte impulso | força; influência dominante; forças poderosas}
\end{EntryWithPhonetic}

\begin{EntryWithPhonetic}{强壮}{qiang2 zhuang4}{12,6}{⼸、⼠}[HSK 6]
  \definition{s.}{(corpo) forte, poderoso, robusto, resistente}
  \definition{v.}{fortalecer; construir}
\end{EntryWithPhonetic}

\begin{EntryWithPhonetic}{墙}{qiang2}{14}{⼟}[HSK 2]
  \definition[面,堵,道]{s.}{parede; barreira ou perímetro construído com tijolos, pedras, etc. | qualquer coisa com a forma ou função de uma parede; a parte de um objeto que funciona como parede ou divisória}
  \definition{v.}{(gíria) bloquear (um website) (usado geralmente na voz passiva: 被墙)}
\end{EntryWithPhonetic}

\begin{EntryWithPhonetic}{墙壁}{qiang2 bi4}{14,16}{⼟、⼟}[HSK 5]
  \definition[面,堵,道]{s.}{parede; barreira ou perímetro construído com tijolos, pedras ou terra}
\end{EntryWithPhonetic}

\begin{EntryWithPhonetic}{墙纸}{qiang2zhi3}{14,7}{⼟、⽷}
  \definition{s.}{papel de parede}
\end{EntryWithPhonetic}

\begin{EntryWithPhonetic}{抢}{qiang3}{7}{⼿}[HSK 5]
  \definition{v.}{roubar; saquear | agarrar; apanhar; arrebatar | disputar; lutar por; ser o primeiro; competir para ser o primeiro | correr; apressar-se; fazer uma incursão | raspar; arranhar; raspar ou esfregar uma camada da superfície de um objeto}
  \seeref{qiang1}
\end{EntryWithPhonetic}

\begin{EntryWithPhonetic}{抢救}{qiang3jiu4}{7,11}{⼿、⽁}[HSK 5]
  \definition{v.}{salvar; resgatar; prestar de socorro ou assistência rápidos em situações de emergência | salvar; tomar medidas rápidas para evitar ou minimizar perdas iminentes.}
\end{EntryWithPhonetic}

\begin{EntryWithPhonetic}{抢掠}{qiang3lve4}{7,11}{⼿、⼿}
  \definition{s.}{saque | pilhagem}
  \definition{v.}{saquear | pilhar}
\end{EntryWithPhonetic}

\begin{EntryWithPhonetic}{强}{qiang3}{12}{⼸}
  \definition{v.}{fazer um esforço; esforçar-se}
  \seeref{jiang4}
  \seeref{qiang2}
\end{EntryWithPhonetic}

\begin{EntryWithPhonetic}{强迫}{qiang3po4}{12,8}{⼸、⾡}[HSK 5]
  \definition{v.}{impelir; forçar; impor; compelir; aplicar pessão para obedecer}
\end{EntryWithPhonetic}

\begin{EntryWithPhonetic}{悄}{qiao1}{10}{⼼}
  \definition{adj.}{quieto; silencioso}
  \seeref{qiao3}
\end{EntryWithPhonetic}

\begin{EntryWithPhonetic}{悄悄}{qiao1qiao1}{10,10}{⼼、⼼}[HSK 5]
  \definition{adv.}{silenciosamente; em silêncio; aos sussuros; sem som ou em voz baixa; com o mínimo de ruído possível}
\end{EntryWithPhonetic}

\begin{EntryWithPhonetic}{敲}{qiao1}{14}{⽁}[HSK 5]
  \definition{v.}{bater; dar uma pancada; golpear | explorar alguém; cobrar a mais; extorquir; chantagear | lembrar; criticar; alertar; advertir}
\end{EntryWithPhonetic}

\begin{EntryWithPhonetic}{敲门}{qiao1 men2}{14,3}{⽁、⾨}[HSK 5]
  \definition{v.}{bater na porta}
\end{EntryWithPhonetic}

\begin{EntryWithPhonetic}{桥}{qiao2}{10}{⽊}[HSK 3]
  \definition*{s.}{Sobrenome Qiao}
  \definition[座]{s.}{ponte; construção que atravessa a água conectando as duas margens}
\end{EntryWithPhonetic}

\begin{EntryWithPhonetic}{桥梁}{qiao2liang2}{10,11}{⽊、⽊}[HSK 6]
  \definition[座]{s.}{ponte; acesso; uma obra construída na superfície do rio, conectando as duas margens | ponte; metáfora para pessoas ou coisas que podem se comunicar}
\end{EntryWithPhonetic}

\begin{EntryWithPhonetic}{瞧}{qiao2}{17}{⽬}[HSK 5]
  \definition{v.}{ver; olhar | tratar; diagnosticar e tratar | ver; visitar; fazer uma visita}
\end{EntryWithPhonetic}

\begin{EntryWithPhonetic}{巧}{qiao3}{5}{⼯}[HSK 3]
  \definition{adj.}{habilidoso; engenhoso; esperto | oportuno; coincidente; fortuito | astuto; enganoso; enganador; traiçoeiro; ardiloso | (de mão, língua) hábil; loquaz}
  \definition{s.}{(tecnologia, artesanato) habilidade; destreza}
\end{EntryWithPhonetic}

\begin{EntryWithPhonetic}{巧合}{qiao3he2}{5,6}{⼯、⼝}
  \definition{s.}{coincidência; (coisas) coincidentes ou idênticas}
\end{EntryWithPhonetic}

\begin{EntryWithPhonetic}{巧克力}{qiao3ke4li4}{5,7,2}{⼯、⼗、⼒}[HSK 4]
  \definition[块,颗,盒,包]{s.}{Empréstimo linguístico: chocolate; alimentos feitos com cacau em pó como principal matéria-prima, açúcar e especiarias}
\end{EntryWithPhonetic}

\begin{EntryWithPhonetic}{巧妙}{qiao3miao4}{5,7}{⼯、⼥}[HSK 6]
  \definition{adj.}{inteligente; engenhoso; (método ou técnica, etc.) inteligente, além do comum}
\end{EntryWithPhonetic}

\begin{EntryWithPhonetic}{悄}{qiao3}{10}{⼼}
  \definition{adj.}{quieto; silencioso | triste; preocupado; aflito}
  \seeref{qiao1}
\end{EntryWithPhonetic}

\begin{EntryWithPhonetic}{壳}{qiao4}{7}{⼠}
  \definition[层,个]{s.}{Coloquial: concha | invólucro; caixa; carapaça | empresa de fachada (ou corporação) | superfície dura}
  \seeref{ke2}
\end{EntryWithPhonetic}

\begin{EntryWithPhonetic}{切}{qie1}{4}{⼑}[HSK 4]
  \definition{v.}{cortar; fatiar; separar itens com uma faca | cortar ou romper; truncar | Geometria: refere-se a quando uma linha, círculo ou superfície intercepta um círculo, arco ou esfera em apenas um ponto}
  \seeref{qie4}
\end{EntryWithPhonetic}

\begin{EntryWithPhonetic}{切割}{qie1ge1}{4,12}{⼑、⼑}
  \definition{v.}{esculpir; cortar algo com uma faca | cortar algo com máquina, fogo, arco voltaico; refere-se especificamente ao corte de materiais metálicos com máquinas-ferramentas ou à queima deles com arco elétrico, laser, etc.}
\end{EntryWithPhonetic}

\begin{EntryWithPhonetic}{茄}{qie2}{8}{⾋}
  \definition[只]{s.}{berinjela}
  \seeref{jia1}
\end{EntryWithPhonetic}

\begin{EntryWithPhonetic}{茄子}{qie2 zi5}{8,3}{⾋、⼦}[HSK 6]
  \definition{interj.}{Onomatopéia: ``xis'' fonético (ao ser fotografado), equivale ao ``diga xis''}
  \definition[个,根]{s.}{berinjela (fruto e planta)}
\end{EntryWithPhonetic}

\begin{EntryWithPhonetic}{且}{qie3}{5}{⼀}
  \definition*{s.}{Sobrenome Qie}
  \definition{adv.}{apenas; por enquanto | por um longo tempo}
  \definition{conj.}{mesmo; até; até mesmo; usado na primeira cláusula de uma frase complexa para expressar concessão, equivalente a 尚且 | ambos\dots e\dots; conecta adjetivos ou verbos para expressar relacionamento paralelo, equivalente a 而且 e 又……又……}
  \seealsoref{而且}{er2 qie3}
  \seealsoref{尚且}{shang4 qie3}
  \seealsoref{又……又……}{you4 you4}
\end{EntryWithPhonetic}

\begin{EntryWithPhonetic}{切}{qie4}{4}{⼑}
  \definition{adj.}{ansioso; sério | duro; severo; rude; áspero}
  \definition{adv.}{com certeza; certamente}
  \definition{s.}{limiar; degrau}
  \definition{v.}{ser prático ou realista | ajustar-se ou corresponder | ser próximo ou íntimo | cortar algo em pedaços com uma faca | tomar o pulso (medicina tradicional chinesa)}
  \seeref{qie1}
\end{EntryWithPhonetic}

\begin{EntryWithPhonetic}{切实}{qie4shi2}{4,8}{⼑、⼧}[HSK 6]
  \definition{adj.}{prático; viável; realista}
\end{EntryWithPhonetic}

\begin{EntryWithPhonetic}{亲}{qin1}{9}{⼇}[HSK 3]
  \definition{adj.}{parente próximo; relacionado por sangue; de ​​parentesco consanguíneo; parente consanguíneo mais próximo | querido; próximo; íntimo; relações próximas entre pessoas; sentimentos profundos (em oposição a 疏) | em si mesmo; pessoalmente}
  \definition[位]{s.}{pais; refere-se aos pais; também se refere apenas ao pai ou à mãe | parente; refere-se a pessoas que são relacionadas por sangue ou casamento| casal; casamento; refere-se ao casamento ou relacionamento conjugal | noiva; refere-se especificamente à noiva}
  \definition{v.}{beijar | (de países, partidos, etc.) a favor de; apoiar; estar perto de}
  \seeref{qing4}
  \seealsoref{疏}{shu1}
\end{EntryWithPhonetic}

\begin{EntryWithPhonetic}{亲爱}{qin1'ai4}{9,10}{⼇、⽖}[HSK 4]
  \definition{adj.}{querido; amado; termo carinhoso que expressa intimidade e afeto}
\end{EntryWithPhonetic}

\begin{EntryWithPhonetic}{亲密}{qin1mi4}{9,11}{⼇、⼧}[HSK 4]
  \definition{adj.}{próximo; íntimo; relacionamento afetuoso e próximo}
\end{EntryWithPhonetic}

\begin{EntryWithPhonetic}{亲切}{qin1qie4}{9,4}{⼇、⼑}[HSK 3]
  \definition{adj.}{gentil; cordial; cheio de sinceridade e cuidado, fazendo com que as pessoas se sintam acolhidas e acessíveis | próximo; íntimo; por familiaridade e afeição}
\end{EntryWithPhonetic}

\begin{EntryWithPhonetic}{亲人}{qin1 ren2}{9,2}{⼇、⼈}[HSK 3]
  \definition[个,位]{s.}{um membro da família; os pais, o cônjuge, os filhos, etc.; refere-se a parentes ou cônjuges | queridos; entes queridos; aqueles queridos para alguém; uma metáfora para pessoas que têm um relacionamento próximo e sentimentos profundos}
\end{EntryWithPhonetic}

\begin{EntryWithPhonetic}{亲属}{qin1 shu3}{9,12}{⼇、⼫}[HSK 6]
  \definition{s.}{parentes; cognatos}
\end{EntryWithPhonetic}

\begin{EntryWithPhonetic}{亲眼}{qin1 yan3}{9,11}{⼇、⽬}[HSK 6]
  \definition{adv.}{pessoalmente; com os próprios olhos}
\end{EntryWithPhonetic}

\begin{EntryWithPhonetic}{亲自}{qin1zi4}{9,6}{⼇、⾃}[HSK 3]
  \definition{adv.}{pessoalmente; em pessoa; si mesmo; fazer algo diretamente por si mesmo}
\end{EntryWithPhonetic}

\begin{EntryWithPhonetic}{侵}{qin1}{9}{⼈}
  \definition*{s.}{Sobrenome Qin}
  \definition{prep.}{aproximando-se; aproximar}
  \definition{v.}{invadir; intrometer-se em; infringir | aproximar-se (amanhecer)}
\end{EntryWithPhonetic}

\begin{EntryWithPhonetic}{侵犯}{qin1fan4}{9,5}{⼈、⽝}[HSK 6]
  \definition{v.}{violar; invadir; infringir; interferência ilegal com terceiros e violação de seus direitos | violar; fazer incursões; invadir o território de outro país}
\end{EntryWithPhonetic}

\begin{EntryWithPhonetic}{侵略}{qin1lve4}{9,11}{⼈、⽥}
  \definition{s.}{invasão}
  \definition{v.}{invadir}
\end{EntryWithPhonetic}

\begin{EntryWithPhonetic}{芹}{qin2}{7}{⾋}
  \definition[把,棵]{s.}{aipo | aipo chinês}
\end{EntryWithPhonetic}

\begin{EntryWithPhonetic}{芹菜}{qin2cai4}{7,11}{⾋、⾋}
  \definition{s.}{salsão}
\end{EntryWithPhonetic}

\begin{EntryWithPhonetic}{琴}{qin2}{12}{⽟}[HSK 5]
  \definition*{s.}{Sobrenome Qin}
  \definition[架,台]{s.}{cítara; qin; guqin (um instrumento de cordas dedilhadas com sete cordas, em alguns aspectos semelhante à cítara)  | nome genérico para certos instrumentos musicais}
\end{EntryWithPhonetic}

\begin{EntryWithPhonetic}{琴键}{qin2jian4}{12,13}{⽟、⾦}
  \definition{s.}{tecla de piano}
\end{EntryWithPhonetic}

\begin{EntryWithPhonetic}{禽}{qin2}{12}{⽱}
  \definition*{s.}{Sobrenome Qin}
  \definition[只]{s.}{aves; pássaros | termo genérico para aves e animais}
\end{EntryWithPhonetic}

\begin{EntryWithPhonetic}{勤}{qin2}{13}{⼒}
  \definition*{s.}{Sobrenome Qin}
  \definition{adj.}{diligente; industrial; trabalhador}
  \definition{adv.}{frequentemente}
  \definition{s.}{dever; serviço | presença; trabalhadores que chegam ao trabalho no horário especificado}
\end{EntryWithPhonetic}

\begin{EntryWithPhonetic}{勤奋}{qin2fen4}{13,8}{⼒、⼤}[HSK 5]
  \definition{adj.}{diligente; assíduo; trabalhador; descreve alguém que se esforça continuamente nos estudos ou no trabalho}
\end{EntryWithPhonetic}

\begin{EntryWithPhonetic}{擒}{qin2}{15}{⼿}
  \definition{v.}{capturar; pegar; apreender}
\end{EntryWithPhonetic}

\begin{EntryWithPhonetic}{擒获}{qin2huo4}{15,10}{⼿、⾋}
  \definition{v.}{apreender | capturar}
\end{EntryWithPhonetic}

\begin{EntryWithPhonetic}{青}{qing1}{8}{⾭}[HSK 5][Kangxi 174]
  \definition*{s.}{Província de Qinghai, abreviação de 青海 | Sobrenome Qing}
  \definition{adj.}{azul ou verde | preto | jovens (pessoas)}
  \definition{s.}{grama verde | colheitas jovens (não maduras) | tiras de bambu verde}
  \seealsoref{青海}{qing1hai3}
\end{EntryWithPhonetic}

\begin{EntryWithPhonetic}{青菜}{qing1cai4}{8,11}{⾭、⾋}
  \definition{s.}{verduras}
\end{EntryWithPhonetic}

\begin{EntryWithPhonetic}{青春}{qing1chun1}{8,9}{⾭、⽇}[HSK 4]
  \definition[个]{s.}{juventude; jovialidade}
\end{EntryWithPhonetic}

\begin{EntryWithPhonetic}{青海}{qing1hai3}{8,10}{⾭、⽔}
  \definition*{s.}{Província de Qinghai}
\end{EntryWithPhonetic}

\begin{EntryWithPhonetic}{青椒}{qing1jiao1}{8,12}{⾭、⽊}
  \definition{s.}{pimenta verde}
\end{EntryWithPhonetic}

\begin{EntryWithPhonetic}{青年}{qing1 nian2}{8,6}{⾭、⼲}[HSK 2]
  \definition[个,位,名,些]{s.}{juventude; jovem; refere-se ao período entre os 15 e os 30 anos de idade.}
\end{EntryWithPhonetic}

\begin{EntryWithPhonetic}{青年节}{qing1nian2jie2}{8,6,5}{⾭、⼲、⾋}
  \definition*{s.}{Dia da Juventude (4 de maio)}
\end{EntryWithPhonetic}

\begin{EntryWithPhonetic}{青少年}{qing1shao4nian2}{8,4,6}{⾭、⼩、⼲}[HSK 2]
  \definition[位,名,个,些]{s.}{adolescentes}
\end{EntryWithPhonetic}

\begin{EntryWithPhonetic}{青天}{qing1tian1}{8,4}{⾭、⼤}
  \definition{s.}{céu claro, limpo ou azul}
\end{EntryWithPhonetic}

\begin{EntryWithPhonetic}{青铜}{qing1tong2}{8,11}{⾭、⾦}
  \definition{s.}{bronze (liga de cobre, 銅, e estanho, 锡)}
\end{EntryWithPhonetic}

\begin{EntryWithPhonetic}{青蛙}{qing1wa1}{8,12}{⾭、⾍}
  \definition{adj.}{(gíria velha) cara feio}
  \definition[只]{s.}{sapo}
\end{EntryWithPhonetic}

\begin{EntryWithPhonetic}{青玉米}{qing1yu4mi3}{8,5,6}{⾭、⽟、⽶}
  \definition{s.}{milho verde}
\end{EntryWithPhonetic}

\begin{EntryWithPhonetic}{轻}{qing1}{9}{⾞}[HSK 2]
  \definition{adj.}{de pouco peso; leve (oposto de 重) | (de carga, equipamento, etc.) pequeno; simples | pequeno em número, grau, etc. | não sério; relaxante; leve | sem importância | suave; delicado | levianos, crédulos | leve; peso leve; densidade baixa | leve; descontraído; fácil | imprudente; descuidado | inconstante; frívolo}
  \definition{v.}{menosprezar; subestimar}
  \seealsoref{重}{zhong4}
\end{EntryWithPhonetic}

\begin{EntryWithPhonetic}{轻松}{qing1song1}{9,8}{⾞、⽊}[HSK 4]
  \definition{adj.}{leve; relaxado; livre de fardos; não nervoso; não cansado}
  \definition{v.}{sentir-se livre de fardos; não se sentir nervoso ou cansado}
\end{EntryWithPhonetic}

\begin{EntryWithPhonetic}{轻易}{qing1yi4}{9,8}{⾞、⽇}[HSK 4]
  \definition{adv.}{facilmente; prontamente | facilmente; precipitadamente; indica que uma ação é realizada casualmente, geralmente usado em frases negativas}
\end{EntryWithPhonetic}

\begin{EntryWithPhonetic}{倾}{qing1}{10}{⼈}
  \definition{s.}{desvio; tendência}
  \definition{v.}{inclinar; inclinar-se; dobrar-se | colapsar | virar e despejar; esvaziar | fazer tudo o que puder; usar todos os recursos | sobrecarregar; dominar; dominar | admirar | superar}
\end{EntryWithPhonetic}

\begin{EntryWithPhonetic}{倾城}{qing1cheng2}{10,9}{⼈、⼟}
  \definition{adj.}{sedutora (mulher)}
  \definition{adv.}{de todo o lugar | vindo de todos os lugares}
  \definition{v.}{arruinar e derrubar o estado}
\end{EntryWithPhonetic}

\begin{EntryWithPhonetic}{倾向}{qing1xiang4}{10,6}{⼈、⼝}[HSK 6]
  \definition{s.}{tendência; desvio; inclinação; direção do desenvolvimento}
  \definition{v.}{preferir; estar inclinado a; concordar com uma determinada opinião}
\end{EntryWithPhonetic}

\begin{EntryWithPhonetic}{清}{qing1}{11}{⽔}[HSK 6]
  \definition*{s.}{Dinastia Qing (1644-1911) | Sobrenome Qing}
  \definition{adj.}{claro; não misturado; (líquido ou gasoso) puro e sem mistura (em oposição a 浊) | silencioso; quieto | justo e honesto | distinto; claro; esclarecido | simples; puro, sem qualquer adulteração ou combinação | limpo; puro}
  \definition{v.}{limpar; tornar limpo | resolver; esclarecer; pagar; liquidar | contar; inspecionar}
  \seealsoref{浊}{zhuo2}
\end{EntryWithPhonetic}

\begin{EntryWithPhonetic}{清唱}{qing1chang4}{11,11}{⽔、⼝}
  \definition{v.}{cantar à capela}
\end{EntryWithPhonetic}

\begin{EntryWithPhonetic}{清彻}{qing1che4}{11,7}{⽔、⼻}
  \variantof{清澈}
\end{EntryWithPhonetic}

\begin{EntryWithPhonetic}{清澈}{qing1che4}{11,15}{⽔、⽔}
  \definition{adj.}{claro | límpido}
\end{EntryWithPhonetic}

\begin{EntryWithPhonetic}{清晨}{qing1chen2}{11,11}{⽔、⽇}[HSK 5]
  \definition{s.}{matinal; manhã cedo; geralmente se refere ao período do amanhecer até logo após o nascer do sol}
\end{EntryWithPhonetic}

\begin{EntryWithPhonetic}{清楚}{qing1chu5}{11,13}{⽔、⽊}[HSK 2]
  \definition{adj.}{claro; distinto; compreensível; organizado; fácil de identificar e entender | plenamente consciente de; claro sobre}
  \definition{v.}{ter clareza sobre; compreender; ação que expressa compreensão e conhecimento}
\end{EntryWithPhonetic}

\begin{EntryWithPhonetic}{清洁}{qing1jie2}{11,9}{⽔、⽔}[HSK 6]
  \definition{adj.}{limpo; sem poeira, gordura, etc.}
  \definition{v.}{limpar}
\end{EntryWithPhonetic}

\begin{EntryWithPhonetic}{清洁工}{qing1 jie2 gong1}{11,9,3}{⽔、⽔、⼯}[HSK 6]
  \definition{s.}{coletor de lixo; trabalhador de saneamento; limpador de rua; trabalhadores envolvidos na limpeza do ambiente, remoção de lixo e fezes, etc.}
\end{EntryWithPhonetic}

\begin{EntryWithPhonetic}{清理}{qing1li3}{11,11}{⽔、⽟}[HSK 5]
  \definition{v.}{esclarecer; resolver; verificar; colocar em ordem; organizar tudo e jogar fora o que não for útil}
\end{EntryWithPhonetic}

\begin{EntryWithPhonetic}{清凉}{qing1liang2}{11,10}{⽔、⼎}
  \definition{adj.}{fresco | refrescante | (roupa) ousada, reveladora}
\end{EntryWithPhonetic}

\begin{EntryWithPhonetic}{清明节}{qing1 ming2 jie2}{11,8,5}{⽔、⽇、⾋}[HSK 6]
  \definition*{s.}{Qingming ou Festival do Brilho Puro ou Dia da Varredura de Túmulos, Dia dos Finados (uma das 24~divisões do ano solar no calendário lunar chinês:~dia~4 ou 5~de~abril solar)}
\end{EntryWithPhonetic}

\begin{EntryWithPhonetic}{清爽}{qing1shuang3}{11,11}{⽔、⽘}
  \definition{adj.}{refrescante | relaxado}
\end{EntryWithPhonetic}

\begin{EntryWithPhonetic}{清晰}{qing1xi1}{11,12}{⽔、⽇}
  \definition{adj.}{claro | distinto}
\end{EntryWithPhonetic}

\begin{EntryWithPhonetic}{清洗}{qing1 xi3}{11,9}{⽔、⽔}[HSK 6]
  \definition{v.}{enxaguar; lavar; limpar | purgar; limpar | eliminar}
\end{EntryWithPhonetic}

\begin{EntryWithPhonetic}{清醒}{qing1xing3}{11,16}{⽔、⾣}[HSK 4]
  \definition{adj.}{sóbrio; lúcido}
  \definition{v.}{recuperar a consciência; recuperar-se de um coma}
\end{EntryWithPhonetic}

\begin{EntryWithPhonetic}{蜻}{qing1}{14}{⾍}
  \definition[只]{s.}{libélula, 蜻蜓}
  \seealsoref{蜻蜓}{qing1ting2}
\end{EntryWithPhonetic}

\begin{EntryWithPhonetic}{蜻蜓}{qing1ting2}{14,12}{⾍、⾍}
  \definition{s.}{libélula}
\end{EntryWithPhonetic}

\begin{EntryWithPhonetic}{蜻蝏}{qing1ting2}{14,15}{⾍、⾍}
  \variantof{蜻蜓}
\end{EntryWithPhonetic}

\begin{EntryWithPhonetic}{情}{qing2}{11}{⼼}
  \definition{s.}{sentimento; afeição | amor; paixão | paixão sexual; luxúria | favor; gentileza | situação; circunstâncias; condição | razão; sentido | sensibilidades; sentimentos}
\end{EntryWithPhonetic}

\begin{EntryWithPhonetic}{情感}{qing2 gan3}{11,13}{⼼、⼼}[HSK 3]
  \definition[份]{s.}{emoção; sentimento | afeição; apego; reações psicológicas positivas ou negativas a estímulos externos, como gosto, raiva, tristeza, medo, amor, nojo, etc.}
\end{EntryWithPhonetic}

\begin{EntryWithPhonetic}{情节}{qing2jie2}{11,5}{⼼、⾋}[HSK 5]
  \definition[个,段]{s.}{enredo; trama; desenrolar específico dos acontecimentos | circunstância; detalhes do crime ou erro | enredo; roteiro; refere-se especificamente ao processo de desenvolvimento e evolução dos conflitos e contradições em obras literárias narrativas}
\end{EntryWithPhonetic}

\begin{EntryWithPhonetic}{情景}{qing2jing3}{11,12}{⼼、⽇}[HSK 4]
  \definition[个,幕,种]{s.}{cena; vista; circunstâncias}
\end{EntryWithPhonetic}

\begin{EntryWithPhonetic}{情况}{qing2kuang4}{11,7}{⼼、⼎}[HSK 3]
  \definition[种,个,些]{s.}{condição; situação; circunstâncias; estado das coisas | mudanças notáveis e impactantes}
\end{EntryWithPhonetic}

\begin{EntryWithPhonetic}{情形}{qing2xing2}{11,7}{⼼、⼺}[HSK 5]
  \definition[个,种]{s.}{situação; condição; circunstâncias; estado de coisas; a situação específica das coisas}
\end{EntryWithPhonetic}

\begin{EntryWithPhonetic}{情绪}{qing2xu4}{11,11}{⼼、⽷}[HSK 6]
  \definition[种,片,股,丝]{s.}{mau humor; depressão; um sentimento ruim no coração, especialmente um estado mental desagradável quando se sente injusto | emoção; humor; moral; sentimento; o estado mental de uma pessoa ao longo de um período de tempo}
\end{EntryWithPhonetic}

\begin{EntryWithPhonetic}{晴}{qing2}{12}{⽇}[HSK 2]
  \definition{adj.}{ensolarado; bom; claro; não há nuvens no céu ou há poucas nuvens}
\end{EntryWithPhonetic}

\begin{EntryWithPhonetic}{晴朗}{qing2lang3}{12,10}{⽇、⽉}[HSK 5]
  \definition{adj.}{bom; claro; ensolarado; céu limpo e sem nuvens}
\end{EntryWithPhonetic}

\begin{EntryWithPhonetic}{晴天}{qing2 tian1}{12,4}{⽇、⼤}[HSK 2]
  \definition[个]{s.}{dia ensolarado; tempo sem nuvens ou com poucas nuvens; em meteorologia, refere-se a um tempo em que a cobertura de nuvens no céu é inferior a 10\%}
\end{EntryWithPhonetic}

\begin{EntryWithPhonetic}{请}{qing3}{10}{⾔}[HSK 1]
  \definition*{s.}{Sobrenome Qing}
  \definition{v.}{solicitar; perguntar | convidar; envolver | por favor; uma expressão educada usada quando você quer que alguém faça algo | comprar coisas sagradas para sacrifício, como incenso, velas, cavalos de papel e santuários de Buda; superstição se refere à compra de estátuas de Buda, santuários, etc. | entreter}
\end{EntryWithPhonetic}

\begin{EntryWithPhonetic}{请假}{qing3/jia4}{10,11}{⾔、⼈}[HSK 1]
  \definition{v.+compl.}{pedir licença para sair; solicitar permissão para não trabalhar ou estudar por um determinado período de tempo devido a doença ou outros motivos}
\end{EntryWithPhonetic}

\begin{EntryWithPhonetic}{请假条}{qing3jia4tiao2}{10,11,7}{⾔、⼈、⽊}
  \definition{s.}{pedido de licença de ausência (do trabalho ou da escola)}
\end{EntryWithPhonetic}

\begin{EntryWithPhonetic}{请教}{qing3jiao4}{10,11}{⾔、⽁}[HSK 3]
  \definition{v.}{consultar; pedir conselho}
\end{EntryWithPhonetic}

\begin{EntryWithPhonetic}{请进}{qing3 jin4}{10,7}{⾔、⾡}[HSK 1]
  \definition{v.}{por favor entre; convidar alguém para um espaço ou lugar}
\end{EntryWithPhonetic}

\begin{EntryWithPhonetic}{请客}{qing3/ke4}{10,9}{⾔、⼧}[HSK 2]
  \definition{v.+compl.}{receber convidados; hospedar convidados | oferecer; convidar; pagar a conta; arcar com os custos; convidar alguém para comer, tomar chá, etc.}
\end{EntryWithPhonetic}

\begin{EntryWithPhonetic}{请求}{qing3qiu2}{10,7}{⾔、⽔}[HSK 2]
  \definition[个,次]{s.}{pedido; petição; solicitação; refere-se à exigência apresentada}
  \definition{v.}{pedir; solicitar; requerer; peticionar; fazer uma solicitação e pedir que a outra parte concorde com ela}
\end{EntryWithPhonetic}

\begin{EntryWithPhonetic}{请问}{qing3 wen4}{10,6}{⾔、⾨}[HSK 1]
  \definition{expr.}{Com licença, posso perguntar\dots? (para perguntar por qualquer coisa); uma maneira educada de pedir para alguém responder a uma pergunta}
\end{EntryWithPhonetic}

\begin{EntryWithPhonetic}{请坐}{qing3 zuo4}{10,7}{⾔、⼟}[HSK 1]
  \definition{v.}{por favor, sente-se; convidar outras pessoas para sentar ou descansar}
\end{EntryWithPhonetic}

\begin{EntryWithPhonetic}{庆}{qing4}{6}{⼴}
  \definition*{s.}{Sobrenome Qing}
  \definition{s.}{celebração | ocasião para celebração; um aniversário que vale a pena comemorar}
  \definition{v.}{celebrar; felicitar; comemorar}
\end{EntryWithPhonetic}

\begin{EntryWithPhonetic}{庆祝}{qing4zhu4}{6,9}{⼴、⽰}[HSK 3]
  \definition{v.}{celebrar; comemorar; festejar; realizar atividades para comemorar ou celebrar festivais comuns e eventos felizes}
\end{EntryWithPhonetic}

\begin{EntryWithPhonetic}{亲}{qing4}{9}{⼇}
  \definition{s.}{parentes por afinidade; parentes por casamento}
  \seeref{qin1}
\end{EntryWithPhonetic}

\begin{EntryWithPhonetic}{穷}{qiong2}{7}{⽳}[HSK 4]
  \definition{adj.}{remoto; isolado; de difícil acesso | pobre; atingido pela pobreza | situação difícil, sem saída}
  \definition{adv.}{completamente | extremamente}
  \definition{v.}{exaurir; esgotar; consmir | ir até o fim; perseguir completamente perseguido; sondar profundamente | gastar}
\end{EntryWithPhonetic}

\begin{EntryWithPhonetic}{穷人}{qiong2 ren2}{7,2}{⽳、⼈}[HSK 4]
  \definition[个]{s.}{os pobres; pessoas pobres}
\end{EntryWithPhonetic}

\begin{EntryWithPhonetic}{丘}{qiu1}{5}{⼀}
  \definition*{s.}{Sobrenome Qiu}
  \definition[个]{s.}{monte; outeiro | (literário) sepultura}
\end{EntryWithPhonetic}

\begin{EntryWithPhonetic}{丘陵}{qiu1ling2}{5,10}{⼀、⾩}
  \definition{s.}{colinas}
\end{EntryWithPhonetic}

\begin{EntryWithPhonetic}{秋}{qiu1}{9}{⽲}
  \definition*{s.}{Sobrenome Qiu}
  \definition{s.}{outono | época da colheita; a estação em que as colheitas amadurecem; colheitas maduras no outono | ano; refere-se a um ano | um período de tempo (geralmente conturbado)}
\end{EntryWithPhonetic}

\begin{EntryWithPhonetic}{秋季}{qiu1 ji4}{9,8}{⽲、⼦}[HSK 4]
  \definition[个]{s.}{outono; terceiro trimestre do ano, segundo o costume chinês, refere-se ao período de três meses entre o outono e o inverno, também se refere aos sétimo, oitavo e nono meses do calendário lunar}
\end{EntryWithPhonetic}

\begin{EntryWithPhonetic}{秋天}{qiu1 tian1}{9,4}{⽲、⼤}[HSK 2]
  \definition[个,段,季,番]{s.}{outono}
\end{EntryWithPhonetic}

\begin{EntryWithPhonetic}{仇}{qiu2}{4}{⼈}
  \definition*{s.}{Sobrenome Qiu}
  \definition{s.}{Literário: cônjuge; esposa; companheira}
  \seeref{chou2}
\end{EntryWithPhonetic}

\begin{EntryWithPhonetic}{囚}{qiu2}{5}{⼞}
  \definition[个,群,位,名,些,批]{s.}{prisioneiro; condenado}
  \definition{v.}{aprisionar}
\end{EntryWithPhonetic}

\begin{EntryWithPhonetic}{囚犯}{qiu2fan4}{5,5}{⼞、⽝}
  \definition[名]{s.}{prisioneiro; condenado}
\end{EntryWithPhonetic}

\begin{EntryWithPhonetic}{求}{qiu2}{7}{⽔}[HSK 2]
  \definition*{s.}{Sobrenome Qiu}
  \definition{v.}{implorar; solicitar; suplicar; rogar | lutar por; buscar; investigar | tentar; procurar; tentar obter | demandar}
\end{EntryWithPhonetic}

\begin{EntryWithPhonetic}{求职}{qiu2 zhi2}{7,11}{⽔、⽿}[HSK 6]
  \definition{v.}{procurar emprego; candidatar-se a um emprego; encontrar um emprego}
\end{EntryWithPhonetic}

\begin{EntryWithPhonetic}{球}{qiu2}{11}{⽟}[HSK 1]
  \definition[个,颗,筐]{s.}{esfera; globo; equipamento de jogo antigo, objeto tridimensional circular, feito de couro, recheado com penas, para ser chutado com os pés ou batido com um bastão | qualquer coisa com formato de bola; algo esférico ou quase esférico | bola; refere-se a certos artigos esportivos (geralmente redondos e tridimensionais) | jogo; partida; referência a esportes com bola | o Globo; a Terra; referindo-se especificamente à Terra}
\end{EntryWithPhonetic}

\begin{EntryWithPhonetic}{球场}{qiu2 chang3}{11,6}{⽟、⼟}[HSK 2]
  \definition[个,座]{s.}{quadra; campo; terreno para jogos com bola; campos para a prática de esportes com bola, como basquete, futebol, tênis e vôlei, cuja forma, tamanho e equipamentos variam de acordo com as exigências de cada esporte}
\end{EntryWithPhonetic}

\begin{EntryWithPhonetic}{球队}{qiu2 dui4}{11,4}{⽟、⾩}[HSK 2]
  \definition[个,支]{s.}{equipe (basquete, futebol, etc.); equipe de atletas formada para competições esportivas com bola, como times de basquete, futebol, etc.}
\end{EntryWithPhonetic}

\begin{EntryWithPhonetic}{球迷}{qiu2mi2}{11,9}{⽟、⾡}[HSK 3]
  \definition[个,位,名,些]{s.}{fã (de esportes de bola); pessoas obcecadas por jogar ou assistir jogos de bola}
\end{EntryWithPhonetic}

\begin{EntryWithPhonetic}{球拍}{qiu2 pai1}{11,8}{⽟、⼿}[HSK 6]
  \definition[支]{s.}{(tênis, badminton, etc.) raquete}
\end{EntryWithPhonetic}

\begin{EntryWithPhonetic}{球鞋}{qiu2 xie2}{11,15}{⽟、⾰}[HSK 2]
  \definition[双,只,款]{s.}{tênis de ginástica; tênis de tênis; tênis esportivos}
\end{EntryWithPhonetic}

\begin{EntryWithPhonetic}{球星}{qiu2 xing1}{11,9}{⽟、⽇}[HSK 6]
  \definition[位,名]{s.}{estrela do esporte (esporte com bola)}
\end{EntryWithPhonetic}

\begin{EntryWithPhonetic}{球员}{qiu2 yuan2}{11,7}{⽟、⼝}[HSK 6]
  \definition[名,位,个]{s.}{Esporte: jogador | membro do clube esportivo}
\end{EntryWithPhonetic}

\begin{EntryWithPhonetic}{区}{qu1}{4}{⼖}[HSK 3]
  \definition{s.}{área; distrito; região; zona; uma determinada área em terra, água ou ar | uma divisão administrativa; as divisões administrativas incluem regiões autônomas étnicas de nível provincial, distritos municipais e de condado e distritos de condado; grandes regiões administrativas, regiões, zonas especiais e regiões administrativas especiais}
  \definition{v.}{classificar; subdividir; distinguir}
  \seeref{ou1}
\end{EntryWithPhonetic}

\begin{EntryWithPhonetic}{区别}{qu1bie2}{4,7}{⼖、⼑}[HSK 3]
  \definition[种,个]{s.}{diferença; distinção; discriminação}
  \definition{v.}{distinguir; diferenciar; fazer distinção entre}
\end{EntryWithPhonetic}

\begin{EntryWithPhonetic}{区分}{qu1fen1}{4,4}{⼖、⼑}[HSK 6]
  \definition{v.}{discriminar; diferenciar; distinguir; comparar dois ou mais objetos; reconhecer suas diferenças}
\end{EntryWithPhonetic}

\begin{EntryWithPhonetic}{区域}{qu1yu4}{4,11}{⼖、⼟}[HSK 5]
  \definition[片,块,个]{s.}{área; setor; região; faixa; inclui áreas regionais com condições naturais, culturais, administrativas, etc.}
\end{EntryWithPhonetic}

\begin{EntryWithPhonetic}{曲}{qu1}{6}{⽈}
  \definition*{s.}{Sobrenome Qu}
  \definition{adj.}{dobrado; curvado (oposto a 直) | errado; injustificável | torto}
  \definition{v.}{dobrar | torcer}
  \seealsoref{直}{zhi2}
\end{EntryWithPhonetic}

\begin{EntryWithPhonetic}{曲棍球}{qu1gun4qiu2}{6,12,11}{⽈、⽊、⽟}
  \definition{s.}{hóquei em campo; hóquei | bola de hóquei}
\end{EntryWithPhonetic}

\begin{EntryWithPhonetic}{驱}{qu1}{7}{⾺}
  \definition{v.}{dirigir (um cavalo, um carro, etc.) | expulsar; dispersar | correr rápido}
\end{EntryWithPhonetic}

\begin{EntryWithPhonetic}{屈}{qu1}{8}{⼫}
  \definition*{s.}{Sobrenome Qu}
  \definition[个]{s.}{injustiça; tratamento injusto | erro; queixa; injustiça}
  \definition{v.}{dobrar; curvar; encurvar | subjugar; submeter | tratar mal; tratar injustamente (ou deslealmente) | estar errado}
\end{EntryWithPhonetic}

\begin{EntryWithPhonetic}{屈原}{qu1yuan2}{8,10}{⼫、⼚}
  \definition*{s.}{Qu Yuan, poeta, é uma figura histórica famosa na cultura chinesa que viveu durante o Período dos Reinos Combatentes (340-278 a.C.).}
\end{EntryWithPhonetic}

\begin{EntryWithPhonetic}{趋}{qu1}{12}{⾛}
  \definition{v.}{apressar-se | tender para; tender a se tornar | (de um ganso, cobra, etc.) estalar a cabeça e morder as pessoas}
\end{EntryWithPhonetic}

\begin{EntryWithPhonetic}{趋势}{qu1shi4}{12,8}{⾛、⼒}[HSK 4]
  \definition{s.}{tendência; tendência; direção; impulso das coisas que se movem em uma direção ou outra}
\end{EntryWithPhonetic}

\begin{EntryWithPhonetic}{渠}{qu2}{11}{⽊}
  \definition*{s.}{Sobrenome Qu}
  \definition{adj.}{Literário: grande}
  \definition{pron.}{Dialeto: ele; ela}
  \definition[条]{s.}{canal; vala; fosso; trincheira | borda externa da roda | escudo}
\end{EntryWithPhonetic}

\begin{EntryWithPhonetic}{渠道}{qu2dao4}{11,12}{⽊、⾡}[HSK 6]
  \definition[条,个,种]{s.}{vala de irrigação; os cursos de água escavados pelos trabalhadores para drenagem e irrigação | maneira; meio; caminho}
\end{EntryWithPhonetic}

\begin{EntryWithPhonetic}{取}{qu3}{8}{⼜}[HSK 2]
  \definition{v.}{pegar; obter; buscar; pegar de um lugar; pegar nas mãos | visar; procurar; obter; provocar | adotar; assumir; escolher; selecionar}
\end{EntryWithPhonetic}

\begin{EntryWithPhonetic}{取得}{qu3 de2}{8,11}{⼜、⼻}[HSK 2]
  \definition{v.}{ganhar; adquirir; obter; ser o primeiro a conseguir}
\end{EntryWithPhonetic}

\begin{EntryWithPhonetic}{取款}{qu3kuan3}{8,12}{⼜、⽋}[HSK 6]
  \definition{v.}{sacar dinheiro (de um banco); retirar o dinheiro que você depositou (geralmente se refere a retirar dinheiro do banco)}
\end{EntryWithPhonetic}

\begin{EntryWithPhonetic}{取款机}{qu3 kuan3 ji1}{8,12,6}{⼜、⽋、⽊}[HSK 6]
  \definition{s.}{ATM; caixa eletrônico; um caixa eletrônico é uma máquina que pode concluir automaticamente operações bancárias, como saques e consultas de saldo}
\end{EntryWithPhonetic}

\begin{EntryWithPhonetic}{取胜}{qu3sheng4}{8,9}{⼜、⾁}
  \definition{v.}{prevalecer sobre os oponentes | marcar uma vitória}
\end{EntryWithPhonetic}

\begin{EntryWithPhonetic}{取水}{qu3shui3}{8,4}{⼜、⽔}
  \definition{v.}{obter água (de um poço, etc.)}
\end{EntryWithPhonetic}

\begin{EntryWithPhonetic}{取现}{qu3xian4}{8,8}{⼜、⾒}
  \definition{v.}{sacar dinheiro}
\end{EntryWithPhonetic}

\begin{EntryWithPhonetic}{取消}{qu3xiao1}{8,10}{⼜、⽔}[HSK 3]
  \definition{v.}{cancelar; suspender; anular; abolir; revogar; rescindir; tornar o sistema original, regulamentos, qualificações, direitos, etc. inválidos}
\end{EntryWithPhonetic}

\begin{EntryWithPhonetic}{取悦}{qu3yue4}{8,10}{⼜、⼼}
  \definition{v.}{tentar agradar}
\end{EntryWithPhonetic}

\begin{EntryWithPhonetic}{厺}{qu4}{5}{⼤}
  \variantof{去}
\end{EntryWithPhonetic}

\begin{EntryWithPhonetic}{去}{qu4}{5}{⼛}[HSK 1]
  \definition{adj.}{passado; último; refere-se ao tempo passado (um ano)}
  \definition{adv.}{muito; extremamente; usado depois de adjetivos como 大, 多 e 远, significa 极 ou 非常}
  \definition{s.}{tom descendente, um dos quatro tons do chinês clássico e o quarto tom na pronúncia padrão do chinês moderno}
  \definition{v.}{ir; partir; sair | estar separado de | perder | remover; livrar-se de | ir (a algum lugar) para fazer algo; sair do local onde o interlocutor se encontra para outro lugar (oposto a 来) | ir para; estar indo para (fazer algo lá); usado antes de outro verbo para indicar fazer algo | desempenhar o papel de; representar o papel de; interpretar papéis em óperas | enviar; fazer ir; despachar}
  \definition{v.aux.}{usado entre uma frase verbal (ou frase preposicional) e um verbo para indicar que o primeiro é um método ou atitude e o último é um propósito | usado depois de um verbo para indicar que a ação está longe da localização do falante}
  \seealsoref{大}{da4}
  \seealsoref{多}{duo1}
  \seealsoref{非常}{fei1chang2}
  \seealsoref{极}{ji2}
  \seealsoref{来}{lai2}
  \seealsoref{远}{yuan3}
\end{EntryWithPhonetic}

\begin{EntryWithPhonetic}{去掉}{qu4 diao4}{5,11}{⼛、⼿}[HSK 6]
  \definition{v.}{livrar-se de; tirar; acabar com; abandonar; erradicar}
\end{EntryWithPhonetic}

\begin{EntryWithPhonetic}{去年}{qu4nian2}{5,6}{⼛、⼲}[HSK 1]
  \definition{s.}{ano passado}
\end{EntryWithPhonetic}

\begin{EntryWithPhonetic}{去世}{qu4shi4}{5,5}{⼛、⼀}[HSK 3]
  \definition{v.}{(usado apenas para adultos, com conotações solenes) morrer; falecer; deixar este mundo}
\end{EntryWithPhonetic}

\begin{EntryWithPhonetic}{去死}{qu4si3}{5,6}{⼛、⽍}
  \definition{interj.}{Caia morto! | Vá para o Inferno!}
\end{EntryWithPhonetic}

\begin{EntryWithPhonetic}{圈}{quan1}{11}{⼞}[HSK 4]
  \definition[个]{s.}{anel; círculo; refere-se a algo em forma de anel | domínio; grupo; escopo; círculo(s)}
  \definition{v.}{cercar; rodear; circundar | marcar com um círculo}
  \seeref{juan1}
  \seeref{juan4}
\end{EntryWithPhonetic}

\begin{EntryWithPhonetic}{圈粉}{quan1fen3}{11,10}{⼞、⽶}
  \definition{s.}{(neologismo, coloquial) ganhar alguém como fã, obter novos fãs}
\end{EntryWithPhonetic}

\begin{EntryWithPhonetic}{全}{quan2}{6}{⼊}[HSK 2]
  \definition*{s.}{Sobrenome Quan}
  \definition{adj.}{completo; total; inteiro}
  \definition{adv.}{inteiramente; totalmente; completamente; significa 100\%; equivalente a 完全 ou 全然}
  \definition{v.}{manter intacto; tornar perfeito ou completo; completar}
  \seealsoref{全然}{quan2ran2}
  \seealsoref{完全}{wan2quan2}
\end{EntryWithPhonetic}

\begin{EntryWithPhonetic}{全部}{quan2bu4}{6,10}{⼊、⾢}[HSK 2]
  \definition{adv.}{tudo; total; inteiro; completo; aplica-se a todos, sem exceção}
  \definition{s.}{totalidade; total; integridade; a soma de todas as partes; o todo}
\end{EntryWithPhonetic}

\begin{EntryWithPhonetic}{全场}{quan2 chang3}{6,6}{⼊、⼟}[HSK 3]
  \definition{s.}{toda a audiência; todos os presentes; todo o público}
\end{EntryWithPhonetic}

\begin{EntryWithPhonetic}{全称特命全权大使}{quan2cheng1 te4ming4 quan2quan2 da4shi3}{6,10,10,8,6,6,3,8}{⼊、⽲、⽜、⼝、⼊、⽊、⼤、⼈}
  \definition*{s.}{Embaixador Extraordinário e Plenipotenciário}
\end{EntryWithPhonetic}

\begin{EntryWithPhonetic}{全都}{quan2 dou1}{6,10}{⼊、⾢}[HSK 5]
  \definition{adv.}{tudo; todos; sem exceção}
\end{EntryWithPhonetic}

\begin{EntryWithPhonetic}{全都不}{quan2dou1 bu4}{6,10,4}{⼊、⾢、⼀}
  \definition{adj.}{nada; nenhum; nenhum deles; nada disso}
\end{EntryWithPhonetic}

\begin{EntryWithPhonetic}{全国}{quan2 guo2}{6,8}{⼊、⼞}[HSK 2]
  \definition{s.}{toda a nação (ou país); em todo o país; em todo o território nacional | toda a nação; todo o país}
\end{EntryWithPhonetic}

\begin{EntryWithPhonetic}{全家}{quan2 jia1}{6,10}{⼊、⼧}[HSK 2]
  \definition{s.}{toda a família; a família inteira}
\end{EntryWithPhonetic}

\begin{EntryWithPhonetic}{全力}{quan2 li4}{6,2}{⼊、⼒}[HSK 6]
  \definition{s.}{exercendo todos os seus esforços; energia ou força total; toda força ou energia}
\end{EntryWithPhonetic}

\begin{EntryWithPhonetic}{全面}{quan2mian4}{6,9}{⼊、⾯}[HSK 3]
  \definition{adj.}{geral; completo; abrangente; onipotente}
  \definition{s.}{todos os aspectos; cada aspecto}
  \seealsoref{片面}{pian4mian4}
\end{EntryWithPhonetic}

\begin{EntryWithPhonetic}{全年}{quan2 nian2}{6,6}{⼊、⼲}[HSK 2]
  \definition{s.}{ano inteiro | anual; todo ano}
\end{EntryWithPhonetic}

\begin{EntryWithPhonetic}{全球}{quan2 qiu2}{6,11}{⼊、⽟}[HSK 3]
  \definition[门]{s.}{o mundo inteiro; a Terra inteira}
\end{EntryWithPhonetic}

\begin{EntryWithPhonetic}{全然}{quan2ran2}{6,12}{⼊、⽕}
  \definition{adv.}{completamente; inteiramente}
\end{EntryWithPhonetic}

\begin{EntryWithPhonetic}{全身}{quan2 shen1}{6,7}{⼊、⾝}[HSK 2]
  \definition{s.}{corpo inteiro; por todo o corpo; todo o corpo}
\end{EntryWithPhonetic}

\begin{EntryWithPhonetic}{全世界}{quan2 shi4 jie4}{6,5,9}{⼊、⼀、⽥}[HSK 5]
  \definition[种]{s.}{mundo inteiro; mundo todo | em todo o mundo}
\end{EntryWithPhonetic}

\begin{EntryWithPhonetic}{全体}{quan2 ti3}{6,7}{⼊、⼈}[HSK 2]
  \definition{s.}{(frequentemente referido a pessoas) todos; número total; todos | por todo o corpo | todos; inteiro; a soma de todas as partes; a soma de todos os indivíduos (geralmente se refere a pessoas)}
\end{EntryWithPhonetic}

\begin{EntryWithPhonetic}{全新}{quan2 xin1}{6,13}{⼊、⽄}[HSK 6]
  \definition{adj.}{totalmente novo; inteiramente/completamente novo; refere-se a algo completamente novo, especialmente algo que não foi usado}
\end{EntryWithPhonetic}

\begin{EntryWithPhonetic}{全职}{quan2zhi2}{6,11}{⼊、⽿}
  \definition{s.}{período integral | tempo inteiro | (trabalho) \emph{full-time}}
\end{EntryWithPhonetic}

\begin{EntryWithPhonetic}{权}{quan2}{6}{⽊}[HSK 6]
  \definition*{s.}{Sobrenome Quan}
  \definition{adv.}{provisoriamente; por enquanto}
  \definition{s.}{Lliterário: contrapeso; peso deslizante de uma balança romana | poder; autoridade | direito | posição vantajosa | conveniência}
  \definition{v.}{pesar; medir o peso}
\end{EntryWithPhonetic}

\begin{EntryWithPhonetic}{权力}{quan2li4}{6,2}{⽊、⼒}[HSK 6]
  \definition[种]{s.}{poder; autoridade; o poder de liderança no âmbito da responsabilidade | poder; coerção política; o poder coercitivo do status social e político}
\end{EntryWithPhonetic}

\begin{EntryWithPhonetic}{权利}{quan2li4}{6,7}{⽊、⼑}[HSK 4]
  \definition[项,种,个,条,份]{s.}{direito; interesse; os poderes e benefícios (em oposição a 义务) exercidos por um cidadão ou pessoa jurídica de acordo com a lei}
  \seealsoref{义务}{yi4wu4}
\end{EntryWithPhonetic}

\begin{EntryWithPhonetic}{泉}{quan2}{9}{⽔}[HSK 5]
  \definition*{s.}{Sobrenome Quan}
  \definition[股,眼,汪]{s.}{fonte (de água mineral) | a nascente de um rio | termo antigo para moeda}
\end{EntryWithPhonetic}

\begin{EntryWithPhonetic}{拳}{quan2}{10}{⼿}
  \definition*{s.}{Sobrenome Quan}
  \definition[个,记,套]{s.}{punho | boxe; pugilismo}
  \definition{v.}{enrolar}
\end{EntryWithPhonetic}

\begin{EntryWithPhonetic}{拳法}{quan2fa3}{10,8}{⼿、⽔}
  \definition{s.}{boxe | luta}
\end{EntryWithPhonetic}

\begin{EntryWithPhonetic}{拳王}{quan2wang2}{10,4}{⼿、⽟}
  \definition{s.}{pugilista | boxeador}
\end{EntryWithPhonetic}

\begin{EntryWithPhonetic}{犬}{quan3}{4}{⽝}[Kangxi 94]
  \definition{s.}{cachorro}
\end{EntryWithPhonetic}

\begin{EntryWithPhonetic}{劝}{quan4}{4}{⼒}[HSK 5]
  \definition*{s.}{Sobrenome Quan}
  \definition{v.}{insistir; aconselhar; tentar persuadir; persuadir, argumentar para que as pessoas obedeçam | incentivar; encorajar}
\end{EntryWithPhonetic}

\begin{EntryWithPhonetic}{券}{quan4}{8}{⼑}[HSK 6]
  \definition[张]{s.}{certificado; bilhete; ingresso; uma conta ou pedaço de papel que serve como recibo}
\end{EntryWithPhonetic}

\begin{EntryWithPhonetic}{缺}{que1}{10}{⽸}[HSK 3]
  \definition{adj.}{incompleto; imperfeito}
  \definition[种]{s.}{vaga; abertura; falta}
  \definition{v.}{estar com falta de; faltar | estar ausente}
\end{EntryWithPhonetic}

\begin{EntryWithPhonetic}{缺点}{que1dian3}{10,9}{⽸、⽕}[HSK 3]
  \definition[个,些]{s.}{desvantagem; deficiência; inconveniência; ponto fraco; uma deficiência ou imperfeição (em oposição a 优点)}
  \seealsoref{优点}{you1dian3}
\end{EntryWithPhonetic}

\begin{EntryWithPhonetic}{缺乏}{que1fa2}{10,4}{⽸、⼃}[HSK 5]
  \definition{v.}{faltar; estar em falta de; não ter ou não ter totalmente (algo que deveria possuir ou é desejaria possuir)}
\end{EntryWithPhonetic}

\begin{EntryWithPhonetic}{缺勤}{que1/qin2}{10,13}{⽸、⼒}
  \definition{v.+compl.}{ausentar-se do dever (trabalho)}
\end{EntryWithPhonetic}

\begin{EntryWithPhonetic}{缺少}{que1shao3}{10,4}{⽸、⼩}[HSK 3]
  \definition{v.}{falta; estar com falta de; estar em falta de; geralmente se refere à falta de pessoas ou coisas}
\end{EntryWithPhonetic}

\begin{EntryWithPhonetic}{缺陷}{que1xian4}{10,10}{⽸、⾩}[HSK 6]
  \definition[个,处,项]{pron.}{defeito; falha; inconveniência; mancha; um lugar onde uma pessoa ou coisa está incompleta ou tem falhas porque algo está faltando}
\end{EntryWithPhonetic}

\begin{EntryWithPhonetic}{却}{que4}{7}{⼙}[HSK 4]
  \definition{adv.}{mas; contudo; no entanto; enquanto; indica um ponto de virada}
  \definition{v.}{recuar; retroceder | afastar; repelir; desencorajar | declinar; recusar; rejeitar}
  \definition{v.aux.}{usado depois de certos verbos para indicar a conclusão de uma ação, resultado, equivalente a 去 ou 掉}
  \seealsoref{掉}{diao4}
  \seealsoref{去}{qu4}
\end{EntryWithPhonetic}

\begin{EntryWithPhonetic}{却是}{que4 shi4}{7,9}{⼙、⽇}[HSK 6]
  \definition{conj.}{na verdade; no entanto; o fato é\dots; indica um ponto de virada, contrário às suas expectativas anteriores}
\end{EntryWithPhonetic}

\begin{EntryWithPhonetic}{确}{que4}{12}{⽯}
  \definition{adj.}{autenticado | sólido | firme | real | verdadeiro}
\end{EntryWithPhonetic}

\begin{EntryWithPhonetic}{确保}{que4bao3}{12,9}{⽯、⼈}[HSK 3]
  \definition{v.}{assegurar; garantir; manter ou garantir com certeza}
\end{EntryWithPhonetic}

\begin{EntryWithPhonetic}{确定}{que4ding4}{12,8}{⽯、⼧}[HSK 3]
  \definition{adj.}{definido; certo; claro}
  \definition{v.}{firmar; definir; determinar; tomar uma decisão clara e não mudar}
\end{EntryWithPhonetic}

\begin{EntryWithPhonetic}{确立}{que4li4}{12,5}{⽯、⽴}[HSK 5]
  \definition{v.}{estabelecer; criar; construir; estabelecer ou consolidar firmemente}
\end{EntryWithPhonetic}

\begin{EntryWithPhonetic}{确认}{que4ren4}{12,4}{⽯、⾔}[HSK 4]
  \definition{v.}{afirmar; confirmar; reconhecer; confirmar explicitamente (fatos, princípios, etc.)}
\end{EntryWithPhonetic}

\begin{EntryWithPhonetic}{确实}{que4shi2}{12,8}{⽯、⼧}[HSK 3]
  \definition{adj.}{verdadeiro; confiável; autêntico}
  \definition{adv.}{verdadeiramente; realmente; de ​​fato; afirmar a autenticidade de fatos objetivos}
\end{EntryWithPhonetic}

\begin{EntryWithPhonetic}{裙}{qun2}{12}{⾐}
  \definition[条]{s.}{saia | avental | algo como uma saia}
\end{EntryWithPhonetic}

\begin{EntryWithPhonetic}{裙子}{qun2zi5}{12,3}{⾐、⼦}[HSK 3]
  \definition[条,件]{s.}{saia (peça de vestuário); uma vestimenta usada abaixo da cintura}
\end{EntryWithPhonetic}

\begin{EntryWithPhonetic}{群}{qun2}{13}{⽺}[HSK 3]
  \definition*{s.}{Sobrenome Qun}
  \definition{adj.}{em grupos; numerosos}
  \definition{clas.}{usado para grupos de pessoas ou coisas; grupo; rebanho; manada}
  \definition{s.}{multidão; grupo; muitas pessoas ou coisas reunidas | as massas; um grupo de pessoas; refere-se a um grande número de pessoas}
\end{EntryWithPhonetic}

\begin{EntryWithPhonetic}{群山}{qun2shan1}{13,3}{⽺、⼭}
  \definition{s.}{montanhas | uma cadeia de colinas}
\end{EntryWithPhonetic}

\begin{EntryWithPhonetic}{群体}{qun2 ti3}{13,7}{⽺、⼈}[HSK 5]
  \definition[个]{s.}{colônia; um conjunto composto por muitos indivíduos da mesma espécie que estão fisicamente conectados, exemplos incluem corais entre os animais e certas algas entre as plantas | grupos; refere-se, de maneira geral, ao conjunto formado por muitos indivíduos interligados que compartilham características essenciais em comum}
\end{EntryWithPhonetic}

\begin{EntryWithPhonetic}{群众}{qun2zhong4}{13,6}{⽺、⼈}[HSK 5]
  \definition[个,名,位]{s.}{as massas; refere-se ao povo em geral | não filiado; apartidário; refere-se a pessoas que não são membros do Partido Comunista Chinês nem da Liga da Juventude Comunista | alguém que não ocupa uma posição de liderança}
\end{EntryWithPhonetic}

%%%%% EOF %%%%%

