%%%
%%% S
%%%

\section*{S}\addcontentsline{toc}{section}{S}

\begin{EntryWithPhonetic}{撒}{sa1}{15}{⼿}
  \definition{v.}{lançar; deixar ir; deixar sair; liberar | livrar-se de todas as restrições; deixar-se levar; tentar usá-lo ou exibi-lo o máximo possível}
\end{EntryWithPhonetic}

\begin{EntryWithPhonetic}{撒旦}{sa1dan4}{15,5}{⼿、⽇}
  \definition*{s.}{Satã}
\end{EntryWithPhonetic}

\begin{EntryWithPhonetic}{撒旦主义}{sa1dan4 zhu3yi4}{15,5,5,3}{⼿、⽇、⼂、⼂}
  \definition*{s.}{Satanismo}
\end{EntryWithPhonetic}

\begin{EntryWithPhonetic}{撒但}{sa1dan4}{15,7}{⼿、⼈}
  \variantof{撒旦}
\end{EntryWithPhonetic}

\begin{EntryWithPhonetic}{洒}{sa3}{9}{⽔}[HSK 5]
  \definition{adj.}{natural e sem restrições; confortável (sem restrições)}
  \definition{v.}{derramar; espalhar; borrifar; salpicar; fazer com que (água ou outra coisa) caia de forma dispersa | derramar; cair de forma dispersa}
\end{EntryWithPhonetic}

\begin{EntryWithPhonetic}{洒水}{sa3shui3}{9,4}{⽔、⽔}
  \definition{v.}{borrifar}
\end{EntryWithPhonetic}

\begin{EntryWithPhonetic}{飒}{sa4}{9}{⾵}
  \definition{adj.}{(das mulheres) natural e desenfreada; elegante; valente}
  \definition{interj.}{(onomatopéia) farfalhar; sussurrar | (onomatopéia) som do vento}
  \definition{v.}{murchar}
\end{EntryWithPhonetic}

\begin{EntryWithPhonetic}{飒飒}{sa4sa4}{9,9}{⾵、⾵}
  \definition{s.}{o farfalhar | sussurro | murmúrio (do vento nas árvores, o mar, etc.)}
\end{EntryWithPhonetic}

\begin{EntryWithPhonetic}{塞}{sai1}{13}{⼟}[HSK 6]
  \definition{s.}{rolha; plugue}
  \definition{v.}{encher; conectar; preencher; espremer; bloquear | superar (para comparação)}
\end{EntryWithPhonetic}

\begin{EntryWithPhonetic}{赛}{sai4}{14}{⾙}[HSK 6]
  \definition*{s.}{Sobrenome Sai}
  \definition{s.}{jogo; partida; competição | sacrifício; cerimônia de sacrifício; antigamente, sacrifícios eram feitos para agradecer aos deuses por suas dádivas}
  \definition{v.}{ter uma competição (comparando alto e baixo, forte e fraco) | superar; ser comparável a; comparar com}
\end{EntryWithPhonetic}

\begin{EntryWithPhonetic}{赛场}{sai4 chang3}{14,6}{⾙、⼟}[HSK 6]
  \definition{s.}{local de competição; arena; ringue; terreno | campo (para competição de atletismo) | pista de corrida}
\end{EntryWithPhonetic}

\begin{EntryWithPhonetic}{赛车}{sai4che1}{14,4}{⾙、⾞}
  \definition{s.}{corrida de automóvel | corrida de bicicleta | carro de corrida}
\end{EntryWithPhonetic}

\begin{EntryWithPhonetic}{三}{san1}{3}{⼀}[HSK 1]
  \definition*{s.}{Sobrenome San}
  \definition{num.}{três; 3 | muitos; vários; mais de dois; referindo-se a muitos ou à maioria | alguns; poucos; menos; não muitos}
\end{EntryWithPhonetic}

\begin{EntryWithPhonetic}{三角}{san1jiao3}{3,7}{⼀、⾓}
  \definition{s.}{triângulo}
\end{EntryWithPhonetic}

\begin{EntryWithPhonetic}{三角恋爱}{san1jiao3lian4'ai4}{3,7,10,10}{⼀、⾓、⼼、⽖}
  \definition{s.}{triângulo amoroso}
\end{EntryWithPhonetic}

\begin{EntryWithPhonetic}{三轮车}{san1lun2che1}{3,8,4}{⼀、⾞、⾞}
  \definition{s.}{triciclo}
\end{EntryWithPhonetic}

\begin{EntryWithPhonetic}{三明治}{san1 ming2 zhi4}{3,8,8}{⼀、⽇、⽔}[HSK 6]
  \definition[个,些,块]{s.}{Empréstimo linguístico: sanduíche, \emph{sandwich}}
\end{EntryWithPhonetic}

\begin{EntryWithPhonetic}{伞}{san3}{6}{⼈}[HSK 4]
  \definition*{s.}{Sobrenome San}
  \definition[把]{s.}{guarda-chuva; proteção contra chuva ou sol | algo que tem o formato de um guarda-chuva}
\end{EntryWithPhonetic}

\begin{EntryWithPhonetic}{散}{san3}{12}{⽁}[HSK 5]
  \definition{adj.}{disperso; fragmentado; não integrado}
  \definition{s.}{medicamento em forma de pó}
  \definition{v.}{divergir; espalhar-se; separar-se; soltar-se; não se manter unido;  desintegrar}
  \seeref{san4}
\end{EntryWithPhonetic}

\begin{EntryWithPhonetic}{散文}{san3wen2}{12,4}{⽁、⽂}[HSK 5]
  \definition[篇,种]{s.}{ensaio; prosa; gênero literário, na antiguidade, referia-se a textos em prosa, em oposição à poesia e à prosa paralela; atualmente, refere-se a obras literárias que não sejam poesia, teatro ou romance, incluindo ensaios, contos, crônicas, relatos de viagem, etc.}
\end{EntryWithPhonetic}

\begin{EntryWithPhonetic}{散}{san4}{12}{⽁}
  \definition{v.}{quebrar; fragmentar; dispersar | dar; distribuir; disseminar; divulgar | dissipar; deixar sai  | terminar um acordo ou contrato; demitir}
  \seeref{san3}
\end{EntryWithPhonetic}

\begin{EntryWithPhonetic}{散步}{san4/bu4}{12,7}{⽁、⽌}[HSK 3]
  \definition{v.+compl.}{dar uma volta; dar um passeio; dar uma caminhada}
\end{EntryWithPhonetic}

\begin{EntryWithPhonetic}{散心}{san4/xin1}{12,4}{⽁、⼼}
  \definition{v.+compl.}{aliviar o tédio | desfrutar de uma diversão | estar despreocupado}
\end{EntryWithPhonetic}

\begin{EntryWithPhonetic}{丧}{sang1}{8}{⼗}
  \definition{adj.}{decepcionado; deprimido; desanimado}
  \definition{v.}{perder | desanimar; frustrar}
  \seeref{sang4}
\end{EntryWithPhonetic}

\begin{EntryWithPhonetic}{丧钟}{sang1zhong1}{8,9}{⼗、⾦}
  \definition{s.}{sentença de morte}
\end{EntryWithPhonetic}

\begin{EntryWithPhonetic}{桑}{sang1}{10}{⽊}
  \definition*{s.}{Sobrenome Sang}
  \definition[棵]{s.}{amoreira}
\end{EntryWithPhonetic}

\begin{EntryWithPhonetic}{桑巴舞}{sang1ba1wu3}{10,4,14}{⽊、⼰、⾇}
  \definition{s.}{samba}
\end{EntryWithPhonetic}

\begin{EntryWithPhonetic}{桑树}{sang1shu4}{10,9}{⽊、⽊}
  \definition{s.}{amoreira, suas folhas são utilizadas para alimentar bichos-da-seda}
\end{EntryWithPhonetic}

\begin{EntryWithPhonetic}{丧}{sang4}{8}{⼗}
  \definition{adj.}{decepcionado | desanimado}
  \definition{v.}{estar enlutado (do cônjuge etc.) | morrer}
  \seeref{sang1}
\end{EntryWithPhonetic}

\begin{EntryWithPhonetic}{丧失}{sang4shi1}{8,5}{⼗、⼤}[HSK 6]
  \definition{v.}{perder (algo que se tem)}
\end{EntryWithPhonetic}

\begin{EntryWithPhonetic}{骚}{sao1}{12}{⾺}
  \definition*{s.}{Abreviação de Li Sao (Encontrando a Tristeza), um poema do poeta e estadista do século IV a.C. Qu Yuan (屈原)}
  \definition{adj.}{coquete; (de uma mulher) lasciva | masculino (de alguns animais domésticos)}
  \definition{s.}{escritos literários; geralmente se refere à poesia | o cheiro de urina; mau cheiro}
  \definition{v.}{perturbar}
  \seealsoref{屈原}{qu1yuan2}
\end{EntryWithPhonetic}

\begin{EntryWithPhonetic}{骚乱}{sao1luan4}{12,7}{⾺、⼄}
  \definition{s.}{rebelião | perturbação | tumulto}
  \definition{v.}{criar um distúrbio}
\end{EntryWithPhonetic}

\begin{EntryWithPhonetic}{扫}{sao3}{6}{⼿}[HSK 4]
  \definition{v.}{varrer; limpar | passar rapidamente ao longo ou sobre; varrer | juntar tudo | Computação: scanear}
  \seeref{sao4}
\end{EntryWithPhonetic}

\begin{EntryWithPhonetic}{扫兴}{sao3/xing4}{6,6}{⼿、⼋}
  \definition{v.+compl.}{sentir-se decepcionado | entristecer alguém}
\end{EntryWithPhonetic}

\begin{EntryWithPhonetic}{嫂}{sao3}{12}{⼥}
  \definition[个,位,名,些]{s.}{esposa do irmão mais velho; cunhada | irmã (uma forma de tratamento para uma mulher casada, mais ou menos da mesma idade)}
\end{EntryWithPhonetic}

\begin{EntryWithPhonetic}{嫂子}{sao3zi5}{12,3}{⼥、⼦}
  \definition{s.}{esposa do irmão mais velho}
\end{EntryWithPhonetic}

\begin{EntryWithPhonetic}{扫}{sao4}{6}{⼿}
  \definition{s.}{elemento formadore de palavra}
  \seeref{sao3}
  \seealsoref{扫帚}{sao4zhou5}
\end{EntryWithPhonetic}

\begin{EntryWithPhonetic}{扫帚}{sao4zhou5}{6,8}{⼿、⼱}
  \definition[把,个]{s.}{vassoura; ferramenta de varredura feita de varas de bambu, etc., maior que uma vassora}
\end{EntryWithPhonetic}

\begin{EntryWithPhonetic}{色}{se4}{6}{⾊}[HSK 4][Kangxi 139]
  \definition*{s.}{Sobrenome Se}
  \definition[种]{s.}{cor | aparência; semblante; expressão | tipo; gênero; descrição | cena; cenário;  paisagem | qualidade (de metais preciosos, mercadorias, etc.) | aparência feminina; beleza feminina | erotismo; apetite sexual; luxúria; desejo sexual}
  \seeref{shai3}
\end{EntryWithPhonetic}

\begin{EntryWithPhonetic}{色彩}{se4cai3}{6,11}{⾊、⼺}[HSK 4]
  \definition[种,丝]{s.}{cor; matiz; tonalidade | cor; sabor; característica; metáfora para um determinado estado de espírito ou tendência de pensamento}
\end{EntryWithPhonetic}

\begin{EntryWithPhonetic}{色狼}{se4lang2}{6,10}{⾊、⽝}
  \definition*{s.}{Sátiro}
  \definition{adj.}{depravado | tarado}
\end{EntryWithPhonetic}

\begin{EntryWithPhonetic}{森}{sen1}{12}{⽊}
  \definition{adj.}{cheio de árvores | multitudinário; em multidões | escuro; sombrio}
\end{EntryWithPhonetic}

\begin{EntryWithPhonetic}{森林}{sen1lin2}{12,8}{⽊、⽊}[HSK 4]
  \definition[片,座,处]{s.}{floresta; bosque; normalmente, refere-se a uma grande área de árvores em crescimento; na silvicultura, refere-se a um grande número de árvores que crescem em uma área razoavelmente grande de terra, juntamente com os animais e outras plantas}
\end{EntryWithPhonetic}

\begin{EntryWithPhonetic}{僧}{seng1}{14}{⼈}
  \definition*{s.}{Sobrenome Seng}
  \definition[位,名,个]{s.}{monge Budista, abreviação de 僧伽}
  \seealsoref{僧伽}{seng1qie2}
\end{EntryWithPhonetic}

\begin{EntryWithPhonetic}{僧伽}{seng1qie2}{14,7}{⼈、⼈}
  \definition{s.}{sangha ou sanga (Budismo) | a comunidade monástica | monge}
\end{EntryWithPhonetic}

\begin{EntryWithPhonetic}{杀}{sha1}{6}{⽊}[HSK 5]
  \definition{adv.}{em extremo; excessivamente; usado após um verbo, indica grau intenso}
  \definition{v.}{matar; abater; esquartejar | lutar; entrar em batalha | enfraquecer; reduzir; diminuir | decolar; neutralizar}
\end{EntryWithPhonetic}

\begin{EntryWithPhonetic}{杀毒}{sha1 du2}{6,9}{⽊、⽏}[HSK 5]
  \definition{s.}{Computação: antivírus}
  \definition{v.}{esterilizar; desinfetar | Computação: eliminar um vírus}
\end{EntryWithPhonetic}

\begin{EntryWithPhonetic}{杀气}{sha1qi4}{6,4}{⽊、⽓}
  \definition{s.}{espírito assassino | aura de morte}
  \definition{v.}{desabafar a raiva de alguém}
\end{EntryWithPhonetic}

\begin{EntryWithPhonetic}{沙}{sha1}{7}{⽔}
  \definition*{s.}{Sobrenome Sha}
  \definition{adj.}{granulado; em pó | rouco}[我今天感冒了,嗓音有点沙哑。===Estou resfriado hoje e minha voz está um pouco rouca.]
  \definition[车,把,袋,吨]{s.}{areia; cascalho; grânulo; pó}
\end{EntryWithPhonetic}

\begin{EntryWithPhonetic}{沙发}{sha1fa1}{7,5}{⽔、⼜}[HSK 3]
  \definition[套,组,个,张]{s.}{sofá; assentos com molas ou espuma plástica espessa, etc., com apoios de braços em ambos os lados}
\end{EntryWithPhonetic}

\begin{EntryWithPhonetic}{沙漠}{sha1mo4}{7,13}{⽔、⽔}[HSK 5]
  \definition[个,片]{s.}{deserto; superfície totalmente coberta por areia, sem água corrente, clima seco e vegetação escassa}
\end{EntryWithPhonetic}

\begin{EntryWithPhonetic}{沙特}{sha1te4}{7,10}{⽔、⽜}
  \definition*{s.}{Saudita | Arábia Saudita, abreviação de 沙特阿拉伯}
  \seealsoref{沙特阿拉伯}{sha1te4 a1la1bo2}
\end{EntryWithPhonetic}

\begin{EntryWithPhonetic}{沙特阿拉伯}{sha1te4 a1la1bo2}{7,10,7,8,7}{⽔、⽜、⾩、⼿、⼈}
  \definition*{s.}{Arábia Saudita}
\end{EntryWithPhonetic}

\begin{EntryWithPhonetic}{沙鱼}{sha1yu2}{7,8}{⽔、⿂}
  \variantof{鲨鱼}
\end{EntryWithPhonetic}

\begin{EntryWithPhonetic}{沙子}{sha1 zi5}{7,3}{⽔、⼦}[HSK 3]
  \definition[粒,把,堆,袋,车]{s.}{areia; grão; pequenas pedras | \emph{pellets}; grãos pequenos; coisas parecidas com areia}
\end{EntryWithPhonetic}

\begin{EntryWithPhonetic}{刹}{sha1}{8}{⼑}
  \definition{v.}{acionar o(s) freio(s); frear; brecar}
  \seeref{cha4}
\end{EntryWithPhonetic}

\begin{EntryWithPhonetic}{刹多罗}{sha1duo1luo2}{8,6,8}{⼑、⼣、⽹}
  \definition*{s.}{Kshatara, sânscrito ``ksetra''}
\end{EntryWithPhonetic}

\begin{EntryWithPhonetic}{砂}{sha1}{9}{⽯}
  \variantof{沙}
\end{EntryWithPhonetic}

\begin{EntryWithPhonetic}{莎}{sha1}{10}{⾋}
  \definition{s.}{em nomes pessoais e de lugares | cigarra | fonético "sha" usado na transliteração}
  \seeref{suo1}
\end{EntryWithPhonetic}

\begin{EntryWithPhonetic}{莎莎舞}{sha1sha1wu3}{10,10,14}{⾋、⾋、⾇}
  \definition{s.}{salsa (dança)}
\end{EntryWithPhonetic}

\begin{EntryWithPhonetic}{鲨}{sha1}{15}{⿂}
  \definition[只,条]{s.}{tubarão}
\end{EntryWithPhonetic}

\begin{EntryWithPhonetic}{鲨鱼}{sha1yu2}{15,8}{⿂、⿂}
  \definition{s.}{tubarão}
\end{EntryWithPhonetic}

\begin{EntryWithPhonetic}{啥}{sha2}{11}{⼝}
  \definition{pron.}{Dialeto: O que?; equivalente a 什么}
\end{EntryWithPhonetic}

\begin{EntryWithPhonetic}{傻}{sha3}{13}{⼈}[HSK 5]
  \definition{adj.}{estúpido; confuso; burro; idiota; inflexível; (ação ou pensamento) mecânico}
\end{EntryWithPhonetic}

\begin{EntryWithPhonetic}{傻瓜}{sha3gua1}{13,5}{⼈、⽠}
  \definition{adj.}{tolo | burro | simplório | idiota}
  \definition{v.}{enganar | iludir | lograr}
\end{EntryWithPhonetic}

\begin{EntryWithPhonetic}{傻眼}{sha3yan3}{13,11}{⼈、⽬}
  \definition{adj.}{estupefato | atordoado}
\end{EntryWithPhonetic}

\begin{EntryWithPhonetic}{嗄}{sha4}{13}{⼝}
  \definition{adj.}{rouco}
  \seeref{a2}
\end{EntryWithPhonetic}

\begin{EntryWithPhonetic}{色}{shai3}{6}{⾊}[Kangxi 139]
  \definition[4]{s.}{cor; (~儿) tem o mesmo significado que "色", usado em algumas palavras faladas}
  \seeref{se4}
\end{EntryWithPhonetic}

\begin{EntryWithPhonetic}{晒}{shai4}{10}{⽇}[HSK 4]
  \definition{v.}{(sol) brilhar sobre | aquecer-se; secar ao sol; colocar algo sob a luz do sol para secar | ignorar (alguém) | mostrar; divulgar o conteúdo de sua vida privada na Internet}
\end{EntryWithPhonetic}

\begin{EntryWithPhonetic}{晒干}{shai4gan1}{10,3}{⽇、⼲}
  \definition{v.}{secar ao sol}
\end{EntryWithPhonetic}

\begin{EntryWithPhonetic}{山}{shan1}{3}{⼭}[HSK 1][Kangxi 46]
  \definition*{s.}{Sobrenome Shan}
  \definition[座]{s.}{colina; maciço; montanha | qualquer coisa que se assemelhe a uma montanha | arbustos nos quais os bichos-da-seda tecem seus casulos; referindo-se a casulos de bicho-da-seda | eco; metáfora para um som muito alto}
\end{EntryWithPhonetic}

\begin{EntryWithPhonetic}{山顶}{shan1ding3}{3,8}{⼭、⾴}
  \definition{s.}{cume da montanha}
\end{EntryWithPhonetic}

\begin{EntryWithPhonetic}{山东}{shan1dong1}{3,5}{⼭、⼀}
  \definition*{s.}{Província de Shandong (Shantung) no nordeste da China}
\end{EntryWithPhonetic}

\begin{EntryWithPhonetic}{山峰}{shan1 feng1}{3,10}{⼭、⼭}[HSK 6]
  \definition[座,个]{s.}{pico (montanha); topo alto e pontudo da montanha}
\end{EntryWithPhonetic}

\begin{EntryWithPhonetic}{山谷}{shan1 gu3}{3,7}{⼭、⾕}[HSK 6]
  \definition[条,个]{s.}{vale; desfiladeiro; ravina; a área baixa e estreita entre duas montanhas geralmente tem riachos no meio}
\end{EntryWithPhonetic}

\begin{EntryWithPhonetic}{山坡}{shan1 po1}{3,8}{⼭、⼟}[HSK 6]
  \definition[个,座,片]{s.}{encosta; encosta da montanha; a inclinação entre o topo da montanha e o terreno plano}
\end{EntryWithPhonetic}

\begin{EntryWithPhonetic}{山区}{shan1 qu1}{3,4}{⼭、⼖}[HSK 5]
  \definition[片]{s.}{área montanhosa; região montanhosa | colina; serra; montanha | distrito montanhoso}
\end{EntryWithPhonetic}

\begin{EntryWithPhonetic}{山体}{shan1ti3}{3,7}{⼭、⼈}
  \definition{s.}{forma de uma montanha}
\end{EntryWithPhonetic}

\begin{EntryWithPhonetic}{山羊}{shan1yang2}{3,6}{⼭、⽺}
  \definition{s.}{cabra | (ginástica) cavalo de salto de pequeno porte}
\end{EntryWithPhonetic}

\begin{EntryWithPhonetic}{山阴}{shan1yin1}{3,6}{⼭、⾩}
  \definition*{s.}{Condado de Shanyin em Shuozhou, Shanxi}
  \definition{s.}{lado norte (ou sombreado) de uma montanha}
\end{EntryWithPhonetic}

\begin{EntryWithPhonetic}{山寨}{shan1zhai4}{3,14}{⼭、⼧}
  \definition{s.}{fortaleza fortificada da vila | fortaleza da montanha (especialmente de bandidos) | falsificação | imitação | (fig.) pechincha}
\end{EntryWithPhonetic}

\begin{EntryWithPhonetic}{扇}{shan1}{10}{⼾}[HSK 5]
  \definition{s.}{ventilar; agitar um leque para fazer o ar circular | dar um tapa; bater com a palma da mão | bater asas; esvoaçar | incitar; instigar; estimular; agitar}
  \seeref{shan4}
\end{EntryWithPhonetic}

\begin{EntryWithPhonetic}{煽}{shan1}{14}{⽕}
  \definition{v.}{abanar (fogo); agitar um leque ou outra folha | incitar; instigar; agitar | vangloriar-se de; esbanjar prêmios em}
\end{EntryWithPhonetic}

\begin{EntryWithPhonetic}{煽动}{shan1dong4}{14,6}{⽕、⼒}
  \definition{v.}{instigar; incitar; agitar; atiçar | incitar; açoitar; conduzir; chicotear}
\end{EntryWithPhonetic}

\begin{EntryWithPhonetic}{闪}{shan3}{5}{⾨}[HSK 4]
  \definition*{s.}{Sobrenome Shan}
  \definition{s.}{relâmpago}
  \definition{v.}{esquivar-se; desviar; sair do caminho | torcer; distender | surgir de repente | cintilar; brilhar | deixar para trás; abandonar | (corpo) oscilar dramaticamente}
\end{EntryWithPhonetic}

\begin{EntryWithPhonetic}{闪存盘}{shan3cun2pan2}{5,6,11}{⾨、⼦、⽫}
  \definition{s.}{unidade de memória \emph{USB} | cartão de memória}
  \seealsoref{优盘}{you1pan2}
\end{EntryWithPhonetic}

\begin{EntryWithPhonetic}{闪电}{shan3dian4}{5,5}{⾨、⽥}[HSK 4]
  \definition[道]{s.}{relâmpago; descargas elétricas entre nuvens ou entre nuvens e o solo}
  \seealsoref{雷电}{lei2dian4}
\end{EntryWithPhonetic}

\begin{EntryWithPhonetic}{掺}{shan3}{11}{⼿}
  \definition{v.}{misturar; mesclar | conter; reter}
  \seeref{can4}
  \seeref{chan1}
\end{EntryWithPhonetic}

\begin{EntryWithPhonetic}{单}{shan4}{8}{⼗}
  \definition*{s.}{Sobrenome Shan}
  \definition{s.}{material de tecido de largura simples (dupla) | número singular (plural)}
  \seeref{chan2}
  \seeref{dan1}
\end{EntryWithPhonetic}

\begin{EntryWithPhonetic}{扇}{shan4}{10}{⼾}[HSK 5]
  \definition{clas.}{usado para portas, janelas, etc.}
  \definition[把]{s.}{leque | folha; algo em forma de placa ou folha}
  \seeref{shan1}
\end{EntryWithPhonetic}

\begin{EntryWithPhonetic}{扇子}{shan4zi5}{10,3}{⼾、⼦}[HSK 5]
  \definition[把,个]{s.}{leque; abano; abanador; utensílios que produzem vento ao serem agitados}
\end{EntryWithPhonetic}

\begin{EntryWithPhonetic}{善}{shan4}{12}{⼝}
  \definition*{s.}{Sobrenome Shan}
  \definition{adj.}{bom; bem | bom; satisfatório | gentil; amigável | familiar}
  \definition{adv.}{bom; bem}
  \definition{s.}{boa ação; ato benevolente; coisas boas (em oposição a 恶)}
  \definition{v.}{fazer sucesso; fazer bem; fazer acontecer | ser bom em; ser especialista (versado) em | ser apto a}
  \seealsoref{恶}{e4}
\end{EntryWithPhonetic}

\begin{EntryWithPhonetic}{善良}{shan4liang2}{12,7}{⼝、⾉}[HSK 4]
  \definition{adj.}{de bom coração; bom e honesto; de bom coração e cheio de boa vontade}
\end{EntryWithPhonetic}

\begin{EntryWithPhonetic}{善意}{shan4yi4}{12,13}{⼝、⼼}
  \definition{s.}{boa vontade | benevolência | bondade}
\end{EntryWithPhonetic}

\begin{EntryWithPhonetic}{善于}{shan4yu2}{12,3}{⼝、⼆}[HSK 4]
  \definition{adv./v.}{ser bom em; ser hábil em}
\end{EntryWithPhonetic}

\begin{EntryWithPhonetic}{禅}{shan4}{12}{⽰}
  \definition{v.}{abdicar e entregar a coroa a outra pessoa}
  \seeref{chan2}
\end{EntryWithPhonetic}

\begin{EntryWithPhonetic}{擅}{shan4}{16}{⼿}
  \definition{adv.}{sem autorização; arbitrariamente | fazer algo por conta própria}
  \definition{v.}{ser bom em; ser especialista em | arrogar-se a si mesmo; fazer algo por conta própria | reivindicar arbitrariamente; ir além do escopo e ajir arbitrariamente}
\end{EntryWithPhonetic}

\begin{EntryWithPhonetic}{擅自}{shan4zi4}{16,6}{⼿、⾃}
  \definition{adv.}{sem permissão ou autorização | por iniciativa própria}
\end{EntryWithPhonetic}

\begin{EntryWithPhonetic}{伤}{shang1}{6}{⼈}[HSK 3]
  \definition*{s.}{Sobrenome Shang}
  \definition[处]{s.}{ferida; ferimento}
  \definition{v.}{ferir; machucar | ter os sentimentos feridos | estar angustiado | enjoar de algo; desenvolver aversão a algo | ser prejudicial a; entravar}
\end{EntryWithPhonetic}

\begin{EntryWithPhonetic}{伤害}{shang1hai4}{6,10}{⼈、⼧}[HSK 4]
  \definition[种]{v.}{ferir; prejudicar; machucar; magoar; causar danos físicos ou mentais}
\end{EntryWithPhonetic}

\begin{EntryWithPhonetic}{伤口}{shang1 kou3}{6,3}{⼈、⼝}[HSK 6]
  \definition[处]{s.}{corte; ferida; onde a pele, os músculos, etc. são feridos, rompidos ou onde são realizadas aberturas cirúrgicas}
\end{EntryWithPhonetic}

\begin{EntryWithPhonetic}{伤亡}{shang1 wang2}{6,3}{⼈、⼇}[HSK 6]
  \definition{s.}{ferimentos e mortes; feridos e mortos; pessoas feridas e mortas; baixas}
  \definition{v.}{ser ferido e morto}
\end{EntryWithPhonetic}

\begin{EntryWithPhonetic}{伤心}{shang1/xin1}{6,4}{⼈、⼼}[HSK 3]
  \definition{v.+compl.}{estar triste; lamentar; estar com o coração partido; sentir-se triste por causa de infortúnio ou decepção}
\end{EntryWithPhonetic}

\begin{EntryWithPhonetic}{伤员}{shang1 yuan2}{6,7}{⼈、⼝}[HSK 6]
  \definition[名,位,个]{s.}{Exército: pessoal ferido; os feridos}
\end{EntryWithPhonetic}

\begin{EntryWithPhonetic}{汤}{shang1}{6}{⽔}
  \definition{s.}{correnteza forte}
  \seeref{tang1}
\end{EntryWithPhonetic}

\begin{EntryWithPhonetic}{商}{shang1}{11}{⼝}
  \definition*{s.}{Dinastia Shang (1600-1046 a.C.) | Shang, nome da estrela da constelação do coração entre as vinte e oito constelações | Sobrenome Shang}
  \definition{s.}{comércio; negócio; a atividade econômica de compra e venda de mercadorias | comerciante; negociante; comerciante; empresário; pessoas que compram e vendem mercadorias | (matemática) quociente;  o resultado de uma operação de divisão em aritmética | uma nota da antiga escala chinesa de cinco tons, correspondente a 2 na notação musical numerada}
  \definition{v.}{discutir; consultar; trocar ideias}
\end{EntryWithPhonetic}

\begin{EntryWithPhonetic}{商标}{shang1biao1}{11,9}{⼝、⽊}[HSK 5]
  \definition[个]{s.}{marca; marca registrada; \emph{trademark}; marca ou símbolo (desenho, padrão, texto, etc.) gravado ou impresso na superfície ou embalagem de um produto, para diferenciá-lo de outros produtos semelhantes}
\end{EntryWithPhonetic}

\begin{EntryWithPhonetic}{商场}{shang1 chang3}{11,6}{⼝、⼟}[HSK 1]
  \definition[家]{s.}{mercado; shopping center; loja de departamentos; loja de grande área com uma variedade completa de produtos | o mundo dos negócios; referindo-se ao mundo dos negócios | mercado; mercado composto por várias lojas reunidas em um ou vários edifícios interligados}
\end{EntryWithPhonetic}

\begin{EntryWithPhonetic}{商城}{shang1 cheng2}{11,9}{⼝、⼟}[HSK 6]
  \definition{s.}{um mercado; um centro comercial; um \emph{shopping center}; refere-se a um complexo comercial contíguo com um grande espaço de construção}
\end{EntryWithPhonetic}

\begin{EntryWithPhonetic}{商店}{shang1dian4}{11,8}{⼝、⼴}[HSK 1]
  \definition[间,家,个]{s.}{loja; armazém; local de venda de mercadorias em recinto fechado}
\end{EntryWithPhonetic}

\begin{EntryWithPhonetic}{商量}{shang1liang5}{11,12}{⼝、⾥}[HSK 2]
  \definition{v.}{consultar; discutir; conversar sobre; discutir e trocar opiniões}
\end{EntryWithPhonetic}

\begin{EntryWithPhonetic}{商贸}{shang1mao4}{11,9}{⼝、⾙}
  \definition{s.}{comércio}
\end{EntryWithPhonetic}

\begin{EntryWithPhonetic}{商品}{shang1pin3}{11,9}{⼝、⼝}[HSK 3]
  \definition[种,个,件,批]{s.}{bens; mercadoria; \emph{merchande}; os produtos do trabalho produzidos para troca têm a dupla natureza de valor de uso e valor; as mercadorias incorporam diferentes relações de produção em diferentes sistemas sociais}
\end{EntryWithPhonetic}

\begin{EntryWithPhonetic}{商人}{shang1 ren2}{11,2}{⼝、⼈}[HSK 2]
  \definition[位,名]{s.}{comerciante; mercador; empresário; homem de negócios; pessoas que trabalham com a distribuição de mercadorias}
\end{EntryWithPhonetic}

\begin{EntryWithPhonetic}{商务}{shang1wu4}{11,5}{⼝、⼒}[HSK 4]
  \definition[种,类,项]{s.}{negócios; assuntos de negócios; assuntos comerciais}
\end{EntryWithPhonetic}

\begin{EntryWithPhonetic}{商业}{shang1ye4}{11,5}{⼝、⼀}[HSK 3]
  \definition[个,种]{s.}{barganha; negócio; comércio; atividade econômica que circula mercadorias por meio de compra e venda}
\end{EntryWithPhonetic}

\begin{EntryWithPhonetic}{上}{shang3}{3}{⼀}
  \definition{s.}{tom descendente-ascendente; significa o segundo tom dos quatro tons do mandarim, e também se refere ao terceiro tom do mandarim padrão}
  \seeref{shang4}
\end{EntryWithPhonetic}

\begin{EntryWithPhonetic}{上声}{shang3sheng1}{3,7}{⼀、⼠}
  \definition{s.}{tom descendente e ascendente | terceiro tom no mandarim moderno}
\end{EntryWithPhonetic}

\begin{EntryWithPhonetic}{赏}{shang3}{12}{⾙}[HSK 4]
  \definition*{s.}{Sobrenome Shang}
  \definition{s.}{recompensa; prêmio}
  \definition{v.}{conceder (outorgar) uma recompensa; recompensar; premiar | admirar; desfrutar; apreciar; valorizar}
\end{EntryWithPhonetic}

\begin{EntryWithPhonetic}{赏赐}{shang3ci4}{12,12}{⾙、⾙}
  \definition{s.}{recompensa | prêmio}
  \definition{v.}{recompensar | premiar}
\end{EntryWithPhonetic}

\begin{EntryWithPhonetic}{赏心悦目}{shang3xin1yue4mu4}{12,4,10,5}{⾙、⼼、⼼、⽬}
  \definition{expr.}{``Aquece o coração e encanta os olhos.''; achar a paisagem agradável tanto aos olhos quanto à mente}
\end{EntryWithPhonetic}

\begin{EntryWithPhonetic}{上}{shang4}{3}{⼀}[HSK 1]
  \definition{adj.}{mais recente; último; anterior; tempo ou a sequência anterior | superior; mais alto; melhor; indica uma posição elevada em termos de qualidade, nível, etc. | lugar elevado; posição superior (em oposição a 下)}
  \definition{s.}{superior; acima; para cima; um lugar alto ou mais alto do que um determinado local | na superfície de um objeto; usado após um substantivo, indica a superfície de um objeto | indica estar dentro do escopo de algo; usado após um substantivo, indica que algo está dentro do âmbito de determinada coisa | indica um aspecto específico | antigamente, referia-se ao imperador | usado após palavras que indicam idade, equivale a ``\dots 的时候'' | o primeiro nível da escala da música folclórica chinesa, usado como um símbolo de nota na notação musical, equivalente ao '1' na notação simplificada.}
  \definition{v.}{subir; montar; embarcar; entrar | ir para; partir para | estar ocupado (com trabalho, estudos, etc.) em um horário fixo; começar a trabalhar ou estudar na hora marcada, etc. | seguir em frente; prosseguir | encher; abastecer; servir; melhorar; aumentar | aparecer no palco; entrar | colocar algo em posição; ajustar; fixar; montar as duas partes de algo | aplicar; pintar; espalhar | ser registrado; ser publicado (em uma publicação) | atingir; ser suficiente (uma determinada quantidade ou grau) | submeter; enviar; apresentar; submeter à aprovação superior | ventilar; apertar; torcer | trazer; servir; colocar comida, pratos, chá e outras coisas na mesa para os convidados | indicar que uma ação tem um resultado | pesquisar na \emph{Internet} | emaranhar-se; ficar emaranhado; enredar-se}
  \definition{v.aux.}{usado após um verbo para indicar início e continuidade}
  \seeref{shang3}
  \seealsoref{的时候}{de5 shi2hou4}
  \seealsoref{下}{xia4}
\end{EntryWithPhonetic}

\begin{EntryWithPhonetic}{上班}{shang4/ban1}{3,10}{⼀、⽟}[HSK 1]
  \definition{v.+compl.}{ir trabalhar; começar a trabalhar; estar de plantão; ir trabalhar no local de trabalho regular no horário especificado}
\end{EntryWithPhonetic}

\begin{EntryWithPhonetic}{上班族}{shang4 ban1 zu2}{3,10,11}{⼀、⽟、⽅}
  \definition[本]{s.}{trabalhadores de escritório (como grupo social)}
\end{EntryWithPhonetic}

\begin{EntryWithPhonetic}{上边}{shang4 bian5}{3,5}{⼀、⾡}[HSK 1]
  \definition{s.}{topo; acima; sobre; superior}
\end{EntryWithPhonetic}

\begin{EntryWithPhonetic}{上车}{shang4 che1}{3,4}{⼀、⾞}[HSK 1]
  \definition{v.}{entrar; subir (em um ônibus, trem, carro etc.)}
\end{EntryWithPhonetic}

\begin{EntryWithPhonetic}{上次}{shang4 ci4}{3,6}{⼀、⽋}[HSK 1]
  \definition{adv.}{última vez}
\end{EntryWithPhonetic}

\begin{EntryWithPhonetic}{上当}{shang4/dang4}{3,6}{⼀、⼹}[HSK 6]
  \definition{v.+compl.}{ser enganado; ser ludibriado; morder a isca; cair nas mãos de alguém}
\end{EntryWithPhonetic}

\begin{EntryWithPhonetic}{上帝}{shang4 di4}{3,9}{⼀、⼱}[HSK 6]
  \definition*{s.}{Deus; O Deus Supremo no Cristianismo | O Imperador do Céu; um deus na antiga crença chinesa que pode controlar tudo no mundo}
  \definition[个]{s.}{(figurado) cliente; metáfora para consumidores}
\end{EntryWithPhonetic}

\begin{EntryWithPhonetic}{上访}{shang4fang3}{3,6}{⼀、⾔}
  \definition{v.}{buscar uma audiência com superiores (especialmente funcionários do governo) para fazer uma petição por algo}
\end{EntryWithPhonetic}

\begin{EntryWithPhonetic}{上个月}{shang4 ge4 yue4}{3,3,4}{⼀、⼈、⽉}[HSK 4]
  \definition{s.}{mês passado; refere-se à hora de um mês atrás, ou seja, o mês passado mais próximo da hora atual}
\end{EntryWithPhonetic}

\begin{EntryWithPhonetic}{上古}{shang4gu3}{3,5}{⼀、⼝}
  \definition{s.}{o passado distante | tempos antigos | antiguidade}
\end{EntryWithPhonetic}

\begin{EntryWithPhonetic}{上海}{shang4hai3}{3,10}{⼀、⽔}
  \definition*{s.}{Município de Xangai (Shanghai), centro-leste da China}
\end{EntryWithPhonetic}

\begin{EntryWithPhonetic}{上级}{shang4ji2}{3,6}{⼀、⽷}[HSK 5]
  \definition[个,位]{s.}{nível superior; organização ou pessoa em nível superior; organizações ou pessoas de nível superior dentro do mesmo sistema organizacional}
\end{EntryWithPhonetic}

\begin{EntryWithPhonetic}{上课}{shang4/ke4}{3,10}{⼀、⾔}[HSK 1]
  \definition{v.+compl.}{frequentar aulas; ir às aulas; dar uma aula}
\end{EntryWithPhonetic}

\begin{EntryWithPhonetic}{上来}{shang4 lai2}{3,7}{⼀、⽊}[HSK 3]
  \definition{v.}{subir (para a minha localização) | estar no começo; começar; iniciar | surgir; de um lugar baixo para um lugar alto (o interlocutor está em um lugar alto) | usado após o verbo, indica que algo foi concluído com sucesso}
\end{EntryWithPhonetic}

\begin{EntryWithPhonetic}{上楼}{shang4 lou2}{3,13}{⼀、⽊}[HSK 4]
  \definition{v.}{subir as escadas; ir para o andar de cima}
\end{EntryWithPhonetic}

\begin{EntryWithPhonetic}{上门}{shang4 men2}{3,3}{⼀、⾨}[HSK 4]
  \definition{v.}{chamar; visitar; aparecer; ir ou vir para ver alguém; ir até a porta; ir até a casa de alguém | trancar a porta; fechar a porta durante a noite | casar-se e morar com a família da noiva}
\end{EntryWithPhonetic}

\begin{EntryWithPhonetic}{上面}{shang4 mian4}{3,9}{⼀、⾯}[HSK 3]
  \definition{s.}{uma posição mais alta que algo; uma posição acima/acima de algo | superfície do objeto | aspecto | a parte acima mencionada; a parte que vem primeiro na ordem; a parte de um artigo ou discurso que vem antes da presente | autoridades superiores | os mais velhos; a geração mais velha da família}
\end{EntryWithPhonetic}

\begin{EntryWithPhonetic}{上坡路}{shang4po1lu4}{3,8,13}{⼀、⼟、⾜}
  \definition{s.}{aclive | progresso | (fig.) tendência ascendente}
\end{EntryWithPhonetic}

\begin{EntryWithPhonetic}{上去}{shang4 qu4}{3,5}{⼀、⼛}[HSK 3]
  \definition{v.}{subir (a partir da minha localização) | ascender a um lugar (ou estado) considerado mais elevado (ou acima); usado depois de um verbo para indicar movimento, de baixo para cima ou de perto para longe}
\end{EntryWithPhonetic}

\begin{EntryWithPhonetic}{上升}{shang4 sheng1}{3,4}{⼀、⼗}[HSK 3]
  \definition{v.}{elevar; subir; mover-se para cima; mover de baixo para cima; aumentar em nível, grau, quantidade, etc.}
\end{EntryWithPhonetic}

\begin{EntryWithPhonetic}{上市}{shang4 shi4}{3,5}{⼀、⼱}[HSK 6]
  \definition{v.}{listar; abrir o capital; ser listado (na bolsa de valores) | estar na estação; estar (aparecer) no mercado | ir ao mercado (para fazer compras)}
\end{EntryWithPhonetic}

\begin{EntryWithPhonetic}{上台}{shang4 tai2}{3,5}{⼀、⼝}[HSK 6]
  \definition{v.}{aparecer no palco; subir na plataforma; ir para o palco ou pódio | assumir o poder; chegar (subir) ao poder; começar a assumir papéis de liderança ou a ganhar algum tipo de poder}
\end{EntryWithPhonetic}

\begin{EntryWithPhonetic}{上网}{shang4wang3}{3,6}{⼀、⽹}[HSK 1]
  \definition{v.}{conectar-se à \emph{Internet}; acessar a \emph{Internet}; entrar na \emph{Internet}; acessar a rede; refere-se especificamente ao computador do usuário conectado à Internet para pesquisar e consultar informações, etc.}
\end{EntryWithPhonetic}

\begin{EntryWithPhonetic}{上午}{shang4wu3}{3,4}{⼀、⼗}[HSK 1]
  \definition[个]{s.}{manhã; \emph{ante meridiem} (a.m.); geralmente refere-se ao período entre a manhã e o meio-dia}
\end{EntryWithPhonetic}

\begin{EntryWithPhonetic}{上下}{shang4 xia4}{3,3}{⼀、⼀}[HSK 5]
  \definition{adv.}{para cima e para baixo}
  \definition[顶]{s.}{alto e baixo | de cima para baixo; para cima e para baixo | superioridade ou inferioridade relativa | (após números redondos) aproximadamente; mais ou menos; por aí | velhos e jovens; hierarquia em termos de cargo e posição social}
  \definition{v.}{subir ou descer | subir e descer; da alta para a baixa ou da baixa para a alta}
\end{EntryWithPhonetic}

\begin{EntryWithPhonetic}{上学}{shang4 xue2}{3,8}{⼀、⼦}[HSK 1]
  \definition{v.}{ir à escola; frequentar a escola; estar na escola; ir à escola para estudar | começar a escola; começar a estudar no ensino fundamental}
\end{EntryWithPhonetic}

\begin{EntryWithPhonetic}{上询}{shang4 xun2}{3,8}{⼀、⾔}
  \definition{adv.}{primeira dezena do mês}
\end{EntryWithPhonetic}

\begin{EntryWithPhonetic}{上演}{shang4 yan3}{3,14}{⼀、⽔}[HSK 6]
  \definition{s.}{exibição | encenação}
  \definition{v.}{exibir (um filme); encenar (uma peça); atuar; colocar no palco}
\end{EntryWithPhonetic}

\begin{EntryWithPhonetic}{上衣}{shang4 yi1}{3,6}{⼀、⾐}[HSK 3]
  \definition[件]{s.}{jaqueta; roupas para a parte superior do corpo}
\end{EntryWithPhonetic}

\begin{EntryWithPhonetic}{上涨}{shang4 zhang3}{3,10}{⼀、⽔}[HSK 5]
  \definition{v.}{subir; ir para cima; ascender}
\end{EntryWithPhonetic}

\begin{EntryWithPhonetic}{上周}{shang4 zhou1}{3,8}{⼀、⼝}[HSK 2]
  \definition{s.}{semana passada}
\end{EntryWithPhonetic}

\begin{EntryWithPhonetic}{尚}{shang4}{8}{⼩}
  \definition*{s.}{Sobrenome Shang}
  \definition{adv.}{ainda}
  \definition{s.}{costume predominante; refere-se à tendência predominante na sociedade; coisas que geralmente são admiradas pelas pessoas}
  \definition{v.}{valorizar; estimar; dar grande importância a}
\end{EntryWithPhonetic}

\begin{EntryWithPhonetic}{尚且}{shang4 qie3}{8,5}{⼩、⼀}
  \definition{conj.}{nem\dots; muito menos\dots; é usado antes do verbo da primeira oração de uma frase complexa para apresentar alguns exemplos óbvios para comparação, a segunda oração frequentemente usa 何况 ou 更 para ecoar e tirar conclusões inevitáveis ​​sobre exemplos semelhantes com diferentes graus de gravidade}
  \seealsoref{更}{geng4}
  \seealsoref{何况}{he2kuang4}
\end{EntryWithPhonetic}

\begin{EntryWithPhonetic}{尚且……何况……}{shang4qie3 he2kuang4}{8,5,7,7}{⼩、⼀、⼈、⼎}
  \definition{conj.}{ainda que\dots, \dots; além do mais\dots e muito menos\dots}
\end{EntryWithPhonetic}

\begin{EntryWithPhonetic}{烧}{shao1}{10}{⽕}[HSK 4]
  \definition[次]{s.}{febre; temperatura corporal mais alta do que o normal}
  \definition{v.}{queimar; pegar fogo | cozinhar; aquecer; assar | guisar depois de fritar ou fritar depois de guisar | assar; grelhar os ingredientes dos alimentos diretamente sobre o fogo | ter febre; estar com febre | danificar (matar ou murchar) as plantas pelo uso excessivo (ou inadequado) de fertilizantes | tornar-se arrogante ou presunçoso; metáfora de estar em uma boa posição e se deixar levar}
\end{EntryWithPhonetic}

\begin{EntryWithPhonetic}{烧烤}{shao1kao3}{10,10}{⽕、⽕}
  \definition{s.}{churrasco}
  \definition{v.}{assar}
\end{EntryWithPhonetic}

\begin{EntryWithPhonetic}{稍}{shao1}{12}{⽲}[HSK 5]
  \definition{adv.}{ligeiramente; um pouco; um pouquinho}
\end{EntryWithPhonetic}

\begin{EntryWithPhonetic}{稍微}{shao1wei1}{12,13}{⽲、⼻}[HSK 5]
  \definition{adv.}{um pouco; um pouquinho; uma ninharia; indica que a quantidade é pequena ou o grau é superficial}
\end{EntryWithPhonetic}

\begin{EntryWithPhonetic}{勺}{shao2}{3}{⼓}[HSK 6]
  \definition{clas.}{shao; uma unidade tradicional de volume, igual a 0,01 市升, e equivalente a 1 centilitro ou 0,018 \emph{pint}}
  \definition{s.}{colher; concha}
  \seealsoref{市升}{shi4sheng1}
\end{EntryWithPhonetic}

\begin{EntryWithPhonetic}{少}{shao3}{4}{⼩}[HSK 1]
  \definition{adj.}{menos; pouco (oposto a 多); escasso; não atingir a quantidade original ou esperada}
  \definition{adv.}{um momento; um instante; provisoriamente; ligeiramente}
  \definition{v.}{faltar; ser insuficiente | dever | perder; desaparecer; extraviar | parar; desistir}
  \seeref{shao4}
  \seealsoref{多}{duo1}
\end{EntryWithPhonetic}

\begin{EntryWithPhonetic}{少见}{shao3jian4}{4,4}{⼩、⾒}
  \definition{adj.}{raramente visto; infrequente; raro}
  \definition{v.}{(forma de saudação) ``Não te vejo há muito tempo.''; ou ``Tenho te visto muito pouco ultimamente.''--``Estou muito feliz em te ver novamente.'' | difícil de ver | não familiar (para o falante) | ser raro (algo)}
\end{EntryWithPhonetic}

\begin{EntryWithPhonetic}{少数}{shao3 shu4}{4,13}{⼩、⽁}[HSK 2]
  \definition{s.}{número pequeno; poucos; minoria}
\end{EntryWithPhonetic}

\begin{EntryWithPhonetic}{少}{shao4}{4}{⼩}
  \definition*{s.}{Sobrenome Shao}
  \definition{s.}{jovem (em oposição a 老)}
  \definition{s.}{jovem mestre; filho de uma família rica}
  \seeref{shao3}
  \seealsoref{老}{lao3}
\end{EntryWithPhonetic}

\begin{EntryWithPhonetic}{少儿}{shao4 er2}{4,2}{⼩、⼉}[HSK 6]
  \definition{s.}{criança}
\end{EntryWithPhonetic}

\begin{EntryWithPhonetic}{少年}{shao4 nian2}{4,6}{⼩、⼲}[HSK 2]
  \definition[个,名,位]{s.}{adolescente; juventude; atualmente, a faixa etária geralmente referida é de 10 anos ou mais a 18 anos ou mais | menor; jovem; juvenil; refere-se a menores na faixa etária anterior | jovem; adolescente; rapaz}
\end{EntryWithPhonetic}

\begin{EntryWithPhonetic}{召}{shao4}{5}{⼝}
  \definition*{s.}{Sobrenome Shao}
  \definition{s.}{(frequentemente em nomes de lugares mongóis) templo; mosteiro}
  \definition{v.}{convocar; intimar; invocar}
  \seeref{zhao4}
\end{EntryWithPhonetic}

\begin{EntryWithPhonetic}{绍}{shao4}{8}{⽷}
  \definition*{s.}{Shaoxing, abreviação de 绍兴 | Sobrenome Shao}
  \definition{v.}{continuar; herdar}
  \seealsoref{绍兴}{shao4xing1}
\end{EntryWithPhonetic}

\begin{EntryWithPhonetic}{绍兴}{shao4xing1}{8,6}{⽷、⼋}
  \definition*{s.}{Shaoxing, anteriormente conhecida como Kuaiji, é uma cidade de nível de prefeitura na província de Zhejiang, na China; é uma grande cidade localizada na parte centro-norte da província de Zhejiang}
\end{EntryWithPhonetic}

\begin{EntryWithPhonetic}{舌}{she2}{6}{⾆}[Kangxi 135]
  \definition*{s.}{Sobrenome She}
  \definition[片,条]{s.}{língua (de um ser humano ou animal); glossa | algo em forma de língua | língua de sino; badalo}
\end{EntryWithPhonetic}

\begin{EntryWithPhonetic}{舌头}{she2tou5}{6,5}{⾆、⼤}[HSK 6]
  \definition[个]{s.}{língua; órgão que auxilia no paladar, na mastigação e na pronúncia | espião}
\end{EntryWithPhonetic}

\begin{EntryWithPhonetic}{折}{she2}{7}{⼿}
  \definition*{s.}{Sobrenome Zhe}
  \definition{clas.}{um ato de zaju | um parágrafo em um drama da Dinastia Yuan, aproximadamente equivalente a uma cena ou ato em uma ópera moderna}
  \definition[张,个,些]{s.}{abatimento; desconto | os traços dos caracteres chineses têm o formato de 𠃍 e 乚 | pasta; livreto}
  \definition{v.}{estalar; quebrar; fazer quebrar | perder; sofrer a perda de | dobrar; torcer; curvar-se | voltar; mudar de direção; retornar | estar convencido; estar cheio de admiração | equivaler a; converter em}
  \seeref{zhe1}
  \seeref{zhe2}
\end{EntryWithPhonetic}

\begin{EntryWithPhonetic}{蛇}{she2}{11}{⾍}[HSK 5]
  \definition[条]{s.}{cobra; serpente; répteis}
\end{EntryWithPhonetic}

\begin{EntryWithPhonetic}{舍}{she3}{8}{⾆}
  \definition{v.}{abandonar; desistir; descartar; jogar fora | dar esmola; dispensar caridade}
  \seeref{she4}
\end{EntryWithPhonetic}

\begin{EntryWithPhonetic}{舍不得}{she3bu5de5}{8,4,11}{⾆、⼀、⼻}[HSK 5]
  \definition{v.}{não se pode abandonar ou deixar, não se quer usar ou descartar; detestar separar-me ou usar}
\end{EntryWithPhonetic}

\begin{EntryWithPhonetic}{舍得}{she3 de5}{8,11}{⾆、⼻}[HSK 5]
  \definition{v.}{não guardar rancor; estar disposto a abrir mão de algo; estar disposto a gastar dinheiro, tempo, etc.; estar disposto a abrir mão de pessoas, oportunidades, coisas, etc. que são importantes para você}
\end{EntryWithPhonetic}

\begin{EntryWithPhonetic}{设}{she4}{6}{⾔}
  \definition*{s.}{Sobrenome She}
  \definition{conj.}{se; no caso | (matemática) dado; suponha; se}
  \definition{v.}{configurar; estabelecer; encontrar; colocar em prática}
\end{EntryWithPhonetic}

\begin{EntryWithPhonetic}{设备}{she4bei4}{6,8}{⾔、⼡}[HSK 3]
  \definition[台,套]{s.}{instalação; equipamento; montagem; um conjunto de edifícios ou equipamentos necessários para executar uma determinada tarefa ou suprir uma determinada necessidade}
\end{EntryWithPhonetic}

\begin{EntryWithPhonetic}{设计}{she4ji4}{6,4}{⾔、⾔}[HSK 3]
  \definition[份]{s.}{plano; esquema; refere-se a um plano de design ou a um projeto para um plano, etc.}
  \definition{v.}{planejar; projetar; formular métodos, desenhos, etc. com antecedência, de acordo com determinados requisitos de finalidade, antes de iniciar oficialmente um trabalho | arquitetar; idear; tramar; fazer um plano}
\end{EntryWithPhonetic}

\begin{EntryWithPhonetic}{设计师}{she4 ji4 shi1}{6,4,6}{⾔、⾔、⼱}[HSK 6]
  \definition[个,位,名,些]{s.}{planejador de projeto; designer | arquiteto}
\end{EntryWithPhonetic}

\begin{EntryWithPhonetic}{设立}{she4li4}{6,5}{⾔、⽴}[HSK 3]
  \definition{v.}{fundar; estabelecer; começar}
\end{EntryWithPhonetic}

\begin{EntryWithPhonetic}{设施}{she4shi1}{6,9}{⾔、⽅}[HSK 4]
  \definition{s.}{facilidade; instalação; instituições, sistemas, organizações, edifícios, etc., estabelecidos para realizar um trabalho ou atender a uma necessidade}
\end{EntryWithPhonetic}

\begin{EntryWithPhonetic}{设想}{she4xiang3}{6,13}{⾔、⼼}[HSK 5]
  \definition[个,种]{s.}{plano provisório (ou ideia); (item, tipo) refere-se a algo hipotético ou imaginário}
  \definition{v.}{imaginar; prever; conceber; supor | ter consideração por}
\end{EntryWithPhonetic}

\begin{EntryWithPhonetic}{设置}{she4zhi4}{6,13}{⾔、⽹}[HSK 4]
  \definition{v.}{estabelecer; colocar em prática; estabelecer ou criar instituições, empregos, profissões ou códigos, etc. | encaixar; ajustar; instalar; configurar; colocar}
\end{EntryWithPhonetic}

\begin{EntryWithPhonetic}{社}{she4}{7}{⽰}[HSK 5]
  \definition[个,家]{s.}{agência; sociedade; órgão organizado; organização; comunidade | comuna popular | o deus da terra, sacrifícios a ele ou altares para tais sacrifícios; na antiguidade, o deus da terra, o local onde ele era venerado, o dia da veneração e o ritual eram chamados de 社 | agência de notícias |  imprensa}
\end{EntryWithPhonetic}

\begin{EntryWithPhonetic}{社会}{she4hui4}{7,6}{⽰、⼈}[HSK 3]
  \definition[个,种]{s.}{sociedade; em um determinado estágio do desenvolvimento histórico, a relação geral entre as pessoas nas atividades de produção | comunidade; geralmente se refere a um grupo de pessoas que estão conectadas por atividades comuns}
\end{EntryWithPhonetic}

\begin{EntryWithPhonetic}{社区}{she4qu1}{7,4}{⽰、⼖}[HSK 5]
  \definition[个]{s.}{bairro; comunidade residencial; bairros da cidade, divididos de acordo com a localização geográfica | distrito; comunidade (para pessoas da mesma classe social, etc.) ; lugar onde pessoas com características comuns, como classe social, vivem juntas}
\end{EntryWithPhonetic}

\begin{EntryWithPhonetic}{舍}{she4}{8}{⾆}
  \definition*{s.}{Sobrenome She}
  \definition{clas.}{uma unidade antiga de distância igual a 30 li, 里}
  \definition{pron.}{meu, uma palavra humilde usada para se referir aos parentes mais jovens ou de geração inferior}
  \definition{s.}{cabana; casa | minha casa; minha humilde morada | chiqueiro; galpão; curral de gado}
  \seeref{she3}
  \seealsoref{里}{li3}
\end{EntryWithPhonetic}

\begin{EntryWithPhonetic}{射}{she4}{10}{⼨}[HSK 5]
  \definition*{s.}{Sobrenome She}
  \definition{v.}{atirar; disparar | descarregar em jato; jorrar | emitir (luz, calor, etc.) | irradiar | aludir a algo ou alguém; insinuar}
\end{EntryWithPhonetic}

\begin{EntryWithPhonetic}{射击}{she4ji1}{10,5}{⼨、⼐}[HSK 5]
  \definition{s.}{tiro; tiro ao alvo}
  \definition{v.}{disparar; atirar}
\end{EntryWithPhonetic}

\begin{EntryWithPhonetic}{涉}{she4}{10}{⽔}[HSK 6]
  \definition*{s.}{Sobrenome She}
  \definition{v.}{vadear; atravessar ou passar um rio ou um obstáculo | passar por; experimentar | envolver; implicar}
\end{EntryWithPhonetic}

\begin{EntryWithPhonetic}{涉及}{she4ji2}{10,3}{⽔、⼃}[HSK 6]
  \definition{v.}{envolver; relacionar-se com; referir-se a; tocar em}
\end{EntryWithPhonetic}

\begin{EntryWithPhonetic}{摄}{she4}{13}{⼿}
  \definition*{s.}{Sobrenome She}
  \definition{v.}{absorver; assimilar | tirar uma fotografia de; fotografar | conservar (a saúde) | atuar}
\end{EntryWithPhonetic}

\begin{EntryWithPhonetic}{摄氏}{she4shi4}{13,4}{⼿、⽒}
  \definition{s.}{graus Celsius (°C), centígrado}
\end{EntryWithPhonetic}

\begin{EntryWithPhonetic}{摄像}{she4 xiang4}{13,13}{⼿、⼈}[HSK 5]
  \definition{v.}{gravar; filmar; filmar com câmera; fazer uma gravação de vídeo (com uma câmera de vídeo ou TV)}
\end{EntryWithPhonetic}

\begin{EntryWithPhonetic}{摄像机}{she4 xiang4 ji1}{13,13,6}{⼿、⼈、⽊}[HSK 5]
  \definition[个,部,台]{s.}{câmera de vídeo; dispositivo que pode ser usado para converter imagens captadas em sinais de imagem de televisão}
\end{EntryWithPhonetic}

\begin{EntryWithPhonetic}{摄影}{she4ying3}{13,15}{⼿、⼺}[HSK 5]
  \definition{v.}{fotografar; tirar uma foto; tirar fotos ou filmar}
\end{EntryWithPhonetic}

\begin{EntryWithPhonetic}{摄影师}{she4 ying3 shi1}{13,15,6}{⼿、⼺、⼱}[HSK 5]
  \definition[个,名,位]{s.}{fotógrafo; cinegrafista; operador de câmera; técnico de fotografia em estúdio fotográfico}
\end{EntryWithPhonetic}

\begin{EntryWithPhonetic}{谁}{shei2}{10}{⾔}[HSK 1]
  \definition{pron.}{quem? | (em pergunta retórica) quem?; usado em perguntas retóricas, para indicar que não há ninguém | refere-se a pessoas que não têm certeza, incluindo aquelas que não sabem | alguém; qualquer pessoa; indica qualquer pessoa ou qualquer um | repetido em uma frase para se referir a uma pessoa | (repetido em duas frases) quem quer que seja; fazer com que o sujeito e o objeto se refiram a duas pessoas diferentes}
  \seeref{shui2}
\end{EntryWithPhonetic}

\begin{EntryWithPhonetic}{申}{shen1}{5}{⽥}
  \definition*{s.}{O nono dos doze Ramos Terrestres | Outro nome para Xangai, 上海 | Sobrenome Shen}
  \definition{v.}{declarar; explicar; expressar}
  \seealsoref{上海}{shang4hai3}
\end{EntryWithPhonetic}

\begin{EntryWithPhonetic}{申请}{shen1qing3}{5,10}{⽥、⾔}[HSK 4]
  \definition[份,批,项]{s.}{a solicitação para; o requerimento para; um pedido para ser visto pelos superiores ou departamentos relevantes}
  \definition{v.}{solicitar; apresentar uma solicitação; apresentar os motivos e fazer o pedido aos superiores ou aos departamentos competentes}
\end{EntryWithPhonetic}

\begin{EntryWithPhonetic}{伸}{shen1}{7}{⼈}[HSK 5]
  \definition{v.}{alongar; esticar; estender}
\end{EntryWithPhonetic}

\begin{EntryWithPhonetic}{身}{shen1}{7}{⾝}[Kangxi 158]
  \definition*{s.}{Sobrenome Shen}
  \definition{adv.}{eu mesmo; a si mesmo; pessoalmente}
  \definition{s.}{corpo humano ou animal | vida | o caráter moral e a conduta de alguém; cultivo moral | corpo; a parte principal de uma estrutura; o corpo principal ou tronco de um objeto |  uma vida inteira; a vida inteira de alguém | \emph{status} social; identidade}
\end{EntryWithPhonetic}

\begin{EntryWithPhonetic}{身边}{shen1 bian1}{7,5}{⾝、⾡}[HSK 2]
  \definition{adv.}{ao redor; ao lado de alguém; perto do corpo | carregar consigo (transportar); à mão}
\end{EntryWithPhonetic}

\begin{EntryWithPhonetic}{身材}{shen1cai2}{7,7}{⾝、⽊}[HSK 4]
  \definition[种,个,具]{s.}{figura; estatura; altura e peso corporal}
\end{EntryWithPhonetic}

\begin{EntryWithPhonetic}{身份}{shen1fen4}{7,6}{⾝、⼈}[HSK 4]
  \definition[种]{s.}{status; capacidade; identidade; refere-se à origem, ao status e às qualificações de uma pessoa | dignidade; posição honrada; referência especial ao status respeitável}
\end{EntryWithPhonetic}

\begin{EntryWithPhonetic}{身份证}{shen1 fen4 zheng4}{7,6,7}{⾝、⼈、⾔}[HSK 3]
  \definition[张]{s.}{ID; bilhete de identidade; carteira de identidade}
\end{EntryWithPhonetic}

\begin{EntryWithPhonetic}{身高}{shen1 gao1}{7,10}{⾝、⾼}[HSK 4]
  \definition[个,种,段]{s.}{estatura; altura (de uma pessoa)}
\end{EntryWithPhonetic}

\begin{EntryWithPhonetic}{身上}{shen1 shang5}{7,3}{⾝、⼀}[HSK 1]
  \definition{s.}{no corpo de alguém | em um;  com um}
\end{EntryWithPhonetic}

\begin{EntryWithPhonetic}{身体}{shen1ti3}{7,7}{⾝、⼈}[HSK 1]
  \definition[具,个]{s.}{corpo | saúde; saúde das pessoas}
\end{EntryWithPhonetic}

\begin{EntryWithPhonetic}{身体能力}{shen1ti3 neng2li4}{7,7,10,2}{⾝、⼈、⾁、⼒}
  \definition{s.}{habilidade física}
\end{EntryWithPhonetic}

\begin{EntryWithPhonetic}{身体乳}{shen1ti3 ru3}{7,7,8}{⾝、⼈、⼄}
  \definition{s.}{loção corporal}
\end{EntryWithPhonetic}

\begin{EntryWithPhonetic}{身亡}{shen1wang2}{7,3}{⾝、⼇}
  \definition{v.}{morrer}
\end{EntryWithPhonetic}

\begin{EntryWithPhonetic}{深}{shen1}{11}{⽔}[HSK 3]
  \definition*{s.}{Sobrenome Shen}
  \definition{adj.}{profundo | difícil; intenso; profundo | completo; penetrante; intenso; profundo | próximo; íntimo; afeição profunda; relacionamento próximo | escuro; profundo | tardio}
  \definition{adv.}{muito; grandemente; profundamente}
  \definition{s.}{profundidade}
  \seealsoref{浅}{qian3}
\end{EntryWithPhonetic}

\begin{EntryWithPhonetic}{深处}{shen1 chu4}{11,5}{⽔、⼡}[HSK 5]
  \definition{s.}{profundidades; recantos; recessos | profundezas}
\end{EntryWithPhonetic}

\begin{EntryWithPhonetic}{深度}{shen1 du4}{11,9}{⽔、⼴}[HSK 5]
  \definition{adj.}{(em grau ou extensão) profundo; sério; grave}
  \definition{s.}{profundidade; grau de profundidade; | profundidade; rigor; meticulosidade; grau de contato com a essência das coisas | estágio avançado (ou em deterioração) de desenvolvimento; grau de crescimento e desenvolvimento das coisas}
\end{EntryWithPhonetic}

\begin{EntryWithPhonetic}{深厚}{shen1hou4}{11,9}{⽔、⼚}[HSK 4]
  \definition{adj.}{profundo; sentimentos fortes | sólido; profundamente enraizado; fundação sólida}
\end{EntryWithPhonetic}

\begin{EntryWithPhonetic}{深化}{shen1 hua4}{11,4}{⽔、⼔}[HSK 6]
  \definition{v.}{aprofundar; avançar; intensificar; tornar-se mais profundo; tornar mais profundo}
\end{EntryWithPhonetic}

\begin{EntryWithPhonetic}{深刻}{shen1ke4}{11,8}{⽔、⼑}[HSK 3]
  \definition{adj.}{profundo; instenso; chegar à essência de um assunto ou problema}
\end{EntryWithPhonetic}

\begin{EntryWithPhonetic}{深入}{shen1 ru4}{11,2}{⽔、⼊}[HSK 3]
  \definition{adj.}{profundo; completo}
  \definition{v.}{ir fundo em; penetrar em; penetrar o exterior; alcançar o interior ou o centro de algo}
\end{EntryWithPhonetic}

\begin{EntryWithPhonetic}{深深}{shen1 shen1}{11,11}{⽔、⽔}[HSK 6]
  \definition{adj.}{profundo; intenso}
  \definition{adv.}{profundamente; intensamente; descreve um grau profundo ou forte}
\end{EntryWithPhonetic}

\begin{EntryWithPhonetic}{深夜}{shen1ye4}{11,8}{⽔、⼣}
  \definition{adv.}{tarde da noite}
\end{EntryWithPhonetic}

\begin{EntryWithPhonetic}{什}{shen2}{4}{⼈}
  \definition{pron.}{o que; qualquer coisa}
  \seeref{shi2}
  \seealsoref{什么}{shen2me5}
\end{EntryWithPhonetic}

\begin{EntryWithPhonetic}{什么}{shen2me5}{4,3}{⼈、⼃}[HSK 1]
  \definition{pron.}{o que?; expressar dúvida, perguntar sobre o mundo, locais, pessoas ou coisas | usado para se referir a algo indefinido; expressar incerteza | qualquer; todos; refere-se a todas as pessoas ou coisas | dois 什么 são usados juntos, indicando que o primeiro determina o segundo | usado para expressar surpresa ou insatisfação | usado para expressar discordância com o que foi dito; expressar negação | usado antes de elementos paralelos para indicar que a lista é infinita}
\end{EntryWithPhonetic}

\begin{EntryWithPhonetic}{什么时候}{shen2me5shi2hou5}{4,3,7,10}{⼈、⼃、⽇、⼈}
  \definition{adv.}{quando? | a que horas?}
\end{EntryWithPhonetic}

\begin{EntryWithPhonetic}{什么样}{shen2 me5 yang4}{4,3,10}{⼈、⼃、⽊}[HSK 2]
  \definition{pron.}{que tipo?; usado para perguntar sobre a natureza, características ou aparência de algo |  o quê?; de que tipo?; usado para perguntar sobre a situação ou o estado de alguém ou algo}
\end{EntryWithPhonetic}

\begin{EntryWithPhonetic}{神}{shen2}{9}{⽰}[HSK 5]
  \definition*{s.}{Deus | Sobrenome Shen}
  \definition{adj.}{inteligente; esperto | mágico; sobrenatural}
  \definition[个,位,尊,名]{s.}{divindade; deidade | espírito; mente; refere-se ao espírito, energia ou atenção de uma pessoa | olhar; expressão; expressões que refletem o estado interior}
\end{EntryWithPhonetic}

\begin{EntryWithPhonetic}{神话}{shen2hua4}{9,8}{⽰、⾔}[HSK 4]
  \definition[段,篇]{s.}{mito; mitologia; conto de fadas; refere-se a deuses e deusas lendários e histórias de heróis antigos deificados | lorota; refere-se a alegações ridículas e infundadas}
\end{EntryWithPhonetic}

\begin{EntryWithPhonetic}{神经}{shen2jing1}{9,8}{⽰、⽷}[HSK 5]
  \definition{adj.}{excêntrico; estranho; peculiar; descreve anormalidade neurológica}
  \definition[根,条]{s.}{nervo; um tipo de tecido presente no corpo humano ou animal que conecta o cérebro aos órgãos, transmitindo as sensações ao cérebro e as informações do cérebro aos órgãos}
\end{EntryWithPhonetic}

\begin{EntryWithPhonetic}{神经病的}{shen2jing1bing4 de5}{9,8,10,8}{⽰、⽷、⽧、⽩}
  \definition{adj.}{neuropático; neurótico}
\end{EntryWithPhonetic}

\begin{EntryWithPhonetic}{神经病学}{shen2jing1bing4 xue2}{9,8,10,8}{⽰、⽷、⽧、⼦}
  \definition{s.}{neurologia}
\end{EntryWithPhonetic}

\begin{EntryWithPhonetic}{神秘}{shen2mi4}{9,10}{⽰、⽲}[HSK 4]
  \definition{adj.}{místico; misterioso}
\end{EntryWithPhonetic}

\begin{EntryWithPhonetic}{神明}{shen2ming2}{9,8}{⽰、⽇}
  \definition{s.}{divindades | deuses}
\end{EntryWithPhonetic}

\begin{EntryWithPhonetic}{神奇}{shen2qi2}{9,8}{⽰、⼤}[HSK 5]
  \definition{adj.}{mágico; peculiar; místico; milagroso; faz as pessoas se sentirem muito revigoradas; é completamente inesperado e geralmente traz boas influências}
  \definition{adj.}{mágico; peculiar; místico; milagroso; algo que parece muito novo; algo que ninguém imaginaria, mas que geralmente traz bons resultados}
\end{EntryWithPhonetic}

\begin{EntryWithPhonetic}{神器}{shen2qi4}{9,16}{⽰、⼝}
  \definition{s.}{objeto mágico | objeto simbólico do poder imperial | arma fina | ferramenta muito útil}
\end{EntryWithPhonetic}

\begin{EntryWithPhonetic}{神情}{shen2 qing2}{9,11}{⽰、⼼}[HSK 5]
  \definition{s.}{aparência; expressão; atividades internas reveladas no rosto das pessoas}
\end{EntryWithPhonetic}

\begin{EntryWithPhonetic}{神兽}{shen2shou4}{9,11}{⽰、⼋}
  \definition{s.}{animal mitológico | fera}
\end{EntryWithPhonetic}

\begin{EntryWithPhonetic}{审}{shen3}{8}{⼧}[HSK 6]
  \definition*{s.}{Sobrenome Shen}
  \definition{adj.}{cuidadoso; detalhado; completo}
  \definition{adv.}{Literpario: realmente; de ​​fato; como esperado}
  \definition{v.}{examinar; analizar | julgar; interrogar | Literário: saber}
\end{EntryWithPhonetic}

\begin{EntryWithPhonetic}{审查}{shen3cha2}{8,9}{⼧、⽊}[HSK 6]
  \definition{v.}{examinar; investigar; verificar se algo está correto e apropriado (geralmente referindo-se a planos, propostas, escritos, qualificações pessoais, etc.); ler e avaliar (provas ou trabalhos de exame)}
\end{EntryWithPhonetic}

\begin{EntryWithPhonetic}{甚}{shen4}{9}{⽢}
  \definition{adv.}{muito; extremamente}
  \definition{pron.}{o que}
  \definition{v.}{exceder; superar}
  \seealsoref{什么}{shen2me5}
\end{EntryWithPhonetic}

\begin{EntryWithPhonetic}{甚而}{shen4'er2}{9,6}{⽢、⽽}
  \definition{conj.}{(ir) tão longe quanto | tanto que}
\end{EntryWithPhonetic}

\begin{EntryWithPhonetic}{甚或}{shen4huo4}{9,8}{⽢、⼽}
  \definition{conj.}{(ir) tão longe quanto | tanto que}
\end{EntryWithPhonetic}

\begin{EntryWithPhonetic}{甚至}{shen4zhi4}{9,6}{⽢、⾄}[HSK 4]
  \definition{conj.}{e até mesmo; nem mesmo; para apresentar uma situação típica e especial, para enfatizar a profundidade e a seriedade de uma situação}
\end{EntryWithPhonetic}

\begin{EntryWithPhonetic}{升}{sheng1}{4}{⼗}[HSK 3]
  \definition*{s.}{Sobrenome Sheng}
  \definition{clas.}{litro (l)}
  \definition{s.}{sheng, uma unidade de medida seca para grãos (= 1 litro), um décimo de 斗}
  \definition{v.}{elevar; içar; subir; ascender; subir ou subir mais alto (oposto de 降) | promover; melhorar (nível)}
  \seealsoref{斗}{dou4}
  \seealsoref{降}{jiang4}
\end{EntryWithPhonetic}

\begin{EntryWithPhonetic}{升高}{sheng1 gao1}{4,10}{⼗、⾼}[HSK 5]
  \definition{v.}{subir; ascender | promover; elevar; intensificar; potencializar; melhorar}
\end{EntryWithPhonetic}

\begin{EntryWithPhonetic}{升级}{sheng1/ji2}{4,6}{⼗、⽷}[HSK 6]
  \definition{v.+compl.}{atualizar (software) | (guerra) escalar; (tensão) aprofundar | subir um ou mais níveis; passar de uma série ou classe inferior para uma série ou classe superior}
\end{EntryWithPhonetic}

\begin{EntryWithPhonetic}{升起}{sheng1qi3}{4,10}{⼗、⾛}
  \definition{v.}{levantar | içar | subir}
\end{EntryWithPhonetic}

\begin{EntryWithPhonetic}{升学}{sheng1 xue2}{4,8}{⼗、⼦}[HSK 6]
  \definition{v.}{ir para uma universidade, faculdade; entrar em uma universidade, faculdade}
\end{EntryWithPhonetic}

\begin{EntryWithPhonetic}{升值}{sheng1 zhi2}{4,10}{⼗、⼈}[HSK 6]
  \definition{v.}{Economia: reavaliar; apreciar | Figurativo: aumento de valor | valorização; apreciação; aumentar o valor; aumentar os preços}
\end{EntryWithPhonetic}

\begin{EntryWithPhonetic}{生}{sheng1}{5}{⽣}[HSK 2,3][Kangxi 100]
  \definition*{s.}{Sobrenome Sheng}
  \definition{adj.}{vivo; vital | verde; não maduro | cru; não cozido; mal cozido | bruto; não refinado; não processado | estranho; desconhecido; não familiarizado | rígido; mecânico; forçado}
  \definition{adv.}{muito; usado antes de certas palavras que expressam emoções e sentimentos | verdadeiramente; realmente; forçosamente}
  \definition{s.}{vida | meio de subsistência | aluno; estudante | estudioso; antigamente chamados de eruditos | o tipo de personagem masculino na ópera de Pequim, etc.}
  \definition{suf.}{certos sufixos substantivos que se referem a pessoas (学生) | sufixos de certos advérbios (好生)}
  \definition{v.}{dar à luz; ter um filho | nascer | crescer; cultivar | viver; existir; sobreviver | favorecer; gerar; ocorrer | acender (uma fogueira); fazer o combustível queimar}
  \seealsoref{好生}{hao3sheng1}
  \seealsoref{学生}{xue2sheng5}
\end{EntryWithPhonetic}

\begin{EntryWithPhonetic}{生病}{sheng1bing4}{5,10}{⽣、⽧}[HSK 1]
  \definition{v.}{adoecer; ficar doente; ficar mal; contrair uma doença}
\end{EntryWithPhonetic}

\begin{EntryWithPhonetic}{生菜}{sheng1cai4}{5,11}{⽣、⾋}
  \definition{s.}{alface}
\end{EntryWithPhonetic}

\begin{EntryWithPhonetic}{生产}{sheng1chan3}{5,6}{⽣、⼇}[HSK 3]
  \definition{v.}{produzir; fabricar; utilizar ferramentas para mudar o objeto de trabalho e criar meios de produção e meios de subsistência | dar à luz uma criança; ter filhos}
\end{EntryWithPhonetic}

\begin{EntryWithPhonetic}{生成}{sheng1 cheng2}{5,6}{⽣、⼽}[HSK 5]
  \definition{v.}{formar; gerar; produzir | ter por natureza; nascer com}
\end{EntryWithPhonetic}

\begin{EntryWithPhonetic}{生词}{sheng1 ci2}{5,7}{⽣、⾔}[HSK 2]
  \definition[个,组,堆,条]{s.}{nova palavra; palavras que não aprendi, não conheço ou não entendo}
\end{EntryWithPhonetic}

\begin{EntryWithPhonetic}{生存}{sheng1cun2}{5,6}{⽣、⼦}[HSK 3]
  \definition{v.}{viver; sobreviver; subsistir; manter a vida; estar vivo}
\end{EntryWithPhonetic}

\begin{EntryWithPhonetic}{生的}{sheng1de5}{5,8}{⽣、⽩}
  \definition{conj.}{para evitar isso | para que\dots não\dots}
\end{EntryWithPhonetic}

\begin{EntryWithPhonetic}{生动}{sheng1dong4}{5,6}{⽣、⼒}[HSK 3]
  \definition{adj.}{vívido; animado; descreve a linguagem e as formas de expressão como sendo ativas e em movimento}
\end{EntryWithPhonetic}

\begin{EntryWithPhonetic}{生活}{sheng1huo2}{5,9}{⽣、⽔}[HSK 2]
  \definition[个,段,种]{s.}{vida; subsistência; as diversas atividades realizadas por pessoas ou seres vivos para sobreviver e se desenvolver | estilo de vida; condições de vida; situação em termos de vestuário, alimentação, habitação e transporte | trabalho (principalmente nas áreas industrial, agrícola e artesanal)}
  \definition{v.}{viver; realizar várias atividades | sobreviver}
\end{EntryWithPhonetic}

\begin{EntryWithPhonetic}{生活费}{sheng1 huo2 fei4}{5,9,9}{⽣、⽔、⾙}[HSK 6]
  \definition{s.}{subsídio; despesas de subsistência; despesas necessárias para manter a vida diária}
\end{EntryWithPhonetic}

\begin{EntryWithPhonetic}{生活垃圾}{sheng1huo2la1ji1}{5,9,8,6}{⽣、⽔、⼟、⼟}
  \definition{s.}{lixo doméstico}
\end{EntryWithPhonetic}

\begin{EntryWithPhonetic}{生活型}{sheng1huo2 xing2}{5,9,9}{⽣、⽔、⼟}
  \definition{s.}{forma de vida}
\end{EntryWithPhonetic}

\begin{EntryWithPhonetic}{生理}{sheng1li3}{5,11}{⽣、⽟}
  \definition{adj.}{fisiológico}
  \definition{s.}{fisiologia}
\end{EntryWithPhonetic}

\begin{EntryWithPhonetic}{生命}{sheng1ming4}{5,8}{⽣、⼝}[HSK 3]
  \definition{s.}{vida; não envolve apenas a existência e as atividades dos organismos, mas também inclui experiências de vida humana, valores e elementos-chave da sobrevivência e do desenvolvimento de várias coisas}
\end{EntryWithPhonetic}

\begin{EntryWithPhonetic}{生气}{sheng1/qi4}{5,4}{⽣、⽓}[HSK 1]
  \definition{s.}{vitalidade; vigor; energia da vida}
  \definition{v.+compl.}{ficar com raiva; ficar ofendido; ficar zangado; encontrar algo que não é do seu agrado e sentir-se descontente}
\end{EntryWithPhonetic}

\begin{EntryWithPhonetic}{生日}{sheng1ri4}{5,4}{⽣、⽇}[HSK 1]
  \definition[个,次]{s.}{aniversário; dia de nascimento, também se refere ao dia em que se completa um ano de idade a cada ano}
\end{EntryWithPhonetic}

\begin{EntryWithPhonetic}{生态}{sheng1tai4}{5,8}{⽣、⼼}
  \definition{adj.}{ecológico}
  \definition{s.}{ecologia}
\end{EntryWithPhonetic}

\begin{EntryWithPhonetic}{生物}{sheng1wu4}{5,8}{⽣、⽜}
  \definition{adj.}{biológico}
  \definition{s.}{biologia (disciplina) | organismo | ser vivo}
\end{EntryWithPhonetic}

\begin{EntryWithPhonetic}{生意}{sheng1yi4}{5,13}{⽣、⼼}
  \definition[笔,种,次]{s.}{tendência a crescer; vitalidade; vigor; energia}
  \seeref{sheng1yi5}
\end{EntryWithPhonetic}

\begin{EntryWithPhonetic}{生意}{sheng1yi5}{5,13}{⽣、⼼}[HSK 3]
  \definition[笔,种,次]{s.}{comércio, compra e venda; negócios; indústria; colegas do mesmo setor}
  \seeref{sheng1yi4}
\end{EntryWithPhonetic}

\begin{EntryWithPhonetic}{生鱼片}{sheng1yu2pian4}{5,8,4}{⽣、⿂、⽚}
  \definition{s.}{fatias de peixe cru, \emph{sashimi}}
\end{EntryWithPhonetic}

\begin{EntryWithPhonetic}{生长}{sheng1zhang3}{5,4}{⽣、⾧}[HSK 3]
  \definition{v.}{cresçer; sob certas condições de vida, o volume e o peso dos organismos aumentam gradualmente | nascer e crescer}
\end{EntryWithPhonetic}

\begin{EntryWithPhonetic}{声}{sheng1}{7}{⼠}[HSK 5]
  \definition{clas.}{indica o número de vezes que um som é emitido}
  \definition{s.}{som; voz | reputação | consoante inicial (de uma sílaba chinesa) | tom; tom de voz | informação; notícia}
  \definition{v.}{declarar; anunciar; emitir um som}
\end{EntryWithPhonetic}

\begin{EntryWithPhonetic}{声明}{sheng1ming2}{7,8}{⼠、⽇}[HSK 3]
  \definition[项,份]{s.}{declaração}
  \definition{v.}{declarar; anunciar; expressar publicamente a sua atitude ou dizer a verdade}
\end{EntryWithPhonetic}

\begin{EntryWithPhonetic}{声音}{sheng1yin1}{7,9}{⼠、⾳}[HSK 2]
  \definition[个,种]{s.}{som; voz; a percepção auditiva das ondas sonoras}
\end{EntryWithPhonetic}

\begin{EntryWithPhonetic}{绳}{sheng2}{11}{⽷}
  \definition*{s.}{Sobrenome Sheng}
  \definition[根]{s.}{corda; cordão; barbante | a linha no marcador de tinta de carpinteiro}
  \definition{v.}{restringir; corrigir; sancionar | medir | continuar}
\end{EntryWithPhonetic}

\begin{EntryWithPhonetic}{绳子}{sheng2zi5}{11,3}{⽷、⼦}
  \definition[条]{s.}{corda | cordão}
\end{EntryWithPhonetic}

\begin{EntryWithPhonetic}{省}{sheng3}{9}{⽬}[HSK 2]
  \definition*{s.}{Sobrenome Sheng}
  \definition{s.}{província; unidade administrativa, subordinada diretamente ao governo central | capital provincial; refere-se à capital da província, localização da administração provincial | abreviação (de palavras)}
  \definition{v.}{economizar; poupar; reduzir o consumo (em oposição a 费) | omitir; deixar de fora}
  \seeref{xing3}
  \seealsoref{费}{fei4}
\end{EntryWithPhonetic}

\begin{EntryWithPhonetic}{省城}{sheng3cheng2}{9,9}{⽬、⼟}
  \definition{s.}{capital da província}
\end{EntryWithPhonetic}

\begin{EntryWithPhonetic}{省会}{sheng3hui4}{9,6}{⽬、⼈}
  \definition{s.}{capital da província}
\end{EntryWithPhonetic}

\begin{EntryWithPhonetic}{省俭}{sheng3jian3}{9,9}{⽬、⼈}
  \definition{s.}{econômico | frugal}
  \definition{v.}{economizar}
\end{EntryWithPhonetic}

\begin{EntryWithPhonetic}{省力}{sheng3li4}{9,2}{⽬、⼒}
  \definition{v.}{economizar esforço ou trabalho}
\end{EntryWithPhonetic}

\begin{EntryWithPhonetic}{省钱}{sheng3 qian2}{9,10}{⽬、⾦}[HSK 6]
  \definition{adj.}{barato; não caro}
  \definition{v.}{economizar dinheiro}
\end{EntryWithPhonetic}

\begin{EntryWithPhonetic}{省却}{sheng3que4}{9,7}{⽬、⼙}
  \definition{v.}{livrar-se (para economizar espaço) | salvar}
\end{EntryWithPhonetic}

\begin{EntryWithPhonetic}{省心}{sheng3xin1}{9,4}{⽬、⼼}
  \definition{adj.}{despreocupado}
  \definition{v.}{ser poupado de preocupações | despreocupar-se}
\end{EntryWithPhonetic}

\begin{EntryWithPhonetic}{省长}{sheng3zhang3}{9,4}{⽬、⾧}
  \definition[位,任]{s.}{governador; governador de uma província}
\end{EntryWithPhonetic}

\begin{EntryWithPhonetic}{圣}{sheng4}{5}{⼟}
  \definition*{s.}{Sobrenome Sheng}
  \definition{adj.}{santo; sagrado | imperial}
  \definition{s.}{santo; sábio | imperador | o maior mestre de uma determinada arte ou habilidade}
\end{EntryWithPhonetic}

\begin{EntryWithPhonetic}{圣诞节}{sheng4 dan4 jie2}{5,8,5}{⼟、⾔、⾋}[HSK 6]
  \definition*{s.}{Natal; Nascimento de Jesus Cristo em 25 de dezembro}
\end{EntryWithPhonetic}

\begin{EntryWithPhonetic}{圣地}{sheng4di4}{5,6}{⼟、⼟}
  \definition{s.}{terra santa (de uma religião) | lugar sagrado | santuário | cidade santa (como Jerusalém, Meca, etc.) | centro de interesse histórico}
\end{EntryWithPhonetic}

\begin{EntryWithPhonetic}{胜}{sheng4}{9}{⾁}[HSK 3]
  \definition{adj.}{soberbo; maravilhoso; adorável}
  \definition[场]{s.}{vitória; sucesso | penteado de mulher; joias usadas pelas mulheres na antiguidade}
  \definition{v.}{vencer (oposto de 负, 败) | derrotar | (frequentemente seguido por 于, etc.) superar; ser superior a; levar a melhor sobre | vencer; ter sucesso; derrotar o adversário | ultrapassar; ser superior ao outro | suportar; ser capaz de suportar ou aguentar}
  \seealsoref{败}{bai4}
  \seealsoref{负}{fu4}
  \seealsoref{于}{yu2}
\end{EntryWithPhonetic}

\begin{EntryWithPhonetic}{胜负}{sheng4fu4}{9,6}{⾁、⾙}[HSK 5]
  \definition{s.}{vitória ou derrota; sucesso ou fracasso}
\end{EntryWithPhonetic}

\begin{EntryWithPhonetic}{胜利}{sheng4li4}{9,7}{⾁、⼑}[HSK 3]
  \definition{adv.}{com sucesso; triunfantemente; atingir o objetivo previsto}
  \definition{v.}{ganhar; vencer; triunfar; ter sucesso}
\end{EntryWithPhonetic}

\begin{EntryWithPhonetic}{胜算}{sheng4suan4}{9,14}{⾁、⽵}
  \definition{s.}{probabilidade de sucesso | estratégia que garante o sucesso}
  \definition{v.}{ter certeza do sucesso}
\end{EntryWithPhonetic}

\begin{EntryWithPhonetic}{乘}{sheng4}{10}{⽲}
  \definition{clas.}{usado para carruagens de guerra puxada por quatro cavalos}
  \definition{s.}{obras históricas; livros de história geral | antigamente, uma carruagem puxada por quatro cavalos}
  \seeref{cheng2}
\end{EntryWithPhonetic}

\begin{EntryWithPhonetic}{盛}{sheng4}{11}{⽫}
  \definition*{s.}{Sobrenome Sheng}
  \definition{adj.}{florescente; próspero | vigoroso; enérgico | grandioso; magnífico | abundante; profundo | popular; comum; difundido; universal | amplo; generoso; abundante; suficiente | ótimo}
  \definition{adv.}{muito; profundamente}
  \seeref{cheng2}
\end{EntryWithPhonetic}

\begin{EntryWithPhonetic}{盛行}{sheng4xing2}{11,6}{⽫、⾏}[HSK 6]
  \definition{v.}{predominar; estar atual; estar na moda; ser amplamente popular}
\end{EntryWithPhonetic}

\begin{EntryWithPhonetic}{盛宴}{sheng4yan4}{11,10}{⽫、⼧}
  \definition{s.}{celebração}
\end{EntryWithPhonetic}

\begin{EntryWithPhonetic}{剩}{sheng4}{12}{⼑}[HSK 5]
  \definition*{s.}{Sobrenome Sheng}
  \definition{v.}{permanecer; ser deixado (para trás)}
\end{EntryWithPhonetic}

\begin{EntryWithPhonetic}{剩下}{sheng4 xia4}{12,3}{⼑、⼀}[HSK 5]
  \definition{v.}{permanecer; ser deixado (para trás); consumir e utilizar, restando apenas os resíduos}
\end{EntryWithPhonetic}

\begin{EntryWithPhonetic}{失}{shi1}{5}{⼤}
  \definition{s.}{deslize; erro; defeito; acidente}
  \definition{v.}{perder (oposto de 得) | perder; deixar escapar | não agir de acordo com; negligenciar; violar | perder o controle de | errar; cometer um deslize; apresentar defeito em | não consiguir encontrar | não conseguir atingir o objetivo | desviar-se do normal | quebrar (uma promessa); voltar atrás (na palavra dada) | não conseguir obter | se perder}
  \seealsoref{得}{de2}
\end{EntryWithPhonetic}

\begin{EntryWithPhonetic}{失败}{shi1bai4}{5,8}{⼤、⾒}[HSK 4]
  \definition{adj.}{insatisfatório; a maneira como as coisas aconteceram deixou muito a desejar; o resultado final deixou muito a desejar}
  \definition{v.}{perder; ser derrotado; não vencer em uma guerra ou competição | falhar; fracassar; não dar em nada; falhar em atingir um objetivo ou meta desejada (trabalho, carreira, etc.)}
\end{EntryWithPhonetic}

\begin{EntryWithPhonetic}{失落}{shi1luo4}{5,12}{⼤、⾋}
  \definition{s.}{frustração | decepção | perda}
  \definition{v.}{perder (algo) | cair (algo) | sentir uma sensação de perda}
\end{EntryWithPhonetic}

\begin{EntryWithPhonetic}{失眠}{shi1mian2}{5,10}{⼤、⽬}
  \definition{s.}{insônia}
  \definition{v.}{ter insônia}
\end{EntryWithPhonetic}

\begin{EntryWithPhonetic}{失去}{shi1qu4}{5,5}{⼤、⼛}[HSK 3]
  \definition{v.}{perder}
\end{EntryWithPhonetic}

\begin{EntryWithPhonetic}{失望}{shi1wang4}{5,11}{⼤、⽉}[HSK 4]
  \definition{adj.}{desapontado; decepcionado}
  \definition{v.}{ficar desapontado; ficar decepcionado; estar desapontado; sentir-se sem esperança; perder a confiança}
\end{EntryWithPhonetic}

\begin{EntryWithPhonetic}{失误}{shi1wu4}{5,9}{⼤、⾔}[HSK 5]
  \definition[个]{s.}{erro; engano; equívoco; erros causados por negligência ou medidas inadequadas}
  \definition{v.}{cometer um erro; cometer um equívoco}
\end{EntryWithPhonetic}

\begin{EntryWithPhonetic}{失业}{shi1ye4}{5,5}{⼤、⼀}[HSK 4]
  \definition{v.}{não ter emprego; estar desempregado; estar sem trabalho; refere-se àqueles que estão dentro da idade legal para trabalhar, têm capacidade para trabalhar, estão desempregados e querem encontrar um emprego, mas não conseguem; embora se envolvam em certos trabalhos sociais, sua remuneração é menor do que o padrão mínimo de vida urbano local e são considerados desempregados}
\end{EntryWithPhonetic}

\begin{EntryWithPhonetic}{失意}{shi1yi4}{5,13}{⼤、⼼}
  \definition{adj.}{desapontado | frustrado}
\end{EntryWithPhonetic}

\begin{EntryWithPhonetic}{师}{shi1}{6}{⼱}
  \definition*{s.}{Sobrenome Shi}
  \definition[位,名,个]{s.}{professor; tutor; mestre | exemplo; modelo a seguir | título honorífico para um monge budista; (termo de respeito para um monge ou freira) mestre; mãe | do seu mestre ou professor | divisão; tropas; exército}
  \definition{suf.}{pessoa qualificada em determinada profissão}
  \definition{v.}{Literário: imitar; aprender}
\end{EntryWithPhonetic}

\begin{EntryWithPhonetic}{师父}{shi1 fu5}{6,4}{⼱、⽗}[HSK 6]
  \definition[个,位,名,些]{s.}{mestre; mestre trabalhador; um título respeitoso dado por um aprendiz ao seu mestre | um título respeitoso para monges, freiras e sacerdotes taoístas}
\end{EntryWithPhonetic}

\begin{EntryWithPhonetic}{师傅}{shi1fu5}{6,12}{⼱、⼈}[HSK 5]
  \definition[个,位,名]{s.}{mestre; um trabalhador qualificado; título honorífico para pessoas habilidosas | mestre; professor (em certos ofícios); pessoas que ensinam técnicas em áreas como engenharia, comércio e teatro}
\end{EntryWithPhonetic}

\begin{EntryWithPhonetic}{师生}{shi1 sheng1}{6,5}{⼱、⽣}[HSK 6]
  \definition{s.}{mestre e discípulo; professores e alunos; um nome combinado para professores e alunos}
\end{EntryWithPhonetic}

\begin{EntryWithPhonetic}{诗}{shi1}{8}{⾔}[HSK 4]
  \definition[首,句,行]{s.}{poesia; verso; poema; um gênero literário que reflete a vida e expressa emoções por meio de uma linguagem rítmica e rimada}
  \seealsoref{诗经}{shi1jing1}
\end{EntryWithPhonetic}

\begin{EntryWithPhonetic}{诗词}{shi1ci2}{8,7}{⾔、⾔}
  \definition{s.}{verso}
\end{EntryWithPhonetic}

\begin{EntryWithPhonetic}{诗歌}{shi1 ge1}{8,14}{⾔、⽋}[HSK 5]
  \definition[本,首,段]{s.}{poesia; poemas e canções; refere-se a todos os tipos de poesia}
\end{EntryWithPhonetic}

\begin{EntryWithPhonetic}{诗经}{shi1jing1}{8,8}{⾔、⽷}
  \definition*{s.}{Shijing, o Livro das Canções, antiga coleção de poemas chineses e um dos Cinco Clássicos do Confucionismo}
\end{EntryWithPhonetic}

\begin{EntryWithPhonetic}{诗句}{shi1ju4}{8,5}{⾔、⼝}
  \definition[行]{s.}{verso | versículo}
\end{EntryWithPhonetic}

\begin{EntryWithPhonetic}{诗人}{shi1 ren2}{8,2}{⾔、⼈}[HSK 4]
  \definition[个,位,名,些]{s.}{poeta; escritor de poesia}
\end{EntryWithPhonetic}

\begin{EntryWithPhonetic}{诗意}{shi1yi4}{8,13}{⾔、⼼}
  \definition{adj.}{poético}
  \definition{s.}{poesia}
\end{EntryWithPhonetic}

\begin{EntryWithPhonetic}{湿}{shi1}{12}{⽔}[HSK 4]
  \definition{adj.}{molhado; úmido; algo com água ou com muita água dentro}
\end{EntryWithPhonetic}

\begin{EntryWithPhonetic}{十}{shi2}{2}{⼗}[HSK 1][Kangxi 24]
  \definition*{s.}{Sobrenome Shi}
  \definition{num.}{dez; 10 | dezena | completo; no topo; máximo; referindo-se a algo que atingiu o ápice da perfeição ou plenitude | um monte de; indica que há muitos}
\end{EntryWithPhonetic}

\begin{EntryWithPhonetic}{十分}{shi2fen1}{2,4}{⼗、⼑}[HSK 2]
  \definition{adv.}{muito; totalmente; completamente; extremamente; indica um nível muito alto}
\end{EntryWithPhonetic}

\begin{EntryWithPhonetic}{十足}{shi2zu2}{2,7}{⼗、⾜}[HSK 5]
  \definition{adj.}{puro e simples; apenas este componente ou esta característica é muito evidente | 100\%; completo; total; muito satisfatório; muito adequado}
\end{EntryWithPhonetic}

\begin{EntryWithPhonetic}{什}{shi2}{4}{⼈}
  \definition*{s.}{Sobrenome Shi}
  \definition{adj.}{variado; sortido; diverso; vários; misturados}
  \definition{num.}{(em frações ou múltiplos) dez}
  \definition{s.}{várias coisas; artigos diversos}
  \seeref{shen2}
\end{EntryWithPhonetic}

\begin{EntryWithPhonetic}{石}{shi2}{5}{⽯}[Kangxi 112]
  \definition*{s.}{Sobrenome Shi}
  \definition{s.}{pedra; rocha; o material duro que constitui a crosta terrestre é composto por uma coleção de minerais | inscrição em pedra; esculturas em pedra}
  \seeref{dan4}
\end{EntryWithPhonetic}

\begin{EntryWithPhonetic}{石头}{shi2tou5}{5,5}{⽯、⼤}[HSK 3]
  \definition[块,堆,些]{s.}{rocha; pedra; uma substância muito dura que é o principal material da superfície da Terra}
\end{EntryWithPhonetic}

\begin{EntryWithPhonetic}{石油}{shi2you2}{5,8}{⽯、⽔}[HSK 3]
  \definition[桶,吨,升]{s.}{óleo; óleo fóssil; petróleo; um líquido inflamável extraído do solo, geralmente marrom escuro, preto ou verde escuro, do qual gasolina e outras substâncias podem ser obtidas}
\end{EntryWithPhonetic}

\begin{EntryWithPhonetic}{时}{shi2}{7}{⽇}[HSK 3]
  \definition*{s.}{Sobrenome Shi}
  \definition{adj.}{atual; presente | temporário; oportuno}
  \definition{adv.}{de vez em quando; ocasionalmente; de ​​tempos em tempos; equivalente a 常常 ou 经常 | às vezes\dots às vezes\dots; dois caracteres 时 usados juntos são equivalentes a ``有时……有时……'' e ``一会儿……一会儿……''}
  \definition{clas.}{hora, cada uma das 24 partes iguais de um dia e uma noite; também usada como unidade legal de tempo}
  \definition{s.}{dias; tempos; longo período de tempo; refere-se a um período de tempo | tempo; tempo fixo; refere-se ao tempo especificado | hora; hora do dia | temporada | chance; oportunidade; momento oportuno | atual; presente | tempo verbal; uma categoria gramatical que utiliza certas formas gramaticais para indicar o momento em que uma ação ocorre; geralmente é dividida em presente, pretérito e futuro}
  \seealsoref{常常}{chang2 chang2}
  \seealsoref{经常}{jing1chang2}
  \seealsoref{一会儿……一会儿……}{yi1hui4r5 yi1hui4r5}
  \seealsoref{有时……有时……}{you3shi2 you3shi2}
\end{EntryWithPhonetic}

\begin{EntryWithPhonetic}{时差}{shi2cha1}{7,9}{⽇、⼯}
  \definition{s.}{diferença de tempo | \emph{jet lag}}
\end{EntryWithPhonetic}

\begin{EntryWithPhonetic}{时常}{shi2chang2}{7,11}{⽇、⼱}[HSK 5]
  \definition{adv.}{frequentemente; com frequência}
\end{EntryWithPhonetic}

\begin{EntryWithPhonetic}{时代}{shi2dai4}{7,5}{⽇、⼈}[HSK 3]
  \definition[个]{s.}{idade; era; tempos; época; períodos e fases históricas divididas de acordo com condições econômicas, políticas, culturais e outras | um período na vida de alguém; uma fase na vida de uma pessoa}
\end{EntryWithPhonetic}

\begin{EntryWithPhonetic}{时而}{shi2'er2}{7,6}{⽇、⽽}[HSK 6]
  \definition{adv.}{às vezes; de tempos em tempos; indica que algo acontece repetidamente em intervalos irregulares}
\end{EntryWithPhonetic}

\begin{EntryWithPhonetic}{时而……,时而……}{shi2'er2 shi2'er2}{7,6,7,6}{⽇、⽽、⽇、⽽}[HSK 6]
  \definition{adv.}{agora\dots, agora\dots; às vezes\dots, às vezes\dots; usado antes e depois; indica que diferentes fenômenos ou coisas ocorrem alternadamente ou mudam continuamente dentro de um determinado período de tempo}[\underline{时而}下雨,\underline{时而}晴天。===Às vezes chove, às vezes faz sol. | 这个地方\underline{时而}热,\underline{时而}冷。===Este lugar às vezes é quente e às vezes frio.]
\end{EntryWithPhonetic}

\begin{EntryWithPhonetic}{时光}{shi2guang1}{7,6}{⽇、⼉}[HSK 5]
  \definition[台]{s.}{tempo; passagem do tempo | dias; horas; anos; épocas; períodos}
\end{EntryWithPhonetic}

\begin{EntryWithPhonetic}{时候}{shi2hou5}{7,10}{⽇、⼈}[HSK 1]
  \definition[个]{s.}{(um ponto no) tempo; momento; um determinado momento no tempo | (a duração do) tempo; um período de tempo com início e fim}
\end{EntryWithPhonetic}

\begin{EntryWithPhonetic}{时机}{shi2ji1}{7,6}{⽇、⽊}[HSK 5]
  \definition[个]{s.}{oportunidade; momento oportuno}
\end{EntryWithPhonetic}

\begin{EntryWithPhonetic}{时间}{shi2jian1}{7,7}{⽇、⾨}[HSK 1]
  \definition[段]{s.}{tempo; refere-se à forma de existência do movimento da matéria, um sistema contínuo composto pelo passado, presente e futuro | tempo; período (duração); um período de tempo com início e fim | tempo (um ponto); em algum momento do tempo}
\end{EntryWithPhonetic}

\begin{EntryWithPhonetic}{时节}{shi2 jie2}{7,5}{⽇、⾋}[HSK 6]
  \definition{s.}{temporada; um período de tempo em um ano com certas características, geralmente relacionadas à estação ou ao termo solar | época; tempo}
\end{EntryWithPhonetic}

\begin{EntryWithPhonetic}{时刻}{shi2ke4}{7,8}{⽇、⼑}[HSK 3]
  \definition{adv.}{constantemente; sempre; a cada momento; frequentemente}
  \definition[个,段]{s.}{tempo; hora; momento; conjuntura; um ponto no tempo}
\end{EntryWithPhonetic}

\begin{EntryWithPhonetic}{时期}{shi2qi1}{7,12}{⽇、⽉}[HSK 6]
  \definition[个,段]{s.}{um período específico; um período de tempo com uma certa característica}
\end{EntryWithPhonetic}

\begin{EntryWithPhonetic}{时时}{shi2 shi2}{7,7}{⽇、⽇}[HSK 6]
  \definition{adv.}{frequentemente; sempre; constantemente; indica que algo acontece várias vezes dentro de um determinado período de tempo}
\end{EntryWithPhonetic}

\begin{EntryWithPhonetic}{时事}{shi2shi4}{7,8}{⽇、⼅}[HSK 5]
  \definition{s.}{acontecimentos atuais; assuntos atuais; eventos atuais | tendências atuais | como as coisas estão indo | a situação atual}
\end{EntryWithPhonetic}

\begin{EntryWithPhonetic}{时装}{shi2 zhuang1}{7,12}{⽇、⾐}[HSK 6]
  \definition{s.}{vestido da moda; a última moda; os últimos estilos de roupas | roupas contemporâneas (em oposição ao 古装)}
  \seealsoref{古装}{gu3 zhuang1}
\end{EntryWithPhonetic}

\begin{EntryWithPhonetic}{识}{shi2}{7}{⾔}[HSK 6]
  \definition{s.}{percepção; conhecimento}
  \definition{v.}{saber; reconhecer | saber; entender}
  \seeref{zhi4}
\end{EntryWithPhonetic}

\begin{EntryWithPhonetic}{识字}{shi2 zi4}{7,6}{⾔、⼦}[HSK 6]
  \definition{v.}{aprender a ler; tornar-se alfabetizado; reconhecer caracteres}
\end{EntryWithPhonetic}

\begin{EntryWithPhonetic}{实}{shi2}{8}{⼧}
  \definition{adj.}{sólido; cheio por dentro; sem espaços vazios (oposto de 虚) | verdadeiro; real; atual; sincero | forte; eficaz; concreto; real}
  \definition{adv.}{verdadeiramente; realmente; de fato; originalmente}
  \definition{s.}{fato; realidade | semente; fruto}
  \definition{v.}{preencher}
  \seealsoref{虚}{xu1}
\end{EntryWithPhonetic}

\begin{EntryWithPhonetic}{实惠}{shi2hui4}{8,12}{⼧、⼼}[HSK 5]
  \definition{adj.}{sólido; substancial; benefícios práticos}
  \definition{s.}{benefício material; benefícios tangíveis; benefícios reais}
\end{EntryWithPhonetic}

\begin{EntryWithPhonetic}{实际}{shi2ji4}{8,7}{⼧、⾩}[HSK 2]
  \definition{adj.}{real; efetivo; concreto; prático | factual; prático; realista; de acordo com os fatos}
  \definition{s.}{realidade; prática; coisas e situações que existem objetivamente}
\end{EntryWithPhonetic}

\begin{EntryWithPhonetic}{实际上}{shi2 ji4 shang4}{8,7,3}{⼧、⾩、⼀}[HSK 3]
  \definition{adv.}{de fato; na verdade}
\end{EntryWithPhonetic}

\begin{EntryWithPhonetic}{实践}{shi2jian4}{8,12}{⼧、⾜}[HSK 6]
  \definition{s.}{prática; filosoficamente, refere-se às ações conscientes das pessoas para transformar a natureza e a sociedade; as atividades de produção são as atividades práticas mais básicas e também incluem atividades políticas, experimentos científicos, educação cultural, etc.}
  \definition{v.}{praticar; realizar; implementar planos e intenções em ações específicas}
\end{EntryWithPhonetic}

\begin{EntryWithPhonetic}{实力}{shi2li4}{8,2}{⼧、⼒}[HSK 3]
  \definition{s.}{força real; geralmente se refere à força militar e econômica de um país, grupo ou indivíduo, e também se refere à capacidade de um indivíduo ou grupo em uma competição}
\end{EntryWithPhonetic}

\begin{EntryWithPhonetic}{实施}{shi2shi1}{8,9}{⼧、⽅}[HSK 4]
  \definition{v.}{colocar em vigor; implementar (leis, políticas, etc.); executar; trazer (colocar) algo em vigor; fazer cumprir; colocar algo em (prática)}
\end{EntryWithPhonetic}

\begin{EntryWithPhonetic}{实习}{shi2xi2}{8,3}{⼧、⼄}[HSK 2]
  \definition{s.}{estagiário; prática; estágio}
  \definition{v.}{aplicar e testar os conhecimentos teóricos aprendidos no trabalho prático, a fim de exercitar a capacidade profissional}
\end{EntryWithPhonetic}

\begin{EntryWithPhonetic}{实现}{shi2xian4}{8,8}{⼧、⾒}[HSK 2]
  \definition{v.}{alcançar; atingir; realizar; concretizar; tornar (ideais, planos, etc.) realidade}
\end{EntryWithPhonetic}

\begin{EntryWithPhonetic}{实行}{shi2xing2}{8,6}{⼧、⾏}[HSK 3]
  \definition{v.}{praticar; implementar; executar; colocar em prática; realizar (programa, política, plano, etc.) por meio de ação}
\end{EntryWithPhonetic}

\begin{EntryWithPhonetic}{实验}{shi2yan4}{8,10}{⼧、⾺}[HSK 3]
  \definition[个,次]{s.}{teste; experimento; trabalho de laboratório}
  \definition{v.}{testar; experimentar; realizar uma operação ou se envolver em uma atividade para testar uma teoria ou hipótese científica}
\end{EntryWithPhonetic}

\begin{EntryWithPhonetic}{实验室}{shi2 yan4 shi4}{8,10,9}{⼧、⾺、⼧}[HSK 3]
  \definition[个,间]{s.}{laboratório; salas especiais para experimentos científicos}
\end{EntryWithPhonetic}

\begin{EntryWithPhonetic}{实用}{shi2yong4}{8,5}{⼧、⽤}[HSK 4]
  \definition{adj.}{prático; pragmático; funcional; atende aos requisitos reais da aplicação}
  \definition{v.}{colocar em uso prático}
\end{EntryWithPhonetic}

\begin{EntryWithPhonetic}{实在}{shi2zai4}{8,6}{⼧、⼟}[HSK 2]
  \definition{adj.}{honesto; sincero | verdadeiro; honesto; realista; não é falso, não é enganador}
  \definition{adv.}{verdadeiramente; de fato; na verdade; usado para reforçar o tom afirmativo, enfatizando que a situação é realmente assim}
\end{EntryWithPhonetic}

\begin{EntryWithPhonetic}{拾}{shi2}{9}{⼿}[HSK 5]
  \definition{num.}{dez (usado no lugar do numeral 十 em cheques, notas bancárias, etc., para evitar erros ou alterações)}
  \definition{v.}{pegar (do chão); recolher}
\end{EntryWithPhonetic}

\begin{EntryWithPhonetic}{食}{shi2}{9}{⾷}[Kangxi 184]
  \definition{adj.}{para cozinhar; comestível}
  \definition{s.}{refeição; comida; o que as pessoas e os animais comem | alimentação; alimento para animais; ração | eclipse solar; eclipse lunar}
  \definition{v.}{comer}
  \seeref{si4}
\end{EntryWithPhonetic}

\begin{EntryWithPhonetic}{食品}{shi2 pin3}{9,9}{⾷、⼝}[HSK 3]
  \definition[种]{s.}{comida; gêneros alimentícios; provisões; alimentos vendidos em lojas que passaram por algum processamento}
\end{EntryWithPhonetic}

\begin{EntryWithPhonetic}{食堂}{shi2 tang2}{9,11}{⾷、⼟}[HSK 4]
  \definition[个,间]{s.}{cantina; refeitório}
\end{EntryWithPhonetic}

\begin{EntryWithPhonetic}{食物}{shi2wu4}{9,8}{⾷、⽜}[HSK 2]
  \definition[种]{s.}{comida; alimentos; comestíveis}
\end{EntryWithPhonetic}

\begin{EntryWithPhonetic}{食欲}{shi2 yu4}{9,11}{⾷、⽋}[HSK 6]
  \definition{adj.}{apetitoso}
  \definition{s.}{apetite; desejo humano de comer}
\end{EntryWithPhonetic}

\begin{EntryWithPhonetic}{使}{shi3}{8}{⼈}[HSK 3]
  \definition{conj.}{se; supondo; usado como a primeira cláusula de uma frase complexa; indica uma relação hipotética; equivalente a 假如}
  \definition{s.}{enviado; mensageiro; pessoas em uma missão}
  \definition{v.}{enviar; despachar; dizer a alguém para fazer algo | usar; empregar; aplicar | deixar; chamar; habilitar}
  \seealsoref{假如}{jia3ru2}
\end{EntryWithPhonetic}

\begin{EntryWithPhonetic}{使得}{shi3 de5}{8,11}{⼈、⼻}[HSK 5]
  \definition{v.}{ser utilizável; poder ser usado | ser viável; ser exequível; ser possível;  poder fazer | fazer; tornar; causar um determinado resultado (intenção, plano, coisa)}
\end{EntryWithPhonetic}

\begin{EntryWithPhonetic}{使劲}{shi3/jin4}{8,7}{⼈、⼒}[HSK 4]
  \definition{v.+compl.}{colocar energia; exercer toda a sua força | esforçar-se para ajudar; colocar energia para ajudar}
\end{EntryWithPhonetic}

\begin{EntryWithPhonetic}{使用}{shi3yong4}{8,5}{⼈、⽤}[HSK 2]
  \definition{v.}{usar; empregar; aplicar; fazer com que pessoas, equipamentos, fundos, etc. sirvam a um determinado propósito}
\end{EntryWithPhonetic}

\begin{EntryWithPhonetic}{始}{shi3}{8}{⼥}
  \definition*{s.}{Sobrenome Shi}
  \definition{adv.}{somente então; não\dots até}
  \definition{s.}{começo; início}
  \definition{v.}{começar; iniciar}
\end{EntryWithPhonetic}

\begin{EntryWithPhonetic}{始终}{shi3zhong1}{8,8}{⼥、⽷}[HSK 3]
  \definition{adv.}{sempre; o tempo todo; durante todo; do começo ao fim; indica continuidade do início ao fim}
  \definition{s.}{todo o processo do começo ao fim}
\end{EntryWithPhonetic}

\begin{EntryWithPhonetic}{屎}{shi3}{9}{⼫}
  \definition{s.}{fezes | excrementos | (forma ligada) secreção (do ouvido, olho, etc.)}
\end{EntryWithPhonetic}

\begin{EntryWithPhonetic}{士}{shi4}{3}{⼠}[Kangxi 33]
  \definition*{s.}{Sobrenome Shi}
  \definition[位,名,个]{s.}{soldado; militar | oficial não comissionado; primeira classe de soldados | pessoa treinada em uma determinada área; algum tipo de técnico | pessoa (louvável) | bacharel (na China antiga) | classe social, entre os oficiais, 大夫, e o povo comum, 庶民 | estudioso | guarda-costas, uma das peças do xadrez chinês}
  \seealsoref{大夫}{da4fu1}
  \seealsoref{庶民}{shu4min2}
\end{EntryWithPhonetic}

\begin{EntryWithPhonetic}{士兵}{shi4bing1}{3,7}{⼠、⼋}[HSK 4]
  \definition[个,名,位,批,群]{s.}{soldado; militar; termo coletivo para oficiais não comissionados e soldados; os membros mais jovens do exército}
\end{EntryWithPhonetic}

\begin{EntryWithPhonetic}{世}{shi4}{5}{⼀}
  \definition*{s.}{Sobrenome Shi}
  \definition{s.}{vida; tempo de vida; vida humana | geração; geração após geração | idade; era | o mundo; sociedade | (geologia) época, abaixo de ``período''}
\end{EntryWithPhonetic}

\begin{EntryWithPhonetic}{世代}{shi4dai4}{5,5}{⼀、⼈}
  \definition{adv.}{por muitas gerações, eras}
  \definition{s.}{geração | era}
\end{EntryWithPhonetic}

\begin{EntryWithPhonetic}{世纪}{shi4ji4}{5,6}{⼀、⽷}[HSK 3]
  \definition[个,段]{s.}{século; uma unidade para calcular anos, cem anos é um século}
\end{EntryWithPhonetic}

\begin{EntryWithPhonetic}{世界}{shi4jie4}{5,9}{⼀、⽥}[HSK 3]
  \definition[个,片,种]{s.}{mundo; todos os lugares da Terra | a soma da natureza e da sociedade humana; refere-se à soma de toda a existência objetiva na natureza e na sociedade humana | campo; refere-se a uma determinada área ou campo | o universo sem limites; costumava ser um termo budista, mas agora também se refere ao mundo natural ilimitado e à sociedade humana | situação social; a situação ou atmosfera social de um determinado período}
\end{EntryWithPhonetic}

\begin{EntryWithPhonetic}{世界杯}{shi4jie4bei1}{5,9,8}{⼀、⽥、⽊}[HSK 3]
  \definition*{s.}{Copa do Mundo; Troféu da Copa do Mundo}
\end{EntryWithPhonetic}

\begin{EntryWithPhonetic}{世锦赛}{shi4jin3sai4}{5,13,14}{⼀、⾦、⾙}
  \definition*{s.}{Campeonato Mundial}
\end{EntryWithPhonetic}

\begin{EntryWithPhonetic}{市}{shi4}{5}{⼱}[HSK 2]
  \definition{s.}{mercado; lugar onde se concentra o comércio | cidade; município; áreas densamente povoadas, com indústrias, comércio e cultura desenvolvidos | relativo ao sistema tradicional chinês de pesos e medidas; unidades administrativas, incluindo cidades sob jurisdição direta e cidades sob jurisdição provincial (ou autônoma) | unidade padrão de mercado; pertencente ao sistema municipal (unidades de medida) | preço de transação no mercado}
  \definition{v.}{comprar ou vender; fazer transações}
\end{EntryWithPhonetic}

\begin{EntryWithPhonetic}{市场}{shi4chang3}{5,6}{⼱、⼟}[HSK 3]
  \definition[家]{s.}{mercado (também no abstrato); um lugar fixo onde as pessoas compram e vendem coisas juntas | área de \emph{marketing}; região onde o produto é vendido | âmbito de influência (figurado); uma metáfora para o escopo e o grau em que uma determinada ideia ou comportamento é aceito por outros}
\end{EntryWithPhonetic}

\begin{EntryWithPhonetic}{市尺}{shi4 chi3}{5,4}{⼱、⼫}
  \definition{clas.}{chi, uma unidade tradicional de comprimento, equivalente a 0,333 metros ou 1,094 pés}
\end{EntryWithPhonetic}

\begin{EntryWithPhonetic}{市斤}{shi4jin1}{5,4}{⼱、⽄}
  \definition{clas.}{jin, uma unidade tradicional de peso, cada uma contendo 10 liang (市两)  e equivalente a 0,5 quilogramas ou 1,102 libras}
  \seealsoref{市两}{shi4liang3}
\end{EntryWithPhonetic}

\begin{EntryWithPhonetic}{市两}{shi4liang3}{5,7}{⼱、⼀}
  \definition{clas.}{liang, uma unidade tradicional de peso, igual a 0,1 jin (市斤), e equivalente a 50 gramas ou 1,763 onças}
  \seealsoref{市斤}{shi4jin1}
\end{EntryWithPhonetic}

\begin{EntryWithPhonetic}{市民}{shi4 min2}{5,5}{⼱、⽒}[HSK 6]
  \definition[位,名]{s.}{habitantes da cidade; residente da cidade; moradores da cidade | cidadão; refere-se especificamente aos artesãos e comerciantes de pequeno e médio porte nas cidades da sociedade feudal tardia}
\end{EntryWithPhonetic}

\begin{EntryWithPhonetic}{市亩}{shi4mu3}{5,7}{⼱、⼇}
  \definition{clas.}{mu, uma unidade tradicional de área, igual a 60 zhang quadrados (平方市丈) e equivalente a 6,667 ares ou 0,165 acre}
  \seealsoref{平方市丈}{ping2fang1 shi4 zhang4}
\end{EntryWithPhonetic}

\begin{EntryWithPhonetic}{市区}{shi4 qu1}{5,4}{⼱、⼖}[HSK 4]
  \definition[个]{s.}{\emph{downtown}; centro da cidade; distrito urbano; áreas que ficam dentro dos limites da cidade e geralmente têm uma alta concentração de população e estoque de moradias}
\end{EntryWithPhonetic}

\begin{EntryWithPhonetic}{市升}{shi4sheng1}{5,4}{⼱、⼗}
  \definition{clas.}{sheng; uma unidade tradicional de volume, equivalente a 1 litro ou 1,76 \emph{pints} ou 0,22 galão}
\end{EntryWithPhonetic}

\begin{EntryWithPhonetic}{市长}{shi4 zhang3}{5,4}{⼱、⾧}[HSK 2]
  \definition[个,位,名]{s.}{prefeito; chefe administrativo responsável pela administração de uma cidade}
\end{EntryWithPhonetic}

\begin{EntryWithPhonetic}{市中心}{shi4zhong1xin1}{5,4,4}{⼱、⼁、⼼}
  \definition{s.}{centro da cidade}
\end{EntryWithPhonetic}

\begin{EntryWithPhonetic}{示}{shi4}{5}{⽰}[Kangxi 113]
  \definition*{s.}{Sobrenome Shi}
  \definition{s.}{(sua) carta  | missiva; instruções; palavras ou escritos para subordinados ou gerações mais jovens}
  \definition{v.}{mostrar; notificar; instruir | indicar; significar; mostrar ou apontar, fazer conhecido}
\end{EntryWithPhonetic}

\begin{EntryWithPhonetic}{示范}{shi4fan4}{5,9}{⽰、⾋}[HSK 5]
  \definition{v.}{demonstrar; dar o exemplo; criar um modelo que todos possam aprender}
\end{EntryWithPhonetic}

\begin{EntryWithPhonetic}{似}{shi4}{6}{⼈}
  \definition{v.}{ver; parecer}
  \seeref{si4}
\end{EntryWithPhonetic}

\begin{EntryWithPhonetic}{似的}{shi4de5}{6,8}{⼈、⽩}[HSK 4]
  \definition{part.}{como; como\dots como; como se (embora); usada após uma palavra ou frase para indicar uma semelhança com algo ou uma situação | usada para indicar alto grau}
\end{EntryWithPhonetic}

\begin{EntryWithPhonetic}{式}{shi4}{6}{⼷}[HSK 5]
  \definition*{s.}{Sobrenome Shi}
  \definition{s.}{tipo; estilo | forma; padrão | ritual; cerimônia | fórmula; conjunto de símbolos que expressam uma lei natural na ciência natural | humor; modo; categoria gramatical que expressa a atitude subjetiva do falante em relação ao que está sendo dito, como narrativa, imperativa e condicional}
\end{EntryWithPhonetic}

\begin{EntryWithPhonetic}{事}{shi4}{8}{⼅}[HSK 1]
  \definition[件,桩,回]{s.}{assunto; questão; coisa; negócio | problema; acidente | emprego; trabalho | responsabilidade; envolvimento | caso, coisa; o que aconteceu}
  \definition{v.}{servir; atender | estar envolvido em; dedicar-se a}
\end{EntryWithPhonetic}

\begin{EntryWithPhonetic}{事故}{shi4gu4}{8,9}{⼅、⽁}[HSK 3]
  \definition[起,桩,次,场]{s.}{acidente; perdas ou desastres repentinos, muitas vezes relacionados ao transporte, produção, trabalho e segurança pessoal}
\end{EntryWithPhonetic}

\begin{EntryWithPhonetic}{事后}{shi4 hou4}{8,6}{⼅、⼝}[HSK 6]
  \definition{s.}{depois; depois do evento; após o incidente ocorrer ou o problema ser resolvido}
\end{EntryWithPhonetic}

\begin{EntryWithPhonetic}{事件}{shi4jian4}{8,6}{⼅、⼈}[HSK 3]
  \definition[个,件,次]{s.}{evento; incidente; grandes eventos na história ou na sociedade}
\end{EntryWithPhonetic}

\begin{EntryWithPhonetic}{事情}{shi4qing5}{8,11}{⼅、⼼}[HSK 2]
  \definition[件,个,些,种]{s.}{assunto; questão; coisa; negócio | erro; acidente; infortúnio | (coloquial) emprego; trabalho}
\end{EntryWithPhonetic}

\begin{EntryWithPhonetic}{事儿}{shi4r5}{8,2}{⼅、⼉}
  \definition[件,桩]{s.}{o emprego | negócio | afazeres | assunto que precisa ser resolvido | matéria}
\end{EntryWithPhonetic}

\begin{EntryWithPhonetic}{事实}{shi4shi2}{8,8}{⼅、⼧}[HSK 3]
  \definition[个,件]{s.}{mito; lenda; uma narrativa sobre alguém ou algo que foi transmitida oralmente}
  \definition{v.}{dizer; contar; ser dito; contar a história}
\end{EntryWithPhonetic}

\begin{EntryWithPhonetic}{事实上}{shi4 shi2 shang4}{8,8,3}{⼅、⼧、⼀}[HSK 3]
  \definition{adv.}{realmente; de fato; na realidade; na verdade; de fato}
\end{EntryWithPhonetic}

\begin{EntryWithPhonetic}{事物}{shi4wu4}{8,8}{⼅、⽜}[HSK 4]
  \definition[种,类,个]{s.}{coisa; objeto; todos os objetos e fenômenos que existem objetivamente}
\end{EntryWithPhonetic}

\begin{EntryWithPhonetic}{事先}{shi4xian1}{8,6}{⼅、⼉}[HSK 4]
  \definition{adv.}{antes; de antemão; com antecedência; antecipadamente}
\end{EntryWithPhonetic}

\begin{EntryWithPhonetic}{事业}{shi4ye4}{8,5}{⼅、⼀}[HSK 3]
  \definition[个]{s.}{causa; carreira; empreendimento; atividades regulares realizadas por pessoas com um determinado objetivo, escala e sistema que têm impacto no desenvolvimento social | instituição; instalações; unidade de trabalho apoiada financeiramente pelo governo; refere-se especificamente a empresas que não têm rendimentos de produção, são financiadas pelo Estado e não realizam contabilidade económica}
\end{EntryWithPhonetic}

\begin{EntryWithPhonetic}{势}{shi4}{8}{⼒}
  \definition{s.}{poder; força; influência | momentum; tendência | aparência externa de um objeto natural; fenômenos ou situações naturais | situação; estado de coisas; circunstâncias | sinal; gesto | genitais masculinos}
\end{EntryWithPhonetic}

\begin{EntryWithPhonetic}{势力}{shi4li4}{8,2}{⼒、⼒}[HSK 5]
  \definition[股]{s.}{força; poder; influência; forças políticas, econômicas, militares, etc.}
\end{EntryWithPhonetic}

\begin{EntryWithPhonetic}{视}{shi4}{8}{⾒}
  \definition{v.}{olhar para | considerar; olhar para | inspecionar; observar}
\end{EntryWithPhonetic}

\begin{EntryWithPhonetic}{视角}{shi4jiao3}{8,7}{⾒、⾓}
  \definition{s.}{ângulo do qual se observa um objeto | (figurativo) perspectiva, ponto de vista, quadro de referência | (cinematografia) ângulo da câmera | (percepção visual) ângulo visual (o ângulo que um objeto visto subtende no olho) | (fotografia) ângulo de visão}
\end{EntryWithPhonetic}

\begin{EntryWithPhonetic}{视频}{shi4pin2}{8,13}{⾒、⾴}[HSK 5]
  \definition[个,段,条]{s.}{vídeo; videoclipe}
\end{EntryWithPhonetic}

\begin{EntryWithPhonetic}{视为}{shi4 wei2}{8,4}{⾒、⼂}[HSK 5]
  \definition{v.}{considerar; ver como; considerar como; considerar ser; achar que é}
\end{EntryWithPhonetic}

\begin{EntryWithPhonetic}{试}{shi4}{8}{⾔}[HSK 1]
  \definition{s.}{teste; exame; avaliação de conhecimentos ou habilidades através de métodos específicos}
  \definition{v.}{tentar; investigar resultados ou verificar a natureza, não se envolver formalmente (em determinada atividade)}
\end{EntryWithPhonetic}

\begin{EntryWithPhonetic}{试点}{shi4 dian3}{8,9}{⾔、⽕}[HSK 6]
  \definition[个]{s.}{local onde um experimento é conduzido; unidade experimental; local de teste; um lugar para pequenos experimentos}
  \definition{v.}{experimentar; fazer experimentos; realizar testes em pontos selecionados; lançar um projeto piloto}
\end{EntryWithPhonetic}

\begin{EntryWithPhonetic}{试卷}{shi4juan4}{8,8}{⾔、⼙}[HSK 4]
  \definition[分,张]{s.}{folha de teste; folha de exame; papel usado para escrever as respostas nos exames}
\end{EntryWithPhonetic}

\begin{EntryWithPhonetic}{试题}{shi4 ti2}{8,15}{⾔、⾴}[HSK 3]
  \definition[道]{s.}{questões de um exame}
\end{EntryWithPhonetic}

\begin{EntryWithPhonetic}{试图}{shi4tu2}{8,8}{⾔、⼞}[HSK 5]
  \definition{v.}{tentar; pretender, fazer o possível para realizar algo}
\end{EntryWithPhonetic}

\begin{EntryWithPhonetic}{试验}{shi4yan4}{8,10}{⾔、⾺}[HSK 3]
  \definition{v.}{testar; fazer um teste; fazer um experimento; para examinar o efeito ou desempenho de algo, primeiro experimente em um laboratório ou em uma escala menor}
\end{EntryWithPhonetic}

\begin{EntryWithPhonetic}{室}{shi4}{9}{⼧}[HSK 3]
  \definition*{s.}{Shi, a décima terceira das vinte e oito constelações da esfera celeste, composta por duas estrelas em linha reta na constelação de Pégaso | Sobrenome Shi}
  \definition{s.}{sala; quarto; casa | departamento; sala como unidade administrativa ou de trabalho; órgãos públicos, fábricas, escolas e outras unidades de trabalho internas | esposa; familiares ou esposa | família; clã | cavidade; órgão com forma semelhante a uma câmara}
\end{EntryWithPhonetic}

\begin{EntryWithPhonetic}{是}{shi4}{9}{⽇}[HSK 1]
  \definition*{s.}{Sobrenome Shi}
  \definition{adj.}{correto; certo | verdadeiro}
  \definition{adv.}{(expressar afirmação firme) de fato; realmente}
  \definition{pron.}{isso; isto |  todos; qualquer um; usado antes de substantivos, tem o significado de 凡是}
  \definition{s.}{assuntos (importantes); grandes planos}
  \definition{v.}{usado como “ser” antes de substantivos ou pronomes para identificar, descrever ou ampliar o sujeito; indica que duas coisas são iguais, ou que a segunda explica a primeira | usado entre duas palavras idênticas; relacionar duas palavras semelhantes |  (usado antes de substantivos) ser exatamente; ser corretamente; usado antes de substantivos, tem o significado de 适合 | elogiar; justificar | expressar afirmação ou concordância (frequentemente usado sozinho) | usado para escolher perguntas, perguntas sim/não ou perguntas retóricas | (usado no início de uma frase) enfatizar uma determinada parte de uma frase | usado em perguntas sim-não}
  \seealsoref{凡是}{fan2shi4}
  \seealsoref{适合}{shi4he2}
\end{EntryWithPhonetic}

\begin{EntryWithPhonetic}{是不是}{shi4 bu2 shi4}{9,4,9}{⽇、⼀、⽇}[HSK 1]
  \definition{expr.}{sim ou não; é ou não é; se ou não; questões levantadas sobre a confirmação e a negação dos fatos}
\end{EntryWithPhonetic}

\begin{EntryWithPhonetic}{是的}{shi4de5}{9,8}{⽇、⽩}
  \definition{adv.}{sim | está certo}
\end{EntryWithPhonetic}

\begin{EntryWithPhonetic}{是否}{shi4fou3}{9,7}{⽇、⼝}[HSK 4]
  \definition{adv.}{se; se ou não; sim ou não}
\end{EntryWithPhonetic}

\begin{EntryWithPhonetic}{适}{shi4}{9}{⾡}
  \definition*{s.}{Sobrenome Shi}
  \definition{adj.}{confortável; bem | adequado; apropriado | certo; oportuno}
  \definition{v.}{ser apto; ser adequado; ser apropriado | ir; seguir; perseguir | (de uma mulher) casar}
\end{EntryWithPhonetic}

\begin{EntryWithPhonetic}{适当}{shi4 dang4}{9,6}{⾡、⼹}[HSK 6]
  \definition{s.}{adequado; apropriado}
\end{EntryWithPhonetic}

\begin{EntryWithPhonetic}{适合}{shi4he2}{9,6}{⾡、⼝}[HSK 3]
  \definition{v.}{servir; caber; se adequar; atender às necessidades de uma determinada situação ou pessoa}
\end{EntryWithPhonetic}

\begin{EntryWithPhonetic}{适应}{shi4ying4}{9,7}{⾡、⼴}[HSK 3]
  \definition{v.}{ajustar-se; adequar-se; adaptar-se; fazer as alterações correspondentes para se adequar à medida que as condições mudam}
\end{EntryWithPhonetic}

\begin{EntryWithPhonetic}{适用}{shi4 yong4}{9,5}{⾡、⽤}[HSK 3]
  \definition{adj.}{adequado; aplicável}
\end{EntryWithPhonetic}

\begin{EntryWithPhonetic}{收}{shou1}{6}{⽁}[HSK 2]
  \definition{expr.}{aos cuidados de (usado na linha de endereço após o nome)}
  \definition{v.}{recolocar; juntar; reunir e juntar coisas espalhadas ou dispersas | recolher; cobrar | ganhar; obter (benefícios econômicos) | colher; recolher; colher ou cortar frutas, legumes, cereais maduros, etc. | aceitar; receber; acolher | controlar; restringir; restringir, controlar os sentimentos ou ações, para voltar ao estado normal | finalizar; parar; concluir; encerrar | prender; deter; colocar sob custódia}
\end{EntryWithPhonetic}

\begin{EntryWithPhonetic}{收藏}{shou1cang2}{6,17}{⽁、⾋}[HSK 6]
  \definition{v.}{coletar; armazenar; consagrar}
\end{EntryWithPhonetic}

\begin{EntryWithPhonetic}{收到}{shou1 dao4}{6,8}{⽁、⼑}[HSK 2]
  \definition{v.}{conseguir; obter; receber; alcançar}
\end{EntryWithPhonetic}

\begin{EntryWithPhonetic}{收费}{shou1 fei4}{6,9}{⽁、⾙}[HSK 3]
  \definition{v.}{cobrar; cobrar taxas}
\end{EntryWithPhonetic}

\begin{EntryWithPhonetic}{收购}{shou1 gou4}{6,8}{⽁、⾙}[HSK 5]
  \definition{v.}{comprar; adquirir; comprar muito em vários lugares | adquirir uma empresa; obter o controle efetivo de uma empresa por meio de dinheiro, transações de ações, etc.}
\end{EntryWithPhonetic}

\begin{EntryWithPhonetic}{收回}{shou1 hui2}{6,6}{⽁、⼞}[HSK 4]
  \definition{v.}{retomar; recuperar; relembrar; recordar; receber de volta o que foi enviado ou emprestado, ou o dinheiro que foi emprestado ou usado | sacar; retirar; recolher; rescindir; cancelar (uma opinião, ordem, etc.)}
\end{EntryWithPhonetic}

\begin{EntryWithPhonetic}{收获}{shou1huo4}{6,10}{⽁、⾋}[HSK 4]
  \definition[次,番,份]{s.}{resultados; ganhos; metaforicamente falando, conhecimento, experiência, etc. obtidos em estudo ou trabalho; os resultados obtidos por meio de trabalho árduo | colheita; colheita de safras}
  \definition{v.}{colher; juntar as colheitas}
\end{EntryWithPhonetic}

\begin{EntryWithPhonetic}{收集}{shou1 ji2}{6,12}{⽁、⾫}[HSK 5]
  \definition{v.}{coletar; reunir; recolher}
\end{EntryWithPhonetic}

\begin{EntryWithPhonetic}{收据}{shou1ju4}{6,11}{⽁、⼿}
  \definition[张]{s.}{recibo | \emph{voucher}}
\end{EntryWithPhonetic}

\begin{EntryWithPhonetic}{收看}{shou1 kan4}{6,9}{⽁、⽬}[HSK 3]
  \definition{v.}{assistir (a um programa de TV)}
\end{EntryWithPhonetic}

\begin{EntryWithPhonetic}{收敛}{shou1lian3}{6,11}{⽁、⽁}
  \definition{v.}{diminuir | desaparecer | fazer desaparecer | exercer restrição | conter (alegria, arrogância, etc.) | constringir | (matemática) convergir}
\end{EntryWithPhonetic}

\begin{EntryWithPhonetic}{收买}{shou1mai3}{6,6}{⽁、⼄}
  \definition{v.}{subornar | comprar}
\end{EntryWithPhonetic}

\begin{EntryWithPhonetic}{收取}{shou1 qu3}{6,8}{⽁、⼜}[HSK 6]
  \definition{v.}{obter; coletar; receber; aceitar o dinheiro pago pela outra parte}
\end{EntryWithPhonetic}

\begin{EntryWithPhonetic}{收入}{shou1ru4}{6,2}{⽁、⼊}[HSK 2]
  \definition[笔,个]{s.}{renda; salário; dinheiro recebido}
  \definition{v.}{receber dinheiro | coletar; receber}
\end{EntryWithPhonetic}

\begin{EntryWithPhonetic}{收拾}{shou1shi5}{6,9}{⽁、⼿}[HSK 5]
  \definition{v.}{arrumar; empacotar; limpar; organizar, policiar, restaurar a normalidade em situações adversas | consertar; reparar; restaurar algo que está danificado ao seu estado ou função original |  punir; punir alguém, geralmente com medidas mais severas | matar}
\end{EntryWithPhonetic}

\begin{EntryWithPhonetic}{收听}{shou1 ting1}{6,7}{⽁、⼝}[HSK 3]
  \definition{v.}{ouvir (rádio)}
\end{EntryWithPhonetic}

\begin{EntryWithPhonetic}{收养}{shou1 yang3}{6,9}{⽁、⼋}[HSK 6]
  \definition{v.}{acolher e criar; adotar; acolher os filhos dos outros e criá-los como se fossem da sua própria família}
\end{EntryWithPhonetic}

\begin{EntryWithPhonetic}{收益}{shou1yi4}{6,10}{⽁、⽫}[HSK 4]
  \definition{s.}{lucro; renda; benefício; ganhos; vantagens ou benefícios obtidos}
\end{EntryWithPhonetic}

\begin{EntryWithPhonetic}{收音机}{shou1yin1ji1}{6,9,6}{⽁、⾳、⽊}[HSK 3]
  \definition[部,台]{s.}{rádio; sem fio; um termo geral para receptores de rádio}
\end{EntryWithPhonetic}

\begin{EntryWithPhonetic}{手}{shou3}{4}{⼿}[HSK 1][Kangxi 64]
  \definition{adj.}{prático; conveniente}
  \definition{adv.}{pessoalmente | para habilidade ou destreza}
  \definition{clas.}{usado para habilidades e competências | usado para indicar o número de vezes em que algo foi feito}
  \definition[双,只]{s.}{mão | pessoa proficiente em determinada atividade | habilidade; meios; referência a habilidades, técnicas ou meios | uma pessoa que faz ou é boa em determinado trabalho}
  \definition{v.}{ter na mão; segurar}
\end{EntryWithPhonetic}

\begin{EntryWithPhonetic}{手臂}{shou3bi4}{4,17}{⼿、⾁}
  \definition{s.}{braço}
\end{EntryWithPhonetic}

\begin{EntryWithPhonetic}{手边}{shou3bian1}{4,5}{⼿、⾡}
  \definition{adv.}{à mão | na mão}
\end{EntryWithPhonetic}

\begin{EntryWithPhonetic}{手表}{shou3biao3}{4,8}{⼿、⾐}[HSK 2]
  \definition[块,只,个]{s.}{relógio de pulso}
\end{EntryWithPhonetic}

\begin{EntryWithPhonetic}{手段}{shou3 duan4}{4,9}{⼿、⽎}[HSK 5]
  \definition[种,个]{s.}{meios; meio; medida; método; métodos e técnicas utilizados para atingir um determinado objetivo | truque; artifício; métodos inadequados de lidar com as pessoas | habilidade; capacidade; delicadeza; sutileza; técnica}
\end{EntryWithPhonetic}

\begin{EntryWithPhonetic}{手法}{shou3fa3}{4,8}{⼿、⽔}[HSK 5]
  \definition[种,个]{s.}{habilidade; técnica; técnicas de criação (de obras literárias e artísticas) | truque; artifício; artimanha; refere-se a métodos inadequados usados para lidar com as pessoas}
\end{EntryWithPhonetic}

\begin{EntryWithPhonetic}{手工}{shou3gong1}{4,3}{⼿、⼯}[HSK 4]
  \definition{s.}{trabalho manual; trabalho feito à mão | método de operação manual; método manual, sem máquina | remuneração por trabalho manual, braçal; custo de mão de obra braçal}
\end{EntryWithPhonetic}

\begin{EntryWithPhonetic}{手工艺人}{shou3gong1 yi4ren2}{4,3,4,2}{⼿、⼯、⾋、⼈}
  \definition{s.}{artesão}
\end{EntryWithPhonetic}

\begin{EntryWithPhonetic}{手机}{shou3ji1}{4,6}{⼿、⽊}[HSK 1]
  \definition[部,台,个]{s.}{celular; telefone celular; telefone móvel}
\end{EntryWithPhonetic}

\begin{EntryWithPhonetic}{手里}{shou3 li3}{4,7}{⼿、⾥}[HSK 4]
  \definition[个]{s.}{(uma situação está) nas mãos de alguém | em mãos}
\end{EntryWithPhonetic}

\begin{EntryWithPhonetic}{手刹}{shou3sha1}{4,8}{⼿、⼑}
  \definition{s.}{freio de mão}
\end{EntryWithPhonetic}

\begin{EntryWithPhonetic}{手术}{shou3shu4}{4,5}{⼿、⽊}[HSK 4]
  \definition[个,次]{s.}{cirurgia; operação (cirúrgica); método de tratamento no qual o médico usa uma faca, tesoura etc. para fazer uma incisão em uma parte do corpo do paciente}
  \definition{v.}{realizar uma cirurgia}
\end{EntryWithPhonetic}

\begin{EntryWithPhonetic}{手套}{shou3tao4}{4,10}{⼿、⼤}[HSK 4]
  \definition[副,套,双,种]{s.}{luvas; itens usados ​​nas mãos, feitos de algodão, lã, couro, etc., para proteger as mãos ou manter o frio longe}
\end{EntryWithPhonetic}

\begin{EntryWithPhonetic}{手续}{shou3xu4}{4,11}{⼿、⽷}[HSK 3]
  \definition[项]{s.}{processo; formalidade; procedimento; procedimentos realizados de acordo com os regulamentos}
\end{EntryWithPhonetic}

\begin{EntryWithPhonetic}{手续费}{shou3 xu4 fei4}{4,11,9}{⼿、⽷、⾙}[HSK 6]
  \definition{s.}{comissão; corretagem; taxa de serviço; taxas a pagar pelos procedimentos de manuseio}
\end{EntryWithPhonetic}

\begin{EntryWithPhonetic}{手指}{shou3zhi3}{4,9}{⼿、⼿}[HSK 3]
  \definition[个,根,只]{s.}{dedo da mão}
\end{EntryWithPhonetic}

\begin{EntryWithPhonetic}{守}{shou3}{6}{⼧}[HSK 4]
  \definition*{s.}{Sobrenome Shou}
  \definition{adv.}{próximo; perto de; perto de algum lugar em posição, perto de algum lugar}
  \definition{v.}{guardar; defender; estar presente para cuidar; não ir embora | manter vigilância; defender do ataque do oponente em uma luta ou confronto | observar; cumprir; respeitar; fazer as coisas como elas devem ser feitas | manter, observar a integridade; honrar a palavra de alguém; manter a palavra de alguém}
\end{EntryWithPhonetic}

\begin{EntryWithPhonetic}{守门员}{shou3men2yuan2}{6,3,7}{⼧、⾨、⼝}
  \definition{s.}{goleiro}
\end{EntryWithPhonetic}

\begin{EntryWithPhonetic}{首}{shou3}{9}{⾸}[HSK 4,6][Kangxi 185]
  \definition*{s.}{Sobrenome Shou}
  \definition{adj.}{primeiro}
  \definition{adv.}{inicialmente; como o primeiro; em primeiro lugar}
  \definition{clas.}{usado para canções e poemas}
  \definition{s.}{cabeça | cabeça; chefe; líder | capital (cidade)}
  \definition{v.}{apresentar acusações contra alguém}
\end{EntryWithPhonetic}

\begin{EntryWithPhonetic}{首次}{shou3 ci4}{9,6}{⾸、⽋}[HSK 6]
  \definition{s.}{o primeiro; pela primeira vez}
\end{EntryWithPhonetic}

\begin{EntryWithPhonetic}{首都}{shou3du1}{9,10}{⾸、⾢}[HSK 3]
  \definition[个,座]{s.}{capital (cidade); a sede do mais alto poder político do país e o centro político do país}
\end{EntryWithPhonetic}

\begin{EntryWithPhonetic}{首脑}{shou3 nao3}{9,10}{⾸、⾁}[HSK 6]
  \definition[位]{s.}{cabeça; líder; chefe}
\end{EntryWithPhonetic}

\begin{EntryWithPhonetic}{首席}{shou3 xi2}{9,10}{⾸、⼱}[HSK 6]
  \definition{adj.}{chefe; a primeira; a posição mais alta}
  \definition{s.}{assento de honra; o assento mais honroso}
\end{EntryWithPhonetic}

\begin{EntryWithPhonetic}{首席执行官}{shou3xi2 zhi2xing2 guan1}{9,10,6,6,8}{⾸、⼱、⼿、⾏、⼧}
  \definition{s.}{\emph{chief executive officer}, CEO}
\end{EntryWithPhonetic}

\begin{EntryWithPhonetic}{首先}{shou3xian1}{9,6}{⾸、⼉}[HSK 3]
  \definition{adv.}{primeiramente; antes de todos os outros}
  \definition{conj.}{acima de tudo; primeiramente; em primeiro lugar}
\end{EntryWithPhonetic}

\begin{EntryWithPhonetic}{首相}{shou3 xiang4}{9,9}{⾸、⽬}[HSK 6]
  \definition*[个,名,位]{s.}{Primeiro-Ministro (Japão, UK, etc.); o mais alto cargo oficial no gabinete de uma monarquia; o chefe do governo central de alguns países não monárquicos às vezes usa esse nome}
\end{EntryWithPhonetic}

\begin{EntryWithPhonetic}{掱}{shou3}{12}{⼿}
  \variantof{手}
\end{EntryWithPhonetic}

\begin{EntryWithPhonetic}{寿}{shou4}{7}{⼨}
  \definition[个,份]{s.}{vida longa; velhice | vida; idade | aniversário | (eufenismo) funerário; preparado antes da morte | longevidade}
\end{EntryWithPhonetic}

\begin{EntryWithPhonetic}{寿司}{shou4 si1}{7,5}{⼨、⼝}[HSK 5]
  \definition[份]{s.}{\emph{sushi}; iguaria tradicional japonesa}
\end{EntryWithPhonetic}

\begin{EntryWithPhonetic}{受}{shou4}{8}{⼜}[HSK 3]
  \definition{v.}{receber; aceitar | sofrer; ser submetido a | aguentar; suportar; tolerar | ser agradável}
\end{EntryWithPhonetic}

\begin{EntryWithPhonetic}{受不了}{shou4bu5liao3}{8,4,2}{⼜、⼀、⼅}[HSK 4]
  \definition{v.}{ser insuportável; não poder suportar algo; não suportar algo}
\end{EntryWithPhonetic}

\begin{EntryWithPhonetic}{受到}{shou4dao4}{8,8}{⼜、⼑}[HSK 2]
  \definition{v.}{receber; receber itens, mensagens, instruções, etc. fornecidos por outras pessoas}
\end{EntryWithPhonetic}

\begin{EntryWithPhonetic}{受得了}{shou4de5liao3}{8,11,2}{⼜、⼻、⼅}
  \definition{v.}{suportar | aguentar}
\end{EntryWithPhonetic}

\begin{EntryWithPhonetic}{受伤}{shou4shang1}{8,6}{⼜、⼈}[HSK 3]
  \definition{v.}{ser ferido; sofrer uma lesão}
\end{EntryWithPhonetic}

\begin{EntryWithPhonetic}{受限}{shou4xian4}{8,8}{⼜、⾩}
  \definition{v.}{ser limitado | ser restrito | ser constrangido}
\end{EntryWithPhonetic}

\begin{EntryWithPhonetic}{受灾}{shou4 zai1}{8,7}{⼜、⽕}[HSK 5]
  \definition{v.}{ser atingido por um desastre natural (ou calamidade) | ser atingido por uma adversidade natural}
\end{EntryWithPhonetic}

\begin{EntryWithPhonetic}{兽}{shou4}{11}{⼋}
  \definition{adj.}{bestial; brutal}
  \definition{s.}{besta; animal}
\end{EntryWithPhonetic}

\begin{EntryWithPhonetic}{兽力车}{shou4 li4 che1}{11,2,4}{⼋、⼒、⾞}
  \definition{s.}{veículo puxado por animais  (oposto a 人力车) | carruagem; carroça}
  \seealsoref{人力车}{ren2 li4 che1}
\end{EntryWithPhonetic}

\begin{EntryWithPhonetic}{兽行}{shou4xing2}{11,6}{⼋、⾏}
  \definition{s.}{ato brutal; brutalidade | bestialidade}
\end{EntryWithPhonetic}

\begin{EntryWithPhonetic}{售}{shou4}{11}{⼝}
  \definition{v.}{vender | fazer (o plano, truque, etc.) funcionar; continuar (as intrigas) | realizar (intrigas)}
\end{EntryWithPhonetic}

\begin{EntryWithPhonetic}{售货员}{shou4huo4yuan2}{11,8,7}{⼝、⾙、⼝}[HSK 4]
  \definition[名,位]{s.}{vendedor; balconista; assistente de loja; equipe que vende produtos em lojas}
\end{EntryWithPhonetic}

\begin{EntryWithPhonetic}{瘦}{shou4}{14}{⽧}[HSK 5]
  \definition{adj.}{magro; esquelético (oposto de 胖, 肥) | magro (oposto de 肥) | apertado (oposto de 肥) | infértil; pobre | esquelético; pouca gordura; pouca carne (em oposição a 或 ou 肥) | (roupas, sapatos, meias, etc.) apertado (em oposição a 肥) |magra; (carne comestível) com baixo teor de gordura (em oposição a 肥)}
  \definition{v.}{perder peso}
  \seealsoref{肥}{fei2}
  \seealsoref{或}{huo4}
  \seealsoref{胖}{pang4}
\end{EntryWithPhonetic}

\begin{EntryWithPhonetic}{书}{shu1}{4}{⼄}[HSK 1]
  \definition*{s.}{Sobrenome Shu}
  \definition[本,册,部,套,卷]{s.}{livro; obras encadernadas | carta; carta especial | documento | estilo de caligrafia; escrita}
  \definition{v.}{escrever; registrar}
\end{EntryWithPhonetic}

\begin{EntryWithPhonetic}{书包}{shu1 bao1}{4,5}{⼄、⼓}[HSK 1]
  \definition[个,款]{s.}{mochila para guardar livros e materiais escolares}
\end{EntryWithPhonetic}

\begin{EntryWithPhonetic}{书店}{shu1 dian4}{4,8}{⼄、⼴}[HSK 1]
  \definition[个,家]{s.}{livraria; lojas que vendem livros}
\end{EntryWithPhonetic}

\begin{EntryWithPhonetic}{书法}{shu1fa3}{4,8}{⼄、⽔}[HSK 5]
  \definition[幅,卷,种,派]{s.}{caligrafia; arte de escrever caracteres, especialmente arte de escrever caracteres chineses com um pincel}
\end{EntryWithPhonetic}

\begin{EntryWithPhonetic}{书房}{shu1 fang2}{4,8}{⼄、⼾}[HSK 6]
  \definition[间]{s.}{uma biblioteca (em uma residência privada); espaço para leitura e escrita}
\end{EntryWithPhonetic}

\begin{EntryWithPhonetic}{书柜}{shu1 gui4}{4,8}{⼄、⽊}[HSK 5]
  \definition{s.}{estante; armário de livros}
\end{EntryWithPhonetic}

\begin{EntryWithPhonetic}{书记}{shu1ji5}{4,5}{⼄、⾔}
  \definition{s.}{secretário (chefe de um ramo de um partido socialista ou comunista) | atendente | balconista | escriturário}
\end{EntryWithPhonetic}

\begin{EntryWithPhonetic}{书架}{shu1jia4}{4,9}{⼄、⽊}[HSK 3]
  \definition[个,种,套]{s.}{estante de livros}
\end{EntryWithPhonetic}

\begin{EntryWithPhonetic}{书桌}{shu1 zhuo1}{4,10}{⼄、⽊}[HSK 5]
  \definition[个,张]{s.}{escrivaninha; mesa para ler e escrever}
\end{EntryWithPhonetic}

\begin{EntryWithPhonetic}{叔}{shu1}{8}{⼜}
  \definition*{s.}{Sobrenome Shu}
  \definition{s.}{irmão mais novo do pai; tio (por parte de pai)| irmão mais novo do marido | terceiro entre irmãos | tio | uma forma de tratamento para um homem um pouco mais jovem que o pai; tio | terceiro tio (de quatro irmãos) | primo mais novo da mãe}
\end{EntryWithPhonetic}

\begin{EntryWithPhonetic}{叔叔}{shu1shu5}{8,8}{⼜、⼜}
  \definition[个,位,名]{s.}{tio; irmão mais novo do pai | tio, dirigindo-se a um homem da mesma geração que o pai e mais jovem em idade}
\end{EntryWithPhonetic}

\begin{EntryWithPhonetic}{疏}{shu1}{12}{⽦}
  \definition*{s.}{Sobrenome Shu}
  \definition{adj.}{fino; esparso; disperso (oposto a 密) | espalhado; disperso; difuso; a distância entre as coisas é grande; as lacunas entre as partes das coisas são grandes | distante; relacionamento distante; não próximo (de relações familiares ou sociais) | não familiarizado com; desconhecido | escasso; vazio}
  \definition{s.}{memorial; memorial ao trono; um texto em que um ministro na era feudal apresentava seus assuntos ao monarca em detalhes | comentário; anotações mais detalhadas de livros antigos do que 注}
  \definition{v.}{dragar (um rio, etc.) | negligenciar | dispersar; espalhar}
  \seealsoref{密}{mi4}
  \seealsoref{注}{zhu4}
\end{EntryWithPhonetic}

\begin{EntryWithPhonetic}{舒}{shu1}{12}{⾆}
  \definition*{s.}{Sobrenome Shu}
  \definition{adj.}{lento; vagaroso; sem pressa | confortável; relaxado e feliz}
  \definition{v.}{esticar; desdobrar | alongar; relaxar}
\end{EntryWithPhonetic}

\begin{EntryWithPhonetic}{舒服}{shu1fu5}{12,8}{⾆、⽉}[HSK 2]
  \definition{adj.}{confortável; sentir-se relaxado e feliz, tanto física quanto mentalmente}
\end{EntryWithPhonetic}

\begin{EntryWithPhonetic}{舒适}{shu1shi4}{12,9}{⾆、⾡}[HSK 4]
  \definition{adj.}{aconchegante; confortável; acolhedor; cômodo}
\end{EntryWithPhonetic}

\begin{EntryWithPhonetic}{输}{shu1}{13}{⾞}[HSK 3]
  \definition{v.}{transportar; entregar | contribuir com dinheiro; doar | perder; falhar; ser batido; ser derrotado}
\end{EntryWithPhonetic}

\begin{EntryWithPhonetic}{输出}{shu1 chu1}{13,5}{⾞、⼐}[HSK 5]
  \definition{v.}{exportar (de dentro para fora); transportar (de dentro) para fora | exportar; vender ou distribuir no exterior ou fora do país | emitir informações, programas, dados, sinais, etc. a partir de uma máquina; enviar por uma determinada instituição ou dispositivo (energia, sinal, etc.)}
\end{EntryWithPhonetic}

\begin{EntryWithPhonetic}{输入}{shu1ru4}{13,2}{⾞、⼊}[HSK 3]
  \definition{v.}{introduzir; importar; comprar bens, introduzir tecnologia, contratar mão de obra, introduzir capital, etc. | inserir informações, programas, dados, sinais, etc. em uma máquina}
\end{EntryWithPhonetic}

\begin{EntryWithPhonetic}{蔬}{shu1}{15}{⾋}
  \definition{s.}{vegetais}
\end{EntryWithPhonetic}

\begin{EntryWithPhonetic}{蔬菜}{shu1cai4}{15,11}{⾋、⾋}[HSK 5]
  \definition[样,种]{s.}{verduras; legumes; vegetais; ervas que podem ser usadas na culinária}
\end{EntryWithPhonetic}

\begin{EntryWithPhonetic}{熟}{shu2}{15}{⽕}[HSK 2]
  \definition{adj.}{maduro (frutos) | pronto; cozido | processado, fabricado ou exercitado | familiar, bem conhecido; conhecido por ser comum ou frequentemente utilizado | habilidoso;  (trabalho, tecnologia) experiente; não é novato | profundo; sólido}
\end{EntryWithPhonetic}

\begin{EntryWithPhonetic}{熟练}{shu2lian4}{15,8}{⽕、⽷}[HSK 4]
  \definition{adj.}{especializado; proficiente; qualificado; habilidoso}
\end{EntryWithPhonetic}

\begin{EntryWithPhonetic}{熟人}{shu2 ren2}{15,2}{⽕、⼈}[HSK 3]
  \definition[位,名,个,些]{s.}{amigo; conhecido; pessoas que se conhecem há muito tempo; pessoas que são muito familiares}
\end{EntryWithPhonetic}

\begin{EntryWithPhonetic}{熟悉}{shu2xi1}{15,11}{⽕、⼼}[HSK 5]
  \definition{adj.}{familiarizado com; não ser estranho}
  \definition{v.}{estar familiarizado com; saber claramente que | conhecer bem algo ou alguém; compreender e dominar (a situação) através da observação ou da experiência}
\end{EntryWithPhonetic}

\begin{EntryWithPhonetic}{属}{shu3}{12}{⼫}[HSK 3]
  \definition{s.}{categoria | gênero | membros da família; dependentes; familiares; parentes}
  \definition{v.}{estar sob; subordinado a | pertencer a | nascer no ano de (um dos doze animais do zodíaco)}
  \seeref{zhu3}
\end{EntryWithPhonetic}

\begin{EntryWithPhonetic}{属于}{shu3yu2}{12,3}{⼫、⼆}[HSK 3]
  \definition{v.}{pertencer a; fazer parte de; pertencer ou ser propriedade de uma determinada parte}
\end{EntryWithPhonetic}

\begin{EntryWithPhonetic}{暑}{shu3}{12}{⽇}
  \definition{adj.}{calor; clima quente; quente (em oposição a 寒)}
  \definition{s.}{verão}
  \seealsoref{寒}{han2}
\end{EntryWithPhonetic}

\begin{EntryWithPhonetic}{暑假}{shu3 jia4}{12,11}{⽇、⼈}[HSK 4]
  \definition[个]{s.}{férias de verão; feriado de verão; férias escolares de verão, na China, durante o sétimo e o oitavo meses do calendário gregoriano}
\end{EntryWithPhonetic}

\begin{EntryWithPhonetic}{黍}{shu3}{12}{⿉}[Kangxi 202]
  \definition{s.}{painço}
\end{EntryWithPhonetic}

\begin{EntryWithPhonetic}{数}{shu3}{13}{⽁}[HSK 2]
  \definition{v.}{contar (número); contar (número) um a um | ser considerado excepcionalmente (bom, ruim, etc.) | enumerar; listar}
  \seeref{shu4}
  \seeref{shuo4}
\end{EntryWithPhonetic}

\begin{EntryWithPhonetic}{鼠}{shu3}{13}{⿏}[HSK 5][Kangxi 208]
  \definition[只]{s.}{rato; camundongo}
\end{EntryWithPhonetic}

\begin{EntryWithPhonetic}{鼠标}{shu3biao1}{13,9}{⿏、⽊}[HSK 5]
  \definition[个,只]{s.}{\emph{mouse} (de computador); dispositivo de entrada externo para computadores, usado para controlar o movimento do cursor na tela do computador, selecionar objetos de operação, executar vários comandos, etc.}
\end{EntryWithPhonetic}

\begin{EntryWithPhonetic}{薯}{shu3}{16}{⾋}
  \definition{s.}{batata | inhame}
\end{EntryWithPhonetic}

\begin{EntryWithPhonetic}{薯片}{shu3 pian4}{16,4}{⾋、⽚}[HSK 6]
  \definition{s.}{batatas fritas (\emph{chips}); batatas fritas crocantes ; flocos finos feitos de batatas}
\end{EntryWithPhonetic}

\begin{EntryWithPhonetic}{薯条}{shu3 tiao2}{16,7}{⾋、⽊}[HSK 6]
  \definition{s.}{batatas fritas (palito)}
\end{EntryWithPhonetic}

\begin{EntryWithPhonetic}{术}{shu4}{5}{⽊}
  \definition*{s.}{Sobrenome Shu}
  \definition{s.}{arte; habilidade; técnica; tecnologia; acadêmico | método; tática; estratégia}
  \seeref{zhu2}
\end{EntryWithPhonetic}

\begin{EntryWithPhonetic}{术科}{shu4ke1}{5,9}{⽊、⽲}
  \definition{s.}{cursos técnicos oferecidos em treinamento militar ou físico (oposto a 学科)}
  \seealsoref{学科}{xue2 ke1}
\end{EntryWithPhonetic}

\begin{EntryWithPhonetic}{束}{shu4}{7}{⽊}[HSK 3]
  \definition*{s.}{Sobrenome Shu}
  \definition{clas.}{usado para cachos, molhos, feixes, feixes de luz, etc.}
  \definition{s.}{monte; pacote; maço; feixe; cacho; coisas agrupadas ou reunidas em tiras}
  \definition{v.}{atar; amarrar; vincular | controlar; restringir}
\end{EntryWithPhonetic}

\begin{EntryWithPhonetic}{束腰}{shu4yao1}{7,13}{⽊、⾁}
  \definition{s.}{cinto | cinta | cinturão}
\end{EntryWithPhonetic}

\begin{EntryWithPhonetic}{树}{shu4}{9}{⽊}[HSK 1]
  \definition*{s.}{Sobrenome Shu}
  \definition[棵,株]{s.}{árvore; nome comum das plantas lenhosas}
  \definition{v.}{plantar; cultivar | configurar; manter; estabelecer}
\end{EntryWithPhonetic}

\begin{EntryWithPhonetic}{树林}{shu4 lin2}{9,8}{⽊、⽊}[HSK 4]
  \definition[片,座]{s.}{bosque; muitas árvores que crescem em fragmentos, menores que as florestas}
\end{EntryWithPhonetic}

\begin{EntryWithPhonetic}{树莓}{shu4mei2}{9,10}{⽊、⾋}
  \definition{s.}{framboesa}
\end{EntryWithPhonetic}

\begin{EntryWithPhonetic}{树木}{shu4mu4}{9,4}{⽊、⽊}
  \definition{s.}{árvore}
\end{EntryWithPhonetic}

\begin{EntryWithPhonetic}{树叶}{shu4ye4}{9,5}{⽊、⼝}[HSK 4]
  \definition[片,枚,堆]{s.}{folha; folhagem}
\end{EntryWithPhonetic}

\begin{EntryWithPhonetic}{竖}{shu4}{9}{⽴}
  \definition*{s.}{Sobrenome Shu}
  \definition{adj.}{vertical; ereto; perpendicular ao solo}
  \definition{s.}{traço vertical (em caracteres chineses) | empregados domésticos; jovens criados}
  \definition{v.}{colocar em pé; erguer; ficar de pé; colocar o objeto perpendicular ao solo}
\end{EntryWithPhonetic}

\begin{EntryWithPhonetic}{竖向}{shu4xiang4}{9,6}{⽴、⼝}
  \definition{adj.}{vertical}
\end{EntryWithPhonetic}

\begin{EntryWithPhonetic}{庶}{shu4}{11}{⼴}
  \definition*{s.}{Sobrenome Shu}
  \definition{adj.}{multitudinário; numeroso}
  \definition{conj.}{para que; de ​​modo a}
  \definition{s.}{da ou pela concubina (diferentemente da esposa legal); no sistema patriarcal, refere-se ao ramo lateral da família}
\end{EntryWithPhonetic}

\begin{EntryWithPhonetic}{庶民}{shu4min2}{11,5}{⼴、⽒}
  \definition{s.}{(antigo) pessoas comuns | (antigo) plebeu; plebeus | (antigo) a multidão de pessoas comuns (na literatura erudita)}
\end{EntryWithPhonetic}

\begin{EntryWithPhonetic}{数}{shu4}{13}{⽁}
  \definition{num.}{vários; alguns}
  \definition{s.}{número; cifra; figura | número (conceito matemático básico que representa a quantidade de coisas) | número; indica a quantidade de coisas a que se referem os substantivos ou pronomes | destino; sorte}
  \seeref{shu3}
  \seeref{shuo4}
\end{EntryWithPhonetic}

\begin{EntryWithPhonetic}{数据}{shu4ju4}{13,11}{⽁、⼿}[HSK 4]
  \definition[组,个,条]{s.}{dados; valores com base nos quais são realizadas estatísticas, cálculos, pesquisas científicas ou projetos técnicos}
\end{EntryWithPhonetic}

\begin{EntryWithPhonetic}{数量}{shu4liang4}{13,12}{⽁、⾥}[HSK 3]
  \definition[个,种]{s.}{quantidade; quantum; quantia; magnitude; número}
\end{EntryWithPhonetic}

\begin{EntryWithPhonetic}{数码}{shu4ma3}{13,8}{⽁、⽯}[HSK 4]
  \definition{s.}{dígito; numeral; algarismo | número; quantidade (usado principalmente na linguagem falada)}
  \definition{v.}{digitalizar}
\end{EntryWithPhonetic}

\begin{EntryWithPhonetic}{数目}{shu4 mu4}{13,5}{⽁、⽬}[HSK 5]
  \definition{s.}{número; quantidade; quantidade de algo expressa em uma determinada medida padrão (como unidades de medida, etc.)}
\end{EntryWithPhonetic}

\begin{EntryWithPhonetic}{数学}{shu4xue2}{13,8}{⽁、⼦}
  \definition{s.}{matemática; a ciência que estuda as formas espaciais e as relações quantitativas do mundo real, incluindo matemática elementar e matemática superior}
\end{EntryWithPhonetic}

\begin{EntryWithPhonetic}{数字}{shu4zi4}{13,6}{⽁、⼦}[HSK 2]
  \definition{adj.}{digital; usando tecnologia digital}
  \definition[个,串]{s.}{dígito; número; um caractere que representa um número | numeral; símbolos que representam números, como algarismos arábicos, algarismos romanos, etc. | quantidade; montante}
\end{EntryWithPhonetic}

\begin{EntryWithPhonetic}{刷}{shua1}{8}{⼑}[HSK 4]
  \definition{s.}{escova; pincel | (onomatopéia) farfalhar; descreve o som de uma passagem rápida}
  \definition{v.}{escovar; esfregar; remover com uma escova | borrar; colar; aplicar com um pincel | eliminar; remover; limpar | rolar; navegar; visualizar grandes quantidades de informações muito rapidamente em um curto período de tempo online ou em dispositivos móveis | deslizar (passar o cartão magnético)}
  \seeref{shua4}
\end{EntryWithPhonetic}

\begin{EntryWithPhonetic}{刷牙}{shua1ya2}{8,4}{⼑、⽛}[HSK 4]
  \definition{s.}{escovar os dentes}
\end{EntryWithPhonetic}

\begin{EntryWithPhonetic}{刷子}{shua1zi5}{8,3}{⼑、⼦}[HSK 4]
  \definition[把,个]{s.}{escova; escovão; utensílio feito de lã, fio de plástico, fio de metal, etc., para remover sujeira ou aplicar óleo de unção, etc., geralmente longo ou oval, alguns com alças}
\end{EntryWithPhonetic}

\begin{EntryWithPhonetic}{耍}{shua3}{9}{⽽}
  \definition{v.}{brincar com | empunhar | agir (legal, calmo, tranquilo, descolado, etc.) | exibir (uma habilidade, o temperamento de alguém, etc.)}
\end{EntryWithPhonetic}

\begin{EntryWithPhonetic}{耍赖}{shua3lai4}{9,13}{⽽、⾙}
  \definition{v.}{agir descaradamente | recusar -se a reconhecer que alguém perdeu o jogo ou fez uma promessa, etc. | agir como um idiota | agir como se algo nunca tivesse acontecido}
\end{EntryWithPhonetic}

\begin{EntryWithPhonetic}{刷}{shua4}{8}{⼑}
  \definition{adj.}{pálido ou branco-azulado}
  \definition{adv.}{bastante; completamente; extremamente; descreve movimentos ágeis}
  \seeref{shua1}
\end{EntryWithPhonetic}

\begin{EntryWithPhonetic}{摔}{shuai1}{14}{⼿}[HSK 5]
  \definition{v.}{cair; tropeçar; perder o equilíbrio | mergulhar; precipitar-se; cair de uma altura elevada | quebrar; fazer cair e quebrar | lançar; atirar; arremessar; joguar coisas com força e para baixo | bater; golpear; bater com força para que o que está grudado cair}
\end{EntryWithPhonetic}

\begin{EntryWithPhonetic}{摔倒}{shuai1dao3}{14,10}{⼿、⼈}[HSK 5]
  \definition{v.}{cair; tropeçar; perder o equilíbrio e cair}
\end{EntryWithPhonetic}

\begin{EntryWithPhonetic}{帅}{shuai4}{5}{⼱}[HSK 4]
  \definition*{s.}{Sobrenome Shuai}
  \definition{adj.}{bonito; arrojado; elegante; inteligente}
  \definition{interj.}{Legal!}
  \definition[位,名,个,些]{s.}{comandante em chefe; o mais alto comandante do exército | comandante em chefe, a peça principal no xadrez chinês}
\end{EntryWithPhonetic}

\begin{EntryWithPhonetic}{帅哥}{shuai4 ge1}{5,10}{⼱、⼝}[HSK 4]
  \definition[个,位,名,些]{s.}{rapaz bonito; um garoto que é bonito e atraente na aparência}
\end{EntryWithPhonetic}

\begin{EntryWithPhonetic}{率}{shuai4}{11}{⽞}
  \definition*{s.}{Sobrenome Shuai}
  \definition{adj.}{precipitado; não cuidadoso; não cauteloso | franco; direto | elegante; bonito; o mesmo que 帅}
  \definition{adv.}{geralmente; expressa uma estimativa incerta, equivalente a 大约 e 大抵}
  \definition{s.}{modelo; exemplo}
  \definition{v.}{liderar; comandar | obedecer; seguir}
  \seeref{lv4}
  \seealsoref{大抵}{da4di3}
  \seealsoref{大约}{da4yue1}
  \seealsoref{帅}{shuai4}
\end{EntryWithPhonetic}

\begin{EntryWithPhonetic}{率领}{shuai4ling3}{11,11}{⽞、⾴}[HSK 5]
  \definition{v.}{liderar (equipe ou grupo); chefiar; comandar}
\end{EntryWithPhonetic}

\begin{EntryWithPhonetic}{率先}{shuai4 xian1}{11,6}{⽞、⼉}[HSK 4]
  \definition{v.}{tomar a iniciativa de fazer algo; ser o primeiro a fazer algo; assumir a liderança}
\end{EntryWithPhonetic}

\begin{EntryWithPhonetic}{双}{shuang1}{4}{⼜}[HSK 3]
  \definition*{s.}{Sobrenome Shuang}
  \definition{adj.}{dois; gêmeo; par; dual; em oposição a 单 | números pares | duplo; dobro}
  \definition{clas.}{usado para certos membros, órgãos ou coisas pareadas que são bilateralmente simétricas, por exemplo, sapatos, meias, pauzinhos, etc.}
  \seealsoref{单}{dan1}
\end{EntryWithPhonetic}

\begin{EntryWithPhonetic}{双层床}{shuang1ceng2chuang2}{4,7,7}{⼜、⼫、⼴}
  \definition{s.}{beliche}
\end{EntryWithPhonetic}

\begin{EntryWithPhonetic}{双打}{shuang1 da3}{4,5}{⼜、⼿}[HSK 6]
  \definition[场,局,次]{s.}{duplas (em esportes)}
\end{EntryWithPhonetic}

\begin{EntryWithPhonetic}{双方}{shuang1fang1}{4,4}{⼜、⽅}[HSK 3]
  \definition{s.}{ambos os lados; as duas partes; duas pessoas ou dois grupos frente a frente em um determinado relacionamento ou situação}
\end{EntryWithPhonetic}

\begin{EntryWithPhonetic}{双方同意}{shuang1fang1tong2yi4}{4,4,6,13}{⼜、⽅、⼝、⼼}
  \definition{s.}{acordo bilateral}
\end{EntryWithPhonetic}

\begin{EntryWithPhonetic}{双手}{shuang1 shou3}{4,4}{⼜、⼿}[HSK 5]
  \definition{s.}{com as duas mãos; ambas as mãos; par de mãos}
\end{EntryWithPhonetic}

\begin{EntryWithPhonetic}{霜}{shuang1}{17}{⾬}
  \definition{s.}{geada | pó branco ou creme espalhado por uma superfície | glacê | creme de pele}
\end{EntryWithPhonetic}

\begin{EntryWithPhonetic}{爽}{shuang3}{11}{⽘}[HSK 6]
  \definition{adj.}{claro; nítido; brilhante | franco; de coração aberto; direto | relaxado; confortável}
  \definition{v.}{desviar; afastar | tornar confortável; ficar confortável}
\end{EntryWithPhonetic}

\begin{EntryWithPhonetic}{谁}{shui2}{10}{⾔}[HSK 1]
  \seeref{shei2}
\end{EntryWithPhonetic}

\begin{EntryWithPhonetic}{水}{shui3}{4}{⽔}[HSK 1][Kangxi 85]
  \definition*{s.}{Etnia Shui, que vive principalmente em Guizhou | Sobrenome Shui}
  \definition{adj.}{de má qualidade; mal feito; de baixa qualidade e conteúdo}
  \definition{clas.}{usado para número de lavagens}
  \definition[条,杯]{s.}{água | rio | termo geral para rios, lagos, mares, etc.; água | corrente; fluxo de água | um líquido; suco ralo | teor de prata nas moedas | encargos adicionais ou receitas | água, um dos cinco elementos}
\end{EntryWithPhonetic}

\begin{EntryWithPhonetic}{水边}{shui3bian1}{4,5}{⽔、⾡}
  \definition{s.}{beira d'água | beira-mar | costa (de mar, lago ou rio)}
\end{EntryWithPhonetic}

\begin{EntryWithPhonetic}{水波}{shui3bo1}{4,8}{⽔、⽔}
  \definition{s.}{ondulação (na água) | onda}
\end{EntryWithPhonetic}

\begin{EntryWithPhonetic}{水槽}{shui3cao2}{4,15}{⽔、⽊}
  \definition{s.}{pia (de cozinha)}
\end{EntryWithPhonetic}

\begin{EntryWithPhonetic}{水产品}{shui3 chan3 pin3}{4,6,9}{⽔、⼇、⼝}[HSK 5]
  \definition{s.}{produto aquático (peixes, camarões, etc.)}
\end{EntryWithPhonetic}

\begin{EntryWithPhonetic}{水分}{shui3 fen4}{4,4}{⽔、⼑}[HSK 5]
  \definition{s.}{teor de umidade; água contida em um objeto | exagero; metáfora de algo falso}
\end{EntryWithPhonetic}

\begin{EntryWithPhonetic}{水果}{shui3guo3}{4,8}{⽔、⽊}[HSK 1]
  \definition[个]{s.}{fruta; um nome genérico para frutas com alto teor de água que podem ser consumidas, como peras, pêssegos, maçãs, etc.}
\end{EntryWithPhonetic}

\begin{EntryWithPhonetic}{水饺}{shui3jiao3}{4,9}{⽔、⾷}
  \definition{s.}{\emph{dumplings} | pastéis chineses cozidos}
\end{EntryWithPhonetic}

\begin{EntryWithPhonetic}{水库}{shui3 ku4}{4,7}{⽔、⼴}[HSK 5]
  \definition[座]{s.}{reservatório; lago artificial construído pelo homem, que utiliza barragens e outras estruturas para represar a água e regular o fluxo, podendo ser utilizado para armazenamento de água, geração de energia e piscicultura, entre outros fins}
\end{EntryWithPhonetic}

\begin{EntryWithPhonetic}{水灵}{shui3ling2}{4,7}{⽔、⽕}
  \definition{adj.}{cheio de vida (sobre uma pessoa, etc.) | úmido e brilhante (sobre os olhos) | fresco (sobre frutas, etc.) | brilhante | aparência saudável}
\end{EntryWithPhonetic}

\begin{EntryWithPhonetic}{水路}{shui3lu4}{4,13}{⽔、⾜}
  \definition{s.}{hidrovia}
\end{EntryWithPhonetic}

\begin{EntryWithPhonetic}{水泥}{shui3ni2}{4,8}{⽔、⽔}[HSK 6]
  \definition[袋,层]{s.}{cimento; um tipo de material mineral em pó que pode endurecer gradualmente no ar e na água após a mistura com água}
\end{EntryWithPhonetic}

\begin{EntryWithPhonetic}{水培}{shui3pei2}{4,11}{⽔、⼟}
  \definition{v.}{cultivar plantas hidroponicamente}
\end{EntryWithPhonetic}

\begin{EntryWithPhonetic}{水平}{shui3ping2}{4,5}{⽔、⼲}[HSK 2]
  \definition{adj.}{horizontal; nivelado; paralelo à superfície da água}
  \definition{s.}{padrão; nível; o nível alcançado em determinado aspecto}
\end{EntryWithPhonetic}

\begin{EntryWithPhonetic}{水平尺}{shui3ping2chi3}{4,5,4}{⽔、⼲、⼫}
  \definition{s.}{nível espiritual}
\end{EntryWithPhonetic}

\begin{EntryWithPhonetic}{水平度}{shui3ping2 du4}{4,5,9}{⽔、⼲、⼴}
  \definition{s.}{nivelamento}
\end{EntryWithPhonetic}

\begin{EntryWithPhonetic}{水平面}{shui3ping2mian4}{4,5,9}{⽔、⼲、⾯}
  \definition{s.}{plano horizontal | nível-da-água | superfície horizontal}
\end{EntryWithPhonetic}

\begin{EntryWithPhonetic}{水平视差}{shui3ping2 shi4cha1}{4,5,8,9}{⽔、⼲、⾒、⼯}
  \definition{s.}{paralaxe horizontal}
\end{EntryWithPhonetic}

\begin{EntryWithPhonetic}{水平仪}{shui3ping2yi2}{4,5,5}{⽔、⼲、⼈}
  \definition{s.}{nível (dispositivo para determinar horizontal) | nível espiritual | nível de topógrafo}
\end{EntryWithPhonetic}

\begin{EntryWithPhonetic}{水平以下}{shui3ping2 yi3xia4}{4,5,4,3}{⽔、⼲、⼈、⼀}
  \definition{s.}{sub-nível}
\end{EntryWithPhonetic}

\begin{EntryWithPhonetic}{水平轴}{shui3ping2zhou2}{4,5,9}{⽔、⼲、⾞}
  \definition{s.}{eixo horizontal}
\end{EntryWithPhonetic}

\begin{EntryWithPhonetic}{水瓶}{shui3 ping2}{4,10}{⽔、⽡}
  \definition{s.}{garrada de água}
\end{EntryWithPhonetic}

\begin{EntryWithPhonetic}{水豚}{shui3tun2}{4,11}{⽔、⾗}
  \definition{s.}{capivara}
\end{EntryWithPhonetic}

\begin{EntryWithPhonetic}{水污染}{shui3wu1ran3}{4,6,9}{⽔、⽔、⽊}
  \definition{s.}{poluição da água}
\end{EntryWithPhonetic}

\begin{EntryWithPhonetic}{水灾}{shui3zai1}{4,7}{⽔、⽕}[HSK 5]
  \definition[场,次]{s.}{inundação; desastres causados por excesso de chuvas, entre outros motivos}
\end{EntryWithPhonetic}

\begin{EntryWithPhonetic}{说}{shui4}{9}{⾔}
  \definition{v.}{persuadir}
  \seeref{shuo1}
\end{EntryWithPhonetic}

\begin{EntryWithPhonetic}{税}{shui4}{12}{⽲}[HSK 6]
  \definition*{s.}{Sobrenome Shui}
  \definition{s.}{imposto; taxa; tarifa}
\end{EntryWithPhonetic}

\begin{EntryWithPhonetic}{睡}{shui4}{13}{⽬}[HSK 1]
  \definition{v.}{dormir | deitar-se}
\end{EntryWithPhonetic}

\begin{EntryWithPhonetic}{睡觉}{shui4/jiao4}{13,9}{⽬、⾒}[HSK 1]
  \definition{v.+compl.}{dormir; ir para a cama; entrar em estado de sono}
\end{EntryWithPhonetic}

\begin{EntryWithPhonetic}{睡懒觉}{shui4lan3jiao4}{13,16,9}{⽬、⼼、⾒}
  \definition{v.}{levantar-se tarde | passar o tempo a dormir}
\end{EntryWithPhonetic}

\begin{EntryWithPhonetic}{睡眠}{shui4 mian2}{13,10}{⽬、⽬}[HSK 5]
  \definition{s.}{sono; \emph{somnus}; sonolência}
\end{EntryWithPhonetic}

\begin{EntryWithPhonetic}{睡衣}{shui4yi1}{13,6}{⽬、⾐}
  \definition{s.}{pijamas | roupas de dormir}
\end{EntryWithPhonetic}

\begin{EntryWithPhonetic}{睡着}{shui4 zhao2}{13,11}{⽬、⽬}[HSK 4]
  \definition{v.}{dormir; adormecer; cair no sono}
\end{EntryWithPhonetic}

\begin{EntryWithPhonetic}{顺}{shun4}{9}{⾴}[HSK 6]
  \definition{adj.}{(de escritos) legível; claro e bem escrito; organizado | favorável; harmonioso | favorável; bem-sucedido}
  \definition{prep.}{conforme a conveniência de alguém | ao longo; a introdução da rota, situação ou oportunidade que a ação segue pode ser seguida por 着 | com a corrente; na mesma direção |  com; na mesma direção que}
  \definition{v.}{organizar; colocar em ordem; tornar as coisas organizadas ou ordenadas | obedecer; ceder a; agir em submissão a | ser adequado; ser agradável}
  \seealsoref{着}{zhe5}
\end{EntryWithPhonetic}

\begin{EntryWithPhonetic}{顺便}{shun4bian4}{9,9}{⾴、⼈}
  \definition{adv.}{convenientemente | de passagem | sem muito esforço extra}
\end{EntryWithPhonetic}

\begin{EntryWithPhonetic}{顺畅}{shun4chang4}{9,8}{⾴、⽥}
  \definition{adj.}{liso e sem obstáculos | fluente}
\end{EntryWithPhonetic}

\begin{EntryWithPhonetic}{顺从}{shun4cong2}{9,4}{⾴、⼈}
  \definition{v.}{obedecer | submeter-se}
\end{EntryWithPhonetic}

\begin{EntryWithPhonetic}{顺当}{shun4dang5}{9,6}{⾴、⼹}
  \definition{adv.}{suavemente}
\end{EntryWithPhonetic}

\begin{EntryWithPhonetic}{顺耳}{shun4'er3}{9,6}{⾴、⽿}
  \definition{adj.}{agradável ao ouvido}
\end{EntryWithPhonetic}

\begin{EntryWithPhonetic}{顺境}{shun4jing4}{9,14}{⾴、⼟}
  \definition{s.}{circunstâncias favoráveis}
\end{EntryWithPhonetic}

\begin{EntryWithPhonetic}{顺利}{shun4li4}{9,7}{⾴、⼑}[HSK 2]
  \definition{adj.}{sem problemas; com sucesso; sem dificuldades; sem contratempos; sem obstáculos; sem obstáculos ou dificuldades significativas no desempenho das tarefas}
\end{EntryWithPhonetic}

\begin{EntryWithPhonetic}{顺水}{shun4shui3}{9,4}{⾴、⽔}
  \definition{v.}{ir com o fluxo}
\end{EntryWithPhonetic}

\begin{EntryWithPhonetic}{顺心}{shun4xin1}{9,4}{⾴、⼼}
  \definition{adj.}{satisfatório | satisfeito}
\end{EntryWithPhonetic}

\begin{EntryWithPhonetic}{顺序}{shun4xu4}{9,7}{⾴、⼴}[HSK 4]
  \definition{adv.}{por sua vez; na ordem correta; na devida ordem; na ordem adequada; na ordem apropriada}
  \definition[个]{s.}{ordem; sequência; sucessão; subsequência; sequência simples; ordem de prioridade}
\end{EntryWithPhonetic}

\begin{EntryWithPhonetic}{顺叙}{shun4xu4}{9,9}{⾴、⼜}
  \definition{s.}{narrativa cronológica}
\end{EntryWithPhonetic}

\begin{EntryWithPhonetic}{顺延}{shun4yan2}{9,6}{⾴、⼵}
  \definition{v.}{adiar | procrastinar}
\end{EntryWithPhonetic}

\begin{EntryWithPhonetic}{顺眼}{shun4yan3}{9,11}{⾴、⽬}
  \definition{adj.}{agradável aos olhos}
\end{EntryWithPhonetic}

\begin{EntryWithPhonetic}{顺嘴}{shun4zui3}{9,16}{⾴、⼝}
  \definition{v.}{deixar escapar (sem pensar) | ler suavemente (texto) | adequar-se  ao gosto (comida)}
\end{EntryWithPhonetic}

\begin{EntryWithPhonetic}{舜}{shun4}{12}{⾇}
  \definition*{s.}{Shun, o nome de um monarca lendário da China antiga | Sobrenome Shun}
\end{EntryWithPhonetic}

\begin{EntryWithPhonetic}{说}{shuo1}{9}{⾔}[HSK 1]
  \definition{s.}{uma teoria (normalmente o último caractere, como em 日心说, teoria heliocêntrica); ensinamentos; doutrina}
  \definition{v.}{falar; conversar; dizer | explicar | repreender | atuar como casamenteiro | referir-se a; indicar | criticar; aconselhar | fazer uma combinação; conciliar; mediar | discutir; falar sobre; conversar sobre | uma forma de expressão linguística da arte cênica}
  \seeref{shui4}
  \seealsoref{日心说}{ri4 xin1 shuo1}
\end{EntryWithPhonetic}

\begin{EntryWithPhonetic}{说不定}{shuo1bu5ding4}{9,4,8}{⾔、⼀、⼧}[HSK 4]
  \definition{adv.}{talvez; indica uma estimativa, possivelmente, provavelmente}
  \definition{v.}{não ter certeza; não estar certo; ser impreciso}
\end{EntryWithPhonetic}

\begin{EntryWithPhonetic}{说法}{shuo1 fa3}{9,8}{⾔、⽔}[HSK 5]
  \definition[种,个]{s.}{formulação; maneira de dizer uma coisa; formas de expressar opiniões | versão; argumento; declaração; opinião | explicação; acordo; palavras justas; razões ou fundamentos legítimos}
\end{EntryWithPhonetic}

\begin{EntryWithPhonetic}{说服}{shuo1fu2}{9,8}{⾔、⽉}[HSK 4]
  \definition{v.}{persuadir; convencer; convencer a outra parte com palavras bem fundamentadas}
\end{EntryWithPhonetic}

\begin{EntryWithPhonetic}{说好}{shuo1hao3}{9,6}{⾔、⼥}
  \definition{v.}{chegar a um acordo | concluir negociações}
\end{EntryWithPhonetic}

\begin{EntryWithPhonetic}{说话}{shuo1hua4}{9,8}{⾔、⾔}[HSK 1]
  \definition{adv.}{imediatamente; em um minuto; refere-se ao tempo que leva para falar, indicando um período muito curto}
  \definition{v.}{falar; conversar; dizer; expressar o significado através da linguagem | conversar (conversa fiada); bater papo | fofocar; conversar; criticar; censurar}
\end{EntryWithPhonetic}

\begin{EntryWithPhonetic}{说谎}{shuo1/huang3}{9,11}{⾔、⾔}
  \definition{v.+compl.}{mentir | contar uma mentira}
\end{EntryWithPhonetic}

\begin{EntryWithPhonetic}{说理}{shuo1li3}{9,11}{⾔、⽟}
  \definition{v.}{racionalizar | discutir logicamente}
\end{EntryWithPhonetic}

\begin{EntryWithPhonetic}{说明}{shuo1ming2}{9,8}{⾔、⽇}[HSK 2]
  \definition[本,个]{s.}{legenda; instrução; explicação}
  \definition{v.}{mostrar; explicar; ilustrar | indicar; mostrar; provar; demonstrar; usar materiais confiáveis para demonstrar ou determinar a autenticidade de pessoas ou coisas}
\end{EntryWithPhonetic}

\begin{EntryWithPhonetic}{说明书}{shuo1 ming2 shu1}{9,8,4}{⾔、⽇、⼄}[HSK 6]
  \definition[本]{s.}{manual; livro de instruções; descrições textuais da finalidade, especificações, desempenho e uso de itens, bem como enredos de peças e filmes, etc.}
\end{EntryWithPhonetic}

\begin{EntryWithPhonetic}{说实话}{shuo1 shi2 hua4}{9,8,8}{⾔、⼧、⾔}[HSK 6]
  \definition{v.}{falar a verdade; dizer a verdade sobre (os próprios erros ou crimes)}
\end{EntryWithPhonetic}

\begin{EntryWithPhonetic}{说完}{shuo1-wan2}{9,7}{⾔、⼧}
  \definition{expr.}{acabar/terminar palavras}
\end{EntryWithPhonetic}

\begin{EntryWithPhonetic}{硕}{shuo4}{11}{⽯}
  \definition*{s.}{Sobrenome Shuo}
  \definition{adj.}{grande; enorme}
  \definition{s.}{mestrado (MBA)}
\end{EntryWithPhonetic}

\begin{EntryWithPhonetic}{硕士}{shuo4shi4}{11,3}{⽯、⼠}[HSK 5]
  \definition[个,位,名]{s.}{mestrado; um diploma concedido por uma universidade ou faculdade a um aluno após um ou dois anos de estudo adicional após o bacharelado}
\end{EntryWithPhonetic}

\begin{EntryWithPhonetic}{数}{shuo4}{13}{⽁}
  \definition{adv.}{com frequência; repetidamente; indica uma ação frequente, equivalente a 屡次}
  \seeref{shu3}
  \seeref{shu4}
  \seealsoref{屡次}{lv3ci4}
\end{EntryWithPhonetic}

\begin{EntryWithPhonetic}{丝}{si1}{5}{⼀}
  \definition{clas.}{si, uma unidade de peso (=0,0005 gramas) | usado para expressar a aparência ou expressão de uma pessoa | um décimo de milésimo de certas unidades de medida (medida de comprimento) | usado para representar coisas abstratas}
  \definition[些,种,类,跟,缕]{s.}{seda | uma coisa semelhante a um fio; itens semelhantes à seda | cordas; instrumentos de corda}
\end{EntryWithPhonetic}

\begin{EntryWithPhonetic}{司}{si1}{5}{⼝}
  \definition*{s.}{Sobrenome Si}
  \definition{s.}{departamento (sob um ministério); um departamento dentro de uma agência de nível ministerial}
  \definition{v.}{assumir o comando de; atender; administrar; operar; gerenciar}
\end{EntryWithPhonetic}

\begin{EntryWithPhonetic}{司机}{si1ji1}{5,6}{⼝、⽊}[HSK 2]
  \definition[个,名,位]{s.}{motorista; motorista particular; chofer; motoristas de veículos de transporte público, como trens, ônibus e bondes}
\end{EntryWithPhonetic}

\begin{EntryWithPhonetic}{司长}{si1 zhang3}{5,4}{⼝、⾧}[HSK 6]
  \definition[位,名]{s.}{diretor-geral | chefe de gabinete}
\end{EntryWithPhonetic}

\begin{EntryWithPhonetic}{私}{si1}{7}{⽲}
  \definition*{s.}{Sobrenome Si}
  \definition{adj.}{pessoal; privado (oposição a 公) | egoísta (oposto a 公) | secreto; privado | ilícito; ilegal}
  \definition{s.}{interesse privado (ou egoísta); motivo (ou ideia) egoísta (oposição a 公) | contrabando; mercadorias contrabandeadas | propriedade privada | interesses privados; ganho pessoal}
  \seealsoref{公}{gong1}
\end{EntryWithPhonetic}

\begin{EntryWithPhonetic}{私函}{si1han2}{7,8}{⽲、⼐}
  \definition{s.}{carta privada}
\end{EntryWithPhonetic}

\begin{EntryWithPhonetic}{私立}{si1li4}{7,5}{⽲、⽴}
  \definition{s.}{privado; estabelecido privadamente}[这是一所私立学校。===Esta é uma escola particular.]
  \definition{v.}{estabelecer-se ilegalmente}
\end{EntryWithPhonetic}

\begin{EntryWithPhonetic}{私人}{si1ren2}{7,2}{⽲、⼈}[HSK 5]
  \definition{adj.}{privado; pertencente a um indivíduo ou exercido a título individual; não público | pessoal; entre indivíduos}
  \definition[个]{s.}{algo privado; pessoas que se aproximam de você por motivos pessoais ou interesses próprios}
\end{EntryWithPhonetic}

\begin{EntryWithPhonetic}{私人信件}{si1ren2 xin4jian4}{7,2,9,6}{⽲、⼈、⼈、⼈}
  \definition{s.}{carta pessoal}
\end{EntryWithPhonetic}

\begin{EntryWithPhonetic}{私人钥匙}{si1ren2yao4shi5}{7,2,9,11}{⽲、⼈、⾦、⼔}
  \definition{s.}{(criptografia) chave privada}
\end{EntryWithPhonetic}

\begin{EntryWithPhonetic}{私人诊所}{si1ren2 zhen3suo3}{7,2,7,8}{⽲、⼈、⾔、⼾}
  \definition[些]{s.}{clínica privada}
\end{EntryWithPhonetic}

\begin{EntryWithPhonetic}{私生活}{si1sheng1huo2}{7,5,9}{⽲、⽣、⽔}
  \definition{s.}{vida privada}
\end{EntryWithPhonetic}

\begin{EntryWithPhonetic}{私事}{si1shi4}{7,8}{⽲、⼅}
  \definition[件,桩]{s.}{privacidade; assuntos privados; assuntos pessoais (oposto a 公事)}
  \seealsoref{公事}{gong1shi4}
\end{EntryWithPhonetic}

\begin{EntryWithPhonetic}{私自}{si1zi4}{7,6}{⽲、⾃}
  \definition{adj.}{privado | pessoal}
  \definition{adv.}{secretamente | sem aprovação explícita}
\end{EntryWithPhonetic}

\begin{EntryWithPhonetic}{思}{si1}{9}{⼼}
  \definition*{s.}{Sobrenome Si}
  \definition{s.}{pensamento; ideias | pensamentos; emoções; humor}
  \definition{v.}{pensar; considerar; deliberar | pensar em; ansiar por}
\end{EntryWithPhonetic}

\begin{EntryWithPhonetic}{思考}{si1kao3}{9,6}{⼼、⽼}[HSK 4]
  \definition{v.}{pensar; ponderar; considerar; deliberar; envolver-se em atividades de pensamento, como análise, síntese, julgamento, raciocínio e generalização}
\end{EntryWithPhonetic}

\begin{EntryWithPhonetic}{思维}{si1wei2}{9,11}{⼼、⽷}[HSK 5]
  \definition[种]{s.}{pensamento; reflexão; organizar e transformar os materiais obtidos através do conhecimento sensorial para formar conceitos, julgamentos e raciocínios}
  \definition{v.}{pensar}
\end{EntryWithPhonetic}

\begin{EntryWithPhonetic}{思想}{si1xiang3}{9,13}{⼼、⼼}[HSK 3]
  \definition[个,种]{s.}{reflexão; pensamento; ideologia; a existência objetiva é refletida na consciência das pessoas por meio de atividades de pensamento, que pertencem à cognição racional | ideia; pensamento}
\end{EntryWithPhonetic}

\begin{EntryWithPhonetic}{斯}{si1}{12}{⽄}
  \definition*{s.}{Sobrenome Si}
  \definition{adv.}{então; assim}
  \definition{pron.}{isto; aqui}
\end{EntryWithPhonetic}

\begin{EntryWithPhonetic}{斯巴达}{si1ba1da2}{12,4,6}{⽄、⼰、⾡}
  \definition*{s.}{Esparta}
\end{EntryWithPhonetic}

\begin{EntryWithPhonetic}{死}{si3}{6}{⽍}[HSK 3]
  \definition{adj.}{até a morte | implacável; mortal | fixo; rígido; inflexível | intransitável; fechado | (expressando raiva, reclamação, etc., às vezes jocosamente) maldito}
  \definition{adv.}{(frequentemente no negativo) teimosamente; inflexivelmente}
  \definition{v.}{morrer; estar morto (oposto a 生 e 活)}
  \seealsoref{活}{huo2}
  \seealsoref{生}{sheng1}
\end{EntryWithPhonetic}

\begin{EntryWithPhonetic}{死亡}{si3wang2}{6,3}{⽍、⼇}[HSK 6]
  \definition{s.}{morte; condenação; dar o último suspiro; refere-se ao estado de vida desaparecendo |}
  \definition{v.}{morrer; estar morto; perder a vida (em oposição à 生存)}
  \seealsoref{生存}{sheng1cun2}
\end{EntryWithPhonetic}

\begin{EntryWithPhonetic}{四}{si4}{5}{⼞}[HSK 1]
  \definition*{s.}{Sobrenome Si}
  \definition{num.}{quatro; 4}
  \definition{s.}{uma nota da escala em Gongchepu (工尺谱), correspondente ao 6 na notação musical numerada}
  \seealsoref{工尺谱}{gong1 che3 pu3}
\end{EntryWithPhonetic}

\begin{EntryWithPhonetic}{四处}{si4 chu4}{5,5}{⼞、⼡}[HSK 6]
  \definition{adv.}{em volta; ao redor; em todos os lugares; em todas as direções}
\end{EntryWithPhonetic}

\begin{EntryWithPhonetic}{四川}{si4chuan1}{5,3}{⼞、⼮}
  \definition*{s.}{Província de Sichuan}
\end{EntryWithPhonetic}

\begin{EntryWithPhonetic}{四季分明}{si4ji4-fen1ming2}{5,8,4,8}{⼞、⼦、⼑、⽇}
  \definition{expr.}{as quatro estações são muito distintas}
\end{EntryWithPhonetic}

\begin{EntryWithPhonetic}{四季如春}{si4ji4-ru2chun1}{5,8,6,9}{⼞、⼦、⼥、⽇}
  \definition{expr.}{é primavera todo o ano | clima favorável durante todo o ano | quatro estações como a primavera}
\end{EntryWithPhonetic}

\begin{EntryWithPhonetic}{四周}{si4 zhou1}{5,8}{⼞、⼝}[HSK 5]
  \definition{s.}{ao redor; por todos os lados; a parte que circunda o centro}
\end{EntryWithPhonetic}

\begin{EntryWithPhonetic}{似}{si4}{6}{⼈}
  \definition*{s.}{Sobrenome Si}
  \definition{adv.}{parece; como se}
  \definition{v.}{ser semelhante; parecer-se com | parecer; aparecer | exceder}
  \seeref{shi4}
\end{EntryWithPhonetic}

\begin{EntryWithPhonetic}{似曾相识}{si4ceng2xiang1shi2}{6,12,9,7}{⼈、⽈、⽬、⾔}
  \definition{s.}{\emph{déjà vu} (a experiência de ver exatamente a mesma situação pela segunda vez) | situação aparentemente familiar}
\end{EntryWithPhonetic}

\begin{EntryWithPhonetic}{似乎}{si4hu1}{6,5}{⼈、⼃}[HSK 4]
  \definition{adv.}{como se; aparentemente; se parece como}
\end{EntryWithPhonetic}

\begin{EntryWithPhonetic}{寺}{si4}{6}{⼨}[HSK 6]
  \definition*{s.}{Sobrenome Si}
  \definition[座]{s.}{templo | (Islã) mesquita | (datado) ministério; agência governamental na China antiga}
\end{EntryWithPhonetic}

\begin{EntryWithPhonetic}{寺庙}{si4miao4}{6,8}{⼨、⼴}
  \definition{s.}{templo | mosteiro | santuário}
\end{EntryWithPhonetic}

\begin{EntryWithPhonetic}{伺}{si4}{7}{⼈}
  \definition{v.}{aguardar; observar; esperar por}
  \seeref{ci4}
\end{EntryWithPhonetic}

\begin{EntryWithPhonetic}{食}{si4}{9}{⾷}[Kangxi 184]
  \definition{v.}{alimentar; dar comida a}
  \seeref{shi2}
\end{EntryWithPhonetic}

\begin{EntryWithPhonetic}{肆}{si4}{13}{⾀}
  \definition*{s.}{Sobrenome Si}
  \definition{adj.}{desenfreado; sem limites; descuidado; imprudente}
  \definition{num.}{quatro (usado para o numeral 四 em cheques, etc., para evitar erros ou alterações)}
  \definition{s.}{Literário: loja; armazém}
  \seealsoref{四}{si4}
\end{EntryWithPhonetic}

\begin{EntryWithPhonetic}{厕}{si5}{8}{⼚}
  \definition{s.}{componente formador de palavras | latrina; fossa sanitária}
  \seeref{ce4}
  \seealsoref{茅厕}{mao2ce4}
\end{EntryWithPhonetic}

\begin{EntryWithPhonetic}{松}{song1}{8}{⽊}[HSK 4]
  \definition*{s.}{Sobrenome Song}
  \definition{adj.}{solto; frouxo; folgado | leve e crocante; macio | relaxado; confortável}
  \definition[棵]{s.}{pinheiro | fio de carne seca; carne moída seca; alimentos macios ou quebradiços}
  \definition{v.}{afrouxar; relaxar; abrandar | desamarrar; desatar; liberar}
\end{EntryWithPhonetic}

\begin{EntryWithPhonetic}{松木}{song1mu4}{8,4}{⽊、⽊}
  \definition{s.}{pinheiro}
\end{EntryWithPhonetic}

\begin{EntryWithPhonetic}{松树}{song1 shu4}{8,9}{⽊、⽊}[HSK 4]
  \definition[棵]{s.}{pinheiro; conífera comum, geralmente com folhas longas e pontiagudas e cones lenhosos}
\end{EntryWithPhonetic}

\begin{EntryWithPhonetic}{宋}{song4}{7}{⼧}
  \definition*{s.}{Dinastia Song (960-1279) | Song das dinastias do sul (420-479) | Sobrenome Song}
  \definition{clas.}{sone; unidade de intensidade sonora}
\end{EntryWithPhonetic}

\begin{EntryWithPhonetic}{送}{song4}{9}{⾡}[HSK 1]
  \definition*{s.}{Sobrenome Song}
  \definition{v.}{transportar; entregar | dar; dar como presente; presentear | acompanhar; despedir-se de alguém (ao sair); acompanhar a pessoa que está partindo até o destino ou caminhar um trecho com ela | escoltar}
\end{EntryWithPhonetic}

\begin{EntryWithPhonetic}{送到}{song4 dao4}{9,8}{⾡、⼑}[HSK 2]
  \definition{v.}{enviar para (lugar)}
\end{EntryWithPhonetic}

\begin{EntryWithPhonetic}{送给}{song4 gei3}{9,9}{⾡、⽷}[HSK 2]
  \definition{v.}{dar a (alguém ou organização); dar como algo gratuito; dar como presente}
\end{EntryWithPhonetic}

\begin{EntryWithPhonetic}{送礼}{song4 li3}{9,5}{⾡、⽰}[HSK 6]
  \definition{v.}{dar um presente a alguém; presentear alguém com um presente | enviar presentes (para obter favores) | dar um presente; enviar um presente}
\end{EntryWithPhonetic}

\begin{EntryWithPhonetic}{送行}{song4 xing2}{9,6}{⾡、⾏}[HSK 6]
  \definition{v.}{ver alguém partir; ir até o local onde o viajante iniciou sua jornada, despedir-se dele e observar ele partir | dar uma festa de despedida; realizar uma festa de despedida | despedir-se do falecido}
\end{EntryWithPhonetic}

\begin{EntryWithPhonetic}{㮸}{song4}{14}{⽊}
  \variantof{送}
\end{EntryWithPhonetic}

\begin{EntryWithPhonetic}{搜}{sou1}{12}{⼿}[HSK 5]
  \definition{v.}{procurar | pesquisar | coletar; reunir | procurar ou revistar um lugar de forma completa e desordenada}
\end{EntryWithPhonetic}

\begin{EntryWithPhonetic}{搜索}{sou1suo3}{12,10}{⼿、⽷}[HSK 5]
  \definition{v.}{procurar; caçar; explorar; pesquisar cuidadosamente; refere-se especificamente à busca militar para identificar situações suspeitas em determinada região, área marítima ou aérea}
\end{EntryWithPhonetic}

\begin{EntryWithPhonetic}{苏}{su1}{7}{⾋}
  \definition*{s.}{Suzhou, abreviação de 苏州 | Província de Jiangsu, abreviação de 江苏 | União Soviética, abreviação de 苏联 | Sobrenome Su}
  \definition{s.}{perilla planta da família das mentas}
  \definition{v.}{reviver; vir a; acordar}
  \seealsoref{江苏}{jiang1su1}
  \seealsoref{苏联}{su1lian2}
  \seealsoref{苏州}{su1zhou1}
\end{EntryWithPhonetic}

\begin{EntryWithPhonetic}{苏格兰}{su1ge2lan2}{7,10,5}{⾋、⽊、⼋}
  \definition*{s.}{Escócia}
\end{EntryWithPhonetic}

\begin{EntryWithPhonetic}{苏联}{su1lian2}{7,12}{⾋、⽿}
  \definition*{s.}{União das Repúblicas Socialistas Soviéticas (1922-1991)}
\end{EntryWithPhonetic}

\begin{EntryWithPhonetic}{苏州}{su1zhou1}{7,6}{⾋、⼮}
  \definition*{s.}{Suzhou, cidade na Província de Jiangsu}
\end{EntryWithPhonetic}

\begin{EntryWithPhonetic}{素}{su4}{10}{⽷}
  \definition{adj.}{branco; de cor natural | simples; natural; singelo; de cor simples | nativo; original | normal; usual; geral}
  \definition{adv.}{geralmente; sempre; habitualmente}
  \definition{s.}{vegetais, frutas e outros alimentos (em oposição à 荤) | matéria-prima; matéria-prima básico; tecidos de seda naturais e não processados | elemento; os componentes básicos de algo}
  \seealsoref{荤}{hun1}
\end{EntryWithPhonetic}

\begin{EntryWithPhonetic}{素质}{su4zhi4}{10,8}{⽷、⾙}[HSK 6]
  \definition[个,种]{s.}{qualidade; características; caráter; o nível físico, moral, mental, intelectual e cultural de uma pessoa}
\end{EntryWithPhonetic}

\begin{EntryWithPhonetic}{速}{su4}{10}{⾡}
  \definition{adj.}{rápido; veloz}
  \definition{s.}{velocidade}
  \definition{v.aux.}{convidar}
\end{EntryWithPhonetic}

\begin{EntryWithPhonetic}{速度}{su4du4}{10,9}{⾡、⼴}[HSK 3]
  \definition[个,种]{s.}{velocidade; taxa; ritmo; andamento; uma quantidade física que indica a velocidade e a direção do movimento de um objeto, ou seja, a distância que um objeto percorre em uma direção por unidade de tempo | velocidade; rapidez; geralmente se refere ao grau de velocidade}
\end{EntryWithPhonetic}

\begin{EntryWithPhonetic}{宿}{su4}{11}{⼧}
  \definition*{s.}{Sobrenome Su}
  \definition{adj.}{de longa data; antigo; velho | veterano; velho; experiente}
  \definition{v.}{hospedar-se para passar a noite; passar a noite}
  \seeref{xiu3}
  \seeref{xiu4}
\end{EntryWithPhonetic}

\begin{EntryWithPhonetic}{宿舍}{su4she4}{11,8}{⼧、⾆}[HSK 5]
  \definition[间,幢]{s.}{alojamento; dormitório; república; albergue; casas onde escolas, empresas, etc. acomodam seus alunos ou funcionários}
\end{EntryWithPhonetic}

\begin{EntryWithPhonetic}{塑}{su4}{13}{⼟}
  \definition{s.}{plástico; material plástico}
  \definition{v.}{modelo; molde; forma}
\end{EntryWithPhonetic}

\begin{EntryWithPhonetic}{塑料}{su4 liao4}{13,10}{⼟、⽃}[HSK 4]
  \definition[块,种]{s.}{plástico; compostos de polímeros feitos de resinas naturais ou sintéticas como componente principal}
\end{EntryWithPhonetic}

\begin{EntryWithPhonetic}{塑料袋}{su4liao4dai4}{13,10,11}{⼟、⽃、⾐}[HSK 4]
  \definition[个,只]{s.}{saco plástico; sacola de plástico}
\end{EntryWithPhonetic}

\begin{EntryWithPhonetic}{痠}{suan1}{12}{⽧}
  \definition{v.}{doer | estar dolorido}
  \variantof{酸}
\end{EntryWithPhonetic}

\begin{EntryWithPhonetic}{酸}{suan1}{14}{⾣}[HSK 4]
  \definition{adj.}{azedo; ácido | aflito; angustiado; doente do coração | pedante; descreve uma pessoa que finge ser culta e também descreve uma pessoa que é muito inflexível com suas próprias ideias e não está disposta a mudá-las para atender às exigências da época, é usado principalmente para satirizar intelectuais que fingem ser capazes de escrever poemas e artigos | ciumento; invejoso; sentimentos desconfortáveis porque outra pessoa é melhor do que você e, em geral, também apresenta comportamento hostil}
  \definition{s.}{ácido; produto químico que tem um sabor ácido quando misturado com água}
  \definition{v.}{estar dolorido (devido à fadiga ou doença); descreve a sensação de não ter força muscular e um pouco de dor por estar doente ou muito cansado}
\end{EntryWithPhonetic}

\begin{EntryWithPhonetic}{酸辣汤}{suan1la4tang1}{14,14,6}{⾣、⾟、⽔}
  \definition{s.}{sopa avinagrada e picante (prato)}
\end{EntryWithPhonetic}

\begin{EntryWithPhonetic}{酸奶}{suan1 nai3}{14,5}{⾣、⼥}[HSK 4]
  \definition[瓶,杯,盒,袋]{s.}{iogurte; produto lácteo fermentado por bactérias de ácido láctico}
\end{EntryWithPhonetic}

\begin{EntryWithPhonetic}{酸甜苦辣}{suan1 tian2 ku3 la4}{14,11,8,14}{⾣、⽢、⾋、⾟}[HSK 5]
  \definition{expr.}{os altos e baixos da vida; as experiências agridoces da vida; os aspectos doces, azedos, amargos e picantes da vida; refere-se a todos os tipos de sabores, como metáfora para experiências diversas, como felicidade, sofrimento, etc. | azedo, doce, amargo, picante --- alegrias e tristezas da vida}
\end{EntryWithPhonetic}

\begin{EntryWithPhonetic}{算}{suan4}{14}{⽵}[HSK 2]
  \definition{adv.}{finalmente; por fim; no final; significa que, após um longo período de tempo ou muitas dificuldades, finalmente se alcançou o objetivo, equivalente a 总算}
  \definition{v.}{calcular; estimar; computar | contar; incluir | planejar; calcular; projetar | pensar; supor; especular | considerar; considerar como; contar como; reconhecer como | (aritmética) contar; ter peso | deixe estar; deixe passar; seguido por 了: desistir, não se importar mais}
  \seealsoref{了}{le5}
  \seealsoref{总算}{zong3suan4}
\end{EntryWithPhonetic}

\begin{EntryWithPhonetic}{算了}{suan4 le5}{14,2}{⽵、⼅}[HSK 6]
  \definition{part.}{deixe estar; deixe passar; usado no final de uma frase para expressar imperativo, término, etc.}
  \definition{v.}{deixar;  deixe estar; deixe passar; esquecer isso; não querer continuar; é usado para persuadir os outros ou para expressar que posso aceitar a situação atual, para encerrar o assunto ou assunto atual, ou para dizer "esqueça"}
\end{EntryWithPhonetic}

\begin{EntryWithPhonetic}{算命}{suan4ming4}{14,8}{⽵、⼝}
  \definition{s.}{cartomante}
  \definition{v.}{ler a sorte | fazer advinhações}
\end{EntryWithPhonetic}

\begin{EntryWithPhonetic}{算是}{suan4 shi4}{14,9}{⽵、⽇}[HSK 6]
  \definition{adv.}{finalmente; por fim; depois de muito tempo, o objetivo foi finalmente alcançado}
  \definition{v.}{contar como; pensar que; ser considerado}
\end{EntryWithPhonetic}

\begin{EntryWithPhonetic}{尿}{sui1}{7}{⼫}
  \definition{s.}{(coloquial) urina}
  \seeref{niao4}
\end{EntryWithPhonetic}

\begin{EntryWithPhonetic}{虽}{sui1}{9}{⾍}[HSK 6]
  \definition{conj.}{no entanto; embora | mesmo se}
\end{EntryWithPhonetic}

\begin{EntryWithPhonetic}{虽然}{sui1 ran2}{9,12}{⾍、⽕}[HSK 2]
  \definition{conj.}{apesar de; embora (frequentemente usado correlativamente com 可是, 但是, etc); geralmente é usado no início de uma frase para indicar que o fato anterior foi reconhecido, mas não mudará o que acontecerá em seguida}
  \seealsoref{但是}{dan4 shi4}
  \seealsoref{可是}{ke3shi4}
\end{EntryWithPhonetic}

\begin{EntryWithPhonetic}{随}{sui2}{11}{⾩}[HSK 3]
  \definition*{s.}{Sobrenome Sui}
  \definition{adv.}{fazer algo imediatamente assim que ocorre, sem demora ou hesitação; usado antes de dois verbos ou frases verbais para indicar que a última ação segue a anterior}
  \definition{prep.}{junto com (alguma outra ação) | apresentando as condições das quais a ação depende}
  \definition{v.}{seguir; vir (ou ir) junto com | concordar com; adaptar-se a | deixar (alguém fazer o que quiser) | (dialeto) parecer-se com; assemelhar-se a | seguir ou agir de acordo com a condição ou circunstância da qual a ação depende}
\end{EntryWithPhonetic}

\begin{EntryWithPhonetic}{随便}{sui2bian4}{11,9}{⾩、⼈}[HSK 2]
  \definition{adj.}{relaxado; descontraído; sem restrições; sem limitações | aleatório; casual; descuidado; indiferente; distraído, não pensa bem antes de falar ou agir | casual; informal; não dá importância aos detalhes}
  \definition{conj.}{qualquer; qualquer que seja; não importa}
  \definition{v.}{deixar alguém à vontade}
\end{EntryWithPhonetic}

\begin{EntryWithPhonetic}{随处}{sui2chu4}{11,5}{⾩、⼡}
  \definition{adv.}{em qualquer lugar}
\end{EntryWithPhonetic}

\begin{EntryWithPhonetic}{随地}{sui2di4}{11,6}{⾩、⼟}
  \definition{adv.}{qualquer lugar | todo lugar}
\end{EntryWithPhonetic}

\begin{EntryWithPhonetic}{随后}{sui2 hou4}{11,6}{⾩、⼝}[HSK 5]
  \definition{adv.}{logo em seguida; logo depois; indica que segue imediatamente após a ação ou situação anterior (geralmente usado em conjunto com 就)}
  \seealsoref{就}{jiu4}
\end{EntryWithPhonetic}

\begin{EntryWithPhonetic}{随机存取存储器}{sui2ji1cun2qu3cun2chu3qi4}{11,6,6,8,6,12,16}{⾩、⽊、⼦、⼜、⼦、⼈、⼝}
  \definition{s.}{RAM (\emph{random access memory})}
  \seealsoref{内存}{nei4cun2}
  \seealsoref{随机存取记忆体}{sui2ji1cun2qu3ji4yi4ti3}
\end{EntryWithPhonetic}

\begin{EntryWithPhonetic}{随机存取记忆体}{sui2ji1cun2qu3ji4yi4ti3}{11,6,6,8,5,4,7}{⾩、⽊、⼦、⼜、⾔、⼼、⼈}
  \definition{s.}{RAM (\emph{random access memory})}
  \seealsoref{内存}{nei4cun2}
  \seealsoref{随机存取存储器}{sui2ji1cun2qu3cun2chu3qi4}
\end{EntryWithPhonetic}

\begin{EntryWithPhonetic}{随时}{sui2shi2}{11,7}{⾩、⽇}[HSK 2]
  \definition{adv.}{a qualquer momento; em todos os momentos}
\end{EntryWithPhonetic}

\begin{EntryWithPhonetic}{随手}{sui2shou3}{11,4}{⾩、⼿}[HSK 4]
  \definition{adv.}{convenientemente; sem problemas adicionais; casualmente}
\end{EntryWithPhonetic}

\begin{EntryWithPhonetic}{随意}{sui2yi4}{11,13}{⾩、⼼}[HSK 5]
  \definition{adj.}{aleatório; casual; à vontade; como se deseja}
\end{EntryWithPhonetic}

\begin{EntryWithPhonetic}{随着}{sui2zhe5}{11,11}{⾩、⽬}[HSK 5]
  \definition{prep.}{junto com; na esteira de; em sintonia com; usado no início da frase ou antes do verbo, indica as condições necessárias para que uma ação, comportamento ou evento ocorra}
\end{EntryWithPhonetic}

\begin{EntryWithPhonetic}{岁}{sui4}{6}{⼭}[HSK 1]
  \definition{clas.}{usado para anos (de idade)}
  \definition{s.}{ano (literário) | colheita do ano (literário) | idade | tempo (literário) | ano (de idade) | ano (para as colheitas)}
\end{EntryWithPhonetic}

\begin{EntryWithPhonetic}{岁数}{sui4 shu4}{6,13}{⼭、⽁}[HSK 6]
  \definition{s.}{idade; anos; a idade de uma pessoa}
\end{EntryWithPhonetic}

\begin{EntryWithPhonetic}{岁月}{sui4yue4}{6,4}{⼭、⽉}[HSK 5]
  \definition[段,番]{s.}{anos; ano e mês; refere-se a tempo em geral}
\end{EntryWithPhonetic}

\begin{EntryWithPhonetic}{碎}{sui4}{13}{⽯}[HSK 5]
  \definition*{s.}{Sobrenome Sui}
  \definition{adj.}{quebrado; fragmentado | tagarela; falante}
  \definition{v.}{quebrar em pedaços; esmagar}
\end{EntryWithPhonetic}

\begin{EntryWithPhonetic}{隧}{sui4}{14}{⾩}
  \definition{s.}{túnel; passagem subterrânea | estrada | subúrbios; áreas suburbanas}
  \definition{v.}{virar}
\end{EntryWithPhonetic}

\begin{EntryWithPhonetic}{隧道}{sui4dao4}{14,12}{⾩、⾡}
  \definition{s.}{túnel}
\end{EntryWithPhonetic}

\begin{EntryWithPhonetic}{孙}{sun1}{6}{⼦}
  \definition*{s.}{Sobrenome Sun}
  \definition{s.}{neto; neta | gerações abaixo da do neto | parentes pertencentes à geração do neto | segundo crescimento das plantas}
\end{EntryWithPhonetic}

\begin{EntryWithPhonetic}{孙女}{sun1nv3}{6,3}{⼦、⼥}[HSK 4]
  \definition[个]{s.}{filha do filho; neta}
\end{EntryWithPhonetic}

\begin{EntryWithPhonetic}{孙武}{sun1wu3}{6,8}{⼦、⽌}
  \definition*{s.}{Sun Wu, também conhecido por Sun Tzu, 孙子, general, estrategista e filósofo autor do ``Arte da Guerra'', 《孙子兵法》}
  \seealsoref{孙子}{sun1zi3}
  \seealsoref{孙子兵法}{sun1zi3 bing1fa3}
\end{EntryWithPhonetic}

\begin{EntryWithPhonetic}{孙子}{sun1zi3}{6,3}{⼦、⼦}
  \definition*{s.}{Sun Tzu, também conhecido por Sun Wu, 孙武, general, estrategista e filósofo autor do ``Arte da Guerra'', 《孙子兵法》}
  \seeref{sun1zi5}
  \seealsoref{孙武}{sun1wu3}
  \seealsoref{孙子兵法}{sun1zi3 bing1fa3}
\end{EntryWithPhonetic}

\begin{EntryWithPhonetic}{孙子兵法}{sun1zi3 bing1fa3}{6,3,7,8}{⼦、⼦、⼋、⽔}
  \definition*{s.}{``Arte da Guerra'', o antigo clássico chinês sobre estratégia militar, escrito por Sun Tzu, 孫子}
  \seealsoref{孙武}{sun1wu3}
  \seealsoref{孙子}{sun1zi3}
\end{EntryWithPhonetic}

\begin{EntryWithPhonetic}{孙子}{sun1zi5}{6,3}{⼦、⼦}[HSK 4]
  \definition[个]{s.}{filho do filho; neto}
  \seeref{sun1zi3}
\end{EntryWithPhonetic}

\begin{EntryWithPhonetic}{损}{sun3}{10}{⼿}
  \definition{adj.}{sarcástico; cortante; de ​​língua afiada; maldoso; mau; cruel}
  \definition{v.}{diminuir; perder; reduzir | prejudicar; danificar; degradar; destruir; arruinar; destruir o estado original ou fazê-lo perder sua eficácia original | ser sarcástico; ser cáustico; ser cortante; ferir; insultar; usar palavras duras para zombar de alguém}
\end{EntryWithPhonetic}

\begin{EntryWithPhonetic}{损害}{sun3 hai4}{10,10}{⼿、⼧}[HSK 5]
  \definition{v.}{prejudicar; danificar; ferir; causar danos; causar perdas}
\end{EntryWithPhonetic}

\begin{EntryWithPhonetic}{损失}{sun3shi1}{10,5}{⼿、⼤}[HSK 5]
  \definition{s.}{perda; desperdício; algo que se consome ou se perde sem custo algum}
  \definition{v.}{perder; consumir ou perder}
\end{EntryWithPhonetic}

\begin{EntryWithPhonetic}{笋}{sun3}{10}{⽵}
  \definition{s.}{broto de bambu}
\end{EntryWithPhonetic}

\begin{EntryWithPhonetic}{莎}{suo1}{10}{⾋}
  \seeref{sha1}
\end{EntryWithPhonetic}

\begin{EntryWithPhonetic}{缩}{suo1}{14}{⽷}
  \definition*{s.}{Sobrenome Suo}
  \definition{v.}{contrair; encolher | recuar; retirar-se | economizar}
\end{EntryWithPhonetic}

\begin{EntryWithPhonetic}{缩短}{suo1duan3}{14,12}{⽷、⽮}[HSK 4]
  \definition{v.}{encurtar; reduzir; diminuir}
\end{EntryWithPhonetic}

\begin{EntryWithPhonetic}{缩小}{suo1 xiao3}{14,3}{⽷、⼩}[HSK 4]
  \definition{v.}{reduzir, estreitar, encolher;  tornar menor (em oposição a 放大)}
  \seealsoref{放大}{fang4da4}
\end{EntryWithPhonetic}

\begin{EntryWithPhonetic}{缩影卡片}{suo1ying3 ka3pian4}{14,15,5,4}{⽷、⼺、⼘、⽚}
  \definition{s.}{cartão em miniatura}
\end{EntryWithPhonetic}

\begin{EntryWithPhonetic}{所}{suo3}{8}{⼾}[HSK 3,6]
  \definition*{s.}{Sobrenome Suo}
  \definition{clas.}{usado para casas, etc.}
  \definition{part.}{usado com 为 ou 被 para indicar voz passiva | usado antes do verbo para formar um substantivo ou para qualificar um substantivo | usado antes do verbo na estrutura sujeito-predicado usada como complemento, indica que o termo central é o objeto}
  \definition{s.}{lugar | usado como nome de órgãos governamentais ou outros locais de trabalho}
  \seealsoref{被}{bei4}
  \seealsoref{为}{wei4}
\end{EntryWithPhonetic}

\begin{EntryWithPhonetic}{所长}{suo3 chang2}{8,4}{⼾、⾧}
  \definition{s.}{aquilo em que alguém é bom; o ponto forte de alguém; o forte de alguém}
  \seeref{suo3 zhang3}
\end{EntryWithPhonetic}

\begin{EntryWithPhonetic}{所以}{suo3 yi3}{8,4}{⼾、⼈}[HSK 2]
  \definition{conj.}{assim; portanto; como resultado; conecta frases, expressa resultados e costuma corresponder a expressões como 因为 e 由于}
  \definition[个]{s.}{motivo real; causa real; comportamento adequado}
  \seealsoref{因为}{yin1wei4}
  \seealsoref{由于}{you2yu2}
\end{EntryWithPhonetic}

\begin{EntryWithPhonetic}{所有}{suo3you3}{8,6}{⼾、⽉}[HSK 2]
  \definition{adj.}{todo | tudo}
  \definition{adj.}{tudo}
  \definition{s.}{bens; posses;}
  \definition{v.}{possuir; ter}
\end{EntryWithPhonetic}

\begin{EntryWithPhonetic}{所在}{suo3 zai4}{8,6}{⼾、⼟}[HSK 5]
  \definition[个]{s.}{lugar; local; localização | o lugar onde alguém ou algo está}
\end{EntryWithPhonetic}

\begin{EntryWithPhonetic}{所长}{suo3 zhang3}{8,4}{⼾、⾧}[HSK 3]
  \definition{s.}{chefe de um instituto, etc. | superintendente}
  \seeref{suo3 chang2}
\end{EntryWithPhonetic}

\begin{EntryWithPhonetic}{索}{suo3}{10}{⽷}
  \definition*{s.}{Sobrenome Suo}
  \definition{adj.}{completamente sozinho; sozinho | maçante; insípido; sem significado}
  \definition[根]{s.}{corda; cabo; cordão; corrente | uma corda grande}
  \definition{v.}{(literário) pesquisar | exigir; pedir}
\end{EntryWithPhonetic}

\begin{EntryWithPhonetic}{索性}{suo3xing4}{10,8}{⽷、⼼}
  \definition{adv.}{poderia muito bem | simplesmente | apenas}
\end{EntryWithPhonetic}

\begin{EntryWithPhonetic}{锁}{suo3}{12}{⾦}[HSK 5]
  \definition[把]{s.}{fechadura; dispositivo que impede que as pessoas abram facilmente a parte que se abre e fecha | correntes; cadeado e correntes | qualquer coisa com a forma de um cadeado antigo}
  \definition{v.}{trancar; trancar com chave | costurar com ponto fixo | tricotar}
\end{EntryWithPhonetic}

%%%%% EOF %%%%%

