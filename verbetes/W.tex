%%%
%%% W
%%%
\section*{W}
\addcontentsline{toc}{section}{W}
\begin{multicols}{2}

\begin{hanzi}[外号]{wai4hao4}
\entry{wai4hao4}{n.}{
    apelido
}
\end{hanzi}

\begin{hanzi}[外边]{wai4bian0}
\entry{wai4bian0}{n.}{
    fora; por fora; exterior
}
\end{hanzi}

\begin{hanzi}[外公]{wai4gong1}
\entry{wai4gong1}{n.}{
    avô materno
}
\end{hanzi}

\begin{hanzi}[外面]{wai4mian0}
\entry{wai4mian0}{n.}{
    fora; por fora; exterior
}
\end{hanzi}

\begin{hanzi}[外婆]{wai4po2}
\entry{wai4po2}{n.}{
    avó materna
}
\end{hanzi}

\begin{hanzi}[外事]{wai4shi4}
\entry{wai4shi4}{n.}{
    assuntos/relações exteriores
}
\end{hanzi}

\begin{hanzi}[外孙]{wai4sun1}
\entry{wai4sun1}{n.}{
    filho da filha
}
\end{hanzi}

\begin{hanzi}[外孙女]{wai4sun1nv3}
\entry{wai4sun1nv3}{n.}{
    filha da filha
}
\end{hanzi}

\begin{hanzi}[玩]{wan2}
\entry{wan2}{v.}{
    brincar|
    tocar (intrumento musical)
}
\end{hanzi}

\begin{hanzi}[玩儿]{wanr2}
\entry{wanr2}{v.}{
    divertir-se
}
\end{hanzi}

\begin{hanzi}[完]{wan2}
\entry{wan2}{v.}{
    acabar; terminar
}
\end{hanzi}

\begin{hanzi}[晚]{wan3}
\entry{wan3}{adj.}{
    tarde
}
\end{hanzi}

\begin{hanzi}[晚饭]{wan3fan4}
\entry{wan3fan4}{n.}{
    jantar
}
\end{hanzi}

\begin{hanzi}[晚上]{wan3shang0}
\entry{wan3shang0}{p.t.}{
    noite; à noite
}
\end{hanzi}

\begin{hanzi}[碗]{wan3}
\entry{wan3}{n}{
    tigela
}
\entry{wan3}{p.c.}{
    tigelas
}
\end{hanzi}

\begin{hanzi}[碗子]{wan3zi0}
\entry{wan3zi0}{n}{
    tigela
}
\end{hanzi}

\begin{hanzi}[万]{wan4}
\entry{wan4}{num.}{
    10.000|
    dez mil
}
\end{hanzi}

\begin{hanzi}[往]{wang3}
\entry{wang3}{prep.}{
    para; em direção a
}
\end{hanzi}

\begin{hanzi}[网球]{wang3qiu2}
\entry{wang3qiu2}{n.}{
    tênis (esporte)
}
\end{hanzi}

\begin{hanzi}[温度]{wen1du4}
\entry{wen1du4}{n.}{
    temperatura
}
\end{hanzi}

\begin{hanzi}[喂]{wei2}
\entry{wei2}{interjeição}{
    ei!|
    chamar atenção (alô, telefone)
}
\entry{wei4}{interjeição}{
    ei!|
    chamar atenção (alô, telefone)
}
\end{hanzi}

\begin{hanzi}[喂]{wei4}
\entry{wei4}{interjeição}{
    ei!|
    chamar atenção (alô, telefone)
}
\entry{wei2}{interjeição}{
    ei!|
    chamar atenção (alô, telefone)
}
\end{hanzi}

\begin{hanzi}[为]{wei4}
\entry{wei4}{prep.}{
    para
}
\end{hanzi}

\begin{hanzi}[位]{wei4}
\entry{wei4}{p.c.}{
    pessoas (com cortesia)
}
\end{hanzi}

\begin{hanzi}[味道]{wei4dao0}
\entry{wei4dao0}{n.}{
    sabor
}
\end{hanzi}

\begin{hanzi}[为什么]{wei4shen2me0}
\entry{wei4shen2me0}{interr.}{
    por que?
}
\end{hanzi}

\begin{hanzi}[卫生间]{wei4sheng1jian1}
\entry{wei4sheng1jian1}{n.}{
    banheiro; toilette
}
\end{hanzi}

\begin{hanzi}[文化]{wen2hua4}
\entry{wen2hua4}{n.}{
    cultura
}
\end{hanzi}

\begin{hanzi}[文学系]{Wen2xue2xi4}
\entry{Wen2xue2xi4}{n.}{
    Faculdade de Letras
}
\end{hanzi}

\begin{hanzi}[问]{wen4}
\entry{wen4}{v.}{
    perguntar
}
\end{hanzi}

\begin{hanzi}[问题]{wen4ti2}
\entry{wen4ti2}{n.}{
    pergunta; questão; problema
}
\end{hanzi}

\begin{hanzi}[我]{wo3}
\entry{wo3}{pron.}{
    eu
}
\end{hanzi}

\begin{hanzi}[我的]{wo3de0}
\entry{wo3de0}{pron.}{
    meu, meus, minha, minhas
}
\end{hanzi}

\begin{hanzi}[我们]{wo3men0}
\entry{wo3men0}{pron.}{
    nós
}
\end{hanzi}

\begin{hanzi}[我们的]{wo3men0de0}
\entry{wo3men0de0}{pron.}{
    nosso, nossos, nossa, nossas
}
\end{hanzi}

\begin{hanzi}[五]{wu3}
\entry{wu3}{num.}{
    5|
    cinco
}
\end{hanzi}

\begin{hanzi}[舞]{wu3}
\entry{wu3}{n.}{
    dança
}
\end{hanzi}

\begin{hanzi}[午饭]{wu3fan4}
\entry{wu3fan4}{n.}{
    almoço
}
\end{hanzi}

\end{multicols}
