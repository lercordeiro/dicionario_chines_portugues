\begin{verbete}[6;7]{优伶}{you1ling2}
  \significado{s.}{ator}
\end{verbete}
